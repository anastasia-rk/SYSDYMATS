\documentclass[a4paper,11pt,twoside]{article}
%%%%%%%%%%%%%%%%%%%%%%%%%%%%%%%%%%%%%%%%%%%%%%%%%%%%%%%%%%%%%%%%%%%%%%%%%%%
% Packages
\usepackage[final]{pdfpages}
\usepackage{verbatim}
\usepackage{inputenc}
\usepackage{graphicx} 
\usepackage{amsmath,amssymb,mathrsfs,amsfonts}
\usepackage{mathtools}
\usepackage{amsthm}
\usepackage{mathtools}
\usepackage{calrsfs}
\usepackage{graphicx}
\usepackage{subfig}
\usepackage{eucal}    
\usepackage{amssymb}  
\usepackage{pifont}
\usepackage{color} 
\usepackage{cancel}
\usepackage[toc,page]{appendix}
\usepackage{pgfplots}
\pgfplotsset{every axis/.append style={line width=0.5pt},label style={font=\scriptsize},tick label style={font=\scriptsize},x tick label style={/pgf/number format/.cd,fixed,precision=3, set thousands separator={}},z tick label style={/pgf/number format/.cd,fixed,precision=3, set thousands separator={}}}
\usetikzlibrary{shapes,shadows,arrows,backgrounds,patterns,positioning,automata,calc,decorations.markings,decorations.pathreplacing,bayesnet,arrows.meta}
%\usepackage{tikzexternal}
%\usepackage{tikz}
%\usepgfplotslibrary{external} 
%\tikzexternalize
\usepackage{varwidth}
\usepackage{lscape}
\usepackage{array} 
\usepackage[colorlinks=false,pdfborder={0 0 0}]{hyperref}
\usepackage{tabularx}
\usepackage{textcomp}
\usepackage{multicol} 
\usepackage{booktabs}
\usepackage{multirow}
\usepackage[font=small,labelfont=bf]{caption}                                                           
\usepackage{textcase}
\usepackage{bbm} 
\usepackage{fancyhdr}
\usepackage{enumitem}
\usepackage{soul}
%\usepackage[british]{babel}
\usepackage{wrapfig}
%\usepackage{glossaries}
%%%%%%%%%%%%%%%%%%%%%%%%%%%%%%%%%%%%%%%%%%%%%%%%%%%%%%%%%%%%%%%%%%%%%%%%%%%
\setlength{\parindent}{2em}
\setlength{\parskip}{0.5em}
\renewcommand{\baselinestretch}{1.2}
\usepackage[left=2cm, right=2cm, top=2.5cm, bottom=3cm, headheight=13.6pt]{geometry}
\allowdisplaybreaks 
% Bibliography
\usepackage[backend=bibtex,style=ieee,sorting=none]{biblatex} 
\bibliography{bibliography}
\renewcommand*{\bibfont}{\scriptsize}
\makeatletter
\def\input@path{{../Results_set_C_ny_0_nu_4/}}
\newcommand*{\rom}[1]{\expandafter\@slowromancap\romannumeral #1@}
\newcommand{\ie}{\textit{i.e.} }
\newcommand{\eg}{\textit{e.g.} }
\newcommand\id{\ensuremath{\mathbbm{1}}} 
\DeclareMathOperator{\E}{\mathbb{E}}
\DeclareMathOperator{\eye}{\mathbb{I}}
\DeclareMathOperator{\zeros}{\mathbb{O}}
\DeclareMathOperator{\tr}{\textrm{tr}}
\DeclareMathOperator{\vvec}{\textrm{vec}}
\DeclareMathOperator{\ik}{\mathrm{k}}
\DeclareMathOperator{\ip}{\mathrm{p}}
\DeclareMathOperator{\inn}{\mathrm{n}}
\DeclareMathOperator{\im}{\mathrm{m}}
\DeclareMathOperator{\td}{\mathrm{t}}
\DeclareMathOperator{\kd}{\mathrm{k}}
\DeclareMathOperator{\T}{\mathrm{T}}
\DeclareMathOperator{\K}{\mathrm{K}}
\DeclareMathOperator{\rk}{\mathrm{rk}}
\DeclareMathOperator{\vc}{\mathrm{vec}}
\DeclareSymbolFontAlphabet{\mathcal} {symbols}
\DeclareSymbolFont{symbols}{OMS}{cm}{m}{n}
\DeclareMathAlphabet{\mathbfit}{OML}{cmm}{b}{it}
\makeatother
% Number equations
%\numberwithin{equation}{section}

%%%%%%%%%%%%%%%%%%%%%%%%%%%%%%%%%%%%%%%%%%%%%%%%%%%%%%%%%%%%%%%%%%%%%%%%%%%
%Theorems
\newtheoremstyle{mytheoremstyle} % name
{.5em}                    % Space above
{.8em}                    % Space below
{\itshape}                % Body font
{1em}                           % Indent amount
{\bfseries}                   % Theorem head font
{:}                          % Punctuation after theorem head
{.5em}                       % Space after theorem head
{}  % Theorem head spec (can be left empty, meaning ‘normal’)

\theoremstyle{mytheoremstyle}
\newtheorem{theorem}{Theorem}[section]
\newtheorem{remark}{Remark}[section]
\newtheorem{assumption}{Assumption}[section]
\newtheorem{lemma}{Lemma}[section]
\newtheorem{condition}{Condition}[section]
\newtheorem{definition}{Definition}[section]
\newtheorem{property}{Property}[section]
\newtheorem{corollary}{Corollary}[section]
\renewcommand\qedsymbol{$\blacksquare$}
%%%%%%%%%%%%%%%%%%%%%%%%%%%%%%%%%%%%%%%%%%%%%%%%%%%%%%%%%%%%%%%%%%%%%%%%%%%
% Nomenclature
\usepackage[intoc]{nomencl}
\makenomenclature

%\usepackage[ruled,chapter]{algorithm}
%\usepackage{float}

%\usepackage{algorithmic}
%\algsetup{linenosize=\scriptsize}
%\usepackage{etoolbox}
%\AtBeginEnvironment{algorithmic}{\scriptsize}
%\renewcommand{\thealgorithm}{\thechapter.\arabic{algorithm}} 
%\usepackage{chngcntr}
%\counterwithin{algorithm}{section}

% correct bad hyphenation here
\hyphenation{op-tical net-works semi-conduc-tor}
%%%%%%%%%%%%%%%%%%%%%%%%%%%%%%%%%%%%%%%%%%%%%%%%%%%%%%%%%%%%%%%%%%%%%%%%%%%%%
% Captions
%\newcommand{\xLanguage}{british}          % <-- Added this command
%
%\usepackage[\xLanguage]{babel}             % <-- Implemented here
%
%\expandafter\addto\csname captions\xLanguage\endcsname{% <-- and here
%	\renewcommand{\tableshortname}{Table}%
%	\renewcommand{\figureshortname}{Figure}%
%}
%
%\captionsetup[table]{format=plain,indention=1.15cm,justification=justified}
%\captionsetup[figure]{format=plain,indention=1.25cm,justification=justified}

%%%%%%%%%%%%%%%%%%%%%%%%%%%%%%%%%%%%%%%%%%%%%%%%%%%%%%%%%%%%%%%%%%%%%%%%%%%%%
\usepackage[explicit]{titlesec}
\usepackage{titletoc}
%\titleformat{\section}[block]{\normalfont\Large\rm\filright\bfseries}{\thesection}{1em}{#1}
%\titlecontents{chapter}[1.5em]{}{\scshape\contentslabel{2.3em}}{}{\titlerule*[1pc]{}\contentspage}
%\definecolor{gray75}{gray}{0.5}
%\newcommand{\hsp}{\hspace{20pt}}
%\titleformat{\chapter}[hang]{\huge\scshape\filright\bfseries}{\color{gray75}\thechapter}{20pt}{\begin{tabular}[t]{@{\color{gray75}\vrule width 2pt\hsp}p{0.85\textwidth}}\raggedright#1\end{tabular}}
%\titleformat{name=\chapter,numberless}[display]{}{}{0pt}{\normalfont\huge\bfseries #1} % format for numberless chapters
%\titleformat{\subsection}[block]{\normalfont\large\rm\filright\bfseries}{\thesubsection}{1em}{#1}
%\renewcommand{\sectionmark}[1]{\markright{\thesection ~ \ #1}}
%%\renewcommand{\chaptermark}[1]{\markboth{\chaptername\ \thechapter ~ \ #1}{}} 
%\pagestyle{fancy}
%%\fancyhf{}
%\fancyhead[RO,LE]{\thepage}
%\fancyhead[RE]{\itshape \nouppercase \rightmark}      % chaptertitle left
%\fancyhead[LO]{\itshape \nouppercase \leftmark}       % sectiontitle right
%\renewcommand{\headrulewidth}{0.5pt} 				   % no rule
%\cfoot{}
\interfootnotelinepenalty=10000

\title{Structure and parameter identification of the dynamical model of auxetic foam}


\begin{document}
	\maketitle
%	\section{Experimental setup}
%	
\section{Experimental data}
\par The experiments is conducted on ten samples of authentic elastomeric foams prepared in different manufacturing settings. In this report, the influence of two manufacturing parameters on the dynamical behaviour of the foam is evaluated: the length of the foam sample after cutting, $L_{cut}$ (mm), and the relaxed density of the sample, $D_{rlx}$ (g/cm$^3$). The parameter values for each specimen are presented in Table \ref{tab:externparams}.
\begin{table}[!h]
	\centering
	\caption{Manufacturing parameters of auxetic form used in 10 experiments.}\label{tab:externparams}
	\scriptsize
	\begin{tabular}{rrrrrrrrrrr}
		Parameter  & C1 & C2 & C3 & C4 & C5 & C6 & C7 & C8 & C9 & C10 \\ 
		\hline 
		$L_{cut}$ & 74 & 72 & 69 & 61 & 56 & 67 & 66 & 64 & 62 & 57 \\ 
		$D_{rlx}$ & 0.123 & 0.113 & 0.104 & 0.095 & 0.093 & 0.276 & 0.255 & 0.230 & 0.209 & 0.193 \\ 
		\hline 
	\end{tabular}
\end{table}
The random input signal adopted for all foam sample is illustrated in Figure \ref{fig:input}. 
\begin{figure}[!h]
	\centering
	% This file was created by matlab2tikz.
%
\definecolor{mycolor1}{rgb}{0.00000,0.44700,0.74100}%
%
\begin{tikzpicture}

\begin{axis}[%
width=11.411cm,
height=5cm,
at={(0cm,0cm)},
scale only axis,
xmin=1,
xmax=1500,
xlabel style={font=\color{white!15!black}},
xlabel={Sample index},
ymin=-7,
ymax=-4.999,
ylabel style={font=\color{white!15!black}},
ylabel={$\Delta x$, mm},
axis background/.style={fill=white},
legend style={legend cell align=left, align=left, draw=white!15!black}
]
\addplot [color=mycolor1]
  table[row sep=crcr]{%
1	-5.951\\
3	-5.933\\
5	-5.933\\
7	-5.914\\
9	-5.933\\
11	-5.933\\
13	-5.914\\
15	-5.914\\
17	-5.933\\
19	-5.933\\
21	-5.933\\
23	-5.933\\
25	-5.914\\
27	-5.914\\
29	-5.933\\
31	-5.933\\
33	-5.933\\
35	-5.914\\
37	-5.933\\
39	-5.933\\
41	-5.933\\
43	-5.933\\
45	-5.933\\
47	-5.933\\
49	-5.933\\
51	-5.933\\
53	-5.914\\
55	-5.914\\
57	-5.914\\
59	-5.933\\
61	-5.933\\
63	-5.933\\
65	-5.933\\
67	-5.914\\
69	-5.914\\
71	-5.914\\
73	-5.933\\
75	-5.914\\
77	-5.914\\
79	-5.933\\
81	-5.933\\
83	-5.951\\
85	-5.933\\
87	-5.933\\
89	-5.914\\
91	-5.933\\
93	-5.933\\
95	-5.933\\
97	-5.933\\
99	-5.914\\
101	-5.914\\
103	-5.933\\
105	-5.933\\
107	-5.933\\
109	-5.933\\
111	-5.914\\
113	-5.896\\
115	-5.878\\
117	-5.841\\
119	-5.859\\
121	-6.061\\
123	-6.226\\
125	-6.116\\
127	-6.171\\
129	-6.281\\
131	-6.061\\
133	-5.878\\
135	-6.226\\
137	-6.061\\
139	-6.226\\
141	-5.859\\
143	-5.933\\
145	-6.354\\
147	-6.079\\
149	-5.896\\
151	-5.64\\
153	-5.64\\
155	-5.365\\
157	-5.621\\
159	-5.878\\
161	-5.786\\
163	-5.548\\
165	-5.988\\
167	-6.207\\
169	-5.933\\
171	-6.097\\
173	-5.713\\
175	-5.511\\
177	-5.548\\
179	-5.42\\
181	-5.511\\
183	-5.2\\
185	-4.999\\
187	-5.219\\
189	-5.731\\
191	-5.566\\
193	-5.585\\
195	-5.621\\
197	-5.823\\
199	-6.079\\
201	-6.006\\
203	-5.475\\
205	-5.64\\
207	-5.786\\
209	-5.951\\
211	-5.841\\
213	-6.171\\
215	-6.207\\
217	-6.006\\
219	-5.731\\
221	-5.64\\
223	-5.969\\
225	-5.768\\
227	-5.841\\
229	-5.933\\
231	-6.299\\
233	-5.859\\
235	-5.878\\
237	-5.75\\
239	-5.493\\
241	-5.768\\
243	-5.951\\
245	-5.896\\
247	-5.933\\
249	-6.207\\
251	-5.914\\
253	-6.042\\
255	-5.64\\
257	-5.896\\
259	-5.823\\
261	-5.658\\
263	-5.585\\
265	-6.097\\
267	-5.768\\
269	-5.658\\
271	-5.695\\
273	-5.676\\
275	-5.914\\
277	-6.299\\
279	-6.299\\
281	-5.859\\
283	-5.804\\
285	-5.878\\
287	-6.024\\
289	-6.207\\
291	-5.969\\
293	-6.024\\
295	-6.061\\
297	-6.244\\
299	-6.097\\
301	-5.878\\
303	-5.804\\
305	-5.475\\
307	-5.328\\
309	-5.621\\
311	-5.621\\
313	-5.64\\
315	-5.804\\
317	-5.988\\
319	-6.335\\
321	-6.647\\
323	-6.354\\
325	-5.933\\
327	-6.042\\
329	-5.951\\
331	-5.896\\
333	-6.152\\
335	-6.793\\
337	-6.573\\
339	-6.519\\
341	-6.061\\
343	-5.75\\
345	-5.951\\
347	-5.64\\
349	-5.2\\
351	-5.585\\
353	-6.024\\
355	-6.464\\
357	-6.445\\
359	-6.299\\
361	-6.207\\
363	-6.354\\
365	-6.097\\
367	-6.171\\
369	-5.914\\
371	-5.859\\
373	-5.969\\
375	-6.537\\
377	-6.409\\
379	-6.061\\
381	-5.42\\
383	-5.383\\
385	-5.676\\
387	-6.024\\
389	-6.244\\
391	-6.72\\
393	-6.555\\
395	-6.152\\
397	-6.628\\
399	-6.39\\
401	-6.042\\
403	-5.804\\
405	-5.804\\
407	-5.493\\
409	-5.53\\
411	-5.878\\
413	-5.75\\
415	-6.152\\
417	-6.281\\
419	-6.354\\
421	-6.079\\
423	-5.969\\
425	-5.896\\
427	-5.969\\
429	-5.841\\
431	-5.951\\
433	-5.548\\
435	-6.006\\
437	-6.573\\
439	-6.592\\
441	-6.647\\
443	-6.281\\
445	-6.171\\
447	-5.878\\
449	-5.53\\
451	-5.53\\
453	-5.878\\
455	-6.226\\
457	-6.024\\
459	-6.317\\
461	-6.354\\
463	-6.024\\
465	-5.878\\
467	-5.768\\
469	-6.006\\
471	-5.933\\
473	-5.969\\
475	-6.024\\
477	-6.189\\
479	-6.189\\
481	-6.317\\
483	-6.006\\
485	-5.804\\
487	-5.896\\
489	-6.042\\
491	-6.042\\
493	-5.988\\
495	-5.75\\
497	-6.061\\
499	-5.914\\
501	-5.695\\
503	-5.585\\
505	-5.621\\
507	-5.933\\
509	-6.226\\
511	-6.354\\
513	-6.464\\
515	-6.573\\
517	-6.134\\
519	-5.841\\
521	-6.079\\
523	-5.713\\
525	-5.53\\
527	-5.548\\
529	-5.731\\
531	-5.933\\
533	-6.226\\
535	-6.061\\
537	-6.024\\
539	-6.006\\
541	-5.933\\
543	-6.116\\
545	-6.427\\
547	-6.244\\
549	-6.281\\
551	-5.969\\
553	-6.39\\
555	-5.914\\
557	-6.116\\
559	-6.207\\
561	-5.75\\
563	-5.713\\
565	-5.585\\
567	-5.658\\
569	-5.402\\
571	-5.219\\
573	-5.273\\
575	-5.566\\
577	-5.878\\
579	-6.097\\
581	-5.713\\
583	-5.493\\
585	-5.566\\
587	-5.237\\
589	-5.164\\
591	-5.493\\
593	-5.603\\
595	-5.475\\
597	-5.823\\
599	-6.079\\
601	-6.152\\
603	-6.097\\
605	-5.969\\
607	-5.786\\
609	-5.603\\
611	-5.786\\
613	-5.548\\
615	-5.951\\
617	-6.079\\
619	-6.079\\
621	-6.189\\
623	-6.042\\
625	-5.841\\
627	-5.804\\
629	-6.189\\
631	-6.207\\
633	-6.207\\
635	-5.878\\
637	-6.024\\
639	-6.244\\
641	-6.006\\
643	-5.878\\
645	-5.621\\
647	-5.914\\
649	-5.896\\
651	-5.493\\
653	-5.585\\
655	-5.878\\
657	-5.804\\
659	-5.951\\
661	-5.603\\
663	-5.53\\
665	-5.713\\
667	-5.823\\
669	-6.171\\
671	-6.262\\
673	-6.024\\
675	-6.116\\
677	-6.006\\
679	-6.006\\
681	-5.988\\
683	-6.189\\
685	-6.079\\
687	-6.189\\
689	-6.335\\
691	-6.372\\
693	-6.189\\
695	-6.39\\
697	-6.702\\
699	-6.262\\
701	-6.171\\
703	-6.134\\
705	-6.5\\
707	-6.226\\
709	-6.171\\
711	-6.024\\
713	-6.207\\
715	-6.317\\
717	-6.006\\
719	-5.878\\
721	-5.75\\
723	-5.75\\
725	-6.079\\
727	-6.281\\
729	-6.702\\
731	-6.39\\
733	-6.116\\
735	-5.786\\
737	-5.988\\
739	-6.189\\
741	-5.896\\
743	-6.042\\
745	-6.134\\
747	-6.189\\
749	-5.841\\
751	-5.969\\
753	-5.603\\
755	-5.951\\
757	-5.695\\
759	-5.768\\
761	-5.859\\
763	-6.354\\
765	-6.152\\
767	-6.042\\
769	-6.171\\
771	-6.079\\
773	-5.603\\
775	-5.713\\
777	-5.878\\
779	-5.988\\
781	-6.152\\
783	-5.933\\
785	-6.262\\
787	-6.207\\
789	-6.207\\
791	-6.207\\
793	-6.116\\
795	-5.75\\
797	-6.061\\
799	-6.006\\
801	-5.823\\
803	-5.896\\
805	-5.878\\
807	-5.786\\
809	-5.713\\
811	-5.64\\
813	-5.585\\
815	-5.841\\
817	-5.75\\
819	-6.134\\
821	-5.988\\
823	-6.061\\
825	-5.53\\
827	-5.676\\
829	-6.134\\
831	-6.061\\
833	-6.024\\
835	-6.207\\
837	-5.988\\
839	-5.786\\
841	-5.75\\
843	-5.768\\
845	-5.75\\
847	-6.244\\
849	-5.878\\
851	-5.457\\
853	-5.42\\
855	-5.493\\
857	-5.566\\
859	-6.024\\
861	-6.372\\
863	-6.244\\
865	-6.409\\
867	-6.042\\
869	-5.731\\
871	-5.713\\
873	-5.621\\
875	-5.768\\
877	-5.896\\
879	-5.988\\
881	-6.189\\
883	-6.061\\
885	-6.061\\
887	-6.226\\
889	-6.281\\
891	-6.39\\
893	-6.061\\
895	-5.951\\
897	-5.658\\
899	-5.621\\
901	-5.328\\
903	-5.731\\
905	-5.621\\
907	-5.511\\
909	-5.53\\
911	-5.786\\
913	-5.566\\
915	-6.134\\
917	-6.097\\
919	-6.372\\
921	-6.39\\
923	-6.226\\
925	-6.262\\
927	-5.768\\
929	-5.658\\
931	-5.676\\
933	-5.969\\
935	-5.676\\
937	-5.731\\
939	-5.383\\
941	-5.859\\
943	-5.676\\
945	-5.823\\
947	-5.804\\
949	-6.189\\
951	-6.244\\
953	-6.244\\
955	-6.061\\
957	-6.281\\
959	-6.262\\
961	-6.006\\
963	-5.988\\
965	-5.786\\
967	-5.64\\
969	-5.676\\
971	-5.823\\
973	-5.695\\
975	-5.896\\
977	-5.676\\
979	-5.695\\
981	-5.841\\
983	-5.804\\
985	-6.317\\
987	-5.933\\
989	-6.152\\
991	-6.189\\
993	-6.299\\
995	-6.537\\
997	-6.39\\
999	-6.152\\
1001	-6.244\\
1003	-6.134\\
1005	-6.281\\
1007	-6.244\\
1009	-5.914\\
1011	-5.969\\
1013	-5.585\\
1015	-5.145\\
1017	-5.896\\
1019	-6.061\\
1021	-5.731\\
1023	-5.914\\
1025	-5.933\\
1027	-5.878\\
1029	-6.079\\
1031	-6.079\\
1033	-6.061\\
1035	-5.878\\
1037	-5.969\\
1039	-5.969\\
1041	-6.006\\
1043	-6.207\\
1045	-5.969\\
1047	-5.658\\
1049	-5.676\\
1051	-5.804\\
1053	-5.859\\
1055	-6.042\\
1057	-5.658\\
1059	-5.786\\
1061	-5.493\\
1063	-5.566\\
1065	-5.548\\
1067	-5.841\\
1069	-5.969\\
1071	-6.024\\
1073	-5.951\\
1075	-5.878\\
1077	-6.116\\
1079	-6.519\\
1081	-6.335\\
1083	-6.61\\
1085	-6.555\\
1087	-6.866\\
1089	-6.665\\
1091	-6.244\\
1093	-6.152\\
1095	-5.786\\
1097	-5.695\\
1099	-5.896\\
1101	-6.006\\
1103	-6.226\\
1105	-6.281\\
1107	-6.097\\
1109	-6.042\\
1111	-6.006\\
1113	-5.914\\
1115	-5.804\\
1117	-5.768\\
1119	-5.676\\
1121	-5.969\\
1123	-6.116\\
1125	-5.988\\
1127	-6.226\\
1129	-6.281\\
1131	-5.859\\
1133	-5.969\\
1135	-6.409\\
1137	-6.354\\
1139	-6.281\\
1141	-6.262\\
1143	-5.969\\
1145	-5.841\\
1147	-6.006\\
1149	-6.207\\
1151	-5.951\\
1153	-5.695\\
1155	-5.457\\
1157	-5.64\\
1159	-5.658\\
1161	-5.53\\
1163	-5.219\\
1165	-5.676\\
1167	-6.134\\
1169	-5.804\\
1171	-5.438\\
1173	-5.713\\
1175	-6.024\\
1177	-6.61\\
1179	-6.555\\
1181	-6.335\\
1183	-6.427\\
1185	-6.244\\
1187	-6.171\\
1189	-6.116\\
1191	-6.427\\
1193	-6.079\\
1195	-6.226\\
1197	-6.537\\
1199	-6.683\\
1201	-6.5\\
1203	-6.61\\
1205	-6.152\\
1207	-6.171\\
1209	-5.969\\
1211	-5.969\\
1213	-5.896\\
1215	-5.914\\
1217	-6.134\\
1219	-5.969\\
1221	-6.427\\
1223	-6.152\\
1225	-5.951\\
1227	-5.804\\
1229	-5.859\\
1231	-6.006\\
1233	-6.042\\
1235	-6.207\\
1237	-6.134\\
1239	-6.152\\
1241	-5.731\\
1243	-5.896\\
1245	-5.786\\
1247	-5.933\\
1249	-5.878\\
1251	-5.804\\
1253	-5.676\\
1255	-5.548\\
1257	-5.768\\
1259	-6.024\\
1261	-6.152\\
1263	-5.695\\
1265	-5.75\\
1267	-5.585\\
1269	-5.896\\
1271	-6.134\\
1273	-5.951\\
1275	-5.566\\
1277	-5.328\\
1279	-5.603\\
1281	-5.768\\
1283	-6.097\\
1285	-6.024\\
1287	-6.116\\
1289	-6.226\\
1291	-5.695\\
1293	-5.383\\
1295	-5.42\\
1297	-5.457\\
1299	-5.731\\
1301	-5.786\\
1303	-5.402\\
1305	-5.804\\
1307	-5.914\\
1309	-5.75\\
1311	-6.024\\
1313	-5.988\\
1315	-6.134\\
1317	-6.134\\
1319	-6.061\\
1321	-6.171\\
1323	-6.39\\
1325	-6.006\\
1327	-6.079\\
1329	-6.079\\
1331	-6.171\\
1333	-6.116\\
1335	-6.024\\
1337	-6.299\\
1339	-6.097\\
1341	-5.969\\
1343	-5.511\\
1345	-5.2\\
1347	-5.676\\
1349	-5.768\\
1351	-5.841\\
1353	-5.621\\
1355	-5.511\\
1357	-5.768\\
1359	-5.768\\
1361	-5.548\\
1363	-5.457\\
1365	-6.079\\
1367	-6.189\\
1369	-6.39\\
1371	-6.134\\
1373	-6.061\\
1375	-6.207\\
1377	-6.39\\
1379	-6.628\\
1381	-6.519\\
1383	-6.482\\
1385	-6.482\\
1387	-5.933\\
1389	-5.969\\
1391	-5.75\\
1393	-5.713\\
1395	-5.566\\
1397	-5.53\\
1399	-5.896\\
1401	-6.024\\
1403	-6.354\\
1405	-6.39\\
1407	-6.226\\
1409	-5.969\\
1411	-5.731\\
1413	-5.676\\
1415	-5.365\\
1417	-5.859\\
1419	-6.042\\
1421	-6.097\\
1423	-5.951\\
1425	-5.841\\
1427	-5.914\\
1429	-5.878\\
1431	-6.006\\
1433	-5.804\\
1435	-5.713\\
1437	-5.676\\
1439	-5.878\\
1441	-5.878\\
1443	-5.695\\
1445	-5.896\\
1447	-6.207\\
1449	-6.116\\
1451	-5.951\\
1453	-5.951\\
1455	-5.768\\
1457	-5.914\\
1459	-6.079\\
1461	-6.262\\
1463	-6.592\\
1465	-6.097\\
1467	-5.603\\
1469	-5.621\\
1471	-5.841\\
1473	-5.896\\
1475	-6.079\\
1477	-6.372\\
1479	-6.317\\
1481	-6.061\\
1483	-6.152\\
1485	-6.006\\
1487	-5.75\\
1489	-6.061\\
1491	-5.969\\
1493	-6.189\\
1495	-6.299\\
1497	-6.134\\
1499	-5.823\\
1501	-6.281\\
};
\addlegendentry{data1}

\end{axis}
\end{tikzpicture}%
	\caption{The random input signal for the set of authentic foams is the displacement of the hydraulic actuator of the machine in mm.}\label{fig:input}
\end{figure}
Samples of recorded foam responses are presented in  Figure \ref{fig:output}. The ability to predict the response of a foam specimen to a known input based on manufacturing parameters is crucial in for optimal design. The aim of this report is to identify the dynamical model that could describe the relationship between measured input and output time series conditioned on the external parameters. Datasets C1-C2, C4-C7, and C9-C10 are used for model identification while the datasets C3 and C8 are used to validate the identified model.
\begin{figure}[!t]
	\centering
	% This file was created by matlab2tikz.
% Minimal pgfplots version: 1.3
%
\definecolor{mycolor1}{rgb}{0.00000,0.44700,0.74100}%
%
\begin{tikzpicture}

\begin{axis}[%
width=5.090835cm,
height=2.240107cm,
at={(0cm,0cm)},
scale only axis,
xmin=1000,
xmax=1500,
xlabel={Sample index},
ymin=-200,
ymax=0,
ylabel={C9 load, kN},
legend style={legend cell align=left,align=left,draw=white!15!black}
]
\addplot [color=mycolor1,solid,forget plot]
  table[row sep=crcr]{%
1000	-68.359\\
1001	-85.449\\
1002	-69.58\\
1003	-65.918\\
1004	-90.332\\
1005	-86.67\\
1006	-103.76\\
1007	-79.346\\
1008	-45.166\\
1009	-57.373\\
1010	-51.27\\
1011	-63.477\\
1012	-48.828\\
1013	-24.414\\
1014	-17.09\\
1015	-17.09\\
1016	-43.945\\
1017	-73.242\\
1018	-76.904\\
1019	-79.346\\
1020	-58.594\\
1021	-36.621\\
1022	-76.904\\
1023	-53.711\\
1024	-43.945\\
1025	-69.58\\
1026	-61.035\\
1027	-48.828\\
1028	-84.229\\
1029	-79.346\\
1030	-59.814\\
1031	-87.891\\
1032	-85.449\\
1033	-67.139\\
1034	-54.932\\
1035	-46.387\\
1036	-56.152\\
1037	-64.697\\
1038	-64.697\\
1039	-67.139\\
1040	-68.359\\
1041	-73.242\\
1042	-100.098\\
1043	-87.891\\
1044	-61.035\\
1045	-56.152\\
1046	-34.18\\
1047	-34.18\\
1048	-48.828\\
1049	-37.842\\
1050	-39.063\\
1051	-56.152\\
1052	-62.256\\
1053	-58.594\\
1054	-90.332\\
1055	-64.697\\
1056	-41.504\\
1057	-34.18\\
1058	-45.166\\
1059	-51.27\\
1060	-28.076\\
1061	-25.635\\
1062	-41.504\\
1063	-34.18\\
1064	-29.297\\
1065	-40.283\\
1066	-43.945\\
1067	-68.359\\
1068	-63.477\\
1069	-79.346\\
1070	-69.58\\
1071	-79.346\\
1072	-63.477\\
1073	-62.256\\
1074	-54.932\\
1075	-53.711\\
1076	-53.711\\
1077	-91.553\\
1078	-126.953\\
1079	-131.836\\
1080	-129.395\\
1081	-83.008\\
1082	-114.746\\
1083	-141.602\\
1084	-147.705\\
1085	-111.084\\
1086	-146.484\\
1087	-185.547\\
1088	-131.836\\
1089	-112.305\\
1090	-75.684\\
1091	-61.035\\
1092	-52.49\\
1093	-64.697\\
1094	-46.387\\
1095	-31.738\\
1096	-31.738\\
1097	-40.283\\
1098	-57.373\\
1099	-52.49\\
1100	-53.711\\
1101	-72.021\\
1102	-72.021\\
1103	-97.656\\
1104	-79.346\\
1105	-101.318\\
1106	-72.021\\
1107	-64.697\\
1108	-73.242\\
1109	-54.932\\
1110	-51.27\\
1111	-59.814\\
1112	-62.256\\
1113	-43.945\\
1114	-47.607\\
1115	-34.18\\
1116	-40.283\\
1117	-42.725\\
1118	-30.518\\
1119	-39.063\\
1120	-47.607\\
1121	-72.021\\
1122	-74.463\\
1123	-81.787\\
1124	-48.828\\
1125	-70.801\\
1126	-101.318\\
1127	-84.229\\
1128	-93.994\\
1129	-91.553\\
1130	-54.932\\
1131	-40.283\\
1132	-56.152\\
1133	-61.035\\
1134	-97.656\\
1135	-119.629\\
1136	-117.188\\
1137	-89.111\\
1138	-92.773\\
1139	-81.787\\
1140	-80.566\\
1141	-79.346\\
1142	-54.932\\
1143	-48.828\\
1144	-45.166\\
1145	-40.283\\
1146	-45.166\\
1147	-64.697\\
1148	-96.436\\
1149	-74.463\\
1150	-52.49\\
1151	-45.166\\
1152	-46.387\\
1153	-29.297\\
1154	-23.193\\
1155	-26.855\\
1156	-46.387\\
1157	-35.4\\
1158	-41.504\\
1159	-40.283\\
1160	-39.063\\
1161	-28.076\\
1162	-21.973\\
1163	-17.09\\
1164	-29.297\\
1165	-58.594\\
1166	-80.566\\
1167	-91.553\\
1168	-59.814\\
1169	-43.945\\
1170	-32.959\\
1171	-23.193\\
1172	-47.607\\
1173	-46.387\\
1174	-67.139\\
1175	-81.787\\
1176	-131.836\\
1177	-147.705\\
1178	-120.85\\
1179	-117.188\\
1180	-78.125\\
1181	-81.787\\
1182	-91.553\\
1183	-98.877\\
1184	-73.242\\
1185	-68.359\\
1186	-69.58\\
1187	-62.256\\
1188	-75.684\\
1189	-53.711\\
1190	-93.994\\
1191	-108.643\\
1192	-80.566\\
1193	-51.27\\
1194	-54.932\\
1195	-92.773\\
1196	-109.863\\
1197	-134.277\\
1198	-140.381\\
1199	-140.381\\
1200	-106.201\\
1201	-96.436\\
1202	-111.084\\
1203	-129.395\\
1204	-78.125\\
1205	-50.049\\
1206	-73.242\\
1207	-57.373\\
1208	-39.063\\
1209	-56.152\\
1210	-61.035\\
1211	-46.387\\
1212	-37.842\\
1213	-50.049\\
1214	-39.063\\
1215	-53.711\\
1216	-76.904\\
1217	-72.021\\
1218	-51.27\\
1219	-51.27\\
1220	-78.125\\
1221	-123.291\\
1222	-86.67\\
1223	-56.152\\
1224	-43.945\\
1225	-48.828\\
1226	-43.945\\
1227	-34.18\\
1228	-37.842\\
1229	-47.607\\
1230	-59.814\\
1231	-64.697\\
1232	-46.387\\
1233	-74.463\\
1234	-85.449\\
1235	-83.008\\
1236	-65.918\\
1237	-73.242\\
1238	-96.436\\
1239	-61.035\\
1240	-29.297\\
1241	-42.725\\
1242	-43.945\\
1243	-58.594\\
1244	-50.049\\
1245	-40.283\\
1246	-58.594\\
1247	-54.932\\
1248	-39.063\\
1249	-48.828\\
1250	-43.945\\
1251	-39.063\\
1252	-50.049\\
1253	-29.297\\
1254	-36.621\\
1255	-26.855\\
1256	-31.738\\
1257	-53.711\\
1258	-61.035\\
1259	-76.904\\
1260	-101.318\\
1261	-67.139\\
1262	-42.725\\
1263	-32.959\\
1264	-31.738\\
1265	-47.607\\
1266	-30.518\\
1267	-36.621\\
1268	-43.945\\
1269	-65.918\\
1270	-62.256\\
1271	-86.67\\
1272	-59.814\\
1273	-56.152\\
1274	-31.738\\
1275	-32.959\\
1276	-20.752\\
1277	-24.414\\
1278	-26.855\\
1279	-47.607\\
1280	-47.607\\
1281	-56.152\\
1282	-59.814\\
1283	-93.994\\
1284	-79.346\\
1285	-62.256\\
1286	-73.242\\
1287	-79.346\\
1288	-101.318\\
1289	-80.566\\
1290	-52.49\\
1291	-30.518\\
1292	-21.973\\
1293	-23.193\\
1294	-29.297\\
1295	-29.297\\
1296	-26.855\\
1297	-36.621\\
1298	-53.711\\
1299	-47.607\\
1300	-51.27\\
1301	-57.373\\
1302	-37.842\\
1303	-20.752\\
1304	-42.725\\
1305	-63.477\\
1306	-61.035\\
1307	-69.58\\
1308	-58.594\\
1309	-45.166\\
1310	-59.814\\
1311	-84.229\\
1312	-85.449\\
1313	-59.814\\
1314	-102.539\\
1315	-76.904\\
1316	-74.463\\
1317	-84.229\\
1318	-83.008\\
1319	-63.477\\
1320	-54.932\\
1321	-90.332\\
1322	-124.512\\
1323	-95.215\\
1324	-57.373\\
1325	-53.711\\
1326	-54.932\\
1327	-72.021\\
1328	-83.008\\
1329	-59.814\\
1330	-64.697\\
1331	-83.008\\
1332	-90.332\\
1333	-61.035\\
1334	-50.049\\
1335	-63.477\\
1336	-104.98\\
1337	-92.773\\
1338	-95.215\\
1339	-58.594\\
1340	-53.711\\
1341	-51.27\\
1342	-34.18\\
1343	-23.193\\
1344	-20.752\\
1345	-17.09\\
1346	-40.283\\
1347	-54.932\\
1348	-62.256\\
1349	-47.607\\
1350	-52.49\\
1351	-57.373\\
1352	-37.842\\
1353	-39.063\\
1354	-39.063\\
1355	-28.076\\
1356	-39.063\\
1357	-57.373\\
1358	-73.242\\
1359	-48.828\\
1360	-35.4\\
1361	-31.738\\
1362	-36.621\\
1363	-26.855\\
1364	-54.932\\
1365	-93.994\\
1366	-72.021\\
1367	-97.656\\
1368	-113.525\\
1369	-114.746\\
1370	-85.449\\
1371	-63.477\\
1372	-63.477\\
1373	-63.477\\
1374	-75.684\\
1375	-89.111\\
1376	-86.67\\
1377	-115.967\\
1378	-125.732\\
1379	-150.146\\
1380	-96.436\\
1381	-109.863\\
1382	-122.07\\
1383	-92.773\\
1384	-107.422\\
1385	-92.773\\
1386	-54.932\\
1387	-39.063\\
1388	-40.283\\
1389	-57.373\\
1390	-42.725\\
1391	-32.959\\
1392	-45.166\\
1393	-34.18\\
1394	-26.855\\
1395	-34.18\\
1396	-37.842\\
1397	-25.635\\
1398	-48.828\\
1399	-64.697\\
1400	-46.387\\
1401	-81.787\\
1402	-115.967\\
1403	-106.201\\
1404	-133.057\\
1405	-92.773\\
1406	-97.656\\
1407	-69.58\\
1408	-43.945\\
1409	-53.711\\
1410	-42.725\\
1411	-34.18\\
1412	-45.166\\
1413	-26.855\\
1414	-20.752\\
1415	-25.635\\
1416	-50.049\\
1417	-61.035\\
1418	-78.125\\
1419	-74.463\\
1420	-69.58\\
1421	-80.566\\
1422	-64.697\\
1423	-53.711\\
1424	-54.932\\
1425	-43.945\\
1426	-69.58\\
1427	-50.049\\
1428	-40.283\\
1429	-58.594\\
1430	-80.566\\
1431	-59.814\\
1432	-46.387\\
1433	-41.504\\
1434	-47.607\\
1435	-32.959\\
1436	-31.738\\
1437	-43.945\\
1438	-53.711\\
1439	-59.814\\
1440	-48.828\\
1441	-61.035\\
1442	-57.373\\
1443	-34.18\\
1444	-37.842\\
1445	-65.918\\
1446	-91.553\\
1447	-90.332\\
1448	-75.684\\
1449	-73.242\\
1450	-57.373\\
1451	-53.711\\
1452	-43.945\\
1453	-61.035\\
1454	-54.932\\
1455	-36.621\\
1456	-52.49\\
1457	-61.035\\
1458	-70.801\\
1459	-76.904\\
1460	-76.904\\
1461	-102.539\\
1462	-124.512\\
1463	-140.381\\
1464	-91.553\\
1465	-52.49\\
1466	-34.18\\
1467	-25.635\\
1468	-36.621\\
1469	-32.959\\
1470	-58.594\\
1471	-53.711\\
1472	-46.387\\
1473	-62.256\\
1474	-72.021\\
1475	-81.787\\
1476	-86.67\\
1477	-122.07\\
1478	-101.318\\
1479	-81.787\\
1480	-58.594\\
1481	-58.594\\
1482	-59.814\\
1483	-75.684\\
1484	-54.932\\
1485	-56.152\\
1486	-53.711\\
1487	-30.518\\
1488	-54.932\\
1489	-70.801\\
1490	-50.049\\
1491	-53.711\\
1492	-100.098\\
1493	-75.684\\
1494	-72.021\\
1495	-102.539\\
1496	-98.877\\
1497	-62.256\\
1498	-40.283\\
1499	-42.725\\
1500	-70.801\\
};
\end{axis}

\begin{axis}[%
width=5.090835cm,
height=2.240107cm,
at={(6.698467cm,15.759893cm)},
scale only axis,
xmin=1000,
xmax=1500,
xlabel={Sample index},
ymin=-50,
ymax=0,
ylabel={C2 load, kN},
legend style={legend cell align=left,align=left,draw=white!15!black}
]
\addplot [color=mycolor1,solid,forget plot]
  table[row sep=crcr]{%
1000	-19.531\\
1001	-24.414\\
1002	-19.531\\
1003	-20.752\\
1004	-25.635\\
1005	-24.414\\
1006	-28.076\\
1007	-23.193\\
1008	-13.428\\
1009	-17.09\\
1010	-17.09\\
1011	-18.311\\
1012	-15.869\\
1013	-8.545\\
1014	-6.104\\
1015	-7.324\\
1016	-9.766\\
1017	-21.973\\
1018	-23.193\\
1019	-23.193\\
1020	-19.531\\
1021	-10.986\\
1022	-17.09\\
1023	-19.531\\
1024	-13.428\\
1025	-18.311\\
1026	-19.531\\
1027	-14.648\\
1028	-24.414\\
1029	-21.973\\
1030	-15.869\\
1031	-23.193\\
1032	-25.635\\
1033	-19.531\\
1034	-17.09\\
1035	-14.648\\
1036	-17.09\\
1037	-20.752\\
1038	-18.311\\
1039	-18.311\\
1040	-19.531\\
1041	-20.752\\
1042	-28.076\\
1043	-25.635\\
1044	-17.09\\
1045	-15.869\\
1046	-12.207\\
1047	-10.986\\
1048	-15.869\\
1049	-12.207\\
1050	-12.207\\
1051	-15.869\\
1052	-18.311\\
1053	-15.869\\
1054	-25.635\\
1055	-21.973\\
1056	-12.207\\
1057	-9.766\\
1058	-13.428\\
1059	-15.869\\
1060	-10.986\\
1061	-10.986\\
1062	-12.207\\
1063	-9.766\\
1064	-8.545\\
1065	-12.207\\
1066	-14.648\\
1067	-17.09\\
1068	-18.311\\
1069	-23.193\\
1070	-21.973\\
1071	-21.973\\
1072	-17.09\\
1073	-17.09\\
1074	-17.09\\
1075	-15.869\\
1076	-17.09\\
1077	-24.414\\
1078	-35.4\\
1079	-36.621\\
1080	-34.18\\
1081	-24.414\\
1082	-29.297\\
1083	-36.621\\
1084	-40.283\\
1085	-31.738\\
1086	-37.842\\
1087	-48.828\\
1088	-39.063\\
1089	-30.518\\
1090	-25.635\\
1091	-19.531\\
1092	-14.648\\
1093	-19.531\\
1094	-15.869\\
1095	-9.766\\
1096	-9.766\\
1097	-12.207\\
1098	-17.09\\
1099	-15.869\\
1100	-15.869\\
1101	-19.531\\
1102	-19.531\\
1103	-28.076\\
1104	-23.193\\
1105	-25.635\\
1106	-23.193\\
1107	-18.311\\
1108	-20.752\\
1109	-17.09\\
1110	-14.648\\
1111	-18.311\\
1112	-17.09\\
1113	-15.869\\
1114	-13.428\\
1115	-10.986\\
1116	-12.207\\
1117	-13.428\\
1118	-10.986\\
1119	-9.766\\
1120	-14.648\\
1121	-19.531\\
1122	-23.193\\
1123	-23.193\\
1124	-17.09\\
1125	-18.311\\
1126	-30.518\\
1127	-23.193\\
1128	-24.414\\
1129	-26.855\\
1130	-17.09\\
1131	-10.986\\
1132	-17.09\\
1133	-18.311\\
1134	-26.855\\
1135	-34.18\\
1136	-31.738\\
1137	-25.635\\
1138	-24.414\\
1139	-23.193\\
1140	-23.193\\
1141	-21.973\\
1142	-18.311\\
1143	-14.648\\
1144	-13.428\\
1145	-12.207\\
1146	-13.428\\
1147	-19.531\\
1148	-25.635\\
1149	-21.973\\
1150	-14.648\\
1151	-13.428\\
1152	-14.648\\
1153	-12.207\\
1154	-8.545\\
1155	-8.545\\
1156	-13.428\\
1157	-12.207\\
1158	-12.207\\
1159	-12.207\\
1160	-12.207\\
1161	-8.545\\
1162	-7.324\\
1163	-4.883\\
1164	-7.324\\
1165	-17.09\\
1166	-24.414\\
1167	-24.414\\
1168	-19.531\\
1169	-12.207\\
1170	-12.207\\
1171	-8.545\\
1172	-10.986\\
1173	-14.648\\
1174	-14.648\\
1175	-24.414\\
1176	-34.18\\
1177	-40.283\\
1178	-32.959\\
1179	-31.738\\
1180	-23.193\\
1181	-25.635\\
1182	-26.855\\
1183	-28.076\\
1184	-21.973\\
1185	-20.752\\
1186	-19.531\\
1187	-19.531\\
1188	-21.973\\
1189	-19.531\\
1190	-23.193\\
1191	-29.297\\
1192	-24.414\\
1193	-14.648\\
1194	-15.869\\
1195	-25.635\\
1196	-32.959\\
1197	-35.4\\
1198	-39.063\\
1199	-37.842\\
1200	-30.518\\
1201	-26.855\\
1202	-30.518\\
1203	-35.4\\
1204	-25.635\\
1205	-14.648\\
1206	-19.531\\
1207	-20.752\\
1208	-12.207\\
1209	-13.428\\
1210	-18.311\\
1211	-14.648\\
1212	-12.207\\
1213	-18.311\\
1214	-10.986\\
1215	-14.648\\
1216	-24.414\\
1217	-21.973\\
1218	-14.648\\
1219	-14.648\\
1220	-20.752\\
1221	-34.18\\
1222	-28.076\\
1223	-15.869\\
1224	-14.648\\
1225	-14.648\\
1226	-12.207\\
1227	-10.986\\
1228	-10.986\\
1229	-13.428\\
1230	-17.09\\
1231	-19.531\\
1232	-14.648\\
1233	-19.531\\
1234	-26.855\\
1235	-23.193\\
1236	-17.09\\
1237	-21.973\\
1238	-25.635\\
1239	-21.973\\
1240	-8.545\\
1241	-12.207\\
1242	-13.428\\
1243	-15.869\\
1244	-15.869\\
1245	-10.986\\
1246	-17.09\\
1247	-15.869\\
1248	-10.986\\
1249	-14.648\\
1250	-14.648\\
1251	-12.207\\
1252	-14.648\\
1253	-10.986\\
1254	-9.766\\
1255	-12.207\\
1256	-8.545\\
1257	-17.09\\
1258	-18.311\\
1259	-20.752\\
1260	-29.297\\
1261	-20.752\\
1262	-10.986\\
1263	-10.986\\
1264	-8.545\\
1265	-13.428\\
1266	-12.207\\
1267	-9.766\\
1268	-15.869\\
1269	-19.531\\
1270	-20.752\\
1271	-23.193\\
1272	-23.193\\
1273	-17.09\\
1274	-15.869\\
1275	-10.986\\
1276	-10.986\\
1277	-7.324\\
1278	-8.545\\
1279	-10.986\\
1280	-17.09\\
1281	-18.311\\
1282	-17.09\\
1283	-24.414\\
1284	-25.635\\
1285	-15.869\\
1286	-19.531\\
1287	-23.193\\
1288	-26.855\\
1289	-25.635\\
1290	-17.09\\
1291	-10.986\\
1292	-7.324\\
1293	-6.104\\
1294	-8.545\\
1295	-9.766\\
1296	-8.545\\
1297	-10.986\\
1298	-17.09\\
1299	-15.869\\
1300	-13.428\\
1301	-15.869\\
1302	-12.207\\
1303	-6.104\\
1304	-9.766\\
1305	-20.752\\
1306	-17.09\\
1307	-18.311\\
1308	-15.869\\
1309	-10.986\\
1310	-17.09\\
1311	-23.193\\
1312	-23.193\\
1313	-17.09\\
1314	-26.855\\
1315	-25.635\\
1316	-18.311\\
1317	-23.193\\
1318	-25.635\\
1319	-19.531\\
1320	-15.869\\
1321	-24.414\\
1322	-35.4\\
1323	-29.297\\
1324	-17.09\\
1325	-17.09\\
1326	-18.311\\
1327	-20.752\\
1328	-23.193\\
1329	-19.531\\
1330	-17.09\\
1331	-24.414\\
1332	-25.635\\
1333	-18.311\\
1334	-15.869\\
1335	-18.311\\
1336	-28.076\\
1337	-26.855\\
1338	-24.414\\
1339	-19.531\\
1340	-17.09\\
1341	-18.311\\
1342	-13.428\\
1343	-6.104\\
1344	-6.104\\
1345	-6.104\\
1346	-9.766\\
1347	-19.531\\
1348	-15.869\\
1349	-14.648\\
1350	-13.428\\
1351	-17.09\\
1352	-13.428\\
1353	-9.766\\
1354	-12.207\\
1355	-9.766\\
1356	-8.545\\
1357	-17.09\\
1358	-20.752\\
1359	-17.09\\
1360	-10.986\\
1361	-10.986\\
1362	-13.428\\
1363	-9.766\\
1364	-10.986\\
1365	-28.076\\
1366	-20.752\\
1367	-25.635\\
1368	-35.4\\
1369	-30.518\\
1370	-24.414\\
1371	-18.311\\
1372	-18.311\\
1373	-19.531\\
1374	-21.973\\
1375	-25.635\\
1376	-25.635\\
1377	-31.738\\
1378	-34.18\\
1379	-41.504\\
1380	-32.959\\
1381	-28.076\\
1382	-35.4\\
1383	-26.855\\
1384	-29.297\\
1385	-28.076\\
1386	-17.09\\
1387	-10.986\\
1388	-12.207\\
1389	-17.09\\
1390	-14.648\\
1391	-9.766\\
1392	-13.428\\
1393	-13.428\\
1394	-7.324\\
1395	-8.545\\
1396	-12.207\\
1397	-7.324\\
1398	-14.648\\
1399	-20.752\\
1400	-13.428\\
1401	-21.973\\
1402	-30.518\\
1403	-29.297\\
1404	-35.4\\
1405	-29.297\\
1406	-28.076\\
1407	-23.193\\
1408	-12.207\\
1409	-14.648\\
1410	-15.869\\
1411	-10.986\\
1412	-12.207\\
1413	-13.428\\
1414	-3.662\\
1415	-8.545\\
1416	-17.09\\
1417	-19.531\\
1418	-20.752\\
1419	-23.193\\
1420	-20.752\\
1421	-23.193\\
1422	-18.311\\
1423	-15.869\\
1424	-15.869\\
1425	-12.207\\
1426	-18.311\\
1427	-17.09\\
1428	-12.207\\
1429	-17.09\\
1430	-23.193\\
1431	-17.09\\
1432	-13.428\\
1433	-12.207\\
1434	-14.648\\
1435	-7.324\\
1436	-8.545\\
1437	-12.207\\
1438	-17.09\\
1439	-17.09\\
1440	-14.648\\
1441	-15.869\\
1442	-18.311\\
1443	-12.207\\
1444	-9.766\\
1445	-18.311\\
1446	-26.855\\
1447	-26.855\\
1448	-21.973\\
1449	-19.531\\
1450	-17.09\\
1451	-14.648\\
1452	-13.428\\
1453	-17.09\\
1454	-17.09\\
1455	-10.986\\
1456	-14.648\\
1457	-18.311\\
1458	-19.531\\
1459	-21.973\\
1460	-21.973\\
1461	-29.297\\
1462	-35.4\\
1463	-36.621\\
1464	-26.855\\
1465	-14.648\\
1466	-10.986\\
1467	-7.324\\
1468	-10.986\\
1469	-10.986\\
1470	-14.648\\
1471	-13.428\\
1472	-12.207\\
1473	-15.869\\
1474	-20.752\\
1475	-20.752\\
1476	-24.414\\
1477	-31.738\\
1478	-29.297\\
1479	-26.855\\
1480	-18.311\\
1481	-17.09\\
1482	-20.752\\
1483	-20.752\\
1484	-17.09\\
1485	-14.648\\
1486	-18.311\\
1487	-12.207\\
1488	-13.428\\
1489	-20.752\\
1490	-17.09\\
1491	-18.311\\
1492	-32.959\\
1493	-25.635\\
1494	-20.752\\
1495	-28.076\\
1496	-28.076\\
1497	-19.531\\
1498	-13.428\\
1499	-13.428\\
1500	-19.531\\
};
\end{axis}

\begin{axis}[%
width=5.090835cm,
height=2.240107cm,
at={(0cm,15.759893cm)},
scale only axis,
xmin=1000,
xmax=1500,
xlabel={Sample index},
ymin=-58.594,
ymax=0,
ylabel={C1 load, kN},
legend style={legend cell align=left,align=left,draw=white!15!black}
]
\addplot [color=mycolor1,solid,forget plot]
  table[row sep=crcr]{%
1000	-23.193\\
1001	-28.076\\
1002	-26.855\\
1003	-20.752\\
1004	-29.297\\
1005	-28.076\\
1006	-32.959\\
1007	-25.635\\
1008	-14.648\\
1009	-20.752\\
1010	-17.09\\
1011	-18.311\\
1012	-20.752\\
1013	-7.324\\
1014	-4.883\\
1015	-4.883\\
1016	-13.428\\
1017	-23.193\\
1018	-24.414\\
1019	-25.635\\
1020	-19.531\\
1021	-12.207\\
1022	-24.414\\
1023	-19.531\\
1024	-14.648\\
1025	-20.752\\
1026	-18.311\\
1027	-17.09\\
1028	-29.297\\
1029	-25.635\\
1030	-18.311\\
1031	-28.076\\
1032	-28.076\\
1033	-21.973\\
1034	-18.311\\
1035	-15.869\\
1036	-15.869\\
1037	-21.973\\
1038	-21.973\\
1039	-21.973\\
1040	-21.973\\
1041	-23.193\\
1042	-30.518\\
1043	-29.297\\
1044	-19.531\\
1045	-18.311\\
1046	-10.986\\
1047	-14.648\\
1048	-15.869\\
1049	-12.207\\
1050	-13.428\\
1051	-18.311\\
1052	-19.531\\
1053	-18.311\\
1054	-29.297\\
1055	-21.973\\
1056	-12.207\\
1057	-12.207\\
1058	-13.428\\
1059	-17.09\\
1060	-9.766\\
1061	-10.986\\
1062	-14.648\\
1063	-9.766\\
1064	-7.324\\
1065	-13.428\\
1066	-13.428\\
1067	-21.973\\
1068	-20.752\\
1069	-25.635\\
1070	-21.973\\
1071	-25.635\\
1072	-20.752\\
1073	-20.752\\
1074	-18.311\\
1075	-18.311\\
1076	-17.09\\
1077	-30.518\\
1078	-41.504\\
1079	-41.504\\
1080	-42.725\\
1081	-28.076\\
1082	-37.842\\
1083	-46.387\\
1084	-46.387\\
1085	-35.4\\
1086	-47.607\\
1087	-58.594\\
1088	-43.945\\
1089	-34.18\\
1090	-24.414\\
1091	-20.752\\
1092	-17.09\\
1093	-21.973\\
1094	-17.09\\
1095	-12.207\\
1096	-9.766\\
1097	-12.207\\
1098	-18.311\\
1099	-18.311\\
1100	-18.311\\
1101	-21.973\\
1102	-23.193\\
1103	-30.518\\
1104	-28.076\\
1105	-30.518\\
1106	-25.635\\
1107	-20.752\\
1108	-24.414\\
1109	-18.311\\
1110	-13.428\\
1111	-19.531\\
1112	-20.752\\
1113	-12.207\\
1114	-15.869\\
1115	-12.207\\
1116	-13.428\\
1117	-13.428\\
1118	-9.766\\
1119	-10.986\\
1120	-17.09\\
1121	-23.193\\
1122	-25.635\\
1123	-26.855\\
1124	-15.869\\
1125	-20.752\\
1126	-32.959\\
1127	-24.414\\
1128	-28.076\\
1129	-29.297\\
1130	-18.311\\
1131	-12.207\\
1132	-17.09\\
1133	-18.311\\
1134	-30.518\\
1135	-36.621\\
1136	-36.621\\
1137	-28.076\\
1138	-28.076\\
1139	-25.635\\
1140	-25.635\\
1141	-25.635\\
1142	-20.752\\
1143	-15.869\\
1144	-14.648\\
1145	-13.428\\
1146	-13.428\\
1147	-20.752\\
1148	-31.738\\
1149	-26.855\\
1150	-17.09\\
1151	-15.869\\
1152	-15.869\\
1153	-9.766\\
1154	-7.324\\
1155	-7.324\\
1156	-15.869\\
1157	-10.986\\
1158	-14.648\\
1159	-14.648\\
1160	-13.428\\
1161	-9.766\\
1162	-6.104\\
1163	-4.883\\
1164	-7.324\\
1165	-18.311\\
1166	-26.855\\
1167	-28.076\\
1168	-20.752\\
1169	-13.428\\
1170	-10.986\\
1171	-6.104\\
1172	-12.207\\
1173	-14.648\\
1174	-18.311\\
1175	-26.855\\
1176	-39.063\\
1177	-47.607\\
1178	-42.725\\
1179	-37.842\\
1180	-28.076\\
1181	-30.518\\
1182	-31.738\\
1183	-34.18\\
1184	-24.414\\
1185	-23.193\\
1186	-23.193\\
1187	-21.973\\
1188	-23.193\\
1189	-18.311\\
1190	-30.518\\
1191	-37.842\\
1192	-28.076\\
1193	-18.311\\
1194	-18.311\\
1195	-30.518\\
1196	-35.4\\
1197	-40.283\\
1198	-45.166\\
1199	-45.166\\
1200	-34.18\\
1201	-32.959\\
1202	-36.621\\
1203	-41.504\\
1204	-29.297\\
1205	-17.09\\
1206	-23.193\\
1207	-20.752\\
1208	-13.428\\
1209	-18.311\\
1210	-19.531\\
1211	-14.648\\
1212	-13.428\\
1213	-15.869\\
1214	-13.428\\
1215	-17.09\\
1216	-25.635\\
1217	-25.635\\
1218	-15.869\\
1219	-14.648\\
1220	-24.414\\
1221	-40.283\\
1222	-29.297\\
1223	-20.752\\
1224	-14.648\\
1225	-18.311\\
1226	-15.869\\
1227	-12.207\\
1228	-13.428\\
1229	-15.869\\
1230	-18.311\\
1231	-20.752\\
1232	-15.869\\
1233	-23.193\\
1234	-29.297\\
1235	-25.635\\
1236	-20.752\\
1237	-23.193\\
1238	-30.518\\
1239	-21.973\\
1240	-8.545\\
1241	-13.428\\
1242	-13.428\\
1243	-17.09\\
1244	-17.09\\
1245	-13.428\\
1246	-17.09\\
1247	-19.531\\
1248	-13.428\\
1249	-15.869\\
1250	-15.869\\
1251	-12.207\\
1252	-15.869\\
1253	-9.766\\
1254	-10.986\\
1255	-8.545\\
1256	-8.545\\
1257	-14.648\\
1258	-20.752\\
1259	-24.414\\
1260	-35.4\\
1261	-24.414\\
1262	-13.428\\
1263	-12.207\\
1264	-10.986\\
1265	-13.428\\
1266	-10.986\\
1267	-12.207\\
1268	-14.648\\
1269	-24.414\\
1270	-21.973\\
1271	-26.855\\
1272	-23.193\\
1273	-17.09\\
1274	-13.428\\
1275	-10.986\\
1276	-8.545\\
1277	-6.104\\
1278	-9.766\\
1279	-14.648\\
1280	-18.311\\
1281	-18.311\\
1282	-19.531\\
1283	-29.297\\
1284	-25.635\\
1285	-18.311\\
1286	-24.414\\
1287	-24.414\\
1288	-31.738\\
1289	-26.855\\
1290	-17.09\\
1291	-9.766\\
1292	-6.104\\
1293	-7.324\\
1294	-9.766\\
1295	-8.545\\
1296	-8.545\\
1297	-12.207\\
1298	-17.09\\
1299	-15.869\\
1300	-15.869\\
1301	-19.531\\
1302	-13.428\\
1303	-4.883\\
1304	-9.766\\
1305	-21.973\\
1306	-19.531\\
1307	-21.973\\
1308	-18.311\\
1309	-14.648\\
1310	-19.531\\
1311	-25.635\\
1312	-25.635\\
1313	-19.531\\
1314	-26.855\\
1315	-26.855\\
1316	-23.193\\
1317	-28.076\\
1318	-26.855\\
1319	-20.752\\
1320	-19.531\\
1321	-29.297\\
1322	-40.283\\
1323	-30.518\\
1324	-18.311\\
1325	-17.09\\
1326	-18.311\\
1327	-21.973\\
1328	-26.855\\
1329	-19.531\\
1330	-19.531\\
1331	-26.855\\
1332	-28.076\\
1333	-20.752\\
1334	-14.648\\
1335	-20.752\\
1336	-34.18\\
1337	-30.518\\
1338	-31.738\\
1339	-21.973\\
1340	-17.09\\
1341	-17.09\\
1342	-10.986\\
1343	-6.104\\
1344	-6.104\\
1345	-4.883\\
1346	-10.986\\
1347	-19.531\\
1348	-18.311\\
1349	-15.869\\
1350	-14.648\\
1351	-20.752\\
1352	-14.648\\
1353	-9.766\\
1354	-13.428\\
1355	-8.545\\
1356	-10.986\\
1357	-18.311\\
1358	-23.193\\
1359	-17.09\\
1360	-9.766\\
1361	-10.986\\
1362	-12.207\\
1363	-9.766\\
1364	-13.428\\
1365	-31.738\\
1366	-25.635\\
1367	-31.738\\
1368	-37.842\\
1369	-36.621\\
1370	-26.855\\
1371	-20.752\\
1372	-18.311\\
1373	-21.973\\
1374	-23.193\\
1375	-29.297\\
1376	-29.297\\
1377	-36.621\\
1378	-42.725\\
1379	-46.387\\
1380	-37.842\\
1381	-35.4\\
1382	-40.283\\
1383	-31.738\\
1384	-34.18\\
1385	-31.738\\
1386	-18.311\\
1387	-12.207\\
1388	-13.428\\
1389	-18.311\\
1390	-17.09\\
1391	-8.545\\
1392	-14.648\\
1393	-13.428\\
1394	-6.104\\
1395	-10.986\\
1396	-12.207\\
1397	-7.324\\
1398	-18.311\\
1399	-20.752\\
1400	-14.648\\
1401	-23.193\\
1402	-37.842\\
1403	-32.959\\
1404	-37.842\\
1405	-32.959\\
1406	-29.297\\
1407	-20.752\\
1408	-14.648\\
1409	-17.09\\
1410	-13.428\\
1411	-10.986\\
1412	-14.648\\
1413	-13.428\\
1414	-3.662\\
1415	-7.324\\
1416	-15.869\\
1417	-20.752\\
1418	-21.973\\
1419	-23.193\\
1420	-21.973\\
1421	-26.855\\
1422	-21.973\\
1423	-17.09\\
1424	-17.09\\
1425	-14.648\\
1426	-20.752\\
1427	-18.311\\
1428	-13.428\\
1429	-18.311\\
1430	-26.855\\
1431	-20.752\\
1432	-15.869\\
1433	-13.428\\
1434	-14.648\\
1435	-12.207\\
1436	-8.545\\
1437	-14.648\\
1438	-19.531\\
1439	-18.311\\
1440	-15.869\\
1441	-19.531\\
1442	-18.311\\
1443	-10.986\\
1444	-10.986\\
1445	-21.973\\
1446	-30.518\\
1447	-29.297\\
1448	-23.193\\
1449	-21.973\\
1450	-19.531\\
1451	-18.311\\
1452	-15.869\\
1453	-19.531\\
1454	-17.09\\
1455	-13.428\\
1456	-17.09\\
1457	-19.531\\
1458	-21.973\\
1459	-25.635\\
1460	-25.635\\
1461	-32.959\\
1462	-41.504\\
1463	-45.166\\
1464	-31.738\\
1465	-17.09\\
1466	-13.428\\
1467	-8.545\\
1468	-10.986\\
1469	-13.428\\
1470	-17.09\\
1471	-18.311\\
1472	-13.428\\
1473	-19.531\\
1474	-24.414\\
1475	-25.635\\
1476	-26.855\\
1477	-39.063\\
1478	-35.4\\
1479	-25.635\\
1480	-20.752\\
1481	-20.752\\
1482	-19.531\\
1483	-23.193\\
1484	-19.531\\
1485	-15.869\\
1486	-17.09\\
1487	-10.986\\
1488	-14.648\\
1489	-28.076\\
1490	-19.531\\
1491	-15.869\\
1492	-34.18\\
1493	-28.076\\
1494	-21.973\\
1495	-34.18\\
1496	-31.738\\
1497	-20.752\\
1498	-15.869\\
1499	-13.428\\
1500	-23.193\\
};
\end{axis}

\begin{axis}[%
width=5.090835cm,
height=2.240107cm,
at={(0cm,3.939973cm)},
scale only axis,
xmin=1000,
xmax=1500,
xlabel={Sample index},
ymin=-400,
ymax=0,
ylabel={C7 load, kN},
legend style={legend cell align=left,align=left,draw=white!15!black}
]
\addplot [color=mycolor1,solid,forget plot]
  table[row sep=crcr]{%
1000	-112.305\\
1001	-144.043\\
1002	-117.188\\
1003	-111.084\\
1004	-156.25\\
1005	-147.705\\
1006	-175.781\\
1007	-134.277\\
1008	-68.359\\
1009	-80.566\\
1010	-80.566\\
1011	-100.098\\
1012	-81.787\\
1013	-34.18\\
1014	-24.414\\
1015	-23.193\\
1016	-73.242\\
1017	-128.174\\
1018	-133.057\\
1019	-137.939\\
1020	-92.773\\
1021	-56.152\\
1022	-125.732\\
1023	-86.67\\
1024	-68.359\\
1025	-114.746\\
1026	-100.098\\
1027	-85.449\\
1028	-142.822\\
1029	-133.057\\
1030	-102.539\\
1031	-150.146\\
1032	-145.264\\
1033	-113.525\\
1034	-92.773\\
1035	-72.021\\
1036	-86.67\\
1037	-112.305\\
1038	-103.76\\
1039	-109.863\\
1040	-112.305\\
1041	-122.07\\
1042	-175.781\\
1043	-155.029\\
1044	-102.539\\
1045	-92.773\\
1046	-53.711\\
1047	-58.594\\
1048	-79.346\\
1049	-59.814\\
1050	-62.256\\
1051	-89.111\\
1052	-102.539\\
1053	-95.215\\
1054	-151.367\\
1055	-111.084\\
1056	-64.697\\
1057	-51.27\\
1058	-69.58\\
1059	-81.787\\
1060	-43.945\\
1061	-45.166\\
1062	-67.139\\
1063	-51.27\\
1064	-42.725\\
1065	-61.035\\
1066	-70.801\\
1067	-109.863\\
1068	-104.98\\
1069	-129.395\\
1070	-120.85\\
1071	-137.939\\
1072	-109.863\\
1073	-107.422\\
1074	-92.773\\
1075	-91.553\\
1076	-87.891\\
1077	-157.471\\
1078	-224.609\\
1079	-231.934\\
1080	-225.83\\
1081	-140.381\\
1082	-205.078\\
1083	-252.686\\
1084	-261.23\\
1085	-200.195\\
1086	-270.996\\
1087	-335.693\\
1088	-235.596\\
1089	-191.65\\
1090	-128.174\\
1091	-98.877\\
1092	-84.229\\
1093	-104.98\\
1094	-74.463\\
1095	-50.049\\
1096	-48.828\\
1097	-63.477\\
1098	-95.215\\
1099	-86.67\\
1100	-89.111\\
1101	-119.629\\
1102	-123.291\\
1103	-167.236\\
1104	-134.277\\
1105	-169.678\\
1106	-122.07\\
1107	-108.643\\
1108	-125.732\\
1109	-89.111\\
1110	-80.566\\
1111	-97.656\\
1112	-98.877\\
1113	-68.359\\
1114	-73.242\\
1115	-54.932\\
1116	-61.035\\
1117	-64.697\\
1118	-45.166\\
1119	-58.594\\
1120	-72.021\\
1121	-117.188\\
1122	-122.07\\
1123	-136.719\\
1124	-76.904\\
1125	-109.863\\
1126	-173.34\\
1127	-131.836\\
1128	-157.471\\
1129	-157.471\\
1130	-85.449\\
1131	-59.814\\
1132	-85.449\\
1133	-98.877\\
1134	-161.133\\
1135	-202.637\\
1136	-198.975\\
1137	-145.264\\
1138	-150.146\\
1139	-135.498\\
1140	-131.836\\
1141	-126.953\\
1142	-85.449\\
1143	-76.904\\
1144	-69.58\\
1145	-62.256\\
1146	-70.801\\
1147	-107.422\\
1148	-164.795\\
1149	-125.732\\
1150	-86.67\\
1151	-73.242\\
1152	-70.801\\
1153	-42.725\\
1154	-31.738\\
1155	-39.063\\
1156	-72.021\\
1157	-57.373\\
1158	-59.814\\
1159	-67.139\\
1160	-63.477\\
1161	-43.945\\
1162	-29.297\\
1163	-25.635\\
1164	-43.945\\
1165	-96.436\\
1166	-140.381\\
1167	-155.029\\
1168	-101.318\\
1169	-69.58\\
1170	-50.049\\
1171	-34.18\\
1172	-83.008\\
1173	-81.787\\
1174	-119.629\\
1175	-145.264\\
1176	-230.713\\
1177	-262.451\\
1178	-211.182\\
1179	-202.637\\
1180	-136.719\\
1181	-151.367\\
1182	-159.912\\
1183	-172.119\\
1184	-129.395\\
1185	-118.408\\
1186	-115.967\\
1187	-103.76\\
1188	-123.291\\
1189	-87.891\\
1190	-153.809\\
1191	-189.209\\
1192	-133.057\\
1193	-78.125\\
1194	-85.449\\
1195	-146.484\\
1196	-181.885\\
1197	-229.492\\
1198	-241.699\\
1199	-240.479\\
1200	-180.664\\
1201	-163.574\\
1202	-187.988\\
1203	-222.168\\
1204	-139.16\\
1205	-81.787\\
1206	-122.07\\
1207	-102.539\\
1208	-65.918\\
1209	-87.891\\
1210	-98.877\\
1211	-76.904\\
1212	-58.594\\
1213	-80.566\\
1214	-65.918\\
1215	-91.553\\
1216	-133.057\\
1217	-122.07\\
1218	-79.346\\
1219	-83.008\\
1220	-126.953\\
1221	-209.961\\
1222	-150.146\\
1223	-92.773\\
1224	-69.58\\
1225	-83.008\\
1226	-74.463\\
1227	-56.152\\
1228	-62.256\\
1229	-74.463\\
1230	-96.436\\
1231	-107.422\\
1232	-72.021\\
1233	-122.07\\
1234	-144.043\\
1235	-137.939\\
1236	-106.201\\
1237	-118.408\\
1238	-158.691\\
1239	-101.318\\
1240	-41.504\\
1241	-58.594\\
1242	-65.918\\
1243	-91.553\\
1244	-81.787\\
1245	-59.814\\
1246	-96.436\\
1247	-91.553\\
1248	-65.918\\
1249	-83.008\\
1250	-76.904\\
1251	-67.139\\
1252	-79.346\\
1253	-46.387\\
1254	-53.711\\
1255	-40.283\\
1256	-43.945\\
1257	-86.67\\
1258	-100.098\\
1259	-137.939\\
1260	-177.002\\
1261	-115.967\\
1262	-68.359\\
1263	-48.828\\
1264	-48.828\\
1265	-73.242\\
1266	-46.387\\
1267	-52.49\\
1268	-70.801\\
1269	-119.629\\
1270	-107.422\\
1271	-150.146\\
1272	-102.539\\
1273	-91.553\\
1274	-47.607\\
1275	-47.607\\
1276	-31.738\\
1277	-34.18\\
1278	-41.504\\
1279	-75.684\\
1280	-83.008\\
1281	-92.773\\
1282	-97.656\\
1283	-152.588\\
1284	-130.615\\
1285	-97.656\\
1286	-119.629\\
1287	-129.395\\
1288	-167.236\\
1289	-133.057\\
1290	-84.229\\
1291	-43.945\\
1292	-31.738\\
1293	-34.18\\
1294	-41.504\\
1295	-43.945\\
1296	-40.283\\
1297	-56.152\\
1298	-87.891\\
1299	-79.346\\
1300	-81.787\\
1301	-96.436\\
1302	-58.594\\
1303	-29.297\\
1304	-64.697\\
1305	-97.656\\
1306	-96.436\\
1307	-109.863\\
1308	-93.994\\
1309	-69.58\\
1310	-97.656\\
1311	-137.939\\
1312	-139.16\\
1313	-97.656\\
1314	-162.354\\
1315	-128.174\\
1316	-118.408\\
1317	-137.939\\
1318	-137.939\\
1319	-103.76\\
1320	-90.332\\
1321	-153.809\\
1322	-214.844\\
1323	-164.795\\
1324	-92.773\\
1325	-92.773\\
1326	-91.553\\
1327	-113.525\\
1328	-136.719\\
1329	-100.098\\
1330	-104.98\\
1331	-140.381\\
1332	-153.809\\
1333	-103.76\\
1334	-79.346\\
1335	-104.98\\
1336	-175.781\\
1337	-159.912\\
1338	-162.354\\
1339	-95.215\\
1340	-84.229\\
1341	-83.008\\
1342	-52.49\\
1343	-34.18\\
1344	-28.076\\
1345	-23.193\\
1346	-59.814\\
1347	-86.67\\
1348	-104.98\\
1349	-76.904\\
1350	-87.891\\
1351	-97.656\\
1352	-59.814\\
1353	-63.477\\
1354	-61.035\\
1355	-45.166\\
1356	-57.373\\
1357	-97.656\\
1358	-124.512\\
1359	-78.125\\
1360	-56.152\\
1361	-46.387\\
1362	-59.814\\
1363	-41.504\\
1364	-91.553\\
1365	-158.691\\
1366	-122.07\\
1367	-167.236\\
1368	-194.092\\
1369	-197.754\\
1370	-140.381\\
1371	-106.201\\
1372	-106.201\\
1373	-104.98\\
1374	-120.85\\
1375	-150.146\\
1376	-147.705\\
1377	-200.195\\
1378	-220.947\\
1379	-263.672\\
1380	-178.223\\
1381	-190.43\\
1382	-212.402\\
1383	-163.574\\
1384	-189.209\\
1385	-163.574\\
1386	-91.553\\
1387	-58.594\\
1388	-62.256\\
1389	-95.215\\
1390	-75.684\\
1391	-51.27\\
1392	-72.021\\
1393	-52.49\\
1394	-40.283\\
1395	-48.828\\
1396	-56.152\\
1397	-40.283\\
1398	-76.904\\
1399	-109.863\\
1400	-78.125\\
1401	-134.277\\
1402	-195.313\\
1403	-178.223\\
1404	-222.168\\
1405	-150.146\\
1406	-161.133\\
1407	-111.084\\
1408	-67.139\\
1409	-85.449\\
1410	-67.139\\
1411	-48.828\\
1412	-72.021\\
1413	-41.504\\
1414	-25.635\\
1415	-36.621\\
1416	-79.346\\
1417	-104.98\\
1418	-128.174\\
1419	-124.512\\
1420	-117.188\\
1421	-136.719\\
1422	-111.084\\
1423	-90.332\\
1424	-90.332\\
1425	-73.242\\
1426	-115.967\\
1427	-78.125\\
1428	-67.139\\
1429	-100.098\\
1430	-137.939\\
1431	-103.76\\
1432	-76.904\\
1433	-64.697\\
1434	-74.463\\
1435	-51.27\\
1436	-50.049\\
1437	-72.021\\
1438	-86.67\\
1439	-98.877\\
1440	-79.346\\
1441	-101.318\\
1442	-92.773\\
1443	-52.49\\
1444	-54.932\\
1445	-112.305\\
1446	-158.691\\
1447	-153.809\\
1448	-129.395\\
1449	-122.07\\
1450	-92.773\\
1451	-89.111\\
1452	-70.801\\
1453	-102.539\\
1454	-90.332\\
1455	-54.932\\
1456	-84.229\\
1457	-101.318\\
1458	-118.408\\
1459	-129.395\\
1460	-129.395\\
1461	-175.781\\
1462	-216.064\\
1463	-244.141\\
1464	-153.809\\
1465	-85.449\\
1466	-54.932\\
1467	-37.842\\
1468	-57.373\\
1469	-53.711\\
1470	-95.215\\
1471	-85.449\\
1472	-75.684\\
1473	-102.539\\
1474	-118.408\\
1475	-141.602\\
1476	-148.926\\
1477	-208.74\\
1478	-181.885\\
1479	-136.719\\
1480	-90.332\\
1481	-92.773\\
1482	-95.215\\
1483	-123.291\\
1484	-87.891\\
1485	-89.111\\
1486	-87.891\\
1487	-47.607\\
1488	-81.787\\
1489	-128.174\\
1490	-90.332\\
1491	-96.436\\
1492	-170.898\\
1493	-129.395\\
1494	-122.07\\
1495	-177.002\\
1496	-166.016\\
1497	-102.539\\
1498	-64.697\\
1499	-65.918\\
1500	-114.746\\
};
\end{axis}

\begin{axis}[%
width=5.090835cm,
height=2.240107cm,
at={(6.698467cm,7.879947cm)},
scale only axis,
xmin=1000,
xmax=1500,
xlabel={Sample index},
ymin=-404.053,
ymax=0,
ylabel={C6 load, kN},
legend style={legend cell align=left,align=left,draw=white!15!black}
]
\addplot [color=mycolor1,solid,forget plot]
  table[row sep=crcr]{%
1000	-140.381\\
1001	-178.223\\
1002	-147.705\\
1003	-139.16\\
1004	-192.871\\
1005	-185.547\\
1006	-219.727\\
1007	-169.678\\
1008	-86.67\\
1009	-107.422\\
1010	-102.539\\
1011	-126.953\\
1012	-103.76\\
1013	-41.504\\
1014	-31.738\\
1015	-28.076\\
1016	-89.111\\
1017	-157.471\\
1018	-167.236\\
1019	-174.561\\
1020	-124.512\\
1021	-74.463\\
1022	-158.691\\
1023	-111.084\\
1024	-83.008\\
1025	-142.822\\
1026	-123.291\\
1027	-103.76\\
1028	-173.34\\
1029	-164.795\\
1030	-124.512\\
1031	-180.664\\
1032	-179.443\\
1033	-142.822\\
1034	-117.188\\
1035	-89.111\\
1036	-108.643\\
1037	-137.939\\
1038	-130.615\\
1039	-136.719\\
1040	-140.381\\
1041	-151.367\\
1042	-213.623\\
1043	-191.65\\
1044	-126.953\\
1045	-115.967\\
1046	-64.697\\
1047	-69.58\\
1048	-96.436\\
1049	-74.463\\
1050	-78.125\\
1051	-111.084\\
1052	-128.174\\
1053	-119.629\\
1054	-190.43\\
1055	-139.16\\
1056	-81.787\\
1057	-63.477\\
1058	-85.449\\
1059	-101.318\\
1060	-51.27\\
1061	-58.594\\
1062	-81.787\\
1063	-65.918\\
1064	-56.152\\
1065	-74.463\\
1066	-86.67\\
1067	-139.16\\
1068	-130.615\\
1069	-163.574\\
1070	-146.484\\
1071	-170.898\\
1072	-139.16\\
1073	-133.057\\
1074	-115.967\\
1075	-111.084\\
1076	-111.084\\
1077	-196.533\\
1078	-280.762\\
1079	-291.748\\
1080	-280.762\\
1081	-175.781\\
1082	-240.479\\
1083	-303.955\\
1084	-317.383\\
1085	-239.258\\
1086	-313.721\\
1087	-404.053\\
1088	-289.307\\
1089	-238.037\\
1090	-158.691\\
1091	-122.07\\
1092	-102.539\\
1093	-129.395\\
1094	-92.773\\
1095	-61.035\\
1096	-59.814\\
1097	-75.684\\
1098	-118.408\\
1099	-107.422\\
1100	-111.084\\
1101	-146.484\\
1102	-152.588\\
1103	-207.52\\
1104	-167.236\\
1105	-208.74\\
1106	-148.926\\
1107	-130.615\\
1108	-151.367\\
1109	-108.643\\
1110	-98.877\\
1111	-122.07\\
1112	-123.291\\
1113	-85.449\\
1114	-95.215\\
1115	-67.139\\
1116	-76.904\\
1117	-81.787\\
1118	-57.373\\
1119	-75.684\\
1120	-93.994\\
1121	-151.367\\
1122	-152.588\\
1123	-170.898\\
1124	-97.656\\
1125	-140.381\\
1126	-211.182\\
1127	-161.133\\
1128	-186.768\\
1129	-191.65\\
1130	-112.305\\
1131	-75.684\\
1132	-107.422\\
1133	-123.291\\
1134	-205.078\\
1135	-252.686\\
1136	-252.686\\
1137	-184.326\\
1138	-189.209\\
1139	-167.236\\
1140	-163.574\\
1141	-162.354\\
1142	-108.643\\
1143	-96.436\\
1144	-86.67\\
1145	-78.125\\
1146	-87.891\\
1147	-133.057\\
1148	-203.857\\
1149	-161.133\\
1150	-109.863\\
1151	-92.773\\
1152	-89.111\\
1153	-52.49\\
1154	-41.504\\
1155	-47.607\\
1156	-90.332\\
1157	-67.139\\
1158	-78.125\\
1159	-85.449\\
1160	-80.566\\
1161	-54.932\\
1162	-37.842\\
1163	-30.518\\
1164	-54.932\\
1165	-119.629\\
1166	-172.119\\
1167	-194.092\\
1168	-136.719\\
1169	-91.553\\
1170	-64.697\\
1171	-43.945\\
1172	-98.877\\
1173	-91.553\\
1174	-137.939\\
1175	-168.457\\
1176	-275.879\\
1177	-316.162\\
1178	-260.01\\
1179	-249.023\\
1180	-159.912\\
1181	-185.547\\
1182	-195.313\\
1183	-212.402\\
1184	-158.691\\
1185	-144.043\\
1186	-142.822\\
1187	-129.395\\
1188	-156.25\\
1189	-109.863\\
1190	-195.313\\
1191	-234.375\\
1192	-175.781\\
1193	-104.98\\
1194	-111.084\\
1195	-192.871\\
1196	-234.375\\
1197	-289.307\\
1198	-303.955\\
1199	-302.734\\
1200	-228.271\\
1201	-205.078\\
1202	-235.596\\
1203	-275.879\\
1204	-170.898\\
1205	-104.98\\
1206	-152.588\\
1207	-122.07\\
1208	-79.346\\
1209	-109.863\\
1210	-122.07\\
1211	-95.215\\
1212	-75.684\\
1213	-101.318\\
1214	-80.566\\
1215	-115.967\\
1216	-162.354\\
1217	-147.705\\
1218	-100.098\\
1219	-101.318\\
1220	-157.471\\
1221	-261.23\\
1222	-185.547\\
1223	-118.408\\
1224	-85.449\\
1225	-101.318\\
1226	-90.332\\
1227	-67.139\\
1228	-76.904\\
1229	-92.773\\
1230	-120.85\\
1231	-134.277\\
1232	-89.111\\
1233	-156.25\\
1234	-179.443\\
1235	-170.898\\
1236	-140.381\\
1237	-150.146\\
1238	-202.637\\
1239	-125.732\\
1240	-53.711\\
1241	-78.125\\
1242	-85.449\\
1243	-119.629\\
1244	-101.318\\
1245	-74.463\\
1246	-118.408\\
1247	-112.305\\
1248	-79.346\\
1249	-102.539\\
1250	-90.332\\
1251	-80.566\\
1252	-95.215\\
1253	-54.932\\
1254	-68.359\\
1255	-47.607\\
1256	-51.27\\
1257	-106.201\\
1258	-122.07\\
1259	-167.236\\
1260	-219.727\\
1261	-142.822\\
1262	-83.008\\
1263	-63.477\\
1264	-62.256\\
1265	-91.553\\
1266	-54.932\\
1267	-70.801\\
1268	-85.449\\
1269	-150.146\\
1270	-133.057\\
1271	-190.43\\
1272	-124.512\\
1273	-117.188\\
1274	-56.152\\
1275	-62.256\\
1276	-34.18\\
1277	-46.387\\
1278	-51.27\\
1279	-95.215\\
1280	-100.098\\
1281	-117.188\\
1282	-123.291\\
1283	-197.754\\
1284	-170.898\\
1285	-128.174\\
1286	-153.809\\
1287	-163.574\\
1288	-213.623\\
1289	-161.133\\
1290	-104.98\\
1291	-53.711\\
1292	-40.283\\
1293	-41.504\\
1294	-52.49\\
1295	-54.932\\
1296	-51.27\\
1297	-70.801\\
1298	-108.643\\
1299	-98.877\\
1300	-103.76\\
1301	-118.408\\
1302	-69.58\\
1303	-36.621\\
1304	-81.787\\
1305	-125.732\\
1306	-125.732\\
1307	-142.822\\
1308	-118.408\\
1309	-87.891\\
1310	-123.291\\
1311	-172.119\\
1312	-172.119\\
1313	-120.85\\
1314	-207.52\\
1315	-155.029\\
1316	-151.367\\
1317	-173.34\\
1318	-172.119\\
1319	-130.615\\
1320	-112.305\\
1321	-192.871\\
1322	-266.113\\
1323	-202.637\\
1324	-124.512\\
1325	-118.408\\
1326	-115.967\\
1327	-150.146\\
1328	-173.34\\
1329	-125.732\\
1330	-134.277\\
1331	-177.002\\
1332	-192.871\\
1333	-120.85\\
1334	-97.656\\
1335	-130.615\\
1336	-218.506\\
1337	-195.313\\
1338	-203.857\\
1339	-115.967\\
1340	-108.643\\
1341	-101.318\\
1342	-63.477\\
1343	-41.504\\
1344	-34.18\\
1345	-29.297\\
1346	-78.125\\
1347	-108.643\\
1348	-131.836\\
1349	-93.994\\
1350	-107.422\\
1351	-119.629\\
1352	-72.021\\
1353	-79.346\\
1354	-73.242\\
1355	-54.932\\
1356	-75.684\\
1357	-120.85\\
1358	-156.25\\
1359	-95.215\\
1360	-73.242\\
1361	-58.594\\
1362	-74.463\\
1363	-45.166\\
1364	-117.188\\
1365	-195.313\\
1366	-163.574\\
1367	-216.064\\
1368	-241.699\\
1369	-249.023\\
1370	-181.885\\
1371	-131.836\\
1372	-130.615\\
1373	-129.395\\
1374	-155.029\\
1375	-184.326\\
1376	-181.885\\
1377	-247.803\\
1378	-270.996\\
1379	-328.369\\
1380	-217.285\\
1381	-238.037\\
1382	-264.893\\
1383	-205.078\\
1384	-239.258\\
1385	-200.195\\
1386	-113.525\\
1387	-74.463\\
1388	-79.346\\
1389	-117.188\\
1390	-97.656\\
1391	-64.697\\
1392	-90.332\\
1393	-63.477\\
1394	-48.828\\
1395	-64.697\\
1396	-72.021\\
1397	-48.828\\
1398	-100.098\\
1399	-135.498\\
1400	-97.656\\
1401	-173.34\\
1402	-241.699\\
1403	-220.947\\
1404	-279.541\\
1405	-187.988\\
1406	-206.299\\
1407	-134.277\\
1408	-85.449\\
1409	-108.643\\
1410	-87.891\\
1411	-63.477\\
1412	-92.773\\
1413	-50.049\\
1414	-32.959\\
1415	-43.945\\
1416	-98.877\\
1417	-125.732\\
1418	-157.471\\
1419	-152.588\\
1420	-146.484\\
1421	-168.457\\
1422	-135.498\\
1423	-108.643\\
1424	-112.305\\
1425	-89.111\\
1426	-146.484\\
1427	-96.436\\
1428	-80.566\\
1429	-120.85\\
1430	-164.795\\
1431	-123.291\\
1432	-93.994\\
1433	-80.566\\
1434	-91.553\\
1435	-61.035\\
1436	-61.035\\
1437	-87.891\\
1438	-109.863\\
1439	-124.512\\
1440	-96.436\\
1441	-128.174\\
1442	-114.746\\
1443	-61.035\\
1444	-69.58\\
1445	-135.498\\
1446	-190.43\\
1447	-186.768\\
1448	-157.471\\
1449	-152.588\\
1450	-114.746\\
1451	-109.863\\
1452	-87.891\\
1453	-125.732\\
1454	-117.188\\
1455	-70.801\\
1456	-107.422\\
1457	-126.953\\
1458	-150.146\\
1459	-164.795\\
1460	-163.574\\
1461	-222.168\\
1462	-267.334\\
1463	-302.734\\
1464	-190.43\\
1465	-106.201\\
1466	-68.359\\
1467	-45.166\\
1468	-70.801\\
1469	-64.697\\
1470	-119.629\\
1471	-102.539\\
1472	-93.994\\
1473	-124.512\\
1474	-145.264\\
1475	-172.119\\
1476	-183.105\\
1477	-258.789\\
1478	-231.934\\
1479	-177.002\\
1480	-120.85\\
1481	-120.85\\
1482	-123.291\\
1483	-157.471\\
1484	-108.643\\
1485	-113.525\\
1486	-107.422\\
1487	-58.594\\
1488	-102.539\\
1489	-152.588\\
1490	-107.422\\
1491	-115.967\\
1492	-216.064\\
1493	-159.912\\
1494	-156.25\\
1495	-219.727\\
1496	-206.299\\
1497	-129.395\\
1498	-85.449\\
1499	-85.449\\
1500	-146.484\\
};
\end{axis}

\begin{axis}[%
width=5.090835cm,
height=2.240107cm,
at={(0cm,7.879947cm)},
scale only axis,
xmin=1000,
xmax=1500,
xlabel={Sample index},
ymin=-28.076,
ymax=0,
ylabel={C5 load, kN},
legend style={legend cell align=left,align=left,draw=white!15!black}
]
\addplot [color=mycolor1,solid,forget plot]
  table[row sep=crcr]{%
1000	-10.986\\
1001	-15.869\\
1002	-10.986\\
1003	-12.207\\
1004	-13.428\\
1005	-15.869\\
1006	-14.648\\
1007	-14.648\\
1008	-7.324\\
1009	-8.545\\
1010	-10.986\\
1011	-9.766\\
1012	-9.766\\
1013	-6.104\\
1014	-2.441\\
1015	-3.662\\
1016	-2.441\\
1017	-12.207\\
1018	-13.428\\
1019	-12.207\\
1020	-10.986\\
1021	-6.104\\
1022	-6.104\\
1023	-1.221\\
1024	-6.104\\
1025	-10.986\\
1026	-12.207\\
1027	-8.545\\
1028	-14.648\\
1029	-14.648\\
1030	-9.766\\
1031	-13.428\\
1032	-12.207\\
1033	-9.766\\
1034	-9.766\\
1035	-8.545\\
1036	-8.545\\
1037	-12.207\\
1038	-10.986\\
1039	-10.986\\
1040	-10.986\\
1041	-10.986\\
1042	-14.648\\
1043	-15.869\\
1044	-9.766\\
1045	-8.545\\
1046	-8.545\\
1047	-6.104\\
1048	-7.324\\
1049	-7.324\\
1050	-4.883\\
1051	-9.766\\
1052	-10.986\\
1053	-8.545\\
1054	-13.428\\
1055	-15.869\\
1056	-8.545\\
1057	-6.104\\
1058	-6.104\\
1059	-9.766\\
1060	-6.104\\
1061	-4.883\\
1062	-7.324\\
1063	-7.324\\
1064	-3.662\\
1065	-6.104\\
1066	-8.545\\
1067	-10.986\\
1068	-10.986\\
1069	-9.766\\
1070	-13.428\\
1071	-10.986\\
1072	-12.207\\
1073	-10.986\\
1074	-10.986\\
1075	-7.324\\
1076	-8.545\\
1077	-14.648\\
1078	-20.752\\
1079	-20.752\\
1080	-19.531\\
1081	-13.428\\
1082	-15.869\\
1083	-24.414\\
1084	-23.193\\
1085	-15.869\\
1086	-21.973\\
1087	-28.076\\
1088	-23.193\\
1089	-17.09\\
1090	-14.648\\
1091	-13.428\\
1092	-9.766\\
1093	-10.986\\
1094	-12.207\\
1095	-4.883\\
1096	-4.883\\
1097	-7.324\\
1098	-10.986\\
1099	-9.766\\
1100	-9.766\\
1101	-10.986\\
1102	-13.428\\
1103	-14.648\\
1104	-15.869\\
1105	-14.648\\
1106	-17.09\\
1107	-12.207\\
1108	-12.207\\
1109	-10.986\\
1110	-8.545\\
1111	-9.766\\
1112	-12.207\\
1113	-8.545\\
1114	-9.766\\
1115	-7.324\\
1116	-6.104\\
1117	-8.545\\
1118	-6.104\\
1119	-4.883\\
1120	-8.545\\
1121	-13.428\\
1122	-12.207\\
1123	-10.986\\
1124	-7.324\\
1125	-8.545\\
1126	-19.531\\
1127	-14.648\\
1128	-10.986\\
1129	-17.09\\
1130	-12.207\\
1131	-6.104\\
1132	-9.766\\
1133	-8.545\\
1134	-14.648\\
1135	-15.869\\
1136	-19.531\\
1137	-14.648\\
1138	-14.648\\
1139	-13.428\\
1140	-13.428\\
1141	-14.648\\
1142	-10.986\\
1143	-7.324\\
1144	-7.324\\
1145	-6.104\\
1146	-8.545\\
1147	-10.986\\
1148	-15.869\\
1149	-14.648\\
1150	-8.545\\
1151	-8.545\\
1152	-9.766\\
1153	-6.104\\
1154	-2.441\\
1155	-4.883\\
1156	-7.324\\
1157	-8.545\\
1158	-6.104\\
1159	-7.324\\
1160	-6.104\\
1161	-4.883\\
1162	-4.883\\
1163	-1.221\\
1164	-3.662\\
1165	-12.207\\
1166	-14.648\\
1167	-15.869\\
1168	-12.207\\
1169	-7.324\\
1170	-6.104\\
1171	-3.662\\
1172	-7.324\\
1173	-8.545\\
1174	-9.766\\
1175	-12.207\\
1176	-18.311\\
1177	-18.311\\
1178	-19.531\\
1179	-18.311\\
1180	-15.869\\
1181	-15.869\\
1182	-17.09\\
1183	-18.311\\
1184	-12.207\\
1185	-10.986\\
1186	-12.207\\
1187	-10.986\\
1188	-12.207\\
1189	-9.766\\
1190	-15.869\\
1191	-18.311\\
1192	-12.207\\
1193	-7.324\\
1194	-9.766\\
1195	-17.09\\
1196	-18.311\\
1197	-18.311\\
1198	-21.973\\
1199	-20.752\\
1200	-17.09\\
1201	-15.869\\
1202	-18.311\\
1203	-20.752\\
1204	-17.09\\
1205	-8.545\\
1206	-14.648\\
1207	-12.207\\
1208	-7.324\\
1209	-10.986\\
1210	-9.766\\
1211	-8.545\\
1212	-7.324\\
1213	-8.545\\
1214	-8.545\\
1215	-9.766\\
1216	-13.428\\
1217	-14.648\\
1218	-7.324\\
1219	-8.545\\
1220	-12.207\\
1221	-17.09\\
1222	-15.869\\
1223	-8.545\\
1224	-8.545\\
1225	-9.766\\
1226	-8.545\\
1227	-7.324\\
1228	-6.104\\
1229	-8.545\\
1230	-9.766\\
1231	-9.766\\
1232	-9.766\\
1233	-12.207\\
1234	-14.648\\
1235	-15.869\\
1236	-10.986\\
1237	-13.428\\
1238	-17.09\\
1239	-12.207\\
1240	-3.662\\
1241	-9.766\\
1242	-8.545\\
1243	-9.766\\
1244	-8.545\\
1245	-8.545\\
1246	-8.545\\
1247	-10.986\\
1248	-7.324\\
1249	-7.324\\
1250	-9.766\\
1251	-6.104\\
1252	-8.545\\
1253	-6.104\\
1254	-3.662\\
1255	-6.104\\
1256	-4.883\\
1257	-10.986\\
1258	-12.207\\
1259	-12.207\\
1260	-18.311\\
1261	-10.986\\
1262	-4.883\\
1263	-6.104\\
1264	-6.104\\
1265	-8.545\\
1266	-6.104\\
1267	-8.545\\
1268	-7.324\\
1269	-13.428\\
1270	-9.766\\
1271	-14.648\\
1272	-10.986\\
1273	-9.766\\
1274	-6.104\\
1275	-3.662\\
1276	-3.662\\
1277	-1.221\\
1278	-2.441\\
1279	-8.545\\
1280	-8.545\\
1281	-7.324\\
1282	-9.766\\
1283	-15.869\\
1284	-10.986\\
1285	-9.766\\
1286	-9.766\\
1287	-13.428\\
1288	-14.648\\
1289	-12.207\\
1290	-12.207\\
1291	-6.104\\
1292	-2.441\\
1293	-4.883\\
1294	-4.883\\
1295	-6.104\\
1296	-3.662\\
1297	-4.883\\
1298	-9.766\\
1299	-7.324\\
1300	-7.324\\
1301	-10.986\\
1302	-7.324\\
1303	-4.883\\
1304	-9.766\\
1305	-12.207\\
1306	-9.766\\
1307	-10.986\\
1308	-7.324\\
1309	-7.324\\
1310	-10.986\\
1311	-14.648\\
1312	-13.428\\
1313	-8.545\\
1314	-14.648\\
1315	-13.428\\
1316	-13.428\\
1317	-13.428\\
1318	-13.428\\
1319	-10.986\\
1320	-10.986\\
1321	-12.207\\
1322	-18.311\\
1323	-17.09\\
1324	-10.986\\
1325	-10.986\\
1326	-10.986\\
1327	-13.428\\
1328	-13.428\\
1329	-9.766\\
1330	-12.207\\
1331	-14.648\\
1332	-14.648\\
1333	-9.766\\
1334	-8.545\\
1335	-10.986\\
1336	-18.311\\
1337	-14.648\\
1338	-13.428\\
1339	-9.766\\
1340	-10.986\\
1341	-9.766\\
1342	-6.104\\
1343	-4.883\\
1344	-3.662\\
1345	-3.662\\
1346	-7.324\\
1347	-10.986\\
1348	-9.766\\
1349	-6.104\\
1350	-7.324\\
1351	-9.766\\
1352	-6.104\\
1353	-6.104\\
1354	-9.766\\
1355	-6.104\\
1356	-7.324\\
1357	-12.207\\
1358	-12.207\\
1359	-8.545\\
1360	-4.883\\
1361	-6.104\\
1362	-7.324\\
1363	-6.104\\
1364	-7.324\\
1365	-14.648\\
1366	-9.766\\
1367	-17.09\\
1368	-20.752\\
1369	-18.311\\
1370	-13.428\\
1371	-10.986\\
1372	-10.986\\
1373	-12.207\\
1374	-13.428\\
1375	-14.648\\
1376	-15.869\\
1377	-20.752\\
1378	-20.752\\
1379	-24.414\\
1380	-15.869\\
1381	-20.752\\
1382	-20.752\\
1383	-14.648\\
1384	-17.09\\
1385	-18.311\\
1386	-9.766\\
1387	-7.324\\
1388	-8.545\\
1389	-9.766\\
1390	-7.324\\
1391	-6.104\\
1392	-7.324\\
1393	-6.104\\
1394	-3.662\\
1395	-4.883\\
1396	-7.324\\
1397	-2.441\\
1398	-10.986\\
1399	-8.545\\
1400	-7.324\\
1401	-15.869\\
1402	-19.531\\
1403	-19.531\\
1404	-19.531\\
1405	-14.648\\
1406	-17.09\\
1407	-12.207\\
1408	-7.324\\
1409	-10.986\\
1410	-7.324\\
1411	-3.662\\
1412	-6.104\\
1413	-4.883\\
1414	-2.441\\
1415	-4.883\\
1416	-10.986\\
1417	-9.766\\
1418	-10.986\\
1419	-10.986\\
1420	-12.207\\
1421	-14.648\\
1422	-10.986\\
1423	-10.986\\
1424	-8.545\\
1425	-7.324\\
1426	-13.428\\
1427	-9.766\\
1428	-6.104\\
1429	-10.986\\
1430	-13.428\\
1431	-10.986\\
1432	-7.324\\
1433	-7.324\\
1434	-7.324\\
1435	-4.883\\
1436	-6.104\\
1437	-7.324\\
1438	-10.986\\
1439	-8.545\\
1440	-7.324\\
1441	-10.986\\
1442	-8.545\\
1443	-6.104\\
1444	-7.324\\
1445	-13.428\\
1446	-14.648\\
1447	-13.428\\
1448	-12.207\\
1449	-12.207\\
1450	-8.545\\
1451	-9.766\\
1452	-8.545\\
1453	-12.207\\
1454	-7.324\\
1455	-4.883\\
1456	-9.766\\
1457	-9.766\\
1458	-10.986\\
1459	-12.207\\
1460	-10.986\\
1461	-17.09\\
1462	-20.752\\
1463	-23.193\\
1464	-15.869\\
1465	-9.766\\
1466	-6.104\\
1467	-6.104\\
1468	-9.766\\
1469	-6.104\\
1470	-9.766\\
1471	-6.104\\
1472	-7.324\\
1473	-8.545\\
1474	-10.986\\
1475	-13.428\\
1476	-14.648\\
1477	-19.531\\
1478	-14.648\\
1479	-13.428\\
1480	-10.986\\
1481	-10.986\\
1482	-10.986\\
1483	-12.207\\
1484	-9.766\\
1485	-10.986\\
1486	-9.766\\
1487	-7.324\\
1488	-12.207\\
1489	-13.428\\
1490	-8.545\\
1491	-9.766\\
1492	-18.311\\
1493	-12.207\\
1494	-12.207\\
1495	-17.09\\
1496	-14.648\\
1497	-10.986\\
1498	-7.324\\
1499	-7.324\\
1500	-12.207\\
};
\end{axis}

\begin{axis}[%
width=5.090835cm,
height=2.240107cm,
at={(0cm,11.81992cm)},
scale only axis,
xmin=1000,
xmax=1500,
xlabel={Sample index},
ymin=-40.283,
ymax=0,
ylabel={C3 load, kN},
legend style={legend cell align=left,align=left,draw=white!15!black}
]
\addplot [color=mycolor1,solid,forget plot]
  table[row sep=crcr]{%
1000	-17.09\\
1001	-19.531\\
1002	-14.648\\
1003	-14.648\\
1004	-19.531\\
1005	-18.311\\
1006	-23.193\\
1007	-18.311\\
1008	-9.766\\
1009	-15.869\\
1010	-12.207\\
1011	-14.648\\
1012	-10.986\\
1013	-3.662\\
1014	-2.441\\
1015	-4.883\\
1016	-6.104\\
1017	-14.648\\
1018	-17.09\\
1019	-17.09\\
1020	-13.428\\
1021	-9.766\\
1022	-18.311\\
1023	-14.648\\
1024	-10.986\\
1025	-13.428\\
1026	-12.207\\
1027	-10.986\\
1028	-20.752\\
1029	-19.531\\
1030	-12.207\\
1031	-20.752\\
1032	-20.752\\
1033	-15.869\\
1034	-13.428\\
1035	-9.766\\
1036	-14.648\\
1037	-14.648\\
1038	-14.648\\
1039	-14.648\\
1040	-14.648\\
1041	-15.869\\
1042	-21.973\\
1043	-20.752\\
1044	-13.428\\
1045	-13.428\\
1046	-8.545\\
1047	-12.207\\
1048	-12.207\\
1049	-13.428\\
1050	-10.986\\
1051	-13.428\\
1052	-14.648\\
1053	-13.428\\
1054	-20.752\\
1055	-13.428\\
1056	-9.766\\
1057	-7.324\\
1058	-8.545\\
1059	-10.986\\
1060	-6.104\\
1061	-9.766\\
1062	-10.986\\
1063	-6.104\\
1064	-6.104\\
1065	-9.766\\
1066	-9.766\\
1067	-15.869\\
1068	-14.648\\
1069	-17.09\\
1070	-15.869\\
1071	-18.311\\
1072	-13.428\\
1073	-14.648\\
1074	-12.207\\
1075	-10.986\\
1076	-12.207\\
1077	-23.193\\
1078	-28.076\\
1079	-30.518\\
1080	-29.297\\
1081	-18.311\\
1082	-28.076\\
1083	-32.959\\
1084	-31.738\\
1085	-20.752\\
1086	-34.18\\
1087	-40.283\\
1088	-29.297\\
1089	-24.414\\
1090	-19.531\\
1091	-15.869\\
1092	-12.207\\
1093	-14.648\\
1094	-9.766\\
1095	-7.324\\
1096	-7.324\\
1097	-10.986\\
1098	-13.428\\
1099	-12.207\\
1100	-12.207\\
1101	-15.869\\
1102	-18.311\\
1103	-19.531\\
1104	-15.869\\
1105	-21.973\\
1106	-15.869\\
1107	-15.869\\
1108	-17.09\\
1109	-12.207\\
1110	-12.207\\
1111	-15.869\\
1112	-14.648\\
1113	-8.545\\
1114	-12.207\\
1115	-8.545\\
1116	-9.766\\
1117	-9.766\\
1118	-7.324\\
1119	-9.766\\
1120	-12.207\\
1121	-17.09\\
1122	-17.09\\
1123	-17.09\\
1124	-10.986\\
1125	-12.207\\
1126	-23.193\\
1127	-15.869\\
1128	-20.752\\
1129	-21.973\\
1130	-12.207\\
1131	-9.766\\
1132	-12.207\\
1133	-13.428\\
1134	-21.973\\
1135	-25.635\\
1136	-26.855\\
1137	-19.531\\
1138	-20.752\\
1139	-18.311\\
1140	-17.09\\
1141	-18.311\\
1142	-13.428\\
1143	-12.207\\
1144	-9.766\\
1145	-9.766\\
1146	-10.986\\
1147	-15.869\\
1148	-23.193\\
1149	-17.09\\
1150	-13.428\\
1151	-10.986\\
1152	-10.986\\
1153	-7.324\\
1154	-6.104\\
1155	-6.104\\
1156	-12.207\\
1157	-7.324\\
1158	-8.545\\
1159	-8.545\\
1160	-8.545\\
1161	-7.324\\
1162	-4.883\\
1163	-3.662\\
1164	-8.545\\
1165	-15.869\\
1166	-18.311\\
1167	-20.752\\
1168	-15.869\\
1169	-10.986\\
1170	-8.545\\
1171	-4.883\\
1172	-9.766\\
1173	-8.545\\
1174	-15.869\\
1175	-17.09\\
1176	-29.297\\
1177	-32.959\\
1178	-25.635\\
1179	-25.635\\
1180	-18.311\\
1181	-23.193\\
1182	-21.973\\
1183	-24.414\\
1184	-15.869\\
1185	-17.09\\
1186	-17.09\\
1187	-15.869\\
1188	-15.869\\
1189	-13.428\\
1190	-23.193\\
1191	-25.635\\
1192	-19.531\\
1193	-13.428\\
1194	-14.648\\
1195	-23.193\\
1196	-24.414\\
1197	-28.076\\
1198	-30.518\\
1199	-29.297\\
1200	-23.193\\
1201	-21.973\\
1202	-25.635\\
1203	-28.076\\
1204	-18.311\\
1205	-12.207\\
1206	-19.531\\
1207	-14.648\\
1208	-9.766\\
1209	-13.428\\
1210	-14.648\\
1211	-10.986\\
1212	-10.986\\
1213	-10.986\\
1214	-8.545\\
1215	-10.986\\
1216	-18.311\\
1217	-17.09\\
1218	-12.207\\
1219	-12.207\\
1220	-18.311\\
1221	-26.855\\
1222	-17.09\\
1223	-14.648\\
1224	-10.986\\
1225	-13.428\\
1226	-10.986\\
1227	-8.545\\
1228	-8.545\\
1229	-10.986\\
1230	-13.428\\
1231	-14.648\\
1232	-12.207\\
1233	-15.869\\
1234	-20.752\\
1235	-17.09\\
1236	-12.207\\
1237	-15.869\\
1238	-20.752\\
1239	-13.428\\
1240	-7.324\\
1241	-13.428\\
1242	-10.986\\
1243	-12.207\\
1244	-9.766\\
1245	-8.545\\
1246	-13.428\\
1247	-13.428\\
1248	-9.766\\
1249	-12.207\\
1250	-9.766\\
1251	-7.324\\
1252	-12.207\\
1253	-8.545\\
1254	-9.766\\
1255	-7.324\\
1256	-4.883\\
1257	-12.207\\
1258	-15.869\\
1259	-17.09\\
1260	-23.193\\
1261	-15.869\\
1262	-9.766\\
1263	-8.545\\
1264	-8.545\\
1265	-10.986\\
1266	-8.545\\
1267	-4.883\\
1268	-10.986\\
1269	-15.869\\
1270	-13.428\\
1271	-20.752\\
1272	-15.869\\
1273	-14.648\\
1274	-8.545\\
1275	-10.986\\
1276	-4.883\\
1277	-3.662\\
1278	-4.883\\
1279	-9.766\\
1280	-10.986\\
1281	-12.207\\
1282	-13.428\\
1283	-20.752\\
1284	-17.09\\
1285	-14.648\\
1286	-17.09\\
1287	-19.531\\
1288	-20.752\\
1289	-17.09\\
1290	-13.428\\
1291	-7.324\\
1292	-6.104\\
1293	-4.883\\
1294	-7.324\\
1295	-7.324\\
1296	-4.883\\
1297	-8.545\\
1298	-12.207\\
1299	-10.986\\
1300	-12.207\\
1301	-12.207\\
1302	-8.545\\
1303	-4.883\\
1304	-9.766\\
1305	-15.869\\
1306	-13.428\\
1307	-15.869\\
1308	-13.428\\
1309	-9.766\\
1310	-14.648\\
1311	-18.311\\
1312	-18.311\\
1313	-13.428\\
1314	-20.752\\
1315	-15.869\\
1316	-17.09\\
1317	-18.311\\
1318	-19.531\\
1319	-13.428\\
1320	-13.428\\
1321	-18.311\\
1322	-26.855\\
1323	-21.973\\
1324	-13.428\\
1325	-13.428\\
1326	-13.428\\
1327	-15.869\\
1328	-17.09\\
1329	-12.207\\
1330	-14.648\\
1331	-18.311\\
1332	-20.752\\
1333	-13.428\\
1334	-12.207\\
1335	-15.869\\
1336	-23.193\\
1337	-20.752\\
1338	-21.973\\
1339	-14.648\\
1340	-14.648\\
1341	-12.207\\
1342	-9.766\\
1343	-4.883\\
1344	-3.662\\
1345	-3.662\\
1346	-7.324\\
1347	-13.428\\
1348	-14.648\\
1349	-9.766\\
1350	-12.207\\
1351	-13.428\\
1352	-9.766\\
1353	-8.545\\
1354	-8.545\\
1355	-6.104\\
1356	-9.766\\
1357	-14.648\\
1358	-15.869\\
1359	-10.986\\
1360	-8.545\\
1361	-8.545\\
1362	-8.545\\
1363	-7.324\\
1364	-10.986\\
1365	-20.752\\
1366	-18.311\\
1367	-18.311\\
1368	-24.414\\
1369	-23.193\\
1370	-18.311\\
1371	-14.648\\
1372	-14.648\\
1373	-14.648\\
1374	-17.09\\
1375	-19.531\\
1376	-19.531\\
1377	-25.635\\
1378	-26.855\\
1379	-31.738\\
1380	-24.414\\
1381	-25.635\\
1382	-26.855\\
1383	-21.973\\
1384	-24.414\\
1385	-20.752\\
1386	-13.428\\
1387	-9.766\\
1388	-9.766\\
1389	-12.207\\
1390	-9.766\\
1391	-7.324\\
1392	-10.986\\
1393	-8.545\\
1394	-6.104\\
1395	-7.324\\
1396	-9.766\\
1397	-6.104\\
1398	-12.207\\
1399	-14.648\\
1400	-10.986\\
1401	-17.09\\
1402	-24.414\\
1403	-24.414\\
1404	-28.076\\
1405	-23.193\\
1406	-21.973\\
1407	-17.09\\
1408	-9.766\\
1409	-10.986\\
1410	-10.986\\
1411	-7.324\\
1412	-10.986\\
1413	-9.766\\
1414	-2.441\\
1415	-6.104\\
1416	-9.766\\
1417	-12.207\\
1418	-14.648\\
1419	-17.09\\
1420	-14.648\\
1421	-17.09\\
1422	-14.648\\
1423	-13.428\\
1424	-12.207\\
1425	-10.986\\
1426	-14.648\\
1427	-13.428\\
1428	-10.986\\
1429	-13.428\\
1430	-18.311\\
1431	-13.428\\
1432	-10.986\\
1433	-10.986\\
1434	-10.986\\
1435	-8.545\\
1436	-7.324\\
1437	-10.986\\
1438	-13.428\\
1439	-13.428\\
1440	-10.986\\
1441	-13.428\\
1442	-13.428\\
1443	-8.545\\
1444	-7.324\\
1445	-15.869\\
1446	-19.531\\
1447	-18.311\\
1448	-18.311\\
1449	-15.869\\
1450	-13.428\\
1451	-10.986\\
1452	-9.766\\
1453	-13.428\\
1454	-13.428\\
1455	-8.545\\
1456	-12.207\\
1457	-14.648\\
1458	-14.648\\
1459	-17.09\\
1460	-17.09\\
1461	-20.752\\
1462	-28.076\\
1463	-30.518\\
1464	-20.752\\
1465	-13.428\\
1466	-8.545\\
1467	-6.104\\
1468	-8.545\\
1469	-7.324\\
1470	-10.986\\
1471	-12.207\\
1472	-12.207\\
1473	-13.428\\
1474	-15.869\\
1475	-19.531\\
1476	-19.531\\
1477	-28.076\\
1478	-23.193\\
1479	-18.311\\
1480	-14.648\\
1481	-14.648\\
1482	-14.648\\
1483	-17.09\\
1484	-14.648\\
1485	-12.207\\
1486	-13.428\\
1487	-8.545\\
1488	-7.324\\
1489	-17.09\\
1490	-14.648\\
1491	-12.207\\
1492	-23.193\\
1493	-18.311\\
1494	-15.869\\
1495	-21.973\\
1496	-23.193\\
1497	-14.648\\
1498	-8.545\\
1499	-9.766\\
1500	-15.869\\
};
\end{axis}

\begin{axis}[%
width=5.090835cm,
height=2.240107cm,
at={(6.698467cm,11.81992cm)},
scale only axis,
xmin=1000,
xmax=1500,
xlabel={Sample index},
ymin=-26.855,
ymax=0,
ylabel={C4 load, kN},
legend style={legend cell align=left,align=left,draw=white!15!black}
]
\addplot [color=mycolor1,solid,forget plot]
  table[row sep=crcr]{%
1000	-10.986\\
1001	-15.869\\
1002	-10.986\\
1003	-10.986\\
1004	-14.648\\
1005	-14.648\\
1006	-17.09\\
1007	-10.986\\
1008	-7.324\\
1009	-14.648\\
1010	-8.545\\
1011	-9.766\\
1012	-6.104\\
1013	-3.662\\
1014	-3.662\\
1015	-3.662\\
1016	-2.441\\
1017	-13.428\\
1018	-13.428\\
1019	-13.428\\
1020	-9.766\\
1021	-7.324\\
1022	-10.986\\
1023	-12.207\\
1024	-6.104\\
1025	-9.766\\
1026	-13.428\\
1027	-7.324\\
1028	-15.869\\
1029	-13.428\\
1030	-10.986\\
1031	-17.09\\
1032	-13.428\\
1033	-10.986\\
1034	-8.545\\
1035	-8.545\\
1036	-10.986\\
1037	-12.207\\
1038	-10.986\\
1039	-9.766\\
1040	-10.986\\
1041	-10.986\\
1042	-15.869\\
1043	-14.648\\
1044	-10.986\\
1045	-10.986\\
1046	-6.104\\
1047	-4.883\\
1048	-8.545\\
1049	-7.324\\
1050	-7.324\\
1051	-10.986\\
1052	-9.766\\
1053	-9.766\\
1054	-15.869\\
1055	-8.545\\
1056	-6.104\\
1057	-7.324\\
1058	-8.545\\
1059	-9.766\\
1060	-4.883\\
1061	-7.324\\
1062	-7.324\\
1063	-7.324\\
1064	-6.104\\
1065	-7.324\\
1066	-8.545\\
1067	-9.766\\
1068	-10.986\\
1069	-10.986\\
1070	-12.207\\
1071	-13.428\\
1072	-10.986\\
1073	-12.207\\
1074	-9.766\\
1075	-9.766\\
1076	-9.766\\
1077	-17.09\\
1078	-20.752\\
1079	-19.531\\
1080	-19.531\\
1081	-15.869\\
1082	-20.752\\
1083	-23.193\\
1084	-23.193\\
1085	-14.648\\
1086	-23.193\\
1087	-26.855\\
1088	-21.973\\
1089	-15.869\\
1090	-14.648\\
1091	-10.986\\
1092	-8.545\\
1093	-10.986\\
1094	-8.545\\
1095	-6.104\\
1096	-6.104\\
1097	-8.545\\
1098	-9.766\\
1099	-10.986\\
1100	-9.766\\
1101	-12.207\\
1102	-10.986\\
1103	-15.869\\
1104	-13.428\\
1105	-18.311\\
1106	-12.207\\
1107	-10.986\\
1108	-12.207\\
1109	-9.766\\
1110	-8.545\\
1111	-10.986\\
1112	-10.986\\
1113	-8.545\\
1114	-8.545\\
1115	-7.324\\
1116	-7.324\\
1117	-7.324\\
1118	-4.883\\
1119	-6.104\\
1120	-8.545\\
1121	-13.428\\
1122	-12.207\\
1123	-13.428\\
1124	-8.545\\
1125	-15.869\\
1126	-17.09\\
1127	-13.428\\
1128	-14.648\\
1129	-14.648\\
1130	-10.986\\
1131	-7.324\\
1132	-8.545\\
1133	-8.545\\
1134	-17.09\\
1135	-20.752\\
1136	-19.531\\
1137	-14.648\\
1138	-15.869\\
1139	-13.428\\
1140	-12.207\\
1141	-12.207\\
1142	-9.766\\
1143	-8.545\\
1144	-8.545\\
1145	-9.766\\
1146	-8.545\\
1147	-13.428\\
1148	-15.869\\
1149	-10.986\\
1150	-8.545\\
1151	-9.766\\
1152	-8.545\\
1153	-6.104\\
1154	-3.662\\
1155	-4.883\\
1156	-8.545\\
1157	-9.766\\
1158	-4.883\\
1159	-7.324\\
1160	-7.324\\
1161	-3.662\\
1162	-3.662\\
1163	-2.441\\
1164	-4.883\\
1165	-10.986\\
1166	-12.207\\
1167	-14.648\\
1168	-13.428\\
1169	-8.545\\
1170	-7.324\\
1171	-4.883\\
1172	-7.324\\
1173	-6.104\\
1174	-10.986\\
1175	-12.207\\
1176	-21.973\\
1177	-23.193\\
1178	-21.973\\
1179	-18.311\\
1180	-13.428\\
1181	-17.09\\
1182	-15.869\\
1183	-19.531\\
1184	-12.207\\
1185	-13.428\\
1186	-12.207\\
1187	-13.428\\
1188	-13.428\\
1189	-10.986\\
1190	-18.311\\
1191	-17.09\\
1192	-13.428\\
1193	-7.324\\
1194	-12.207\\
1195	-15.869\\
1196	-17.09\\
1197	-19.531\\
1198	-21.973\\
1199	-20.752\\
1200	-15.869\\
1201	-14.648\\
1202	-18.311\\
1203	-19.531\\
1204	-13.428\\
1205	-9.766\\
1206	-13.428\\
1207	-9.766\\
1208	-7.324\\
1209	-9.766\\
1210	-10.986\\
1211	-7.324\\
1212	-7.324\\
1213	-9.766\\
1214	-6.104\\
1215	-10.986\\
1216	-13.428\\
1217	-13.428\\
1218	-8.545\\
1219	-9.766\\
1220	-12.207\\
1221	-18.311\\
1222	-15.869\\
1223	-9.766\\
1224	-8.545\\
1225	-9.766\\
1226	-7.324\\
1227	-8.545\\
1228	-7.324\\
1229	-8.545\\
1230	-10.986\\
1231	-10.986\\
1232	-6.104\\
1233	-10.986\\
1234	-13.428\\
1235	-13.428\\
1236	-9.766\\
1237	-12.207\\
1238	-13.428\\
1239	-10.986\\
1240	-4.883\\
1241	-10.986\\
1242	-8.545\\
1243	-8.545\\
1244	-6.104\\
1245	-6.104\\
1246	-10.986\\
1247	-10.986\\
1248	-6.104\\
1249	-8.545\\
1250	-8.545\\
1251	-4.883\\
1252	-8.545\\
1253	-4.883\\
1254	-7.324\\
1255	-4.883\\
1256	-4.883\\
1257	-10.986\\
1258	-12.207\\
1259	-12.207\\
1260	-15.869\\
1261	-10.986\\
1262	-6.104\\
1263	-7.324\\
1264	-4.883\\
1265	-7.324\\
1266	-6.104\\
1267	-3.662\\
1268	-8.545\\
1269	-9.766\\
1270	-9.766\\
1271	-17.09\\
1272	-10.986\\
1273	-13.428\\
1274	-7.324\\
1275	-8.545\\
1276	-3.662\\
1277	-1.221\\
1278	-6.104\\
1279	-9.766\\
1280	-8.545\\
1281	-9.766\\
1282	-9.766\\
1283	-17.09\\
1284	-10.986\\
1285	-10.986\\
1286	-10.986\\
1287	-13.428\\
1288	-15.869\\
1289	-15.869\\
1290	-10.986\\
1291	-6.104\\
1292	-3.662\\
1293	-4.883\\
1294	-6.104\\
1295	-4.883\\
1296	-6.104\\
1297	-6.104\\
1298	-9.766\\
1299	-9.766\\
1300	-8.545\\
1301	-9.766\\
1302	-6.104\\
1303	-3.662\\
1304	-7.324\\
1305	-10.986\\
1306	-8.545\\
1307	-12.207\\
1308	-8.545\\
1309	-7.324\\
1310	-8.545\\
1311	-12.207\\
1312	-14.648\\
1313	-10.986\\
1314	-17.09\\
1315	-9.766\\
1316	-12.207\\
1317	-13.428\\
1318	-13.428\\
1319	-9.766\\
1320	-9.766\\
1321	-12.207\\
1322	-18.311\\
1323	-15.869\\
1324	-8.545\\
1325	-12.207\\
1326	-9.766\\
1327	-12.207\\
1328	-13.428\\
1329	-9.766\\
1330	-9.766\\
1331	-14.648\\
1332	-13.428\\
1333	-12.207\\
1334	-8.545\\
1335	-9.766\\
1336	-15.869\\
1337	-14.648\\
1338	-15.869\\
1339	-12.207\\
1340	-10.986\\
1341	-9.766\\
1342	-8.545\\
1343	-4.883\\
1344	-3.662\\
1345	-3.662\\
1346	-7.324\\
1347	-10.986\\
1348	-9.766\\
1349	-8.545\\
1350	-10.986\\
1351	-9.766\\
1352	-8.545\\
1353	-3.662\\
1354	-6.104\\
1355	-6.104\\
1356	-6.104\\
1357	-8.545\\
1358	-10.986\\
1359	-7.324\\
1360	-6.104\\
1361	-7.324\\
1362	-7.324\\
1363	-6.104\\
1364	-8.545\\
1365	-15.869\\
1366	-14.648\\
1367	-17.09\\
1368	-17.09\\
1369	-17.09\\
1370	-13.428\\
1371	-10.986\\
1372	-9.766\\
1373	-10.986\\
1374	-12.207\\
1375	-14.648\\
1376	-14.648\\
1377	-17.09\\
1378	-19.531\\
1379	-21.973\\
1380	-18.311\\
1381	-21.973\\
1382	-19.531\\
1383	-14.648\\
1384	-17.09\\
1385	-14.648\\
1386	-10.986\\
1387	-8.545\\
1388	-8.545\\
1389	-10.986\\
1390	-7.324\\
1391	-7.324\\
1392	-7.324\\
1393	-7.324\\
1394	-3.662\\
1395	-7.324\\
1396	-7.324\\
1397	-6.104\\
1398	-8.545\\
1399	-12.207\\
1400	-8.545\\
1401	-12.207\\
1402	-18.311\\
1403	-17.09\\
1404	-19.531\\
1405	-17.09\\
1406	-14.648\\
1407	-14.648\\
1408	-7.324\\
1409	-9.766\\
1410	-8.545\\
1411	-6.104\\
1412	-7.324\\
1413	-8.545\\
1414	-2.441\\
1415	-4.883\\
1416	-8.545\\
1417	-9.766\\
1418	-12.207\\
1419	-12.207\\
1420	-10.986\\
1421	-12.207\\
1422	-14.648\\
1423	-10.986\\
1424	-9.766\\
1425	-7.324\\
1426	-10.986\\
1427	-10.986\\
1428	-7.324\\
1429	-10.986\\
1430	-13.428\\
1431	-10.986\\
1432	-8.545\\
1433	-8.545\\
1434	-8.545\\
1435	-6.104\\
1436	-7.324\\
1437	-6.104\\
1438	-9.766\\
1439	-8.545\\
1440	-10.986\\
1441	-10.986\\
1442	-8.545\\
1443	-6.104\\
1444	-6.104\\
1445	-10.986\\
1446	-14.648\\
1447	-14.648\\
1448	-12.207\\
1449	-10.986\\
1450	-9.766\\
1451	-8.545\\
1452	-8.545\\
1453	-9.766\\
1454	-12.207\\
1455	-7.324\\
1456	-7.324\\
1457	-10.986\\
1458	-10.986\\
1459	-13.428\\
1460	-13.428\\
1461	-14.648\\
1462	-20.752\\
1463	-21.973\\
1464	-14.648\\
1465	-9.766\\
1466	-7.324\\
1467	-6.104\\
1468	-7.324\\
1469	-4.883\\
1470	-7.324\\
1471	-7.324\\
1472	-7.324\\
1473	-10.986\\
1474	-12.207\\
1475	-13.428\\
1476	-13.428\\
1477	-18.311\\
1478	-17.09\\
1479	-12.207\\
1480	-12.207\\
1481	-9.766\\
1482	-9.766\\
1483	-13.428\\
1484	-9.766\\
1485	-8.545\\
1486	-10.986\\
1487	-6.104\\
1488	-10.986\\
1489	-14.648\\
1490	-8.545\\
1491	-8.545\\
1492	-15.869\\
1493	-13.428\\
1494	-10.986\\
1495	-17.09\\
1496	-17.09\\
1497	-10.986\\
1498	-8.545\\
1499	-9.766\\
1500	-12.207\\
};
\end{axis}

\begin{axis}[%
width=5.090835cm,
height=2.240107cm,
at={(6.698467cm,3.939973cm)},
scale only axis,
xmin=1000,
xmax=1500,
xlabel={Sample index},
ymin=-279.541,
ymax=0,
ylabel={C8 load, kN},
legend style={legend cell align=left,align=left,draw=white!15!black}
]
\addplot [color=mycolor1,solid,forget plot]
  table[row sep=crcr]{%
1000	-96.436\\
1001	-122.07\\
1002	-100.098\\
1003	-93.994\\
1004	-130.615\\
1005	-125.732\\
1006	-153.809\\
1007	-115.967\\
1008	-61.035\\
1009	-74.463\\
1010	-73.242\\
1011	-86.67\\
1012	-73.242\\
1013	-34.18\\
1014	-20.752\\
1015	-23.193\\
1016	-58.594\\
1017	-100.098\\
1018	-109.863\\
1019	-114.746\\
1020	-83.008\\
1021	-52.49\\
1022	-104.98\\
1023	-81.787\\
1024	-59.814\\
1025	-97.656\\
1026	-83.008\\
1027	-72.021\\
1028	-118.408\\
1029	-107.422\\
1030	-83.008\\
1031	-119.629\\
1032	-123.291\\
1033	-95.215\\
1034	-80.566\\
1035	-62.256\\
1036	-75.684\\
1037	-95.215\\
1038	-87.891\\
1039	-91.553\\
1040	-96.436\\
1041	-102.539\\
1042	-141.602\\
1043	-128.174\\
1044	-85.449\\
1045	-79.346\\
1046	-47.607\\
1047	-48.828\\
1048	-68.359\\
1049	-52.49\\
1050	-57.373\\
1051	-75.684\\
1052	-89.111\\
1053	-81.787\\
1054	-128.174\\
1055	-98.877\\
1056	-57.373\\
1057	-45.166\\
1058	-62.256\\
1059	-72.021\\
1060	-40.283\\
1061	-42.725\\
1062	-56.152\\
1063	-46.387\\
1064	-40.283\\
1065	-52.49\\
1066	-62.256\\
1067	-92.773\\
1068	-90.332\\
1069	-107.422\\
1070	-98.877\\
1071	-115.967\\
1072	-91.553\\
1073	-91.553\\
1074	-79.346\\
1075	-75.684\\
1076	-76.904\\
1077	-133.057\\
1078	-189.209\\
1079	-194.092\\
1080	-189.209\\
1081	-119.629\\
1082	-170.898\\
1083	-213.623\\
1084	-219.727\\
1085	-169.678\\
1086	-219.727\\
1087	-279.541\\
1088	-197.754\\
1089	-161.133\\
1090	-109.863\\
1091	-85.449\\
1092	-73.242\\
1093	-91.553\\
1094	-62.256\\
1095	-46.387\\
1096	-42.725\\
1097	-54.932\\
1098	-79.346\\
1099	-74.463\\
1100	-75.684\\
1101	-100.098\\
1102	-103.76\\
1103	-141.602\\
1104	-114.746\\
1105	-142.822\\
1106	-104.98\\
1107	-90.332\\
1108	-103.76\\
1109	-76.904\\
1110	-69.58\\
1111	-84.229\\
1112	-86.67\\
1113	-59.814\\
1114	-64.697\\
1115	-48.828\\
1116	-53.711\\
1117	-57.373\\
1118	-41.504\\
1119	-52.49\\
1120	-65.918\\
1121	-101.318\\
1122	-104.98\\
1123	-114.746\\
1124	-68.359\\
1125	-93.994\\
1126	-150.146\\
1127	-114.746\\
1128	-125.732\\
1129	-128.174\\
1130	-80.566\\
1131	-53.711\\
1132	-75.684\\
1133	-85.449\\
1134	-137.939\\
1135	-172.119\\
1136	-170.898\\
1137	-123.291\\
1138	-130.615\\
1139	-115.967\\
1140	-113.525\\
1141	-111.084\\
1142	-76.904\\
1143	-68.359\\
1144	-61.035\\
1145	-57.373\\
1146	-62.256\\
1147	-91.553\\
1148	-140.381\\
1149	-111.084\\
1150	-79.346\\
1151	-64.697\\
1152	-62.256\\
1153	-40.283\\
1154	-29.297\\
1155	-36.621\\
1156	-63.477\\
1157	-51.27\\
1158	-52.49\\
1159	-58.594\\
1160	-54.932\\
1161	-37.842\\
1162	-28.076\\
1163	-23.193\\
1164	-39.063\\
1165	-83.008\\
1166	-117.188\\
1167	-130.615\\
1168	-86.67\\
1169	-58.594\\
1170	-45.166\\
1171	-32.959\\
1172	-67.139\\
1173	-65.918\\
1174	-91.553\\
1175	-117.188\\
1176	-186.768\\
1177	-216.064\\
1178	-177.002\\
1179	-168.457\\
1180	-109.863\\
1181	-122.07\\
1182	-131.836\\
1183	-146.484\\
1184	-108.643\\
1185	-100.098\\
1186	-98.877\\
1187	-89.111\\
1188	-106.201\\
1189	-78.125\\
1190	-130.615\\
1191	-159.912\\
1192	-113.525\\
1193	-70.801\\
1194	-73.242\\
1195	-123.291\\
1196	-153.809\\
1197	-189.209\\
1198	-202.637\\
1199	-202.637\\
1200	-153.809\\
1201	-137.939\\
1202	-158.691\\
1203	-187.988\\
1204	-109.863\\
1205	-72.021\\
1206	-101.318\\
1207	-86.67\\
1208	-53.711\\
1209	-74.463\\
1210	-84.229\\
1211	-67.139\\
1212	-51.27\\
1213	-70.801\\
1214	-58.594\\
1215	-79.346\\
1216	-107.422\\
1217	-100.098\\
1218	-69.58\\
1219	-70.801\\
1220	-107.422\\
1221	-177.002\\
1222	-128.174\\
1223	-79.346\\
1224	-61.035\\
1225	-70.801\\
1226	-64.697\\
1227	-48.828\\
1228	-53.711\\
1229	-65.918\\
1230	-83.008\\
1231	-91.553\\
1232	-63.477\\
1233	-104.98\\
1234	-124.512\\
1235	-118.408\\
1236	-91.553\\
1237	-100.098\\
1238	-135.498\\
1239	-87.891\\
1240	-42.725\\
1241	-57.373\\
1242	-62.256\\
1243	-81.787\\
1244	-72.021\\
1245	-52.49\\
1246	-79.346\\
1247	-80.566\\
1248	-57.373\\
1249	-70.801\\
1250	-63.477\\
1251	-56.152\\
1252	-67.139\\
1253	-41.504\\
1254	-48.828\\
1255	-36.621\\
1256	-40.283\\
1257	-75.684\\
1258	-84.229\\
1259	-113.525\\
1260	-146.484\\
1261	-98.877\\
1262	-58.594\\
1263	-46.387\\
1264	-43.945\\
1265	-63.477\\
1266	-42.725\\
1267	-47.607\\
1268	-61.035\\
1269	-102.539\\
1270	-93.994\\
1271	-134.277\\
1272	-89.111\\
1273	-81.787\\
1274	-46.387\\
1275	-45.166\\
1276	-28.076\\
1277	-35.4\\
1278	-37.842\\
1279	-67.139\\
1280	-70.801\\
1281	-79.346\\
1282	-85.449\\
1283	-125.732\\
1284	-112.305\\
1285	-84.229\\
1286	-102.539\\
1287	-109.863\\
1288	-144.043\\
1289	-115.967\\
1290	-75.684\\
1291	-40.283\\
1292	-29.297\\
1293	-29.297\\
1294	-36.621\\
1295	-39.063\\
1296	-37.842\\
1297	-50.049\\
1298	-74.463\\
1299	-68.359\\
1300	-70.801\\
1301	-83.008\\
1302	-51.27\\
1303	-30.518\\
1304	-58.594\\
1305	-90.332\\
1306	-87.891\\
1307	-97.656\\
1308	-83.008\\
1309	-61.035\\
1310	-85.449\\
1311	-118.408\\
1312	-118.408\\
1313	-84.229\\
1314	-136.719\\
1315	-100.098\\
1316	-101.318\\
1317	-117.188\\
1318	-115.967\\
1319	-86.67\\
1320	-76.904\\
1321	-128.174\\
1322	-178.223\\
1323	-139.16\\
1324	-79.346\\
1325	-76.904\\
1326	-79.346\\
1327	-98.877\\
1328	-114.746\\
1329	-85.449\\
1330	-90.332\\
1331	-120.85\\
1332	-131.836\\
1333	-87.891\\
1334	-68.359\\
1335	-91.553\\
1336	-156.25\\
1337	-137.939\\
1338	-140.381\\
1339	-84.229\\
1340	-76.904\\
1341	-72.021\\
1342	-47.607\\
1343	-30.518\\
1344	-26.855\\
1345	-23.193\\
1346	-54.932\\
1347	-70.801\\
1348	-87.891\\
1349	-65.918\\
1350	-73.242\\
1351	-83.008\\
1352	-53.711\\
1353	-56.152\\
1354	-53.711\\
1355	-40.283\\
1356	-51.27\\
1357	-84.229\\
1358	-106.201\\
1359	-63.477\\
1360	-48.828\\
1361	-40.283\\
1362	-50.049\\
1363	-35.4\\
1364	-76.904\\
1365	-130.615\\
1366	-104.98\\
1367	-139.16\\
1368	-162.354\\
1369	-164.795\\
1370	-124.512\\
1371	-90.332\\
1372	-89.111\\
1373	-89.111\\
1374	-106.201\\
1375	-125.732\\
1376	-125.732\\
1377	-168.457\\
1378	-183.105\\
1379	-219.727\\
1380	-147.705\\
1381	-166.016\\
1382	-184.326\\
1383	-140.381\\
1384	-162.354\\
1385	-139.16\\
1386	-80.566\\
1387	-52.49\\
1388	-56.152\\
1389	-81.787\\
1390	-65.918\\
1391	-43.945\\
1392	-62.256\\
1393	-47.607\\
1394	-36.621\\
1395	-46.387\\
1396	-52.49\\
1397	-36.621\\
1398	-70.801\\
1399	-92.773\\
1400	-68.359\\
1401	-114.746\\
1402	-164.795\\
1403	-150.146\\
1404	-196.533\\
1405	-131.836\\
1406	-144.043\\
1407	-96.436\\
1408	-63.477\\
1409	-76.904\\
1410	-59.814\\
1411	-45.166\\
1412	-64.697\\
1413	-36.621\\
1414	-28.076\\
1415	-34.18\\
1416	-70.801\\
1417	-86.67\\
1418	-107.422\\
1419	-103.76\\
1420	-100.098\\
1421	-114.746\\
1422	-93.994\\
1423	-76.904\\
1424	-76.904\\
1425	-62.256\\
1426	-97.656\\
1427	-68.359\\
1428	-56.152\\
1429	-81.787\\
1430	-113.525\\
1431	-86.67\\
1432	-65.918\\
1433	-56.152\\
1434	-65.918\\
1435	-45.166\\
1436	-45.166\\
1437	-59.814\\
1438	-75.684\\
1439	-84.229\\
1440	-65.918\\
1441	-86.67\\
1442	-79.346\\
1443	-47.607\\
1444	-50.049\\
1445	-95.215\\
1446	-131.836\\
1447	-131.836\\
1448	-107.422\\
1449	-103.76\\
1450	-79.346\\
1451	-76.904\\
1452	-61.035\\
1453	-89.111\\
1454	-75.684\\
1455	-50.049\\
1456	-74.463\\
1457	-85.449\\
1458	-101.318\\
1459	-109.863\\
1460	-109.863\\
1461	-148.926\\
1462	-181.885\\
1463	-205.078\\
1464	-129.395\\
1465	-73.242\\
1466	-47.607\\
1467	-34.18\\
1468	-54.932\\
1469	-46.387\\
1470	-80.566\\
1471	-72.021\\
1472	-64.697\\
1473	-85.449\\
1474	-98.877\\
1475	-117.188\\
1476	-123.291\\
1477	-175.781\\
1478	-150.146\\
1479	-117.188\\
1480	-80.566\\
1481	-80.566\\
1482	-83.008\\
1483	-106.201\\
1484	-75.684\\
1485	-79.346\\
1486	-74.463\\
1487	-42.725\\
1488	-74.463\\
1489	-107.422\\
1490	-73.242\\
1491	-80.566\\
1492	-147.705\\
1493	-109.863\\
1494	-106.201\\
1495	-147.705\\
1496	-140.381\\
1497	-87.891\\
1498	-56.152\\
1499	-58.594\\
1500	-97.656\\
};
\end{axis}

\begin{axis}[%
width=5.090835cm,
height=2.240107cm,
at={(6.698467cm,0cm)},
scale only axis,
xmin=1000,
xmax=1500,
xlabel={Sample index},
ymin=-158.691,
ymax=-17.09,
ylabel={C10 load, kN},
legend style={legend cell align=left,align=left,draw=white!15!black}
]
\addplot [color=mycolor1,solid,forget plot]
  table[row sep=crcr]{%
1000	-63.477\\
1001	-75.684\\
1002	-62.256\\
1003	-59.814\\
1004	-80.566\\
1005	-76.904\\
1006	-90.332\\
1007	-68.359\\
1008	-40.283\\
1009	-48.828\\
1010	-46.387\\
1011	-54.932\\
1012	-45.166\\
1013	-21.973\\
1014	-17.09\\
1015	-18.311\\
1016	-40.283\\
1017	-62.256\\
1018	-67.139\\
1019	-69.58\\
1020	-52.49\\
1021	-34.18\\
1022	-64.697\\
1023	-52.49\\
1024	-39.063\\
1025	-61.035\\
1026	-53.711\\
1027	-45.166\\
1028	-73.242\\
1029	-67.139\\
1030	-52.49\\
1031	-75.684\\
1032	-74.463\\
1033	-58.594\\
1034	-50.049\\
1035	-41.504\\
1036	-47.607\\
1037	-58.594\\
1038	-54.932\\
1039	-59.814\\
1040	-59.814\\
1041	-65.918\\
1042	-85.449\\
1043	-76.904\\
1044	-54.932\\
1045	-50.049\\
1046	-35.4\\
1047	-34.18\\
1048	-45.166\\
1049	-34.18\\
1050	-36.621\\
1051	-51.27\\
1052	-54.932\\
1053	-51.27\\
1054	-78.125\\
1055	-62.256\\
1056	-36.621\\
1057	-30.518\\
1058	-40.283\\
1059	-46.387\\
1060	-29.297\\
1061	-29.297\\
1062	-36.621\\
1063	-31.738\\
1064	-26.855\\
1065	-35.4\\
1066	-40.283\\
1067	-58.594\\
1068	-56.152\\
1069	-65.918\\
1070	-61.035\\
1071	-70.801\\
1072	-57.373\\
1073	-58.594\\
1074	-51.27\\
1075	-50.049\\
1076	-48.828\\
1077	-80.566\\
1078	-111.084\\
1079	-114.746\\
1080	-112.305\\
1081	-74.463\\
1082	-100.098\\
1083	-125.732\\
1084	-128.174\\
1085	-96.436\\
1086	-124.512\\
1087	-158.691\\
1088	-115.967\\
1089	-97.656\\
1090	-68.359\\
1091	-53.711\\
1092	-47.607\\
1093	-56.152\\
1094	-45.166\\
1095	-30.518\\
1096	-30.518\\
1097	-36.621\\
1098	-50.049\\
1099	-48.828\\
1100	-47.607\\
1101	-62.256\\
1102	-64.697\\
1103	-85.449\\
1104	-72.021\\
1105	-85.449\\
1106	-63.477\\
1107	-54.932\\
1108	-64.697\\
1109	-48.828\\
1110	-43.945\\
1111	-53.711\\
1112	-54.932\\
1113	-40.283\\
1114	-43.945\\
1115	-32.959\\
1116	-36.621\\
1117	-40.283\\
1118	-30.518\\
1119	-34.18\\
1120	-43.945\\
1121	-63.477\\
1122	-65.918\\
1123	-72.021\\
1124	-45.166\\
1125	-56.152\\
1126	-89.111\\
1127	-70.801\\
1128	-81.787\\
1129	-80.566\\
1130	-50.049\\
1131	-36.621\\
1132	-47.607\\
1133	-52.49\\
1134	-81.787\\
1135	-103.76\\
1136	-102.539\\
1137	-76.904\\
1138	-80.566\\
1139	-72.021\\
1140	-69.58\\
1141	-68.359\\
1142	-51.27\\
1143	-45.166\\
1144	-39.063\\
1145	-37.842\\
1146	-41.504\\
1147	-56.152\\
1148	-84.229\\
1149	-70.801\\
1150	-50.049\\
1151	-42.725\\
1152	-40.283\\
1153	-28.076\\
1154	-23.193\\
1155	-24.414\\
1156	-41.504\\
1157	-32.959\\
1158	-34.18\\
1159	-37.842\\
1160	-35.4\\
1161	-26.855\\
1162	-21.973\\
1163	-18.311\\
1164	-25.635\\
1165	-51.27\\
1166	-73.242\\
1167	-78.125\\
1168	-54.932\\
1169	-39.063\\
1170	-31.738\\
1171	-24.414\\
1172	-41.504\\
1173	-42.725\\
1174	-54.932\\
1175	-73.242\\
1176	-108.643\\
1177	-125.732\\
1178	-103.76\\
1179	-101.318\\
1180	-67.139\\
1181	-75.684\\
1182	-79.346\\
1183	-86.67\\
1184	-69.58\\
1185	-62.256\\
1186	-61.035\\
1187	-56.152\\
1188	-67.139\\
1189	-52.49\\
1190	-76.904\\
1191	-97.656\\
1192	-72.021\\
1193	-45.166\\
1194	-47.607\\
1195	-78.125\\
1196	-93.994\\
1197	-112.305\\
1198	-119.629\\
1199	-119.629\\
1200	-91.553\\
1201	-84.229\\
1202	-96.436\\
1203	-111.084\\
1204	-74.463\\
1205	-46.387\\
1206	-62.256\\
1207	-54.932\\
1208	-34.18\\
1209	-47.607\\
1210	-53.711\\
1211	-42.725\\
1212	-34.18\\
1213	-43.945\\
1214	-40.283\\
1215	-52.49\\
1216	-65.918\\
1217	-62.256\\
1218	-45.166\\
1219	-45.166\\
1220	-67.139\\
1221	-104.98\\
1222	-79.346\\
1223	-51.27\\
1224	-40.283\\
1225	-47.607\\
1226	-40.283\\
1227	-31.738\\
1228	-35.4\\
1229	-42.725\\
1230	-51.27\\
1231	-57.373\\
1232	-42.725\\
1233	-63.477\\
1234	-75.684\\
1235	-70.801\\
1236	-57.373\\
1237	-62.256\\
1238	-81.787\\
1239	-52.49\\
1240	-28.076\\
1241	-36.621\\
1242	-42.725\\
1243	-51.27\\
1244	-46.387\\
1245	-36.621\\
1246	-52.49\\
1247	-52.49\\
1248	-37.842\\
1249	-47.607\\
1250	-41.504\\
1251	-37.842\\
1252	-45.166\\
1253	-29.297\\
1254	-31.738\\
1255	-29.297\\
1256	-26.855\\
1257	-47.607\\
1258	-53.711\\
1259	-69.58\\
1260	-89.111\\
1261	-62.256\\
1262	-36.621\\
1263	-31.738\\
1264	-29.297\\
1265	-40.283\\
1266	-31.738\\
1267	-30.518\\
1268	-39.063\\
1269	-61.035\\
1270	-53.711\\
1271	-79.346\\
1272	-54.932\\
1273	-48.828\\
1274	-32.959\\
1275	-29.297\\
1276	-23.193\\
1277	-23.193\\
1278	-25.635\\
1279	-40.283\\
1280	-46.387\\
1281	-48.828\\
1282	-51.27\\
1283	-79.346\\
1284	-73.242\\
1285	-53.711\\
1286	-63.477\\
1287	-69.58\\
1288	-85.449\\
1289	-72.021\\
1290	-47.607\\
1291	-28.076\\
1292	-21.973\\
1293	-21.973\\
1294	-26.855\\
1295	-26.855\\
1296	-25.635\\
1297	-32.959\\
1298	-47.607\\
1299	-43.945\\
1300	-45.166\\
1301	-52.49\\
1302	-35.4\\
1303	-20.752\\
1304	-35.4\\
1305	-56.152\\
1306	-54.932\\
1307	-59.814\\
1308	-53.711\\
1309	-40.283\\
1310	-51.27\\
1311	-72.021\\
1312	-70.801\\
1313	-51.27\\
1314	-81.787\\
1315	-62.256\\
1316	-63.477\\
1317	-73.242\\
1318	-70.801\\
1319	-54.932\\
1320	-47.607\\
1321	-78.125\\
1322	-108.643\\
1323	-85.449\\
1324	-50.049\\
1325	-51.27\\
1326	-51.27\\
1327	-61.035\\
1328	-72.021\\
1329	-54.932\\
1330	-56.152\\
1331	-73.242\\
1332	-79.346\\
1333	-56.152\\
1334	-45.166\\
1335	-57.373\\
1336	-87.891\\
1337	-78.125\\
1338	-80.566\\
1339	-54.932\\
1340	-45.166\\
1341	-47.607\\
1342	-31.738\\
1343	-20.752\\
1344	-20.752\\
1345	-17.09\\
1346	-31.738\\
1347	-48.828\\
1348	-54.932\\
1349	-40.283\\
1350	-45.166\\
1351	-52.49\\
1352	-35.4\\
1353	-35.4\\
1354	-35.4\\
1355	-29.297\\
1356	-31.738\\
1357	-51.27\\
1358	-64.697\\
1359	-41.504\\
1360	-32.959\\
1361	-28.076\\
1362	-32.959\\
1363	-26.855\\
1364	-45.166\\
1365	-80.566\\
1366	-63.477\\
1367	-81.787\\
1368	-98.877\\
1369	-97.656\\
1370	-76.904\\
1371	-58.594\\
1372	-54.932\\
1373	-57.373\\
1374	-65.918\\
1375	-76.904\\
1376	-76.904\\
1377	-100.098\\
1378	-108.643\\
1379	-130.615\\
1380	-91.553\\
1381	-98.877\\
1382	-108.643\\
1383	-85.449\\
1384	-96.436\\
1385	-85.449\\
1386	-51.27\\
1387	-34.18\\
1388	-36.621\\
1389	-51.27\\
1390	-43.945\\
1391	-31.738\\
1392	-41.504\\
1393	-32.959\\
1394	-23.193\\
1395	-31.738\\
1396	-34.18\\
1397	-25.635\\
1398	-42.725\\
1399	-58.594\\
1400	-45.166\\
1401	-68.359\\
1402	-100.098\\
1403	-87.891\\
1404	-109.863\\
1405	-81.787\\
1406	-81.787\\
1407	-62.256\\
1408	-37.842\\
1409	-46.387\\
1410	-42.725\\
1411	-29.297\\
1412	-40.283\\
1413	-21.973\\
1414	-19.531\\
1415	-24.414\\
1416	-45.166\\
1417	-57.373\\
1418	-65.918\\
1419	-65.918\\
1420	-62.256\\
1421	-69.58\\
1422	-57.373\\
1423	-48.828\\
1424	-50.049\\
1425	-43.945\\
1426	-59.814\\
1427	-46.387\\
1428	-36.621\\
1429	-51.27\\
1430	-69.58\\
1431	-54.932\\
1432	-41.504\\
1433	-37.842\\
1434	-41.504\\
1435	-32.959\\
1436	-30.518\\
1437	-41.504\\
1438	-47.607\\
1439	-52.49\\
1440	-42.725\\
1441	-53.711\\
1442	-51.27\\
1443	-32.959\\
1444	-32.959\\
1445	-58.594\\
1446	-80.566\\
1447	-78.125\\
1448	-67.139\\
1449	-62.256\\
1450	-50.049\\
1451	-48.828\\
1452	-41.504\\
1453	-54.932\\
1454	-50.049\\
1455	-32.959\\
1456	-45.166\\
1457	-52.49\\
1458	-61.035\\
1459	-68.359\\
1460	-68.359\\
1461	-87.891\\
1462	-107.422\\
1463	-119.629\\
1464	-80.566\\
1465	-46.387\\
1466	-32.959\\
1467	-24.414\\
1468	-34.18\\
1469	-29.297\\
1470	-51.27\\
1471	-48.828\\
1472	-40.283\\
1473	-52.49\\
1474	-63.477\\
1475	-70.801\\
1476	-75.684\\
1477	-103.76\\
1478	-92.773\\
1479	-72.021\\
1480	-51.27\\
1481	-52.49\\
1482	-54.932\\
1483	-65.918\\
1484	-51.27\\
1485	-50.049\\
1486	-48.828\\
1487	-30.518\\
1488	-45.166\\
1489	-68.359\\
1490	-48.828\\
1491	-52.49\\
1492	-86.67\\
1493	-63.477\\
1494	-63.477\\
1495	-85.449\\
1496	-81.787\\
1497	-53.711\\
1498	-35.4\\
1499	-36.621\\
1500	-61.035\\
};
\end{axis}
\end{tikzpicture}%
	\caption{The measured output for each foam sample is the force of the specimen response in kN.}
\end{figure}\label{fig:output}
\section{Structure identification}
\par The following model structure is assumed. The output of the NARX model $\mathbfit{y}(t)$ is the measured load. The input vector is composed as
\begin{equation}
	\mathbfit{x}(t) = \{x_i(t)\}^{d}_{i=1} = \left[\{y(t - k + 1)\}^{n_y}_{k=1} \quad \{u(t - k + n_y + 1)\}^{n_y + n_u}_{k= n_y + 1} \right]^{\top},
\end{equation}
where $n_u$ is the length of the input lag and $n_y$ is the length of the output lag in discrete time, and where $d = n_u + n_y$. In this case, the identification is performed under the following assumptions:
\begin{itemize}[noitemsep,topsep=0.3pt,parsep=0.3pt,partopsep=0.2pt,labelindent=1cm] 
	\item only the input signal affects the output ($n_y = 0$).
	\item the input signal has a lag of length $n_u = 4$.
\end{itemize}
The resultant input vector of the NARX model then takes the following form:
\begin{equation}
\mathbfit{x}(t) = \left[\begin{array}{cccc}
u(t-3) & u(t-2) & u(t-1) & u(t)
\end{array}\right]^{\top}.
\end{equation}
The unknown model is approximated with a sum of polynomial basis functions up to second degree ($\lambda = 2$), rendering the following structure
\begin{equation}\label{eq:narx}
	\mathbfit{y}(t) = \theta^0 + \sum_{i=1}^{d} \theta_i x_i(t) + \sum_{i=1}^{d} \sum_{j=1}^{d} \theta_{i,j} x_i(t) x_j(t) + e(t).
\end{equation}
The number and order of significant terms are identified within the EFOR-CMSS algorithm based on the data from 8 out of 10 datasets. Figure \ref{fig:aamdl} illustrates the relationship between the number of model terms and the selected criterion of significance, AAMDL.
\begin{figure}[!h]
	\centering
	\definecolor{mycolor1}{rgb}{0.00000,0.44700,0.74100}%
	\definecolor{mycolor2}{rgb}{0.85000,0.32500,0.09800}%
	\subfloat[Sample size 2000.]{% This file was created by matlab2tikz.
% Minimal pgfplots version: 1.3
%
\definecolor{mycolor1}{rgb}{0.00000,0.44700,0.74100}%
\definecolor{mycolor2}{rgb}{0.85000,0.32500,0.09800}%
%
\begin{tikzpicture}

\begin{axis}[%
width=5.979382cm,
height=4cm,
at={(0cm,0cm)},
scale only axis,
xmin=1,
xmax=15,
xlabel={Number of terms},
ymin=-3.00104696019893,
ymax=-2,
ylabel={AAMDL},
legend style={legend cell align=left,align=left,draw=white!15!black}
]
\addplot [color=mycolor1,only marks,mark=o,mark options={solid},forget plot]
  table[row sep=crcr]{%
1	-2.11570305645112\\
2	-2.79305839626823\\
3	-2.82935915360095\\
4	-2.83995463691255\\
5	-2.84820086237127\\
6	-2.84802935769127\\
7	-2.95749685829162\\
8	-3.00104696019893\\
9	-3.00045381599073\\
10	-3.00058067310023\\
11	-2.99929045654952\\
12	-2.99893492528062\\
13	-2.99723787908848\\
14	-2.99552868329925\\
15	-2.99382083512231\\
};
\addplot [color=mycolor2,line width=5.0pt,only marks,mark=asterisk,mark options={solid},forget plot]
  table[row sep=crcr]
	\subfloat[Sample size 4000.]{% This file was created by matlab2tikz.
% Minimal pgfplots version: 1.3
%
\definecolor{mycolor1}{rgb}{0.00000,0.44700,0.74100}%
\definecolor{mycolor2}{rgb}{0.85000,0.32500,0.09800}%
%
\begin{tikzpicture}

\begin{axis}[%
width=5.979382cm,
height=4cm,
at={(0cm,0cm)},
scale only axis,
xmin=1,
xmax=15,
xlabel={Number of terms},
ymin=-2.8,
ymax=-1.8,
ylabel={AAMDL},
legend style={legend cell align=left,align=left,draw=white!15!black}
]
\addplot [color=mycolor1,only marks,mark=o,mark options={solid},forget plot]
  table[row sep=crcr]{%
1	-1.85238639353984\\
2	-2.51661709688035\\
3	-2.54325120703526\\
4	-2.55128539901149\\
5	-2.55878900600603\\
6	-2.55830335858106\\
7	-2.65674313484709\\
8	-2.69442575885501\\
9	-2.69503767764109\\
10	-2.69462917143197\\
11	-2.69378515117799\\
12	-2.69252058304669\\
13	-2.69122725766994\\
14	-2.68930444709195\\
15	-2.68735035007295\\
};
\addplot [color=mycolor2,line width=5.0pt,only marks,mark=asterisk,mark options={solid},forget plot]
  table[row sep=crcr]
	\caption{Evolution of AAMDL with the respect to the number of terms for samples of different size. The optimal number of terms (\ref{tikz:nterms}) increases with the growing sample size.}\label{fig:aamdl}
\end{figure}	
\section{Parameter estimation}
\par The results of internal parameter estimation via the EFOR-CMSS method for sample sizes of 2000 and 4000 points are presented in Tables \ref{tab:thetas2000} and \ref{tab:thetas4000}, respectively. It can be seen that the first 7 significant terms are the same for both samples; moreover, the estimated values are of the same order in both cases. The further analysis deals only with the parameter estimates obtained from the smaller data sample. 
\begin{table}[!h]
	\centering
	\caption{Estimated parameters for the sample length 2000.}\label{tab:thetas2000}
	\small
	\begin{tabular}{rrrrrrrrrrr}
Step & Terms & C1 & C2 & C4 & C5 & C6 & C7 & C9 & C10 & AEER($\%$) \\ 
\hline 
1 & $x_4 x_4$ & -26.04 & -20.99 & -10.69 & -10.96 & -191.78 & -157.64 & -87.42 & -69.8 & 89.511 \\ 
2 & $x_3$ & 75.42 & 59.58 & 33 & 26.06 & 508.94 & 419.54 & 242.35 & 195.15 & 8.849 \\ 
3 & $x_1 x_4$ & 0.62 & 0.76 & 0.32 & 0.48 & 8.55 & 7.83 & 2.66 & 1.15 & 0.139 \\ 
4 & $x_1 x_1$ & 0.01 & -0.19 & -0.15 & -0.22 & 0.05 & -0.48 & 0.44 & 0.76 & 0.045 \\ 
5 & $x_2$ & 0.71 & -0.73 & -2.24 & -0.66 & 45.94 & 36.57 & 18.4 & 12.68 & 0.032 \\ 
6 & $x_4$ & -171.24 & -139.22 & -69.61 & -73.69 & -1273.02 & -1046.38 & -579.72 & -465.59 & 0.006 \\ 
7 & $c$ & -233.16 & -200.83 & -93.74 & -119.7 & -1805.9 & -1488.55 & -803.7 & -648.8 & 0.308 \\ 
8 & $x_3 x_4$ & 15.47 & 12.1 & 6.36 & 5.68 & 110.13 & 90.43 & 51.77 & 41.43 & 0.093 \\ 
\hline 
\end{tabular}
\end{table}
\begin{table}[!h]
	\centering
	\caption{Estimated parameters for the sample length 4000.}\label{tab:thetas4000}
	\small
	\begin{tabular}{rrrrrrrrrrr}
Step & Terms & C1 & C2 & C4 & C5 & C6 & C7 & C9 & C10 & AEER($\%$) \\ 
\hline 
1 & $x_4 x_4$ & -18.97 & -15.33 & -8.34 & -8.94 & -138.21 & -114.23 & -63.58 & -50.3 & 88.667 \\ 
2 & $x_3$ & 63.2 & 53.09 & 30.09 & 24.55 & 426.09 & 362.14 & 208.4 & 168.78 & 9.494 \\ 
3 & $x_1 x_4$ & 4.29 & 3.45 & 2.03 & 3.11 & 34.39 & 27.1 & 14.45 & 10.2 & 0.12 \\ 
4 & $x_1 x_1$ & 0.6 & 0.87 & 0.31 & 0.56 & 9.24 & 5.54 & 4.15 & 4.1 & 0.042 \\ 
5 & $x_2$ & 2.34 & 0.07 & -1.9 & -2.13 & 43.4 & 34.96 & 18.62 & 12.71 & 0.036 \\ 
6 & $x_4$ & -157.87 & -128.81 & -69.1 & -69.29 & -1153.95 & -964.72 & -539.51 & -431.83 & 0.006 \\ 
7 & $c$ & -226.03 & -187.69 & -100.5 & -116.41 & -1722.47 & -1432.4 & -789.68 & -632.3 & 0.335 \\ 
8 & $x_3 x_3$ & 9.21 & 7.92 & 4.31 & 4.67 & 70 & 55.98 & 32.65 & 26.32 & 0.103 \\ 
9 & $x_1 x_3$ & -4.8 & -4.8 & -2.64 & -4.26 & -45.05 & -31.95 & -19.47 & -15.95 & 0.007 \\ 
\hline 
\end{tabular}
\end{table}
\par In order to link the external and internal parameters, an arbitrary polynomial function of two arguments is formed
\begin{equation}
\theta_i(L_{cut},D_{rlx}) = \beta_0 + \beta_1 L_{cut} + \beta_2 D_{rlx} + \beta_3 L_{cut}^{2} + \beta_4 D_{rlx}^{2} + \beta_5 L_{cut} D_{rlx}, \qquad i=1,\dots,N_s,
\end{equation}
$L_{cut}$ and $D_{rlx}$ are the external parameters of manufacturing process, and where $N_s$ is the estimated number of the significant terms. The coefficients $B = \left[ \beta_0 \dots \beta_5 \right]$ are unknown and must be estimated from the internal parameter values available. The linear relationship between the batch of internal parameter values corresponding to each significant term $\Theta_i$ and the surface coefficients can be established in the following form
\begin{equation}\label{eq:linrel}
\left[\begin{array}{c}
\theta_{i}^{1} \\
\vdots \\
\theta_{i}^{k} \\
\vdots \\
\theta_{i}^{K}
\end{array}\right] =
\left[\begin{array}{cccccc}
1 & L& D& L\times L& D\times D& L\times D
\end{array}\right] B,
\end{equation}
where $k = 1, \dots, K$ is the index of the dataset, $i =1, \dots, N_s$ is the element index of the internal parameter vector, and where $\times$ denotes the by-element product such that
\begin{equation*}
L\times L = \Big[ L_{cut}^{(j)} L_{cut}^{(j)}\Big]^{N}_{j=1}.
\end{equation*}
Given the equation \eqref{eq:linrel}, the vector of unknown coefficients $B$ can be identified via the classical least squares (LS) method. The estimated coefficients are presented in Table \ref{tab:betas_all}, and the surface fitting results for each internal parameter are illustrated in Figure \ref{fig:surfaces_all} 
\begin{table}[!h]
	\centering
	\caption{Estimated polynomial coefficients for the sample length 2000.}\label{tab:betas_all}
	\small
	\begin{tabular}{rrrrrrrr}
NARMAX terms & Input terms & $\beta_0$ & $\beta_1$ & $\beta_2$ & $\beta_3$ & $\beta_4$ & $\beta_5$ \\ 
\hline 
$x_4,x_4$ & $u^2(t)$ & -170.31 & 4.27 & 831.86 & -2.02 & -0.03 & -4562.38 \\ 
$x_3$ & $u(t-1)$& 143.25 & -3.37 & -1382.17 & -3.9 & 0.04 & 11446.44 \\ 
$x_1,x_4$ & $u(t-3)u(t)$ & 5.32 & 0.17 & -203.84 & 2.88 & 0 & 161.25 \\ 
$x_1,x_1$ & $u^2(t-3)$ & 1.85 & -0.2 & 78.58 & -1.4 & 0 & 39.42 \\ 
$x_2$ & $u(t-2)$ & 76.05 & -1.76 & -434.89 & 5.27 & 0.01 & 953.45 \\ 
$x_4$ & $u(t)$ & -1216.06 & 31.23 & 5342.46 & -8.72 & -0.25 & -30672.05 \\ 
$c$ & $const$ & -2427.76 & 63.16 & 8882.74 & -21.5 & -0.49 & -45523.06 \\ 
$x_3,x_4$ & $u(t-1)u(t)$& 61.26 & -1.53 & -370.55 & 0.3 & 0.01 & 2478.05 \\ 
\hline 
\end{tabular}
\end{table}
The figure also shows values of the internal parameters computed for the external settings of experiments C3 and C8. The obtained internal parameters are substituted in the modified version of model \eqref{eq:narx} that only includes the identified significant  polynomial terms to validate the identified model structure. The simulation results are compared with true system outputs for C3 and C8 in Figure \ref{fig:c3all} and Figure \ref{fig:c8all}, respectively. 
\begin{figure}[!t]
	\centering
	% This file was created by matlab2tikz.
% Minimal pgfplots version: 1.3
%
\definecolor{mycolor1}{rgb}{0.00000,0.44700,0.74100}%
\definecolor{mycolor2}{rgb}{0.85000,0.32500,0.09800}%
%
\begin{tikzpicture}

\begin{axis}[%
width=4.527496cm,
height=3.870968cm,
at={(0cm,0cm)},
scale only axis,
xmin=56,
xmax=74,
tick align=outside,
xlabel={$L_{cut}$},
xmajorgrids,
ymin=0.093,
ymax=0.276,
ylabel={$D_{rlx}$},
ymajorgrids,
zmin=-2000,
zmax=0,
zlabel={$c$},
zmajorgrids,
view={-140}{50},
legend style={at={(1.03,1)},anchor=north west,legend cell align=left,align=left,draw=white!15!black}
]
\addplot3[only marks,mark=*,mark options={},mark size=1.5000pt,color=mycolor1] plot table[row sep=crcr,]{%
74	0.123	-233.157966898601\\
72	0.113	-200.830593420783\\
61	0.095	-93.7391747783605\\
56	0.093	-119.696332564413\\
};\label{tikz:thetas1}
\addplot3[only marks,mark=*,mark options={},mark size=1.5000pt,color=mycolor2] plot table[row sep=crcr,]{%
67	0.276	-1805.89673785913\\
66	0.255	-1488.55090215304\\
62	0.209	-803.703476355143\\
57	0.193	-648.796609601896\\
};\label{tikz:thetas2}
\addplot3[only marks,mark=*,mark options={},mark size=1.5000pt,color=black] plot table[row sep=crcr,]{%
69	0.104	-138.628962092727\\
};\label{tikz:thetaidentified}
\addplot3[only marks,mark=*,mark options={},mark size=1.5000pt,color=black] plot table[row sep=crcr,]{%
64	0.23	-1085.45181546038\\
};

\addplot3[%
surf,
opacity=0.7,
shader=interp,
colormap={mymap}{[1pt] rgb(0pt)=(0.0901961,0.239216,0.0745098); rgb(1pt)=(0.0945149,0.242058,0.0739522); rgb(2pt)=(0.0988592,0.244894,0.0733566); rgb(3pt)=(0.103229,0.247724,0.0727241); rgb(4pt)=(0.107623,0.250549,0.0720557); rgb(5pt)=(0.112043,0.253367,0.0713525); rgb(6pt)=(0.116487,0.25618,0.0706154); rgb(7pt)=(0.120956,0.258986,0.0698456); rgb(8pt)=(0.125449,0.261787,0.0690441); rgb(9pt)=(0.129967,0.264581,0.0682118); rgb(10pt)=(0.134508,0.26737,0.06735); rgb(11pt)=(0.139074,0.270152,0.0664596); rgb(12pt)=(0.143663,0.272929,0.0655416); rgb(13pt)=(0.148275,0.275699,0.0645971); rgb(14pt)=(0.152911,0.278463,0.0636271); rgb(15pt)=(0.15757,0.281221,0.0626328); rgb(16pt)=(0.162252,0.283973,0.0616151); rgb(17pt)=(0.166957,0.286719,0.060575); rgb(18pt)=(0.171685,0.289458,0.0595136); rgb(19pt)=(0.176434,0.292191,0.0584321); rgb(20pt)=(0.181207,0.294918,0.0573313); rgb(21pt)=(0.186001,0.297639,0.0562123); rgb(22pt)=(0.190817,0.300353,0.0550763); rgb(23pt)=(0.195655,0.303061,0.0539242); rgb(24pt)=(0.200514,0.305763,0.052757); rgb(25pt)=(0.205395,0.308459,0.0515759); rgb(26pt)=(0.210296,0.311149,0.0503624); rgb(27pt)=(0.215212,0.313846,0.0490067); rgb(28pt)=(0.220142,0.316548,0.0475043); rgb(29pt)=(0.22509,0.319254,0.0458704); rgb(30pt)=(0.230056,0.321962,0.0441205); rgb(31pt)=(0.235042,0.324671,0.04227); rgb(32pt)=(0.240048,0.327379,0.0403343); rgb(33pt)=(0.245078,0.330085,0.0383287); rgb(34pt)=(0.250131,0.332786,0.0362688); rgb(35pt)=(0.25521,0.335482,0.0341698); rgb(36pt)=(0.260317,0.33817,0.0320472); rgb(37pt)=(0.265451,0.340849,0.0299163); rgb(38pt)=(0.270616,0.343517,0.0277927); rgb(39pt)=(0.275813,0.346172,0.0256916); rgb(40pt)=(0.281043,0.348814,0.0236284); rgb(41pt)=(0.286307,0.35144,0.0216186); rgb(42pt)=(0.291607,0.354048,0.0196776); rgb(43pt)=(0.296945,0.356637,0.0178207); rgb(44pt)=(0.302322,0.359206,0.0160634); rgb(45pt)=(0.307739,0.361753,0.0144211); rgb(46pt)=(0.313198,0.364275,0.0129091); rgb(47pt)=(0.318701,0.366772,0.0115428); rgb(48pt)=(0.324249,0.369242,0.0103377); rgb(49pt)=(0.329843,0.371682,0.00930909); rgb(50pt)=(0.335485,0.374093,0.00847245); rgb(51pt)=(0.341176,0.376471,0.00784314); rgb(52pt)=(0.346925,0.378826,0.00732741); rgb(53pt)=(0.352735,0.381168,0.00682184); rgb(54pt)=(0.358605,0.383497,0.00632729); rgb(55pt)=(0.364532,0.385812,0.00584464); rgb(56pt)=(0.370516,0.388113,0.00537476); rgb(57pt)=(0.376552,0.390399,0.00491852); rgb(58pt)=(0.38264,0.39267,0.00447681); rgb(59pt)=(0.388777,0.394925,0.00405048); rgb(60pt)=(0.394962,0.397164,0.00364042); rgb(61pt)=(0.401191,0.399386,0.00324749); rgb(62pt)=(0.407464,0.401592,0.00287258); rgb(63pt)=(0.413777,0.40378,0.00251655); rgb(64pt)=(0.420129,0.40595,0.00218028); rgb(65pt)=(0.426518,0.408102,0.00186463); rgb(66pt)=(0.432942,0.410234,0.00157049); rgb(67pt)=(0.439399,0.412348,0.00129873); rgb(68pt)=(0.445885,0.414441,0.00105022); rgb(69pt)=(0.452401,0.416515,0.000825833); rgb(70pt)=(0.458942,0.418567,0.000626441); rgb(71pt)=(0.465508,0.420599,0.00045292); rgb(72pt)=(0.472096,0.422609,0.000306141); rgb(73pt)=(0.478704,0.424596,0.000186979); rgb(74pt)=(0.485331,0.426562,9.63073e-05); rgb(75pt)=(0.491973,0.428504,3.49981e-05); rgb(76pt)=(0.498628,0.430422,3.92506e-06); rgb(77pt)=(0.505323,0.432315,0); rgb(78pt)=(0.512206,0.434168,0); rgb(79pt)=(0.519282,0.435983,0); rgb(80pt)=(0.526529,0.437764,0); rgb(81pt)=(0.533922,0.439512,0); rgb(82pt)=(0.54144,0.441232,0); rgb(83pt)=(0.549059,0.442927,0); rgb(84pt)=(0.556756,0.444599,0); rgb(85pt)=(0.564508,0.446252,0); rgb(86pt)=(0.572292,0.447889,0); rgb(87pt)=(0.580084,0.449514,0); rgb(88pt)=(0.587863,0.451129,0); rgb(89pt)=(0.595604,0.452737,0); rgb(90pt)=(0.603284,0.454343,0); rgb(91pt)=(0.610882,0.455948,0); rgb(92pt)=(0.618373,0.457556,0); rgb(93pt)=(0.625734,0.459171,0); rgb(94pt)=(0.632943,0.460795,0); rgb(95pt)=(0.639976,0.462432,0); rgb(96pt)=(0.64681,0.464084,0); rgb(97pt)=(0.653423,0.465756,0); rgb(98pt)=(0.659791,0.46745,0); rgb(99pt)=(0.665891,0.469169,0); rgb(100pt)=(0.6717,0.470916,0); rgb(101pt)=(0.677195,0.472696,0); rgb(102pt)=(0.682353,0.47451,0); rgb(103pt)=(0.687242,0.476355,0); rgb(104pt)=(0.691952,0.478225,0); rgb(105pt)=(0.696497,0.480118,0); rgb(106pt)=(0.700887,0.482033,0); rgb(107pt)=(0.705134,0.483968,0); rgb(108pt)=(0.709251,0.485921,0); rgb(109pt)=(0.713249,0.487891,0); rgb(110pt)=(0.71714,0.489876,0); rgb(111pt)=(0.720936,0.491875,0); rgb(112pt)=(0.724649,0.493887,0); rgb(113pt)=(0.72829,0.495909,0); rgb(114pt)=(0.731872,0.49794,0); rgb(115pt)=(0.735406,0.499979,0); rgb(116pt)=(0.738904,0.502025,0); rgb(117pt)=(0.742378,0.504075,0); rgb(118pt)=(0.74584,0.506128,0); rgb(119pt)=(0.749302,0.508182,0); rgb(120pt)=(0.752775,0.510237,0); rgb(121pt)=(0.756272,0.51229,0); rgb(122pt)=(0.759804,0.514339,0); rgb(123pt)=(0.763384,0.516385,0); rgb(124pt)=(0.767022,0.518424,0); rgb(125pt)=(0.770731,0.520455,0); rgb(126pt)=(0.774523,0.522478,0); rgb(127pt)=(0.77841,0.524489,0); rgb(128pt)=(0.782391,0.526491,0); rgb(129pt)=(0.786402,0.528496,0); rgb(130pt)=(0.790431,0.530506,0); rgb(131pt)=(0.794478,0.532521,0); rgb(132pt)=(0.798541,0.534539,0); rgb(133pt)=(0.802619,0.53656,0); rgb(134pt)=(0.806712,0.538584,0); rgb(135pt)=(0.81082,0.540609,0); rgb(136pt)=(0.81494,0.542635,0); rgb(137pt)=(0.819074,0.54466,0); rgb(138pt)=(0.823219,0.546686,0); rgb(139pt)=(0.827374,0.548709,0); rgb(140pt)=(0.831541,0.55073,0); rgb(141pt)=(0.835716,0.552749,0); rgb(142pt)=(0.8399,0.554763,0); rgb(143pt)=(0.844092,0.556774,0); rgb(144pt)=(0.848292,0.558779,0); rgb(145pt)=(0.852497,0.560778,0); rgb(146pt)=(0.856708,0.562771,0); rgb(147pt)=(0.860924,0.564756,0); rgb(148pt)=(0.865143,0.566733,0); rgb(149pt)=(0.869366,0.568701,0); rgb(150pt)=(0.873592,0.57066,0); rgb(151pt)=(0.877819,0.572608,0); rgb(152pt)=(0.882047,0.574545,0); rgb(153pt)=(0.886275,0.576471,0); rgb(154pt)=(0.890659,0.578362,0); rgb(155pt)=(0.895333,0.580203,0); rgb(156pt)=(0.900258,0.581999,0); rgb(157pt)=(0.905397,0.583755,0); rgb(158pt)=(0.910711,0.585479,0); rgb(159pt)=(0.916164,0.587176,0); rgb(160pt)=(0.921717,0.588852,0); rgb(161pt)=(0.927333,0.590513,0); rgb(162pt)=(0.932974,0.592166,0); rgb(163pt)=(0.938602,0.593815,0); rgb(164pt)=(0.94418,0.595468,0); rgb(165pt)=(0.949669,0.59713,0); rgb(166pt)=(0.955033,0.598808,0); rgb(167pt)=(0.960233,0.600507,0); rgb(168pt)=(0.965232,0.602233,0); rgb(169pt)=(0.969992,0.603992,0); rgb(170pt)=(0.974475,0.605791,0); rgb(171pt)=(0.978643,0.607636,0); rgb(172pt)=(0.98246,0.609532,0); rgb(173pt)=(0.985886,0.611486,0); rgb(174pt)=(0.988885,0.613503,0); rgb(175pt)=(0.991419,0.61559,0); rgb(176pt)=(0.99345,0.617753,0); rgb(177pt)=(0.99494,0.619997,0); rgb(178pt)=(0.995851,0.622329,0); rgb(179pt)=(0.996226,0.624763,0); rgb(180pt)=(0.996512,0.627352,0); rgb(181pt)=(0.996788,0.630095,0); rgb(182pt)=(0.997053,0.632982,0); rgb(183pt)=(0.997308,0.636004,0); rgb(184pt)=(0.997552,0.639152,0); rgb(185pt)=(0.997785,0.642416,0); rgb(186pt)=(0.998006,0.645786,0); rgb(187pt)=(0.998217,0.649253,0); rgb(188pt)=(0.998416,0.652807,0); rgb(189pt)=(0.998605,0.656439,0); rgb(190pt)=(0.998781,0.660138,0); rgb(191pt)=(0.998946,0.663897,0); rgb(192pt)=(0.9991,0.667704,0); rgb(193pt)=(0.999242,0.67155,0); rgb(194pt)=(0.999372,0.675427,0); rgb(195pt)=(0.99949,0.679323,0); rgb(196pt)=(0.999596,0.68323,0); rgb(197pt)=(0.99969,0.687139,0); rgb(198pt)=(0.999771,0.691039,0); rgb(199pt)=(0.999841,0.694921,0); rgb(200pt)=(0.999898,0.698775,0); rgb(201pt)=(0.999942,0.702592,0); rgb(202pt)=(0.999974,0.706363,0); rgb(203pt)=(0.999994,0.710077,0); rgb(204pt)=(1,0.713725,0); rgb(205pt)=(1,0.717341,0); rgb(206pt)=(1,0.720963,0); rgb(207pt)=(1,0.724591,0); rgb(208pt)=(1,0.728226,0); rgb(209pt)=(1,0.731867,0); rgb(210pt)=(1,0.735514,0); rgb(211pt)=(1,0.739167,0); rgb(212pt)=(1,0.742827,0); rgb(213pt)=(1,0.746493,0); rgb(214pt)=(1,0.750165,0); rgb(215pt)=(1,0.753843,0); rgb(216pt)=(1,0.757527,0); rgb(217pt)=(1,0.761217,0); rgb(218pt)=(1,0.764913,0); rgb(219pt)=(1,0.768615,0); rgb(220pt)=(1,0.772324,0); rgb(221pt)=(1,0.776038,0); rgb(222pt)=(1,0.779758,0); rgb(223pt)=(1,0.783484,0); rgb(224pt)=(1,0.787215,0); rgb(225pt)=(1,0.790953,0); rgb(226pt)=(1,0.794696,0); rgb(227pt)=(1,0.798445,0); rgb(228pt)=(1,0.8022,0); rgb(229pt)=(1,0.805961,0); rgb(230pt)=(1,0.809727,0); rgb(231pt)=(1,0.8135,0); rgb(232pt)=(1,0.817278,0); rgb(233pt)=(1,0.821063,0); rgb(234pt)=(1,0.824854,0); rgb(235pt)=(1,0.828652,0); rgb(236pt)=(1,0.832455,0); rgb(237pt)=(1,0.836265,0); rgb(238pt)=(1,0.840081,0); rgb(239pt)=(1,0.843903,0); rgb(240pt)=(1,0.847732,0); rgb(241pt)=(1,0.851566,0); rgb(242pt)=(1,0.855406,0); rgb(243pt)=(1,0.859253,0); rgb(244pt)=(1,0.863106,0); rgb(245pt)=(1,0.866964,0); rgb(246pt)=(1,0.870829,0); rgb(247pt)=(1,0.8747,0); rgb(248pt)=(1,0.878577,0); rgb(249pt)=(1,0.88246,0); rgb(250pt)=(1,0.886349,0); rgb(251pt)=(1,0.890243,0); rgb(252pt)=(1,0.894144,0); rgb(253pt)=(1,0.898051,0); rgb(254pt)=(1,0.901964,0); rgb(255pt)=(1,0.905882,0)},
mesh/rows=49]
table[row sep=crcr,header=false] {%
%
56	0.093	-115.68341116986\\
56	0.09666	-119.178384949931\\
56	0.10032	-123.892976330784\\
56	0.10398	-129.827185312418\\
56	0.10764	-136.981011894832\\
56	0.1113	-145.354456078027\\
56	0.11496	-154.947517862003\\
56	0.11862	-165.76019724676\\
56	0.12228	-177.792494232298\\
56	0.12594	-191.044408818617\\
56	0.1296	-205.515941005717\\
56	0.13326	-221.207090793597\\
56	0.13692	-238.117858182259\\
56	0.14058	-256.248243171702\\
56	0.14424	-275.598245761925\\
56	0.1479	-296.16786595293\\
56	0.15156	-317.957103744714\\
56	0.15522	-340.965959137281\\
56	0.15888	-365.194432130628\\
56	0.16254	-390.642522724757\\
56	0.1662	-417.310230919665\\
56	0.16986	-445.197556715355\\
56	0.17352	-474.304500111826\\
56	0.17718	-504.631061109078\\
56	0.18084	-536.177239707109\\
56	0.1845	-568.943035905924\\
56	0.18816	-602.928449705518\\
56	0.19182	-638.133481105893\\
56	0.19548	-674.558130107049\\
56	0.19914	-712.202396708986\\
56	0.2028	-751.066280911705\\
56	0.20646	-791.149782715203\\
56	0.21012	-832.452902119484\\
56	0.21378	-874.975639124543\\
56	0.21744	-918.717993730386\\
56	0.2211	-963.679965937008\\
56	0.22476	-1009.86155574441\\
56	0.22842	-1057.2627631526\\
56	0.23208	-1105.88358816156\\
56	0.23574	-1155.72403077131\\
56	0.2394	-1206.78409098184\\
56	0.24306	-1259.06376879314\\
56	0.24672	-1312.56306420523\\
56	0.25038	-1367.2819772181\\
56	0.25404	-1423.22050783175\\
56	0.2577	-1480.37865604619\\
56	0.26136	-1538.7564218614\\
56	0.26502	-1598.35380527739\\
56	0.26868	-1659.17080629417\\
56	0.27234	-1721.20742491172\\
56	0.276	-1784.46366113006\\
56.375	0.093	-113.514447153983\\
56.375	0.09666	-117.038923411893\\
56.375	0.10032	-121.783017270585\\
56.375	0.10398	-127.746728730057\\
56.375	0.10764	-134.93005779031\\
56.375	0.1113	-143.333004451344\\
56.375	0.11496	-152.955568713159\\
56.375	0.11862	-163.797750575755\\
56.375	0.12228	-175.859550039132\\
56.375	0.12594	-189.140967103289\\
56.375	0.1296	-203.642001768228\\
56.375	0.13326	-219.362654033947\\
56.375	0.13692	-236.302923900447\\
56.375	0.14058	-254.462811367729\\
56.375	0.14424	-273.842316435791\\
56.375	0.1479	-294.441439104634\\
56.375	0.15156	-316.260179374258\\
56.375	0.15522	-339.298537244663\\
56.375	0.15888	-363.556512715849\\
56.375	0.16254	-389.034105787816\\
56.375	0.1662	-415.731316460563\\
56.375	0.16986	-443.648144734093\\
56.375	0.17352	-472.784590608401\\
56.375	0.17718	-503.140654083492\\
56.375	0.18084	-534.716335159363\\
56.375	0.1845	-567.511633836016\\
56.375	0.18816	-601.526550113449\\
56.375	0.19182	-636.761083991663\\
56.375	0.19548	-673.215235470658\\
56.375	0.19914	-710.889004550434\\
56.375	0.2028	-749.782391230991\\
56.375	0.20646	-789.895395512328\\
56.375	0.21012	-831.228017394447\\
56.375	0.21378	-873.780256877346\\
56.375	0.21744	-917.552113961027\\
56.375	0.2211	-962.543588645488\\
56.375	0.22476	-1008.75468093073\\
56.375	0.22842	-1056.18539081675\\
56.375	0.23208	-1104.83571830356\\
56.375	0.23574	-1154.70566339114\\
56.375	0.2394	-1205.79522607951\\
56.375	0.24306	-1258.10440636866\\
56.375	0.24672	-1311.63320425858\\
56.375	0.25038	-1366.38161974929\\
56.375	0.25404	-1422.34965284078\\
56.375	0.2577	-1479.53730353305\\
56.375	0.26136	-1537.9445718261\\
56.375	0.26502	-1597.57145771994\\
56.375	0.26868	-1658.41796121455\\
56.375	0.27234	-1720.48408230994\\
56.375	0.276	-1783.76982100612\\
56.75	0.093	-111.484086745226\\
56.75	0.09666	-115.038065480975\\
56.75	0.10032	-119.811661817505\\
56.75	0.10398	-125.804875754816\\
56.75	0.10764	-133.017707292908\\
56.75	0.1113	-141.45015643178\\
56.75	0.11496	-151.102223171434\\
56.75	0.11862	-161.973907511869\\
56.75	0.12228	-174.065209453084\\
56.75	0.12594	-187.376128995081\\
56.75	0.1296	-201.906666137858\\
56.75	0.13326	-217.656820881416\\
56.75	0.13692	-234.626593225755\\
56.75	0.14058	-252.815983170875\\
56.75	0.14424	-272.224990716777\\
56.75	0.1479	-292.853615863458\\
56.75	0.15156	-314.701858610921\\
56.75	0.15522	-337.769718959164\\
56.75	0.15888	-362.05719690819\\
56.75	0.16254	-387.564292457995\\
56.75	0.1662	-414.291005608582\\
56.75	0.16986	-442.237336359949\\
56.75	0.17352	-471.403284712098\\
56.75	0.17718	-501.788850665026\\
56.75	0.18084	-533.394034218737\\
56.75	0.1845	-566.218835373227\\
56.75	0.18816	-600.2632541285\\
56.75	0.19182	-635.527290484552\\
56.75	0.19548	-672.010944441386\\
56.75	0.19914	-709.714215999001\\
56.75	0.2028	-748.637105157396\\
56.75	0.20646	-788.779611916573\\
56.75	0.21012	-830.14173627653\\
56.75	0.21378	-872.723478237269\\
56.75	0.21744	-916.524837798787\\
56.75	0.2211	-961.545814961088\\
56.75	0.22476	-1007.78640972417\\
56.75	0.22842	-1055.24662208803\\
56.75	0.23208	-1103.92645205267\\
56.75	0.23574	-1153.8258996181\\
56.75	0.2394	-1204.9449647843\\
56.75	0.24306	-1257.28364755129\\
56.75	0.24672	-1310.84194791905\\
56.75	0.25038	-1365.6198658876\\
56.75	0.25404	-1421.61740145693\\
56.75	0.2577	-1478.83455462704\\
56.75	0.26136	-1537.27132539793\\
56.75	0.26502	-1596.9277137696\\
56.75	0.26868	-1657.80371974205\\
56.75	0.27234	-1719.89934331529\\
56.75	0.276	-1783.2145844893\\
57.125	0.093	-109.592329943587\\
57.125	0.09666	-113.175811157175\\
57.125	0.10032	-117.978909971543\\
57.125	0.10398	-124.001626386693\\
57.125	0.10764	-131.243960402624\\
57.125	0.1113	-139.705912019335\\
57.125	0.11496	-149.387481236828\\
57.125	0.11862	-160.288668055101\\
57.125	0.12228	-172.409472474155\\
57.125	0.12594	-185.749894493991\\
57.125	0.1296	-200.309934114607\\
57.125	0.13326	-216.089591336004\\
57.125	0.13692	-233.088866158182\\
57.125	0.14058	-251.30775858114\\
57.125	0.14424	-270.746268604881\\
57.125	0.1479	-291.404396229401\\
57.125	0.15156	-313.282141454703\\
57.125	0.15522	-336.379504280785\\
57.125	0.15888	-360.696484707648\\
57.125	0.16254	-386.233082735293\\
57.125	0.1662	-412.989298363718\\
57.125	0.16986	-440.965131592924\\
57.125	0.17352	-470.160582422912\\
57.125	0.17718	-500.575650853679\\
57.125	0.18084	-532.210336885228\\
57.125	0.1845	-565.064640517558\\
57.125	0.18816	-599.138561750668\\
57.125	0.19182	-634.43210058456\\
57.125	0.19548	-670.945257019233\\
57.125	0.19914	-708.678031054686\\
57.125	0.2028	-747.630422690921\\
57.125	0.20646	-787.802431927936\\
57.125	0.21012	-829.194058765732\\
57.125	0.21378	-871.805303204309\\
57.125	0.21744	-915.636165243666\\
57.125	0.2211	-960.686644883806\\
57.125	0.22476	-1006.95674212473\\
57.125	0.22842	-1054.44645696643\\
57.125	0.23208	-1103.15578940891\\
57.125	0.23574	-1153.08473945217\\
57.125	0.2394	-1204.23330709621\\
57.125	0.24306	-1256.60149234104\\
57.125	0.24672	-1310.18929518664\\
57.125	0.25038	-1364.99671563303\\
57.125	0.25404	-1421.0237536802\\
57.125	0.2577	-1478.27040932815\\
57.125	0.26136	-1536.73668257687\\
57.125	0.26502	-1596.42257342638\\
57.125	0.26868	-1657.32808187668\\
57.125	0.27234	-1719.45320792775\\
57.125	0.276	-1782.7979515796\\
57.5	0.093	-107.839176749067\\
57.5	0.09666	-111.452160440494\\
57.5	0.10032	-116.284761732701\\
57.5	0.10398	-122.33698062569\\
57.5	0.10764	-129.608817119459\\
57.5	0.1113	-138.10027121401\\
57.5	0.11496	-147.811342909341\\
57.5	0.11862	-158.742032205453\\
57.5	0.12228	-170.892339102346\\
57.5	0.12594	-184.26226360002\\
57.5	0.1296	-198.851805698475\\
57.5	0.13326	-214.66096539771\\
57.5	0.13692	-231.689742697727\\
57.5	0.14058	-249.938137598525\\
57.5	0.14424	-269.406150100104\\
57.5	0.1479	-290.093780202463\\
57.5	0.15156	-312.001027905603\\
57.5	0.15522	-335.127893209524\\
57.5	0.15888	-359.474376114227\\
57.5	0.16254	-385.04047661971\\
57.5	0.1662	-411.826194725974\\
57.5	0.16986	-439.831530433019\\
57.5	0.17352	-469.056483740845\\
57.5	0.17718	-499.501054649451\\
57.5	0.18084	-531.165243158839\\
57.5	0.1845	-564.049049269008\\
57.5	0.18816	-598.152472979957\\
57.5	0.19182	-633.475514291687\\
57.5	0.19548	-670.018173204199\\
57.5	0.19914	-707.780449717491\\
57.5	0.2028	-746.762343831564\\
57.5	0.20646	-786.963855546418\\
57.5	0.21012	-828.384984862053\\
57.5	0.21378	-871.025731778469\\
57.5	0.21744	-914.886096295665\\
57.5	0.2211	-959.966078413643\\
57.5	0.22476	-1006.2656781324\\
57.5	0.22842	-1053.78489545194\\
57.5	0.23208	-1102.52373037226\\
57.5	0.23574	-1152.48218289336\\
57.5	0.2394	-1203.66025301525\\
57.5	0.24306	-1256.05794073791\\
57.5	0.24672	-1309.67524606135\\
57.5	0.25038	-1364.51216898558\\
57.5	0.25404	-1420.56870951058\\
57.5	0.2577	-1477.84486763637\\
57.5	0.26136	-1536.34064336294\\
57.5	0.26502	-1596.05603669029\\
57.5	0.26868	-1656.99104761842\\
57.5	0.27234	-1719.14567614733\\
57.5	0.276	-1782.51992227702\\
57.875	0.093	-106.224627161667\\
57.875	0.09666	-109.867113330933\\
57.875	0.10032	-114.729217100979\\
57.875	0.10398	-120.810938471806\\
57.875	0.10764	-128.112277443414\\
57.875	0.1113	-136.633234015804\\
57.875	0.11496	-146.373808188973\\
57.875	0.11862	-157.333999962924\\
57.875	0.12228	-169.513809337656\\
57.875	0.12594	-182.913236313169\\
57.875	0.1296	-197.532280889462\\
57.875	0.13326	-213.370943066537\\
57.875	0.13692	-230.429222844393\\
57.875	0.14058	-248.707120223029\\
57.875	0.14424	-268.204635202446\\
57.875	0.1479	-288.921767782645\\
57.875	0.15156	-310.858517963623\\
57.875	0.15522	-334.014885745384\\
57.875	0.15888	-358.390871127924\\
57.875	0.16254	-383.986474111247\\
57.875	0.1662	-410.801694695349\\
57.875	0.16986	-438.836532880234\\
57.875	0.17352	-468.090988665897\\
57.875	0.17718	-498.565062052343\\
57.875	0.18084	-530.258753039569\\
57.875	0.1845	-563.172061627577\\
57.875	0.18816	-597.304987816365\\
57.875	0.19182	-632.657531605935\\
57.875	0.19548	-669.229692996284\\
57.875	0.19914	-707.021471987415\\
57.875	0.2028	-746.032868579328\\
57.875	0.20646	-786.26388277202\\
57.875	0.21012	-827.714514565494\\
57.875	0.21378	-870.384763959748\\
57.875	0.21744	-914.274630954784\\
57.875	0.2211	-959.3841155506\\
57.875	0.22476	-1005.7132177472\\
57.875	0.22842	-1053.26193754458\\
57.875	0.23208	-1102.03027494274\\
57.875	0.23574	-1152.01822994168\\
57.875	0.2394	-1203.2258025414\\
57.875	0.24306	-1255.6529927419\\
57.875	0.24672	-1309.29980054318\\
57.875	0.25038	-1364.16622594525\\
57.875	0.25404	-1420.25226894809\\
57.875	0.2577	-1477.55792955172\\
57.875	0.26136	-1536.08320775612\\
57.875	0.26502	-1595.82810356131\\
57.875	0.26868	-1656.79261696728\\
57.875	0.27234	-1718.97674797403\\
57.875	0.276	-1782.38049658156\\
58.25	0.093	-104.748681181386\\
58.25	0.09666	-108.42066982849\\
58.25	0.10032	-113.312276076375\\
58.25	0.10398	-119.423499925041\\
58.25	0.10764	-126.754341374488\\
58.25	0.1113	-135.304800424716\\
58.25	0.11496	-145.074877075725\\
58.25	0.11862	-156.064571327514\\
58.25	0.12228	-168.273883180085\\
58.25	0.12594	-181.702812633437\\
58.25	0.1296	-196.351359687569\\
58.25	0.13326	-212.219524342482\\
58.25	0.13692	-229.307306598176\\
58.25	0.14058	-247.614706454652\\
58.25	0.14424	-267.141723911908\\
58.25	0.1479	-287.888358969945\\
58.25	0.15156	-309.854611628762\\
58.25	0.15522	-333.040481888362\\
58.25	0.15888	-357.445969748741\\
58.25	0.16254	-383.071075209902\\
58.25	0.1662	-409.915798271843\\
58.25	0.16986	-437.980138934566\\
58.25	0.17352	-467.264097198069\\
58.25	0.17718	-497.767673062354\\
58.25	0.18084	-529.490866527418\\
58.25	0.1845	-562.433677593265\\
58.25	0.18816	-596.596106259892\\
58.25	0.19182	-631.9781525273\\
58.25	0.19548	-668.579816395489\\
58.25	0.19914	-706.401097864459\\
58.25	0.2028	-745.44199693421\\
58.25	0.20646	-785.702513604741\\
58.25	0.21012	-827.182647876054\\
58.25	0.21378	-869.882399748146\\
58.25	0.21744	-913.801769221021\\
58.25	0.2211	-958.940756294676\\
58.25	0.22476	-1005.29936096911\\
58.25	0.22842	-1052.87758324433\\
58.25	0.23208	-1101.67542312033\\
58.25	0.23574	-1151.69288059711\\
58.25	0.2394	-1202.92995567467\\
58.25	0.24306	-1255.38664835301\\
58.25	0.24672	-1309.06295863213\\
58.25	0.25038	-1363.95888651203\\
58.25	0.25404	-1420.07443199271\\
58.25	0.2577	-1477.40959507418\\
58.25	0.26136	-1535.96437575642\\
58.25	0.26502	-1595.73877403945\\
58.25	0.26868	-1656.73278992326\\
58.25	0.27234	-1718.94642340785\\
58.25	0.276	-1782.37967449321\\
58.625	0.093	-103.411338808224\\
58.625	0.09666	-107.112829933167\\
58.625	0.10032	-112.033938658891\\
58.625	0.10398	-118.174664985396\\
58.625	0.10764	-125.535008912681\\
58.625	0.1113	-134.114970440748\\
58.625	0.11496	-143.914549569595\\
58.625	0.11862	-154.933746299224\\
58.625	0.12228	-167.172560629633\\
58.625	0.12594	-180.630992560824\\
58.625	0.1296	-195.309042092795\\
58.625	0.13326	-211.206709225547\\
58.625	0.13692	-228.32399395908\\
58.625	0.14058	-246.660896293393\\
58.625	0.14424	-266.217416228489\\
58.625	0.1479	-286.993553764364\\
58.625	0.15156	-308.989308901021\\
58.625	0.15522	-332.204681638459\\
58.625	0.15888	-356.639671976677\\
58.625	0.16254	-382.294279915676\\
58.625	0.1662	-409.168505455457\\
58.625	0.16986	-437.262348596018\\
58.625	0.17352	-466.575809337361\\
58.625	0.17718	-497.108887679483\\
58.625	0.18084	-528.861583622388\\
58.625	0.1845	-561.833897166072\\
58.625	0.18816	-596.025828310538\\
58.625	0.19182	-631.437377055785\\
58.625	0.19548	-668.068543401812\\
58.625	0.19914	-705.919327348621\\
58.625	0.2028	-744.98972889621\\
58.625	0.20646	-785.279748044581\\
58.625	0.21012	-826.789384793732\\
58.625	0.21378	-869.518639143664\\
58.625	0.21744	-913.467511094377\\
58.625	0.2211	-958.636000645871\\
58.625	0.22476	-1005.02410779815\\
58.625	0.22842	-1052.6318325512\\
58.625	0.23208	-1101.45917490504\\
58.625	0.23574	-1151.50613485966\\
58.625	0.2394	-1202.77271241506\\
58.625	0.24306	-1255.25890757123\\
58.625	0.24672	-1308.96472032819\\
58.625	0.25038	-1363.89015068594\\
58.625	0.25404	-1420.03519864446\\
58.625	0.2577	-1477.39986420376\\
58.625	0.26136	-1535.98414736385\\
58.625	0.26502	-1595.78804812471\\
58.625	0.26868	-1656.81156648636\\
58.625	0.27234	-1719.05470244878\\
58.625	0.276	-1782.51745601199\\
59	0.093	-102.212600042181\\
59	0.09666	-105.943593644963\\
59	0.10032	-110.894204848525\\
59	0.10398	-117.064433652869\\
59	0.10764	-124.454280057993\\
59	0.1113	-133.063744063899\\
59	0.11496	-142.892825670585\\
59	0.11862	-153.941524878052\\
59	0.12228	-166.2098416863\\
59	0.12594	-179.697776095329\\
59	0.1296	-194.405328105139\\
59	0.13326	-210.33249771573\\
59	0.13692	-227.479284927102\\
59	0.14058	-245.845689739254\\
59	0.14424	-265.431712152189\\
59	0.1479	-286.237352165903\\
59	0.15156	-308.262609780399\\
59	0.15522	-331.507484995674\\
59	0.15888	-355.971977811732\\
59	0.16254	-381.65608822857\\
59	0.1662	-408.559816246189\\
59	0.16986	-436.683161864589\\
59	0.17352	-466.02612508377\\
59	0.17718	-496.588705903732\\
59	0.18084	-528.370904324475\\
59	0.1845	-561.372720345998\\
59	0.18816	-595.594153968303\\
59	0.19182	-631.035205191389\\
59	0.19548	-667.695874015255\\
59	0.19914	-705.576160439902\\
59	0.2028	-744.67606446533\\
59	0.20646	-784.99558609154\\
59	0.21012	-826.534725318529\\
59	0.21378	-869.293482146301\\
59	0.21744	-913.271856574852\\
59	0.2211	-958.469848604185\\
59	0.22476	-1004.8874582343\\
59	0.22842	-1052.52468546519\\
59	0.23208	-1101.38153029687\\
59	0.23574	-1151.45799272933\\
59	0.2394	-1202.75407276256\\
59	0.24306	-1255.26977039658\\
59	0.24672	-1309.00508563138\\
59	0.25038	-1363.96001846696\\
59	0.25404	-1420.13456890332\\
59	0.2577	-1477.52873694046\\
59	0.26136	-1536.14252257839\\
59	0.26502	-1595.97592581709\\
59	0.26868	-1657.02894665657\\
59	0.27234	-1719.30158509684\\
59	0.276	-1782.79384113789\\
59.375	0.093	-101.152464883257\\
59.375	0.09666	-104.912960963878\\
59.375	0.10032	-109.893074645279\\
59.375	0.10398	-116.092805927461\\
59.375	0.10764	-123.512154810424\\
59.375	0.1113	-132.151121294169\\
59.375	0.11496	-142.009705378694\\
59.375	0.11862	-153.087907064\\
59.375	0.12228	-165.385726350087\\
59.375	0.12594	-178.903163236955\\
59.375	0.1296	-193.640217724603\\
59.375	0.13326	-209.596889813033\\
59.375	0.13692	-226.773179502243\\
59.375	0.14058	-245.169086792235\\
59.375	0.14424	-264.784611683007\\
59.375	0.1479	-285.619754174561\\
59.375	0.15156	-307.674514266894\\
59.375	0.15522	-330.94889196001\\
59.375	0.15888	-355.442887253906\\
59.375	0.16254	-381.156500148584\\
59.375	0.1662	-408.089730644041\\
59.375	0.16986	-436.24257874028\\
59.375	0.17352	-465.615044437299\\
59.375	0.17718	-496.207127735101\\
59.375	0.18084	-528.018828633681\\
59.375	0.1845	-561.050147133044\\
59.375	0.18816	-595.301083233187\\
59.375	0.19182	-630.771636934112\\
59.375	0.19548	-667.461808235817\\
59.375	0.19914	-705.371597138303\\
59.375	0.2028	-744.50100364157\\
59.375	0.20646	-784.850027745618\\
59.375	0.21012	-826.418669450447\\
59.375	0.21378	-869.206928756056\\
59.375	0.21744	-913.214805662447\\
59.375	0.2211	-958.442300169618\\
59.375	0.22476	-1004.88941227757\\
59.375	0.22842	-1052.5561419863\\
59.375	0.23208	-1101.44248929582\\
59.375	0.23574	-1151.54845420611\\
59.375	0.2394	-1202.87403671719\\
59.375	0.24306	-1255.41923682905\\
59.375	0.24672	-1309.18405454168\\
59.375	0.25038	-1364.1684898551\\
59.375	0.25404	-1420.3725427693\\
59.375	0.2577	-1477.79621328428\\
59.375	0.26136	-1536.43950140005\\
59.375	0.26502	-1596.30240711659\\
59.375	0.26868	-1657.38493043391\\
59.375	0.27234	-1719.68707135202\\
59.375	0.276	-1783.2088298709\\
59.75	0.093	-100.230933331452\\
59.75	0.09666	-104.020931889911\\
59.75	0.10032	-109.030548049152\\
59.75	0.10398	-115.259781809173\\
59.75	0.10764	-122.708633169975\\
59.75	0.1113	-131.377102131558\\
59.75	0.11496	-141.265188693922\\
59.75	0.11862	-152.372892857067\\
59.75	0.12228	-164.700214620992\\
59.75	0.12594	-178.247153985699\\
59.75	0.1296	-193.013710951186\\
59.75	0.13326	-208.999885517455\\
59.75	0.13692	-226.205677684504\\
59.75	0.14058	-244.631087452335\\
59.75	0.14424	-264.276114820945\\
59.75	0.1479	-285.140759790338\\
59.75	0.15156	-307.22502236051\\
59.75	0.15522	-330.528902531465\\
59.75	0.15888	-355.052400303199\\
59.75	0.16254	-380.795515675716\\
59.75	0.1662	-407.758248649012\\
59.75	0.16986	-435.94059922309\\
59.75	0.17352	-465.342567397948\\
59.75	0.17718	-495.964153173587\\
59.75	0.18084	-527.805356550007\\
59.75	0.1845	-560.866177527209\\
59.75	0.18816	-595.146616105191\\
59.75	0.19182	-630.646672283954\\
59.75	0.19548	-667.366346063498\\
59.75	0.19914	-705.305637443822\\
59.75	0.2028	-744.464546424929\\
59.75	0.20646	-784.843073006815\\
59.75	0.21012	-826.441217189483\\
59.75	0.21378	-869.258978972931\\
59.75	0.21744	-913.296358357161\\
59.75	0.2211	-958.55335534217\\
59.75	0.22476	-1005.02996992796\\
59.75	0.22842	-1052.72620211453\\
59.75	0.23208	-1101.64205190189\\
59.75	0.23574	-1151.77751929002\\
59.75	0.2394	-1203.13260427894\\
59.75	0.24306	-1255.70730686863\\
59.75	0.24672	-1309.50162705911\\
59.75	0.25038	-1364.51556485037\\
59.75	0.25404	-1420.74912024241\\
59.75	0.2577	-1478.20229323522\\
59.75	0.26136	-1536.87508382882\\
59.75	0.26502	-1596.76749202321\\
59.75	0.26868	-1657.87951781837\\
59.75	0.27234	-1720.21116121431\\
59.75	0.276	-1783.76242221104\\
60.125	0.093	-99.448005386767\\
60.125	0.09666	-103.267506423065\\
60.125	0.10032	-108.306625060144\\
60.125	0.10398	-114.565361298004\\
60.125	0.10764	-122.043715136645\\
60.125	0.1113	-130.741686576066\\
60.125	0.11496	-140.659275616269\\
60.125	0.11862	-151.796482257253\\
60.125	0.12228	-164.153306499017\\
60.125	0.12594	-177.729748341563\\
60.125	0.1296	-192.525807784889\\
60.125	0.13326	-208.541484828996\\
60.125	0.13692	-225.776779473884\\
60.125	0.14058	-244.231691719553\\
60.125	0.14424	-263.906221566003\\
60.125	0.1479	-284.800369013234\\
60.125	0.15156	-306.914134061246\\
60.125	0.15522	-330.247516710038\\
60.125	0.15888	-354.800516959612\\
60.125	0.16254	-380.573134809966\\
60.125	0.1662	-407.565370261102\\
60.125	0.16986	-435.777223313018\\
60.125	0.17352	-465.208693965716\\
60.125	0.17718	-495.859782219193\\
60.125	0.18084	-527.730488073453\\
60.125	0.1845	-560.820811528492\\
60.125	0.18816	-595.130752584313\\
60.125	0.19182	-630.660311240915\\
60.125	0.19548	-667.409487498298\\
60.125	0.19914	-705.378281356461\\
60.125	0.2028	-744.566692815406\\
60.125	0.20646	-784.974721875132\\
60.125	0.21012	-826.602368535638\\
60.125	0.21378	-869.449632796925\\
60.125	0.21744	-913.516514658993\\
60.125	0.2211	-958.803014121842\\
60.125	0.22476	-1005.30913118547\\
60.125	0.22842	-1053.03486584988\\
60.125	0.23208	-1101.98021811507\\
60.125	0.23574	-1152.14518798105\\
60.125	0.2394	-1203.5297754478\\
60.125	0.24306	-1256.13398051534\\
60.125	0.24672	-1309.95780318365\\
60.125	0.25038	-1365.00124345275\\
60.125	0.25404	-1421.26430132263\\
60.125	0.2577	-1478.74697679328\\
60.125	0.26136	-1537.44926986472\\
60.125	0.26502	-1597.37118053694\\
60.125	0.26868	-1658.51270880994\\
60.125	0.27234	-1720.87385468373\\
60.125	0.276	-1784.45461815829\\
60.5	0.093	-98.8036810492002\\
60.5	0.09666	-102.652684563337\\
60.5	0.10032	-107.721305678255\\
60.5	0.10398	-114.009544393953\\
60.5	0.10764	-121.517400710433\\
60.5	0.1113	-130.244874627694\\
60.5	0.11496	-140.191966145735\\
60.5	0.11862	-151.358675264557\\
60.5	0.12228	-163.74500198416\\
60.5	0.12594	-177.350946304545\\
60.5	0.1296	-192.17650822571\\
60.5	0.13326	-208.221687747656\\
60.5	0.13692	-225.486484870382\\
60.5	0.14058	-243.97089959389\\
60.5	0.14424	-263.674931918179\\
60.5	0.1479	-284.598581843248\\
60.5	0.15156	-306.741849369099\\
60.5	0.15522	-330.10473449573\\
60.5	0.15888	-354.687237223143\\
60.5	0.16254	-380.489357551336\\
60.5	0.1662	-407.511095480311\\
60.5	0.16986	-435.752451010066\\
60.5	0.17352	-465.213424140602\\
60.5	0.17718	-495.894014871918\\
60.5	0.18084	-527.794223204016\\
60.5	0.1845	-560.914049136895\\
60.5	0.18816	-595.253492670555\\
60.5	0.19182	-630.812553804995\\
60.5	0.19548	-667.591232540217\\
60.5	0.19914	-705.589528876219\\
60.5	0.2028	-744.807442813003\\
60.5	0.20646	-785.244974350567\\
60.5	0.21012	-826.902123488912\\
60.5	0.21378	-869.778890228038\\
60.5	0.21744	-913.875274567944\\
60.5	0.2211	-959.191276508633\\
60.5	0.22476	-1005.7268960501\\
60.5	0.22842	-1053.48213319235\\
60.5	0.23208	-1102.45698793538\\
60.5	0.23574	-1152.65146027919\\
60.5	0.2394	-1204.06555022379\\
60.5	0.24306	-1256.69925776916\\
60.5	0.24672	-1310.55258291531\\
60.5	0.25038	-1365.62552566225\\
60.5	0.25404	-1421.91808600996\\
60.5	0.2577	-1479.43026395846\\
60.5	0.26136	-1538.16205950774\\
60.5	0.26502	-1598.1134726578\\
60.5	0.26868	-1659.28450340864\\
60.5	0.27234	-1721.67515176026\\
60.5	0.276	-1785.28541771266\\
60.875	0.093	-98.2979603187528\\
60.875	0.09666	-102.176466310728\\
60.875	0.10032	-107.274589903485\\
60.875	0.10398	-113.592331097022\\
60.875	0.10764	-121.129689891341\\
60.875	0.1113	-129.88666628644\\
60.875	0.11496	-139.86326028232\\
60.875	0.11862	-151.059471878981\\
60.875	0.12228	-163.475301076423\\
60.875	0.12594	-177.110747874646\\
60.875	0.1296	-191.96581227365\\
60.875	0.13326	-208.040494273434\\
60.875	0.13692	-225.334793874\\
60.875	0.14058	-243.848711075346\\
60.875	0.14424	-263.582245877475\\
60.875	0.1479	-284.535398280383\\
60.875	0.15156	-306.708168284072\\
60.875	0.15522	-330.100555888542\\
60.875	0.15888	-354.712561093794\\
60.875	0.16254	-380.544183899826\\
60.875	0.1662	-407.595424306639\\
60.875	0.16986	-435.866282314232\\
60.875	0.17352	-465.356757922607\\
60.875	0.17718	-496.066851131763\\
60.875	0.18084	-527.996561941699\\
60.875	0.1845	-561.145890352417\\
60.875	0.18816	-595.514836363915\\
60.875	0.19182	-631.103399976195\\
60.875	0.19548	-667.911581189255\\
60.875	0.19914	-705.939380003096\\
60.875	0.2028	-745.186796417718\\
60.875	0.20646	-785.653830433122\\
60.875	0.21012	-827.340482049305\\
60.875	0.21378	-870.24675126627\\
60.875	0.21744	-914.372638084015\\
60.875	0.2211	-959.718142502542\\
60.875	0.22476	-1006.28326452185\\
60.875	0.22842	-1054.06800414194\\
60.875	0.23208	-1103.07236136281\\
60.875	0.23574	-1153.29633618446\\
60.875	0.2394	-1204.73992860689\\
60.875	0.24306	-1257.4031386301\\
60.875	0.24672	-1311.28596625409\\
60.875	0.25038	-1366.38841147887\\
60.875	0.25404	-1422.71047430442\\
60.875	0.2577	-1480.25215473076\\
60.875	0.26136	-1539.01345275788\\
60.875	0.26502	-1598.99436838577\\
60.875	0.26868	-1660.19490161445\\
60.875	0.27234	-1722.61505244391\\
60.875	0.276	-1786.25482087415\\
61.25	0.093	-97.9308431954249\\
61.25	0.09666	-101.838851665239\\
61.25	0.10032	-106.966477735835\\
61.25	0.10398	-113.31372140721\\
61.25	0.10764	-120.880582679368\\
61.25	0.1113	-129.667061552306\\
61.25	0.11496	-139.673158026025\\
61.25	0.11862	-150.898872100525\\
61.25	0.12228	-163.344203775805\\
61.25	0.12594	-177.009153051867\\
61.25	0.1296	-191.89371992871\\
61.25	0.13326	-207.997904406333\\
61.25	0.13692	-225.321706484738\\
61.25	0.14058	-243.865126163923\\
61.25	0.14424	-263.628163443889\\
61.25	0.1479	-284.610818324637\\
61.25	0.15156	-306.813090806164\\
61.25	0.15522	-330.234980888474\\
61.25	0.15888	-354.876488571563\\
61.25	0.16254	-380.737613855435\\
61.25	0.1662	-407.818356740086\\
61.25	0.16986	-436.118717225519\\
61.25	0.17352	-465.638695311732\\
61.25	0.17718	-496.378290998727\\
61.25	0.18084	-528.337504286502\\
61.25	0.1845	-561.516335175059\\
61.25	0.18816	-595.914783664396\\
61.25	0.19182	-631.532849754514\\
61.25	0.19548	-668.370533445413\\
61.25	0.19914	-706.427834737093\\
61.25	0.2028	-745.704753629554\\
61.25	0.20646	-786.201290122795\\
61.25	0.21012	-827.917444216818\\
61.25	0.21378	-870.853215911621\\
61.25	0.21744	-915.008605207206\\
61.25	0.2211	-960.383612103571\\
61.25	0.22476	-1006.97823660072\\
61.25	0.22842	-1054.79247869865\\
61.25	0.23208	-1103.82633839735\\
61.25	0.23574	-1154.07981569684\\
61.25	0.2394	-1205.55291059711\\
61.25	0.24306	-1258.24562309816\\
61.25	0.24672	-1312.15795319999\\
61.25	0.25038	-1367.28990090261\\
61.25	0.25404	-1423.641466206\\
61.25	0.2577	-1481.21264911018\\
61.25	0.26136	-1540.00344961513\\
61.25	0.26502	-1600.01386772087\\
61.25	0.26868	-1661.24390342738\\
61.25	0.27234	-1723.69355673468\\
61.25	0.276	-1787.36282764276\\
61.625	0.093	-97.7023296792157\\
61.625	0.09666	-101.639840626868\\
61.625	0.10032	-106.796969175303\\
61.625	0.10398	-113.173715324518\\
61.625	0.10764	-120.770079074514\\
61.625	0.1113	-129.586060425291\\
61.625	0.11496	-139.621659376848\\
61.625	0.11862	-150.876875929187\\
61.625	0.12228	-163.351710082306\\
61.625	0.12594	-177.046161836207\\
61.625	0.1296	-191.960231190888\\
61.625	0.13326	-208.09391814635\\
61.625	0.13692	-225.447222702594\\
61.625	0.14058	-244.020144859618\\
61.625	0.14424	-263.812684617423\\
61.625	0.1479	-284.824841976009\\
61.625	0.15156	-307.056616935375\\
61.625	0.15522	-330.508009495524\\
61.625	0.15888	-355.179019656452\\
61.625	0.16254	-381.069647418162\\
61.625	0.1662	-408.179892780652\\
61.625	0.16986	-436.509755743924\\
61.625	0.17352	-466.059236307976\\
61.625	0.17718	-496.82833447281\\
61.625	0.18084	-528.817050238423\\
61.625	0.1845	-562.025383604818\\
61.625	0.18816	-596.453334571994\\
61.625	0.19182	-632.100903139951\\
61.625	0.19548	-668.968089308689\\
61.625	0.19914	-707.054893078208\\
61.625	0.2028	-746.361314448508\\
61.625	0.20646	-786.887353419588\\
61.625	0.21012	-828.63300999145\\
61.625	0.21378	-871.598284164091\\
61.625	0.21744	-915.783175937515\\
61.625	0.2211	-961.187685311718\\
61.625	0.22476	-1007.8118122867\\
61.625	0.22842	-1055.65555686247\\
61.625	0.23208	-1104.71891903902\\
61.625	0.23574	-1155.00189881635\\
61.625	0.2394	-1206.50449619445\\
61.625	0.24306	-1259.22671117334\\
61.625	0.24672	-1313.16854375301\\
61.625	0.25038	-1368.32999393347\\
61.625	0.25404	-1424.7110617147\\
61.625	0.2577	-1482.31174709671\\
61.625	0.26136	-1541.13205007951\\
61.625	0.26502	-1601.17197066308\\
61.625	0.26868	-1662.43150884744\\
61.625	0.27234	-1724.91066463257\\
61.625	0.276	-1788.60943801849\\
62	0.093	-97.6124197701259\\
62	0.09666	-101.579433195618\\
62	0.10032	-106.766064221891\\
62	0.10398	-113.172312848944\\
62	0.10764	-120.798179076779\\
62	0.1113	-129.643662905395\\
62	0.11496	-139.708764334791\\
62	0.11862	-150.993483364968\\
62	0.12228	-163.497819995927\\
62	0.12594	-177.221774227666\\
62	0.1296	-192.165346060186\\
62	0.13326	-208.328535493487\\
62	0.13692	-225.711342527569\\
62	0.14058	-244.313767162432\\
62	0.14424	-264.135809398076\\
62	0.1479	-285.1774692345\\
62	0.15156	-307.438746671706\\
62	0.15522	-330.919641709693\\
62	0.15888	-355.62015434846\\
62	0.16254	-381.540284588008\\
62	0.1662	-408.680032428338\\
62	0.16986	-437.039397869448\\
62	0.17352	-466.618380911339\\
62	0.17718	-497.416981554011\\
62	0.18084	-529.435199797464\\
62	0.1845	-562.673035641698\\
62	0.18816	-597.130489086713\\
62	0.19182	-632.807560132508\\
62	0.19548	-669.704248779085\\
62	0.19914	-707.820555026442\\
62	0.2028	-747.15647887458\\
62	0.20646	-787.7120203235\\
62	0.21012	-829.4871793732\\
62	0.21378	-872.481956023681\\
62	0.21744	-916.696350274943\\
62	0.2211	-962.130362126986\\
62	0.22476	-1008.78399157981\\
62	0.22842	-1056.65723863342\\
62	0.23208	-1105.7501032878\\
62	0.23574	-1156.06258554297\\
62	0.2394	-1207.59468539892\\
62	0.24306	-1260.34640285564\\
62	0.24672	-1314.31773791315\\
62	0.25038	-1369.50869057144\\
62	0.25404	-1425.91926083051\\
62	0.2577	-1483.54944869037\\
62	0.26136	-1542.399254151\\
62	0.26502	-1602.46867721241\\
62	0.26868	-1663.75771787461\\
62	0.27234	-1726.26637613758\\
62	0.276	-1789.99465200134\\
62.375	0.093	-97.6611134681548\\
62.375	0.09666	-101.657629371485\\
62.375	0.10032	-106.873762875597\\
62.375	0.10398	-113.309513980489\\
62.375	0.10764	-120.964882686163\\
62.375	0.1113	-129.839868992617\\
62.375	0.11496	-139.934472899853\\
62.375	0.11862	-151.248694407869\\
62.375	0.12228	-163.782533516666\\
62.375	0.12594	-177.535990226244\\
62.375	0.1296	-192.509064536603\\
62.375	0.13326	-208.701756447742\\
62.375	0.13692	-226.114065959663\\
62.375	0.14058	-244.745993072365\\
62.375	0.14424	-264.597537785848\\
62.375	0.1479	-285.668700100111\\
62.375	0.15156	-307.959480015156\\
62.375	0.15522	-331.46987753098\\
62.375	0.15888	-356.199892647587\\
62.375	0.16254	-382.149525364974\\
62.375	0.1662	-409.318775683143\\
62.375	0.16986	-437.707643602091\\
62.375	0.17352	-467.316129121821\\
62.375	0.17718	-498.144232242332\\
62.375	0.18084	-530.191952963623\\
62.375	0.1845	-563.459291285696\\
62.375	0.18816	-597.946247208549\\
62.375	0.19182	-633.652820732184\\
62.375	0.19548	-670.579011856599\\
62.375	0.19914	-708.724820581796\\
62.375	0.2028	-748.090246907773\\
62.375	0.20646	-788.675290834531\\
62.375	0.21012	-830.47995236207\\
62.375	0.21378	-873.50423149039\\
62.375	0.21744	-917.74812821949\\
62.375	0.2211	-963.211642549372\\
62.375	0.22476	-1009.89477448003\\
62.375	0.22842	-1057.79752401148\\
62.375	0.23208	-1106.9198911437\\
62.375	0.23574	-1157.26187587671\\
62.375	0.2394	-1208.8234782105\\
62.375	0.24306	-1261.60469814506\\
62.375	0.24672	-1315.60553568041\\
62.375	0.25038	-1370.82599081654\\
62.375	0.25404	-1427.26606355345\\
62.375	0.2577	-1484.92575389114\\
62.375	0.26136	-1543.80506182961\\
62.375	0.26502	-1603.90398736886\\
62.375	0.26868	-1665.2225305089\\
62.375	0.27234	-1727.76069124971\\
62.375	0.276	-1791.51846959131\\
62.75	0.093	-97.848410773303\\
62.75	0.09666	-101.874429154473\\
62.75	0.10032	-107.120065136423\\
62.75	0.10398	-113.585318719155\\
62.75	0.10764	-121.270189902667\\
62.75	0.1113	-130.17467868696\\
62.75	0.11496	-140.298785072034\\
62.75	0.11862	-151.642509057889\\
62.75	0.12228	-164.205850644525\\
62.75	0.12594	-177.988809831941\\
62.75	0.1296	-192.991386620139\\
62.75	0.13326	-209.213581009118\\
62.75	0.13692	-226.655392998877\\
62.75	0.14058	-245.316822589418\\
62.75	0.14424	-265.197869780739\\
62.75	0.1479	-286.298534572841\\
62.75	0.15156	-308.618816965724\\
62.75	0.15522	-332.158716959389\\
62.75	0.15888	-356.918234553834\\
62.75	0.16254	-382.89736974906\\
62.75	0.1662	-410.096122545066\\
62.75	0.16986	-438.514492941855\\
62.75	0.17352	-468.152480939422\\
62.75	0.17718	-499.010086537772\\
62.75	0.18084	-531.087309736902\\
62.75	0.1845	-564.384150536814\\
62.75	0.18816	-598.900608937506\\
62.75	0.19182	-634.636684938979\\
62.75	0.19548	-671.592378541233\\
62.75	0.19914	-709.767689744269\\
62.75	0.2028	-749.162618548085\\
62.75	0.20646	-789.777164952681\\
62.75	0.21012	-831.611328958059\\
62.75	0.21378	-874.665110564217\\
62.75	0.21744	-918.938509771157\\
62.75	0.2211	-964.431526578877\\
62.75	0.22476	-1011.14416098738\\
62.75	0.22842	-1059.07641299666\\
62.75	0.23208	-1108.22828260672\\
62.75	0.23574	-1158.59976981757\\
62.75	0.2394	-1210.19087462919\\
62.75	0.24306	-1263.0015970416\\
62.75	0.24672	-1317.03193705479\\
62.75	0.25038	-1372.28189466876\\
62.75	0.25404	-1428.7514698835\\
62.75	0.2577	-1486.44066269903\\
62.75	0.26136	-1545.34947311534\\
62.75	0.26502	-1605.47790113244\\
62.75	0.26868	-1666.82594675031\\
62.75	0.27234	-1729.39360996896\\
62.75	0.276	-1793.18089078839\\
63.125	0.093	-98.1743116855703\\
63.125	0.09666	-102.229832544579\\
63.125	0.10032	-107.504971004368\\
63.125	0.10398	-113.999727064938\\
63.125	0.10764	-121.714100726289\\
63.125	0.1113	-130.648091988421\\
63.125	0.11496	-140.801700851334\\
63.125	0.11862	-152.174927315027\\
63.125	0.12228	-164.767771379502\\
63.125	0.12594	-178.580233044758\\
63.125	0.1296	-193.612312310794\\
63.125	0.13326	-209.864009177611\\
63.125	0.13692	-227.335323645209\\
63.125	0.14058	-246.026255713589\\
63.125	0.14424	-265.936805382749\\
63.125	0.1479	-287.06697265269\\
63.125	0.15156	-309.416757523412\\
63.125	0.15522	-332.986159994915\\
63.125	0.15888	-357.775180067199\\
63.125	0.16254	-383.783817740264\\
63.125	0.1662	-411.012073014109\\
63.125	0.16986	-439.459945888736\\
63.125	0.17352	-469.127436364143\\
63.125	0.17718	-500.014544440331\\
63.125	0.18084	-532.1212701173\\
63.125	0.1845	-565.447613395051\\
63.125	0.18816	-599.993574273582\\
63.125	0.19182	-635.759152752893\\
63.125	0.19548	-672.744348832987\\
63.125	0.19914	-710.94916251386\\
63.125	0.2028	-750.373593795515\\
63.125	0.20646	-791.01764267795\\
63.125	0.21012	-832.881309161168\\
63.125	0.21378	-875.964593245164\\
63.125	0.21744	-920.267494929943\\
63.125	0.2211	-965.790014215502\\
63.125	0.22476	-1012.53215110184\\
63.125	0.22842	-1060.49390558896\\
63.125	0.23208	-1109.67527767687\\
63.125	0.23574	-1160.07626736555\\
63.125	0.2394	-1211.69687465501\\
63.125	0.24306	-1264.53709954526\\
63.125	0.24672	-1318.59694203628\\
63.125	0.25038	-1373.87640212809\\
63.125	0.25404	-1430.37547982068\\
63.125	0.2577	-1488.09417511405\\
63.125	0.26136	-1547.03248800819\\
63.125	0.26502	-1607.19041850313\\
63.125	0.26868	-1668.56796659884\\
63.125	0.27234	-1731.16513229533\\
63.125	0.276	-1794.9819155926\\
63.5	0.093	-98.6388162049566\\
63.5	0.09666	-102.723839541804\\
63.5	0.10032	-108.028480479432\\
63.5	0.10398	-114.552739017841\\
63.5	0.10764	-122.29661515703\\
63.5	0.1113	-131.260108897001\\
63.5	0.11496	-141.443220237753\\
63.5	0.11862	-152.845949179285\\
63.5	0.12228	-165.468295721598\\
63.5	0.12594	-179.310259864693\\
63.5	0.1296	-194.371841608568\\
63.5	0.13326	-210.653040953224\\
63.5	0.13692	-228.153857898661\\
63.5	0.14058	-246.874292444879\\
63.5	0.14424	-266.814344591878\\
63.5	0.1479	-287.974014339658\\
63.5	0.15156	-310.353301688218\\
63.5	0.15522	-333.952206637561\\
63.5	0.15888	-358.770729187683\\
63.5	0.16254	-384.808869338587\\
63.5	0.1662	-412.066627090271\\
63.5	0.16986	-440.544002442736\\
63.5	0.17352	-470.240995395982\\
63.5	0.17718	-501.157605950009\\
63.5	0.18084	-533.293834104817\\
63.5	0.1845	-566.649679860406\\
63.5	0.18816	-601.225143216776\\
63.5	0.19182	-637.020224173927\\
63.5	0.19548	-674.034922731858\\
63.5	0.19914	-712.269238890571\\
63.5	0.2028	-751.723172650065\\
63.5	0.20646	-792.396724010339\\
63.5	0.21012	-834.289892971394\\
63.5	0.21378	-877.40267953323\\
63.5	0.21744	-921.735083695848\\
63.5	0.2211	-967.287105459245\\
63.5	0.22476	-1014.05874482342\\
63.5	0.22842	-1062.05000178838\\
63.5	0.23208	-1111.26087635412\\
63.5	0.23574	-1161.69136852065\\
63.5	0.2394	-1213.34147828795\\
63.5	0.24306	-1266.21120565603\\
63.5	0.24672	-1320.3005506249\\
63.5	0.25038	-1375.60951319454\\
63.5	0.25404	-1432.13809336497\\
63.5	0.2577	-1489.88629113618\\
63.5	0.26136	-1548.85410650816\\
63.5	0.26502	-1609.04153948093\\
63.5	0.26868	-1670.44859005448\\
63.5	0.27234	-1733.07525822881\\
63.5	0.276	-1796.92154400393\\
63.875	0.093	-99.2419243314626\\
63.875	0.09666	-103.356450146148\\
63.875	0.10032	-108.690593561615\\
63.875	0.10398	-115.244354577863\\
63.875	0.10764	-123.017733194891\\
63.875	0.1113	-132.010729412701\\
63.875	0.11496	-142.223343231291\\
63.875	0.11862	-153.655574650662\\
63.875	0.12228	-166.307423670815\\
63.875	0.12594	-180.178890291748\\
63.875	0.1296	-195.269974513462\\
63.875	0.13326	-211.580676335956\\
63.875	0.13692	-229.110995759232\\
63.875	0.14058	-247.860932783289\\
63.875	0.14424	-267.830487408127\\
63.875	0.1479	-289.019659633745\\
63.875	0.15156	-311.428449460145\\
63.875	0.15522	-335.056856887325\\
63.875	0.15888	-359.904881915287\\
63.875	0.16254	-385.972524544029\\
63.875	0.1662	-413.259784773552\\
63.875	0.16986	-441.766662603856\\
63.875	0.17352	-471.493158034941\\
63.875	0.17718	-502.439271066807\\
63.875	0.18084	-534.605001699454\\
63.875	0.1845	-567.990349932881\\
63.875	0.18816	-602.59531576709\\
63.875	0.19182	-638.419899202079\\
63.875	0.19548	-675.46410023785\\
63.875	0.19914	-713.727918874401\\
63.875	0.2028	-753.211355111733\\
63.875	0.20646	-793.914408949847\\
63.875	0.21012	-835.837080388741\\
63.875	0.21378	-878.979369428416\\
63.875	0.21744	-923.341276068871\\
63.875	0.2211	-968.922800310108\\
63.875	0.22476	-1015.72394215213\\
63.875	0.22842	-1063.74470159493\\
63.875	0.23208	-1112.9850786385\\
63.875	0.23574	-1163.44507328286\\
63.875	0.2394	-1215.12468552801\\
63.875	0.24306	-1268.02391537393\\
63.875	0.24672	-1322.14276282063\\
63.875	0.25038	-1377.48122786812\\
63.875	0.25404	-1434.03931051638\\
63.875	0.2577	-1491.81701076543\\
63.875	0.26136	-1550.81432861525\\
63.875	0.26502	-1611.03126406586\\
63.875	0.26868	-1672.46781711725\\
63.875	0.27234	-1735.12398776942\\
63.875	0.276	-1798.99977602237\\
64.25	0.093	-99.9836360650873\\
64.25	0.09666	-104.127664357612\\
64.25	0.10032	-109.491310250918\\
64.25	0.10398	-116.074573745004\\
64.25	0.10764	-123.877454839871\\
64.25	0.1113	-132.89995353552\\
64.25	0.11496	-143.142069831948\\
64.25	0.11862	-154.603803729158\\
64.25	0.12228	-167.28515522715\\
64.25	0.12594	-181.186124325921\\
64.25	0.1296	-196.306711025474\\
64.25	0.13326	-212.646915325808\\
64.25	0.13692	-230.206737226923\\
64.25	0.14058	-248.986176728817\\
64.25	0.14424	-268.985233831494\\
64.25	0.1479	-290.203908534951\\
64.25	0.15156	-312.64220083919\\
64.25	0.15522	-336.300110744209\\
64.25	0.15888	-361.177638250009\\
64.25	0.16254	-387.27478335659\\
64.25	0.1662	-414.591546063952\\
64.25	0.16986	-443.127926372095\\
64.25	0.17352	-472.883924281019\\
64.25	0.17718	-503.859539790723\\
64.25	0.18084	-536.054772901209\\
64.25	0.1845	-569.469623612475\\
64.25	0.18816	-604.104091924523\\
64.25	0.19182	-639.958177837351\\
64.25	0.19548	-677.03188135096\\
64.25	0.19914	-715.32520246535\\
64.25	0.2028	-754.838141180521\\
64.25	0.20646	-795.570697496474\\
64.25	0.21012	-837.522871413206\\
64.25	0.21378	-880.69466293072\\
64.25	0.21744	-925.086072049014\\
64.25	0.2211	-970.69709876809\\
64.25	0.22476	-1017.52774308795\\
64.25	0.22842	-1065.57800500858\\
64.25	0.23208	-1114.84788453\\
64.25	0.23574	-1165.3373816522\\
64.25	0.2394	-1217.04649637518\\
64.25	0.24306	-1269.97522869894\\
64.25	0.24672	-1324.12357862348\\
64.25	0.25038	-1379.49154614881\\
64.25	0.25404	-1436.07913127491\\
64.25	0.2577	-1493.8863340018\\
64.25	0.26136	-1552.91315432946\\
64.25	0.26502	-1613.15959225791\\
64.25	0.26868	-1674.62564778714\\
64.25	0.27234	-1737.31132091714\\
64.25	0.276	-1801.21661164793\\
64.625	0.093	-100.863951405831\\
64.625	0.09666	-105.037482176195\\
64.625	0.10032	-110.430630547339\\
64.625	0.10398	-117.043396519264\\
64.625	0.10764	-124.87578009197\\
64.625	0.1113	-133.927781265457\\
64.625	0.11496	-144.199400039726\\
64.625	0.11862	-155.690636414774\\
64.625	0.12228	-168.401490390604\\
64.625	0.12594	-182.331961967215\\
64.625	0.1296	-197.482051144606\\
64.625	0.13326	-213.851757922778\\
64.625	0.13692	-231.441082301732\\
64.625	0.14058	-250.250024281466\\
64.625	0.14424	-270.278583861981\\
64.625	0.1479	-291.526761043277\\
64.625	0.15156	-313.994555825354\\
64.625	0.15522	-337.681968208213\\
64.625	0.15888	-362.588998191851\\
64.625	0.16254	-388.715645776271\\
64.625	0.1662	-416.061910961472\\
64.625	0.16986	-444.627793747454\\
64.625	0.17352	-474.413294134215\\
64.625	0.17718	-505.41841212176\\
64.625	0.18084	-537.643147710083\\
64.625	0.1845	-571.087500899189\\
64.625	0.18816	-605.751471689075\\
64.625	0.19182	-641.635060079742\\
64.625	0.19548	-678.73826607119\\
64.625	0.19914	-717.061089663419\\
64.625	0.2028	-756.603530856429\\
64.625	0.20646	-797.365589650219\\
64.625	0.21012	-839.347266044791\\
64.625	0.21378	-882.548560040143\\
64.625	0.21744	-926.969471636277\\
64.625	0.2211	-972.610000833191\\
64.625	0.22476	-1019.47014763089\\
64.625	0.22842	-1067.54991202936\\
64.625	0.23208	-1116.84929402862\\
64.625	0.23574	-1167.36829362866\\
64.625	0.2394	-1219.10691082948\\
64.625	0.24306	-1272.06514563108\\
64.625	0.24672	-1326.24299803346\\
64.625	0.25038	-1381.64046803662\\
64.625	0.25404	-1438.25755564056\\
64.625	0.2577	-1496.09426084529\\
64.625	0.26136	-1555.15058365079\\
64.625	0.26502	-1615.42652405707\\
64.625	0.26868	-1676.92208206414\\
64.625	0.27234	-1739.63725767199\\
64.625	0.276	-1803.57205088062\\
65	0.093	-101.882870353695\\
65	0.09666	-106.085903601897\\
65	0.10032	-111.50855445088\\
65	0.10398	-118.150822900644\\
65	0.10764	-126.012708951189\\
65	0.1113	-135.094212602515\\
65	0.11496	-145.395333854621\\
65	0.11862	-156.916072707509\\
65	0.12228	-169.656429161177\\
65	0.12594	-183.616403215627\\
65	0.1296	-198.795994870857\\
65	0.13326	-215.195204126868\\
65	0.13692	-232.81403098366\\
65	0.14058	-251.652475441233\\
65	0.14424	-271.710537499588\\
65	0.1479	-292.988217158722\\
65	0.15156	-315.485514418638\\
65	0.15522	-339.202429279334\\
65	0.15888	-364.138961740812\\
65	0.16254	-390.295111803071\\
65	0.1662	-417.670879466111\\
65	0.16986	-446.266264729931\\
65	0.17352	-476.081267594532\\
65	0.17718	-507.115888059914\\
65	0.18084	-539.370126126077\\
65	0.1845	-572.843981793021\\
65	0.18816	-607.537455060746\\
65	0.19182	-643.450545929252\\
65	0.19548	-680.583254398539\\
65	0.19914	-718.935580468606\\
65	0.2028	-758.507524139455\\
65	0.20646	-799.299085411085\\
65	0.21012	-841.310264283495\\
65	0.21378	-884.541060756686\\
65	0.21744	-928.991474830658\\
65	0.2211	-974.661506505412\\
65	0.22476	-1021.55115578095\\
65	0.22842	-1069.66042265726\\
65	0.23208	-1118.98930713436\\
65	0.23574	-1169.53780921223\\
65	0.2394	-1221.30592889089\\
65	0.24306	-1274.29366617033\\
65	0.24672	-1328.50102105055\\
65	0.25038	-1383.92799353155\\
65	0.25404	-1440.57458361333\\
65	0.2577	-1498.44079129589\\
65	0.26136	-1557.52661657924\\
65	0.26502	-1617.83205946336\\
65	0.26868	-1679.35711994827\\
65	0.27234	-1742.10179803395\\
65	0.276	-1806.06609372042\\
65.375	0.093	-103.040392908677\\
65.375	0.09666	-107.272928634718\\
65.375	0.10032	-112.725081961539\\
65.375	0.10398	-119.396852889142\\
65.375	0.10764	-127.288241417526\\
65.375	0.1113	-136.39924754669\\
65.375	0.11496	-146.729871276636\\
65.375	0.11862	-158.280112607362\\
65.375	0.12228	-171.049971538869\\
65.375	0.12594	-185.039448071157\\
65.375	0.1296	-200.248542204227\\
65.375	0.13326	-216.677253938076\\
65.375	0.13692	-234.325583272707\\
65.375	0.14058	-253.193530208119\\
65.375	0.14424	-273.281094744312\\
65.375	0.1479	-294.588276881285\\
65.375	0.15156	-317.11507661904\\
65.375	0.15522	-340.861493957575\\
65.375	0.15888	-365.827528896892\\
65.375	0.16254	-392.013181436989\\
65.375	0.1662	-419.418451577868\\
65.375	0.16986	-448.043339319527\\
65.375	0.17352	-477.887844661967\\
65.375	0.17718	-508.951967605188\\
65.375	0.18084	-541.23570814919\\
65.375	0.1845	-574.739066293972\\
65.375	0.18816	-609.462042039536\\
65.375	0.19182	-645.404635385881\\
65.375	0.19548	-682.566846333006\\
65.375	0.19914	-720.948674880912\\
65.375	0.2028	-760.5501210296\\
65.375	0.20646	-801.371184779068\\
65.375	0.21012	-843.411866129317\\
65.375	0.21378	-886.672165080347\\
65.375	0.21744	-931.152081632158\\
65.375	0.2211	-976.85161578475\\
65.375	0.22476	-1023.77076753812\\
65.375	0.22842	-1071.90953689228\\
65.375	0.23208	-1121.26792384721\\
65.375	0.23574	-1171.84592840293\\
65.375	0.2394	-1223.64355055942\\
65.375	0.24306	-1276.6607903167\\
65.375	0.24672	-1330.89764767476\\
65.375	0.25038	-1386.3541226336\\
65.375	0.25404	-1443.03021519322\\
65.375	0.2577	-1500.92592535362\\
65.375	0.26136	-1560.0412531148\\
65.375	0.26502	-1620.37619847676\\
65.375	0.26868	-1681.93076143951\\
65.375	0.27234	-1744.70494200303\\
65.375	0.276	-1808.69874016734\\
65.75	0.093	-104.336519070778\\
65.75	0.09666	-108.598557274657\\
65.75	0.10032	-114.080213079318\\
65.75	0.10398	-120.78148648476\\
65.75	0.10764	-128.702377490982\\
65.75	0.1113	-137.842886097985\\
65.75	0.11496	-148.20301230577\\
65.75	0.11862	-159.782756114334\\
65.75	0.12228	-172.582117523681\\
65.75	0.12594	-186.601096533808\\
65.75	0.1296	-201.839693144715\\
65.75	0.13326	-218.297907356404\\
65.75	0.13692	-235.975739168874\\
65.75	0.14058	-254.873188582124\\
65.75	0.14424	-274.990255596156\\
65.75	0.1479	-296.326940210968\\
65.75	0.15156	-318.883242426562\\
65.75	0.15522	-342.659162242936\\
65.75	0.15888	-367.654699660091\\
65.75	0.16254	-393.869854678027\\
65.75	0.1662	-421.304627296745\\
65.75	0.16986	-449.959017516242\\
65.75	0.17352	-479.833025336521\\
65.75	0.17718	-510.926650757581\\
65.75	0.18084	-543.239893779421\\
65.75	0.1845	-576.772754402043\\
65.75	0.18816	-611.525232625446\\
65.75	0.19182	-647.497328449629\\
65.75	0.19548	-684.689041874593\\
65.75	0.19914	-723.100372900338\\
65.75	0.2028	-762.731321526864\\
65.75	0.20646	-803.581887754172\\
65.75	0.21012	-845.652071582259\\
65.75	0.21378	-888.941873011128\\
65.75	0.21744	-933.451292040777\\
65.75	0.2211	-979.180328671209\\
65.75	0.22476	-1026.12898290242\\
65.75	0.22842	-1074.29725473441\\
65.75	0.23208	-1123.68514416719\\
65.75	0.23574	-1174.29265120074\\
65.75	0.2394	-1226.11977583508\\
65.75	0.24306	-1279.16651807019\\
65.75	0.24672	-1333.43287790609\\
65.75	0.25038	-1388.91885534277\\
65.75	0.25404	-1445.62445038023\\
65.75	0.2577	-1503.54966301847\\
65.75	0.26136	-1562.69449325749\\
65.75	0.26502	-1623.05894109729\\
65.75	0.26868	-1684.64300653787\\
65.75	0.27234	-1747.44668957923\\
65.75	0.276	-1811.46999022138\\
66.125	0.093	-105.771248839998\\
66.125	0.09666	-110.062789521716\\
66.125	0.10032	-115.573947804216\\
66.125	0.10398	-122.304723687496\\
66.125	0.10764	-130.255117171557\\
66.125	0.1113	-139.425128256399\\
66.125	0.11496	-149.814756942022\\
66.125	0.11862	-161.424003228426\\
66.125	0.12228	-174.252867115611\\
66.125	0.12594	-188.301348603577\\
66.125	0.1296	-203.569447692323\\
66.125	0.13326	-220.057164381851\\
66.125	0.13692	-237.764498672159\\
66.125	0.14058	-256.691450563249\\
66.125	0.14424	-276.838020055119\\
66.125	0.1479	-298.20420714777\\
66.125	0.15156	-320.790011841202\\
66.125	0.15522	-344.595434135415\\
66.125	0.15888	-369.620474030409\\
66.125	0.16254	-395.865131526184\\
66.125	0.1662	-423.32940662274\\
66.125	0.16986	-452.013299320077\\
66.125	0.17352	-481.916809618194\\
66.125	0.17718	-513.039937517092\\
66.125	0.18084	-545.382683016772\\
66.125	0.1845	-578.945046117232\\
66.125	0.18816	-613.727026818474\\
66.125	0.19182	-649.728625120496\\
66.125	0.19548	-686.949841023299\\
66.125	0.19914	-725.390674526883\\
66.125	0.2028	-765.051125631248\\
66.125	0.20646	-805.931194336394\\
66.125	0.21012	-848.03088064232\\
66.125	0.21378	-891.350184549028\\
66.125	0.21744	-935.889106056516\\
66.125	0.2211	-981.647645164786\\
66.125	0.22476	-1028.62580187384\\
66.125	0.22842	-1076.82357618367\\
66.125	0.23208	-1126.24096809428\\
66.125	0.23574	-1176.87797760567\\
66.125	0.2394	-1228.73460471785\\
66.125	0.24306	-1281.8108494308\\
66.125	0.24672	-1336.10671174454\\
66.125	0.25038	-1391.62219165905\\
66.125	0.25404	-1448.35728917435\\
66.125	0.2577	-1506.31200429043\\
66.125	0.26136	-1565.48633700729\\
66.125	0.26502	-1625.88028732493\\
66.125	0.26868	-1687.49385524335\\
66.125	0.27234	-1750.32704076255\\
66.125	0.276	-1814.37984388254\\
66.5	0.093	-107.344582216338\\
66.5	0.09666	-111.665625375895\\
66.5	0.10032	-117.206286136233\\
66.5	0.10398	-123.966564497352\\
66.5	0.10764	-131.946460459252\\
66.5	0.1113	-141.145974021933\\
66.5	0.11496	-151.565105185394\\
66.5	0.11862	-163.203853949637\\
66.5	0.12228	-176.062220314661\\
66.5	0.12594	-190.140204280465\\
66.5	0.1296	-205.437805847051\\
66.5	0.13326	-221.955025014417\\
66.5	0.13692	-239.691861782564\\
66.5	0.14058	-258.648316151492\\
66.5	0.14424	-278.824388121202\\
66.5	0.1479	-300.220077691691\\
66.5	0.15156	-322.835384862962\\
66.5	0.15522	-346.670309635014\\
66.5	0.15888	-371.724852007847\\
66.5	0.16254	-397.99901198146\\
66.5	0.1662	-425.492789555855\\
66.5	0.16986	-454.20618473103\\
66.5	0.17352	-484.139197506987\\
66.5	0.17718	-515.291827883724\\
66.5	0.18084	-547.664075861242\\
66.5	0.1845	-581.255941439541\\
66.5	0.18816	-616.067424618622\\
66.5	0.19182	-652.098525398482\\
66.5	0.19548	-689.349243779124\\
66.5	0.19914	-727.819579760547\\
66.5	0.2028	-767.509533342751\\
66.5	0.20646	-808.419104525736\\
66.5	0.21012	-850.5482933095\\
66.5	0.21378	-893.897099694047\\
66.5	0.21744	-938.465523679374\\
66.5	0.2211	-984.253565265482\\
66.5	0.22476	-1031.26122445237\\
66.5	0.22842	-1079.48850124004\\
66.5	0.23208	-1128.93539562849\\
66.5	0.23574	-1179.60190761772\\
66.5	0.2394	-1231.48803720774\\
66.5	0.24306	-1284.59378439853\\
66.5	0.24672	-1338.91914919011\\
66.5	0.25038	-1394.46413158246\\
66.5	0.25404	-1451.2287315756\\
66.5	0.2577	-1509.21294916952\\
66.5	0.26136	-1568.41678436421\\
66.5	0.26502	-1628.84023715969\\
66.5	0.26868	-1690.48330755595\\
66.5	0.27234	-1753.34599555299\\
66.5	0.276	-1817.42830115082\\
66.875	0.093	-109.056519199797\\
66.875	0.09666	-113.407064837193\\
66.875	0.10032	-118.97722807537\\
66.875	0.10398	-125.767008914328\\
66.875	0.10764	-133.776407354067\\
66.875	0.1113	-143.005423394586\\
66.875	0.11496	-153.454057035887\\
66.875	0.11862	-165.122308277968\\
66.875	0.12228	-178.01017712083\\
66.875	0.12594	-192.117663564474\\
66.875	0.1296	-207.444767608898\\
66.875	0.13326	-223.991489254103\\
66.875	0.13692	-241.757828500089\\
66.875	0.14058	-260.743785346856\\
66.875	0.14424	-280.949359794404\\
66.875	0.1479	-302.374551842732\\
66.875	0.15156	-325.019361491842\\
66.875	0.15522	-348.883788741733\\
66.875	0.15888	-373.967833592404\\
66.875	0.16254	-400.271496043857\\
66.875	0.1662	-427.79477609609\\
66.875	0.16986	-456.537673749104\\
66.875	0.17352	-486.500189002899\\
66.875	0.17718	-517.682321857475\\
66.875	0.18084	-550.084072312832\\
66.875	0.1845	-583.70544036897\\
66.875	0.18816	-618.546426025889\\
66.875	0.19182	-654.607029283589\\
66.875	0.19548	-691.887250142069\\
66.875	0.19914	-730.387088601331\\
66.875	0.2028	-770.106544661373\\
66.875	0.20646	-811.045618322197\\
66.875	0.21012	-853.204309583801\\
66.875	0.21378	-896.582618446186\\
66.875	0.21744	-941.180544909352\\
66.875	0.2211	-986.998088973299\\
66.875	0.22476	-1034.03525063803\\
66.875	0.22842	-1082.29202990354\\
66.875	0.23208	-1131.76842676983\\
66.875	0.23574	-1182.4644412369\\
66.875	0.2394	-1234.38007330475\\
66.875	0.24306	-1287.51532297338\\
66.875	0.24672	-1341.87019024279\\
66.875	0.25038	-1397.44467511299\\
66.875	0.25404	-1454.23877758396\\
66.875	0.2577	-1512.25249765572\\
66.875	0.26136	-1571.48583532826\\
66.875	0.26502	-1631.93879060157\\
66.875	0.26868	-1693.61136347567\\
66.875	0.27234	-1756.50355395055\\
66.875	0.276	-1820.61536202621\\
67.25	0.093	-110.907059790375\\
67.25	0.09666	-115.28710790561\\
67.25	0.10032	-120.886773621625\\
67.25	0.10398	-127.706056938422\\
67.25	0.10764	-135.744957855999\\
67.25	0.1113	-145.003476374358\\
67.25	0.11496	-155.481612493497\\
67.25	0.11862	-167.179366213417\\
67.25	0.12228	-180.096737534118\\
67.25	0.12594	-194.2337264556\\
67.25	0.1296	-209.590332977863\\
67.25	0.13326	-226.166557100907\\
67.25	0.13692	-243.962398824732\\
67.25	0.14058	-262.977858149337\\
67.25	0.14424	-283.212935074724\\
67.25	0.1479	-304.667629600891\\
67.25	0.15156	-327.34194172784\\
67.25	0.15522	-351.235871455569\\
67.25	0.15888	-376.349418784079\\
67.25	0.16254	-402.682583713371\\
67.25	0.1662	-430.235366243443\\
67.25	0.16986	-459.007766374296\\
67.25	0.17352	-488.99978410593\\
67.25	0.17718	-520.211419438344\\
67.25	0.18084	-552.64267237154\\
67.25	0.1845	-586.293542905517\\
67.25	0.18816	-621.164031040274\\
67.25	0.19182	-657.254136775813\\
67.25	0.19548	-694.563860112132\\
67.25	0.19914	-733.093201049233\\
67.25	0.2028	-772.842159587114\\
67.25	0.20646	-813.810735725777\\
67.25	0.21012	-855.998929465219\\
67.25	0.21378	-899.406740805443\\
67.25	0.21744	-944.034169746447\\
67.25	0.2211	-989.881216288234\\
67.25	0.22476	-1036.9478804308\\
67.25	0.22842	-1085.23416217415\\
67.25	0.23208	-1134.74006151828\\
67.25	0.23574	-1185.46557846319\\
67.25	0.2394	-1237.41071300888\\
67.25	0.24306	-1290.57546515535\\
67.25	0.24672	-1344.9598349026\\
67.25	0.25038	-1400.56382225063\\
67.25	0.25404	-1457.38742719945\\
67.25	0.2577	-1515.43064974904\\
67.25	0.26136	-1574.69348989942\\
67.25	0.26502	-1635.17594765057\\
67.25	0.26868	-1696.87802300251\\
67.25	0.27234	-1759.79971595523\\
67.25	0.276	-1823.94102650873\\
67.625	0.093	-112.896203988071\\
67.625	0.09666	-117.305754581144\\
67.625	0.10032	-122.934922774999\\
67.625	0.10398	-129.783708569635\\
67.625	0.10764	-137.852111965051\\
67.625	0.1113	-147.140132961248\\
67.625	0.11496	-157.647771558226\\
67.625	0.11862	-169.375027755985\\
67.625	0.12228	-182.321901554525\\
67.625	0.12594	-196.488392953846\\
67.625	0.1296	-211.874501953947\\
67.625	0.13326	-228.48022855483\\
67.625	0.13692	-246.305572756494\\
67.625	0.14058	-265.350534558938\\
67.625	0.14424	-285.615113962164\\
67.625	0.1479	-307.099310966169\\
67.625	0.15156	-329.803125570957\\
67.625	0.15522	-353.726557776525\\
67.625	0.15888	-378.869607582874\\
67.625	0.16254	-405.232274990004\\
67.625	0.1662	-432.814559997915\\
67.625	0.16986	-461.616462606607\\
67.625	0.17352	-491.637982816079\\
67.625	0.17718	-522.879120626333\\
67.625	0.18084	-555.339876037368\\
67.625	0.1845	-589.020249049182\\
67.625	0.18816	-623.920239661779\\
67.625	0.19182	-660.039847875157\\
67.625	0.19548	-697.379073689314\\
67.625	0.19914	-735.937917104254\\
67.625	0.2028	-775.716378119974\\
67.625	0.20646	-816.714456736475\\
67.625	0.21012	-858.932152953756\\
67.625	0.21378	-902.369466771819\\
67.625	0.21744	-947.026398190662\\
67.625	0.2211	-992.902947210287\\
67.625	0.22476	-1039.99911383069\\
67.625	0.22842	-1088.31489805188\\
67.625	0.23208	-1137.85029987385\\
67.625	0.23574	-1188.60531929659\\
67.625	0.2394	-1240.57995632012\\
67.625	0.24306	-1293.77421094443\\
67.625	0.24672	-1348.18808316952\\
67.625	0.25038	-1403.8215729954\\
67.625	0.25404	-1460.67468042205\\
67.625	0.2577	-1518.74740544948\\
67.625	0.26136	-1578.0397480777\\
67.625	0.26502	-1638.55170830669\\
67.625	0.26868	-1700.28328613647\\
67.625	0.27234	-1763.23448156703\\
67.625	0.276	-1827.40529459836\\
68	0.093	-115.023951792887\\
68	0.09666	-119.463004863799\\
68	0.10032	-125.121675535493\\
68	0.10398	-131.999963807967\\
68	0.10764	-140.097869681222\\
68	0.1113	-149.415393155258\\
68	0.11496	-159.952534230075\\
68	0.11862	-171.709292905672\\
68	0.12228	-184.685669182051\\
68	0.12594	-198.88166305921\\
68	0.1296	-214.297274537151\\
68	0.13326	-230.932503615873\\
68	0.13692	-248.787350295375\\
68	0.14058	-267.861814575658\\
68	0.14424	-288.155896456722\\
68	0.1479	-309.669595938567\\
68	0.15156	-332.402913021193\\
68	0.15522	-356.3558477046\\
68	0.15888	-381.528399988788\\
68	0.16254	-407.920569873757\\
68	0.1662	-435.532357359506\\
68	0.16986	-464.363762446037\\
68	0.17352	-494.414785133348\\
68	0.17718	-525.68542542144\\
68	0.18084	-558.175683310314\\
68	0.1845	-591.885558799968\\
68	0.18816	-626.815051890403\\
68	0.19182	-662.964162581619\\
68	0.19548	-700.332890873616\\
68	0.19914	-738.921236766394\\
68	0.2028	-778.729200259953\\
68	0.20646	-819.756781354293\\
68	0.21012	-862.003980049413\\
68	0.21378	-905.470796345314\\
68	0.21744	-950.157230241996\\
68	0.2211	-996.06328173946\\
68	0.22476	-1043.1889508377\\
68	0.22842	-1091.53423753673\\
68	0.23208	-1141.09914183654\\
68	0.23574	-1191.88366373712\\
68	0.2394	-1243.88780323849\\
68	0.24306	-1297.11156034064\\
68	0.24672	-1351.55493504357\\
68	0.25038	-1407.21792734728\\
68	0.25404	-1464.10053725177\\
68	0.2577	-1522.20276475704\\
68	0.26136	-1581.5246098631\\
68	0.26502	-1642.06607256993\\
68	0.26868	-1703.82715287755\\
68	0.27234	-1766.80785078594\\
68	0.276	-1831.00816629512\\
68.375	0.093	-117.290303204823\\
68.375	0.09666	-121.758858753574\\
68.375	0.10032	-127.447031903106\\
68.375	0.10398	-134.354822653419\\
68.375	0.10764	-142.482231004513\\
68.375	0.1113	-151.829256956388\\
68.375	0.11496	-162.395900509043\\
68.375	0.11862	-174.182161662479\\
68.375	0.12228	-187.188040416697\\
68.375	0.12594	-201.413536771695\\
68.375	0.1296	-216.858650727475\\
68.375	0.13326	-233.523382284035\\
68.375	0.13692	-251.407731441376\\
68.375	0.14058	-270.511698199498\\
68.375	0.14424	-290.835282558401\\
68.375	0.1479	-312.378484518084\\
68.375	0.15156	-335.141304078549\\
68.375	0.15522	-359.123741239795\\
68.375	0.15888	-384.325796001822\\
68.375	0.16254	-410.747468364629\\
68.375	0.1662	-438.388758328218\\
68.375	0.16986	-467.249665892587\\
68.375	0.17352	-497.330191057737\\
68.375	0.17718	-528.630333823668\\
68.375	0.18084	-561.15009419038\\
68.375	0.1845	-594.889472157873\\
68.375	0.18816	-629.848467726147\\
68.375	0.19182	-666.027080895202\\
68.375	0.19548	-703.425311665037\\
68.375	0.19914	-742.043160035654\\
68.375	0.2028	-781.880626007052\\
68.375	0.20646	-822.93770957923\\
68.375	0.21012	-865.214410752189\\
68.375	0.21378	-908.71072952593\\
68.375	0.21744	-953.42666590045\\
68.375	0.2211	-999.362219875753\\
68.375	0.22476	-1046.51739145184\\
68.375	0.22842	-1094.8921806287\\
68.375	0.23208	-1144.48658740634\\
68.375	0.23574	-1195.30061178477\\
68.375	0.2394	-1247.33425376398\\
68.375	0.24306	-1300.58751334396\\
68.375	0.24672	-1355.06039052473\\
68.375	0.25038	-1410.75288530628\\
68.375	0.25404	-1467.66499768861\\
68.375	0.2577	-1525.79672767172\\
68.375	0.26136	-1585.14807525562\\
68.375	0.26502	-1645.71904044029\\
68.375	0.26868	-1707.50962322574\\
68.375	0.27234	-1770.51982361198\\
68.375	0.276	-1834.74964159899\\
68.75	0.093	-119.695258223877\\
68.75	0.09666	-124.193316250467\\
68.75	0.10032	-129.910991877838\\
68.75	0.10398	-136.848285105989\\
68.75	0.10764	-145.005195934922\\
68.75	0.1113	-154.381724364636\\
68.75	0.11496	-164.97787039513\\
68.75	0.11862	-176.793634026405\\
68.75	0.12228	-189.829015258462\\
68.75	0.12594	-204.084014091299\\
68.75	0.1296	-219.558630524916\\
68.75	0.13326	-236.252864559315\\
68.75	0.13692	-254.166716194495\\
68.75	0.14058	-273.300185430456\\
68.75	0.14424	-293.653272267198\\
68.75	0.1479	-315.22597670472\\
68.75	0.15156	-338.018298743024\\
68.75	0.15522	-362.030238382108\\
68.75	0.15888	-387.261795621974\\
68.75	0.16254	-413.71297046262\\
68.75	0.1662	-441.383762904048\\
68.75	0.16986	-470.274172946255\\
68.75	0.17352	-500.384200589244\\
68.75	0.17718	-531.713845833014\\
68.75	0.18084	-564.263108677565\\
68.75	0.1845	-598.031989122896\\
68.75	0.18816	-633.020487169009\\
68.75	0.19182	-669.228602815903\\
68.75	0.19548	-706.656336063577\\
68.75	0.19914	-745.303686912033\\
68.75	0.2028	-785.170655361269\\
68.75	0.20646	-826.257241411287\\
68.75	0.21012	-868.563445062084\\
68.75	0.21378	-912.089266313663\\
68.75	0.21744	-956.834705166023\\
68.75	0.2211	-1002.79976161916\\
68.75	0.22476	-1049.98443567309\\
68.75	0.22842	-1098.38872732779\\
68.75	0.23208	-1148.01263658327\\
68.75	0.23574	-1198.85616343954\\
68.75	0.2394	-1250.91930789658\\
68.75	0.24306	-1304.20206995441\\
68.75	0.24672	-1358.70444961302\\
68.75	0.25038	-1414.4264468724\\
68.75	0.25404	-1471.36806173257\\
68.75	0.2577	-1529.52929419352\\
68.75	0.26136	-1588.91014425525\\
68.75	0.26502	-1649.51061191777\\
68.75	0.26868	-1711.33069718106\\
68.75	0.27234	-1774.37040004513\\
68.75	0.276	-1838.62972050999\\
69.125	0.093	-122.238816850051\\
69.125	0.09666	-126.766377354479\\
69.125	0.10032	-132.513555459689\\
69.125	0.10398	-139.480351165679\\
69.125	0.10764	-147.66676447245\\
69.125	0.1113	-157.072795380003\\
69.125	0.11496	-167.698443888336\\
69.125	0.11862	-179.54370999745\\
69.125	0.12228	-192.608593707345\\
69.125	0.12594	-206.893095018021\\
69.125	0.1296	-222.397213929478\\
69.125	0.13326	-239.120950441715\\
69.125	0.13692	-257.064304554734\\
69.125	0.14058	-276.227276268533\\
69.125	0.14424	-296.609865583114\\
69.125	0.1479	-318.212072498475\\
69.125	0.15156	-341.033897014618\\
69.125	0.15522	-365.075339131541\\
69.125	0.15888	-390.336398849245\\
69.125	0.16254	-416.81707616773\\
69.125	0.1662	-444.517371086996\\
69.125	0.16986	-473.437283607043\\
69.125	0.17352	-503.576813727871\\
69.125	0.17718	-534.935961449479\\
69.125	0.18084	-567.514726771869\\
69.125	0.1845	-601.313109695039\\
69.125	0.18816	-636.331110218991\\
69.125	0.19182	-672.568728343723\\
69.125	0.19548	-710.025964069236\\
69.125	0.19914	-748.70281739553\\
69.125	0.2028	-788.599288322606\\
69.125	0.20646	-829.715376850462\\
69.125	0.21012	-872.051082979098\\
69.125	0.21378	-915.606406708516\\
69.125	0.21744	-960.381348038714\\
69.125	0.2211	-1006.37590696969\\
69.125	0.22476	-1053.59008350145\\
69.125	0.22842	-1102.023877634\\
69.125	0.23208	-1151.67728936732\\
69.125	0.23574	-1202.55031870142\\
69.125	0.2394	-1254.64296563631\\
69.125	0.24306	-1307.95523017197\\
69.125	0.24672	-1362.48711230842\\
69.125	0.25038	-1418.23861204565\\
69.125	0.25404	-1475.20972938365\\
69.125	0.2577	-1533.40046432244\\
69.125	0.26136	-1592.81081686201\\
69.125	0.26502	-1653.44078700236\\
69.125	0.26868	-1715.29037474349\\
69.125	0.27234	-1778.3595800854\\
69.125	0.276	-1842.6484030281\\
69.5	0.093	-124.920979083343\\
69.5	0.09666	-129.47804206561\\
69.5	0.10032	-135.254722648658\\
69.5	0.10398	-142.251020832488\\
69.5	0.10764	-150.466936617098\\
69.5	0.1113	-159.902470002489\\
69.5	0.11496	-170.557620988661\\
69.5	0.11862	-182.432389575614\\
69.5	0.12228	-195.526775763347\\
69.5	0.12594	-209.840779551862\\
69.5	0.1296	-225.374400941157\\
69.5	0.13326	-242.127639931234\\
69.5	0.13692	-260.100496522091\\
69.5	0.14058	-279.292970713729\\
69.5	0.14424	-299.705062506149\\
69.5	0.1479	-321.336771899349\\
69.5	0.15156	-344.18809889333\\
69.5	0.15522	-368.259043488092\\
69.5	0.15888	-393.549605683635\\
69.5	0.16254	-420.059785479959\\
69.5	0.1662	-447.789582877064\\
69.5	0.16986	-476.738997874949\\
69.5	0.17352	-506.908030473616\\
69.5	0.17718	-538.296680673063\\
69.5	0.18084	-570.904948473291\\
69.5	0.1845	-604.7328338743\\
69.5	0.18816	-639.780336876091\\
69.5	0.19182	-676.047457478662\\
69.5	0.19548	-713.534195682014\\
69.5	0.19914	-752.240551486147\\
69.5	0.2028	-792.166524891061\\
69.5	0.20646	-833.312115896756\\
69.5	0.21012	-875.677324503231\\
69.5	0.21378	-919.262150710488\\
69.5	0.21744	-964.066594518525\\
69.5	0.2211	-1010.09065592734\\
69.5	0.22476	-1057.33433493694\\
69.5	0.22842	-1105.79763154732\\
69.5	0.23208	-1155.48054575848\\
69.5	0.23574	-1206.38307757043\\
69.5	0.2394	-1258.50522698315\\
69.5	0.24306	-1311.84699399665\\
69.5	0.24672	-1366.40837861094\\
69.5	0.25038	-1422.189380826\\
69.5	0.25404	-1479.19000064185\\
69.5	0.2577	-1537.41023805848\\
69.5	0.26136	-1596.85009307589\\
69.5	0.26502	-1657.50956569408\\
69.5	0.26868	-1719.38865591305\\
69.5	0.27234	-1782.4873637328\\
69.5	0.276	-1846.80568915333\\
69.875	0.093	-127.741744923754\\
69.875	0.09666	-132.32831038386\\
69.875	0.10032	-138.134493444747\\
69.875	0.10398	-145.160294106415\\
69.875	0.10764	-153.405712368864\\
69.875	0.1113	-162.870748232094\\
69.875	0.11496	-173.555401696105\\
69.875	0.11862	-185.459672760896\\
69.875	0.12228	-198.583561426469\\
69.875	0.12594	-212.927067692822\\
69.875	0.1296	-228.490191559956\\
69.875	0.13326	-245.272933027872\\
69.875	0.13692	-263.275292096568\\
69.875	0.14058	-282.497268766045\\
69.875	0.14424	-302.938863036303\\
69.875	0.1479	-324.600074907341\\
69.875	0.15156	-347.480904379162\\
69.875	0.15522	-371.581351451762\\
69.875	0.15888	-396.901416125144\\
69.875	0.16254	-423.441098399307\\
69.875	0.1662	-451.200398274251\\
69.875	0.16986	-480.179315749975\\
69.875	0.17352	-510.37785082648\\
69.875	0.17718	-541.796003503766\\
69.875	0.18084	-574.433773781833\\
69.875	0.1845	-608.291161660681\\
69.875	0.18816	-643.36816714031\\
69.875	0.19182	-679.66479022072\\
69.875	0.19548	-717.181030901911\\
69.875	0.19914	-755.916889183883\\
69.875	0.2028	-795.872365066635\\
69.875	0.20646	-837.047458550169\\
69.875	0.21012	-879.442169634483\\
69.875	0.21378	-923.056498319579\\
69.875	0.21744	-967.890444605454\\
69.875	0.2211	-1013.94400849211\\
69.875	0.22476	-1061.21718997955\\
69.875	0.22842	-1109.70998906777\\
69.875	0.23208	-1159.42240575677\\
69.875	0.23574	-1210.35444004655\\
69.875	0.2394	-1262.50609193711\\
69.875	0.24306	-1315.87736142845\\
69.875	0.24672	-1370.46824852058\\
69.875	0.25038	-1426.27875321348\\
69.875	0.25404	-1483.30887550717\\
69.875	0.2577	-1541.55861540163\\
69.875	0.26136	-1601.02797289688\\
69.875	0.26502	-1661.71694799291\\
69.875	0.26868	-1723.62554068972\\
69.875	0.27234	-1786.75375098731\\
69.875	0.276	-1851.10157888568\\
70.25	0.093	-130.701114371285\\
70.25	0.09666	-135.31718230923\\
70.25	0.10032	-141.152867847956\\
70.25	0.10398	-148.208170987463\\
70.25	0.10764	-156.483091727751\\
70.25	0.1113	-165.977630068819\\
70.25	0.11496	-176.691786010669\\
70.25	0.11862	-188.625559553299\\
70.25	0.12228	-201.77895069671\\
70.25	0.12594	-216.151959440903\\
70.25	0.1296	-231.744585785876\\
70.25	0.13326	-248.556829731629\\
70.25	0.13692	-266.588691278165\\
70.25	0.14058	-285.84017042548\\
70.25	0.14424	-306.311267173578\\
70.25	0.1479	-328.001981522455\\
70.25	0.15156	-350.912313472114\\
70.25	0.15522	-375.042263022553\\
70.25	0.15888	-400.391830173774\\
70.25	0.16254	-426.961014925775\\
70.25	0.1662	-454.749817278558\\
70.25	0.16986	-483.758237232121\\
70.25	0.17352	-513.986274786464\\
70.25	0.17718	-545.433929941589\\
70.25	0.18084	-578.101202697495\\
70.25	0.1845	-611.988093054182\\
70.25	0.18816	-647.09460101165\\
70.25	0.19182	-683.420726569899\\
70.25	0.19548	-720.966469728928\\
70.25	0.19914	-759.731830488739\\
70.25	0.2028	-799.71680884933\\
70.25	0.20646	-840.921404810703\\
70.25	0.21012	-883.345618372855\\
70.25	0.21378	-926.98944953579\\
70.25	0.21744	-971.852898299504\\
70.25	0.2211	-1017.935964664\\
70.25	0.22476	-1065.23864862928\\
70.25	0.22842	-1113.76095019534\\
70.25	0.23208	-1163.50286936217\\
70.25	0.23574	-1214.46440612979\\
70.25	0.2394	-1266.64556049819\\
70.25	0.24306	-1320.04633246738\\
70.25	0.24672	-1374.66672203734\\
70.25	0.25038	-1430.50672920808\\
70.25	0.25404	-1487.56635397961\\
70.25	0.2577	-1545.84559635191\\
70.25	0.26136	-1605.344456325\\
70.25	0.26502	-1666.06293389886\\
70.25	0.26868	-1728.00102907351\\
70.25	0.27234	-1791.15874184894\\
70.25	0.276	-1855.53607222515\\
70.625	0.093	-133.799087425935\\
70.625	0.09666	-138.444657841719\\
70.625	0.10032	-144.309845858284\\
70.625	0.10398	-151.394651475629\\
70.625	0.10764	-159.699074693756\\
70.625	0.1113	-169.223115512663\\
70.625	0.11496	-179.966773932352\\
70.625	0.11862	-191.930049952821\\
70.625	0.12228	-205.112943574071\\
70.625	0.12594	-219.515454796102\\
70.625	0.1296	-235.137583618913\\
70.625	0.13326	-251.979330042506\\
70.625	0.13692	-270.04069406688\\
70.625	0.14058	-289.321675692034\\
70.625	0.14424	-309.822274917971\\
70.625	0.1479	-331.542491744686\\
70.625	0.15156	-354.482326172184\\
70.625	0.15522	-378.641778200462\\
70.625	0.15888	-404.020847829521\\
70.625	0.16254	-430.619535059362\\
70.625	0.1662	-458.437839889983\\
70.625	0.16986	-487.475762321385\\
70.625	0.17352	-517.733302353568\\
70.625	0.17718	-549.210459986531\\
70.625	0.18084	-581.907235220276\\
70.625	0.1845	-615.823628054801\\
70.625	0.18816	-650.959638490108\\
70.625	0.19182	-687.315266526195\\
70.625	0.19548	-724.890512163064\\
70.625	0.19914	-763.685375400713\\
70.625	0.2028	-803.699856239143\\
70.625	0.20646	-844.933954678355\\
70.625	0.21012	-887.387670718346\\
70.625	0.21378	-931.061004359119\\
70.625	0.21744	-975.953955600673\\
70.625	0.2211	-1022.06652444301\\
70.625	0.22476	-1069.39871088612\\
70.625	0.22842	-1117.95051493002\\
70.625	0.23208	-1167.7219365747\\
70.625	0.23574	-1218.71297582016\\
70.625	0.2394	-1270.92363266639\\
70.625	0.24306	-1324.35390711342\\
70.625	0.24672	-1379.00379916122\\
70.625	0.25038	-1434.8733088098\\
70.625	0.25404	-1491.96243605916\\
70.625	0.2577	-1550.27118090931\\
70.625	0.26136	-1609.79954336023\\
70.625	0.26502	-1670.54752341193\\
70.625	0.26868	-1732.51512106442\\
70.625	0.27234	-1795.70233631769\\
70.625	0.276	-1860.10916917174\\
71	0.093	-137.035664087704\\
71	0.09666	-141.710736981327\\
71	0.10032	-147.60542747573\\
71	0.10398	-154.719735570914\\
71	0.10764	-163.053661266879\\
71	0.1113	-172.607204563626\\
71	0.11496	-183.380365461153\\
71	0.11862	-195.373143959461\\
71	0.12228	-208.585540058549\\
71	0.12594	-223.017553758419\\
71	0.1296	-238.66918505907\\
71	0.13326	-255.540433960501\\
71	0.13692	-273.631300462714\\
71	0.14058	-292.941784565707\\
71	0.14424	-313.471886269482\\
71	0.1479	-335.221605574036\\
71	0.15156	-358.190942479373\\
71	0.15522	-382.37989698549\\
71	0.15888	-407.788469092388\\
71	0.16254	-434.416658800067\\
71	0.1662	-462.264466108527\\
71	0.16986	-491.331891017767\\
71	0.17352	-521.618933527789\\
71	0.17718	-553.125593638591\\
71	0.18084	-585.851871350175\\
71	0.1845	-619.797766662539\\
71	0.18816	-654.963279575685\\
71	0.19182	-691.348410089611\\
71	0.19548	-728.953158204318\\
71	0.19914	-767.777523919806\\
71	0.2028	-807.821507236075\\
71	0.20646	-849.085108153125\\
71	0.21012	-891.568326670955\\
71	0.21378	-935.271162789567\\
71	0.21744	-980.193616508959\\
71	0.2211	-1026.33568782913\\
71	0.22476	-1073.69737675009\\
71	0.22842	-1122.27868327182\\
71	0.23208	-1172.07960739434\\
71	0.23574	-1223.10014911764\\
71	0.2394	-1275.34030844171\\
71	0.24306	-1328.80008536657\\
71	0.24672	-1383.47947989221\\
71	0.25038	-1439.37849201863\\
71	0.25404	-1496.49712174584\\
71	0.2577	-1554.83536907382\\
71	0.26136	-1614.39323400258\\
71	0.26502	-1675.17071653213\\
71	0.26868	-1737.16781666245\\
71	0.27234	-1800.38453439356\\
71	0.276	-1864.82086972544\\
71.375	0.093	-140.410844356592\\
71.375	0.09666	-145.115419728054\\
71.375	0.10032	-151.039612700296\\
71.375	0.10398	-158.183423273318\\
71.375	0.10764	-166.546851447123\\
71.375	0.1113	-176.129897221708\\
71.375	0.11496	-186.932560597073\\
71.375	0.11862	-198.95484157322\\
71.375	0.12228	-212.196740150148\\
71.375	0.12594	-226.658256327856\\
71.375	0.1296	-242.339390106345\\
71.375	0.13326	-259.240141485616\\
71.375	0.13692	-277.360510465667\\
71.375	0.14058	-296.700497046499\\
71.375	0.14424	-317.260101228113\\
71.375	0.1479	-339.039323010506\\
71.375	0.15156	-362.038162393681\\
71.375	0.15522	-386.256619377637\\
71.375	0.15888	-411.694693962374\\
71.375	0.16254	-438.352386147892\\
71.375	0.1662	-466.229695934191\\
71.375	0.16986	-495.32662332127\\
71.375	0.17352	-525.64316830913\\
71.375	0.17718	-557.179330897771\\
71.375	0.18084	-589.935111087193\\
71.375	0.1845	-623.910508877396\\
71.375	0.18816	-659.105524268381\\
71.375	0.19182	-695.520157260146\\
71.375	0.19548	-733.154407852692\\
71.375	0.19914	-772.008276046018\\
71.375	0.2028	-812.081761840126\\
71.375	0.20646	-853.374865235015\\
71.375	0.21012	-895.887586230684\\
71.375	0.21378	-939.619924827135\\
71.375	0.21744	-984.571881024365\\
71.375	0.2211	-1030.74345482238\\
71.375	0.22476	-1078.13464622117\\
71.375	0.22842	-1126.74545522075\\
71.375	0.23208	-1176.5758818211\\
71.375	0.23574	-1227.62592602224\\
71.375	0.2394	-1279.89558782415\\
71.375	0.24306	-1333.38486722685\\
71.375	0.24672	-1388.09376423033\\
71.375	0.25038	-1444.02227883459\\
71.375	0.25404	-1501.17041103963\\
71.375	0.2577	-1559.53816084545\\
71.375	0.26136	-1619.12552825205\\
71.375	0.26502	-1679.93251325944\\
71.375	0.26868	-1741.9591158676\\
71.375	0.27234	-1805.20533607655\\
71.375	0.276	-1869.67117388627\\
71.75	0.093	-143.924628232599\\
71.75	0.09666	-148.658706081899\\
71.75	0.10032	-154.612401531981\\
71.75	0.10398	-161.785714582842\\
71.75	0.10764	-170.178645234485\\
71.75	0.1113	-179.791193486909\\
71.75	0.11496	-190.623359340113\\
71.75	0.11862	-202.675142794099\\
71.75	0.12228	-215.946543848865\\
71.75	0.12594	-230.437562504413\\
71.75	0.1296	-246.148198760741\\
71.75	0.13326	-263.07845261785\\
71.75	0.13692	-281.228324075739\\
71.75	0.14058	-300.597813134411\\
71.75	0.14424	-321.186919793863\\
71.75	0.1479	-342.995644054095\\
71.75	0.15156	-366.023985915109\\
71.75	0.15522	-390.271945376903\\
71.75	0.15888	-415.739522439479\\
71.75	0.16254	-442.426717102836\\
71.75	0.1662	-470.333529366973\\
71.75	0.16986	-499.459959231891\\
71.75	0.17352	-529.80600669759\\
71.75	0.17718	-561.37167176407\\
71.75	0.18084	-594.156954431332\\
71.75	0.1845	-628.161854699373\\
71.75	0.18816	-663.386372568196\\
71.75	0.19182	-699.8305080378\\
71.75	0.19548	-737.494261108185\\
71.75	0.19914	-776.37763177935\\
71.75	0.2028	-816.480620051296\\
71.75	0.20646	-857.803225924024\\
71.75	0.21012	-900.345449397532\\
71.75	0.21378	-944.107290471822\\
71.75	0.21744	-989.088749146891\\
71.75	0.2211	-1035.28982542274\\
71.75	0.22476	-1082.71051929937\\
71.75	0.22842	-1131.35083077679\\
71.75	0.23208	-1181.21075985498\\
71.75	0.23574	-1232.29030653396\\
71.75	0.2394	-1284.58947081371\\
71.75	0.24306	-1338.10825269425\\
71.75	0.24672	-1392.84665217556\\
71.75	0.25038	-1448.80466925766\\
71.75	0.25404	-1505.98230394054\\
71.75	0.2577	-1564.3795562242\\
71.75	0.26136	-1623.99642610864\\
71.75	0.26502	-1684.83291359387\\
71.75	0.26868	-1746.88901867987\\
71.75	0.27234	-1810.16474136665\\
71.75	0.276	-1874.66008165422\\
72.125	0.093	-147.577015715726\\
72.125	0.09666	-152.340596042865\\
72.125	0.10032	-158.323793970785\\
72.125	0.10398	-165.526609499485\\
72.125	0.10764	-173.949042628967\\
72.125	0.1113	-183.59109335923\\
72.125	0.11496	-194.452761690273\\
72.125	0.11862	-206.534047622097\\
72.125	0.12228	-219.834951154702\\
72.125	0.12594	-234.355472288088\\
72.125	0.1296	-250.095611022255\\
72.125	0.13326	-267.055367357203\\
72.125	0.13692	-285.234741292932\\
72.125	0.14058	-304.633732829441\\
72.125	0.14424	-325.252341966733\\
72.125	0.1479	-347.090568704803\\
72.125	0.15156	-370.148413043656\\
72.125	0.15522	-394.425874983289\\
72.125	0.15888	-419.922954523704\\
72.125	0.16254	-446.639651664899\\
72.125	0.1662	-474.575966406876\\
72.125	0.16986	-503.731898749633\\
72.125	0.17352	-534.10744869317\\
72.125	0.17718	-565.702616237489\\
72.125	0.18084	-598.517401382589\\
72.125	0.1845	-632.551804128469\\
72.125	0.18816	-667.805824475131\\
72.125	0.19182	-704.279462422574\\
72.125	0.19548	-741.972717970797\\
72.125	0.19914	-780.885591119801\\
72.125	0.2028	-821.018081869586\\
72.125	0.20646	-862.370190220153\\
72.125	0.21012	-904.9419161715\\
72.125	0.21378	-948.733259723628\\
72.125	0.21744	-993.744220876536\\
72.125	0.2211	-1039.97479963023\\
72.125	0.22476	-1087.4249959847\\
72.125	0.22842	-1136.09480993995\\
72.125	0.23208	-1185.98424149598\\
72.125	0.23574	-1237.0932906528\\
72.125	0.2394	-1289.42195741039\\
72.125	0.24306	-1342.97024176876\\
72.125	0.24672	-1397.73814372792\\
72.125	0.25038	-1453.72566328786\\
72.125	0.25404	-1510.93280044858\\
72.125	0.2577	-1569.35955521007\\
72.125	0.26136	-1629.00592757235\\
72.125	0.26502	-1689.87191753542\\
72.125	0.26868	-1751.95752509926\\
72.125	0.27234	-1815.26275026388\\
72.125	0.276	-1879.78759302928\\
72.5	0.093	-151.368006805972\\
72.5	0.09666	-156.161089610949\\
72.5	0.10032	-162.173790016708\\
72.5	0.10398	-169.406108023247\\
72.5	0.10764	-177.858043630568\\
72.5	0.1113	-187.529596838669\\
72.5	0.11496	-198.420767647551\\
72.5	0.11862	-210.531556057214\\
72.5	0.12228	-223.861962067658\\
72.5	0.12594	-238.411985678882\\
72.5	0.1296	-254.181626890888\\
72.5	0.13326	-271.170885703675\\
72.5	0.13692	-289.379762117242\\
72.5	0.14058	-308.808256131591\\
72.5	0.14424	-329.456367746721\\
72.5	0.1479	-351.32409696263\\
72.5	0.15156	-374.411443779322\\
72.5	0.15522	-398.718408196794\\
72.5	0.15888	-424.244990215047\\
72.5	0.16254	-450.991189834081\\
72.5	0.1662	-478.957007053897\\
72.5	0.16986	-508.142441874492\\
72.5	0.17352	-538.547494295869\\
72.5	0.17718	-570.172164318026\\
72.5	0.18084	-603.016451940965\\
72.5	0.1845	-637.080357164684\\
72.5	0.18816	-672.363879989184\\
72.5	0.19182	-708.867020414466\\
72.5	0.19548	-746.589778440528\\
72.5	0.19914	-785.532154067371\\
72.5	0.2028	-825.694147294995\\
72.5	0.20646	-867.0757581234\\
72.5	0.21012	-909.676986552586\\
72.5	0.21378	-953.497832582553\\
72.5	0.21744	-998.5382962133\\
72.5	0.2211	-1044.79837744483\\
72.5	0.22476	-1092.27807627714\\
72.5	0.22842	-1140.97739271023\\
72.5	0.23208	-1190.8963267441\\
72.5	0.23574	-1242.03487837875\\
72.5	0.2394	-1294.39304761419\\
72.5	0.24306	-1347.9708344504\\
72.5	0.24672	-1402.76823888739\\
72.5	0.25038	-1458.78526092517\\
72.5	0.25404	-1516.02190056373\\
72.5	0.2577	-1574.47815780306\\
72.5	0.26136	-1634.15403264318\\
72.5	0.26502	-1695.04952508408\\
72.5	0.26868	-1757.16463512576\\
72.5	0.27234	-1820.49936276822\\
72.5	0.276	-1885.05370801147\\
72.875	0.093	-155.297601503336\\
72.875	0.09666	-160.120186786153\\
72.875	0.10032	-166.16238966975\\
72.875	0.10398	-173.424210154128\\
72.875	0.10764	-181.905648239287\\
72.875	0.1113	-191.606703925227\\
72.875	0.11496	-202.527377211948\\
72.875	0.11862	-214.66766809945\\
72.875	0.12228	-228.027576587732\\
72.875	0.12594	-242.607102676796\\
72.875	0.1296	-258.406246366641\\
72.875	0.13326	-275.425007657266\\
72.875	0.13692	-293.663386548672\\
72.875	0.14058	-313.121383040859\\
72.875	0.14424	-333.798997133828\\
72.875	0.1479	-355.696228827577\\
72.875	0.15156	-378.813078122107\\
72.875	0.15522	-403.149545017417\\
72.875	0.15888	-428.70562951351\\
72.875	0.16254	-455.481331610383\\
72.875	0.1662	-483.476651308037\\
72.875	0.16986	-512.691588606471\\
72.875	0.17352	-543.126143505686\\
72.875	0.17718	-574.780316005682\\
72.875	0.18084	-607.65410610646\\
72.875	0.1845	-641.747513808018\\
72.875	0.18816	-677.060539110357\\
72.875	0.19182	-713.593182013477\\
72.875	0.19548	-751.345442517378\\
72.875	0.19914	-790.31732062206\\
72.875	0.2028	-830.508816327523\\
72.875	0.20646	-871.919929633767\\
72.875	0.21012	-914.550660540791\\
72.875	0.21378	-958.401009048597\\
72.875	0.21744	-1003.47097515718\\
72.875	0.2211	-1049.76055886655\\
72.875	0.22476	-1097.2697601767\\
72.875	0.22842	-1145.99857908763\\
72.875	0.23208	-1195.94701559934\\
72.875	0.23574	-1247.11506971183\\
72.875	0.2394	-1299.5027414251\\
72.875	0.24306	-1353.11003073915\\
72.875	0.24672	-1407.93693765399\\
72.875	0.25038	-1463.9834621696\\
72.875	0.25404	-1521.249604286\\
72.875	0.2577	-1579.73536400317\\
72.875	0.26136	-1639.44074132113\\
72.875	0.26502	-1700.36573623987\\
72.875	0.26868	-1762.51034875939\\
72.875	0.27234	-1825.87457887969\\
72.875	0.276	-1890.45842660077\\
73.25	0.093	-159.36579980782\\
73.25	0.09666	-164.217887568475\\
73.25	0.10032	-170.289592929911\\
73.25	0.10398	-177.580915892128\\
73.25	0.10764	-186.091856455126\\
73.25	0.1113	-195.822414618904\\
73.25	0.11496	-206.772590383464\\
73.25	0.11862	-218.942383748805\\
73.25	0.12228	-232.331794714926\\
73.25	0.12594	-246.940823281829\\
73.25	0.1296	-262.769469449512\\
73.25	0.13326	-279.817733217976\\
73.25	0.13692	-298.085614587221\\
73.25	0.14058	-317.573113557247\\
73.25	0.14424	-338.280230128054\\
73.25	0.1479	-360.206964299642\\
73.25	0.15156	-383.353316072011\\
73.25	0.15522	-407.71928544516\\
73.25	0.15888	-433.304872419091\\
73.25	0.16254	-460.110076993803\\
73.25	0.1662	-488.134899169295\\
73.25	0.16986	-517.379338945568\\
73.25	0.17352	-547.843396322623\\
73.25	0.17718	-579.527071300457\\
73.25	0.18084	-612.430363879074\\
73.25	0.1845	-646.55327405847\\
73.25	0.18816	-681.895801838649\\
73.25	0.19182	-718.457947219608\\
73.25	0.19548	-756.239710201347\\
73.25	0.19914	-795.241090783868\\
73.25	0.2028	-835.462088967169\\
73.25	0.20646	-876.902704751252\\
73.25	0.21012	-919.562938136115\\
73.25	0.21378	-963.44278912176\\
73.25	0.21744	-1008.54225770818\\
73.25	0.2211	-1054.86134389539\\
73.25	0.22476	-1102.40004768338\\
73.25	0.22842	-1151.15836907215\\
73.25	0.23208	-1201.13630806169\\
73.25	0.23574	-1252.33386465203\\
73.25	0.2394	-1304.75103884314\\
73.25	0.24306	-1358.38783063503\\
73.25	0.24672	-1413.2442400277\\
73.25	0.25038	-1469.32026702115\\
73.25	0.25404	-1526.61591161539\\
73.25	0.2577	-1585.1311738104\\
73.25	0.26136	-1644.8660536062\\
73.25	0.26502	-1705.82055100278\\
73.25	0.26868	-1767.99466600013\\
73.25	0.27234	-1831.38839859827\\
73.25	0.276	-1896.00174879719\\
73.625	0.093	-163.572601719423\\
73.625	0.09666	-168.454191957917\\
73.625	0.10032	-174.555399797192\\
73.625	0.10398	-181.876225237248\\
73.625	0.10764	-190.416668278084\\
73.625	0.1113	-200.176728919702\\
73.625	0.11496	-211.1564071621\\
73.625	0.11862	-223.35570300528\\
73.625	0.12228	-236.77461644924\\
73.625	0.12594	-251.413147493981\\
73.625	0.1296	-267.271296139503\\
73.625	0.13326	-284.349062385806\\
73.625	0.13692	-302.64644623289\\
73.625	0.14058	-322.163447680755\\
73.625	0.14424	-342.900066729401\\
73.625	0.1479	-364.856303378827\\
73.625	0.15156	-388.032157629035\\
73.625	0.15522	-412.427629480023\\
73.625	0.15888	-438.042718931793\\
73.625	0.16254	-464.877425984343\\
73.625	0.1662	-492.931750637675\\
73.625	0.16986	-522.205692891786\\
73.625	0.17352	-552.699252746679\\
73.625	0.17718	-584.412430202353\\
73.625	0.18084	-617.345225258808\\
73.625	0.1845	-651.497637916043\\
73.625	0.18816	-686.869668174061\\
73.625	0.19182	-723.461316032858\\
73.625	0.19548	-761.272581492437\\
73.625	0.19914	-800.303464552796\\
73.625	0.2028	-840.553965213936\\
73.625	0.20646	-882.024083475858\\
73.625	0.21012	-924.713819338559\\
73.625	0.21378	-968.623172802043\\
73.625	0.21744	-1013.75214386631\\
73.625	0.2211	-1060.10073253135\\
73.625	0.22476	-1107.66893879718\\
73.625	0.22842	-1156.45676266378\\
73.625	0.23208	-1206.46420413117\\
73.625	0.23574	-1257.69126319934\\
73.625	0.2394	-1310.13793986829\\
73.625	0.24306	-1363.80423413802\\
73.625	0.24672	-1418.69014600853\\
73.625	0.25038	-1474.79567547982\\
73.625	0.25404	-1532.1208225519\\
73.625	0.2577	-1590.66558722475\\
73.625	0.26136	-1650.42996949839\\
73.625	0.26502	-1711.4139693728\\
73.625	0.26868	-1773.617586848\\
73.625	0.27234	-1837.04082192398\\
73.625	0.276	-1901.68367460073\\
74	0.093	-167.918007238146\\
74	0.09666	-172.829099954478\\
74	0.10032	-178.959810271592\\
74	0.10398	-186.310138189486\\
74	0.10764	-194.880083708161\\
74	0.1113	-204.669646827618\\
74	0.11496	-215.678827547855\\
74	0.11862	-227.907625868873\\
74	0.12228	-241.356041790672\\
74	0.12594	-256.024075313252\\
74	0.1296	-271.911726436613\\
74	0.13326	-289.018995160754\\
74	0.13692	-307.345881485677\\
74	0.14058	-326.892385411381\\
74	0.14424	-347.658506937865\\
74	0.1479	-369.644246065131\\
74	0.15156	-392.849602793177\\
74	0.15522	-417.274577122004\\
74	0.15888	-442.919169051612\\
74	0.16254	-469.783378582002\\
74	0.1662	-497.867205713172\\
74	0.16986	-527.170650445122\\
74	0.17352	-557.693712777855\\
74	0.17718	-589.436392711367\\
74	0.18084	-622.39869024566\\
74	0.1845	-656.580605380735\\
74	0.18816	-691.98213811659\\
74	0.19182	-728.603288453227\\
74	0.19548	-766.444056390644\\
74	0.19914	-805.504441928842\\
74	0.2028	-845.784445067821\\
74	0.20646	-887.284065807582\\
74	0.21012	-930.003304148122\\
74	0.21378	-973.942160089444\\
74	0.21744	-1019.10063363155\\
74	0.2211	-1065.47872477443\\
74	0.22476	-1113.0764335181\\
74	0.22842	-1161.89375986254\\
74	0.23208	-1211.93070380777\\
74	0.23574	-1263.18726535377\\
74	0.2394	-1315.66344450056\\
74	0.24306	-1369.35924124813\\
74	0.24672	-1424.27465559648\\
74	0.25038	-1480.40968754561\\
74	0.25404	-1537.76433709553\\
74	0.2577	-1596.33860424622\\
74	0.26136	-1656.13248899769\\
74	0.26502	-1717.14599134995\\
74	0.26868	-1779.37911130298\\
74	0.27234	-1842.8318488568\\
74	0.276	-1907.50420401139\\
};
\end{axis}

\begin{axis}[%
width=4.527496cm,
height=3.870968cm,
at={(6.483547cm,5.376344cm)},
scale only axis,
xmin=56,
xmax=74,
tick align=outside,
xlabel={$L_{cut}$},
xmajorgrids,
ymin=0.093,
ymax=0.276,
ylabel={$D_{rlx}$},
ymajorgrids,
zmin=-1328.01931269086,
zmax=0,
zlabel={$x_4$},
zmajorgrids,
view={-140}{50},
legend style={at={(1.03,1)},anchor=north west,legend cell align=left,align=left,draw=white!15!black}
]
\addplot3[only marks,mark=*,mark options={},mark size=1.5000pt,color=mycolor1] plot table[row sep=crcr,]{%
74	0.123	-171.244880190644\\
72	0.113	-139.217460396701\\
61	0.095	-69.6051966848396\\
56	0.093	-73.6870961160989\\
};
\addplot3[only marks,mark=*,mark options={},mark size=1.5000pt,color=mycolor2] plot table[row sep=crcr,]{%
67	0.276	-1273.01999820883\\
66	0.255	-1046.38133574319\\
62	0.209	-579.722288788632\\
57	0.193	-465.590859275979\\
};
\addplot3[only marks,mark=*,mark options={},mark size=1.5000pt,color=black] plot table[row sep=crcr,]{%
69	0.104	-102.299172261903\\
};
\addplot3[only marks,mark=*,mark options={},mark size=1.5000pt,color=black] plot table[row sep=crcr,]{%
64	0.23	-773.899909377579\\
};

\addplot3[%
surf,
opacity=0.7,
shader=interp,
colormap={mymap}{[1pt] rgb(0pt)=(0.0901961,0.239216,0.0745098); rgb(1pt)=(0.0945149,0.242058,0.0739522); rgb(2pt)=(0.0988592,0.244894,0.0733566); rgb(3pt)=(0.103229,0.247724,0.0727241); rgb(4pt)=(0.107623,0.250549,0.0720557); rgb(5pt)=(0.112043,0.253367,0.0713525); rgb(6pt)=(0.116487,0.25618,0.0706154); rgb(7pt)=(0.120956,0.258986,0.0698456); rgb(8pt)=(0.125449,0.261787,0.0690441); rgb(9pt)=(0.129967,0.264581,0.0682118); rgb(10pt)=(0.134508,0.26737,0.06735); rgb(11pt)=(0.139074,0.270152,0.0664596); rgb(12pt)=(0.143663,0.272929,0.0655416); rgb(13pt)=(0.148275,0.275699,0.0645971); rgb(14pt)=(0.152911,0.278463,0.0636271); rgb(15pt)=(0.15757,0.281221,0.0626328); rgb(16pt)=(0.162252,0.283973,0.0616151); rgb(17pt)=(0.166957,0.286719,0.060575); rgb(18pt)=(0.171685,0.289458,0.0595136); rgb(19pt)=(0.176434,0.292191,0.0584321); rgb(20pt)=(0.181207,0.294918,0.0573313); rgb(21pt)=(0.186001,0.297639,0.0562123); rgb(22pt)=(0.190817,0.300353,0.0550763); rgb(23pt)=(0.195655,0.303061,0.0539242); rgb(24pt)=(0.200514,0.305763,0.052757); rgb(25pt)=(0.205395,0.308459,0.0515759); rgb(26pt)=(0.210296,0.311149,0.0503624); rgb(27pt)=(0.215212,0.313846,0.0490067); rgb(28pt)=(0.220142,0.316548,0.0475043); rgb(29pt)=(0.22509,0.319254,0.0458704); rgb(30pt)=(0.230056,0.321962,0.0441205); rgb(31pt)=(0.235042,0.324671,0.04227); rgb(32pt)=(0.240048,0.327379,0.0403343); rgb(33pt)=(0.245078,0.330085,0.0383287); rgb(34pt)=(0.250131,0.332786,0.0362688); rgb(35pt)=(0.25521,0.335482,0.0341698); rgb(36pt)=(0.260317,0.33817,0.0320472); rgb(37pt)=(0.265451,0.340849,0.0299163); rgb(38pt)=(0.270616,0.343517,0.0277927); rgb(39pt)=(0.275813,0.346172,0.0256916); rgb(40pt)=(0.281043,0.348814,0.0236284); rgb(41pt)=(0.286307,0.35144,0.0216186); rgb(42pt)=(0.291607,0.354048,0.0196776); rgb(43pt)=(0.296945,0.356637,0.0178207); rgb(44pt)=(0.302322,0.359206,0.0160634); rgb(45pt)=(0.307739,0.361753,0.0144211); rgb(46pt)=(0.313198,0.364275,0.0129091); rgb(47pt)=(0.318701,0.366772,0.0115428); rgb(48pt)=(0.324249,0.369242,0.0103377); rgb(49pt)=(0.329843,0.371682,0.00930909); rgb(50pt)=(0.335485,0.374093,0.00847245); rgb(51pt)=(0.341176,0.376471,0.00784314); rgb(52pt)=(0.346925,0.378826,0.00732741); rgb(53pt)=(0.352735,0.381168,0.00682184); rgb(54pt)=(0.358605,0.383497,0.00632729); rgb(55pt)=(0.364532,0.385812,0.00584464); rgb(56pt)=(0.370516,0.388113,0.00537476); rgb(57pt)=(0.376552,0.390399,0.00491852); rgb(58pt)=(0.38264,0.39267,0.00447681); rgb(59pt)=(0.388777,0.394925,0.00405048); rgb(60pt)=(0.394962,0.397164,0.00364042); rgb(61pt)=(0.401191,0.399386,0.00324749); rgb(62pt)=(0.407464,0.401592,0.00287258); rgb(63pt)=(0.413777,0.40378,0.00251655); rgb(64pt)=(0.420129,0.40595,0.00218028); rgb(65pt)=(0.426518,0.408102,0.00186463); rgb(66pt)=(0.432942,0.410234,0.00157049); rgb(67pt)=(0.439399,0.412348,0.00129873); rgb(68pt)=(0.445885,0.414441,0.00105022); rgb(69pt)=(0.452401,0.416515,0.000825833); rgb(70pt)=(0.458942,0.418567,0.000626441); rgb(71pt)=(0.465508,0.420599,0.00045292); rgb(72pt)=(0.472096,0.422609,0.000306141); rgb(73pt)=(0.478704,0.424596,0.000186979); rgb(74pt)=(0.485331,0.426562,9.63073e-05); rgb(75pt)=(0.491973,0.428504,3.49981e-05); rgb(76pt)=(0.498628,0.430422,3.92506e-06); rgb(77pt)=(0.505323,0.432315,0); rgb(78pt)=(0.512206,0.434168,0); rgb(79pt)=(0.519282,0.435983,0); rgb(80pt)=(0.526529,0.437764,0); rgb(81pt)=(0.533922,0.439512,0); rgb(82pt)=(0.54144,0.441232,0); rgb(83pt)=(0.549059,0.442927,0); rgb(84pt)=(0.556756,0.444599,0); rgb(85pt)=(0.564508,0.446252,0); rgb(86pt)=(0.572292,0.447889,0); rgb(87pt)=(0.580084,0.449514,0); rgb(88pt)=(0.587863,0.451129,0); rgb(89pt)=(0.595604,0.452737,0); rgb(90pt)=(0.603284,0.454343,0); rgb(91pt)=(0.610882,0.455948,0); rgb(92pt)=(0.618373,0.457556,0); rgb(93pt)=(0.625734,0.459171,0); rgb(94pt)=(0.632943,0.460795,0); rgb(95pt)=(0.639976,0.462432,0); rgb(96pt)=(0.64681,0.464084,0); rgb(97pt)=(0.653423,0.465756,0); rgb(98pt)=(0.659791,0.46745,0); rgb(99pt)=(0.665891,0.469169,0); rgb(100pt)=(0.6717,0.470916,0); rgb(101pt)=(0.677195,0.472696,0); rgb(102pt)=(0.682353,0.47451,0); rgb(103pt)=(0.687242,0.476355,0); rgb(104pt)=(0.691952,0.478225,0); rgb(105pt)=(0.696497,0.480118,0); rgb(106pt)=(0.700887,0.482033,0); rgb(107pt)=(0.705134,0.483968,0); rgb(108pt)=(0.709251,0.485921,0); rgb(109pt)=(0.713249,0.487891,0); rgb(110pt)=(0.71714,0.489876,0); rgb(111pt)=(0.720936,0.491875,0); rgb(112pt)=(0.724649,0.493887,0); rgb(113pt)=(0.72829,0.495909,0); rgb(114pt)=(0.731872,0.49794,0); rgb(115pt)=(0.735406,0.499979,0); rgb(116pt)=(0.738904,0.502025,0); rgb(117pt)=(0.742378,0.504075,0); rgb(118pt)=(0.74584,0.506128,0); rgb(119pt)=(0.749302,0.508182,0); rgb(120pt)=(0.752775,0.510237,0); rgb(121pt)=(0.756272,0.51229,0); rgb(122pt)=(0.759804,0.514339,0); rgb(123pt)=(0.763384,0.516385,0); rgb(124pt)=(0.767022,0.518424,0); rgb(125pt)=(0.770731,0.520455,0); rgb(126pt)=(0.774523,0.522478,0); rgb(127pt)=(0.77841,0.524489,0); rgb(128pt)=(0.782391,0.526491,0); rgb(129pt)=(0.786402,0.528496,0); rgb(130pt)=(0.790431,0.530506,0); rgb(131pt)=(0.794478,0.532521,0); rgb(132pt)=(0.798541,0.534539,0); rgb(133pt)=(0.802619,0.53656,0); rgb(134pt)=(0.806712,0.538584,0); rgb(135pt)=(0.81082,0.540609,0); rgb(136pt)=(0.81494,0.542635,0); rgb(137pt)=(0.819074,0.54466,0); rgb(138pt)=(0.823219,0.546686,0); rgb(139pt)=(0.827374,0.548709,0); rgb(140pt)=(0.831541,0.55073,0); rgb(141pt)=(0.835716,0.552749,0); rgb(142pt)=(0.8399,0.554763,0); rgb(143pt)=(0.844092,0.556774,0); rgb(144pt)=(0.848292,0.558779,0); rgb(145pt)=(0.852497,0.560778,0); rgb(146pt)=(0.856708,0.562771,0); rgb(147pt)=(0.860924,0.564756,0); rgb(148pt)=(0.865143,0.566733,0); rgb(149pt)=(0.869366,0.568701,0); rgb(150pt)=(0.873592,0.57066,0); rgb(151pt)=(0.877819,0.572608,0); rgb(152pt)=(0.882047,0.574545,0); rgb(153pt)=(0.886275,0.576471,0); rgb(154pt)=(0.890659,0.578362,0); rgb(155pt)=(0.895333,0.580203,0); rgb(156pt)=(0.900258,0.581999,0); rgb(157pt)=(0.905397,0.583755,0); rgb(158pt)=(0.910711,0.585479,0); rgb(159pt)=(0.916164,0.587176,0); rgb(160pt)=(0.921717,0.588852,0); rgb(161pt)=(0.927333,0.590513,0); rgb(162pt)=(0.932974,0.592166,0); rgb(163pt)=(0.938602,0.593815,0); rgb(164pt)=(0.94418,0.595468,0); rgb(165pt)=(0.949669,0.59713,0); rgb(166pt)=(0.955033,0.598808,0); rgb(167pt)=(0.960233,0.600507,0); rgb(168pt)=(0.965232,0.602233,0); rgb(169pt)=(0.969992,0.603992,0); rgb(170pt)=(0.974475,0.605791,0); rgb(171pt)=(0.978643,0.607636,0); rgb(172pt)=(0.98246,0.609532,0); rgb(173pt)=(0.985886,0.611486,0); rgb(174pt)=(0.988885,0.613503,0); rgb(175pt)=(0.991419,0.61559,0); rgb(176pt)=(0.99345,0.617753,0); rgb(177pt)=(0.99494,0.619997,0); rgb(178pt)=(0.995851,0.622329,0); rgb(179pt)=(0.996226,0.624763,0); rgb(180pt)=(0.996512,0.627352,0); rgb(181pt)=(0.996788,0.630095,0); rgb(182pt)=(0.997053,0.632982,0); rgb(183pt)=(0.997308,0.636004,0); rgb(184pt)=(0.997552,0.639152,0); rgb(185pt)=(0.997785,0.642416,0); rgb(186pt)=(0.998006,0.645786,0); rgb(187pt)=(0.998217,0.649253,0); rgb(188pt)=(0.998416,0.652807,0); rgb(189pt)=(0.998605,0.656439,0); rgb(190pt)=(0.998781,0.660138,0); rgb(191pt)=(0.998946,0.663897,0); rgb(192pt)=(0.9991,0.667704,0); rgb(193pt)=(0.999242,0.67155,0); rgb(194pt)=(0.999372,0.675427,0); rgb(195pt)=(0.99949,0.679323,0); rgb(196pt)=(0.999596,0.68323,0); rgb(197pt)=(0.99969,0.687139,0); rgb(198pt)=(0.999771,0.691039,0); rgb(199pt)=(0.999841,0.694921,0); rgb(200pt)=(0.999898,0.698775,0); rgb(201pt)=(0.999942,0.702592,0); rgb(202pt)=(0.999974,0.706363,0); rgb(203pt)=(0.999994,0.710077,0); rgb(204pt)=(1,0.713725,0); rgb(205pt)=(1,0.717341,0); rgb(206pt)=(1,0.720963,0); rgb(207pt)=(1,0.724591,0); rgb(208pt)=(1,0.728226,0); rgb(209pt)=(1,0.731867,0); rgb(210pt)=(1,0.735514,0); rgb(211pt)=(1,0.739167,0); rgb(212pt)=(1,0.742827,0); rgb(213pt)=(1,0.746493,0); rgb(214pt)=(1,0.750165,0); rgb(215pt)=(1,0.753843,0); rgb(216pt)=(1,0.757527,0); rgb(217pt)=(1,0.761217,0); rgb(218pt)=(1,0.764913,0); rgb(219pt)=(1,0.768615,0); rgb(220pt)=(1,0.772324,0); rgb(221pt)=(1,0.776038,0); rgb(222pt)=(1,0.779758,0); rgb(223pt)=(1,0.783484,0); rgb(224pt)=(1,0.787215,0); rgb(225pt)=(1,0.790953,0); rgb(226pt)=(1,0.794696,0); rgb(227pt)=(1,0.798445,0); rgb(228pt)=(1,0.8022,0); rgb(229pt)=(1,0.805961,0); rgb(230pt)=(1,0.809727,0); rgb(231pt)=(1,0.8135,0); rgb(232pt)=(1,0.817278,0); rgb(233pt)=(1,0.821063,0); rgb(234pt)=(1,0.824854,0); rgb(235pt)=(1,0.828652,0); rgb(236pt)=(1,0.832455,0); rgb(237pt)=(1,0.836265,0); rgb(238pt)=(1,0.840081,0); rgb(239pt)=(1,0.843903,0); rgb(240pt)=(1,0.847732,0); rgb(241pt)=(1,0.851566,0); rgb(242pt)=(1,0.855406,0); rgb(243pt)=(1,0.859253,0); rgb(244pt)=(1,0.863106,0); rgb(245pt)=(1,0.866964,0); rgb(246pt)=(1,0.870829,0); rgb(247pt)=(1,0.8747,0); rgb(248pt)=(1,0.878577,0); rgb(249pt)=(1,0.88246,0); rgb(250pt)=(1,0.886349,0); rgb(251pt)=(1,0.890243,0); rgb(252pt)=(1,0.894144,0); rgb(253pt)=(1,0.898051,0); rgb(254pt)=(1,0.901964,0); rgb(255pt)=(1,0.905882,0)},
mesh/rows=49]
table[row sep=crcr,header=false] {%
%
56	0.093	-73.0247201609133\\
56	0.09666	-76.5488775261567\\
56	0.10032	-80.8947757668137\\
56	0.10398	-86.0624148828836\\
56	0.10764	-92.0517948743666\\
56	0.1113	-98.8629157412631\\
56	0.11496	-106.495777483573\\
56	0.11862	-114.950380101296\\
56	0.12228	-124.226723594431\\
56	0.12594	-134.324807962981\\
56	0.1296	-145.244633206943\\
56	0.13326	-156.986199326319\\
56	0.13692	-169.549506321107\\
56	0.14058	-182.934554191309\\
56	0.14424	-197.141342936925\\
56	0.1479	-212.169872557953\\
56	0.15156	-228.020143054394\\
56	0.15522	-244.692154426249\\
56	0.15888	-262.185906673517\\
56	0.16254	-280.501399796198\\
56	0.1662	-299.638633794293\\
56	0.16986	-319.5976086678\\
56	0.17352	-340.378324416721\\
56	0.17718	-361.980781041055\\
56	0.18084	-384.404978540802\\
56	0.1845	-407.650916915963\\
56	0.18816	-431.718596166536\\
56	0.19182	-456.608016292523\\
56	0.19548	-482.319177293923\\
56	0.19914	-508.852079170736\\
56	0.2028	-536.206721922963\\
56	0.20646	-564.383105550602\\
56	0.21012	-593.381230053655\\
56	0.21378	-623.201095432121\\
56	0.21744	-653.842701686\\
56	0.2211	-685.306048815292\\
56	0.22476	-717.591136819998\\
56	0.22842	-750.697965700116\\
56	0.23208	-784.626535455648\\
56	0.23574	-819.376846086594\\
56	0.2394	-854.948897592952\\
56	0.24306	-891.342689974723\\
56	0.24672	-928.558223231908\\
56	0.25038	-966.595497364506\\
56	0.25404	-1005.45451237252\\
56	0.2577	-1045.13526825594\\
56	0.26136	-1085.63776501478\\
56	0.26502	-1126.96200264903\\
56	0.26868	-1169.10798115869\\
56	0.27234	-1212.07570054377\\
56	0.276	-1255.86516080426\\
56.375	0.093	-72.2605225046986\\
56.375	0.09666	-75.7966422256901\\
56.375	0.10032	-80.1545028220945\\
56.375	0.10398	-85.3341042939122\\
56.375	0.10764	-91.3354466411432\\
56.375	0.1113	-98.1585298637872\\
56.375	0.11496	-105.803353961845\\
56.375	0.11862	-114.269918935315\\
56.375	0.12228	-123.558224784199\\
56.375	0.12594	-133.668271508496\\
56.375	0.1296	-144.600059108206\\
56.375	0.13326	-156.353587583329\\
56.375	0.13692	-168.928856933866\\
56.375	0.14058	-182.325867159816\\
56.375	0.14424	-196.544618261178\\
56.375	0.1479	-211.585110237955\\
56.375	0.15156	-227.447343090144\\
56.375	0.15522	-244.131316817747\\
56.375	0.15888	-261.637031420762\\
56.375	0.16254	-279.964486899192\\
56.375	0.1662	-299.113683253033\\
56.375	0.16986	-319.084620482289\\
56.375	0.17352	-339.877298586957\\
56.375	0.17718	-361.491717567039\\
56.375	0.18084	-383.927877422534\\
56.375	0.1845	-407.185778153442\\
56.375	0.18816	-431.265419759763\\
56.375	0.19182	-456.166802241498\\
56.375	0.19548	-481.889925598646\\
56.375	0.19914	-508.434789831206\\
56.375	0.2028	-535.801394939181\\
56.375	0.20646	-563.989740922568\\
56.375	0.21012	-592.999827781368\\
56.375	0.21378	-622.831655515582\\
56.375	0.21744	-653.485224125209\\
56.375	0.2211	-684.960533610249\\
56.375	0.22476	-717.257583970702\\
56.375	0.22842	-750.376375206569\\
56.375	0.23208	-784.316907317849\\
56.375	0.23574	-819.079180304542\\
56.375	0.2394	-854.663194166648\\
56.375	0.24306	-891.068948904167\\
56.375	0.24672	-928.296444517099\\
56.375	0.25038	-966.345681005445\\
56.375	0.25404	-1005.2166583692\\
56.375	0.2577	-1044.90937660838\\
56.375	0.26136	-1085.42383572296\\
56.375	0.26502	-1126.76003571296\\
56.375	0.26868	-1168.91797657837\\
56.375	0.27234	-1211.8976583192\\
56.375	0.276	-1255.69908093543\\
56.75	0.093	-71.5673585236526\\
56.75	0.09666	-75.1154406003918\\
56.75	0.10032	-79.485263552544\\
56.75	0.10398	-84.6768273801095\\
56.75	0.10764	-90.6901320830883\\
56.75	0.1113	-97.52517766148\\
56.75	0.11496	-105.181964115285\\
56.75	0.11862	-113.660491444503\\
56.75	0.12228	-122.960759649135\\
56.75	0.12594	-133.08276872918\\
56.75	0.1296	-144.026518684638\\
56.75	0.13326	-155.792009515509\\
56.75	0.13692	-168.379241221793\\
56.75	0.14058	-181.788213803491\\
56.75	0.14424	-196.018927260601\\
56.75	0.1479	-211.071381593125\\
56.75	0.15156	-226.945576801062\\
56.75	0.15522	-243.641512884412\\
56.75	0.15888	-261.159189843176\\
56.75	0.16254	-279.498607677353\\
56.75	0.1662	-298.659766386942\\
56.75	0.16986	-318.642665971946\\
56.75	0.17352	-339.447306432362\\
56.75	0.17718	-361.073687768191\\
56.75	0.18084	-383.521809979434\\
56.75	0.1845	-406.79167306609\\
56.75	0.18816	-430.883277028159\\
56.75	0.19182	-455.796621865641\\
56.75	0.19548	-481.531707578537\\
56.75	0.19914	-508.088534166845\\
56.75	0.2028	-535.467101630567\\
56.75	0.20646	-563.667409969703\\
56.75	0.21012	-592.689459184251\\
56.75	0.21378	-622.533249274212\\
56.75	0.21744	-653.198780239587\\
56.75	0.2211	-684.686052080375\\
56.75	0.22476	-716.995064796576\\
56.75	0.22842	-750.12581838819\\
56.75	0.23208	-784.078312855218\\
56.75	0.23574	-818.852548197658\\
56.75	0.2394	-854.448524415512\\
56.75	0.24306	-890.866241508779\\
56.75	0.24672	-928.105699477459\\
56.75	0.25038	-966.166898321553\\
56.75	0.25404	-1005.04983804106\\
56.75	0.2577	-1044.75451863598\\
56.75	0.26136	-1085.28094010631\\
56.75	0.26502	-1126.62910245206\\
56.75	0.26868	-1168.79900567322\\
56.75	0.27234	-1211.79064976979\\
56.75	0.276	-1255.60403474178\\
57.125	0.093	-70.9452282177755\\
57.125	0.09666	-74.5052726502622\\
57.125	0.10032	-78.8870579581624\\
57.125	0.10398	-84.0905841414756\\
57.125	0.10764	-90.1158512002021\\
57.125	0.1113	-96.9628591343419\\
57.125	0.11496	-104.631607943894\\
57.125	0.11862	-113.122097628861\\
57.125	0.12228	-122.43432818924\\
57.125	0.12594	-132.568299625032\\
57.125	0.1296	-143.524011936238\\
57.125	0.13326	-155.301465122857\\
57.125	0.13692	-167.900659184889\\
57.125	0.14058	-181.321594122334\\
57.125	0.14424	-195.564269935193\\
57.125	0.1479	-210.628686623464\\
57.125	0.15156	-226.514844187149\\
57.125	0.15522	-243.222742626247\\
57.125	0.15888	-260.752381940759\\
57.125	0.16254	-279.103762130683\\
57.125	0.1662	-298.276883196021\\
57.125	0.16986	-318.271745136771\\
57.125	0.17352	-339.088347952935\\
57.125	0.17718	-360.726691644512\\
57.125	0.18084	-383.186776211503\\
57.125	0.1845	-406.468601653907\\
57.125	0.18816	-430.572167971724\\
57.125	0.19182	-455.497475164954\\
57.125	0.19548	-481.244523233597\\
57.125	0.19914	-507.813312177653\\
57.125	0.2028	-535.203841997123\\
57.125	0.20646	-563.416112692006\\
57.125	0.21012	-592.450124262302\\
57.125	0.21378	-622.305876708011\\
57.125	0.21744	-652.983370029133\\
57.125	0.2211	-684.482604225669\\
57.125	0.22476	-716.803579297618\\
57.125	0.22842	-749.94629524498\\
57.125	0.23208	-783.910752067755\\
57.125	0.23574	-818.696949765944\\
57.125	0.2394	-854.304888339545\\
57.125	0.24306	-890.73456778856\\
57.125	0.24672	-927.985988112988\\
57.125	0.25038	-966.059149312829\\
57.125	0.25404	-1004.95405138808\\
57.125	0.2577	-1044.67069433875\\
57.125	0.26136	-1085.20907816483\\
57.125	0.26502	-1126.56920286633\\
57.125	0.26868	-1168.75106844323\\
57.125	0.27234	-1211.75467489555\\
57.125	0.276	-1255.58002222329\\
57.5	0.093	-70.394131587067\\
57.5	0.09666	-73.9661383753014\\
57.5	0.10032	-78.3598860389494\\
57.5	0.10398	-83.5753745780103\\
57.5	0.10764	-89.6126039924844\\
57.5	0.1113	-96.4715742823719\\
57.5	0.11496	-104.152285447672\\
57.5	0.11862	-112.654737488386\\
57.5	0.12228	-121.978930404513\\
57.5	0.12594	-132.124864196054\\
57.5	0.1296	-143.092538863007\\
57.5	0.13326	-154.881954405374\\
57.5	0.13692	-167.493110823153\\
57.5	0.14058	-180.926008116346\\
57.5	0.14424	-195.180646284953\\
57.5	0.1479	-210.257025328972\\
57.5	0.15156	-226.155145248405\\
57.5	0.15522	-242.87500604325\\
57.5	0.15888	-260.41660771351\\
57.5	0.16254	-278.779950259182\\
57.5	0.1662	-297.965033680267\\
57.5	0.16986	-317.971857976766\\
57.5	0.17352	-338.800423148678\\
57.5	0.17718	-360.450729196002\\
57.5	0.18084	-382.92277611874\\
57.5	0.1845	-406.216563916892\\
57.5	0.18816	-430.332092590457\\
57.5	0.19182	-455.269362139434\\
57.5	0.19548	-481.028372563825\\
57.5	0.19914	-507.60912386363\\
57.5	0.2028	-535.011616038847\\
57.5	0.20646	-563.235849089478\\
57.5	0.21012	-592.281823015521\\
57.5	0.21378	-622.149537816978\\
57.5	0.21744	-652.838993493849\\
57.5	0.2211	-684.350190046132\\
57.5	0.22476	-716.683127473828\\
57.5	0.22842	-749.837805776938\\
57.5	0.23208	-783.814224955461\\
57.5	0.23574	-818.612385009398\\
57.5	0.2394	-854.232285938747\\
57.5	0.24306	-890.673927743509\\
57.5	0.24672	-927.937310423685\\
57.5	0.25038	-966.022433979274\\
57.5	0.25404	-1004.92929841028\\
57.5	0.2577	-1044.65790371669\\
57.5	0.26136	-1085.20824989852\\
57.5	0.26502	-1126.58033695576\\
57.5	0.26868	-1168.77416488842\\
57.5	0.27234	-1211.78973369648\\
57.5	0.276	-1255.62704337997\\
57.875	0.093	-69.914068631527\\
57.875	0.09666	-73.4980377755096\\
57.875	0.10032	-77.9037477949049\\
57.875	0.10398	-83.1311986897136\\
57.875	0.10764	-89.1803904599356\\
57.875	0.1113	-96.0513231055706\\
57.875	0.11496	-103.743996626619\\
57.875	0.11862	-112.258411023081\\
57.875	0.12228	-121.594566294956\\
57.875	0.12594	-131.752462442243\\
57.875	0.1296	-142.732099464945\\
57.875	0.13326	-154.533477363059\\
57.875	0.13692	-167.156596136587\\
57.875	0.14058	-180.601455785527\\
57.875	0.14424	-194.868056309881\\
57.875	0.1479	-209.956397709649\\
57.875	0.15156	-225.866479984829\\
57.875	0.15522	-242.598303135422\\
57.875	0.15888	-260.151867161429\\
57.875	0.16254	-278.527172062849\\
57.875	0.1662	-297.724217839682\\
57.875	0.16986	-317.743004491929\\
57.875	0.17352	-338.583532019588\\
57.875	0.17718	-360.245800422661\\
57.875	0.18084	-382.729809701147\\
57.875	0.1845	-406.035559855046\\
57.875	0.18816	-430.163050884359\\
57.875	0.19182	-455.112282789084\\
57.875	0.19548	-480.883255569223\\
57.875	0.19914	-507.475969224775\\
57.875	0.2028	-534.89042375574\\
57.875	0.20646	-563.126619162118\\
57.875	0.21012	-592.18455544391\\
57.875	0.21378	-622.064232601115\\
57.875	0.21744	-652.765650633733\\
57.875	0.2211	-684.288809541764\\
57.875	0.22476	-716.633709325208\\
57.875	0.22842	-749.800349984065\\
57.875	0.23208	-783.788731518336\\
57.875	0.23574	-818.59885392802\\
57.875	0.2394	-854.230717213117\\
57.875	0.24306	-890.684321373628\\
57.875	0.24672	-927.959666409551\\
57.875	0.25038	-966.056752320888\\
57.875	0.25404	-1004.97557910764\\
57.875	0.2577	-1044.7161467698\\
57.875	0.26136	-1085.27845530738\\
57.875	0.26502	-1126.66250472037\\
57.875	0.26868	-1168.86829500877\\
57.875	0.27234	-1211.89582617259\\
57.875	0.276	-1255.74509821181\\
58.25	0.093	-69.5050393511555\\
58.25	0.09666	-73.1009708508859\\
58.25	0.10032	-77.5186432260292\\
58.25	0.10398	-82.7580564765856\\
58.25	0.10764	-88.8192106025554\\
58.25	0.1113	-95.7021056039382\\
58.25	0.11496	-103.406741480735\\
58.25	0.11862	-111.933118232944\\
58.25	0.12228	-121.281235860567\\
58.25	0.12594	-131.451094363602\\
58.25	0.1296	-142.442693742051\\
58.25	0.13326	-154.256033995913\\
58.25	0.13692	-166.891115125188\\
58.25	0.14058	-180.347937129877\\
58.25	0.14424	-194.626500009979\\
58.25	0.1479	-209.726803765494\\
58.25	0.15156	-225.648848396422\\
58.25	0.15522	-242.392633902763\\
58.25	0.15888	-259.958160284517\\
58.25	0.16254	-278.345427541685\\
58.25	0.1662	-297.554435674266\\
58.25	0.16986	-317.585184682261\\
58.25	0.17352	-338.437674565668\\
58.25	0.17718	-360.111905324488\\
58.25	0.18084	-382.607876958722\\
58.25	0.1845	-405.925589468369\\
58.25	0.18816	-430.065042853429\\
58.25	0.19182	-455.026237113902\\
58.25	0.19548	-480.809172249789\\
58.25	0.19914	-507.413848261088\\
58.25	0.2028	-534.840265147801\\
58.25	0.20646	-563.088422909928\\
58.25	0.21012	-592.158321547467\\
58.25	0.21378	-622.049961060419\\
58.25	0.21744	-652.763341448785\\
58.25	0.2211	-684.298462712564\\
58.25	0.22476	-716.655324851756\\
58.25	0.22842	-749.833927866361\\
58.25	0.23208	-783.83427175638\\
58.25	0.23574	-818.656356521812\\
58.25	0.2394	-854.300182162656\\
58.25	0.24306	-890.765748678914\\
58.25	0.24672	-928.053056070586\\
58.25	0.25038	-966.16210433767\\
58.25	0.25404	-1005.09289348017\\
58.25	0.2577	-1044.84542349808\\
58.25	0.26136	-1085.4196943914\\
58.25	0.26502	-1126.81570616014\\
58.25	0.26868	-1169.03345880429\\
58.25	0.27234	-1212.07295232385\\
58.25	0.276	-1255.93418671883\\
58.625	0.093	-69.1670437459534\\
58.625	0.09666	-72.7749376014311\\
58.625	0.10032	-77.2045723323224\\
58.625	0.10398	-82.4559479386266\\
58.625	0.10764	-88.5290644203441\\
58.625	0.1113	-95.4239217774749\\
58.625	0.11496	-103.140520010019\\
58.625	0.11862	-111.678859117976\\
58.625	0.12228	-121.038939101346\\
58.625	0.12594	-131.22075996013\\
58.625	0.1296	-142.224321694326\\
58.625	0.13326	-154.049624303936\\
58.625	0.13692	-166.696667788959\\
58.625	0.14058	-180.165452149395\\
58.625	0.14424	-194.455977385245\\
58.625	0.1479	-209.568243496508\\
58.625	0.15156	-225.502250483184\\
58.625	0.15522	-242.257998345272\\
58.625	0.15888	-259.835487082775\\
58.625	0.16254	-278.23471669569\\
58.625	0.1662	-297.455687184019\\
58.625	0.16986	-317.498398547761\\
58.625	0.17352	-338.362850786916\\
58.625	0.17718	-360.049043901484\\
58.625	0.18084	-382.556977891466\\
58.625	0.1845	-405.88665275686\\
58.625	0.18816	-430.038068497668\\
58.625	0.19182	-455.01122511389\\
58.625	0.19548	-480.806122605523\\
58.625	0.19914	-507.422760972571\\
58.625	0.2028	-534.861140215032\\
58.625	0.20646	-563.121260332906\\
58.625	0.21012	-592.203121326193\\
58.625	0.21378	-622.106723194893\\
58.625	0.21744	-652.832065939006\\
58.625	0.2211	-684.379149558533\\
58.625	0.22476	-716.747974053473\\
58.625	0.22842	-749.938539423826\\
58.625	0.23208	-783.950845669592\\
58.625	0.23574	-818.784892790772\\
58.625	0.2394	-854.440680787364\\
58.625	0.24306	-890.91820965937\\
58.625	0.24672	-928.217479406789\\
58.625	0.25038	-966.338490029622\\
58.625	0.25404	-1005.28124152787\\
58.625	0.2577	-1045.04573390153\\
58.625	0.26136	-1085.6319671506\\
58.625	0.26502	-1127.03994127508\\
58.625	0.26868	-1169.26965627498\\
58.625	0.27234	-1212.32111215029\\
58.625	0.276	-1256.19430890102\\
59	0.093	-68.9000818159197\\
59	0.09666	-72.5199380271451\\
59	0.10032	-76.9615351137841\\
59	0.10398	-82.2248730758361\\
59	0.10764	-88.3099519133012\\
59	0.1113	-95.2167716261797\\
59	0.11496	-102.945332214471\\
59	0.11862	-111.495633678176\\
59	0.12228	-120.867676017294\\
59	0.12594	-131.061459231826\\
59	0.1296	-142.07698332177\\
59	0.13326	-153.914248287128\\
59	0.13692	-166.573254127898\\
59	0.14058	-180.054000844083\\
59	0.14424	-194.35648843568\\
59	0.1479	-209.48071690269\\
59	0.15156	-225.426686245114\\
59	0.15522	-242.194396462951\\
59	0.15888	-259.783847556201\\
59	0.16254	-278.195039524864\\
59	0.1662	-297.42797236894\\
59	0.16986	-317.48264608843\\
59	0.17352	-338.359060683333\\
59	0.17718	-360.057216153649\\
59	0.18084	-382.577112499378\\
59	0.1845	-405.918749720521\\
59	0.18816	-430.082127817076\\
59	0.19182	-455.067246789045\\
59	0.19548	-480.874106636427\\
59	0.19914	-507.502707359222\\
59	0.2028	-534.953048957431\\
59	0.20646	-563.225131431052\\
59	0.21012	-592.318954780087\\
59	0.21378	-622.234519004535\\
59	0.21744	-652.971824104396\\
59	0.2211	-684.530870079671\\
59	0.22476	-716.911656930358\\
59	0.22842	-750.114184656459\\
59	0.23208	-784.138453257973\\
59	0.23574	-818.984462734901\\
59	0.2394	-854.652213087241\\
59	0.24306	-891.141704314994\\
59	0.24672	-928.452936418161\\
59	0.25038	-966.585909396741\\
59	0.25404	-1005.54062325073\\
59	0.2577	-1045.31707798014\\
59	0.26136	-1085.91527358496\\
59	0.26502	-1127.33521006519\\
59	0.26868	-1169.57688742084\\
59	0.27234	-1212.6403056519\\
59	0.276	-1256.52546475837\\
59.375	0.093	-68.7041535610545\\
59.375	0.09666	-72.3359721280279\\
59.375	0.10032	-76.7895315704145\\
59.375	0.10398	-82.0648318882142\\
59.375	0.10764	-88.1618730814272\\
59.375	0.1113	-95.0806551500533\\
59.375	0.11496	-102.821178094093\\
59.375	0.11862	-111.383441913545\\
59.375	0.12228	-120.767446608411\\
59.375	0.12594	-130.97319217869\\
59.375	0.1296	-142.000678624383\\
59.375	0.13326	-153.849905945488\\
59.375	0.13692	-166.520874142007\\
59.375	0.14058	-180.013583213938\\
59.375	0.14424	-194.328033161283\\
59.375	0.1479	-209.464223984042\\
59.375	0.15156	-225.422155682213\\
59.375	0.15522	-242.201828255797\\
59.375	0.15888	-259.803241704795\\
59.375	0.16254	-278.226396029207\\
59.375	0.1662	-297.47129122903\\
59.375	0.16986	-317.537927304268\\
59.375	0.17352	-338.426304254918\\
59.375	0.17718	-360.136422080982\\
59.375	0.18084	-382.668280782459\\
59.375	0.1845	-406.021880359349\\
59.375	0.18816	-430.197220811653\\
59.375	0.19182	-455.19430213937\\
59.375	0.19548	-481.013124342499\\
59.375	0.19914	-507.653687421042\\
59.375	0.2028	-535.115991374998\\
59.375	0.20646	-563.400036204368\\
59.375	0.21012	-592.50582190915\\
59.375	0.21378	-622.433348489346\\
59.375	0.21744	-653.182615944955\\
59.375	0.2211	-684.753624275977\\
59.375	0.22476	-717.146373482413\\
59.375	0.22842	-750.360863564261\\
59.375	0.23208	-784.397094521523\\
59.375	0.23574	-819.255066354198\\
59.375	0.2394	-854.934779062286\\
59.375	0.24306	-891.436232645788\\
59.375	0.24672	-928.759427104702\\
59.375	0.25038	-966.90436243903\\
59.375	0.25404	-1005.87103864877\\
59.375	0.2577	-1045.65945573393\\
59.375	0.26136	-1086.26961369449\\
59.375	0.26502	-1127.70151253047\\
59.375	0.26868	-1169.95515224187\\
59.375	0.27234	-1213.03053282867\\
59.375	0.276	-1256.92765429089\\
59.75	0.093	-68.5792589813578\\
59.75	0.09666	-72.2230399040792\\
59.75	0.10032	-76.6885617022135\\
59.75	0.10398	-81.975824375761\\
59.75	0.10764	-88.0848279247218\\
59.75	0.1113	-95.0155723490956\\
59.75	0.11496	-102.768057648883\\
59.75	0.11862	-111.342283824083\\
59.75	0.12228	-120.738250874697\\
59.75	0.12594	-130.955958800724\\
59.75	0.1296	-141.995407602164\\
59.75	0.13326	-153.856597279017\\
59.75	0.13692	-166.539527831283\\
59.75	0.14058	-180.044199258963\\
59.75	0.14424	-194.370611562056\\
59.75	0.1479	-209.518764740561\\
59.75	0.15156	-225.48865879448\\
59.75	0.15522	-242.280293723813\\
59.75	0.15888	-259.893669528558\\
59.75	0.16254	-278.328786208717\\
59.75	0.1662	-297.585643764289\\
59.75	0.16986	-317.664242195274\\
59.75	0.17352	-338.564581501672\\
59.75	0.17718	-360.286661683484\\
59.75	0.18084	-382.830482740709\\
59.75	0.1845	-406.196044673347\\
59.75	0.18816	-430.383347481398\\
59.75	0.19182	-455.392391164862\\
59.75	0.19548	-481.22317572374\\
59.75	0.19914	-507.875701158031\\
59.75	0.2028	-535.349967467735\\
59.75	0.20646	-563.645974652852\\
59.75	0.21012	-592.763722713382\\
59.75	0.21378	-622.703211649325\\
59.75	0.21744	-653.464441460682\\
59.75	0.2211	-685.047412147452\\
59.75	0.22476	-717.452123709635\\
59.75	0.22842	-750.678576147232\\
59.75	0.23208	-784.726769460241\\
59.75	0.23574	-819.596703648664\\
59.75	0.2394	-855.2883787125\\
59.75	0.24306	-891.801794651749\\
59.75	0.24672	-929.136951466411\\
59.75	0.25038	-967.293849156487\\
59.75	0.25404	-1006.27248772198\\
59.75	0.2577	-1046.07286716288\\
59.75	0.26136	-1086.69498747919\\
59.75	0.26502	-1128.13884867092\\
59.75	0.26868	-1170.40445073806\\
59.75	0.27234	-1213.49179368062\\
59.75	0.276	-1257.40087749859\\
60.125	0.093	-68.5253980768301\\
60.125	0.09666	-72.1811413552992\\
60.125	0.10032	-76.6586255091813\\
60.125	0.10398	-81.9578505384765\\
60.125	0.10764	-88.0788164431851\\
60.125	0.1113	-95.0215232233066\\
60.125	0.11496	-102.785970878842\\
60.125	0.11862	-111.37215940979\\
60.125	0.12228	-120.780088816151\\
60.125	0.12594	-131.009759097926\\
60.125	0.1296	-142.061170255113\\
60.125	0.13326	-153.934322287714\\
60.125	0.13692	-166.629215195728\\
60.125	0.14058	-180.145848979156\\
60.125	0.14424	-194.484223637996\\
60.125	0.1479	-209.64433917225\\
60.125	0.15156	-225.626195581917\\
60.125	0.15522	-242.429792866997\\
60.125	0.15888	-260.05513102749\\
60.125	0.16254	-278.502210063397\\
60.125	0.1662	-297.771029974716\\
60.125	0.16986	-317.86159076145\\
60.125	0.17352	-338.773892423595\\
60.125	0.17718	-360.507934961155\\
60.125	0.18084	-383.063718374127\\
60.125	0.1845	-406.441242662513\\
60.125	0.18816	-430.640507826312\\
60.125	0.19182	-455.661513865524\\
60.125	0.19548	-481.504260780149\\
60.125	0.19914	-508.168748570188\\
60.125	0.2028	-535.65497723564\\
60.125	0.20646	-563.962946776505\\
60.125	0.21012	-593.092657192783\\
60.125	0.21378	-623.044108484474\\
60.125	0.21744	-653.817300651578\\
60.125	0.2211	-685.412233694096\\
60.125	0.22476	-717.828907612027\\
60.125	0.22842	-751.067322405371\\
60.125	0.23208	-785.127478074128\\
60.125	0.23574	-820.009374618299\\
60.125	0.2394	-855.713012037883\\
60.125	0.24306	-892.238390332879\\
60.125	0.24672	-929.585509503289\\
60.125	0.25038	-967.754369549113\\
60.125	0.25404	-1006.74497047035\\
60.125	0.2577	-1046.557312267\\
60.125	0.26136	-1087.19139493906\\
60.125	0.26502	-1128.64721848654\\
60.125	0.26868	-1170.92478290943\\
60.125	0.27234	-1214.02408820773\\
60.125	0.276	-1257.94513438145\\
60.5	0.093	-68.5425708474715\\
60.5	0.09666	-72.210276481688\\
60.5	0.10032	-76.699722991318\\
60.5	0.10398	-82.010910376361\\
60.5	0.10764	-88.1438386368173\\
60.5	0.1113	-95.0985077726868\\
60.5	0.11496	-102.874917783969\\
60.5	0.11862	-111.473068670665\\
60.5	0.12228	-120.892960432774\\
60.5	0.12594	-131.134593070297\\
60.5	0.1296	-142.197966583232\\
60.5	0.13326	-154.083080971581\\
60.5	0.13692	-166.789936235342\\
60.5	0.14058	-180.318532374518\\
60.5	0.14424	-194.668869389106\\
60.5	0.1479	-209.840947279107\\
60.5	0.15156	-225.834766044522\\
60.5	0.15522	-242.65032568535\\
60.5	0.15888	-260.287626201591\\
60.5	0.16254	-278.746667593245\\
60.5	0.1662	-298.027449860313\\
60.5	0.16986	-318.129973002793\\
60.5	0.17352	-339.054237020687\\
60.5	0.17718	-360.800241913994\\
60.5	0.18084	-383.367987682714\\
60.5	0.1845	-406.757474326848\\
60.5	0.18816	-430.968701846395\\
60.5	0.19182	-456.001670241355\\
60.5	0.19548	-481.856379511728\\
60.5	0.19914	-508.532829657514\\
60.5	0.2028	-536.031020678714\\
60.5	0.20646	-564.350952575326\\
60.5	0.21012	-593.492625347352\\
60.5	0.21378	-623.456038994791\\
60.5	0.21744	-654.241193517643\\
60.5	0.2211	-685.848088915909\\
60.5	0.22476	-718.276725189587\\
60.5	0.22842	-751.527102338679\\
60.5	0.23208	-785.599220363184\\
60.5	0.23574	-820.493079263103\\
60.5	0.2394	-856.208679038434\\
60.5	0.24306	-892.746019689178\\
60.5	0.24672	-930.105101215336\\
60.5	0.25038	-968.285923616907\\
60.5	0.25404	-1007.28848689389\\
60.5	0.2577	-1047.11279104629\\
60.5	0.26136	-1087.7588360741\\
60.5	0.26502	-1129.22662197732\\
60.5	0.26868	-1171.51614875596\\
60.5	0.27234	-1214.62741641001\\
60.5	0.276	-1258.56042493947\\
60.875	0.093	-68.6307772932809\\
60.875	0.09666	-72.3104452832451\\
60.875	0.10032	-76.8118541486229\\
60.875	0.10398	-82.1350038894137\\
60.875	0.10764	-88.2798945056177\\
60.875	0.1113	-95.246525997235\\
60.875	0.11496	-103.034898364265\\
60.875	0.11862	-111.645011606709\\
60.875	0.12228	-121.076865724566\\
60.875	0.12594	-131.330460717836\\
60.875	0.1296	-142.405796586519\\
60.875	0.13326	-154.302873330616\\
60.875	0.13692	-167.021690950125\\
60.875	0.14058	-180.562249445048\\
60.875	0.14424	-194.924548815384\\
60.875	0.1479	-210.108589061133\\
60.875	0.15156	-226.114370182296\\
60.875	0.15522	-242.941892178871\\
60.875	0.15888	-260.59115505086\\
60.875	0.16254	-279.062158798262\\
60.875	0.1662	-298.354903421078\\
60.875	0.16986	-318.469388919306\\
60.875	0.17352	-339.405615292948\\
60.875	0.17718	-361.163582542002\\
60.875	0.18084	-383.74329066647\\
60.875	0.1845	-407.144739666352\\
60.875	0.18816	-431.367929541646\\
60.875	0.19182	-456.412860292354\\
60.875	0.19548	-482.279531918475\\
60.875	0.19914	-508.967944420009\\
60.875	0.2028	-536.478097796956\\
60.875	0.20646	-564.809992049316\\
60.875	0.21012	-593.96362717709\\
60.875	0.21378	-623.939003180277\\
60.875	0.21744	-654.736120058877\\
60.875	0.2211	-686.35497781289\\
60.875	0.22476	-718.795576442316\\
60.875	0.22842	-752.057915947156\\
60.875	0.23208	-786.141996327409\\
60.875	0.23574	-821.047817583075\\
60.875	0.2394	-856.775379714154\\
60.875	0.24306	-893.324682720646\\
60.875	0.24672	-930.695726602552\\
60.875	0.25038	-968.888511359871\\
60.875	0.25404	-1007.9030369926\\
60.875	0.2577	-1047.73930350075\\
60.875	0.26136	-1088.39731088431\\
60.875	0.26502	-1129.87705914328\\
60.875	0.26868	-1172.17854827766\\
60.875	0.27234	-1215.30177828746\\
60.875	0.276	-1259.24674917267\\
61.25	0.093	-68.7900174142592\\
61.25	0.09666	-72.4816477599717\\
61.25	0.10032	-76.995018981097\\
61.25	0.10398	-82.3301310776355\\
61.25	0.10764	-88.4869840495873\\
61.25	0.1113	-95.4655778969521\\
61.25	0.11496	-103.265912619731\\
61.25	0.11862	-111.887988217922\\
61.25	0.12228	-121.331804691527\\
61.25	0.12594	-131.597362040544\\
61.25	0.1296	-142.684660264975\\
61.25	0.13326	-154.593699364819\\
61.25	0.13692	-167.324479340077\\
61.25	0.14058	-180.877000190747\\
61.25	0.14424	-195.251261916831\\
61.25	0.1479	-210.447264518328\\
61.25	0.15156	-226.465007995238\\
61.25	0.15522	-243.304492347562\\
61.25	0.15888	-260.965717575298\\
61.25	0.16254	-279.448683678448\\
61.25	0.1662	-298.753390657011\\
61.25	0.16986	-318.879838510988\\
61.25	0.17352	-339.828027240377\\
61.25	0.17718	-361.597956845179\\
61.25	0.18084	-384.189627325395\\
61.25	0.1845	-407.603038681024\\
61.25	0.18816	-431.838190912066\\
61.25	0.19182	-456.895084018522\\
61.25	0.19548	-482.77371800039\\
61.25	0.19914	-509.474092857672\\
61.25	0.2028	-536.996208590367\\
61.25	0.20646	-565.340065198475\\
61.25	0.21012	-594.505662681997\\
61.25	0.21378	-624.493001040931\\
61.25	0.21744	-655.302080275279\\
61.25	0.2211	-686.93290038504\\
61.25	0.22476	-719.385461370214\\
61.25	0.22842	-752.659763230801\\
61.25	0.23208	-786.755805966802\\
61.25	0.23574	-821.673589578216\\
61.25	0.2394	-857.413114065043\\
61.25	0.24306	-893.974379427283\\
61.25	0.24672	-931.357385664936\\
61.25	0.25038	-969.562132778003\\
61.25	0.25404	-1008.58862076648\\
61.25	0.2577	-1048.43684963038\\
61.25	0.26136	-1089.10681936968\\
61.25	0.26502	-1130.5985299844\\
61.25	0.26868	-1172.91198147453\\
61.25	0.27234	-1216.04717384008\\
61.25	0.276	-1260.00410708104\\
61.625	0.093	-69.0202912104062\\
61.625	0.09666	-72.7238839118664\\
61.625	0.10032	-77.2492174887395\\
61.625	0.10398	-82.5962919410258\\
61.625	0.10764	-88.7651072687253\\
61.625	0.1113	-95.7556634718379\\
61.625	0.11496	-103.567960550364\\
61.625	0.11862	-112.201998504303\\
61.625	0.12228	-121.657777333656\\
61.625	0.12594	-131.935297038421\\
61.625	0.1296	-143.0345576186\\
61.625	0.13326	-154.955559074192\\
61.625	0.13692	-167.698301405197\\
61.625	0.14058	-181.262784611615\\
61.625	0.14424	-195.649008693447\\
61.625	0.1479	-210.856973650692\\
61.625	0.15156	-226.886679483349\\
61.625	0.15522	-243.738126191421\\
61.625	0.15888	-261.411313774905\\
61.625	0.16254	-279.906242233803\\
61.625	0.1662	-299.222911568113\\
61.625	0.16986	-319.361321777837\\
61.625	0.17352	-340.321472862974\\
61.625	0.17718	-362.103364823525\\
61.625	0.18084	-384.706997659488\\
61.625	0.1845	-408.132371370865\\
61.625	0.18816	-432.379485957655\\
61.625	0.19182	-457.448341419858\\
61.625	0.19548	-483.338937757474\\
61.625	0.19914	-510.051274970504\\
61.625	0.2028	-537.585353058947\\
61.625	0.20646	-565.941172022803\\
61.625	0.21012	-595.118731862072\\
61.625	0.21378	-625.118032576754\\
61.625	0.21744	-655.93907416685\\
61.625	0.2211	-687.581856632358\\
61.625	0.22476	-720.04637997328\\
61.625	0.22842	-753.332644189615\\
61.625	0.23208	-787.440649281364\\
61.625	0.23574	-822.370395248525\\
61.625	0.2394	-858.1218820911\\
61.625	0.24306	-894.695109809088\\
61.625	0.24672	-932.090078402489\\
61.625	0.25038	-970.306787871303\\
61.625	0.25404	-1009.34523821553\\
61.625	0.2577	-1049.20542943517\\
61.625	0.26136	-1089.88736153023\\
61.625	0.26502	-1131.39103450069\\
61.625	0.26868	-1173.71644834657\\
61.625	0.27234	-1216.86360306787\\
61.625	0.276	-1260.83249866457\\
62	0.093	-69.3215986817225\\
62	0.09666	-73.03715373893\\
62	0.10032	-77.574449671551\\
62	0.10398	-82.933486479585\\
62	0.10764	-89.1142641630324\\
62	0.1113	-96.1167827218929\\
62	0.11496	-103.941042156166\\
62	0.11862	-112.587042465854\\
62	0.12228	-122.054783650953\\
62	0.12594	-132.344265711467\\
62	0.1296	-143.455488647393\\
62	0.13326	-155.388452458733\\
62	0.13692	-168.143157145486\\
62	0.14058	-181.719602707652\\
62	0.14424	-196.117789145231\\
62	0.1479	-211.337716458224\\
62	0.15156	-227.37938464663\\
62	0.15522	-244.242793710448\\
62	0.15888	-261.927943649681\\
62	0.16254	-280.434834464326\\
62	0.1662	-299.763466154385\\
62	0.16986	-319.913838719856\\
62	0.17352	-340.885952160741\\
62	0.17718	-362.679806477039\\
62	0.18084	-385.29540166875\\
62	0.1845	-408.732737735875\\
62	0.18816	-432.991814678412\\
62	0.19182	-458.072632496364\\
62	0.19548	-483.975191189728\\
62	0.19914	-510.699490758505\\
62	0.2028	-538.245531202695\\
62	0.20646	-566.613312522299\\
62	0.21012	-595.802834717316\\
62	0.21378	-625.814097787746\\
62	0.21744	-656.647101733589\\
62	0.2211	-688.301846554846\\
62	0.22476	-720.778332251515\\
62	0.22842	-754.076558823598\\
62	0.23208	-788.196526271094\\
62	0.23574	-823.138234594004\\
62	0.2394	-858.901683792326\\
62	0.24306	-895.486873866062\\
62	0.24672	-932.893804815211\\
62	0.25038	-971.122476639773\\
62	0.25404	-1010.17288933975\\
62	0.2577	-1050.04504291514\\
62	0.26136	-1090.73893736594\\
62	0.26502	-1132.25457269215\\
62	0.26868	-1174.59194889378\\
62	0.27234	-1217.75106597082\\
62	0.276	-1261.73192392328\\
62.375	0.093	-69.6939398282066\\
62.375	0.09666	-73.4214572411619\\
62.375	0.10032	-77.9707155295307\\
62.375	0.10398	-83.3417146933124\\
62.375	0.10764	-89.5344547325075\\
62.375	0.1113	-96.5489356471159\\
62.375	0.11496	-104.385157437137\\
62.375	0.11862	-113.043120102572\\
62.375	0.12228	-122.52282364342\\
62.375	0.12594	-132.824268059681\\
62.375	0.1296	-143.947453351355\\
62.375	0.13326	-155.892379518443\\
62.375	0.13692	-168.659046560943\\
62.375	0.14058	-182.247454478857\\
62.375	0.14424	-196.657603272184\\
62.375	0.1479	-211.889492940925\\
62.375	0.15156	-227.943123485078\\
62.375	0.15522	-244.818494904645\\
62.375	0.15888	-262.515607199625\\
62.375	0.16254	-281.034460370018\\
62.375	0.1662	-300.375054415824\\
62.375	0.16986	-320.537389337043\\
62.375	0.17352	-341.521465133676\\
62.375	0.17718	-363.327281805722\\
62.375	0.18084	-385.954839353181\\
62.375	0.1845	-409.404137776053\\
62.375	0.18816	-433.675177074338\\
62.375	0.19182	-458.767957248037\\
62.375	0.19548	-484.682478297149\\
62.375	0.19914	-511.418740221674\\
62.375	0.2028	-538.976743021612\\
62.375	0.20646	-567.356486696964\\
62.375	0.21012	-596.557971247728\\
62.375	0.21378	-626.581196673906\\
62.375	0.21744	-657.426162975497\\
62.375	0.2211	-689.092870152501\\
62.375	0.22476	-721.581318204919\\
62.375	0.22842	-754.89150713275\\
62.375	0.23208	-789.023436935993\\
62.375	0.23574	-823.977107614651\\
62.375	0.2394	-859.752519168721\\
62.375	0.24306	-896.349671598204\\
62.375	0.24672	-933.768564903101\\
62.375	0.25038	-972.009199083411\\
62.375	0.25404	-1011.07157413913\\
62.375	0.2577	-1050.95569007027\\
62.375	0.26136	-1091.66154687682\\
62.375	0.26502	-1133.18914455878\\
62.375	0.26868	-1175.53848311616\\
62.375	0.27234	-1218.70956254895\\
62.375	0.276	-1262.70238285715\\
62.75	0.093	-70.1373146498598\\
62.75	0.09666	-73.8767944185632\\
62.75	0.10032	-78.4380150626795\\
62.75	0.10398	-83.8209765822091\\
62.75	0.10764	-90.0256789771519\\
62.75	0.1113	-97.0521222475078\\
62.75	0.11496	-104.900306393277\\
62.75	0.11862	-113.570231414459\\
62.75	0.12228	-123.061897311055\\
62.75	0.12594	-133.375304083064\\
62.75	0.1296	-144.510451730486\\
62.75	0.13326	-156.467340253321\\
62.75	0.13692	-169.24596965157\\
62.75	0.14058	-182.846339925232\\
62.75	0.14424	-197.268451074306\\
62.75	0.1479	-212.512303098794\\
62.75	0.15156	-228.577895998695\\
62.75	0.15522	-245.46522977401\\
62.75	0.15888	-263.174304424738\\
62.75	0.16254	-281.705119950879\\
62.75	0.1662	-301.057676352432\\
62.75	0.16986	-321.2319736294\\
62.75	0.17352	-342.22801178178\\
62.75	0.17718	-364.045790809573\\
62.75	0.18084	-386.68531071278\\
62.75	0.1845	-410.1465714914\\
62.75	0.18816	-434.429573145434\\
62.75	0.19182	-459.53431567488\\
62.75	0.19548	-485.46079907974\\
62.75	0.19914	-512.209023360012\\
62.75	0.2028	-539.778988515698\\
62.75	0.20646	-568.170694546798\\
62.75	0.21012	-597.38414145331\\
62.75	0.21378	-627.419329235236\\
62.75	0.21744	-658.276257892574\\
62.75	0.2211	-689.954927425326\\
62.75	0.22476	-722.455337833492\\
62.75	0.22842	-755.77748911707\\
62.75	0.23208	-789.921381276062\\
62.75	0.23574	-824.887014310467\\
62.75	0.2394	-860.674388220285\\
62.75	0.24306	-897.283503005516\\
62.75	0.24672	-934.71435866616\\
62.75	0.25038	-972.966955202218\\
62.75	0.25404	-1012.04129261369\\
62.75	0.2577	-1051.93737090057\\
62.75	0.26136	-1092.65519006287\\
62.75	0.26502	-1134.19475010058\\
62.75	0.26868	-1176.5560510137\\
62.75	0.27234	-1219.73909280224\\
62.75	0.276	-1263.74387546619\\
63.125	0.093	-70.6517231466816\\
63.125	0.09666	-74.4031652711328\\
63.125	0.10032	-78.9763482709969\\
63.125	0.10398	-84.3712721462742\\
63.125	0.10764	-90.5879368969648\\
63.125	0.1113	-97.6263425230684\\
63.125	0.11496	-105.486489024586\\
63.125	0.11862	-114.168376401516\\
63.125	0.12228	-123.672004653859\\
63.125	0.12594	-133.997373781616\\
63.125	0.1296	-145.144483784786\\
63.125	0.13326	-157.113334663368\\
63.125	0.13692	-169.903926417365\\
63.125	0.14058	-183.516259046774\\
63.125	0.14424	-197.950332551596\\
63.125	0.1479	-213.206146931832\\
63.125	0.15156	-229.283702187481\\
63.125	0.15522	-246.182998318544\\
63.125	0.15888	-263.904035325019\\
63.125	0.16254	-282.446813206908\\
63.125	0.1662	-301.811331964209\\
63.125	0.16986	-321.997591596925\\
63.125	0.17352	-343.005592105052\\
63.125	0.17718	-364.835333488594\\
63.125	0.18084	-387.486815747548\\
63.125	0.1845	-410.960038881916\\
63.125	0.18816	-435.255002891697\\
63.125	0.19182	-460.371707776891\\
63.125	0.19548	-486.310153537499\\
63.125	0.19914	-513.070340173519\\
63.125	0.2028	-540.652267684953\\
63.125	0.20646	-569.0559360718\\
63.125	0.21012	-598.28134533406\\
63.125	0.21378	-628.328495471734\\
63.125	0.21744	-659.19738648482\\
63.125	0.2211	-690.88801837332\\
63.125	0.22476	-723.400391137233\\
63.125	0.22842	-756.734504776559\\
63.125	0.23208	-790.890359291298\\
63.125	0.23574	-825.867954681451\\
63.125	0.2394	-861.667290947017\\
63.125	0.24306	-898.288368087996\\
63.125	0.24672	-935.731186104387\\
63.125	0.25038	-973.995744996193\\
63.125	0.25404	-1013.08204476341\\
63.125	0.2577	-1052.99008540604\\
63.125	0.26136	-1093.71986692409\\
63.125	0.26502	-1135.27138931755\\
63.125	0.26868	-1177.64465258642\\
63.125	0.27234	-1220.8396567307\\
63.125	0.276	-1264.8564017504\\
63.5	0.093	-71.2371653186723\\
63.5	0.09666	-75.0005697988711\\
63.5	0.10032	-79.5857151544831\\
63.5	0.10398	-84.9926013855082\\
63.5	0.10764	-91.2212284919465\\
63.5	0.1113	-98.2715964737981\\
63.5	0.11496	-106.143705331063\\
63.5	0.11862	-114.837555063741\\
63.5	0.12228	-124.353145671832\\
63.5	0.12594	-134.690477155336\\
63.5	0.1296	-145.849549514254\\
63.5	0.13326	-157.830362748584\\
63.5	0.13692	-170.632916858328\\
63.5	0.14058	-184.257211843485\\
63.5	0.14424	-198.703247704056\\
63.5	0.1479	-213.971024440039\\
63.5	0.15156	-230.060542051436\\
63.5	0.15522	-246.971800538246\\
63.5	0.15888	-264.70479990047\\
63.5	0.16254	-283.259540138106\\
63.5	0.1662	-302.636021251155\\
63.5	0.16986	-322.834243239618\\
63.5	0.17352	-343.854206103494\\
63.5	0.17718	-365.695909842783\\
63.5	0.18084	-388.359354457485\\
63.5	0.1845	-411.844539947601\\
63.5	0.18816	-436.151466313129\\
63.5	0.19182	-461.280133554071\\
63.5	0.19548	-487.230541670426\\
63.5	0.19914	-514.002690662195\\
63.5	0.2028	-541.596580529376\\
63.5	0.20646	-570.012211271971\\
63.5	0.21012	-599.249582889979\\
63.5	0.21378	-629.3086953834\\
63.5	0.21744	-660.189548752234\\
63.5	0.2211	-691.892142996482\\
63.5	0.22476	-724.416478116143\\
63.5	0.22842	-757.762554111217\\
63.5	0.23208	-791.930370981704\\
63.5	0.23574	-826.919928727604\\
63.5	0.2394	-862.731227348918\\
63.5	0.24306	-899.364266845644\\
63.5	0.24672	-936.819047217784\\
63.5	0.25038	-975.095568465337\\
63.5	0.25404	-1014.1938305883\\
63.5	0.2577	-1054.11383358668\\
63.5	0.26136	-1094.85557746048\\
63.5	0.26502	-1136.41906220968\\
63.5	0.26868	-1178.8042878343\\
63.5	0.27234	-1222.01125433433\\
63.5	0.276	-1266.03996170978\\
63.875	0.093	-71.8936411658314\\
63.875	0.09666	-75.6690080017777\\
63.875	0.10032	-80.2661157131375\\
63.875	0.10398	-85.6849642999103\\
63.875	0.10764	-91.9255537620966\\
63.875	0.1113	-98.987884099696\\
63.875	0.11496	-106.871955312708\\
63.875	0.11862	-115.577767401134\\
63.875	0.12228	-125.105320364973\\
63.875	0.12594	-135.454614204225\\
63.875	0.1296	-146.62564891889\\
63.875	0.13326	-158.618424508969\\
63.875	0.13692	-171.43294097446\\
63.875	0.14058	-185.069198315365\\
63.875	0.14424	-199.527196531683\\
63.875	0.1479	-214.806935623415\\
63.875	0.15156	-230.908415590559\\
63.875	0.15522	-247.831636433117\\
63.875	0.15888	-265.576598151088\\
63.875	0.16254	-284.143300744472\\
63.875	0.1662	-303.531744213269\\
63.875	0.16986	-323.74192855748\\
63.875	0.17352	-344.773853777104\\
63.875	0.17718	-366.62751987214\\
63.875	0.18084	-389.30292684259\\
63.875	0.1845	-412.800074688454\\
63.875	0.18816	-437.11896340973\\
63.875	0.19182	-462.25959300642\\
63.875	0.19548	-488.221963478523\\
63.875	0.19914	-515.006074826039\\
63.875	0.2028	-542.611927048968\\
63.875	0.20646	-571.039520147311\\
63.875	0.21012	-600.288854121066\\
63.875	0.21378	-630.359928970235\\
63.875	0.21744	-661.252744694817\\
63.875	0.2211	-692.967301294812\\
63.875	0.22476	-725.503598770221\\
63.875	0.22842	-758.861637121043\\
63.875	0.23208	-793.041416347277\\
63.875	0.23574	-828.042936448926\\
63.875	0.2394	-863.866197425987\\
63.875	0.24306	-900.511199278461\\
63.875	0.24672	-937.977942006348\\
63.875	0.25038	-976.26642560965\\
63.875	0.25404	-1015.37665008836\\
63.875	0.2577	-1055.30861544249\\
63.875	0.26136	-1096.06232167203\\
63.875	0.26502	-1137.63776877698\\
63.875	0.26868	-1180.03495675735\\
63.875	0.27234	-1223.25388561313\\
63.875	0.276	-1267.29455534433\\
64.25	0.093	-72.6211506881593\\
64.25	0.09666	-76.4084798798538\\
64.25	0.10032	-81.0175499469611\\
64.25	0.10398	-86.4483608894816\\
64.25	0.10764	-92.7009127074155\\
64.25	0.1113	-99.7752054007624\\
64.25	0.11496	-107.671238969523\\
64.25	0.11862	-116.389013413696\\
64.25	0.12228	-125.928528733283\\
64.25	0.12594	-136.289784928283\\
64.25	0.1296	-147.472781998696\\
64.25	0.13326	-159.477519944522\\
64.25	0.13692	-172.303998765762\\
64.25	0.14058	-185.952218462414\\
64.25	0.14424	-200.42217903448\\
64.25	0.1479	-215.713880481959\\
64.25	0.15156	-231.827322804851\\
64.25	0.15522	-248.762506003157\\
64.25	0.15888	-266.519430076876\\
64.25	0.16254	-285.098095026008\\
64.25	0.1662	-304.498500850552\\
64.25	0.16986	-324.720647550511\\
64.25	0.17352	-345.764535125882\\
64.25	0.17718	-367.630163576667\\
64.25	0.18084	-390.317532902864\\
64.25	0.1845	-413.826643104476\\
64.25	0.18816	-438.1574941815\\
64.25	0.19182	-463.310086133937\\
64.25	0.19548	-489.284418961788\\
64.25	0.19914	-516.080492665052\\
64.25	0.2028	-543.698307243729\\
64.25	0.20646	-572.137862697819\\
64.25	0.21012	-601.399159027323\\
64.25	0.21378	-631.482196232239\\
64.25	0.21744	-662.386974312569\\
64.25	0.2211	-694.113493268312\\
64.25	0.22476	-726.661753099468\\
64.25	0.22842	-760.031753806038\\
64.25	0.23208	-794.22349538802\\
64.25	0.23574	-829.236977845417\\
64.25	0.2394	-865.072201178225\\
64.25	0.24306	-901.729165386447\\
64.25	0.24672	-939.207870470083\\
64.25	0.25038	-977.508316429132\\
64.25	0.25404	-1016.63050326359\\
64.25	0.2577	-1056.57443097347\\
64.25	0.26136	-1097.34009955876\\
64.25	0.26502	-1138.92750901946\\
64.25	0.26868	-1181.33665935557\\
64.25	0.27234	-1224.5675505671\\
64.25	0.276	-1268.62018265404\\
64.625	0.093	-73.4196938856559\\
64.625	0.09666	-77.2189854330982\\
64.625	0.10032	-81.8400178559532\\
64.625	0.10398	-87.2827911542216\\
64.625	0.10764	-93.5473053279032\\
64.625	0.1113	-100.633560376998\\
64.625	0.11496	-108.541556301506\\
64.625	0.11862	-117.271293101427\\
64.625	0.12228	-126.822770776762\\
64.625	0.12594	-137.195989327509\\
64.625	0.1296	-148.39094875367\\
64.625	0.13326	-160.407649055244\\
64.625	0.13692	-173.246090232231\\
64.625	0.14058	-186.906272284632\\
64.625	0.14424	-201.388195212445\\
64.625	0.1479	-216.691859015672\\
64.625	0.15156	-232.817263694312\\
64.625	0.15522	-249.764409248365\\
64.625	0.15888	-267.533295677832\\
64.625	0.16254	-286.123922982712\\
64.625	0.1662	-305.536291163004\\
64.625	0.16986	-325.770400218711\\
64.625	0.17352	-346.826250149829\\
64.625	0.17718	-368.703840956362\\
64.625	0.18084	-391.403172638307\\
64.625	0.1845	-414.924245195666\\
64.625	0.18816	-439.267058628438\\
64.625	0.19182	-464.431612936623\\
64.625	0.19548	-490.417908120222\\
64.625	0.19914	-517.225944179233\\
64.625	0.2028	-544.855721113658\\
64.625	0.20646	-573.307238923496\\
64.625	0.21012	-602.580497608747\\
64.625	0.21378	-632.675497169412\\
64.625	0.21744	-663.592237605489\\
64.625	0.2211	-695.33071891698\\
64.625	0.22476	-727.890941103884\\
64.625	0.22842	-761.272904166201\\
64.625	0.23208	-795.476608103932\\
64.625	0.23574	-830.502052917076\\
64.625	0.2394	-866.349238605632\\
64.625	0.24306	-903.018165169602\\
64.625	0.24672	-940.508832608985\\
64.625	0.25038	-978.821240923782\\
64.625	0.25404	-1017.95539011399\\
64.625	0.2577	-1057.91128017961\\
64.625	0.26136	-1098.68891112065\\
64.625	0.26502	-1140.2882829371\\
64.625	0.26868	-1182.70939562896\\
64.625	0.27234	-1225.95224919624\\
64.625	0.276	-1270.01684363893\\
65	0.093	-74.2892707583216\\
65	0.09666	-78.1005246615111\\
65	0.10032	-82.7335194401144\\
65	0.10398	-88.1882550941305\\
65	0.10764	-94.4647316235599\\
65	0.1113	-101.562949028402\\
65	0.11496	-109.482907308658\\
65	0.11862	-118.224606464327\\
65	0.12228	-127.788046495409\\
65	0.12594	-138.173227401905\\
65	0.1296	-149.380149183813\\
65	0.13326	-161.408811841135\\
65	0.13692	-174.25921537387\\
65	0.14058	-187.931359782018\\
65	0.14424	-202.42524506558\\
65	0.1479	-217.740871224554\\
65	0.15156	-233.878238258942\\
65	0.15522	-250.837346168743\\
65	0.15888	-268.618194953957\\
65	0.16254	-287.220784614585\\
65	0.1662	-306.645115150625\\
65	0.16986	-326.891186562079\\
65	0.17352	-347.958998848946\\
65	0.17718	-369.848552011226\\
65	0.18084	-392.559846048919\\
65	0.1845	-416.092880962026\\
65	0.18816	-440.447656750545\\
65	0.19182	-465.624173414478\\
65	0.19548	-491.622430953825\\
65	0.19914	-518.442429368584\\
65	0.2028	-546.084168658757\\
65	0.20646	-574.547648824343\\
65	0.21012	-603.832869865341\\
65	0.21378	-633.939831781753\\
65	0.21744	-664.868534573579\\
65	0.2211	-696.618978240817\\
65	0.22476	-729.191162783469\\
65	0.22842	-762.585088201534\\
65	0.23208	-796.800754495012\\
65	0.23574	-831.838161663904\\
65	0.2394	-867.697309708208\\
65	0.24306	-904.378198627926\\
65	0.24672	-941.880828423057\\
65	0.25038	-980.205199093601\\
65	0.25404	-1019.35131063956\\
65	0.2577	-1059.31916306093\\
65	0.26136	-1100.10875635771\\
65	0.26502	-1141.72009052991\\
65	0.26868	-1184.15316557752\\
65	0.27234	-1227.40798150054\\
65	0.276	-1271.48453829898\\
65.375	0.093	-75.2298813061554\\
65.375	0.09666	-79.0530975650927\\
65.375	0.10032	-83.6980546994435\\
65.375	0.10398	-89.1647527092073\\
65.375	0.10764	-95.4531915943847\\
65.375	0.1113	-102.563371354975\\
65.375	0.11496	-110.495291990978\\
65.375	0.11862	-119.248953502395\\
65.375	0.12228	-128.824355889225\\
65.375	0.12594	-139.221499151468\\
65.375	0.1296	-150.440383289124\\
65.375	0.13326	-162.481008302194\\
65.375	0.13692	-175.343374190677\\
65.375	0.14058	-189.027480954573\\
65.375	0.14424	-203.533328593882\\
65.375	0.1479	-218.860917108604\\
65.375	0.15156	-235.01024649874\\
65.375	0.15522	-251.981316764288\\
65.375	0.15888	-269.774127905251\\
65.375	0.16254	-288.388679921625\\
65.375	0.1662	-307.824972813414\\
65.375	0.16986	-328.083006580615\\
65.375	0.17352	-349.16278122323\\
65.375	0.17718	-371.064296741258\\
65.375	0.18084	-393.787553134699\\
65.375	0.1845	-417.332550403553\\
65.375	0.18816	-441.699288547821\\
65.375	0.19182	-466.887767567502\\
65.375	0.19548	-492.897987462596\\
65.375	0.19914	-519.729948233103\\
65.375	0.2028	-547.383649879023\\
65.375	0.20646	-575.859092400357\\
65.375	0.21012	-605.156275797103\\
65.375	0.21378	-635.275200069263\\
65.375	0.21744	-666.215865216836\\
65.375	0.2211	-697.978271239822\\
65.375	0.22476	-730.562418138222\\
65.375	0.22842	-763.968305912035\\
65.375	0.23208	-798.195934561261\\
65.375	0.23574	-833.2453040859\\
65.375	0.2394	-869.116414485952\\
65.375	0.24306	-905.809265761417\\
65.375	0.24672	-943.323857912296\\
65.375	0.25038	-981.660190938589\\
65.375	0.25404	-1020.81826484029\\
65.375	0.2577	-1060.79807961741\\
65.375	0.26136	-1101.59963526994\\
65.375	0.26502	-1143.22293179789\\
65.375	0.26868	-1185.66796920125\\
65.375	0.27234	-1228.93474748002\\
65.375	0.276	-1273.0232666342\\
65.75	0.093	-76.2415255291582\\
65.75	0.09666	-80.0767041438437\\
65.75	0.10032	-84.733623633942\\
65.75	0.10398	-90.2122839994536\\
65.75	0.10764	-96.5126852403785\\
65.75	0.1113	-103.634827356717\\
65.75	0.11496	-111.578710348468\\
65.75	0.11862	-120.344334215632\\
65.75	0.12228	-129.93169895821\\
65.75	0.12594	-140.340804576201\\
65.75	0.1296	-151.571651069605\\
65.75	0.13326	-163.624238438422\\
65.75	0.13692	-176.498566682653\\
65.75	0.14058	-190.194635802297\\
65.75	0.14424	-204.712445797354\\
65.75	0.1479	-220.051996667824\\
65.75	0.15156	-236.213288413707\\
65.75	0.15522	-253.196321035003\\
65.75	0.15888	-271.001094531713\\
65.75	0.16254	-289.627608903836\\
65.75	0.1662	-309.075864151372\\
65.75	0.16986	-329.345860274322\\
65.75	0.17352	-350.437597272684\\
65.75	0.17718	-372.351075146459\\
65.75	0.18084	-395.086293895648\\
65.75	0.1845	-418.64325352025\\
65.75	0.18816	-443.021954020266\\
65.75	0.19182	-468.222395395694\\
65.75	0.19548	-494.244577646536\\
65.75	0.19914	-521.088500772791\\
65.75	0.2028	-548.754164774459\\
65.75	0.20646	-577.24156965154\\
65.75	0.21012	-606.550715404034\\
65.75	0.21378	-636.681602031942\\
65.75	0.21744	-667.634229535263\\
65.75	0.2211	-699.408597913997\\
65.75	0.22476	-732.004707168144\\
65.75	0.22842	-765.422557297705\\
65.75	0.23208	-799.662148302678\\
65.75	0.23574	-834.723480183065\\
65.75	0.2394	-870.606552938865\\
65.75	0.24306	-907.311366570079\\
65.75	0.24672	-944.837921076705\\
65.75	0.25038	-983.186216458745\\
65.75	0.25404	-1022.3562527162\\
65.75	0.2577	-1062.34802984906\\
65.75	0.26136	-1103.16154785734\\
65.75	0.26502	-1144.79680674104\\
65.75	0.26868	-1187.25380650014\\
65.75	0.27234	-1230.53254713466\\
65.75	0.276	-1274.63302864459\\
66.125	0.093	-77.3242034273296\\
66.125	0.09666	-81.1713443977628\\
66.125	0.10032	-85.8402262436089\\
66.125	0.10398	-91.3308489648683\\
66.125	0.10764	-97.6432125615409\\
66.125	0.1113	-104.777317033627\\
66.125	0.11496	-112.733162381126\\
66.125	0.11862	-121.510748604038\\
66.125	0.12228	-131.110075702364\\
66.125	0.12594	-141.531143676102\\
66.125	0.1296	-152.773952525254\\
66.125	0.13326	-164.838502249819\\
66.125	0.13692	-177.724792849797\\
66.125	0.14058	-191.432824325189\\
66.125	0.14424	-205.962596675993\\
66.125	0.1479	-221.314109902211\\
66.125	0.15156	-237.487364003842\\
66.125	0.15522	-254.482358980886\\
66.125	0.15888	-272.299094833344\\
66.125	0.16254	-290.937571561215\\
66.125	0.1662	-310.397789164498\\
66.125	0.16986	-330.679747643196\\
66.125	0.17352	-351.783446997306\\
66.125	0.17718	-373.708887226829\\
66.125	0.18084	-396.456068331766\\
66.125	0.1845	-420.024990312116\\
66.125	0.18816	-444.415653167879\\
66.125	0.19182	-469.628056899055\\
66.125	0.19548	-495.662201505644\\
66.125	0.19914	-522.518086987647\\
66.125	0.2028	-550.195713345063\\
66.125	0.20646	-578.695080577892\\
66.125	0.21012	-608.016188686134\\
66.125	0.21378	-638.159037669789\\
66.125	0.21744	-669.123627528858\\
66.125	0.2211	-700.90995826334\\
66.125	0.22476	-733.518029873235\\
66.125	0.22842	-766.947842358543\\
66.125	0.23208	-801.199395719264\\
66.125	0.23574	-836.272689955399\\
66.125	0.2394	-872.167725066947\\
66.125	0.24306	-908.884501053908\\
66.125	0.24672	-946.423017916282\\
66.125	0.25038	-984.78327565407\\
66.125	0.25404	-1023.96527426727\\
66.125	0.2577	-1063.96901375588\\
66.125	0.26136	-1104.79449411991\\
66.125	0.26502	-1146.44171535935\\
66.125	0.26868	-1188.9106774742\\
66.125	0.27234	-1232.20138046447\\
66.125	0.276	-1276.31382433015\\
66.5	0.093	-78.4779150006693\\
66.5	0.09666	-82.3370183268503\\
66.5	0.10032	-87.0178625284444\\
66.5	0.10398	-92.5204476054515\\
66.5	0.10764	-98.8447735578719\\
66.5	0.1113	-105.990840385705\\
66.5	0.11496	-113.958648088952\\
66.5	0.11862	-122.748196667612\\
66.5	0.12228	-132.359486121686\\
66.5	0.12594	-142.792516451172\\
66.5	0.1296	-154.047287656072\\
66.5	0.13326	-166.123799736384\\
66.5	0.13692	-179.02205269211\\
66.5	0.14058	-192.742046523249\\
66.5	0.14424	-207.283781229802\\
66.5	0.1479	-222.647256811768\\
66.5	0.15156	-238.832473269146\\
66.5	0.15522	-255.839430601938\\
66.5	0.15888	-273.668128810143\\
66.5	0.16254	-292.318567893762\\
66.5	0.1662	-311.790747852793\\
66.5	0.16986	-332.084668687239\\
66.5	0.17352	-353.200330397096\\
66.5	0.17718	-375.137732982367\\
66.5	0.18084	-397.896876443052\\
66.5	0.1845	-421.477760779149\\
66.5	0.18816	-445.88038599066\\
66.5	0.19182	-471.104752077584\\
66.5	0.19548	-497.150859039921\\
66.5	0.19914	-524.018706877672\\
66.5	0.2028	-551.708295590835\\
66.5	0.20646	-580.219625179412\\
66.5	0.21012	-609.552695643402\\
66.5	0.21378	-639.707506982805\\
66.5	0.21744	-670.684059197622\\
66.5	0.2211	-702.482352287851\\
66.5	0.22476	-735.102386253494\\
66.5	0.22842	-768.54416109455\\
66.5	0.23208	-802.807676811019\\
66.5	0.23574	-837.892933402901\\
66.5	0.2394	-873.799930870197\\
66.5	0.24306	-910.528669212906\\
66.5	0.24672	-948.079148431028\\
66.5	0.25038	-986.451368524563\\
66.5	0.25404	-1025.64532949351\\
66.5	0.2577	-1065.66103133787\\
66.5	0.26136	-1106.49847405765\\
66.5	0.26502	-1148.15765765284\\
66.5	0.26868	-1190.63858212344\\
66.5	0.27234	-1233.94124746945\\
66.5	0.276	-1278.06565369088\\
66.875	0.093	-79.7026602491782\\
66.875	0.09666	-83.573725931107\\
66.875	0.10032	-88.2665324884486\\
66.875	0.10398	-93.7810799212035\\
66.875	0.10764	-100.117368229372\\
66.875	0.1113	-107.275397412953\\
66.875	0.11496	-115.255167471948\\
66.875	0.11862	-124.056678406355\\
66.875	0.12228	-133.679930216176\\
66.875	0.12594	-144.12492290141\\
66.875	0.1296	-155.391656462058\\
66.875	0.13326	-167.480130898118\\
66.875	0.13692	-180.390346209592\\
66.875	0.14058	-194.122302396479\\
66.875	0.14424	-208.675999458779\\
66.875	0.1479	-224.051437396493\\
66.875	0.15156	-240.248616209619\\
66.875	0.15522	-257.267535898159\\
66.875	0.15888	-275.108196462112\\
66.875	0.16254	-293.770597901478\\
66.875	0.1662	-313.254740216257\\
66.875	0.16986	-333.56062340645\\
66.875	0.17352	-354.688247472056\\
66.875	0.17718	-376.637612413075\\
66.875	0.18084	-399.408718229507\\
66.875	0.1845	-423.001564921352\\
66.875	0.18816	-447.416152488611\\
66.875	0.19182	-472.652480931283\\
66.875	0.19548	-498.710550249368\\
66.875	0.19914	-525.590360442866\\
66.875	0.2028	-553.291911511777\\
66.875	0.20646	-581.815203456102\\
66.875	0.21012	-611.160236275839\\
66.875	0.21378	-641.32700997099\\
66.875	0.21744	-672.315524541554\\
66.875	0.2211	-704.125779987532\\
66.875	0.22476	-736.757776308922\\
66.875	0.22842	-770.211513505726\\
66.875	0.23208	-804.486991577942\\
66.875	0.23574	-839.584210525573\\
66.875	0.2394	-875.503170348617\\
66.875	0.24306	-912.243871047073\\
66.875	0.24672	-949.806312620942\\
66.875	0.25038	-988.190495070226\\
66.875	0.25404	-1027.39641839492\\
66.875	0.2577	-1067.42408259503\\
66.875	0.26136	-1108.27348767055\\
66.875	0.26502	-1149.94463362149\\
66.875	0.26868	-1192.43752044784\\
66.875	0.27234	-1235.7521481496\\
66.875	0.276	-1279.88851672678\\
67.25	0.093	-80.998439172856\\
67.25	0.09666	-84.8814672105325\\
67.25	0.10032	-89.5862361236219\\
67.25	0.10398	-95.1127459121245\\
67.25	0.10764	-101.46099657604\\
67.25	0.1113	-108.63098811537\\
67.25	0.11496	-116.622720530112\\
67.25	0.11862	-125.436193820267\\
67.25	0.12228	-135.071407985836\\
67.25	0.12594	-145.528363026818\\
67.25	0.1296	-156.807058943213\\
67.25	0.13326	-168.907495735021\\
67.25	0.13692	-181.829673402243\\
67.25	0.14058	-195.573591944878\\
67.25	0.14424	-210.139251362926\\
67.25	0.1479	-225.526651656387\\
67.25	0.15156	-241.735792825261\\
67.25	0.15522	-258.766674869549\\
67.25	0.15888	-276.619297789249\\
67.25	0.16254	-295.293661584363\\
67.25	0.1662	-314.78976625489\\
67.25	0.16986	-335.107611800831\\
67.25	0.17352	-356.247198222184\\
67.25	0.17718	-378.208525518951\\
67.25	0.18084	-400.991593691131\\
67.25	0.1845	-424.596402738724\\
67.25	0.18816	-449.02295266173\\
67.25	0.19182	-474.27124346015\\
67.25	0.19548	-500.341275133982\\
67.25	0.19914	-527.233047683228\\
67.25	0.2028	-554.946561107887\\
67.25	0.20646	-583.48181540796\\
67.25	0.21012	-612.838810583445\\
67.25	0.21378	-643.017546634344\\
67.25	0.21744	-674.018023560656\\
67.25	0.2211	-705.840241362381\\
67.25	0.22476	-738.484200039519\\
67.25	0.22842	-771.949899592071\\
67.25	0.23208	-806.237340020036\\
67.25	0.23574	-841.346521323414\\
67.25	0.2394	-877.277443502205\\
67.25	0.24306	-914.030106556408\\
67.25	0.24672	-951.604510486026\\
67.25	0.25038	-990.000655291057\\
67.25	0.25404	-1029.2185409715\\
67.25	0.2577	-1069.25816752736\\
67.25	0.26136	-1110.11953495863\\
67.25	0.26502	-1151.80264326531\\
67.25	0.26868	-1194.30749244741\\
67.25	0.27234	-1237.63408250492\\
67.25	0.276	-1281.78241343784\\
67.625	0.093	-82.365251771702\\
67.625	0.09666	-86.2602421651266\\
67.625	0.10032	-90.9769734339637\\
67.625	0.10398	-96.515445578214\\
67.625	0.10764	-102.875658597878\\
67.625	0.1113	-110.057612492954\\
67.625	0.11496	-118.061307263445\\
67.625	0.11862	-126.886742909348\\
67.625	0.12228	-136.533919430665\\
67.625	0.12594	-147.002836827394\\
67.625	0.1296	-158.293495099537\\
67.625	0.13326	-170.405894247093\\
67.625	0.13692	-183.340034270062\\
67.625	0.14058	-197.095915168445\\
67.625	0.14424	-211.673536942241\\
67.625	0.1479	-227.072899591449\\
67.625	0.15156	-243.294003116072\\
67.625	0.15522	-260.336847516107\\
67.625	0.15888	-278.201432791555\\
67.625	0.16254	-296.887758942417\\
67.625	0.1662	-316.395825968692\\
67.625	0.16986	-336.72563387038\\
67.625	0.17352	-357.877182647481\\
67.625	0.17718	-379.850472299995\\
67.625	0.18084	-402.645502827923\\
67.625	0.1845	-426.262274231264\\
67.625	0.18816	-450.700786510018\\
67.625	0.19182	-475.961039664186\\
67.625	0.19548	-502.043033693766\\
67.625	0.19914	-528.946768598759\\
67.625	0.2028	-556.672244379166\\
67.625	0.20646	-585.219461034987\\
67.625	0.21012	-614.58841856622\\
67.625	0.21378	-644.779116972866\\
67.625	0.21744	-675.791556254926\\
67.625	0.2211	-707.625736412399\\
67.625	0.22476	-740.281657445284\\
67.625	0.22842	-773.759319353584\\
67.625	0.23208	-808.058722137297\\
67.625	0.23574	-843.179865796423\\
67.625	0.2394	-879.122750330961\\
67.625	0.24306	-915.887375740913\\
67.625	0.24672	-953.473742026278\\
67.625	0.25038	-991.881849187057\\
67.625	0.25404	-1031.11169722325\\
67.625	0.2577	-1071.16328613485\\
67.625	0.26136	-1112.03661592187\\
67.625	0.26502	-1153.7316865843\\
67.625	0.26868	-1196.24849812215\\
67.625	0.27234	-1239.5870505354\\
67.625	0.276	-1283.74734382407\\
68	0.093	-83.8030980457169\\
68	0.09666	-87.710050794889\\
68	0.10032	-92.4387444194741\\
68	0.10398	-97.9891789194722\\
68	0.10764	-104.361354294884\\
68	0.1113	-111.555270545708\\
68	0.11496	-119.570927671946\\
68	0.11862	-128.408325673597\\
68	0.12228	-138.067464550661\\
68	0.12594	-148.548344303139\\
68	0.1296	-159.850964931029\\
68	0.13326	-171.975326434333\\
68	0.13692	-184.92142881305\\
68	0.14058	-198.689272067181\\
68	0.14424	-213.278856196724\\
68	0.1479	-228.690181201681\\
68	0.15156	-244.92324708205\\
68	0.15522	-261.978053837833\\
68	0.15888	-279.85460146903\\
68	0.16254	-298.552889975639\\
68	0.1662	-318.072919357662\\
68	0.16986	-338.414689615098\\
68	0.17352	-359.578200747947\\
68	0.17718	-381.563452756209\\
68	0.18084	-404.370445639884\\
68	0.1845	-427.999179398973\\
68	0.18816	-452.449654033475\\
68	0.19182	-477.72186954339\\
68	0.19548	-503.815825928718\\
68	0.19914	-530.731523189459\\
68	0.2028	-558.468961325614\\
68	0.20646	-587.028140337182\\
68	0.21012	-616.409060224163\\
68	0.21378	-646.611720986557\\
68	0.21744	-677.636122624365\\
68	0.2211	-709.482265137585\\
68	0.22476	-742.150148526219\\
68	0.22842	-775.639772790266\\
68	0.23208	-809.951137929726\\
68	0.23574	-845.0842439446\\
68	0.2394	-881.039090834886\\
68	0.24306	-917.815678600586\\
68	0.24672	-955.414007241699\\
68	0.25038	-993.834076758225\\
68	0.25404	-1033.07588715016\\
68	0.2577	-1073.13943841752\\
68	0.26136	-1114.02473056028\\
68	0.26502	-1155.73176357846\\
68	0.26868	-1198.26053747205\\
68	0.27234	-1241.61105224106\\
68	0.276	-1285.78330788548\\
68.375	0.093	-85.3119779949004\\
68.375	0.09666	-89.2308930998202\\
68.375	0.10032	-93.9715490801528\\
68.375	0.10398	-99.5339459358989\\
68.375	0.10764	-105.918083667058\\
68.375	0.1113	-113.12396227363\\
68.375	0.11496	-121.151581755616\\
68.375	0.11862	-130.000942113015\\
68.375	0.12228	-139.672043345827\\
68.375	0.12594	-150.164885454052\\
68.375	0.1296	-161.47946843769\\
68.375	0.13326	-173.615792296742\\
68.375	0.13692	-186.573857031207\\
68.375	0.14058	-200.353662641085\\
68.375	0.14424	-214.955209126376\\
68.375	0.1479	-230.37849648708\\
68.375	0.15156	-246.623524723198\\
68.375	0.15522	-263.690293834729\\
68.375	0.15888	-281.578803821673\\
68.375	0.16254	-300.28905468403\\
68.375	0.1662	-319.8210464218\\
68.375	0.16986	-340.174779034984\\
68.375	0.17352	-361.350252523581\\
68.375	0.17718	-383.34746688759\\
68.375	0.18084	-406.166422127014\\
68.375	0.1845	-429.80711824185\\
68.375	0.18816	-454.2695552321\\
68.375	0.19182	-479.553733097763\\
68.375	0.19548	-505.659651838838\\
68.375	0.19914	-532.587311455328\\
68.375	0.2028	-560.33671194723\\
68.375	0.20646	-588.907853314546\\
68.375	0.21012	-618.300735557274\\
68.375	0.21378	-648.515358675416\\
68.375	0.21744	-679.551722668972\\
68.375	0.2211	-711.40982753794\\
68.375	0.22476	-744.089673282321\\
68.375	0.22842	-777.591259902116\\
68.375	0.23208	-811.914587397324\\
68.375	0.23574	-847.059655767946\\
68.375	0.2394	-883.02646501398\\
68.375	0.24306	-919.815015135427\\
68.375	0.24672	-957.425306132288\\
68.375	0.25038	-995.857338004562\\
68.375	0.25404	-1035.11111075225\\
68.375	0.2577	-1075.18662437535\\
68.375	0.26136	-1116.08387887386\\
68.375	0.26502	-1157.80287424779\\
68.375	0.26868	-1200.34361049713\\
68.375	0.27234	-1243.70608762188\\
68.375	0.276	-1287.89030562205\\
68.75	0.093	-86.891891619253\\
68.75	0.09666	-90.8227690799206\\
68.75	0.10032	-95.575387416001\\
68.75	0.10398	-101.149746627495\\
68.75	0.10764	-107.545846714402\\
68.75	0.1113	-114.763687676722\\
68.75	0.11496	-122.803269514455\\
68.75	0.11862	-131.664592227602\\
68.75	0.12228	-141.347655816161\\
68.75	0.12594	-151.852460280134\\
68.75	0.1296	-163.179005619521\\
68.75	0.13326	-175.32729183432\\
68.75	0.13692	-188.297318924532\\
68.75	0.14058	-202.089086890158\\
68.75	0.14424	-216.702595731197\\
68.75	0.1479	-232.137845447649\\
68.75	0.15156	-248.394836039515\\
68.75	0.15522	-265.473567506793\\
68.75	0.15888	-283.374039849485\\
68.75	0.16254	-302.09625306759\\
68.75	0.1662	-321.640207161108\\
68.75	0.16986	-342.00590213004\\
68.75	0.17352	-363.193337974384\\
68.75	0.17718	-385.202514694142\\
68.75	0.18084	-408.033432289312\\
68.75	0.1845	-431.686090759897\\
68.75	0.18816	-456.160490105894\\
68.75	0.19182	-481.456630327305\\
68.75	0.19548	-507.574511424128\\
68.75	0.19914	-534.514133396365\\
68.75	0.2028	-562.275496244016\\
68.75	0.20646	-590.858599967079\\
68.75	0.21012	-620.263444565556\\
68.75	0.21378	-650.490030039445\\
68.75	0.21744	-681.538356388748\\
68.75	0.2211	-713.408423613464\\
68.75	0.22476	-746.100231713594\\
68.75	0.22842	-779.613780689136\\
68.75	0.23208	-813.949070540091\\
68.75	0.23574	-849.106101266461\\
68.75	0.2394	-885.084872868243\\
68.75	0.24306	-921.885385345438\\
68.75	0.24672	-959.507638698046\\
68.75	0.25038	-997.951632926068\\
68.75	0.25404	-1037.2173680295\\
68.75	0.2577	-1077.30484400835\\
68.75	0.26136	-1118.21406086261\\
68.75	0.26502	-1159.94501859229\\
68.75	0.26868	-1202.49771719737\\
68.75	0.27234	-1245.87215667788\\
68.75	0.276	-1290.06833703379\\
69.125	0.093	-88.5428389187738\\
69.125	0.09666	-92.4856787351894\\
69.125	0.10032	-97.2502594270175\\
69.125	0.10398	-102.836580994259\\
69.125	0.10764	-109.244643436914\\
69.125	0.1113	-116.474446754981\\
69.125	0.11496	-124.525990948463\\
69.125	0.11862	-133.399276017357\\
69.125	0.12228	-143.094301961665\\
69.125	0.12594	-153.611068781385\\
69.125	0.1296	-164.949576476519\\
69.125	0.13326	-177.109825047066\\
69.125	0.13692	-190.091814493027\\
69.125	0.14058	-203.8955448144\\
69.125	0.14424	-218.521016011187\\
69.125	0.1479	-233.968228083387\\
69.125	0.15156	-250.237181031\\
69.125	0.15522	-267.327874854026\\
69.125	0.15888	-285.240309552466\\
69.125	0.16254	-303.974485126319\\
69.125	0.1662	-323.530401575584\\
69.125	0.16986	-343.908058900263\\
69.125	0.17352	-365.107457100356\\
69.125	0.17718	-387.128596175861\\
69.125	0.18084	-409.97147612678\\
69.125	0.1845	-433.636096953112\\
69.125	0.18816	-458.122458654857\\
69.125	0.19182	-483.430561232015\\
69.125	0.19548	-509.560404684587\\
69.125	0.19914	-536.511989012571\\
69.125	0.2028	-564.285314215969\\
69.125	0.20646	-592.88038029478\\
69.125	0.21012	-622.297187249005\\
69.125	0.21378	-652.535735078642\\
69.125	0.21744	-683.596023783693\\
69.125	0.2211	-715.478053364156\\
69.125	0.22476	-748.181823820034\\
69.125	0.22842	-781.707335151324\\
69.125	0.23208	-816.054587358027\\
69.125	0.23574	-851.223580440144\\
69.125	0.2394	-887.214314397674\\
69.125	0.24306	-924.026789230617\\
69.125	0.24672	-961.661004938973\\
69.125	0.25038	-1000.11696152274\\
69.125	0.25404	-1039.39465898193\\
69.125	0.2577	-1079.49409731652\\
69.125	0.26136	-1120.41527652653\\
69.125	0.26502	-1162.15819661195\\
69.125	0.26868	-1204.72285757279\\
69.125	0.27234	-1248.10925940904\\
69.125	0.276	-1292.3174021207\\
69.5	0.093	-90.2648198934634\\
69.5	0.09666	-94.2196220656265\\
69.5	0.10032	-98.9961651132026\\
69.5	0.10398	-104.594449036192\\
69.5	0.10764	-111.014473834594\\
69.5	0.1113	-118.25623950841\\
69.5	0.11496	-126.319746057639\\
69.5	0.11862	-135.204993482281\\
69.5	0.12228	-144.911981782336\\
69.5	0.12594	-155.440710957804\\
69.5	0.1296	-166.791181008686\\
69.5	0.13326	-178.963391934981\\
69.5	0.13692	-191.957343736689\\
69.5	0.14058	-205.77303641381\\
69.5	0.14424	-220.410469966345\\
69.5	0.1479	-235.869644394293\\
69.5	0.15156	-252.150559697653\\
69.5	0.15522	-269.253215876427\\
69.5	0.15888	-287.177612930615\\
69.5	0.16254	-305.923750860215\\
69.5	0.1662	-325.491629665229\\
69.5	0.16986	-345.881249345656\\
69.5	0.17352	-367.092609901496\\
69.5	0.17718	-389.125711332749\\
69.5	0.18084	-411.980553639415\\
69.5	0.1845	-435.657136821495\\
69.5	0.18816	-460.155460878988\\
69.5	0.19182	-485.475525811894\\
69.5	0.19548	-511.617331620213\\
69.5	0.19914	-538.580878303946\\
69.5	0.2028	-566.366165863092\\
69.5	0.20646	-594.97319429765\\
69.5	0.21012	-624.401963607623\\
69.5	0.21378	-654.652473793008\\
69.5	0.21744	-685.724724853806\\
69.5	0.2211	-717.618716790018\\
69.5	0.22476	-750.334449601643\\
69.5	0.22842	-783.871923288681\\
69.5	0.23208	-818.231137851132\\
69.5	0.23574	-853.412093288996\\
69.5	0.2394	-889.414789602274\\
69.5	0.24306	-926.239226790965\\
69.5	0.24672	-963.885404855068\\
69.5	0.25038	-1002.35332379459\\
69.5	0.25404	-1041.64298360952\\
69.5	0.2577	-1081.75438429986\\
69.5	0.26136	-1122.68752586562\\
69.5	0.26502	-1164.44240830679\\
69.5	0.26868	-1207.01903162337\\
69.5	0.27234	-1250.41739581537\\
69.5	0.276	-1294.63750088278\\
69.875	0.093	-92.0578345433218\\
69.875	0.09666	-96.0245990712326\\
69.875	0.10032	-100.813104474556\\
69.875	0.10398	-106.423350753293\\
69.875	0.10764	-112.855337907444\\
69.875	0.1113	-120.109065937007\\
69.875	0.11496	-128.184534841984\\
69.875	0.11862	-137.081744622373\\
69.875	0.12228	-146.800695278177\\
69.875	0.12594	-157.341386809393\\
69.875	0.1296	-168.703819216022\\
69.875	0.13326	-180.887992498065\\
69.875	0.13692	-193.893906655521\\
69.875	0.14058	-207.72156168839\\
69.875	0.14424	-222.370957596672\\
69.875	0.1479	-237.842094380367\\
69.875	0.15156	-254.134972039476\\
69.875	0.15522	-271.249590573998\\
69.875	0.15888	-289.185949983933\\
69.875	0.16254	-307.944050269281\\
69.875	0.1662	-327.523891430042\\
69.875	0.16986	-347.925473466217\\
69.875	0.17352	-369.148796377805\\
69.875	0.17718	-391.193860164806\\
69.875	0.18084	-414.06066482722\\
69.875	0.1845	-437.749210365047\\
69.875	0.18816	-462.259496778288\\
69.875	0.19182	-487.591524066942\\
69.875	0.19548	-513.745292231009\\
69.875	0.19914	-540.720801270489\\
69.875	0.2028	-568.518051185382\\
69.875	0.20646	-597.137041975689\\
69.875	0.21012	-626.577773641409\\
69.875	0.21378	-656.840246182542\\
69.875	0.21744	-687.924459599088\\
69.875	0.2211	-719.830413891047\\
69.875	0.22476	-752.55810905842\\
69.875	0.22842	-786.107545101206\\
69.875	0.23208	-820.478722019405\\
69.875	0.23574	-855.671639813017\\
69.875	0.2394	-891.686298482043\\
69.875	0.24306	-928.522698026481\\
69.875	0.24672	-966.180838446333\\
69.875	0.25038	-1004.6607197416\\
69.875	0.25404	-1043.96234191228\\
69.875	0.2577	-1084.08570495837\\
69.875	0.26136	-1125.03080887987\\
69.875	0.26502	-1166.79765367679\\
69.875	0.26868	-1209.38623934912\\
69.875	0.27234	-1252.79656589686\\
69.875	0.276	-1297.02863332002\\
70.25	0.093	-93.921882868349\\
70.25	0.09666	-97.9006097520078\\
70.25	0.10032	-102.701077511079\\
70.25	0.10398	-108.323286145564\\
70.25	0.10764	-114.767235655462\\
70.25	0.1113	-122.032926040773\\
70.25	0.11496	-130.120357301498\\
70.25	0.11862	-139.029529437635\\
70.25	0.12228	-148.760442449186\\
70.25	0.12594	-159.31309633615\\
70.25	0.1296	-170.687491098527\\
70.25	0.13326	-182.883626736317\\
70.25	0.13692	-195.901503249521\\
70.25	0.14058	-209.741120638138\\
70.25	0.14424	-224.402478902168\\
70.25	0.1479	-239.885578041611\\
70.25	0.15156	-256.190418056467\\
70.25	0.15522	-273.316998946737\\
70.25	0.15888	-291.26532071242\\
70.25	0.16254	-310.035383353516\\
70.25	0.1662	-329.627186870025\\
70.25	0.16986	-350.040731261947\\
70.25	0.17352	-371.276016529283\\
70.25	0.17718	-393.333042672031\\
70.25	0.18084	-416.211809690193\\
70.25	0.1845	-439.912317583769\\
70.25	0.18816	-464.434566352757\\
70.25	0.19182	-489.778555997159\\
70.25	0.19548	-515.944286516973\\
70.25	0.19914	-542.931757912201\\
70.25	0.2028	-570.740970182843\\
70.25	0.20646	-599.371923328897\\
70.25	0.21012	-628.824617350364\\
70.25	0.21378	-659.099052247245\\
70.25	0.21744	-690.195228019539\\
70.25	0.2211	-722.113144667246\\
70.25	0.22476	-754.852802190366\\
70.25	0.22842	-788.4142005889\\
70.25	0.23208	-822.797339862847\\
70.25	0.23574	-858.002220012207\\
70.25	0.2394	-894.02884103698\\
70.25	0.24306	-930.877202937166\\
70.25	0.24672	-968.547305712766\\
70.25	0.25038	-1007.03914936378\\
70.25	0.25404	-1046.3527338902\\
70.25	0.2577	-1086.48805929204\\
70.25	0.26136	-1127.4451255693\\
70.25	0.26502	-1169.22393272196\\
70.25	0.26868	-1211.82448075004\\
70.25	0.27234	-1255.24676965353\\
70.25	0.276	-1299.49079943244\\
70.625	0.093	-95.8569648685447\\
70.625	0.09666	-99.847654107951\\
70.625	0.10032	-104.66008422277\\
70.625	0.10398	-110.294255213003\\
70.625	0.10764	-116.750167078649\\
70.625	0.1113	-124.027819819707\\
70.625	0.11496	-132.12721343618\\
70.625	0.11862	-141.048347928065\\
70.625	0.12228	-150.791223295364\\
70.625	0.12594	-161.355839538075\\
70.625	0.1296	-172.7421966562\\
70.625	0.13326	-184.950294649738\\
70.625	0.13692	-197.98013351869\\
70.625	0.14058	-211.831713263054\\
70.625	0.14424	-226.505033882832\\
70.625	0.1479	-242.000095378023\\
70.625	0.15156	-258.316897748627\\
70.625	0.15522	-275.455440994644\\
70.625	0.15888	-293.415725116075\\
70.625	0.16254	-312.197750112919\\
70.625	0.1662	-331.801515985176\\
70.625	0.16986	-352.227022732846\\
70.625	0.17352	-373.474270355929\\
70.625	0.17718	-395.543258854426\\
70.625	0.18084	-418.433988228335\\
70.625	0.1845	-442.146458477658\\
70.625	0.18816	-466.680669602395\\
70.625	0.19182	-492.036621602544\\
70.625	0.19548	-518.214314478106\\
70.625	0.19914	-545.213748229082\\
70.625	0.2028	-573.034922855471\\
70.625	0.20646	-601.677838357273\\
70.625	0.21012	-631.142494734488\\
70.625	0.21378	-661.428891987117\\
70.625	0.21744	-692.537030115159\\
70.625	0.2211	-724.466909118613\\
70.625	0.22476	-757.218528997481\\
70.625	0.22842	-790.791889751763\\
70.625	0.23208	-825.186991381458\\
70.625	0.23574	-860.403833886566\\
70.625	0.2394	-896.442417267086\\
70.625	0.24306	-933.30274152302\\
70.625	0.24672	-970.984806654368\\
70.625	0.25038	-1009.48861266113\\
70.625	0.25404	-1048.8141595433\\
70.625	0.2577	-1088.96144730089\\
70.625	0.26136	-1129.93047593389\\
70.625	0.26502	-1171.7212454423\\
70.625	0.26868	-1214.33375582613\\
70.625	0.27234	-1257.76800708537\\
70.625	0.276	-1302.02399922002\\
71	0.093	-97.863080543909\\
71	0.09666	-101.865732139063\\
71	0.10032	-106.69012460963\\
71	0.10398	-112.33625795561\\
71	0.10764	-118.804132177004\\
71	0.1113	-126.09374727381\\
71	0.11496	-134.205103246031\\
71	0.11862	-143.138200093663\\
71	0.12228	-152.89303781671\\
71	0.12594	-163.469616415169\\
71	0.1296	-174.867935889042\\
71	0.13326	-187.087996238328\\
71	0.13692	-200.129797463027\\
71	0.14058	-213.993339563139\\
71	0.14424	-228.678622538665\\
71	0.1479	-244.185646389604\\
71	0.15156	-260.514411115955\\
71	0.15522	-277.664916717721\\
71	0.15888	-295.637163194899\\
71	0.16254	-314.431150547491\\
71	0.1662	-334.046878775495\\
71	0.16986	-354.484347878913\\
71	0.17352	-375.743557857744\\
71	0.17718	-397.824508711988\\
71	0.18084	-420.727200441646\\
71	0.1845	-444.451633046717\\
71	0.18816	-468.997806527201\\
71	0.19182	-494.365720883098\\
71	0.19548	-520.555376114408\\
71	0.19914	-547.566772221131\\
71	0.2028	-575.399909203268\\
71	0.20646	-604.054787060818\\
71	0.21012	-633.531405793781\\
71	0.21378	-663.829765402157\\
71	0.21744	-694.949865885947\\
71	0.2211	-726.891707245149\\
71	0.22476	-759.655289479765\\
71	0.22842	-793.240612589794\\
71	0.23208	-827.647676575237\\
71	0.23574	-862.876481436092\\
71	0.2394	-898.927027172361\\
71	0.24306	-935.799313784042\\
71	0.24672	-973.493341271138\\
71	0.25038	-1012.00910963365\\
71	0.25404	-1051.34661887157\\
71	0.2577	-1091.5058689849\\
71	0.26136	-1132.48685997365\\
71	0.26502	-1174.28959183781\\
71	0.26868	-1216.91406457739\\
71	0.27234	-1260.36027819237\\
71	0.276	-1304.62823268277\\
71.375	0.093	-99.9402298944421\\
71.375	0.09666	-103.954843845344\\
71.375	0.10032	-108.791198671659\\
71.375	0.10398	-114.449294373387\\
71.375	0.10764	-120.929130950528\\
71.375	0.1113	-128.230708403082\\
71.375	0.11496	-136.35402673105\\
71.375	0.11862	-145.299085934431\\
71.375	0.12228	-155.065886013225\\
71.375	0.12594	-165.654426967432\\
71.375	0.1296	-177.064708797053\\
71.375	0.13326	-189.296731502086\\
71.375	0.13692	-202.350495082533\\
71.375	0.14058	-216.225999538393\\
71.375	0.14424	-230.923244869667\\
71.375	0.1479	-246.442231076353\\
71.375	0.15156	-262.782958158453\\
71.375	0.15522	-279.945426115965\\
71.375	0.15888	-297.929634948891\\
71.375	0.16254	-316.735584657231\\
71.375	0.1662	-336.363275240983\\
71.375	0.16986	-356.812706700149\\
71.375	0.17352	-378.083879034728\\
71.375	0.17718	-400.17679224472\\
71.375	0.18084	-423.091446330125\\
71.375	0.1845	-446.827841290944\\
71.375	0.18816	-471.385977127175\\
71.375	0.19182	-496.76585383882\\
71.375	0.19548	-522.967471425878\\
71.375	0.19914	-549.990829888349\\
71.375	0.2028	-577.835929226234\\
71.375	0.20646	-606.502769439532\\
71.375	0.21012	-635.991350528242\\
71.375	0.21378	-666.301672492366\\
71.375	0.21744	-697.433735331904\\
71.375	0.2211	-729.387539046854\\
71.375	0.22476	-762.163083637217\\
71.375	0.22842	-795.760369102994\\
71.375	0.23208	-830.179395444184\\
71.375	0.23574	-865.420162660788\\
71.375	0.2394	-901.482670752804\\
71.375	0.24306	-938.366919720233\\
71.375	0.24672	-976.072909563076\\
71.375	0.25038	-1014.60064028133\\
71.375	0.25404	-1053.950111875\\
71.375	0.2577	-1094.12132434408\\
71.375	0.26136	-1135.11427768858\\
71.375	0.26502	-1176.92897190849\\
71.375	0.26868	-1219.56540700381\\
71.375	0.27234	-1263.02358297455\\
71.375	0.276	-1307.30349982069\\
71.75	0.093	-102.088412920144\\
71.75	0.09666	-106.114989226794\\
71.75	0.10032	-110.963306408856\\
71.75	0.10398	-116.633364466332\\
71.75	0.10764	-123.125163399221\\
71.75	0.1113	-130.438703207523\\
71.75	0.11496	-138.573983891239\\
71.75	0.11862	-147.531005450367\\
71.75	0.12228	-157.309767884909\\
71.75	0.12594	-167.910271194864\\
71.75	0.1296	-179.332515380233\\
71.75	0.13326	-191.576500441014\\
71.75	0.13692	-204.642226377209\\
71.75	0.14058	-218.529693188817\\
71.75	0.14424	-233.238900875837\\
71.75	0.1479	-248.769849438272\\
71.75	0.15156	-265.122538876119\\
71.75	0.15522	-282.29696918938\\
71.75	0.15888	-300.293140378053\\
71.75	0.16254	-319.111052442141\\
71.75	0.1662	-338.750705381641\\
71.75	0.16986	-359.212099196554\\
71.75	0.17352	-380.495233886881\\
71.75	0.17718	-402.60010945262\\
71.75	0.18084	-425.526725893773\\
71.75	0.1845	-449.27508321034\\
71.75	0.18816	-473.845181402319\\
71.75	0.19182	-499.237020469712\\
71.75	0.19548	-525.450600412518\\
71.75	0.19914	-552.485921230736\\
71.75	0.2028	-580.342982924368\\
71.75	0.20646	-609.021785493414\\
71.75	0.21012	-638.522328937873\\
71.75	0.21378	-668.844613257744\\
71.75	0.21744	-699.988638453029\\
71.75	0.2211	-731.954404523728\\
71.75	0.22476	-764.741911469839\\
71.75	0.22842	-798.351159291364\\
71.75	0.23208	-832.782147988301\\
71.75	0.23574	-868.034877560653\\
71.75	0.2394	-904.109348008417\\
71.75	0.24306	-941.005559331594\\
71.75	0.24672	-978.723511530184\\
71.75	0.25038	-1017.26320460419\\
71.75	0.25404	-1056.6246385536\\
71.75	0.2577	-1096.80781337844\\
71.75	0.26136	-1137.81272907868\\
71.75	0.26502	-1179.63938565434\\
71.75	0.26868	-1222.2877831054\\
71.75	0.27234	-1265.75792143189\\
71.75	0.276	-1310.04980063378\\
72.125	0.093	-104.307629621015\\
72.125	0.09666	-108.346168283412\\
72.125	0.10032	-113.206447821222\\
72.125	0.10398	-118.888468234446\\
72.125	0.10764	-125.392229523083\\
72.125	0.1113	-132.717731687133\\
72.125	0.11496	-140.864974726596\\
72.125	0.11862	-149.833958641472\\
72.125	0.12228	-159.624683431762\\
72.125	0.12594	-170.237149097464\\
72.125	0.1296	-181.671355638581\\
72.125	0.13326	-193.92730305511\\
72.125	0.13692	-207.004991347052\\
72.125	0.14058	-220.904420514408\\
72.125	0.14424	-235.625590557176\\
72.125	0.1479	-251.168501475358\\
72.125	0.15156	-267.533153268953\\
72.125	0.15522	-284.719545937962\\
72.125	0.15888	-302.727679482383\\
72.125	0.16254	-321.557553902219\\
72.125	0.1662	-341.209169197466\\
72.125	0.16986	-361.682525368128\\
72.125	0.17352	-382.977622414202\\
72.125	0.17718	-405.094460335689\\
72.125	0.18084	-428.03303913259\\
72.125	0.1845	-451.793358804904\\
72.125	0.18816	-476.375419352631\\
72.125	0.19182	-501.779220775772\\
72.125	0.19548	-528.004763074325\\
72.125	0.19914	-555.052046248292\\
72.125	0.2028	-582.921070297672\\
72.125	0.20646	-611.611835222465\\
72.125	0.21012	-641.124341022672\\
72.125	0.21378	-671.458587698291\\
72.125	0.21744	-702.614575249323\\
72.125	0.2211	-734.59230367577\\
72.125	0.22476	-767.391772977629\\
72.125	0.22842	-801.012983154901\\
72.125	0.23208	-835.455934207587\\
72.125	0.23574	-870.720626135685\\
72.125	0.2394	-906.807058939198\\
72.125	0.24306	-943.715232618123\\
72.125	0.24672	-981.445147172461\\
72.125	0.25038	-1019.99680260221\\
72.125	0.25404	-1059.37019890738\\
72.125	0.2577	-1099.56533608796\\
72.125	0.26136	-1140.58221414395\\
72.125	0.26502	-1182.42083307535\\
72.125	0.26868	-1225.08119288217\\
72.125	0.27234	-1268.5632935644\\
72.125	0.276	-1312.86713512204\\
72.5	0.093	-106.597879997054\\
72.5	0.09666	-110.648381015199\\
72.5	0.10032	-115.520622908757\\
72.5	0.10398	-121.214605677728\\
72.5	0.10764	-127.730329322113\\
72.5	0.1113	-135.067793841911\\
72.5	0.11496	-143.226999237122\\
72.5	0.11862	-152.207945507745\\
72.5	0.12228	-162.010632653783\\
72.5	0.12594	-172.635060675233\\
72.5	0.1296	-184.081229572097\\
72.5	0.13326	-196.349139344374\\
72.5	0.13692	-209.438789992064\\
72.5	0.14058	-223.350181515168\\
72.5	0.14424	-238.083313913684\\
72.5	0.1479	-253.638187187614\\
72.5	0.15156	-270.014801336957\\
72.5	0.15522	-287.213156361713\\
72.5	0.15888	-305.233252261882\\
72.5	0.16254	-324.075089037465\\
72.5	0.1662	-343.738666688461\\
72.5	0.16986	-364.22398521487\\
72.5	0.17352	-385.531044616692\\
72.5	0.17718	-407.659844893927\\
72.5	0.18084	-430.610386046575\\
72.5	0.1845	-454.382668074637\\
72.5	0.18816	-478.976690978112\\
72.5	0.19182	-504.392454757\\
72.5	0.19548	-530.629959411302\\
72.5	0.19914	-557.689204941016\\
72.5	0.2028	-585.570191346143\\
72.5	0.20646	-614.272918626685\\
72.5	0.21012	-643.797386782639\\
72.5	0.21378	-674.143595814006\\
72.5	0.21744	-705.311545720786\\
72.5	0.2211	-737.30123650298\\
72.5	0.22476	-770.112668160587\\
72.5	0.22842	-803.745840693607\\
72.5	0.23208	-838.20075410204\\
72.5	0.23574	-873.477408385887\\
72.5	0.2394	-909.575803545147\\
72.5	0.24306	-946.49593957982\\
72.5	0.24672	-984.237816489906\\
72.5	0.25038	-1022.80143427541\\
72.5	0.25404	-1062.18679293632\\
72.5	0.2577	-1102.39389247264\\
72.5	0.26136	-1143.42273288438\\
72.5	0.26502	-1185.27331417153\\
72.5	0.26868	-1227.9456363341\\
72.5	0.27234	-1271.43969937208\\
72.5	0.276	-1315.75550328547\\
72.875	0.093	-108.959164048261\\
72.875	0.09666	-113.021627422155\\
72.875	0.10032	-117.90583167146\\
72.875	0.10398	-123.611776796179\\
72.875	0.10764	-130.139462796312\\
72.875	0.1113	-137.488889671857\\
72.875	0.11496	-145.660057422816\\
72.875	0.11862	-154.652966049188\\
72.875	0.12228	-164.467615550973\\
72.875	0.12594	-175.104005928171\\
72.875	0.1296	-186.562137180783\\
72.875	0.13326	-198.842009308807\\
72.875	0.13692	-211.943622312245\\
72.875	0.14058	-225.866976191096\\
72.875	0.14424	-240.612070945361\\
72.875	0.1479	-256.178906575038\\
72.875	0.15156	-272.567483080129\\
72.875	0.15522	-289.777800460633\\
72.875	0.15888	-307.80985871655\\
72.875	0.16254	-326.66365784788\\
72.875	0.1662	-346.339197854623\\
72.875	0.16986	-366.83647873678\\
72.875	0.17352	-388.15550049435\\
72.875	0.17718	-410.296263127333\\
72.875	0.18084	-433.258766635729\\
72.875	0.1845	-457.043011019539\\
72.875	0.18816	-481.648996278762\\
72.875	0.19182	-507.076722413398\\
72.875	0.19548	-533.326189423447\\
72.875	0.19914	-560.397397308908\\
72.875	0.2028	-588.290346069784\\
72.875	0.20646	-617.005035706073\\
72.875	0.21012	-646.541466217775\\
72.875	0.21378	-676.89963760489\\
72.875	0.21744	-708.079549867418\\
72.875	0.2211	-740.08120300536\\
72.875	0.22476	-772.904597018714\\
72.875	0.22842	-806.549731907482\\
72.875	0.23208	-841.016607671663\\
72.875	0.23574	-876.305224311257\\
72.875	0.2394	-912.415581826265\\
72.875	0.24306	-949.347680216686\\
72.875	0.24672	-987.101519482519\\
72.875	0.25038	-1025.67709962377\\
72.875	0.25404	-1065.07442064043\\
72.875	0.2577	-1105.2934825325\\
72.875	0.26136	-1146.33428529999\\
72.875	0.26502	-1188.19682894289\\
72.875	0.26868	-1230.8811134612\\
72.875	0.27234	-1274.38713885493\\
72.875	0.276	-1318.71490512407\\
73.25	0.093	-111.391481774638\\
73.25	0.09666	-115.465907504279\\
73.25	0.10032	-120.362074109332\\
73.25	0.10398	-126.079981589799\\
73.25	0.10764	-132.619629945679\\
73.25	0.1113	-139.981019176972\\
73.25	0.11496	-148.164149283679\\
73.25	0.11862	-157.169020265798\\
73.25	0.12228	-166.995632123331\\
73.25	0.12594	-177.643984856277\\
73.25	0.1296	-189.114078464637\\
73.25	0.13326	-201.405912948409\\
73.25	0.13692	-214.519488307595\\
73.25	0.14058	-228.454804542194\\
73.25	0.14424	-243.211861652206\\
73.25	0.1479	-258.790659637631\\
73.25	0.15156	-275.191198498469\\
73.25	0.15522	-292.413478234721\\
73.25	0.15888	-310.457498846386\\
73.25	0.16254	-329.323260333464\\
73.25	0.1662	-349.010762695955\\
73.25	0.16986	-369.52000593386\\
73.25	0.17352	-390.850990047177\\
73.25	0.17718	-413.003715035908\\
73.25	0.18084	-435.978180900052\\
73.25	0.1845	-459.774387639609\\
73.25	0.18816	-484.39233525458\\
73.25	0.19182	-509.832023744964\\
73.25	0.19548	-536.09345311076\\
73.25	0.19914	-563.17662335197\\
73.25	0.2028	-591.081534468593\\
73.25	0.20646	-619.80818646063\\
73.25	0.21012	-649.35657932808\\
73.25	0.21378	-679.726713070942\\
73.25	0.21744	-710.918587689218\\
73.25	0.2211	-742.932203182907\\
73.25	0.22476	-775.76755955201\\
73.25	0.22842	-809.424656796526\\
73.25	0.23208	-843.903494916454\\
73.25	0.23574	-879.204073911797\\
73.25	0.2394	-915.326393782552\\
73.25	0.24306	-952.27045452872\\
73.25	0.24672	-990.036256150302\\
73.25	0.25038	-1028.6237986473\\
73.25	0.25404	-1068.0330820197\\
73.25	0.2577	-1108.26410626753\\
73.25	0.26136	-1149.31687139076\\
73.25	0.26502	-1191.19137738941\\
73.25	0.26868	-1233.88762426347\\
73.25	0.27234	-1277.40561201294\\
73.25	0.276	-1321.74534063783\\
73.625	0.093	-113.894833176184\\
73.625	0.09666	-117.981221261572\\
73.625	0.10032	-122.889350222374\\
73.625	0.10398	-128.619220058588\\
73.625	0.10764	-135.170830770216\\
73.625	0.1113	-142.544182357257\\
73.625	0.11496	-150.739274819711\\
73.625	0.11862	-159.756108157579\\
73.625	0.12228	-169.594682370859\\
73.625	0.12594	-180.254997459553\\
73.625	0.1296	-191.73705342366\\
73.625	0.13326	-204.04085026318\\
73.625	0.13692	-217.166387978114\\
73.625	0.14058	-231.11366656846\\
73.625	0.14424	-245.88268603422\\
73.625	0.1479	-261.473446375393\\
73.625	0.15156	-277.885947591979\\
73.625	0.15522	-295.120189683979\\
73.625	0.15888	-313.176172651391\\
73.625	0.16254	-332.053896494217\\
73.625	0.1662	-351.753361212456\\
73.625	0.16986	-372.274566806109\\
73.625	0.17352	-393.617513275174\\
73.625	0.17718	-415.782200619652\\
73.625	0.18084	-438.768628839544\\
73.625	0.1845	-462.576797934849\\
73.625	0.18816	-487.206707905567\\
73.625	0.19182	-512.658358751698\\
73.625	0.19548	-538.931750473243\\
73.625	0.19914	-566.026883070201\\
73.625	0.2028	-593.943756542572\\
73.625	0.20646	-622.682370890356\\
73.625	0.21012	-652.242726113554\\
73.625	0.21378	-682.624822212164\\
73.625	0.21744	-713.828659186188\\
73.625	0.2211	-745.854237035625\\
73.625	0.22476	-778.701555760475\\
73.625	0.22842	-812.370615360738\\
73.625	0.23208	-846.861415836415\\
73.625	0.23574	-882.173957187505\\
73.625	0.2394	-918.308239414008\\
73.625	0.24306	-955.264262515924\\
73.625	0.24672	-993.042026493253\\
73.625	0.25038	-1031.641531346\\
73.625	0.25404	-1071.06277707415\\
73.625	0.2577	-1111.30576367772\\
73.625	0.26136	-1152.3704911567\\
73.625	0.26502	-1194.2569595111\\
73.625	0.26868	-1236.96516874091\\
73.625	0.27234	-1280.49511884613\\
73.625	0.276	-1324.84680982676\\
74	0.093	-116.469218252898\\
74	0.09666	-120.567568694034\\
74	0.10032	-125.487660010583\\
74	0.10398	-131.229492202545\\
74	0.10764	-137.793065269921\\
74	0.1113	-145.17837921271\\
74	0.11496	-153.385434030912\\
74	0.11862	-162.414229724527\\
74	0.12228	-172.264766293555\\
74	0.12594	-182.937043737997\\
74	0.1296	-194.431062057852\\
74	0.13326	-206.746821253119\\
74	0.13692	-219.884321323801\\
74	0.14058	-233.843562269895\\
74	0.14424	-248.624544091403\\
74	0.1479	-264.227266788323\\
74	0.15156	-280.651730360657\\
74	0.15522	-297.897934808405\\
74	0.15888	-315.965880131565\\
74	0.16254	-334.855566330139\\
74	0.1662	-354.566993404125\\
74	0.16986	-375.100161353525\\
74	0.17352	-396.455070178338\\
74	0.17718	-418.631719878564\\
74	0.18084	-441.630110454204\\
74	0.1845	-465.450241905257\\
74	0.18816	-490.092114231723\\
74	0.19182	-515.555727433602\\
74	0.19548	-541.841081510894\\
74	0.19914	-568.9481764636\\
74	0.2028	-596.877012291719\\
74	0.20646	-625.627588995251\\
74	0.21012	-655.199906574196\\
74	0.21378	-685.593965028554\\
74	0.21744	-716.809764358326\\
74	0.2211	-748.84730456351\\
74	0.22476	-781.706585644108\\
74	0.22842	-815.38760760012\\
74	0.23208	-849.890370431544\\
74	0.23574	-885.214874138382\\
74	0.2394	-921.361118720632\\
74	0.24306	-958.329104178296\\
74	0.24672	-996.118830511373\\
74	0.25038	-1034.73029771986\\
74	0.25404	-1074.16350580377\\
74	0.2577	-1114.41845476308\\
74	0.26136	-1155.49514459781\\
74	0.26502	-1197.39357530796\\
74	0.26868	-1240.11374689351\\
74	0.27234	-1283.65565935448\\
74	0.276	-1328.01931269086\\
};
\end{axis}

\begin{axis}[%
width=4.527496cm,
height=3.870968cm,
at={(6.483547cm,10.752688cm)},
scale only axis,
xmin=56,
xmax=74,
tick align=outside,
xlabel={$L_{cut}$},
xmajorgrids,
ymin=0.093,
ymax=0.276,
ylabel={$D_{rlx}$},
ymajorgrids,
zmin=-1.52969162398006,
zmax=2.51190540588646,
zlabel={$x_1,x_1$},
zmajorgrids,
view={-140}{50},
legend style={at={(1.03,1)},anchor=north west,legend cell align=left,align=left,draw=white!15!black}
]
\addplot3[only marks,mark=*,mark options={},mark size=1.5000pt,color=mycolor1] plot table[row sep=crcr,]{%
74	0.123	0.00740370360905991\\
72	0.113	-0.186626587578763\\
61	0.095	-0.150607769785267\\
56	0.093	-0.216975707230133\\
};
\addplot3[only marks,mark=*,mark options={},mark size=1.5000pt,color=mycolor2] plot table[row sep=crcr,]{%
67	0.276	0.0527524251509314\\
66	0.255	-0.475927501616939\\
62	0.209	0.435130634728223\\
57	0.193	0.76117565090573\\
};
\addplot3[only marks,mark=*,mark options={},mark size=1.5000pt,color=black] plot table[row sep=crcr,]{%
69	0.104	-0.0638888311101987\\
};
\addplot3[only marks,mark=*,mark options={},mark size=1.5000pt,color=black] plot table[row sep=crcr,]{%
64	0.23	0.0838528346972987\\
};

\addplot3[%
surf,
opacity=0.7,
shader=interp,
colormap={mymap}{[1pt] rgb(0pt)=(0.0901961,0.239216,0.0745098); rgb(1pt)=(0.0945149,0.242058,0.0739522); rgb(2pt)=(0.0988592,0.244894,0.0733566); rgb(3pt)=(0.103229,0.247724,0.0727241); rgb(4pt)=(0.107623,0.250549,0.0720557); rgb(5pt)=(0.112043,0.253367,0.0713525); rgb(6pt)=(0.116487,0.25618,0.0706154); rgb(7pt)=(0.120956,0.258986,0.0698456); rgb(8pt)=(0.125449,0.261787,0.0690441); rgb(9pt)=(0.129967,0.264581,0.0682118); rgb(10pt)=(0.134508,0.26737,0.06735); rgb(11pt)=(0.139074,0.270152,0.0664596); rgb(12pt)=(0.143663,0.272929,0.0655416); rgb(13pt)=(0.148275,0.275699,0.0645971); rgb(14pt)=(0.152911,0.278463,0.0636271); rgb(15pt)=(0.15757,0.281221,0.0626328); rgb(16pt)=(0.162252,0.283973,0.0616151); rgb(17pt)=(0.166957,0.286719,0.060575); rgb(18pt)=(0.171685,0.289458,0.0595136); rgb(19pt)=(0.176434,0.292191,0.0584321); rgb(20pt)=(0.181207,0.294918,0.0573313); rgb(21pt)=(0.186001,0.297639,0.0562123); rgb(22pt)=(0.190817,0.300353,0.0550763); rgb(23pt)=(0.195655,0.303061,0.0539242); rgb(24pt)=(0.200514,0.305763,0.052757); rgb(25pt)=(0.205395,0.308459,0.0515759); rgb(26pt)=(0.210296,0.311149,0.0503624); rgb(27pt)=(0.215212,0.313846,0.0490067); rgb(28pt)=(0.220142,0.316548,0.0475043); rgb(29pt)=(0.22509,0.319254,0.0458704); rgb(30pt)=(0.230056,0.321962,0.0441205); rgb(31pt)=(0.235042,0.324671,0.04227); rgb(32pt)=(0.240048,0.327379,0.0403343); rgb(33pt)=(0.245078,0.330085,0.0383287); rgb(34pt)=(0.250131,0.332786,0.0362688); rgb(35pt)=(0.25521,0.335482,0.0341698); rgb(36pt)=(0.260317,0.33817,0.0320472); rgb(37pt)=(0.265451,0.340849,0.0299163); rgb(38pt)=(0.270616,0.343517,0.0277927); rgb(39pt)=(0.275813,0.346172,0.0256916); rgb(40pt)=(0.281043,0.348814,0.0236284); rgb(41pt)=(0.286307,0.35144,0.0216186); rgb(42pt)=(0.291607,0.354048,0.0196776); rgb(43pt)=(0.296945,0.356637,0.0178207); rgb(44pt)=(0.302322,0.359206,0.0160634); rgb(45pt)=(0.307739,0.361753,0.0144211); rgb(46pt)=(0.313198,0.364275,0.0129091); rgb(47pt)=(0.318701,0.366772,0.0115428); rgb(48pt)=(0.324249,0.369242,0.0103377); rgb(49pt)=(0.329843,0.371682,0.00930909); rgb(50pt)=(0.335485,0.374093,0.00847245); rgb(51pt)=(0.341176,0.376471,0.00784314); rgb(52pt)=(0.346925,0.378826,0.00732741); rgb(53pt)=(0.352735,0.381168,0.00682184); rgb(54pt)=(0.358605,0.383497,0.00632729); rgb(55pt)=(0.364532,0.385812,0.00584464); rgb(56pt)=(0.370516,0.388113,0.00537476); rgb(57pt)=(0.376552,0.390399,0.00491852); rgb(58pt)=(0.38264,0.39267,0.00447681); rgb(59pt)=(0.388777,0.394925,0.00405048); rgb(60pt)=(0.394962,0.397164,0.00364042); rgb(61pt)=(0.401191,0.399386,0.00324749); rgb(62pt)=(0.407464,0.401592,0.00287258); rgb(63pt)=(0.413777,0.40378,0.00251655); rgb(64pt)=(0.420129,0.40595,0.00218028); rgb(65pt)=(0.426518,0.408102,0.00186463); rgb(66pt)=(0.432942,0.410234,0.00157049); rgb(67pt)=(0.439399,0.412348,0.00129873); rgb(68pt)=(0.445885,0.414441,0.00105022); rgb(69pt)=(0.452401,0.416515,0.000825833); rgb(70pt)=(0.458942,0.418567,0.000626441); rgb(71pt)=(0.465508,0.420599,0.00045292); rgb(72pt)=(0.472096,0.422609,0.000306141); rgb(73pt)=(0.478704,0.424596,0.000186979); rgb(74pt)=(0.485331,0.426562,9.63073e-05); rgb(75pt)=(0.491973,0.428504,3.49981e-05); rgb(76pt)=(0.498628,0.430422,3.92506e-06); rgb(77pt)=(0.505323,0.432315,0); rgb(78pt)=(0.512206,0.434168,0); rgb(79pt)=(0.519282,0.435983,0); rgb(80pt)=(0.526529,0.437764,0); rgb(81pt)=(0.533922,0.439512,0); rgb(82pt)=(0.54144,0.441232,0); rgb(83pt)=(0.549059,0.442927,0); rgb(84pt)=(0.556756,0.444599,0); rgb(85pt)=(0.564508,0.446252,0); rgb(86pt)=(0.572292,0.447889,0); rgb(87pt)=(0.580084,0.449514,0); rgb(88pt)=(0.587863,0.451129,0); rgb(89pt)=(0.595604,0.452737,0); rgb(90pt)=(0.603284,0.454343,0); rgb(91pt)=(0.610882,0.455948,0); rgb(92pt)=(0.618373,0.457556,0); rgb(93pt)=(0.625734,0.459171,0); rgb(94pt)=(0.632943,0.460795,0); rgb(95pt)=(0.639976,0.462432,0); rgb(96pt)=(0.64681,0.464084,0); rgb(97pt)=(0.653423,0.465756,0); rgb(98pt)=(0.659791,0.46745,0); rgb(99pt)=(0.665891,0.469169,0); rgb(100pt)=(0.6717,0.470916,0); rgb(101pt)=(0.677195,0.472696,0); rgb(102pt)=(0.682353,0.47451,0); rgb(103pt)=(0.687242,0.476355,0); rgb(104pt)=(0.691952,0.478225,0); rgb(105pt)=(0.696497,0.480118,0); rgb(106pt)=(0.700887,0.482033,0); rgb(107pt)=(0.705134,0.483968,0); rgb(108pt)=(0.709251,0.485921,0); rgb(109pt)=(0.713249,0.487891,0); rgb(110pt)=(0.71714,0.489876,0); rgb(111pt)=(0.720936,0.491875,0); rgb(112pt)=(0.724649,0.493887,0); rgb(113pt)=(0.72829,0.495909,0); rgb(114pt)=(0.731872,0.49794,0); rgb(115pt)=(0.735406,0.499979,0); rgb(116pt)=(0.738904,0.502025,0); rgb(117pt)=(0.742378,0.504075,0); rgb(118pt)=(0.74584,0.506128,0); rgb(119pt)=(0.749302,0.508182,0); rgb(120pt)=(0.752775,0.510237,0); rgb(121pt)=(0.756272,0.51229,0); rgb(122pt)=(0.759804,0.514339,0); rgb(123pt)=(0.763384,0.516385,0); rgb(124pt)=(0.767022,0.518424,0); rgb(125pt)=(0.770731,0.520455,0); rgb(126pt)=(0.774523,0.522478,0); rgb(127pt)=(0.77841,0.524489,0); rgb(128pt)=(0.782391,0.526491,0); rgb(129pt)=(0.786402,0.528496,0); rgb(130pt)=(0.790431,0.530506,0); rgb(131pt)=(0.794478,0.532521,0); rgb(132pt)=(0.798541,0.534539,0); rgb(133pt)=(0.802619,0.53656,0); rgb(134pt)=(0.806712,0.538584,0); rgb(135pt)=(0.81082,0.540609,0); rgb(136pt)=(0.81494,0.542635,0); rgb(137pt)=(0.819074,0.54466,0); rgb(138pt)=(0.823219,0.546686,0); rgb(139pt)=(0.827374,0.548709,0); rgb(140pt)=(0.831541,0.55073,0); rgb(141pt)=(0.835716,0.552749,0); rgb(142pt)=(0.8399,0.554763,0); rgb(143pt)=(0.844092,0.556774,0); rgb(144pt)=(0.848292,0.558779,0); rgb(145pt)=(0.852497,0.560778,0); rgb(146pt)=(0.856708,0.562771,0); rgb(147pt)=(0.860924,0.564756,0); rgb(148pt)=(0.865143,0.566733,0); rgb(149pt)=(0.869366,0.568701,0); rgb(150pt)=(0.873592,0.57066,0); rgb(151pt)=(0.877819,0.572608,0); rgb(152pt)=(0.882047,0.574545,0); rgb(153pt)=(0.886275,0.576471,0); rgb(154pt)=(0.890659,0.578362,0); rgb(155pt)=(0.895333,0.580203,0); rgb(156pt)=(0.900258,0.581999,0); rgb(157pt)=(0.905397,0.583755,0); rgb(158pt)=(0.910711,0.585479,0); rgb(159pt)=(0.916164,0.587176,0); rgb(160pt)=(0.921717,0.588852,0); rgb(161pt)=(0.927333,0.590513,0); rgb(162pt)=(0.932974,0.592166,0); rgb(163pt)=(0.938602,0.593815,0); rgb(164pt)=(0.94418,0.595468,0); rgb(165pt)=(0.949669,0.59713,0); rgb(166pt)=(0.955033,0.598808,0); rgb(167pt)=(0.960233,0.600507,0); rgb(168pt)=(0.965232,0.602233,0); rgb(169pt)=(0.969992,0.603992,0); rgb(170pt)=(0.974475,0.605791,0); rgb(171pt)=(0.978643,0.607636,0); rgb(172pt)=(0.98246,0.609532,0); rgb(173pt)=(0.985886,0.611486,0); rgb(174pt)=(0.988885,0.613503,0); rgb(175pt)=(0.991419,0.61559,0); rgb(176pt)=(0.99345,0.617753,0); rgb(177pt)=(0.99494,0.619997,0); rgb(178pt)=(0.995851,0.622329,0); rgb(179pt)=(0.996226,0.624763,0); rgb(180pt)=(0.996512,0.627352,0); rgb(181pt)=(0.996788,0.630095,0); rgb(182pt)=(0.997053,0.632982,0); rgb(183pt)=(0.997308,0.636004,0); rgb(184pt)=(0.997552,0.639152,0); rgb(185pt)=(0.997785,0.642416,0); rgb(186pt)=(0.998006,0.645786,0); rgb(187pt)=(0.998217,0.649253,0); rgb(188pt)=(0.998416,0.652807,0); rgb(189pt)=(0.998605,0.656439,0); rgb(190pt)=(0.998781,0.660138,0); rgb(191pt)=(0.998946,0.663897,0); rgb(192pt)=(0.9991,0.667704,0); rgb(193pt)=(0.999242,0.67155,0); rgb(194pt)=(0.999372,0.675427,0); rgb(195pt)=(0.99949,0.679323,0); rgb(196pt)=(0.999596,0.68323,0); rgb(197pt)=(0.99969,0.687139,0); rgb(198pt)=(0.999771,0.691039,0); rgb(199pt)=(0.999841,0.694921,0); rgb(200pt)=(0.999898,0.698775,0); rgb(201pt)=(0.999942,0.702592,0); rgb(202pt)=(0.999974,0.706363,0); rgb(203pt)=(0.999994,0.710077,0); rgb(204pt)=(1,0.713725,0); rgb(205pt)=(1,0.717341,0); rgb(206pt)=(1,0.720963,0); rgb(207pt)=(1,0.724591,0); rgb(208pt)=(1,0.728226,0); rgb(209pt)=(1,0.731867,0); rgb(210pt)=(1,0.735514,0); rgb(211pt)=(1,0.739167,0); rgb(212pt)=(1,0.742827,0); rgb(213pt)=(1,0.746493,0); rgb(214pt)=(1,0.750165,0); rgb(215pt)=(1,0.753843,0); rgb(216pt)=(1,0.757527,0); rgb(217pt)=(1,0.761217,0); rgb(218pt)=(1,0.764913,0); rgb(219pt)=(1,0.768615,0); rgb(220pt)=(1,0.772324,0); rgb(221pt)=(1,0.776038,0); rgb(222pt)=(1,0.779758,0); rgb(223pt)=(1,0.783484,0); rgb(224pt)=(1,0.787215,0); rgb(225pt)=(1,0.790953,0); rgb(226pt)=(1,0.794696,0); rgb(227pt)=(1,0.798445,0); rgb(228pt)=(1,0.8022,0); rgb(229pt)=(1,0.805961,0); rgb(230pt)=(1,0.809727,0); rgb(231pt)=(1,0.8135,0); rgb(232pt)=(1,0.817278,0); rgb(233pt)=(1,0.821063,0); rgb(234pt)=(1,0.824854,0); rgb(235pt)=(1,0.828652,0); rgb(236pt)=(1,0.832455,0); rgb(237pt)=(1,0.836265,0); rgb(238pt)=(1,0.840081,0); rgb(239pt)=(1,0.843903,0); rgb(240pt)=(1,0.847732,0); rgb(241pt)=(1,0.851566,0); rgb(242pt)=(1,0.855406,0); rgb(243pt)=(1,0.859253,0); rgb(244pt)=(1,0.863106,0); rgb(245pt)=(1,0.866964,0); rgb(246pt)=(1,0.870829,0); rgb(247pt)=(1,0.8747,0); rgb(248pt)=(1,0.878577,0); rgb(249pt)=(1,0.88246,0); rgb(250pt)=(1,0.886349,0); rgb(251pt)=(1,0.890243,0); rgb(252pt)=(1,0.894144,0); rgb(253pt)=(1,0.898051,0); rgb(254pt)=(1,0.901964,0); rgb(255pt)=(1,0.905882,0)},
mesh/rows=49]
table[row sep=crcr,header=false] {%
%
56	0.093	-0.183093463550335\\
56	0.09666	-0.15506912383271\\
56	0.10032	-0.125988635638709\\
56	0.10398	-0.095851998968341\\
56	0.10764	-0.0646592138216007\\
56	0.1113	-0.032410280198485\\
56	0.11496	0.000894801900998998\\
56	0.11862	0.0352560324768565\\
56	0.12228	0.0706734115290841\\
56	0.12594	0.107146939057687\\
56	0.1296	0.14467661506266\\
56	0.13326	0.183262439544003\\
56	0.13692	0.22290441250172\\
56	0.14058	0.263602533935807\\
56	0.14424	0.305356803846267\\
56	0.1479	0.348167222233099\\
56	0.15156	0.392033789096303\\
56	0.15522	0.436956504435879\\
56	0.15888	0.482935368251827\\
56	0.16254	0.529970380544145\\
56	0.1662	0.578061541312836\\
56	0.16986	0.627208850557897\\
56	0.17352	0.677412308279332\\
56	0.17718	0.728671914477137\\
56	0.18084	0.780987669151318\\
56	0.1845	0.834359572301868\\
56	0.18816	0.888787623928789\\
56	0.19182	0.944271824032083\\
56	0.19548	1.00081217261175\\
56	0.19914	1.05840866966778\\
56	0.2028	1.11706131520019\\
56	0.20646	1.17677010920897\\
56	0.21012	1.23753505169413\\
56	0.21378	1.29935614265565\\
56	0.21744	1.36223338209355\\
56	0.2211	1.42616677000781\\
56	0.22476	1.49115630639846\\
56	0.22842	1.55720199126547\\
56	0.23208	1.62430382460885\\
56	0.23574	1.69246180642861\\
56	0.2394	1.76167593672473\\
56	0.24306	1.83194621549724\\
56	0.24672	1.90327264274611\\
56	0.25038	1.97565521847135\\
56	0.25404	2.04909394267296\\
56	0.2577	2.12358881535095\\
56	0.26136	2.19913983650531\\
56	0.26502	2.27574700613604\\
56	0.26868	2.35341032424314\\
56	0.27234	2.43212979082662\\
56	0.276	2.51190540588646\\
56.375	0.093	-0.18948708924544\\
56.375	0.09666	-0.163384241142382\\
56.375	0.10032	-0.136225244562956\\
56.375	0.10398	-0.108010099507157\\
56.375	0.10764	-0.0787388059749854\\
56.375	0.1113	-0.0484113639664421\\
56.375	0.11496	-0.0170277734815271\\
56.375	0.11862	0.0154119654797598\\
56.375	0.12228	0.0489078529174185\\
56.375	0.12594	0.0834598888314491\\
56.375	0.1296	0.119068073221851\\
56.375	0.13326	0.155732406088625\\
56.375	0.13692	0.193452887431771\\
56.375	0.14058	0.232229517251288\\
56.375	0.14424	0.272062295547178\\
56.375	0.1479	0.312951222319441\\
56.375	0.15156	0.354896297568072\\
56.375	0.15522	0.397897521293079\\
56.375	0.15888	0.441954893494456\\
56.375	0.16254	0.487068414172203\\
56.375	0.1662	0.533238083326324\\
56.375	0.16986	0.580463900956817\\
56.375	0.17352	0.628745867063681\\
56.375	0.17718	0.678083981646915\\
56.375	0.18084	0.728478244706525\\
56.375	0.1845	0.779928656242505\\
56.375	0.18816	0.832435216254855\\
56.375	0.19182	0.88599792474358\\
56.375	0.19548	0.940616781708675\\
56.375	0.19914	0.99629178715014\\
56.375	0.2028	1.05302294106798\\
56.375	0.20646	1.11081024346219\\
56.375	0.21012	1.16965369433278\\
56.375	0.21378	1.22955329367973\\
56.375	0.21744	1.29050904150305\\
56.375	0.2211	1.35252093780275\\
56.375	0.22476	1.41558898257882\\
56.375	0.22842	1.47971317583126\\
56.375	0.23208	1.54489351756008\\
56.375	0.23574	1.61113000776526\\
56.375	0.2394	1.67842264644682\\
56.375	0.24306	1.74677143360475\\
56.375	0.24672	1.81617636923905\\
56.375	0.25038	1.88663745334972\\
56.375	0.25404	1.95815468593677\\
56.375	0.2577	2.03072806700018\\
56.375	0.26136	2.10435759653997\\
56.375	0.26502	2.17904327455613\\
56.375	0.26868	2.25478510104866\\
56.375	0.27234	2.33158307601756\\
56.375	0.276	2.40943719946284\\
56.75	0.093	-0.195103341821336\\
56.75	0.09666	-0.170921985332851\\
56.75	0.10032	-0.145684480367992\\
56.75	0.10398	-0.119390826926765\\
56.75	0.10764	-0.0920410250091641\\
56.75	0.1113	-0.0636350746151897\\
56.75	0.11496	-0.0341729757448453\\
56.75	0.11862	-0.00365472839812919\\
56.75	0.12228	0.0279196674249589\\
56.75	0.12594	0.0605502117244188\\
56.75	0.1296	0.0942369045002521\\
56.75	0.13326	0.128979745752454\\
56.75	0.13692	0.164778735481031\\
56.75	0.14058	0.201633873685977\\
56.75	0.14424	0.239545160367297\\
56.75	0.1479	0.278512595524988\\
56.75	0.15156	0.318536179159051\\
56.75	0.15522	0.359615911269485\\
56.75	0.15888	0.401751791856294\\
56.75	0.16254	0.44494382091947\\
56.75	0.1662	0.489191998459022\\
56.75	0.16986	0.534496324474942\\
56.75	0.17352	0.580856798967237\\
56.75	0.17718	0.628273421935903\\
56.75	0.18084	0.67674619338094\\
56.75	0.1845	0.726275113302351\\
56.75	0.18816	0.77686018170013\\
56.75	0.19182	0.828501398574285\\
56.75	0.19548	0.881198763924809\\
56.75	0.19914	0.934952277751706\\
56.75	0.2028	0.989761940054972\\
56.75	0.20646	1.04562775083461\\
56.75	0.21012	1.10254971009063\\
56.75	0.21378	1.16052781782301\\
56.75	0.21744	1.21956207403177\\
56.75	0.2211	1.27965247871689\\
56.75	0.22476	1.34079903187839\\
56.75	0.22842	1.40300173351626\\
56.75	0.23208	1.46626058363051\\
56.75	0.23574	1.53057558222112\\
56.75	0.2394	1.59594672928811\\
56.75	0.24306	1.66237402483147\\
56.75	0.24672	1.7298574688512\\
56.75	0.25038	1.7983970613473\\
56.75	0.25404	1.86799280231977\\
56.75	0.2577	1.93864469176862\\
56.75	0.26136	2.01035272969384\\
56.75	0.26502	2.08311691609543\\
56.75	0.26868	2.15693725097339\\
56.75	0.27234	2.23181373432773\\
56.75	0.276	2.30774636615843\\
57.125	0.093	-0.199942221278023\\
57.125	0.09666	-0.17768235640411\\
57.125	0.10032	-0.154366343053822\\
57.125	0.10398	-0.129994181227162\\
57.125	0.10764	-0.104565870924133\\
57.125	0.1113	-0.0780814121447296\\
57.125	0.11496	-0.0505408048889558\\
57.125	0.11862	-0.0219440491568086\\
57.125	0.12228	0.00770885505170882\\
57.125	0.12594	0.038417907736598\\
57.125	0.1296	0.0701831088978607\\
57.125	0.13326	0.103004458535494\\
57.125	0.13692	0.1368819566495\\
57.125	0.14058	0.171815603239875\\
57.125	0.14424	0.207805398306625\\
57.125	0.1479	0.244851341849747\\
57.125	0.15156	0.282953433869239\\
57.125	0.15522	0.322111674365103\\
57.125	0.15888	0.362326063337341\\
57.125	0.16254	0.403596600785948\\
57.125	0.1662	0.445923286710927\\
57.125	0.16986	0.489306121112278\\
57.125	0.17352	0.533745103990003\\
57.125	0.17718	0.579240235344098\\
57.125	0.18084	0.625791515174564\\
57.125	0.1845	0.673398943481404\\
57.125	0.18816	0.722062520264615\\
57.125	0.19182	0.771782245524197\\
57.125	0.19548	0.822558119260153\\
57.125	0.19914	0.874390141472479\\
57.125	0.2028	0.927278312161178\\
57.125	0.20646	0.981222631326246\\
57.125	0.21012	1.03622309896769\\
57.125	0.21378	1.0922797150855\\
57.125	0.21744	1.14939247967969\\
57.125	0.2211	1.20756139275025\\
57.125	0.22476	1.26678645429718\\
57.125	0.22842	1.32706766432047\\
57.125	0.23208	1.38840502282015\\
57.125	0.23574	1.45079852979619\\
57.125	0.2394	1.51424818524861\\
57.125	0.24306	1.5787539891774\\
57.125	0.24672	1.64431594158256\\
57.125	0.25038	1.71093404246409\\
57.125	0.25404	1.77860829182199\\
57.125	0.2577	1.84733868965627\\
57.125	0.26136	1.91712523596692\\
57.125	0.26502	1.98796793075394\\
57.125	0.26868	2.05986677401732\\
57.125	0.27234	2.13282176575709\\
57.125	0.276	2.20683290597322\\
57.5	0.093	-0.204003727615505\\
57.5	0.09666	-0.183665354356161\\
57.5	0.10032	-0.162270832620443\\
57.5	0.10398	-0.139820162408354\\
57.5	0.10764	-0.116313343719895\\
57.5	0.1113	-0.0917503765550616\\
57.5	0.11496	-0.0661312609138586\\
57.5	0.11862	-0.039455996796282\\
57.5	0.12228	-0.0117245842023352\\
57.5	0.12594	0.0170629768679851\\
57.5	0.1296	0.0469066864146771\\
57.5	0.13326	0.0778065444377393\\
57.5	0.13692	0.109762550937175\\
57.5	0.14058	0.142774705912979\\
57.5	0.14424	0.176843009365159\\
57.5	0.1479	0.21196746129371\\
57.5	0.15156	0.248148061698631\\
57.5	0.15522	0.285384810579926\\
57.5	0.15888	0.323677707937593\\
57.5	0.16254	0.36302675377163\\
57.5	0.1662	0.403431948082039\\
57.5	0.16986	0.444893290868819\\
57.5	0.17352	0.487410782131973\\
57.5	0.17718	0.530984421871497\\
57.5	0.18084	0.575614210087395\\
57.5	0.1845	0.621300146779664\\
57.5	0.18816	0.668042231948304\\
57.5	0.19182	0.715840465593317\\
57.5	0.19548	0.764694847714702\\
57.5	0.19914	0.814605378312456\\
57.5	0.2028	0.865572057386585\\
57.5	0.20646	0.917594884937085\\
57.5	0.21012	0.970673860963958\\
57.5	0.21378	1.0248089854672\\
57.5	0.21744	1.08000025844681\\
57.5	0.2211	1.1362476799028\\
57.5	0.22476	1.19355124983516\\
57.5	0.22842	1.25191096824389\\
57.5	0.23208	1.31132683512899\\
57.5	0.23574	1.37179885049047\\
57.5	0.2394	1.43332701432831\\
57.5	0.24306	1.49591132664253\\
57.5	0.24672	1.55955178743312\\
57.5	0.25038	1.62424839670008\\
57.5	0.25404	1.69000115444341\\
57.5	0.2577	1.75681006066312\\
57.5	0.26136	1.8246751153592\\
57.5	0.26502	1.89359631853165\\
57.5	0.26868	1.96357367018047\\
57.5	0.27234	2.03460717030566\\
57.5	0.276	2.10669681890722\\
57.875	0.093	-0.207287860833774\\
57.875	0.09666	-0.188870979188999\\
57.875	0.10032	-0.169397949067852\\
57.875	0.10398	-0.148868770470335\\
57.875	0.10764	-0.127283443396446\\
57.875	0.1113	-0.104641967846182\\
57.875	0.11496	-0.08094434381955\\
57.875	0.11862	-0.056190571316544\\
57.875	0.12228	-0.030380650337168\\
57.875	0.12594	-0.00351458088141832\\
57.875	0.1296	0.024407637050703\\
57.875	0.13326	0.0533860034591945\\
57.875	0.13692	0.0834205183440596\\
57.875	0.14058	0.114511181705295\\
57.875	0.14424	0.146657993542904\\
57.875	0.1479	0.179860953856886\\
57.875	0.15156	0.214120062647237\\
57.875	0.15522	0.249435319913961\\
57.875	0.15888	0.285806725657057\\
57.875	0.16254	0.323234279876524\\
57.875	0.1662	0.361717982572363\\
57.875	0.16986	0.401257833744573\\
57.875	0.17352	0.441853833393156\\
57.875	0.17718	0.48350598151811\\
57.875	0.18084	0.526214278119437\\
57.875	0.1845	0.569978723197137\\
57.875	0.18816	0.614799316751205\\
57.875	0.19182	0.660676058781649\\
57.875	0.19548	0.707608949288463\\
57.875	0.19914	0.755597988271646\\
57.875	0.2028	0.804643175731204\\
57.875	0.20646	0.854744511667134\\
57.875	0.21012	0.905901996079436\\
57.875	0.21378	0.958115628968109\\
57.875	0.21744	1.01138541033315\\
57.875	0.2211	1.06571134017457\\
57.875	0.22476	1.12109341849236\\
57.875	0.22842	1.17753164528652\\
57.875	0.23208	1.23502602055705\\
57.875	0.23574	1.29357654430396\\
57.875	0.2394	1.35318321652723\\
57.875	0.24306	1.41384603722688\\
57.875	0.24672	1.4755650064029\\
57.875	0.25038	1.53834012405529\\
57.875	0.25404	1.60217139018405\\
57.875	0.2577	1.66705880478919\\
57.875	0.26136	1.73300236787069\\
57.875	0.26502	1.80000207942857\\
57.875	0.26868	1.86805793946282\\
57.875	0.27234	1.93716994797345\\
57.875	0.276	2.00733810496044\\
58.25	0.093	-0.209794620932834\\
58.25	0.09666	-0.193299230902629\\
58.25	0.10032	-0.175747692396053\\
58.25	0.10398	-0.157140005413105\\
58.25	0.10764	-0.137476169953787\\
58.25	0.1113	-0.116756186018094\\
58.25	0.11496	-0.0949800536060318\\
58.25	0.11862	-0.0721477727175948\\
58.25	0.12228	-0.0482593433527894\\
58.25	0.12594	-0.0233147655116104\\
58.25	0.1296	0.00268596080594208\\
58.25	0.13326	0.0297428355998629\\
58.25	0.13692	0.0578558588701573\\
58.25	0.14058	0.0870250306168221\\
58.25	0.14424	0.117250350839862\\
58.25	0.1479	0.148531819539272\\
58.25	0.15156	0.180869436715052\\
58.25	0.15522	0.214263202367207\\
58.25	0.15888	0.248713116495733\\
58.25	0.16254	0.284219179100628\\
58.25	0.1662	0.320781390181897\\
58.25	0.16986	0.358399749739538\\
58.25	0.17352	0.397074257773551\\
58.25	0.17718	0.436804914283934\\
58.25	0.18084	0.477591719270692\\
58.25	0.1845	0.51943467273382\\
58.25	0.18816	0.562333774673319\\
58.25	0.19182	0.60628902508919\\
58.25	0.19548	0.651300423981436\\
58.25	0.19914	0.697367971350051\\
58.25	0.2028	0.744491667195039\\
58.25	0.20646	0.792671511516395\\
58.25	0.21012	0.841907504314129\\
58.25	0.21378	0.892199645588232\\
58.25	0.21744	0.943547935338703\\
58.25	0.2211	0.995952373565549\\
58.25	0.22476	1.04941296026877\\
58.25	0.22842	1.10392969544836\\
58.25	0.23208	1.15950257910432\\
58.25	0.23574	1.21613161123665\\
58.25	0.2394	1.27381679184536\\
58.25	0.24306	1.33255812093044\\
58.25	0.24672	1.39235559849189\\
58.25	0.25038	1.45320922452971\\
58.25	0.25404	1.5151189990439\\
58.25	0.2577	1.57808492203447\\
58.25	0.26136	1.6421069935014\\
58.25	0.26502	1.70718521344471\\
58.25	0.26868	1.77331958186439\\
58.25	0.27234	1.84051009876044\\
58.25	0.276	1.90875676413287\\
58.625	0.093	-0.211524007912687\\
58.625	0.09666	-0.196950109497054\\
58.625	0.10032	-0.181320062605044\\
58.625	0.10398	-0.164633867236669\\
58.625	0.10764	-0.14689152339192\\
58.625	0.1113	-0.128093031070799\\
58.625	0.11496	-0.108238390273306\\
58.625	0.11862	-0.0873276009994395\\
58.625	0.12228	-0.0653606632492048\\
58.625	0.12594	-0.0423375770225947\\
58.625	0.1296	-0.0182583423196147\\
58.625	0.13326	0.00687704085973728\\
58.625	0.13692	0.033068572515461\\
58.625	0.14058	0.0603162526475551\\
58.625	0.14424	0.0886200812560244\\
58.625	0.1479	0.117980058340864\\
58.625	0.15156	0.148396183902075\\
58.625	0.15522	0.179868457939659\\
58.625	0.15888	0.212396880453614\\
58.625	0.16254	0.245981451443939\\
58.625	0.1662	0.280622170910639\\
58.625	0.16986	0.316319038853708\\
58.625	0.17352	0.353072055273151\\
58.625	0.17718	0.390881220168964\\
58.625	0.18084	0.429746533541151\\
58.625	0.1845	0.469667995389708\\
58.625	0.18816	0.510645605714638\\
58.625	0.19182	0.552679364515939\\
58.625	0.19548	0.595769271793614\\
58.625	0.19914	0.639915327547659\\
58.625	0.2028	0.685117531778076\\
58.625	0.20646	0.731375884484865\\
58.625	0.21012	0.778690385668025\\
58.625	0.21378	0.827061035327557\\
58.625	0.21744	0.876487833463461\\
58.625	0.2211	0.926970780075737\\
58.625	0.22476	0.978509875164384\\
58.625	0.22842	1.0311051187294\\
58.625	0.23208	1.08475651077079\\
58.625	0.23574	1.13946405128856\\
58.625	0.2394	1.19522774028269\\
58.625	0.24306	1.2520475777532\\
58.625	0.24672	1.30992356370008\\
58.625	0.25038	1.36885569812333\\
58.625	0.25404	1.42884398102295\\
58.625	0.2577	1.48988841239895\\
58.625	0.26136	1.55198899225132\\
58.625	0.26502	1.61514572058005\\
58.625	0.26868	1.67935859738516\\
58.625	0.27234	1.74462762266664\\
58.625	0.276	1.8109527964245\\
59	0.093	-0.212476021773331\\
59	0.09666	-0.199823614972266\\
59	0.10032	-0.186115059694831\\
59	0.10398	-0.171350355941025\\
59	0.10764	-0.155529503710846\\
59	0.1113	-0.138652503004294\\
59	0.11496	-0.120719353821372\\
59	0.11862	-0.101730056162078\\
59	0.12228	-0.0816846100264124\\
59	0.12594	-0.0605830154143729\\
59	0.1296	-0.0384252723259618\\
59	0.13326	-0.0152113807611823\\
59	0.13692	0.00905865927997251\\
59	0.14058	0.034384847797496\\
59	0.14424	0.0607671847913945\\
59	0.1479	0.0882056702616648\\
59	0.15156	0.116700304208305\\
59	0.15522	0.146251086631317\\
59	0.15888	0.176858017530704\\
59	0.16254	0.208521096906459\\
59	0.1662	0.241240324758587\\
59	0.16986	0.275015701087087\\
59	0.17352	0.30984722589196\\
59	0.17718	0.345734899173201\\
59	0.18084	0.382678720930818\\
59	0.1845	0.420678691164807\\
59	0.18816	0.459734809875164\\
59	0.19182	0.499847077061896\\
59	0.19548	0.541015492725\\
59	0.19914	0.583240056864474\\
59	0.2028	0.626520769480321\\
59	0.20646	0.670857630572539\\
59	0.21012	0.716250640141132\\
59	0.21378	0.762699798186093\\
59	0.21744	0.810205104707423\\
59	0.2211	0.858766559705128\\
59	0.22476	0.908384163179208\\
59	0.22842	0.959057915129657\\
59	0.23208	1.01078781555648\\
59	0.23574	1.06357386445967\\
59	0.2394	1.11741606183924\\
59	0.24306	1.17231440769517\\
59	0.24672	1.22826890202748\\
59	0.25038	1.28527954483616\\
59	0.25404	1.34334633612121\\
59	0.2577	1.40246927588264\\
59	0.26136	1.46264836412043\\
59	0.26502	1.5238836008346\\
59	0.26868	1.58617498602514\\
59	0.27234	1.64952251969205\\
59	0.276	1.71392620183533\\
59.375	0.093	-0.212650662514767\\
59.375	0.09666	-0.201919747328275\\
59.375	0.10032	-0.190132683665409\\
59.375	0.10398	-0.177289471526173\\
59.375	0.10764	-0.163390110910565\\
59.375	0.1113	-0.148434601818582\\
59.375	0.11496	-0.132422944250232\\
59.375	0.11862	-0.115355138205507\\
59.375	0.12228	-0.0972311836844122\\
59.375	0.12594	-0.0780510806869434\\
59.375	0.1296	-0.057814829213103\\
59.375	0.13326	-0.0365224292628924\\
59.375	0.13692	-0.0141738808363082\\
59.375	0.14058	0.00923081606664455\\
59.375	0.14424	0.0336916614459725\\
59.375	0.1479	0.0592086553016721\\
59.375	0.15156	0.0857817976337418\\
59.375	0.15522	0.113411088442185\\
59.375	0.15888	0.142096527727001\\
59.375	0.16254	0.171838115488184\\
59.375	0.1662	0.202635851725743\\
59.375	0.16986	0.234489736439672\\
59.375	0.17352	0.267399769629974\\
59.375	0.17718	0.301365951296647\\
59.375	0.18084	0.336388281439693\\
59.375	0.1845	0.372466760059111\\
59.375	0.18816	0.409601387154897\\
59.375	0.19182	0.447792162727059\\
59.375	0.19548	0.487039086775592\\
59.375	0.19914	0.527342159300496\\
59.375	0.2028	0.568701380301772\\
59.375	0.20646	0.611116749779419\\
59.375	0.21012	0.654588267733441\\
59.375	0.21378	0.699115934163832\\
59.375	0.21744	0.744699749070594\\
59.375	0.2211	0.791339712453729\\
59.375	0.22476	0.839035824313239\\
59.375	0.22842	0.887788084649117\\
59.375	0.23208	0.937596493461367\\
59.375	0.23574	0.988461050749989\\
59.375	0.2394	1.04038175651498\\
59.375	0.24306	1.09335861075635\\
59.375	0.24672	1.14739161347409\\
59.375	0.25038	1.2024807646682\\
59.375	0.25404	1.25862606433868\\
59.375	0.2577	1.31582751248554\\
59.375	0.26136	1.37408510910876\\
59.375	0.26502	1.43339885420836\\
59.375	0.26868	1.49376874778432\\
59.375	0.27234	1.55519478983667\\
59.375	0.276	1.61767698036538\\
59.75	0.093	-0.212047930136994\\
59.75	0.09666	-0.203238506565073\\
59.75	0.10032	-0.193372934516779\\
59.75	0.10398	-0.182451213992112\\
59.75	0.10764	-0.170473344991073\\
59.75	0.1113	-0.157439327513662\\
59.75	0.11496	-0.143349161559881\\
59.75	0.11862	-0.128202847129727\\
59.75	0.12228	-0.112000384223202\\
59.75	0.12594	-0.0947417728403044\\
59.75	0.1296	-0.0764270129810346\\
59.75	0.13326	-0.0570561046453929\\
59.75	0.13692	-0.0366290478333794\\
59.75	0.14058	-0.0151458425449973\\
59.75	0.14424	0.00739351121976173\\
59.75	0.1479	0.0309890134608907\\
59.75	0.15156	0.0556406641783898\\
59.75	0.15522	0.0813484633722624\\
59.75	0.15888	0.108112411042507\\
59.75	0.16254	0.135932507189122\\
59.75	0.1662	0.16480875181211\\
59.75	0.16986	0.194741144911468\\
59.75	0.17352	0.2257296864872\\
59.75	0.17718	0.257774376539302\\
59.75	0.18084	0.290875215067777\\
59.75	0.1845	0.325032202072624\\
59.75	0.18816	0.360245337553842\\
59.75	0.19182	0.396514621511433\\
59.75	0.19548	0.433840053945396\\
59.75	0.19914	0.472221634855729\\
59.75	0.2028	0.511659364242437\\
59.75	0.20646	0.552153242105514\\
59.75	0.21012	0.593703268444965\\
59.75	0.21378	0.636309443260785\\
59.75	0.21744	0.679971766552977\\
59.75	0.2211	0.724690238321541\\
59.75	0.22476	0.77046485856648\\
59.75	0.22842	0.817295627287788\\
59.75	0.23208	0.865182544485467\\
59.75	0.23574	0.914125610159519\\
59.75	0.2394	0.964124824309942\\
59.75	0.24306	1.01518018693674\\
59.75	0.24672	1.06729169803991\\
59.75	0.25038	1.12045935761945\\
59.75	0.25404	1.17468316567536\\
59.75	0.2577	1.22996312220764\\
59.75	0.26136	1.2862992272163\\
59.75	0.26502	1.34369148070132\\
59.75	0.26868	1.40213988266272\\
59.75	0.27234	1.46164443310049\\
59.75	0.276	1.52220513201464\\
60.125	0.093	-0.210667824640015\\
60.125	0.09666	-0.203779892682662\\
60.125	0.10032	-0.195835812248937\\
60.125	0.10398	-0.186835583338841\\
60.125	0.10764	-0.176779205952375\\
60.125	0.1113	-0.165666680089534\\
60.125	0.11496	-0.153498005750322\\
60.125	0.11862	-0.140273182934739\\
60.125	0.12228	-0.125992211642785\\
60.125	0.12594	-0.110655091874456\\
60.125	0.1296	-0.0942618236297567\\
60.125	0.13326	-0.0768124069086874\\
60.125	0.13692	-0.0583068417112428\\
60.125	0.14058	-0.0387451280374296\\
60.125	0.14424	-0.018127265887243\\
60.125	0.1479	0.00354674473931527\\
60.125	0.15156	0.0262769038422437\\
60.125	0.15522	0.0500632114215475\\
60.125	0.15888	0.0749056674772216\\
60.125	0.16254	0.100804272009265\\
60.125	0.1662	0.127759025017683\\
60.125	0.16986	0.155769926502472\\
60.125	0.17352	0.184836976463633\\
60.125	0.17718	0.214960174901164\\
60.125	0.18084	0.246139521815069\\
60.125	0.1845	0.278375017205348\\
60.125	0.18816	0.311666661071995\\
60.125	0.19182	0.346014453415015\\
60.125	0.19548	0.381418394234407\\
60.125	0.19914	0.417878483530169\\
60.125	0.2028	0.455394721302307\\
60.125	0.20646	0.493967107550813\\
60.125	0.21012	0.533595642275694\\
60.125	0.21378	0.574280325476943\\
60.125	0.21744	0.616021157154565\\
60.125	0.2211	0.658818137308561\\
60.125	0.22476	0.70267126593893\\
60.125	0.22842	0.747580543045666\\
60.125	0.23208	0.793545968628775\\
60.125	0.23574	0.840567542688256\\
60.125	0.2394	0.888645265224112\\
60.125	0.24306	0.937779136236339\\
60.125	0.24672	0.987969155724935\\
60.125	0.25038	1.0392153236899\\
60.125	0.25404	1.09151764013125\\
60.125	0.2577	1.14487610504896\\
60.125	0.26136	1.19929071844304\\
60.125	0.26502	1.2547614803135\\
60.125	0.26868	1.31128839066033\\
60.125	0.27234	1.36887144948353\\
60.125	0.276	1.4275106567831\\
60.5	0.093	-0.208510346023826\\
60.5	0.09666	-0.203543905681044\\
60.5	0.10032	-0.19752131686189\\
60.5	0.10398	-0.190442579566362\\
60.5	0.10764	-0.182307693794467\\
60.5	0.1113	-0.173116659546195\\
60.5	0.11496	-0.162869476821556\\
60.5	0.11862	-0.151566145620541\\
60.5	0.12228	-0.139206665943158\\
60.5	0.12594	-0.125791037789399\\
60.5	0.1296	-0.111319261159271\\
60.5	0.13326	-0.0957913360527706\\
60.5	0.13692	-0.0792072624698984\\
60.5	0.14058	-0.0615670404106541\\
60.5	0.14424	-0.0428706698750382\\
60.5	0.1479	-0.0231181508630488\\
60.5	0.15156	-0.00230948337469106\\
60.5	0.15522	0.0195553325900421\\
60.5	0.15888	0.0424762970311455\\
60.5	0.16254	0.0664534099486187\\
60.5	0.1662	0.0914866713424674\\
60.5	0.16986	0.117576081212684\\
60.5	0.17352	0.144721639559276\\
60.5	0.17718	0.172923346382237\\
60.5	0.18084	0.202181201681573\\
60.5	0.1845	0.232495205457279\\
60.5	0.18816	0.263865357709355\\
60.5	0.19182	0.296291658437806\\
60.5	0.19548	0.329774107642628\\
60.5	0.19914	0.364312705323819\\
60.5	0.2028	0.399907451481387\\
60.5	0.20646	0.436558346115322\\
60.5	0.21012	0.474265389225636\\
60.5	0.21378	0.513028580812314\\
60.5	0.21744	0.552847920875365\\
60.5	0.2211	0.593723409414787\\
60.5	0.22476	0.635655046430589\\
60.5	0.22842	0.678642831922755\\
60.5	0.23208	0.722686765891293\\
60.5	0.23574	0.767786848336207\\
60.5	0.2394	0.813943079257488\\
60.5	0.24306	0.861155458655145\\
60.5	0.24672	0.90942398652917\\
60.5	0.25038	0.958748662879571\\
60.5	0.25404	1.00912948770634\\
60.5	0.2577	1.06056646100948\\
60.5	0.26136	1.113059582789\\
60.5	0.26502	1.16660885304488\\
60.5	0.26868	1.22121427177714\\
60.5	0.27234	1.27687583898577\\
60.5	0.276	1.33359355467078\\
60.875	0.093	-0.205575494288422\\
60.875	0.09666	-0.202530545560211\\
60.875	0.10032	-0.198429448355626\\
60.875	0.10398	-0.193272202674671\\
60.875	0.10764	-0.187058808517345\\
60.875	0.1113	-0.179789265883645\\
60.875	0.11496	-0.171463574773574\\
60.875	0.11862	-0.16208173518713\\
60.875	0.12228	-0.151643747124316\\
60.875	0.12594	-0.14014961058513\\
60.875	0.1296	-0.12759932556957\\
60.875	0.13326	-0.113992892077641\\
60.875	0.13692	-0.0993303101093391\\
60.875	0.14058	-0.0836115796646655\\
60.875	0.14424	-0.0668367007436202\\
60.875	0.1479	-0.0490056733462015\\
60.875	0.15156	-0.0301184974724127\\
60.875	0.15522	-0.010175173122252\\
60.875	0.15888	0.0108242997042826\\
60.875	0.16254	0.0328799210071868\\
60.875	0.1662	0.055991690786463\\
60.875	0.16986	0.0801596090421111\\
60.875	0.17352	0.105383675774132\\
60.875	0.17718	0.131663890982522\\
60.875	0.18084	0.159000254667288\\
60.875	0.1845	0.187392766828425\\
60.875	0.18816	0.216841427465932\\
60.875	0.19182	0.247346236579811\\
60.875	0.19548	0.278907194170062\\
60.875	0.19914	0.311524300236686\\
60.875	0.2028	0.345197554779679\\
60.875	0.20646	0.379926957799047\\
60.875	0.21012	0.415712509294787\\
60.875	0.21378	0.452554209266895\\
60.875	0.21744	0.490452057715378\\
60.875	0.2211	0.52940605464023\\
60.875	0.22476	0.569416200041457\\
60.875	0.22842	0.610482493919056\\
60.875	0.23208	0.652604936273023\\
60.875	0.23574	0.695783527103367\\
60.875	0.2394	0.740018266410078\\
60.875	0.24306	0.785309154193164\\
60.875	0.24672	0.831656190452622\\
60.875	0.25038	0.879059375188449\\
60.875	0.25404	0.92751870840065\\
60.875	0.2577	0.977034190089224\\
60.875	0.26136	1.02760582025417\\
60.875	0.26502	1.07923359889548\\
60.875	0.26868	1.13191752601317\\
60.875	0.27234	1.18565760160723\\
60.875	0.276	1.24045382567766\\
61.25	0.093	-0.20186326943382\\
61.25	0.09666	-0.200739812320177\\
61.25	0.10032	-0.198560206730163\\
61.25	0.10398	-0.195324452663778\\
61.25	0.10764	-0.191032550121022\\
61.25	0.1113	-0.185684499101892\\
61.25	0.11496	-0.179280299606392\\
61.25	0.11862	-0.171819951634519\\
61.25	0.12228	-0.163303455186275\\
61.25	0.12594	-0.15373081026166\\
61.25	0.1296	-0.143102016860671\\
61.25	0.13326	-0.131417074983312\\
61.25	0.13692	-0.118675984629579\\
61.25	0.14058	-0.104878745799476\\
61.25	0.14424	-0.0900253584930016\\
61.25	0.1479	-0.0741158227101536\\
61.25	0.15156	-0.0571501384509354\\
61.25	0.15522	-0.0391283057153436\\
61.25	0.15888	-0.0200503245033797\\
61.25	0.16254	8.38051849521015e-05\\
61.25	0.1662	0.0212740833496594\\
61.25	0.16986	0.0435205099907368\\
61.25	0.17352	0.0668230851081875\\
61.25	0.17718	0.0911818087020084\\
61.25	0.18084	0.116596680772203\\
61.25	0.1845	0.143067701318769\\
61.25	0.18816	0.170594870341706\\
61.25	0.19182	0.199178187841014\\
61.25	0.19548	0.228817653816694\\
61.25	0.19914	0.259513268268748\\
61.25	0.2028	0.291265031197174\\
61.25	0.20646	0.324072942601968\\
61.25	0.21012	0.357937002483141\\
61.25	0.21378	0.392857210840678\\
61.25	0.21744	0.428833567674587\\
61.25	0.2211	0.465866072984872\\
61.25	0.22476	0.503954726771528\\
61.25	0.22842	0.543099529034556\\
61.25	0.23208	0.583300479773953\\
61.25	0.23574	0.624557578989726\\
61.25	0.2394	0.666870826681866\\
61.25	0.24306	0.710240222850385\\
61.25	0.24672	0.754665767495269\\
61.25	0.25038	0.800147460616528\\
61.25	0.25404	0.846685302214156\\
61.25	0.2577	0.894279292288163\\
61.25	0.26136	0.942929430838534\\
61.25	0.26502	0.99263571786528\\
61.25	0.26868	1.04339815336839\\
61.25	0.27234	1.09521673734789\\
61.25	0.276	1.14809146980375\\
61.625	0.093	-0.19737367146\\
61.625	0.09666	-0.19817170596093\\
61.625	0.10032	-0.197913591985488\\
61.625	0.10398	-0.196599329533673\\
61.625	0.10764	-0.194228918605487\\
61.625	0.1113	-0.190802359200926\\
61.625	0.11496	-0.186319651319997\\
61.625	0.11862	-0.180780794962694\\
61.625	0.12228	-0.174185790129021\\
61.625	0.12594	-0.166534636818975\\
61.625	0.1296	-0.157827335032556\\
61.625	0.13326	-0.148063884769768\\
61.625	0.13692	-0.137244286030606\\
61.625	0.14058	-0.125368538815074\\
61.625	0.14424	-0.112436643123168\\
61.625	0.1479	-0.0984485989548908\\
61.625	0.15156	-0.0834044063102433\\
61.625	0.15522	-0.0673040651892222\\
61.625	0.15888	-0.0501475755918289\\
61.625	0.16254	-0.031934937518066\\
61.625	0.1662	-0.0126661509679293\\
61.625	0.16986	0.00765878405857734\\
61.625	0.17352	0.0290398675614574\\
61.625	0.17718	0.0514770995407077\\
61.625	0.18084	0.0749704799963316\\
61.625	0.1845	0.0995200089283272\\
61.625	0.18816	0.125125686336692\\
61.625	0.19182	0.151787512221433\\
61.625	0.19548	0.179505486582542\\
61.625	0.19914	0.208279609420025\\
61.625	0.2028	0.23810988073388\\
61.625	0.20646	0.268996300524104\\
61.625	0.21012	0.300938868790706\\
61.625	0.21378	0.333937585533672\\
61.625	0.21744	0.367992450753014\\
61.625	0.2211	0.403103464448728\\
61.625	0.22476	0.439270626620814\\
61.625	0.22842	0.476493937269272\\
61.625	0.23208	0.514773396394097\\
61.625	0.23574	0.5541090039953\\
61.625	0.2394	0.594500760072873\\
61.625	0.24306	0.635948664626818\\
61.625	0.24672	0.678452717657134\\
61.625	0.25038	0.72201291916382\\
61.625	0.25404	0.76662926914688\\
61.625	0.2577	0.812301767606316\\
61.625	0.26136	0.859030414542116\\
61.625	0.26502	0.906815209954292\\
61.625	0.26868	0.955656153842836\\
61.625	0.27234	1.00555324620776\\
61.625	0.276	1.05650648704905\\
62	0.093	-0.192106700366977\\
62	0.09666	-0.194826226482477\\
62	0.10032	-0.196489604121602\\
62	0.10398	-0.197096833284359\\
62	0.10764	-0.196647913970744\\
62	0.1113	-0.195142846180754\\
62	0.11496	-0.192581629914396\\
62	0.11862	-0.188964265171662\\
62	0.12228	-0.184290751952559\\
62	0.12594	-0.178561090257084\\
62	0.1296	-0.171775280085236\\
62	0.13326	-0.163933321437017\\
62	0.13692	-0.155035214312425\\
62	0.14058	-0.145080958711464\\
62	0.14424	-0.134070554634129\\
62	0.1479	-0.122004002080422\\
62	0.15156	-0.108881301050343\\
62	0.15522	-0.0947024515438929\\
62	0.15888	-0.0794674535610703\\
62	0.16254	-0.063176307101878\\
62	0.1662	-0.0458290121663121\\
62	0.16986	-0.0274255687543743\\
62	0.17352	-0.00796597686606493\\
62	0.17718	0.0125497634986147\\
62	0.18084	0.0341216523396679\\
62	0.1845	0.0567496896570929\\
62	0.18816	0.0804338754508884\\
62	0.19182	0.105174209721059\\
62	0.19548	0.130970692467601\\
62	0.19914	0.15782332369051\\
62	0.2028	0.185732103389795\\
62	0.20646	0.214697031565451\\
62	0.21012	0.244718108217478\\
62	0.21378	0.275795333345878\\
62	0.21744	0.307928706950649\\
62	0.2211	0.341118229031789\\
62	0.22476	0.375363899589308\\
62	0.22842	0.410665718623195\\
62	0.23208	0.44702368613345\\
62	0.23574	0.484437802120081\\
62	0.2394	0.522908066583084\\
62	0.24306	0.562434479522458\\
62	0.24672	0.603017040938204\\
62	0.25038	0.644655750830322\\
62	0.25404	0.687350609198812\\
62	0.2577	0.731101616043674\\
62	0.26136	0.775908771364907\\
62	0.26502	0.821772075162508\\
62	0.26868	0.868691527436485\\
62	0.27234	0.916667128186838\\
62	0.276	0.965698877413555\\
62.375	0.093	-0.186062356154745\\
62.375	0.09666	-0.190703373884815\\
62.375	0.10032	-0.194288243138512\\
62.375	0.10398	-0.196816963915838\\
62.375	0.10764	-0.198289536216792\\
62.375	0.1113	-0.198705960041374\\
62.375	0.11496	-0.198066235389585\\
62.375	0.11862	-0.196370362261423\\
62.375	0.12228	-0.19361834065689\\
62.375	0.12594	-0.189810170575985\\
62.375	0.1296	-0.184945852018706\\
62.375	0.13326	-0.179025384985059\\
62.375	0.13692	-0.172048769475037\\
62.375	0.14058	-0.164016005488646\\
62.375	0.14424	-0.154927093025881\\
62.375	0.1479	-0.144782032086744\\
62.375	0.15156	-0.133580822671237\\
62.375	0.15522	-0.121323464779356\\
62.375	0.15888	-0.108009958411104\\
62.375	0.16254	-0.0936403035664823\\
62.375	0.1662	-0.0782145002454853\\
62.375	0.16986	-0.0617325484481199\\
62.375	0.17352	-0.0441944481743795\\
62.375	0.17718	-0.0256001994242705\\
62.375	0.18084	-0.00594980219778618\\
62.375	0.1845	0.0147567435050682\\
62.375	0.18816	0.036519437684293\\
62.375	0.19182	0.0593382803398927\\
62.375	0.19548	0.0832132714718643\\
62.375	0.19914	0.108144411080203\\
62.375	0.2028	0.134131699164916\\
62.375	0.20646	0.161175135726002\\
62.375	0.21012	0.189274720763462\\
62.375	0.21378	0.218430454277291\\
62.375	0.21744	0.248642336267489\\
62.375	0.2211	0.279910366734061\\
62.375	0.22476	0.312234545677009\\
62.375	0.22842	0.345614873096322\\
62.375	0.23208	0.38005134899201\\
62.375	0.23574	0.415543973364071\\
62.375	0.2394	0.452092746212502\\
62.375	0.24306	0.489697667537309\\
62.375	0.24672	0.528358737338481\\
62.375	0.25038	0.568075955616029\\
62.375	0.25404	0.608849322369948\\
62.375	0.2577	0.650678837600243\\
62.375	0.26136	0.693564501306905\\
62.375	0.26502	0.737506313489936\\
62.375	0.26868	0.782504274149342\\
62.375	0.27234	0.828558383285124\\
62.375	0.276	0.875668640897274\\
62.75	0.093	-0.179240638823306\\
62.75	0.09666	-0.185803148167946\\
62.75	0.10032	-0.191309509036215\\
62.75	0.10398	-0.195759721428111\\
62.75	0.10764	-0.199153785343636\\
62.75	0.1113	-0.201491700782787\\
62.75	0.11496	-0.202773467745568\\
62.75	0.11862	-0.202999086231977\\
62.75	0.12228	-0.202168556242014\\
62.75	0.12594	-0.200281877775678\\
62.75	0.1296	-0.197339050832972\\
62.75	0.13326	-0.193340075413894\\
62.75	0.13692	-0.188284951518444\\
62.75	0.14058	-0.182173679146622\\
62.75	0.14424	-0.175006258298428\\
62.75	0.1479	-0.166782688973861\\
62.75	0.15156	-0.157502971172924\\
62.75	0.15522	-0.147167104895615\\
62.75	0.15888	-0.135775090141932\\
62.75	0.16254	-0.123326926911881\\
62.75	0.1662	-0.109822615205454\\
62.75	0.16986	-0.0952621550226596\\
62.75	0.17352	-0.0796455463634897\\
62.75	0.17718	-0.0629727892279497\\
62.75	0.18084	-0.0452438836160378\\
62.75	0.1845	-0.0264588295277524\\
62.75	0.18816	-0.0066176269630982\\
62.75	0.19182	0.0142797240779309\\
62.75	0.19548	0.0362332235953318\\
62.75	0.19914	0.0592428715891029\\
62.75	0.2028	0.0833086680592461\\
62.75	0.20646	0.108430613005761\\
62.75	0.21012	0.134608706428647\\
62.75	0.21378	0.161842948327905\\
62.75	0.21744	0.190133338703536\\
62.75	0.2211	0.219479877555538\\
62.75	0.22476	0.249882564883915\\
62.75	0.22842	0.281341400688657\\
62.75	0.23208	0.313856384969774\\
62.75	0.23574	0.347427517727264\\
62.75	0.2394	0.382054798961125\\
62.75	0.24306	0.417738228671362\\
62.75	0.24672	0.454477806857966\\
62.75	0.25038	0.492273533520943\\
62.75	0.25404	0.531125408660292\\
62.75	0.2577	0.571033432276016\\
62.75	0.26136	0.611997604368104\\
62.75	0.26502	0.654017924936568\\
62.75	0.26868	0.697094393981403\\
62.75	0.27234	0.741227011502615\\
62.75	0.276	0.786415777500194\\
63.125	0.093	-0.171641548372656\\
63.125	0.09666	-0.180125549331865\\
63.125	0.10032	-0.187553401814702\\
63.125	0.10398	-0.193925105821169\\
63.125	0.10764	-0.199240661351266\\
63.125	0.1113	-0.203500068404986\\
63.125	0.11496	-0.20670332698234\\
63.125	0.11862	-0.208850437083318\\
63.125	0.12228	-0.209941398707926\\
63.125	0.12594	-0.20997621185616\\
63.125	0.1296	-0.208954876528023\\
63.125	0.13326	-0.206877392723515\\
63.125	0.13692	-0.203743760442636\\
63.125	0.14058	-0.199553979685385\\
63.125	0.14424	-0.19430805045176\\
63.125	0.1479	-0.188005972741765\\
63.125	0.15156	-0.180647746555399\\
63.125	0.15522	-0.172233371892658\\
63.125	0.15888	-0.162762848753546\\
63.125	0.16254	-0.152236177138066\\
63.125	0.1662	-0.14065335704621\\
63.125	0.16986	-0.128014388477984\\
63.125	0.17352	-0.114319271433385\\
63.125	0.17718	-0.0995680059124158\\
63.125	0.18084	-0.0837605919150728\\
63.125	0.1845	-0.066897029441358\\
63.125	0.18816	-0.0489773184912745\\
63.125	0.19182	-0.0300014590648161\\
63.125	0.19548	-0.00996945116198589\\
63.125	0.19914	0.0111187052172146\\
63.125	0.2028	0.0332630100727871\\
63.125	0.20646	0.0564634634047312\\
63.125	0.21012	0.0807200652130504\\
63.125	0.21378	0.106032815497738\\
63.125	0.21744	0.132401714258797\\
63.125	0.2211	0.159826761496229\\
63.125	0.22476	0.188307957210032\\
63.125	0.22842	0.217845301400207\\
63.125	0.23208	0.248438794066753\\
63.125	0.23574	0.280088435209673\\
63.125	0.2394	0.312794224828963\\
63.125	0.24306	0.346556162924629\\
63.125	0.24672	0.381374249496663\\
63.125	0.25038	0.417248484545069\\
63.125	0.25404	0.454178868069847\\
63.125	0.2577	0.492165400071\\
63.125	0.26136	0.531208080548522\\
63.125	0.26502	0.571306909502415\\
63.125	0.26868	0.612461886932679\\
63.125	0.27234	0.65467301283932\\
63.125	0.276	0.697940287222329\\
63.5	0.093	-0.163265084802796\\
63.5	0.09666	-0.173670577376576\\
63.5	0.10032	-0.183019921473983\\
63.5	0.10398	-0.191313117095019\\
63.5	0.10764	-0.198550164239685\\
63.5	0.1113	-0.204731062907978\\
63.5	0.11496	-0.2098558130999\\
63.5	0.11862	-0.213924414815449\\
63.5	0.12228	-0.216936868054628\\
63.5	0.12594	-0.218893172817433\\
63.5	0.1296	-0.219793329103866\\
63.5	0.13326	-0.219637336913929\\
63.5	0.13692	-0.218425196247619\\
63.5	0.14058	-0.216156907104938\\
63.5	0.14424	-0.212832469485884\\
63.5	0.1479	-0.208451883390458\\
63.5	0.15156	-0.203015148818662\\
63.5	0.15522	-0.196522265770493\\
63.5	0.15888	-0.188973234245951\\
63.5	0.16254	-0.18036805424504\\
63.5	0.1662	-0.170706725767755\\
63.5	0.16986	-0.159989248814099\\
63.5	0.17352	-0.148215623384071\\
63.5	0.17718	-0.135385849477672\\
63.5	0.18084	-0.121499927094902\\
63.5	0.1845	-0.106557856235758\\
63.5	0.18816	-0.0905596369002413\\
63.5	0.19182	-0.0735052690883535\\
63.5	0.19548	-0.055394752800094\\
63.5	0.19914	-0.0362280880354642\\
63.5	0.2028	-0.0160052747944623\\
63.5	0.20646	0.00527368692291108\\
63.5	0.21012	0.0276087971166596\\
63.5	0.21378	0.0510000557867765\\
63.5	0.21744	0.0754474629332653\\
63.5	0.2211	0.100951018556126\\
63.5	0.22476	0.127510722655362\\
63.5	0.22842	0.155126575230966\\
63.5	0.23208	0.183798576282942\\
63.5	0.23574	0.213526725811295\\
63.5	0.2394	0.244311023816014\\
63.5	0.24306	0.276151470297109\\
63.5	0.24672	0.309048065254573\\
63.5	0.25038	0.343000808688408\\
63.5	0.25404	0.378009700598615\\
63.5	0.2577	0.414074740985198\\
63.5	0.26136	0.451195929848148\\
63.5	0.26502	0.489373267187471\\
63.5	0.26868	0.528606753003165\\
63.5	0.27234	0.568896387295235\\
63.5	0.276	0.610242170063673\\
63.875	0.093	-0.154111248113726\\
63.875	0.09666	-0.166438232302078\\
63.875	0.10032	-0.177709068014059\\
63.875	0.10398	-0.187923755249664\\
63.875	0.10764	-0.1970822940089\\
63.875	0.1113	-0.205184684291763\\
63.875	0.11496	-0.212230926098255\\
63.875	0.11862	-0.218221019428374\\
63.875	0.12228	-0.223154964282123\\
63.875	0.12594	-0.227032760659499\\
63.875	0.1296	-0.229854408560501\\
63.875	0.13326	-0.231619907985135\\
63.875	0.13692	-0.232329258933396\\
63.875	0.14058	-0.231982461405285\\
63.875	0.14424	-0.230579515400802\\
63.875	0.1479	-0.228120420919947\\
63.875	0.15156	-0.22460517796272\\
63.875	0.15522	-0.220033786529121\\
63.875	0.15888	-0.21440624661915\\
63.875	0.16254	-0.207722558232809\\
63.875	0.1662	-0.199982721370095\\
63.875	0.16986	-0.19118673603101\\
63.875	0.17352	-0.181334602215551\\
63.875	0.17718	-0.170426319923723\\
63.875	0.18084	-0.158461889155521\\
63.875	0.1845	-0.145441309910948\\
63.875	0.18816	-0.131364582190002\\
63.875	0.19182	-0.116231705992685\\
63.875	0.19548	-0.100042681318996\\
63.875	0.19914	-0.0827975081689369\\
63.875	0.2028	-0.0644961865425058\\
63.875	0.20646	-0.045138716439703\\
63.875	0.21012	-0.0247250978605216\\
63.875	0.21378	-0.00325533080497542\\
63.875	0.21744	0.0192705847269428\\
63.875	0.2211	0.0428526487352328\\
63.875	0.22476	0.0674908612198981\\
63.875	0.22842	0.0931852221809315\\
63.875	0.23208	0.119935731618337\\
63.875	0.23574	0.147742389532119\\
63.875	0.2394	0.176605195922268\\
63.875	0.24306	0.206524150788792\\
63.875	0.24672	0.237499254131685\\
63.875	0.25038	0.26953050595095\\
63.875	0.25404	0.302617906246589\\
63.875	0.2577	0.336761455018602\\
63.875	0.26136	0.371961152266981\\
63.875	0.26502	0.408216997991733\\
63.875	0.26868	0.445528992192857\\
63.875	0.27234	0.483897134870356\\
63.875	0.276	0.523321426024223\\
64.25	0.093	-0.144180038305451\\
64.25	0.09666	-0.158428514108372\\
64.25	0.10032	-0.171620841434921\\
64.25	0.10398	-0.183757020285098\\
64.25	0.10764	-0.194837050658906\\
64.25	0.1113	-0.204860932556338\\
64.25	0.11496	-0.213828665977401\\
64.25	0.11862	-0.22174025092209\\
64.25	0.12228	-0.22859568739041\\
64.25	0.12594	-0.234394975382354\\
64.25	0.1296	-0.239138114897929\\
64.25	0.13326	-0.242825105937132\\
64.25	0.13692	-0.245455948499963\\
64.25	0.14058	-0.247030642586423\\
64.25	0.14424	-0.247549188196511\\
64.25	0.1479	-0.247011585330224\\
64.25	0.15156	-0.24541783398757\\
64.25	0.15522	-0.24276793416854\\
64.25	0.15888	-0.239061885873139\\
64.25	0.16254	-0.234299689101369\\
64.25	0.1662	-0.228481343853224\\
64.25	0.16986	-0.22160685012871\\
64.25	0.17352	-0.213676207927821\\
64.25	0.17718	-0.204689417250564\\
64.25	0.18084	-0.194646478096931\\
64.25	0.1845	-0.183547390466928\\
64.25	0.18816	-0.171392154360557\\
64.25	0.19182	-0.158180769777807\\
64.25	0.19548	-0.143913236718689\\
64.25	0.19914	-0.1285895551832\\
64.25	0.2028	-0.112209725171336\\
64.25	0.20646	-0.094773746683104\\
64.25	0.21012	-0.0762816197184968\\
64.25	0.21378	-0.0567333442775213\\
64.25	0.21744	-0.0361289203601738\\
64.25	0.2211	-0.0144683479664509\\
64.25	0.22476	0.00824837290364377\\
64.25	0.22842	0.0320212422501065\\
64.25	0.23208	0.0568502600729412\\
64.25	0.23574	0.0827354263721523\\
64.25	0.2394	0.109676741147731\\
64.25	0.24306	0.137674204399684\\
64.25	0.24672	0.166727816128006\\
64.25	0.25038	0.196837576332704\\
64.25	0.25404	0.22800348501377\\
64.25	0.2577	0.260225542171211\\
64.25	0.26136	0.293503747805024\\
64.25	0.26502	0.327838101915205\\
64.25	0.26868	0.363228604501758\\
64.25	0.27234	0.399675255564687\\
64.25	0.276	0.437178055103987\\
64.625	0.093	-0.133471455377967\\
64.625	0.09666	-0.149641422795459\\
64.625	0.10032	-0.164755241736579\\
64.625	0.10398	-0.178812912201327\\
64.625	0.10764	-0.191814434189703\\
64.625	0.1113	-0.203759807701706\\
64.625	0.11496	-0.21464903273734\\
64.625	0.11862	-0.224482109296599\\
64.625	0.12228	-0.23325903737949\\
64.625	0.12594	-0.240979816986005\\
64.625	0.1296	-0.247644448116149\\
64.625	0.13326	-0.253252930769924\\
64.625	0.13692	-0.257805264947324\\
64.625	0.14058	-0.261301450648355\\
64.625	0.14424	-0.263741487873011\\
64.625	0.1479	-0.265125376621297\\
64.625	0.15156	-0.265453116893212\\
64.625	0.15522	-0.264724708688754\\
64.625	0.15888	-0.262940152007923\\
64.625	0.16254	-0.260099446850723\\
64.625	0.1662	-0.256202593217149\\
64.625	0.16986	-0.251249591107205\\
64.625	0.17352	-0.245240440520887\\
64.625	0.17718	-0.238175141458201\\
64.625	0.18084	-0.230053693919137\\
64.625	0.1845	-0.220876097903705\\
64.625	0.18816	-0.210642353411904\\
64.625	0.19182	-0.199352460443724\\
64.625	0.19548	-0.187006418999177\\
64.625	0.19914	-0.173604229078259\\
64.625	0.2028	-0.159145890680966\\
64.625	0.20646	-0.143631403807304\\
64.625	0.21012	-0.127060768457264\\
64.625	0.21378	-0.109433984630859\\
64.625	0.21744	-0.0907510523280826\\
64.625	0.2211	-0.0710119715489339\\
64.625	0.22476	-0.0502167422934063\\
64.625	0.22842	-0.0283653645615143\\
64.625	0.23208	-0.0054578383532502\\
64.625	0.23574	0.0185058363313901\\
64.625	0.2394	0.0435256594923978\\
64.625	0.24306	0.0696016311297845\\
64.625	0.24672	0.0967337512435358\\
64.625	0.25038	0.124922019833659\\
64.625	0.25404	0.154166436900158\\
64.625	0.2577	0.184467002443029\\
64.625	0.26136	0.215823716462267\\
64.625	0.26502	0.248236578957881\\
64.625	0.26868	0.281705589929863\\
64.625	0.27234	0.316230749378221\\
64.625	0.276	0.351812057302951\\
65	0.093	-0.121985499331276\\
65	0.09666	-0.140076958363339\\
65	0.10032	-0.157112268919029\\
65	0.10398	-0.173091430998346\\
65	0.10764	-0.188014444601293\\
65	0.1113	-0.201881309727866\\
65	0.11496	-0.21469202637807\\
65	0.11862	-0.226446594551899\\
65	0.12228	-0.237145014249361\\
65	0.12594	-0.246787285470446\\
65	0.1296	-0.255373408215161\\
65	0.13326	-0.262903382483505\\
65	0.13692	-0.269377208275475\\
65	0.14058	-0.274794885591077\\
65	0.14424	-0.279156414430304\\
65	0.1479	-0.282461794793159\\
65	0.15156	-0.284711026679646\\
65	0.15522	-0.285904110089757\\
65	0.15888	-0.286041045023497\\
65	0.16254	-0.285121831480868\\
65	0.1662	-0.283146469461864\\
65	0.16986	-0.28011495896649\\
65	0.17352	-0.276027299994742\\
65	0.17718	-0.270883492546624\\
65	0.18084	-0.264683536622135\\
65	0.1845	-0.25742743222127\\
65	0.18816	-0.249115179344039\\
65	0.19182	-0.239746777990431\\
65	0.19548	-0.229322228160454\\
65	0.19914	-0.217841529854103\\
65	0.2028	-0.205304683071384\\
65	0.20646	-0.19171168781229\\
65	0.21012	-0.177062544076824\\
65	0.21378	-0.16135725186499\\
65	0.21744	-0.14459581117678\\
65	0.2211	-0.126778222012202\\
65	0.22476	-0.107904484371245\\
65	0.22842	-0.0879745982539237\\
65	0.23208	-0.0669885636602303\\
65	0.23574	-0.0449463805901607\\
65	0.2394	-0.0218480490437201\\
65	0.24306	0.00230643097909233\\
65	0.24672	0.027517059478273\\
65	0.25038	0.0537838364538294\\
65	0.25404	0.0811067619057537\\
65	0.2577	0.109485835834057\\
65	0.26136	0.138921058238725\\
65	0.26502	0.169412429119769\\
65	0.26868	0.20095994847718\\
65	0.27234	0.233563616310967\\
65	0.276	0.267223432621126\\
65.375	0.093	-0.109722170165376\\
65.375	0.09666	-0.129735120812007\\
65.375	0.10032	-0.148691922982266\\
65.375	0.10398	-0.166592576676156\\
65.375	0.10764	-0.183437081893673\\
65.375	0.1113	-0.199225438634815\\
65.375	0.11496	-0.213957646899589\\
65.375	0.11862	-0.22763370668799\\
65.375	0.12228	-0.24025361800002\\
65.375	0.12594	-0.251817380835677\\
65.375	0.1296	-0.262324995194961\\
65.375	0.13326	-0.271776461077876\\
65.375	0.13692	-0.280171778484417\\
65.375	0.14058	-0.28751094741459\\
65.375	0.14424	-0.293793967868388\\
65.375	0.1479	-0.299020839845813\\
65.375	0.15156	-0.303191563346869\\
65.375	0.15522	-0.306306138371551\\
65.375	0.15888	-0.308364564919861\\
65.375	0.16254	-0.309366842991801\\
65.375	0.1662	-0.309312972587368\\
65.375	0.16986	-0.308202953706564\\
65.375	0.17352	-0.306036786349387\\
65.375	0.17718	-0.302814470515842\\
65.375	0.18084	-0.298536006205919\\
65.375	0.1845	-0.293201393419628\\
65.375	0.18816	-0.286810632156965\\
65.375	0.19182	-0.279363722417928\\
65.375	0.19548	-0.270860664202521\\
65.375	0.19914	-0.261301457510741\\
65.375	0.2028	-0.250686102342593\\
65.375	0.20646	-0.239014598698069\\
65.375	0.21012	-0.226286946577174\\
65.375	0.21378	-0.212503145979907\\
65.375	0.21744	-0.197663196906271\\
65.375	0.2211	-0.181767099356261\\
65.375	0.22476	-0.164814853329878\\
65.375	0.22842	-0.146806458827124\\
65.375	0.23208	-0.127741915848001\\
65.375	0.23574	-0.107621224392502\\
65.375	0.2394	-0.086444384460632\\
65.375	0.24306	-0.0642113960523902\\
65.375	0.24672	-0.0409222591677767\\
65.375	0.25038	-0.0165769738067909\\
65.375	0.25404	0.0088244600305627\\
65.375	0.2577	0.0352820423442957\\
65.375	0.26136	0.0627957731343929\\
65.375	0.26502	0.0913656524008655\\
65.375	0.26868	0.120991680143706\\
65.375	0.27234	0.151673856362927\\
65.375	0.276	0.183412181058511\\
65.75	0.093	-0.0966814678802672\\
65.75	0.09666	-0.118615910141467\\
65.75	0.10032	-0.139494203926299\\
65.75	0.10398	-0.159316349234758\\
65.75	0.10764	-0.178082346066846\\
65.75	0.1113	-0.195792194422559\\
65.75	0.11496	-0.212445894301903\\
65.75	0.11862	-0.228043445704874\\
65.75	0.12228	-0.242584848631473\\
65.75	0.12594	-0.256070103081701\\
65.75	0.1296	-0.268499209055556\\
65.75	0.13326	-0.279872166553042\\
65.75	0.13692	-0.290188975574153\\
65.75	0.14058	-0.299449636118895\\
65.75	0.14424	-0.307654148187263\\
65.75	0.1479	-0.314802511779258\\
65.75	0.15156	-0.320894726894884\\
65.75	0.15522	-0.325930793534137\\
65.75	0.15888	-0.329910711697017\\
65.75	0.16254	-0.332834481383528\\
65.75	0.1662	-0.334702102593665\\
65.75	0.16986	-0.335513575327433\\
65.75	0.17352	-0.335268899584825\\
65.75	0.17718	-0.333968075365848\\
65.75	0.18084	-0.3316111026705\\
65.75	0.1845	-0.328197981498776\\
65.75	0.18816	-0.323728711850684\\
65.75	0.19182	-0.318203293726216\\
65.75	0.19548	-0.311621727125381\\
65.75	0.19914	-0.303984012048172\\
65.75	0.2028	-0.29529014849459\\
65.75	0.20646	-0.285540136464641\\
65.75	0.21012	-0.274733975958313\\
65.75	0.21378	-0.26287166697562\\
65.75	0.21744	-0.249953209516552\\
65.75	0.2211	-0.235978603581111\\
65.75	0.22476	-0.220947849169299\\
65.75	0.22842	-0.204860946281116\\
65.75	0.23208	-0.187717894916564\\
65.75	0.23574	-0.169518695075635\\
65.75	0.2394	-0.150263346758336\\
65.75	0.24306	-0.129951849964661\\
65.75	0.24672	-0.108584204694622\\
65.75	0.25038	-0.086160410948207\\
65.75	0.25404	-0.0626804687254205\\
65.75	0.2577	-0.0381443780262618\\
65.75	0.26136	-0.0125521388507317\\
65.75	0.26502	0.0140962488011667\\
65.75	0.26868	0.0418007849294404\\
65.75	0.27234	0.0705614695340899\\
65.75	0.276	0.100378302615104\\
66.125	0.093	-0.0828633924759457\\
66.125	0.09666	-0.106719326351716\\
66.125	0.10032	-0.129519111751119\\
66.125	0.10398	-0.151262748674146\\
66.125	0.10764	-0.171950237120805\\
66.125	0.1113	-0.191581577091089\\
66.125	0.11496	-0.210156768585004\\
66.125	0.11862	-0.227675811602545\\
66.125	0.12228	-0.244138706143715\\
66.125	0.12594	-0.259545452208513\\
66.125	0.1296	-0.273896049796939\\
66.125	0.13326	-0.287190498908994\\
66.125	0.13692	-0.299428799544676\\
66.125	0.14058	-0.310610951703989\\
66.125	0.14424	-0.320736955386927\\
66.125	0.1479	-0.329806810593493\\
66.125	0.15156	-0.33782051732369\\
66.125	0.15522	-0.344778075577511\\
66.125	0.15888	-0.350679485354963\\
66.125	0.16254	-0.355524746656044\\
66.125	0.1662	-0.359313859480752\\
66.125	0.16986	-0.362046823829088\\
66.125	0.17352	-0.363723639701052\\
66.125	0.17718	-0.364344307096645\\
66.125	0.18084	-0.363908826015867\\
66.125	0.1845	-0.362417196458714\\
66.125	0.18816	-0.359869418425192\\
66.125	0.19182	-0.356265491915296\\
66.125	0.19548	-0.351605416929031\\
66.125	0.19914	-0.345889193466392\\
66.125	0.2028	-0.339116821527382\\
66.125	0.20646	-0.331288301111999\\
66.125	0.21012	-0.322403632220245\\
66.125	0.21378	-0.31246281485212\\
66.125	0.21744	-0.301465849007622\\
66.125	0.2211	-0.289412734686756\\
66.125	0.22476	-0.276303471889511\\
66.125	0.22842	-0.262138060615898\\
66.125	0.23208	-0.246916500865913\\
66.125	0.23574	-0.230638792639559\\
66.125	0.2394	-0.213304935936831\\
66.125	0.24306	-0.194914930757727\\
66.125	0.24672	-0.175468777102255\\
66.125	0.25038	-0.15496647497041\\
66.125	0.25404	-0.133408024362198\\
66.125	0.2577	-0.110793425277606\\
66.125	0.26136	-0.0871226777166467\\
66.125	0.26502	-0.062395781679319\\
66.125	0.26868	-0.036612737165616\\
66.125	0.27234	-0.00977354417553711\\
66.125	0.276	0.018121797290906\\
66.5	0.093	-0.0682679439524163\\
66.5	0.09666	-0.0940453694427614\\
66.5	0.10032	-0.118766646456731\\
66.5	0.10398	-0.142431774994331\\
66.5	0.10764	-0.16504075505556\\
66.5	0.1113	-0.186593586640414\\
66.5	0.11496	-0.207090269748898\\
66.5	0.11862	-0.226530804381011\\
66.5	0.12228	-0.244915190536751\\
66.5	0.12594	-0.26224342821612\\
66.5	0.1296	-0.278515517419115\\
66.5	0.13326	-0.293731458145742\\
66.5	0.13692	-0.307891250395993\\
66.5	0.14058	-0.320994894169876\\
66.5	0.14424	-0.333042389467384\\
66.5	0.1479	-0.344033736288522\\
66.5	0.15156	-0.353968934633287\\
66.5	0.15522	-0.362847984501682\\
66.5	0.15888	-0.370670885893702\\
66.5	0.16254	-0.377437638809354\\
66.5	0.1662	-0.383148243248631\\
66.5	0.16986	-0.387802699211539\\
66.5	0.17352	-0.391401006698074\\
66.5	0.17718	-0.393943165708237\\
66.5	0.18084	-0.395429176242027\\
66.5	0.1845	-0.395859038299444\\
66.5	0.18816	-0.395232751880493\\
66.5	0.19182	-0.393550316985168\\
66.5	0.19548	-0.390811733613473\\
66.5	0.19914	-0.387017001765405\\
66.5	0.2028	-0.382166121440965\\
66.5	0.20646	-0.376259092640154\\
66.5	0.21012	-0.369295915362967\\
66.5	0.21378	-0.361276589609412\\
66.5	0.21744	-0.352201115379488\\
66.5	0.2211	-0.34206949267319\\
66.5	0.22476	-0.330881721490515\\
66.5	0.22842	-0.318637801831473\\
66.5	0.23208	-0.305337733696059\\
66.5	0.23574	-0.290981517084272\\
66.5	0.2394	-0.275569151996117\\
66.5	0.24306	-0.259100638431584\\
66.5	0.24672	-0.241575976390683\\
66.5	0.25038	-0.222995165873409\\
66.5	0.25404	-0.203358206879764\\
66.5	0.2577	-0.182665099409743\\
66.5	0.26136	-0.160915843463358\\
66.5	0.26502	-0.138110439040597\\
66.5	0.26868	-0.114248886141465\\
66.5	0.27234	-0.0893311847659564\\
66.5	0.276	-0.0633573349140804\\
66.875	0.093	-0.052895122309681\\
66.875	0.09666	-0.0805940394145949\\
66.875	0.10032	-0.107236808043137\\
66.875	0.10398	-0.132823428195306\\
66.875	0.10764	-0.157353899871106\\
66.875	0.1113	-0.180828223070529\\
66.875	0.11496	-0.203246397793585\\
66.875	0.11862	-0.224608424040267\\
66.875	0.12228	-0.244914301810578\\
66.875	0.12594	-0.264164031104515\\
66.875	0.1296	-0.282357611922083\\
66.875	0.13326	-0.299495044263279\\
66.875	0.13692	-0.315576328128101\\
66.875	0.14058	-0.330601463516554\\
66.875	0.14424	-0.344570450428633\\
66.875	0.1479	-0.357483288864339\\
66.875	0.15156	-0.369339978823676\\
66.875	0.15522	-0.38014052030664\\
66.875	0.15888	-0.389884913313231\\
66.875	0.16254	-0.398573157843454\\
66.875	0.1662	-0.406205253897302\\
66.875	0.16986	-0.412781201474779\\
66.875	0.17352	-0.418301000575883\\
66.875	0.17718	-0.422764651200617\\
66.875	0.18084	-0.426172153348977\\
66.875	0.1845	-0.428523507020965\\
66.875	0.18816	-0.429818712216585\\
66.875	0.19182	-0.43005776893583\\
66.875	0.19548	-0.429240677178706\\
66.875	0.19914	-0.427367436945209\\
66.875	0.2028	-0.424438048235339\\
66.875	0.20646	-0.420452511049098\\
66.875	0.21012	-0.415410825386482\\
66.875	0.21378	-0.409312991247498\\
66.875	0.21744	-0.402159008632142\\
66.875	0.2211	-0.393948877540413\\
66.875	0.22476	-0.38468259797231\\
66.875	0.22842	-0.374360169927838\\
66.875	0.23208	-0.362981593406995\\
66.875	0.23574	-0.350546868409778\\
66.875	0.2394	-0.337055994936191\\
66.875	0.24306	-0.322508972986232\\
66.875	0.24672	-0.306905802559901\\
66.875	0.25038	-0.290246483657195\\
66.875	0.25404	-0.272531016278124\\
66.875	0.2577	-0.253759400422673\\
66.875	0.26136	-0.233931636090855\\
66.875	0.26502	-0.213047723282665\\
66.875	0.26868	-0.191107661998104\\
66.875	0.27234	-0.168111452237166\\
66.875	0.276	-0.144059093999861\\
67.25	0.093	-0.0367449275477378\\
67.25	0.09666	-0.0663653362672207\\
67.25	0.10032	-0.0949295965103316\\
67.25	0.10398	-0.122437708277073\\
67.25	0.10764	-0.148889671567442\\
67.25	0.1113	-0.174285486381437\\
67.25	0.11496	-0.198625152719063\\
67.25	0.11862	-0.221908670580315\\
67.25	0.12228	-0.244136039965196\\
67.25	0.12594	-0.265307260873705\\
67.25	0.1296	-0.285422333305841\\
67.25	0.13326	-0.304481257261607\\
67.25	0.13692	-0.322484032741\\
67.25	0.14058	-0.339430659744024\\
67.25	0.14424	-0.355321138270673\\
67.25	0.1479	-0.370155468320951\\
67.25	0.15156	-0.383933649894858\\
67.25	0.15522	-0.396655682992392\\
67.25	0.15888	-0.408321567613553\\
67.25	0.16254	-0.418931303758345\\
67.25	0.1662	-0.428484891426763\\
67.25	0.16986	-0.436982330618811\\
67.25	0.17352	-0.444423621334487\\
67.25	0.17718	-0.450808763573792\\
67.25	0.18084	-0.456137757336723\\
67.25	0.1845	-0.460410602623282\\
67.25	0.18816	-0.463627299433472\\
67.25	0.19182	-0.465787847767287\\
67.25	0.19548	-0.466892247624731\\
67.25	0.19914	-0.466940499005804\\
67.25	0.2028	-0.465932601910502\\
67.25	0.20646	-0.463868556338832\\
67.25	0.21012	-0.460748362290786\\
67.25	0.21378	-0.456572019766373\\
67.25	0.21744	-0.451339528765587\\
67.25	0.2211	-0.44505088928843\\
67.25	0.22476	-0.437706101334897\\
67.25	0.22842	-0.429305164904996\\
67.25	0.23208	-0.419848079998723\\
67.25	0.23574	-0.409334846616077\\
67.25	0.2394	-0.397765464757061\\
67.25	0.24306	-0.385139934421669\\
67.25	0.24672	-0.371458255609908\\
67.25	0.25038	-0.356720428321776\\
67.25	0.25404	-0.340926452557272\\
67.25	0.2577	-0.324076328316393\\
67.25	0.26136	-0.306170055599145\\
67.25	0.26502	-0.287207634405526\\
67.25	0.26868	-0.267189064735535\\
67.25	0.27234	-0.246114346589168\\
67.25	0.276	-0.223983479966433\\
67.625	0.093	-0.0198173596665869\\
67.625	0.09666	-0.0513592600006404\\
67.625	0.10032	-0.081845011858322\\
67.625	0.10398	-0.111274615239632\\
67.625	0.10764	-0.139648070144572\\
67.625	0.1113	-0.166965376573138\\
67.625	0.11496	-0.193226534525332\\
67.625	0.11862	-0.218431544001155\\
67.625	0.12228	-0.242580405000607\\
67.625	0.12594	-0.265673117523686\\
67.625	0.1296	-0.287709681570393\\
67.625	0.13326	-0.308690097140729\\
67.625	0.13692	-0.328614364234692\\
67.625	0.14058	-0.347482482852287\\
67.625	0.14424	-0.365294452993505\\
67.625	0.1479	-0.382050274658353\\
67.625	0.15156	-0.397749947846831\\
67.625	0.15522	-0.412393472558935\\
67.625	0.15888	-0.425980848794667\\
67.625	0.16254	-0.43851207655403\\
67.625	0.1662	-0.449987155837018\\
67.625	0.16986	-0.460406086643635\\
67.625	0.17352	-0.469768868973881\\
67.625	0.17718	-0.478075502827756\\
67.625	0.18084	-0.485325988205257\\
67.625	0.1845	-0.491520325106387\\
67.625	0.18816	-0.496658513531148\\
67.625	0.19182	-0.500740553479534\\
67.625	0.19548	-0.503766444951548\\
67.625	0.19914	-0.505736187947189\\
67.625	0.2028	-0.506649782466461\\
67.625	0.20646	-0.506507228509361\\
67.625	0.21012	-0.505308526075886\\
67.625	0.21378	-0.503053675166043\\
67.625	0.21744	-0.499742675779828\\
67.625	0.2211	-0.495375527917241\\
67.625	0.22476	-0.489952231578279\\
67.625	0.22842	-0.483472786762945\\
67.625	0.23208	-0.475937193471243\\
67.625	0.23574	-0.467345451703168\\
67.625	0.2394	-0.457697561458722\\
67.625	0.24306	-0.446993522737901\\
67.625	0.24672	-0.435233335540711\\
67.625	0.25038	-0.422416999867146\\
67.625	0.25404	-0.408544515717213\\
67.625	0.2577	-0.393615883090904\\
67.625	0.26136	-0.377631101988227\\
67.625	0.26502	-0.360590172409179\\
67.625	0.26868	-0.342493094353758\\
67.625	0.27234	-0.323339867821962\\
67.625	0.276	-0.303130492813798\\
68	0.093	-0.00211241866622647\\
68	0.09666	-0.0355758106148507\\
68	0.10032	-0.0679830540871029\\
68	0.10398	-0.0993341490829835\\
68	0.10764	-0.129629095602494\\
68	0.1113	-0.158867893645629\\
68	0.11496	-0.187050543212396\\
68	0.11862	-0.214177044302787\\
68	0.12228	-0.24024739691681\\
68	0.12594	-0.26526160105446\\
68	0.1296	-0.289219656715736\\
68	0.13326	-0.312121563900644\\
68	0.13692	-0.333967322609178\\
68	0.14058	-0.354756932841342\\
68	0.14424	-0.37449039459713\\
68	0.1479	-0.393167707876549\\
68	0.15156	-0.410788872679597\\
68	0.15522	-0.427353889006271\\
68	0.15888	-0.442862756856573\\
68	0.16254	-0.457315476230507\\
68	0.1662	-0.470712047128066\\
68	0.16986	-0.483052469549255\\
68	0.17352	-0.494336743494071\\
68	0.17718	-0.504564868962514\\
68	0.18084	-0.513736845954586\\
68	0.1845	-0.521852674470286\\
68	0.18816	-0.528912354509618\\
68	0.19182	-0.534915886072571\\
68	0.19548	-0.539863269159156\\
68	0.19914	-0.543754503769371\\
68	0.2028	-0.546589589903213\\
68	0.20646	-0.548368527560684\\
68	0.21012	-0.549091316741777\\
68	0.21378	-0.548757957446504\\
68	0.21744	-0.54736844967486\\
68	0.2211	-0.544922793426844\\
68	0.22476	-0.541420988702452\\
68	0.22842	-0.536863035501689\\
68	0.23208	-0.531248933824557\\
68	0.23574	-0.524578683671053\\
68	0.2394	-0.516852285041178\\
68	0.24306	-0.508069737934927\\
68	0.24672	-0.498231042352305\\
68	0.25038	-0.487336198293314\\
68	0.25404	-0.475385205757951\\
68	0.2577	-0.462378064746213\\
68	0.26136	-0.448314775258103\\
68	0.26502	-0.433195337293625\\
68	0.26868	-0.417019750852776\\
68	0.27234	-0.39978801593555\\
68	0.276	-0.381500132541957\\
68.375	0.093	0.0163698954533436\\
68.375	0.09666	-0.0190149881098495\\
68.375	0.10032	-0.0533437231966706\\
68.375	0.10398	-0.0866163098071237\\
68.375	0.10764	-0.118832747941203\\
68.375	0.1113	-0.149993037598909\\
68.375	0.11496	-0.180097178780246\\
68.375	0.11862	-0.209145171485208\\
68.375	0.12228	-0.237137015713802\\
68.375	0.12594	-0.264072711466021\\
68.375	0.1296	-0.289952258741869\\
68.375	0.13326	-0.314775657541346\\
68.375	0.13692	-0.338542907864451\\
68.375	0.14058	-0.361254009711185\\
68.375	0.14424	-0.382908963081544\\
68.375	0.1479	-0.403507767975534\\
68.375	0.15156	-0.423050424393151\\
68.375	0.15522	-0.441536932334395\\
68.375	0.15888	-0.458967291799268\\
68.375	0.16254	-0.475341502787772\\
68.375	0.1662	-0.4906595652999\\
68.375	0.16986	-0.504921479335659\\
68.375	0.17352	-0.518127244895045\\
68.375	0.17718	-0.530276861978062\\
68.375	0.18084	-0.541370330584705\\
68.375	0.1845	-0.551407650714972\\
68.375	0.18816	-0.560388822368874\\
68.375	0.19182	-0.568313845546402\\
68.375	0.19548	-0.575182720247554\\
68.375	0.19914	-0.580995446472339\\
68.375	0.2028	-0.585752024220749\\
68.375	0.20646	-0.589452453492791\\
68.375	0.21012	-0.592096734288457\\
68.375	0.21378	-0.593684866607756\\
68.375	0.21744	-0.594216850450678\\
68.375	0.2211	-0.593692685817233\\
68.375	0.22476	-0.592112372707412\\
68.375	0.22842	-0.589475911121223\\
68.375	0.23208	-0.585783301058659\\
68.375	0.23574	-0.581034542519725\\
68.375	0.2394	-0.57522963550442\\
68.375	0.24306	-0.568368580012737\\
68.375	0.24672	-0.560451376044689\\
68.375	0.25038	-0.551478023600265\\
68.375	0.25404	-0.541448522679473\\
68.375	0.2577	-0.530362873282305\\
68.375	0.26136	-0.51822107540877\\
68.375	0.26502	-0.505023129058859\\
68.375	0.26868	-0.49076903423258\\
68.375	0.27234	-0.475458790929925\\
68.375	0.276	-0.459092399150903\\
68.75	0.093	0.0356295826921285\\
68.75	0.09666	-0.00167679248563707\\
68.75	0.10032	-0.0379270191870306\\
68.75	0.10398	-0.0731210974120526\\
68.75	0.10764	-0.107259027160703\\
68.75	0.1113	-0.140340808432979\\
68.75	0.11496	-0.172366441228885\\
68.75	0.11862	-0.203335925548418\\
68.75	0.12228	-0.233249261391583\\
68.75	0.12594	-0.262106448758372\\
68.75	0.1296	-0.289907487648789\\
68.75	0.13326	-0.316652378062838\\
68.75	0.13692	-0.342341120000512\\
68.75	0.14058	-0.366973713461817\\
68.75	0.14424	-0.390550158446747\\
68.75	0.1479	-0.413070454955305\\
68.75	0.15156	-0.434534602987493\\
68.75	0.15522	-0.45494260254331\\
68.75	0.15888	-0.474294453622752\\
68.75	0.16254	-0.492590156225826\\
68.75	0.1662	-0.509829710352525\\
68.75	0.16986	-0.526013116002854\\
68.75	0.17352	-0.541140373176812\\
68.75	0.17718	-0.555211481874395\\
68.75	0.18084	-0.568226442095609\\
68.75	0.1845	-0.58018525384045\\
68.75	0.18816	-0.59108791710892\\
68.75	0.19182	-0.600934431901018\\
68.75	0.19548	-0.609724798216741\\
68.75	0.19914	-0.617459016056097\\
68.75	0.2028	-0.624137085419077\\
68.75	0.20646	-0.629759006305689\\
68.75	0.21012	-0.634324778715927\\
68.75	0.21378	-0.637834402649792\\
68.75	0.21744	-0.640287878107289\\
68.75	0.2211	-0.641685205088411\\
68.75	0.22476	-0.642026383593161\\
68.75	0.22842	-0.641311413621542\\
68.75	0.23208	-0.639540295173548\\
68.75	0.23574	-0.636713028249186\\
68.75	0.2394	-0.632829612848448\\
68.75	0.24306	-0.627890048971339\\
68.75	0.24672	-0.621894336617861\\
68.75	0.25038	-0.614842475788008\\
68.75	0.25404	-0.606734466481787\\
68.75	0.2577	-0.597570308699186\\
68.75	0.26136	-0.587350002440222\\
68.75	0.26502	-0.576073547704882\\
68.75	0.26868	-0.563740944493173\\
68.75	0.27234	-0.550352192805089\\
68.75	0.276	-0.535907292640633\\
69.125	0.093	0.055666643050114\\
69.125	0.09666	0.0164387762577778\\
69.125	0.10032	-0.0217329420581864\\
69.125	0.10398	-0.0588485118977772\\
69.125	0.10764	-0.0949079332609979\\
69.125	0.1113	-0.129911206147843\\
69.125	0.11496	-0.163858330558322\\
69.125	0.11862	-0.196749306492426\\
69.125	0.12228	-0.228584133950159\\
69.125	0.12594	-0.259362812931519\\
69.125	0.1296	-0.289085343436507\\
69.125	0.13326	-0.317751725465125\\
69.125	0.13692	-0.345361959017369\\
69.125	0.14058	-0.371916044093243\\
69.125	0.14424	-0.397413980692745\\
69.125	0.1479	-0.421855768815874\\
69.125	0.15156	-0.445241408462633\\
69.125	0.15522	-0.467570899633018\\
69.125	0.15888	-0.488844242327031\\
69.125	0.16254	-0.509061436544675\\
69.125	0.1662	-0.528222482285944\\
69.125	0.16986	-0.546327379550846\\
69.125	0.17352	-0.56337612833937\\
69.125	0.17718	-0.579368728651528\\
69.125	0.18084	-0.594305180487309\\
69.125	0.1845	-0.608185483846721\\
69.125	0.18816	-0.621009638729761\\
69.125	0.19182	-0.63277764513643\\
69.125	0.19548	-0.643489503066723\\
69.125	0.19914	-0.65314521252065\\
69.125	0.2028	-0.661744773498201\\
69.125	0.20646	-0.669288185999384\\
69.125	0.21012	-0.675775450024188\\
69.125	0.21378	-0.681206565572628\\
69.125	0.21744	-0.685581532644692\\
69.125	0.2211	-0.688900351240388\\
69.125	0.22476	-0.691163021359705\\
69.125	0.22842	-0.692369543002657\\
69.125	0.23208	-0.692519916169234\\
69.125	0.23574	-0.691614140859442\\
69.125	0.2394	-0.689652217073275\\
69.125	0.24306	-0.686634144810736\\
69.125	0.24672	-0.682559924071826\\
69.125	0.25038	-0.677429554856543\\
69.125	0.25404	-0.671243037164893\\
69.125	0.2577	-0.664000370996863\\
69.125	0.26136	-0.655701556352469\\
69.125	0.26502	-0.646346593231703\\
69.125	0.26868	-0.635935481634562\\
69.125	0.27234	-0.624468221561048\\
69.125	0.276	-0.611944813011164\\
69.5	0.093	0.0764810765273092\\
69.5	0.09666	0.0353317181204023\\
69.5	0.10032	-0.00476149181012903\\
69.5	0.10398	-0.0437985532642923\\
69.5	0.10764	-0.0817794662420837\\
69.5	0.1113	-0.118704230743501\\
69.5	0.11496	-0.154572846768549\\
69.5	0.11862	-0.189385314317223\\
69.5	0.12228	-0.223141633389527\\
69.5	0.12594	-0.255841803985456\\
69.5	0.1296	-0.287485826105015\\
69.5	0.13326	-0.318073699748203\\
69.5	0.13692	-0.347605424915018\\
69.5	0.14058	-0.376081001605465\\
69.5	0.14424	-0.403500429819534\\
69.5	0.1479	-0.429863709557234\\
69.5	0.15156	-0.455170840818563\\
69.5	0.15522	-0.479421823603519\\
69.5	0.15888	-0.502616657912105\\
69.5	0.16254	-0.524755343744317\\
69.5	0.1662	-0.545837881100159\\
69.5	0.16986	-0.565864269979627\\
69.5	0.17352	-0.584834510382726\\
69.5	0.17718	-0.602748602309451\\
69.5	0.18084	-0.619606545759802\\
69.5	0.1845	-0.635408340733785\\
69.5	0.18816	-0.650153987231396\\
69.5	0.19182	-0.663843485252636\\
69.5	0.19548	-0.6764768347975\\
69.5	0.19914	-0.688054035865993\\
69.5	0.2028	-0.698575088458119\\
69.5	0.20646	-0.708039992573869\\
69.5	0.21012	-0.716448748213248\\
69.5	0.21378	-0.723801355376254\\
69.5	0.21744	-0.730097814062889\\
69.5	0.2211	-0.735338124273156\\
69.5	0.22476	-0.739522286007043\\
69.5	0.22842	-0.742650299264566\\
69.5	0.23208	-0.744722164045714\\
69.5	0.23574	-0.745737880350489\\
69.5	0.2394	-0.745697448178896\\
69.5	0.24306	-0.744600867530925\\
69.5	0.24672	-0.742448138406588\\
69.5	0.25038	-0.739239260805876\\
69.5	0.25404	-0.734974234728793\\
69.5	0.2577	-0.729653060175337\\
69.5	0.26136	-0.723275737145511\\
69.5	0.26502	-0.715842265639315\\
69.5	0.26868	-0.707352645656745\\
69.5	0.27234	-0.697806877197802\\
69.5	0.276	-0.687204960262488\\
69.875	0.093	0.0980728831237121\\
69.875	0.09666	0.0550020331022363\\
69.875	0.10032	0.0129873315571326\\
69.875	0.10398	-0.0279712215115996\\
69.875	0.10764	-0.0678736261039616\\
69.875	0.1113	-0.106719882219948\\
69.875	0.11496	-0.144509989859567\\
69.875	0.11862	-0.181243949022811\\
69.875	0.12228	-0.216921759709686\\
69.875	0.12594	-0.251543421920186\\
69.875	0.1296	-0.285108935654315\\
69.875	0.13326	-0.317618300912074\\
69.875	0.13692	-0.349071517693458\\
69.875	0.14058	-0.379468585998473\\
69.875	0.14424	-0.408809505827115\\
69.875	0.1479	-0.437094277179386\\
69.875	0.15156	-0.464322900055286\\
69.875	0.15522	-0.490495374454811\\
69.875	0.15888	-0.515611700377965\\
69.875	0.16254	-0.53967187782475\\
69.875	0.1662	-0.562675906795159\\
69.875	0.16986	-0.584623787289202\\
69.875	0.17352	-0.605515519306867\\
69.875	0.17718	-0.625351102848163\\
69.875	0.18084	-0.644130537913088\\
69.875	0.1845	-0.661853824501639\\
69.875	0.18816	-0.67852096261382\\
69.875	0.19182	-0.69413195224963\\
69.875	0.19548	-0.708686793409065\\
69.875	0.19914	-0.722185486092129\\
69.875	0.2028	-0.734628030298822\\
69.875	0.20646	-0.746014426029146\\
69.875	0.21012	-0.756344673283092\\
69.875	0.21378	-0.765618772060669\\
69.875	0.21744	-0.773836722361878\\
69.875	0.2211	-0.780998524186712\\
69.875	0.22476	-0.78710417753517\\
69.875	0.22842	-0.792153682407264\\
69.875	0.23208	-0.796147038802982\\
69.875	0.23574	-0.799084246722328\\
69.875	0.2394	-0.800965306165306\\
69.875	0.24306	-0.801790217131905\\
69.875	0.24672	-0.801558979622136\\
69.875	0.25038	-0.800271593635995\\
69.875	0.25404	-0.797928059173485\\
69.875	0.2577	-0.794528376234597\\
69.875	0.26136	-0.790072544819341\\
69.875	0.26502	-0.784560564927716\\
69.875	0.26868	-0.777992436559716\\
69.875	0.27234	-0.77036815971534\\
69.875	0.276	-0.7616877343946\\
70.25	0.093	0.120442062839325\\
70.25	0.09666	0.0754497212032799\\
70.25	0.10032	0.0315135280436037\\
70.25	0.10398	-0.0113665166396974\\
70.25	0.10764	-0.0531904128466301\\
70.25	0.1113	-0.0939581605771874\\
70.25	0.11496	-0.133669759831375\\
70.25	0.11862	-0.17232521060919\\
70.25	0.12228	-0.209924512910634\\
70.25	0.12594	-0.246467666735706\\
70.25	0.1296	-0.281954672084404\\
70.25	0.13326	-0.316385528956734\\
70.25	0.13692	-0.349760237352688\\
70.25	0.14058	-0.382078797272274\\
70.25	0.14424	-0.413341208715487\\
70.25	0.1479	-0.443547471682326\\
70.25	0.15156	-0.472697586172797\\
70.25	0.15522	-0.500791552186892\\
70.25	0.15888	-0.527829369724617\\
70.25	0.16254	-0.553811038785973\\
70.25	0.1662	-0.578736559370953\\
70.25	0.16986	-0.602605931479562\\
70.25	0.17352	-0.625419155111802\\
70.25	0.17718	-0.647176230267669\\
70.25	0.18084	-0.667877156947161\\
70.25	0.1845	-0.687521935150282\\
70.25	0.18816	-0.706110564877034\\
70.25	0.19182	-0.723643046127415\\
70.25	0.19548	-0.74011937890142\\
70.25	0.19914	-0.755539563199056\\
70.25	0.2028	-0.769903599020319\\
70.25	0.20646	-0.78321148636521\\
70.25	0.21012	-0.79546322523373\\
70.25	0.21378	-0.806658815625878\\
70.25	0.21744	-0.816798257541654\\
70.25	0.2211	-0.825881550981059\\
70.25	0.22476	-0.833908695944091\\
70.25	0.22842	-0.840879692430752\\
70.25	0.23208	-0.846794540441041\\
70.25	0.23574	-0.851653239974957\\
70.25	0.2394	-0.855455791032502\\
70.25	0.24306	-0.858202193613676\\
70.25	0.24672	-0.859892447718477\\
70.25	0.25038	-0.860526553346907\\
70.25	0.25404	-0.860104510498965\\
70.25	0.2577	-0.858626319174647\\
70.25	0.26136	-0.856091979373965\\
70.25	0.26502	-0.852501491096907\\
70.25	0.26868	-0.847854854343478\\
70.25	0.27234	-0.842152069113673\\
70.25	0.276	-0.835393135407504\\
70.625	0.093	0.143588615674148\\
70.625	0.09666	0.0966747824235295\\
70.625	0.10032	0.0508170976492862\\
70.625	0.10398	0.00601556135141262\\
70.625	0.10764	-0.0377298264700889\\
70.625	0.1113	-0.0804190658152169\\
70.625	0.11496	-0.122052156683977\\
70.625	0.11862	-0.162629099076361\\
70.625	0.12228	-0.202149892992375\\
70.625	0.12594	-0.240614538432018\\
70.625	0.1296	-0.278023035395287\\
70.625	0.13326	-0.314375383882186\\
70.625	0.13692	-0.349671583892713\\
70.625	0.14058	-0.383911635426867\\
70.625	0.14424	-0.417095538484651\\
70.625	0.1479	-0.449223293066061\\
70.625	0.15156	-0.4802948991711\\
70.625	0.15522	-0.510310356799768\\
70.625	0.15888	-0.539269665952062\\
70.625	0.16254	-0.567172826627988\\
70.625	0.1662	-0.594019838827539\\
70.625	0.16986	-0.619810702550719\\
70.625	0.17352	-0.644545417797526\\
70.625	0.17718	-0.668223984567963\\
70.625	0.18084	-0.69084640286203\\
70.625	0.1845	-0.712412672679722\\
70.625	0.18816	-0.732922794021044\\
70.625	0.19182	-0.752376766885992\\
70.625	0.19548	-0.770774591274568\\
70.625	0.19914	-0.788116267186774\\
70.625	0.2028	-0.804401794622608\\
70.625	0.20646	-0.81963117358207\\
70.625	0.21012	-0.833804404065157\\
70.625	0.21378	-0.846921486071876\\
70.625	0.21744	-0.858982419602226\\
70.625	0.2211	-0.869987204656201\\
70.625	0.22476	-0.879935841233801\\
70.625	0.22842	-0.888828329335032\\
70.625	0.23208	-0.896664668959892\\
70.625	0.23574	-0.903444860108379\\
70.625	0.2394	-0.909168902780495\\
70.625	0.24306	-0.913836796976235\\
70.625	0.24672	-0.917448542695607\\
70.625	0.25038	-0.920004139938607\\
70.625	0.25404	-0.92150358870524\\
70.625	0.2577	-0.921946888995492\\
70.625	0.26136	-0.921334040809378\\
70.625	0.26502	-0.919665044146891\\
70.625	0.26868	-0.916939899008032\\
70.625	0.27234	-0.913158605392798\\
70.625	0.276	-0.908321163301196\\
71	0.093	0.167512541628178\\
71	0.09666	0.11867721676299\\
71	0.10032	0.0708980403741746\\
71	0.10398	0.0241750124617322\\
71	0.10764	-0.02149186697434\\
71	0.1113	-0.0661025979340387\\
71	0.11496	-0.109657180417367\\
71	0.11862	-0.152155614424322\\
71	0.12228	-0.193597899954907\\
71	0.12594	-0.233984037009119\\
71	0.1296	-0.273314025586958\\
71	0.13326	-0.31158786568843\\
71	0.13692	-0.348805557313525\\
71	0.14058	-0.384967100462251\\
71	0.14424	-0.420072495134603\\
71	0.1479	-0.454121741330584\\
71	0.15156	-0.487114839050196\\
71	0.15522	-0.519051788293433\\
71	0.15888	-0.549932589060297\\
71	0.16254	-0.579757241350794\\
71	0.1662	-0.608525745164915\\
71	0.16986	-0.636238100502666\\
71	0.17352	-0.662894307364044\\
71	0.17718	-0.688494365749052\\
71	0.18084	-0.713038275657686\\
71	0.1845	-0.73652603708995\\
71	0.18816	-0.758957650045841\\
71	0.19182	-0.780333114525362\\
71	0.19548	-0.800652430528507\\
71	0.19914	-0.819915598055283\\
71	0.2028	-0.838122617105687\\
71	0.20646	-0.85527348767972\\
71	0.21012	-0.871368209777378\\
71	0.21378	-0.886406783398668\\
71	0.21744	-0.900389208543585\\
71	0.2211	-0.913315485212131\\
71	0.22476	-0.925185613404301\\
71	0.22842	-0.935999593120103\\
71	0.23208	-0.945757424359533\\
71	0.23574	-0.954459107122591\\
71	0.2394	-0.962104641409277\\
71	0.24306	-0.968694027219589\\
71	0.24672	-0.974227264553531\\
71	0.25038	-0.978704353411102\\
71	0.25404	-0.982125293792302\\
71	0.2577	-0.984490085697125\\
71	0.26136	-0.985798729125581\\
71	0.26502	-0.986051224077665\\
71	0.26868	-0.985247570553377\\
71	0.27234	-0.983387768552713\\
71	0.276	-0.980471818075682\\
71.375	0.093	0.192213840701419\\
71.375	0.09666	0.141457024221661\\
71.375	0.10032	0.0917563562182744\\
71.375	0.10398	0.0431118366912613\\
71.375	0.10764	-0.00447653435938156\\
71.375	0.1113	-0.0510087569336491\\
71.375	0.11496	-0.0964848310315484\\
71.375	0.11862	-0.140904756653074\\
71.375	0.12228	-0.18426853379823\\
71.375	0.12594	-0.226576162467012\\
71.375	0.1296	-0.267827642659422\\
71.375	0.13326	-0.308022974375462\\
71.375	0.13692	-0.347162157615129\\
71.375	0.14058	-0.385245192378425\\
71.375	0.14424	-0.422272078665348\\
71.375	0.1479	-0.458242816475899\\
71.375	0.15156	-0.49315740581008\\
71.375	0.15522	-0.527015846667887\\
71.375	0.15888	-0.559818139049325\\
71.375	0.16254	-0.591564282954388\\
71.375	0.1662	-0.62225427838308\\
71.375	0.16986	-0.651888125335402\\
71.375	0.17352	-0.680465823811351\\
71.375	0.17718	-0.707987373810927\\
71.375	0.18084	-0.734452775334133\\
71.375	0.1845	-0.759862028380968\\
71.375	0.18816	-0.78421513295143\\
71.375	0.19182	-0.807512089045521\\
71.375	0.19548	-0.829752896663237\\
71.375	0.19914	-0.850937555804584\\
71.375	0.2028	-0.871066066469556\\
71.375	0.20646	-0.890138428658159\\
71.375	0.21012	-0.908154642370388\\
71.375	0.21378	-0.925114707606248\\
71.375	0.21744	-0.941018624365736\\
71.375	0.2211	-0.955866392648852\\
71.375	0.22476	-0.969658012455593\\
71.375	0.22842	-0.982393483785966\\
71.375	0.23208	-0.994072806639967\\
71.375	0.23574	-1.0046959810176\\
71.375	0.2394	-1.01426300691885\\
71.375	0.24306	-1.02277388434373\\
71.375	0.24672	-1.03022861329225\\
71.375	0.25038	-1.03662719376439\\
71.375	0.25404	-1.04196962576016\\
71.375	0.2577	-1.04625590927955\\
71.375	0.26136	-1.04948604432258\\
71.375	0.26502	-1.05166003088923\\
71.375	0.26868	-1.05277786897951\\
71.375	0.27234	-1.05283955859342\\
71.375	0.276	-1.05184509973096\\
71.75	0.093	0.217692512893865\\
71.75	0.09666	0.165014204799537\\
71.75	0.10032	0.113392045181582\\
71.75	0.10398	0.0628260340399965\\
71.75	0.10764	0.0133161713747847\\
71.75	0.1113	-0.0351375428140553\\
71.75	0.11496	-0.0825351085265252\\
71.75	0.11862	-0.12887652576262\\
71.75	0.12228	-0.174161794522346\\
71.75	0.12594	-0.218390914805699\\
71.75	0.1296	-0.26156388661268\\
71.75	0.13326	-0.303680709943289\\
71.75	0.13692	-0.344741384797526\\
71.75	0.14058	-0.384745911175393\\
71.75	0.14424	-0.423694289076887\\
71.75	0.1479	-0.461586518502008\\
71.75	0.15156	-0.498422599450758\\
71.75	0.15522	-0.534202531923136\\
71.75	0.15888	-0.568926315919142\\
71.75	0.16254	-0.602593951438779\\
71.75	0.1662	-0.635205438482041\\
71.75	0.16986	-0.666760777048934\\
71.75	0.17352	-0.697259967139455\\
71.75	0.17718	-0.726703008753603\\
71.75	0.18084	-0.755089901891373\\
71.75	0.1845	-0.782420646552779\\
71.75	0.18816	-0.808695242737812\\
71.75	0.19182	-0.833913690446473\\
71.75	0.19548	-0.85807598967876\\
71.75	0.19914	-0.881182140434674\\
71.75	0.2028	-0.90323214271422\\
71.75	0.20646	-0.924225996517394\\
71.75	0.21012	-0.944163701844193\\
71.75	0.21378	-0.963045258694624\\
71.75	0.21744	-0.980870667068683\\
71.75	0.2211	-0.997639926966366\\
71.75	0.22476	-1.01335303838768\\
71.75	0.22842	-1.02801000133262\\
71.75	0.23208	-1.04161081580119\\
71.75	0.23574	-1.05415548179339\\
71.75	0.2394	-1.06564399930922\\
71.75	0.24306	-1.07607636834867\\
71.75	0.24672	-1.08545258891175\\
71.75	0.25038	-1.09377266099846\\
71.75	0.25404	-1.10103658460881\\
71.75	0.2577	-1.10724435974277\\
71.75	0.26136	-1.11239598640036\\
71.75	0.26502	-1.11649146458159\\
71.75	0.26868	-1.11953079428644\\
71.75	0.27234	-1.12151397551492\\
71.75	0.276	-1.12244100826703\\
72.125	0.093	0.24394855820552\\
72.125	0.09666	0.189348758496622\\
72.125	0.10032	0.135805107264096\\
72.125	0.10398	0.0833176045079412\\
72.125	0.10764	0.0318862502281587\\
72.125	0.1113	-0.0184889555752502\\
72.125	0.11496	-0.0678080129022908\\
72.125	0.11862	-0.116070921752958\\
72.125	0.12228	-0.163277682127253\\
72.125	0.12594	-0.209428294025176\\
72.125	0.1296	-0.254522757446726\\
72.125	0.13326	-0.298561072391908\\
72.125	0.13692	-0.341543238860714\\
72.125	0.14058	-0.383469256853151\\
72.125	0.14424	-0.424339126369216\\
72.125	0.1479	-0.464152847408906\\
72.125	0.15156	-0.502910419972229\\
72.125	0.15522	-0.540611844059176\\
72.125	0.15888	-0.577257119669752\\
72.125	0.16254	-0.612846246803961\\
72.125	0.1662	-0.647379225461792\\
72.125	0.16986	-0.680856055643252\\
72.125	0.17352	-0.71327673734834\\
72.125	0.17718	-0.744641270577063\\
72.125	0.18084	-0.774949655329408\\
72.125	0.1845	-0.80420189160538\\
72.125	0.18816	-0.832397979404984\\
72.125	0.19182	-0.859537918728216\\
72.125	0.19548	-0.885621709575073\\
72.125	0.19914	-0.910649351945558\\
72.125	0.2028	-0.934620845839674\\
72.125	0.20646	-0.957536191257419\\
72.125	0.21012	-0.979395388198785\\
72.125	0.21378	-1.00019843666379\\
72.125	0.21744	-1.01994533665242\\
72.125	0.2211	-1.03863608816467\\
72.125	0.22476	-1.05627069120055\\
72.125	0.22842	-1.07284914576007\\
72.125	0.23208	-1.08837145184321\\
72.125	0.23574	-1.10283760944998\\
72.125	0.2394	-1.11624761858037\\
72.125	0.24306	-1.1286014792344\\
72.125	0.24672	-1.13989919141205\\
72.125	0.25038	-1.15014075511333\\
72.125	0.25404	-1.15932617033824\\
72.125	0.2577	-1.16745543708678\\
72.125	0.26136	-1.17452855535895\\
72.125	0.26502	-1.18054552515474\\
72.125	0.26868	-1.18550634647416\\
72.125	0.27234	-1.18941101931721\\
72.125	0.276	-1.19225954368389\\
72.5	0.093	0.270981976636384\\
72.5	0.09666	0.214460685312917\\
72.5	0.10032	0.158995542465822\\
72.5	0.10398	0.104586548095095\\
72.5	0.10764	0.0512337022007423\\
72.5	0.1113	-0.00106299521723724\\
72.5	0.11496	-0.0523035441588467\\
72.5	0.11862	-0.102487944624085\\
72.5	0.12228	-0.15161619661295\\
72.5	0.12594	-0.199688300125443\\
72.5	0.1296	-0.246704255161565\\
72.5	0.13326	-0.292664061721315\\
72.5	0.13692	-0.337567719804692\\
72.5	0.14058	-0.3814152294117\\
72.5	0.14424	-0.424206590542333\\
72.5	0.1479	-0.465941803196597\\
72.5	0.15156	-0.506620867374488\\
72.5	0.15522	-0.546243783076006\\
72.5	0.15888	-0.584810550301151\\
72.5	0.16254	-0.622321169049929\\
72.5	0.1662	-0.658775639322331\\
72.5	0.16986	-0.694173961118365\\
72.5	0.17352	-0.728516134438023\\
72.5	0.17718	-0.761802159281314\\
72.5	0.18084	-0.794032035648228\\
72.5	0.1845	-0.825205763538772\\
72.5	0.18816	-0.855323342952946\\
72.5	0.19182	-0.884384773890749\\
72.5	0.19548	-0.912390056352177\\
72.5	0.19914	-0.939339190337232\\
72.5	0.2028	-0.965232175845919\\
72.5	0.20646	-0.990069012878231\\
72.5	0.21012	-1.01384970143417\\
72.5	0.21378	-1.03657424151374\\
72.5	0.21744	-1.05824263311694\\
72.5	0.2211	-1.07885487624377\\
72.5	0.22476	-1.09841097089422\\
72.5	0.22842	-1.1169109170683\\
72.5	0.23208	-1.13435471476602\\
72.5	0.23574	-1.15074236398735\\
72.5	0.2394	-1.16607386473232\\
72.5	0.24306	-1.18034921700092\\
72.5	0.24672	-1.19356842079314\\
72.5	0.25038	-1.20573147610899\\
72.5	0.25404	-1.21683838294847\\
72.5	0.2577	-1.22688914131158\\
72.5	0.26136	-1.23588375119831\\
72.5	0.26502	-1.24382221260868\\
72.5	0.26868	-1.25070452554267\\
72.5	0.27234	-1.25653069000029\\
72.5	0.276	-1.26130070598154\\
72.875	0.093	0.298792768186456\\
72.875	0.09666	0.240349985248419\\
72.875	0.10032	0.182963350786753\\
72.875	0.10398	0.126632864801457\\
72.875	0.10764	0.0713585272925337\\
72.875	0.1113	0.0171403382599835\\
72.875	0.11496	-0.0360217022961985\\
72.875	0.11862	-0.0881275943760051\\
72.875	0.12228	-0.139177337979442\\
72.875	0.12594	-0.189170933106505\\
72.875	0.1296	-0.238108379757196\\
72.875	0.13326	-0.285989677931517\\
72.875	0.13692	-0.332814827629464\\
72.875	0.14058	-0.378583828851043\\
72.875	0.14424	-0.423296681596247\\
72.875	0.1479	-0.466953385865079\\
72.875	0.15156	-0.509553941657543\\
72.875	0.15522	-0.551098348973631\\
72.875	0.15888	-0.591586607813351\\
72.875	0.16254	-0.631018718176696\\
72.875	0.1662	-0.669394680063668\\
72.875	0.16986	-0.706714493474273\\
72.875	0.17352	-0.742978158408499\\
72.875	0.17718	-0.778185674866363\\
72.875	0.18084	-0.812337042847845\\
72.875	0.1845	-0.845432262352962\\
72.875	0.18816	-0.877471333381704\\
72.875	0.19182	-0.908454255934074\\
72.875	0.19548	-0.938381030010072\\
72.875	0.19914	-0.967251655609702\\
72.875	0.2028	-0.995066132732956\\
72.875	0.20646	-1.02182446137984\\
72.875	0.21012	-1.04752664155035\\
72.875	0.21378	-1.07217267324449\\
72.875	0.21744	-1.09576255646226\\
72.875	0.2211	-1.11829629120366\\
72.875	0.22476	-1.13977387746868\\
72.875	0.22842	-1.16019531525733\\
72.875	0.23208	-1.17956060456962\\
72.875	0.23574	-1.19786974540553\\
72.875	0.2394	-1.21512273776507\\
72.875	0.24306	-1.23131958164823\\
72.875	0.24672	-1.24646027705502\\
72.875	0.25038	-1.26054482398544\\
72.875	0.25404	-1.2735732224395\\
72.875	0.2577	-1.28554547241717\\
72.875	0.26136	-1.29646157391848\\
72.875	0.26502	-1.30632152694342\\
72.875	0.26868	-1.31512533149198\\
72.875	0.27234	-1.32287298756417\\
72.875	0.276	-1.32956449515999\\
73.25	0.093	0.327380932855736\\
73.25	0.09666	0.26701665830313\\
73.25	0.10032	0.207708532226892\\
73.25	0.10398	0.149456554627027\\
73.25	0.10764	0.0922607255035328\\
73.25	0.1113	0.0361210448564119\\
73.25	0.11496	-0.0189624873143389\\
73.25	0.11862	-0.0729898710087162\\
73.25	0.12228	-0.125961106226723\\
73.25	0.12594	-0.177876192968357\\
73.25	0.1296	-0.228735131233619\\
73.25	0.13326	-0.278537921022511\\
73.25	0.13692	-0.327284562335029\\
73.25	0.14058	-0.374975055171177\\
73.25	0.14424	-0.421609399530949\\
73.25	0.1479	-0.467187595414356\\
73.25	0.15156	-0.511709642821386\\
73.25	0.15522	-0.555175541752045\\
73.25	0.15888	-0.597585292206336\\
73.25	0.16254	-0.638938894184248\\
73.25	0.1662	-0.679236347685795\\
73.25	0.16986	-0.718477652710966\\
73.25	0.17352	-0.756662809259766\\
73.25	0.17718	-0.793791817332198\\
73.25	0.18084	-0.829864676928254\\
73.25	0.1845	-0.864881388047938\\
73.25	0.18816	-0.898841950691251\\
73.25	0.19182	-0.931746364858191\\
73.25	0.19548	-0.96359463054876\\
73.25	0.19914	-0.994386747762961\\
73.25	0.2028	-1.02412271650079\\
73.25	0.20646	-1.05280253676224\\
73.25	0.21012	-1.08042620854732\\
73.25	0.21378	-1.10699373185603\\
73.25	0.21744	-1.13250510668837\\
73.25	0.2211	-1.15696033304434\\
73.25	0.22476	-1.18035941092393\\
73.25	0.22842	-1.20270234032716\\
73.25	0.23208	-1.22398912125401\\
73.25	0.23574	-1.24421975370449\\
73.25	0.2394	-1.2633942376786\\
73.25	0.24306	-1.28151257317633\\
73.25	0.24672	-1.2985747601977\\
73.25	0.25038	-1.31458079874269\\
73.25	0.25404	-1.32953068881131\\
73.25	0.2577	-1.34342443040356\\
73.25	0.26136	-1.35626202351943\\
73.25	0.26502	-1.36804346815894\\
73.25	0.26868	-1.37876876432207\\
73.25	0.27234	-1.38843791200883\\
73.25	0.276	-1.39705091121922\\
73.625	0.093	0.35674647064423\\
73.625	0.09666	0.294460704477049\\
73.625	0.10032	0.233231086786244\\
73.625	0.10398	0.173057617571807\\
73.625	0.10764	0.113940296833742\\
73.625	0.1113	0.0558791245720517\\
73.625	0.11496	-0.00112589921326978\\
73.625	0.11862	-0.0570747745222178\\
73.625	0.12228	-0.111967501354796\\
73.625	0.12594	-0.165804079710998\\
73.625	0.1296	-0.218584509590831\\
73.625	0.13326	-0.270308790994293\\
73.625	0.13692	-0.32097692392138\\
73.625	0.14058	-0.3705889083721\\
73.625	0.14424	-0.419144744346445\\
73.625	0.1479	-0.466644431844419\\
73.625	0.15156	-0.51308797086602\\
73.625	0.15522	-0.55847536141125\\
73.625	0.15888	-0.602806603480111\\
73.625	0.16254	-0.646081697072597\\
73.625	0.1662	-0.688300642188708\\
73.625	0.16986	-0.729463438828454\\
73.625	0.17352	-0.769570086991824\\
73.625	0.17718	-0.808620586678823\\
73.625	0.18084	-0.84661493788945\\
73.625	0.1845	-0.883553140623705\\
73.625	0.18816	-0.919435194881591\\
73.625	0.19182	-0.954261100663099\\
73.625	0.19548	-0.988030857968239\\
73.625	0.19914	-1.02074446679701\\
73.625	0.2028	-1.05240192714941\\
73.625	0.20646	-1.08300323902543\\
73.625	0.21012	-1.11254840242508\\
73.625	0.21378	-1.14103741734836\\
73.625	0.21744	-1.16847028379527\\
73.625	0.2211	-1.19484700176581\\
73.625	0.22476	-1.22016757125997\\
73.625	0.22842	-1.24443199227777\\
73.625	0.23208	-1.26764026481919\\
73.625	0.23574	-1.28979238888424\\
73.625	0.2394	-1.31088836447292\\
73.625	0.24306	-1.33092819158522\\
73.625	0.24672	-1.34991187022116\\
73.625	0.25038	-1.36783940038072\\
73.625	0.25404	-1.38471078206391\\
73.625	0.2577	-1.40052601527073\\
73.625	0.26136	-1.41528510000117\\
73.625	0.26502	-1.42898803625525\\
73.625	0.26868	-1.44163482403296\\
73.625	0.27234	-1.45322546333429\\
73.625	0.276	-1.46375995415925\\
74	0.093	0.386889381551931\\
74	0.09666	0.322682123770181\\
74	0.10032	0.259531014464803\\
74	0.10398	0.197436053635797\\
74	0.10764	0.136397241283162\\
74	0.1113	0.0764145774069011\\
74	0.11496	0.0174880620070089\\
74	0.11862	-0.040382304916508\\
74	0.12228	-0.0971965233636565\\
74	0.12594	-0.15295459333443\\
74	0.1296	-0.207656514828833\\
74	0.13326	-0.261302287846864\\
74	0.13692	-0.313891912388524\\
74	0.14058	-0.365425388453815\\
74	0.14424	-0.415902716042727\\
74	0.1479	-0.465323895155271\\
74	0.15156	-0.513688925791443\\
74	0.15522	-0.560997807951247\\
74	0.15888	-0.607250541634672\\
74	0.16254	-0.652447126841732\\
74	0.1662	-0.696587563572413\\
74	0.16986	-0.73967185182673\\
74	0.17352	-0.781699991604667\\
74	0.17718	-0.82267198290624\\
74	0.18084	-0.862587825731434\\
74	0.1845	-0.901447520080264\\
74	0.18816	-0.939251065952717\\
74	0.19182	-0.975998463348799\\
74	0.19548	-1.01168971226851\\
74	0.19914	-1.04632481271185\\
74	0.2028	-1.07990376467881\\
74	0.20646	-1.11242656816941\\
74	0.21012	-1.14389322318363\\
74	0.21378	-1.17430372972148\\
74	0.21744	-1.20365808778296\\
74	0.2211	-1.23195629736807\\
74	0.22476	-1.2591983584768\\
74	0.22842	-1.28538427110917\\
74	0.23208	-1.31051403526516\\
74	0.23574	-1.33458765094478\\
74	0.2394	-1.35760511814803\\
74	0.24306	-1.3795664368749\\
74	0.24672	-1.40047160712541\\
74	0.25038	-1.42032062889954\\
74	0.25404	-1.4391135021973\\
74	0.2577	-1.45685022701869\\
74	0.26136	-1.47353080336371\\
74	0.26502	-1.48915523123236\\
74	0.26868	-1.50372351062463\\
74	0.27234	-1.51723564154053\\
74	0.276	-1.52969162398006\\
};
\end{axis}

\begin{axis}[%
width=4.527496cm,
height=3.870968cm,
at={(6.483547cm,0cm)},
scale only axis,
xmin=56,
xmax=74,
tick align=outside,
xlabel={$L_{cut}$},
xmajorgrids,
ymin=0.093,
ymax=0.276,
ylabel={$D_{rlx}$},
ymajorgrids,
zmin=0,
zmax=113.453205752381,
zlabel={$x_3,x_4$},
zmajorgrids,
view={-140}{50},
legend style={at={(1.03,1)},anchor=north west,legend cell align=left,align=left,draw=white!15!black}
]
\addplot3[only marks,mark=*,mark options={},mark size=1.5000pt,color=mycolor1] plot table[row sep=crcr,]{%
74	0.123	15.4680684467597\\
72	0.113	12.0981268654473\\
61	0.095	6.35969255839677\\
56	0.093	5.67589105085895\\
};
\addplot3[only marks,mark=*,mark options={},mark size=1.5000pt,color=mycolor2] plot table[row sep=crcr,]{%
67	0.276	110.131370256376\\
66	0.255	90.4271696447197\\
62	0.209	51.7720891508705\\
57	0.193	41.4264697981385\\
};
\addplot3[only marks,mark=*,mark options={},mark size=1.5000pt,color=black] plot table[row sep=crcr,]{%
69	0.104	9.34303234279614\\
};
\addplot3[only marks,mark=*,mark options={},mark size=1.5000pt,color=black] plot table[row sep=crcr,]{%
64	0.23	68.0446111694493\\
};

\addplot3[%
surf,
opacity=0.7,
shader=interp,
colormap={mymap}{[1pt] rgb(0pt)=(0.0901961,0.239216,0.0745098); rgb(1pt)=(0.0945149,0.242058,0.0739522); rgb(2pt)=(0.0988592,0.244894,0.0733566); rgb(3pt)=(0.103229,0.247724,0.0727241); rgb(4pt)=(0.107623,0.250549,0.0720557); rgb(5pt)=(0.112043,0.253367,0.0713525); rgb(6pt)=(0.116487,0.25618,0.0706154); rgb(7pt)=(0.120956,0.258986,0.0698456); rgb(8pt)=(0.125449,0.261787,0.0690441); rgb(9pt)=(0.129967,0.264581,0.0682118); rgb(10pt)=(0.134508,0.26737,0.06735); rgb(11pt)=(0.139074,0.270152,0.0664596); rgb(12pt)=(0.143663,0.272929,0.0655416); rgb(13pt)=(0.148275,0.275699,0.0645971); rgb(14pt)=(0.152911,0.278463,0.0636271); rgb(15pt)=(0.15757,0.281221,0.0626328); rgb(16pt)=(0.162252,0.283973,0.0616151); rgb(17pt)=(0.166957,0.286719,0.060575); rgb(18pt)=(0.171685,0.289458,0.0595136); rgb(19pt)=(0.176434,0.292191,0.0584321); rgb(20pt)=(0.181207,0.294918,0.0573313); rgb(21pt)=(0.186001,0.297639,0.0562123); rgb(22pt)=(0.190817,0.300353,0.0550763); rgb(23pt)=(0.195655,0.303061,0.0539242); rgb(24pt)=(0.200514,0.305763,0.052757); rgb(25pt)=(0.205395,0.308459,0.0515759); rgb(26pt)=(0.210296,0.311149,0.0503624); rgb(27pt)=(0.215212,0.313846,0.0490067); rgb(28pt)=(0.220142,0.316548,0.0475043); rgb(29pt)=(0.22509,0.319254,0.0458704); rgb(30pt)=(0.230056,0.321962,0.0441205); rgb(31pt)=(0.235042,0.324671,0.04227); rgb(32pt)=(0.240048,0.327379,0.0403343); rgb(33pt)=(0.245078,0.330085,0.0383287); rgb(34pt)=(0.250131,0.332786,0.0362688); rgb(35pt)=(0.25521,0.335482,0.0341698); rgb(36pt)=(0.260317,0.33817,0.0320472); rgb(37pt)=(0.265451,0.340849,0.0299163); rgb(38pt)=(0.270616,0.343517,0.0277927); rgb(39pt)=(0.275813,0.346172,0.0256916); rgb(40pt)=(0.281043,0.348814,0.0236284); rgb(41pt)=(0.286307,0.35144,0.0216186); rgb(42pt)=(0.291607,0.354048,0.0196776); rgb(43pt)=(0.296945,0.356637,0.0178207); rgb(44pt)=(0.302322,0.359206,0.0160634); rgb(45pt)=(0.307739,0.361753,0.0144211); rgb(46pt)=(0.313198,0.364275,0.0129091); rgb(47pt)=(0.318701,0.366772,0.0115428); rgb(48pt)=(0.324249,0.369242,0.0103377); rgb(49pt)=(0.329843,0.371682,0.00930909); rgb(50pt)=(0.335485,0.374093,0.00847245); rgb(51pt)=(0.341176,0.376471,0.00784314); rgb(52pt)=(0.346925,0.378826,0.00732741); rgb(53pt)=(0.352735,0.381168,0.00682184); rgb(54pt)=(0.358605,0.383497,0.00632729); rgb(55pt)=(0.364532,0.385812,0.00584464); rgb(56pt)=(0.370516,0.388113,0.00537476); rgb(57pt)=(0.376552,0.390399,0.00491852); rgb(58pt)=(0.38264,0.39267,0.00447681); rgb(59pt)=(0.388777,0.394925,0.00405048); rgb(60pt)=(0.394962,0.397164,0.00364042); rgb(61pt)=(0.401191,0.399386,0.00324749); rgb(62pt)=(0.407464,0.401592,0.00287258); rgb(63pt)=(0.413777,0.40378,0.00251655); rgb(64pt)=(0.420129,0.40595,0.00218028); rgb(65pt)=(0.426518,0.408102,0.00186463); rgb(66pt)=(0.432942,0.410234,0.00157049); rgb(67pt)=(0.439399,0.412348,0.00129873); rgb(68pt)=(0.445885,0.414441,0.00105022); rgb(69pt)=(0.452401,0.416515,0.000825833); rgb(70pt)=(0.458942,0.418567,0.000626441); rgb(71pt)=(0.465508,0.420599,0.00045292); rgb(72pt)=(0.472096,0.422609,0.000306141); rgb(73pt)=(0.478704,0.424596,0.000186979); rgb(74pt)=(0.485331,0.426562,9.63073e-05); rgb(75pt)=(0.491973,0.428504,3.49981e-05); rgb(76pt)=(0.498628,0.430422,3.92506e-06); rgb(77pt)=(0.505323,0.432315,0); rgb(78pt)=(0.512206,0.434168,0); rgb(79pt)=(0.519282,0.435983,0); rgb(80pt)=(0.526529,0.437764,0); rgb(81pt)=(0.533922,0.439512,0); rgb(82pt)=(0.54144,0.441232,0); rgb(83pt)=(0.549059,0.442927,0); rgb(84pt)=(0.556756,0.444599,0); rgb(85pt)=(0.564508,0.446252,0); rgb(86pt)=(0.572292,0.447889,0); rgb(87pt)=(0.580084,0.449514,0); rgb(88pt)=(0.587863,0.451129,0); rgb(89pt)=(0.595604,0.452737,0); rgb(90pt)=(0.603284,0.454343,0); rgb(91pt)=(0.610882,0.455948,0); rgb(92pt)=(0.618373,0.457556,0); rgb(93pt)=(0.625734,0.459171,0); rgb(94pt)=(0.632943,0.460795,0); rgb(95pt)=(0.639976,0.462432,0); rgb(96pt)=(0.64681,0.464084,0); rgb(97pt)=(0.653423,0.465756,0); rgb(98pt)=(0.659791,0.46745,0); rgb(99pt)=(0.665891,0.469169,0); rgb(100pt)=(0.6717,0.470916,0); rgb(101pt)=(0.677195,0.472696,0); rgb(102pt)=(0.682353,0.47451,0); rgb(103pt)=(0.687242,0.476355,0); rgb(104pt)=(0.691952,0.478225,0); rgb(105pt)=(0.696497,0.480118,0); rgb(106pt)=(0.700887,0.482033,0); rgb(107pt)=(0.705134,0.483968,0); rgb(108pt)=(0.709251,0.485921,0); rgb(109pt)=(0.713249,0.487891,0); rgb(110pt)=(0.71714,0.489876,0); rgb(111pt)=(0.720936,0.491875,0); rgb(112pt)=(0.724649,0.493887,0); rgb(113pt)=(0.72829,0.495909,0); rgb(114pt)=(0.731872,0.49794,0); rgb(115pt)=(0.735406,0.499979,0); rgb(116pt)=(0.738904,0.502025,0); rgb(117pt)=(0.742378,0.504075,0); rgb(118pt)=(0.74584,0.506128,0); rgb(119pt)=(0.749302,0.508182,0); rgb(120pt)=(0.752775,0.510237,0); rgb(121pt)=(0.756272,0.51229,0); rgb(122pt)=(0.759804,0.514339,0); rgb(123pt)=(0.763384,0.516385,0); rgb(124pt)=(0.767022,0.518424,0); rgb(125pt)=(0.770731,0.520455,0); rgb(126pt)=(0.774523,0.522478,0); rgb(127pt)=(0.77841,0.524489,0); rgb(128pt)=(0.782391,0.526491,0); rgb(129pt)=(0.786402,0.528496,0); rgb(130pt)=(0.790431,0.530506,0); rgb(131pt)=(0.794478,0.532521,0); rgb(132pt)=(0.798541,0.534539,0); rgb(133pt)=(0.802619,0.53656,0); rgb(134pt)=(0.806712,0.538584,0); rgb(135pt)=(0.81082,0.540609,0); rgb(136pt)=(0.81494,0.542635,0); rgb(137pt)=(0.819074,0.54466,0); rgb(138pt)=(0.823219,0.546686,0); rgb(139pt)=(0.827374,0.548709,0); rgb(140pt)=(0.831541,0.55073,0); rgb(141pt)=(0.835716,0.552749,0); rgb(142pt)=(0.8399,0.554763,0); rgb(143pt)=(0.844092,0.556774,0); rgb(144pt)=(0.848292,0.558779,0); rgb(145pt)=(0.852497,0.560778,0); rgb(146pt)=(0.856708,0.562771,0); rgb(147pt)=(0.860924,0.564756,0); rgb(148pt)=(0.865143,0.566733,0); rgb(149pt)=(0.869366,0.568701,0); rgb(150pt)=(0.873592,0.57066,0); rgb(151pt)=(0.877819,0.572608,0); rgb(152pt)=(0.882047,0.574545,0); rgb(153pt)=(0.886275,0.576471,0); rgb(154pt)=(0.890659,0.578362,0); rgb(155pt)=(0.895333,0.580203,0); rgb(156pt)=(0.900258,0.581999,0); rgb(157pt)=(0.905397,0.583755,0); rgb(158pt)=(0.910711,0.585479,0); rgb(159pt)=(0.916164,0.587176,0); rgb(160pt)=(0.921717,0.588852,0); rgb(161pt)=(0.927333,0.590513,0); rgb(162pt)=(0.932974,0.592166,0); rgb(163pt)=(0.938602,0.593815,0); rgb(164pt)=(0.94418,0.595468,0); rgb(165pt)=(0.949669,0.59713,0); rgb(166pt)=(0.955033,0.598808,0); rgb(167pt)=(0.960233,0.600507,0); rgb(168pt)=(0.965232,0.602233,0); rgb(169pt)=(0.969992,0.603992,0); rgb(170pt)=(0.974475,0.605791,0); rgb(171pt)=(0.978643,0.607636,0); rgb(172pt)=(0.98246,0.609532,0); rgb(173pt)=(0.985886,0.611486,0); rgb(174pt)=(0.988885,0.613503,0); rgb(175pt)=(0.991419,0.61559,0); rgb(176pt)=(0.99345,0.617753,0); rgb(177pt)=(0.99494,0.619997,0); rgb(178pt)=(0.995851,0.622329,0); rgb(179pt)=(0.996226,0.624763,0); rgb(180pt)=(0.996512,0.627352,0); rgb(181pt)=(0.996788,0.630095,0); rgb(182pt)=(0.997053,0.632982,0); rgb(183pt)=(0.997308,0.636004,0); rgb(184pt)=(0.997552,0.639152,0); rgb(185pt)=(0.997785,0.642416,0); rgb(186pt)=(0.998006,0.645786,0); rgb(187pt)=(0.998217,0.649253,0); rgb(188pt)=(0.998416,0.652807,0); rgb(189pt)=(0.998605,0.656439,0); rgb(190pt)=(0.998781,0.660138,0); rgb(191pt)=(0.998946,0.663897,0); rgb(192pt)=(0.9991,0.667704,0); rgb(193pt)=(0.999242,0.67155,0); rgb(194pt)=(0.999372,0.675427,0); rgb(195pt)=(0.99949,0.679323,0); rgb(196pt)=(0.999596,0.68323,0); rgb(197pt)=(0.99969,0.687139,0); rgb(198pt)=(0.999771,0.691039,0); rgb(199pt)=(0.999841,0.694921,0); rgb(200pt)=(0.999898,0.698775,0); rgb(201pt)=(0.999942,0.702592,0); rgb(202pt)=(0.999974,0.706363,0); rgb(203pt)=(0.999994,0.710077,0); rgb(204pt)=(1,0.713725,0); rgb(205pt)=(1,0.717341,0); rgb(206pt)=(1,0.720963,0); rgb(207pt)=(1,0.724591,0); rgb(208pt)=(1,0.728226,0); rgb(209pt)=(1,0.731867,0); rgb(210pt)=(1,0.735514,0); rgb(211pt)=(1,0.739167,0); rgb(212pt)=(1,0.742827,0); rgb(213pt)=(1,0.746493,0); rgb(214pt)=(1,0.750165,0); rgb(215pt)=(1,0.753843,0); rgb(216pt)=(1,0.757527,0); rgb(217pt)=(1,0.761217,0); rgb(218pt)=(1,0.764913,0); rgb(219pt)=(1,0.768615,0); rgb(220pt)=(1,0.772324,0); rgb(221pt)=(1,0.776038,0); rgb(222pt)=(1,0.779758,0); rgb(223pt)=(1,0.783484,0); rgb(224pt)=(1,0.787215,0); rgb(225pt)=(1,0.790953,0); rgb(226pt)=(1,0.794696,0); rgb(227pt)=(1,0.798445,0); rgb(228pt)=(1,0.8022,0); rgb(229pt)=(1,0.805961,0); rgb(230pt)=(1,0.809727,0); rgb(231pt)=(1,0.8135,0); rgb(232pt)=(1,0.817278,0); rgb(233pt)=(1,0.821063,0); rgb(234pt)=(1,0.824854,0); rgb(235pt)=(1,0.828652,0); rgb(236pt)=(1,0.832455,0); rgb(237pt)=(1,0.836265,0); rgb(238pt)=(1,0.840081,0); rgb(239pt)=(1,0.843903,0); rgb(240pt)=(1,0.847732,0); rgb(241pt)=(1,0.851566,0); rgb(242pt)=(1,0.855406,0); rgb(243pt)=(1,0.859253,0); rgb(244pt)=(1,0.863106,0); rgb(245pt)=(1,0.866964,0); rgb(246pt)=(1,0.870829,0); rgb(247pt)=(1,0.8747,0); rgb(248pt)=(1,0.878577,0); rgb(249pt)=(1,0.88246,0); rgb(250pt)=(1,0.886349,0); rgb(251pt)=(1,0.890243,0); rgb(252pt)=(1,0.894144,0); rgb(253pt)=(1,0.898051,0); rgb(254pt)=(1,0.901964,0); rgb(255pt)=(1,0.905882,0)},
mesh/rows=49]
table[row sep=crcr,header=false] {%
%
56	0.093	5.74613142666568\\
56	0.09666	6.17235695419101\\
56	0.10032	6.66497241887921\\
56	0.10398	7.22397782073029\\
56	0.10764	7.84937315974424\\
56	0.1113	8.54115843592107\\
56	0.11496	9.29933364926077\\
56	0.11862	10.1238987997633\\
56	0.12228	11.0148538874288\\
56	0.12594	11.9721989122571\\
56	0.1296	12.9959338742483\\
56	0.13326	14.0860587734024\\
56	0.13692	15.2425736097193\\
56	0.14058	16.4654783831992\\
56	0.14424	17.7547730938419\\
56	0.1479	19.1104577416474\\
56	0.15156	20.5325323266159\\
56	0.15522	22.0209968487472\\
56	0.15888	23.5758513080414\\
56	0.16254	25.1970957044985\\
56	0.1662	26.8847300381184\\
56	0.16986	28.6387543089012\\
56	0.17352	30.4591685168469\\
56	0.17718	32.3459726619555\\
56	0.18084	34.2991667442269\\
56	0.1845	36.3187507636612\\
56	0.18816	38.4047247202584\\
56	0.19182	40.5570886140185\\
56	0.19548	42.7758424449414\\
56	0.19914	45.0609862130273\\
56	0.2028	47.412519918276\\
56	0.20646	49.8304435606875\\
56	0.21012	52.314757140262\\
56	0.21378	54.8654606569993\\
56	0.21744	57.4825541108994\\
56	0.2211	60.1660375019625\\
56	0.22476	62.9159108301884\\
56	0.22842	65.7321740955772\\
56	0.23208	68.6148272981289\\
56	0.23574	71.5638704378435\\
56	0.2394	74.5793035147209\\
56	0.24306	77.6611265287612\\
56	0.24672	80.8093394799644\\
56	0.25038	84.0239423683305\\
56	0.25404	87.3049351938594\\
56	0.2577	90.6523179565513\\
56	0.26136	94.0660906564059\\
56	0.26502	97.5462532934235\\
56	0.26868	101.092805867604\\
56	0.27234	104.705748378947\\
56	0.276	108.385080827453\\
56.375	0.093	5.74291435535881\\
56.375	0.09666	6.16955702951079\\
56.375	0.10032	6.66258964082565\\
56.375	0.10398	7.22201218930338\\
56.375	0.10764	7.847824674944\\
56.375	0.1113	8.54002709774747\\
56.375	0.11496	9.29861945771383\\
56.375	0.11862	10.1236017548431\\
56.375	0.12228	11.0149739891352\\
56.375	0.12594	11.9727361605902\\
56.375	0.1296	12.996888269208\\
56.375	0.13326	14.0874303149887\\
56.375	0.13692	15.2443622979323\\
56.375	0.14058	16.4676842180388\\
56.375	0.14424	17.7573960753082\\
56.375	0.1479	19.1134978697404\\
56.375	0.15156	20.5359896013355\\
56.375	0.15522	22.0248712700935\\
56.375	0.15888	23.5801428760143\\
56.375	0.16254	25.201804419098\\
56.375	0.1662	26.8898558993446\\
56.375	0.16986	28.6442973167541\\
56.375	0.17352	30.4651286713265\\
56.375	0.17718	32.3523499630617\\
56.375	0.18084	34.3059611919598\\
56.375	0.1845	36.3259623580207\\
56.375	0.18816	38.4123534612446\\
56.375	0.19182	40.5651345016313\\
56.375	0.19548	42.7843054791809\\
56.375	0.19914	45.0698663938934\\
56.375	0.2028	47.4218172457687\\
56.375	0.20646	49.840158034807\\
56.375	0.21012	52.324888761008\\
56.375	0.21378	54.876009424372\\
56.375	0.21744	57.4935200248988\\
56.375	0.2211	60.1774205625886\\
56.375	0.22476	62.9277110374411\\
56.375	0.22842	65.7443914494566\\
56.375	0.23208	68.6274617986349\\
56.375	0.23574	71.5769220849762\\
56.375	0.2394	74.5927723084802\\
56.375	0.24306	77.6750124691472\\
56.375	0.24672	80.823642566977\\
56.375	0.25038	84.0386626019698\\
56.375	0.25404	87.3200725741253\\
56.375	0.2577	90.6678724834438\\
56.375	0.26136	94.0820623299252\\
56.375	0.26502	97.5626421135694\\
56.375	0.26868	101.109611834376\\
56.375	0.27234	104.722971492346\\
56.375	0.276	108.402721087479\\
56.75	0.093	5.74343965323961\\
56.75	0.09666	6.17049947401825\\
56.75	0.10032	6.66394923195977\\
56.75	0.10398	7.22378892706414\\
56.75	0.10764	7.85001855933142\\
56.75	0.1113	8.54263812876155\\
56.75	0.11496	9.30164763535456\\
56.75	0.11862	10.1270470791104\\
56.75	0.12228	11.0188364600292\\
56.75	0.12594	11.9770157781109\\
56.75	0.1296	13.0015850333553\\
56.75	0.13326	14.0925442257627\\
56.75	0.13692	15.249893355333\\
56.75	0.14058	16.4736324220661\\
56.75	0.14424	17.7637614259621\\
56.75	0.1479	19.120280367021\\
56.75	0.15156	20.5431892452428\\
56.75	0.15522	22.0324880606274\\
56.75	0.15888	23.5881768131749\\
56.75	0.16254	25.2102555028853\\
56.75	0.1662	26.8987241297585\\
56.75	0.16986	28.6535826937947\\
56.75	0.17352	30.4748311949937\\
56.75	0.17718	32.3624696333555\\
56.75	0.18084	34.3164980088803\\
56.75	0.1845	36.3369163215679\\
56.75	0.18816	38.4237245714184\\
56.75	0.19182	40.5769227584318\\
56.75	0.19548	42.796510882608\\
56.75	0.19914	45.0824889439472\\
56.75	0.2028	47.4348569424492\\
56.75	0.20646	49.8536148781141\\
56.75	0.21012	52.3387627509418\\
56.75	0.21378	54.8903005609324\\
56.75	0.21744	57.5082283080859\\
56.75	0.2211	60.1925459924023\\
56.75	0.22476	62.9432536138815\\
56.75	0.22842	65.7603511725236\\
56.75	0.23208	68.6438386683286\\
56.75	0.23574	71.5937161012965\\
56.75	0.2394	74.6099834714272\\
56.75	0.24306	77.6926407787209\\
56.75	0.24672	80.8416880231773\\
56.75	0.25038	84.0571252047967\\
56.75	0.25404	87.338952323579\\
56.75	0.2577	90.6871693795241\\
56.75	0.26136	94.1017763726321\\
56.75	0.26502	97.5827733029029\\
56.75	0.26868	101.130160170337\\
56.75	0.27234	104.743936974933\\
56.75	0.276	108.424103716693\\
57.125	0.093	5.74770732030808\\
57.125	0.09666	6.17518428771337\\
57.125	0.10032	6.66905119228155\\
57.125	0.10398	7.22930803401258\\
57.125	0.10764	7.85595481290651\\
57.125	0.1113	8.54899152896329\\
57.125	0.11496	9.30841818218295\\
57.125	0.11862	10.1342347725655\\
57.125	0.12228	11.0264413001109\\
57.125	0.12594	11.9850377648192\\
57.125	0.1296	13.0100241666904\\
57.125	0.13326	14.1014005057244\\
57.125	0.13692	15.2591667819213\\
57.125	0.14058	16.4833229952811\\
57.125	0.14424	17.7738691458038\\
57.125	0.1479	19.1308052334893\\
57.125	0.15156	20.5541312583377\\
57.125	0.15522	22.043847220349\\
57.125	0.15888	23.5999531195232\\
57.125	0.16254	25.2224489558602\\
57.125	0.1662	26.9113347293601\\
57.125	0.16986	28.6666104400229\\
57.125	0.17352	30.4882760878485\\
57.125	0.17718	32.376331672837\\
57.125	0.18084	34.3307771949885\\
57.125	0.1845	36.3516126543027\\
57.125	0.18816	38.4388380507799\\
57.125	0.19182	40.5924533844199\\
57.125	0.19548	42.8124586552228\\
57.125	0.19914	45.0988538631886\\
57.125	0.2028	47.4516390083173\\
57.125	0.20646	49.8708140906088\\
57.125	0.21012	52.3563791100632\\
57.125	0.21378	54.9083340666805\\
57.125	0.21744	57.5266789604606\\
57.125	0.2211	60.2114137914037\\
57.125	0.22476	62.9625385595095\\
57.125	0.22842	65.7800532647783\\
57.125	0.23208	68.6639579072099\\
57.125	0.23574	71.6142524868045\\
57.125	0.2394	74.6309370035619\\
57.125	0.24306	77.7140114574822\\
57.125	0.24672	80.8634758485653\\
57.125	0.25038	84.0793301768114\\
57.125	0.25404	87.3615744422202\\
57.125	0.2577	90.710208644792\\
57.125	0.26136	94.1252327845267\\
57.125	0.26502	97.6066468614242\\
57.125	0.26868	101.154450875485\\
57.125	0.27234	104.768644826708\\
57.125	0.276	108.449228715094\\
57.5	0.093	5.75571735656422\\
57.5	0.09666	6.18361147059617\\
57.5	0.10032	6.677895521791\\
57.5	0.10398	7.23856951014869\\
57.5	0.10764	7.86563343566927\\
57.5	0.1113	8.55908729835271\\
57.5	0.11496	9.31893109819903\\
57.5	0.11862	10.1451648352082\\
57.5	0.12228	11.0377885093803\\
57.5	0.12594	11.9968021207152\\
57.5	0.1296	13.022205669213\\
57.5	0.13326	14.1139991548737\\
57.5	0.13692	15.2721825776973\\
57.5	0.14058	16.4967559376837\\
57.5	0.14424	17.7877192348331\\
57.5	0.1479	19.1450724691453\\
57.5	0.15156	20.5688156406203\\
57.5	0.15522	22.0589487492583\\
57.5	0.15888	23.6154717950591\\
57.5	0.16254	25.2383847780228\\
57.5	0.1662	26.9276876981493\\
57.5	0.16986	28.6833805554388\\
57.5	0.17352	30.5054633498911\\
57.5	0.17718	32.3939360815063\\
57.5	0.18084	34.3487987502843\\
57.5	0.1845	36.3700513562252\\
57.5	0.18816	38.4576938993291\\
57.5	0.19182	40.6117263795957\\
57.5	0.19548	42.8321487970253\\
57.5	0.19914	45.1189611516177\\
57.5	0.2028	47.4721634433731\\
57.5	0.20646	49.8917556722913\\
57.5	0.21012	52.3777378383723\\
57.5	0.21378	54.9301099416162\\
57.5	0.21744	57.548871982023\\
57.5	0.2211	60.2340239595927\\
57.5	0.22476	62.9855658743252\\
57.5	0.22842	65.8034977262207\\
57.5	0.23208	68.687819515279\\
57.5	0.23574	71.6385312415002\\
57.5	0.2394	74.6556329048842\\
57.5	0.24306	77.7391245054311\\
57.5	0.24672	80.8890060431409\\
57.5	0.25038	84.1052775180137\\
57.5	0.25404	87.3879389300492\\
57.5	0.2577	90.7369902792476\\
57.5	0.26136	94.1524315656089\\
57.5	0.26502	97.6342627891331\\
57.5	0.26868	101.18248394982\\
57.5	0.27234	104.79709504767\\
57.5	0.276	108.478096082683\\
57.875	0.093	5.76746976200801\\
57.875	0.09666	6.19578102266661\\
57.875	0.10032	6.69048222048808\\
57.875	0.10398	7.25157335547244\\
57.875	0.10764	7.87905442761965\\
57.875	0.1113	8.57292543692976\\
57.875	0.11496	9.33318638340273\\
57.875	0.11862	10.1598372670386\\
57.875	0.12228	11.0528780878373\\
57.875	0.12594	12.0123088457989\\
57.875	0.1296	13.0381295409234\\
57.875	0.13326	14.1303401732107\\
57.875	0.13692	15.2889407426609\\
57.875	0.14058	16.5139312492741\\
57.875	0.14424	17.80531169305\\
57.875	0.1479	19.1630820739889\\
57.875	0.15156	20.5872423920906\\
57.875	0.15522	22.0777926473552\\
57.875	0.15888	23.6347328397826\\
57.875	0.16254	25.258062969373\\
57.875	0.1662	26.9477830361262\\
57.875	0.16986	28.7038930400423\\
57.875	0.17352	30.5263929811213\\
57.875	0.17718	32.4152828593631\\
57.875	0.18084	34.3705626747678\\
57.875	0.1845	36.3922324273354\\
57.875	0.18816	38.4802921170659\\
57.875	0.19182	40.6347417439592\\
57.875	0.19548	42.8555813080154\\
57.875	0.19914	45.1428108092345\\
57.875	0.2028	47.4964302476165\\
57.875	0.20646	49.9164396231613\\
57.875	0.21012	52.402838935869\\
57.875	0.21378	54.9556281857396\\
57.875	0.21744	57.5748073727731\\
57.875	0.2211	60.2603764969694\\
57.875	0.22476	63.0123355583286\\
57.875	0.22842	65.8306845568507\\
57.875	0.23208	68.7154234925356\\
57.875	0.23574	71.6665523653835\\
57.875	0.2394	74.6840711753942\\
57.875	0.24306	77.7679799225678\\
57.875	0.24672	80.9182786069042\\
57.875	0.25038	84.1349672284036\\
57.875	0.25404	87.4180457870658\\
57.875	0.2577	90.7675142828909\\
57.875	0.26136	94.1833727158788\\
57.875	0.26502	97.6656210860296\\
57.875	0.26868	101.214259393343\\
57.875	0.27234	104.82928763782\\
57.875	0.276	108.510705819459\\
58.25	0.093	5.78296453663947\\
58.25	0.09666	6.21169294392473\\
58.25	0.10032	6.70681128837286\\
58.25	0.10398	7.26831956998387\\
58.25	0.10764	7.89621778875775\\
58.25	0.1113	8.59050594469451\\
58.25	0.11496	9.35118403779413\\
58.25	0.11862	10.1782520680566\\
58.25	0.12228	11.071710035482\\
58.25	0.12594	12.0315579400703\\
58.25	0.1296	13.0577957818214\\
58.25	0.13326	14.1504235607354\\
58.25	0.13692	15.3094412768123\\
58.25	0.14058	16.534848930052\\
58.25	0.14424	17.8266465204546\\
58.25	0.1479	19.1848340480201\\
58.25	0.15156	20.6094115127485\\
58.25	0.15522	22.1003789146398\\
58.25	0.15888	23.6577362536939\\
58.25	0.16254	25.2814835299109\\
58.25	0.1662	26.9716207432908\\
58.25	0.16986	28.7281478938335\\
58.25	0.17352	30.5510649815391\\
58.25	0.17718	32.4403720064076\\
58.25	0.18084	34.396068968439\\
58.25	0.1845	36.4181558676332\\
58.25	0.18816	38.5066327039904\\
58.25	0.19182	40.6614994775103\\
58.25	0.19548	42.8827561881932\\
58.25	0.19914	45.170402836039\\
58.25	0.2028	47.5244394210476\\
58.25	0.20646	49.9448659432191\\
58.25	0.21012	52.4316824025534\\
58.25	0.21378	54.9848887990507\\
58.25	0.21744	57.6044851327108\\
58.25	0.2211	60.2904714035338\\
58.25	0.22476	63.0428476115196\\
58.25	0.22842	65.8616137566684\\
58.25	0.23208	68.74676983898\\
58.25	0.23574	71.6983158584545\\
58.25	0.2394	74.7162518150918\\
58.25	0.24306	77.8005777088921\\
58.25	0.24672	80.9512935398552\\
58.25	0.25038	84.1683993079812\\
58.25	0.25404	87.45189501327\\
58.25	0.2577	90.8017806557218\\
58.25	0.26136	94.2180562353364\\
58.25	0.26502	97.7007217521139\\
58.25	0.26868	101.249777206054\\
58.25	0.27234	104.865222597157\\
58.25	0.276	108.547057925424\\
58.625	0.093	5.80220168045861\\
58.625	0.09666	6.23134723437052\\
58.625	0.10032	6.7268827254453\\
58.625	0.10398	7.28880815368297\\
58.625	0.10764	7.9171235190835\\
58.625	0.1113	8.61182882164692\\
58.625	0.11496	9.37292406137319\\
58.625	0.11862	10.2004092382623\\
58.625	0.12228	11.0942843523144\\
58.625	0.12594	12.0545494035293\\
58.625	0.1296	13.0812043919071\\
58.625	0.13326	14.1742493174477\\
58.625	0.13692	15.3336841801513\\
58.625	0.14058	16.5595089800177\\
58.625	0.14424	17.851723717047\\
58.625	0.1479	19.2103283912391\\
58.625	0.15156	20.6353230025941\\
58.625	0.15522	22.126707551112\\
58.625	0.15888	23.6844820367928\\
58.625	0.16254	25.3086464596365\\
58.625	0.1662	26.999200819643\\
58.625	0.16986	28.7561451168124\\
58.625	0.17352	30.5794793511447\\
58.625	0.17718	32.4692035226398\\
58.625	0.18084	34.4253176312978\\
58.625	0.1845	36.4478216771187\\
58.625	0.18816	38.5367156601025\\
58.625	0.19182	40.6919995802492\\
58.625	0.19548	42.9136734375587\\
58.625	0.19914	45.2017372320311\\
58.625	0.2028	47.5561909636664\\
58.625	0.20646	49.9770346324645\\
58.625	0.21012	52.4642682384255\\
58.625	0.21378	55.0178917815494\\
58.625	0.21744	57.6379052618362\\
58.625	0.2211	60.3243086792858\\
58.625	0.22476	63.0771020338983\\
58.625	0.22842	65.8962853256737\\
58.625	0.23208	68.781858554612\\
58.625	0.23574	71.7338217207131\\
58.625	0.2394	74.7521748239772\\
58.625	0.24306	77.8369178644041\\
58.625	0.24672	80.9880508419938\\
58.625	0.25038	84.2055737567465\\
58.625	0.25404	87.489486608662\\
58.625	0.2577	90.8397893977404\\
58.625	0.26136	94.2564821239816\\
58.625	0.26502	97.7395647873858\\
58.625	0.26868	101.289037387953\\
58.625	0.27234	104.904899925683\\
58.625	0.276	108.587152400575\\
59	0.093	5.8251811934654\\
59	0.09666	6.25474389400397\\
59	0.10032	6.7506965317054\\
59	0.10398	7.31303910656973\\
59	0.10764	7.94177161859691\\
59	0.1113	8.63689406778698\\
59	0.11496	9.39840645413992\\
59	0.11862	10.2263087776557\\
59	0.12228	11.1206010383344\\
59	0.12594	12.081283236176\\
59	0.1296	13.1083553711804\\
59	0.13326	14.2018174433477\\
59	0.13692	15.3616694526779\\
59	0.14058	16.587911399171\\
59	0.14424	17.8805432828269\\
59	0.1479	19.2395651036457\\
59	0.15156	20.6649768616274\\
59	0.15522	22.1567785567719\\
59	0.15888	23.7149701890794\\
59	0.16254	25.3395517585497\\
59	0.1662	27.0305232651829\\
59	0.16986	28.7878847089789\\
59	0.17352	30.6116360899379\\
59	0.17718	32.5017774080597\\
59	0.18084	34.4583086633443\\
59	0.1845	36.4812298557919\\
59	0.18816	38.5705409854023\\
59	0.19182	40.7262420521756\\
59	0.19548	42.9483330561118\\
59	0.19914	45.2368139972108\\
59	0.2028	47.5916848754728\\
59	0.20646	50.0129456908976\\
59	0.21012	52.5005964434853\\
59	0.21378	55.0546371332358\\
59	0.21744	57.6750677601492\\
59	0.2211	60.3618883242255\\
59	0.22476	63.1150988254647\\
59	0.22842	65.9346992638667\\
59	0.23208	68.8206896394316\\
59	0.23574	71.7730699521595\\
59	0.2394	74.7918402020501\\
59	0.24306	77.8770003891037\\
59	0.24672	81.0285505133201\\
59	0.25038	84.2464905746994\\
59	0.25404	87.5308205732416\\
59	0.2577	90.8815405089466\\
59	0.26136	94.2986503818145\\
59	0.26502	97.7821501918453\\
59	0.26868	101.332039939039\\
59	0.27234	104.948319623396\\
59	0.276	108.630989244915\\
59.375	0.093	5.85190307565984\\
59.375	0.09666	6.28188292282507\\
59.375	0.10032	6.77825270715317\\
59.375	0.10398	7.34101242864412\\
59.375	0.10764	7.97016208729798\\
59.375	0.1113	8.66570168311469\\
59.375	0.11496	9.42763121609428\\
59.375	0.11862	10.2559506862368\\
59.375	0.12228	11.1506600935421\\
59.375	0.12594	12.1117594380103\\
59.375	0.1296	13.1392487196414\\
59.375	0.13326	14.2331279384354\\
59.375	0.13692	15.3933970943922\\
59.375	0.14058	16.6200561875119\\
59.375	0.14424	17.9131052177945\\
59.375	0.1479	19.27254418524\\
59.375	0.15156	20.6983730898483\\
59.375	0.15522	22.1905919316195\\
59.375	0.15888	23.7492007105536\\
59.375	0.16254	25.3741994266506\\
59.375	0.1662	27.0655880799104\\
59.375	0.16986	28.8233666703331\\
59.375	0.17352	30.6475351979187\\
59.375	0.17718	32.5380936626672\\
59.375	0.18084	34.4950420645785\\
59.375	0.1845	36.5183804036527\\
59.375	0.18816	38.6081086798898\\
59.375	0.19182	40.7642268932897\\
59.375	0.19548	42.9867350438526\\
59.375	0.19914	45.2756331315783\\
59.375	0.2028	47.6309211564669\\
59.375	0.20646	50.0525991185183\\
59.375	0.21012	52.5406670177326\\
59.375	0.21378	55.0951248541098\\
59.375	0.21744	57.7159726276499\\
59.375	0.2211	60.4032103383529\\
59.375	0.22476	63.1568379862187\\
59.375	0.22842	65.9768555712474\\
59.375	0.23208	68.863263093439\\
59.375	0.23574	71.8160605527934\\
59.375	0.2394	74.8352479493108\\
59.375	0.24306	77.920825282991\\
59.375	0.24672	81.072792553834\\
59.375	0.25038	84.29114976184\\
59.375	0.25404	87.5758969070088\\
59.375	0.2577	90.9270339893405\\
59.375	0.26136	94.3445610088351\\
59.375	0.26502	97.8284779654925\\
59.375	0.26868	101.378784859313\\
59.375	0.27234	104.995481690296\\
59.375	0.276	108.678568458442\\
59.75	0.093	5.88236732704198\\
59.75	0.09666	6.31276432083386\\
59.75	0.10032	6.80955125178861\\
59.75	0.10398	7.37272811990622\\
59.75	0.10764	8.00229492518674\\
59.75	0.1113	8.6982516676301\\
59.75	0.11496	9.46059834723635\\
59.75	0.11862	10.2893349640055\\
59.75	0.12228	11.1844615179375\\
59.75	0.12594	12.1459780090323\\
59.75	0.1296	13.1738844372901\\
59.75	0.13326	14.2681808027107\\
59.75	0.13692	15.4288671052942\\
59.75	0.14058	16.6559433450406\\
59.75	0.14424	17.9494095219498\\
59.75	0.1479	19.3092656360219\\
59.75	0.15156	20.7355116872569\\
59.75	0.15522	22.2281476756548\\
59.75	0.15888	23.7871736012155\\
59.75	0.16254	25.4125894639391\\
59.75	0.1662	27.1043952638256\\
59.75	0.16986	28.862591000875\\
59.75	0.17352	30.6871766750873\\
59.75	0.17718	32.5781522864623\\
59.75	0.18084	34.5355178350003\\
59.75	0.1845	36.5592733207012\\
59.75	0.18816	38.6494187435649\\
59.75	0.19182	40.8059541035915\\
59.75	0.19548	43.028879400781\\
59.75	0.19914	45.3181946351334\\
59.75	0.2028	47.6738998066486\\
59.75	0.20646	50.0959949153268\\
59.75	0.21012	52.5844799611677\\
59.75	0.21378	55.1393549441716\\
59.75	0.21744	57.7606198643383\\
59.75	0.2211	60.4482747216679\\
59.75	0.22476	63.2023195161604\\
59.75	0.22842	66.0227542478158\\
59.75	0.23208	68.909578916634\\
59.75	0.23574	71.8627935226151\\
59.75	0.2394	74.8823980657591\\
59.75	0.24306	77.9683925460659\\
59.75	0.24672	81.1207769635357\\
59.75	0.25038	84.3395513181683\\
59.75	0.25404	87.6247156099637\\
59.75	0.2577	90.9762698389221\\
59.75	0.26136	94.3942140050434\\
59.75	0.26502	97.8785481083274\\
59.75	0.26868	101.429272148774\\
59.75	0.27234	105.046386126384\\
59.75	0.276	108.729890041157\\
60.125	0.093	5.91657394761176\\
60.125	0.09666	6.3473880880303\\
60.125	0.10032	6.8445921656117\\
60.125	0.10398	7.40818618035596\\
60.125	0.10764	8.03817013226314\\
60.125	0.1113	8.73454402133316\\
60.125	0.11496	9.49730784756606\\
60.125	0.11862	10.3264616109618\\
60.125	0.12228	11.2220053115205\\
60.125	0.12594	12.183938949242\\
60.125	0.1296	13.2122625241264\\
60.125	0.13326	14.3069760361737\\
60.125	0.13692	15.4680794853838\\
60.125	0.14058	16.6955728717569\\
60.125	0.14424	17.9894561952928\\
60.125	0.1479	19.3497294559915\\
60.125	0.15156	20.7763926538532\\
60.125	0.15522	22.2694457888777\\
60.125	0.15888	23.8288888610651\\
60.125	0.16254	25.4547218704154\\
60.125	0.1662	27.1469448169285\\
60.125	0.16986	28.9055577006045\\
60.125	0.17352	30.7305605214434\\
60.125	0.17718	32.6219532794452\\
60.125	0.18084	34.5797359746098\\
60.125	0.1845	36.6039086069373\\
60.125	0.18816	38.6944711764277\\
60.125	0.19182	40.851423683081\\
60.125	0.19548	43.0747661268971\\
60.125	0.19914	45.3644985078762\\
60.125	0.2028	47.7206208260181\\
60.125	0.20646	50.1431330813228\\
60.125	0.21012	52.6320352737905\\
60.125	0.21378	55.187327403421\\
60.125	0.21744	57.8090094702144\\
60.125	0.2211	60.4970814741706\\
60.125	0.22476	63.2515434152898\\
60.125	0.22842	66.0723952935718\\
60.125	0.23208	68.9596371090166\\
60.125	0.23574	71.9132688616244\\
60.125	0.2394	74.9332905513951\\
60.125	0.24306	78.0197021783286\\
60.125	0.24672	81.1725037424249\\
60.125	0.25038	84.3916952436842\\
60.125	0.25404	87.6772766821063\\
60.125	0.2577	91.0292480576914\\
60.125	0.26136	94.4476093704392\\
60.125	0.26502	97.93236062035\\
60.125	0.26868	101.483501807424\\
60.125	0.27234	105.10103293166\\
60.125	0.276	108.78495399306\\
60.5	0.093	5.95452293736921\\
60.5	0.09666	6.3857542244144\\
60.5	0.10032	6.88337544862247\\
60.5	0.10398	7.44738660999339\\
60.5	0.10764	8.07778770852721\\
60.5	0.1113	8.77457874422389\\
60.5	0.11496	9.53775971708345\\
60.5	0.11862	10.3673306271059\\
60.5	0.12228	11.2632914742912\\
60.5	0.12594	12.2256422586394\\
60.5	0.1296	13.2543829801504\\
60.5	0.13326	14.3495136388244\\
60.5	0.13692	15.5110342346611\\
60.5	0.14058	16.7389447676608\\
60.5	0.14424	18.0332452378234\\
60.5	0.1479	19.3939356451488\\
60.5	0.15156	20.8210159896371\\
60.5	0.15522	22.3144862712883\\
60.5	0.15888	23.8743464901023\\
60.5	0.16254	25.5005966460793\\
60.5	0.1662	27.1932367392191\\
60.5	0.16986	28.9522667695217\\
60.5	0.17352	30.7776867369873\\
60.5	0.17718	32.6694966416157\\
60.5	0.18084	34.627696483407\\
60.5	0.1845	36.6522862623612\\
60.5	0.18816	38.7432659784782\\
60.5	0.19182	40.9006356317581\\
60.5	0.19548	43.1243952222009\\
60.5	0.19914	45.4145447498066\\
60.5	0.2028	47.7710842145752\\
60.5	0.20646	50.1940136165066\\
60.5	0.21012	52.6833329556009\\
60.5	0.21378	55.239042231858\\
60.5	0.21744	57.8611414452781\\
60.5	0.2211	60.549630595861\\
60.5	0.22476	63.3045096836068\\
60.5	0.22842	66.1257787085154\\
60.5	0.23208	69.013437670587\\
60.5	0.23574	71.9674865698214\\
60.5	0.2394	74.9879254062187\\
60.5	0.24306	78.0747541797789\\
60.5	0.24672	81.2279728905019\\
60.5	0.25038	84.4475815383878\\
60.5	0.25404	87.7335801234366\\
60.5	0.2577	91.0859686456483\\
60.5	0.26136	94.5047471050228\\
60.5	0.26502	97.9899155015602\\
60.5	0.26868	101.54147383526\\
60.5	0.27234	105.159422106124\\
60.5	0.276	108.84376031415\\
60.875	0.093	5.99621429631434\\
60.875	0.09666	6.42786272998619\\
60.875	0.10032	6.92590110082091\\
60.875	0.10398	7.49032940881848\\
60.875	0.10764	8.12114765397896\\
60.875	0.1113	8.81835583630229\\
60.875	0.11496	9.58195395578849\\
60.875	0.11862	10.4119420124376\\
60.875	0.12228	11.3083200062495\\
60.875	0.12594	12.2710879372244\\
60.875	0.1296	13.3002458053621\\
60.875	0.13326	14.3957936106627\\
60.875	0.13692	15.5577313531261\\
60.875	0.14058	16.7860590327524\\
60.875	0.14424	18.0807766495417\\
60.875	0.1479	19.4418842034937\\
60.875	0.15156	20.8693816946087\\
60.875	0.15522	22.3632691228865\\
60.875	0.15888	23.9235464883272\\
60.875	0.16254	25.5502137909308\\
60.875	0.1662	27.2432710306973\\
60.875	0.16986	29.0027182076266\\
60.875	0.17352	30.8285553217188\\
60.875	0.17718	32.7207823729739\\
60.875	0.18084	34.6793993613918\\
60.875	0.1845	36.7044062869727\\
60.875	0.18816	38.7958031497164\\
60.875	0.19182	40.9535899496229\\
60.875	0.19548	43.1777666866924\\
60.875	0.19914	45.4683333609247\\
60.875	0.2028	47.8252899723199\\
60.875	0.20646	50.248636520878\\
60.875	0.21012	52.7383730065989\\
60.875	0.21378	55.2944994294828\\
60.875	0.21744	57.9170157895294\\
60.875	0.2211	60.605922086739\\
60.875	0.22476	63.3612183211115\\
60.875	0.22842	66.1829044926468\\
60.875	0.23208	69.070980601345\\
60.875	0.23574	72.0254466472061\\
60.875	0.2394	75.04630263023\\
60.875	0.24306	78.1335485504168\\
60.875	0.24672	81.2871844077665\\
60.875	0.25038	84.5072102022791\\
60.875	0.25404	87.7936259339545\\
60.875	0.2577	91.1464316027929\\
60.875	0.26136	94.565627208794\\
60.875	0.26502	98.0512127519581\\
60.875	0.26868	101.603188232285\\
60.875	0.27234	105.221553649775\\
60.875	0.276	108.906309004428\\
61.25	0.093	6.04164802444712\\
61.25	0.09666	6.47371360474562\\
61.25	0.10032	6.97216912220697\\
61.25	0.10398	7.53701457683123\\
61.25	0.10764	8.16824996861833\\
61.25	0.1113	8.86587529756834\\
61.25	0.11496	9.6298905636812\\
61.25	0.11862	10.4602957669569\\
61.25	0.12228	11.3570909073956\\
61.25	0.12594	12.320275984997\\
61.25	0.1296	13.3498509997614\\
61.25	0.13326	14.4458159516886\\
61.25	0.13692	15.6081708407787\\
61.25	0.14058	16.8369156670317\\
61.25	0.14424	18.1320504304476\\
61.25	0.1479	19.4935751310263\\
61.25	0.15156	20.9214897687679\\
61.25	0.15522	22.4157943436724\\
61.25	0.15888	23.9764888557398\\
61.25	0.16254	25.60357330497\\
61.25	0.1662	27.2970476913631\\
61.25	0.16986	29.0569120149191\\
61.25	0.17352	30.883166275638\\
61.25	0.17718	32.7758104735197\\
61.25	0.18084	34.7348446085643\\
61.25	0.1845	36.7602686807718\\
61.25	0.18816	38.8520826901422\\
61.25	0.19182	41.0102866366754\\
61.25	0.19548	43.2348805203715\\
61.25	0.19914	45.5258643412305\\
61.25	0.2028	47.8832380992523\\
61.25	0.20646	50.3070017944371\\
61.25	0.21012	52.7971554267847\\
61.25	0.21378	55.3536989962951\\
61.25	0.21744	57.9766325029685\\
61.25	0.2211	60.6659559468047\\
61.25	0.22476	63.4216693278038\\
61.25	0.22842	66.2437726459658\\
61.25	0.23208	69.1322659012906\\
61.25	0.23574	72.0871490937784\\
61.25	0.2394	75.108422223429\\
61.25	0.24306	78.1960852902424\\
61.25	0.24672	81.3501382942188\\
61.25	0.25038	84.570581235358\\
61.25	0.25404	87.8574141136601\\
61.25	0.2577	91.2106369291251\\
61.25	0.26136	94.6302496817529\\
61.25	0.26502	98.1162523715436\\
61.25	0.26868	101.668644998497\\
61.25	0.27234	105.287427562614\\
61.25	0.276	108.972600063893\\
61.625	0.093	6.09082412176756\\
61.625	0.09666	6.52330684869272\\
61.625	0.10032	7.02217951278073\\
61.625	0.10398	7.58744211403165\\
61.625	0.10764	8.21909465244541\\
61.625	0.1113	8.91713712802206\\
61.625	0.11496	9.68156954076157\\
61.625	0.11862	10.512391890664\\
61.625	0.12228	11.4096041777292\\
61.625	0.12594	12.3732064019574\\
61.625	0.1296	13.4031985633484\\
61.625	0.13326	14.4995806619023\\
61.625	0.13692	15.662352697619\\
61.625	0.14058	16.8915146704987\\
61.625	0.14424	18.1870665805412\\
61.625	0.1479	19.5490084277466\\
61.625	0.15156	20.9773402121149\\
61.625	0.15522	22.472061933646\\
61.625	0.15888	24.03317359234\\
61.625	0.16254	25.6606751881969\\
61.625	0.1662	27.3545667212167\\
61.625	0.16986	29.1148481913993\\
61.625	0.17352	30.9415195987448\\
61.625	0.17718	32.8345809432532\\
61.625	0.18084	34.7940322249245\\
61.625	0.1845	36.8198734437586\\
61.625	0.18816	38.9121045997556\\
61.625	0.19182	41.0707256929155\\
61.625	0.19548	43.2957367232383\\
61.625	0.19914	45.5871376907239\\
61.625	0.2028	47.9449285953724\\
61.625	0.20646	50.3691094371838\\
61.625	0.21012	52.8596802161581\\
61.625	0.21378	55.4166409322952\\
61.625	0.21744	58.0399915855952\\
61.625	0.2211	60.7297321760581\\
61.625	0.22476	63.4858627036838\\
61.625	0.22842	66.3083831684724\\
61.625	0.23208	69.1972935704239\\
61.625	0.23574	72.1525939095383\\
61.625	0.2394	75.1742841858156\\
61.625	0.24306	78.2623643992557\\
61.625	0.24672	81.4168345498587\\
61.625	0.25038	84.6376946376246\\
61.625	0.25404	87.9249446625533\\
61.625	0.2577	91.278584624645\\
61.625	0.26136	94.6986145238995\\
61.625	0.26502	98.1850343603169\\
61.625	0.26868	101.737844133897\\
61.625	0.27234	105.35704384464\\
61.625	0.276	109.042633492546\\
62	0.093	6.14374258827568\\
62	0.09666	6.57664246182749\\
62	0.10032	7.07593227254216\\
62	0.10398	7.64161202041971\\
62	0.10764	8.27368170546014\\
62	0.1113	8.97214132766344\\
62	0.11496	9.73699088702962\\
62	0.11862	10.5682303835586\\
62	0.12228	11.4658598172506\\
62	0.12594	12.4298791881054\\
62	0.1296	13.4602884961231\\
62	0.13326	14.5570877413036\\
62	0.13692	15.720276923647\\
62	0.14058	16.9498560431533\\
62	0.14424	18.2458250998225\\
62	0.1479	19.6081840936545\\
62	0.15156	21.0369330246495\\
62	0.15522	22.5320718928073\\
62	0.15888	24.0936006981279\\
62	0.16254	25.7215194406115\\
62	0.1662	27.4158281202579\\
62	0.16986	29.1765267370672\\
62	0.17352	31.0036152910394\\
62	0.17718	32.8970937821744\\
62	0.18084	34.8569622104723\\
62	0.1845	36.8832205759331\\
62	0.18816	38.9758688785568\\
62	0.19182	41.1349071183433\\
62	0.19548	43.3603352952927\\
62	0.19914	45.652153409405\\
62	0.2028	48.0103614606802\\
62	0.20646	50.4349594491182\\
62	0.21012	52.9259473747191\\
62	0.21378	55.4833252374829\\
62	0.21744	58.1070930374096\\
62	0.2211	60.7972507744991\\
62	0.22476	63.5537984487515\\
62	0.22842	66.3767360601668\\
62	0.23208	69.2660636087449\\
62	0.23574	72.221781094486\\
62	0.2394	75.2438885173899\\
62	0.24306	78.3323858774567\\
62	0.24672	81.4872731746863\\
62	0.25038	84.7085504090789\\
62	0.25404	87.9962175806343\\
62	0.2577	91.3502746893526\\
62	0.26136	94.7707217352337\\
62	0.26502	98.2575587182777\\
62	0.26868	101.810785638485\\
62	0.27234	105.430402495854\\
62	0.276	109.116409290387\\
62.375	0.093	6.20040342397147\\
62.375	0.09666	6.63372044414993\\
62.375	0.10032	7.13342740149125\\
62.375	0.10398	7.69952429599547\\
62.375	0.10764	8.33201112766254\\
62.375	0.1113	9.03088789649249\\
62.375	0.11496	9.79615460248532\\
62.375	0.11862	10.627811245641\\
62.375	0.12228	11.5258578259596\\
62.375	0.12594	12.4902943434411\\
62.375	0.1296	13.5211207980854\\
62.375	0.13326	14.6183371898926\\
62.375	0.13692	15.7819435188627\\
62.375	0.14058	17.0119397849956\\
62.375	0.14424	18.3083259882914\\
62.375	0.1479	19.6711021287501\\
62.375	0.15156	21.1002682063717\\
62.375	0.15522	22.5958242211562\\
62.375	0.15888	24.1577701731035\\
62.375	0.16254	25.7861060622137\\
62.375	0.1662	27.4808318884868\\
62.375	0.16986	29.2419476519227\\
62.375	0.17352	31.0694533525215\\
62.375	0.17718	32.9633489902832\\
62.375	0.18084	34.9236345652078\\
62.375	0.1845	36.9503100772952\\
62.375	0.18816	39.0433755265456\\
62.375	0.19182	41.2028309129588\\
62.375	0.19548	43.4286762365348\\
62.375	0.19914	45.7209114972738\\
62.375	0.2028	48.0795366951756\\
62.375	0.20646	50.5045518302403\\
62.375	0.21012	52.9959569024678\\
62.375	0.21378	55.5537519118583\\
62.375	0.21744	58.1779368584116\\
62.375	0.2211	60.8685117421278\\
62.375	0.22476	63.6254765630068\\
62.375	0.22842	66.4488313210488\\
62.375	0.23208	69.3385760162536\\
62.375	0.23574	72.2947106486213\\
62.375	0.2394	75.3172352181519\\
62.375	0.24306	78.4061497248453\\
62.375	0.24672	81.5614541687016\\
62.375	0.25038	84.7831485497208\\
62.375	0.25404	88.0712328679028\\
62.375	0.2577	91.4257071232478\\
62.375	0.26136	94.8465713157556\\
62.375	0.26502	98.3338254454263\\
62.375	0.26868	101.88746951226\\
62.375	0.27234	105.507503516256\\
62.375	0.276	109.193927457416\\
62.75	0.093	6.26080662885489\\
62.75	0.09666	6.69454079566001\\
62.75	0.10032	7.19466489962801\\
62.75	0.10398	7.76117894075885\\
62.75	0.10764	8.3940829190526\\
62.75	0.1113	9.0933768345092\\
62.75	0.11496	9.85906068712869\\
62.75	0.11862	10.6911344769111\\
62.75	0.12228	11.5895982038563\\
62.75	0.12594	12.5544518679644\\
62.75	0.1296	13.5856954692354\\
62.75	0.13326	14.6833290076692\\
62.75	0.13692	15.8473524832659\\
62.75	0.14058	17.0777658960256\\
62.75	0.14424	18.374569245948\\
62.75	0.1479	19.7377625330334\\
62.75	0.15156	21.1673457572816\\
62.75	0.15522	22.6633189186927\\
62.75	0.15888	24.2256820172667\\
62.75	0.16254	25.8544350530036\\
62.75	0.1662	27.5495780259033\\
62.75	0.16986	29.3111109359659\\
62.75	0.17352	31.1390337831914\\
62.75	0.17718	33.0333465675797\\
62.75	0.18084	34.9940492891309\\
62.75	0.1845	37.021141947845\\
62.75	0.18816	39.114624543722\\
62.75	0.19182	41.2744970767618\\
62.75	0.19548	43.5007595469646\\
62.75	0.19914	45.7934119543302\\
62.75	0.2028	48.1524542988587\\
62.75	0.20646	50.57788658055\\
62.75	0.21012	53.0697087994042\\
62.75	0.21378	55.6279209554213\\
62.75	0.21744	58.2525230486013\\
62.75	0.2211	60.9435150789441\\
62.75	0.22476	63.7008970464498\\
62.75	0.22842	66.5246689511184\\
62.75	0.23208	69.4148307929499\\
62.75	0.23574	72.3713825719443\\
62.75	0.2394	75.3943242881015\\
62.75	0.24306	78.4836559414216\\
62.75	0.24672	81.6393775319045\\
62.75	0.25038	84.8614890595504\\
62.75	0.25404	88.1499905243591\\
62.75	0.2577	91.5048819263307\\
62.75	0.26136	94.9261632654652\\
62.75	0.26502	98.4138345417625\\
62.75	0.26868	101.967895755223\\
62.75	0.27234	105.588346905846\\
62.75	0.276	109.275187993632\\
63.125	0.093	6.324952202926\\
63.125	0.09666	6.75910351635778\\
63.125	0.10032	7.25964476695242\\
63.125	0.10398	7.82657595470992\\
63.125	0.10764	8.45989707963033\\
63.125	0.1113	9.15960814171358\\
63.125	0.11496	9.92570914095972\\
63.125	0.11862	10.7582000773687\\
63.125	0.12228	11.6570809509406\\
63.125	0.12594	12.6223517616754\\
63.125	0.1296	13.654012509573\\
63.125	0.13326	14.7520631946335\\
63.125	0.13692	15.9165038168569\\
63.125	0.14058	17.1473343762432\\
63.125	0.14424	18.4445548727923\\
63.125	0.1479	19.8081653065043\\
63.125	0.15156	21.2381656773792\\
63.125	0.15522	22.734555985417\\
63.125	0.15888	24.2973362306176\\
63.125	0.16254	25.9265064129811\\
63.125	0.1662	27.6220665325075\\
63.125	0.16986	29.3840165891967\\
63.125	0.17352	31.2123565830489\\
63.125	0.17718	33.1070865140639\\
63.125	0.18084	35.0682063822417\\
63.125	0.1845	37.0957161875825\\
63.125	0.18816	39.1896159300861\\
63.125	0.19182	41.3499056097526\\
63.125	0.19548	43.576585226582\\
63.125	0.19914	45.8696547805743\\
63.125	0.2028	48.2291142717294\\
63.125	0.20646	50.6549637000474\\
63.125	0.21012	53.1472030655283\\
63.125	0.21378	55.705832368172\\
63.125	0.21744	58.3308516079787\\
63.125	0.2211	61.0222607849482\\
63.125	0.22476	63.7800598990805\\
63.125	0.22842	66.6042489503758\\
63.125	0.23208	69.4948279388339\\
63.125	0.23574	72.4517968644549\\
63.125	0.2394	75.4751557272388\\
63.125	0.24306	78.5649045271855\\
63.125	0.24672	81.7210432642951\\
63.125	0.25038	84.9435719385677\\
63.125	0.25404	88.232490550003\\
63.125	0.2577	91.5877990986013\\
63.125	0.26136	95.0094975843624\\
63.125	0.26502	98.4975860072864\\
63.125	0.26868	102.052064367373\\
63.125	0.27234	105.672932664623\\
63.125	0.276	109.360190899036\\
63.5	0.093	6.39284014618478\\
63.5	0.09666	6.82740860624321\\
63.5	0.10032	7.32836700346451\\
63.5	0.10398	7.89571533784866\\
63.5	0.10764	8.52945360939573\\
63.5	0.1113	9.22958181810564\\
63.5	0.11496	9.99609996397843\\
63.5	0.11862	10.8290080470141\\
63.5	0.12228	11.7283060672126\\
63.5	0.12594	12.6939940245741\\
63.5	0.1296	13.7260719190983\\
63.5	0.13326	14.8245397507855\\
63.5	0.13692	15.9893975196355\\
63.5	0.14058	17.2206452256485\\
63.5	0.14424	18.5182828688243\\
63.5	0.1479	19.8823104491629\\
63.5	0.15156	21.3127279666645\\
63.5	0.15522	22.8095354213289\\
63.5	0.15888	24.3727328131562\\
63.5	0.16254	26.0023201421463\\
63.5	0.1662	27.6982974082994\\
63.5	0.16986	29.4606646116153\\
63.5	0.17352	31.2894217520941\\
63.5	0.17718	33.1845688297357\\
63.5	0.18084	35.1461058445402\\
63.5	0.1845	37.1740327965077\\
63.5	0.18816	39.2683496856379\\
63.5	0.19182	41.4290565119311\\
63.5	0.19548	43.6561532753871\\
63.5	0.19914	45.949639976006\\
63.5	0.2028	48.3095166137878\\
63.5	0.20646	50.7357831887325\\
63.5	0.21012	53.22843970084\\
63.5	0.21378	55.7874861501104\\
63.5	0.21744	58.4129225365437\\
63.5	0.2211	61.1047488601398\\
63.5	0.22476	63.8629651208989\\
63.5	0.22842	66.6875713188207\\
63.5	0.23208	69.5785674539055\\
63.5	0.23574	72.5359535261532\\
63.5	0.2394	75.5597295355637\\
63.5	0.24306	78.6498954821371\\
63.5	0.24672	81.8064513658734\\
63.5	0.25038	85.0293971867726\\
63.5	0.25404	88.3187329448346\\
63.5	0.2577	91.6744586400595\\
63.5	0.26136	95.0965742724473\\
63.5	0.26502	98.5850798419979\\
63.5	0.26868	102.139975348711\\
63.5	0.27234	105.761260792588\\
63.5	0.276	109.448936173627\\
63.875	0.093	6.46447045863123\\
63.875	0.09666	6.8994560653163\\
63.875	0.10032	7.40083160916426\\
63.875	0.10398	7.96859709017506\\
63.875	0.10764	8.60275250834878\\
63.875	0.1113	9.30329786368535\\
63.875	0.11496	10.0702331561848\\
63.875	0.11862	10.9035583858471\\
63.875	0.12228	11.8032735526723\\
63.875	0.12594	12.7693786566604\\
63.875	0.1296	13.8018736978113\\
63.875	0.13326	14.9007586761252\\
63.875	0.13692	16.0660335916018\\
63.875	0.14058	17.2976984442414\\
63.875	0.14424	18.5957532340439\\
63.875	0.1479	19.9601979610092\\
63.875	0.15156	21.3910326251374\\
63.875	0.15522	22.8882572264285\\
63.875	0.15888	24.4518717648824\\
63.875	0.16254	26.0818762404992\\
63.875	0.1662	27.7782706532789\\
63.875	0.16986	29.5410550032215\\
63.875	0.17352	31.3702292903269\\
63.875	0.17718	33.2657935145952\\
63.875	0.18084	35.2277476760264\\
63.875	0.1845	37.2560917746205\\
63.875	0.18816	39.3508258103774\\
63.875	0.19182	41.5119497832972\\
63.875	0.19548	43.7394636933799\\
63.875	0.19914	46.0333675406255\\
63.875	0.2028	48.3936613250339\\
63.875	0.20646	50.8203450466052\\
63.875	0.21012	53.3134187053394\\
63.875	0.21378	55.8728823012365\\
63.875	0.21744	58.4987358342964\\
63.875	0.2211	61.1909793045192\\
63.875	0.22476	63.9496127119049\\
63.875	0.22842	66.7746360564534\\
63.875	0.23208	69.6660493381648\\
63.875	0.23574	72.6238525570392\\
63.875	0.2394	75.6480457130764\\
63.875	0.24306	78.7386288062764\\
63.875	0.24672	81.8956018366393\\
63.875	0.25038	85.1189648041652\\
63.875	0.25404	88.4087177088538\\
63.875	0.2577	91.7648605507054\\
63.875	0.26136	95.1873933297198\\
63.875	0.26502	98.6763160458971\\
63.875	0.26868	102.231628699237\\
63.875	0.27234	105.85333128974\\
63.875	0.276	109.541423817406\\
64.25	0.093	6.53984314026532\\
64.25	0.09666	6.97524589357706\\
64.25	0.10032	7.47703858405167\\
64.25	0.10398	8.04522121168913\\
64.25	0.10764	8.6797937764895\\
64.25	0.1113	9.38075627845273\\
64.25	0.11496	10.1481087175788\\
64.25	0.11862	10.9818510938678\\
64.25	0.12228	11.8819834073196\\
64.25	0.12594	12.8485056579344\\
64.25	0.1296	13.881417845712\\
64.25	0.13326	14.9807199706525\\
64.25	0.13692	16.1464120327558\\
64.25	0.14058	17.378494032022\\
64.25	0.14424	18.6769659684511\\
64.25	0.1479	20.0418278420431\\
64.25	0.15156	21.4730796527979\\
64.25	0.15522	22.9707214007157\\
64.25	0.15888	24.5347530857963\\
64.25	0.16254	26.1651747080397\\
64.25	0.1662	27.8619862674461\\
64.25	0.16986	29.6251877640153\\
64.25	0.17352	31.4547791977474\\
64.25	0.17718	33.3507605686424\\
64.25	0.18084	35.3131318767002\\
64.25	0.1845	37.3418931219209\\
64.25	0.18816	39.4370443043045\\
64.25	0.19182	41.598585423851\\
64.25	0.19548	43.8265164805603\\
64.25	0.19914	46.1208374744325\\
64.25	0.2028	48.4815484054677\\
64.25	0.20646	50.9086492736656\\
64.25	0.21012	53.4021400790264\\
64.25	0.21378	55.9620208215501\\
64.25	0.21744	58.5882915012367\\
64.25	0.2211	61.2809521180862\\
64.25	0.22476	64.0400026720985\\
64.25	0.22842	66.8654431632737\\
64.25	0.23208	69.7572735916118\\
64.25	0.23574	72.7154939571128\\
64.25	0.2394	75.7401042597766\\
64.25	0.24306	78.8311044996034\\
64.25	0.24672	81.9884946765929\\
64.25	0.25038	85.2122747907454\\
64.25	0.25404	88.5024448420607\\
64.25	0.2577	91.859004830539\\
64.25	0.26136	95.2819547561801\\
64.25	0.26502	98.771294618984\\
64.25	0.26868	102.327024418951\\
64.25	0.27234	105.949144156081\\
64.25	0.276	109.637653830373\\
64.625	0.093	6.61895819108708\\
64.625	0.09666	7.05477809102547\\
64.625	0.10032	7.55698792812673\\
64.625	0.10398	8.12558770239087\\
64.625	0.10764	8.76057741381787\\
64.625	0.1113	9.46195706240776\\
64.625	0.11496	10.2297266481605\\
64.625	0.11862	11.0638861710761\\
64.625	0.12228	11.9644356311547\\
64.625	0.12594	12.931375028396\\
64.625	0.1296	13.9647043628003\\
64.625	0.13326	15.0644236343674\\
64.625	0.13692	16.2305328430974\\
64.625	0.14058	17.4630319889903\\
64.625	0.14424	18.7619210720461\\
64.625	0.1479	20.1272000922647\\
64.625	0.15156	21.5588690496462\\
64.625	0.15522	23.0569279441906\\
64.625	0.15888	24.6213767758978\\
64.625	0.16254	26.2522155447679\\
64.625	0.1662	27.9494442508009\\
64.625	0.16986	29.7130628939968\\
64.625	0.17352	31.5430714743556\\
64.625	0.17718	33.4394699918772\\
64.625	0.18084	35.4022584465617\\
64.625	0.1845	37.4314368384091\\
64.625	0.18816	39.5270051674193\\
64.625	0.19182	41.6889634335924\\
64.625	0.19548	43.9173116369284\\
64.625	0.19914	46.2120497774273\\
64.625	0.2028	48.573177855089\\
64.625	0.20646	51.0006958699137\\
64.625	0.21012	53.4946038219012\\
64.625	0.21378	56.0549017110515\\
64.625	0.21744	58.6815895373648\\
64.625	0.2211	61.3746673008409\\
64.625	0.22476	64.1341350014799\\
64.625	0.22842	66.9599926392817\\
64.625	0.23208	69.8522402142465\\
64.625	0.23574	72.8108777263741\\
64.625	0.2394	75.8359051756646\\
64.625	0.24306	78.927322562118\\
64.625	0.24672	82.0851298857342\\
64.625	0.25038	85.3093271465133\\
64.625	0.25404	88.5999143444553\\
64.625	0.2577	91.9568914795602\\
64.625	0.26136	95.3802585518279\\
64.625	0.26502	98.8700155612585\\
64.625	0.26868	102.426162507852\\
64.625	0.27234	106.048699391608\\
64.625	0.276	109.737626212528\\
65	0.093	6.70181561109652\\
65	0.09666	7.13805265766156\\
65	0.10032	7.64067964138947\\
65	0.10398	8.20969656228026\\
65	0.10764	8.84510342033392\\
65	0.1113	9.54690021555046\\
65	0.11496	10.3150869479299\\
65	0.11862	11.1496636174721\\
65	0.12228	12.0506302241773\\
65	0.12594	13.0179867680454\\
65	0.1296	14.0517332490763\\
65	0.13326	15.15186966727\\
65	0.13692	16.3183960226267\\
65	0.14058	17.5513123151463\\
65	0.14424	18.8506185448286\\
65	0.1479	20.2163147116739\\
65	0.15156	21.6484008156821\\
65	0.15522	23.1468768568531\\
65	0.15888	24.711742835187\\
65	0.16254	26.3429987506838\\
65	0.1662	28.0406446033435\\
65	0.16986	29.804680393166\\
65	0.17352	31.6351061201514\\
65	0.17718	33.5319217842997\\
65	0.18084	35.4951273856108\\
65	0.1845	37.5247229240848\\
65	0.18816	39.6207083997218\\
65	0.19182	41.7830838125215\\
65	0.19548	44.0118491624842\\
65	0.19914	46.3070044496097\\
65	0.2028	48.6685496738981\\
65	0.20646	51.0964848353494\\
65	0.21012	53.5908099339635\\
65	0.21378	56.1515249697406\\
65	0.21744	58.7786299426804\\
65	0.2211	61.4721248527832\\
65	0.22476	64.2320097000488\\
65	0.22842	67.0582844844774\\
65	0.23208	69.9509492060687\\
65	0.23574	72.9100038648231\\
65	0.2394	75.9354484607402\\
65	0.24306	79.0272829938203\\
65	0.24672	82.1855074640631\\
65	0.25038	85.4101218714689\\
65	0.25404	88.7011262160375\\
65	0.2577	92.0585204977691\\
65	0.26136	95.4823047166635\\
65	0.26502	98.9724788727207\\
65	0.26868	102.529042965941\\
65	0.27234	106.151996996324\\
65	0.276	109.84134096387\\
65.375	0.093	6.78841540029362\\
65.375	0.09666	7.22506959348531\\
65.375	0.10032	7.72811372383989\\
65.375	0.10398	8.29754779135734\\
65.375	0.10764	8.93337179603764\\
65.375	0.1113	9.63558573788084\\
65.375	0.11496	10.4041896168869\\
65.375	0.11862	11.2391834330558\\
65.375	0.12228	12.1405671863877\\
65.375	0.12594	13.1083408768824\\
65.375	0.1296	14.1425045045399\\
65.375	0.13326	15.2430580693603\\
65.375	0.13692	16.4100015713437\\
65.375	0.14058	17.6433350104899\\
65.375	0.14424	18.9430583867989\\
65.375	0.1479	20.3091717002709\\
65.375	0.15156	21.7416749509057\\
65.375	0.15522	23.2405681387034\\
65.375	0.15888	24.8058512636639\\
65.375	0.16254	26.4375243257874\\
65.375	0.1662	28.1355873250737\\
65.375	0.16986	29.9000402615229\\
65.375	0.17352	31.7308831351349\\
65.375	0.17718	33.6281159459098\\
65.375	0.18084	35.5917386938476\\
65.375	0.1845	37.6217513789483\\
65.375	0.18816	39.7181540012119\\
65.375	0.19182	41.8809465606383\\
65.375	0.19548	44.1101290572276\\
65.375	0.19914	46.4057014909798\\
65.375	0.2028	48.7676638618949\\
65.375	0.20646	51.1960161699728\\
65.375	0.21012	53.6907584152136\\
65.375	0.21378	56.2518905976173\\
65.375	0.21744	58.8794127171838\\
65.375	0.2211	61.5733247739132\\
65.375	0.22476	64.3336267678055\\
65.375	0.22842	67.1603186988607\\
65.375	0.23208	70.0534005670787\\
65.375	0.23574	73.0128723724597\\
65.375	0.2394	76.0387341150035\\
65.375	0.24306	79.1309857947102\\
65.375	0.24672	82.2896274115797\\
65.375	0.25038	85.5146589656122\\
65.375	0.25404	88.8060804568075\\
65.375	0.2577	92.1638918851656\\
65.375	0.26136	95.5880932506867\\
65.375	0.26502	99.0786845533706\\
65.375	0.26868	102.635665793217\\
65.375	0.27234	106.259036970227\\
65.375	0.276	109.9487980844\\
65.75	0.093	6.87875755867838\\
65.75	0.09666	7.31582889849674\\
65.75	0.10032	7.81929017547796\\
65.75	0.10398	8.38914138962205\\
65.75	0.10764	9.02538254092902\\
65.75	0.1113	9.72801362939887\\
65.75	0.11496	10.4970346550316\\
65.75	0.11862	11.3324456178272\\
65.75	0.12228	12.2342465177857\\
65.75	0.12594	13.202437354907\\
65.75	0.1296	14.2370181291912\\
65.75	0.13326	15.3379888406383\\
65.75	0.13692	16.5053494892483\\
65.75	0.14058	17.7391000750211\\
65.75	0.14424	19.0392405979568\\
65.75	0.1479	20.4057710580554\\
65.75	0.15156	21.8386914553169\\
65.75	0.15522	23.3380017897412\\
65.75	0.15888	24.9037020613285\\
65.75	0.16254	26.5357922700786\\
65.75	0.1662	28.2342724159915\\
65.75	0.16986	29.9991424990674\\
65.75	0.17352	31.8304025193061\\
65.75	0.17718	33.7280524767076\\
65.75	0.18084	35.6920923712721\\
65.75	0.1845	37.7225222029994\\
65.75	0.18816	39.8193419718897\\
65.75	0.19182	41.9825516779427\\
65.75	0.19548	44.2121513211587\\
65.75	0.19914	46.5081409015375\\
65.75	0.2028	48.8705204190793\\
65.75	0.20646	51.2992898737838\\
65.75	0.21012	53.7944492656513\\
65.75	0.21378	56.3559985946816\\
65.75	0.21744	58.9839378608748\\
65.75	0.2211	61.6782670642309\\
65.75	0.22476	64.4389862047499\\
65.75	0.22842	67.2660952824317\\
65.75	0.23208	70.1595942972764\\
65.75	0.23574	73.119483249284\\
65.75	0.2394	76.1457621384544\\
65.75	0.24306	79.2384309647878\\
65.75	0.24672	82.397489728284\\
65.75	0.25038	85.6229384289431\\
65.75	0.25404	88.914777066765\\
65.75	0.2577	92.2730056417498\\
65.75	0.26136	95.6976241538976\\
65.75	0.26502	99.1886326032081\\
65.75	0.26868	102.746030989682\\
65.75	0.27234	106.369819313318\\
65.75	0.276	110.059997574117\\
66.125	0.093	6.9728420862508\\
66.125	0.09666	7.4103305726958\\
66.125	0.10032	7.9142089963037\\
66.125	0.10398	8.48447735707443\\
66.125	0.10764	9.12113565500808\\
66.125	0.1113	9.82418389010456\\
66.125	0.11496	10.5936220623639\\
66.125	0.11862	11.4294501717862\\
66.125	0.12228	12.3316682183713\\
66.125	0.12594	13.3002762021193\\
66.125	0.1296	14.3352741230302\\
66.125	0.13326	15.4366619811039\\
66.125	0.13692	16.6044397763405\\
66.125	0.14058	17.8386075087401\\
66.125	0.14424	19.1391651783024\\
66.125	0.1479	20.5061127850277\\
66.125	0.15156	21.9394503289158\\
66.125	0.15522	23.4391778099668\\
66.125	0.15888	25.0052952281807\\
66.125	0.16254	26.6378025835574\\
66.125	0.1662	28.336699876097\\
66.125	0.16986	30.1019871057995\\
66.125	0.17352	31.9336642726649\\
66.125	0.17718	33.8317313766931\\
66.125	0.18084	35.7961884178842\\
66.125	0.1845	37.8270353962382\\
66.125	0.18816	39.9242723117551\\
66.125	0.19182	42.0878991644348\\
66.125	0.19548	44.3179159542774\\
66.125	0.19914	46.6143226812829\\
66.125	0.2028	48.9771193454513\\
66.125	0.20646	51.4063059467826\\
66.125	0.21012	53.9018824852767\\
66.125	0.21378	56.4638489609336\\
66.125	0.21744	59.0922053737535\\
66.125	0.2211	61.7869517237362\\
66.125	0.22476	64.5480880108818\\
66.125	0.22842	67.3756142351903\\
66.125	0.23208	70.2695303966617\\
66.125	0.23574	73.2298364952959\\
66.125	0.2394	76.2565325310931\\
66.125	0.24306	79.349618504053\\
66.125	0.24672	82.5090944141758\\
66.125	0.25038	85.7349602614616\\
66.125	0.25404	89.0272160459102\\
66.125	0.2577	92.3858617675217\\
66.125	0.26136	95.8108974262961\\
66.125	0.26502	99.3023230222333\\
66.125	0.26868	102.860138555333\\
66.125	0.27234	106.484344025596\\
66.125	0.276	110.174939433022\\
66.5	0.093	7.07066898301089\\
66.5	0.09666	7.50857461608255\\
66.5	0.10032	8.01287018631711\\
66.5	0.10398	8.58355569371449\\
66.5	0.10764	9.22063113827479\\
66.5	0.1113	9.92409651999794\\
66.5	0.11496	10.693951838884\\
66.5	0.11862	11.5301970949329\\
66.5	0.12228	12.4328322881446\\
66.5	0.12594	13.4018574185193\\
66.5	0.1296	14.4372724860568\\
66.5	0.13326	15.5390774907572\\
66.5	0.13692	16.7072724326205\\
66.5	0.14058	17.9418573116467\\
66.5	0.14424	19.2428321278357\\
66.5	0.1479	20.6101968811876\\
66.5	0.15156	22.0439515717024\\
66.5	0.15522	23.54409619938\\
66.5	0.15888	25.1106307642206\\
66.5	0.16254	26.7435552662239\\
66.5	0.1662	28.4428697053902\\
66.5	0.16986	30.2085740817194\\
66.5	0.17352	32.0406683952114\\
66.5	0.17718	33.9391526458663\\
66.5	0.18084	35.904026833684\\
66.5	0.1845	37.9352909586647\\
66.5	0.18816	40.0329450208082\\
66.5	0.19182	42.1969890201146\\
66.5	0.19548	44.4274229565839\\
66.5	0.19914	46.724246830216\\
66.5	0.2028	49.0874606410111\\
66.5	0.20646	51.517064388969\\
66.5	0.21012	54.0130580740897\\
66.5	0.21378	56.5754416963733\\
66.5	0.21744	59.2042152558199\\
66.5	0.2211	61.8993787524293\\
66.5	0.22476	64.6609321862015\\
66.5	0.22842	67.4888755571366\\
66.5	0.23208	70.3832088652346\\
66.5	0.23574	73.3439321104956\\
66.5	0.2394	76.3710452929193\\
66.5	0.24306	79.464548412506\\
66.5	0.24672	82.6244414692555\\
66.5	0.25038	85.8507244631679\\
66.5	0.25404	89.1433973942431\\
66.5	0.2577	92.5024602624813\\
66.5	0.26136	95.9279130678823\\
66.5	0.26502	99.4197558104462\\
66.5	0.26868	102.977988490173\\
66.5	0.27234	106.602611107063\\
66.5	0.276	110.293623661115\\
66.875	0.093	7.17223824895865\\
66.875	0.09666	7.61056102865697\\
66.875	0.10032	8.11527374551817\\
66.875	0.10398	8.6863763995422\\
66.875	0.10764	9.32386899072916\\
66.875	0.1113	10.027751519079\\
66.875	0.11496	10.7980239845916\\
66.875	0.11862	11.6346863872672\\
66.875	0.12228	12.5377387271056\\
66.875	0.12594	13.507181004107\\
66.875	0.1296	14.5430132182711\\
66.875	0.13326	15.6452353695982\\
66.875	0.13692	16.8138474580881\\
66.875	0.14058	18.0488494837409\\
66.875	0.14424	19.3502414465566\\
66.875	0.1479	20.7180233465352\\
66.875	0.15156	22.1521951836766\\
66.875	0.15522	23.6527569579809\\
66.875	0.15888	25.2197086694481\\
66.875	0.16254	26.8530503180781\\
66.875	0.1662	28.5527819038711\\
66.875	0.16986	30.3189034268269\\
66.875	0.17352	32.1514148869456\\
66.875	0.17718	34.0503162842271\\
66.875	0.18084	36.0156076186715\\
66.875	0.1845	38.0472888902788\\
66.875	0.18816	40.145360099049\\
66.875	0.19182	42.309821244982\\
66.875	0.19548	44.540672328078\\
66.875	0.19914	46.8379133483368\\
66.875	0.2028	49.2015443057585\\
66.875	0.20646	51.631565200343\\
66.875	0.21012	54.1279760320904\\
66.875	0.21378	56.6907768010007\\
66.875	0.21744	59.3199675070739\\
66.875	0.2211	62.0155481503099\\
66.875	0.22476	64.7775187307088\\
66.875	0.22842	67.6058792482706\\
66.875	0.23208	70.5006297029953\\
66.875	0.23574	73.4617700948828\\
66.875	0.2394	76.4893004239333\\
66.875	0.24306	79.5832206901466\\
66.875	0.24672	82.7435308935227\\
66.875	0.25038	85.9702310340618\\
66.875	0.25404	89.2633211117637\\
66.875	0.2577	92.6228011266285\\
66.875	0.26136	96.0486710786562\\
66.875	0.26502	99.5409309678467\\
66.875	0.26868	103.0995807942\\
66.875	0.27234	106.724620557716\\
66.875	0.276	110.416050258396\\
67.25	0.093	7.27754988409409\\
67.25	0.09666	7.71628981041905\\
67.25	0.10032	8.22141967390691\\
67.25	0.10398	8.79293947455761\\
67.25	0.10764	9.43084921237121\\
67.25	0.1113	10.1351488873477\\
67.25	0.11496	10.905838499487\\
67.25	0.11862	11.7429180487892\\
67.25	0.12228	12.6463875352543\\
67.25	0.12594	13.6162469588823\\
67.25	0.1296	14.6524963196731\\
67.25	0.13326	15.7551356176268\\
67.25	0.13692	16.9241648527434\\
67.25	0.14058	18.1595840250229\\
67.25	0.14424	19.4613931344652\\
67.25	0.1479	20.8295921810704\\
67.25	0.15156	22.2641811648385\\
67.25	0.15522	23.7651600857695\\
67.25	0.15888	25.3325289438633\\
67.25	0.16254	26.96628773912\\
67.25	0.1662	28.6664364715396\\
67.25	0.16986	30.432975141122\\
67.25	0.17352	32.2659037478674\\
67.25	0.17718	34.1652222917756\\
67.25	0.18084	36.1309307728466\\
67.25	0.1845	38.1630291910806\\
67.25	0.18816	40.2615175464774\\
67.25	0.19182	42.4263958390371\\
67.25	0.19548	44.6576640687597\\
67.25	0.19914	46.9553222356452\\
67.25	0.2028	49.3193703396935\\
67.25	0.20646	51.7498083809047\\
67.25	0.21012	54.2466363592788\\
67.25	0.21378	56.8098542748157\\
67.25	0.21744	59.4394621275156\\
67.25	0.2211	62.1354599173783\\
67.25	0.22476	64.8978476444038\\
67.25	0.22842	67.7266253085923\\
67.25	0.23208	70.6217929099436\\
67.25	0.23574	73.5833504484578\\
67.25	0.2394	76.6112979241349\\
67.25	0.24306	79.7056353369748\\
67.25	0.24672	82.8663626869777\\
67.25	0.25038	86.0934799741434\\
67.25	0.25404	89.3869871984719\\
67.25	0.2577	92.7468843599634\\
67.25	0.26136	96.1731714586177\\
67.25	0.26502	99.6658484944349\\
67.25	0.26868	103.224915467415\\
67.25	0.27234	106.850372377558\\
67.25	0.276	110.542219224864\\
67.625	0.093	7.38660388841716\\
67.625	0.09666	7.82576096136878\\
67.625	0.10032	8.33130797148327\\
67.625	0.10398	8.90324491876065\\
67.625	0.10764	9.54157180320088\\
67.625	0.1113	10.246288624804\\
67.625	0.11496	11.01739538357\\
67.625	0.11862	11.8548920794989\\
67.625	0.12228	12.7587787125906\\
67.625	0.12594	13.7290552828452\\
67.625	0.1296	14.7657217902627\\
67.625	0.13326	15.8687782348431\\
67.625	0.13692	17.0382246165863\\
67.625	0.14058	18.2740609354925\\
67.625	0.14424	19.5762871915614\\
67.625	0.1479	20.9449033847933\\
67.625	0.15156	22.379909515188\\
67.625	0.15522	23.8813055827457\\
67.625	0.15888	25.4490915874662\\
67.625	0.16254	27.0832675293495\\
67.625	0.1662	28.7838334083958\\
67.625	0.16986	30.5507892246049\\
67.625	0.17352	32.3841349779769\\
67.625	0.17718	34.2838706685117\\
67.625	0.18084	36.2499962962094\\
67.625	0.1845	38.28251186107\\
67.625	0.18816	40.3814173630935\\
67.625	0.19182	42.5467128022799\\
67.625	0.19548	44.7783981786291\\
67.625	0.19914	47.0764734921412\\
67.625	0.2028	49.4409387428162\\
67.625	0.20646	51.8717939306541\\
67.625	0.21012	54.3690390556548\\
67.625	0.21378	56.9326741178184\\
67.625	0.21744	59.5626991171449\\
67.625	0.2211	62.2591140536342\\
67.625	0.22476	65.0219189272865\\
67.625	0.22842	67.8511137381016\\
67.625	0.23208	70.7466984860795\\
67.625	0.23574	73.7086731712204\\
67.625	0.2394	76.7370377935241\\
67.625	0.24306	79.8317923529908\\
67.625	0.24672	82.9929368496202\\
67.625	0.25038	86.2204712834126\\
67.625	0.25404	89.5143956543678\\
67.625	0.2577	92.8747099624859\\
67.625	0.26136	96.3014142077669\\
67.625	0.26502	99.7945083902107\\
67.625	0.26868	103.353992509817\\
67.625	0.27234	106.979866566587\\
67.625	0.276	110.672130560519\\
68	0.093	7.49940026192791\\
68	0.09666	7.9389744815062\\
68	0.10032	8.44493863824733\\
68	0.10398	9.01729273215138\\
68	0.10764	9.65603676321827\\
68	0.1113	10.361170731448\\
68	0.11496	11.1326946368407\\
68	0.11862	11.9706084793962\\
68	0.12228	12.8749122591146\\
68	0.12594	13.8456059759959\\
68	0.1296	14.88268963004\\
68	0.13326	15.986163221247\\
68	0.13692	17.1560267496169\\
68	0.14058	18.3922802151497\\
68	0.14424	19.6949236178454\\
68	0.1479	21.0639569577039\\
68	0.15156	22.4993802347253\\
68	0.15522	24.0011934489096\\
68	0.15888	25.5693966002567\\
68	0.16254	27.2039896887667\\
68	0.1662	28.9049727144396\\
68	0.16986	30.6723456772754\\
68	0.17352	32.506108577274\\
68	0.17718	34.4062614144355\\
68	0.18084	36.3728041887599\\
68	0.1845	38.4057369002472\\
68	0.18816	40.5050595488973\\
68	0.19182	42.6707721347103\\
68	0.19548	44.9028746576862\\
68	0.19914	47.201367117825\\
68	0.2028	49.5662495151266\\
68	0.20646	51.9975218495911\\
68	0.21012	54.4951841212185\\
68	0.21378	57.0592363300088\\
68	0.21744	59.6896784759619\\
68	0.2211	62.3865105590779\\
68	0.22476	65.1497325793568\\
68	0.22842	67.9793445367985\\
68	0.23208	70.8753464314032\\
68	0.23574	73.8377382631707\\
68	0.2394	76.8665200321011\\
68	0.24306	79.9616917381943\\
68	0.24672	83.1232533814505\\
68	0.25038	86.3512049618695\\
68	0.25404	89.6455464794514\\
68	0.2577	93.0062779341961\\
68	0.26136	96.4333993261037\\
68	0.26502	99.9269106551743\\
68	0.26868	103.486811921408\\
68	0.27234	107.113103124804\\
68	0.276	110.805784265363\\
68.375	0.093	7.61593900462633\\
68.375	0.09666	8.05593037083127\\
68.375	0.10032	8.56231167419906\\
68.375	0.10398	9.13508291472976\\
68.375	0.10764	9.77424409242331\\
68.375	0.1113	10.4797952072797\\
68.375	0.11496	11.251736259299\\
68.375	0.11862	12.0900672484812\\
68.375	0.12228	12.9947881748263\\
68.375	0.12594	13.9658990383342\\
68.375	0.1296	15.003399839005\\
68.375	0.13326	16.1072905768387\\
68.375	0.13692	17.2775712518352\\
68.375	0.14058	18.5142418639947\\
68.375	0.14424	19.8173024133169\\
68.375	0.1479	21.1867528998021\\
68.375	0.15156	22.6225933234502\\
68.375	0.15522	24.1248236842611\\
68.375	0.15888	25.6934439822349\\
68.375	0.16254	27.3284542173716\\
68.375	0.1662	29.0298543896711\\
68.375	0.16986	30.7976444991335\\
68.375	0.17352	32.6318245457588\\
68.375	0.17718	34.532394529547\\
68.375	0.18084	36.499354450498\\
68.375	0.1845	38.532704308612\\
68.375	0.18816	40.6324441038887\\
68.375	0.19182	42.7985738363284\\
68.375	0.19548	45.031093505931\\
68.375	0.19914	47.3300031126964\\
68.375	0.2028	49.6953026566247\\
68.375	0.20646	52.1269921377159\\
68.375	0.21012	54.6250715559699\\
68.375	0.21378	57.1895409113868\\
68.375	0.21744	59.8204002039666\\
68.375	0.2211	62.5176494337093\\
68.375	0.22476	65.2812886006148\\
68.375	0.22842	68.1113177046832\\
68.375	0.23208	71.0077367459144\\
68.375	0.23574	73.9705457243086\\
68.375	0.2394	76.9997446398657\\
68.375	0.24306	80.0953334925856\\
68.375	0.24672	83.2573122824684\\
68.375	0.25038	86.4856810095141\\
68.375	0.25404	89.7804396737226\\
68.375	0.2577	93.141588275094\\
68.375	0.26136	96.5691268136283\\
68.375	0.26502	100.063055289325\\
68.375	0.26868	103.623373702185\\
68.375	0.27234	107.250082052208\\
68.375	0.276	110.943180339394\\
68.75	0.093	7.73622011651242\\
68.75	0.09666	8.17662862934401\\
68.75	0.10032	8.68342707933845\\
68.75	0.10398	9.2566154664958\\
68.75	0.10764	9.896193790816\\
68.75	0.1113	10.6021620522991\\
68.75	0.11496	11.3745202509451\\
68.75	0.11862	12.2132683867539\\
68.75	0.12228	13.1184064597256\\
68.75	0.12594	14.0899344698602\\
68.75	0.1296	15.1278524171576\\
68.75	0.13326	16.2321603016179\\
68.75	0.13692	17.4028581232412\\
68.75	0.14058	18.6399458820272\\
68.75	0.14424	19.9434235779762\\
68.75	0.1479	21.313291211088\\
68.75	0.15156	22.7495487813627\\
68.75	0.15522	24.2521962888003\\
68.75	0.15888	25.8212337334008\\
68.75	0.16254	27.4566611151641\\
68.75	0.1662	29.1584784340903\\
68.75	0.16986	30.9266856901794\\
68.75	0.17352	32.7612828834313\\
68.75	0.17718	34.6622700138461\\
68.75	0.18084	36.6296470814238\\
68.75	0.1845	38.6634140861644\\
68.75	0.18816	40.7635710280679\\
68.75	0.19182	42.9301179071342\\
68.75	0.19548	45.1630547233634\\
68.75	0.19914	47.4623814767554\\
68.75	0.2028	49.8280981673104\\
68.75	0.20646	52.2602047950282\\
68.75	0.21012	54.7587013599089\\
68.75	0.21378	57.3235878619525\\
68.75	0.21744	59.9548643011589\\
68.75	0.2211	62.6525306775282\\
68.75	0.22476	65.4165869910604\\
68.75	0.22842	68.2470332417555\\
68.75	0.23208	71.1438694296134\\
68.75	0.23574	74.1070955546343\\
68.75	0.2394	77.1367116168179\\
68.75	0.24306	80.2327176161645\\
68.75	0.24672	83.3951135526739\\
68.75	0.25038	86.6238994263463\\
68.75	0.25404	89.9190752371815\\
68.75	0.2577	93.2806409851795\\
68.75	0.26136	96.7085966703405\\
68.75	0.26502	100.202942292664\\
68.75	0.26868	103.763677852151\\
68.75	0.27234	107.390803348801\\
68.75	0.276	111.084318782613\\
69.125	0.093	7.86024359758616\\
69.125	0.09666	8.30106925704441\\
69.125	0.10032	8.80828485366552\\
69.125	0.10398	9.38189038744952\\
69.125	0.10764	10.0218858583964\\
69.125	0.1113	10.7282712665061\\
69.125	0.11496	11.5010466117787\\
69.125	0.11862	12.3402118942142\\
69.125	0.12228	13.2457671138126\\
69.125	0.12594	14.2177122705738\\
69.125	0.1296	15.2560473644979\\
69.125	0.13326	16.3607723955849\\
69.125	0.13692	17.5318873638348\\
69.125	0.14058	18.7693922692475\\
69.125	0.14424	20.0732871118231\\
69.125	0.1479	21.4435718915616\\
69.125	0.15156	22.880246608463\\
69.125	0.15522	24.3833112625272\\
69.125	0.15888	25.9527658537543\\
69.125	0.16254	27.5886103821443\\
69.125	0.1662	29.2908448476971\\
69.125	0.16986	31.0594692504129\\
69.125	0.17352	32.8944835902915\\
69.125	0.17718	34.7958878673329\\
69.125	0.18084	36.7636820815373\\
69.125	0.1845	38.7978662329045\\
69.125	0.18816	40.8984403214346\\
69.125	0.19182	43.0654043471276\\
69.125	0.19548	45.2987583099834\\
69.125	0.19914	47.5985022100022\\
69.125	0.2028	49.9646360471838\\
69.125	0.20646	52.3971598215283\\
69.125	0.21012	54.8960735330356\\
69.125	0.21378	57.4613771817058\\
69.125	0.21744	60.0930707675389\\
69.125	0.2211	62.7911542905349\\
69.125	0.22476	65.5556277506937\\
69.125	0.22842	68.3864911480154\\
69.125	0.23208	71.2837444825\\
69.125	0.23574	74.2473877541475\\
69.125	0.2394	77.2774209629579\\
69.125	0.24306	80.3738441089311\\
69.125	0.24672	83.5366571920672\\
69.125	0.25038	86.7658602123662\\
69.125	0.25404	90.061453169828\\
69.125	0.2577	93.4234360644527\\
69.125	0.26136	96.8518088962404\\
69.125	0.26502	100.346571665191\\
69.125	0.26868	103.907724371304\\
69.125	0.27234	107.53526701458\\
69.125	0.276	111.229199595019\\
69.5	0.093	7.98800944784757\\
69.5	0.09666	8.42925225393246\\
69.5	0.10032	8.93688499718024\\
69.5	0.10398	9.51090767759086\\
69.5	0.10764	10.1513202951644\\
69.5	0.1113	10.8581228499008\\
69.5	0.11496	11.6313153418001\\
69.5	0.11862	12.4708977708622\\
69.5	0.12228	13.3768701370872\\
69.5	0.12594	14.3492324404751\\
69.5	0.1296	15.3879846810259\\
69.5	0.13326	16.4931268587395\\
69.5	0.13692	17.664658973616\\
69.5	0.14058	18.9025810256554\\
69.5	0.14424	20.2068930148577\\
69.5	0.1479	21.5775949412228\\
69.5	0.15156	23.0146868047508\\
69.5	0.15522	24.5181686054417\\
69.5	0.15888	26.0880403432955\\
69.5	0.16254	27.7243020183121\\
69.5	0.1662	29.4269536304916\\
69.5	0.16986	31.195995179834\\
69.5	0.17352	33.0314266663393\\
69.5	0.17718	34.9332480900074\\
69.5	0.18084	36.9014594508384\\
69.5	0.1845	38.9360607488323\\
69.5	0.18816	41.037051983989\\
69.5	0.19182	43.2044331563087\\
69.5	0.19548	45.4382042657912\\
69.5	0.19914	47.7383653124366\\
69.5	0.2028	50.1049162962448\\
69.5	0.20646	52.537857217216\\
69.5	0.21012	55.03718807535\\
69.5	0.21378	57.6029088706468\\
69.5	0.21744	60.2350196031066\\
69.5	0.2211	62.9335202727292\\
69.5	0.22476	65.6984108795147\\
69.5	0.22842	68.5296914234631\\
69.5	0.23208	71.4273619045743\\
69.5	0.23574	74.3914223228485\\
69.5	0.2394	77.4218726782855\\
69.5	0.24306	80.5187129708853\\
69.5	0.24672	83.6819432006481\\
69.5	0.25038	86.9115633675737\\
69.5	0.25404	90.2075734716622\\
69.5	0.2577	93.5699735129136\\
69.5	0.26136	96.9987634913279\\
69.5	0.26502	100.493943406905\\
69.5	0.26868	104.055513259645\\
69.5	0.27234	107.683473049548\\
69.5	0.276	111.377822776614\\
69.875	0.093	8.11951766729666\\
69.875	0.09666	8.5611776200082\\
69.875	0.10032	9.06922750988263\\
69.875	0.10398	9.64366733691992\\
69.875	0.10764	10.2844971011201\\
69.875	0.1113	10.9917168024832\\
69.875	0.11496	11.7653264410091\\
69.875	0.11862	12.6053260166979\\
69.875	0.12228	13.5117155295495\\
69.875	0.12594	14.4844949795641\\
69.875	0.1296	15.5236643667415\\
69.875	0.13326	16.6292236910818\\
69.875	0.13692	17.801172952585\\
69.875	0.14058	19.039512151251\\
69.875	0.14424	20.3442412870799\\
69.875	0.1479	21.7153603600717\\
69.875	0.15156	23.1528693702264\\
69.875	0.15522	24.6567683175439\\
69.875	0.15888	26.2270572020244\\
69.875	0.16254	27.8637360236676\\
69.875	0.1662	29.5668047824738\\
69.875	0.16986	31.3362634784428\\
69.875	0.17352	33.1721121115748\\
69.875	0.17718	35.0743506818695\\
69.875	0.18084	37.0429791893272\\
69.875	0.1845	39.0779976339477\\
69.875	0.18816	41.1794060157311\\
69.875	0.19182	43.3472043346774\\
69.875	0.19548	45.5813925907866\\
69.875	0.19914	47.8819707840586\\
69.875	0.2028	50.2489389144936\\
69.875	0.20646	52.6822969820913\\
69.875	0.21012	55.182044986852\\
69.875	0.21378	57.7481829287755\\
69.875	0.21744	60.3807108078619\\
69.875	0.2211	63.0796286241112\\
69.875	0.22476	65.8449363775234\\
69.875	0.22842	68.6766340680984\\
69.875	0.23208	71.5747216958363\\
69.875	0.23574	74.5391992607371\\
69.875	0.2394	77.5700667628007\\
69.875	0.24306	80.6673242020273\\
69.875	0.24672	83.8309715784167\\
69.875	0.25038	87.061008891969\\
69.875	0.25404	90.3574361426841\\
69.875	0.2577	93.7202533305622\\
69.875	0.26136	97.149460455603\\
69.875	0.26502	100.645057517807\\
69.875	0.26868	104.207044517173\\
69.875	0.27234	107.835421453703\\
69.875	0.276	111.530188327395\\
70.25	0.093	8.25476825593339\\
70.25	0.09666	8.69684535527161\\
70.25	0.10032	9.20531239177269\\
70.25	0.10398	9.78016936543663\\
70.25	0.10764	10.4214162762635\\
70.25	0.1113	11.1290531242532\\
70.25	0.11496	11.9030799094057\\
70.25	0.11862	12.7434966317212\\
70.25	0.12228	13.6503032911995\\
70.25	0.12594	14.6234998878407\\
70.25	0.1296	15.6630864216448\\
70.25	0.13326	16.7690628926117\\
70.25	0.13692	17.9414293007416\\
70.25	0.14058	19.1801856460343\\
70.25	0.14424	20.4853319284898\\
70.25	0.1479	21.8568681481083\\
70.25	0.15156	23.2947943048896\\
70.25	0.15522	24.7991103988338\\
70.25	0.15888	26.3698164299409\\
70.25	0.16254	28.0069123982108\\
70.25	0.1662	29.7103983036436\\
70.25	0.16986	31.4802741462393\\
70.25	0.17352	33.3165399259979\\
70.25	0.17718	35.2191956429193\\
70.25	0.18084	37.1882412970036\\
70.25	0.1845	39.2236768882508\\
70.25	0.18816	41.3255024166609\\
70.25	0.19182	43.4937178822339\\
70.25	0.19548	45.7283232849697\\
70.25	0.19914	48.0293186248684\\
70.25	0.2028	50.3967039019299\\
70.25	0.20646	52.8304791161544\\
70.25	0.21012	55.3306442675417\\
70.25	0.21378	57.8971993560919\\
70.25	0.21744	60.5301443818049\\
70.25	0.2211	63.2294793446809\\
70.25	0.22476	65.9952042447197\\
70.25	0.22842	68.8273190819213\\
70.25	0.23208	71.7258238562859\\
70.25	0.23574	74.6907185678133\\
70.25	0.2394	77.7220032165037\\
70.25	0.24306	80.8196778023569\\
70.25	0.24672	83.9837423253729\\
70.25	0.25038	87.2141967855518\\
70.25	0.25404	90.5110411828936\\
70.25	0.2577	93.8742755173984\\
70.25	0.26136	97.3038997890659\\
70.25	0.26502	100.799913997896\\
70.25	0.26868	104.36231814389\\
70.25	0.27234	107.991112227046\\
70.25	0.276	111.686296247365\\
70.625	0.093	8.39376121375781\\
70.625	0.09666	8.83625545972267\\
70.625	0.10032	9.34513964285041\\
70.625	0.10398	9.920413763141\\
70.625	0.10764	10.5620778205945\\
70.625	0.1113	11.2701318152108\\
70.625	0.11496	12.0445757469901\\
70.625	0.11862	12.8854096159322\\
70.625	0.12228	13.7926334220372\\
70.625	0.12594	14.766247165305\\
70.625	0.1296	15.8062508457357\\
70.625	0.13326	16.9126444633294\\
70.625	0.13692	18.0854280180858\\
70.625	0.14058	19.3246015100052\\
70.625	0.14424	20.6301649390874\\
70.625	0.1479	22.0021183053325\\
70.625	0.15156	23.4404616087405\\
70.625	0.15522	24.9451948493113\\
70.625	0.15888	26.5163180270451\\
70.625	0.16254	28.1538311419417\\
70.625	0.1662	29.8577341940011\\
70.625	0.16986	31.6280271832235\\
70.625	0.17352	33.4647101096087\\
70.625	0.17718	35.3677829731568\\
70.625	0.18084	37.3372457738678\\
70.625	0.1845	39.3730985117416\\
70.625	0.18816	41.4753411867784\\
70.625	0.19182	43.6439737989779\\
70.625	0.19548	45.8789963483404\\
70.625	0.19914	48.1804088348658\\
70.625	0.2028	50.548211258554\\
70.625	0.20646	52.9824036194051\\
70.625	0.21012	55.482985917419\\
70.625	0.21378	58.0499581525959\\
70.625	0.21744	60.6833203249356\\
70.625	0.2211	63.3830724344382\\
70.625	0.22476	66.1492144811036\\
70.625	0.22842	68.981746464932\\
70.625	0.23208	71.8806683859232\\
70.625	0.23574	74.8459802440773\\
70.625	0.2394	77.8776820393942\\
70.625	0.24306	80.9757737718741\\
70.625	0.24672	84.1402554415168\\
70.625	0.25038	87.3711270483224\\
70.625	0.25404	90.6683885922909\\
70.625	0.2577	94.0320400734222\\
70.625	0.26136	97.4620814917164\\
70.625	0.26502	100.958512847174\\
70.625	0.26868	104.521334139793\\
70.625	0.27234	108.150545369576\\
70.625	0.276	111.846146536522\\
71	0.093	8.53649654076987\\
71	0.09666	8.97940793336138\\
71	0.10032	9.48870926311576\\
71	0.10398	10.064400530033\\
71	0.10764	10.7064817341132\\
71	0.1113	11.4149528753562\\
71	0.11496	12.1898139537621\\
71	0.11862	13.0310649693308\\
71	0.12228	13.9387059220625\\
71	0.12594	14.912736811957\\
71	0.1296	15.9531576390144\\
71	0.13326	17.0599684032346\\
71	0.13692	18.2331691046177\\
71	0.14058	19.4727597431638\\
71	0.14424	20.7787403188726\\
71	0.1479	22.1511108317444\\
71	0.15156	23.589871281779\\
71	0.15522	25.0950216689765\\
71	0.15888	26.6665619933369\\
71	0.16254	28.3044922548602\\
71	0.1662	30.0088124535463\\
71	0.16986	31.7795225893953\\
71	0.17352	33.6166226624072\\
71	0.17718	35.5201126725819\\
71	0.18084	37.4899926199195\\
71	0.1845	39.52626250442\\
71	0.18816	41.6289223260834\\
71	0.19182	43.7979720849097\\
71	0.19548	46.0334117808988\\
71	0.19914	48.3352414140508\\
71	0.2028	50.7034609843657\\
71	0.20646	53.1380704918434\\
71	0.21012	55.639069936484\\
71	0.21378	58.2064593182875\\
71	0.21744	60.8402386372539\\
71	0.2211	63.5404078933832\\
71	0.22476	66.3069670866753\\
71	0.22842	69.1399162171302\\
71	0.23208	72.0392552847481\\
71	0.23574	75.0049842895289\\
71	0.2394	78.0371032314725\\
71	0.24306	81.135612110579\\
71	0.24672	84.3005109268484\\
71	0.25038	87.5317996802806\\
71	0.25404	90.8294783708757\\
71	0.2577	94.1935469986337\\
71	0.26136	97.6240055635546\\
71	0.26502	101.120854065638\\
71	0.26868	104.684092504885\\
71	0.27234	108.313720881294\\
71	0.276	112.009739194867\\
71.375	0.093	8.68297423696959\\
71.375	0.09666	9.12630277618777\\
71.375	0.10032	9.63602125256881\\
71.375	0.10398	10.2121296661127\\
71.375	0.10764	10.8546280168195\\
71.375	0.1113	11.5635163046892\\
71.375	0.11496	12.3387945297217\\
71.375	0.11862	13.1804626919171\\
71.375	0.12228	14.0885207912754\\
71.375	0.12594	15.0629688277966\\
71.375	0.1296	16.1038068014806\\
71.375	0.13326	17.2110347123275\\
71.375	0.13692	18.3846525603373\\
71.375	0.14058	19.62466034551\\
71.375	0.14424	20.9310580678455\\
71.375	0.1479	22.3038457273439\\
71.375	0.15156	23.7430233240052\\
71.375	0.15522	25.2485908578294\\
71.375	0.15888	26.8205483288164\\
71.375	0.16254	28.4588957369663\\
71.375	0.1662	30.1636330822791\\
71.375	0.16986	31.9347603647548\\
71.375	0.17352	33.7722775843933\\
71.375	0.17718	35.6761847411947\\
71.375	0.18084	37.646481835159\\
71.375	0.1845	39.6831688662862\\
71.375	0.18816	41.7862458345762\\
71.375	0.19182	43.9557127400291\\
71.375	0.19548	46.1915695826449\\
71.375	0.19914	48.4938163624235\\
71.375	0.2028	50.8624530793651\\
71.375	0.20646	53.2974797334695\\
71.375	0.21012	55.7988963247367\\
71.375	0.21378	58.3667028531669\\
71.375	0.21744	61.0008993187599\\
71.375	0.2211	63.7014857215158\\
71.375	0.22476	66.4684620614346\\
71.375	0.22842	69.3018283385162\\
71.375	0.23208	72.2015845527607\\
71.375	0.23574	75.1677307041682\\
71.375	0.2394	78.2002667927384\\
71.375	0.24306	81.2991928184716\\
71.375	0.24672	84.4645087813676\\
71.375	0.25038	87.6962146814265\\
71.375	0.25404	90.9943105186483\\
71.375	0.2577	94.3587962930329\\
71.375	0.26136	97.7896720045804\\
71.375	0.26502	101.286937653291\\
71.375	0.26868	104.850593239164\\
71.375	0.27234	108.4806387622\\
71.375	0.276	112.177074222399\\
71.75	0.093	8.83319430235701\\
71.75	0.09666	9.27693998820184\\
71.75	0.10032	9.78707561120953\\
71.75	0.10398	10.3636011713801\\
71.75	0.10764	11.0065166687135\\
71.75	0.1113	11.7158221032099\\
71.75	0.11496	12.4915174748691\\
71.75	0.11862	13.3336027836911\\
71.75	0.12228	14.2420780296761\\
71.75	0.12594	15.2169432128239\\
71.75	0.1296	16.2581983331346\\
71.75	0.13326	17.3658433906082\\
71.75	0.13692	18.5398783852446\\
71.75	0.14058	19.7803033170439\\
71.75	0.14424	21.0871181860061\\
71.75	0.1479	22.4603229921312\\
71.75	0.15156	23.8999177354191\\
71.75	0.15522	25.4059024158699\\
71.75	0.15888	26.9782770334836\\
71.75	0.16254	28.6170415882602\\
71.75	0.1662	30.3221960801996\\
71.75	0.16986	32.0937405093019\\
71.75	0.17352	33.9316748755671\\
71.75	0.17718	35.8359991789952\\
71.75	0.18084	37.8067134195861\\
71.75	0.1845	39.8438175973399\\
71.75	0.18816	41.9473117122566\\
71.75	0.19182	44.1171957643362\\
71.75	0.19548	46.3534697535786\\
71.75	0.19914	48.6561336799839\\
71.75	0.2028	51.0251875435521\\
71.75	0.20646	53.4606313442832\\
71.75	0.21012	55.9624650821771\\
71.75	0.21378	58.5306887572339\\
71.75	0.21744	61.1653023694536\\
71.75	0.2211	63.8663059188361\\
71.75	0.22476	66.6336994053816\\
71.75	0.22842	69.4674828290898\\
71.75	0.23208	72.367656189961\\
71.75	0.23574	75.3342194879951\\
71.75	0.2394	78.367172723192\\
71.75	0.24306	81.4665158955518\\
71.75	0.24672	84.6322490050745\\
71.75	0.25038	87.8643720517601\\
71.75	0.25404	91.1628850356085\\
71.75	0.2577	94.5277879566198\\
71.75	0.26136	97.959080814794\\
71.75	0.26502	101.456763610131\\
71.75	0.26868	105.020836342631\\
71.75	0.27234	108.651299012294\\
71.75	0.276	112.348151619119\\
72.125	0.093	8.98715673693209\\
72.125	0.09666	9.43131956940356\\
72.125	0.10032	9.94187233903791\\
72.125	0.10398	10.5188150458352\\
72.125	0.10764	11.1621476897952\\
72.125	0.1113	11.8718702709182\\
72.125	0.11496	12.6479827892041\\
72.125	0.11862	13.4904852446528\\
72.125	0.12228	14.3993776372644\\
72.125	0.12594	15.3746599670389\\
72.125	0.1296	16.4163322339762\\
72.125	0.13326	17.5243944380764\\
72.125	0.13692	18.6988465793395\\
72.125	0.14058	19.9396886577655\\
72.125	0.14424	21.2469206733543\\
72.125	0.1479	22.6205426261061\\
72.125	0.15156	24.0605545160207\\
72.125	0.15522	25.5669563430981\\
72.125	0.15888	27.1397481073385\\
72.125	0.16254	28.7789298087417\\
72.125	0.1662	30.4845014473078\\
72.125	0.16986	32.2564630230368\\
72.125	0.17352	34.0948145359286\\
72.125	0.17718	35.9995559859833\\
72.125	0.18084	37.9706873732009\\
72.125	0.1845	40.0082086975814\\
72.125	0.18816	42.1121199591247\\
72.125	0.19182	44.2824211578309\\
72.125	0.19548	46.5191122937\\
72.125	0.19914	48.822193366732\\
72.125	0.2028	51.1916643769268\\
72.125	0.20646	53.6275253242845\\
72.125	0.21012	56.1297762088051\\
72.125	0.21378	58.6984170304886\\
72.125	0.21744	61.3334477893349\\
72.125	0.2211	64.0348684853441\\
72.125	0.22476	66.8026791185162\\
72.125	0.22842	69.6368796888511\\
72.125	0.23208	72.537470196349\\
72.125	0.23574	75.5044506410097\\
72.125	0.2394	78.5378210228333\\
72.125	0.24306	81.6375813418197\\
72.125	0.24672	84.8037315979691\\
72.125	0.25038	88.0362717912813\\
72.125	0.25404	91.3352019217564\\
72.125	0.2577	94.7005219893943\\
72.125	0.26136	98.1322319941951\\
72.125	0.26502	101.630331936159\\
72.125	0.26868	105.194821815285\\
72.125	0.27234	108.825701631575\\
72.125	0.276	112.522971385027\\
72.5	0.093	9.14486154069483\\
72.5	0.09666	9.58944151979296\\
72.5	0.10032	10.100411436054\\
72.5	0.10398	10.6777712894778\\
72.5	0.10764	11.3215210800646\\
72.5	0.1113	12.0316608078142\\
72.5	0.11496	12.8081904727267\\
72.5	0.11862	13.6511100748021\\
72.5	0.12228	14.5604196140404\\
72.5	0.12594	15.5361190904415\\
72.5	0.1296	16.5782085040055\\
72.5	0.13326	17.6866878547324\\
72.5	0.13692	18.8615571426221\\
72.5	0.14058	20.1028163676748\\
72.5	0.14424	21.4104655298903\\
72.5	0.1479	22.7845046292686\\
72.5	0.15156	24.2249336658099\\
72.5	0.15522	25.731752639514\\
72.5	0.15888	27.304961550381\\
72.5	0.16254	28.9445603984109\\
72.5	0.1662	30.6505491836036\\
72.5	0.16986	32.4229279059592\\
72.5	0.17352	34.2616965654777\\
72.5	0.17718	36.1668551621591\\
72.5	0.18084	38.1384036960033\\
72.5	0.1845	40.1763421670105\\
72.5	0.18816	42.2806705751805\\
72.5	0.19182	44.4513889205133\\
72.5	0.19548	46.6884972030091\\
72.5	0.19914	48.9919954226677\\
72.5	0.2028	51.3618835794892\\
72.5	0.20646	53.7981616734736\\
72.5	0.21012	56.3008297046208\\
72.5	0.21378	58.8698876729309\\
72.5	0.21744	61.5053355784039\\
72.5	0.2211	64.2071734210398\\
72.5	0.22476	66.9754012008385\\
72.5	0.22842	69.8100189178001\\
72.5	0.23208	72.7110265719245\\
72.5	0.23574	75.678424163212\\
72.5	0.2394	78.7122116916622\\
72.5	0.24306	81.8123891572753\\
72.5	0.24672	84.9789565600513\\
72.5	0.25038	88.2119138999902\\
72.5	0.25404	91.5112611770919\\
72.5	0.2577	94.8769983913565\\
72.5	0.26136	98.309125542784\\
72.5	0.26502	101.807642631374\\
72.5	0.26868	105.372549657128\\
72.5	0.27234	109.003846620044\\
72.5	0.276	112.701533520123\\
72.875	0.093	9.3063087136452\\
72.875	0.09666	9.75130583936999\\
72.875	0.10032	10.2626929022577\\
72.875	0.10398	10.8404699023082\\
72.875	0.10764	11.4846368395216\\
72.875	0.1113	12.1951937138979\\
72.875	0.11496	12.972140525437\\
72.875	0.11862	13.8154772741391\\
72.875	0.12228	14.725203960004\\
72.875	0.12594	15.7013205830318\\
72.875	0.1296	16.7438271432224\\
72.875	0.13326	17.852723640576\\
72.875	0.13692	19.0280100750924\\
72.875	0.14058	20.2696864467716\\
72.875	0.14424	21.5777527556138\\
72.875	0.1479	22.9522090016188\\
72.875	0.15156	24.3930551847867\\
72.875	0.15522	25.9002913051175\\
72.875	0.15888	27.4739173626112\\
72.875	0.16254	29.1139333572677\\
72.875	0.1662	30.8203392890871\\
72.875	0.16986	32.5931351580694\\
72.875	0.17352	34.4323209642145\\
72.875	0.17718	36.3378967075225\\
72.875	0.18084	38.3098623879934\\
72.875	0.1845	40.3482180056272\\
72.875	0.18816	42.4529635604239\\
72.875	0.19182	44.6240990523834\\
72.875	0.19548	46.8616244815058\\
72.875	0.19914	49.1655398477911\\
72.875	0.2028	51.5358451512392\\
72.875	0.20646	53.9725403918502\\
72.875	0.21012	56.4756255696241\\
72.875	0.21378	59.0451006845609\\
72.875	0.21744	61.6809657366605\\
72.875	0.2211	64.3832207259231\\
72.875	0.22476	67.1518656523484\\
72.875	0.22842	69.9869005159367\\
72.875	0.23208	72.8883253166878\\
72.875	0.23574	75.8561400546019\\
72.875	0.2394	78.8903447296788\\
72.875	0.24306	81.9909393419185\\
72.875	0.24672	85.1579238913211\\
72.875	0.25038	88.3912983778867\\
72.875	0.25404	91.6910628016151\\
72.875	0.2577	95.0572171625064\\
72.875	0.26136	98.4897614605605\\
72.875	0.26502	101.988695695777\\
72.875	0.26868	105.554019868157\\
72.875	0.27234	109.1857339777\\
72.875	0.276	112.883838024406\\
73.25	0.093	9.47149825578327\\
73.25	0.09666	9.91691252813471\\
73.25	0.10032	10.428716737649\\
73.25	0.10398	11.0069108843262\\
73.25	0.10764	11.6514949681663\\
73.25	0.1113	12.3624689891692\\
73.25	0.11496	13.139832947335\\
73.25	0.11862	13.9835868426637\\
73.25	0.12228	14.8937306751553\\
73.25	0.12594	15.8702644448097\\
73.25	0.1296	16.913188151627\\
73.25	0.13326	18.0225017956072\\
73.25	0.13692	19.1982053767503\\
73.25	0.14058	20.4402988950562\\
73.25	0.14424	21.748782350525\\
73.25	0.1479	23.1236557431567\\
73.25	0.15156	24.5649190729513\\
73.25	0.15522	26.0725723399087\\
73.25	0.15888	27.646615544029\\
73.25	0.16254	29.2870486853122\\
73.25	0.1662	30.9938717637582\\
73.25	0.16986	32.7670847793672\\
73.25	0.17352	34.606687732139\\
73.25	0.17718	36.5126806220736\\
73.25	0.18084	38.4850634491712\\
73.25	0.1845	40.5238362134316\\
73.25	0.18816	42.6289989148549\\
73.25	0.19182	44.8005515534411\\
73.25	0.19548	47.0384941291902\\
73.25	0.19914	49.3428266421021\\
73.25	0.2028	51.7135490921769\\
73.25	0.20646	54.1506614794146\\
73.25	0.21012	56.6541638038152\\
73.25	0.21378	59.2240560653785\\
73.25	0.21744	61.8603382641048\\
73.25	0.2211	64.563010399994\\
73.25	0.22476	67.3320724730461\\
73.25	0.22842	70.167524483261\\
73.25	0.23208	73.0693664306388\\
73.25	0.23574	76.0375983151795\\
73.25	0.2394	79.072220136883\\
73.25	0.24306	82.1732318957495\\
73.25	0.24672	85.3406335917787\\
73.25	0.25038	88.5744252249709\\
73.25	0.25404	91.8746067953259\\
73.25	0.2577	95.2411783028439\\
73.25	0.26136	98.6741397475247\\
73.25	0.26502	102.173491129368\\
73.25	0.26868	105.739232448375\\
73.25	0.27234	109.371363704544\\
73.25	0.276	113.069884897877\\
73.625	0.093	9.640430167109\\
73.625	0.09666	10.0862615860871\\
73.625	0.10032	10.5984829422281\\
73.625	0.10398	11.1770942355319\\
73.625	0.10764	11.8220954659986\\
73.625	0.1113	12.5334866336282\\
73.625	0.11496	13.3112677384207\\
73.625	0.11862	14.155438780376\\
73.625	0.12228	15.0659997594942\\
73.625	0.12594	16.0429506757754\\
73.625	0.1296	17.0862915292193\\
73.625	0.13326	18.1960223198261\\
73.625	0.13692	19.3721430475958\\
73.625	0.14058	20.6146537125284\\
73.625	0.14424	21.9235543146239\\
73.625	0.1479	23.2988448538823\\
73.625	0.15156	24.7405253303035\\
73.625	0.15522	26.2485957438876\\
73.625	0.15888	27.8230560946345\\
73.625	0.16254	29.4639063825444\\
73.625	0.1662	31.1711466076171\\
73.625	0.16986	32.9447767698526\\
73.625	0.17352	34.7847968692511\\
73.625	0.17718	36.6912069058124\\
73.625	0.18084	38.6640068795366\\
73.625	0.1845	40.7031967904237\\
73.625	0.18816	42.8087766384737\\
73.625	0.19182	44.9807464236865\\
73.625	0.19548	47.2191061460622\\
73.625	0.19914	49.5238558056008\\
73.625	0.2028	51.8949954023023\\
73.625	0.20646	54.3325249361666\\
73.625	0.21012	56.8364444071938\\
73.625	0.21378	59.4067538153839\\
73.625	0.21744	62.0434531607368\\
73.625	0.2211	64.7465424432527\\
73.625	0.22476	67.5160216629314\\
73.625	0.22842	70.3518908197729\\
73.625	0.23208	73.2541499137774\\
73.625	0.23574	76.2227989449447\\
73.625	0.2394	79.2578379132749\\
73.625	0.24306	82.359266818768\\
73.625	0.24672	85.5270856614239\\
73.625	0.25038	88.7612944412428\\
73.625	0.25404	92.0618931582245\\
73.625	0.2577	95.4288818123691\\
73.625	0.26136	98.8622604036765\\
73.625	0.26502	102.362028932147\\
73.625	0.26868	105.92818739778\\
73.625	0.27234	109.560735800576\\
73.625	0.276	113.259674140535\\
74	0.093	9.81310444762239\\
74	0.09666	10.2593530132271\\
74	0.10032	10.7719915159948\\
74	0.10398	11.3510199559253\\
74	0.10764	11.9964383330186\\
74	0.1113	12.7082466472749\\
74	0.11496	13.486444898694\\
74	0.11862	14.331033087276\\
74	0.12228	15.2420112130209\\
74	0.12594	16.2193792759286\\
74	0.1296	17.2631372759992\\
74	0.13326	18.3732852132328\\
74	0.13692	19.5498230876291\\
74	0.14058	20.7927508991883\\
74	0.14424	22.1020686479105\\
74	0.1479	23.4777763337955\\
74	0.15156	24.9198739568433\\
74	0.15522	26.4283615170541\\
74	0.15888	28.0032390144277\\
74	0.16254	29.6445064489642\\
74	0.1662	31.3521638206636\\
74	0.16986	33.1262111295258\\
74	0.17352	34.9666483755509\\
74	0.17718	36.8734755587389\\
74	0.18084	38.8466926790898\\
74	0.1845	40.8862997366035\\
74	0.18816	42.9922967312801\\
74	0.19182	45.1646836631196\\
74	0.19548	47.403460532122\\
74	0.19914	49.7086273382872\\
74	0.2028	52.0801840816153\\
74	0.20646	54.5181307621063\\
74	0.21012	57.0224673797602\\
74	0.21378	59.5931939345769\\
74	0.21744	62.2303104265565\\
74	0.2211	64.933816855699\\
74	0.22476	67.7037132220043\\
74	0.22842	70.5399995254726\\
74	0.23208	73.4426757661036\\
74	0.23574	76.4117419438977\\
74	0.2394	79.4471980588545\\
74	0.24306	82.5490441109742\\
74	0.24672	85.7172801002568\\
74	0.25038	88.9519060267024\\
74	0.25404	92.2529218903107\\
74	0.2577	95.6203276910819\\
74	0.26136	99.054123429016\\
74	0.26502	102.554309104113\\
74	0.26868	106.120884716373\\
74	0.27234	109.753850265796\\
74	0.276	113.453205752381\\
};
\end{axis}

\begin{axis}[%
width=4.527496cm,
height=3.870968cm,
at={(0cm,5.376344cm)},
scale only axis,
xmin=56,
xmax=74,
tick align=outside,
xlabel={$L_{cut}$},
xmajorgrids,
ymin=0.093,
ymax=0.276,
ylabel={$D_{rlx}$},
ymajorgrids,
zmin=-3.90013578541794,
zmax=52.2197059680429,
zlabel={$x_2$},
zmajorgrids,
view={-140}{50},
legend style={at={(1.03,1)},anchor=north west,legend cell align=left,align=left,draw=white!15!black}
]
\addplot3[only marks,mark=*,mark options={},mark size=1.5000pt,color=mycolor1] plot table[row sep=crcr,]{%
74	0.123	0.706917161850943\\
72	0.113	-0.725188933625476\\
61	0.095	-2.23969084196957\\
56	0.093	-0.663995623424443\\
};
\addplot3[only marks,mark=*,mark options={},mark size=1.5000pt,color=mycolor2] plot table[row sep=crcr,]{%
67	0.276	45.9418866067325\\
66	0.255	36.5734993512285\\
62	0.209	18.3955084875498\\
57	0.193	12.6775369279536\\
};
\addplot3[only marks,mark=*,mark options={},mark size=1.5000pt,color=black] plot table[row sep=crcr,]{%
69	0.104	-2.29954839560715\\
};
\addplot3[only marks,mark=*,mark options={},mark size=1.5000pt,color=black] plot table[row sep=crcr,]{%
64	0.23	25.9519553252518\\
};

\addplot3[%
surf,
opacity=0.7,
shader=interp,
colormap={mymap}{[1pt] rgb(0pt)=(0.0901961,0.239216,0.0745098); rgb(1pt)=(0.0945149,0.242058,0.0739522); rgb(2pt)=(0.0988592,0.244894,0.0733566); rgb(3pt)=(0.103229,0.247724,0.0727241); rgb(4pt)=(0.107623,0.250549,0.0720557); rgb(5pt)=(0.112043,0.253367,0.0713525); rgb(6pt)=(0.116487,0.25618,0.0706154); rgb(7pt)=(0.120956,0.258986,0.0698456); rgb(8pt)=(0.125449,0.261787,0.0690441); rgb(9pt)=(0.129967,0.264581,0.0682118); rgb(10pt)=(0.134508,0.26737,0.06735); rgb(11pt)=(0.139074,0.270152,0.0664596); rgb(12pt)=(0.143663,0.272929,0.0655416); rgb(13pt)=(0.148275,0.275699,0.0645971); rgb(14pt)=(0.152911,0.278463,0.0636271); rgb(15pt)=(0.15757,0.281221,0.0626328); rgb(16pt)=(0.162252,0.283973,0.0616151); rgb(17pt)=(0.166957,0.286719,0.060575); rgb(18pt)=(0.171685,0.289458,0.0595136); rgb(19pt)=(0.176434,0.292191,0.0584321); rgb(20pt)=(0.181207,0.294918,0.0573313); rgb(21pt)=(0.186001,0.297639,0.0562123); rgb(22pt)=(0.190817,0.300353,0.0550763); rgb(23pt)=(0.195655,0.303061,0.0539242); rgb(24pt)=(0.200514,0.305763,0.052757); rgb(25pt)=(0.205395,0.308459,0.0515759); rgb(26pt)=(0.210296,0.311149,0.0503624); rgb(27pt)=(0.215212,0.313846,0.0490067); rgb(28pt)=(0.220142,0.316548,0.0475043); rgb(29pt)=(0.22509,0.319254,0.0458704); rgb(30pt)=(0.230056,0.321962,0.0441205); rgb(31pt)=(0.235042,0.324671,0.04227); rgb(32pt)=(0.240048,0.327379,0.0403343); rgb(33pt)=(0.245078,0.330085,0.0383287); rgb(34pt)=(0.250131,0.332786,0.0362688); rgb(35pt)=(0.25521,0.335482,0.0341698); rgb(36pt)=(0.260317,0.33817,0.0320472); rgb(37pt)=(0.265451,0.340849,0.0299163); rgb(38pt)=(0.270616,0.343517,0.0277927); rgb(39pt)=(0.275813,0.346172,0.0256916); rgb(40pt)=(0.281043,0.348814,0.0236284); rgb(41pt)=(0.286307,0.35144,0.0216186); rgb(42pt)=(0.291607,0.354048,0.0196776); rgb(43pt)=(0.296945,0.356637,0.0178207); rgb(44pt)=(0.302322,0.359206,0.0160634); rgb(45pt)=(0.307739,0.361753,0.0144211); rgb(46pt)=(0.313198,0.364275,0.0129091); rgb(47pt)=(0.318701,0.366772,0.0115428); rgb(48pt)=(0.324249,0.369242,0.0103377); rgb(49pt)=(0.329843,0.371682,0.00930909); rgb(50pt)=(0.335485,0.374093,0.00847245); rgb(51pt)=(0.341176,0.376471,0.00784314); rgb(52pt)=(0.346925,0.378826,0.00732741); rgb(53pt)=(0.352735,0.381168,0.00682184); rgb(54pt)=(0.358605,0.383497,0.00632729); rgb(55pt)=(0.364532,0.385812,0.00584464); rgb(56pt)=(0.370516,0.388113,0.00537476); rgb(57pt)=(0.376552,0.390399,0.00491852); rgb(58pt)=(0.38264,0.39267,0.00447681); rgb(59pt)=(0.388777,0.394925,0.00405048); rgb(60pt)=(0.394962,0.397164,0.00364042); rgb(61pt)=(0.401191,0.399386,0.00324749); rgb(62pt)=(0.407464,0.401592,0.00287258); rgb(63pt)=(0.413777,0.40378,0.00251655); rgb(64pt)=(0.420129,0.40595,0.00218028); rgb(65pt)=(0.426518,0.408102,0.00186463); rgb(66pt)=(0.432942,0.410234,0.00157049); rgb(67pt)=(0.439399,0.412348,0.00129873); rgb(68pt)=(0.445885,0.414441,0.00105022); rgb(69pt)=(0.452401,0.416515,0.000825833); rgb(70pt)=(0.458942,0.418567,0.000626441); rgb(71pt)=(0.465508,0.420599,0.00045292); rgb(72pt)=(0.472096,0.422609,0.000306141); rgb(73pt)=(0.478704,0.424596,0.000186979); rgb(74pt)=(0.485331,0.426562,9.63073e-05); rgb(75pt)=(0.491973,0.428504,3.49981e-05); rgb(76pt)=(0.498628,0.430422,3.92506e-06); rgb(77pt)=(0.505323,0.432315,0); rgb(78pt)=(0.512206,0.434168,0); rgb(79pt)=(0.519282,0.435983,0); rgb(80pt)=(0.526529,0.437764,0); rgb(81pt)=(0.533922,0.439512,0); rgb(82pt)=(0.54144,0.441232,0); rgb(83pt)=(0.549059,0.442927,0); rgb(84pt)=(0.556756,0.444599,0); rgb(85pt)=(0.564508,0.446252,0); rgb(86pt)=(0.572292,0.447889,0); rgb(87pt)=(0.580084,0.449514,0); rgb(88pt)=(0.587863,0.451129,0); rgb(89pt)=(0.595604,0.452737,0); rgb(90pt)=(0.603284,0.454343,0); rgb(91pt)=(0.610882,0.455948,0); rgb(92pt)=(0.618373,0.457556,0); rgb(93pt)=(0.625734,0.459171,0); rgb(94pt)=(0.632943,0.460795,0); rgb(95pt)=(0.639976,0.462432,0); rgb(96pt)=(0.64681,0.464084,0); rgb(97pt)=(0.653423,0.465756,0); rgb(98pt)=(0.659791,0.46745,0); rgb(99pt)=(0.665891,0.469169,0); rgb(100pt)=(0.6717,0.470916,0); rgb(101pt)=(0.677195,0.472696,0); rgb(102pt)=(0.682353,0.47451,0); rgb(103pt)=(0.687242,0.476355,0); rgb(104pt)=(0.691952,0.478225,0); rgb(105pt)=(0.696497,0.480118,0); rgb(106pt)=(0.700887,0.482033,0); rgb(107pt)=(0.705134,0.483968,0); rgb(108pt)=(0.709251,0.485921,0); rgb(109pt)=(0.713249,0.487891,0); rgb(110pt)=(0.71714,0.489876,0); rgb(111pt)=(0.720936,0.491875,0); rgb(112pt)=(0.724649,0.493887,0); rgb(113pt)=(0.72829,0.495909,0); rgb(114pt)=(0.731872,0.49794,0); rgb(115pt)=(0.735406,0.499979,0); rgb(116pt)=(0.738904,0.502025,0); rgb(117pt)=(0.742378,0.504075,0); rgb(118pt)=(0.74584,0.506128,0); rgb(119pt)=(0.749302,0.508182,0); rgb(120pt)=(0.752775,0.510237,0); rgb(121pt)=(0.756272,0.51229,0); rgb(122pt)=(0.759804,0.514339,0); rgb(123pt)=(0.763384,0.516385,0); rgb(124pt)=(0.767022,0.518424,0); rgb(125pt)=(0.770731,0.520455,0); rgb(126pt)=(0.774523,0.522478,0); rgb(127pt)=(0.77841,0.524489,0); rgb(128pt)=(0.782391,0.526491,0); rgb(129pt)=(0.786402,0.528496,0); rgb(130pt)=(0.790431,0.530506,0); rgb(131pt)=(0.794478,0.532521,0); rgb(132pt)=(0.798541,0.534539,0); rgb(133pt)=(0.802619,0.53656,0); rgb(134pt)=(0.806712,0.538584,0); rgb(135pt)=(0.81082,0.540609,0); rgb(136pt)=(0.81494,0.542635,0); rgb(137pt)=(0.819074,0.54466,0); rgb(138pt)=(0.823219,0.546686,0); rgb(139pt)=(0.827374,0.548709,0); rgb(140pt)=(0.831541,0.55073,0); rgb(141pt)=(0.835716,0.552749,0); rgb(142pt)=(0.8399,0.554763,0); rgb(143pt)=(0.844092,0.556774,0); rgb(144pt)=(0.848292,0.558779,0); rgb(145pt)=(0.852497,0.560778,0); rgb(146pt)=(0.856708,0.562771,0); rgb(147pt)=(0.860924,0.564756,0); rgb(148pt)=(0.865143,0.566733,0); rgb(149pt)=(0.869366,0.568701,0); rgb(150pt)=(0.873592,0.57066,0); rgb(151pt)=(0.877819,0.572608,0); rgb(152pt)=(0.882047,0.574545,0); rgb(153pt)=(0.886275,0.576471,0); rgb(154pt)=(0.890659,0.578362,0); rgb(155pt)=(0.895333,0.580203,0); rgb(156pt)=(0.900258,0.581999,0); rgb(157pt)=(0.905397,0.583755,0); rgb(158pt)=(0.910711,0.585479,0); rgb(159pt)=(0.916164,0.587176,0); rgb(160pt)=(0.921717,0.588852,0); rgb(161pt)=(0.927333,0.590513,0); rgb(162pt)=(0.932974,0.592166,0); rgb(163pt)=(0.938602,0.593815,0); rgb(164pt)=(0.94418,0.595468,0); rgb(165pt)=(0.949669,0.59713,0); rgb(166pt)=(0.955033,0.598808,0); rgb(167pt)=(0.960233,0.600507,0); rgb(168pt)=(0.965232,0.602233,0); rgb(169pt)=(0.969992,0.603992,0); rgb(170pt)=(0.974475,0.605791,0); rgb(171pt)=(0.978643,0.607636,0); rgb(172pt)=(0.98246,0.609532,0); rgb(173pt)=(0.985886,0.611486,0); rgb(174pt)=(0.988885,0.613503,0); rgb(175pt)=(0.991419,0.61559,0); rgb(176pt)=(0.99345,0.617753,0); rgb(177pt)=(0.99494,0.619997,0); rgb(178pt)=(0.995851,0.622329,0); rgb(179pt)=(0.996226,0.624763,0); rgb(180pt)=(0.996512,0.627352,0); rgb(181pt)=(0.996788,0.630095,0); rgb(182pt)=(0.997053,0.632982,0); rgb(183pt)=(0.997308,0.636004,0); rgb(184pt)=(0.997552,0.639152,0); rgb(185pt)=(0.997785,0.642416,0); rgb(186pt)=(0.998006,0.645786,0); rgb(187pt)=(0.998217,0.649253,0); rgb(188pt)=(0.998416,0.652807,0); rgb(189pt)=(0.998605,0.656439,0); rgb(190pt)=(0.998781,0.660138,0); rgb(191pt)=(0.998946,0.663897,0); rgb(192pt)=(0.9991,0.667704,0); rgb(193pt)=(0.999242,0.67155,0); rgb(194pt)=(0.999372,0.675427,0); rgb(195pt)=(0.99949,0.679323,0); rgb(196pt)=(0.999596,0.68323,0); rgb(197pt)=(0.99969,0.687139,0); rgb(198pt)=(0.999771,0.691039,0); rgb(199pt)=(0.999841,0.694921,0); rgb(200pt)=(0.999898,0.698775,0); rgb(201pt)=(0.999942,0.702592,0); rgb(202pt)=(0.999974,0.706363,0); rgb(203pt)=(0.999994,0.710077,0); rgb(204pt)=(1,0.713725,0); rgb(205pt)=(1,0.717341,0); rgb(206pt)=(1,0.720963,0); rgb(207pt)=(1,0.724591,0); rgb(208pt)=(1,0.728226,0); rgb(209pt)=(1,0.731867,0); rgb(210pt)=(1,0.735514,0); rgb(211pt)=(1,0.739167,0); rgb(212pt)=(1,0.742827,0); rgb(213pt)=(1,0.746493,0); rgb(214pt)=(1,0.750165,0); rgb(215pt)=(1,0.753843,0); rgb(216pt)=(1,0.757527,0); rgb(217pt)=(1,0.761217,0); rgb(218pt)=(1,0.764913,0); rgb(219pt)=(1,0.768615,0); rgb(220pt)=(1,0.772324,0); rgb(221pt)=(1,0.776038,0); rgb(222pt)=(1,0.779758,0); rgb(223pt)=(1,0.783484,0); rgb(224pt)=(1,0.787215,0); rgb(225pt)=(1,0.790953,0); rgb(226pt)=(1,0.794696,0); rgb(227pt)=(1,0.798445,0); rgb(228pt)=(1,0.8022,0); rgb(229pt)=(1,0.805961,0); rgb(230pt)=(1,0.809727,0); rgb(231pt)=(1,0.8135,0); rgb(232pt)=(1,0.817278,0); rgb(233pt)=(1,0.821063,0); rgb(234pt)=(1,0.824854,0); rgb(235pt)=(1,0.828652,0); rgb(236pt)=(1,0.832455,0); rgb(237pt)=(1,0.836265,0); rgb(238pt)=(1,0.840081,0); rgb(239pt)=(1,0.843903,0); rgb(240pt)=(1,0.847732,0); rgb(241pt)=(1,0.851566,0); rgb(242pt)=(1,0.855406,0); rgb(243pt)=(1,0.859253,0); rgb(244pt)=(1,0.863106,0); rgb(245pt)=(1,0.866964,0); rgb(246pt)=(1,0.870829,0); rgb(247pt)=(1,0.8747,0); rgb(248pt)=(1,0.878577,0); rgb(249pt)=(1,0.88246,0); rgb(250pt)=(1,0.886349,0); rgb(251pt)=(1,0.890243,0); rgb(252pt)=(1,0.894144,0); rgb(253pt)=(1,0.898051,0); rgb(254pt)=(1,0.901964,0); rgb(255pt)=(1,0.905882,0)},
mesh/rows=49]
table[row sep=crcr,header=false] {%
%
56	0.093	-0.774334427159586\\
56	0.09666	-0.624731683335815\\
56	0.10032	-0.449584953080116\\
56	0.10398	-0.248894236392514\\
56	0.10764	-0.0226595332730053\\
56	0.1113	0.229119156278419\\
56	0.11496	0.506441832261782\\
56	0.11862	0.80930849467703\\
56	0.12228	1.1377191435242\\
56	0.12594	1.49167377880329\\
56	0.1296	1.87117240051427\\
56	0.13326	2.27621500865718\\
56	0.13692	2.706801603232\\
56	0.14058	3.16293218423874\\
56	0.14424	3.64460675167738\\
56	0.1479	4.15182530554794\\
56	0.15156	4.68458784585042\\
56	0.15522	5.24289437258479\\
56	0.15888	5.82674488575109\\
56	0.16254	6.43613938534931\\
56	0.1662	7.07107787137943\\
56	0.16986	7.73156034384146\\
56	0.17352	8.41758680273542\\
56	0.17718	9.12915724806128\\
56	0.18084	9.86627167981906\\
56	0.1845	10.6289300980087\\
56	0.18816	11.4171325026303\\
56	0.19182	12.2308788936839\\
56	0.19548	13.0701692711693\\
56	0.19914	13.9350036350866\\
56	0.2028	14.8253819854359\\
56	0.20646	15.7413043222171\\
56	0.21012	16.6827706454301\\
56	0.21378	17.6497809550751\\
56	0.21744	18.642335251152\\
56	0.2211	19.6604335336609\\
56	0.22476	20.7040758026016\\
56	0.22842	21.7732620579742\\
56	0.23208	22.8679922997788\\
56	0.23574	23.9882665280153\\
56	0.2394	25.1340847426837\\
56	0.24306	26.3054469437839\\
56	0.24672	27.5023531313161\\
56	0.25038	28.7248033052803\\
56	0.25404	29.9727974656763\\
56	0.2577	31.2463356125043\\
56	0.26136	32.5454177457641\\
56	0.26502	33.8700438654559\\
56	0.26868	35.2202139715796\\
56	0.27234	36.5959280641352\\
56	0.276	37.9971861431227\\
56.375	0.093	-0.89511921239191\\
56.375	0.09666	-0.738288001408483\\
56.375	0.10032	-0.555912803993127\\
56.375	0.10398	-0.347993620145862\\
56.375	0.10764	-0.114530449866697\\
56.375	0.1113	0.144476706844383\\
56.375	0.11496	0.429027849987403\\
56.375	0.11862	0.739122979562307\\
56.375	0.12228	1.07476209556914\\
56.375	0.12594	1.43594519800788\\
56.375	0.1296	1.82267228687852\\
56.375	0.13326	2.23494336218108\\
56.375	0.13692	2.67275842391557\\
56.375	0.14058	3.13611747208196\\
56.375	0.14424	3.62502050668026\\
56.375	0.1479	4.13946752771048\\
56.375	0.15156	4.67945853517261\\
56.375	0.15522	5.24499352906665\\
56.375	0.15888	5.83607250939261\\
56.375	0.16254	6.45269547615048\\
56.375	0.1662	7.09486242934026\\
56.375	0.16986	7.76257336896195\\
56.375	0.17352	8.45582829501556\\
56.375	0.17718	9.17462720750109\\
56.375	0.18084	9.91897010641852\\
56.375	0.1845	10.6888569917679\\
56.375	0.18816	11.4842878635491\\
56.375	0.19182	12.3052627217623\\
56.375	0.19548	13.1517815664074\\
56.375	0.19914	14.0238443974844\\
56.375	0.2028	14.9214512149933\\
56.375	0.20646	15.8446020189341\\
56.375	0.21012	16.7932968093069\\
56.375	0.21378	17.7675355861115\\
56.375	0.21744	18.7673183493481\\
56.375	0.2211	19.7926450990166\\
56.375	0.22476	20.8435158351169\\
56.375	0.22842	21.9199305576492\\
56.375	0.23208	23.0218892666134\\
56.375	0.23574	24.1493919620096\\
56.375	0.2394	25.3024386438376\\
56.375	0.24306	26.4810293120976\\
56.375	0.24672	27.6851639667895\\
56.375	0.25038	28.9148426079132\\
56.375	0.25404	30.1700652354689\\
56.375	0.2577	31.4508318494565\\
56.375	0.26136	32.757142449876\\
56.375	0.26502	34.0889970367275\\
56.375	0.26868	35.4463956100108\\
56.375	0.27234	36.8293381697261\\
56.375	0.276	38.2378247158733\\
56.75	0.093	-1.01353532002414\\
56.75	0.09666	-0.849475641881041\\
56.75	0.10032	-0.659871977306043\\
56.75	0.10398	-0.444724326299129\\
56.75	0.10764	-0.204032688860279\\
56.75	0.1113	0.0622029350104576\\
56.75	0.11496	0.353982545313105\\
56.75	0.11862	0.67130614204768\\
56.75	0.12228	1.01417372521417\\
56.75	0.12594	1.38258529481255\\
56.75	0.1296	1.77654085084288\\
56.75	0.13326	2.1960403933051\\
56.75	0.13692	2.64108392219922\\
56.75	0.14058	3.11167143752527\\
56.75	0.14424	3.60780293928326\\
56.75	0.1479	4.12947842747312\\
56.75	0.15156	4.67669790209491\\
56.75	0.15522	5.2494613631486\\
56.75	0.15888	5.84776881063422\\
56.75	0.16254	6.47162024455175\\
56.75	0.1662	7.12101566490119\\
56.75	0.16986	7.79595507168253\\
56.75	0.17352	8.4964384648958\\
56.75	0.17718	9.22246584454098\\
56.75	0.18084	9.97403721061808\\
56.75	0.1845	10.7511525631271\\
56.75	0.18816	11.553811902068\\
56.75	0.19182	12.3820152274408\\
56.75	0.19548	13.2357625392456\\
56.75	0.19914	14.1150538374822\\
56.75	0.2028	15.0198891221508\\
56.75	0.20646	15.9502683932513\\
56.75	0.21012	16.9061916507837\\
56.75	0.21378	17.887658894748\\
56.75	0.21744	18.8946701251442\\
56.75	0.2211	19.9272253419723\\
56.75	0.22476	20.9853245452324\\
56.75	0.22842	22.0689677349243\\
56.75	0.23208	23.1781549110482\\
56.75	0.23574	24.312886073604\\
56.75	0.2394	25.4731612225917\\
56.75	0.24306	26.6589803580113\\
56.75	0.24672	27.8703434798628\\
56.75	0.25038	29.1072505881463\\
56.75	0.25404	30.3697016828617\\
56.75	0.2577	31.6576967640089\\
56.75	0.26136	32.9712358315881\\
56.75	0.26502	34.3103188855992\\
56.75	0.26868	35.6749459260421\\
56.75	0.27234	37.0651169529171\\
56.75	0.276	38.4808319662239\\
57.125	0.093	-1.12958275005631\\
57.125	0.09666	-0.958294604753545\\
57.125	0.10032	-0.761462473018884\\
57.125	0.10398	-0.539086354852321\\
57.125	0.10764	-0.291166250253815\\
57.125	0.1113	-0.0177021592234148\\
57.125	0.11496	0.281305918238889\\
57.125	0.11862	0.605857982133127\\
57.125	0.12228	0.955954032459271\\
57.125	0.12594	1.33159406921731\\
57.125	0.1296	1.73277809240729\\
57.125	0.13326	2.15950610202917\\
57.125	0.13692	2.61177809808295\\
57.125	0.14058	3.08959408056866\\
57.125	0.14424	3.59295404948631\\
57.125	0.1479	4.12185800483582\\
57.125	0.15156	4.67630594661727\\
57.125	0.15522	5.25629787483062\\
57.125	0.15888	5.8618337894759\\
57.125	0.16254	6.49291369055308\\
57.125	0.1662	7.14953757806218\\
57.125	0.16986	7.83170545200318\\
57.125	0.17352	8.53941731237611\\
57.125	0.17718	9.27267315918095\\
57.125	0.18084	10.0314729924177\\
57.125	0.1845	10.8158168120864\\
57.125	0.18816	11.6257046181869\\
57.125	0.19182	12.4611364107194\\
57.125	0.19548	13.3221121896838\\
57.125	0.19914	14.2086319550801\\
57.125	0.2028	15.1206957069084\\
57.125	0.20646	16.0583034451685\\
57.125	0.21012	17.0214551698606\\
57.125	0.21378	18.0101508809845\\
57.125	0.21744	19.0243905785404\\
57.125	0.2211	20.0641742625282\\
57.125	0.22476	21.1295019329479\\
57.125	0.22842	22.2203735897995\\
57.125	0.23208	23.336789233083\\
57.125	0.23574	24.4787488627985\\
57.125	0.2394	25.6462524789459\\
57.125	0.24306	26.8393000815251\\
57.125	0.24672	28.0578916705363\\
57.125	0.25038	29.3020272459794\\
57.125	0.25404	30.5717068078544\\
57.125	0.2577	31.8669303561614\\
57.125	0.26136	33.1876978909002\\
57.125	0.26502	34.5340094120709\\
57.125	0.26868	35.9058649196736\\
57.125	0.27234	37.3032644137081\\
57.125	0.276	38.7262078941746\\
57.5	0.093	-1.24326150248837\\
57.5	0.09666	-1.06474489002598\\
57.5	0.10032	-0.860684291131644\\
57.5	0.10398	-0.631079705805403\\
57.5	0.10764	-0.375931134047269\\
57.5	0.1113	-0.0952385758572127\\
57.5	0.11496	0.210997968764783\\
57.5	0.11862	0.542778499818656\\
57.5	0.12228	0.900103017304456\\
57.5	0.12594	1.28297152122218\\
57.5	0.1296	1.69138401157178\\
57.5	0.13326	2.12534048835333\\
57.5	0.13692	2.58484095156678\\
57.5	0.14058	3.06988540121215\\
57.5	0.14424	3.58047383728942\\
57.5	0.1479	4.11660625979862\\
57.5	0.15156	4.67828266873972\\
57.5	0.15522	5.26550306411272\\
57.5	0.15888	5.87826744591766\\
57.5	0.16254	6.5165758141545\\
57.5	0.1662	7.18042816882327\\
57.5	0.16986	7.86982450992392\\
57.5	0.17352	8.58476483745651\\
57.5	0.17718	9.325249151421\\
57.5	0.18084	10.0912774518174\\
57.5	0.1845	10.8828497386457\\
57.5	0.18816	11.699966011906\\
57.5	0.19182	12.5426262715981\\
57.5	0.19548	13.4108305177222\\
57.5	0.19914	14.3045787502781\\
57.5	0.2028	15.223870969266\\
57.5	0.20646	16.1687071746858\\
57.5	0.21012	17.1390873665375\\
57.5	0.21378	18.1350115448212\\
57.5	0.21744	19.1564797095367\\
57.5	0.2211	20.2034918606841\\
57.5	0.22476	21.2760479982635\\
57.5	0.22842	22.3741481222748\\
57.5	0.23208	23.497792232718\\
57.5	0.23574	24.6469803295931\\
57.5	0.2394	25.8217124129001\\
57.5	0.24306	27.021988482639\\
57.5	0.24672	28.2478085388099\\
57.5	0.25038	29.4991725814126\\
57.5	0.25404	30.7760806104473\\
57.5	0.2577	32.0785326259139\\
57.5	0.26136	33.4065286278123\\
57.5	0.26502	34.7600686161428\\
57.5	0.26868	36.1391525909051\\
57.5	0.27234	37.5437805520993\\
57.5	0.276	38.9739524997255\\
57.875	0.093	-1.35457157732037\\
57.875	0.09666	-1.16882649769831\\
57.875	0.10032	-0.957537431644319\\
57.875	0.10398	-0.720704379158429\\
57.875	0.10764	-0.458327340240631\\
57.875	0.1113	-0.170406314890919\\
57.875	0.11496	0.143058696890733\\
57.875	0.11862	0.482067695104263\\
57.875	0.12228	0.846620679749726\\
57.875	0.12594	1.2367176508271\\
57.875	0.1296	1.65235860833637\\
57.875	0.13326	2.09354355227757\\
57.875	0.13692	2.56027248265068\\
57.875	0.14058	3.05254539945571\\
57.875	0.14424	3.57036230269264\\
57.875	0.1479	4.11372319236148\\
57.875	0.15156	4.68262806846225\\
57.875	0.15522	5.27707693099492\\
57.875	0.15888	5.89706977995951\\
57.875	0.16254	6.54260661535601\\
57.875	0.1662	7.21368743718443\\
57.875	0.16986	7.91031224544474\\
57.875	0.17352	8.63248104013698\\
57.875	0.17718	9.38019382126114\\
57.875	0.18084	10.1534505888172\\
57.875	0.1845	10.9522513428052\\
57.875	0.18816	11.7765960832251\\
57.875	0.19182	12.6264848100769\\
57.875	0.19548	13.5019175233606\\
57.875	0.19914	14.4028942230762\\
57.875	0.2028	15.3294149092238\\
57.875	0.20646	16.2814795818032\\
57.875	0.21012	17.2590882408146\\
57.875	0.21378	18.2622408862579\\
57.875	0.21744	19.2909375181331\\
57.875	0.2211	20.3451781364402\\
57.875	0.22476	21.4249627411792\\
57.875	0.22842	22.5302913323501\\
57.875	0.23208	23.661163909953\\
57.875	0.23574	24.8175804739877\\
57.875	0.2394	25.9995410244544\\
57.875	0.24306	27.207045561353\\
57.875	0.24672	28.4400940846835\\
57.875	0.25038	29.6986865944459\\
57.875	0.25404	30.9828230906402\\
57.875	0.2577	32.2925035732665\\
57.875	0.26136	33.6277280423246\\
57.875	0.26502	34.9884964978147\\
57.875	0.26868	36.3748089397366\\
57.875	0.27234	37.7866653680905\\
57.875	0.276	39.2240657828764\\
58.25	0.093	-1.46351297455227\\
58.25	0.09666	-1.27053942777054\\
58.25	0.10032	-1.0520218945569\\
58.25	0.10398	-0.807960374911373\\
58.25	0.10764	-0.538354868833883\\
58.25	0.1113	-0.243205376324514\\
58.25	0.11496	0.0774881026167655\\
58.25	0.11862	0.423725567989973\\
58.25	0.12228	0.795507019795085\\
58.25	0.12594	1.1928324580321\\
58.25	0.1296	1.61570188270106\\
58.25	0.13326	2.0641152938019\\
58.25	0.13692	2.53807269133466\\
58.25	0.14058	3.03757407529935\\
58.25	0.14424	3.56261944569596\\
58.25	0.1479	4.11320880252445\\
58.25	0.15156	4.68934214578487\\
58.25	0.15522	5.2910194754772\\
58.25	0.15888	5.91824079160144\\
58.25	0.16254	6.57100609415761\\
58.25	0.1662	7.24931538314568\\
58.25	0.16986	7.95316865856566\\
58.25	0.17352	8.68256592041755\\
58.25	0.17718	9.43750716870136\\
58.25	0.18084	10.2179924034171\\
58.25	0.1845	11.0240216245647\\
58.25	0.18816	11.8555948321443\\
58.25	0.19182	12.7127120261557\\
58.25	0.19548	13.5953732065991\\
58.25	0.19914	14.5035783734744\\
58.25	0.2028	15.4373275267816\\
58.25	0.20646	16.3966206665207\\
58.25	0.21012	17.3814577926918\\
58.25	0.21378	18.3918389052947\\
58.25	0.21744	19.4277640043295\\
58.25	0.2211	20.4892330897963\\
58.25	0.22476	21.576246161695\\
58.25	0.22842	22.6888032200256\\
58.25	0.23208	23.8269042647881\\
58.25	0.23574	24.9905492959825\\
58.25	0.2394	26.1797383136088\\
58.25	0.24306	27.394471317667\\
58.25	0.24672	28.6347483081572\\
58.25	0.25038	29.9005692850793\\
58.25	0.25404	31.1919342484333\\
58.25	0.2577	32.5088431982192\\
58.25	0.26136	33.851296134437\\
58.25	0.26502	35.2192930570867\\
58.25	0.26868	36.6128339661683\\
58.25	0.27234	38.0319188616819\\
58.25	0.276	39.4765477436273\\
58.625	0.093	-1.57008569418409\\
58.625	0.09666	-1.36988368024271\\
58.625	0.10032	-1.14413767986942\\
58.625	0.10398	-0.892847693064224\\
58.625	0.10764	-0.616013719827086\\
58.625	0.1113	-0.313635760158054\\
58.625	0.11496	0.0142861859428827\\
58.625	0.11862	0.367752118475746\\
58.625	0.12228	0.746762037440522\\
58.625	0.12594	1.1513159428372\\
58.625	0.1296	1.58141383466581\\
58.625	0.13326	2.03705571292631\\
58.625	0.13692	2.51824157761872\\
58.625	0.14058	3.02497142874306\\
58.625	0.14424	3.55724526629934\\
58.625	0.1479	4.11506309028749\\
58.625	0.15156	4.69842490070756\\
58.625	0.15522	5.30733069755955\\
58.625	0.15888	5.94178048084346\\
58.625	0.16254	6.60177425055928\\
58.625	0.1662	7.28731200670701\\
58.625	0.16986	7.99839374928664\\
58.625	0.17352	8.7350194782982\\
58.625	0.17718	9.49718919374167\\
58.625	0.18084	10.284902895617\\
58.625	0.1845	11.0981605839243\\
58.625	0.18816	11.9369622586635\\
58.625	0.19182	12.8013079198347\\
58.625	0.19548	13.6911975674377\\
58.625	0.19914	14.6066312014726\\
58.625	0.2028	15.5476088219395\\
58.625	0.20646	16.5141304288383\\
58.625	0.21012	17.506196022169\\
58.625	0.21378	18.5238056019316\\
58.625	0.21744	19.5669591681261\\
58.625	0.2211	20.6356567207525\\
58.625	0.22476	21.7298982598108\\
58.625	0.22842	22.8496837853011\\
58.625	0.23208	23.9950132972232\\
58.625	0.23574	25.1658867955773\\
58.625	0.2394	26.3623042803633\\
58.625	0.24306	27.5842657515812\\
58.625	0.24672	28.831771209231\\
58.625	0.25038	30.1048206533127\\
58.625	0.25404	31.4034140838264\\
58.625	0.2577	32.7275515007719\\
58.625	0.26136	34.0772329041494\\
58.625	0.26502	35.4524582939588\\
58.625	0.26868	36.8532276702\\
58.625	0.27234	38.2795410328733\\
58.625	0.276	39.7313983819784\\
59	0.093	-1.67428973621583\\
59	0.09666	-1.46685925511481\\
59	0.10032	-1.23388478758184\\
59	0.10398	-0.97536633361697\\
59	0.10764	-0.691303893220203\\
59	0.1113	-0.381697466391515\\
59	0.11496	-0.0465470531308938\\
59	0.11862	0.314147346561612\\
59	0.12228	0.700385732686044\\
59	0.12594	1.1121681052424\\
59	0.1296	1.54949446423064\\
59	0.13326	2.01236480965081\\
59	0.13692	2.50077914150289\\
59	0.14058	3.0147374597869\\
59	0.14424	3.5542397645028\\
59	0.1479	4.11928605565062\\
59	0.15156	4.70987633323036\\
59	0.15522	5.326010597242\\
59	0.15888	5.96768884768556\\
59	0.16254	6.63491108456103\\
59	0.1662	7.32767730786843\\
59	0.16986	8.04598751760771\\
59	0.17352	8.78984171377893\\
59	0.17718	9.55923989638205\\
59	0.18084	10.3541820654171\\
59	0.1845	11.174668220884\\
59	0.18816	12.0206983627829\\
59	0.19182	12.8922724911137\\
59	0.19548	13.7893906058764\\
59	0.19914	14.712052707071\\
59	0.2028	15.6602587946975\\
59	0.20646	16.6340088687559\\
59	0.21012	17.6333029292463\\
59	0.21378	18.6581409761685\\
59	0.21744	19.7085230095227\\
59	0.2211	20.7844490293088\\
59	0.22476	21.8859190355268\\
59	0.22842	23.0129330281767\\
59	0.23208	24.1654910072585\\
59	0.23574	25.3435929727722\\
59	0.2394	26.5472389247179\\
59	0.24306	27.7764288630954\\
59	0.24672	29.0311627879049\\
59	0.25038	30.3114406991463\\
59	0.25404	31.6172625968196\\
59	0.2577	32.9486284809248\\
59	0.26136	34.3055383514619\\
59	0.26502	35.687992208431\\
59	0.26868	37.0959900518319\\
59	0.27234	38.5295318816648\\
59	0.276	39.9886176979295\\
59.375	0.093	-1.7761251006475\\
59.375	0.09666	-1.56146615238682\\
59.375	0.10032	-1.3212632176942\\
59.375	0.10398	-1.05551629656967\\
59.375	0.10764	-0.764225389013236\\
59.375	0.1113	-0.447390495024891\\
59.375	0.11496	-0.105011614604614\\
59.375	0.11862	0.262911252247555\\
59.375	0.12228	0.656378105531644\\
59.375	0.12594	1.07538894524765\\
59.375	0.1296	1.51994377139555\\
59.375	0.13326	1.99004258397538\\
59.375	0.13692	2.48568538298712\\
59.375	0.14058	3.00687216843078\\
59.375	0.14424	3.55360294030634\\
59.375	0.1479	4.12587769861382\\
59.375	0.15156	4.72369644335322\\
59.375	0.15522	5.34705917452451\\
59.375	0.15888	5.99596589212774\\
59.375	0.16254	6.67041659616287\\
59.375	0.1662	7.37041128662992\\
59.375	0.16986	8.09594996352886\\
59.375	0.17352	8.84703262685974\\
59.375	0.17718	9.62365927662252\\
59.375	0.18084	10.4258299128172\\
59.375	0.1845	11.2535445354438\\
59.375	0.18816	12.1068031445023\\
59.375	0.19182	12.9856057399928\\
59.375	0.19548	13.8899523219151\\
59.375	0.19914	14.8198428902694\\
59.375	0.2028	15.7752774450556\\
59.375	0.20646	16.7562559862737\\
59.375	0.21012	17.7627785139237\\
59.375	0.21378	18.7948450280056\\
59.375	0.21744	19.8524555285194\\
59.375	0.2211	20.9356100154651\\
59.375	0.22476	22.0443084888428\\
59.375	0.22842	23.1785509486523\\
59.375	0.23208	24.3383373948938\\
59.375	0.23574	25.5236678275672\\
59.375	0.2394	26.7345422466725\\
59.375	0.24306	27.9709606522097\\
59.375	0.24672	29.2329230441789\\
59.375	0.25038	30.5204294225799\\
59.375	0.25404	31.8334797874129\\
59.375	0.2577	33.1720741386777\\
59.375	0.26136	34.5362124763745\\
59.375	0.26502	35.9258948005032\\
59.375	0.26868	37.3411211110638\\
59.375	0.27234	38.7818914080563\\
59.375	0.276	40.2482056914808\\
59.75	0.093	-1.87559178747907\\
59.75	0.09666	-1.65370437205871\\
59.75	0.10032	-1.40627297020644\\
59.75	0.10398	-1.13329758192227\\
59.75	0.10764	-0.834778207206165\\
59.75	0.1113	-0.510714846058157\\
59.75	0.11496	-0.161107498478252\\
59.75	0.11862	0.214043835533587\\
59.75	0.12228	0.614739155977333\\
59.75	0.12594	1.04097846285298\\
59.75	0.1296	1.49276175616057\\
59.75	0.13326	1.97008903590004\\
59.75	0.13692	2.47296030207143\\
59.75	0.14058	3.00137555467475\\
59.75	0.14424	3.55533479371\\
59.75	0.1479	4.13483801917711\\
59.75	0.15156	4.73988523107617\\
59.75	0.15522	5.37047642940712\\
59.75	0.15888	6.02661161417\\
59.75	0.16254	6.7082907853648\\
59.75	0.1662	7.4155139429915\\
59.75	0.16986	8.1482810870501\\
59.75	0.17352	8.90659221754063\\
59.75	0.17718	9.69044733446308\\
59.75	0.18084	10.4998464378174\\
59.75	0.1845	11.3347895276037\\
59.75	0.18816	12.1952766038219\\
59.75	0.19182	13.081307666472\\
59.75	0.19548	13.992882715554\\
59.75	0.19914	14.9300017510679\\
59.75	0.2028	15.8926647730137\\
59.75	0.20646	16.8808717813915\\
59.75	0.21012	17.8946227762011\\
59.75	0.21378	18.9339177574427\\
59.75	0.21744	19.9987567251162\\
59.75	0.2211	21.0891396792216\\
59.75	0.22476	22.2050666197589\\
59.75	0.22842	23.3465375467281\\
59.75	0.23208	24.5135524601292\\
59.75	0.23574	25.7061113599623\\
59.75	0.2394	26.9242142462273\\
59.75	0.24306	28.1678611189241\\
59.75	0.24672	29.4370519780529\\
59.75	0.25038	30.7317868236136\\
59.75	0.25404	32.0520656556063\\
59.75	0.2577	33.3978884740308\\
59.75	0.26136	34.7692552788872\\
59.75	0.26502	36.1661660701756\\
59.75	0.26868	37.5886208478958\\
59.75	0.27234	39.036619612048\\
59.75	0.276	40.5101623626321\\
60.125	0.093	-1.97268979671057\\
60.125	0.09666	-1.74357391413055\\
60.125	0.10032	-1.48891404511862\\
60.125	0.10398	-1.2087101896748\\
60.125	0.10764	-0.902962347799031\\
60.125	0.1113	-0.571670519491374\\
60.125	0.11496	-0.214834704751805\\
60.125	0.11862	0.167545096419691\\
60.125	0.12228	0.575468884023099\\
60.125	0.12594	1.00893665805841\\
60.125	0.1296	1.46794841852565\\
60.125	0.13326	1.95250416542478\\
60.125	0.13692	2.46260389875582\\
60.125	0.14058	2.9982476185188\\
60.125	0.14424	3.5594353247137\\
60.125	0.1479	4.14616701734048\\
60.125	0.15156	4.75844269639919\\
60.125	0.15522	5.39626236188981\\
60.125	0.15888	6.05962601381234\\
60.125	0.16254	6.7485336521668\\
60.125	0.1662	7.46298527695316\\
60.125	0.16986	8.20298088817141\\
60.125	0.17352	8.9685204858216\\
60.125	0.17718	9.75960406990371\\
60.125	0.18084	10.5762316404177\\
60.125	0.1845	11.4184031973636\\
60.125	0.18816	12.2861187407415\\
60.125	0.19182	13.1793782705512\\
60.125	0.19548	14.0981817867929\\
60.125	0.19914	15.0425292894665\\
60.125	0.2028	16.012420778572\\
60.125	0.20646	17.0078562541094\\
60.125	0.21012	18.0288357160787\\
60.125	0.21378	19.0753591644799\\
60.125	0.21744	20.147426599313\\
60.125	0.2211	21.2450380205781\\
60.125	0.22476	22.3681934282751\\
60.125	0.22842	23.516892822404\\
60.125	0.23208	24.6911362029647\\
60.125	0.23574	25.8909235699575\\
60.125	0.2394	27.1162549233821\\
60.125	0.24306	28.3671302632386\\
60.125	0.24672	29.643549589527\\
60.125	0.25038	30.9455129022474\\
60.125	0.25404	32.2730202013997\\
60.125	0.2577	33.6260714869839\\
60.125	0.26136	35.0046667589999\\
60.125	0.26502	36.4088060174479\\
60.125	0.26868	37.8384892623279\\
60.125	0.27234	39.2937164936397\\
60.125	0.276	40.7744877113835\\
60.5	0.093	-2.06741912834199\\
60.5	0.09666	-1.83107477860231\\
60.5	0.10032	-1.56918644243074\\
60.5	0.10398	-1.28175411982725\\
60.5	0.10764	-0.968777810791819\\
60.5	0.1113	-0.630257515324505\\
60.5	0.11496	-0.26619323342528\\
60.5	0.11862	0.123415034905872\\
60.5	0.12228	0.538567289668936\\
60.5	0.12594	0.979263530863907\\
60.5	0.1296	1.44550375849081\\
60.5	0.13326	1.9372879725496\\
60.5	0.13692	2.4546161730403\\
60.5	0.14058	2.99748835996293\\
60.5	0.14424	3.56590453331749\\
60.5	0.1479	4.15986469310393\\
60.5	0.15156	4.7793688393223\\
60.5	0.15522	5.42441697197257\\
60.5	0.15888	6.09500909105476\\
60.5	0.16254	6.79114519656887\\
60.5	0.1662	7.51282528851489\\
60.5	0.16986	8.26004936689281\\
60.5	0.17352	9.03281743170266\\
60.5	0.17718	9.83112948294442\\
60.5	0.18084	10.6549855206181\\
60.5	0.1845	11.5043855447237\\
60.5	0.18816	12.3793295552612\\
60.5	0.19182	13.2798175522306\\
60.5	0.19548	14.2058495356319\\
60.5	0.19914	15.1574255054651\\
60.5	0.2028	16.1345454617303\\
60.5	0.20646	17.1372094044273\\
60.5	0.21012	18.1654173335563\\
60.5	0.21378	19.2191692491172\\
60.5	0.21744	20.29846515111\\
60.5	0.2211	21.4033050395347\\
60.5	0.22476	22.5336889143913\\
60.5	0.22842	23.6896167756798\\
60.5	0.23208	24.8710886234003\\
60.5	0.23574	26.0781044575527\\
60.5	0.2394	27.310664278137\\
60.5	0.24306	28.5687680851532\\
60.5	0.24672	29.8524158786013\\
60.5	0.25038	31.1616076584813\\
60.5	0.25404	32.4963434247932\\
60.5	0.2577	33.856623177537\\
60.5	0.26136	35.2424469167128\\
60.5	0.26502	36.6538146423205\\
60.5	0.26868	38.09072635436\\
60.5	0.27234	39.5531820528315\\
60.5	0.276	41.041181737735\\
60.875	0.093	-2.15977978237331\\
60.875	0.09666	-1.91620696547399\\
60.875	0.10032	-1.64709016214274\\
60.875	0.10398	-1.35242937237958\\
60.875	0.10764	-1.03222459618453\\
60.875	0.1113	-0.686475833557552\\
60.875	0.11496	-0.315183084498642\\
60.875	0.11862	0.081653650992159\\
60.875	0.12228	0.50403437291488\\
60.875	0.12594	0.951959081269514\\
60.875	0.1296	1.42542777605605\\
60.875	0.13326	1.92444045727451\\
60.875	0.13692	2.44899712492488\\
60.875	0.14058	2.99909777900718\\
60.875	0.14424	3.57474241952136\\
60.875	0.1479	4.17593104646747\\
60.875	0.15156	4.8026636598455\\
60.875	0.15522	5.45494025965543\\
60.875	0.15888	6.13276084589728\\
60.875	0.16254	6.83612541857104\\
60.875	0.1662	7.56503397767673\\
60.875	0.16986	8.3194865232143\\
60.875	0.17352	9.0994830551838\\
60.875	0.17718	9.90502357358522\\
60.875	0.18084	10.7361080784186\\
60.875	0.1845	11.5927365696838\\
60.875	0.18816	12.4749090473809\\
60.875	0.19182	13.38262551151\\
60.875	0.19548	14.315885962071\\
60.875	0.19914	15.2746903990639\\
60.875	0.2028	16.2590388224887\\
60.875	0.20646	17.2689312323454\\
60.875	0.21012	18.304367628634\\
60.875	0.21378	19.3653480113546\\
60.875	0.21744	20.451872380507\\
60.875	0.2211	21.5639407360914\\
60.875	0.22476	22.7015530781077\\
60.875	0.22842	23.8647094065559\\
60.875	0.23208	25.053409721436\\
60.875	0.23574	26.267654022748\\
60.875	0.2394	27.507442310492\\
60.875	0.24306	28.7727745846678\\
60.875	0.24672	30.0636508452755\\
60.875	0.25038	31.3800710923152\\
60.875	0.25404	32.7220353257868\\
60.875	0.2577	34.0895435456903\\
60.875	0.26136	35.4825957520257\\
60.875	0.26502	36.9011919447931\\
60.875	0.26868	38.3453321239923\\
60.875	0.27234	39.8150162896234\\
60.875	0.276	41.3102444416865\\
61.25	0.093	-2.24977175880457\\
61.25	0.09666	-1.9989704747456\\
61.25	0.10032	-1.72262520425469\\
61.25	0.10398	-1.42073594733187\\
61.25	0.10764	-1.09330270397716\\
61.25	0.1113	-0.740325474190524\\
61.25	0.11496	-0.361804257971951\\
61.25	0.11862	0.0422609446784996\\
61.25	0.12228	0.471870133760884\\
61.25	0.12594	0.927023309275175\\
61.25	0.1296	1.40772047122136\\
61.25	0.13326	1.91396161959948\\
61.25	0.13692	2.44574675440951\\
61.25	0.14058	3.00307587565146\\
61.25	0.14424	3.58594898332531\\
61.25	0.1479	4.19436607743108\\
61.25	0.15156	4.82832715796876\\
61.25	0.15522	5.48783222493835\\
61.25	0.15888	6.17288127833985\\
61.25	0.16254	6.88347431817328\\
61.25	0.1662	7.61961134443861\\
61.25	0.16986	8.38129235713585\\
61.25	0.17352	9.16851735626501\\
61.25	0.17718	9.98128634182608\\
61.25	0.18084	10.8195993138191\\
61.25	0.1845	11.683456272244\\
61.25	0.18816	12.5728572171008\\
61.25	0.19182	13.4878021483895\\
61.25	0.19548	14.4282910661101\\
61.25	0.19914	15.3943239702627\\
61.25	0.2028	16.3859008608471\\
61.25	0.20646	17.4030217378635\\
61.25	0.21012	18.4456866013118\\
61.25	0.21378	19.513895451192\\
61.25	0.21744	20.6076482875041\\
61.25	0.2211	21.7269451102482\\
61.25	0.22476	22.8717859194241\\
61.25	0.22842	24.0421707150319\\
61.25	0.23208	25.2380994970717\\
61.25	0.23574	26.4595722655434\\
61.25	0.2394	27.706589020447\\
61.25	0.24306	28.9791497617825\\
61.25	0.24672	30.2772544895499\\
61.25	0.25038	31.6009032037492\\
61.25	0.25404	32.9500959043805\\
61.25	0.2577	34.3248325914437\\
61.25	0.26136	35.7251132649387\\
61.25	0.26502	37.1509379248657\\
61.25	0.26868	38.6023065712246\\
61.25	0.27234	40.0792192040154\\
61.25	0.276	41.5816758232381\\
61.625	0.093	-2.33739505763572\\
61.625	0.09666	-2.07936530641708\\
61.625	0.10032	-1.79579156876652\\
61.625	0.10398	-1.48667384468407\\
61.625	0.10764	-1.15201213416966\\
61.625	0.1113	-0.791806437223375\\
61.625	0.11496	-0.406056753845174\\
61.625	0.11862	0.00523691596494658\\
61.625	0.12228	0.442074572206987\\
61.625	0.12594	0.904456214880927\\
61.625	0.1296	1.3923818439868\\
61.625	0.13326	1.90585145952456\\
61.625	0.13692	2.44486506149424\\
61.625	0.14058	3.00942264989584\\
61.625	0.14424	3.59952422472938\\
61.625	0.1479	4.21516978599479\\
61.625	0.15156	4.85635933369213\\
61.625	0.15522	5.52309286782137\\
61.625	0.15888	6.21537038838254\\
61.625	0.16254	6.93319189537563\\
61.625	0.1662	7.67655738880061\\
61.625	0.16986	8.44546686865751\\
61.625	0.17352	9.23992033494632\\
61.625	0.17718	10.0599177876671\\
61.625	0.18084	10.9054592268197\\
61.625	0.1845	11.7765446524043\\
61.625	0.18816	12.6731740644207\\
61.625	0.19182	13.5953474628691\\
61.625	0.19548	14.5430648477494\\
61.625	0.19914	15.5163262190616\\
61.625	0.2028	16.5151315768057\\
61.625	0.20646	17.5394809209818\\
61.625	0.21012	18.5893742515897\\
61.625	0.21378	19.6648115686296\\
61.625	0.21744	20.7657928721013\\
61.625	0.2211	21.892318162005\\
61.625	0.22476	23.0443874383406\\
61.625	0.22842	24.2220007011081\\
61.625	0.23208	25.4251579503075\\
61.625	0.23574	26.6538591859389\\
61.625	0.2394	27.9081044080021\\
61.625	0.24306	29.1878936164973\\
61.625	0.24672	30.4932268114244\\
61.625	0.25038	31.8241039927834\\
61.625	0.25404	33.1805251605743\\
61.625	0.2577	34.5624903147971\\
61.625	0.26136	35.9699994554518\\
61.625	0.26502	37.4030525825385\\
61.625	0.26868	38.861649696057\\
61.625	0.27234	40.3457907960075\\
61.625	0.276	41.8554758823899\\
62	0.093	-2.42264967886682\\
62	0.09666	-2.15739146048851\\
62	0.10032	-1.8665892556783\\
62	0.10398	-1.55024306443618\\
62	0.10764	-1.20835288676212\\
62	0.1113	-0.840918722656177\\
62	0.11496	-0.44794057211832\\
62	0.11862	-0.0294184351485356\\
62	0.12228	0.414647688253162\\
62	0.12594	0.88425779808675\\
62	0.1296	1.37941189435228\\
62	0.13326	1.90010997704971\\
62	0.13692	2.44635204617904\\
62	0.14058	3.0181381017403\\
62	0.14424	3.61546814373349\\
62	0.1479	4.23834217215856\\
62	0.15156	4.88676018701556\\
62	0.15522	5.56072218830446\\
62	0.15888	6.26022817602529\\
62	0.16254	6.98527815017803\\
62	0.1662	7.73587211076268\\
62	0.16986	8.51201005777923\\
62	0.17352	9.31369199122771\\
62	0.17718	10.1409179111081\\
62	0.18084	10.9936878174204\\
62	0.1845	11.8720017101646\\
62	0.18816	12.7758595893407\\
62	0.19182	13.7052614549488\\
62	0.19548	14.6602073069887\\
62	0.19914	15.6406971454606\\
62	0.2028	16.6467309703644\\
62	0.20646	17.6783087817001\\
62	0.21012	18.7354305794677\\
62	0.21378	19.8180963636672\\
62	0.21744	20.9263061342986\\
62	0.2211	22.060059891362\\
62	0.22476	23.2193576348572\\
62	0.22842	24.4041993647844\\
62	0.23208	25.6145850811434\\
62	0.23574	26.8505147839345\\
62	0.2394	28.1119884731574\\
62	0.24306	29.3990061488122\\
62	0.24672	30.7115678108989\\
62	0.25038	32.0496734594176\\
62	0.25404	33.4133230943681\\
62	0.2577	34.8025167157506\\
62	0.26136	36.217254323565\\
62	0.26502	37.6575359178113\\
62	0.26868	39.1233614984895\\
62	0.27234	40.6147310655996\\
62	0.276	42.1316446191416\\
62.375	0.093	-2.5055356224978\\
62.375	0.09666	-2.23304893695986\\
62.375	0.10032	-1.93501826498998\\
62.375	0.10398	-1.61144360658819\\
62.375	0.10764	-1.2623249617545\\
62.375	0.1113	-0.887662330488894\\
62.375	0.11496	-0.487455712791352\\
62.375	0.11862	-0.0617051086619256\\
62.375	0.12228	0.389589481899428\\
62.375	0.12594	0.866428058892694\\
62.375	0.1296	1.36881062231786\\
62.375	0.13326	1.89673717217495\\
62.375	0.13692	2.45020770846395\\
62.375	0.14058	3.02922223118488\\
62.375	0.14424	3.6337807403377\\
62.375	0.1479	4.26388323592244\\
62.375	0.15156	4.9195297179391\\
62.375	0.15522	5.60072018638765\\
62.375	0.15888	6.30745464126814\\
62.375	0.16254	7.03973308258053\\
62.375	0.1662	7.79755551032484\\
62.375	0.16986	8.58092192450105\\
62.375	0.17352	9.38983232510918\\
62.375	0.17718	10.2242867121492\\
62.375	0.18084	11.0842850856212\\
62.375	0.1845	11.9698274455251\\
62.375	0.18816	12.8809137918608\\
62.375	0.19182	13.8175441246285\\
62.375	0.19548	14.7797184438282\\
62.375	0.19914	15.7674367494597\\
62.375	0.2028	16.7806990415231\\
62.375	0.20646	17.8195053200185\\
62.375	0.21012	18.8838555849457\\
62.375	0.21378	19.9737498363049\\
62.375	0.21744	21.089188074096\\
62.375	0.2211	22.230170298319\\
62.375	0.22476	23.3966965089739\\
62.375	0.22842	24.5887667060607\\
62.375	0.23208	25.8063808895794\\
62.375	0.23574	27.0495390595301\\
62.375	0.2394	28.3182412159127\\
62.375	0.24306	29.6124873587272\\
62.375	0.24672	30.9322774879736\\
62.375	0.25038	32.2776116036519\\
62.375	0.25404	33.6484897057621\\
62.375	0.2577	35.0449117943042\\
62.375	0.26136	36.4668778692782\\
62.375	0.26502	37.9143879306842\\
62.375	0.26868	39.3874419785221\\
62.375	0.27234	40.8860400127918\\
62.375	0.276	42.4101820334935\\
62.75	0.093	-2.58605288852874\\
62.75	0.09666	-2.30633773583113\\
62.75	0.10032	-2.00107859670159\\
62.75	0.10398	-1.67027547114014\\
62.75	0.10764	-1.31392835914679\\
62.75	0.1113	-0.932037260721529\\
62.75	0.11496	-0.524602175864331\\
62.75	0.11862	-0.0916231045752482\\
62.75	0.12228	0.366899953145769\\
62.75	0.12594	0.850966997298691\\
62.75	0.1296	1.36057802788351\\
62.75	0.13326	1.89573304490026\\
62.75	0.13692	2.45643204834893\\
62.75	0.14058	3.0426750382295\\
62.75	0.14424	3.65446201454198\\
62.75	0.1479	4.29179297728638\\
62.75	0.15156	4.95466792646269\\
62.75	0.15522	5.64308686207091\\
62.75	0.15888	6.35704978411106\\
62.75	0.16254	7.09655669258311\\
62.75	0.1662	7.86160758748707\\
62.75	0.16986	8.65220246882294\\
62.75	0.17352	9.46834133659073\\
62.75	0.17718	10.3100241907904\\
62.75	0.18084	11.1772510314221\\
62.75	0.1845	12.0700218584856\\
62.75	0.18816	12.988336671981\\
62.75	0.19182	13.9321954719084\\
62.75	0.19548	14.9015982582676\\
62.75	0.19914	15.8965450310588\\
62.75	0.2028	16.9170357902819\\
62.75	0.20646	17.9630705359369\\
62.75	0.21012	19.0346492680238\\
62.75	0.21378	20.1317719865427\\
62.75	0.21744	21.2544386914934\\
62.75	0.2211	22.4026493828761\\
62.75	0.22476	23.5764040606907\\
62.75	0.22842	24.7757027249371\\
62.75	0.23208	26.0005453756155\\
62.75	0.23574	27.2509320127258\\
62.75	0.2394	28.5268626362681\\
62.75	0.24306	29.8283372462422\\
62.75	0.24672	31.1553558426482\\
62.75	0.25038	32.5079184254862\\
62.75	0.25404	33.8860249947561\\
62.75	0.2577	35.2896755504579\\
62.75	0.26136	36.7188700925916\\
62.75	0.26502	38.1736086211572\\
62.75	0.26868	39.6538911361547\\
62.75	0.27234	41.1597176375842\\
62.75	0.276	42.6910881254455\\
63.125	0.093	-2.66420147695956\\
63.125	0.09666	-2.37725785710228\\
63.125	0.10032	-2.0647702508131\\
63.125	0.10398	-1.72673865809201\\
63.125	0.10764	-1.36316307893898\\
63.125	0.1113	-0.974043513354054\\
63.125	0.11496	-0.559379961337228\\
63.125	0.11862	-0.119172422888468\\
63.125	0.12228	0.346579101992198\\
63.125	0.12594	0.83787461330477\\
63.125	0.1296	1.35471411104927\\
63.125	0.13326	1.89709759522566\\
63.125	0.13692	2.46502506583397\\
63.125	0.14058	3.05849652287421\\
63.125	0.14424	3.67751196634637\\
63.125	0.1479	4.32207139625041\\
63.125	0.15156	4.99217481258639\\
63.125	0.15522	5.68782221535426\\
63.125	0.15888	6.40901360455406\\
63.125	0.16254	7.15574898018578\\
63.125	0.1662	7.9280283422494\\
63.125	0.16986	8.72585169074492\\
63.125	0.17352	9.54921902567237\\
63.125	0.17718	10.3981303470317\\
63.125	0.18084	11.272585654823\\
63.125	0.1845	12.1725849490462\\
63.125	0.18816	13.0981282297013\\
63.125	0.19182	14.0492154967883\\
63.125	0.19548	15.0258467503072\\
63.125	0.19914	16.0280219902581\\
63.125	0.2028	17.0557412166408\\
63.125	0.20646	18.1090044294555\\
63.125	0.21012	19.1878116287021\\
63.125	0.21378	20.2921628143806\\
63.125	0.21744	21.422057986491\\
63.125	0.2211	22.5774971450333\\
63.125	0.22476	23.7584802900075\\
63.125	0.22842	24.9650074214137\\
63.125	0.23208	26.1970785392517\\
63.125	0.23574	27.4546936435217\\
63.125	0.2394	28.7378527342236\\
63.125	0.24306	30.0465558113573\\
63.125	0.24672	31.380802874923\\
63.125	0.25038	32.7405939249207\\
63.125	0.25404	34.1259289613502\\
63.125	0.2577	35.5368079842117\\
63.125	0.26136	36.973230993505\\
63.125	0.26502	38.4351979892303\\
63.125	0.26868	39.9227089713875\\
63.125	0.27234	41.4357639399766\\
63.125	0.276	42.9743628949976\\
63.5	0.093	-2.73998138779032\\
63.5	0.09666	-2.44580930077338\\
63.5	0.10032	-2.12609322732454\\
63.5	0.10398	-1.78083316744379\\
63.5	0.10764	-1.4100291211311\\
63.5	0.1113	-1.01368108838652\\
63.5	0.11496	-0.591789069210037\\
63.5	0.11862	-0.144353063601621\\
63.5	0.12228	0.328626928438709\\
63.5	0.12594	0.82715090691093\\
63.5	0.1296	1.3512188718151\\
63.5	0.13326	1.90083082315115\\
63.5	0.13692	2.47598676091911\\
63.5	0.14058	3.076686685119\\
63.5	0.14424	3.70293059575083\\
63.5	0.1479	4.35471849281452\\
63.5	0.15156	5.03205037631016\\
63.5	0.15522	5.73492624623769\\
63.5	0.15888	6.46334610259715\\
63.5	0.16254	7.21730994538851\\
63.5	0.1662	7.9968177746118\\
63.5	0.16986	8.80186959026698\\
63.5	0.17352	9.63246539235409\\
63.5	0.17718	10.4886051808731\\
63.5	0.18084	11.370288955824\\
63.5	0.1845	12.2775167172069\\
63.5	0.18816	13.2102884650216\\
63.5	0.19182	14.1686041992683\\
63.5	0.19548	15.1524639199469\\
63.5	0.19914	16.1618676270574\\
63.5	0.2028	17.1968153205998\\
63.5	0.20646	18.2573070005741\\
63.5	0.21012	19.3433426669804\\
63.5	0.21378	20.4549223198185\\
63.5	0.21744	21.5920459590885\\
63.5	0.2211	22.7547135847905\\
63.5	0.22476	23.9429251969244\\
63.5	0.22842	25.1566807954902\\
63.5	0.23208	26.3959803804879\\
63.5	0.23574	27.6608239519176\\
63.5	0.2394	28.9512115097791\\
63.5	0.24306	30.2671430540726\\
63.5	0.24672	31.6086185847979\\
63.5	0.25038	32.9756381019552\\
63.5	0.25404	34.3682016055444\\
63.5	0.2577	35.7863090955655\\
63.5	0.26136	37.2299605720185\\
63.5	0.26502	38.6991560349035\\
63.5	0.26868	40.1938954842202\\
63.5	0.27234	41.714178919969\\
63.5	0.276	43.2600063421497\\
63.875	0.093	-2.81339262102098\\
63.875	0.09666	-2.5119920668444\\
63.875	0.10032	-2.18504752623589\\
63.875	0.10398	-1.83255899919547\\
63.875	0.10764	-1.45452648572315\\
63.875	0.1113	-1.05094998581891\\
63.875	0.11496	-0.621829499482743\\
63.875	0.11862	-0.167165026714684\\
63.875	0.12228	0.313043432485301\\
63.875	0.12594	0.8187958781172\\
63.875	0.1296	1.35009231018099\\
63.875	0.13326	1.90693272867672\\
63.875	0.13692	2.48931713360436\\
63.875	0.14058	3.09724552496391\\
63.875	0.14424	3.73071790275536\\
63.875	0.1479	4.38973426697872\\
63.875	0.15156	5.07429461763401\\
63.875	0.15522	5.7843989547212\\
63.875	0.15888	6.52004727824032\\
63.875	0.16254	7.28123958819135\\
63.875	0.1662	8.06797588457429\\
63.875	0.16986	8.88025616738913\\
63.875	0.17352	9.71808043663589\\
63.875	0.17718	10.5814486923146\\
63.875	0.18084	11.4703609344252\\
63.875	0.1845	12.3848171629677\\
63.875	0.18816	13.3248173779421\\
63.875	0.19182	14.2903615793484\\
63.875	0.19548	15.2814497671867\\
63.875	0.19914	16.2980819414568\\
63.875	0.2028	17.3402581021589\\
63.875	0.20646	18.4079782492929\\
63.875	0.21012	19.5012423828587\\
63.875	0.21378	20.6200505028565\\
63.875	0.21744	21.7644026092863\\
63.875	0.2211	22.9342987021479\\
63.875	0.22476	24.1297387814414\\
63.875	0.22842	25.3507228471669\\
63.875	0.23208	26.5972508993243\\
63.875	0.23574	27.8693229379136\\
63.875	0.2394	29.1669389629348\\
63.875	0.24306	30.4900989743878\\
63.875	0.24672	31.8388029722729\\
63.875	0.25038	33.2130509565898\\
63.875	0.25404	34.6128429273387\\
63.875	0.2577	36.0381788845194\\
63.875	0.26136	37.4890588281321\\
63.875	0.26502	38.9654827581767\\
63.875	0.26868	40.4674506746532\\
63.875	0.27234	41.9949625775616\\
63.875	0.276	43.5480184669019\\
64.25	0.093	-2.88443517665157\\
64.25	0.09666	-2.57580615531534\\
64.25	0.10032	-2.24163314754716\\
64.25	0.10398	-1.88191615334709\\
64.25	0.10764	-1.49665517271511\\
64.25	0.1113	-1.08585020565122\\
64.25	0.11496	-0.649501252155389\\
64.25	0.11862	-0.187608312227667\\
64.25	0.12228	0.299828614131975\\
64.25	0.12594	0.81280952692353\\
64.25	0.1296	1.35133442614698\\
64.25	0.13326	1.91540331180236\\
64.25	0.13692	2.50501618388965\\
64.25	0.14058	3.12017304240886\\
64.25	0.14424	3.76087388735997\\
64.25	0.1479	4.427118718743\\
64.25	0.15156	5.11890753655794\\
64.25	0.15522	5.83624034080479\\
64.25	0.15888	6.57911713148357\\
64.25	0.16254	7.34753790859425\\
64.25	0.1662	8.14150267213685\\
64.25	0.16986	8.96101142211135\\
64.25	0.17352	9.80606415851777\\
64.25	0.17718	10.6766608813561\\
64.25	0.18084	11.5728015906264\\
64.25	0.1845	12.4944862863285\\
64.25	0.18816	13.4417149684626\\
64.25	0.19182	14.4144876370286\\
64.25	0.19548	15.4128042920265\\
64.25	0.19914	16.4366649334563\\
64.25	0.2028	17.486069561318\\
64.25	0.20646	18.5610181756117\\
64.25	0.21012	19.6615107763372\\
64.25	0.21378	20.7875473634947\\
64.25	0.21744	21.939127937084\\
64.25	0.2211	23.1162524971053\\
64.25	0.22476	24.3189210435585\\
64.25	0.22842	25.5471335764436\\
64.25	0.23208	26.8008900957606\\
64.25	0.23574	28.0801906015096\\
64.25	0.2394	29.3850350936905\\
64.25	0.24306	30.7154235723032\\
64.25	0.24672	32.0713560373479\\
64.25	0.25038	33.4528324888245\\
64.25	0.25404	34.859852926733\\
64.25	0.2577	36.2924173510734\\
64.25	0.26136	37.7505257618458\\
64.25	0.26502	39.23417815905\\
64.25	0.26868	40.7433745426862\\
64.25	0.27234	42.2781149127542\\
64.25	0.276	43.8383992692542\\
64.625	0.093	-2.95310905468209\\
64.625	0.09666	-2.6372515661862\\
64.625	0.10032	-2.29585009125837\\
64.625	0.10398	-1.92890462989863\\
64.625	0.10764	-1.53641518210699\\
64.625	0.1113	-1.11838174788345\\
64.625	0.11496	-0.674804327227953\\
64.625	0.11862	-0.205682920140582\\
64.625	0.12228	0.288982473378717\\
64.625	0.12594	0.809191853329928\\
64.625	0.1296	1.35494521971304\\
64.625	0.13326	1.92624257252808\\
64.625	0.13692	2.52308391177503\\
64.625	0.14058	3.1454692374539\\
64.625	0.14424	3.79339854956466\\
64.625	0.1479	4.46687184810735\\
64.625	0.15156	5.16588913308195\\
64.625	0.15522	5.89045040448846\\
64.625	0.15888	6.64055566232689\\
64.625	0.16254	7.41620490659723\\
64.625	0.1662	8.21739813729949\\
64.625	0.16986	9.04413535443364\\
64.625	0.17352	9.89641655799972\\
64.625	0.17718	10.7742417479977\\
64.625	0.18084	11.6776109244276\\
64.625	0.1845	12.6065240872894\\
64.625	0.18816	13.5609812365832\\
64.625	0.19182	14.5409823723088\\
64.625	0.19548	15.5465274944664\\
64.625	0.19914	16.5776166030558\\
64.625	0.2028	17.6342496980772\\
64.625	0.20646	18.7164267795305\\
64.625	0.21012	19.8241478474157\\
64.625	0.21378	20.9574129017328\\
64.625	0.21744	22.1162219424819\\
64.625	0.2211	23.3005749696628\\
64.625	0.22476	24.5104719832757\\
64.625	0.22842	25.7459129833205\\
64.625	0.23208	27.0068979697972\\
64.625	0.23574	28.2934269427057\\
64.625	0.2394	29.6054999020463\\
64.625	0.24306	30.9431168478187\\
64.625	0.24672	32.306277780023\\
64.625	0.25038	33.6949826986593\\
64.625	0.25404	35.1092316037274\\
64.625	0.2577	36.5490244952275\\
64.625	0.26136	38.0143613731595\\
64.625	0.26502	39.5052422375234\\
64.625	0.26868	41.0216670883192\\
64.625	0.27234	42.5636359255469\\
64.625	0.276	44.1311487492066\\
65	0.093	-3.0194142551125\\
65	0.09666	-2.69632829945694\\
65	0.10032	-2.34769835736947\\
65	0.10398	-1.97352442885009\\
65	0.10764	-1.57380651389877\\
65	0.1113	-1.14854461251556\\
65	0.11496	-0.697738724700436\\
65	0.11862	-0.221388850453394\\
65	0.12228	0.280505010225568\\
65	0.12594	0.807942857336421\\
65	0.1296	1.36092469087922\\
65	0.13326	1.9394505108539\\
65	0.13692	2.54352031726049\\
65	0.14058	3.17313411009902\\
65	0.14424	3.82829188936947\\
65	0.1479	4.5089936550718\\
65	0.15156	5.21523940720606\\
65	0.15522	5.94702914577223\\
65	0.15888	6.70436287077032\\
65	0.16254	7.48724058220031\\
65	0.1662	8.29566228006223\\
65	0.16986	9.12962796435604\\
65	0.17352	9.98913763508178\\
65	0.17718	10.8741912922394\\
65	0.18084	11.784788935829\\
65	0.1845	12.7209305658505\\
65	0.18816	13.6826161823039\\
65	0.19182	14.6698457851891\\
65	0.19548	15.6826193745064\\
65	0.19914	16.7209369502555\\
65	0.2028	17.7847985124365\\
65	0.20646	18.8742040610495\\
65	0.21012	19.9891535960944\\
65	0.21378	21.1296471175711\\
65	0.21744	22.2956846254798\\
65	0.2211	23.4872661198204\\
65	0.22476	24.7043916005929\\
65	0.22842	25.9470610677974\\
65	0.23208	27.2152745214337\\
65	0.23574	28.509031961502\\
65	0.2394	29.8283333880022\\
65	0.24306	31.1731788009342\\
65	0.24672	32.5435682002982\\
65	0.25038	33.9395015860941\\
65	0.25404	35.360978958322\\
65	0.2577	36.8080003169817\\
65	0.26136	38.2805656620733\\
65	0.26502	39.7786749935969\\
65	0.26868	41.3023283115524\\
65	0.27234	42.8515256159397\\
65	0.276	44.426266906759\\
65.375	0.093	-3.08335077794285\\
65.375	0.09666	-2.75303635512763\\
65.375	0.10032	-2.39717794588049\\
65.375	0.10398	-2.01577555020146\\
65.375	0.10764	-1.60882916809048\\
65.375	0.1113	-1.17633879954762\\
65.375	0.11496	-0.718304444572837\\
65.375	0.11862	-0.234726103166132\\
65.375	0.12228	0.274396224672479\\
65.375	0.12594	0.809062538942996\\
65.375	0.1296	1.36927283964544\\
65.375	0.13326	1.95502712677979\\
65.375	0.13692	2.56632540034604\\
65.375	0.14058	3.20316766034422\\
65.375	0.14424	3.86555390677433\\
65.375	0.1479	4.55348413963632\\
65.375	0.15156	5.26695835893024\\
65.375	0.15522	6.00597656465606\\
65.375	0.15888	6.77053875681381\\
65.375	0.16254	7.56064493540347\\
65.375	0.1662	8.37629510042504\\
65.375	0.16986	9.2174892518785\\
65.375	0.17352	10.0842273897639\\
65.375	0.17718	10.9765095140812\\
65.375	0.18084	11.8943356248304\\
65.375	0.1845	12.8377057220116\\
65.375	0.18816	13.8066198056246\\
65.375	0.19182	14.8010778756696\\
65.375	0.19548	15.8210799321464\\
65.375	0.19914	16.8666259750552\\
65.375	0.2028	17.9377160043959\\
65.375	0.20646	19.0343500201685\\
65.375	0.21012	20.1565280223731\\
65.375	0.21378	21.3042500110095\\
65.375	0.21744	22.4775159860778\\
65.375	0.2211	23.6763259475781\\
65.375	0.22476	24.9006798955103\\
65.375	0.22842	26.1505778298744\\
65.375	0.23208	27.4260197506704\\
65.375	0.23574	28.7270056578983\\
65.375	0.2394	30.0535355515581\\
65.375	0.24306	31.4056094316499\\
65.375	0.24672	32.7832272981735\\
65.375	0.25038	34.1863891511291\\
65.375	0.25404	35.6150949905166\\
65.375	0.2577	37.0693448163359\\
65.375	0.26136	38.5491386285872\\
65.375	0.26502	40.0544764272705\\
65.375	0.26868	41.5853582123856\\
65.375	0.27234	43.1417839839326\\
65.375	0.276	44.7237537419116\\
65.75	0.093	-3.1449186231731\\
65.75	0.09666	-2.80737573319824\\
65.75	0.10032	-2.44428885679143\\
65.75	0.10398	-2.05565799395272\\
65.75	0.10764	-1.64148314468211\\
65.75	0.1113	-1.20176430897959\\
65.75	0.11496	-0.736501486845128\\
65.75	0.11862	-0.245694678278781\\
65.75	0.12228	0.270656116719493\\
65.75	0.12594	0.812550898149674\\
65.75	0.1296	1.37998966601176\\
65.75	0.13326	1.97297242030577\\
65.75	0.13692	2.5914991610317\\
65.75	0.14058	3.23556988818953\\
65.75	0.14424	3.90518460177928\\
65.75	0.1479	4.60034330180094\\
65.75	0.15156	5.32104598825451\\
65.75	0.15522	6.06729266113999\\
65.75	0.15888	6.8390833204574\\
65.75	0.16254	7.63641796620671\\
65.75	0.1662	8.45929659838794\\
65.75	0.16986	9.30771921700106\\
65.75	0.17352	10.1816858220461\\
65.75	0.17718	11.0811964135231\\
65.75	0.18084	12.006250991432\\
65.75	0.1845	12.9568495557728\\
65.75	0.18816	13.9329921065455\\
65.75	0.19182	14.9346786437501\\
65.75	0.19548	15.9619091673866\\
65.75	0.19914	17.014683677455\\
65.75	0.2028	18.0930021739554\\
65.75	0.20646	19.1968646568877\\
65.75	0.21012	20.3262711262519\\
65.75	0.21378	21.4812215820479\\
65.75	0.21744	22.661716024276\\
65.75	0.2211	23.8677544529359\\
65.75	0.22476	25.0993368680277\\
65.75	0.22842	26.3564632695515\\
65.75	0.23208	27.6391336575071\\
65.75	0.23574	28.9473480318947\\
65.75	0.2394	30.2811063927142\\
65.75	0.24306	31.6404087399656\\
65.75	0.24672	33.0252550736489\\
65.75	0.25038	34.4356453937641\\
65.75	0.25404	35.8715797003113\\
65.75	0.2577	37.3330579932903\\
65.75	0.26136	38.8200802727013\\
65.75	0.26502	40.3326465385441\\
65.75	0.26868	41.8707567908189\\
65.75	0.27234	43.4344110295256\\
65.75	0.276	45.0236092546642\\
66.125	0.093	-3.20411779080329\\
66.125	0.09666	-2.85934643366876\\
66.125	0.10032	-2.4890310901023\\
66.125	0.10398	-2.09317176010394\\
66.125	0.10764	-1.67176844367367\\
66.125	0.1113	-1.22482114081149\\
66.125	0.11496	-0.752329851517366\\
66.125	0.11862	-0.254294575791363\\
66.125	0.12228	0.269284686366568\\
66.125	0.12594	0.818407934956412\\
66.125	0.1296	1.39307516997815\\
66.125	0.13326	1.99328639143182\\
66.125	0.13692	2.6190415993174\\
66.125	0.14058	3.2703407936349\\
66.125	0.14424	3.9471839743843\\
66.125	0.1479	4.64957114156562\\
66.125	0.15156	5.37750229517885\\
66.125	0.15522	6.13097743522399\\
66.125	0.15888	6.90999656170105\\
66.125	0.16254	7.71455967461002\\
66.125	0.1662	8.54466677395091\\
66.125	0.16986	9.4003178597237\\
66.125	0.17352	10.2815129319284\\
66.125	0.17718	11.188251990565\\
66.125	0.18084	12.1205350356336\\
66.125	0.1845	13.078362067134\\
66.125	0.18816	14.0617330850664\\
66.125	0.19182	15.0706480894307\\
66.125	0.19548	16.1051070802268\\
66.125	0.19914	17.1651100574549\\
66.125	0.2028	18.250657021115\\
66.125	0.20646	19.3617479712069\\
66.125	0.21012	20.4983829077307\\
66.125	0.21378	21.6605618306865\\
66.125	0.21744	22.8482847400741\\
66.125	0.2211	24.0615516358937\\
66.125	0.22476	25.3003625181452\\
66.125	0.22842	26.5647173868286\\
66.125	0.23208	27.8546162419439\\
66.125	0.23574	29.1700590834912\\
66.125	0.2394	30.5110459114703\\
66.125	0.24306	31.8775767258813\\
66.125	0.24672	33.2696515267243\\
66.125	0.25038	34.6872703139992\\
66.125	0.25404	36.130433087706\\
66.125	0.2577	37.5991398478447\\
66.125	0.26136	39.0933905944153\\
66.125	0.26502	40.6131853274179\\
66.125	0.26868	42.1585240468523\\
66.125	0.27234	43.7294067527186\\
66.125	0.276	45.3258334450169\\
66.5	0.093	-3.26094828083337\\
66.5	0.09666	-2.90894845653917\\
66.5	0.10032	-2.53140464581307\\
66.5	0.10398	-2.12831684865506\\
66.5	0.10764	-1.69968506506511\\
66.5	0.1113	-1.24550929504326\\
66.5	0.11496	-0.765789538589516\\
66.5	0.11862	-0.260525795703842\\
66.5	0.12228	0.270281933613752\\
66.5	0.12594	0.826633649363238\\
66.5	0.1296	1.40852935154466\\
66.5	0.13326	2.01596904015797\\
66.5	0.13692	2.6489527152032\\
66.5	0.14058	3.30748037668036\\
66.5	0.14424	3.99155202458945\\
66.5	0.1479	4.70116765893041\\
66.5	0.15156	5.4363272797033\\
66.5	0.15522	6.19703088690809\\
66.5	0.15888	6.98327848054481\\
66.5	0.16254	7.79507006061344\\
66.5	0.1662	8.63240562711398\\
66.5	0.16986	9.49528518004643\\
66.5	0.17352	10.3837087194108\\
66.5	0.17718	11.2976762452071\\
66.5	0.18084	12.2371877574353\\
66.5	0.1845	13.2022432560954\\
66.5	0.18816	14.1928427411874\\
66.5	0.19182	15.2089862127113\\
66.5	0.19548	16.2506736706672\\
66.5	0.19914	17.3179051150549\\
66.5	0.2028	18.4106805458746\\
66.5	0.20646	19.5289999631262\\
66.5	0.21012	20.6728633668097\\
66.5	0.21378	21.8422707569251\\
66.5	0.21744	23.0372221334724\\
66.5	0.2211	24.2577174964517\\
66.5	0.22476	25.5037568458628\\
66.5	0.22842	26.7753401817058\\
66.5	0.23208	28.0724675039808\\
66.5	0.23574	29.3951388126877\\
66.5	0.2394	30.7433541078265\\
66.5	0.24306	32.1171133893972\\
66.5	0.24672	33.5164166573999\\
66.5	0.25038	34.9412639118344\\
66.5	0.25404	36.3916551527009\\
66.5	0.2577	37.8675903799992\\
66.5	0.26136	39.3690695937295\\
66.5	0.26502	40.8960927938917\\
66.5	0.26868	42.4486599804858\\
66.5	0.27234	44.0267711535118\\
66.5	0.276	45.6304263129698\\
66.875	0.093	-3.31541009326339\\
66.875	0.09666	-2.95618180180953\\
66.875	0.10032	-2.57140952392377\\
66.875	0.10398	-2.16109325960611\\
66.875	0.10764	-1.72523300885649\\
66.875	0.1113	-1.26382877167499\\
66.875	0.11496	-0.776880548061591\\
66.875	0.11862	-0.264388338016253\\
66.875	0.12228	0.27364785846099\\
66.875	0.12594	0.837228041370139\\
66.875	0.1296	1.42635221071122\\
66.875	0.13326	2.04102036648419\\
66.875	0.13692	2.68123250868907\\
66.875	0.14058	3.34698863732589\\
66.875	0.14424	4.03828875239463\\
66.875	0.1479	4.75513285389525\\
66.875	0.15156	5.4975209418278\\
66.875	0.15522	6.26545301619225\\
66.875	0.15888	7.05892907698863\\
66.875	0.16254	7.87794912421692\\
66.875	0.1662	8.72251315787712\\
66.875	0.16986	9.59262117796922\\
66.875	0.17352	10.4882731844932\\
66.875	0.17718	11.4094691774492\\
66.875	0.18084	12.356209156837\\
66.875	0.1845	13.3284931226568\\
66.875	0.18816	14.3263210749085\\
66.875	0.19182	15.3496930135921\\
66.875	0.19548	16.3986089387076\\
66.875	0.19914	17.473068850255\\
66.875	0.2028	18.5730727482343\\
66.875	0.20646	19.6986206326456\\
66.875	0.21012	20.8497125034887\\
66.875	0.21378	22.0263483607638\\
66.875	0.21744	23.2285282044708\\
66.875	0.2211	24.4562520346096\\
66.875	0.22476	25.7095198511804\\
66.875	0.22842	26.9883316541832\\
66.875	0.23208	28.2926874436178\\
66.875	0.23574	29.6225872194844\\
66.875	0.2394	30.9780309817828\\
66.875	0.24306	32.3590187305132\\
66.875	0.24672	33.7655504656755\\
66.875	0.25038	35.1976261872697\\
66.875	0.25404	36.6552458952958\\
66.875	0.2577	38.1384095897538\\
66.875	0.26136	39.6471172706437\\
66.875	0.26502	41.1813689379656\\
66.875	0.26868	42.7411645917193\\
66.875	0.27234	44.326504231905\\
66.875	0.276	45.9373878585226\\
67.25	0.093	-3.36750322809331\\
67.25	0.09666	-3.0010464694798\\
67.25	0.10032	-2.60904572443437\\
67.25	0.10398	-2.19150099295703\\
67.25	0.10764	-1.7484122750478\\
67.25	0.1113	-1.27977957070664\\
67.25	0.11496	-0.785602879933549\\
67.25	0.11862	-0.265882202728569\\
67.25	0.12228	0.279382460908337\\
67.25	0.12594	0.85019111097715\\
67.25	0.1296	1.44654374747786\\
67.25	0.13326	2.06844037041051\\
67.25	0.13692	2.71588097977506\\
67.25	0.14058	3.38886557557153\\
67.25	0.14424	4.08739415779991\\
67.25	0.1479	4.8114667264602\\
67.25	0.15156	5.5610832815524\\
67.25	0.15522	6.33624382307652\\
67.25	0.15888	7.13694835103255\\
67.25	0.16254	7.9631968654205\\
67.25	0.1662	8.81498936624035\\
67.25	0.16986	9.69232585349211\\
67.25	0.17352	10.5952063271758\\
67.25	0.17718	11.5236307872914\\
67.25	0.18084	12.4775992338389\\
67.25	0.1845	13.4571116668183\\
67.25	0.18816	14.4621680862297\\
67.25	0.19182	15.4927684920729\\
67.25	0.19548	16.5489128843481\\
67.25	0.19914	17.6306012630551\\
67.25	0.2028	18.7378336281941\\
67.25	0.20646	19.870609979765\\
67.25	0.21012	21.0289303177678\\
67.25	0.21378	22.2127946422026\\
67.25	0.21744	23.4222029530692\\
67.25	0.2211	24.6571552503677\\
67.25	0.22476	25.9176515340982\\
67.25	0.22842	27.2036918042606\\
67.25	0.23208	28.5152760608549\\
67.25	0.23574	29.8524043038811\\
67.25	0.2394	31.2150765333392\\
67.25	0.24306	32.6032927492292\\
67.25	0.24672	34.0170529515512\\
67.25	0.25038	35.456357140305\\
67.25	0.25404	36.9212053154908\\
67.25	0.2577	38.4115974771085\\
67.25	0.26136	39.9275336251581\\
67.25	0.26502	41.4690137596396\\
67.25	0.26868	43.036037880553\\
67.25	0.27234	44.6286059878983\\
67.25	0.276	46.2467180816756\\
67.625	0.093	-3.41722768532316\\
67.625	0.09666	-3.04354245955\\
67.625	0.10032	-2.64431324734491\\
67.625	0.10398	-2.21954004870791\\
67.625	0.10764	-1.76922286363902\\
67.625	0.1113	-1.2933616921382\\
67.625	0.11496	-0.791956534205454\\
67.625	0.11862	-0.265007389840818\\
67.625	0.12228	0.287485740955745\\
67.625	0.12594	0.865522858184221\\
67.625	0.1296	1.46910396184459\\
67.625	0.13326	2.09822905193689\\
67.625	0.13692	2.75289812846111\\
67.625	0.14058	3.43311119141724\\
67.625	0.14424	4.13886824080527\\
67.625	0.1479	4.87016927662522\\
67.625	0.15156	5.62701429887708\\
67.625	0.15522	6.40940330756084\\
67.625	0.15888	7.21733630267654\\
67.625	0.16254	8.05081328422414\\
67.625	0.1662	8.90983425220366\\
67.625	0.16986	9.79439920661508\\
67.625	0.17352	10.7045081474584\\
67.625	0.17718	11.6401610747337\\
67.625	0.18084	12.6013579884408\\
67.625	0.1845	13.5880988885799\\
67.625	0.18816	14.6003837751509\\
67.625	0.19182	15.6382126481538\\
67.625	0.19548	16.7015855075886\\
67.625	0.19914	17.7905023534554\\
67.625	0.2028	18.904963185754\\
67.625	0.20646	20.0449680044846\\
67.625	0.21012	21.2105168096471\\
67.625	0.21378	22.4016096012414\\
67.625	0.21744	23.6182463792677\\
67.625	0.2211	24.8604271437259\\
67.625	0.22476	26.128151894616\\
67.625	0.22842	27.4214206319381\\
67.625	0.23208	28.740233355692\\
67.625	0.23574	30.0845900658779\\
67.625	0.2394	31.4544907624957\\
67.625	0.24306	32.8499354455454\\
67.625	0.24672	34.2709241150269\\
67.625	0.25038	35.7174567709405\\
67.625	0.25404	37.1895334132859\\
67.625	0.2577	38.6871540420632\\
67.625	0.26136	40.2103186572725\\
67.625	0.26502	41.7590272589136\\
67.625	0.26868	43.3332798469867\\
67.625	0.27234	44.9330764214917\\
67.625	0.276	46.5584169824286\\
68	0.093	-3.46458346495291\\
68	0.09666	-3.08366977202008\\
68	0.10032	-2.67721209265535\\
68	0.10398	-2.24521042685871\\
68	0.10764	-1.78766477463012\\
68	0.1113	-1.30457513596965\\
68	0.11496	-0.79594151087727\\
68	0.11862	-0.261763899352964\\
68	0.12228	0.297957698603255\\
68	0.12594	0.883223282991374\\
68	0.1296	1.49403285381143\\
68	0.13326	2.13038641106337\\
68	0.13692	2.79228395474723\\
68	0.14058	3.47972548486301\\
68	0.14424	4.19271100141074\\
68	0.1479	4.93124050439033\\
68	0.15156	5.69531399380185\\
68	0.15522	6.48493146964528\\
68	0.15888	7.30009293192062\\
68	0.16254	8.14079838062788\\
68	0.1662	9.00704781576706\\
68	0.16986	9.89884123733813\\
68	0.17352	10.8161786453411\\
68	0.17718	11.759060039776\\
68	0.18084	12.7274854206429\\
68	0.1845	13.7214547879416\\
68	0.18816	14.7409681416723\\
68	0.19182	15.7860254818348\\
68	0.19548	16.8566268084293\\
68	0.19914	17.9527721214557\\
68	0.2028	19.074461420914\\
68	0.20646	20.2216947068042\\
68	0.21012	21.3944719791263\\
68	0.21378	22.5927932378804\\
68	0.21744	23.8166584830663\\
68	0.2211	25.0660677146842\\
68	0.22476	26.341020932734\\
68	0.22842	27.6415181372157\\
68	0.23208	28.9675593281293\\
68	0.23574	30.3191445054748\\
68	0.2394	31.6962736692522\\
68	0.24306	33.0989468194616\\
68	0.24672	34.5271639561028\\
68	0.25038	35.980925079176\\
68	0.25404	37.4602301886811\\
68	0.2577	38.9650792846181\\
68	0.26136	40.495472366987\\
68	0.26502	42.0514094357878\\
68	0.26868	43.6328904910205\\
68	0.27234	45.2399155326852\\
68	0.276	46.8724845607817\\
68.375	0.093	-3.5095705669826\\
68.375	0.09666	-3.12142840689012\\
68.375	0.10032	-2.70774226036573\\
68.375	0.10398	-2.26851212740943\\
68.375	0.10764	-1.80373800802119\\
68.375	0.1113	-1.31341990220105\\
68.375	0.11496	-0.797557809949019\\
68.375	0.11862	-0.256151731265057\\
68.375	0.12228	0.310798333850819\\
68.375	0.12594	0.903292385398593\\
68.375	0.1296	1.52133042337831\\
68.375	0.13326	2.16491244778991\\
68.375	0.13692	2.83403845863343\\
68.375	0.14058	3.52870845590887\\
68.375	0.14424	4.24892243961624\\
68.375	0.1479	4.99468040975549\\
68.375	0.15156	5.76598236632668\\
68.375	0.15522	6.56282830932976\\
68.375	0.15888	7.38521823876476\\
68.375	0.16254	8.23315215463169\\
68.375	0.1662	9.10663005693052\\
68.375	0.16986	10.0056519456612\\
68.375	0.17352	10.9302178208239\\
68.375	0.17718	11.8803276824185\\
68.375	0.18084	12.855981530445\\
68.375	0.1845	13.8571793649034\\
68.375	0.18816	14.8839211857937\\
68.375	0.19182	15.9362069931159\\
68.375	0.19548	17.01403678687\\
68.375	0.19914	18.1174105670561\\
68.375	0.2028	19.246328333674\\
68.375	0.20646	20.4007900867239\\
68.375	0.21012	21.5807958262057\\
68.375	0.21378	22.7863455521194\\
68.375	0.21744	24.017439264465\\
68.375	0.2211	25.2740769632425\\
68.375	0.22476	26.556258648452\\
68.375	0.22842	27.8639843200933\\
68.375	0.23208	29.1972539781666\\
68.375	0.23574	30.5560676226718\\
68.375	0.2394	31.9404252536088\\
68.375	0.24306	33.3503268709778\\
68.375	0.24672	34.7857724747787\\
68.375	0.25038	36.2467620650116\\
68.375	0.25404	37.7332956416763\\
68.375	0.2577	39.245373204773\\
68.375	0.26136	40.7829947543015\\
68.375	0.26502	42.346160290262\\
68.375	0.26868	43.9348698126544\\
68.375	0.27234	45.5491233214787\\
68.375	0.276	47.188920816735\\
68.75	0.093	-3.55218899141221\\
68.75	0.09666	-3.15681836416007\\
68.75	0.10032	-2.73590375047602\\
68.75	0.10398	-2.28944515036006\\
68.75	0.10764	-1.81744256381217\\
68.75	0.1113	-1.31989599083238\\
68.75	0.11496	-0.79680543142068\\
68.75	0.11862	-0.248170885577061\\
68.75	0.12228	0.326007646698478\\
68.75	0.12594	0.925730165405909\\
68.75	0.1296	1.55099667054528\\
68.75	0.13326	2.20180716211654\\
68.75	0.13692	2.87816164011972\\
68.75	0.14058	3.58006010455481\\
68.75	0.14424	4.30750255542185\\
68.75	0.1479	5.06048899272075\\
68.75	0.15156	5.8390194164516\\
68.75	0.15522	6.64309382661433\\
68.75	0.15888	7.472712223209\\
68.75	0.16254	8.32787460623558\\
68.75	0.1662	9.20858097569407\\
68.75	0.16986	10.1148313315845\\
68.75	0.17352	11.0466256739068\\
68.75	0.17718	12.003964002661\\
68.75	0.18084	12.9868463178471\\
68.75	0.1845	13.9952726194652\\
68.75	0.18816	15.0292429075152\\
68.75	0.19182	16.088757181997\\
68.75	0.19548	17.1738154429108\\
68.75	0.19914	18.2844176902565\\
68.75	0.2028	19.4205639240342\\
68.75	0.20646	20.5822541442437\\
68.75	0.21012	21.7694883508851\\
68.75	0.21378	22.9822665439585\\
68.75	0.21744	24.2205887234638\\
68.75	0.2211	25.4844548894009\\
68.75	0.22476	26.77386504177\\
68.75	0.22842	28.0888191805711\\
68.75	0.23208	29.4293173058039\\
68.75	0.23574	30.7953594174688\\
68.75	0.2394	32.1869455155656\\
68.75	0.24306	33.6040756000942\\
68.75	0.24672	35.0467496710548\\
68.75	0.25038	36.5149677284473\\
68.75	0.25404	38.0087297722716\\
68.75	0.2577	39.528035802528\\
68.75	0.26136	41.0728858192162\\
68.75	0.26502	42.6432798223363\\
68.75	0.26868	44.2392178118884\\
68.75	0.27234	45.8606997878723\\
68.75	0.276	47.5077257502882\\
69.125	0.093	-3.59243873824172\\
69.125	0.09666	-3.18983964382994\\
69.125	0.10032	-2.76169656298621\\
69.125	0.10398	-2.30800949571059\\
69.125	0.10764	-1.82877844200306\\
69.125	0.1113	-1.32400340186361\\
69.125	0.11496	-0.79368437529223\\
69.125	0.11862	-0.237821362288962\\
69.125	0.12228	0.343585637146226\\
69.125	0.12594	0.950536623013335\\
69.125	0.1296	1.58303159531234\\
69.125	0.13326	2.24107055404327\\
69.125	0.13692	2.92465349920611\\
69.125	0.14058	3.63378043080087\\
69.125	0.14424	4.36845134882753\\
69.125	0.1479	5.12866625328611\\
69.125	0.15156	5.91442514417661\\
69.125	0.15522	6.72572802149901\\
69.125	0.15888	7.56257488525333\\
69.125	0.16254	8.42496573543956\\
69.125	0.1662	9.31290057205771\\
69.125	0.16986	10.2263793951078\\
69.125	0.17352	11.1654022045897\\
69.125	0.17718	12.1299690005036\\
69.125	0.18084	13.1200797828494\\
69.125	0.1845	14.1357345516271\\
69.125	0.18816	15.1769333068367\\
69.125	0.19182	16.2436760484783\\
69.125	0.19548	17.3359627765518\\
69.125	0.19914	18.4537934910571\\
69.125	0.2028	19.5971681919944\\
69.125	0.20646	20.7660868793636\\
69.125	0.21012	21.9605495531647\\
69.125	0.21378	23.1805562133977\\
69.125	0.21744	24.4261068600626\\
69.125	0.2211	25.6972014931594\\
69.125	0.22476	26.9938401126882\\
69.125	0.22842	28.3160227186489\\
69.125	0.23208	29.6637493110414\\
69.125	0.23574	31.037019889866\\
69.125	0.2394	32.4358344551223\\
69.125	0.24306	33.8601930068106\\
69.125	0.24672	35.3100955449309\\
69.125	0.25038	36.785542069483\\
69.125	0.25404	38.2865325804671\\
69.125	0.2577	39.8130670778831\\
69.125	0.26136	41.3651455617309\\
69.125	0.26502	42.9427680320107\\
69.125	0.26868	44.5459344887224\\
69.125	0.27234	46.174644931866\\
69.125	0.276	47.8288993614416\\
69.5	0.093	-3.63031980747115\\
69.5	0.09666	-3.22049224589971\\
69.5	0.10032	-2.78512069789633\\
69.5	0.10398	-2.32420516346105\\
69.5	0.10764	-1.83774564259386\\
69.5	0.1113	-1.32574213529476\\
69.5	0.11496	-0.78819464156372\\
69.5	0.11862	-0.225103161400796\\
69.5	0.12228	0.363532305194056\\
69.5	0.12594	0.977711758220821\\
69.5	0.1296	1.61743519767947\\
69.5	0.13326	2.28270262357006\\
69.5	0.13692	2.97351403589257\\
69.5	0.14058	3.68986943464698\\
69.5	0.14424	4.4317688198333\\
69.5	0.1479	5.19921219145154\\
69.5	0.15156	5.9921995495017\\
69.5	0.15522	6.81073089398375\\
69.5	0.15888	7.65480622489773\\
69.5	0.16254	8.52442554224363\\
69.5	0.1662	9.41958884602143\\
69.5	0.16986	10.3402961362311\\
69.5	0.17352	11.2865474128728\\
69.5	0.17718	12.2583426759463\\
69.5	0.18084	13.2556819254518\\
69.5	0.1845	14.2785651613891\\
69.5	0.18816	15.3269923837584\\
69.5	0.19182	16.4009635925596\\
69.5	0.19548	17.5004787877927\\
69.5	0.19914	18.6255379694577\\
69.5	0.2028	19.7761411375547\\
69.5	0.20646	20.9522882920835\\
69.5	0.21012	22.1539794330443\\
69.5	0.21378	23.3812145604369\\
69.5	0.21744	24.6339936742615\\
69.5	0.2211	25.912316774518\\
69.5	0.22476	27.2161838612064\\
69.5	0.22842	28.5455949343268\\
69.5	0.23208	29.900549993879\\
69.5	0.23574	31.2810490398631\\
69.5	0.2394	32.6870920722792\\
69.5	0.24306	34.1186790911272\\
69.5	0.24672	35.5758100964071\\
69.5	0.25038	37.0584850881189\\
69.5	0.25404	38.5667040662626\\
69.5	0.2577	40.1004670308382\\
69.5	0.26136	41.6597739818458\\
69.5	0.26502	43.2446249192852\\
69.5	0.26868	44.8550198431566\\
69.5	0.27234	46.4909587534599\\
69.5	0.276	48.1524416501951\\
69.875	0.093	-3.66583219910051\\
69.875	0.09666	-3.24877617036939\\
69.875	0.10032	-2.80617615520637\\
69.875	0.10398	-2.33803215361144\\
69.875	0.10764	-1.84434416558457\\
69.875	0.1113	-1.3251121911258\\
69.875	0.11496	-0.780336230235136\\
69.875	0.11862	-0.210016282912541\\
69.875	0.12228	0.385847650841967\\
69.875	0.12594	1.00725557102837\\
69.875	0.1296	1.65420747764672\\
69.875	0.13326	2.32670337069695\\
69.875	0.13692	3.0247432501791\\
69.875	0.14058	3.74832711609318\\
69.875	0.14424	4.49745496843918\\
69.875	0.1479	5.27212680721706\\
69.875	0.15156	6.07234263242687\\
69.875	0.15522	6.89810244406858\\
69.875	0.15888	7.74940624214223\\
69.875	0.16254	8.62625402664778\\
69.875	0.1662	9.52864579758524\\
69.875	0.16986	10.4565815549546\\
69.875	0.17352	11.4100612987559\\
69.875	0.17718	12.3890850289891\\
69.875	0.18084	13.3936527456542\\
69.875	0.1845	14.4237644487512\\
69.875	0.18816	15.4794201382802\\
69.875	0.19182	16.560619814241\\
69.875	0.19548	17.6673634766338\\
69.875	0.19914	18.7996511254585\\
69.875	0.2028	19.9574827607151\\
69.875	0.20646	21.1408583824036\\
69.875	0.21012	22.349777990524\\
69.875	0.21378	23.5842415850763\\
69.875	0.21744	24.8442491660606\\
69.875	0.2211	26.1298007334767\\
69.875	0.22476	27.4408962873248\\
69.875	0.22842	28.7775358276048\\
69.875	0.23208	30.1397193543166\\
69.875	0.23574	31.5274468674605\\
69.875	0.2394	32.9407183670362\\
69.875	0.24306	34.3795338530438\\
69.875	0.24672	35.8438933254834\\
69.875	0.25038	37.3337967843548\\
69.875	0.25404	38.8492442296582\\
69.875	0.2577	40.3902356613935\\
69.875	0.26136	41.9567710795607\\
69.875	0.26502	43.5488504841598\\
69.875	0.26868	45.1664738751908\\
69.875	0.27234	46.8096412526537\\
69.875	0.276	48.4783526165486\\
70.25	0.093	-3.69897591312978\\
70.25	0.09666	-3.274691417239\\
70.25	0.10032	-2.82486293491632\\
70.25	0.10398	-2.34949046616173\\
70.25	0.10764	-1.84857401097521\\
70.25	0.1113	-1.32211356935679\\
70.25	0.11496	-0.770109141306463\\
70.25	0.11862	-0.192560726824205\\
70.25	0.12228	0.41053167408996\\
70.25	0.12594	1.03916806143602\\
70.25	0.1296	1.69334843521402\\
70.25	0.13326	2.37307279542392\\
70.25	0.13692	3.07834114206572\\
70.25	0.14058	3.80915347513946\\
70.25	0.14424	4.56550979464512\\
70.25	0.1479	5.34741010058265\\
70.25	0.15156	6.15485439295212\\
70.25	0.15522	6.9878426717535\\
70.25	0.15888	7.84637493698679\\
70.25	0.16254	8.730451188652\\
70.25	0.1662	9.64007142674912\\
70.25	0.16986	10.5752356512781\\
70.25	0.17352	11.5359438622391\\
70.25	0.17718	12.5221960596319\\
70.25	0.18084	13.5339922434567\\
70.25	0.1845	14.5713324137134\\
70.25	0.18816	15.634216570402\\
70.25	0.19182	16.7226447135225\\
70.25	0.19548	17.8366168430749\\
70.25	0.19914	18.9761329590593\\
70.25	0.2028	20.1411930614755\\
70.25	0.20646	21.3317971503237\\
70.25	0.21012	22.5479452256038\\
70.25	0.21378	23.7896372873157\\
70.25	0.21744	25.0568733354596\\
70.25	0.2211	26.3496533700354\\
70.25	0.22476	27.6679773910432\\
70.25	0.22842	29.0118453984828\\
70.25	0.23208	30.3812573923544\\
70.25	0.23574	31.7762133726578\\
70.25	0.2394	33.1967133393932\\
70.25	0.24306	34.6427572925605\\
70.25	0.24672	36.1143452321597\\
70.25	0.25038	37.6114771581908\\
70.25	0.25404	39.1341530706539\\
70.25	0.2577	40.6823729695488\\
70.25	0.26136	42.2561368548756\\
70.25	0.26502	43.8554447266344\\
70.25	0.26868	45.4802965848251\\
70.25	0.27234	47.1306924294477\\
70.25	0.276	48.8066322605022\\
70.625	0.093	-3.72975094955896\\
70.625	0.09666	-3.29823798650853\\
70.625	0.10032	-2.84118103702618\\
70.625	0.10398	-2.35858010111193\\
70.625	0.10764	-1.85043517876577\\
70.625	0.1113	-1.31674626998769\\
70.625	0.11496	-0.75751337477768\\
70.625	0.11862	-0.17273649313578\\
70.625	0.12228	0.437584374938041\\
70.625	0.12594	1.07344922944378\\
70.625	0.1296	1.73485807038141\\
70.625	0.13326	2.42181089775098\\
70.625	0.13692	3.13430771155245\\
70.625	0.14058	3.87234851178584\\
70.625	0.14424	4.63593329845114\\
70.625	0.1479	5.42506207154834\\
70.625	0.15156	6.23973483107747\\
70.625	0.15522	7.0799515770385\\
70.625	0.15888	7.94571230943146\\
70.625	0.16254	8.83701702825632\\
70.625	0.1662	9.7538657335131\\
70.625	0.16986	10.6962584252018\\
70.625	0.17352	11.6641951033224\\
70.625	0.17718	12.6576757678749\\
70.625	0.18084	13.6767004188593\\
70.625	0.1845	14.7212690562757\\
70.625	0.18816	15.7913816801239\\
70.625	0.19182	16.8870382904041\\
70.625	0.19548	18.0082388871162\\
70.625	0.19914	19.1549834702602\\
70.625	0.2028	20.3272720398361\\
70.625	0.20646	21.5251045958439\\
70.625	0.21012	22.7484811382837\\
70.625	0.21378	23.9974016671552\\
70.625	0.21744	25.2718661824588\\
70.625	0.2211	26.5718746841943\\
70.625	0.22476	27.8974271723617\\
70.625	0.22842	29.248523646961\\
70.625	0.23208	30.6251641079922\\
70.625	0.23574	32.0273485554553\\
70.625	0.2394	33.4550769893504\\
70.625	0.24306	34.9083494096773\\
70.625	0.24672	36.3871658164361\\
70.625	0.25038	37.891526209627\\
70.625	0.25404	39.4214305892496\\
70.625	0.2577	40.9768789553042\\
70.625	0.26136	42.5578713077907\\
70.625	0.26502	44.1644076467092\\
70.625	0.26868	45.7964879720595\\
70.625	0.27234	47.4541122838417\\
70.625	0.276	49.1372805820559\\
71	0.093	-3.75815730838807\\
71	0.09666	-3.31941587817799\\
71	0.10032	-2.85513046153598\\
71	0.10398	-2.36530105846206\\
71	0.10764	-1.84992766895625\\
71	0.1113	-1.30901029301851\\
71	0.11496	-0.742548930648844\\
71	0.11862	-0.150543581847288\\
71	0.12228	0.467005753386196\\
71	0.12594	1.11009907505159\\
71	0.1296	1.77873638314888\\
71	0.13326	2.4729176776781\\
71	0.13692	3.19264295863923\\
71	0.14058	3.93791222603229\\
71	0.14424	4.70872547985723\\
71	0.1479	5.5050827201141\\
71	0.15156	6.32698394680289\\
71	0.15522	7.17442915992357\\
71	0.15888	8.04741835947618\\
71	0.16254	8.94595154546071\\
71	0.1662	9.87002871787715\\
71	0.16986	10.8196498767255\\
71	0.17352	11.7948150220057\\
71	0.17718	12.7955241537179\\
71	0.18084	13.821777271862\\
71	0.1845	14.873574376438\\
71	0.18816	15.9509154674459\\
71	0.19182	17.0538005448857\\
71	0.19548	18.1822296087575\\
71	0.19914	19.3362026590611\\
71	0.2028	20.5157196957967\\
71	0.20646	21.7207807189642\\
71	0.21012	22.9513857285636\\
71	0.21378	24.2075347245949\\
71	0.21744	25.4892277070581\\
71	0.2211	26.7964646759532\\
71	0.22476	28.1292456312803\\
71	0.22842	29.4875705730392\\
71	0.23208	30.87143950123\\
71	0.23574	32.2808524158529\\
71	0.2394	33.7158093169076\\
71	0.24306	35.1763102043941\\
71	0.24672	36.6623550783127\\
71	0.25038	38.1739439386631\\
71	0.25404	39.7110767854455\\
71	0.2577	41.2737536186597\\
71	0.26136	42.8619744383059\\
71	0.26502	44.475739244384\\
71	0.26868	46.1150480368939\\
71	0.27234	47.7799008158358\\
71	0.276	49.4702975812097\\
71.375	0.093	-3.78419498961708\\
71.375	0.09666	-3.33822509224733\\
71.375	0.10032	-2.86671120844568\\
71.375	0.10398	-2.36965333821212\\
71.375	0.10764	-1.84705148154662\\
71.375	0.1113	-1.29890563844922\\
71.375	0.11496	-0.72521580891992\\
71.375	0.11862	-0.1259819929587\\
71.375	0.12228	0.498795809434441\\
71.375	0.12594	1.14911759825948\\
71.375	0.1296	1.82498337351646\\
71.375	0.13326	2.52639313520532\\
71.375	0.13692	3.2533468833261\\
71.375	0.14058	4.0058446178788\\
71.375	0.14424	4.78388633886344\\
71.375	0.1479	5.58747204627995\\
71.375	0.15156	6.4166017401284\\
71.375	0.15522	7.27127542040874\\
71.375	0.15888	8.15149308712101\\
71.375	0.16254	9.05725474026519\\
71.375	0.1662	9.98856037984129\\
71.375	0.16986	10.9454100058493\\
71.375	0.17352	11.9278036182892\\
71.375	0.17718	12.935741217161\\
71.375	0.18084	13.9692228024648\\
71.375	0.1845	15.0282483742004\\
71.375	0.18816	16.112817932368\\
71.375	0.19182	17.2229314769675\\
71.375	0.19548	18.3585890079989\\
71.375	0.19914	19.5197905254622\\
71.375	0.2028	20.7065360293574\\
71.375	0.20646	21.9188255196845\\
71.375	0.21012	23.1566589964436\\
71.375	0.21378	24.4200364596346\\
71.375	0.21744	25.7089579092574\\
71.375	0.2211	27.0234233453122\\
71.375	0.22476	28.3634327677989\\
71.375	0.22842	29.7289861767175\\
71.375	0.23208	31.120083572068\\
71.375	0.23574	32.5367249538505\\
71.375	0.2394	33.9789103220648\\
71.375	0.24306	35.4466396767111\\
71.375	0.24672	36.9399130177893\\
71.375	0.25038	38.4587303452994\\
71.375	0.25404	40.0030916592414\\
71.375	0.2577	41.5729969596153\\
71.375	0.26136	43.1684462464211\\
71.375	0.26502	44.7894395196589\\
71.375	0.26868	46.4359767793285\\
71.375	0.27234	48.1080580254301\\
71.375	0.276	49.8056832579636\\
71.75	0.093	-3.80786399324603\\
71.75	0.09666	-3.35466562871662\\
71.75	0.10032	-2.87592327775531\\
71.75	0.10398	-2.37163694036209\\
71.75	0.10764	-1.84180661653693\\
71.75	0.1113	-1.28643230627988\\
71.75	0.11496	-0.705514009590921\\
71.75	0.11862	-0.0990517264700372\\
71.75	0.12228	0.532954543082759\\
71.75	0.12594	1.19050479906745\\
71.75	0.1296	1.87359904148409\\
71.75	0.13326	2.58223727033261\\
71.75	0.13692	3.31641948561304\\
71.75	0.14058	4.07614568732541\\
71.75	0.14424	4.8614158754697\\
71.75	0.1479	5.67223005004587\\
71.75	0.15156	6.50858821105397\\
71.75	0.15522	7.37049035849397\\
71.75	0.15888	8.25793649236591\\
71.75	0.16254	9.17092661266975\\
71.75	0.1662	10.1094607194055\\
71.75	0.16986	11.0735388125731\\
71.75	0.17352	12.0631608921727\\
71.75	0.17718	13.0783269582042\\
71.75	0.18084	14.1190370106676\\
71.75	0.1845	15.1852910495629\\
71.75	0.18816	16.2770890748902\\
71.75	0.19182	17.3944310866493\\
71.75	0.19548	18.5373170848404\\
71.75	0.19914	19.7057470694633\\
71.75	0.2028	20.8997210405182\\
71.75	0.20646	22.119238998005\\
71.75	0.21012	23.3643009419237\\
71.75	0.21378	24.6349068722743\\
71.75	0.21744	25.9310567890569\\
71.75	0.2211	27.2527506922713\\
71.75	0.22476	28.5999885819176\\
71.75	0.22842	29.9727704579959\\
71.75	0.23208	31.3710963205061\\
71.75	0.23574	32.7949661694482\\
71.75	0.2394	34.2443800048222\\
71.75	0.24306	35.7193378266281\\
71.75	0.24672	37.219839634866\\
71.75	0.25038	38.7458854295357\\
71.75	0.25404	40.2974752106374\\
71.75	0.2577	41.874608978171\\
71.75	0.26136	43.4772867321364\\
71.75	0.26502	45.1055084725338\\
71.75	0.26868	46.7592741993631\\
71.75	0.27234	48.4385839126244\\
71.75	0.276	50.1434376123175\\
72.125	0.093	-3.82916431927489\\
72.125	0.09666	-3.36873748758583\\
72.125	0.10032	-2.88276666946486\\
72.125	0.10398	-2.37125186491198\\
72.125	0.10764	-1.83419307392716\\
72.125	0.1113	-1.27159029651045\\
72.125	0.11496	-0.68344353266184\\
72.125	0.11862	-0.0697527823813004\\
72.125	0.12228	0.569481954331152\\
72.125	0.12594	1.23426067747551\\
72.125	0.1296	1.9245833870518\\
72.125	0.13326	2.64045008305998\\
72.125	0.13692	3.38186076550007\\
72.125	0.14058	4.14881543437209\\
72.125	0.14424	4.94131408967604\\
72.125	0.1479	5.75935673141187\\
72.125	0.15156	6.60294335957963\\
72.125	0.15522	7.47207397417929\\
72.125	0.15888	8.36674857521088\\
72.125	0.16254	9.28696716267437\\
72.125	0.1662	10.2327297365698\\
72.125	0.16986	11.2040362968971\\
72.125	0.17352	12.2008868436563\\
72.125	0.17718	13.2232813768475\\
72.125	0.18084	14.2712198964705\\
72.125	0.1845	15.3447024025255\\
72.125	0.18816	16.4437288950124\\
72.125	0.19182	17.5682993739312\\
72.125	0.19548	18.7184138392819\\
72.125	0.19914	19.8940722910645\\
72.125	0.2028	21.0952747292791\\
72.125	0.20646	22.3220211539255\\
72.125	0.21012	23.5743115650039\\
72.125	0.21378	24.8521459625142\\
72.125	0.21744	26.1555243464563\\
72.125	0.2211	27.4844467168304\\
72.125	0.22476	28.8389130736365\\
72.125	0.22842	30.2189234168744\\
72.125	0.23208	31.6244777465442\\
72.125	0.23574	33.055576062646\\
72.125	0.2394	34.5122183651796\\
72.125	0.24306	35.9944046541452\\
72.125	0.24672	37.5021349295427\\
72.125	0.25038	39.0354091913721\\
72.125	0.25404	40.5942274396335\\
72.125	0.2577	42.1785896743267\\
72.125	0.26136	43.7884958954518\\
72.125	0.26502	45.4239461030089\\
72.125	0.26868	47.0849402969978\\
72.125	0.27234	48.7714784774187\\
72.125	0.276	50.4835606442715\\
72.5	0.093	-3.84809596770366\\
72.5	0.09666	-3.38044066885495\\
72.5	0.10032	-2.88724138357431\\
72.5	0.10398	-2.36849811186176\\
72.5	0.10764	-1.82421085371732\\
72.5	0.1113	-1.25437960914095\\
72.5	0.11496	-0.65900437813265\\
72.5	0.11862	-0.0380851606924679\\
72.5	0.12228	0.608378043179648\\
72.5	0.12594	1.28038523348367\\
72.5	0.1296	1.97793641021959\\
72.5	0.13326	2.70103157338744\\
72.5	0.13692	3.44967072298721\\
72.5	0.14058	4.22385385901889\\
72.5	0.14424	5.02358098148247\\
72.5	0.1479	5.84885209037797\\
72.5	0.15156	6.69966718570538\\
72.5	0.15522	7.5760262674647\\
72.5	0.15888	8.47792933565594\\
72.5	0.16254	9.4053763902791\\
72.5	0.1662	10.3583674313342\\
72.5	0.16986	11.3369024588211\\
72.5	0.17352	12.34098147274\\
72.5	0.17718	13.3706044730908\\
72.5	0.18084	14.4257714598736\\
72.5	0.1845	15.5064824330882\\
72.5	0.18816	16.6127373927347\\
72.5	0.19182	17.7445363388132\\
72.5	0.19548	18.9018792713235\\
72.5	0.19914	20.0847661902658\\
72.5	0.2028	21.29319709564\\
72.5	0.20646	22.5271719874461\\
72.5	0.21012	23.7866908656842\\
72.5	0.21378	25.0717537303541\\
72.5	0.21744	26.3823605814559\\
72.5	0.2211	27.7185114189897\\
72.5	0.22476	29.0802062429554\\
72.5	0.22842	30.4674450533529\\
72.5	0.23208	31.8802278501825\\
72.5	0.23574	33.3185546334439\\
72.5	0.2394	34.7824254031372\\
72.5	0.24306	36.2718401592624\\
72.5	0.24672	37.7867989018196\\
72.5	0.25038	39.3273016308087\\
72.5	0.25404	40.8933483462296\\
72.5	0.2577	42.4849390480825\\
72.5	0.26136	44.1020737363673\\
72.5	0.26502	45.744752411084\\
72.5	0.26868	47.4129750722326\\
72.5	0.27234	49.1067417198132\\
72.5	0.276	50.8260523538256\\
72.875	0.093	-3.86465893853234\\
72.875	0.09666	-3.38977517252396\\
72.875	0.10032	-2.88934742008367\\
72.875	0.10398	-2.36337568121149\\
72.875	0.10764	-1.81185995590735\\
72.875	0.1113	-1.23480024417133\\
72.875	0.11496	-0.632196546003399\\
72.875	0.11862	-0.00404886140353966\\
72.875	0.12228	0.649642809628226\\
72.875	0.12594	1.3288784670919\\
72.875	0.1296	2.03365811098751\\
72.875	0.13326	2.763981741315\\
72.875	0.13692	3.51984935807441\\
72.875	0.14058	4.30126096126574\\
72.875	0.14424	5.10821655088901\\
72.875	0.1479	5.94071612694415\\
72.875	0.15156	6.79875968943123\\
72.875	0.15522	7.6823472383502\\
72.875	0.15888	8.59147877370111\\
72.875	0.16254	9.52615429548392\\
72.875	0.1662	10.4863738036986\\
72.875	0.16986	11.4721372983453\\
72.875	0.17352	12.4834447794238\\
72.875	0.17718	13.5202962469343\\
72.875	0.18084	14.5826917008767\\
72.875	0.1845	15.6706311412509\\
72.875	0.18816	16.7841145680571\\
72.875	0.19182	17.9231419812952\\
72.875	0.19548	19.0877133809653\\
72.875	0.19914	20.2778287670672\\
72.875	0.2028	21.4934881396011\\
72.875	0.20646	22.7346914985669\\
72.875	0.21012	24.0014388439645\\
72.875	0.21378	25.2937301757941\\
72.875	0.21744	26.6115654940556\\
72.875	0.2211	27.954944798749\\
72.875	0.22476	29.3238680898743\\
72.875	0.22842	30.7183353674316\\
72.875	0.23208	32.1383466314208\\
72.875	0.23574	33.5839018818418\\
72.875	0.2394	35.0550011186948\\
72.875	0.24306	36.5516443419797\\
72.875	0.24672	38.0738315516965\\
72.875	0.25038	39.6215627478452\\
72.875	0.25404	41.1948379304259\\
72.875	0.2577	42.7936570994384\\
72.875	0.26136	44.4180202548829\\
72.875	0.26502	46.0679273967592\\
72.875	0.26868	47.7433785250675\\
72.875	0.27234	49.4443736398077\\
72.875	0.276	51.1709127409798\\
73.25	0.093	-3.87885323176096\\
73.25	0.09666	-3.39674099859292\\
73.25	0.10032	-2.88908477899297\\
73.25	0.10398	-2.35588457296113\\
73.25	0.10764	-1.79714038049733\\
73.25	0.1113	-1.21285220160166\\
73.25	0.11496	-0.603020036274067\\
73.25	0.11862	0.0323561154854488\\
73.25	0.12228	0.693276253676878\\
73.25	0.12594	1.37974037830021\\
73.25	0.1296	2.09174848935547\\
73.25	0.13326	2.82930058684262\\
73.25	0.13692	3.59239667076168\\
73.25	0.14058	4.38103674111268\\
73.25	0.14424	5.19522079789561\\
73.25	0.1479	6.0349488411104\\
73.25	0.15156	6.90022087075713\\
73.25	0.15522	7.79103688683577\\
73.25	0.15888	8.70739688934633\\
73.25	0.16254	9.6493008782888\\
73.25	0.1662	10.6167488536632\\
73.25	0.16986	11.6097408154695\\
73.25	0.17352	12.6282767637077\\
73.25	0.17718	13.6723566983778\\
73.25	0.18084	14.7419806194798\\
73.25	0.1845	15.8371485270138\\
73.25	0.18816	16.9578604209796\\
73.25	0.19182	18.1041163013774\\
73.25	0.19548	19.2759161682071\\
73.25	0.19914	20.4732600214687\\
73.25	0.2028	21.6961478611622\\
73.25	0.20646	22.9445796872876\\
73.25	0.21012	24.218555499845\\
73.25	0.21378	25.5180752988342\\
73.25	0.21744	26.8431390842554\\
73.25	0.2211	28.1937468561084\\
73.25	0.22476	29.5698986143934\\
73.25	0.22842	30.9715943591103\\
73.25	0.23208	32.3988340902591\\
73.25	0.23574	33.8516178078399\\
73.25	0.2394	35.3299455118525\\
73.25	0.24306	36.8338172022971\\
73.25	0.24672	38.3632328791735\\
73.25	0.25038	39.9181925424819\\
73.25	0.25404	41.4986961922222\\
73.25	0.2577	43.1047438283944\\
73.25	0.26136	44.7363354509985\\
73.25	0.26502	46.3934710600346\\
73.25	0.26868	48.0761506555025\\
73.25	0.27234	49.7843742374023\\
73.25	0.276	51.5181418057341\\
73.625	0.093	-3.89067884738949\\
73.625	0.09666	-3.40133814706179\\
73.625	0.10032	-2.88645346030219\\
73.625	0.10398	-2.34602478711069\\
73.625	0.10764	-1.78005212748724\\
73.625	0.1113	-1.1885354814319\\
73.625	0.11496	-0.571474848944654\\
73.625	0.11862	0.0711297699745188\\
73.625	0.12228	0.739278375325604\\
73.625	0.12594	1.43297096710859\\
73.625	0.1296	2.15220754532352\\
73.625	0.13326	2.89698810997033\\
73.625	0.13692	3.66731266104905\\
73.625	0.14058	4.4631811985597\\
73.625	0.14424	5.28459372250228\\
73.625	0.1479	6.13155023287674\\
73.625	0.15156	7.00405072968313\\
73.625	0.15522	7.90209521292142\\
73.625	0.15888	8.82568368259164\\
73.625	0.16254	9.77481613869377\\
73.625	0.1662	10.7494925812278\\
73.625	0.16986	11.7497130101937\\
73.625	0.17352	12.7754774255916\\
73.625	0.17718	13.8267858274214\\
73.625	0.18084	14.9036382156831\\
73.625	0.1845	16.0060345903767\\
73.625	0.18816	17.1339749515022\\
73.625	0.19182	18.2874592990596\\
73.625	0.19548	19.466487633049\\
73.625	0.19914	20.6710599534702\\
73.625	0.2028	21.9011762603234\\
73.625	0.20646	23.1568365536085\\
73.625	0.21012	24.4380408333255\\
73.625	0.21378	25.7447890994744\\
73.625	0.21744	27.0770813520552\\
73.625	0.2211	28.4349175910679\\
73.625	0.22476	29.8182978165126\\
73.625	0.22842	31.2272220283891\\
73.625	0.23208	32.6616902266976\\
73.625	0.23574	34.121702411438\\
73.625	0.2394	35.6072585826103\\
73.625	0.24306	37.1183587402145\\
73.625	0.24672	38.6550028842506\\
73.625	0.25038	40.2171910147187\\
73.625	0.25404	41.8049231316186\\
73.625	0.2577	43.4181992349505\\
73.625	0.26136	45.0570193247142\\
73.625	0.26502	46.7213834009099\\
73.625	0.26868	48.4112914635375\\
73.625	0.27234	50.126743512597\\
73.625	0.276	51.8677395480885\\
74	0.093	-3.90013578541794\\
74	0.09666	-3.40356661793058\\
74	0.10032	-2.88145346401132\\
74	0.10398	-2.33379632366016\\
74	0.10764	-1.76059519687706\\
74	0.1113	-1.16185008366206\\
74	0.11496	-0.537560984015158\\
74	0.11862	0.112272102063677\\
74	0.12228	0.787649174574419\\
74	0.12594	1.48857023351706\\
74	0.1296	2.21503527889164\\
74	0.13326	2.96704431069811\\
74	0.13692	3.74459732893649\\
74	0.14058	4.5476943336068\\
74	0.14424	5.37633532470904\\
74	0.1479	6.23052030224315\\
74	0.15156	7.11024926620921\\
74	0.15522	8.01552221660715\\
74	0.15888	8.94633915343703\\
74	0.16254	9.90270007669881\\
74	0.1662	10.8846049863925\\
74	0.16986	11.8920538825181\\
74	0.17352	12.9250467650756\\
74	0.17718	13.983583634065\\
74	0.18084	15.0676644894864\\
74	0.1845	16.1772893313397\\
74	0.18816	17.3124581596249\\
74	0.19182	18.473170974342\\
74	0.19548	19.6594277754909\\
74	0.19914	20.8712285630718\\
74	0.2028	22.1085733370847\\
74	0.20646	23.3714620975294\\
74	0.21012	24.6598948444061\\
74	0.21378	25.9738715777146\\
74	0.21744	27.3133922974551\\
74	0.2211	28.6784570036275\\
74	0.22476	30.0690656962318\\
74	0.22842	31.485218375268\\
74	0.23208	32.9269150407361\\
74	0.23574	34.3941556926362\\
74	0.2394	35.8869403309681\\
74	0.24306	37.405268955732\\
74	0.24672	38.9491415669278\\
74	0.25038	40.5185581645555\\
74	0.25404	42.1135187486151\\
74	0.2577	43.7340233191066\\
74	0.26136	45.3800718760301\\
74	0.26502	47.0516644193854\\
74	0.26868	48.7488009491726\\
74	0.27234	50.4714814653918\\
74	0.276	52.2197059680429\\
};
\end{axis}

\begin{axis}[%
width=4.527496cm,
height=3.870968cm,
at={(6.483547cm,16.129032cm)},
scale only axis,
xmin=56,
xmax=74,
tick align=outside,
xlabel={$L_{cut}$},
xmajorgrids,
ymin=0.093,
ymax=0.276,
ylabel={$D_{rlx}$},
ymajorgrids,
zmin=0,
zmax=517.903625289077,
zlabel={$x_3$},
zmajorgrids,
view={-140}{50},
legend style={at={(1.03,1)},anchor=north west,legend cell align=left,align=left,draw=white!15!black}
]
\addplot3[only marks,mark=*,mark options={},mark size=1.5000pt,color=mycolor1] plot table[row sep=crcr,]{%
74	0.123	75.4245651900459\\
72	0.113	59.5782863482935\\
61	0.095	33.0003406266977\\
56	0.093	26.0569470415911\\
};
\addplot3[only marks,mark=*,mark options={},mark size=1.5000pt,color=mycolor2] plot table[row sep=crcr,]{%
67	0.276	508.93731875208\\
66	0.255	419.53506953587\\
62	0.209	242.35439857544\\
57	0.193	195.15231965962\\
};
\addplot3[only marks,mark=*,mark options={},mark size=1.5000pt,color=black] plot table[row sep=crcr,]{%
69	0.104	48.1087944305374\\
};
\addplot3[only marks,mark=*,mark options={},mark size=1.5000pt,color=black] plot table[row sep=crcr,]{%
64	0.23	317.238786352579\\
};

\addplot3[%
surf,
opacity=0.7,
shader=interp,
colormap={mymap}{[1pt] rgb(0pt)=(0.0901961,0.239216,0.0745098); rgb(1pt)=(0.0945149,0.242058,0.0739522); rgb(2pt)=(0.0988592,0.244894,0.0733566); rgb(3pt)=(0.103229,0.247724,0.0727241); rgb(4pt)=(0.107623,0.250549,0.0720557); rgb(5pt)=(0.112043,0.253367,0.0713525); rgb(6pt)=(0.116487,0.25618,0.0706154); rgb(7pt)=(0.120956,0.258986,0.0698456); rgb(8pt)=(0.125449,0.261787,0.0690441); rgb(9pt)=(0.129967,0.264581,0.0682118); rgb(10pt)=(0.134508,0.26737,0.06735); rgb(11pt)=(0.139074,0.270152,0.0664596); rgb(12pt)=(0.143663,0.272929,0.0655416); rgb(13pt)=(0.148275,0.275699,0.0645971); rgb(14pt)=(0.152911,0.278463,0.0636271); rgb(15pt)=(0.15757,0.281221,0.0626328); rgb(16pt)=(0.162252,0.283973,0.0616151); rgb(17pt)=(0.166957,0.286719,0.060575); rgb(18pt)=(0.171685,0.289458,0.0595136); rgb(19pt)=(0.176434,0.292191,0.0584321); rgb(20pt)=(0.181207,0.294918,0.0573313); rgb(21pt)=(0.186001,0.297639,0.0562123); rgb(22pt)=(0.190817,0.300353,0.0550763); rgb(23pt)=(0.195655,0.303061,0.0539242); rgb(24pt)=(0.200514,0.305763,0.052757); rgb(25pt)=(0.205395,0.308459,0.0515759); rgb(26pt)=(0.210296,0.311149,0.0503624); rgb(27pt)=(0.215212,0.313846,0.0490067); rgb(28pt)=(0.220142,0.316548,0.0475043); rgb(29pt)=(0.22509,0.319254,0.0458704); rgb(30pt)=(0.230056,0.321962,0.0441205); rgb(31pt)=(0.235042,0.324671,0.04227); rgb(32pt)=(0.240048,0.327379,0.0403343); rgb(33pt)=(0.245078,0.330085,0.0383287); rgb(34pt)=(0.250131,0.332786,0.0362688); rgb(35pt)=(0.25521,0.335482,0.0341698); rgb(36pt)=(0.260317,0.33817,0.0320472); rgb(37pt)=(0.265451,0.340849,0.0299163); rgb(38pt)=(0.270616,0.343517,0.0277927); rgb(39pt)=(0.275813,0.346172,0.0256916); rgb(40pt)=(0.281043,0.348814,0.0236284); rgb(41pt)=(0.286307,0.35144,0.0216186); rgb(42pt)=(0.291607,0.354048,0.0196776); rgb(43pt)=(0.296945,0.356637,0.0178207); rgb(44pt)=(0.302322,0.359206,0.0160634); rgb(45pt)=(0.307739,0.361753,0.0144211); rgb(46pt)=(0.313198,0.364275,0.0129091); rgb(47pt)=(0.318701,0.366772,0.0115428); rgb(48pt)=(0.324249,0.369242,0.0103377); rgb(49pt)=(0.329843,0.371682,0.00930909); rgb(50pt)=(0.335485,0.374093,0.00847245); rgb(51pt)=(0.341176,0.376471,0.00784314); rgb(52pt)=(0.346925,0.378826,0.00732741); rgb(53pt)=(0.352735,0.381168,0.00682184); rgb(54pt)=(0.358605,0.383497,0.00632729); rgb(55pt)=(0.364532,0.385812,0.00584464); rgb(56pt)=(0.370516,0.388113,0.00537476); rgb(57pt)=(0.376552,0.390399,0.00491852); rgb(58pt)=(0.38264,0.39267,0.00447681); rgb(59pt)=(0.388777,0.394925,0.00405048); rgb(60pt)=(0.394962,0.397164,0.00364042); rgb(61pt)=(0.401191,0.399386,0.00324749); rgb(62pt)=(0.407464,0.401592,0.00287258); rgb(63pt)=(0.413777,0.40378,0.00251655); rgb(64pt)=(0.420129,0.40595,0.00218028); rgb(65pt)=(0.426518,0.408102,0.00186463); rgb(66pt)=(0.432942,0.410234,0.00157049); rgb(67pt)=(0.439399,0.412348,0.00129873); rgb(68pt)=(0.445885,0.414441,0.00105022); rgb(69pt)=(0.452401,0.416515,0.000825833); rgb(70pt)=(0.458942,0.418567,0.000626441); rgb(71pt)=(0.465508,0.420599,0.00045292); rgb(72pt)=(0.472096,0.422609,0.000306141); rgb(73pt)=(0.478704,0.424596,0.000186979); rgb(74pt)=(0.485331,0.426562,9.63073e-05); rgb(75pt)=(0.491973,0.428504,3.49981e-05); rgb(76pt)=(0.498628,0.430422,3.92506e-06); rgb(77pt)=(0.505323,0.432315,0); rgb(78pt)=(0.512206,0.434168,0); rgb(79pt)=(0.519282,0.435983,0); rgb(80pt)=(0.526529,0.437764,0); rgb(81pt)=(0.533922,0.439512,0); rgb(82pt)=(0.54144,0.441232,0); rgb(83pt)=(0.549059,0.442927,0); rgb(84pt)=(0.556756,0.444599,0); rgb(85pt)=(0.564508,0.446252,0); rgb(86pt)=(0.572292,0.447889,0); rgb(87pt)=(0.580084,0.449514,0); rgb(88pt)=(0.587863,0.451129,0); rgb(89pt)=(0.595604,0.452737,0); rgb(90pt)=(0.603284,0.454343,0); rgb(91pt)=(0.610882,0.455948,0); rgb(92pt)=(0.618373,0.457556,0); rgb(93pt)=(0.625734,0.459171,0); rgb(94pt)=(0.632943,0.460795,0); rgb(95pt)=(0.639976,0.462432,0); rgb(96pt)=(0.64681,0.464084,0); rgb(97pt)=(0.653423,0.465756,0); rgb(98pt)=(0.659791,0.46745,0); rgb(99pt)=(0.665891,0.469169,0); rgb(100pt)=(0.6717,0.470916,0); rgb(101pt)=(0.677195,0.472696,0); rgb(102pt)=(0.682353,0.47451,0); rgb(103pt)=(0.687242,0.476355,0); rgb(104pt)=(0.691952,0.478225,0); rgb(105pt)=(0.696497,0.480118,0); rgb(106pt)=(0.700887,0.482033,0); rgb(107pt)=(0.705134,0.483968,0); rgb(108pt)=(0.709251,0.485921,0); rgb(109pt)=(0.713249,0.487891,0); rgb(110pt)=(0.71714,0.489876,0); rgb(111pt)=(0.720936,0.491875,0); rgb(112pt)=(0.724649,0.493887,0); rgb(113pt)=(0.72829,0.495909,0); rgb(114pt)=(0.731872,0.49794,0); rgb(115pt)=(0.735406,0.499979,0); rgb(116pt)=(0.738904,0.502025,0); rgb(117pt)=(0.742378,0.504075,0); rgb(118pt)=(0.74584,0.506128,0); rgb(119pt)=(0.749302,0.508182,0); rgb(120pt)=(0.752775,0.510237,0); rgb(121pt)=(0.756272,0.51229,0); rgb(122pt)=(0.759804,0.514339,0); rgb(123pt)=(0.763384,0.516385,0); rgb(124pt)=(0.767022,0.518424,0); rgb(125pt)=(0.770731,0.520455,0); rgb(126pt)=(0.774523,0.522478,0); rgb(127pt)=(0.77841,0.524489,0); rgb(128pt)=(0.782391,0.526491,0); rgb(129pt)=(0.786402,0.528496,0); rgb(130pt)=(0.790431,0.530506,0); rgb(131pt)=(0.794478,0.532521,0); rgb(132pt)=(0.798541,0.534539,0); rgb(133pt)=(0.802619,0.53656,0); rgb(134pt)=(0.806712,0.538584,0); rgb(135pt)=(0.81082,0.540609,0); rgb(136pt)=(0.81494,0.542635,0); rgb(137pt)=(0.819074,0.54466,0); rgb(138pt)=(0.823219,0.546686,0); rgb(139pt)=(0.827374,0.548709,0); rgb(140pt)=(0.831541,0.55073,0); rgb(141pt)=(0.835716,0.552749,0); rgb(142pt)=(0.8399,0.554763,0); rgb(143pt)=(0.844092,0.556774,0); rgb(144pt)=(0.848292,0.558779,0); rgb(145pt)=(0.852497,0.560778,0); rgb(146pt)=(0.856708,0.562771,0); rgb(147pt)=(0.860924,0.564756,0); rgb(148pt)=(0.865143,0.566733,0); rgb(149pt)=(0.869366,0.568701,0); rgb(150pt)=(0.873592,0.57066,0); rgb(151pt)=(0.877819,0.572608,0); rgb(152pt)=(0.882047,0.574545,0); rgb(153pt)=(0.886275,0.576471,0); rgb(154pt)=(0.890659,0.578362,0); rgb(155pt)=(0.895333,0.580203,0); rgb(156pt)=(0.900258,0.581999,0); rgb(157pt)=(0.905397,0.583755,0); rgb(158pt)=(0.910711,0.585479,0); rgb(159pt)=(0.916164,0.587176,0); rgb(160pt)=(0.921717,0.588852,0); rgb(161pt)=(0.927333,0.590513,0); rgb(162pt)=(0.932974,0.592166,0); rgb(163pt)=(0.938602,0.593815,0); rgb(164pt)=(0.94418,0.595468,0); rgb(165pt)=(0.949669,0.59713,0); rgb(166pt)=(0.955033,0.598808,0); rgb(167pt)=(0.960233,0.600507,0); rgb(168pt)=(0.965232,0.602233,0); rgb(169pt)=(0.969992,0.603992,0); rgb(170pt)=(0.974475,0.605791,0); rgb(171pt)=(0.978643,0.607636,0); rgb(172pt)=(0.98246,0.609532,0); rgb(173pt)=(0.985886,0.611486,0); rgb(174pt)=(0.988885,0.613503,0); rgb(175pt)=(0.991419,0.61559,0); rgb(176pt)=(0.99345,0.617753,0); rgb(177pt)=(0.99494,0.619997,0); rgb(178pt)=(0.995851,0.622329,0); rgb(179pt)=(0.996226,0.624763,0); rgb(180pt)=(0.996512,0.627352,0); rgb(181pt)=(0.996788,0.630095,0); rgb(182pt)=(0.997053,0.632982,0); rgb(183pt)=(0.997308,0.636004,0); rgb(184pt)=(0.997552,0.639152,0); rgb(185pt)=(0.997785,0.642416,0); rgb(186pt)=(0.998006,0.645786,0); rgb(187pt)=(0.998217,0.649253,0); rgb(188pt)=(0.998416,0.652807,0); rgb(189pt)=(0.998605,0.656439,0); rgb(190pt)=(0.998781,0.660138,0); rgb(191pt)=(0.998946,0.663897,0); rgb(192pt)=(0.9991,0.667704,0); rgb(193pt)=(0.999242,0.67155,0); rgb(194pt)=(0.999372,0.675427,0); rgb(195pt)=(0.99949,0.679323,0); rgb(196pt)=(0.999596,0.68323,0); rgb(197pt)=(0.99969,0.687139,0); rgb(198pt)=(0.999771,0.691039,0); rgb(199pt)=(0.999841,0.694921,0); rgb(200pt)=(0.999898,0.698775,0); rgb(201pt)=(0.999942,0.702592,0); rgb(202pt)=(0.999974,0.706363,0); rgb(203pt)=(0.999994,0.710077,0); rgb(204pt)=(1,0.713725,0); rgb(205pt)=(1,0.717341,0); rgb(206pt)=(1,0.720963,0); rgb(207pt)=(1,0.724591,0); rgb(208pt)=(1,0.728226,0); rgb(209pt)=(1,0.731867,0); rgb(210pt)=(1,0.735514,0); rgb(211pt)=(1,0.739167,0); rgb(212pt)=(1,0.742827,0); rgb(213pt)=(1,0.746493,0); rgb(214pt)=(1,0.750165,0); rgb(215pt)=(1,0.753843,0); rgb(216pt)=(1,0.757527,0); rgb(217pt)=(1,0.761217,0); rgb(218pt)=(1,0.764913,0); rgb(219pt)=(1,0.768615,0); rgb(220pt)=(1,0.772324,0); rgb(221pt)=(1,0.776038,0); rgb(222pt)=(1,0.779758,0); rgb(223pt)=(1,0.783484,0); rgb(224pt)=(1,0.787215,0); rgb(225pt)=(1,0.790953,0); rgb(226pt)=(1,0.794696,0); rgb(227pt)=(1,0.798445,0); rgb(228pt)=(1,0.8022,0); rgb(229pt)=(1,0.805961,0); rgb(230pt)=(1,0.809727,0); rgb(231pt)=(1,0.8135,0); rgb(232pt)=(1,0.817278,0); rgb(233pt)=(1,0.821063,0); rgb(234pt)=(1,0.824854,0); rgb(235pt)=(1,0.828652,0); rgb(236pt)=(1,0.832455,0); rgb(237pt)=(1,0.836265,0); rgb(238pt)=(1,0.840081,0); rgb(239pt)=(1,0.843903,0); rgb(240pt)=(1,0.847732,0); rgb(241pt)=(1,0.851566,0); rgb(242pt)=(1,0.855406,0); rgb(243pt)=(1,0.859253,0); rgb(244pt)=(1,0.863106,0); rgb(245pt)=(1,0.866964,0); rgb(246pt)=(1,0.870829,0); rgb(247pt)=(1,0.8747,0); rgb(248pt)=(1,0.878577,0); rgb(249pt)=(1,0.88246,0); rgb(250pt)=(1,0.886349,0); rgb(251pt)=(1,0.890243,0); rgb(252pt)=(1,0.894144,0); rgb(253pt)=(1,0.898051,0); rgb(254pt)=(1,0.901964,0); rgb(255pt)=(1,0.905882,0)},
mesh/rows=49]
table[row sep=crcr,header=false] {%
%
56	0.093	26.7123045686334\\
56	0.09666	28.8008271328639\\
56	0.10032	31.1960133736808\\
56	0.10398	33.897863291084\\
56	0.10764	36.9063768850736\\
56	0.1113	40.2215541556496\\
56	0.11496	43.8433951028119\\
56	0.11862	47.7718997265605\\
56	0.12228	52.0070680268955\\
56	0.12594	56.5489000038169\\
56	0.1296	61.3973956573246\\
56	0.13326	66.5525549874186\\
56	0.13692	72.014377994099\\
56	0.14058	77.7828646773658\\
56	0.14424	83.8580150372189\\
56	0.1479	90.2398290736583\\
56	0.15156	96.9283067866842\\
56	0.15522	103.923448176296\\
56	0.15888	111.225253242495\\
56	0.16254	118.83372198528\\
56	0.1662	126.748854404651\\
56	0.16986	134.970650500609\\
56	0.17352	143.499110273152\\
56	0.17718	152.334233722283\\
56	0.18084	161.476020847999\\
56	0.1845	170.924471650302\\
56	0.18816	180.679586129192\\
56	0.19182	190.741364284667\\
56	0.19548	201.109806116729\\
56	0.19914	211.784911625378\\
56	0.2028	222.766680810613\\
56	0.20646	234.055113672434\\
56	0.21012	245.650210210841\\
56	0.21378	257.551970425835\\
56	0.21744	269.760394317415\\
56	0.2211	282.275481885581\\
56	0.22476	295.097233130334\\
56	0.22842	308.225648051674\\
56	0.23208	321.660726649599\\
56	0.23574	335.402468924111\\
56	0.2394	349.450874875209\\
56	0.24306	363.805944502894\\
56	0.24672	378.467677807165\\
56	0.25038	393.436074788022\\
56	0.25404	408.711135445466\\
56	0.2577	424.292859779496\\
56	0.26136	440.181247790112\\
56	0.26502	456.376299477315\\
56	0.26868	472.878014841104\\
56	0.27234	489.686393881479\\
56	0.276	506.801436598441\\
56.375	0.093	26.9534015004779\\
56.375	0.09666	29.0365780221604\\
56.375	0.10032	31.4264182204294\\
56.375	0.10398	34.1229220952847\\
56.375	0.10764	37.1260896467263\\
56.375	0.1113	40.4359208747543\\
56.375	0.11496	44.0524157793686\\
56.375	0.11862	47.9755743605693\\
56.375	0.12228	52.2053966183563\\
56.375	0.12594	56.7418825527297\\
56.375	0.1296	61.5850321636895\\
56.375	0.13326	66.7348454512355\\
56.375	0.13692	72.191322415368\\
56.375	0.14058	77.9544630560869\\
56.375	0.14424	84.024267373392\\
56.375	0.1479	90.4007353672835\\
56.375	0.15156	97.0838670377614\\
56.375	0.15522	104.073662384826\\
56.375	0.15888	111.370121408476\\
56.375	0.16254	118.973244108713\\
56.375	0.1662	126.883030485536\\
56.375	0.16986	135.099480538946\\
56.375	0.17352	143.622594268942\\
56.375	0.17718	152.452371675524\\
56.375	0.18084	161.588812758693\\
56.375	0.1845	171.031917518448\\
56.375	0.18816	180.781685954789\\
56.375	0.19182	190.838118067717\\
56.375	0.19548	201.201213857231\\
56.375	0.19914	211.870973323332\\
56.375	0.2028	222.847396466018\\
56.375	0.20646	234.130483285291\\
56.375	0.21012	245.720233781151\\
56.375	0.21378	257.616647953597\\
56.375	0.21744	269.819725802629\\
56.375	0.2211	282.329467328247\\
56.375	0.22476	295.145872530452\\
56.375	0.22842	308.268941409244\\
56.375	0.23208	321.698673964621\\
56.375	0.23574	335.435070196585\\
56.375	0.2394	349.478130105136\\
56.375	0.24306	363.827853690272\\
56.375	0.24672	378.484240951995\\
56.375	0.25038	393.447291890305\\
56.375	0.25404	408.7170065052\\
56.375	0.2577	424.293384796682\\
56.375	0.26136	440.176426764751\\
56.375	0.26502	456.366132409405\\
56.375	0.26868	472.862501730646\\
56.375	0.27234	489.665534728474\\
56.375	0.276	506.775231402888\\
56.75	0.093	27.2054559128872\\
56.75	0.09666	29.2832863920218\\
56.75	0.10032	31.6677805477428\\
56.75	0.10398	34.3589383800501\\
56.75	0.10764	37.3567598889438\\
56.75	0.1113	40.6612450744238\\
56.75	0.11496	44.2723939364902\\
56.75	0.11862	48.1902064751429\\
56.75	0.12228	52.414682690382\\
56.75	0.12594	56.9458225822074\\
56.75	0.1296	61.7836261506192\\
56.75	0.13326	66.9280933956174\\
56.75	0.13692	72.3792243172018\\
56.75	0.14058	78.1370189153727\\
56.75	0.14424	84.20147719013\\
56.75	0.1479	90.5725991414735\\
56.75	0.15156	97.2503847694034\\
56.75	0.15522	104.23483407392\\
56.75	0.15888	111.525947055022\\
56.75	0.16254	119.123723712711\\
56.75	0.1662	127.028164046987\\
56.75	0.16986	135.239268057848\\
56.75	0.17352	143.757035745296\\
56.75	0.17718	152.581467109331\\
56.75	0.18084	161.712562149951\\
56.75	0.1845	171.150320867158\\
56.75	0.18816	180.894743260952\\
56.75	0.19182	190.945829331331\\
56.75	0.19548	201.303579078298\\
56.75	0.19914	211.96799250185\\
56.75	0.2028	222.939069601989\\
56.75	0.20646	234.216810378714\\
56.75	0.21012	245.801214832026\\
56.75	0.21378	257.692282961923\\
56.75	0.21744	269.890014768408\\
56.75	0.2211	282.394410251478\\
56.75	0.22476	295.205469411135\\
56.75	0.22842	308.323192247379\\
56.75	0.23208	321.747578760208\\
56.75	0.23574	335.478628949624\\
56.75	0.2394	349.516342815627\\
56.75	0.24306	363.860720358215\\
56.75	0.24672	378.51176157739\\
56.75	0.25038	393.469466473152\\
56.75	0.25404	408.733835045499\\
56.75	0.2577	424.304867294434\\
56.75	0.26136	440.182563219954\\
56.75	0.26502	456.366922822061\\
56.75	0.26868	472.857946100754\\
56.75	0.27234	489.655633056033\\
56.75	0.276	506.759983687899\\
57.125	0.093	27.4684678058615\\
57.125	0.09666	29.5409522424481\\
57.125	0.10032	31.9201003556211\\
57.125	0.10398	34.6059121453805\\
57.125	0.10764	37.5983876117262\\
57.125	0.1113	40.8975267546583\\
57.125	0.11496	44.5033295741767\\
57.125	0.11862	48.4157960702815\\
57.125	0.12228	52.6349262429726\\
57.125	0.12594	57.1607200922501\\
57.125	0.1296	61.9931776181139\\
57.125	0.13326	67.1322988205641\\
57.125	0.13692	72.5780836996007\\
57.125	0.14058	78.3305322552235\\
57.125	0.14424	84.3896444874328\\
57.125	0.1479	90.7554203962284\\
57.125	0.15156	97.4278599816103\\
57.125	0.15522	104.406963243579\\
57.125	0.15888	111.692730182133\\
57.125	0.16254	119.285160797274\\
57.125	0.1662	127.184255089002\\
57.125	0.16986	135.390013057315\\
57.125	0.17352	143.902434702215\\
57.125	0.17718	152.721520023702\\
57.125	0.18084	161.847269021774\\
57.125	0.1845	171.279681696434\\
57.125	0.18816	181.018758047679\\
57.125	0.19182	191.064498075511\\
57.125	0.19548	201.416901779929\\
57.125	0.19914	212.075969160934\\
57.125	0.2028	223.041700218524\\
57.125	0.20646	234.314094952702\\
57.125	0.21012	245.893153363465\\
57.125	0.21378	257.778875450815\\
57.125	0.21744	269.971261214751\\
57.125	0.2211	282.470310655274\\
57.125	0.22476	295.276023772383\\
57.125	0.22842	308.388400566078\\
57.125	0.23208	321.80744103636\\
57.125	0.23574	335.533145183228\\
57.125	0.2394	349.565513006683\\
57.125	0.24306	363.904544506723\\
57.125	0.24672	378.55023968335\\
57.125	0.25038	393.502598536564\\
57.125	0.25404	408.761621066364\\
57.125	0.2577	424.32730727275\\
57.125	0.26136	440.199657155722\\
57.125	0.26502	456.378670715281\\
57.125	0.26868	472.864347951426\\
57.125	0.27234	489.656688864158\\
57.125	0.276	506.755693453476\\
57.5	0.093	27.7424371794006\\
57.5	0.09666	29.8095755734393\\
57.5	0.10032	32.1833776440644\\
57.5	0.10398	34.8638433912757\\
57.5	0.10764	37.8509728150735\\
57.5	0.1113	41.1447659154576\\
57.5	0.11496	44.7452226924281\\
57.5	0.11862	48.6523431459849\\
57.5	0.12228	52.8661272761281\\
57.5	0.12594	57.3865750828576\\
57.5	0.1296	62.2136865661734\\
57.5	0.13326	67.3474617260757\\
57.5	0.13692	72.7879005625643\\
57.5	0.14058	78.5350030756393\\
57.5	0.14424	84.5887692653005\\
57.5	0.1479	90.9491991315482\\
57.5	0.15156	97.6162926743822\\
57.5	0.15522	104.590049893802\\
57.5	0.15888	111.870470789809\\
57.5	0.16254	119.457555362402\\
57.5	0.1662	127.351303611582\\
57.5	0.16986	135.551715537347\\
57.5	0.17352	144.0587911397\\
57.5	0.17718	152.872530418638\\
57.5	0.18084	161.992933374163\\
57.5	0.1845	171.420000006274\\
57.5	0.18816	181.153730314971\\
57.5	0.19182	191.194124300255\\
57.5	0.19548	201.541181962125\\
57.5	0.19914	212.194903300582\\
57.5	0.2028	223.155288315625\\
57.5	0.20646	234.422337007254\\
57.5	0.21012	245.99604937547\\
57.5	0.21378	257.876425420272\\
57.5	0.21744	270.06346514166\\
57.5	0.2211	282.557168539635\\
57.5	0.22476	295.357535614196\\
57.5	0.22842	308.464566365343\\
57.5	0.23208	321.878260793077\\
57.5	0.23574	335.598618897397\\
57.5	0.2394	349.625640678303\\
57.5	0.24306	363.959326135796\\
57.5	0.24672	378.599675269875\\
57.5	0.25038	393.546688080541\\
57.5	0.25404	408.800364567793\\
57.5	0.2577	424.360704731631\\
57.5	0.26136	440.227708572055\\
57.5	0.26502	456.401376089066\\
57.5	0.26868	472.881707282663\\
57.5	0.27234	489.668702152847\\
57.5	0.276	506.762360699617\\
57.875	0.093	28.0273640335046\\
57.875	0.09666	30.0891563849953\\
57.875	0.10032	32.4576124130725\\
57.875	0.10398	35.1327321177359\\
57.875	0.10764	38.1145154989857\\
57.875	0.1113	41.4029625568218\\
57.875	0.11496	44.9980732912444\\
57.875	0.11862	48.8998477022533\\
57.875	0.12228	53.1082857898485\\
57.875	0.12594	57.62338755403\\
57.875	0.1296	62.445152994798\\
57.875	0.13326	67.5735821121522\\
57.875	0.13692	73.0086749060928\\
57.875	0.14058	78.7504313766198\\
57.875	0.14424	84.7988515237332\\
57.875	0.1479	91.1539353474328\\
57.875	0.15156	97.8156828477188\\
57.875	0.15522	104.784094024591\\
57.875	0.15888	112.05916887805\\
57.875	0.16254	119.640907408095\\
57.875	0.1662	127.529309614727\\
57.875	0.16986	135.724375497944\\
57.875	0.17352	144.226105057748\\
57.875	0.17718	153.034498294139\\
57.875	0.18084	162.149555207116\\
57.875	0.1845	171.571275796679\\
57.875	0.18816	181.299660062829\\
57.875	0.19182	191.334708005564\\
57.875	0.19548	201.676419624887\\
57.875	0.19914	212.324794920795\\
57.875	0.2028	223.27983389329\\
57.875	0.20646	234.541536542372\\
57.875	0.21012	246.109902868039\\
57.875	0.21378	257.984932870293\\
57.875	0.21744	270.166626549133\\
57.875	0.2211	282.65498390456\\
57.875	0.22476	295.450004936573\\
57.875	0.22842	308.551689645173\\
57.875	0.23208	321.960038030358\\
57.875	0.23574	335.675050092131\\
57.875	0.2394	349.696725830489\\
57.875	0.24306	364.025065245434\\
57.875	0.24672	378.660068336965\\
57.875	0.25038	393.601735105083\\
57.875	0.25404	408.850065549787\\
57.875	0.2577	424.405059671077\\
57.875	0.26136	440.266717468953\\
57.875	0.26502	456.435038943416\\
57.875	0.26868	472.910024094466\\
57.875	0.27234	489.691672922101\\
57.875	0.276	506.779985426323\\
58.25	0.093	28.3232483681735\\
58.25	0.09666	30.3796946771163\\
58.25	0.10032	32.7428046626454\\
58.25	0.10398	35.4125783247609\\
58.25	0.10764	38.3890156634628\\
58.25	0.1113	41.672116678751\\
58.25	0.11496	45.2618813706256\\
58.25	0.11862	49.1583097390864\\
58.25	0.12228	53.3614017841337\\
58.25	0.12594	57.8711575057673\\
58.25	0.1296	62.6875769039872\\
58.25	0.13326	67.8106599787936\\
58.25	0.13692	73.2404067301862\\
58.25	0.14058	78.9768171581653\\
58.25	0.14424	85.0198912627307\\
58.25	0.1479	91.3696290438824\\
58.25	0.15156	98.0260305016205\\
58.25	0.15522	104.989095635945\\
58.25	0.15888	112.258824446856\\
58.25	0.16254	119.835216934353\\
58.25	0.1662	127.718273098436\\
58.25	0.16986	135.907992939106\\
58.25	0.17352	144.404376456362\\
58.25	0.17718	153.207423650205\\
58.25	0.18084	162.317134520634\\
58.25	0.1845	171.733509067649\\
58.25	0.18816	181.456547291251\\
58.25	0.19182	191.486249191438\\
58.25	0.19548	201.822614768213\\
58.25	0.19914	212.465644021573\\
58.25	0.2028	223.41533695152\\
58.25	0.20646	234.671693558054\\
58.25	0.21012	246.234713841174\\
58.25	0.21378	258.104397800879\\
58.25	0.21744	270.280745437172\\
58.25	0.2211	282.763756750051\\
58.25	0.22476	295.553431739516\\
58.25	0.22842	308.649770405567\\
58.25	0.23208	322.052772748205\\
58.25	0.23574	335.762438767429\\
58.25	0.2394	349.77876846324\\
58.25	0.24306	364.101761835637\\
58.25	0.24672	378.73141888462\\
58.25	0.25038	393.66773961019\\
58.25	0.25404	408.910724012345\\
58.25	0.2577	424.460372091088\\
58.25	0.26136	440.316683846416\\
58.25	0.26502	456.479659278331\\
58.25	0.26868	472.949298386833\\
58.25	0.27234	489.72560117192\\
58.25	0.276	506.808567633594\\
58.625	0.093	28.6300901834073\\
58.625	0.09666	30.6811904498022\\
58.625	0.10032	33.0389543927833\\
58.625	0.10398	35.7033820123509\\
58.625	0.10764	38.6744733085048\\
58.625	0.1113	41.952228281245\\
58.625	0.11496	45.5366469305716\\
58.625	0.11862	49.4277292564846\\
58.625	0.12228	53.6254752589839\\
58.625	0.12594	58.1298849380695\\
58.625	0.1296	62.9409582937415\\
58.625	0.13326	68.0586953259999\\
58.625	0.13692	73.4830960348446\\
58.625	0.14058	79.2141604202757\\
58.625	0.14424	85.2518884822931\\
58.625	0.1479	91.5962802208968\\
58.625	0.15156	98.2473356360869\\
58.625	0.15522	105.205054727863\\
58.625	0.15888	112.469437496226\\
58.625	0.16254	120.040483941175\\
58.625	0.1662	127.918194062711\\
58.625	0.16986	136.102567860833\\
58.625	0.17352	144.593605335541\\
58.625	0.17718	153.391306486836\\
58.625	0.18084	162.495671314717\\
58.625	0.1845	171.906699819184\\
58.625	0.18816	181.624392000237\\
58.625	0.19182	191.648747857877\\
58.625	0.19548	201.979767392104\\
58.625	0.19914	212.617450602917\\
58.625	0.2028	223.561797490316\\
58.625	0.20646	234.812808054301\\
58.625	0.21012	246.370482294873\\
58.625	0.21378	258.234820212031\\
58.625	0.21744	270.405821805775\\
58.625	0.2211	282.883487076106\\
58.625	0.22476	295.667816023023\\
58.625	0.22842	308.758808646527\\
58.625	0.23208	322.156464946617\\
58.625	0.23574	335.860784923293\\
58.625	0.2394	349.871768576555\\
58.625	0.24306	364.189415906404\\
58.625	0.24672	378.813726912839\\
58.625	0.25038	393.744701595861\\
58.625	0.25404	408.982339955469\\
58.625	0.2577	424.526641991664\\
58.625	0.26136	440.377607704444\\
58.625	0.26502	456.535237093811\\
58.625	0.26868	472.999530159764\\
58.625	0.27234	489.770486902304\\
58.625	0.276	506.84810732143\\
59	0.093	28.947889479206\\
59	0.09666	30.9936437030529\\
59	0.10032	33.3460616034861\\
59	0.10398	36.0051431805057\\
59	0.10764	38.9708884341117\\
59	0.1113	42.2432973643039\\
59	0.11496	45.8223699710826\\
59	0.11862	49.7081062544476\\
59	0.12228	53.9005062143989\\
59	0.12594	58.3995698509366\\
59	0.1296	63.2052971640606\\
59	0.13326	68.3176881537711\\
59	0.13692	73.7367428200678\\
59	0.14058	79.462461162951\\
59	0.14424	85.4948431824204\\
59	0.1479	91.8338888784762\\
59	0.15156	98.4795982511184\\
59	0.15522	105.431971300347\\
59	0.15888	112.691008026162\\
59	0.16254	120.256708428563\\
59	0.1662	128.129072507551\\
59	0.16986	136.308100263124\\
59	0.17352	144.793791695285\\
59	0.17718	153.586146804031\\
59	0.18084	162.685165589364\\
59	0.1845	172.090848051284\\
59	0.18816	181.803194189789\\
59	0.19182	191.822204004881\\
59	0.19548	202.14787749656\\
59	0.19914	212.780214664824\\
59	0.2028	223.719215509676\\
59	0.20646	234.964880031113\\
59	0.21012	246.517208229137\\
59	0.21378	258.376200103747\\
59	0.21744	270.541855654943\\
59	0.2211	283.014174882726\\
59	0.22476	295.793157787095\\
59	0.22842	308.878804368051\\
59	0.23208	322.271114625593\\
59	0.23574	335.970088559721\\
59	0.2394	349.975726170436\\
59	0.24306	364.288027457737\\
59	0.24672	378.906992421624\\
59	0.25038	393.832621062098\\
59	0.25404	409.064913379158\\
59	0.2577	424.603869372804\\
59	0.26136	440.449489043037\\
59	0.26502	456.601772389856\\
59	0.26868	473.060719413261\\
59	0.27234	489.826330113253\\
59	0.276	506.898604489831\\
59.375	0.093	29.2766462555696\\
59.375	0.09666	31.3170544368685\\
59.375	0.10032	33.6641262947538\\
59.375	0.10398	36.3178618292254\\
59.375	0.10764	39.2782610402834\\
59.375	0.1113	42.5453239279277\\
59.375	0.11496	46.1190504921584\\
59.375	0.11862	49.9994407329754\\
59.375	0.12228	54.1864946503788\\
59.375	0.12594	58.6802122443686\\
59.375	0.1296	63.4805935149446\\
59.375	0.13326	68.5876384621071\\
59.375	0.13692	74.0013470858559\\
59.375	0.14058	79.7217193861911\\
59.375	0.14424	85.7487553631126\\
59.375	0.1479	92.0824550166204\\
59.375	0.15156	98.7228183467146\\
59.375	0.15522	105.669845353395\\
59.375	0.15888	112.923536036662\\
59.375	0.16254	120.483890396515\\
59.375	0.1662	128.350908432955\\
59.375	0.16986	136.524590145981\\
59.375	0.17352	145.004935535593\\
59.375	0.17718	153.791944601792\\
59.375	0.18084	162.885617344577\\
59.375	0.1845	172.285953763948\\
59.375	0.18816	181.992953859906\\
59.375	0.19182	192.00661763245\\
59.375	0.19548	202.326945081581\\
59.375	0.19914	212.953936207297\\
59.375	0.2028	223.887591009601\\
59.375	0.20646	235.12790948849\\
59.375	0.21012	246.674891643966\\
59.375	0.21378	258.528537476028\\
59.375	0.21744	270.688846984677\\
59.375	0.2211	283.155820169911\\
59.375	0.22476	295.929457031733\\
59.375	0.22842	309.00975757014\\
59.375	0.23208	322.396721785134\\
59.375	0.23574	336.090349676715\\
59.375	0.2394	350.090641244881\\
59.375	0.24306	364.397596489634\\
59.375	0.24672	379.011215410973\\
59.375	0.25038	393.931498008899\\
59.375	0.25404	409.158444283411\\
59.375	0.2577	424.69205423451\\
59.375	0.26136	440.532327862195\\
59.375	0.26502	456.679265166465\\
59.375	0.26868	473.132866147323\\
59.375	0.27234	489.893130804767\\
59.375	0.276	506.960059138797\\
59.75	0.093	29.6163605124981\\
59.75	0.09666	31.651422651249\\
59.75	0.10032	33.9931484665864\\
59.75	0.10398	36.6415379585101\\
59.75	0.10764	39.5965911270201\\
59.75	0.1113	42.8583079721164\\
59.75	0.11496	46.4266884937992\\
59.75	0.11862	50.3017326920683\\
59.75	0.12228	54.4834405669237\\
59.75	0.12594	58.9718121183655\\
59.75	0.1296	63.7668473463936\\
59.75	0.13326	68.8685462510081\\
59.75	0.13692	74.2769088322089\\
59.75	0.14058	79.9919350899962\\
59.75	0.14424	86.0136250243697\\
59.75	0.1479	92.3419786353296\\
59.75	0.15156	98.9769959228759\\
59.75	0.15522	105.918676887008\\
59.75	0.15888	113.167021527728\\
59.75	0.16254	120.722029845033\\
59.75	0.1662	128.583701838924\\
59.75	0.16986	136.752037509402\\
59.75	0.17352	145.227036856467\\
59.75	0.17718	154.008699880117\\
59.75	0.18084	163.097026580354\\
59.75	0.1845	172.492016957178\\
59.75	0.18816	182.193671010588\\
59.75	0.19182	192.201988740584\\
59.75	0.19548	202.516970147166\\
59.75	0.19914	213.138615230335\\
59.75	0.2028	224.06692399009\\
59.75	0.20646	235.301896426432\\
59.75	0.21012	246.84353253936\\
59.75	0.21378	258.691832328874\\
59.75	0.21744	270.846795794974\\
59.75	0.2211	283.308422937661\\
59.75	0.22476	296.076713756935\\
59.75	0.22842	309.151668252794\\
59.75	0.23208	322.53328642524\\
59.75	0.23574	336.221568274273\\
59.75	0.2394	350.216513799891\\
59.75	0.24306	364.518123002096\\
59.75	0.24672	379.126395880888\\
59.75	0.25038	394.041332436266\\
59.75	0.25404	409.26293266823\\
59.75	0.2577	424.79119657678\\
59.75	0.26136	440.626124161917\\
59.75	0.26502	456.76771542364\\
59.75	0.26868	473.215970361949\\
59.75	0.27234	489.970888976845\\
59.75	0.276	507.032471268328\\
60.125	0.093	29.9670322499914\\
60.125	0.09666	31.9967483461945\\
60.125	0.10032	34.3331281189838\\
60.125	0.10398	36.9761715683595\\
60.125	0.10764	39.9258786943216\\
60.125	0.1113	43.18224949687\\
60.125	0.11496	46.7452839760048\\
60.125	0.11862	50.6149821317259\\
60.125	0.12228	54.7913439640334\\
60.125	0.12594	59.2743694729272\\
60.125	0.1296	64.0640586584074\\
60.125	0.13326	69.1604115204739\\
60.125	0.13692	74.5634280591268\\
60.125	0.14058	80.2731082743661\\
60.125	0.14424	86.2894521661917\\
60.125	0.1479	92.6124597346036\\
60.125	0.15156	99.2421309796019\\
60.125	0.15522	106.178465901187\\
60.125	0.15888	113.421464499358\\
60.125	0.16254	120.971126774115\\
60.125	0.1662	128.827452725459\\
60.125	0.16986	136.990442353389\\
60.125	0.17352	145.460095657905\\
60.125	0.17718	154.236412639008\\
60.125	0.18084	163.319393296697\\
60.125	0.1845	172.709037630972\\
60.125	0.18816	182.405345641834\\
60.125	0.19182	192.408317329282\\
60.125	0.19548	202.717952693317\\
60.125	0.19914	213.334251733938\\
60.125	0.2028	224.257214451145\\
60.125	0.20646	235.486840844939\\
60.125	0.21012	247.023130915318\\
60.125	0.21378	258.866084662285\\
60.125	0.21744	271.015702085837\\
60.125	0.2211	283.471983185976\\
60.125	0.22476	296.234927962702\\
60.125	0.22842	309.304536416013\\
60.125	0.23208	322.680808545911\\
60.125	0.23574	336.363744352396\\
60.125	0.2394	350.353343835466\\
60.125	0.24306	364.649606995124\\
60.125	0.24672	379.252533831367\\
60.125	0.25038	394.162124344197\\
60.125	0.25404	409.378378533613\\
60.125	0.2577	424.901296399616\\
60.125	0.26136	440.730877942204\\
60.125	0.26502	456.867123161379\\
60.125	0.26868	473.310032057141\\
60.125	0.27234	490.059604629489\\
60.125	0.276	507.115840878423\\
60.5	0.093	30.3286614680497\\
60.5	0.09666	32.3530315217048\\
60.5	0.10032	34.6840652519462\\
60.5	0.10398	37.3217626587739\\
60.5	0.10764	40.2661237421881\\
60.5	0.1113	43.5171485021885\\
60.5	0.11496	47.0748369387753\\
60.5	0.11862	50.9391890519485\\
60.5	0.12228	55.1102048417081\\
60.5	0.12594	59.5878843080539\\
60.5	0.1296	64.3722274509861\\
60.5	0.13326	69.4632342705046\\
60.5	0.13692	74.8609047666096\\
60.5	0.14058	80.5652389393009\\
60.5	0.14424	86.5762367885786\\
60.5	0.1479	92.8938983144425\\
60.5	0.15156	99.5182235168928\\
60.5	0.15522	106.44921239593\\
60.5	0.15888	113.686864951553\\
60.5	0.16254	121.231181183762\\
60.5	0.1662	129.082161092558\\
60.5	0.16986	137.23980467794\\
60.5	0.17352	145.704111939908\\
60.5	0.17718	154.475082878463\\
60.5	0.18084	163.552717493604\\
60.5	0.1845	172.937015785332\\
60.5	0.18816	182.627977753646\\
60.5	0.19182	192.625603398546\\
60.5	0.19548	202.929892720032\\
60.5	0.19914	213.540845718105\\
60.5	0.2028	224.458462392765\\
60.5	0.20646	235.68274274401\\
60.5	0.21012	247.213686771842\\
60.5	0.21378	259.05129447626\\
60.5	0.21744	271.195565857265\\
60.5	0.2211	283.646500914856\\
60.5	0.22476	296.404099649033\\
60.5	0.22842	309.468362059797\\
60.5	0.23208	322.839288147147\\
60.5	0.23574	336.516877911084\\
60.5	0.2394	350.501131351606\\
60.5	0.24306	364.792048468716\\
60.5	0.24672	379.389629262411\\
60.5	0.25038	394.293873732693\\
60.5	0.25404	409.504781879561\\
60.5	0.2577	425.022353703016\\
60.5	0.26136	440.846589203057\\
60.5	0.26502	456.977488379684\\
60.5	0.26868	473.415051232897\\
60.5	0.27234	490.159277762697\\
60.5	0.276	507.210167969083\\
60.875	0.093	30.7012481666729\\
60.875	0.09666	32.72027217778\\
60.875	0.10032	35.0459598654734\\
60.875	0.10398	37.6783112297532\\
60.875	0.10764	40.6173262706194\\
60.875	0.1113	43.8630049880719\\
60.875	0.11496	47.4153473821108\\
60.875	0.11862	51.274353452736\\
60.875	0.12228	55.4400231999475\\
60.875	0.12594	59.9123566237454\\
60.875	0.1296	64.6913537241297\\
60.875	0.13326	69.7770145011003\\
60.875	0.13692	75.1693389546573\\
60.875	0.14058	80.8683270848006\\
60.875	0.14424	86.8739788915304\\
60.875	0.1479	93.1862943748463\\
60.875	0.15156	99.8052735347487\\
60.875	0.15522	106.730916371237\\
60.875	0.15888	113.963222884313\\
60.875	0.16254	121.502193073974\\
60.875	0.1662	129.347826940222\\
60.875	0.16986	137.500124483056\\
60.875	0.17352	145.959085702476\\
60.875	0.17718	154.724710598483\\
60.875	0.18084	163.796999171076\\
60.875	0.1845	173.175951420256\\
60.875	0.18816	182.861567346022\\
60.875	0.19182	192.853846948374\\
60.875	0.19548	203.152790227313\\
60.875	0.19914	213.758397182838\\
60.875	0.2028	224.670667814949\\
60.875	0.20646	235.889602123647\\
60.875	0.21012	247.415200108931\\
60.875	0.21378	259.247461770801\\
60.875	0.21744	271.386387109258\\
60.875	0.2211	283.831976124301\\
60.875	0.22476	296.58422881593\\
60.875	0.22842	309.643145184146\\
60.875	0.23208	323.008725228948\\
60.875	0.23574	336.680968950337\\
60.875	0.2394	350.659876348311\\
60.875	0.24306	364.945447422872\\
60.875	0.24672	379.53768217402\\
60.875	0.25038	394.436580601754\\
60.875	0.25404	409.642142706074\\
60.875	0.2577	425.154368486981\\
60.875	0.26136	440.973257944474\\
60.875	0.26502	457.098811078553\\
60.875	0.26868	473.531027889219\\
60.875	0.27234	490.269908376471\\
60.875	0.276	507.315452540309\\
61.25	0.093	31.0847923458609\\
61.25	0.09666	33.0984703144201\\
61.25	0.10032	35.4188119595656\\
61.25	0.10398	38.0458172812974\\
61.25	0.10764	40.9794862796156\\
61.25	0.1113	44.2198189545201\\
61.25	0.11496	47.7668153060111\\
61.25	0.11862	51.6204753340883\\
61.25	0.12228	55.7807990387519\\
61.25	0.12594	60.2477864200019\\
61.25	0.1296	65.0214374778382\\
61.25	0.13326	70.1017522122608\\
61.25	0.13692	75.4887306232698\\
61.25	0.14058	81.1823727108653\\
61.25	0.14424	87.182678475047\\
61.25	0.1479	93.489647915815\\
61.25	0.15156	100.103281033169\\
61.25	0.15522	107.02357782711\\
61.25	0.15888	114.250538297637\\
61.25	0.16254	121.784162444751\\
61.25	0.1662	129.624450268451\\
61.25	0.16986	137.771401768737\\
61.25	0.17352	146.225016945609\\
61.25	0.17718	154.985295799068\\
61.25	0.18084	164.052238329113\\
61.25	0.1845	173.425844535745\\
61.25	0.18816	183.106114418963\\
61.25	0.19182	193.093047978767\\
61.25	0.19548	203.386645215158\\
61.25	0.19914	213.986906128135\\
61.25	0.2028	224.893830717698\\
61.25	0.20646	236.107418983848\\
61.25	0.21012	247.627670926584\\
61.25	0.21378	259.454586545906\\
61.25	0.21744	271.588165841815\\
61.25	0.2211	284.02840881431\\
61.25	0.22476	296.775315463392\\
61.25	0.22842	309.82888578906\\
61.25	0.23208	323.189119791314\\
61.25	0.23574	336.856017470154\\
61.25	0.2394	350.829578825581\\
61.25	0.24306	365.109803857594\\
61.25	0.24672	379.696692566194\\
61.25	0.25038	394.59024495138\\
61.25	0.25404	409.790461013152\\
61.25	0.2577	425.297340751511\\
61.25	0.26136	441.110884166456\\
61.25	0.26502	457.231091257987\\
61.25	0.26868	473.657962026105\\
61.25	0.27234	490.391496470809\\
61.25	0.276	507.431694592099\\
61.625	0.093	31.4792940056139\\
61.625	0.09666	33.487625931625\\
61.625	0.10032	35.8026215342226\\
61.625	0.10398	38.4242808134065\\
61.625	0.10764	41.3526037691767\\
61.625	0.1113	44.5875904015333\\
61.625	0.11496	48.1292407104763\\
61.625	0.11862	51.9775546960055\\
61.625	0.12228	56.1325323581212\\
61.625	0.12594	60.5941736968232\\
61.625	0.1296	65.3624787121115\\
61.625	0.13326	70.4374474039862\\
61.625	0.13692	75.8190797724473\\
61.625	0.14058	81.5073758174948\\
61.625	0.14424	87.5023355391285\\
61.625	0.1479	93.8039589373486\\
61.625	0.15156	100.412246012155\\
61.625	0.15522	107.327196763548\\
61.625	0.15888	114.548811191527\\
61.625	0.16254	122.077089296093\\
61.625	0.1662	129.912031077245\\
61.625	0.16986	138.053636534983\\
61.625	0.17352	146.501905669307\\
61.625	0.17718	155.256838480218\\
61.625	0.18084	164.318434967715\\
61.625	0.1845	173.686695131799\\
61.625	0.18816	183.361618972469\\
61.625	0.19182	193.343206489725\\
61.625	0.19548	203.631457683568\\
61.625	0.19914	214.226372553997\\
61.625	0.2028	225.127951101013\\
61.625	0.20646	236.336193324614\\
61.625	0.21012	247.851099224802\\
61.625	0.21378	259.672668801577\\
61.625	0.21744	271.800902054938\\
61.625	0.2211	284.235798984885\\
61.625	0.22476	296.977359591418\\
61.625	0.22842	310.025583874538\\
61.625	0.23208	323.380471834244\\
61.625	0.23574	337.042023470537\\
61.625	0.2394	351.010238783416\\
61.625	0.24306	365.285117772881\\
61.625	0.24672	379.866660438933\\
61.625	0.25038	394.754866781571\\
61.625	0.25404	409.949736800795\\
61.625	0.2577	425.451270496606\\
61.625	0.26136	441.259467869003\\
61.625	0.26502	457.374328917986\\
61.625	0.26868	473.795853643556\\
61.625	0.27234	490.524042045712\\
61.625	0.276	507.558894124454\\
62	0.093	31.8847531459317\\
62	0.09666	33.8877390293949\\
62	0.10032	36.1973885894445\\
62	0.10398	38.8137018260804\\
62	0.10764	41.7366787393027\\
62	0.1113	44.9663193291113\\
62	0.11496	48.5026235955064\\
62	0.11862	52.3455915384877\\
62	0.12228	56.4952231580554\\
62	0.12594	60.9515184542095\\
62	0.1296	65.7144774269498\\
62	0.13326	70.7841000762766\\
62	0.13692	76.1603864021897\\
62	0.14058	81.8433364046892\\
62	0.14424	87.832950083775\\
62	0.1479	94.1292274394471\\
62	0.15156	100.732168471706\\
62	0.15522	107.641773180551\\
62	0.15888	114.858041565982\\
62	0.16254	122.380973627999\\
62	0.1662	130.210569366603\\
62	0.16986	138.346828781793\\
62	0.17352	146.78975187357\\
62	0.17718	155.539338641933\\
62	0.18084	164.595589086882\\
62	0.1845	173.958503208418\\
62	0.18816	183.62808100654\\
62	0.19182	193.604322481248\\
62	0.19548	203.887227632543\\
62	0.19914	214.476796460424\\
62	0.2028	225.373028964892\\
62	0.20646	236.575925145946\\
62	0.21012	248.085485003586\\
62	0.21378	259.901708537812\\
62	0.21744	272.024595748625\\
62	0.2211	284.454146636024\\
62	0.22476	297.19036120001\\
62	0.22842	310.233239440582\\
62	0.23208	323.58278135774\\
62	0.23574	337.238986951485\\
62	0.2394	351.201856221815\\
62	0.24306	365.471389168733\\
62	0.24672	380.047585792236\\
62	0.25038	394.930446092326\\
62	0.25404	410.119970069003\\
62	0.2577	425.616157722266\\
62	0.26136	441.419009052115\\
62	0.26502	457.52852405855\\
62	0.26868	473.944702741572\\
62	0.27234	490.66754510118\\
62	0.276	507.697051137374\\
62.375	0.093	32.3011697668144\\
62.375	0.09666	34.2988096077297\\
62.375	0.10032	36.6031131252313\\
62.375	0.10398	39.2140803193193\\
62.375	0.10764	42.1317111899936\\
62.375	0.1113	45.3560057372543\\
62.375	0.11496	48.8869639611013\\
62.375	0.11862	52.7245858615347\\
62.375	0.12228	56.8688714385544\\
62.375	0.12594	61.3198206921606\\
62.375	0.1296	66.077433622353\\
62.375	0.13326	71.1417102291317\\
62.375	0.13692	76.5126505124969\\
62.375	0.14058	82.1902544724485\\
62.375	0.14424	88.1745221089863\\
62.375	0.1479	94.4654534221105\\
62.375	0.15156	101.063048411821\\
62.375	0.15522	107.967307078118\\
62.375	0.15888	115.178229421001\\
62.375	0.16254	122.695815440471\\
62.375	0.1662	130.520065136527\\
62.375	0.16986	138.650978509169\\
62.375	0.17352	147.088555558398\\
62.375	0.17718	155.832796284213\\
62.375	0.18084	164.883700686614\\
62.375	0.1845	174.241268765602\\
62.375	0.18816	183.905500521176\\
62.375	0.19182	193.876395953336\\
62.375	0.19548	204.153955062083\\
62.375	0.19914	214.738177847416\\
62.375	0.2028	225.629064309336\\
62.375	0.20646	236.826614447842\\
62.375	0.21012	248.330828262934\\
62.375	0.21378	260.141705754612\\
62.375	0.21744	272.259246922877\\
62.375	0.2211	284.683451767728\\
62.375	0.22476	297.414320289166\\
62.375	0.22842	310.45185248719\\
62.375	0.23208	323.7960483618\\
62.375	0.23574	337.446907912997\\
62.375	0.2394	351.40443114078\\
62.375	0.24306	365.668618045149\\
62.375	0.24672	380.239468626105\\
62.375	0.25038	395.116982883647\\
62.375	0.25404	410.301160817775\\
62.375	0.2577	425.79200242849\\
62.375	0.26136	441.589507715791\\
62.375	0.26502	457.693676679679\\
62.375	0.26868	474.104509320152\\
62.375	0.27234	490.822005637213\\
62.375	0.276	507.846165630859\\
62.75	0.093	32.728543868262\\
62.75	0.09666	34.7208376666293\\
62.75	0.10032	37.019795141583\\
62.75	0.10398	39.625416293123\\
62.75	0.10764	42.5377011212494\\
62.75	0.1113	45.7566496259621\\
62.75	0.11496	49.2822618072612\\
62.75	0.11862	53.1145376651466\\
62.75	0.12228	57.2534771996184\\
62.75	0.12594	61.6990804106765\\
62.75	0.1296	66.451347298321\\
62.75	0.13326	71.5102778625518\\
62.75	0.13692	76.875872103369\\
62.75	0.14058	82.5481300207726\\
62.75	0.14424	88.5270516147625\\
62.75	0.1479	94.8126368853387\\
62.75	0.15156	101.404885832501\\
62.75	0.15522	108.30379845625\\
62.75	0.15888	115.509374756586\\
62.75	0.16254	123.021614733507\\
62.75	0.1662	130.840518387015\\
62.75	0.16986	138.96608571711\\
62.75	0.17352	147.39831672379\\
62.75	0.17718	156.137211407057\\
62.75	0.18084	165.182769766911\\
62.75	0.1845	174.534991803351\\
62.75	0.18816	184.193877516377\\
62.75	0.19182	194.159426905989\\
62.75	0.19548	204.431639972188\\
62.75	0.19914	215.010516714973\\
62.75	0.2028	225.896057134345\\
62.75	0.20646	237.088261230303\\
62.75	0.21012	248.587129002847\\
62.75	0.21378	260.392660451977\\
62.75	0.21744	272.504855577694\\
62.75	0.2211	284.923714379997\\
62.75	0.22476	297.649236858887\\
62.75	0.22842	310.681423014363\\
62.75	0.23208	324.020272846425\\
62.75	0.23574	337.665786355074\\
62.75	0.2394	351.617963540309\\
62.75	0.24306	365.876804402131\\
62.75	0.24672	380.442308940538\\
62.75	0.25038	395.314477155533\\
62.75	0.25404	410.493309047113\\
62.75	0.2577	425.97880461528\\
62.75	0.26136	441.770963860033\\
62.75	0.26502	457.869786781372\\
62.75	0.26868	474.275273379298\\
62.75	0.27234	490.98742365381\\
62.75	0.276	508.006237604909\\
63.125	0.093	33.1668754502745\\
63.125	0.09666	35.1538232060939\\
63.125	0.10032	37.4474346384996\\
63.125	0.10398	40.0477097474917\\
63.125	0.10764	42.9546485330701\\
63.125	0.1113	46.1682509952349\\
63.125	0.11496	49.688517133986\\
63.125	0.11862	53.5154469493235\\
63.125	0.12228	57.6490404412473\\
63.125	0.12594	62.0892976097575\\
63.125	0.1296	66.836218454854\\
63.125	0.13326	71.8898029765369\\
63.125	0.13692	77.2500511748061\\
63.125	0.14058	82.9169630496617\\
63.125	0.14424	88.8905386011037\\
63.125	0.1479	95.1707778291319\\
63.125	0.15156	101.757680733747\\
63.125	0.15522	108.651247314948\\
63.125	0.15888	115.851477572735\\
63.125	0.16254	123.358371507109\\
63.125	0.1662	131.171929118069\\
63.125	0.16986	139.292150405615\\
63.125	0.17352	147.719035369748\\
63.125	0.17718	156.452584010467\\
63.125	0.18084	165.492796327772\\
63.125	0.1845	174.839672321664\\
63.125	0.18816	184.493211992142\\
63.125	0.19182	194.453415339207\\
63.125	0.19548	204.720282362858\\
63.125	0.19914	215.293813063095\\
63.125	0.2028	226.174007439918\\
63.125	0.20646	237.360865493328\\
63.125	0.21012	248.854387223325\\
63.125	0.21378	260.654572629907\\
63.125	0.21744	272.761421713076\\
63.125	0.2211	285.174934472831\\
63.125	0.22476	297.895110909173\\
63.125	0.22842	310.921951022101\\
63.125	0.23208	324.255454811616\\
63.125	0.23574	337.895622277716\\
63.125	0.2394	351.842453420403\\
63.125	0.24306	366.095948239677\\
63.125	0.24672	380.656106735537\\
63.125	0.25038	395.522928907983\\
63.125	0.25404	410.696414757015\\
63.125	0.2577	426.176564282634\\
63.125	0.26136	441.963377484839\\
63.125	0.26502	458.056854363631\\
63.125	0.26868	474.456994919009\\
63.125	0.27234	491.163799150973\\
63.125	0.276	508.177267059524\\
63.5	0.093	33.616164512852\\
63.5	0.09666	35.5977662261234\\
63.5	0.10032	37.8860316159811\\
63.5	0.10398	40.4809606824252\\
63.5	0.10764	43.3825534254557\\
63.5	0.1113	46.5908098450725\\
63.5	0.11496	50.1057299412757\\
63.5	0.11862	53.9273137140651\\
63.5	0.12228	58.0555611634411\\
63.5	0.12594	62.4904722894033\\
63.5	0.1296	67.2320470919518\\
63.5	0.13326	72.2802855710867\\
63.5	0.13692	77.635187726808\\
63.5	0.14058	83.2967535591157\\
63.5	0.14424	89.2649830680097\\
63.5	0.1479	95.53987625349\\
63.5	0.15156	102.121433115557\\
63.5	0.15522	109.00965365421\\
63.5	0.15888	116.204537869449\\
63.5	0.16254	123.706085761275\\
63.5	0.1662	131.514297329687\\
63.5	0.16986	139.629172574685\\
63.5	0.17352	148.05071149627\\
63.5	0.17718	156.778914094441\\
63.5	0.18084	165.813780369199\\
63.5	0.1845	175.155310320543\\
63.5	0.18816	184.803503948473\\
63.5	0.19182	194.758361252989\\
63.5	0.19548	205.019882234092\\
63.5	0.19914	215.588066891782\\
63.5	0.2028	226.462915226057\\
63.5	0.20646	237.644427236919\\
63.5	0.21012	249.132602924367\\
63.5	0.21378	260.927442288402\\
63.5	0.21744	273.028945329023\\
63.5	0.2211	285.43711204623\\
63.5	0.22476	298.151942440024\\
63.5	0.22842	311.173436510404\\
63.5	0.23208	324.501594257371\\
63.5	0.23574	338.136415680924\\
63.5	0.2394	352.077900781063\\
63.5	0.24306	366.326049557788\\
63.5	0.24672	380.8808620111\\
63.5	0.25038	395.742338140998\\
63.5	0.25404	410.910477947483\\
63.5	0.2577	426.385281430554\\
63.5	0.26136	442.166748590211\\
63.5	0.26502	458.254879426454\\
63.5	0.26868	474.649673939284\\
63.5	0.27234	491.351132128701\\
63.5	0.276	508.359253994703\\
63.875	0.093	34.0764110559942\\
63.875	0.09666	36.0526667267177\\
63.875	0.10032	38.3355860740275\\
63.875	0.10398	40.9251690979236\\
63.875	0.10764	43.8214157984061\\
63.875	0.1113	47.024326175475\\
63.875	0.11496	50.5339002291302\\
63.875	0.11862	54.3501379593718\\
63.875	0.12228	58.4730393661997\\
63.875	0.12594	62.9026044496139\\
63.875	0.1296	67.6388332096145\\
63.875	0.13326	72.6817256462014\\
63.875	0.13692	78.0312817593748\\
63.875	0.14058	83.6875015491345\\
63.875	0.14424	89.6503850154806\\
63.875	0.1479	95.9199321584129\\
63.875	0.15156	102.496142977932\\
63.875	0.15522	109.379017474037\\
63.875	0.15888	116.568555646728\\
63.875	0.16254	124.064757496006\\
63.875	0.1662	131.86762302187\\
63.875	0.16986	139.977152224321\\
63.875	0.17352	148.393345103357\\
63.875	0.17718	157.116201658981\\
63.875	0.18084	166.14572189119\\
63.875	0.1845	175.481905799986\\
63.875	0.18816	185.124753385368\\
63.875	0.19182	195.074264647337\\
63.875	0.19548	205.330439585892\\
63.875	0.19914	215.893278201033\\
63.875	0.2028	226.762780492761\\
63.875	0.20646	237.938946461075\\
63.875	0.21012	249.421776105975\\
63.875	0.21378	261.211269427462\\
63.875	0.21744	273.307426425535\\
63.875	0.2211	285.710247100194\\
63.875	0.22476	298.41973145144\\
63.875	0.22842	311.435879479272\\
63.875	0.23208	324.758691183691\\
63.875	0.23574	338.388166564696\\
63.875	0.2394	352.324305622287\\
63.875	0.24306	366.567108356464\\
63.875	0.24672	381.116574767228\\
63.875	0.25038	395.972704854578\\
63.875	0.25404	411.135498618515\\
63.875	0.2577	426.604956059038\\
63.875	0.26136	442.381077176147\\
63.875	0.26502	458.463861969843\\
63.875	0.26868	474.853310440125\\
63.875	0.27234	491.549422586993\\
63.875	0.276	508.552198410448\\
64.25	0.093	34.5476150797014\\
64.25	0.09666	36.5185247078769\\
64.25	0.10032	38.7960980126388\\
64.25	0.10398	41.3803349939869\\
64.25	0.10764	44.2712356519215\\
64.25	0.1113	47.4687999864424\\
64.25	0.11496	50.9730279975497\\
64.25	0.11862	54.7839196852433\\
64.25	0.12228	58.9014750495232\\
64.25	0.12594	63.3256940903896\\
64.25	0.1296	68.0565768078422\\
64.25	0.13326	73.0941232018812\\
64.25	0.13692	78.4383332725066\\
64.25	0.14058	84.0892070197183\\
64.25	0.14424	90.0467444435164\\
64.25	0.1479	96.3109455439007\\
64.25	0.15156	102.881810320872\\
64.25	0.15522	109.759338774429\\
64.25	0.15888	116.943530904572\\
64.25	0.16254	124.434386711302\\
64.25	0.1662	132.231906194618\\
64.25	0.16986	140.336089354521\\
64.25	0.17352	148.74693619101\\
64.25	0.17718	157.464446704085\\
64.25	0.18084	166.488620893746\\
64.25	0.1845	175.819458759994\\
64.25	0.18816	185.456960302829\\
64.25	0.19182	195.401125522249\\
64.25	0.19548	205.651954418256\\
64.25	0.19914	216.20944699085\\
64.25	0.2028	227.073603240029\\
64.25	0.20646	238.244423165795\\
64.25	0.21012	249.721906768148\\
64.25	0.21378	261.506054047086\\
64.25	0.21744	273.596865002612\\
64.25	0.2211	285.994339634723\\
64.25	0.22476	298.698477943421\\
64.25	0.22842	311.709279928705\\
64.25	0.23208	325.026745590576\\
64.25	0.23574	338.650874929033\\
64.25	0.2394	352.581667944076\\
64.25	0.24306	366.819124635705\\
64.25	0.24672	381.363245003921\\
64.25	0.25038	396.214029048723\\
64.25	0.25404	411.371476770112\\
64.25	0.2577	426.835588168087\\
64.25	0.26136	442.606363242648\\
64.25	0.26502	458.683801993796\\
64.25	0.26868	475.06790442153\\
64.25	0.27234	491.75867052585\\
64.25	0.276	508.756100306757\\
64.625	0.093	35.0297765839735\\
64.625	0.09666	36.995340169601\\
64.625	0.10032	39.2675674318149\\
64.625	0.10398	41.8464583706151\\
64.625	0.10764	44.7320129860017\\
64.625	0.1113	47.9242312779747\\
64.625	0.11496	51.423113246534\\
64.625	0.11862	55.2286588916796\\
64.625	0.12228	59.3408682134116\\
64.625	0.12594	63.75974121173\\
64.625	0.1296	68.4852778866346\\
64.625	0.13326	73.5174782381257\\
64.625	0.13692	78.8563422662031\\
64.625	0.14058	84.501869970867\\
64.625	0.14424	90.454061352117\\
64.625	0.1479	96.7129164099535\\
64.625	0.15156	103.278435144376\\
64.625	0.15522	110.150617555386\\
64.625	0.15888	117.329463642981\\
64.625	0.16254	124.814973407163\\
64.625	0.1662	132.607146847931\\
64.625	0.16986	140.705983965286\\
64.625	0.17352	149.111484759227\\
64.625	0.17718	157.823649229754\\
64.625	0.18084	166.842477376867\\
64.625	0.1845	176.167969200567\\
64.625	0.18816	185.800124700854\\
64.625	0.19182	195.738943877726\\
64.625	0.19548	205.984426731186\\
64.625	0.19914	216.536573261231\\
64.625	0.2028	227.395383467863\\
64.625	0.20646	238.560857351081\\
64.625	0.21012	250.032994910885\\
64.625	0.21378	261.811796147276\\
64.625	0.21744	273.897261060253\\
64.625	0.2211	286.289389649817\\
64.625	0.22476	298.988181915966\\
64.625	0.22842	311.993637858703\\
64.625	0.23208	325.305757478025\\
64.625	0.23574	338.924540773934\\
64.625	0.2394	352.849987746429\\
64.625	0.24306	367.082098395511\\
64.625	0.24672	381.620872721179\\
64.625	0.25038	396.466310723433\\
64.625	0.25404	411.618412402274\\
64.625	0.2577	427.077177757701\\
64.625	0.26136	442.842606789714\\
64.625	0.26502	458.914699498314\\
64.625	0.26868	475.2934558835\\
64.625	0.27234	491.978875945273\\
64.625	0.276	508.970959683631\\
65	0.093	35.5228955688105\\
65	0.09666	37.48311311189\\
65	0.10032	39.749994331556\\
65	0.10398	42.3235392278083\\
65	0.10764	45.2037478006469\\
65	0.1113	48.3906200500719\\
65	0.11496	51.8841559760832\\
65	0.11862	55.6843555786809\\
65	0.12228	59.7912188578649\\
65	0.12594	64.2047458136353\\
65	0.1296	68.9249364459921\\
65	0.13326	73.9517907549352\\
65	0.13692	79.2853087404646\\
65	0.14058	84.9254904025805\\
65	0.14424	90.8723357412826\\
65	0.1479	97.1258447565712\\
65	0.15156	103.686017448446\\
65	0.15522	110.552853816907\\
65	0.15888	117.726353861955\\
65	0.16254	125.206517583589\\
65	0.1662	132.993344981809\\
65	0.16986	141.086836056616\\
65	0.17352	149.486990808009\\
65	0.17718	158.193809235988\\
65	0.18084	167.207291340553\\
65	0.1845	176.527437121706\\
65	0.18816	186.154246579444\\
65	0.19182	196.087719713769\\
65	0.19548	206.32785652468\\
65	0.19914	216.874657012177\\
65	0.2028	227.728121176261\\
65	0.20646	238.888249016931\\
65	0.21012	250.355040534188\\
65	0.21378	262.12849572803\\
65	0.21744	274.20861459846\\
65	0.2211	286.595397145475\\
65	0.22476	299.288843369077\\
65	0.22842	312.288953269265\\
65	0.23208	325.59572684604\\
65	0.23574	339.209164099401\\
65	0.2394	353.129265029348\\
65	0.24306	367.356029635882\\
65	0.24672	381.889457919002\\
65	0.25038	396.729549878708\\
65	0.25404	411.876305515001\\
65	0.2577	427.32972482788\\
65	0.26136	443.089807817345\\
65	0.26502	459.156554483397\\
65	0.26868	475.529964826035\\
65	0.27234	492.21003884526\\
65	0.276	509.19677654107\\
65.375	0.093	36.0269720342123\\
65.375	0.09666	37.9818435347439\\
65.375	0.10032	40.2433787118619\\
65.375	0.10398	42.8115775655662\\
65.375	0.10764	45.6864400958569\\
65.375	0.1113	48.8679663027339\\
65.375	0.11496	52.3561561861974\\
65.375	0.11862	56.1510097462471\\
65.375	0.12228	60.2525269828832\\
65.375	0.12594	64.6607078961056\\
65.375	0.1296	69.3755524859143\\
65.375	0.13326	74.3970607523095\\
65.375	0.13692	79.725232695291\\
65.375	0.14058	85.3600683148589\\
65.375	0.14424	91.301567611013\\
65.375	0.1479	97.5497305837536\\
65.375	0.15156	104.104557233081\\
65.375	0.15522	110.966047558994\\
65.375	0.15888	118.134201561493\\
65.375	0.16254	125.609019240579\\
65.375	0.1662	133.390500596252\\
65.375	0.16986	141.47864562851\\
65.375	0.17352	149.873454337355\\
65.375	0.17718	158.574926722787\\
65.375	0.18084	167.583062784804\\
65.375	0.1845	176.897862523408\\
65.375	0.18816	186.519325938599\\
65.375	0.19182	196.447453030376\\
65.375	0.19548	206.682243798739\\
65.375	0.19914	217.223698243688\\
65.375	0.2028	228.071816365224\\
65.375	0.20646	239.226598163346\\
65.375	0.21012	250.688043638055\\
65.375	0.21378	262.45615278935\\
65.375	0.21744	274.530925617231\\
65.375	0.2211	286.912362121698\\
65.375	0.22476	299.600462302752\\
65.375	0.22842	312.595226160393\\
65.375	0.23208	325.896653694619\\
65.375	0.23574	339.504744905432\\
65.375	0.2394	353.419499792832\\
65.375	0.24306	367.640918356817\\
65.375	0.24672	382.169000597389\\
65.375	0.25038	397.003746514548\\
65.375	0.25404	412.145156108293\\
65.375	0.2577	427.593229378624\\
65.375	0.26136	443.347966325541\\
65.375	0.26502	459.409366949045\\
65.375	0.26868	475.777431249135\\
65.375	0.27234	492.452159225812\\
65.375	0.276	509.433550879074\\
65.75	0.093	36.5420059801791\\
65.75	0.09666	38.4915314381627\\
65.75	0.10032	40.7477205727327\\
65.75	0.10398	43.3105733838891\\
65.75	0.10764	46.1800898716318\\
65.75	0.1113	49.3562700359609\\
65.75	0.11496	52.8391138768764\\
65.75	0.11862	56.6286213943781\\
65.75	0.12228	60.7247925884662\\
65.75	0.12594	65.1276274591407\\
65.75	0.1296	69.8371260064016\\
65.75	0.13326	74.8532882302487\\
65.75	0.13692	80.1761141306822\\
65.75	0.14058	85.8056037077021\\
65.75	0.14424	91.7417569613085\\
65.75	0.1479	97.9845738915011\\
65.75	0.15156	104.53405449828\\
65.75	0.15522	111.390198781645\\
65.75	0.15888	118.553006741597\\
65.75	0.16254	126.022478378135\\
65.75	0.1662	133.798613691259\\
65.75	0.16986	141.88141268097\\
65.75	0.17352	150.270875347267\\
65.75	0.17718	158.96700169015\\
65.75	0.18084	167.96979170962\\
65.75	0.1845	177.279245405676\\
65.75	0.18816	186.895362778319\\
65.75	0.19182	196.818143827548\\
65.75	0.19548	207.047588553363\\
65.75	0.19914	217.583696955764\\
65.75	0.2028	228.426469034752\\
65.75	0.20646	239.575904790326\\
65.75	0.21012	251.032004222487\\
65.75	0.21378	262.794767331234\\
65.75	0.21744	274.864194116567\\
65.75	0.2211	287.240284578487\\
65.75	0.22476	299.923038716993\\
65.75	0.22842	312.912456532085\\
65.75	0.23208	326.208538023764\\
65.75	0.23574	339.811283192029\\
65.75	0.2394	353.72069203688\\
65.75	0.24306	367.936764558318\\
65.75	0.24672	382.459500756342\\
65.75	0.25038	397.288900630953\\
65.75	0.25404	412.424964182149\\
65.75	0.2577	427.867691409932\\
65.75	0.26136	443.617082314302\\
65.75	0.26502	459.673136895258\\
65.75	0.26868	476.0358551528\\
65.75	0.27234	492.705237086928\\
65.75	0.276	509.681282697643\\
66.125	0.093	37.0679974067107\\
66.125	0.09666	39.0121768221464\\
66.125	0.10032	41.2630199141685\\
66.125	0.10398	43.8205266827769\\
66.125	0.10764	46.6846971279717\\
66.125	0.1113	49.8555312497528\\
66.125	0.11496	53.3330290481203\\
66.125	0.11862	57.1171905230741\\
66.125	0.12228	61.2080156746142\\
66.125	0.12594	65.6055045027408\\
66.125	0.1296	70.3096570074536\\
66.125	0.13326	75.3204731887528\\
66.125	0.13692	80.6379530466385\\
66.125	0.14058	86.2620965811104\\
66.125	0.14424	92.1929037921688\\
66.125	0.1479	98.4303746798133\\
66.125	0.15156	104.974509244044\\
66.125	0.15522	111.825307484862\\
66.125	0.15888	118.982769402265\\
66.125	0.16254	126.446894996255\\
66.125	0.1662	134.217684266832\\
66.125	0.16986	142.295137213995\\
66.125	0.17352	150.679253837744\\
66.125	0.17718	159.370034138079\\
66.125	0.18084	168.367478115001\\
66.125	0.1845	177.671585768509\\
66.125	0.18816	187.282357098604\\
66.125	0.19182	197.199792105284\\
66.125	0.19548	207.423890788552\\
66.125	0.19914	217.954653148405\\
66.125	0.2028	228.792079184845\\
66.125	0.20646	239.936168897871\\
66.125	0.21012	251.386922287484\\
66.125	0.21378	263.144339353683\\
66.125	0.21744	275.208420096468\\
66.125	0.2211	287.57916451584\\
66.125	0.22476	300.256572611798\\
66.125	0.22842	313.240644384342\\
66.125	0.23208	326.531379833473\\
66.125	0.23574	340.12877895919\\
66.125	0.2394	354.032841761494\\
66.125	0.24306	368.243568240383\\
66.125	0.24672	382.760958395859\\
66.125	0.25038	397.585012227922\\
66.125	0.25404	412.715729736571\\
66.125	0.2577	428.153110921806\\
66.125	0.26136	443.897155783628\\
66.125	0.26502	459.947864322036\\
66.125	0.26868	476.30523653703\\
66.125	0.27234	492.96927242861\\
66.125	0.276	509.939971996777\\
66.5	0.093	37.6049463138072\\
66.5	0.09666	39.543779686695\\
66.5	0.10032	41.7892767361691\\
66.5	0.10398	44.3414374622295\\
66.5	0.10764	47.2002618648764\\
66.5	0.1113	50.3657499441095\\
66.5	0.11496	53.8379016999291\\
66.5	0.11862	57.6167171323349\\
66.5	0.12228	61.7021962413271\\
66.5	0.12594	66.0943390269057\\
66.5	0.1296	70.7931454890706\\
66.5	0.13326	75.7986156278218\\
66.5	0.13692	81.1107494431595\\
66.5	0.14058	86.7295469350835\\
66.5	0.14424	92.6550081035939\\
66.5	0.1479	98.8871329486905\\
66.5	0.15156	105.425921470373\\
66.5	0.15522	112.271373668643\\
66.5	0.15888	119.423489543499\\
66.5	0.16254	126.882269094941\\
66.5	0.1662	134.647712322969\\
66.5	0.16986	142.719819227584\\
66.5	0.17352	151.098589808785\\
66.5	0.17718	159.784024066573\\
66.5	0.18084	168.776122000946\\
66.5	0.1845	178.074883611907\\
66.5	0.18816	187.680308899453\\
66.5	0.19182	197.592397863586\\
66.5	0.19548	207.811150504305\\
66.5	0.19914	218.336566821611\\
66.5	0.2028	229.168646815503\\
66.5	0.20646	240.307390485981\\
66.5	0.21012	251.752797833046\\
66.5	0.21378	263.504868856697\\
66.5	0.21744	275.563603556934\\
66.5	0.2211	287.929001933758\\
66.5	0.22476	300.601063987168\\
66.5	0.22842	313.579789717165\\
66.5	0.23208	326.865179123747\\
66.5	0.23574	340.457232206917\\
66.5	0.2394	354.355948966672\\
66.5	0.24306	368.561329403014\\
66.5	0.24672	383.073373515942\\
66.5	0.25038	397.892081305456\\
66.5	0.25404	413.017452771557\\
66.5	0.2577	428.449487914245\\
66.5	0.26136	444.188186733518\\
66.5	0.26502	460.233549229378\\
66.5	0.26868	476.585575401824\\
66.5	0.27234	493.244265250857\\
66.5	0.276	510.209618776476\\
66.875	0.093	38.1528527014687\\
66.875	0.09666	40.0863400318085\\
66.875	0.10032	42.3264910387347\\
66.875	0.10398	44.8733057222471\\
66.875	0.10764	47.726784082346\\
66.875	0.1113	50.8869261190312\\
66.875	0.11496	54.3537318323028\\
66.875	0.11862	58.1272012221607\\
66.875	0.12228	62.2073342886049\\
66.875	0.12594	66.5941310316355\\
66.875	0.1296	71.2875914512525\\
66.875	0.13326	76.2877155474558\\
66.875	0.13692	81.5945033202455\\
66.875	0.14058	87.2079547696216\\
66.875	0.14424	93.1280698955839\\
66.875	0.1479	99.3548486981326\\
66.875	0.15156	105.888291177268\\
66.875	0.15522	112.728397332989\\
66.875	0.15888	119.875167165297\\
66.875	0.16254	127.328600674191\\
66.875	0.1662	135.088697859672\\
66.875	0.16986	143.155458721738\\
66.875	0.17352	151.528883260392\\
66.875	0.17718	160.208971475631\\
66.875	0.18084	169.195723367457\\
66.875	0.1845	178.489138935869\\
66.875	0.18816	188.089218180868\\
66.875	0.19182	197.995961102453\\
66.875	0.19548	208.209367700624\\
66.875	0.19914	218.729437975382\\
66.875	0.2028	229.556171926726\\
66.875	0.20646	240.689569554656\\
66.875	0.21012	252.129630859173\\
66.875	0.21378	263.876355840276\\
66.875	0.21744	275.929744497965\\
66.875	0.2211	288.289796832241\\
66.875	0.22476	300.956512843103\\
66.875	0.22842	313.929892530552\\
66.875	0.23208	327.209935894586\\
66.875	0.23574	340.796642935208\\
66.875	0.2394	354.690013652415\\
66.875	0.24306	368.890048046209\\
66.875	0.24672	383.396746116589\\
66.875	0.25038	398.210107863556\\
66.875	0.25404	413.330133287109\\
66.875	0.2577	428.756822387248\\
66.875	0.26136	444.490175163974\\
66.875	0.26502	460.530191617286\\
66.875	0.26868	476.876871747184\\
66.875	0.27234	493.530215553669\\
66.875	0.276	510.49022303674\\
67.25	0.093	38.711716569695\\
67.25	0.09666	40.6398578574868\\
67.25	0.10032	42.874662821865\\
67.25	0.10398	45.4161314628295\\
67.25	0.10764	48.2642637803804\\
67.25	0.1113	51.4190597745177\\
67.25	0.11496	54.8805194452413\\
67.25	0.11862	58.6486427925513\\
67.25	0.12228	62.7234298164476\\
67.25	0.12594	67.1048805169302\\
67.25	0.1296	71.7929948939992\\
67.25	0.13326	76.7877729476546\\
67.25	0.13692	82.0892146778963\\
67.25	0.14058	87.6973200847244\\
67.25	0.14424	93.6120891681388\\
67.25	0.1479	99.8335219281395\\
67.25	0.15156	106.361618364727\\
67.25	0.15522	113.1963784779\\
67.25	0.15888	120.33780226766\\
67.25	0.16254	127.785889734006\\
67.25	0.1662	135.540640876939\\
67.25	0.16986	143.602055696458\\
67.25	0.17352	151.970134192563\\
67.25	0.17718	160.644876365254\\
67.25	0.18084	169.626282214532\\
67.25	0.1845	178.914351740397\\
67.25	0.18816	188.509084942847\\
67.25	0.19182	198.410481821884\\
67.25	0.19548	208.618542377508\\
67.25	0.19914	219.133266609717\\
67.25	0.2028	229.954654518513\\
67.25	0.20646	241.082706103896\\
67.25	0.21012	252.517421365864\\
67.25	0.21378	264.258800304419\\
67.25	0.21744	276.306842919561\\
67.25	0.2211	288.661549211289\\
67.25	0.22476	301.322919179603\\
67.25	0.22842	314.290952824503\\
67.25	0.23208	327.56565014599\\
67.25	0.23574	341.147011144064\\
67.25	0.2394	355.035035818723\\
67.25	0.24306	369.229724169969\\
67.25	0.24672	383.731076197801\\
67.25	0.25038	398.53909190222\\
67.25	0.25404	413.653771283225\\
67.25	0.2577	429.075114340816\\
67.25	0.26136	444.803121074994\\
67.25	0.26502	460.837791485758\\
67.25	0.26868	477.179125573108\\
67.25	0.27234	493.827123337045\\
67.25	0.276	510.781784777568\\
67.625	0.093	39.2815379184862\\
67.625	0.09666	41.2043331637301\\
67.625	0.10032	43.4337920855603\\
67.625	0.10398	45.9699146839769\\
67.625	0.10764	48.8127009589799\\
67.625	0.1113	51.9621509105691\\
67.625	0.11496	55.4182645387448\\
67.625	0.11862	59.1810418435068\\
67.625	0.12228	63.2504828248551\\
67.625	0.12594	67.6265874827898\\
67.625	0.1296	72.3093558173109\\
67.625	0.13326	77.2987878284183\\
67.625	0.13692	82.594883516112\\
67.625	0.14058	88.1976428803922\\
67.625	0.14424	94.1070659212586\\
67.625	0.1479	100.323152638711\\
67.625	0.15156	106.845903032751\\
67.625	0.15522	113.675317103376\\
67.625	0.15888	120.811394850588\\
67.625	0.16254	128.254136274386\\
67.625	0.1662	136.003541374771\\
67.625	0.16986	144.059610151742\\
67.625	0.17352	152.422342605299\\
67.625	0.17718	161.091738735443\\
67.625	0.18084	170.067798542172\\
67.625	0.1845	179.350522025489\\
67.625	0.18816	188.939909185392\\
67.625	0.19182	198.835960021881\\
67.625	0.19548	209.038674534956\\
67.625	0.19914	219.548052724618\\
67.625	0.2028	230.364094590866\\
67.625	0.20646	241.4868001337\\
67.625	0.21012	252.916169353121\\
67.625	0.21378	264.652202249128\\
67.625	0.21744	276.694898821722\\
67.625	0.2211	289.044259070901\\
67.625	0.22476	301.700282996668\\
67.625	0.22842	314.66297059902\\
67.625	0.23208	327.932321877959\\
67.625	0.23574	341.508336833484\\
67.625	0.2394	355.391015465596\\
67.625	0.24306	369.580357774294\\
67.625	0.24672	384.076363759578\\
67.625	0.25038	398.879033421449\\
67.625	0.25404	413.988366759906\\
67.625	0.2577	429.40436377495\\
67.625	0.26136	445.127024466579\\
67.625	0.26502	461.156348834795\\
67.625	0.26868	477.492336879597\\
67.625	0.27234	494.134988600986\\
67.625	0.276	511.084303998961\\
68	0.093	39.8623167478423\\
68	0.09666	41.7797659505383\\
68	0.10032	44.0038788298206\\
68	0.10398	46.5346553856892\\
68	0.10764	49.3720956181442\\
68	0.1113	52.5161995271854\\
68	0.11496	55.9669671128132\\
68	0.11862	59.7243983750272\\
68	0.12228	63.7884933138276\\
68	0.12594	68.1592519292143\\
68	0.1296	72.8366742211875\\
68	0.13326	77.8207601897468\\
68	0.13692	83.1115098348926\\
68	0.14058	88.7089231566248\\
68	0.14424	94.6130001549434\\
68	0.1479	100.823740829848\\
68	0.15156	107.341145181339\\
68	0.15522	114.165213209417\\
68	0.15888	121.295944914081\\
68	0.16254	128.733340295331\\
68	0.1662	136.477399353168\\
68	0.16986	144.528122087591\\
68	0.17352	152.8855084986\\
68	0.17718	161.549558586196\\
68	0.18084	170.520272350378\\
68	0.1845	179.797649791146\\
68	0.18816	189.381690908501\\
68	0.19182	199.272395702442\\
68	0.19548	209.469764172969\\
68	0.19914	219.973796320083\\
68	0.2028	230.784492143783\\
68	0.20646	241.90185164407\\
68	0.21012	253.325874820943\\
68	0.21378	265.056561674402\\
68	0.21744	277.093912204447\\
68	0.2211	289.437926411079\\
68	0.22476	302.088604294297\\
68	0.22842	315.045945854102\\
68	0.23208	328.309951090493\\
68	0.23574	341.88062000347\\
68	0.2394	355.757952593034\\
68	0.24306	369.941948859184\\
68	0.24672	384.43260880192\\
68	0.25038	399.229932421243\\
68	0.25404	414.333919717152\\
68	0.2577	429.744570689648\\
68	0.26136	445.461885338729\\
68	0.26502	461.485863664397\\
68	0.26868	477.816505666652\\
68	0.27234	494.453811345493\\
68	0.276	511.39778070092\\
68.375	0.093	40.4540530577633\\
68.375	0.09666	42.3661562179113\\
68.375	0.10032	44.5849230546456\\
68.375	0.10398	47.1103535679663\\
68.375	0.10764	49.9424477578733\\
68.375	0.1113	53.0812056243667\\
68.375	0.11496	56.5266271674464\\
68.375	0.11862	60.2787123871125\\
68.375	0.12228	64.3374612833649\\
68.375	0.12594	68.7028738562037\\
68.375	0.1296	73.3749501056288\\
68.375	0.13326	78.3536900316403\\
68.375	0.13692	83.6390936342381\\
68.375	0.14058	89.2311609134224\\
68.375	0.14424	95.129891869193\\
68.375	0.1479	101.33528650155\\
68.375	0.15156	107.847344810493\\
68.375	0.15522	114.666066796023\\
68.375	0.15888	121.791452458139\\
68.375	0.16254	129.223501796841\\
68.375	0.1662	136.96221481213\\
68.375	0.16986	145.007591504005\\
68.375	0.17352	153.359631872466\\
68.375	0.17718	162.018335917514\\
68.375	0.18084	170.983703639148\\
68.375	0.1845	180.255735037368\\
68.375	0.18816	189.834430112175\\
68.375	0.19182	199.719788863568\\
68.375	0.19548	209.911811291548\\
68.375	0.19914	220.410497396113\\
68.375	0.2028	231.215847177266\\
68.375	0.20646	242.327860635004\\
68.375	0.21012	253.746537769329\\
68.375	0.21378	265.47187858024\\
68.375	0.21744	277.503883067738\\
68.375	0.2211	289.842551231822\\
68.375	0.22476	302.487883072492\\
68.375	0.22842	315.439878589749\\
68.375	0.23208	328.698537783592\\
68.375	0.23574	342.263860654021\\
68.375	0.2394	356.135847201037\\
68.375	0.24306	370.314497424639\\
68.375	0.24672	384.799811324827\\
68.375	0.25038	399.591788901602\\
68.375	0.25404	414.690430154963\\
68.375	0.2577	430.09573508491\\
68.375	0.26136	445.807703691444\\
68.375	0.26502	461.826335974564\\
68.375	0.26868	478.151631934271\\
68.375	0.27234	494.783591570564\\
68.375	0.276	511.722214883443\\
68.75	0.093	41.0567468482492\\
68.75	0.09666	42.9635039658492\\
68.75	0.10032	45.1769247600356\\
68.75	0.10398	47.6970092308083\\
68.75	0.10764	50.5237573781674\\
68.75	0.1113	53.6571692021128\\
68.75	0.11496	57.0972447026446\\
68.75	0.11862	60.8439838797627\\
68.75	0.12228	64.8973867334672\\
68.75	0.12594	69.2574532637581\\
68.75	0.1296	73.9241834706352\\
68.75	0.13326	78.8975773540987\\
68.75	0.13692	84.1776349141486\\
68.75	0.14058	89.7643561507849\\
68.75	0.14424	95.6577410640075\\
68.75	0.1479	101.857789653816\\
68.75	0.15156	108.364501920212\\
68.75	0.15522	115.177877863193\\
68.75	0.15888	122.297917482761\\
68.75	0.16254	129.724620778916\\
68.75	0.1662	137.457987751656\\
68.75	0.16986	145.498018400983\\
68.75	0.17352	153.844712726897\\
68.75	0.17718	162.498070729397\\
68.75	0.18084	171.458092408483\\
68.75	0.1845	180.724777764155\\
68.75	0.18816	190.298126796414\\
68.75	0.19182	200.178139505259\\
68.75	0.19548	210.364815890691\\
68.75	0.19914	220.858155952709\\
68.75	0.2028	231.658159691313\\
68.75	0.20646	242.764827106503\\
68.75	0.21012	254.17815819828\\
68.75	0.21378	265.898152966643\\
68.75	0.21744	277.924811411593\\
68.75	0.2211	290.258133533129\\
68.75	0.22476	302.898119331251\\
68.75	0.22842	315.84476880596\\
68.75	0.23208	329.098081957255\\
68.75	0.23574	342.658058785137\\
68.75	0.2394	356.524699289604\\
68.75	0.24306	370.698003470658\\
68.75	0.24672	385.177971328299\\
68.75	0.25038	399.964602862526\\
68.75	0.25404	415.057898073339\\
68.75	0.2577	430.457856960738\\
68.75	0.26136	446.164479524724\\
68.75	0.26502	462.177765765296\\
68.75	0.26868	478.497715682455\\
68.75	0.27234	495.1243292762\\
68.75	0.276	512.057606546531\\
69.125	0.093	41.6703981193\\
69.125	0.09666	43.571809194352\\
69.125	0.10032	45.7798839459904\\
69.125	0.10398	48.2946223742152\\
69.125	0.10764	51.1160244790263\\
69.125	0.1113	54.2440902604238\\
69.125	0.11496	57.6788197184076\\
69.125	0.11862	61.4202128529778\\
69.125	0.12228	65.4682696641343\\
69.125	0.12594	69.8229901518772\\
69.125	0.1296	74.4843743162064\\
69.125	0.13326	79.4524221571219\\
69.125	0.13692	84.7271336746239\\
69.125	0.14058	90.3085088687122\\
69.125	0.14424	96.1965477393869\\
69.125	0.1479	102.391250286648\\
69.125	0.15156	108.892616510495\\
69.125	0.15522	115.700646410929\\
69.125	0.15888	122.815339987949\\
69.125	0.16254	130.236697241555\\
69.125	0.1662	137.964718171748\\
69.125	0.16986	145.999402778527\\
69.125	0.17352	154.340751061893\\
69.125	0.17718	162.988763021844\\
69.125	0.18084	171.943438658382\\
69.125	0.1845	181.204777971507\\
69.125	0.18816	190.772780961218\\
69.125	0.19182	200.647447627515\\
69.125	0.19548	210.828777970399\\
69.125	0.19914	221.316771989869\\
69.125	0.2028	232.111429685925\\
69.125	0.20646	243.212751058567\\
69.125	0.21012	254.620736107796\\
69.125	0.21378	266.335384833612\\
69.125	0.21744	278.356697236013\\
69.125	0.2211	290.684673315001\\
69.125	0.22476	303.319313070576\\
69.125	0.22842	316.260616502737\\
69.125	0.23208	329.508583611484\\
69.125	0.23574	343.063214396817\\
69.125	0.2394	356.924508858737\\
69.125	0.24306	371.092466997243\\
69.125	0.24672	385.567088812335\\
69.125	0.25038	400.348374304014\\
69.125	0.25404	415.436323472279\\
69.125	0.2577	430.830936317131\\
69.125	0.26136	446.532212838569\\
69.125	0.26502	462.540153036593\\
69.125	0.26868	478.854756911204\\
69.125	0.27234	495.476024462401\\
69.125	0.276	512.403955690184\\
69.5	0.093	42.2950068709157\\
69.5	0.09666	44.1910719034198\\
69.5	0.10032	46.3938006125102\\
69.5	0.10398	48.903192998187\\
69.5	0.10764	51.7192490604502\\
69.5	0.1113	54.8419687992997\\
69.5	0.11496	58.2713522147356\\
69.5	0.11862	62.0073993067578\\
69.5	0.12228	66.0501100753664\\
69.5	0.12594	70.3994845205613\\
69.5	0.1296	75.0555226423425\\
69.5	0.13326	80.0182244407101\\
69.5	0.13692	85.2875899156641\\
69.5	0.14058	90.8636190672045\\
69.5	0.14424	96.7463118953312\\
69.5	0.1479	102.935668400044\\
69.5	0.15156	109.431688581344\\
69.5	0.15522	116.234372439229\\
69.5	0.15888	123.343719973701\\
69.5	0.16254	130.75973118476\\
69.5	0.1662	138.482406072405\\
69.5	0.16986	146.511744636636\\
69.5	0.17352	154.847746877453\\
69.5	0.17718	163.490412794857\\
69.5	0.18084	172.439742388847\\
69.5	0.1845	181.695735659424\\
69.5	0.18816	191.258392606587\\
69.5	0.19182	201.127713230336\\
69.5	0.19548	211.303697530672\\
69.5	0.19914	221.786345507594\\
69.5	0.2028	232.575657161102\\
69.5	0.20646	243.671632491197\\
69.5	0.21012	255.074271497878\\
69.5	0.21378	266.783574181145\\
69.5	0.21744	278.799540540999\\
69.5	0.2211	291.122170577439\\
69.5	0.22476	303.751464290465\\
69.5	0.22842	316.687421680078\\
69.5	0.23208	329.930042746277\\
69.5	0.23574	343.479327489062\\
69.5	0.2394	357.335275908434\\
69.5	0.24306	371.497888004392\\
69.5	0.24672	385.967163776937\\
69.5	0.25038	400.743103226068\\
69.5	0.25404	415.825706351785\\
69.5	0.2577	431.214973154089\\
69.5	0.26136	446.910903632979\\
69.5	0.26502	462.913497788455\\
69.5	0.26868	479.222755620517\\
69.5	0.27234	495.838677129166\\
69.5	0.276	512.761262314402\\
69.875	0.093	42.9305731030962\\
69.875	0.09666	44.8212920930523\\
69.875	0.10032	47.0186747595949\\
69.875	0.10398	49.5227211027237\\
69.875	0.10764	52.3334311224389\\
69.875	0.1113	55.4508048187404\\
69.875	0.11496	58.8748421916284\\
69.875	0.11862	62.6055432411026\\
69.875	0.12228	66.6429079671632\\
69.875	0.12594	70.9869363698102\\
69.875	0.1296	75.6376284490434\\
69.875	0.13326	80.5949842048631\\
69.875	0.13692	85.8590036372692\\
69.875	0.14058	91.4296867462616\\
69.875	0.14424	97.3070335318403\\
69.875	0.1479	103.491043994005\\
69.875	0.15156	109.981718132757\\
69.875	0.15522	116.779055948095\\
69.875	0.15888	123.883057440019\\
69.875	0.16254	131.293722608529\\
69.875	0.1662	139.011051453626\\
69.875	0.16986	147.035043975309\\
69.875	0.17352	155.365700173579\\
69.875	0.17718	164.003020048435\\
69.875	0.18084	172.947003599877\\
69.875	0.1845	182.197650827905\\
69.875	0.18816	191.75496173252\\
69.875	0.19182	201.618936313722\\
69.875	0.19548	211.789574571509\\
69.875	0.19914	222.266876505883\\
69.875	0.2028	233.050842116844\\
69.875	0.20646	244.14147140439\\
69.875	0.21012	255.538764368523\\
69.875	0.21378	267.242721009243\\
69.875	0.21744	279.253341326549\\
69.875	0.2211	291.570625320441\\
69.875	0.22476	304.194572990919\\
69.875	0.22842	317.125184337984\\
69.875	0.23208	330.362459361635\\
69.875	0.23574	343.906398061873\\
69.875	0.2394	357.757000438696\\
69.875	0.24306	371.914266492107\\
69.875	0.24672	386.378196222103\\
69.875	0.25038	401.148789628686\\
69.875	0.25404	416.226046711855\\
69.875	0.2577	431.609967471611\\
69.875	0.26136	447.300551907953\\
69.875	0.26502	463.297800020881\\
69.875	0.26868	479.601711810396\\
69.875	0.27234	496.212287276497\\
69.875	0.276	513.129526419184\\
70.25	0.093	43.5770968158417\\
70.25	0.09666	45.4624697632499\\
70.25	0.10032	47.6545063872444\\
70.25	0.10398	50.1532066878253\\
70.25	0.10764	52.9585706649925\\
70.25	0.1113	56.0705983187461\\
70.25	0.11496	59.4892896490861\\
70.25	0.11862	63.2146446560124\\
70.25	0.12228	67.246663339525\\
70.25	0.12594	71.585345699624\\
70.25	0.1296	76.2306917363094\\
70.25	0.13326	81.1827014495811\\
70.25	0.13692	86.4413748394391\\
70.25	0.14058	92.0067119058836\\
70.25	0.14424	97.8787126489143\\
70.25	0.1479	104.057377068531\\
70.25	0.15156	110.542705164735\\
70.25	0.15522	117.334696937525\\
70.25	0.15888	124.433352386901\\
70.25	0.16254	131.838671512863\\
70.25	0.1662	139.550654315412\\
70.25	0.16986	147.569300794548\\
70.25	0.17352	155.894610950269\\
70.25	0.17718	164.526584782577\\
70.25	0.18084	173.465222291471\\
70.25	0.1845	182.710523476952\\
70.25	0.18816	192.262488339019\\
70.25	0.19182	202.121116877672\\
70.25	0.19548	212.286409092912\\
70.25	0.19914	222.758364984738\\
70.25	0.2028	233.53698455315\\
70.25	0.20646	244.622267798149\\
70.25	0.21012	256.014214719734\\
70.25	0.21378	267.712825317906\\
70.25	0.21744	279.718099592663\\
70.25	0.2211	292.030037544008\\
70.25	0.22476	304.648639171938\\
70.25	0.22842	317.573904476455\\
70.25	0.23208	330.805833457558\\
70.25	0.23574	344.344426115248\\
70.25	0.2394	358.189682449524\\
70.25	0.24306	372.341602460386\\
70.25	0.24672	386.800186147835\\
70.25	0.25038	401.56543351187\\
70.25	0.25404	416.637344552491\\
70.25	0.2577	432.015919269699\\
70.25	0.26136	447.701157663493\\
70.25	0.26502	463.693059733873\\
70.25	0.26868	479.99162548084\\
70.25	0.27234	496.596854904393\\
70.25	0.276	513.508748004532\\
70.625	0.093	44.234578009152\\
70.625	0.09666	46.1146049140123\\
70.625	0.10032	48.3012954954589\\
70.625	0.10398	50.7946497534918\\
70.625	0.10764	53.5946676881111\\
70.625	0.1113	56.7013492993167\\
70.625	0.11496	60.1146945871087\\
70.625	0.11862	63.8347035514871\\
70.625	0.12228	67.8613761924518\\
70.625	0.12594	72.1947125100028\\
70.625	0.1296	76.8347125041402\\
70.625	0.13326	81.7813761748639\\
70.625	0.13692	87.034703522174\\
70.625	0.14058	92.5946945460705\\
70.625	0.14424	98.4613492465534\\
70.625	0.1479	104.634667623623\\
70.625	0.15156	111.114649677278\\
70.625	0.15522	117.90129540752\\
70.625	0.15888	124.994604814348\\
70.625	0.16254	132.394577897763\\
70.625	0.1662	140.101214657764\\
70.625	0.16986	148.114515094351\\
70.625	0.17352	156.434479207524\\
70.625	0.17718	165.061106997284\\
70.625	0.18084	173.994398463631\\
70.625	0.1845	183.234353606563\\
70.625	0.18816	192.780972426082\\
70.625	0.19182	202.634254922188\\
70.625	0.19548	212.79420109488\\
70.625	0.19914	223.260810944158\\
70.625	0.2028	234.034084470022\\
70.625	0.20646	245.114021672473\\
70.625	0.21012	256.50062255151\\
70.625	0.21378	268.193887107133\\
70.625	0.21744	280.193815339343\\
70.625	0.2211	292.500407248139\\
70.625	0.22476	305.113662833522\\
70.625	0.22842	318.033582095491\\
70.625	0.23208	331.260165034046\\
70.625	0.23574	344.793411649188\\
70.625	0.2394	358.633321940916\\
70.625	0.24306	372.77989590923\\
70.625	0.24672	387.233133554131\\
70.625	0.25038	401.993034875618\\
70.625	0.25404	417.059599873691\\
70.625	0.2577	432.432828548351\\
70.625	0.26136	448.112720899597\\
70.625	0.26502	464.099276927429\\
70.625	0.26868	480.392496631848\\
70.625	0.27234	496.992380012853\\
70.625	0.276	513.898927070445\\
71	0.093	44.9030166830273\\
71	0.09666	46.7776975453395\\
71	0.10032	48.9590420842382\\
71	0.10398	51.4470502997231\\
71	0.10764	54.2417221917945\\
71	0.1113	57.3430577604521\\
71	0.11496	60.7510570056962\\
71	0.11862	64.4657199275266\\
71	0.12228	68.4870465259433\\
71	0.12594	72.8150368009464\\
71	0.1296	77.4496907525358\\
71	0.13326	82.3910083807116\\
71	0.13692	87.6389896854737\\
71	0.14058	93.1936346668223\\
71	0.14424	99.0549433247572\\
71	0.1479	105.222915659278\\
71	0.15156	111.697551670386\\
71	0.15522	118.47885135808\\
71	0.15888	125.56681472236\\
71	0.16254	132.961441763227\\
71	0.1662	140.66273248068\\
71	0.16986	148.670686874719\\
71	0.17352	156.985304945345\\
71	0.17718	165.606586692557\\
71	0.18084	174.534532116355\\
71	0.1845	183.76914121674\\
71	0.18816	193.310413993711\\
71	0.19182	203.158350447268\\
71	0.19548	213.312950577412\\
71	0.19914	223.774214384142\\
71	0.2028	234.542141867459\\
71	0.20646	245.616733027361\\
71	0.21012	256.997987863851\\
71	0.21378	268.685906376926\\
71	0.21744	280.680488566588\\
71	0.2211	292.981734432836\\
71	0.22476	305.589643975671\\
71	0.22842	318.504217195092\\
71	0.23208	331.725454091099\\
71	0.23574	345.253354663693\\
71	0.2394	359.087918912873\\
71	0.24306	373.229146838639\\
71	0.24672	387.677038440992\\
71	0.25038	402.431593719931\\
71	0.25404	417.492812675456\\
71	0.2577	432.860695307568\\
71	0.26136	448.535241616266\\
71	0.26502	464.516451601551\\
71	0.26868	480.804325263421\\
71	0.27234	497.398862601878\\
71	0.276	514.300063616922\\
71.375	0.093	45.5824128374674\\
71.375	0.09666	47.4517476572317\\
71.375	0.10032	49.6277461535824\\
71.375	0.10398	52.1104083265194\\
71.375	0.10764	54.8997341760428\\
71.375	0.1113	57.9957237021525\\
71.375	0.11496	61.3983769048486\\
71.375	0.11862	65.107693784131\\
71.375	0.12228	69.1236743399998\\
71.375	0.12594	73.4463185724549\\
71.375	0.1296	78.0756264814964\\
71.375	0.13326	83.0115980671242\\
71.375	0.13692	88.2542333293385\\
71.375	0.14058	93.803532268139\\
71.375	0.14424	99.659494883526\\
71.375	0.1479	105.822121175499\\
71.375	0.15156	112.291411144059\\
71.375	0.15522	119.067364789205\\
71.375	0.15888	126.149982110937\\
71.375	0.16254	133.539263109256\\
71.375	0.1662	141.235207784161\\
71.375	0.16986	149.237816135652\\
71.375	0.17352	157.54708816373\\
71.375	0.17718	166.163023868394\\
71.375	0.18084	175.085623249644\\
71.375	0.1845	184.314886307481\\
71.375	0.18816	193.850813041904\\
71.375	0.19182	203.693403452914\\
71.375	0.19548	213.842657540509\\
71.375	0.19914	224.298575304692\\
71.375	0.2028	235.06115674546\\
71.375	0.20646	246.130401862815\\
71.375	0.21012	257.506310656756\\
71.375	0.21378	269.188883127284\\
71.375	0.21744	281.178119274398\\
71.375	0.2211	293.474019098098\\
71.375	0.22476	306.076582598385\\
71.375	0.22842	318.985809775258\\
71.375	0.23208	332.201700628717\\
71.375	0.23574	345.724255158763\\
71.375	0.2394	359.553473365395\\
71.375	0.24306	373.689355248613\\
71.375	0.24672	388.131900808418\\
71.375	0.25038	402.881110044809\\
71.375	0.25404	417.936982957786\\
71.375	0.2577	433.29951954735\\
71.375	0.26136	448.9687198135\\
71.375	0.26502	464.944583756237\\
71.375	0.26868	481.22711137556\\
71.375	0.27234	497.816302671469\\
71.375	0.276	514.712157643964\\
71.75	0.093	46.2727664724724\\
71.75	0.09666	48.1367552496888\\
71.75	0.10032	50.3074077034915\\
71.75	0.10398	52.7847238338806\\
71.75	0.10764	55.568703640856\\
71.75	0.1113	58.6593471244177\\
71.75	0.11496	62.0566542845659\\
71.75	0.11862	65.7606251213003\\
71.75	0.12228	69.7712596346212\\
71.75	0.12594	74.0885578245284\\
71.75	0.1296	78.7125196910218\\
71.75	0.13326	83.6431452341017\\
71.75	0.13692	88.880434453768\\
71.75	0.14058	94.4243873500206\\
71.75	0.14424	100.27500392286\\
71.75	0.1479	106.432284172285\\
71.75	0.15156	112.896228098296\\
71.75	0.15522	119.666835700894\\
71.75	0.15888	126.744106980079\\
71.75	0.16254	134.128041935849\\
71.75	0.1662	141.818640568207\\
71.75	0.16986	149.81590287715\\
71.75	0.17352	158.11982886268\\
71.75	0.17718	166.730418524796\\
71.75	0.18084	175.647671863498\\
71.75	0.1845	184.871588878787\\
71.75	0.18816	194.402169570662\\
71.75	0.19182	204.239413939124\\
71.75	0.19548	214.383321984172\\
71.75	0.19914	224.833893705806\\
71.75	0.2028	235.591129104026\\
71.75	0.20646	246.655028178833\\
71.75	0.21012	258.025590930227\\
71.75	0.21378	269.702817358206\\
71.75	0.21744	281.686707462772\\
71.75	0.2211	293.977261243924\\
71.75	0.22476	306.574478701663\\
71.75	0.22842	319.478359835988\\
71.75	0.23208	332.6889046469\\
71.75	0.23574	346.206113134397\\
71.75	0.2394	360.029985298481\\
71.75	0.24306	374.160521139152\\
71.75	0.24672	388.597720656409\\
71.75	0.25038	403.341583850252\\
71.75	0.25404	418.392110720681\\
71.75	0.2577	433.749301267697\\
71.75	0.26136	449.413155491299\\
71.75	0.26502	465.383673391488\\
71.75	0.26868	481.660854968263\\
71.75	0.27234	498.244700221624\\
71.75	0.276	515.135209151571\\
72.125	0.093	46.9740775880423\\
72.125	0.09666	48.8327203227107\\
72.125	0.10032	50.9980267339655\\
72.125	0.10398	53.4699968218066\\
72.125	0.10764	56.2486305862341\\
72.125	0.1113	59.3339280272479\\
72.125	0.11496	62.7258891448481\\
72.125	0.11862	66.4245139390346\\
72.125	0.12228	70.4298024098075\\
72.125	0.12594	74.7417545571667\\
72.125	0.1296	79.3603703811122\\
72.125	0.13326	84.2856498816441\\
72.125	0.13692	89.5175930587625\\
72.125	0.14058	95.0561999124671\\
72.125	0.14424	100.901470442758\\
72.125	0.1479	107.053404649635\\
72.125	0.15156	113.512002533099\\
72.125	0.15522	120.277264093149\\
72.125	0.15888	127.349189329786\\
72.125	0.16254	134.727778243008\\
72.125	0.1662	142.413030832817\\
72.125	0.16986	150.404947099213\\
72.125	0.17352	158.703527042195\\
72.125	0.17718	167.308770661763\\
72.125	0.18084	176.220677957917\\
72.125	0.1845	185.439248930658\\
72.125	0.18816	194.964483579985\\
72.125	0.19182	204.796381905899\\
72.125	0.19548	214.934943908399\\
72.125	0.19914	225.380169587485\\
72.125	0.2028	236.132058943158\\
72.125	0.20646	247.190611975417\\
72.125	0.21012	258.555828684262\\
72.125	0.21378	270.227709069693\\
72.125	0.21744	282.206253131712\\
72.125	0.2211	294.491460870316\\
72.125	0.22476	307.083332285507\\
72.125	0.22842	319.981867377284\\
72.125	0.23208	333.187066145647\\
72.125	0.23574	346.698928590597\\
72.125	0.2394	360.517454712133\\
72.125	0.24306	374.642644510255\\
72.125	0.24672	389.074497984964\\
72.125	0.25038	403.81301513626\\
72.125	0.25404	418.858195964141\\
72.125	0.2577	434.210040468609\\
72.125	0.26136	449.868548649663\\
72.125	0.26502	465.833720507304\\
72.125	0.26868	482.105556041531\\
72.125	0.27234	498.684055252344\\
72.125	0.276	515.569218139744\\
72.5	0.093	47.6863461841771\\
72.5	0.09666	49.5396428762976\\
72.5	0.10032	51.6996032450044\\
72.5	0.10398	54.1662272902975\\
72.5	0.10764	56.9395150121771\\
72.5	0.1113	60.0194664106429\\
72.5	0.11496	63.4060814856951\\
72.5	0.11862	67.0993602373337\\
72.5	0.12228	71.0993026655586\\
72.5	0.12594	75.4059087703698\\
72.5	0.1296	80.0191785517674\\
72.5	0.13326	84.9391120097514\\
72.5	0.13692	90.1657091443217\\
72.5	0.14058	95.6989699554785\\
72.5	0.14424	101.538894443221\\
72.5	0.1479	107.685482607551\\
72.5	0.15156	114.138734448467\\
72.5	0.15522	120.898649965969\\
72.5	0.15888	127.965229160057\\
72.5	0.16254	135.338472030732\\
72.5	0.1662	143.018378577993\\
72.5	0.16986	151.00494880184\\
72.5	0.17352	159.298182702274\\
72.5	0.17718	167.898080279295\\
72.5	0.18084	176.804641532901\\
72.5	0.1845	186.017866463094\\
72.5	0.18816	195.537755069873\\
72.5	0.19182	205.364307353239\\
72.5	0.19548	215.497523313191\\
72.5	0.19914	225.937402949729\\
72.5	0.2028	236.683946262854\\
72.5	0.20646	247.737153252565\\
72.5	0.21012	259.097023918862\\
72.5	0.21378	270.763558261746\\
72.5	0.21744	282.736756281216\\
72.5	0.2211	295.016617977272\\
72.5	0.22476	307.603143349915\\
72.5	0.22842	320.496332399144\\
72.5	0.23208	333.69618512496\\
72.5	0.23574	347.202701527362\\
72.5	0.2394	361.01588160635\\
72.5	0.24306	375.135725361924\\
72.5	0.24672	389.562232794085\\
72.5	0.25038	404.295403902832\\
72.5	0.25404	419.335238688166\\
72.5	0.2577	434.681737150086\\
72.5	0.26136	450.334899288592\\
72.5	0.26502	466.294725103685\\
72.5	0.26868	482.561214595364\\
72.5	0.27234	499.134367763629\\
72.5	0.276	516.014184608481\\
72.875	0.093	48.4095722608768\\
72.875	0.09666	50.2575229104493\\
72.875	0.10032	52.4121372366082\\
72.875	0.10398	54.8734152393533\\
72.875	0.10764	57.6413569186849\\
72.875	0.1113	60.7159622746028\\
72.875	0.11496	64.0972313071071\\
72.875	0.11862	67.7851640161977\\
72.875	0.12228	71.7797604018747\\
72.875	0.12594	76.0810204641379\\
72.875	0.1296	80.6889442029876\\
72.875	0.13326	85.6035316184236\\
72.875	0.13692	90.824782710446\\
72.875	0.14058	96.3526974790547\\
72.875	0.14424	102.18727592425\\
72.875	0.1479	108.328518046031\\
72.875	0.15156	114.776423844399\\
72.875	0.15522	121.530993319353\\
72.875	0.15888	128.592226470894\\
72.875	0.16254	135.96012329902\\
72.875	0.1662	143.634683803734\\
72.875	0.16986	151.615907985033\\
72.875	0.17352	159.903795842919\\
72.875	0.17718	168.498347377391\\
72.875	0.18084	177.39956258845\\
72.875	0.1845	186.607441476095\\
72.875	0.18816	196.121984040326\\
72.875	0.19182	205.943190281144\\
72.875	0.19548	216.071060198548\\
72.875	0.19914	226.505593792538\\
72.875	0.2028	237.246791063115\\
72.875	0.20646	248.294652010278\\
72.875	0.21012	259.649176634027\\
72.875	0.21378	271.310364934363\\
72.875	0.21744	283.278216911285\\
72.875	0.2211	295.552732564793\\
72.875	0.22476	308.133911894888\\
72.875	0.22842	321.021754901569\\
72.875	0.23208	334.216261584837\\
72.875	0.23574	347.717431944691\\
72.875	0.2394	361.525265981131\\
72.875	0.24306	375.639763694158\\
72.875	0.24672	390.06092508377\\
72.875	0.25038	404.78875014997\\
72.875	0.25404	419.823238892755\\
72.875	0.2577	435.164391312127\\
72.875	0.26136	450.812207408086\\
72.875	0.26502	466.76668718063\\
72.875	0.26868	483.027830629761\\
72.875	0.27234	499.595637755479\\
72.875	0.276	516.470108557782\\
73.25	0.093	49.1437558181414\\
73.25	0.09666	50.986360425166\\
73.25	0.10032	53.1356287087768\\
73.25	0.10398	55.5915606689741\\
73.25	0.10764	58.3541563057577\\
73.25	0.1113	61.4234156191276\\
73.25	0.11496	64.799338609084\\
73.25	0.11862	68.4819252756266\\
73.25	0.12228	72.4711756187556\\
73.25	0.12594	76.767089638471\\
73.25	0.1296	81.3696673347727\\
73.25	0.13326	86.2789087076607\\
73.25	0.13692	91.4948137571351\\
73.25	0.14058	97.0173824831959\\
73.25	0.14424	102.846614885843\\
73.25	0.1479	108.982510965076\\
73.25	0.15156	115.425070720896\\
73.25	0.15522	122.174294153302\\
73.25	0.15888	129.230181262295\\
73.25	0.16254	136.592732047874\\
73.25	0.1662	144.261946510039\\
73.25	0.16986	152.237824648791\\
73.25	0.17352	160.520366464129\\
73.25	0.17718	169.109571956053\\
73.25	0.18084	178.005441124563\\
73.25	0.1845	187.20797396966\\
73.25	0.18816	196.717170491344\\
73.25	0.19182	206.533030689613\\
73.25	0.19548	216.655554564469\\
73.25	0.19914	227.084742115912\\
73.25	0.2028	237.820593343941\\
73.25	0.20646	248.863108248556\\
73.25	0.21012	260.212286829757\\
73.25	0.21378	271.868129087545\\
73.25	0.21744	283.830635021919\\
73.25	0.2211	296.099804632879\\
73.25	0.22476	308.675637920427\\
73.25	0.22842	321.55813488456\\
73.25	0.23208	334.747295525279\\
73.25	0.23574	348.243119842585\\
73.25	0.2394	362.045607836477\\
73.25	0.24306	376.154759506956\\
73.25	0.24672	390.570574854021\\
73.25	0.25038	405.293053877672\\
73.25	0.25404	420.32219657791\\
73.25	0.2577	435.658002954734\\
73.25	0.26136	451.300473008144\\
73.25	0.26502	467.249606738141\\
73.25	0.26868	483.505404144724\\
73.25	0.27234	500.067865227893\\
73.25	0.276	516.936989987649\\
73.625	0.093	49.8888968559709\\
73.625	0.09666	51.7261554204475\\
73.625	0.10032	53.8700776615104\\
73.625	0.10398	56.3206635791597\\
73.625	0.10764	59.0779131733954\\
73.625	0.1113	62.1418264442173\\
73.625	0.11496	65.5124033916257\\
73.625	0.11862	69.1896440156204\\
73.625	0.12228	73.1735483162014\\
73.625	0.12594	77.4641162933688\\
73.625	0.1296	82.0613479471226\\
73.625	0.13326	86.9652432774626\\
73.625	0.13692	92.1758022843891\\
73.625	0.14058	97.6930249679019\\
73.625	0.14424	103.516911328001\\
73.625	0.1479	109.647461364687\\
73.625	0.15156	116.084675077958\\
73.625	0.15522	122.828552467817\\
73.625	0.15888	129.879093534261\\
73.625	0.16254	137.236298277292\\
73.625	0.1662	144.900166696909\\
73.625	0.16986	152.870698793113\\
73.625	0.17352	161.147894565903\\
73.625	0.17718	169.731754015279\\
73.625	0.18084	178.622277141242\\
73.625	0.1845	187.819463943791\\
73.625	0.18816	197.323314422926\\
73.625	0.19182	207.133828578648\\
73.625	0.19548	217.251006410956\\
73.625	0.19914	227.674847919851\\
73.625	0.2028	238.405353105331\\
73.625	0.20646	249.442521967399\\
73.625	0.21012	260.786354506052\\
73.625	0.21378	272.436850721292\\
73.625	0.21744	284.394010613118\\
73.625	0.2211	296.65783418153\\
73.625	0.22476	309.22832142653\\
73.625	0.22842	322.105472348115\\
73.625	0.23208	335.289286946286\\
73.625	0.23574	348.779765221044\\
73.625	0.2394	362.576907172389\\
73.625	0.24306	376.680712800319\\
73.625	0.24672	391.091182104836\\
73.625	0.25038	405.80831508594\\
73.625	0.25404	420.832111743629\\
73.625	0.2577	436.162572077905\\
73.625	0.26136	451.799696088768\\
73.625	0.26502	467.743483776217\\
73.625	0.26868	483.993935140252\\
73.625	0.27234	500.551050180873\\
73.625	0.276	517.414828898081\\
74	0.093	50.6449953743653\\
74	0.09666	52.4769078962939\\
74	0.10032	54.6154840948089\\
74	0.10398	57.0607239699102\\
74	0.10764	59.8126275215979\\
74	0.1113	62.871194749872\\
74	0.11496	66.2364256547323\\
74	0.11862	69.908320236179\\
74	0.12228	73.8868784942122\\
74	0.12594	78.1721004288316\\
74	0.1296	82.7639860400373\\
74	0.13326	87.6625353278295\\
74	0.13692	92.867748292208\\
74	0.14058	98.3796249331729\\
74	0.14424	104.198165250724\\
74	0.1479	110.323369244862\\
74	0.15156	116.755236915585\\
74	0.15522	123.493768262896\\
74	0.15888	130.538963286792\\
74	0.16254	137.890821987275\\
74	0.1662	145.549344364345\\
74	0.16986	153.514530418\\
74	0.17352	161.786380148242\\
74	0.17718	170.364893555071\\
74	0.18084	179.250070638485\\
74	0.1845	188.441911398486\\
74	0.18816	197.940415835074\\
74	0.19182	207.745583948248\\
74	0.19548	217.857415738008\\
74	0.19914	228.275911204354\\
74	0.2028	239.001070347287\\
74	0.20646	250.032893166806\\
74	0.21012	261.371379662912\\
74	0.21378	273.016529835604\\
74	0.21744	284.968343684882\\
74	0.2211	297.226821210746\\
74	0.22476	309.791962413197\\
74	0.22842	322.663767292235\\
74	0.23208	335.842235847858\\
74	0.23574	349.327368080068\\
74	0.2394	363.119163988865\\
74	0.24306	377.217623574247\\
74	0.24672	391.622746836216\\
74	0.25038	406.334533774772\\
74	0.25404	421.352984389914\\
74	0.2577	436.678098681642\\
74	0.26136	452.309876649956\\
74	0.26502	468.248318294857\\
74	0.26868	484.493423616344\\
74	0.27234	501.045192614418\\
74	0.276	517.903625289077\\
};
\end{axis}

\begin{axis}[%
width=4.527496cm,
height=3.870968cm,
at={(0cm,10.752688cm)},
scale only axis,
xmin=56,
xmax=74,
tick align=outside,
xlabel={$L_{cut}$},
xmajorgrids,
ymin=0.093,
ymax=0.276,
ylabel={$D_{rlx}$},
ymajorgrids,
zmin=-0.468526216891017,
zmax=12.0724461325189,
zlabel={$x_1,x_4$},
zmajorgrids,
view={-140}{50},
legend style={at={(1.03,1)},anchor=north west,legend cell align=left,align=left,draw=white!15!black}
]
\addplot3[only marks,mark=*,mark options={},mark size=1.5000pt,color=mycolor1] plot table[row sep=crcr,]{%
74	0.123	0.616147082677301\\
72	0.113	0.756244217181634\\
61	0.095	0.315136730386088\\
56	0.093	0.479225561787365\\
};
\addplot3[only marks,mark=*,mark options={},mark size=1.5000pt,color=mycolor2] plot table[row sep=crcr,]{%
67	0.276	8.54895561081077\\
66	0.255	7.8265872893112\\
62	0.209	2.65659919911432\\
57	0.193	1.15263735041756\\
};
\addplot3[only marks,mark=*,mark options={},mark size=1.5000pt,color=black] plot table[row sep=crcr,]{%
69	0.104	0.308894496859802\\
};
\addplot3[only marks,mark=*,mark options={},mark size=1.5000pt,color=black] plot table[row sep=crcr,]{%
64	0.23	4.75084671442933\\
};

\addplot3[%
surf,
opacity=0.7,
shader=interp,
colormap={mymap}{[1pt] rgb(0pt)=(0.0901961,0.239216,0.0745098); rgb(1pt)=(0.0945149,0.242058,0.0739522); rgb(2pt)=(0.0988592,0.244894,0.0733566); rgb(3pt)=(0.103229,0.247724,0.0727241); rgb(4pt)=(0.107623,0.250549,0.0720557); rgb(5pt)=(0.112043,0.253367,0.0713525); rgb(6pt)=(0.116487,0.25618,0.0706154); rgb(7pt)=(0.120956,0.258986,0.0698456); rgb(8pt)=(0.125449,0.261787,0.0690441); rgb(9pt)=(0.129967,0.264581,0.0682118); rgb(10pt)=(0.134508,0.26737,0.06735); rgb(11pt)=(0.139074,0.270152,0.0664596); rgb(12pt)=(0.143663,0.272929,0.0655416); rgb(13pt)=(0.148275,0.275699,0.0645971); rgb(14pt)=(0.152911,0.278463,0.0636271); rgb(15pt)=(0.15757,0.281221,0.0626328); rgb(16pt)=(0.162252,0.283973,0.0616151); rgb(17pt)=(0.166957,0.286719,0.060575); rgb(18pt)=(0.171685,0.289458,0.0595136); rgb(19pt)=(0.176434,0.292191,0.0584321); rgb(20pt)=(0.181207,0.294918,0.0573313); rgb(21pt)=(0.186001,0.297639,0.0562123); rgb(22pt)=(0.190817,0.300353,0.0550763); rgb(23pt)=(0.195655,0.303061,0.0539242); rgb(24pt)=(0.200514,0.305763,0.052757); rgb(25pt)=(0.205395,0.308459,0.0515759); rgb(26pt)=(0.210296,0.311149,0.0503624); rgb(27pt)=(0.215212,0.313846,0.0490067); rgb(28pt)=(0.220142,0.316548,0.0475043); rgb(29pt)=(0.22509,0.319254,0.0458704); rgb(30pt)=(0.230056,0.321962,0.0441205); rgb(31pt)=(0.235042,0.324671,0.04227); rgb(32pt)=(0.240048,0.327379,0.0403343); rgb(33pt)=(0.245078,0.330085,0.0383287); rgb(34pt)=(0.250131,0.332786,0.0362688); rgb(35pt)=(0.25521,0.335482,0.0341698); rgb(36pt)=(0.260317,0.33817,0.0320472); rgb(37pt)=(0.265451,0.340849,0.0299163); rgb(38pt)=(0.270616,0.343517,0.0277927); rgb(39pt)=(0.275813,0.346172,0.0256916); rgb(40pt)=(0.281043,0.348814,0.0236284); rgb(41pt)=(0.286307,0.35144,0.0216186); rgb(42pt)=(0.291607,0.354048,0.0196776); rgb(43pt)=(0.296945,0.356637,0.0178207); rgb(44pt)=(0.302322,0.359206,0.0160634); rgb(45pt)=(0.307739,0.361753,0.0144211); rgb(46pt)=(0.313198,0.364275,0.0129091); rgb(47pt)=(0.318701,0.366772,0.0115428); rgb(48pt)=(0.324249,0.369242,0.0103377); rgb(49pt)=(0.329843,0.371682,0.00930909); rgb(50pt)=(0.335485,0.374093,0.00847245); rgb(51pt)=(0.341176,0.376471,0.00784314); rgb(52pt)=(0.346925,0.378826,0.00732741); rgb(53pt)=(0.352735,0.381168,0.00682184); rgb(54pt)=(0.358605,0.383497,0.00632729); rgb(55pt)=(0.364532,0.385812,0.00584464); rgb(56pt)=(0.370516,0.388113,0.00537476); rgb(57pt)=(0.376552,0.390399,0.00491852); rgb(58pt)=(0.38264,0.39267,0.00447681); rgb(59pt)=(0.388777,0.394925,0.00405048); rgb(60pt)=(0.394962,0.397164,0.00364042); rgb(61pt)=(0.401191,0.399386,0.00324749); rgb(62pt)=(0.407464,0.401592,0.00287258); rgb(63pt)=(0.413777,0.40378,0.00251655); rgb(64pt)=(0.420129,0.40595,0.00218028); rgb(65pt)=(0.426518,0.408102,0.00186463); rgb(66pt)=(0.432942,0.410234,0.00157049); rgb(67pt)=(0.439399,0.412348,0.00129873); rgb(68pt)=(0.445885,0.414441,0.00105022); rgb(69pt)=(0.452401,0.416515,0.000825833); rgb(70pt)=(0.458942,0.418567,0.000626441); rgb(71pt)=(0.465508,0.420599,0.00045292); rgb(72pt)=(0.472096,0.422609,0.000306141); rgb(73pt)=(0.478704,0.424596,0.000186979); rgb(74pt)=(0.485331,0.426562,9.63073e-05); rgb(75pt)=(0.491973,0.428504,3.49981e-05); rgb(76pt)=(0.498628,0.430422,3.92506e-06); rgb(77pt)=(0.505323,0.432315,0); rgb(78pt)=(0.512206,0.434168,0); rgb(79pt)=(0.519282,0.435983,0); rgb(80pt)=(0.526529,0.437764,0); rgb(81pt)=(0.533922,0.439512,0); rgb(82pt)=(0.54144,0.441232,0); rgb(83pt)=(0.549059,0.442927,0); rgb(84pt)=(0.556756,0.444599,0); rgb(85pt)=(0.564508,0.446252,0); rgb(86pt)=(0.572292,0.447889,0); rgb(87pt)=(0.580084,0.449514,0); rgb(88pt)=(0.587863,0.451129,0); rgb(89pt)=(0.595604,0.452737,0); rgb(90pt)=(0.603284,0.454343,0); rgb(91pt)=(0.610882,0.455948,0); rgb(92pt)=(0.618373,0.457556,0); rgb(93pt)=(0.625734,0.459171,0); rgb(94pt)=(0.632943,0.460795,0); rgb(95pt)=(0.639976,0.462432,0); rgb(96pt)=(0.64681,0.464084,0); rgb(97pt)=(0.653423,0.465756,0); rgb(98pt)=(0.659791,0.46745,0); rgb(99pt)=(0.665891,0.469169,0); rgb(100pt)=(0.6717,0.470916,0); rgb(101pt)=(0.677195,0.472696,0); rgb(102pt)=(0.682353,0.47451,0); rgb(103pt)=(0.687242,0.476355,0); rgb(104pt)=(0.691952,0.478225,0); rgb(105pt)=(0.696497,0.480118,0); rgb(106pt)=(0.700887,0.482033,0); rgb(107pt)=(0.705134,0.483968,0); rgb(108pt)=(0.709251,0.485921,0); rgb(109pt)=(0.713249,0.487891,0); rgb(110pt)=(0.71714,0.489876,0); rgb(111pt)=(0.720936,0.491875,0); rgb(112pt)=(0.724649,0.493887,0); rgb(113pt)=(0.72829,0.495909,0); rgb(114pt)=(0.731872,0.49794,0); rgb(115pt)=(0.735406,0.499979,0); rgb(116pt)=(0.738904,0.502025,0); rgb(117pt)=(0.742378,0.504075,0); rgb(118pt)=(0.74584,0.506128,0); rgb(119pt)=(0.749302,0.508182,0); rgb(120pt)=(0.752775,0.510237,0); rgb(121pt)=(0.756272,0.51229,0); rgb(122pt)=(0.759804,0.514339,0); rgb(123pt)=(0.763384,0.516385,0); rgb(124pt)=(0.767022,0.518424,0); rgb(125pt)=(0.770731,0.520455,0); rgb(126pt)=(0.774523,0.522478,0); rgb(127pt)=(0.77841,0.524489,0); rgb(128pt)=(0.782391,0.526491,0); rgb(129pt)=(0.786402,0.528496,0); rgb(130pt)=(0.790431,0.530506,0); rgb(131pt)=(0.794478,0.532521,0); rgb(132pt)=(0.798541,0.534539,0); rgb(133pt)=(0.802619,0.53656,0); rgb(134pt)=(0.806712,0.538584,0); rgb(135pt)=(0.81082,0.540609,0); rgb(136pt)=(0.81494,0.542635,0); rgb(137pt)=(0.819074,0.54466,0); rgb(138pt)=(0.823219,0.546686,0); rgb(139pt)=(0.827374,0.548709,0); rgb(140pt)=(0.831541,0.55073,0); rgb(141pt)=(0.835716,0.552749,0); rgb(142pt)=(0.8399,0.554763,0); rgb(143pt)=(0.844092,0.556774,0); rgb(144pt)=(0.848292,0.558779,0); rgb(145pt)=(0.852497,0.560778,0); rgb(146pt)=(0.856708,0.562771,0); rgb(147pt)=(0.860924,0.564756,0); rgb(148pt)=(0.865143,0.566733,0); rgb(149pt)=(0.869366,0.568701,0); rgb(150pt)=(0.873592,0.57066,0); rgb(151pt)=(0.877819,0.572608,0); rgb(152pt)=(0.882047,0.574545,0); rgb(153pt)=(0.886275,0.576471,0); rgb(154pt)=(0.890659,0.578362,0); rgb(155pt)=(0.895333,0.580203,0); rgb(156pt)=(0.900258,0.581999,0); rgb(157pt)=(0.905397,0.583755,0); rgb(158pt)=(0.910711,0.585479,0); rgb(159pt)=(0.916164,0.587176,0); rgb(160pt)=(0.921717,0.588852,0); rgb(161pt)=(0.927333,0.590513,0); rgb(162pt)=(0.932974,0.592166,0); rgb(163pt)=(0.938602,0.593815,0); rgb(164pt)=(0.94418,0.595468,0); rgb(165pt)=(0.949669,0.59713,0); rgb(166pt)=(0.955033,0.598808,0); rgb(167pt)=(0.960233,0.600507,0); rgb(168pt)=(0.965232,0.602233,0); rgb(169pt)=(0.969992,0.603992,0); rgb(170pt)=(0.974475,0.605791,0); rgb(171pt)=(0.978643,0.607636,0); rgb(172pt)=(0.98246,0.609532,0); rgb(173pt)=(0.985886,0.611486,0); rgb(174pt)=(0.988885,0.613503,0); rgb(175pt)=(0.991419,0.61559,0); rgb(176pt)=(0.99345,0.617753,0); rgb(177pt)=(0.99494,0.619997,0); rgb(178pt)=(0.995851,0.622329,0); rgb(179pt)=(0.996226,0.624763,0); rgb(180pt)=(0.996512,0.627352,0); rgb(181pt)=(0.996788,0.630095,0); rgb(182pt)=(0.997053,0.632982,0); rgb(183pt)=(0.997308,0.636004,0); rgb(184pt)=(0.997552,0.639152,0); rgb(185pt)=(0.997785,0.642416,0); rgb(186pt)=(0.998006,0.645786,0); rgb(187pt)=(0.998217,0.649253,0); rgb(188pt)=(0.998416,0.652807,0); rgb(189pt)=(0.998605,0.656439,0); rgb(190pt)=(0.998781,0.660138,0); rgb(191pt)=(0.998946,0.663897,0); rgb(192pt)=(0.9991,0.667704,0); rgb(193pt)=(0.999242,0.67155,0); rgb(194pt)=(0.999372,0.675427,0); rgb(195pt)=(0.99949,0.679323,0); rgb(196pt)=(0.999596,0.68323,0); rgb(197pt)=(0.99969,0.687139,0); rgb(198pt)=(0.999771,0.691039,0); rgb(199pt)=(0.999841,0.694921,0); rgb(200pt)=(0.999898,0.698775,0); rgb(201pt)=(0.999942,0.702592,0); rgb(202pt)=(0.999974,0.706363,0); rgb(203pt)=(0.999994,0.710077,0); rgb(204pt)=(1,0.713725,0); rgb(205pt)=(1,0.717341,0); rgb(206pt)=(1,0.720963,0); rgb(207pt)=(1,0.724591,0); rgb(208pt)=(1,0.728226,0); rgb(209pt)=(1,0.731867,0); rgb(210pt)=(1,0.735514,0); rgb(211pt)=(1,0.739167,0); rgb(212pt)=(1,0.742827,0); rgb(213pt)=(1,0.746493,0); rgb(214pt)=(1,0.750165,0); rgb(215pt)=(1,0.753843,0); rgb(216pt)=(1,0.757527,0); rgb(217pt)=(1,0.761217,0); rgb(218pt)=(1,0.764913,0); rgb(219pt)=(1,0.768615,0); rgb(220pt)=(1,0.772324,0); rgb(221pt)=(1,0.776038,0); rgb(222pt)=(1,0.779758,0); rgb(223pt)=(1,0.783484,0); rgb(224pt)=(1,0.787215,0); rgb(225pt)=(1,0.790953,0); rgb(226pt)=(1,0.794696,0); rgb(227pt)=(1,0.798445,0); rgb(228pt)=(1,0.8022,0); rgb(229pt)=(1,0.805961,0); rgb(230pt)=(1,0.809727,0); rgb(231pt)=(1,0.8135,0); rgb(232pt)=(1,0.817278,0); rgb(233pt)=(1,0.821063,0); rgb(234pt)=(1,0.824854,0); rgb(235pt)=(1,0.828652,0); rgb(236pt)=(1,0.832455,0); rgb(237pt)=(1,0.836265,0); rgb(238pt)=(1,0.840081,0); rgb(239pt)=(1,0.843903,0); rgb(240pt)=(1,0.847732,0); rgb(241pt)=(1,0.851566,0); rgb(242pt)=(1,0.855406,0); rgb(243pt)=(1,0.859253,0); rgb(244pt)=(1,0.863106,0); rgb(245pt)=(1,0.866964,0); rgb(246pt)=(1,0.870829,0); rgb(247pt)=(1,0.8747,0); rgb(248pt)=(1,0.878577,0); rgb(249pt)=(1,0.88246,0); rgb(250pt)=(1,0.886349,0); rgb(251pt)=(1,0.890243,0); rgb(252pt)=(1,0.894144,0); rgb(253pt)=(1,0.898051,0); rgb(254pt)=(1,0.901964,0); rgb(255pt)=(1,0.905882,0)},
mesh/rows=49]
table[row sep=crcr,header=false] {%
%
56	0.093	0.414666213324685\\
56	0.09666	0.37013123446904\\
56	0.10032	0.32991630850448\\
56	0.10398	0.294021435431006\\
56	0.10764	0.262446615248612\\
56	0.1113	0.235191847957301\\
56	0.11496	0.212257133557077\\
56	0.11862	0.193642472047932\\
56	0.12228	0.179347863429871\\
56	0.12594	0.169373307702895\\
56	0.1296	0.163718804867\\
56	0.13326	0.162384354922187\\
56	0.13692	0.165369957868462\\
56	0.14058	0.172675613705816\\
56	0.14424	0.184301322434254\\
56	0.1479	0.200247084053774\\
56	0.15156	0.22051289856438\\
56	0.15522	0.245098765966067\\
56	0.15888	0.274004686258834\\
56	0.16254	0.307230659442694\\
56	0.1662	0.34477668551763\\
56	0.16986	0.386642764483645\\
56	0.17352	0.432828896340753\\
56	0.17718	0.483335081088939\\
56	0.18084	0.538161318728203\\
56	0.1845	0.597307609258562\\
56	0.18816	0.660773952679993\\
56	0.19182	0.728560348992515\\
56	0.19548	0.800666798196115\\
56	0.19914	0.877093300290799\\
56	0.2028	0.957839855276571\\
56	0.20646	1.04290646315342\\
56	0.21012	1.13229312392135\\
56	0.21378	1.22599983758037\\
56	0.21744	1.32402660413047\\
56	0.2211	1.42637342357165\\
56	0.22476	1.53304029590393\\
56	0.22842	1.64402722112727\\
56	0.23208	1.7593341992417\\
56	0.23574	1.87896123024722\\
56	0.2394	2.00290831414382\\
56	0.24306	2.1311754509315\\
56	0.24672	2.26376264061027\\
56	0.25038	2.40066988318012\\
56	0.25404	2.54189717864105\\
56	0.2577	2.68744452699306\\
56	0.26136	2.83731192823616\\
56	0.26502	2.99149938237034\\
56	0.26868	3.1500068893956\\
56	0.27234	3.31283444931195\\
56	0.276	3.47998206211939\\
56.375	0.093	0.420828723367943\\
56.375	0.09666	0.380241934720889\\
56.375	0.10032	0.343975198964918\\
56.375	0.10398	0.312028516100034\\
56.375	0.10764	0.284401886126226\\
56.375	0.1113	0.261095309043508\\
56.375	0.11496	0.24210878485187\\
56.375	0.11862	0.227442313551318\\
56.375	0.12228	0.217095895141846\\
56.375	0.12594	0.211069529623457\\
56.375	0.1296	0.209363216996155\\
56.375	0.13326	0.211976957259932\\
56.375	0.13692	0.218910750414792\\
56.375	0.14058	0.230164596460737\\
56.375	0.14424	0.245738495397767\\
56.375	0.1479	0.265632447225877\\
56.375	0.15156	0.289846451945071\\
56.375	0.15522	0.318380509555348\\
56.375	0.15888	0.351234620056704\\
56.375	0.16254	0.388408783449154\\
56.375	0.1662	0.429902999732676\\
56.375	0.16986	0.475717268907283\\
56.375	0.17352	0.525851590972982\\
56.375	0.17718	0.580305965929753\\
56.375	0.18084	0.63908039377761\\
56.375	0.1845	0.702174874516559\\
56.375	0.18816	0.769589408146577\\
56.375	0.19182	0.841323994667691\\
56.375	0.19548	0.917378634079881\\
56.375	0.19914	0.997753326383155\\
56.375	0.2028	1.08244807157752\\
56.375	0.20646	1.17146286966295\\
56.375	0.21012	1.26479772063948\\
56.375	0.21378	1.36245262450709\\
56.375	0.21744	1.46442758126577\\
56.375	0.2211	1.57072259091555\\
56.375	0.22476	1.68133765345641\\
56.375	0.22842	1.79627276888834\\
56.375	0.23208	1.91552793721137\\
56.375	0.23574	2.03910315842547\\
56.375	0.2394	2.16699843253066\\
56.375	0.24306	2.29921375952694\\
56.375	0.24672	2.43574913941429\\
56.375	0.25038	2.57660457219273\\
56.375	0.25404	2.72178005786225\\
56.375	0.2577	2.87127559642285\\
56.375	0.26136	3.02509118787454\\
56.375	0.26502	3.18322683221731\\
56.375	0.26868	3.34568252945116\\
56.375	0.27234	3.5124582795761\\
56.375	0.276	3.68355408259212\\
56.75	0.093	0.425946026928677\\
56.75	0.09666	0.389307428490212\\
56.75	0.10032	0.356988882942833\\
56.75	0.10398	0.328990390286535\\
56.75	0.10764	0.30531195052132\\
56.75	0.1113	0.285953563647189\\
56.75	0.11496	0.270915229664144\\
56.75	0.11862	0.260196948572181\\
56.75	0.12228	0.253798720371296\\
56.75	0.12594	0.2517205450615\\
56.75	0.1296	0.253962422642787\\
56.75	0.13326	0.26052435311515\\
56.75	0.13692	0.271406336478603\\
56.75	0.14058	0.286608372733137\\
56.75	0.14424	0.306130461878757\\
56.75	0.1479	0.329972603915457\\
56.75	0.15156	0.358134798843239\\
56.75	0.15522	0.390617046662109\\
56.75	0.15888	0.427419347372055\\
56.75	0.16254	0.468541700973094\\
56.75	0.1662	0.513984107465205\\
56.75	0.16986	0.563746566848403\\
56.75	0.17352	0.617829079122691\\
56.75	0.17718	0.676231644288052\\
56.75	0.18084	0.738954262344499\\
56.75	0.1845	0.805996933292037\\
56.75	0.18816	0.877359657130648\\
56.75	0.19182	0.953042433860348\\
56.75	0.19548	1.03304526348113\\
56.75	0.19914	1.11736814599299\\
56.75	0.2028	1.20601108139594\\
56.75	0.20646	1.29897406968997\\
56.75	0.21012	1.39625711087509\\
56.75	0.21378	1.49786020495128\\
56.75	0.21744	1.60378335191856\\
56.75	0.2211	1.71402655177693\\
56.75	0.22476	1.82858980452637\\
56.75	0.22842	1.9474731101669\\
56.75	0.23208	2.07067646869852\\
56.75	0.23574	2.1981998801212\\
56.75	0.2394	2.33004334443498\\
56.75	0.24306	2.46620686163985\\
56.75	0.24672	2.60669043173579\\
56.75	0.25038	2.75149405472283\\
56.75	0.25404	2.90061773060094\\
56.75	0.2577	3.05406145937013\\
56.75	0.26136	3.2118252410304\\
56.75	0.26502	3.37390907558176\\
56.75	0.26868	3.5403129630242\\
56.75	0.27234	3.71103690335774\\
56.75	0.276	3.88608089658235\\
57.125	0.093	0.430018124006879\\
57.125	0.09666	0.397327715777004\\
57.125	0.10032	0.368957360438213\\
57.125	0.10398	0.344907057990508\\
57.125	0.10764	0.325176808433883\\
57.125	0.1113	0.309766611768341\\
57.125	0.11496	0.298676467993885\\
57.125	0.11862	0.291906377110509\\
57.125	0.12228	0.289456339118217\\
57.125	0.12594	0.29132635401701\\
57.125	0.1296	0.297516421806883\\
57.125	0.13326	0.308026542487839\\
57.125	0.13692	0.322856716059883\\
57.125	0.14058	0.342006942523006\\
57.125	0.14424	0.365477221877216\\
57.125	0.1479	0.393267554122501\\
57.125	0.15156	0.425377939258875\\
57.125	0.15522	0.461808377286338\\
57.125	0.15888	0.502558868204874\\
57.125	0.16254	0.547629412014499\\
57.125	0.1662	0.597020008715203\\
57.125	0.16986	0.650730658306991\\
57.125	0.17352	0.708761360789865\\
57.125	0.17718	0.771112116163819\\
57.125	0.18084	0.837782924428855\\
57.125	0.1845	0.908773785584979\\
57.125	0.18816	0.98408469963218\\
57.125	0.19182	1.06371566657047\\
57.125	0.19548	1.14766668639984\\
57.125	0.19914	1.23593775912029\\
57.125	0.2028	1.32852888473183\\
57.125	0.20646	1.42544006323446\\
57.125	0.21012	1.52667129462816\\
57.125	0.21378	1.63222257891294\\
57.125	0.21744	1.74209391608881\\
57.125	0.2211	1.85628530615577\\
57.125	0.22476	1.9747967491138\\
57.125	0.22842	2.09762824496292\\
57.125	0.23208	2.22477979370312\\
57.125	0.23574	2.35625139533441\\
57.125	0.2394	2.49204304985677\\
57.125	0.24306	2.63215475727023\\
57.125	0.24672	2.77658651757476\\
57.125	0.25038	2.92533833077038\\
57.125	0.25404	3.07841019685708\\
57.125	0.2577	3.23580211583486\\
57.125	0.26136	3.39751408770373\\
57.125	0.26502	3.56354611246368\\
57.125	0.26868	3.73389819011471\\
57.125	0.27234	3.90857032065683\\
57.125	0.276	4.08756250409002\\
57.5	0.093	0.433045014602555\\
57.5	0.09666	0.40430279658127\\
57.5	0.10032	0.37988063145107\\
57.5	0.10398	0.35977851921195\\
57.5	0.10764	0.343996459863915\\
57.5	0.1113	0.332534453406966\\
57.5	0.11496	0.325392499841097\\
57.5	0.11862	0.322570599166314\\
57.5	0.12228	0.324068751382611\\
57.5	0.12594	0.32988695648999\\
57.5	0.1296	0.340025214488457\\
57.5	0.13326	0.354483525378003\\
57.5	0.13692	0.373261889158635\\
57.5	0.14058	0.396360305830349\\
57.5	0.14424	0.423778775393144\\
57.5	0.1479	0.455517297847023\\
57.5	0.15156	0.491575873191988\\
57.5	0.15522	0.531954501428037\\
57.5	0.15888	0.576653182555162\\
57.5	0.16254	0.625671916573377\\
57.5	0.1662	0.679010703482671\\
57.5	0.16986	0.736669543283048\\
57.5	0.17352	0.798648435974512\\
57.5	0.17718	0.864947381557055\\
57.5	0.18084	0.935566380030681\\
57.5	0.1845	1.0105054313954\\
57.5	0.18816	1.08976453565119\\
57.5	0.19182	1.17334369279807\\
57.5	0.19548	1.26124290283603\\
57.5	0.19914	1.35346216576507\\
57.5	0.2028	1.4500014815852\\
57.5	0.20646	1.55086085029641\\
57.5	0.21012	1.65604027189871\\
57.5	0.21378	1.76553974639208\\
57.5	0.21744	1.87935927377653\\
57.5	0.2211	1.99749885405208\\
57.5	0.22476	2.1199584872187\\
57.5	0.22842	2.24673817327641\\
57.5	0.23208	2.37783791222521\\
57.5	0.23574	2.51325770406508\\
57.5	0.2394	2.65299754879603\\
57.5	0.24306	2.79705744641808\\
57.5	0.24672	2.9454373969312\\
57.5	0.25038	3.09813740033541\\
57.5	0.25404	3.2551574566307\\
57.5	0.2577	3.41649756581707\\
57.5	0.26136	3.58215772789453\\
57.5	0.26502	3.75213794286307\\
57.5	0.26868	3.92643821072269\\
57.5	0.27234	4.1050585314734\\
57.5	0.276	4.28799890511518\\
57.875	0.093	0.435026698715704\\
57.875	0.09666	0.410232670903006\\
57.875	0.10032	0.389758695981394\\
57.875	0.10398	0.373604773950868\\
57.875	0.10764	0.361770904811422\\
57.875	0.1113	0.35425708856306\\
57.875	0.11496	0.351063325205784\\
57.875	0.11862	0.35218961473959\\
57.875	0.12228	0.357635957164473\\
57.875	0.12594	0.367402352480446\\
57.875	0.1296	0.381488800687502\\
57.875	0.13326	0.399895301785637\\
57.875	0.13692	0.422621855774856\\
57.875	0.14058	0.449668462655159\\
57.875	0.14424	0.481035122426548\\
57.875	0.1479	0.516721835089016\\
57.875	0.15156	0.556728600642571\\
57.875	0.15522	0.601055419087206\\
57.875	0.15888	0.649702290422921\\
57.875	0.16254	0.702669214649728\\
57.875	0.1662	0.759956191767612\\
57.875	0.16986	0.821563221776575\\
57.875	0.17352	0.887490304676632\\
57.875	0.17718	0.957737440467765\\
57.875	0.18084	1.03230462914998\\
57.875	0.1845	1.11119187072328\\
57.875	0.18816	1.19439916518766\\
57.875	0.19182	1.28192651254314\\
57.875	0.19548	1.37377391278968\\
57.875	0.19914	1.46994136592732\\
57.875	0.2028	1.57042887195604\\
57.875	0.20646	1.67523643087583\\
57.875	0.21012	1.78436404268672\\
57.875	0.21378	1.89781170738868\\
57.875	0.21744	2.01557942498173\\
57.875	0.2211	2.13766719546587\\
57.875	0.22476	2.26407501884108\\
57.875	0.22842	2.39480289510737\\
57.875	0.23208	2.52985082426477\\
57.875	0.23574	2.66921880631322\\
57.875	0.2394	2.81290684125276\\
57.875	0.24306	2.9609149290834\\
57.875	0.24672	3.11324306980512\\
57.875	0.25038	3.26989126341792\\
57.875	0.25404	3.43085950992179\\
57.875	0.2577	3.59614780931676\\
57.875	0.26136	3.7657561616028\\
57.875	0.26502	3.93968456677992\\
57.875	0.26868	4.11793302484814\\
57.875	0.27234	4.30050153580744\\
57.875	0.276	4.48739009965782\\
58.25	0.093	0.435963176346323\\
58.25	0.09666	0.415117338742217\\
58.25	0.10032	0.398591554029194\\
58.25	0.10398	0.386385822207258\\
58.25	0.10764	0.378500143276398\\
58.25	0.1113	0.374934517236629\\
58.25	0.11496	0.375688944087942\\
58.25	0.11862	0.380763423830334\\
58.25	0.12228	0.390157956463811\\
58.25	0.12594	0.403872541988373\\
58.25	0.1296	0.421907180404018\\
58.25	0.13326	0.44426187171074\\
58.25	0.13692	0.470936615908552\\
58.25	0.14058	0.501931412997445\\
58.25	0.14424	0.537246262977423\\
58.25	0.1479	0.576881165848481\\
58.25	0.15156	0.620836121610622\\
58.25	0.15522	0.669111130263851\\
58.25	0.15888	0.721706191808156\\
58.25	0.16254	0.778621306243553\\
58.25	0.1662	0.839856473570027\\
58.25	0.16986	0.905411693787583\\
58.25	0.17352	0.975286966896226\\
58.25	0.17718	1.04948229289595\\
58.25	0.18084	1.12799767178675\\
58.25	0.1845	1.21083310356865\\
58.25	0.18816	1.29798858824162\\
58.25	0.19182	1.38946412580568\\
58.25	0.19548	1.48525971626082\\
58.25	0.19914	1.58537535960703\\
58.25	0.2028	1.68981105584434\\
58.25	0.20646	1.79856680497274\\
58.25	0.21012	1.91164260699221\\
58.25	0.21378	2.02903846190277\\
58.25	0.21744	2.1507543697044\\
58.25	0.2211	2.27679033039712\\
58.25	0.22476	2.40714634398093\\
58.25	0.22842	2.54182241045581\\
58.25	0.23208	2.68081852982179\\
58.25	0.23574	2.82413470207885\\
58.25	0.2394	2.97177092722698\\
58.25	0.24306	3.1237272052662\\
58.25	0.24672	3.2800035361965\\
58.25	0.25038	3.4405999200179\\
58.25	0.25404	3.60551635673036\\
58.25	0.2577	3.77475284633391\\
58.25	0.26136	3.94830938882855\\
58.25	0.26502	4.12618598421427\\
58.25	0.26868	4.30838263249107\\
58.25	0.27234	4.49489933365896\\
58.25	0.276	4.68573608771792\\
58.625	0.093	0.435854447494409\\
58.625	0.09666	0.418956800098891\\
58.625	0.10032	0.406379205594462\\
58.625	0.10398	0.398121663981112\\
58.625	0.10764	0.394184175258845\\
58.625	0.1113	0.394566739427665\\
58.625	0.11496	0.399269356487565\\
58.625	0.11862	0.40829202643855\\
58.625	0.12228	0.421634749280616\\
58.625	0.12594	0.439297525013768\\
58.625	0.1296	0.461280353638\\
58.625	0.13326	0.487583235153314\\
58.625	0.13692	0.518206169559716\\
58.625	0.14058	0.553149156857198\\
58.625	0.14424	0.592412197045766\\
58.625	0.1479	0.63599529012541\\
58.625	0.15156	0.683898436096144\\
58.625	0.15522	0.736121634957962\\
58.625	0.15888	0.792664886710856\\
58.625	0.16254	0.853528191354843\\
58.625	0.1662	0.918711548889906\\
58.625	0.16986	0.988214959316048\\
58.625	0.17352	1.06203842263328\\
58.625	0.17718	1.1401819388416\\
58.625	0.18084	1.22264550794099\\
58.625	0.1845	1.30942912993147\\
58.625	0.18816	1.40053280481303\\
58.625	0.19182	1.49595653258569\\
58.625	0.19548	1.59570031324941\\
58.625	0.19914	1.69976414680422\\
58.625	0.2028	1.80814803325012\\
58.625	0.20646	1.9208519725871\\
58.625	0.21012	2.03787596481516\\
58.625	0.21378	2.1592200099343\\
58.625	0.21744	2.28488410794453\\
58.625	0.2211	2.41486825884585\\
58.625	0.22476	2.54917246263824\\
58.625	0.22842	2.68779671932172\\
58.625	0.23208	2.83074102889628\\
58.625	0.23574	2.97800539136192\\
58.625	0.2394	3.12958980671865\\
58.625	0.24306	3.28549427496646\\
58.625	0.24672	3.44571879610536\\
58.625	0.25038	3.61026337013534\\
58.625	0.25404	3.77912799705639\\
58.625	0.2577	3.95231267686853\\
58.625	0.26136	4.12981740957176\\
58.625	0.26502	4.31164219516606\\
58.625	0.26868	4.49778703365145\\
58.625	0.27234	4.68825192502793\\
58.625	0.276	4.88303686929549\\
59	0.093	0.434700512159973\\
59	0.09666	0.421751054973047\\
59	0.10032	0.413121650677204\\
59	0.10398	0.408812299272446\\
59	0.10764	0.408823000758769\\
59	0.1113	0.413153755136175\\
59	0.11496	0.421804562404668\\
59	0.11862	0.434775422564243\\
59	0.12228	0.452066335614895\\
59	0.12594	0.473677301556637\\
59	0.1296	0.499608320389461\\
59	0.13326	0.529859392113366\\
59	0.13692	0.564430516728357\\
59	0.14058	0.603321694234429\\
59	0.14424	0.646532924631587\\
59	0.1479	0.69406420791982\\
59	0.15156	0.745915544099144\\
59	0.15522	0.802086933169549\\
59	0.15888	0.862578375131033\\
59	0.16254	0.927389869983609\\
59	0.1662	0.996521417727262\\
59	0.16986	1.06997301836199\\
59	0.17352	1.14774467188782\\
59	0.17718	1.22983637830472\\
59	0.18084	1.31624813761271\\
59	0.1845	1.40697994981178\\
59	0.18816	1.50203181490193\\
59	0.19182	1.60140373288317\\
59	0.19548	1.70509570375549\\
59	0.19914	1.81310772751889\\
59	0.2028	1.92543980417338\\
59	0.20646	2.04209193371894\\
59	0.21012	2.1630641161556\\
59	0.21378	2.28835635148333\\
59	0.21744	2.41796863970215\\
59	0.2211	2.55190098081205\\
59	0.22476	2.69015337481303\\
59	0.22842	2.8327258217051\\
59	0.23208	2.97961832148825\\
59	0.23574	3.13083087416248\\
59	0.2394	3.2863634797278\\
59	0.24306	3.4462161381842\\
59	0.24672	3.61038884953168\\
59	0.25038	3.77888161377025\\
59	0.25404	3.9516944308999\\
59	0.2577	4.12882730092063\\
59	0.26136	4.31028022383245\\
59	0.26502	4.49605319963534\\
59	0.26868	4.68614622832932\\
59	0.27234	4.88055930991439\\
59	0.276	5.07929244439054\\
59.375	0.093	0.432501370343004\\
59.375	0.09666	0.423500103364665\\
59.375	0.10032	0.418818889277415\\
59.375	0.10398	0.418457728081247\\
59.375	0.10764	0.422416619776156\\
59.375	0.1113	0.430695564362156\\
59.375	0.11496	0.443294561839238\\
59.375	0.11862	0.460213612207399\\
59.375	0.12228	0.481452715466644\\
59.375	0.12594	0.507011871616975\\
59.375	0.1296	0.53689108065839\\
59.375	0.13326	0.571090342590884\\
59.375	0.13692	0.609609657414461\\
59.375	0.14058	0.652449025129122\\
59.375	0.14424	0.699608445734869\\
59.375	0.1479	0.751087919231696\\
59.375	0.15156	0.806887445619609\\
59.375	0.15522	0.867007024898606\\
59.375	0.15888	0.93144665706868\\
59.375	0.16254	1.00020634212984\\
59.375	0.1662	1.07328608008208\\
59.375	0.16986	1.15068587092541\\
59.375	0.17352	1.23240571465982\\
59.375	0.17718	1.31844561128532\\
59.375	0.18084	1.40880556080189\\
59.375	0.1845	1.50348556320955\\
59.375	0.18816	1.60248561850829\\
59.375	0.19182	1.70580572669812\\
59.375	0.19548	1.81344588777903\\
59.375	0.19914	1.92540610175102\\
59.375	0.2028	2.0416863686141\\
59.375	0.20646	2.16228668836825\\
59.375	0.21012	2.2872070610135\\
59.375	0.21378	2.41644748654982\\
59.375	0.21744	2.55000796497722\\
59.375	0.2211	2.68788849629572\\
59.375	0.22476	2.83008908050529\\
59.375	0.22842	2.97660971760595\\
59.375	0.23208	3.12745040759769\\
59.375	0.23574	3.28261115048051\\
59.375	0.2394	3.44209194625442\\
59.375	0.24306	3.60589279491941\\
59.375	0.24672	3.77401369647547\\
59.375	0.25038	3.94645465092263\\
59.375	0.25404	4.12321565826087\\
59.375	0.2577	4.30429671849019\\
59.375	0.26136	4.48969783161061\\
59.375	0.26502	4.67941899762209\\
59.375	0.26868	4.87346021652466\\
59.375	0.27234	5.07182148831831\\
59.375	0.276	5.27450281300305\\
59.75	0.093	0.429257022043504\\
59.75	0.09666	0.424203945273756\\
59.75	0.10032	0.423470921395092\\
59.75	0.10398	0.427057950407514\\
59.75	0.10764	0.434965032311016\\
59.75	0.1113	0.447192167105605\\
59.75	0.11496	0.463739354791274\\
59.75	0.11862	0.484606595368028\\
59.75	0.12228	0.509793888835863\\
59.75	0.12594	0.539301235194784\\
59.75	0.1296	0.573128634444784\\
59.75	0.13326	0.611276086585868\\
59.75	0.13692	0.653743591618038\\
59.75	0.14058	0.700531149541289\\
59.75	0.14424	0.751638760355626\\
59.75	0.1479	0.807066424061042\\
59.75	0.15156	0.866814140657545\\
59.75	0.15522	0.930881910145129\\
59.75	0.15888	0.999269732523793\\
59.75	0.16254	1.07197760779355\\
59.75	0.1662	1.14900553595438\\
59.75	0.16986	1.23035351700629\\
59.75	0.17352	1.3160215509493\\
59.75	0.17718	1.40600963778338\\
59.75	0.18084	1.50031777750854\\
59.75	0.1845	1.5989459701248\\
59.75	0.18816	1.70189421563213\\
59.75	0.19182	1.80916251403054\\
59.75	0.19548	1.92075086532004\\
59.75	0.19914	2.03665926950062\\
59.75	0.2028	2.15688772657229\\
59.75	0.20646	2.28143623653503\\
59.75	0.21012	2.41030479938887\\
59.75	0.21378	2.54349341513378\\
59.75	0.21744	2.68100208376977\\
59.75	0.2211	2.82283080529686\\
59.75	0.22476	2.96897957971502\\
59.75	0.22842	3.11944840702427\\
59.75	0.23208	3.2742372872246\\
59.75	0.23574	3.43334622031601\\
59.75	0.2394	3.5967752062985\\
59.75	0.24306	3.76452424517209\\
59.75	0.24672	3.93659333693674\\
59.75	0.25038	4.11298248159249\\
59.75	0.25404	4.29369167913932\\
59.75	0.2577	4.47872092957723\\
59.75	0.26136	4.66807023290622\\
59.75	0.26502	4.8617395891263\\
59.75	0.26868	5.05972899823746\\
59.75	0.27234	5.26203846023971\\
59.75	0.276	5.46866797513303\\
60.125	0.093	0.424967467261482\\
60.125	0.09666	0.423862580700321\\
60.125	0.10032	0.42707774703025\\
60.125	0.10398	0.434612966251262\\
60.125	0.10764	0.446468238363354\\
60.125	0.1113	0.462643563366529\\
60.125	0.11496	0.48313894126079\\
60.125	0.11862	0.507954372046134\\
60.125	0.12228	0.537089855722559\\
60.125	0.12594	0.570545392290065\\
60.125	0.1296	0.608320981748659\\
60.125	0.13326	0.650416624098332\\
60.125	0.13692	0.696832319339092\\
60.125	0.14058	0.747568067470933\\
60.125	0.14424	0.802623868493859\\
60.125	0.1479	0.861999722407865\\
60.125	0.15156	0.925695629212954\\
60.125	0.15522	0.99371158890913\\
60.125	0.15888	1.06604760149638\\
60.125	0.16254	1.14270366697473\\
60.125	0.1662	1.22367978534415\\
60.125	0.16986	1.30897595660465\\
60.125	0.17352	1.39859218075625\\
60.125	0.17718	1.49252845779892\\
60.125	0.18084	1.59078478773267\\
60.125	0.1845	1.69336117055751\\
60.125	0.18816	1.80025760627343\\
60.125	0.19182	1.91147409488044\\
60.125	0.19548	2.02701063637853\\
60.125	0.19914	2.14686723076769\\
60.125	0.2028	2.27104387804795\\
60.125	0.20646	2.39954057821929\\
60.125	0.21012	2.53235733128171\\
60.125	0.21378	2.66949413723522\\
60.125	0.21744	2.8109509960798\\
60.125	0.2211	2.95672790781547\\
60.125	0.22476	3.10682487244223\\
60.125	0.22842	3.26124188996006\\
60.125	0.23208	3.41997896036898\\
60.125	0.23574	3.58303608366898\\
60.125	0.2394	3.75041325986006\\
60.125	0.24306	3.92211048894224\\
60.125	0.24672	4.09812777091549\\
60.125	0.25038	4.27846510577983\\
60.125	0.25404	4.46312249353524\\
60.125	0.2577	4.65209993418174\\
60.125	0.26136	4.84539742771933\\
60.125	0.26502	5.043014974148\\
60.125	0.26868	5.24495257346774\\
60.125	0.27234	5.45121022567858\\
60.125	0.276	5.66178793078049\\
60.5	0.093	0.419632705996928\\
60.5	0.09666	0.422476009644359\\
60.5	0.10032	0.429639366182878\\
60.5	0.10398	0.441122775612476\\
60.5	0.10764	0.456926237933157\\
60.5	0.1113	0.477049753144925\\
60.5	0.11496	0.501493321247776\\
60.5	0.11862	0.530256942241706\\
60.5	0.12228	0.56334061612672\\
60.5	0.12594	0.600744342902821\\
60.5	0.1296	0.642468122570004\\
60.5	0.13326	0.688511955128266\\
60.5	0.13692	0.738875840577616\\
60.5	0.14058	0.793559778918046\\
60.5	0.14424	0.852563770149559\\
60.5	0.1479	0.915887814272154\\
60.5	0.15156	0.983531911285836\\
60.5	0.15522	1.0554960611906\\
60.5	0.15888	1.13178026398644\\
60.5	0.16254	1.21238451967338\\
60.5	0.1662	1.29730882825139\\
60.5	0.16986	1.38655318972048\\
60.5	0.17352	1.48011760408066\\
60.5	0.17718	1.57800207133192\\
60.5	0.18084	1.68020659147427\\
60.5	0.1845	1.7867311645077\\
60.5	0.18816	1.89757579043221\\
60.5	0.19182	2.01274046924781\\
60.5	0.19548	2.13222520095449\\
60.5	0.19914	2.25602998555224\\
60.5	0.2028	2.38415482304109\\
60.5	0.20646	2.51659971342102\\
60.5	0.21012	2.65336465669203\\
60.5	0.21378	2.79444965285412\\
60.5	0.21744	2.93985470190729\\
60.5	0.2211	3.08957980385156\\
60.5	0.22476	3.2436249586869\\
60.5	0.22842	3.40199016641332\\
60.5	0.23208	3.56467542703084\\
60.5	0.23574	3.73168074053943\\
60.5	0.2394	3.9030061069391\\
60.5	0.24306	4.07865152622985\\
60.5	0.24672	4.2586169984117\\
60.5	0.25038	4.44290252348462\\
60.5	0.25404	4.63150810144863\\
60.5	0.2577	4.82443373230372\\
60.5	0.26136	5.0216794160499\\
60.5	0.26502	5.22324515268716\\
60.5	0.26868	5.42913094221549\\
60.5	0.27234	5.63933678463491\\
60.5	0.276	5.85386267994542\\
60.875	0.093	0.413252738249848\\
60.875	0.09666	0.420044232105869\\
60.875	0.10032	0.431155778852974\\
60.875	0.10398	0.446587378491165\\
60.875	0.10764	0.466339031020436\\
60.875	0.1113	0.49041073644079\\
60.875	0.11496	0.518802494752231\\
60.875	0.11862	0.551514305954754\\
60.875	0.12228	0.588546170048358\\
60.875	0.12594	0.629898087033047\\
60.875	0.1296	0.67557005690882\\
60.875	0.13326	0.725562079675669\\
60.875	0.13692	0.779874155333608\\
60.875	0.14058	0.838506283882631\\
60.875	0.14424	0.901458465322734\\
60.875	0.1479	0.968730699653919\\
60.875	0.15156	1.04032298687619\\
60.875	0.15522	1.11623532698955\\
60.875	0.15888	1.19646771999398\\
60.875	0.16254	1.2810201658895\\
60.875	0.1662	1.3698926646761\\
60.875	0.16986	1.46308521635379\\
60.875	0.17352	1.56059782092256\\
60.875	0.17718	1.66243047838241\\
60.875	0.18084	1.76858318873334\\
60.875	0.1845	1.87905595197537\\
60.875	0.18816	1.99384876810846\\
60.875	0.19182	2.11296163713265\\
60.875	0.19548	2.23639455904791\\
60.875	0.19914	2.36414753385426\\
60.875	0.2028	2.4962205615517\\
60.875	0.20646	2.63261364214021\\
60.875	0.21012	2.77332677561982\\
60.875	0.21378	2.9183599619905\\
60.875	0.21744	3.06771320125227\\
60.875	0.2211	3.22138649340512\\
60.875	0.22476	3.37937983844905\\
60.875	0.22842	3.54169323638406\\
60.875	0.23208	3.70832668721016\\
60.875	0.23574	3.87928019092734\\
60.875	0.2394	4.05455374753561\\
60.875	0.24306	4.23414735703496\\
60.875	0.24672	4.41806101942539\\
60.875	0.25038	4.60629473470691\\
60.875	0.25404	4.7988485028795\\
60.875	0.2577	4.99572232394318\\
60.875	0.26136	5.19691619789795\\
60.875	0.26502	5.40243012474379\\
60.875	0.26868	5.61226410448072\\
60.875	0.27234	5.82641813710873\\
60.875	0.276	6.04489222262783\\
61.25	0.093	0.405827564020234\\
61.25	0.09666	0.416567248084845\\
61.25	0.10032	0.431626985040543\\
61.25	0.10398	0.451006774887324\\
61.25	0.10764	0.474706617625184\\
61.25	0.1113	0.502726513254128\\
61.25	0.11496	0.535066461774159\\
61.25	0.11862	0.571726463185271\\
61.25	0.12228	0.612706517487465\\
61.25	0.12594	0.65800662468074\\
61.25	0.1296	0.707626784765103\\
61.25	0.13326	0.761566997740545\\
61.25	0.13692	0.819827263607074\\
61.25	0.14058	0.882407582364683\\
61.25	0.14424	0.949307954013378\\
61.25	0.1479	1.02052837855315\\
61.25	0.15156	1.09606885598401\\
61.25	0.15522	1.17592938630596\\
61.25	0.15888	1.26010996951898\\
61.25	0.16254	1.34861060562309\\
61.25	0.1662	1.44143129461829\\
61.25	0.16986	1.53857203650455\\
61.25	0.17352	1.64003283128192\\
61.25	0.17718	1.74581367895036\\
61.25	0.18084	1.85591457950988\\
61.25	0.1845	1.97033553296049\\
61.25	0.18816	2.08907653930218\\
61.25	0.19182	2.21213759853496\\
61.25	0.19548	2.33951871065881\\
61.25	0.19914	2.47121987567375\\
61.25	0.2028	2.60724109357978\\
61.25	0.20646	2.74758236437688\\
61.25	0.21012	2.89224368806508\\
61.25	0.21378	3.04122506464434\\
61.25	0.21744	3.1945264941147\\
61.25	0.2211	3.35214797647614\\
61.25	0.22476	3.51408951172866\\
61.25	0.22842	3.68035109987227\\
61.25	0.23208	3.85093274090696\\
61.25	0.23574	4.02583443483272\\
61.25	0.2394	4.20505618164958\\
61.25	0.24306	4.38859798135752\\
61.25	0.24672	4.57645983395654\\
61.25	0.25038	4.76864173944665\\
61.25	0.25404	4.96514369782783\\
61.25	0.2577	5.1659657091001\\
61.25	0.26136	5.37110777326346\\
61.25	0.26502	5.58056989031789\\
61.25	0.26868	5.79435206026341\\
61.25	0.27234	6.01245428310001\\
61.25	0.276	6.2348765588277\\
61.625	0.093	0.397357183308098\\
61.625	0.09666	0.412045057581299\\
61.625	0.10032	0.431052984745583\\
61.625	0.10398	0.454380964800953\\
61.625	0.10764	0.482028997747403\\
61.625	0.1113	0.51399708358494\\
61.625	0.11496	0.55028522231356\\
61.625	0.11862	0.590893413933259\\
61.625	0.12228	0.635821658444041\\
61.625	0.12594	0.68506995584591\\
61.625	0.1296	0.738638306138862\\
61.625	0.13326	0.796526709322894\\
61.625	0.13692	0.858735165398012\\
61.625	0.14058	0.925263674364211\\
61.625	0.14424	0.996112236221496\\
61.625	0.1479	1.07128085096986\\
61.625	0.15156	1.15076951860931\\
61.625	0.15522	1.23457823913984\\
61.625	0.15888	1.32270701256146\\
61.625	0.16254	1.41515583887416\\
61.625	0.1662	1.51192471807794\\
61.625	0.16986	1.6130136501728\\
61.625	0.17352	1.71842263515875\\
61.625	0.17718	1.82815167303578\\
61.625	0.18084	1.9422007638039\\
61.625	0.1845	2.06056990746309\\
61.625	0.18816	2.18325910401337\\
61.625	0.19182	2.31026835345474\\
61.625	0.19548	2.44159765578718\\
61.625	0.19914	2.57724701101071\\
61.625	0.2028	2.71721641912533\\
61.625	0.20646	2.86150588013102\\
61.625	0.21012	3.0101153940278\\
61.625	0.21378	3.16304496081567\\
61.625	0.21744	3.32029458049461\\
61.625	0.2211	3.48186425306464\\
61.625	0.22476	3.64775397852575\\
61.625	0.22842	3.81796375687795\\
61.625	0.23208	3.99249358812123\\
61.625	0.23574	4.17134347225558\\
61.625	0.2394	4.35451340928103\\
61.625	0.24306	4.54200339919756\\
61.625	0.24672	4.73381344200517\\
61.625	0.25038	4.92994353770387\\
61.625	0.25404	5.13039368629364\\
61.625	0.2577	5.3351638877745\\
61.625	0.26136	5.54425414214644\\
61.625	0.26502	5.75766444940946\\
61.625	0.26868	5.97539480956357\\
61.625	0.27234	6.19744522260877\\
61.625	0.276	6.42381568854504\\
62	0.093	0.38784159611343\\
62	0.09666	0.406477660595216\\
62	0.10032	0.429433777968093\\
62	0.10398	0.456709948232053\\
62	0.10764	0.488306171387093\\
62	0.1113	0.524222447433216\\
62	0.11496	0.564458776370425\\
62	0.11862	0.609015158198718\\
62	0.12228	0.65789159291809\\
62	0.12594	0.711088080528548\\
62	0.1296	0.76860462103009\\
62	0.13326	0.830441214422711\\
62	0.13692	0.896597860706416\\
62	0.14058	0.967074559881208\\
62	0.14424	1.04187131194708\\
62	0.1479	1.12098811690403\\
62	0.15156	1.20442497475207\\
62	0.15522	1.2921818854912\\
62	0.15888	1.3842588491214\\
62	0.16254	1.48065586564269\\
62	0.1662	1.58137293505506\\
62	0.16986	1.68641005735851\\
62	0.17352	1.79576723255306\\
62	0.17718	1.90944446063867\\
62	0.18084	2.02744174161537\\
62	0.1845	2.14975907548317\\
62	0.18816	2.27639646224203\\
62	0.19182	2.40735390189199\\
62	0.19548	2.54263139443303\\
62	0.19914	2.68222893986514\\
62	0.2028	2.82614653818835\\
62	0.20646	2.97438418940263\\
62	0.21012	3.126941893508\\
62	0.21378	3.28381965050446\\
62	0.21744	3.44501746039199\\
62	0.2211	3.61053532317061\\
62	0.22476	3.78037323884031\\
62	0.22842	3.95453120740109\\
62	0.23208	4.13300922885296\\
62	0.23574	4.31580730319591\\
62	0.2394	4.50292543042994\\
62	0.24306	4.69436361055507\\
62	0.24672	4.89012184357126\\
62	0.25038	5.09020012947856\\
62	0.25404	5.29459846827692\\
62	0.2577	5.50331685996636\\
62	0.26136	5.71635530454689\\
62	0.26502	5.9337138020185\\
62	0.26868	6.1553923523812\\
62	0.27234	6.38139095563499\\
62	0.276	6.61170961177986\\
62.375	0.093	0.377280802436231\\
62.375	0.09666	0.399865057126611\\
62.375	0.10032	0.426769364708078\\
62.375	0.10398	0.457993725180627\\
62.375	0.10764	0.493538138544253\\
62.375	0.1113	0.533402604798969\\
62.375	0.11496	0.577587123944768\\
62.375	0.11862	0.62609169598165\\
62.375	0.12228	0.678916320909612\\
62.375	0.12594	0.73606099872866\\
62.375	0.1296	0.797525729438787\\
62.375	0.13326	0.863310513039998\\
62.375	0.13692	0.933415349532296\\
62.375	0.14058	1.00784023891567\\
62.375	0.14424	1.08658518119014\\
62.375	0.1479	1.16965017635568\\
62.375	0.15156	1.25703522441231\\
62.375	0.15522	1.34874032536003\\
62.375	0.15888	1.44476547919882\\
62.375	0.16254	1.5451106859287\\
62.375	0.1662	1.64977594554966\\
62.375	0.16986	1.7587612580617\\
62.375	0.17352	1.87206662346483\\
62.375	0.17718	1.98969204175904\\
62.375	0.18084	2.11163751294433\\
62.375	0.1845	2.23790303702072\\
62.375	0.18816	2.36848861398817\\
62.375	0.19182	2.50339424384672\\
62.375	0.19548	2.64261992659634\\
62.375	0.19914	2.78616566223705\\
62.375	0.2028	2.93403145076884\\
62.375	0.20646	3.08621729219172\\
62.375	0.21012	3.24272318650568\\
62.375	0.21378	3.40354913371072\\
62.375	0.21744	3.56869513380684\\
62.375	0.2211	3.73816118679405\\
62.375	0.22476	3.91194729267234\\
62.375	0.22842	4.09005345144171\\
62.375	0.23208	4.27247966310218\\
62.375	0.23574	4.45922592765371\\
62.375	0.2394	4.65029224509633\\
62.375	0.24306	4.84567861543005\\
62.375	0.24672	5.04538503865483\\
62.375	0.25038	5.24941151477071\\
62.375	0.25404	5.45775804377766\\
62.375	0.2577	5.6704246256757\\
62.375	0.26136	5.88741126046483\\
62.375	0.26502	6.10871794814503\\
62.375	0.26868	6.33434468871632\\
62.375	0.27234	6.56429148217869\\
62.375	0.276	6.79855832853214\\
62.75	0.093	0.365674802276511\\
62.75	0.09666	0.392207247175481\\
62.75	0.10032	0.423059744965534\\
62.75	0.10398	0.458232295646672\\
62.75	0.10764	0.497724899218891\\
62.75	0.1113	0.541537555682197\\
62.75	0.11496	0.589670265036586\\
62.75	0.11862	0.642123027282054\\
62.75	0.12228	0.698895842418605\\
62.75	0.12594	0.759988710446243\\
62.75	0.1296	0.825401631364964\\
62.75	0.13326	0.895134605174764\\
62.75	0.13692	0.969187631875652\\
62.75	0.14058	1.04756071146762\\
62.75	0.14424	1.13025384395067\\
62.75	0.1479	1.21726702932481\\
62.75	0.15156	1.30860026759003\\
62.75	0.15522	1.40425355874633\\
62.75	0.15888	1.50422690279371\\
62.75	0.16254	1.60852029973218\\
62.75	0.1662	1.71713374956173\\
62.75	0.16986	1.83006725228236\\
62.75	0.17352	1.94732080789408\\
62.75	0.17718	2.06889441639688\\
62.75	0.18084	2.19478807779076\\
62.75	0.1845	2.32500179207573\\
62.75	0.18816	2.45953555925177\\
62.75	0.19182	2.59838937931892\\
62.75	0.19548	2.74156325227712\\
62.75	0.19914	2.88905717812642\\
62.75	0.2028	3.04087115686681\\
62.75	0.20646	3.19700518849827\\
62.75	0.21012	3.35745927302083\\
62.75	0.21378	3.52223341043445\\
62.75	0.21744	3.69132760073917\\
62.75	0.2211	3.86474184393497\\
62.75	0.22476	4.04247614002185\\
62.75	0.22842	4.22453048899981\\
62.75	0.23208	4.41090489086886\\
62.75	0.23574	4.60159934562899\\
62.75	0.2394	4.7966138532802\\
62.75	0.24306	4.9959484138225\\
62.75	0.24672	5.19960302725587\\
62.75	0.25038	5.40757769358034\\
62.75	0.25404	5.61987241279589\\
62.75	0.2577	5.83648718490251\\
62.75	0.26136	6.05742200990023\\
62.75	0.26502	6.28267688778902\\
62.75	0.26868	6.5122518185689\\
62.75	0.27234	6.74614680223986\\
62.75	0.276	6.9843618388019\\
63.125	0.093	0.353023595634254\\
63.125	0.09666	0.383504230741813\\
63.125	0.10032	0.418304918740459\\
63.125	0.10398	0.457425659630188\\
63.125	0.10764	0.500866453410996\\
63.125	0.1113	0.548627300082888\\
63.125	0.11496	0.600708199645867\\
63.125	0.11862	0.657109152099927\\
63.125	0.12228	0.717830157445069\\
63.125	0.12594	0.782871215681296\\
63.125	0.1296	0.852232326808606\\
63.125	0.13326	0.925913490826996\\
63.125	0.13692	1.00391470773647\\
63.125	0.14058	1.08623597753703\\
63.125	0.14424	1.17287730022867\\
63.125	0.1479	1.2638386758114\\
63.125	0.15156	1.35912010428521\\
63.125	0.15522	1.4587215856501\\
63.125	0.15888	1.56264311990607\\
63.125	0.16254	1.67088470705313\\
63.125	0.1662	1.78344634709127\\
63.125	0.16986	1.90032804002049\\
63.125	0.17352	2.0215297858408\\
63.125	0.17718	2.14705158455219\\
63.125	0.18084	2.27689343615466\\
63.125	0.1845	2.41105534064822\\
63.125	0.18816	2.54953729803285\\
63.125	0.19182	2.69233930830858\\
63.125	0.19548	2.83946137147538\\
63.125	0.19914	2.99090348753327\\
63.125	0.2028	3.14666565648225\\
63.125	0.20646	3.3067478783223\\
63.125	0.21012	3.47115015305344\\
63.125	0.21378	3.63987248067566\\
63.125	0.21744	3.81291486118896\\
63.125	0.2211	3.99027729459335\\
63.125	0.22476	4.17195978088882\\
63.125	0.22842	4.35796232007537\\
63.125	0.23208	4.54828491215301\\
63.125	0.23574	4.74292755712173\\
63.125	0.2394	4.94189025498152\\
63.125	0.24306	5.14517300573243\\
63.125	0.24672	5.35277580937439\\
63.125	0.25038	5.56469866590744\\
63.125	0.25404	5.78094157533158\\
63.125	0.2577	6.00150453764679\\
63.125	0.26136	6.2263875528531\\
63.125	0.26502	6.45559062095047\\
63.125	0.26868	6.68911374193894\\
63.125	0.27234	6.9269569158185\\
63.125	0.276	7.16912014258913\\
63.5	0.093	0.339327182509474\\
63.5	0.09666	0.373756007825623\\
63.5	0.10032	0.412504886032855\\
63.5	0.10398	0.455573817131173\\
63.5	0.10764	0.502962801120571\\
63.5	0.1113	0.554671838001056\\
63.5	0.11496	0.610700927772624\\
63.5	0.11862	0.671050070435275\\
63.5	0.12228	0.735719265989005\\
63.5	0.12594	0.804708514433822\\
63.5	0.1296	0.878017815769722\\
63.5	0.13326	0.955647169996702\\
63.5	0.13692	1.03759657711477\\
63.5	0.14058	1.12386603712392\\
63.5	0.14424	1.21445555002415\\
63.5	0.1479	1.30936511581546\\
63.5	0.15156	1.40859473449786\\
63.5	0.15522	1.51214440607134\\
63.5	0.15888	1.6200141305359\\
63.5	0.16254	1.73220390789155\\
63.5	0.1662	1.84871373813828\\
63.5	0.16986	1.96954362127609\\
63.5	0.17352	2.09469355730499\\
63.5	0.17718	2.22416354622497\\
63.5	0.18084	2.35795358803603\\
63.5	0.1845	2.49606368273818\\
63.5	0.18816	2.63849383033141\\
63.5	0.19182	2.78524403081572\\
63.5	0.19548	2.93631428419112\\
63.5	0.19914	3.09170459045759\\
63.5	0.2028	3.25141494961516\\
63.5	0.20646	3.4154453616638\\
63.5	0.21012	3.58379582660353\\
63.5	0.21378	3.75646634443434\\
63.5	0.21744	3.93345691515623\\
63.5	0.2211	4.11476753876921\\
63.5	0.22476	4.30039821527327\\
63.5	0.22842	4.49034894466841\\
63.5	0.23208	4.68461972695464\\
63.5	0.23574	4.88321056213194\\
63.5	0.2394	5.08612145020034\\
63.5	0.24306	5.29335239115981\\
63.5	0.24672	5.50490338501037\\
63.5	0.25038	5.72077443175202\\
63.5	0.25404	5.94096553138474\\
63.5	0.2577	6.16547668390855\\
63.5	0.26136	6.39430788932344\\
63.5	0.26502	6.62745914762941\\
63.5	0.26868	6.86493045882647\\
63.5	0.27234	7.10672182291461\\
63.5	0.276	7.35283323989383\\
63.875	0.093	0.32458556290216\\
63.875	0.09666	0.362962578426899\\
63.875	0.10032	0.405659646842724\\
63.875	0.10398	0.452676768149632\\
63.875	0.10764	0.504013942347619\\
63.875	0.1113	0.559671169436694\\
63.875	0.11496	0.619648449416851\\
63.875	0.11862	0.683945782288088\\
63.875	0.12228	0.752563168050409\\
63.875	0.12594	0.825500606703815\\
63.875	0.1296	0.902758098248305\\
63.875	0.13326	0.984335642683874\\
63.875	0.13692	1.07023324001053\\
63.875	0.14058	1.16045089022827\\
63.875	0.14424	1.25498859333709\\
63.875	0.1479	1.35384634933699\\
63.875	0.15156	1.45702415822798\\
63.875	0.15522	1.56452202001005\\
63.875	0.15888	1.6763399346832\\
63.875	0.16254	1.79247790224745\\
63.875	0.1662	1.91293592270276\\
63.875	0.16986	2.03771399604916\\
63.875	0.17352	2.16681212228665\\
63.875	0.17718	2.30023030141522\\
63.875	0.18084	2.43796853343487\\
63.875	0.1845	2.58002681834561\\
63.875	0.18816	2.72640515614742\\
63.875	0.19182	2.87710354684033\\
63.875	0.19548	3.03212199042431\\
63.875	0.19914	3.19146048689938\\
63.875	0.2028	3.35511903626553\\
63.875	0.20646	3.52309763852277\\
63.875	0.21012	3.69539629367108\\
63.875	0.21378	3.87201500171048\\
63.875	0.21744	4.05295376264096\\
63.875	0.2211	4.23821257646254\\
63.875	0.22476	4.42779144317518\\
63.875	0.22842	4.62169036277891\\
63.875	0.23208	4.81990933527373\\
63.875	0.23574	5.02244836065963\\
63.875	0.2394	5.22930743893661\\
63.875	0.24306	5.44048657010468\\
63.875	0.24672	5.65598575416383\\
63.875	0.25038	5.87580499111406\\
63.875	0.25404	6.09994428095537\\
63.875	0.2577	6.32840362368777\\
63.875	0.26136	6.56118301931125\\
63.875	0.26502	6.79828246782581\\
63.875	0.26868	7.03970196923146\\
63.875	0.27234	7.28544152352819\\
63.875	0.276	7.535501130716\\
64.25	0.093	0.308798736812324\\
64.25	0.09666	0.351123942545651\\
64.25	0.10032	0.397769201170066\\
64.25	0.10398	0.44873451268556\\
64.25	0.10764	0.504019877092137\\
64.25	0.1113	0.563625294389802\\
64.25	0.11496	0.627550764578549\\
64.25	0.11862	0.695796287658379\\
64.25	0.12228	0.768361863629289\\
64.25	0.12594	0.845247492491285\\
64.25	0.1296	0.926453174244364\\
64.25	0.13326	1.01197890888852\\
64.25	0.13692	1.10182469642377\\
64.25	0.14058	1.19599053685009\\
64.25	0.14424	1.29447643016751\\
64.25	0.1479	1.397282376376\\
64.25	0.15156	1.50440837547558\\
64.25	0.15522	1.61585442746624\\
64.25	0.15888	1.73162053234798\\
64.25	0.16254	1.85170669012081\\
64.25	0.1662	1.97611290078472\\
64.25	0.16986	2.10483916433971\\
64.25	0.17352	2.23788548078579\\
64.25	0.17718	2.37525185012294\\
64.25	0.18084	2.51693827235118\\
64.25	0.1845	2.66294474747051\\
64.25	0.18816	2.81327127548091\\
64.25	0.19182	2.96791785638241\\
64.25	0.19548	3.12688449017498\\
64.25	0.19914	3.29017117685864\\
64.25	0.2028	3.45777791643338\\
64.25	0.20646	3.62970470889921\\
64.25	0.21012	3.80595155425612\\
64.25	0.21378	3.98651845250411\\
64.25	0.21744	4.17140540364317\\
64.25	0.2211	4.36061240767333\\
64.25	0.22476	4.55413946459457\\
64.25	0.22842	4.75198657440689\\
64.25	0.23208	4.9541537371103\\
64.25	0.23574	5.16064095270479\\
64.25	0.2394	5.37144822119036\\
64.25	0.24306	5.58657554256702\\
64.25	0.24672	5.80602291683476\\
64.25	0.25038	6.02979034399358\\
64.25	0.25404	6.25787782404348\\
64.25	0.2577	6.49028535698447\\
64.25	0.26136	6.72701294281654\\
64.25	0.26502	6.96806058153969\\
64.25	0.26868	7.21342827315392\\
64.25	0.27234	7.46311601765925\\
64.25	0.276	7.71712381505565\\
64.625	0.093	0.291966704239957\\
64.625	0.09666	0.338240100181874\\
64.625	0.10032	0.388833549014875\\
64.625	0.10398	0.443747050738962\\
64.625	0.10764	0.502980605354129\\
64.625	0.1113	0.566534212860383\\
64.625	0.11496	0.634407873257719\\
64.625	0.11862	0.706601586546139\\
64.625	0.12228	0.783115352725638\\
64.625	0.12594	0.863949171796224\\
64.625	0.1296	0.949103043757893\\
64.625	0.13326	1.03857696861064\\
64.625	0.13692	1.13237094635448\\
64.625	0.14058	1.23048497698939\\
64.625	0.14424	1.33291906051539\\
64.625	0.1479	1.43967319693248\\
64.625	0.15156	1.55074738624064\\
64.625	0.15522	1.66614162843989\\
64.625	0.15888	1.78585592353022\\
64.625	0.16254	1.90989027151164\\
64.625	0.1662	2.03824467238414\\
64.625	0.16986	2.17091912614772\\
64.625	0.17352	2.30791363280239\\
64.625	0.17718	2.44922819234814\\
64.625	0.18084	2.59486280478497\\
64.625	0.1845	2.74481747011288\\
64.625	0.18816	2.89909218833188\\
64.625	0.19182	3.05768695944197\\
64.625	0.19548	3.22060178344312\\
64.625	0.19914	3.38783666033537\\
64.625	0.2028	3.55939159011871\\
64.625	0.20646	3.73526657279311\\
64.625	0.21012	3.91546160835862\\
64.625	0.21378	4.09997669681519\\
64.625	0.21744	4.28881183816285\\
64.625	0.2211	4.4819670324016\\
64.625	0.22476	4.67944227953143\\
64.625	0.22842	4.88123757955234\\
64.625	0.23208	5.08735293246434\\
64.625	0.23574	5.29778833826742\\
64.625	0.2394	5.51254379696158\\
64.625	0.24306	5.73161930854682\\
64.625	0.24672	5.95501487302314\\
64.625	0.25038	6.18273049039056\\
64.625	0.25404	6.41476616064905\\
64.625	0.2577	6.65112188379863\\
64.625	0.26136	6.8917976598393\\
64.625	0.26502	7.13679348877104\\
64.625	0.26868	7.38610937059386\\
64.625	0.27234	7.63974530530777\\
64.625	0.276	7.89770129291276\\
65	0.093	0.274089465185058\\
65	0.09666	0.324311051335565\\
65	0.10032	0.378852690377159\\
65	0.10398	0.437714382309835\\
65	0.10764	0.500896127133592\\
65	0.1113	0.568397924848435\\
65	0.11496	0.640219775454358\\
65	0.11862	0.716361678951367\\
65	0.12228	0.796823635339456\\
65	0.12594	0.881605644618632\\
65	0.1296	0.97070770678889\\
65	0.13326	1.06412982185023\\
65	0.13692	1.16187198980265\\
65	0.14058	1.26393421064616\\
65	0.14424	1.37031648438075\\
65	0.1479	1.48101881100642\\
65	0.15156	1.59604119052318\\
65	0.15522	1.71538362293102\\
65	0.15888	1.83904610822994\\
65	0.16254	1.96702864641995\\
65	0.1662	2.09933123750104\\
65	0.16986	2.23595388147321\\
65	0.17352	2.37689657833646\\
65	0.17718	2.5221593280908\\
65	0.18084	2.67174213073622\\
65	0.1845	2.82564498627273\\
65	0.18816	2.98386789470031\\
65	0.19182	3.14641085601899\\
65	0.19548	3.31327387022874\\
65	0.19914	3.48445693732958\\
65	0.2028	3.6599600573215\\
65	0.20646	3.8397832302045\\
65	0.21012	4.02392645597859\\
65	0.21378	4.21238973464375\\
65	0.21744	4.40517306620001\\
65	0.2211	4.60227645064734\\
65	0.22476	4.80369988798576\\
65	0.22842	5.00944337821526\\
65	0.23208	5.21950692133585\\
65	0.23574	5.43389051734752\\
65	0.2394	5.65259416625026\\
65	0.24306	5.8756178680441\\
65	0.24672	6.10296162272901\\
65	0.25038	6.33462543030502\\
65	0.25404	6.5706092907721\\
65	0.2577	6.81091320413027\\
65	0.26136	7.05553717037952\\
65	0.26502	7.30448118951985\\
65	0.26868	7.55774526155126\\
65	0.27234	7.81532938647376\\
65	0.276	8.07723356428735\\
65.375	0.093	0.255167019647636\\
65.375	0.09666	0.309336796006732\\
65.375	0.10032	0.367826625256912\\
65.375	0.10398	0.430636507398179\\
65.375	0.10764	0.497766442430525\\
65.375	0.1113	0.569216430353958\\
65.375	0.11496	0.644986471168474\\
65.375	0.11862	0.725076564874072\\
65.375	0.12228	0.809486711470751\\
65.375	0.12594	0.898216910958516\\
65.375	0.1296	0.991267163337364\\
65.375	0.13326	1.08863746860729\\
65.375	0.13692	1.19032782676831\\
65.375	0.14058	1.29633823782041\\
65.375	0.14424	1.40666870176359\\
65.375	0.1479	1.52131921859785\\
65.375	0.15156	1.64028978832319\\
65.375	0.15522	1.76358041093962\\
65.375	0.15888	1.89119108644713\\
65.375	0.16254	2.02312181484573\\
65.375	0.1662	2.15937259613541\\
65.375	0.16986	2.29994343031617\\
65.375	0.17352	2.44483431738801\\
65.375	0.17718	2.59404525735094\\
65.375	0.18084	2.74757625020495\\
65.375	0.1845	2.90542729595005\\
65.375	0.18816	3.06759839458622\\
65.375	0.19182	3.23408954611348\\
65.375	0.19548	3.40490075053183\\
65.375	0.19914	3.58003200784125\\
65.375	0.2028	3.75948331804176\\
65.375	0.20646	3.94325468113336\\
65.375	0.21012	4.13134609711603\\
65.375	0.21378	4.32375756598979\\
65.375	0.21744	4.52048908775463\\
65.375	0.2211	4.72154066241056\\
65.375	0.22476	4.92691228995756\\
65.375	0.22842	5.13660397039565\\
65.375	0.23208	5.35061570372483\\
65.375	0.23574	5.56894748994509\\
65.375	0.2394	5.79159932905643\\
65.375	0.24306	6.01857122105886\\
65.375	0.24672	6.24986316595236\\
65.375	0.25038	6.48547516373695\\
65.375	0.25404	6.72540721441263\\
65.375	0.2577	6.96965931797938\\
65.375	0.26136	7.21823147443722\\
65.375	0.26502	7.47112368378614\\
65.375	0.26868	7.72833594602615\\
65.375	0.27234	7.98986826115724\\
65.375	0.276	8.25572062917941\\
65.75	0.093	0.23519936762768\\
65.75	0.09666	0.293317334195366\\
65.75	0.10032	0.35575535365414\\
65.75	0.10398	0.422513426003995\\
65.75	0.10764	0.493591551244931\\
65.75	0.1113	0.568989729376954\\
65.75	0.11496	0.648707960400059\\
65.75	0.11862	0.732746244314248\\
65.75	0.12228	0.821104581119516\\
65.75	0.12594	0.913782970815871\\
65.75	0.1296	1.01078141340331\\
65.75	0.13326	1.11209990888183\\
65.75	0.13692	1.21773845725143\\
65.75	0.14058	1.32769705851211\\
65.75	0.14424	1.44197571266389\\
65.75	0.1479	1.56057441970674\\
65.75	0.15156	1.68349317964067\\
65.75	0.15522	1.81073199246569\\
65.75	0.15888	1.94229085818179\\
65.75	0.16254	2.07816977678898\\
65.75	0.1662	2.21836874828725\\
65.75	0.16986	2.36288777267659\\
65.75	0.17352	2.51172684995703\\
65.75	0.17718	2.66488598012855\\
65.75	0.18084	2.82236516319115\\
65.75	0.1845	2.98416439914483\\
65.75	0.18816	3.1502836879896\\
65.75	0.19182	3.32072302972545\\
65.75	0.19548	3.49548242435239\\
65.75	0.19914	3.6745618718704\\
65.75	0.2028	3.8579613722795\\
65.75	0.20646	4.04568092557968\\
65.75	0.21012	4.23772053177095\\
65.75	0.21378	4.4340801908533\\
65.75	0.21744	4.63475990282672\\
65.75	0.2211	4.83975966769124\\
65.75	0.22476	5.04907948544684\\
65.75	0.22842	5.26271935609352\\
65.75	0.23208	5.48067927963128\\
65.75	0.23574	5.70295925606013\\
65.75	0.2394	5.92955928538006\\
65.75	0.24306	6.16047936759108\\
65.75	0.24672	6.39571950269317\\
65.75	0.25038	6.63527969068635\\
65.75	0.25404	6.87915993157061\\
65.75	0.2577	7.12736022534595\\
65.75	0.26136	7.3798805720124\\
65.75	0.26502	7.6367209715699\\
65.75	0.26868	7.8978814240185\\
65.75	0.27234	8.16336192935817\\
65.75	0.276	8.43316248758893\\
66.125	0.093	0.214186509125203\\
66.125	0.09666	0.276252665901479\\
66.125	0.10032	0.342638875568842\\
66.125	0.10398	0.413345138127287\\
66.125	0.10764	0.488371453576813\\
66.125	0.1113	0.567717821917421\\
66.125	0.11496	0.651384243149117\\
66.125	0.11862	0.739370717271894\\
66.125	0.12228	0.831677244285753\\
66.125	0.12594	0.9283038241907\\
66.125	0.1296	1.02925045698673\\
66.125	0.13326	1.13451714267383\\
66.125	0.13692	1.24410388125203\\
66.125	0.14058	1.3580106727213\\
66.125	0.14424	1.47623751708166\\
66.125	0.1479	1.5987844143331\\
66.125	0.15156	1.72565136447563\\
66.125	0.15522	1.85683836750924\\
66.125	0.15888	1.99234542343393\\
66.125	0.16254	2.13217253224971\\
66.125	0.1662	2.27631969395656\\
66.125	0.16986	2.4247869085545\\
66.125	0.17352	2.57757417604352\\
66.125	0.17718	2.73468149642363\\
66.125	0.18084	2.89610886969482\\
66.125	0.1845	3.0618562958571\\
66.125	0.18816	3.23192377491045\\
66.125	0.19182	3.4063113068549\\
66.125	0.19548	3.58501889169041\\
66.125	0.19914	3.76804652941702\\
66.125	0.2028	3.95539422003471\\
66.125	0.20646	4.14706196354349\\
66.125	0.21012	4.34304975994334\\
66.125	0.21378	4.54335760923428\\
66.125	0.21744	4.74798551141629\\
66.125	0.2211	4.95693346648941\\
66.125	0.22476	5.17020147445359\\
66.125	0.22842	5.38778953530886\\
66.125	0.23208	5.60969764905522\\
66.125	0.23574	5.83592581569265\\
66.125	0.2394	6.06647403522117\\
66.125	0.24306	6.30134230764077\\
66.125	0.24672	6.54053063295146\\
66.125	0.25038	6.78403901115323\\
66.125	0.25404	7.03186744224609\\
66.125	0.2577	7.28401592623002\\
66.125	0.26136	7.54048446310504\\
66.125	0.26502	7.80127305287114\\
66.125	0.26868	8.06638169552832\\
66.125	0.27234	8.33581039107659\\
66.125	0.276	8.60955913951594\\
66.5	0.093	0.192128444140185\\
66.5	0.09666	0.258142791125051\\
66.5	0.10032	0.328477191001003\\
66.5	0.10398	0.403131643768038\\
66.5	0.10764	0.482106149426153\\
66.5	0.1113	0.565400707975355\\
66.5	0.11496	0.65301531941564\\
66.5	0.11862	0.744949983747007\\
66.5	0.12228	0.841204700969455\\
66.5	0.12594	0.941779471082989\\
66.5	0.1296	1.04667429408761\\
66.5	0.13326	1.1558891699833\\
66.5	0.13692	1.26942409877009\\
66.5	0.14058	1.38727908044795\\
66.5	0.14424	1.5094541150169\\
66.5	0.1479	1.63594920247693\\
66.5	0.15156	1.76676434282805\\
66.5	0.15522	1.90189953607025\\
66.5	0.15888	2.04135478220352\\
66.5	0.16254	2.18513008122789\\
66.5	0.1662	2.33322543314334\\
66.5	0.16986	2.48564083794987\\
66.5	0.17352	2.64237629564748\\
66.5	0.17718	2.80343180623618\\
66.5	0.18084	2.96880736971595\\
66.5	0.1845	3.13850298608682\\
66.5	0.18816	3.31251865534877\\
66.5	0.19182	3.4908543775018\\
66.5	0.19548	3.67351015254591\\
66.5	0.19914	3.8604859804811\\
66.5	0.2028	4.05178186130739\\
66.5	0.20646	4.24739779502474\\
66.5	0.21012	4.44733378163319\\
66.5	0.21378	4.65158982113272\\
66.5	0.21744	4.86016591352333\\
66.5	0.2211	5.07306205880502\\
66.5	0.22476	5.2902782569778\\
66.5	0.22842	5.51181450804166\\
66.5	0.23208	5.73767081199661\\
66.5	0.23574	5.96784716884263\\
66.5	0.2394	6.20234357857974\\
66.5	0.24306	6.44116004120793\\
66.5	0.24672	6.68429655672721\\
66.5	0.25038	6.93175312513757\\
66.5	0.25404	7.18352974643901\\
66.5	0.2577	7.43962642063153\\
66.5	0.26136	7.70004314771515\\
66.5	0.26502	7.96477992768983\\
66.5	0.26868	8.23383676055561\\
66.5	0.27234	8.50721364631247\\
66.5	0.276	8.78491058496041\\
66.875	0.093	0.169025172672648\\
66.875	0.09666	0.238987709866103\\
66.875	0.10032	0.313270299950645\\
66.875	0.10398	0.39187294292627\\
66.875	0.10764	0.474795638792975\\
66.875	0.1113	0.562038387550766\\
66.875	0.11496	0.65360118919964\\
66.875	0.11862	0.749484043739598\\
66.875	0.12228	0.849686951170635\\
66.875	0.12594	0.954209911492758\\
66.875	0.1296	1.06305292470596\\
66.875	0.13326	1.17621599081025\\
66.875	0.13692	1.29369910980562\\
66.875	0.14058	1.41550228169208\\
66.875	0.14424	1.54162550646962\\
66.875	0.1479	1.67206878413824\\
66.875	0.15156	1.80683211469794\\
66.875	0.15522	1.94591549814873\\
66.875	0.15888	2.0893189344906\\
66.875	0.16254	2.23704242372356\\
66.875	0.1662	2.3890859658476\\
66.875	0.16986	2.54544956086271\\
66.875	0.17352	2.70613320876892\\
66.875	0.17718	2.8711369095662\\
66.875	0.18084	3.04046066325457\\
66.875	0.1845	3.21410446983402\\
66.875	0.18816	3.39206832930456\\
66.875	0.19182	3.57435224166618\\
66.875	0.19548	3.76095620691888\\
66.875	0.19914	3.95188022506266\\
66.875	0.2028	4.14712429609754\\
66.875	0.20646	4.34668842002349\\
66.875	0.21012	4.55057259684053\\
66.875	0.21378	4.75877682654864\\
66.875	0.21744	4.97130110914784\\
66.875	0.2211	5.18814544463812\\
66.875	0.22476	5.40930983301949\\
66.875	0.22842	5.63479427429194\\
66.875	0.23208	5.86459876845548\\
66.875	0.23574	6.09872331551009\\
66.875	0.2394	6.33716791545579\\
66.875	0.24306	6.57993256829257\\
66.875	0.24672	6.82701727402044\\
66.875	0.25038	7.07842203263939\\
66.875	0.25404	7.33414684414942\\
66.875	0.2577	7.59419170855053\\
66.875	0.26136	7.85855662584274\\
66.875	0.26502	8.12724159602601\\
66.875	0.26868	8.40024661910038\\
66.875	0.27234	8.67757169506583\\
66.875	0.276	8.95921682392236\\
67.25	0.093	0.144876694722584\\
67.25	0.09666	0.218787422124629\\
67.25	0.10032	0.297018202417757\\
67.25	0.10398	0.379569035601971\\
67.25	0.10764	0.466439921677269\\
67.25	0.1113	0.557630860643647\\
67.25	0.11496	0.653141852501111\\
67.25	0.11862	0.752972897249657\\
67.25	0.12228	0.857123994889288\\
67.25	0.12594	0.965595145420001\\
67.25	0.1296	1.0783863488418\\
67.25	0.13326	1.19549760515467\\
67.25	0.13692	1.31692891435864\\
67.25	0.14058	1.44268027645368\\
67.25	0.14424	1.57275169143981\\
67.25	0.1479	1.70714315931702\\
67.25	0.15156	1.84585468008531\\
67.25	0.15522	1.98888625374469\\
67.25	0.15888	2.13623788029515\\
67.25	0.16254	2.2879095597367\\
67.25	0.1662	2.44390129206932\\
67.25	0.16986	2.60421307729303\\
67.25	0.17352	2.76884491540783\\
67.25	0.17718	2.9377968064137\\
67.25	0.18084	3.11106875031066\\
67.25	0.1845	3.2886607470987\\
67.25	0.18816	3.47057279677783\\
67.25	0.19182	3.65680489934804\\
67.25	0.19548	3.84735705480933\\
67.25	0.19914	4.0422292631617\\
67.25	0.2028	4.24142152440516\\
67.25	0.20646	4.4449338385397\\
67.25	0.21012	4.65276620556533\\
67.25	0.21378	4.86491862548203\\
67.25	0.21744	5.08139109828982\\
67.25	0.2211	5.3021836239887\\
67.25	0.22476	5.52729620257866\\
67.25	0.22842	5.75672883405969\\
67.25	0.23208	5.99048151843182\\
67.25	0.23574	6.22855425569502\\
67.25	0.2394	6.47094704584931\\
67.25	0.24306	6.71765988889469\\
67.25	0.24672	6.96869278483114\\
67.25	0.25038	7.22404573365868\\
67.25	0.25404	7.4837187353773\\
67.25	0.2577	7.747711789987\\
67.25	0.26136	8.01602489748779\\
67.25	0.26502	8.28865805787966\\
67.25	0.26868	8.56561127116261\\
67.25	0.27234	8.84688453733666\\
67.25	0.276	9.13247785640178\\
67.625	0.093	0.119683010289987\\
67.625	0.09666	0.197541927900621\\
67.625	0.10032	0.279720898402342\\
67.625	0.10398	0.366219921795146\\
67.625	0.10764	0.45703899807903\\
67.625	0.1113	0.552178127254001\\
67.625	0.11496	0.651637309320054\\
67.625	0.11862	0.755416544277191\\
67.625	0.12228	0.863515832125407\\
67.625	0.12594	0.97593517286471\\
67.625	0.1296	1.0926745664951\\
67.625	0.13326	1.21373401301656\\
67.625	0.13692	1.33911351242912\\
67.625	0.14058	1.46881306473275\\
67.625	0.14424	1.60283266992747\\
67.625	0.1479	1.74117232801327\\
67.625	0.15156	1.88383203899015\\
67.625	0.15522	2.03081180285812\\
67.625	0.15888	2.18211161961717\\
67.625	0.16254	2.3377314892673\\
67.625	0.1662	2.49767141180852\\
67.625	0.16986	2.66193138724082\\
67.625	0.17352	2.8305114155642\\
67.625	0.17718	3.00341149677867\\
67.625	0.18084	3.18063163088421\\
67.625	0.1845	3.36217181788085\\
67.625	0.18816	3.54803205776856\\
67.625	0.19182	3.73821235054737\\
67.625	0.19548	3.93271269621724\\
67.625	0.19914	4.13153309477821\\
67.625	0.2028	4.33467354623026\\
67.625	0.20646	4.54213405057339\\
67.625	0.21012	4.7539146078076\\
67.625	0.21378	4.9700152179329\\
67.625	0.21744	5.19043588094927\\
67.625	0.2211	5.41517659685674\\
67.625	0.22476	5.64423736565528\\
67.625	0.22842	5.87761818734491\\
67.625	0.23208	6.11531906192563\\
67.625	0.23574	6.35733998939743\\
67.625	0.2394	6.6036809697603\\
67.625	0.24306	6.85434200301427\\
67.625	0.24672	7.10932308915931\\
67.625	0.25038	7.36862422819544\\
67.625	0.25404	7.63224542012265\\
67.625	0.2577	7.90018666494094\\
67.625	0.26136	8.17244796265032\\
67.625	0.26502	8.44902931325078\\
67.625	0.26868	8.72993071674232\\
67.625	0.27234	9.01515217312495\\
67.625	0.276	9.30469368239866\\
68	0.093	0.0934441193748627\\
68	0.09666	0.175251227194087\\
68	0.10032	0.261378387904398\\
68	0.10398	0.351825601505791\\
68	0.10764	0.446592867998264\\
68	0.1113	0.545680187381825\\
68	0.11496	0.649087559656468\\
68	0.11862	0.756814984822194\\
68	0.12228	0.868862462879\\
68	0.12594	0.985229993826892\\
68	0.1296	1.10591757766587\\
68	0.13326	1.23092521439592\\
68	0.13692	1.36025290401706\\
68	0.14058	1.49390064652929\\
68	0.14424	1.6318684419326\\
68	0.1479	1.77415629022699\\
68	0.15156	1.92076419141246\\
68	0.15522	2.07169214548902\\
68	0.15888	2.22694015245666\\
68	0.16254	2.38650821231538\\
68	0.1662	2.55039632506519\\
68	0.16986	2.71860449070608\\
68	0.17352	2.89113270923805\\
68	0.17718	3.0679809806611\\
68	0.18084	3.24914930497524\\
68	0.1845	3.43463768218047\\
68	0.18816	3.62444611227677\\
68	0.19182	3.81857459526416\\
68	0.19548	4.01702313114263\\
68	0.19914	4.21979171991218\\
68	0.2028	4.42688036157282\\
68	0.20646	4.63828905612454\\
68	0.21012	4.85401780356734\\
68	0.21378	5.07406660390123\\
68	0.21744	5.2984354571262\\
68	0.2211	5.52712436324226\\
68	0.22476	5.76013332224939\\
68	0.22842	5.9974623341476\\
68	0.23208	6.23911139893691\\
68	0.23574	6.4850805166173\\
68	0.2394	6.73536968718876\\
68	0.24306	6.98997891065132\\
68	0.24672	7.24890818700494\\
68	0.25038	7.51215751624967\\
68	0.25404	7.77972689838547\\
68	0.2577	8.05161633341235\\
68	0.26136	8.32782582133032\\
68	0.26502	8.60835536213937\\
68	0.26868	8.8932049558395\\
68	0.27234	9.18237460243072\\
68	0.276	9.47586430191301\\
68.375	0.093	0.0661600219772049\\
68.375	0.09666	0.151915320005019\\
68.375	0.10032	0.241990670923919\\
68.375	0.10398	0.336386074733902\\
68.375	0.10764	0.435101531434965\\
68.375	0.1113	0.538137041027115\\
68.375	0.11496	0.645492603510348\\
68.375	0.11862	0.757168218884663\\
68.375	0.12228	0.873163887150063\\
68.375	0.12594	0.993479608306544\\
68.375	0.1296	1.11811538235411\\
68.375	0.13326	1.24707120929275\\
68.375	0.13692	1.38034708912249\\
68.375	0.14058	1.5179430218433\\
68.375	0.14424	1.6598590074552\\
68.375	0.1479	1.80609504595818\\
68.375	0.15156	1.95665113735224\\
68.375	0.15522	2.11152728163739\\
68.375	0.15888	2.27072347881361\\
68.375	0.16254	2.43423972888093\\
68.375	0.1662	2.60207603183932\\
68.375	0.16986	2.7742323876888\\
68.375	0.17352	2.95070879642937\\
68.375	0.17718	3.13150525806101\\
68.375	0.18084	3.31662177258373\\
68.375	0.1845	3.50605833999755\\
68.375	0.18816	3.69981496030244\\
68.375	0.19182	3.89789163349842\\
68.375	0.19548	4.10028835958548\\
68.375	0.19914	4.30700513856362\\
68.375	0.2028	4.51804197043285\\
68.375	0.20646	4.73339885519316\\
68.375	0.21012	4.95307579284456\\
68.375	0.21378	5.17707278338704\\
68.375	0.21744	5.40538982682059\\
68.375	0.2211	5.63802692314524\\
68.375	0.22476	5.87498407236096\\
68.375	0.22842	6.11626127446777\\
68.375	0.23208	6.36185852946566\\
68.375	0.23574	6.61177583735464\\
68.375	0.2394	6.86601319813468\\
68.375	0.24306	7.12457061180584\\
68.375	0.24672	7.38744807836806\\
68.375	0.25038	7.65464559782137\\
68.375	0.25404	7.92616317016576\\
68.375	0.2577	8.20200079540123\\
68.375	0.26136	8.48215847352778\\
68.375	0.26502	8.76663620454542\\
68.375	0.26868	9.05543398845414\\
68.375	0.27234	9.34855182525396\\
68.375	0.276	9.64598971494484\\
68.75	0.093	0.0378307180970312\\
68.75	0.09666	0.127534206333434\\
68.75	0.10032	0.221557747460924\\
68.75	0.10398	0.319901341479497\\
68.75	0.10764	0.42256498838915\\
68.75	0.1113	0.529548688189889\\
68.75	0.11496	0.640852440881712\\
68.75	0.11862	0.756476246464617\\
68.75	0.12228	0.876420104938602\\
68.75	0.12594	1.00068401630367\\
68.75	0.1296	1.12926798055983\\
68.75	0.13326	1.26217199770707\\
68.75	0.13692	1.39939606774539\\
68.75	0.14058	1.54094019067479\\
68.75	0.14424	1.68680436649528\\
68.75	0.1479	1.83698859520685\\
68.75	0.15156	1.9914928768095\\
68.75	0.15522	2.15031721130324\\
68.75	0.15888	2.31346159868805\\
68.75	0.16254	2.48092603896396\\
68.75	0.1662	2.65271053213094\\
68.75	0.16986	2.82881507818901\\
68.75	0.17352	3.00923967713817\\
68.75	0.17718	3.1939843289784\\
68.75	0.18084	3.38304903370971\\
68.75	0.1845	3.57643379133212\\
68.75	0.18816	3.7741386018456\\
68.75	0.19182	3.97616346525017\\
68.75	0.19548	4.18250838154582\\
68.75	0.19914	4.39317335073255\\
68.75	0.2028	4.60815837281037\\
68.75	0.20646	4.82746344777927\\
68.75	0.21012	5.05108857563926\\
68.75	0.21378	5.27903375639031\\
68.75	0.21744	5.51129899003246\\
68.75	0.2211	5.7478842765657\\
68.75	0.22476	5.98878961599001\\
68.75	0.22842	6.2340150083054\\
68.75	0.23208	6.48356045351189\\
68.75	0.23574	6.73742595160945\\
68.75	0.2394	6.9956115025981\\
68.75	0.24306	7.25811710647784\\
68.75	0.24672	7.52494276324865\\
68.75	0.25038	7.79608847291055\\
68.75	0.25404	8.07155423546353\\
68.75	0.2577	8.35134005090759\\
68.75	0.26136	8.63544591924274\\
68.75	0.26502	8.92387184046897\\
68.75	0.26868	9.21661781458627\\
68.75	0.27234	9.51368384159468\\
68.75	0.276	9.81506992149415\\
69.125	0.093	0.0084562077343131\\
69.125	0.09666	0.102107886179309\\
69.125	0.10032	0.200079617515386\\
69.125	0.10398	0.302371401742551\\
69.125	0.10764	0.408983238860794\\
69.125	0.1113	0.519915128870123\\
69.125	0.11496	0.635167071770535\\
69.125	0.11862	0.75473906756203\\
69.125	0.12228	0.878631116244605\\
69.125	0.12594	1.00684321781827\\
69.125	0.1296	1.13937537228301\\
69.125	0.13326	1.27622757963883\\
69.125	0.13692	1.41739983988574\\
69.125	0.14058	1.56289215302374\\
69.125	0.14424	1.71270451905282\\
69.125	0.1479	1.86683693797297\\
69.125	0.15156	2.02528940978422\\
69.125	0.15522	2.18806193448655\\
69.125	0.15888	2.35515451207995\\
69.125	0.16254	2.52656714256445\\
69.125	0.1662	2.70229982594002\\
69.125	0.16986	2.88235256220668\\
69.125	0.17352	3.06672535136442\\
69.125	0.17718	3.25541819341324\\
69.125	0.18084	3.44843108835315\\
69.125	0.1845	3.64576403618414\\
69.125	0.18816	3.84741703690621\\
69.125	0.19182	4.05339009051937\\
69.125	0.19548	4.26368319702361\\
69.125	0.19914	4.47829635641893\\
69.125	0.2028	4.69722956870534\\
69.125	0.20646	4.92048283388283\\
69.125	0.21012	5.14805615195141\\
69.125	0.21378	5.37994952291106\\
69.125	0.21744	5.6161629467618\\
69.125	0.2211	5.85669642350362\\
69.125	0.22476	6.10154995313652\\
69.125	0.22842	6.3507235356605\\
69.125	0.23208	6.60421717107558\\
69.125	0.23574	6.86203085938173\\
69.125	0.2394	7.12416460057897\\
69.125	0.24306	7.39061839466729\\
69.125	0.24672	7.6613922416467\\
69.125	0.25038	7.93648614151719\\
69.125	0.25404	8.21590009427876\\
69.125	0.2577	8.4996340999314\\
69.125	0.26136	8.78768815847515\\
69.125	0.26502	9.08006226990996\\
69.125	0.26868	9.37675643423586\\
69.125	0.27234	9.67777065145285\\
69.125	0.276	9.98310492156092\\
69.5	0.093	-0.0219635091109245\\
69.5	0.09666	0.0756363595426579\\
69.5	0.10032	0.177556281087327\\
69.5	0.10398	0.283796255523079\\
69.5	0.10764	0.394356282849911\\
69.5	0.1113	0.50923636306783\\
69.5	0.11496	0.628436496176831\\
69.5	0.11862	0.751956682176916\\
69.5	0.12228	0.879796921068084\\
69.5	0.12594	1.01195721285033\\
69.5	0.1296	1.14843755752367\\
69.5	0.13326	1.28923795508808\\
69.5	0.13692	1.43435840554358\\
69.5	0.14058	1.58379890889017\\
69.5	0.14424	1.73755946512783\\
69.5	0.1479	1.89564007425658\\
69.5	0.15156	2.05804073627641\\
69.5	0.15522	2.22476145118733\\
69.5	0.15888	2.39580221898932\\
69.5	0.16254	2.57116303968241\\
69.5	0.1662	2.75084391326657\\
69.5	0.16986	2.93484483974182\\
69.5	0.17352	3.12316581910815\\
69.5	0.17718	3.31580685136557\\
69.5	0.18084	3.51276793651406\\
69.5	0.1845	3.71404907455365\\
69.5	0.18816	3.9196502654843\\
69.5	0.19182	4.12957150930606\\
69.5	0.19548	4.34381280601888\\
69.5	0.19914	4.56237415562279\\
69.5	0.2028	4.78525555811779\\
69.5	0.20646	5.01245701350387\\
69.5	0.21012	5.24397852178103\\
69.5	0.21378	5.47982008294928\\
69.5	0.21744	5.7199816970086\\
69.5	0.2211	5.96446336395902\\
69.5	0.22476	6.21326508380052\\
69.5	0.22842	6.46638685653309\\
69.5	0.23208	6.72382868215676\\
69.5	0.23574	6.9855905606715\\
69.5	0.2394	7.25167249207732\\
69.5	0.24306	7.52207447637423\\
69.5	0.24672	7.79679651356223\\
69.5	0.25038	8.07583860364131\\
69.5	0.25404	8.35920074661146\\
69.5	0.2577	8.6468829424727\\
69.5	0.26136	8.93888519122503\\
69.5	0.26502	9.23520749286843\\
69.5	0.26868	9.53584984740293\\
69.5	0.27234	9.84081225482851\\
69.5	0.276	10.1500947151452\\
69.875	0.093	-0.0534284324386887\\
69.875	0.09666	0.0481196264234833\\
69.875	0.10032	0.153987738176742\\
69.875	0.10398	0.264175902821084\\
69.875	0.10764	0.378684120356505\\
69.875	0.1113	0.497512390783014\\
69.875	0.11496	0.620660714100605\\
69.875	0.11862	0.748129090309279\\
69.875	0.12228	0.879917519409033\\
69.875	0.12594	1.01602600139987\\
69.875	0.1296	1.1564545362818\\
69.875	0.13326	1.3012031240548\\
69.875	0.13692	1.45027176471889\\
69.875	0.14058	1.60366045827407\\
69.875	0.14424	1.76136920472032\\
69.875	0.1479	1.92339800405766\\
69.875	0.15156	2.08974685628608\\
69.875	0.15522	2.26041576140559\\
69.875	0.15888	2.43540471941617\\
69.875	0.16254	2.61471373031785\\
69.875	0.1662	2.7983427941106\\
69.875	0.16986	2.98629191079443\\
69.875	0.17352	3.17856108036936\\
69.875	0.17718	3.37515030283536\\
69.875	0.18084	3.57605957819244\\
69.875	0.1845	3.78128890644062\\
69.875	0.18816	3.99083828757987\\
69.875	0.19182	4.20470772161021\\
69.875	0.19548	4.42289720853163\\
69.875	0.19914	4.64540674834413\\
69.875	0.2028	4.87223634104772\\
69.875	0.20646	5.10338598664239\\
69.875	0.21012	5.33885568512814\\
69.875	0.21378	5.57864543650497\\
69.875	0.21744	5.82275524077288\\
69.875	0.2211	6.07118509793189\\
69.875	0.22476	6.32393500798197\\
69.875	0.22842	6.58100497092314\\
69.875	0.23208	6.8423949867554\\
69.875	0.23574	7.10810505547873\\
69.875	0.2394	7.37813517709314\\
69.875	0.24306	7.65248535159864\\
69.875	0.24672	7.93115557899522\\
69.875	0.25038	8.21414585928289\\
69.875	0.25404	8.50145619246164\\
69.875	0.2577	8.79308657853147\\
69.875	0.26136	9.08903701749239\\
69.875	0.26502	9.38930750934439\\
69.875	0.26868	9.69389805408747\\
69.875	0.27234	10.0028086517216\\
69.875	0.276	10.3160393022469\\
70.25	0.093	-0.085938562248983\\
70.25	0.09666	0.0195576868217786\\
70.25	0.10032	0.129373988783627\\
70.25	0.10398	0.243510343636558\\
70.25	0.10764	0.361966751380569\\
70.25	0.1113	0.484743212015667\\
70.25	0.11496	0.611839725541852\\
70.25	0.11862	0.743256291959115\\
70.25	0.12228	0.878992911267459\\
70.25	0.12594	1.01904958346689\\
70.25	0.1296	1.1634263085574\\
70.25	0.13326	1.312123086539\\
70.25	0.13692	1.46513991741168\\
70.25	0.14058	1.62247680117544\\
70.25	0.14424	1.78413373783029\\
70.25	0.1479	1.95011072737621\\
70.25	0.15156	2.12040776981322\\
70.25	0.15522	2.29502486514132\\
70.25	0.15888	2.4739620133605\\
70.25	0.16254	2.65721921447076\\
70.25	0.1662	2.8447964684721\\
70.25	0.16986	3.03669377536453\\
70.25	0.17352	3.23291113514804\\
70.25	0.17718	3.43344854782263\\
70.25	0.18084	3.63830601338831\\
70.25	0.1845	3.84748353184507\\
70.25	0.18816	4.06098110319291\\
70.25	0.19182	4.27879872743184\\
70.25	0.19548	4.50093640456185\\
70.25	0.19914	4.72739413458293\\
70.25	0.2028	4.95817191749512\\
70.25	0.20646	5.19326975329837\\
70.25	0.21012	5.43268764199272\\
70.25	0.21378	5.67642558357814\\
70.25	0.21744	5.92448357805464\\
70.25	0.2211	6.17686162542224\\
70.25	0.22476	6.43355972568091\\
70.25	0.22842	6.69457787883067\\
70.25	0.23208	6.95991608487151\\
70.25	0.23574	7.22957434380343\\
70.25	0.2394	7.50355265562643\\
70.25	0.24306	7.78185102034053\\
70.25	0.24672	8.0644694379457\\
70.25	0.25038	8.35140790844196\\
70.25	0.25404	8.6426664318293\\
70.25	0.2577	8.93824500810771\\
70.25	0.26136	9.23814363727723\\
70.25	0.26502	9.54236231933781\\
70.25	0.26868	9.85090105428948\\
70.25	0.27234	10.1637598421322\\
70.25	0.276	10.4809386828661\\
70.625	0.093	-0.119493898541811\\
70.625	0.09666	-0.0100494592624598\\
70.625	0.10032	0.103715032907978\\
70.625	0.10398	0.221799577969499\\
70.625	0.10764	0.3442041759221\\
70.625	0.1113	0.470928826765787\\
70.625	0.11496	0.601973530500558\\
70.625	0.11862	0.737338287126411\\
70.625	0.12228	0.877023096643348\\
70.625	0.12594	1.02102795905137\\
70.625	0.1296	1.16935287435047\\
70.625	0.13326	1.32199784254065\\
70.625	0.13692	1.47896286362192\\
70.625	0.14058	1.64024793759427\\
70.625	0.14424	1.80585306445771\\
70.625	0.1479	1.97577824421223\\
70.625	0.15156	2.15002347685783\\
70.625	0.15522	2.32858876239451\\
70.625	0.15888	2.51147410082228\\
70.625	0.16254	2.69867949214113\\
70.625	0.1662	2.89020493635106\\
70.625	0.16986	3.08605043345208\\
70.625	0.17352	3.28621598344418\\
70.625	0.17718	3.49070158632736\\
70.625	0.18084	3.69950724210162\\
70.625	0.1845	3.91263295076698\\
70.625	0.18816	4.13007871232341\\
70.625	0.19182	4.35184452677093\\
70.625	0.19548	4.57793039410953\\
70.625	0.19914	4.8083363143392\\
70.625	0.2028	5.04306228745998\\
70.625	0.20646	5.28210831347182\\
70.625	0.21012	5.52547439237476\\
70.625	0.21378	5.77316052416877\\
70.625	0.21744	6.02516670885386\\
70.625	0.2211	6.28149294643004\\
70.625	0.22476	6.54213923689731\\
70.625	0.22842	6.80710558025565\\
70.625	0.23208	7.07639197650508\\
70.625	0.23574	7.3499984256456\\
70.625	0.2394	7.62792492767719\\
70.625	0.24306	7.91017148259987\\
70.625	0.24672	8.19673809041363\\
70.625	0.25038	8.48762475111849\\
70.625	0.25404	8.78283146471441\\
70.625	0.2577	9.08235823120142\\
70.625	0.26136	9.38620505057952\\
70.625	0.26502	9.69437192284869\\
70.625	0.26868	10.006858848009\\
70.625	0.27234	10.3236658260603\\
70.625	0.276	10.6447928570027\\
71	0.093	-0.154094441317162\\
71	0.09666	-0.0407018118292213\\
71	0.10032	0.0770108705498065\\
71	0.10398	0.199043605819917\\
71	0.10764	0.325396393981111\\
71	0.1113	0.456069235033388\\
71	0.11496	0.591062128976748\\
71	0.11862	0.730375075811191\\
71	0.12228	0.874008075536714\\
71	0.12594	1.02196112815333\\
71	0.1296	1.17423423366102\\
71	0.13326	1.33082739205979\\
71	0.13692	1.49174060334965\\
71	0.14058	1.65697386753059\\
71	0.14424	1.82652718460262\\
71	0.1479	2.00040055456572\\
71	0.15156	2.17859397741992\\
71	0.15522	2.36110745316519\\
71	0.15888	2.54794098180154\\
71	0.16254	2.73909456332899\\
71	0.1662	2.93456819774751\\
71	0.16986	3.13436188505711\\
71	0.17352	3.33847562525781\\
71	0.17718	3.54690941834957\\
71	0.18084	3.75966326433243\\
71	0.1845	3.97673716320638\\
71	0.18816	4.19813111497139\\
71	0.19182	4.42384511962751\\
71	0.19548	4.65387917717469\\
71	0.19914	4.88823328761296\\
71	0.2028	5.12690745094232\\
71	0.20646	5.36990166716275\\
71	0.21012	5.61721593627427\\
71	0.21378	5.86885025827688\\
71	0.21744	6.12480463317057\\
71	0.2211	6.38507906095534\\
71	0.22476	6.64967354163119\\
71	0.22842	6.91858807519812\\
71	0.23208	7.19182266165615\\
71	0.23574	7.46937730100524\\
71	0.2394	7.75125199324543\\
71	0.24306	8.0374467383767\\
71	0.24672	8.32796153639905\\
71	0.25038	8.62279638731249\\
71	0.25404	8.921951291117\\
71	0.2577	9.2254262478126\\
71	0.26136	9.5332212573993\\
71	0.26502	9.84533631987706\\
71	0.26868	10.1617714352459\\
71	0.27234	10.4825266035058\\
71	0.276	10.8076018246569\\
71.375	0.093	-0.189740190575043\\
71.375	0.09666	-0.0723993708785129\\
71.375	0.10032	0.0492615017091045\\
71.375	0.10398	0.175242427187804\\
71.375	0.10764	0.305543405557584\\
71.375	0.1113	0.440164436818451\\
71.375	0.11496	0.579105520970404\\
71.375	0.11862	0.722366658013437\\
71.375	0.12228	0.869947847947549\\
71.375	0.12594	1.02184909077275\\
71.375	0.1296	1.17807038648903\\
71.375	0.13326	1.33861173509639\\
71.375	0.13692	1.50347313659484\\
71.375	0.14058	1.67265459098438\\
71.375	0.14424	1.84615609826499\\
71.375	0.1479	2.02397765843668\\
71.375	0.15156	2.20611927149946\\
71.375	0.15522	2.39258093745333\\
71.375	0.15888	2.58336265629827\\
71.375	0.16254	2.77846442803431\\
71.375	0.1662	2.97788625266142\\
71.375	0.16986	3.18162813017961\\
71.375	0.17352	3.3896900605889\\
71.375	0.17718	3.60207204388925\\
71.375	0.18084	3.8187740800807\\
71.375	0.1845	4.03979616916323\\
71.375	0.18816	4.26513831113684\\
71.375	0.19182	4.49480050600154\\
71.375	0.19548	4.72878275375732\\
71.375	0.19914	4.96708505440418\\
71.375	0.2028	5.20970740794212\\
71.375	0.20646	5.45664981437115\\
71.375	0.21012	5.70791227369126\\
71.375	0.21378	5.96349478590245\\
71.375	0.21744	6.22339735100473\\
71.375	0.2211	6.48761996899809\\
71.375	0.22476	6.75616263988253\\
71.375	0.22842	7.02902536365805\\
71.375	0.23208	7.30620814032467\\
71.375	0.23574	7.58771096988236\\
71.375	0.2394	7.87353385233113\\
71.375	0.24306	8.163676787671\\
71.375	0.24672	8.45813977590193\\
71.375	0.25038	8.75692281702397\\
71.375	0.25404	9.06002591103707\\
71.375	0.2577	9.36744905794126\\
71.375	0.26136	9.67919225773653\\
71.375	0.26502	9.99525551042289\\
71.375	0.26868	10.3156388160003\\
71.375	0.27234	10.6403421744689\\
71.375	0.276	10.9693655858285\\
71.75	0.093	-0.226431146315458\\
71.75	0.09666	-0.105142136410335\\
71.75	0.10032	0.0204669263858723\\
71.75	0.10398	0.150396042073162\\
71.75	0.10764	0.284645210651531\\
71.75	0.1113	0.423214432120988\\
71.75	0.11496	0.566103706481527\\
71.75	0.11862	0.713313033733153\\
71.75	0.12228	0.864842413875855\\
71.75	0.12594	1.02069184690964\\
71.75	0.1296	1.18086133283451\\
71.75	0.13326	1.34535087165047\\
71.75	0.13692	1.51416046335751\\
71.75	0.14058	1.68729010795563\\
71.75	0.14424	1.86473980544483\\
71.75	0.1479	2.04650955582512\\
71.75	0.15156	2.23259935909649\\
71.75	0.15522	2.42300921525894\\
71.75	0.15888	2.61773912431248\\
71.75	0.16254	2.8167890862571\\
71.75	0.1662	3.0201591010928\\
71.75	0.16986	3.22784916881958\\
71.75	0.17352	3.43985928943745\\
71.75	0.17718	3.6561894629464\\
71.75	0.18084	3.87683968934644\\
71.75	0.1845	4.10180996863756\\
71.75	0.18816	4.33110030081976\\
71.75	0.19182	4.56471068589305\\
71.75	0.19548	4.80264112385741\\
71.75	0.19914	5.04489161471286\\
71.75	0.2028	5.2914621584594\\
71.75	0.20646	5.54235275509702\\
71.75	0.21012	5.79756340462572\\
71.75	0.21378	6.0570941070455\\
71.75	0.21744	6.32094486235636\\
71.75	0.2211	6.58911567055832\\
71.75	0.22476	6.86160653165135\\
71.75	0.22842	7.13841744563546\\
71.75	0.23208	7.41954841251066\\
71.75	0.23574	7.70499943227694\\
71.75	0.2394	7.9947705049343\\
71.75	0.24306	8.28886163048275\\
71.75	0.24672	8.58727280892228\\
71.75	0.25038	8.8900040402529\\
71.75	0.25404	9.1970553244746\\
71.75	0.2577	9.50842666158738\\
71.75	0.26136	9.82411805159124\\
71.75	0.26502	10.1441294944862\\
71.75	0.26868	10.4684609902722\\
71.75	0.27234	10.7971125389493\\
71.75	0.276	11.1300841405175\\
72.125	0.093	-0.264167308538396\\
72.125	0.09666	-0.138930108424687\\
72.125	0.10032	-0.00937285541988997\\
72.125	0.10398	0.124504450475989\\
72.125	0.10764	0.262701809262952\\
72.125	0.1113	0.405219220940998\\
72.125	0.11496	0.552056685510127\\
72.125	0.11862	0.703214202970338\\
72.125	0.12228	0.85869177332163\\
72.125	0.12594	1.01848939656401\\
72.125	0.1296	1.18260707269747\\
72.125	0.13326	1.35104480172201\\
72.125	0.13692	1.52380258363764\\
72.125	0.14058	1.70088041844435\\
72.125	0.14424	1.88227830614215\\
72.125	0.1479	2.06799624673102\\
72.125	0.15156	2.25803424021098\\
72.125	0.15522	2.45239228658203\\
72.125	0.15888	2.65107038584415\\
72.125	0.16254	2.85406853799736\\
72.125	0.1662	3.06138674304166\\
72.125	0.16986	3.27302500097702\\
72.125	0.17352	3.48898331180349\\
72.125	0.17718	3.70926167552103\\
72.125	0.18084	3.93386009212965\\
72.125	0.1845	4.16277856162936\\
72.125	0.18816	4.39601708402015\\
72.125	0.19182	4.63357565930203\\
72.125	0.19548	4.87545428747499\\
72.125	0.19914	5.12165296853902\\
72.125	0.2028	5.37217170249415\\
72.125	0.20646	5.62701048934035\\
72.125	0.21012	5.88616932907765\\
72.125	0.21378	6.14964822170602\\
72.125	0.21744	6.41744716722547\\
72.125	0.2211	6.68956616563602\\
72.125	0.22476	6.96600521693763\\
72.125	0.22842	7.24676432113034\\
72.125	0.23208	7.53184347821413\\
72.125	0.23574	7.821242688189\\
72.125	0.2394	8.11496195105495\\
72.125	0.24306	8.41300126681199\\
72.125	0.24672	8.71536063546011\\
72.125	0.25038	9.02204005699932\\
72.125	0.25404	9.33303953142961\\
72.125	0.2577	9.64835905875098\\
72.125	0.26136	9.96799863896343\\
72.125	0.26502	10.291958272067\\
72.125	0.26868	10.6202379580616\\
72.125	0.27234	10.9528376969473\\
72.125	0.276	11.2897574887241\\
72.5	0.093	-0.302948677243864\\
72.5	0.09666	-0.173763286921565\\
72.5	0.10032	-0.0402578437081789\\
72.5	0.10398	0.0975676523962898\\
72.5	0.10764	0.239713201391839\\
72.5	0.1113	0.386178803278474\\
72.5	0.11496	0.536964458056196\\
72.5	0.11862	0.692070165724997\\
72.5	0.12228	0.851495926284879\\
72.5	0.12594	1.01524173973585\\
72.5	0.1296	1.1833076060779\\
72.5	0.13326	1.35569352531103\\
72.5	0.13692	1.53239949743525\\
72.5	0.14058	1.71342552245055\\
72.5	0.14424	1.89877160035693\\
72.5	0.1479	2.0884377311544\\
72.5	0.15156	2.28242391484295\\
72.5	0.15522	2.48073015142258\\
72.5	0.15888	2.68335644089329\\
72.5	0.16254	2.8903027832551\\
72.5	0.1662	3.10156917850798\\
72.5	0.16986	3.31715562665194\\
72.5	0.17352	3.53706212768699\\
72.5	0.17718	3.76128868161312\\
72.5	0.18084	3.98983528843033\\
72.5	0.1845	4.22270194813864\\
72.5	0.18816	4.45988866073801\\
72.5	0.19182	4.70139542622848\\
72.5	0.19548	4.94722224461002\\
72.5	0.19914	5.19736911588265\\
72.5	0.2028	5.45183604004637\\
72.5	0.20646	5.71062301710116\\
72.5	0.21012	5.97373004704705\\
72.5	0.21378	6.241157129884\\
72.5	0.21744	6.51290426561205\\
72.5	0.2211	6.78897145423119\\
72.5	0.22476	7.06935869574139\\
72.5	0.22842	7.35406599014268\\
72.5	0.23208	7.64309333743506\\
72.5	0.23574	7.93644073761853\\
72.5	0.2394	8.23410819069307\\
72.5	0.24306	8.5360956966587\\
72.5	0.24672	8.84240325551541\\
72.5	0.25038	9.15303086726321\\
72.5	0.25404	9.46797853190208\\
72.5	0.2577	9.78724624943204\\
72.5	0.26136	10.1108340198531\\
72.5	0.26502	10.4387418431652\\
72.5	0.26868	10.7709697193684\\
72.5	0.27234	11.1075176484627\\
72.5	0.276	11.4483856304481\\
72.875	0.093	-0.342775252431855\\
72.875	0.09666	-0.209641671900967\\
72.875	0.10032	-0.0721880384789908\\
72.875	0.10398	0.0695856478340711\\
72.875	0.10764	0.215679387038209\\
72.875	0.1113	0.366093179133435\\
72.875	0.11496	0.520827024119743\\
72.875	0.11862	0.679880921997134\\
72.875	0.12228	0.843254872765608\\
72.875	0.12594	1.01094887642517\\
72.875	0.1296	1.18296293297581\\
72.875	0.13326	1.35929704241753\\
72.875	0.13692	1.53995120475034\\
72.875	0.14058	1.72492541997423\\
72.875	0.14424	1.9142196880892\\
72.875	0.1479	2.10783400909525\\
72.875	0.15156	2.30576838299239\\
72.875	0.15522	2.50802280978061\\
72.875	0.15888	2.71459728945992\\
72.875	0.16254	2.92549182203032\\
72.875	0.1662	3.14070640749178\\
72.875	0.16986	3.36024104584433\\
72.875	0.17352	3.58409573708797\\
72.875	0.17718	3.81227048122269\\
72.875	0.18084	4.0447652782485\\
72.875	0.1845	4.28158012816538\\
72.875	0.18816	4.52271503097336\\
72.875	0.19182	4.76816998667241\\
72.875	0.19548	5.01794499526255\\
72.875	0.19914	5.27204005674377\\
72.875	0.2028	5.53045517111607\\
72.875	0.20646	5.79319033837946\\
72.875	0.21012	6.06024555853392\\
72.875	0.21378	6.33162083157948\\
72.875	0.21744	6.60731615751611\\
72.875	0.2211	6.88733153634383\\
72.875	0.22476	7.17166696806263\\
72.875	0.22842	7.46032245267252\\
72.875	0.23208	7.75329799017348\\
72.875	0.23574	8.05059358056554\\
72.875	0.2394	8.35220922384866\\
72.875	0.24306	8.65814492002289\\
72.875	0.24672	8.96840066908819\\
72.875	0.25038	9.28297647104457\\
72.875	0.25404	9.60187232589204\\
72.875	0.2577	9.92508823363058\\
72.875	0.26136	10.2526241942602\\
72.875	0.26502	10.5844802077809\\
72.875	0.26868	10.9206562741927\\
72.875	0.27234	11.2611523934956\\
72.875	0.276	11.6059685656896\\
73.25	0.093	-0.383647034102384\\
73.25	0.09666	-0.246565263362905\\
73.25	0.10032	-0.10516343973234\\
73.25	0.10398	0.0405584367893079\\
73.25	0.10764	0.190600366202039\\
73.25	0.1113	0.344962348505854\\
73.25	0.11496	0.503644383700752\\
73.25	0.11862	0.666646471786732\\
73.25	0.12228	0.833968612763793\\
73.25	0.12594	1.00561080663194\\
73.25	0.1296	1.18157305339117\\
73.25	0.13326	1.36185535304148\\
73.25	0.13692	1.54645770558288\\
73.25	0.14058	1.73538011101536\\
73.25	0.14424	1.92862256933893\\
73.25	0.1479	2.12618508055357\\
73.25	0.15156	2.3280676446593\\
73.25	0.15522	2.53427026165611\\
73.25	0.15888	2.744792931544\\
73.25	0.16254	2.95963565432298\\
73.25	0.1662	3.17879842999304\\
73.25	0.16986	3.40228125855419\\
73.25	0.17352	3.63008414000642\\
73.25	0.17718	3.86220707434973\\
73.25	0.18084	4.09865006158412\\
73.25	0.1845	4.3394131017096\\
73.25	0.18816	4.58449619472616\\
73.25	0.19182	4.8338993406338\\
73.25	0.19548	5.08762253943253\\
73.25	0.19914	5.34566579112234\\
73.25	0.2028	5.60802909570323\\
73.25	0.20646	5.87471245317521\\
73.25	0.21012	6.14571586353826\\
73.25	0.21378	6.42103932679241\\
73.25	0.21744	6.70068284293763\\
73.25	0.2211	6.98464641197394\\
73.25	0.22476	7.27293003390133\\
73.25	0.22842	7.5655337087198\\
73.25	0.23208	7.86245743642936\\
73.25	0.23574	8.16370121703\\
73.25	0.2394	8.46926505052172\\
73.25	0.24306	8.77914893690453\\
73.25	0.24672	9.09335287617842\\
73.25	0.25038	9.4118768683434\\
73.25	0.25404	9.73472091339945\\
73.25	0.2577	10.0618850113466\\
73.25	0.26136	10.3933691621848\\
73.25	0.26502	10.7291733659141\\
73.25	0.26868	11.0692976225345\\
73.25	0.27234	11.413741932046\\
73.25	0.276	11.7625062944485\\
73.625	0.093	-0.425564022255435\\
73.625	0.09666	-0.284534061307367\\
73.625	0.10032	-0.139184047468212\\
73.625	0.10398	0.0104860192620289\\
73.625	0.10764	0.164476138883346\\
73.625	0.1113	0.322786311395751\\
73.625	0.11496	0.485416536799242\\
73.625	0.11862	0.652366815093812\\
73.625	0.12228	0.823637146279462\\
73.625	0.12594	0.999227530356198\\
73.625	0.1296	1.17913796732402\\
73.625	0.13326	1.36336845718292\\
73.625	0.13692	1.55191899993291\\
73.625	0.14058	1.74478959557398\\
73.625	0.14424	1.94198024410613\\
73.625	0.1479	2.14349094552936\\
73.625	0.15156	2.34932169984368\\
73.625	0.15522	2.55947250704909\\
73.625	0.15888	2.77394336714557\\
73.625	0.16254	2.99273428013314\\
73.625	0.1662	3.21584524601179\\
73.625	0.16986	3.44327626478152\\
73.625	0.17352	3.67502733644234\\
73.625	0.17718	3.91109846099424\\
73.625	0.18084	4.15148963843722\\
73.625	0.1845	4.39620086877129\\
73.625	0.18816	4.64523215199644\\
73.625	0.19182	4.89858348811268\\
73.625	0.19548	5.15625487711999\\
73.625	0.19914	5.41824631901838\\
73.625	0.2028	5.68455781380787\\
73.625	0.20646	5.95518936148843\\
73.625	0.21012	6.23014096206009\\
73.625	0.21378	6.50941261552282\\
73.625	0.21744	6.79300432187663\\
73.625	0.2211	7.08091608112153\\
73.625	0.22476	7.37314789325751\\
73.625	0.22842	7.66969975828457\\
73.625	0.23208	7.97057167620272\\
73.625	0.23574	8.27576364701195\\
73.625	0.2394	8.58527567071226\\
73.625	0.24306	8.89910774730366\\
73.625	0.24672	9.21725987678614\\
73.625	0.25038	9.53973205915971\\
73.625	0.25404	9.86652429442436\\
73.625	0.2577	10.1976365825801\\
73.625	0.26136	10.5330689236269\\
73.625	0.26502	10.8728213175648\\
73.625	0.26868	11.2168937643938\\
73.625	0.27234	11.5652862641138\\
73.625	0.276	11.917998816725\\
74	0.093	-0.468526216891017\\
74	0.09666	-0.323548065734359\\
74	0.10032	-0.174249861686615\\
74	0.10398	-0.0206316047477875\\
74	0.10764	0.137306705082123\\
74	0.1113	0.299565067803117\\
74	0.11496	0.466143483415194\\
74	0.11862	0.637041951918357\\
74	0.12228	0.812260473312597\\
74	0.12594	0.991799047597923\\
74	0.1296	1.17565767477433\\
74	0.13326	1.36383635484182\\
74	0.13692	1.5563350878004\\
74	0.14058	1.75315387365006\\
74	0.14424	1.9542927123908\\
74	0.1479	2.15975160402263\\
74	0.15156	2.36953054854553\\
74	0.15522	2.58362954595953\\
74	0.15888	2.8020485962646\\
74	0.16254	3.02478769946076\\
74	0.1662	3.251846855548\\
74	0.16986	3.48322606452632\\
74	0.17352	3.71892532639573\\
74	0.17718	3.95894464115622\\
74	0.18084	4.20328400880779\\
74	0.1845	4.45194342935045\\
74	0.18816	4.70492290278418\\
74	0.19182	4.96222242910901\\
74	0.19548	5.22384200832491\\
74	0.19914	5.4897816404319\\
74	0.2028	5.76004132542998\\
74	0.20646	6.03462106331913\\
74	0.21012	6.31352085409937\\
74	0.21378	6.59674069777069\\
74	0.21744	6.88428059433309\\
74	0.2211	7.17614054378658\\
74	0.22476	7.47232054613116\\
74	0.22842	7.7728206013668\\
74	0.23208	8.07764070949355\\
74	0.23574	8.38678087051136\\
74	0.2394	8.70024108442026\\
74	0.24306	9.01802135122026\\
74	0.24672	9.34012167091132\\
74	0.25038	9.66654204349348\\
74	0.25404	9.99728246896671\\
74	0.2577	10.332342947331\\
74	0.26136	10.6717234785864\\
74	0.26502	11.0154240627329\\
74	0.26868	11.3634446997705\\
74	0.27234	11.7157853896991\\
74	0.276	12.0724461325189\\
};
\end{axis}

\begin{axis}[%
width=4.527496cm,
height=3.870968cm,
at={(0cm,16.129032cm)},
scale only axis,
xmin=56,
xmax=74,
tick align=outside,
xlabel={$L_{cut}$},
xmajorgrids,
ymin=0.093,
ymax=0.276,
ylabel={$D_{rlx}$},
ymajorgrids,
zmin=-200.569447084128,
zmax=0,
zlabel={$x_4,x_4$},
zmajorgrids,
view={-140}{50},
legend style={at={(1.03,1)},anchor=north west,legend cell align=left,align=left,draw=white!15!black}
]
\addplot3[only marks,mark=*,mark options={},mark size=1.5000pt,color=mycolor1] plot table[row sep=crcr,]{%
74	0.123	-26.0353957891804\\
72	0.113	-20.9879322169279\\
61	0.095	-10.6920630070547\\
56	0.093	-10.9569379219045\\
};
\addplot3[only marks,mark=*,mark options={},mark size=1.5000pt,color=mycolor2] plot table[row sep=crcr,]{%
67	0.276	-191.779551108501\\
66	0.255	-157.643535964989\\
62	0.209	-87.4196213968237\\
57	0.193	-69.8013569503948\\
};
\addplot3[only marks,mark=*,mark options={},mark size=1.5000pt,color=black] plot table[row sep=crcr,]{%
69	0.104	-15.5770582620907\\
};
\addplot3[only marks,mark=*,mark options={},mark size=1.5000pt,color=black] plot table[row sep=crcr,]{%
64	0.23	-116.694090391038\\
};

\addplot3[%
surf,
opacity=0.7,
shader=interp,
colormap={mymap}{[1pt] rgb(0pt)=(0.0901961,0.239216,0.0745098); rgb(1pt)=(0.0945149,0.242058,0.0739522); rgb(2pt)=(0.0988592,0.244894,0.0733566); rgb(3pt)=(0.103229,0.247724,0.0727241); rgb(4pt)=(0.107623,0.250549,0.0720557); rgb(5pt)=(0.112043,0.253367,0.0713525); rgb(6pt)=(0.116487,0.25618,0.0706154); rgb(7pt)=(0.120956,0.258986,0.0698456); rgb(8pt)=(0.125449,0.261787,0.0690441); rgb(9pt)=(0.129967,0.264581,0.0682118); rgb(10pt)=(0.134508,0.26737,0.06735); rgb(11pt)=(0.139074,0.270152,0.0664596); rgb(12pt)=(0.143663,0.272929,0.0655416); rgb(13pt)=(0.148275,0.275699,0.0645971); rgb(14pt)=(0.152911,0.278463,0.0636271); rgb(15pt)=(0.15757,0.281221,0.0626328); rgb(16pt)=(0.162252,0.283973,0.0616151); rgb(17pt)=(0.166957,0.286719,0.060575); rgb(18pt)=(0.171685,0.289458,0.0595136); rgb(19pt)=(0.176434,0.292191,0.0584321); rgb(20pt)=(0.181207,0.294918,0.0573313); rgb(21pt)=(0.186001,0.297639,0.0562123); rgb(22pt)=(0.190817,0.300353,0.0550763); rgb(23pt)=(0.195655,0.303061,0.0539242); rgb(24pt)=(0.200514,0.305763,0.052757); rgb(25pt)=(0.205395,0.308459,0.0515759); rgb(26pt)=(0.210296,0.311149,0.0503624); rgb(27pt)=(0.215212,0.313846,0.0490067); rgb(28pt)=(0.220142,0.316548,0.0475043); rgb(29pt)=(0.22509,0.319254,0.0458704); rgb(30pt)=(0.230056,0.321962,0.0441205); rgb(31pt)=(0.235042,0.324671,0.04227); rgb(32pt)=(0.240048,0.327379,0.0403343); rgb(33pt)=(0.245078,0.330085,0.0383287); rgb(34pt)=(0.250131,0.332786,0.0362688); rgb(35pt)=(0.25521,0.335482,0.0341698); rgb(36pt)=(0.260317,0.33817,0.0320472); rgb(37pt)=(0.265451,0.340849,0.0299163); rgb(38pt)=(0.270616,0.343517,0.0277927); rgb(39pt)=(0.275813,0.346172,0.0256916); rgb(40pt)=(0.281043,0.348814,0.0236284); rgb(41pt)=(0.286307,0.35144,0.0216186); rgb(42pt)=(0.291607,0.354048,0.0196776); rgb(43pt)=(0.296945,0.356637,0.0178207); rgb(44pt)=(0.302322,0.359206,0.0160634); rgb(45pt)=(0.307739,0.361753,0.0144211); rgb(46pt)=(0.313198,0.364275,0.0129091); rgb(47pt)=(0.318701,0.366772,0.0115428); rgb(48pt)=(0.324249,0.369242,0.0103377); rgb(49pt)=(0.329843,0.371682,0.00930909); rgb(50pt)=(0.335485,0.374093,0.00847245); rgb(51pt)=(0.341176,0.376471,0.00784314); rgb(52pt)=(0.346925,0.378826,0.00732741); rgb(53pt)=(0.352735,0.381168,0.00682184); rgb(54pt)=(0.358605,0.383497,0.00632729); rgb(55pt)=(0.364532,0.385812,0.00584464); rgb(56pt)=(0.370516,0.388113,0.00537476); rgb(57pt)=(0.376552,0.390399,0.00491852); rgb(58pt)=(0.38264,0.39267,0.00447681); rgb(59pt)=(0.388777,0.394925,0.00405048); rgb(60pt)=(0.394962,0.397164,0.00364042); rgb(61pt)=(0.401191,0.399386,0.00324749); rgb(62pt)=(0.407464,0.401592,0.00287258); rgb(63pt)=(0.413777,0.40378,0.00251655); rgb(64pt)=(0.420129,0.40595,0.00218028); rgb(65pt)=(0.426518,0.408102,0.00186463); rgb(66pt)=(0.432942,0.410234,0.00157049); rgb(67pt)=(0.439399,0.412348,0.00129873); rgb(68pt)=(0.445885,0.414441,0.00105022); rgb(69pt)=(0.452401,0.416515,0.000825833); rgb(70pt)=(0.458942,0.418567,0.000626441); rgb(71pt)=(0.465508,0.420599,0.00045292); rgb(72pt)=(0.472096,0.422609,0.000306141); rgb(73pt)=(0.478704,0.424596,0.000186979); rgb(74pt)=(0.485331,0.426562,9.63073e-05); rgb(75pt)=(0.491973,0.428504,3.49981e-05); rgb(76pt)=(0.498628,0.430422,3.92506e-06); rgb(77pt)=(0.505323,0.432315,0); rgb(78pt)=(0.512206,0.434168,0); rgb(79pt)=(0.519282,0.435983,0); rgb(80pt)=(0.526529,0.437764,0); rgb(81pt)=(0.533922,0.439512,0); rgb(82pt)=(0.54144,0.441232,0); rgb(83pt)=(0.549059,0.442927,0); rgb(84pt)=(0.556756,0.444599,0); rgb(85pt)=(0.564508,0.446252,0); rgb(86pt)=(0.572292,0.447889,0); rgb(87pt)=(0.580084,0.449514,0); rgb(88pt)=(0.587863,0.451129,0); rgb(89pt)=(0.595604,0.452737,0); rgb(90pt)=(0.603284,0.454343,0); rgb(91pt)=(0.610882,0.455948,0); rgb(92pt)=(0.618373,0.457556,0); rgb(93pt)=(0.625734,0.459171,0); rgb(94pt)=(0.632943,0.460795,0); rgb(95pt)=(0.639976,0.462432,0); rgb(96pt)=(0.64681,0.464084,0); rgb(97pt)=(0.653423,0.465756,0); rgb(98pt)=(0.659791,0.46745,0); rgb(99pt)=(0.665891,0.469169,0); rgb(100pt)=(0.6717,0.470916,0); rgb(101pt)=(0.677195,0.472696,0); rgb(102pt)=(0.682353,0.47451,0); rgb(103pt)=(0.687242,0.476355,0); rgb(104pt)=(0.691952,0.478225,0); rgb(105pt)=(0.696497,0.480118,0); rgb(106pt)=(0.700887,0.482033,0); rgb(107pt)=(0.705134,0.483968,0); rgb(108pt)=(0.709251,0.485921,0); rgb(109pt)=(0.713249,0.487891,0); rgb(110pt)=(0.71714,0.489876,0); rgb(111pt)=(0.720936,0.491875,0); rgb(112pt)=(0.724649,0.493887,0); rgb(113pt)=(0.72829,0.495909,0); rgb(114pt)=(0.731872,0.49794,0); rgb(115pt)=(0.735406,0.499979,0); rgb(116pt)=(0.738904,0.502025,0); rgb(117pt)=(0.742378,0.504075,0); rgb(118pt)=(0.74584,0.506128,0); rgb(119pt)=(0.749302,0.508182,0); rgb(120pt)=(0.752775,0.510237,0); rgb(121pt)=(0.756272,0.51229,0); rgb(122pt)=(0.759804,0.514339,0); rgb(123pt)=(0.763384,0.516385,0); rgb(124pt)=(0.767022,0.518424,0); rgb(125pt)=(0.770731,0.520455,0); rgb(126pt)=(0.774523,0.522478,0); rgb(127pt)=(0.77841,0.524489,0); rgb(128pt)=(0.782391,0.526491,0); rgb(129pt)=(0.786402,0.528496,0); rgb(130pt)=(0.790431,0.530506,0); rgb(131pt)=(0.794478,0.532521,0); rgb(132pt)=(0.798541,0.534539,0); rgb(133pt)=(0.802619,0.53656,0); rgb(134pt)=(0.806712,0.538584,0); rgb(135pt)=(0.81082,0.540609,0); rgb(136pt)=(0.81494,0.542635,0); rgb(137pt)=(0.819074,0.54466,0); rgb(138pt)=(0.823219,0.546686,0); rgb(139pt)=(0.827374,0.548709,0); rgb(140pt)=(0.831541,0.55073,0); rgb(141pt)=(0.835716,0.552749,0); rgb(142pt)=(0.8399,0.554763,0); rgb(143pt)=(0.844092,0.556774,0); rgb(144pt)=(0.848292,0.558779,0); rgb(145pt)=(0.852497,0.560778,0); rgb(146pt)=(0.856708,0.562771,0); rgb(147pt)=(0.860924,0.564756,0); rgb(148pt)=(0.865143,0.566733,0); rgb(149pt)=(0.869366,0.568701,0); rgb(150pt)=(0.873592,0.57066,0); rgb(151pt)=(0.877819,0.572608,0); rgb(152pt)=(0.882047,0.574545,0); rgb(153pt)=(0.886275,0.576471,0); rgb(154pt)=(0.890659,0.578362,0); rgb(155pt)=(0.895333,0.580203,0); rgb(156pt)=(0.900258,0.581999,0); rgb(157pt)=(0.905397,0.583755,0); rgb(158pt)=(0.910711,0.585479,0); rgb(159pt)=(0.916164,0.587176,0); rgb(160pt)=(0.921717,0.588852,0); rgb(161pt)=(0.927333,0.590513,0); rgb(162pt)=(0.932974,0.592166,0); rgb(163pt)=(0.938602,0.593815,0); rgb(164pt)=(0.94418,0.595468,0); rgb(165pt)=(0.949669,0.59713,0); rgb(166pt)=(0.955033,0.598808,0); rgb(167pt)=(0.960233,0.600507,0); rgb(168pt)=(0.965232,0.602233,0); rgb(169pt)=(0.969992,0.603992,0); rgb(170pt)=(0.974475,0.605791,0); rgb(171pt)=(0.978643,0.607636,0); rgb(172pt)=(0.98246,0.609532,0); rgb(173pt)=(0.985886,0.611486,0); rgb(174pt)=(0.988885,0.613503,0); rgb(175pt)=(0.991419,0.61559,0); rgb(176pt)=(0.99345,0.617753,0); rgb(177pt)=(0.99494,0.619997,0); rgb(178pt)=(0.995851,0.622329,0); rgb(179pt)=(0.996226,0.624763,0); rgb(180pt)=(0.996512,0.627352,0); rgb(181pt)=(0.996788,0.630095,0); rgb(182pt)=(0.997053,0.632982,0); rgb(183pt)=(0.997308,0.636004,0); rgb(184pt)=(0.997552,0.639152,0); rgb(185pt)=(0.997785,0.642416,0); rgb(186pt)=(0.998006,0.645786,0); rgb(187pt)=(0.998217,0.649253,0); rgb(188pt)=(0.998416,0.652807,0); rgb(189pt)=(0.998605,0.656439,0); rgb(190pt)=(0.998781,0.660138,0); rgb(191pt)=(0.998946,0.663897,0); rgb(192pt)=(0.9991,0.667704,0); rgb(193pt)=(0.999242,0.67155,0); rgb(194pt)=(0.999372,0.675427,0); rgb(195pt)=(0.99949,0.679323,0); rgb(196pt)=(0.999596,0.68323,0); rgb(197pt)=(0.99969,0.687139,0); rgb(198pt)=(0.999771,0.691039,0); rgb(199pt)=(0.999841,0.694921,0); rgb(200pt)=(0.999898,0.698775,0); rgb(201pt)=(0.999942,0.702592,0); rgb(202pt)=(0.999974,0.706363,0); rgb(203pt)=(0.999994,0.710077,0); rgb(204pt)=(1,0.713725,0); rgb(205pt)=(1,0.717341,0); rgb(206pt)=(1,0.720963,0); rgb(207pt)=(1,0.724591,0); rgb(208pt)=(1,0.728226,0); rgb(209pt)=(1,0.731867,0); rgb(210pt)=(1,0.735514,0); rgb(211pt)=(1,0.739167,0); rgb(212pt)=(1,0.742827,0); rgb(213pt)=(1,0.746493,0); rgb(214pt)=(1,0.750165,0); rgb(215pt)=(1,0.753843,0); rgb(216pt)=(1,0.757527,0); rgb(217pt)=(1,0.761217,0); rgb(218pt)=(1,0.764913,0); rgb(219pt)=(1,0.768615,0); rgb(220pt)=(1,0.772324,0); rgb(221pt)=(1,0.776038,0); rgb(222pt)=(1,0.779758,0); rgb(223pt)=(1,0.783484,0); rgb(224pt)=(1,0.787215,0); rgb(225pt)=(1,0.790953,0); rgb(226pt)=(1,0.794696,0); rgb(227pt)=(1,0.798445,0); rgb(228pt)=(1,0.8022,0); rgb(229pt)=(1,0.805961,0); rgb(230pt)=(1,0.809727,0); rgb(231pt)=(1,0.8135,0); rgb(232pt)=(1,0.817278,0); rgb(233pt)=(1,0.821063,0); rgb(234pt)=(1,0.824854,0); rgb(235pt)=(1,0.828652,0); rgb(236pt)=(1,0.832455,0); rgb(237pt)=(1,0.836265,0); rgb(238pt)=(1,0.840081,0); rgb(239pt)=(1,0.843903,0); rgb(240pt)=(1,0.847732,0); rgb(241pt)=(1,0.851566,0); rgb(242pt)=(1,0.855406,0); rgb(243pt)=(1,0.859253,0); rgb(244pt)=(1,0.863106,0); rgb(245pt)=(1,0.866964,0); rgb(246pt)=(1,0.870829,0); rgb(247pt)=(1,0.8747,0); rgb(248pt)=(1,0.878577,0); rgb(249pt)=(1,0.88246,0); rgb(250pt)=(1,0.886349,0); rgb(251pt)=(1,0.890243,0); rgb(252pt)=(1,0.894144,0); rgb(253pt)=(1,0.898051,0); rgb(254pt)=(1,0.901964,0); rgb(255pt)=(1,0.905882,0)},
mesh/rows=49]
table[row sep=crcr,header=false] {%
%
56	0.093	-10.9049618708685\\
56	0.09666	-11.4423373147976\\
56	0.10032	-12.1019444097205\\
56	0.10398	-12.8837831556373\\
56	0.10764	-13.7878535525481\\
56	0.1113	-14.8141556004528\\
56	0.11496	-15.9626892993514\\
56	0.11862	-17.2334546492439\\
56	0.12228	-18.6264516501303\\
56	0.12594	-20.1416803020107\\
56	0.1296	-21.779140604885\\
56	0.13326	-23.5388325587532\\
56	0.13692	-25.4207561636153\\
56	0.14058	-27.4249114194713\\
56	0.14424	-29.5512983263213\\
56	0.1479	-31.7999168841651\\
56	0.15156	-34.1707670930029\\
56	0.15522	-36.6638489528346\\
56	0.15888	-39.2791624636602\\
56	0.16254	-42.0167076254798\\
56	0.1662	-44.8764844382932\\
56	0.16986	-47.8584929021006\\
56	0.17352	-50.9627330169019\\
56	0.17718	-54.189204782697\\
56	0.18084	-57.5379081994862\\
56	0.1845	-61.0088432672693\\
56	0.18816	-64.6020099860462\\
56	0.19182	-68.3174083558171\\
56	0.19548	-72.1550383765819\\
56	0.19914	-76.1149000483406\\
56	0.2028	-80.1969933710933\\
56	0.20646	-84.4013183448399\\
56	0.21012	-88.7278749695803\\
56	0.21378	-93.1766632453147\\
56	0.21744	-97.747683172043\\
56	0.2211	-102.440934749765\\
56	0.22476	-107.256417978481\\
56	0.22842	-112.194132858191\\
56	0.23208	-117.254079388895\\
56	0.23574	-122.436257570593\\
56	0.2394	-127.740667403285\\
56	0.24306	-133.167308886971\\
56	0.24672	-138.716182021651\\
56	0.25038	-144.387286807324\\
56	0.25404	-150.180623243992\\
56	0.2577	-156.096191331653\\
56	0.26136	-162.133991070308\\
56	0.26502	-168.294022459957\\
56	0.26868	-174.576285500601\\
56	0.27234	-180.980780192238\\
56	0.276	-187.507506534869\\
56.375	0.093	-10.8126291856249\\
56.375	0.09666	-11.3527834706413\\
56.375	0.10032	-12.0151694066517\\
56.375	0.10398	-12.7997869936559\\
56.375	0.10764	-13.7066362316541\\
56.375	0.1113	-14.7357171206462\\
56.375	0.11496	-15.8870296606322\\
56.375	0.11862	-17.1605738516121\\
56.375	0.12228	-18.556349693586\\
56.375	0.12594	-20.0743571865538\\
56.375	0.1296	-21.7145963305155\\
56.375	0.13326	-23.4770671254711\\
56.375	0.13692	-25.3617695714206\\
56.375	0.14058	-27.3687036683641\\
56.375	0.14424	-29.4978694163015\\
56.375	0.1479	-31.7492668152327\\
56.375	0.15156	-34.1228958651579\\
56.375	0.15522	-36.6187565660771\\
56.375	0.15888	-39.2368489179901\\
56.375	0.16254	-41.9771729208971\\
56.375	0.1662	-44.839728574798\\
56.375	0.16986	-47.8245158796927\\
56.375	0.17352	-50.9315348355815\\
56.375	0.17718	-54.1607854424641\\
56.375	0.18084	-57.5122677003406\\
56.375	0.1845	-60.9859816092111\\
56.375	0.18816	-64.5819271690755\\
56.375	0.19182	-68.3001043799338\\
56.375	0.19548	-72.140513241786\\
56.375	0.19914	-76.1031537546321\\
56.375	0.2028	-80.1880259184722\\
56.375	0.20646	-84.3951297333062\\
56.375	0.21012	-88.7244651991341\\
56.375	0.21378	-93.1760323159559\\
56.375	0.21744	-97.7498310837716\\
56.375	0.2211	-102.445861502581\\
56.375	0.22476	-107.264123572385\\
56.375	0.22842	-112.204617293182\\
56.375	0.23208	-117.267342664974\\
56.375	0.23574	-122.452299687759\\
56.375	0.2394	-127.759488361538\\
56.375	0.24306	-133.188908686311\\
56.375	0.24672	-138.740560662079\\
56.375	0.25038	-144.41444428884\\
56.375	0.25404	-150.210559566594\\
56.375	0.2577	-156.128906495343\\
56.375	0.26136	-162.169485075086\\
56.375	0.26502	-168.332295305823\\
56.375	0.26868	-174.617337187553\\
56.375	0.27234	-181.024610720278\\
56.375	0.276	-187.554115903996\\
56.75	0.093	-10.729892857501\\
56.75	0.09666	-11.2728259836048\\
56.75	0.10032	-11.9379907607026\\
56.75	0.10398	-12.7253871887943\\
56.75	0.10764	-13.6350152678799\\
56.75	0.1113	-14.6668749979594\\
56.75	0.11496	-15.8209663790329\\
56.75	0.11862	-17.0972894111002\\
56.75	0.12228	-18.4958440941615\\
56.75	0.12594	-20.0166304282167\\
56.75	0.1296	-21.6596484132658\\
56.75	0.13326	-23.4248980493089\\
56.75	0.13692	-25.3123793363458\\
56.75	0.14058	-27.3220922743767\\
56.75	0.14424	-29.4540368634015\\
56.75	0.1479	-31.7082131034202\\
56.75	0.15156	-34.0846209944328\\
56.75	0.15522	-36.5832605364393\\
56.75	0.15888	-39.2041317294398\\
56.75	0.16254	-41.9472345734342\\
56.75	0.1662	-44.8125690684225\\
56.75	0.16986	-47.8001352144047\\
56.75	0.17352	-50.9099330113808\\
56.75	0.17718	-54.1419624593509\\
56.75	0.18084	-57.4962235583148\\
56.75	0.1845	-60.9727163082728\\
56.75	0.18816	-64.5714407092246\\
56.75	0.19182	-68.2923967611703\\
56.75	0.19548	-72.1355844641099\\
56.75	0.19914	-76.1010038180435\\
56.75	0.2028	-80.188654822971\\
56.75	0.20646	-84.3985374788924\\
56.75	0.21012	-88.7306517858077\\
56.75	0.21378	-93.1849977437169\\
56.75	0.21744	-97.76157535262\\
56.75	0.2211	-102.460384612517\\
56.75	0.22476	-107.281425523408\\
56.75	0.22842	-112.224698085293\\
56.75	0.23208	-117.290202298172\\
56.75	0.23574	-122.477938162045\\
56.75	0.2394	-127.787905676911\\
56.75	0.24306	-133.220104842772\\
56.75	0.24672	-138.774535659626\\
56.75	0.25038	-144.451198127475\\
56.75	0.25404	-150.250092246317\\
56.75	0.2577	-156.171218016153\\
56.75	0.26136	-162.214575436984\\
56.75	0.26502	-168.380164508808\\
56.75	0.26868	-174.667985231626\\
56.75	0.27234	-181.078037605437\\
56.75	0.276	-187.610321630243\\
57.125	0.093	-10.6567528864969\\
57.125	0.09666	-11.2024648536881\\
57.125	0.10032	-11.8704084718733\\
57.125	0.10398	-12.6605837410525\\
57.125	0.10764	-13.5729906612255\\
57.125	0.1113	-14.6076292323924\\
57.125	0.11496	-15.7644994545533\\
57.125	0.11862	-17.0436013277081\\
57.125	0.12228	-18.4449348518568\\
57.125	0.12594	-19.9685000269994\\
57.125	0.1296	-21.614296853136\\
57.125	0.13326	-23.3823253302664\\
57.125	0.13692	-25.2725854583908\\
57.125	0.14058	-27.2850772375091\\
57.125	0.14424	-29.4198006676213\\
57.125	0.1479	-31.6767557487274\\
57.125	0.15156	-34.0559424808275\\
57.125	0.15522	-36.5573608639214\\
57.125	0.15888	-39.1810108980094\\
57.125	0.16254	-41.9268925830912\\
57.125	0.1662	-44.7950059191669\\
57.125	0.16986	-47.7853509062365\\
57.125	0.17352	-50.8979275443\\
57.125	0.17718	-54.1327358333575\\
57.125	0.18084	-57.4897757734089\\
57.125	0.1845	-60.9690473644542\\
57.125	0.18816	-64.5705506064935\\
57.125	0.19182	-68.2942854995266\\
57.125	0.19548	-72.1402520435537\\
57.125	0.19914	-76.1084502385746\\
57.125	0.2028	-80.1988800845895\\
57.125	0.20646	-84.4115415815984\\
57.125	0.21012	-88.7464347296011\\
57.125	0.21378	-93.2035595285978\\
57.125	0.21744	-97.7829159785883\\
57.125	0.2211	-102.484504079573\\
57.125	0.22476	-107.308323831551\\
57.125	0.22842	-112.254375234524\\
57.125	0.23208	-117.32265828849\\
57.125	0.23574	-122.51317299345\\
57.125	0.2394	-127.825919349404\\
57.125	0.24306	-133.260897356352\\
57.125	0.24672	-138.818107014294\\
57.125	0.25038	-144.49754832323\\
57.125	0.25404	-150.29922128316\\
57.125	0.2577	-156.223125894083\\
57.125	0.26136	-162.269262156001\\
57.125	0.26502	-168.437630068912\\
57.125	0.26868	-174.728229632818\\
57.125	0.27234	-181.141060847717\\
57.125	0.276	-187.67612371361\\
57.5	0.093	-10.5932092726126\\
57.5	0.09666	-11.1417000808913\\
57.5	0.10032	-11.8124225401639\\
57.5	0.10398	-12.6053766504304\\
57.5	0.10764	-13.5205624116909\\
57.5	0.1113	-14.5579798239453\\
57.5	0.11496	-15.7176288871936\\
57.5	0.11862	-16.9995096014358\\
57.5	0.12228	-18.4036219666719\\
57.5	0.12594	-19.929965982902\\
57.5	0.1296	-21.5785416501259\\
57.5	0.13326	-23.3493489683438\\
57.5	0.13692	-25.2423879375556\\
57.5	0.14058	-27.2576585577613\\
57.5	0.14424	-29.3951608289609\\
57.5	0.1479	-31.6548947511545\\
57.5	0.15156	-34.036860324342\\
57.5	0.15522	-36.5410575485233\\
57.5	0.15888	-39.1674864236987\\
57.5	0.16254	-41.9161469498679\\
57.5	0.1662	-44.787039127031\\
57.5	0.16986	-47.7801629551881\\
57.5	0.17352	-50.8955184343391\\
57.5	0.17718	-54.1331055644839\\
57.5	0.18084	-57.4929243456227\\
57.5	0.1845	-60.9749747777555\\
57.5	0.18816	-64.5792568608822\\
57.5	0.19182	-68.3057705950027\\
57.5	0.19548	-72.1545159801172\\
57.5	0.19914	-76.1254930162256\\
57.5	0.2028	-80.2187017033279\\
57.5	0.20646	-84.4341420414242\\
57.5	0.21012	-88.7718140305143\\
57.5	0.21378	-93.2317176705984\\
57.5	0.21744	-97.8138529616764\\
57.5	0.2211	-102.518219903748\\
57.5	0.22476	-107.344818496814\\
57.5	0.22842	-112.293648740874\\
57.5	0.23208	-117.364710635927\\
57.5	0.23574	-122.558004181975\\
57.5	0.2394	-127.873529379017\\
57.5	0.24306	-133.311286227052\\
57.5	0.24672	-138.871274726081\\
57.5	0.25038	-144.553494876105\\
57.5	0.25404	-150.357946677122\\
57.5	0.2577	-156.284630129133\\
57.5	0.26136	-162.333545232138\\
57.5	0.26502	-168.504691986137\\
57.5	0.26868	-174.79807039113\\
57.5	0.27234	-181.213680447116\\
57.5	0.276	-187.751522154097\\
57.875	0.093	-10.5392620158482\\
57.875	0.09666	-11.0905316652143\\
57.875	0.10032	-11.7640329655743\\
57.875	0.10398	-12.5597659169282\\
57.875	0.10764	-13.4777305192761\\
57.875	0.1113	-14.5179267726179\\
57.875	0.11496	-15.6803546769536\\
57.875	0.11862	-16.9650142322833\\
57.875	0.12228	-18.3719054386068\\
57.875	0.12594	-19.9010282959243\\
57.875	0.1296	-21.5523828042356\\
57.875	0.13326	-23.325968963541\\
57.875	0.13692	-25.2217867738402\\
57.875	0.14058	-27.2398362351333\\
57.875	0.14424	-29.3801173474204\\
57.875	0.1479	-31.6426301107013\\
57.875	0.15156	-34.0273745249762\\
57.875	0.15522	-36.5343505902451\\
57.875	0.15888	-39.1635583065078\\
57.875	0.16254	-41.9149976737644\\
57.875	0.1662	-44.788668692015\\
57.875	0.16986	-47.7845713612595\\
57.875	0.17352	-50.9027056814979\\
57.875	0.17718	-54.1430716527301\\
57.875	0.18084	-57.5056692749564\\
57.875	0.1845	-60.9904985481766\\
57.875	0.18816	-64.5975594723906\\
57.875	0.19182	-68.3268520475986\\
57.875	0.19548	-72.1783762738005\\
57.875	0.19914	-76.1521321509963\\
57.875	0.2028	-80.2481196791861\\
57.875	0.20646	-84.4663388583698\\
57.875	0.21012	-88.8067896885474\\
57.875	0.21378	-93.2694721697188\\
57.875	0.21744	-97.8543863018843\\
57.875	0.2211	-102.561532085044\\
57.875	0.22476	-107.390909519197\\
57.875	0.22842	-112.342518604344\\
57.875	0.23208	-117.416359340485\\
57.875	0.23574	-122.61243172762\\
57.875	0.2394	-127.930735765749\\
57.875	0.24306	-133.371271454872\\
57.875	0.24672	-138.934038794989\\
57.875	0.25038	-144.619037786099\\
57.875	0.25404	-150.426268428204\\
57.875	0.2577	-156.355730721302\\
57.875	0.26136	-162.407424665395\\
57.875	0.26502	-168.581350260481\\
57.875	0.26868	-174.877507506561\\
57.875	0.27234	-181.295896403636\\
57.875	0.276	-187.836516951704\\
58.25	0.093	-10.4949111162035\\
58.25	0.09666	-11.048959606657\\
58.25	0.10032	-11.7252397481045\\
58.25	0.10398	-12.5237515405458\\
58.25	0.10764	-13.4444949839811\\
58.25	0.1113	-14.4874700784104\\
58.25	0.11496	-15.6526768238335\\
58.25	0.11862	-16.9401152202505\\
58.25	0.12228	-18.3497852676615\\
58.25	0.12594	-19.8816869660664\\
58.25	0.1296	-21.5358203154652\\
58.25	0.13326	-23.3121853158579\\
58.25	0.13692	-25.2107819672446\\
58.25	0.14058	-27.2316102696251\\
58.25	0.14424	-29.3746702229996\\
58.25	0.1479	-31.639961827368\\
58.25	0.15156	-34.0274850827303\\
58.25	0.15522	-36.5372399890866\\
58.25	0.15888	-39.1692265464367\\
58.25	0.16254	-41.9234447547808\\
58.25	0.1662	-44.7998946141188\\
58.25	0.16986	-47.7985761244506\\
58.25	0.17352	-50.9194892857765\\
58.25	0.17718	-54.1626340980962\\
58.25	0.18084	-57.5280105614098\\
58.25	0.1845	-61.0156186757175\\
58.25	0.18816	-64.625458441019\\
58.25	0.19182	-68.3575298573143\\
58.25	0.19548	-72.2118329246037\\
58.25	0.19914	-76.1883676428869\\
58.25	0.2028	-80.2871340121641\\
58.25	0.20646	-84.5081320324352\\
58.25	0.21012	-88.8513617037002\\
58.25	0.21378	-93.3168230259591\\
58.25	0.21744	-97.904515999212\\
58.25	0.2211	-102.614440623459\\
58.25	0.22476	-107.446596898699\\
58.25	0.22842	-112.400984824934\\
58.25	0.23208	-117.477604402162\\
58.25	0.23574	-122.676455630385\\
58.25	0.2394	-127.997538509601\\
58.25	0.24306	-133.440853039812\\
58.25	0.24672	-139.006399221016\\
58.25	0.25038	-144.694177053214\\
58.25	0.25404	-150.504186536406\\
58.25	0.2577	-156.436427670592\\
58.25	0.26136	-162.490900455772\\
58.25	0.26502	-168.667604891945\\
58.25	0.26868	-174.966540979113\\
58.25	0.27234	-181.387708717275\\
58.25	0.276	-187.93110810643\\
58.625	0.093	-10.4601565736786\\
58.625	0.09666	-11.0169839052196\\
58.625	0.10032	-11.6960428877545\\
58.625	0.10398	-12.4973335212833\\
58.625	0.10764	-13.420855805806\\
58.625	0.1113	-14.4666097413226\\
58.625	0.11496	-15.6345953278332\\
58.625	0.11862	-16.9248125653376\\
58.625	0.12228	-18.337261453836\\
58.625	0.12594	-19.8719419933284\\
58.625	0.1296	-21.5288541838146\\
58.625	0.13326	-23.3079980252947\\
58.625	0.13692	-25.2093735177688\\
58.625	0.14058	-27.2329806612368\\
58.625	0.14424	-29.3788194556987\\
58.625	0.1479	-31.6468899011545\\
58.625	0.15156	-34.0371919976042\\
58.625	0.15522	-36.5497257450479\\
58.625	0.15888	-39.1844911434855\\
58.625	0.16254	-41.941488192917\\
58.625	0.1662	-44.8207168933424\\
58.625	0.16986	-47.8221772447617\\
58.625	0.17352	-50.9458692471749\\
58.625	0.17718	-54.1917929005821\\
58.625	0.18084	-57.5599482049831\\
58.625	0.1845	-61.0503351603781\\
58.625	0.18816	-64.6629537667671\\
58.625	0.19182	-68.3978040241499\\
58.625	0.19548	-72.2548859325266\\
58.625	0.19914	-76.2341994918973\\
58.625	0.2028	-80.3357447022619\\
58.625	0.20646	-84.5595215636204\\
58.625	0.21012	-88.9055300759729\\
58.625	0.21378	-93.3737702393192\\
58.625	0.21744	-97.9642420536594\\
58.625	0.2211	-102.676945518994\\
58.625	0.22476	-107.511880635322\\
58.625	0.22842	-112.469047402644\\
58.625	0.23208	-117.54844582096\\
58.625	0.23574	-122.75007589027\\
58.625	0.2394	-128.073937610573\\
58.625	0.24306	-133.520030981871\\
58.625	0.24672	-139.088356004163\\
58.625	0.25038	-144.778912677448\\
58.625	0.25404	-150.591701001728\\
58.625	0.2577	-156.526720977001\\
58.625	0.26136	-162.583972603268\\
58.625	0.26502	-168.763455880529\\
58.625	0.26868	-175.065170808784\\
58.625	0.27234	-181.489117388033\\
58.625	0.276	-188.035295618276\\
59	0.093	-10.4349983882736\\
59	0.09666	-10.994604560902\\
59	0.10032	-11.6764423845243\\
59	0.10398	-12.4805118591405\\
59	0.10764	-13.4068129847506\\
59	0.1113	-14.4553457613547\\
59	0.11496	-15.6261101889527\\
59	0.11862	-16.9191062675446\\
59	0.12228	-18.3343339971304\\
59	0.12594	-19.8717933777101\\
59	0.1296	-21.5314844092838\\
59	0.13326	-23.3134070918513\\
59	0.13692	-25.2175614254128\\
59	0.14058	-27.2439474099682\\
59	0.14424	-29.3925650455176\\
59	0.1479	-31.6634143320608\\
59	0.15156	-34.0564952695979\\
59	0.15522	-36.571807858129\\
59	0.15888	-39.209352097654\\
59	0.16254	-41.9691279881729\\
59	0.1662	-44.8511355296857\\
59	0.16986	-47.8553747221925\\
59	0.17352	-50.9818455656931\\
59	0.17718	-54.2305480601877\\
59	0.18084	-57.6014822056762\\
59	0.1845	-61.0946480021586\\
59	0.18816	-64.710045449635\\
59	0.19182	-68.4476745481052\\
59	0.19548	-72.3075352975694\\
59	0.19914	-76.2896276980275\\
59	0.2028	-80.3939517494795\\
59	0.20646	-84.6205074519255\\
59	0.21012	-88.9692948053654\\
59	0.21378	-93.4403138097991\\
59	0.21744	-98.0335644652267\\
59	0.2211	-102.749046771648\\
59	0.22476	-107.586760729064\\
59	0.22842	-112.546706337473\\
59	0.23208	-117.628883596877\\
59	0.23574	-122.833292507274\\
59	0.2394	-128.159933068665\\
59	0.24306	-133.60880528105\\
59	0.24672	-139.179909144429\\
59	0.25038	-144.873244658802\\
59	0.25404	-150.688811824169\\
59	0.2577	-156.62661064053\\
59	0.26136	-162.686641107885\\
59	0.26502	-168.868903226233\\
59	0.26868	-175.173396995576\\
59	0.27234	-181.600122415912\\
59	0.276	-188.149079487242\\
59.375	0.093	-10.4194365599884\\
59.375	0.09666	-10.9818215737042\\
59.375	0.10032	-11.6664382384139\\
59.375	0.10398	-12.4732865541175\\
59.375	0.10764	-13.4023665208151\\
59.375	0.1113	-14.4536781385066\\
59.375	0.11496	-15.627221407192\\
59.375	0.11862	-16.9229963268713\\
59.375	0.12228	-18.3410028975445\\
59.375	0.12594	-19.8812411192117\\
59.375	0.1296	-21.5437109918728\\
59.375	0.13326	-23.3284125155277\\
59.375	0.13692	-25.2353456901767\\
59.375	0.14058	-27.2645105158195\\
59.375	0.14424	-29.4159069924562\\
59.375	0.1479	-31.6895351200869\\
59.375	0.15156	-34.0853948987114\\
59.375	0.15522	-36.60348632833\\
59.375	0.15888	-39.2438094089424\\
59.375	0.16254	-42.0063641405487\\
59.375	0.1662	-44.891150523149\\
59.375	0.16986	-47.8981685567431\\
59.375	0.17352	-51.0274182413312\\
59.375	0.17718	-54.2788995769132\\
59.375	0.18084	-57.6526125634891\\
59.375	0.1845	-61.148557201059\\
59.375	0.18816	-64.7667334896228\\
59.375	0.19182	-68.5071414291804\\
59.375	0.19548	-72.369781019732\\
59.375	0.19914	-76.3546522612775\\
59.375	0.2028	-80.461755153817\\
59.375	0.20646	-84.6910896973503\\
59.375	0.21012	-89.0426558918776\\
59.375	0.21378	-93.5164537373988\\
59.375	0.21744	-98.1124832339139\\
59.375	0.2211	-102.830744381423\\
59.375	0.22476	-107.671237179926\\
59.375	0.22842	-112.633961629423\\
59.375	0.23208	-117.718917729913\\
59.375	0.23574	-122.926105481398\\
59.375	0.2394	-128.255524883877\\
59.375	0.24306	-133.707175937349\\
59.375	0.24672	-139.281058641816\\
59.375	0.25038	-144.977172997276\\
59.375	0.25404	-150.79551900373\\
59.375	0.2577	-156.736096661179\\
59.375	0.26136	-162.798905969621\\
59.375	0.26502	-168.983946929057\\
59.375	0.26868	-175.291219539487\\
59.375	0.27234	-181.720723800911\\
59.375	0.276	-188.272459713328\\
59.75	0.093	-10.413471088823\\
59.75	0.09666	-10.9786349436262\\
59.75	0.10032	-11.6660304494233\\
59.75	0.10398	-12.4756576062144\\
59.75	0.10764	-13.4075164139994\\
59.75	0.1113	-14.4616068727783\\
59.75	0.11496	-15.6379289825511\\
59.75	0.11862	-16.9364827433178\\
59.75	0.12228	-18.3572681550785\\
59.75	0.12594	-19.9002852178331\\
59.75	0.1296	-21.5655339315816\\
59.75	0.13326	-23.353014296324\\
59.75	0.13692	-25.2627263120603\\
59.75	0.14058	-27.2946699787905\\
59.75	0.14424	-29.4488452965147\\
59.75	0.1479	-31.7252522652328\\
59.75	0.15156	-34.1238908849448\\
59.75	0.15522	-36.6447611556507\\
59.75	0.15888	-39.2878630773506\\
59.75	0.16254	-42.0531966500443\\
59.75	0.1662	-44.940761873732\\
59.75	0.16986	-47.9505587484136\\
59.75	0.17352	-51.0825872740891\\
59.75	0.17718	-54.3368474507585\\
59.75	0.18084	-57.7133392784218\\
59.75	0.1845	-61.2120627570791\\
59.75	0.18816	-64.8330178867303\\
59.75	0.19182	-68.5762046673754\\
59.75	0.19548	-72.4416230990144\\
59.75	0.19914	-76.4292731816473\\
59.75	0.2028	-80.5391549152742\\
59.75	0.20646	-84.771268299895\\
59.75	0.21012	-89.1256133355097\\
59.75	0.21378	-93.6021900221183\\
59.75	0.21744	-98.2009983597208\\
59.75	0.2211	-102.922038348317\\
59.75	0.22476	-107.765309987908\\
59.75	0.22842	-112.730813278492\\
59.75	0.23208	-117.81854822007\\
59.75	0.23574	-123.028514812642\\
59.75	0.2394	-128.360713056208\\
59.75	0.24306	-133.815142950768\\
59.75	0.24672	-139.391804496322\\
59.75	0.25038	-145.09069769287\\
59.75	0.25404	-150.911822540412\\
59.75	0.2577	-156.855179038947\\
59.75	0.26136	-162.920767188477\\
59.75	0.26502	-169.108586989\\
59.75	0.26868	-175.418638440517\\
59.75	0.27234	-181.850921543029\\
59.75	0.276	-188.405436296534\\
60.125	0.093	-10.4171019747773\\
60.125	0.09666	-10.985044670668\\
60.125	0.10032	-11.6752190175526\\
60.125	0.10398	-12.487625015431\\
60.125	0.10764	-13.4222626643034\\
60.125	0.1113	-14.4791319641698\\
60.125	0.11496	-15.65823291503\\
60.125	0.11862	-16.9595655168842\\
60.125	0.12228	-18.3831297697322\\
60.125	0.12594	-19.9289256735743\\
60.125	0.1296	-21.5969532284102\\
60.125	0.13326	-23.38721243424\\
60.125	0.13692	-25.2997032910637\\
60.125	0.14058	-27.3344257988814\\
60.125	0.14424	-29.491379957693\\
60.125	0.1479	-31.7705657674985\\
60.125	0.15156	-34.1719832282979\\
60.125	0.15522	-36.6956323400913\\
60.125	0.15888	-39.3415131028785\\
60.125	0.16254	-42.1096255166597\\
60.125	0.1662	-44.9999695814348\\
60.125	0.16986	-48.0125452972038\\
60.125	0.17352	-51.1473526639667\\
60.125	0.17718	-54.4043916817236\\
60.125	0.18084	-57.7836623504743\\
60.125	0.1845	-61.285164670219\\
60.125	0.18816	-64.9088986409576\\
60.125	0.19182	-68.6548642626901\\
60.125	0.19548	-72.5230615354166\\
60.125	0.19914	-76.5134904591369\\
60.125	0.2028	-80.6261510338512\\
60.125	0.20646	-84.8610432595595\\
60.125	0.21012	-89.2181671362616\\
60.125	0.21378	-93.6975226639576\\
60.125	0.21744	-98.2991098426475\\
60.125	0.2211	-103.022928672331\\
60.125	0.22476	-107.868979153009\\
60.125	0.22842	-112.837261284681\\
60.125	0.23208	-117.927775067346\\
60.125	0.23574	-123.140520501006\\
60.125	0.2394	-128.475497585659\\
60.125	0.24306	-133.932706321307\\
60.125	0.24672	-139.512146707948\\
60.125	0.25038	-145.213818745583\\
60.125	0.25404	-151.037722434212\\
60.125	0.2577	-156.983857773836\\
60.125	0.26136	-163.052224764453\\
60.125	0.26502	-169.242823406063\\
60.125	0.26868	-175.555653698668\\
60.125	0.27234	-181.990715642267\\
60.125	0.276	-188.548009236859\\
60.5	0.093	-10.4303292178515\\
60.5	0.09666	-11.0010507548296\\
60.5	0.10032	-11.6940039428016\\
60.5	0.10398	-12.5091887817675\\
60.5	0.10764	-13.4466052717273\\
60.5	0.1113	-14.5062534126811\\
60.5	0.11496	-15.6881332046288\\
60.5	0.11862	-16.9922446475703\\
60.5	0.12228	-18.4185877415058\\
60.5	0.12594	-19.9671624864353\\
60.5	0.1296	-21.6379688823586\\
60.5	0.13326	-23.4310069292758\\
60.5	0.13692	-25.346276627187\\
60.5	0.14058	-27.3837779760921\\
60.5	0.14424	-29.5435109759911\\
60.5	0.1479	-31.825475626884\\
60.5	0.15156	-34.2296719287709\\
60.5	0.15522	-36.7560998816516\\
60.5	0.15888	-39.4047594855263\\
60.5	0.16254	-42.1756507403949\\
60.5	0.1662	-45.0687736462574\\
60.5	0.16986	-48.0841282031138\\
60.5	0.17352	-51.2217144109642\\
60.5	0.17718	-54.4815322698084\\
60.5	0.18084	-57.8635817796466\\
60.5	0.1845	-61.3678629404788\\
60.5	0.18816	-64.9943757523048\\
60.5	0.19182	-68.7431202151247\\
60.5	0.19548	-72.6140963289386\\
60.5	0.19914	-76.6073040937464\\
60.5	0.2028	-80.7227435095481\\
60.5	0.20646	-84.9604145763437\\
60.5	0.21012	-89.3203172941333\\
60.5	0.21378	-93.8024516629167\\
60.5	0.21744	-98.406817682694\\
60.5	0.2211	-103.133415353465\\
60.5	0.22476	-107.982244675231\\
60.5	0.22842	-112.95330564799\\
60.5	0.23208	-118.046598271743\\
60.5	0.23574	-123.26212254649\\
60.5	0.2394	-128.599878472231\\
60.5	0.24306	-134.059866048965\\
60.5	0.24672	-139.642085276694\\
60.5	0.25038	-145.346536155417\\
60.5	0.25404	-151.173218685133\\
60.5	0.2577	-157.122132865844\\
60.5	0.26136	-163.193278697548\\
60.5	0.26502	-169.386656180246\\
60.5	0.26868	-175.702265313939\\
60.5	0.27234	-182.140106098625\\
60.5	0.276	-188.700178534305\\
60.875	0.093	-10.4531528180456\\
60.875	0.09666	-11.026653196111\\
60.875	0.10032	-11.7223852251704\\
60.875	0.10398	-12.5403489052238\\
60.875	0.10764	-13.480544236271\\
60.875	0.1113	-14.5429712183122\\
60.875	0.11496	-15.7276298513473\\
60.875	0.11862	-17.0345201353763\\
60.875	0.12228	-18.4636420703992\\
60.875	0.12594	-20.014995656416\\
60.875	0.1296	-21.6885808934268\\
60.875	0.13326	-23.4843977814315\\
60.875	0.13692	-25.4024463204301\\
60.875	0.14058	-27.4427265104226\\
60.875	0.14424	-29.605238351409\\
60.875	0.1479	-31.8899818433894\\
60.875	0.15156	-34.2969569863636\\
60.875	0.15522	-36.8261637803318\\
60.875	0.15888	-39.4776022252939\\
60.875	0.16254	-42.2512723212499\\
60.875	0.1662	-45.1471740681999\\
60.875	0.16986	-48.1653074661437\\
60.875	0.17352	-51.3056725150815\\
60.875	0.17718	-54.5682692150131\\
60.875	0.18084	-57.9530975659387\\
60.875	0.1845	-61.4601575678583\\
60.875	0.18816	-65.0894492207718\\
60.875	0.19182	-68.8409725246791\\
60.875	0.19548	-72.7147274795804\\
60.875	0.19914	-76.7107140854756\\
60.875	0.2028	-80.8289323423648\\
60.875	0.20646	-85.0693822502478\\
60.875	0.21012	-89.4320638091248\\
60.875	0.21378	-93.9169770189956\\
60.875	0.21744	-98.5241218798604\\
60.875	0.2211	-103.253498391719\\
60.875	0.22476	-108.105106554572\\
60.875	0.22842	-113.078946368418\\
60.875	0.23208	-118.175017833259\\
60.875	0.23574	-123.393320949093\\
60.875	0.2394	-128.733855715921\\
60.875	0.24306	-134.196622133744\\
60.875	0.24672	-139.78162020256\\
60.875	0.25038	-145.48884992237\\
60.875	0.25404	-151.318311293174\\
60.875	0.2577	-157.270004314972\\
60.875	0.26136	-163.343928987764\\
60.875	0.26502	-169.540085311549\\
60.875	0.26868	-175.858473286329\\
60.875	0.27234	-182.299092912102\\
60.875	0.276	-188.86194418887\\
61.25	0.093	-10.4855727753594\\
61.25	0.09666	-11.0618519945123\\
61.25	0.10032	-11.7603628646591\\
61.25	0.10398	-12.5811053857998\\
61.25	0.10764	-13.5240795579345\\
61.25	0.1113	-14.5892853810631\\
61.25	0.11496	-15.7767228551856\\
61.25	0.11862	-17.086391980302\\
61.25	0.12228	-18.5182927564124\\
61.25	0.12594	-20.0724251835166\\
61.25	0.1296	-21.7487892616148\\
61.25	0.13326	-23.5473849907069\\
61.25	0.13692	-25.4682123707929\\
61.25	0.14058	-27.5112714018729\\
61.25	0.14424	-29.6765620839467\\
61.25	0.1479	-31.9640844170145\\
61.25	0.15156	-34.3738384010762\\
61.25	0.15522	-36.9058240361318\\
61.25	0.15888	-39.5600413221813\\
61.25	0.16254	-42.3364902592248\\
61.25	0.1662	-45.2351708472621\\
61.25	0.16986	-48.2560830862934\\
61.25	0.17352	-51.3992269763186\\
61.25	0.17718	-54.6646025173377\\
61.25	0.18084	-58.0522097093507\\
61.25	0.1845	-61.5620485523577\\
61.25	0.18816	-65.1941190463585\\
61.25	0.19182	-68.9484211913533\\
61.25	0.19548	-72.824954987342\\
61.25	0.19914	-76.8237204343247\\
61.25	0.2028	-80.9447175323012\\
61.25	0.20646	-85.1879462812717\\
61.25	0.21012	-89.5534066812361\\
61.25	0.21378	-94.0410987321943\\
61.25	0.21744	-98.6510224341465\\
61.25	0.2211	-103.383177787093\\
61.25	0.22476	-108.237564791033\\
61.25	0.22842	-113.214183445967\\
61.25	0.23208	-118.313033751895\\
61.25	0.23574	-123.534115708816\\
61.25	0.2394	-128.877429316732\\
61.25	0.24306	-134.342974575642\\
61.25	0.24672	-139.930751485545\\
61.25	0.25038	-145.640760046443\\
61.25	0.25404	-151.473000258334\\
61.25	0.2577	-157.427472121219\\
61.25	0.26136	-163.504175635099\\
61.25	0.26502	-169.703110799972\\
61.25	0.26868	-176.024277615839\\
61.25	0.27234	-182.4676760827\\
61.25	0.276	-189.033306200555\\
61.625	0.093	-10.527589089793\\
61.625	0.09666	-11.1066471500333\\
61.625	0.10032	-11.8079368612676\\
61.625	0.10398	-12.6314582234957\\
61.625	0.10764	-13.5772112367179\\
61.625	0.1113	-14.6451959009339\\
61.625	0.11496	-15.8354122161438\\
61.625	0.11862	-17.1478601823476\\
61.625	0.12228	-18.5825397995454\\
61.625	0.12594	-20.1394510677371\\
61.625	0.1296	-21.8185939869227\\
61.625	0.13326	-23.6199685571022\\
61.625	0.13692	-25.5435747782756\\
61.625	0.14058	-27.589412650443\\
61.625	0.14424	-29.7574821736043\\
61.625	0.1479	-32.0477833477595\\
61.625	0.15156	-34.4603161729086\\
61.625	0.15522	-36.9950806490516\\
61.625	0.15888	-39.6520767761886\\
61.625	0.16254	-42.4313045543194\\
61.625	0.1662	-45.3327639834442\\
61.625	0.16986	-48.3564550635629\\
61.625	0.17352	-51.5023777946755\\
61.625	0.17718	-54.770532176782\\
61.625	0.18084	-58.1609182098824\\
61.625	0.1845	-61.6735358939769\\
61.625	0.18816	-65.3083852290652\\
61.625	0.19182	-69.0654662151473\\
61.625	0.19548	-72.9447788522235\\
61.625	0.19914	-76.9463231402935\\
61.625	0.2028	-81.0700990793575\\
61.625	0.20646	-85.3161066694154\\
61.625	0.21012	-89.6843459104672\\
61.625	0.21378	-94.1748168025129\\
61.625	0.21744	-98.7875193455525\\
61.625	0.2211	-103.522453539586\\
61.625	0.22476	-108.379619384614\\
61.625	0.22842	-113.359016880635\\
61.625	0.23208	-118.46064602765\\
61.625	0.23574	-123.684506825659\\
61.625	0.2394	-129.030599274663\\
61.625	0.24306	-134.49892337466\\
61.625	0.24672	-140.089479125651\\
61.625	0.25038	-145.802266527636\\
61.625	0.25404	-151.637285580614\\
61.625	0.2577	-157.594536284587\\
61.625	0.26136	-163.674018639554\\
61.625	0.26502	-169.875732645514\\
61.625	0.26868	-176.199678302469\\
61.625	0.27234	-182.645855610417\\
61.625	0.276	-189.214264569359\\
62	0.093	-10.5792017613464\\
62	0.09666	-11.1610386626742\\
62	0.10032	-11.8651072149959\\
62	0.10398	-12.6914074183114\\
62	0.10764	-13.6399392726209\\
62	0.1113	-14.7107027779244\\
62	0.11496	-15.9036979342217\\
62	0.11862	-17.218924741513\\
62	0.12228	-18.6563831997982\\
62	0.12594	-20.2160733090773\\
62	0.1296	-21.8979950693503\\
62	0.13326	-23.7021484806173\\
62	0.13692	-25.6285335428781\\
62	0.14058	-27.6771502561329\\
62	0.14424	-29.8479986203816\\
62	0.1479	-32.1410786356242\\
62	0.15156	-34.5563903018607\\
62	0.15522	-37.0939336190912\\
62	0.15888	-39.7537085873156\\
62	0.16254	-42.5357152065338\\
62	0.1662	-45.439953476746\\
62	0.16986	-48.4664233979522\\
62	0.17352	-51.6151249701522\\
62	0.17718	-54.8860581933461\\
62	0.18084	-58.279223067534\\
62	0.1845	-61.7946195927158\\
62	0.18816	-65.4322477688915\\
62	0.19182	-69.1921075960611\\
62	0.19548	-73.0741990742247\\
62	0.19914	-77.0785222033822\\
62	0.2028	-81.2050769835336\\
62	0.20646	-85.4538634146789\\
62	0.21012	-89.8248814968181\\
62	0.21378	-94.3181312299512\\
62	0.21744	-98.9336126140783\\
62	0.2211	-103.671325649199\\
62	0.22476	-108.531270335314\\
62	0.22842	-113.513446672423\\
62	0.23208	-118.617854660526\\
62	0.23574	-123.844494299622\\
62	0.2394	-129.193365589713\\
62	0.24306	-134.664468530797\\
62	0.24672	-140.257803122876\\
62	0.25038	-145.973369365948\\
62	0.25404	-151.811167260014\\
62	0.2577	-157.771196805075\\
62	0.26136	-163.853458001129\\
62	0.26502	-170.057950848176\\
62	0.26868	-176.384675346218\\
62	0.27234	-182.833631495254\\
62	0.276	-189.404819295284\\
62.375	0.093	-10.6404107900197\\
62.375	0.09666	-11.2250265324349\\
62.375	0.10032	-11.931873925844\\
62.375	0.10398	-12.760952970247\\
62.375	0.10764	-13.7122636656439\\
62.375	0.1113	-14.7858060120348\\
62.375	0.11496	-15.9815800094195\\
62.375	0.11862	-17.2995856577982\\
62.375	0.12228	-18.7398229571708\\
62.375	0.12594	-20.3022919075374\\
62.375	0.1296	-21.9869925088978\\
62.375	0.13326	-23.7939247612521\\
62.375	0.13692	-25.7230886646005\\
62.375	0.14058	-27.7744842189426\\
62.375	0.14424	-29.9481114242788\\
62.375	0.1479	-32.2439702806088\\
62.375	0.15156	-34.6620607879327\\
62.375	0.15522	-37.2023829462506\\
62.375	0.15888	-39.8649367555624\\
62.375	0.16254	-42.6497222158681\\
62.375	0.1662	-45.5567393271677\\
62.375	0.16986	-48.5859880894613\\
62.375	0.17352	-51.7374685027487\\
62.375	0.17718	-55.0111805670301\\
62.375	0.18084	-58.4071242823054\\
62.375	0.1845	-61.9252996485746\\
62.375	0.18816	-65.5657066658378\\
62.375	0.19182	-69.3283453340948\\
62.375	0.19548	-73.2132156533458\\
62.375	0.19914	-77.2203176235907\\
62.375	0.2028	-81.3496512448295\\
62.375	0.20646	-85.6012165170623\\
62.375	0.21012	-89.9750134402889\\
62.375	0.21378	-94.4710420145094\\
62.375	0.21744	-99.0893022397238\\
62.375	0.2211	-103.829794115932\\
62.375	0.22476	-108.692517643135\\
62.375	0.22842	-113.677472821331\\
62.375	0.23208	-118.784659650521\\
62.375	0.23574	-124.014078130705\\
62.375	0.2394	-129.365728261883\\
62.375	0.24306	-134.839610044055\\
62.375	0.24672	-140.435723477221\\
62.375	0.25038	-146.154068561381\\
62.375	0.25404	-151.994645296534\\
62.375	0.2577	-157.957453682682\\
62.375	0.26136	-164.042493719823\\
62.375	0.26502	-170.249765407959\\
62.375	0.26868	-176.579268747088\\
62.375	0.27234	-183.031003737211\\
62.375	0.276	-189.604970378328\\
62.75	0.093	-10.7112161758128\\
62.75	0.09666	-11.2986107593154\\
62.75	0.10032	-12.0082369938119\\
62.75	0.10398	-12.8400948793023\\
62.75	0.10764	-13.7941844157867\\
62.75	0.1113	-14.870505603265\\
62.75	0.11496	-16.0690584417372\\
62.75	0.11862	-17.3898429312033\\
62.75	0.12228	-18.8328590716633\\
62.75	0.12594	-20.3981068631172\\
62.75	0.1296	-22.0855863055651\\
62.75	0.13326	-23.8952973990069\\
62.75	0.13692	-25.8272401434426\\
62.75	0.14058	-27.8814145388722\\
62.75	0.14424	-30.0578205852958\\
62.75	0.1479	-32.3564582827132\\
62.75	0.15156	-34.7773276311246\\
62.75	0.15522	-37.3204286305299\\
62.75	0.15888	-39.9857612809291\\
62.75	0.16254	-42.7733255823222\\
62.75	0.1662	-45.6831215347093\\
62.75	0.16986	-48.7151491380902\\
62.75	0.17352	-51.8694083924651\\
62.75	0.17718	-55.1458992978339\\
62.75	0.18084	-58.5446218541966\\
62.75	0.1845	-62.0655760615533\\
62.75	0.18816	-65.7087619199038\\
62.75	0.19182	-69.4741794292482\\
62.75	0.19548	-73.3618285895866\\
62.75	0.19914	-77.3717094009189\\
62.75	0.2028	-81.5038218632452\\
62.75	0.20646	-85.7581659765654\\
62.75	0.21012	-90.1347417408794\\
62.75	0.21378	-94.6335491561874\\
62.75	0.21744	-99.2545882224892\\
62.75	0.2211	-103.997858939785\\
62.75	0.22476	-108.863361308075\\
62.75	0.22842	-113.851095327359\\
62.75	0.23208	-118.961060997636\\
62.75	0.23574	-124.193258318908\\
62.75	0.2394	-129.547687291173\\
62.75	0.24306	-135.024347914432\\
62.75	0.24672	-140.623240188686\\
62.75	0.25038	-146.344364113933\\
62.75	0.25404	-152.187719690174\\
62.75	0.2577	-158.153306917409\\
62.75	0.26136	-164.241125795638\\
62.75	0.26502	-170.451176324861\\
62.75	0.26868	-176.783458505077\\
62.75	0.27234	-183.237972336288\\
62.75	0.276	-189.814717818493\\
63.125	0.093	-10.7916179187256\\
63.125	0.09666	-11.3817913433157\\
63.125	0.10032	-12.0941964188996\\
63.125	0.10398	-12.9288331454774\\
63.125	0.10764	-13.8857015230492\\
63.125	0.1113	-14.9648015516149\\
63.125	0.11496	-16.1661332311746\\
63.125	0.11862	-17.4896965617281\\
63.125	0.12228	-18.9354915432755\\
63.125	0.12594	-20.5035181758169\\
63.125	0.1296	-22.1937764593522\\
63.125	0.13326	-24.0062663938814\\
63.125	0.13692	-25.9409879794045\\
63.125	0.14058	-27.9979412159216\\
63.125	0.14424	-30.1771261034325\\
63.125	0.1479	-32.4785426419374\\
63.125	0.15156	-34.9021908314362\\
63.125	0.15522	-37.4480706719289\\
63.125	0.15888	-40.1161821634156\\
63.125	0.16254	-42.9065253058961\\
63.125	0.1662	-45.8191000993706\\
63.125	0.16986	-48.8539065438389\\
63.125	0.17352	-52.0109446393012\\
63.125	0.17718	-55.2902143857575\\
63.125	0.18084	-58.6917157832076\\
63.125	0.1845	-62.2154488316517\\
63.125	0.18816	-65.8614135310897\\
63.125	0.19182	-69.6296098815215\\
63.125	0.19548	-73.5200378829473\\
63.125	0.19914	-77.532697535367\\
63.125	0.2028	-81.6675888387807\\
63.125	0.20646	-85.9247117931883\\
63.125	0.21012	-90.3040663985898\\
63.125	0.21378	-94.8056526549852\\
63.125	0.21744	-99.4294705623745\\
63.125	0.2211	-104.175520120758\\
63.125	0.22476	-109.043801330135\\
63.125	0.22842	-114.034314190506\\
63.125	0.23208	-119.147058701871\\
63.125	0.23574	-124.38203486423\\
63.125	0.2394	-129.739242677583\\
63.125	0.24306	-135.218682141929\\
63.125	0.24672	-140.82035325727\\
63.125	0.25038	-146.544256023605\\
63.125	0.25404	-152.390390440933\\
63.125	0.2577	-158.358756509256\\
63.125	0.26136	-164.449354228572\\
63.125	0.26502	-170.662183598882\\
63.125	0.26868	-176.997244620186\\
63.125	0.27234	-183.454537292484\\
63.125	0.276	-190.034061615776\\
63.5	0.093	-10.8816160187583\\
63.5	0.09666	-11.4745682844357\\
63.5	0.10032	-12.1897522011071\\
63.5	0.10398	-13.0271677687724\\
63.5	0.10764	-13.9868149874316\\
63.5	0.1113	-15.0686938570847\\
63.5	0.11496	-16.2728043777317\\
63.5	0.11862	-17.5991465493727\\
63.5	0.12228	-19.0477203720076\\
63.5	0.12594	-20.6185258456364\\
63.5	0.1296	-22.3115629702591\\
63.5	0.13326	-24.1268317458757\\
63.5	0.13692	-26.0643321724862\\
63.5	0.14058	-28.1240642500907\\
63.5	0.14424	-30.3060279786891\\
63.5	0.1479	-32.6102233582814\\
63.5	0.15156	-35.0366503888676\\
63.5	0.15522	-37.5853090704477\\
63.5	0.15888	-40.2561994030218\\
63.5	0.16254	-43.0493213865898\\
63.5	0.1662	-45.9646750211517\\
63.5	0.16986	-49.0022603067075\\
63.5	0.17352	-52.1620772432572\\
63.5	0.17718	-55.4441258308008\\
63.5	0.18084	-58.8484060693384\\
63.5	0.1845	-62.3749179588699\\
63.5	0.18816	-66.0236614993953\\
63.5	0.19182	-69.7946366909145\\
63.5	0.19548	-73.6878435334278\\
63.5	0.19914	-77.7032820269349\\
63.5	0.2028	-81.840952171436\\
63.5	0.20646	-86.100853966931\\
63.5	0.21012	-90.4829874134199\\
63.5	0.21378	-94.9873525109027\\
63.5	0.21744	-99.6139492593795\\
63.5	0.2211	-104.36277765885\\
63.5	0.22476	-109.233837709315\\
63.5	0.22842	-114.227129410773\\
63.5	0.23208	-119.342652763226\\
63.5	0.23574	-124.580407766672\\
63.5	0.2394	-129.940394421112\\
63.5	0.24306	-135.422612726546\\
63.5	0.24672	-141.027062682975\\
63.5	0.25038	-146.753744290396\\
63.5	0.25404	-152.602657548812\\
63.5	0.2577	-158.573802458222\\
63.5	0.26136	-164.667179018626\\
63.5	0.26502	-170.882787230024\\
63.5	0.26868	-177.220627092415\\
63.5	0.27234	-183.680698605801\\
63.5	0.276	-190.26300177018\\
63.875	0.093	-10.9812104759108\\
63.875	0.09666	-11.5769415826757\\
63.875	0.10032	-12.2949043404345\\
63.875	0.10398	-13.1350987491871\\
63.875	0.10764	-14.0975248089338\\
63.875	0.1113	-15.1821825196743\\
63.875	0.11496	-16.3890718814088\\
63.875	0.11862	-17.7181928941371\\
63.875	0.12228	-19.1695455578594\\
63.875	0.12594	-20.7431298725757\\
63.875	0.1296	-22.4389458382858\\
63.875	0.13326	-24.2569934549898\\
63.875	0.13692	-26.1972727226878\\
63.875	0.14058	-28.2597836413797\\
63.875	0.14424	-30.4445262110655\\
63.875	0.1479	-32.7515004317452\\
63.875	0.15156	-35.1807063034188\\
63.875	0.15522	-37.7321438260864\\
63.875	0.15888	-40.4058129997479\\
63.875	0.16254	-43.2017138244033\\
63.875	0.1662	-46.1198463000526\\
63.875	0.16986	-49.1602104266958\\
63.875	0.17352	-52.3228062043329\\
63.875	0.17718	-55.607633632964\\
63.875	0.18084	-59.014692712589\\
63.875	0.1845	-62.5439834432079\\
63.875	0.18816	-66.1955058248207\\
63.875	0.19182	-69.9692598574274\\
63.875	0.19548	-73.8652455410281\\
63.875	0.19914	-77.8834628756227\\
63.875	0.2028	-82.0239118612112\\
63.875	0.20646	-86.2865924977936\\
63.875	0.21012	-90.6715047853699\\
63.875	0.21378	-95.1786487239401\\
63.875	0.21744	-99.8080243135043\\
63.875	0.2211	-104.559631554062\\
63.875	0.22476	-109.433470445614\\
63.875	0.22842	-114.42954098816\\
63.875	0.23208	-119.5478431817\\
63.875	0.23574	-124.788377026234\\
63.875	0.2394	-130.151142521762\\
63.875	0.24306	-135.636139668283\\
63.875	0.24672	-141.243368465799\\
63.875	0.25038	-146.972828914308\\
63.875	0.25404	-152.824521013811\\
63.875	0.2577	-158.798444764309\\
63.875	0.26136	-164.8946001658\\
63.875	0.26502	-171.112987218285\\
63.875	0.26868	-177.453605921764\\
63.875	0.27234	-183.916456276237\\
63.875	0.276	-190.501538281704\\
64.25	0.093	-11.0904012901831\\
64.25	0.09666	-11.6889112380354\\
64.25	0.10032	-12.4096528368816\\
64.25	0.10398	-13.2526260867217\\
64.25	0.10764	-14.2178309875557\\
64.25	0.1113	-15.3052675393837\\
64.25	0.11496	-16.5149357422056\\
64.25	0.11862	-17.8468355960214\\
64.25	0.12228	-19.3009671008311\\
64.25	0.12594	-20.8773302566347\\
64.25	0.1296	-22.5759250634323\\
64.25	0.13326	-24.3967515212237\\
64.25	0.13692	-26.3398096300091\\
64.25	0.14058	-28.4050993897884\\
64.25	0.14424	-30.5926208005617\\
64.25	0.1479	-32.9023738623288\\
64.25	0.15156	-35.3343585750899\\
64.25	0.15522	-37.8885749388449\\
64.25	0.15888	-40.5650229535938\\
64.25	0.16254	-43.3637026193366\\
64.25	0.1662	-46.2846139360733\\
64.25	0.16986	-49.3277569038039\\
64.25	0.17352	-52.4931315225285\\
64.25	0.17718	-55.780737792247\\
64.25	0.18084	-59.1905757129594\\
64.25	0.1845	-62.7226452846657\\
64.25	0.18816	-66.3769465073659\\
64.25	0.19182	-70.1534793810601\\
64.25	0.19548	-74.0522439057482\\
64.25	0.19914	-78.0732400814302\\
64.25	0.2028	-82.2164679081061\\
64.25	0.20646	-86.4819273857759\\
64.25	0.21012	-90.8696185144397\\
64.25	0.21378	-95.3795412940973\\
64.25	0.21744	-100.011695724749\\
64.25	0.2211	-104.766081806394\\
64.25	0.22476	-109.642699539034\\
64.25	0.22842	-114.641548922667\\
64.25	0.23208	-119.762629957294\\
64.25	0.23574	-125.005942642916\\
64.25	0.2394	-130.371486979531\\
64.25	0.24306	-135.85926296714\\
64.25	0.24672	-141.469270605743\\
64.25	0.25038	-147.20150989534\\
64.25	0.25404	-153.05598083593\\
64.25	0.2577	-159.032683427515\\
64.25	0.26136	-165.131617670094\\
64.25	0.26502	-171.352783563666\\
64.25	0.26868	-177.696181108233\\
64.25	0.27234	-184.161810303793\\
64.25	0.276	-190.749671150347\\
64.625	0.093	-11.2091884615752\\
64.625	0.09666	-11.8104772505149\\
64.625	0.10032	-12.5339976904485\\
64.625	0.10398	-13.379749781376\\
64.625	0.10764	-14.3477335232975\\
64.625	0.1113	-15.4379489162129\\
64.625	0.11496	-16.6503959601222\\
64.625	0.11862	-17.9850746550254\\
64.625	0.12228	-19.4419850009225\\
64.625	0.12594	-21.0211269978136\\
64.625	0.1296	-22.7225006456986\\
64.625	0.13326	-24.5461059445775\\
64.625	0.13692	-26.4919428944503\\
64.625	0.14058	-28.560011495317\\
64.625	0.14424	-30.7503117471777\\
64.625	0.1479	-33.0628436500322\\
64.625	0.15156	-35.4976072038807\\
64.625	0.15522	-38.0546024087231\\
64.625	0.15888	-40.7338292645595\\
64.625	0.16254	-43.5352877713897\\
64.625	0.1662	-46.4589779292138\\
64.625	0.16986	-49.5048997380319\\
64.625	0.17352	-52.6730531978439\\
64.625	0.17718	-55.9634383086498\\
64.625	0.18084	-59.3760550704496\\
64.625	0.1845	-62.9109034832434\\
64.625	0.18816	-66.567983547031\\
64.625	0.19182	-70.3472952618125\\
64.625	0.19548	-74.2488386275881\\
64.625	0.19914	-78.2726136443575\\
64.625	0.2028	-82.4186203121209\\
64.625	0.20646	-86.6868586308781\\
64.625	0.21012	-91.0773286006293\\
64.625	0.21378	-95.5900302213743\\
64.625	0.21744	-100.224963493113\\
64.625	0.2211	-104.982128415846\\
64.625	0.22476	-109.861524989573\\
64.625	0.22842	-114.863153214294\\
64.625	0.23208	-119.987013090009\\
64.625	0.23574	-125.233104616717\\
64.625	0.2394	-130.60142779442\\
64.625	0.24306	-136.091982623116\\
64.625	0.24672	-141.704769102806\\
64.625	0.25038	-147.439787233491\\
64.625	0.25404	-153.297037015169\\
64.625	0.2577	-159.276518447841\\
64.625	0.26136	-165.378231531507\\
64.625	0.26502	-171.602176266167\\
64.625	0.26868	-177.948352651821\\
64.625	0.27234	-184.416760688469\\
64.625	0.276	-191.00740037611\\
65	0.093	-11.3375719900871\\
65	0.09666	-11.9416396201142\\
65	0.10032	-12.6679389011353\\
65	0.10398	-13.5164698331502\\
65	0.10764	-14.4872324161591\\
65	0.1113	-15.5802266501619\\
65	0.11496	-16.7954525351586\\
65	0.11862	-18.1329100711493\\
65	0.12228	-19.5925992581338\\
65	0.12594	-21.1745200961123\\
65	0.1296	-22.8786725850847\\
65	0.13326	-24.705056725051\\
65	0.13692	-26.6536725160113\\
65	0.14058	-28.7245199579654\\
65	0.14424	-30.9175990509135\\
65	0.1479	-33.2329097948555\\
65	0.15156	-35.6704521897914\\
65	0.15522	-38.2302262357212\\
65	0.15888	-40.9122319326449\\
65	0.16254	-43.7164692805626\\
65	0.1662	-46.6429382794742\\
65	0.16986	-49.6916389293796\\
65	0.17352	-52.8625712302791\\
65	0.17718	-56.1557351821724\\
65	0.18084	-59.5711307850596\\
65	0.1845	-63.1087580389408\\
65	0.18816	-66.7686169438159\\
65	0.19182	-70.5507074996848\\
65	0.19548	-74.4550297065478\\
65	0.19914	-78.4815835644046\\
65	0.2028	-82.6303690732554\\
65	0.20646	-86.9013862331001\\
65	0.21012	-91.2946350439387\\
65	0.21378	-95.8101155057712\\
65	0.21744	-100.447827618598\\
65	0.2211	-105.207771382418\\
65	0.22476	-110.089946797232\\
65	0.22842	-115.09435386304\\
65	0.23208	-120.220992579842\\
65	0.23574	-125.469862947639\\
65	0.2394	-130.840964966428\\
65	0.24306	-136.334298636212\\
65	0.24672	-141.94986395699\\
65	0.25038	-147.687660928762\\
65	0.25404	-153.547689551527\\
65	0.2577	-159.529949825287\\
65	0.26136	-165.63444175004\\
65	0.26502	-171.861165325788\\
65	0.26868	-178.210120552529\\
65	0.27234	-184.681307430264\\
65	0.276	-191.274725958993\\
65.375	0.093	-11.4755518757188\\
65.375	0.09666	-12.0823983468333\\
65.375	0.10032	-12.8114764689418\\
65.375	0.10398	-13.6627862420442\\
65.375	0.10764	-14.6363276661405\\
65.375	0.1113	-15.7321007412307\\
65.375	0.11496	-16.9501054673149\\
65.375	0.11862	-18.290341844393\\
65.375	0.12228	-19.7528098724649\\
65.375	0.12594	-21.3375095515309\\
65.375	0.1296	-23.0444408815907\\
65.375	0.13326	-24.8736038626444\\
65.375	0.13692	-26.824998494692\\
65.375	0.14058	-28.8986247777336\\
65.375	0.14424	-31.0944827117691\\
65.375	0.1479	-33.4125722967985\\
65.375	0.15156	-35.8528935328218\\
65.375	0.15522	-38.4154464198391\\
65.375	0.15888	-41.1002309578503\\
65.375	0.16254	-43.9072471468554\\
65.375	0.1662	-46.8364949868543\\
65.375	0.16986	-49.8879744778473\\
65.375	0.17352	-53.0616856198341\\
65.375	0.17718	-56.3576284128148\\
65.375	0.18084	-59.7758028567894\\
65.375	0.1845	-63.3162089517581\\
65.375	0.18816	-66.9788466977206\\
65.375	0.19182	-70.763716094677\\
65.375	0.19548	-74.6708171426273\\
65.375	0.19914	-78.7001498415716\\
65.375	0.2028	-82.8517141915098\\
65.375	0.20646	-87.1255101924419\\
65.375	0.21012	-91.5215378443679\\
65.375	0.21378	-96.0397971472878\\
65.375	0.21744	-100.680288101202\\
65.375	0.2211	-105.443010706109\\
65.375	0.22476	-110.327964962011\\
65.375	0.22842	-115.335150868907\\
65.375	0.23208	-120.464568426796\\
65.375	0.23574	-125.71621763568\\
65.375	0.2394	-131.090098495557\\
65.375	0.24306	-136.586211006428\\
65.375	0.24672	-142.204555168294\\
65.375	0.25038	-147.945130981153\\
65.375	0.25404	-153.807938445006\\
65.375	0.2577	-159.792977559853\\
65.375	0.26136	-165.900248325694\\
65.375	0.26502	-172.129750742528\\
65.375	0.26868	-178.481484810357\\
65.375	0.27234	-184.95545052918\\
65.375	0.276	-191.551647898996\\
65.75	0.093	-11.6231281184703\\
65.75	0.09666	-12.2327534306723\\
65.75	0.10032	-12.9646103938682\\
65.75	0.10398	-13.818699008058\\
65.75	0.10764	-14.7950192732417\\
65.75	0.1113	-15.8935711894194\\
65.75	0.11496	-17.114354756591\\
65.75	0.11862	-18.4573699747564\\
65.75	0.12228	-19.9226168439158\\
65.75	0.12594	-21.5100953640692\\
65.75	0.1296	-23.2198055352164\\
65.75	0.13326	-25.0517473573575\\
65.75	0.13692	-27.0059208304926\\
65.75	0.14058	-29.0823259546216\\
65.75	0.14424	-31.2809627297446\\
65.75	0.1479	-33.6018311558614\\
65.75	0.15156	-36.0449312329721\\
65.75	0.15522	-38.6102629610768\\
65.75	0.15888	-41.2978263401754\\
65.75	0.16254	-44.1076213702679\\
65.75	0.1662	-47.0396480513543\\
65.75	0.16986	-50.0939063834346\\
65.75	0.17352	-53.2703963665089\\
65.75	0.17718	-56.569118000577\\
65.75	0.18084	-59.9900712856391\\
65.75	0.1845	-63.5332562216952\\
65.75	0.18816	-67.1986728087451\\
65.75	0.19182	-70.9863210467889\\
65.75	0.19548	-74.8962009358266\\
65.75	0.19914	-78.9283124758583\\
65.75	0.2028	-83.082655666884\\
65.75	0.20646	-87.3592305089035\\
65.75	0.21012	-91.7580370019169\\
65.75	0.21378	-96.2790751459242\\
65.75	0.21744	-100.922344940926\\
65.75	0.2211	-105.687846386921\\
65.75	0.22476	-110.57557948391\\
65.75	0.22842	-115.585544231893\\
65.75	0.23208	-120.71774063087\\
65.75	0.23574	-125.972168680841\\
65.75	0.2394	-131.348828381805\\
65.75	0.24306	-136.847719733764\\
65.75	0.24672	-142.468842736717\\
65.75	0.25038	-148.212197390663\\
65.75	0.25404	-154.077783695604\\
65.75	0.2577	-160.065601651538\\
65.75	0.26136	-166.175651258467\\
65.75	0.26502	-172.407932516389\\
65.75	0.26868	-178.762445425305\\
65.75	0.27234	-185.239189985215\\
65.75	0.276	-191.838166196119\\
66.125	0.093	-11.7803007183416\\
66.125	0.09666	-12.392704871631\\
66.125	0.10032	-13.1273406759143\\
66.125	0.10398	-13.9842081311916\\
66.125	0.10764	-14.9633072374627\\
66.125	0.1113	-16.0646379947278\\
66.125	0.11496	-17.2882004029868\\
66.125	0.11862	-18.6339944622397\\
66.125	0.12228	-20.1020201724865\\
66.125	0.12594	-21.6922775337273\\
66.125	0.1296	-23.4047665459619\\
66.125	0.13326	-25.2394872091905\\
66.125	0.13692	-27.196439523413\\
66.125	0.14058	-29.2756234886294\\
66.125	0.14424	-31.4770391048398\\
66.125	0.1479	-33.800686372044\\
66.125	0.15156	-36.2465652902422\\
66.125	0.15522	-38.8146758594342\\
66.125	0.15888	-41.5050180796203\\
66.125	0.16254	-44.3175919508002\\
66.125	0.1662	-47.252397472974\\
66.125	0.16986	-50.3094346461418\\
66.125	0.17352	-53.4887034703034\\
66.125	0.17718	-56.790203945459\\
66.125	0.18084	-60.2139360716085\\
66.125	0.1845	-63.759899848752\\
66.125	0.18816	-67.4280952768893\\
66.125	0.19182	-71.2185223560206\\
66.125	0.19548	-75.1311810861458\\
66.125	0.19914	-79.1660714672649\\
66.125	0.2028	-83.3231934993779\\
66.125	0.20646	-87.6025471824849\\
66.125	0.21012	-92.0041325165857\\
66.125	0.21378	-96.5279495016805\\
66.125	0.21744	-101.173998137769\\
66.125	0.2211	-105.942278424852\\
66.125	0.22476	-110.832790362928\\
66.125	0.22842	-115.845533951999\\
66.125	0.23208	-120.980509192063\\
66.125	0.23574	-126.237716083121\\
66.125	0.2394	-131.617154625174\\
66.125	0.24306	-137.11882481822\\
66.125	0.24672	-142.74272666226\\
66.125	0.25038	-148.488860157294\\
66.125	0.25404	-154.357225303322\\
66.125	0.2577	-160.347822100343\\
66.125	0.26136	-166.460650548359\\
66.125	0.26502	-172.695710647369\\
66.125	0.26868	-179.053002397372\\
66.125	0.27234	-185.53252579837\\
66.125	0.276	-192.134280850361\\
66.5	0.093	-11.9470696753328\\
66.5	0.09666	-12.5622526697096\\
66.5	0.10032	-13.2996673150804\\
66.5	0.10398	-14.159313611445\\
66.5	0.10764	-15.1411915588036\\
66.5	0.1113	-16.2453011571561\\
66.5	0.11496	-17.4716424065025\\
66.5	0.11862	-18.8202153068428\\
66.5	0.12228	-20.291019858177\\
66.5	0.12594	-21.8840560605052\\
66.5	0.1296	-23.5993239138273\\
66.5	0.13326	-25.4368234181433\\
66.5	0.13692	-27.3965545734532\\
66.5	0.14058	-29.478517379757\\
66.5	0.14424	-31.6827118370548\\
66.5	0.1479	-34.0091379453465\\
66.5	0.15156	-36.4577957046321\\
66.5	0.15522	-39.0286851149116\\
66.5	0.15888	-41.721806176185\\
66.5	0.16254	-44.5371588884524\\
66.5	0.1662	-47.4747432517136\\
66.5	0.16986	-50.5345592659688\\
66.5	0.17352	-53.7166069312179\\
66.5	0.17718	-57.0208862474609\\
66.5	0.18084	-60.4473972146978\\
66.5	0.1845	-63.9961398329287\\
66.5	0.18816	-67.6671141021534\\
66.5	0.19182	-71.4603200223721\\
66.5	0.19548	-75.3757575935848\\
66.5	0.19914	-79.4134268157912\\
66.5	0.2028	-83.5733276889917\\
66.5	0.20646	-87.8554602131861\\
66.5	0.21012	-92.2598243883744\\
66.5	0.21378	-96.7864202145566\\
66.5	0.21744	-101.435247691733\\
66.5	0.2211	-106.206306819903\\
66.5	0.22476	-111.099597599067\\
66.5	0.22842	-116.115120029225\\
66.5	0.23208	-121.252874110376\\
66.5	0.23574	-126.512859842522\\
66.5	0.2394	-131.895077225662\\
66.5	0.24306	-137.399526259795\\
66.5	0.24672	-143.026206944923\\
66.5	0.25038	-148.775119281044\\
66.5	0.25404	-154.646263268159\\
66.5	0.2577	-160.639638906269\\
66.5	0.26136	-166.755246195372\\
66.5	0.26502	-172.993085135469\\
66.5	0.26868	-179.35315572656\\
66.5	0.27234	-185.835457968645\\
66.5	0.276	-192.439991861723\\
66.875	0.093	-12.1234349894437\\
66.875	0.09666	-12.741396824908\\
66.875	0.10032	-13.4815903113661\\
66.875	0.10398	-14.3440154488182\\
66.875	0.10764	-15.3286722372642\\
66.875	0.1113	-16.4355606767041\\
66.875	0.11496	-17.664680767138\\
66.875	0.11862	-19.0160325085657\\
66.875	0.12228	-20.4896159009873\\
66.875	0.12594	-22.085430944403\\
66.875	0.1296	-23.8034776388125\\
66.875	0.13326	-25.6437559842159\\
66.875	0.13692	-27.6062659806132\\
66.875	0.14058	-29.6910076280045\\
66.875	0.14424	-31.8979809263897\\
66.875	0.1479	-34.2271858757688\\
66.875	0.15156	-36.6786224761418\\
66.875	0.15522	-39.2522907275087\\
66.875	0.15888	-41.9481906298696\\
66.875	0.16254	-44.7663221832243\\
66.875	0.1662	-47.706685387573\\
66.875	0.16986	-50.7692802429156\\
66.875	0.17352	-53.9541067492521\\
66.875	0.17718	-57.2611649065825\\
66.875	0.18084	-60.6904547149068\\
66.875	0.1845	-64.2419761742252\\
66.875	0.18816	-67.9157292845374\\
66.875	0.19182	-71.7117140458435\\
66.875	0.19548	-75.6299304581435\\
66.875	0.19914	-79.6703785214374\\
66.875	0.2028	-83.8330582357253\\
66.875	0.20646	-88.1179696010071\\
66.875	0.21012	-92.5251126172828\\
66.875	0.21378	-97.0544872845524\\
66.875	0.21744	-101.706093602816\\
66.875	0.2211	-106.479931572073\\
66.875	0.22476	-111.376001192325\\
66.875	0.22842	-116.39430246357\\
66.875	0.23208	-121.534835385809\\
66.875	0.23574	-126.797599959042\\
66.875	0.2394	-132.182596183269\\
66.875	0.24306	-137.68982405849\\
66.875	0.24672	-143.319283584705\\
66.875	0.25038	-149.070974761914\\
66.875	0.25404	-154.944897590117\\
66.875	0.2577	-160.941052069313\\
66.875	0.26136	-167.059438199504\\
66.875	0.26502	-173.300055980688\\
66.875	0.26868	-179.662905412867\\
66.875	0.27234	-186.147986496039\\
66.875	0.276	-192.755299230205\\
67.25	0.093	-12.3093966606745\\
67.25	0.09666	-12.9301373372262\\
67.25	0.10032	-13.6731096647718\\
67.25	0.10398	-14.5383136433112\\
67.25	0.10764	-15.5257492728446\\
67.25	0.1113	-16.635416553372\\
67.25	0.11496	-17.8673154848933\\
67.25	0.11862	-19.2214460674084\\
67.25	0.12228	-20.6978083009175\\
67.25	0.12594	-22.2964021854205\\
67.25	0.1296	-24.0172277209174\\
67.25	0.13326	-25.8602849074083\\
67.25	0.13692	-27.8255737448931\\
67.25	0.14058	-29.9130942333717\\
67.25	0.14424	-32.1228463728443\\
67.25	0.1479	-34.4548301633108\\
67.25	0.15156	-36.9090456047713\\
67.25	0.15522	-39.4854926972256\\
67.25	0.15888	-42.1841714406739\\
67.25	0.16254	-45.0050818351161\\
67.25	0.1662	-47.9482238805522\\
67.25	0.16986	-51.0135975769822\\
67.25	0.17352	-54.2012029244061\\
67.25	0.17718	-57.511039922824\\
67.25	0.18084	-60.9431085722357\\
67.25	0.1845	-64.4974088726415\\
67.25	0.18816	-68.1739408240411\\
67.25	0.19182	-71.9727044264346\\
67.25	0.19548	-75.8936996798221\\
67.25	0.19914	-79.9369265842034\\
67.25	0.2028	-84.1023851395787\\
67.25	0.20646	-88.390075345948\\
67.25	0.21012	-92.7999972033111\\
67.25	0.21378	-97.3321507116681\\
67.25	0.21744	-101.986535871019\\
67.25	0.2211	-106.763152681364\\
67.25	0.22476	-111.662001142703\\
67.25	0.22842	-116.683081255035\\
67.25	0.23208	-121.826393018362\\
67.25	0.23574	-127.091936432683\\
67.25	0.2394	-132.479711497997\\
67.25	0.24306	-137.989718214306\\
67.25	0.24672	-143.621956581608\\
67.25	0.25038	-149.376426599904\\
67.25	0.25404	-155.253128269194\\
67.25	0.2577	-161.252061589478\\
67.25	0.26136	-167.373226560756\\
67.25	0.26502	-173.616623183028\\
67.25	0.26868	-179.982251456294\\
67.25	0.27234	-186.470111380554\\
67.25	0.276	-193.080202955807\\
67.625	0.093	-12.504954689025\\
67.625	0.09666	-13.1284742066641\\
67.625	0.10032	-13.8742253752971\\
67.625	0.10398	-14.742208194924\\
67.625	0.10764	-15.7324226655449\\
67.625	0.1113	-16.8448687871596\\
67.625	0.11496	-18.0795465597683\\
67.625	0.11862	-19.4364559833709\\
67.625	0.12228	-20.9155970579674\\
67.625	0.12594	-22.5169697835579\\
67.625	0.1296	-24.2405741601422\\
67.625	0.13326	-26.0864101877205\\
67.625	0.13692	-28.0544778662927\\
67.625	0.14058	-30.1447771958587\\
67.625	0.14424	-32.3573081764188\\
67.625	0.1479	-34.6920708079727\\
67.625	0.15156	-37.1490650905206\\
67.625	0.15522	-39.7282910240624\\
67.625	0.15888	-42.4297486085981\\
67.625	0.16254	-45.2534378441277\\
67.625	0.1662	-48.1993587306512\\
67.625	0.16986	-51.2675112681686\\
67.625	0.17352	-54.45789545668\\
67.625	0.17718	-57.7705112961852\\
67.625	0.18084	-61.2053587866844\\
67.625	0.1845	-64.7624379281776\\
67.625	0.18816	-68.4417487206646\\
67.625	0.19182	-72.2432911641455\\
67.625	0.19548	-76.1670652586204\\
67.625	0.19914	-80.2130710040892\\
67.625	0.2028	-84.381308400552\\
67.625	0.20646	-88.6717774480086\\
67.625	0.21012	-93.0844781464591\\
67.625	0.21378	-97.6194104959036\\
67.625	0.21744	-102.276574496342\\
67.625	0.2211	-107.055970147774\\
67.625	0.22476	-111.957597450201\\
67.625	0.22842	-116.981456403621\\
67.625	0.23208	-122.127547008035\\
67.625	0.23574	-127.395869263443\\
67.625	0.2394	-132.786423169845\\
67.625	0.24306	-138.29920872724\\
67.625	0.24672	-143.93422593563\\
67.625	0.25038	-149.691474795014\\
67.625	0.25404	-155.570955305391\\
67.625	0.2577	-161.572667466763\\
67.625	0.26136	-167.696611279128\\
67.625	0.26502	-173.942786742487\\
67.625	0.26868	-180.311193856841\\
67.625	0.27234	-186.801832622188\\
67.625	0.276	-193.414703038529\\
68	0.093	-12.7101090744954\\
68	0.09666	-13.3364074332219\\
68	0.10032	-14.0849374429423\\
68	0.10398	-14.9556991036567\\
68	0.10764	-15.9486924153649\\
68	0.1113	-17.0639173780671\\
68	0.11496	-18.3013739917632\\
68	0.11862	-19.6610622564532\\
68	0.12228	-21.1429821721372\\
68	0.12594	-22.747133738815\\
68	0.1296	-24.4735169564868\\
68	0.13326	-26.3221318251525\\
68	0.13692	-28.2929783448121\\
68	0.14058	-30.3860565154656\\
68	0.14424	-32.6013663371131\\
68	0.1479	-34.9389078097544\\
68	0.15156	-37.3986809333897\\
68	0.15522	-39.9806857080189\\
68	0.15888	-42.684922133642\\
68	0.16254	-45.5113902102591\\
68	0.1662	-48.46008993787\\
68	0.16986	-51.5310213164748\\
68	0.17352	-54.7241843460736\\
68	0.17718	-58.0395790266663\\
68	0.18084	-61.4772053582529\\
68	0.1845	-65.0370633408335\\
68	0.18816	-68.719152974408\\
68	0.19182	-72.5234742589763\\
68	0.19548	-76.4500271945386\\
68	0.19914	-80.4988117810948\\
68	0.2028	-84.669828018645\\
68	0.20646	-88.9630759071891\\
68	0.21012	-93.378555446727\\
68	0.21378	-97.9162666372589\\
68	0.21744	-102.576209478785\\
68	0.2211	-107.358383971304\\
68	0.22476	-112.262790114818\\
68	0.22842	-117.289427909326\\
68	0.23208	-122.438297354827\\
68	0.23574	-127.709398451322\\
68	0.2394	-133.102731198812\\
68	0.24306	-138.618295597295\\
68	0.24672	-144.256091646772\\
68	0.25038	-150.016119347243\\
68	0.25404	-155.898378698708\\
68	0.2577	-161.902869701167\\
68	0.26136	-168.02959235462\\
68	0.26502	-174.278546659067\\
68	0.26868	-180.649732614507\\
68	0.27234	-187.143150220942\\
68	0.276	-193.75879947837\\
68.375	0.093	-12.9248598170856\\
68.375	0.09666	-13.5539370168995\\
68.375	0.10032	-14.3052458677074\\
68.375	0.10398	-15.1787863695091\\
68.375	0.10764	-16.1745585223048\\
68.375	0.1113	-17.2925623260944\\
68.375	0.11496	-18.532797780878\\
68.375	0.11862	-19.8952648866554\\
68.375	0.12228	-21.3799636434267\\
68.375	0.12594	-22.986894051192\\
68.375	0.1296	-24.7160561099512\\
68.375	0.13326	-26.5674498197043\\
68.375	0.13692	-28.5410751804513\\
68.375	0.14058	-30.6369321921923\\
68.375	0.14424	-32.8550208549272\\
68.375	0.1479	-35.1953411686559\\
68.375	0.15156	-37.6578931333786\\
68.375	0.15522	-40.2426767490953\\
68.375	0.15888	-42.9496920158058\\
68.375	0.16254	-45.7789389335102\\
68.375	0.1662	-48.7304175022086\\
68.375	0.16986	-51.8041277219009\\
68.375	0.17352	-55.0000695925871\\
68.375	0.17718	-58.3182431142672\\
68.375	0.18084	-61.7586482869412\\
68.375	0.1845	-65.3212851106092\\
68.375	0.18816	-69.0061535852711\\
68.375	0.19182	-72.8132537109269\\
68.375	0.19548	-76.7425854875766\\
68.375	0.19914	-80.7941489152202\\
68.375	0.2028	-84.9679439938578\\
68.375	0.20646	-89.2639707234893\\
68.375	0.21012	-93.6822291041147\\
68.375	0.21378	-98.2227191357339\\
68.375	0.21744	-102.885440818347\\
68.375	0.2211	-107.670394151954\\
68.375	0.22476	-112.577579136555\\
68.375	0.22842	-117.60699577215\\
68.375	0.23208	-122.758644058739\\
68.375	0.23574	-128.032523996322\\
68.375	0.2394	-133.428635584899\\
68.375	0.24306	-138.94697882447\\
68.375	0.24672	-144.587553715034\\
68.375	0.25038	-150.350360256593\\
68.375	0.25404	-156.235398449145\\
68.375	0.2577	-162.242668292691\\
68.375	0.26136	-168.372169787232\\
68.375	0.26502	-174.623902932766\\
68.375	0.26868	-180.997867729294\\
68.375	0.27234	-187.494064176816\\
68.375	0.276	-194.112492275332\\
68.75	0.093	-13.1492069167956\\
68.75	0.09666	-13.781062957697\\
68.75	0.10032	-14.5351506495922\\
68.75	0.10398	-15.4114699924814\\
68.75	0.10764	-16.4100209863645\\
68.75	0.1113	-17.5308036312415\\
68.75	0.11496	-18.7738179271125\\
68.75	0.11862	-20.1390638739773\\
68.75	0.12228	-21.6265414718361\\
68.75	0.12594	-23.2362507206888\\
68.75	0.1296	-24.9681916205354\\
68.75	0.13326	-26.822364171376\\
68.75	0.13692	-28.7987683732104\\
68.75	0.14058	-30.8974042260388\\
68.75	0.14424	-33.1182717298611\\
68.75	0.1479	-35.4613708846773\\
68.75	0.15156	-37.9267016904874\\
68.75	0.15522	-40.5142641472914\\
68.75	0.15888	-43.2240582550894\\
68.75	0.16254	-46.0560840138812\\
68.75	0.1662	-49.0103414236671\\
68.75	0.16986	-52.0868304844468\\
68.75	0.17352	-55.2855511962204\\
68.75	0.17718	-58.6065035589879\\
68.75	0.18084	-62.0496875727493\\
68.75	0.1845	-65.6151032375048\\
68.75	0.18816	-69.3027505532541\\
68.75	0.19182	-73.1126295199973\\
68.75	0.19548	-77.0447401377344\\
68.75	0.19914	-81.0990824064655\\
68.75	0.2028	-85.2756563261904\\
68.75	0.20646	-89.5744618969094\\
68.75	0.21012	-93.9954991186222\\
68.75	0.21378	-98.5387679913289\\
68.75	0.21744	-103.20426851503\\
68.75	0.2211	-107.992000689724\\
68.75	0.22476	-112.901964515413\\
68.75	0.22842	-117.934159992095\\
68.75	0.23208	-123.088587119771\\
68.75	0.23574	-128.365245898442\\
68.75	0.2394	-133.764136328106\\
68.75	0.24306	-139.285258408764\\
68.75	0.24672	-144.928612140416\\
68.75	0.25038	-150.694197523062\\
68.75	0.25404	-156.582014556701\\
68.75	0.2577	-162.592063241335\\
68.75	0.26136	-168.724343576963\\
68.75	0.26502	-174.978855563584\\
68.75	0.26868	-181.3555992012\\
68.75	0.27234	-187.854574489809\\
68.75	0.276	-194.475781429413\\
69.125	0.093	-13.3831503736254\\
69.125	0.09666	-14.0177852556142\\
69.125	0.10032	-14.7746517885969\\
69.125	0.10398	-15.6537499725735\\
69.125	0.10764	-16.655079807544\\
69.125	0.1113	-17.7786412935085\\
69.125	0.11496	-19.0244344304668\\
69.125	0.11862	-20.3924592184191\\
69.125	0.12228	-21.8827156573653\\
69.125	0.12594	-23.4952037473054\\
69.125	0.1296	-25.2299234882395\\
69.125	0.13326	-27.0868748801674\\
69.125	0.13692	-29.0660579230893\\
69.125	0.14058	-31.167472617005\\
69.125	0.14424	-33.3911189619148\\
69.125	0.1479	-35.7369969578184\\
69.125	0.15156	-38.2051066047159\\
69.125	0.15522	-40.7954479026074\\
69.125	0.15888	-43.5080208514928\\
69.125	0.16254	-46.3428254513721\\
69.125	0.1662	-49.2998617022453\\
69.125	0.16986	-52.3791296041124\\
69.125	0.17352	-55.5806291569735\\
69.125	0.17718	-58.9043603608284\\
69.125	0.18084	-62.3503232156773\\
69.125	0.1845	-65.9185177215201\\
69.125	0.18816	-69.6089438783569\\
69.125	0.19182	-73.4216016861875\\
69.125	0.19548	-77.356491145012\\
69.125	0.19914	-81.4136122548305\\
69.125	0.2028	-85.5929650156429\\
69.125	0.20646	-89.8945494274492\\
69.125	0.21012	-94.3183654902495\\
69.125	0.21378	-98.8644132040436\\
69.125	0.21744	-103.532692568832\\
69.125	0.2211	-108.323203584614\\
69.125	0.22476	-113.23594625139\\
69.125	0.22842	-118.270920569159\\
69.125	0.23208	-123.428126537923\\
69.125	0.23574	-128.707564157681\\
69.125	0.2394	-134.109233428432\\
69.125	0.24306	-139.633134350178\\
69.125	0.24672	-145.279266922917\\
69.125	0.25038	-151.047631146651\\
69.125	0.25404	-156.938227021378\\
69.125	0.2577	-162.951054547099\\
69.125	0.26136	-169.086113723814\\
69.125	0.26502	-175.343404551523\\
69.125	0.26868	-181.722927030226\\
69.125	0.27234	-188.224681159923\\
69.125	0.276	-194.848666940614\\
69.5	0.093	-13.626690187575\\
69.5	0.09666	-14.2641039106512\\
69.5	0.10032	-15.0237492847213\\
69.5	0.10398	-15.9056263097854\\
69.5	0.10764	-16.9097349858433\\
69.5	0.1113	-18.0360753128952\\
69.5	0.11496	-19.284647290941\\
69.5	0.11862	-20.6554509199807\\
69.5	0.12228	-22.1484862000143\\
69.5	0.12594	-23.7637531310418\\
69.5	0.1296	-25.5012517130633\\
69.5	0.13326	-27.3609819460787\\
69.5	0.13692	-29.342943830088\\
69.5	0.14058	-31.4471373650911\\
69.5	0.14424	-33.6735625510883\\
69.5	0.1479	-36.0222193880793\\
69.5	0.15156	-38.4931078760643\\
69.5	0.15522	-41.0862280150432\\
69.5	0.15888	-43.801579805016\\
69.5	0.16254	-46.6391632459827\\
69.5	0.1662	-49.5989783379433\\
69.5	0.16986	-52.6810250808979\\
69.5	0.17352	-55.8853034748464\\
69.5	0.17718	-59.2118135197887\\
69.5	0.18084	-62.660555215725\\
69.5	0.1845	-66.2315285626553\\
69.5	0.18816	-69.9247335605795\\
69.5	0.19182	-73.7401702094975\\
69.5	0.19548	-77.6778385094095\\
69.5	0.19914	-81.7377384603153\\
69.5	0.2028	-85.9198700622152\\
69.5	0.20646	-90.2242333151089\\
69.5	0.21012	-94.6508282189966\\
69.5	0.21378	-99.1996547738782\\
69.5	0.21744	-103.870712979754\\
69.5	0.2211	-108.664002836623\\
69.5	0.22476	-113.579524344486\\
69.5	0.22842	-118.617277503344\\
69.5	0.23208	-123.777262313195\\
69.5	0.23574	-129.05947877404\\
69.5	0.2394	-134.463926885879\\
69.5	0.24306	-139.990606648712\\
69.5	0.24672	-145.639518062539\\
69.5	0.25038	-151.410661127359\\
69.5	0.25404	-157.304035843174\\
69.5	0.2577	-163.319642209983\\
69.5	0.26136	-169.457480227785\\
69.5	0.26502	-175.717549896582\\
69.5	0.26868	-182.099851216372\\
69.5	0.27234	-188.604384187156\\
69.5	0.276	-195.231148808934\\
69.875	0.093	-13.8798263586445\\
69.875	0.09666	-14.520018922808\\
69.875	0.10032	-15.2824431379656\\
69.875	0.10398	-16.167099004117\\
69.875	0.10764	-17.1739865212624\\
69.875	0.1113	-18.3031056894017\\
69.875	0.11496	-19.5544565085349\\
69.875	0.11862	-20.928038978662\\
69.875	0.12228	-22.4238530997831\\
69.875	0.12594	-24.041898871898\\
69.875	0.1296	-25.7821762950069\\
69.875	0.13326	-27.6446853691097\\
69.875	0.13692	-29.6294260942064\\
69.875	0.14058	-31.736398470297\\
69.875	0.14424	-33.9656024973816\\
69.875	0.1479	-36.3170381754601\\
69.875	0.15156	-38.7907055045325\\
69.875	0.15522	-41.3866044845987\\
69.875	0.15888	-44.104735115659\\
69.875	0.16254	-46.9450973977131\\
69.875	0.1662	-49.9076913307612\\
69.875	0.16986	-52.9925169148031\\
69.875	0.17352	-56.199574149839\\
69.875	0.17718	-59.5288630358688\\
69.875	0.18084	-62.9803835728925\\
69.875	0.1845	-66.5541357609102\\
69.875	0.18816	-70.2501195999218\\
69.875	0.19182	-74.0683350899273\\
69.875	0.19548	-78.0087822309267\\
69.875	0.19914	-82.0714610229199\\
69.875	0.2028	-86.2563714659073\\
69.875	0.20646	-90.5635135598884\\
69.875	0.21012	-94.9928873048635\\
69.875	0.21378	-99.5444927008324\\
69.875	0.21744	-104.218329747795\\
69.875	0.2211	-109.014398445752\\
69.875	0.22476	-113.932698794703\\
69.875	0.22842	-118.973230794648\\
69.875	0.23208	-124.135994445586\\
69.875	0.23574	-129.420989747519\\
69.875	0.2394	-134.828216700445\\
69.875	0.24306	-140.357675304366\\
69.875	0.24672	-146.00936555928\\
69.875	0.25038	-151.783287465188\\
69.875	0.25404	-157.67944102209\\
69.875	0.2577	-163.697826229986\\
69.875	0.26136	-169.838443088876\\
69.875	0.26502	-176.10129159876\\
69.875	0.26868	-182.486371759638\\
69.875	0.27234	-188.993683571509\\
69.875	0.276	-195.623227034375\\
70.25	0.093	-14.1425588868337\\
70.25	0.09666	-14.7855302920847\\
70.25	0.10032	-15.5507333483297\\
70.25	0.10398	-16.4381680555685\\
70.25	0.10764	-17.4478344138013\\
70.25	0.1113	-18.579732423028\\
70.25	0.11496	-19.8338620832487\\
70.25	0.11862	-21.2102233944632\\
70.25	0.12228	-22.7088163566717\\
70.25	0.12594	-24.3296409698741\\
70.25	0.1296	-26.0726972340703\\
70.25	0.13326	-27.9379851492606\\
70.25	0.13692	-29.9255047154447\\
70.25	0.14058	-32.0352559326227\\
70.25	0.14424	-34.2672388007947\\
70.25	0.1479	-36.6214533199606\\
70.25	0.15156	-39.0978994901204\\
70.25	0.15522	-41.6965773112741\\
70.25	0.15888	-44.4174867834218\\
70.25	0.16254	-47.2606279065634\\
70.25	0.1662	-50.2260006806988\\
70.25	0.16986	-53.3136051058282\\
70.25	0.17352	-56.5234411819515\\
70.25	0.17718	-59.8555089090687\\
70.25	0.18084	-63.3098082871799\\
70.25	0.1845	-66.886339316285\\
70.25	0.18816	-70.585101996384\\
70.25	0.19182	-74.4060963274769\\
70.25	0.19548	-78.3493223095637\\
70.25	0.19914	-82.4147799426444\\
70.25	0.2028	-86.6024692267191\\
70.25	0.20646	-90.9123901617878\\
70.25	0.21012	-95.3445427478502\\
70.25	0.21378	-99.8989269849066\\
70.25	0.21744	-104.575542872957\\
70.25	0.2211	-109.374390412001\\
70.25	0.22476	-114.295469602039\\
70.25	0.22842	-119.338780443071\\
70.25	0.23208	-124.504322935097\\
70.25	0.23574	-129.792097078117\\
70.25	0.2394	-135.202102872131\\
70.25	0.24306	-140.734340317139\\
70.25	0.24672	-146.388809413141\\
70.25	0.25038	-152.165510160136\\
70.25	0.25404	-158.064442558126\\
70.25	0.2577	-164.085606607109\\
70.25	0.26136	-170.229002307087\\
70.25	0.26502	-176.494629658058\\
70.25	0.26868	-182.882488660023\\
70.25	0.27234	-189.392579312982\\
70.25	0.276	-196.024901616935\\
70.625	0.093	-14.4148877721427\\
70.625	0.09666	-15.0606380184812\\
70.625	0.10032	-15.8286199158135\\
70.625	0.10398	-16.7188334641398\\
70.625	0.10764	-17.73127866346\\
70.625	0.1113	-18.8659555137742\\
70.625	0.11496	-20.1228640150822\\
70.625	0.11862	-21.5020041673842\\
70.625	0.12228	-23.0033759706801\\
70.625	0.12594	-24.6269794249699\\
70.625	0.1296	-26.3728145302536\\
70.625	0.13326	-28.2408812865312\\
70.625	0.13692	-30.2311796938028\\
70.625	0.14058	-32.3437097520683\\
70.625	0.14424	-34.5784714613277\\
70.625	0.1479	-36.935464821581\\
70.625	0.15156	-39.4146898328282\\
70.625	0.15522	-42.0161464950693\\
70.625	0.15888	-44.7398348083044\\
70.625	0.16254	-47.5857547725334\\
70.625	0.1662	-50.5539063877563\\
70.625	0.16986	-53.6442896539731\\
70.625	0.17352	-56.8569045711839\\
70.625	0.17718	-60.1917511393885\\
70.625	0.18084	-63.6488293585871\\
70.625	0.1845	-67.2281392287796\\
70.625	0.18816	-70.929680749966\\
70.625	0.19182	-74.7534539221463\\
70.625	0.19548	-78.6994587453206\\
70.625	0.19914	-82.7676952194887\\
70.625	0.2028	-86.9581633446508\\
70.625	0.20646	-91.2708631208069\\
70.625	0.21012	-95.7057945479567\\
70.625	0.21378	-100.262957626101\\
70.625	0.21744	-104.942352355238\\
70.625	0.2211	-109.74397873537\\
70.625	0.22476	-114.667836766496\\
70.625	0.22842	-119.713926448615\\
70.625	0.23208	-124.882247781728\\
70.625	0.23574	-130.172800765836\\
70.625	0.2394	-135.585585400937\\
70.625	0.24306	-141.120601687032\\
70.625	0.24672	-146.777849624121\\
70.625	0.25038	-152.557329212204\\
70.625	0.25404	-158.459040451281\\
70.625	0.2577	-164.482983341352\\
70.625	0.26136	-170.629157882417\\
70.625	0.26502	-176.897564074476\\
70.625	0.26868	-183.288201917528\\
70.625	0.27234	-189.801071411575\\
70.625	0.276	-196.436172556615\\
71	0.093	-14.6968130145715\\
71	0.09666	-15.3453421019974\\
71	0.10032	-16.1161028404172\\
71	0.10398	-17.0090952298309\\
71	0.10764	-18.0243192702386\\
71	0.1113	-19.1617749616401\\
71	0.11496	-20.4214623040356\\
71	0.11862	-21.803381297425\\
71	0.12228	-23.3075319418083\\
71	0.12594	-24.9339142371855\\
71	0.1296	-26.6825281835567\\
71	0.13326	-28.5533737809217\\
71	0.13692	-30.5464510292807\\
71	0.14058	-32.6617599286336\\
71	0.14424	-34.8993004789804\\
71	0.1479	-37.2590726803212\\
71	0.15156	-39.7410765326558\\
71	0.15522	-42.3453120359843\\
71	0.15888	-45.0717791903069\\
71	0.16254	-47.9204779956233\\
71	0.1662	-50.8914084519336\\
71	0.16986	-53.9845705592378\\
71	0.17352	-57.199964317536\\
71	0.17718	-60.537589726828\\
71	0.18084	-63.997446787114\\
71	0.1845	-67.579535498394\\
71	0.18816	-71.2838558606678\\
71	0.19182	-75.1104078739355\\
71	0.19548	-79.0591915381972\\
71	0.19914	-83.1302068534528\\
71	0.2028	-87.3234538197023\\
71	0.20646	-91.6389324369458\\
71	0.21012	-96.0766427051831\\
71	0.21378	-100.636584624414\\
71	0.21744	-105.318758194639\\
71	0.2211	-110.123163415859\\
71	0.22476	-115.049800288072\\
71	0.22842	-120.098668811279\\
71	0.23208	-125.269768985479\\
71	0.23574	-130.563100810674\\
71	0.2394	-135.978664286863\\
71	0.24306	-141.516459414045\\
71	0.24672	-147.176486192222\\
71	0.25038	-152.958744621392\\
71	0.25404	-158.863234701557\\
71	0.2577	-164.889956432715\\
71	0.26136	-171.038909814867\\
71	0.26502	-177.310094848013\\
71	0.26868	-183.703511532153\\
71	0.27234	-190.219159867287\\
71	0.276	-196.857039853415\\
71.375	0.093	-14.9883346141203\\
71.375	0.09666	-15.6396425426336\\
71.375	0.10032	-16.4131821221408\\
71.375	0.10398	-17.3089533526419\\
71.375	0.10764	-18.326956234137\\
71.375	0.1113	-19.467190766626\\
71.375	0.11496	-20.7296569501089\\
71.375	0.11862	-22.1143547845857\\
71.375	0.12228	-23.6212842700564\\
71.375	0.12594	-25.2504454065211\\
71.375	0.1296	-27.0018381939796\\
71.375	0.13326	-28.8754626324321\\
71.375	0.13692	-30.8713187218785\\
71.375	0.14058	-32.9894064623188\\
71.375	0.14424	-35.2297258537531\\
71.375	0.1479	-37.5922768961812\\
71.375	0.15156	-40.0770595896033\\
71.375	0.15522	-42.6840739340193\\
71.375	0.15888	-45.4133199294292\\
71.375	0.16254	-48.2647975758331\\
71.375	0.1662	-51.2385068732308\\
71.375	0.16986	-54.3344478216225\\
71.375	0.17352	-57.552620421008\\
71.375	0.17718	-60.8930246713875\\
71.375	0.18084	-64.3556605727609\\
71.375	0.1845	-67.9405281251283\\
71.375	0.18816	-71.6476273284895\\
71.375	0.19182	-75.4769581828447\\
71.375	0.19548	-79.4285206881938\\
71.375	0.19914	-83.5023148445368\\
71.375	0.2028	-87.6983406518738\\
71.375	0.20646	-92.0165981102046\\
71.375	0.21012	-96.4570872195294\\
71.375	0.21378	-101.019807979848\\
71.375	0.21744	-105.704760391161\\
71.375	0.2211	-110.511944453467\\
71.375	0.22476	-115.441360166768\\
71.375	0.22842	-120.493007531062\\
71.375	0.23208	-125.66688654635\\
71.375	0.23574	-130.962997212632\\
71.375	0.2394	-136.381339529909\\
71.375	0.24306	-141.921913498178\\
71.375	0.24672	-147.584719117442\\
71.375	0.25038	-153.3697563877\\
71.375	0.25404	-159.277025308952\\
71.375	0.2577	-165.306525881198\\
71.375	0.26136	-171.458258104437\\
71.375	0.26502	-177.732221978671\\
71.375	0.26868	-184.128417503898\\
71.375	0.27234	-190.64684468012\\
71.375	0.276	-197.287503507335\\
71.75	0.093	-15.2894525707887\\
71.75	0.09666	-15.9435393403894\\
71.75	0.10032	-16.7198577609841\\
71.75	0.10398	-17.6184078325726\\
71.75	0.10764	-18.6391895551551\\
71.75	0.1113	-19.7822029287315\\
71.75	0.11496	-21.0474479533019\\
71.75	0.11862	-22.4349246288661\\
71.75	0.12228	-23.9446329554242\\
71.75	0.12594	-25.5765729329763\\
71.75	0.1296	-27.3307445615223\\
71.75	0.13326	-29.2071478410622\\
71.75	0.13692	-31.205782771596\\
71.75	0.14058	-33.3266493531237\\
71.75	0.14424	-35.5697475856454\\
71.75	0.1479	-37.935077469161\\
71.75	0.15156	-40.4226390036705\\
71.75	0.15522	-43.0324321891739\\
71.75	0.15888	-45.7644570256713\\
71.75	0.16254	-48.6187135131625\\
71.75	0.1662	-51.5952016516477\\
71.75	0.16986	-54.6939214411267\\
71.75	0.17352	-57.9148728815998\\
71.75	0.17718	-61.2580559730667\\
71.75	0.18084	-64.7234707155275\\
71.75	0.1845	-68.3111171089823\\
71.75	0.18816	-72.020995153431\\
71.75	0.19182	-75.8531048488735\\
71.75	0.19548	-79.80744619531\\
71.75	0.19914	-83.8840191927405\\
71.75	0.2028	-88.0828238411648\\
71.75	0.20646	-92.4038601405831\\
71.75	0.21012	-96.8471280909953\\
71.75	0.21378	-101.412627692401\\
71.75	0.21744	-106.100358944801\\
71.75	0.2211	-110.910321848195\\
71.75	0.22476	-115.842516402583\\
71.75	0.22842	-120.896942607965\\
71.75	0.23208	-126.073600464341\\
71.75	0.23574	-131.37248997171\\
71.75	0.2394	-136.793611130074\\
71.75	0.24306	-142.336963939431\\
71.75	0.24672	-148.002548399783\\
71.75	0.25038	-153.790364511128\\
71.75	0.25404	-159.700412273467\\
71.75	0.2577	-165.7326916868\\
71.75	0.26136	-171.887202751127\\
71.75	0.26502	-178.163945466448\\
71.75	0.26868	-184.562919832763\\
71.75	0.27234	-191.084125850072\\
71.75	0.276	-197.727563518375\\
72.125	0.093	-15.600166884577\\
72.125	0.09666	-16.2570324952651\\
72.125	0.10032	-17.0361297569472\\
72.125	0.10398	-17.9374586696232\\
72.125	0.10764	-18.9610192332931\\
72.125	0.1113	-20.1068114479569\\
72.125	0.11496	-21.3748353136147\\
72.125	0.11862	-22.7650908302663\\
72.125	0.12228	-24.2775779979119\\
72.125	0.12594	-25.9122968165514\\
72.125	0.1296	-27.6692472861848\\
72.125	0.13326	-29.5484294068121\\
72.125	0.13692	-31.5498431784333\\
72.125	0.14058	-33.6734886010485\\
72.125	0.14424	-35.9193656746576\\
72.125	0.1479	-38.2874743992606\\
72.125	0.15156	-40.7778147748575\\
72.125	0.15522	-43.3903868014484\\
72.125	0.15888	-46.1251904790332\\
72.125	0.16254	-48.9822258076118\\
72.125	0.1662	-51.9614927871844\\
72.125	0.16986	-55.0629914177509\\
72.125	0.17352	-58.2867216993113\\
72.125	0.17718	-61.6326836318657\\
72.125	0.18084	-65.1008772154139\\
72.125	0.1845	-68.6913024499561\\
72.125	0.18816	-72.4039593354922\\
72.125	0.19182	-76.2388478720222\\
72.125	0.19548	-80.1959680595461\\
72.125	0.19914	-84.275319898064\\
72.125	0.2028	-88.4769033875758\\
72.125	0.20646	-92.8007185280815\\
72.125	0.21012	-97.2467653195811\\
72.125	0.21378	-101.815043762075\\
72.125	0.21744	-106.505553855562\\
72.125	0.2211	-111.318295600043\\
72.125	0.22476	-116.253268995519\\
72.125	0.22842	-121.310474041988\\
72.125	0.23208	-126.489910739451\\
72.125	0.23574	-131.791579087908\\
72.125	0.2394	-137.215479087359\\
72.125	0.24306	-142.761610737804\\
72.125	0.24672	-148.429974039243\\
72.125	0.25038	-154.220568991675\\
72.125	0.25404	-160.133395595102\\
72.125	0.2577	-166.168453849523\\
72.125	0.26136	-172.325743754937\\
72.125	0.26502	-178.605265311345\\
72.125	0.26868	-185.007018518748\\
72.125	0.27234	-191.531003377144\\
72.125	0.276	-198.177219886534\\
72.5	0.093	-15.9204775554851\\
72.5	0.09666	-16.5801220072606\\
72.5	0.10032	-17.3619981100301\\
72.5	0.10398	-18.2661058637935\\
72.5	0.10764	-19.2924452685509\\
72.5	0.1113	-20.4410163243021\\
72.5	0.11496	-21.7118190310473\\
72.5	0.11862	-23.1048533887863\\
72.5	0.12228	-24.6201193975193\\
72.5	0.12594	-26.2576170572463\\
72.5	0.1296	-28.0173463679671\\
72.5	0.13326	-29.8993073296818\\
72.5	0.13692	-31.9034999423905\\
72.5	0.14058	-34.0299242060931\\
72.5	0.14424	-36.2785801207896\\
72.5	0.1479	-38.64946768648\\
72.5	0.15156	-41.1425869031644\\
72.5	0.15522	-43.7579377708426\\
72.5	0.15888	-46.4955202895148\\
72.5	0.16254	-49.3553344591809\\
72.5	0.1662	-52.3373802798409\\
72.5	0.16986	-55.4416577514948\\
72.5	0.17352	-58.6681668741427\\
72.5	0.17718	-62.0169076477844\\
72.5	0.18084	-65.4878800724201\\
72.5	0.1845	-69.0810841480497\\
72.5	0.18816	-72.7965198746733\\
72.5	0.19182	-76.6341872522906\\
72.5	0.19548	-80.594086280902\\
72.5	0.19914	-84.6762169605073\\
72.5	0.2028	-88.8805792911065\\
72.5	0.20646	-93.2071732726996\\
72.5	0.21012	-97.6559989052867\\
72.5	0.21378	-102.227056188868\\
72.5	0.21744	-106.920345123442\\
72.5	0.2211	-111.735865709011\\
72.5	0.22476	-116.673617945574\\
72.5	0.22842	-121.733601833131\\
72.5	0.23208	-126.915817371681\\
72.5	0.23574	-132.220264561226\\
72.5	0.2394	-137.646943401764\\
72.5	0.24306	-143.195853893296\\
72.5	0.24672	-148.866996035822\\
72.5	0.25038	-154.660369829343\\
72.5	0.25404	-160.575975273857\\
72.5	0.2577	-166.613812369365\\
72.5	0.26136	-172.773881115866\\
72.5	0.26502	-179.056181513362\\
72.5	0.26868	-185.460713561852\\
72.5	0.27234	-191.987477261336\\
72.5	0.276	-198.636472611813\\
72.875	0.093	-16.250384583513\\
72.875	0.09666	-16.912807876376\\
72.875	0.10032	-17.6974628202329\\
72.875	0.10398	-18.6043494150837\\
72.875	0.10764	-19.6334676609285\\
72.875	0.1113	-20.7848175577671\\
72.875	0.11496	-22.0583991055997\\
72.875	0.11862	-23.4542123044262\\
72.875	0.12228	-24.9722571542466\\
72.875	0.12594	-26.6125336550609\\
72.875	0.1296	-28.3750418068692\\
72.875	0.13326	-30.2597816096714\\
72.875	0.13692	-32.2667530634675\\
72.875	0.14058	-34.3959561682574\\
72.875	0.14424	-36.6473909240414\\
72.875	0.1479	-39.0210573308192\\
72.875	0.15156	-41.516955388591\\
72.875	0.15522	-44.1350850973567\\
72.875	0.15888	-46.8754464571163\\
72.875	0.16254	-49.7380394678698\\
72.875	0.1662	-52.7228641296172\\
72.875	0.16986	-55.8299204423586\\
72.875	0.17352	-59.0592084060939\\
72.875	0.17718	-62.410728020823\\
72.875	0.18084	-65.8844792865461\\
72.875	0.1845	-69.4804622032632\\
72.875	0.18816	-73.1986767709741\\
72.875	0.19182	-77.0391229896789\\
72.875	0.19548	-81.0018008593778\\
72.875	0.19914	-85.0867103800704\\
72.875	0.2028	-89.2938515517571\\
72.875	0.20646	-93.6232243744376\\
72.875	0.21012	-98.0748288481121\\
72.875	0.21378	-102.64866497278\\
72.875	0.21744	-107.344732748443\\
72.875	0.2211	-112.163032175099\\
72.875	0.22476	-117.103563252749\\
72.875	0.22842	-122.166325981393\\
72.875	0.23208	-127.351320361031\\
72.875	0.23574	-132.658546391663\\
72.875	0.2394	-138.088004073289\\
72.875	0.24306	-143.639693405908\\
72.875	0.24672	-149.313614389522\\
72.875	0.25038	-155.10976702413\\
72.875	0.25404	-161.028151309731\\
72.875	0.2577	-167.068767246326\\
72.875	0.26136	-173.231614833916\\
72.875	0.26502	-179.516694072499\\
72.875	0.26868	-185.924004962076\\
72.875	0.27234	-192.453547502647\\
72.875	0.276	-199.105321694212\\
73.25	0.093	-16.5898879686607\\
73.25	0.09666	-17.2550901026111\\
73.25	0.10032	-18.0425238875554\\
73.25	0.10398	-18.9521893234936\\
73.25	0.10764	-19.9840864104258\\
73.25	0.1113	-21.1382151483519\\
73.25	0.11496	-22.4145755372719\\
73.25	0.11862	-23.8131675771858\\
73.25	0.12228	-25.3339912680937\\
73.25	0.12594	-26.9770466099954\\
73.25	0.1296	-28.7423336028911\\
73.25	0.13326	-30.6298522467807\\
73.25	0.13692	-32.6396025416642\\
73.25	0.14058	-34.7715844875416\\
73.25	0.14424	-37.025798084413\\
73.25	0.1479	-39.4022433322783\\
73.25	0.15156	-41.9009202311375\\
73.25	0.15522	-44.5218287809906\\
73.25	0.15888	-47.2649689818376\\
73.25	0.16254	-50.1303408336785\\
73.25	0.1662	-53.1179443365134\\
73.25	0.16986	-56.2277794903421\\
73.25	0.17352	-59.4598462951648\\
73.25	0.17718	-62.8141447509814\\
73.25	0.18084	-66.2906748577919\\
73.25	0.1845	-69.8894366155964\\
73.25	0.18816	-73.6104300243948\\
73.25	0.19182	-77.453655084187\\
73.25	0.19548	-81.4191117949732\\
73.25	0.19914	-85.5068001567533\\
73.25	0.2028	-89.7167201695274\\
73.25	0.20646	-94.0488718332954\\
73.25	0.21012	-98.5032551480573\\
73.25	0.21378	-103.079870113813\\
73.25	0.21744	-107.778716730563\\
73.25	0.2211	-112.599794998306\\
73.25	0.22476	-117.543104917044\\
73.25	0.22842	-122.608646486775\\
73.25	0.23208	-127.796419707501\\
73.25	0.23574	-133.10642457922\\
73.25	0.2394	-138.538661101933\\
73.25	0.24306	-144.09312927564\\
73.25	0.24672	-149.769829100341\\
73.25	0.25038	-155.568760576036\\
73.25	0.25404	-161.489923702725\\
73.25	0.2577	-167.533318480408\\
73.25	0.26136	-173.698944909085\\
73.25	0.26502	-179.986802988755\\
73.25	0.26868	-186.39689271942\\
73.25	0.27234	-192.929214101078\\
73.25	0.276	-199.583767133731\\
73.625	0.093	-16.9389877109282\\
73.625	0.09666	-17.606968685966\\
73.625	0.10032	-18.3971813119978\\
73.625	0.10398	-19.3096255890234\\
73.625	0.10764	-20.344301517043\\
73.625	0.1113	-21.5012090960565\\
73.625	0.11496	-22.7803483260639\\
73.625	0.11862	-24.1817192070653\\
73.625	0.12228	-25.7053217390605\\
73.625	0.12594	-27.3511559220497\\
73.625	0.1296	-29.1192217560328\\
73.625	0.13326	-31.0095192410098\\
73.625	0.13692	-33.0220483769808\\
73.625	0.14058	-35.1568091639456\\
73.625	0.14424	-37.4138016019044\\
73.625	0.1479	-39.7930256908571\\
73.625	0.15156	-42.2944814308037\\
73.625	0.15522	-44.9181688217442\\
73.625	0.15888	-47.6640878636787\\
73.625	0.16254	-50.532238556607\\
73.625	0.1662	-53.5226209005293\\
73.625	0.16986	-56.6352348954455\\
73.625	0.17352	-59.8700805413556\\
73.625	0.17718	-63.2271578382596\\
73.625	0.18084	-66.7064667861576\\
73.625	0.1845	-70.3080073850495\\
73.625	0.18816	-74.0317796349353\\
73.625	0.19182	-77.8777835358149\\
73.625	0.19548	-81.8460190876885\\
73.625	0.19914	-85.9364862905561\\
73.625	0.2028	-90.1491851444176\\
73.625	0.20646	-94.484115649273\\
73.625	0.21012	-98.9412778051223\\
73.625	0.21378	-103.520671611965\\
73.625	0.21744	-108.222297069803\\
73.625	0.2211	-113.046154178634\\
73.625	0.22476	-117.992242938459\\
73.625	0.22842	-123.060563349277\\
73.625	0.23208	-128.25111541109\\
73.625	0.23574	-133.563899123897\\
73.625	0.2394	-138.998914487698\\
73.625	0.24306	-144.556161502492\\
73.625	0.24672	-150.235640168281\\
73.625	0.25038	-156.037350485063\\
73.625	0.25404	-161.961292452839\\
73.625	0.2577	-168.00746607161\\
73.625	0.26136	-174.175871341374\\
73.625	0.26502	-180.466508262132\\
73.625	0.26868	-186.879376833884\\
73.625	0.27234	-193.41447705663\\
73.625	0.276	-200.071808930369\\
74	0.093	-17.2976838103155\\
74	0.09666	-17.9684436264408\\
74	0.10032	-18.7614350935599\\
74	0.10398	-19.676658211673\\
74	0.10764	-20.71411298078\\
74	0.1113	-21.873799400881\\
74	0.11496	-23.1557174719758\\
74	0.11862	-24.5598671940646\\
74	0.12228	-26.0862485671472\\
74	0.12594	-27.7348615912239\\
74	0.1296	-29.5057062662944\\
74	0.13326	-31.3987825923588\\
74	0.13692	-33.4140905694171\\
74	0.14058	-35.5516301974694\\
74	0.14424	-37.8114014765157\\
74	0.1479	-40.1934044065557\\
74	0.15156	-42.6976389875898\\
74	0.15522	-45.3241052196177\\
74	0.15888	-48.0728031026396\\
74	0.16254	-50.9437326366554\\
74	0.1662	-53.9368938216651\\
74	0.16986	-57.0522866576686\\
74	0.17352	-60.2899111446662\\
74	0.17718	-63.6497672826576\\
74	0.18084	-67.131855071643\\
74	0.1845	-70.7361745116223\\
74	0.18816	-74.4627256025956\\
74	0.19182	-78.3115083445626\\
74	0.19548	-82.2825227375237\\
74	0.19914	-86.3757687814787\\
74	0.2028	-90.5912464764275\\
74	0.20646	-94.9289558223704\\
74	0.21012	-99.3888968193071\\
74	0.21378	-103.971069467238\\
74	0.21744	-108.675473766162\\
74	0.2211	-113.502109716081\\
74	0.22476	-118.450977316993\\
74	0.22842	-123.522076568899\\
74	0.23208	-128.7154074718\\
74	0.23574	-134.030970025694\\
74	0.2394	-139.468764230582\\
74	0.24306	-145.028790086464\\
74	0.24672	-150.71104759334\\
74	0.25038	-156.515536751209\\
74	0.25404	-162.442257560073\\
74	0.2577	-168.491210019931\\
74	0.26136	-174.662394130782\\
74	0.26502	-180.955809892628\\
74	0.26868	-187.371457305467\\
74	0.27234	-193.909336369301\\
74	0.276	-200.569447084128\\
};\label{tikz:theta_surf}
\end{axis}
\end{tikzpicture}%
	\caption{The quantitative relationship between external parameters ($L_{cut}, D_{rlx}$) and internal parameters $\theta$ is obtained by fitting a polynomial surface to the internal parameter values estimated for the sets C1-C5 (\ref{tikz:thetas1}) and sets C6-C10 (\ref{tikz:thetas2}). The fitted surfaces (\ref{tikz:theta_surf}) are then used to compute the internal model parameters corresponding to the settings of choice (\ref{tikz:thetaidentified}).}\label{fig:surfaces_all}
\end{figure}
\section{Estimating from the separated experimental data}
\par The above report replicates the results of \cite{Wei2008} for a different pair of manufacturing parameters. The authors consider all datasets in model identification, however they are collected from two different foam types: sets C1-C5 are obtained from , while sets C6-C10 are obtained from . It is therefore reasonable to consider individual dynamical models for different foam types. For the first foam type, F1, sets C1, C2, C4, and C5 are used for identification, and C3 is used for validation of the dynamical model. For the second foam type, F2, set C8 is used for validation while the rest are used in the identification process. Application of the EFOR-CMSS framework is applied to two different foam types produces similar results, as can be seen in Tables \ref{tab:thetasC15}-\ref{tab:thetasC610}.
\begin{figure}[!t]
	\centering
	\subfloat[C3]{% This file was created by matlab2tikz.
% Minimal pgfplots version: 1.3
%
\definecolor{mycolor1}{rgb}{0.00000,0.44700,0.74100}%
\definecolor{mycolor2}{rgb}{0.85000,0.32500,0.09800}%
%
\begin{tikzpicture}

\begin{axis}[%
width=11.411043cm,
height=3.8cm,
at={(0cm,0cm)},
scale only axis,
xmin=1000,
xmax=1500,
ymin=-40.283,
ymax=0,
legend style={legend cell align=left,align=left,draw=white!15!black,font=\small}
]
\addplot [color=mycolor1,solid]
  table[row sep=crcr]{%
1001	-17.09\\
1002	-19.531\\
1003	-14.648\\
1004	-14.648\\
1005	-19.531\\
1006	-18.311\\
1007	-23.193\\
1008	-18.311\\
1009	-9.766\\
1010	-15.869\\
1011	-12.207\\
1012	-14.648\\
1013	-10.986\\
1014	-3.662\\
1015	-2.441\\
1016	-4.883\\
1017	-6.104\\
1018	-14.648\\
1019	-17.09\\
1020	-17.09\\
1021	-13.428\\
1022	-9.766\\
1023	-18.311\\
1024	-14.648\\
1025	-10.986\\
1026	-13.428\\
1027	-12.207\\
1028	-10.986\\
1029	-20.752\\
1030	-19.531\\
1031	-12.207\\
1032	-20.752\\
1033	-20.752\\
1034	-15.869\\
1035	-13.428\\
1036	-9.766\\
1037	-14.648\\
1038	-14.648\\
1039	-14.648\\
1040	-14.648\\
1041	-14.648\\
1042	-15.869\\
1043	-21.973\\
1044	-20.752\\
1045	-13.428\\
1046	-13.428\\
1047	-8.545\\
1048	-12.207\\
1049	-12.207\\
1050	-13.428\\
1051	-10.986\\
1052	-13.428\\
1053	-14.648\\
1054	-13.428\\
1055	-20.752\\
1056	-13.428\\
1057	-9.766\\
1058	-7.324\\
1059	-8.545\\
1060	-10.986\\
1061	-6.104\\
1062	-9.766\\
1063	-10.986\\
1064	-6.104\\
1065	-6.104\\
1066	-9.766\\
1067	-9.766\\
1068	-15.869\\
1069	-14.648\\
1070	-17.09\\
1071	-15.869\\
1072	-18.311\\
1073	-13.428\\
1074	-14.648\\
1075	-12.207\\
1076	-10.986\\
1077	-12.207\\
1078	-23.193\\
1079	-28.076\\
1080	-30.518\\
1081	-29.297\\
1082	-18.311\\
1083	-28.076\\
1084	-32.959\\
1085	-31.738\\
1086	-20.752\\
1087	-34.18\\
1088	-40.283\\
1089	-29.297\\
1090	-24.414\\
1091	-19.531\\
1092	-15.869\\
1093	-12.207\\
1094	-14.648\\
1095	-9.766\\
1096	-7.324\\
1097	-7.324\\
1098	-10.986\\
1099	-13.428\\
1100	-12.207\\
1101	-12.207\\
1102	-15.869\\
1103	-18.311\\
1104	-19.531\\
1105	-15.869\\
1106	-21.973\\
1107	-15.869\\
1108	-15.869\\
1109	-17.09\\
1110	-12.207\\
1111	-12.207\\
1112	-15.869\\
1113	-14.648\\
1114	-8.545\\
1115	-12.207\\
1116	-8.545\\
1117	-9.766\\
1118	-9.766\\
1119	-7.324\\
1120	-9.766\\
1121	-12.207\\
1122	-17.09\\
1123	-17.09\\
1124	-17.09\\
1125	-10.986\\
1126	-12.207\\
1127	-23.193\\
1128	-15.869\\
1129	-20.752\\
1130	-21.973\\
1131	-12.207\\
1132	-9.766\\
1133	-12.207\\
1134	-13.428\\
1135	-21.973\\
1136	-25.635\\
1137	-26.855\\
1138	-19.531\\
1139	-20.752\\
1140	-18.311\\
1141	-17.09\\
1142	-18.311\\
1143	-13.428\\
1144	-12.207\\
1145	-9.766\\
1146	-9.766\\
1147	-10.986\\
1148	-15.869\\
1149	-23.193\\
1150	-17.09\\
1151	-13.428\\
1152	-10.986\\
1153	-10.986\\
1154	-7.324\\
1155	-6.104\\
1156	-6.104\\
1157	-12.207\\
1158	-7.324\\
1159	-8.545\\
1160	-8.545\\
1161	-8.545\\
1162	-7.324\\
1163	-4.883\\
1164	-3.662\\
1165	-8.545\\
1166	-15.869\\
1167	-18.311\\
1168	-20.752\\
1169	-15.869\\
1170	-10.986\\
1171	-8.545\\
1172	-4.883\\
1173	-9.766\\
1174	-8.545\\
1175	-15.869\\
1176	-17.09\\
1177	-29.297\\
1178	-32.959\\
1179	-25.635\\
1180	-25.635\\
1181	-18.311\\
1182	-23.193\\
1183	-21.973\\
1184	-24.414\\
1185	-15.869\\
1186	-17.09\\
1187	-17.09\\
1188	-15.869\\
1189	-15.869\\
1190	-13.428\\
1191	-23.193\\
1192	-25.635\\
1193	-19.531\\
1194	-13.428\\
1195	-14.648\\
1196	-23.193\\
1197	-24.414\\
1198	-28.076\\
1199	-30.518\\
1200	-29.297\\
1201	-23.193\\
1202	-21.973\\
1203	-25.635\\
1204	-28.076\\
1205	-18.311\\
1206	-12.207\\
1207	-19.531\\
1208	-14.648\\
1209	-9.766\\
1210	-13.428\\
1211	-14.648\\
1212	-10.986\\
1213	-10.986\\
1214	-10.986\\
1215	-8.545\\
1216	-10.986\\
1217	-18.311\\
1218	-17.09\\
1219	-12.207\\
1220	-12.207\\
1221	-18.311\\
1222	-26.855\\
1223	-17.09\\
1224	-14.648\\
1225	-10.986\\
1226	-13.428\\
1227	-10.986\\
1228	-8.545\\
1229	-8.545\\
1230	-10.986\\
1231	-13.428\\
1232	-14.648\\
1233	-12.207\\
1234	-15.869\\
1235	-20.752\\
1236	-17.09\\
1237	-12.207\\
1238	-15.869\\
1239	-20.752\\
1240	-13.428\\
1241	-7.324\\
1242	-13.428\\
1243	-10.986\\
1244	-12.207\\
1245	-9.766\\
1246	-8.545\\
1247	-13.428\\
1248	-13.428\\
1249	-9.766\\
1250	-12.207\\
1251	-9.766\\
1252	-7.324\\
1253	-12.207\\
1254	-8.545\\
1255	-9.766\\
1256	-7.324\\
1257	-4.883\\
1258	-12.207\\
1259	-15.869\\
1260	-17.09\\
1261	-23.193\\
1262	-15.869\\
1263	-9.766\\
1264	-8.545\\
1265	-8.545\\
1266	-10.986\\
1267	-8.545\\
1268	-4.883\\
1269	-10.986\\
1270	-15.869\\
1271	-13.428\\
1272	-20.752\\
1273	-15.869\\
1274	-14.648\\
1275	-8.545\\
1276	-10.986\\
1277	-4.883\\
1278	-3.662\\
1279	-4.883\\
1280	-9.766\\
1281	-10.986\\
1282	-12.207\\
1283	-13.428\\
1284	-20.752\\
1285	-17.09\\
1286	-14.648\\
1287	-17.09\\
1288	-19.531\\
1289	-20.752\\
1290	-17.09\\
1291	-13.428\\
1292	-7.324\\
1293	-6.104\\
1294	-4.883\\
1295	-7.324\\
1296	-7.324\\
1297	-4.883\\
1298	-8.545\\
1299	-12.207\\
1300	-10.986\\
1301	-12.207\\
1302	-12.207\\
1303	-8.545\\
1304	-4.883\\
1305	-9.766\\
1306	-15.869\\
1307	-13.428\\
1308	-15.869\\
1309	-13.428\\
1310	-9.766\\
1311	-14.648\\
1312	-18.311\\
1313	-18.311\\
1314	-13.428\\
1315	-20.752\\
1316	-15.869\\
1317	-17.09\\
1318	-18.311\\
1319	-19.531\\
1320	-13.428\\
1321	-13.428\\
1322	-18.311\\
1323	-26.855\\
1324	-21.973\\
1325	-13.428\\
1326	-13.428\\
1327	-13.428\\
1328	-15.869\\
1329	-17.09\\
1330	-12.207\\
1331	-14.648\\
1332	-18.311\\
1333	-20.752\\
1334	-13.428\\
1335	-12.207\\
1336	-15.869\\
1337	-23.193\\
1338	-20.752\\
1339	-21.973\\
1340	-14.648\\
1341	-14.648\\
1342	-12.207\\
1343	-9.766\\
1344	-4.883\\
1345	-3.662\\
1346	-3.662\\
1347	-7.324\\
1348	-13.428\\
1349	-14.648\\
1350	-9.766\\
1351	-12.207\\
1352	-13.428\\
1353	-9.766\\
1354	-8.545\\
1355	-8.545\\
1356	-6.104\\
1357	-9.766\\
1358	-14.648\\
1359	-15.869\\
1360	-10.986\\
1361	-8.545\\
1362	-8.545\\
1363	-8.545\\
1364	-7.324\\
1365	-10.986\\
1366	-20.752\\
1367	-18.311\\
1368	-18.311\\
1369	-24.414\\
1370	-23.193\\
1371	-18.311\\
1372	-14.648\\
1373	-14.648\\
1374	-14.648\\
1375	-17.09\\
1376	-19.531\\
1377	-19.531\\
1378	-25.635\\
1379	-26.855\\
1380	-31.738\\
1381	-24.414\\
1382	-25.635\\
1383	-26.855\\
1384	-21.973\\
1385	-24.414\\
1386	-20.752\\
1387	-13.428\\
1388	-9.766\\
1389	-9.766\\
1390	-12.207\\
1391	-9.766\\
1392	-7.324\\
1393	-10.986\\
1394	-8.545\\
1395	-6.104\\
1396	-7.324\\
1397	-9.766\\
1398	-6.104\\
1399	-12.207\\
1400	-14.648\\
1401	-10.986\\
1402	-17.09\\
1403	-24.414\\
1404	-24.414\\
1405	-28.076\\
1406	-23.193\\
1407	-21.973\\
1408	-17.09\\
1409	-9.766\\
1410	-10.986\\
1411	-10.986\\
1412	-7.324\\
1413	-10.986\\
1414	-9.766\\
1415	-2.441\\
1416	-6.104\\
1417	-9.766\\
1418	-12.207\\
1419	-14.648\\
1420	-17.09\\
1421	-14.648\\
1422	-17.09\\
1423	-14.648\\
1424	-13.428\\
1425	-12.207\\
1426	-10.986\\
1427	-14.648\\
1428	-13.428\\
1429	-10.986\\
1430	-13.428\\
1431	-18.311\\
1432	-13.428\\
1433	-10.986\\
1434	-10.986\\
1435	-10.986\\
1436	-8.545\\
1437	-7.324\\
1438	-10.986\\
1439	-13.428\\
1440	-13.428\\
1441	-10.986\\
1442	-13.428\\
1443	-13.428\\
1444	-8.545\\
1445	-7.324\\
1446	-15.869\\
1447	-19.531\\
1448	-18.311\\
1449	-18.311\\
1450	-15.869\\
1451	-13.428\\
1452	-10.986\\
1453	-9.766\\
1454	-13.428\\
1455	-13.428\\
1456	-8.545\\
1457	-12.207\\
1458	-14.648\\
1459	-14.648\\
1460	-17.09\\
1461	-17.09\\
1462	-20.752\\
1463	-28.076\\
1464	-30.518\\
1465	-20.752\\
1466	-13.428\\
1467	-8.545\\
1468	-6.104\\
1469	-8.545\\
1470	-7.324\\
1471	-10.986\\
1472	-12.207\\
1473	-12.207\\
1474	-13.428\\
1475	-15.869\\
1476	-19.531\\
1477	-19.531\\
1478	-28.076\\
1479	-23.193\\
1480	-18.311\\
1481	-14.648\\
1482	-14.648\\
1483	-14.648\\
1484	-17.09\\
1485	-14.648\\
1486	-12.207\\
1487	-13.428\\
1488	-8.545\\
1489	-7.324\\
1490	-17.09\\
1491	-14.648\\
1492	-12.207\\
1493	-23.193\\
1494	-18.311\\
1495	-15.869\\
1496	-21.973\\
1497	-23.193\\
1498	-14.648\\
1499	-8.545\\
1500	-9.766\\
};
\addlegendentry{True output};

\addplot [color=mycolor2,dashed]
  table[row sep=crcr]{%
1001	-16.2540935145274\\
1002	-19.6781161256224\\
1003	-18.4690849743674\\
1004	-16.8081231775558\\
1005	-20.6962893637078\\
1006	-20.9225148450289\\
1007	-22.1584058942941\\
1008	-18.9517795836797\\
1009	-11.8524980717393\\
1010	-12.709369075182\\
1011	-13.817268554564\\
1012	-15.6676711184648\\
1013	-14.3382898812635\\
1014	-7.44371832376557\\
1015	-4.8414907076294\\
1016	-4.54520950726634\\
1017	-10.2801134382359\\
1018	-18.6652210243154\\
1019	-18.1545824419907\\
1020	-17.4188721457343\\
1021	-13.4249274058715\\
1022	-9.17834638982119\\
1023	-16.281506579801\\
1024	-13.7568766946723\\
1025	-10.5803850002584\\
1026	-14.2596696927907\\
1027	-15.1538148188948\\
1028	-12.4171241272467\\
1029	-17.5320414025642\\
1030	-17.7332976590538\\
1031	-13.9932256683414\\
1032	-18.0977365097215\\
1033	-18.7875109070734\\
1034	-15.4976711074121\\
1035	-13.5600002709904\\
1036	-12.1794189418506\\
1037	-13.1048703669504\\
1038	-15.7291494764788\\
1039	-15.6320751020955\\
1040	-14.7762229079978\\
1041	-15.279682345934\\
1042	-16.1264365778233\\
1043	-20.7291711625914\\
1044	-20.183965283507\\
1045	-14.5948517048798\\
1046	-13.292418777022\\
1047	-9.39777429115196\\
1048	-8.85408950655054\\
1049	-11.7980144978102\\
1050	-11.0469235500007\\
1051	-10.7570590658588\\
1052	-13.1987350630046\\
1053	-14.7947650694648\\
1054	-13.9360296042916\\
1055	-19.0825634681976\\
1056	-16.7246785075192\\
1057	-10.227567116386\\
1058	-8.69403522375991\\
1059	-11.1017000978611\\
1060	-12.9869954879128\\
1061	-9.34920015173307\\
1062	-7.73576220716166\\
1063	-10.3740367350549\\
1064	-10.0452277153996\\
1065	-8.48801067378436\\
1066	-9.63121860103035\\
1067	-10.9786153962456\\
1068	-15.0185960804688\\
1069	-15.2803652282633\\
1070	-16.1017053323231\\
1071	-15.1499185891709\\
1072	-16.8882247216557\\
1073	-15.2235285219534\\
1074	-13.9139882462441\\
1075	-13.4035699328454\\
1076	-13.3904916224081\\
1077	-13.2240812577005\\
1078	-19.447690889179\\
1079	-28.0912003265071\\
1080	-28.0496720142235\\
1081	-26.3400534895402\\
1082	-17.8008811006425\\
1083	-23.9060897679816\\
1084	-29.4192072707845\\
1085	-29.4304486303308\\
1086	-24.5330542306618\\
1087	-29.1045055999455\\
1088	-36.7573010783209\\
1089	-31.0424337395091\\
1090	-25.9569909906469\\
1091	-19.3992817539064\\
1092	-16.3203915601897\\
1093	-14.9971977754351\\
1094	-16.8283718465212\\
1095	-15.2463345537692\\
1096	-10.2241459153806\\
1097	-8.87363228006125\\
1098	-10.9631852992353\\
1099	-15.1536470896596\\
1100	-14.9935949409659\\
1101	-14.3162469502607\\
1102	-16.4746526864952\\
1103	-17.2235284745487\\
1104	-22.8150015536454\\
1105	-20.490778408016\\
1106	-21.1850646940185\\
1107	-18.2174603134295\\
1108	-15.5834538330018\\
1109	-18.3552280014343\\
1110	-15.501278784321\\
1111	-13.358225874526\\
1112	-15.8304207518969\\
1113	-16.2278909320124\\
1114	-13.6856822779349\\
1115	-12.9041083615519\\
1116	-11.3622804470723\\
1117	-10.9779509734532\\
1118	-12.7439665180143\\
1119	-9.68104291224882\\
1120	-10.3054817618142\\
1121	-12.8998421047303\\
1122	-17.4200247357197\\
1123	-17.6189392550663\\
1124	-17.4690649299764\\
1125	-12.9318884038412\\
1126	-15.7637612975256\\
1127	-23.3619858489566\\
1128	-18.9651935838016\\
1129	-20.0125388489007\\
1130	-20.0476640141925\\
1131	-14.4369581315353\\
1132	-10.8739751794717\\
1133	-13.0817587388974\\
1134	-15.4731264837244\\
1135	-22.1779365783887\\
1136	-26.5871626848186\\
1137	-24.876384466659\\
1138	-20.5079507512394\\
1139	-19.9677893587743\\
1140	-19.9422950878121\\
1141	-19.1190609171174\\
1142	-19.0098664681093\\
1143	-14.9401956411333\\
1144	-12.8987477119719\\
1145	-12.855227145262\\
1146	-12.3779012082404\\
1147	-13.0021226428845\\
1148	-17.0017624002235\\
1149	-23.1674021419941\\
1150	-20.3948390561294\\
1151	-14.3348937624356\\
1152	-12.3097910285086\\
1153	-12.7577754613764\\
1154	-9.21926986565226\\
1155	-7.03549028826926\\
1156	-7.63686745528799\\
1157	-12.0321238756377\\
1158	-11.0236340599989\\
1159	-10.3255639477249\\
1160	-10.9522468503964\\
1161	-10.4949665524402\\
1162	-8.89763598489923\\
1163	-6.66290946331524\\
1164	-5.69801686231262\\
1165	-7.31918720590012\\
1166	-14.1183060802943\\
1167	-19.9862621332491\\
1168	-20.8174653618558\\
1169	-14.3104919859448\\
1170	-10.4338996191761\\
1171	-8.16960155961766\\
1172	-6.24615431363078\\
1173	-11.1070016953635\\
1174	-12.2884086231907\\
1175	-14.9264454787321\\
1176	-18.3687486492295\\
1177	-27.8508846385197\\
1178	-32.4574143625563\\
1179	-26.714769256636\\
1180	-25.5300276460745\\
1181	-20.2589952400281\\
1182	-20.1579463046446\\
1183	-22.3867171929896\\
1184	-23.5828928351122\\
1185	-19.9501077863819\\
1186	-18.4514844286699\\
1187	-18.804483031569\\
1188	-17.5497470002858\\
1189	-19.9449466898675\\
1190	-16.7206368929492\\
1191	-21.6424467682672\\
1192	-26.219661222302\\
1193	-20.235217849024\\
1194	-13.7329435104998\\
1195	-14.3870733597551\\
1196	-21.704613093604\\
1197	-25.7398675490362\\
1198	-28.932227779186\\
1199	-30.0809341950143\\
1200	-29.7376856070797\\
1201	-24.4598930235694\\
1202	-22.9663871321836\\
1203	-25.5044825761378\\
1204	-28.1967647253629\\
1205	-21.0498932944133\\
1206	-14.0727184984223\\
1207	-18.0995816068168\\
1208	-17.4534884434623\\
1209	-12.5125085847913\\
1210	-14.6037874220264\\
1211	-16.9676712435975\\
1212	-14.5586508377998\\
1213	-11.78868273786\\
1214	-13.631417534713\\
1215	-13.3115984930181\\
1216	-15.3277294519243\\
1217	-20.0026162840161\\
1218	-18.1714620276356\\
1219	-14.127285470939\\
1220	-13.5267065793817\\
1221	-18.8817999353771\\
1222	-28.2612789985512\\
1223	-23.5923165995635\\
1224	-15.2994073684091\\
1225	-12.692048442392\\
1226	-13.6116186012182\\
1227	-14.5219937182695\\
1228	-11.7055516958389\\
1229	-12.1578244941847\\
1230	-13.4087434221936\\
1231	-16.1405868668563\\
1232	-17.2227621917768\\
1233	-12.6122401953053\\
1234	-16.2244606081285\\
1235	-19.8496668780763\\
1236	-18.971642505503\\
1237	-16.1610857006138\\
1238	-16.6111555283554\\
1239	-21.3272904383001\\
1240	-16.0667991366233\\
1241	-8.27931033263347\\
1242	-9.36496170002805\\
1243	-11.7597155512476\\
1244	-15.3974887607976\\
1245	-14.3122156520875\\
1246	-11.4075028966841\\
1247	-14.5268100814281\\
1248	-14.9392092847979\\
1249	-11.9382133881196\\
1250	-13.2739646003353\\
1251	-12.8283405155235\\
1252	-12.1369572319626\\
1253	-12.7998397638547\\
1254	-9.75416495688165\\
1255	-9.78210293070879\\
1256	-8.65775749538272\\
1257	-8.92714309141512\\
1258	-13.6384114109594\\
1259	-15.5369589716981\\
1260	-18.566836235313\\
1261	-23.5019050880935\\
1262	-17.3020843893294\\
1263	-10.9767753483691\\
1264	-8.85122071076948\\
1265	-9.104103302686\\
1266	-12.3754673096294\\
1267	-9.62022972161428\\
1268	-9.36183784102888\\
1269	-11.4386439456839\\
1270	-16.7921899076538\\
1271	-16.8550062884583\\
1272	-19.2788113665031\\
1273	-15.1179463029535\\
1274	-13.1442041955273\\
1275	-8.76735355127342\\
1276	-7.95308233834544\\
1277	-6.24804657007028\\
1278	-6.30469987812323\\
1279	-7.65683172701634\\
1280	-12.5549921415108\\
1281	-12.7642618620243\\
1282	-13.6703823072768\\
1283	-13.9271641531163\\
1284	-20.2783228079909\\
1285	-20.0037398340946\\
1286	-14.4987581678286\\
1287	-16.277009659398\\
1288	-17.2471694014645\\
1289	-21.5790679535949\\
1290	-18.7865044340178\\
1291	-13.5853647071379\\
1292	-8.38739186433385\\
1293	-6.13752318961107\\
1294	-6.72647066109547\\
1295	-8.00908897250292\\
1296	-8.37672593863177\\
1297	-8.15540132314834\\
1298	-9.27439713696989\\
1299	-12.8267799764289\\
1300	-12.8265231783683\\
1301	-11.9868249496469\\
1302	-13.2598504057457\\
1303	-9.79852066159641\\
1304	-6.18491596306425\\
1305	-9.67527253416103\\
1306	-14.3214339379838\\
1307	-14.6436388856859\\
1308	-14.3699959191078\\
1309	-13.2570315264672\\
1310	-10.823096451323\\
1311	-13.0784877764003\\
1312	-17.6012786397144\\
1313	-18.1381038231001\\
1314	-13.6082740234558\\
1315	-19.9848763237007\\
1316	-17.7785374150766\\
1317	-16.0423780210654\\
1318	-18.4950577461941\\
1319	-17.9576322521958\\
1320	-15.2499229846594\\
1321	-13.5016330958777\\
1322	-18.5728502573428\\
1323	-26.7346022983735\\
1324	-21.4403957947488\\
1325	-13.5662683843385\\
1326	-12.6948241395906\\
1327	-14.6811544161622\\
1328	-17.0679122171716\\
1329	-19.2054884874539\\
1330	-15.5593972531267\\
1331	-16.0449218201268\\
1332	-19.4230173580334\\
1333	-20.7793175553385\\
1334	-15.9366804710614\\
1335	-13.3219559667703\\
1336	-16.0121208056094\\
1337	-24.1143704670151\\
1338	-21.9826700876171\\
1339	-21.1314189966024\\
1340	-14.6798323695103\\
1341	-13.0005230463584\\
1342	-13.9707985230354\\
1343	-9.67428084122372\\
1344	-7.41582820765886\\
1345	-5.92968490260749\\
1346	-5.45544279030059\\
1347	-9.28679583125602\\
1348	-13.551186981089\\
1349	-15.2521343040952\\
1350	-11.915081740564\\
1351	-11.7331163618353\\
1352	-12.9443625214193\\
1353	-10.0640243792355\\
1354	-9.53217992120862\\
1355	-10.0150855561666\\
1356	-8.71108662632031\\
1357	-9.3926194224001\\
1358	-13.1624129098267\\
1359	-16.5856566553071\\
1360	-11.8950495609745\\
1361	-9.14177281309779\\
1362	-8.3170933623387\\
1363	-9.76675119483916\\
1364	-7.95314208640526\\
1365	-13.607765847322\\
1366	-21.5591482684397\\
1367	-17.5460344374494\\
1368	-20.4375402689759\\
1369	-24.3407030831828\\
1370	-23.4695204754495\\
1371	-19.449528642617\\
1372	-15.1138132053627\\
1373	-15.226131983169\\
1374	-16.1804759235144\\
1375	-18.0945991476306\\
1376	-19.8945222057792\\
1377	-20.2224459872784\\
1378	-23.3208077660641\\
1379	-26.5664672545925\\
1380	-30.4310739601329\\
1381	-23.7749583938735\\
1382	-23.3010529745657\\
1383	-27.3413998967943\\
1384	-23.5779287308367\\
1385	-24.3335542033364\\
1386	-23.0668685196119\\
1387	-14.1839855873625\\
1388	-10.3085905472884\\
1389	-11.1350918276436\\
1390	-15.7140042184524\\
1391	-14.2587168835076\\
1392	-10.5562829490051\\
1393	-12.2616580260489\\
1394	-10.5431607320274\\
1395	-8.42690148084056\\
1396	-9.50786236091288\\
1397	-10.5177833512145\\
1398	-8.90276514294118\\
1399	-12.2864443179189\\
1400	-16.301914849056\\
1401	-13.3154948536705\\
1402	-17.3046427054393\\
1403	-25.0983770971767\\
1404	-23.5148022225688\\
1405	-26.6856109383342\\
1406	-21.3243320855078\\
1407	-20.7886591219549\\
1408	-17.1257901998322\\
1409	-11.5251448494647\\
1410	-13.825116376832\\
1411	-12.3151370214062\\
1412	-10.0006120694871\\
1413	-12.2032023934612\\
1414	-8.78802876302736\\
1415	-6.14840260379326\\
1416	-6.77079749176925\\
1417	-11.0256219138281\\
1418	-15.4905680906093\\
1419	-17.2155755457636\\
1420	-16.7821137379026\\
1421	-15.7351225885157\\
1422	-16.9707455943554\\
1423	-15.4249759204352\\
1424	-13.2514843055385\\
1425	-13.9446048529254\\
1426	-12.4102617625233\\
1427	-16.307473765553\\
1428	-13.1629662000318\\
1429	-10.7079035542882\\
1430	-14.0981395701634\\
1431	-19.2110571163176\\
1432	-16.3233897821916\\
1433	-12.3389335588335\\
1434	-11.1153395494131\\
1435	-12.4833454427878\\
1436	-10.0524109080527\\
1437	-9.03517627212049\\
1438	-11.4987781439815\\
1439	-13.3353052837696\\
1440	-14.8517096856379\\
1441	-12.63435939169\\
1442	-13.9798240714166\\
1443	-14.0719673982978\\
1444	-9.91705813023684\\
1445	-9.37086602602507\\
1446	-14.9935300761992\\
1447	-21.0311932278284\\
1448	-19.9526785279329\\
1449	-16.9503647185865\\
1450	-16.4824810844158\\
1451	-14.2271117179446\\
1452	-13.6307906346517\\
1453	-12.0474321349579\\
1454	-14.7128009832539\\
1455	-14.4412070947729\\
1456	-10.2626949589823\\
1457	-12.8508953444511\\
1458	-14.5649441178532\\
1459	-16.9703877839933\\
1460	-17.3670542619308\\
1461	-17.0739492898177\\
1462	-21.7840748905499\\
1463	-26.371971068145\\
1464	-29.3744138285765\\
1465	-21.4243947793701\\
1466	-13.436510314024\\
1467	-9.7920687076917\\
1468	-7.23404412014515\\
1469	-9.58463033066043\\
1470	-10.6832547193847\\
1471	-14.8810151122222\\
1472	-14.5216636834786\\
1473	-12.1267642392328\\
1474	-14.5015287036726\\
1475	-16.3743645344054\\
1476	-18.0650735062645\\
1477	-19.1202230859784\\
1478	-26.8693198722355\\
1479	-24.8685389861568\\
1480	-19.9408756129964\\
1481	-15.0991530117526\\
1482	-14.4367278332308\\
1483	-16.256917618954\\
1484	-18.6629248026786\\
1485	-15.4375289050158\\
1486	-14.5312029787649\\
1487	-14.6250611620703\\
1488	-9.92851226064018\\
1489	-12.9610987990231\\
1490	-18.316335041247\\
1491	-14.7779090316173\\
1492	-14.3298923690686\\
1493	-22.6425959947508\\
1494	-19.4058233161287\\
1495	-17.3107385092493\\
1496	-22.3017476268636\\
1497	-22.0852432804429\\
1498	-16.2598265192822\\
1499	-10.410903098244\\
1500	-11.3787057337699\\
};
\addlegendentry{Generated output};

\end{axis}
\end{tikzpicture}%\label{fig:c3all}}\\
	\subfloat[C8]{% This file was created by matlab2tikz.
% Minimal pgfplots version: 1.3
%
\definecolor{mycolor1}{rgb}{0.00000,0.44700,0.74100}%
\definecolor{mycolor2}{rgb}{0.85000,0.32500,0.09800}%
%
\begin{tikzpicture}

\begin{axis}[%
width=11.411043cm,
height=3.8cm,
at={(0cm,0cm)},
scale only axis,
xmin=1000,
xmax=1500,
ymin=-300,
ymax=0,
legend style={legend cell align=left,align=left,draw=white!15!black,font=\small}
]
\addplot [color=mycolor1,solid]
  table[row sep=crcr]{%
1001	-96.436\\
1002	-122.07\\
1003	-100.098\\
1004	-93.994\\
1005	-130.615\\
1006	-125.732\\
1007	-153.809\\
1008	-115.967\\
1009	-61.035\\
1010	-74.463\\
1011	-73.242\\
1012	-86.67\\
1013	-73.242\\
1014	-34.18\\
1015	-20.752\\
1016	-23.193\\
1017	-58.594\\
1018	-100.098\\
1019	-109.863\\
1020	-114.746\\
1021	-83.008\\
1022	-52.49\\
1023	-104.98\\
1024	-81.787\\
1025	-59.814\\
1026	-97.656\\
1027	-83.008\\
1028	-72.021\\
1029	-118.408\\
1030	-107.422\\
1031	-83.008\\
1032	-119.629\\
1033	-123.291\\
1034	-95.215\\
1035	-80.566\\
1036	-62.256\\
1037	-75.684\\
1038	-95.215\\
1039	-87.891\\
1040	-91.553\\
1041	-96.436\\
1042	-102.539\\
1043	-141.602\\
1044	-128.174\\
1045	-85.449\\
1046	-79.346\\
1047	-47.607\\
1048	-48.828\\
1049	-68.359\\
1050	-52.49\\
1051	-57.373\\
1052	-75.684\\
1053	-89.111\\
1054	-81.787\\
1055	-128.174\\
1056	-98.877\\
1057	-57.373\\
1058	-45.166\\
1059	-62.256\\
1060	-72.021\\
1061	-40.283\\
1062	-42.725\\
1063	-56.152\\
1064	-46.387\\
1065	-40.283\\
1066	-52.49\\
1067	-62.256\\
1068	-92.773\\
1069	-90.332\\
1070	-107.422\\
1071	-98.877\\
1072	-115.967\\
1073	-91.553\\
1074	-91.553\\
1075	-79.346\\
1076	-75.684\\
1077	-76.904\\
1078	-133.057\\
1079	-189.209\\
1080	-194.092\\
1081	-189.209\\
1082	-119.629\\
1083	-170.898\\
1084	-213.623\\
1085	-219.727\\
1086	-169.678\\
1087	-219.727\\
1088	-279.541\\
1089	-197.754\\
1090	-161.133\\
1091	-109.863\\
1092	-85.449\\
1093	-73.242\\
1094	-91.553\\
1095	-62.256\\
1096	-46.387\\
1097	-42.725\\
1098	-54.932\\
1099	-79.346\\
1100	-74.463\\
1101	-75.684\\
1102	-100.098\\
1103	-103.76\\
1104	-141.602\\
1105	-114.746\\
1106	-142.822\\
1107	-104.98\\
1108	-90.332\\
1109	-103.76\\
1110	-76.904\\
1111	-69.58\\
1112	-84.229\\
1113	-86.67\\
1114	-59.814\\
1115	-64.697\\
1116	-48.828\\
1117	-53.711\\
1118	-57.373\\
1119	-41.504\\
1120	-52.49\\
1121	-65.918\\
1122	-101.318\\
1123	-104.98\\
1124	-114.746\\
1125	-68.359\\
1126	-93.994\\
1127	-150.146\\
1128	-114.746\\
1129	-125.732\\
1130	-128.174\\
1131	-80.566\\
1132	-53.711\\
1133	-75.684\\
1134	-85.449\\
1135	-137.939\\
1136	-172.119\\
1137	-170.898\\
1138	-123.291\\
1139	-130.615\\
1140	-115.967\\
1141	-113.525\\
1142	-111.084\\
1143	-76.904\\
1144	-68.359\\
1145	-61.035\\
1146	-57.373\\
1147	-62.256\\
1148	-91.553\\
1149	-140.381\\
1150	-111.084\\
1151	-79.346\\
1152	-64.697\\
1153	-62.256\\
1154	-40.283\\
1155	-29.297\\
1156	-36.621\\
1157	-63.477\\
1158	-51.27\\
1159	-52.49\\
1160	-58.594\\
1161	-54.932\\
1162	-37.842\\
1163	-28.076\\
1164	-23.193\\
1165	-39.063\\
1166	-83.008\\
1167	-117.188\\
1168	-130.615\\
1169	-86.67\\
1170	-58.594\\
1171	-45.166\\
1172	-32.959\\
1173	-67.139\\
1174	-65.918\\
1175	-91.553\\
1176	-117.188\\
1177	-186.768\\
1178	-216.064\\
1179	-177.002\\
1180	-168.457\\
1181	-109.863\\
1182	-122.07\\
1183	-131.836\\
1184	-146.484\\
1185	-108.643\\
1186	-100.098\\
1187	-98.877\\
1188	-89.111\\
1189	-106.201\\
1190	-78.125\\
1191	-130.615\\
1192	-159.912\\
1193	-113.525\\
1194	-70.801\\
1195	-73.242\\
1196	-123.291\\
1197	-153.809\\
1198	-189.209\\
1199	-202.637\\
1200	-202.637\\
1201	-153.809\\
1202	-137.939\\
1203	-158.691\\
1204	-187.988\\
1205	-109.863\\
1206	-72.021\\
1207	-101.318\\
1208	-86.67\\
1209	-53.711\\
1210	-74.463\\
1211	-84.229\\
1212	-67.139\\
1213	-51.27\\
1214	-70.801\\
1215	-58.594\\
1216	-79.346\\
1217	-107.422\\
1218	-100.098\\
1219	-69.58\\
1220	-70.801\\
1221	-107.422\\
1222	-177.002\\
1223	-128.174\\
1224	-79.346\\
1225	-61.035\\
1226	-70.801\\
1227	-64.697\\
1228	-48.828\\
1229	-53.711\\
1230	-65.918\\
1231	-83.008\\
1232	-91.553\\
1233	-63.477\\
1234	-104.98\\
1235	-124.512\\
1236	-118.408\\
1237	-91.553\\
1238	-100.098\\
1239	-135.498\\
1240	-87.891\\
1241	-42.725\\
1242	-57.373\\
1243	-62.256\\
1244	-81.787\\
1245	-72.021\\
1246	-52.49\\
1247	-79.346\\
1248	-80.566\\
1249	-57.373\\
1250	-70.801\\
1251	-63.477\\
1252	-56.152\\
1253	-67.139\\
1254	-41.504\\
1255	-48.828\\
1256	-36.621\\
1257	-40.283\\
1258	-75.684\\
1259	-84.229\\
1260	-113.525\\
1261	-146.484\\
1262	-98.877\\
1263	-58.594\\
1264	-46.387\\
1265	-43.945\\
1266	-63.477\\
1267	-42.725\\
1268	-47.607\\
1269	-61.035\\
1270	-102.539\\
1271	-93.994\\
1272	-134.277\\
1273	-89.111\\
1274	-81.787\\
1275	-46.387\\
1276	-45.166\\
1277	-28.076\\
1278	-35.4\\
1279	-37.842\\
1280	-67.139\\
1281	-70.801\\
1282	-79.346\\
1283	-85.449\\
1284	-125.732\\
1285	-112.305\\
1286	-84.229\\
1287	-102.539\\
1288	-109.863\\
1289	-144.043\\
1290	-115.967\\
1291	-75.684\\
1292	-40.283\\
1293	-29.297\\
1294	-29.297\\
1295	-36.621\\
1296	-39.063\\
1297	-37.842\\
1298	-50.049\\
1299	-74.463\\
1300	-68.359\\
1301	-70.801\\
1302	-83.008\\
1303	-51.27\\
1304	-30.518\\
1305	-58.594\\
1306	-90.332\\
1307	-87.891\\
1308	-97.656\\
1309	-83.008\\
1310	-61.035\\
1311	-85.449\\
1312	-118.408\\
1313	-118.408\\
1314	-84.229\\
1315	-136.719\\
1316	-100.098\\
1317	-101.318\\
1318	-117.188\\
1319	-115.967\\
1320	-86.67\\
1321	-76.904\\
1322	-128.174\\
1323	-178.223\\
1324	-139.16\\
1325	-79.346\\
1326	-76.904\\
1327	-79.346\\
1328	-98.877\\
1329	-114.746\\
1330	-85.449\\
1331	-90.332\\
1332	-120.85\\
1333	-131.836\\
1334	-87.891\\
1335	-68.359\\
1336	-91.553\\
1337	-156.25\\
1338	-137.939\\
1339	-140.381\\
1340	-84.229\\
1341	-76.904\\
1342	-72.021\\
1343	-47.607\\
1344	-30.518\\
1345	-26.855\\
1346	-23.193\\
1347	-54.932\\
1348	-70.801\\
1349	-87.891\\
1350	-65.918\\
1351	-73.242\\
1352	-83.008\\
1353	-53.711\\
1354	-56.152\\
1355	-53.711\\
1356	-40.283\\
1357	-51.27\\
1358	-84.229\\
1359	-106.201\\
1360	-63.477\\
1361	-48.828\\
1362	-40.283\\
1363	-50.049\\
1364	-35.4\\
1365	-76.904\\
1366	-130.615\\
1367	-104.98\\
1368	-139.16\\
1369	-162.354\\
1370	-164.795\\
1371	-124.512\\
1372	-90.332\\
1373	-89.111\\
1374	-89.111\\
1375	-106.201\\
1376	-125.732\\
1377	-125.732\\
1378	-168.457\\
1379	-183.105\\
1380	-219.727\\
1381	-147.705\\
1382	-166.016\\
1383	-184.326\\
1384	-140.381\\
1385	-162.354\\
1386	-139.16\\
1387	-80.566\\
1388	-52.49\\
1389	-56.152\\
1390	-81.787\\
1391	-65.918\\
1392	-43.945\\
1393	-62.256\\
1394	-47.607\\
1395	-36.621\\
1396	-46.387\\
1397	-52.49\\
1398	-36.621\\
1399	-70.801\\
1400	-92.773\\
1401	-68.359\\
1402	-114.746\\
1403	-164.795\\
1404	-150.146\\
1405	-196.533\\
1406	-131.836\\
1407	-144.043\\
1408	-96.436\\
1409	-63.477\\
1410	-76.904\\
1411	-59.814\\
1412	-45.166\\
1413	-64.697\\
1414	-36.621\\
1415	-28.076\\
1416	-34.18\\
1417	-70.801\\
1418	-86.67\\
1419	-107.422\\
1420	-103.76\\
1421	-100.098\\
1422	-114.746\\
1423	-93.994\\
1424	-76.904\\
1425	-76.904\\
1426	-62.256\\
1427	-97.656\\
1428	-68.359\\
1429	-56.152\\
1430	-81.787\\
1431	-113.525\\
1432	-86.67\\
1433	-65.918\\
1434	-56.152\\
1435	-65.918\\
1436	-45.166\\
1437	-45.166\\
1438	-59.814\\
1439	-75.684\\
1440	-84.229\\
1441	-65.918\\
1442	-86.67\\
1443	-79.346\\
1444	-47.607\\
1445	-50.049\\
1446	-95.215\\
1447	-131.836\\
1448	-131.836\\
1449	-107.422\\
1450	-103.76\\
1451	-79.346\\
1452	-76.904\\
1453	-61.035\\
1454	-89.111\\
1455	-75.684\\
1456	-50.049\\
1457	-74.463\\
1458	-85.449\\
1459	-101.318\\
1460	-109.863\\
1461	-109.863\\
1462	-148.926\\
1463	-181.885\\
1464	-205.078\\
1465	-129.395\\
1466	-73.242\\
1467	-47.607\\
1468	-34.18\\
1469	-54.932\\
1470	-46.387\\
1471	-80.566\\
1472	-72.021\\
1473	-64.697\\
1474	-85.449\\
1475	-98.877\\
1476	-117.188\\
1477	-123.291\\
1478	-175.781\\
1479	-150.146\\
1480	-117.188\\
1481	-80.566\\
1482	-80.566\\
1483	-83.008\\
1484	-106.201\\
1485	-75.684\\
1486	-79.346\\
1487	-74.463\\
1488	-42.725\\
1489	-74.463\\
1490	-107.422\\
1491	-73.242\\
1492	-80.566\\
1493	-147.705\\
1494	-109.863\\
1495	-106.201\\
1496	-147.705\\
1497	-140.381\\
1498	-87.891\\
1499	-56.152\\
1500	-58.594\\
};
\addlegendentry{True output};

\addplot [color=mycolor2,dashed]
  table[row sep=crcr]{%
1001	-85.8754828737272\\
1002	-108.781804073663\\
1003	-93.9973002769866\\
1004	-89.4771261745714\\
1005	-119.592026582798\\
1006	-105.370289539826\\
1007	-132.748169758881\\
1008	-104.931586406171\\
1009	-50.760359766167\\
1010	-64.7980172835796\\
1011	-63.0914301402359\\
1012	-84.4128274360239\\
1013	-69.2120702544636\\
1014	-26.7701405286225\\
1015	-18.3307677608845\\
1016	-15.8810194720932\\
1017	-53.0749493764915\\
1018	-97.3230195379965\\
1019	-102.863792615792\\
1020	-98.0125190856074\\
1021	-69.5590012515283\\
1022	-38.1339057422107\\
1023	-92.1826196372786\\
1024	-61.2326326765934\\
1025	-51.6238470251255\\
1026	-87.1614374431951\\
1027	-71.4206231583056\\
1028	-63.5637952842214\\
1029	-100.656767639047\\
1030	-87.9958892623159\\
1031	-70.9655144827352\\
1032	-98.0232106565681\\
1033	-94.9397061781745\\
1034	-80.7518645339737\\
1035	-68.4947003777977\\
1036	-54.3484427866507\\
1037	-66.222422873574\\
1038	-77.462815126698\\
1039	-78.0910891930115\\
1040	-77.8275332828053\\
1041	-79.7047737361657\\
1042	-80.87924289625\\
1043	-114.384750781944\\
1044	-105.334932857725\\
1045	-75.0887698962047\\
1046	-67.6857728833781\\
1047	-36.4190407751656\\
1048	-39.5084931817393\\
1049	-59.4443445023673\\
1050	-46.7754374059305\\
1051	-51.5216088476489\\
1052	-68.6412713089979\\
1053	-74.7719858765198\\
1054	-71.5513780825771\\
1055	-109.226953977909\\
1056	-79.7404146839409\\
1057	-48.1116499396354\\
1058	-39.4821581947792\\
1059	-49.3077822679495\\
1060	-63.3150669508728\\
1061	-34.8495237848091\\
1062	-38.5338736268529\\
1063	-49.2207592003144\\
1064	-40.8318311601575\\
1065	-38.4986932546994\\
1066	-47.8753930433802\\
1067	-48.9108062855549\\
1068	-86.5102467204863\\
1069	-73.0382500304068\\
1070	-91.1276352061126\\
1071	-80.5812008516425\\
1072	-87.0681591792763\\
1073	-74.7061252837774\\
1074	-71.5422481630964\\
1075	-66.2450732018288\\
1076	-66.3630561216633\\
1077	-64.9167492461641\\
1078	-105.871820150896\\
1079	-156.124437015537\\
1080	-161.950523163016\\
1081	-152.348615333437\\
1082	-97.4791446268578\\
1083	-138.238726176287\\
1084	-167.52770373281\\
1085	-175.705763629449\\
1086	-145.926358898757\\
1087	-173.466000358245\\
1088	-230.097634785732\\
1089	-177.4866765837\\
1090	-151.995372983929\\
1091	-100.855076266863\\
1092	-78.4155827886561\\
1093	-71.2111071928787\\
1094	-89.2218227288836\\
1095	-64.7730352832251\\
1096	-41.5760671167094\\
1097	-38.1975297860438\\
1098	-51.9330210549152\\
1099	-78.1822631424245\\
1100	-72.7746607516126\\
1101	-73.9525464565967\\
1102	-90.0839237714314\\
1103	-94.0697292631576\\
1104	-126.593999527618\\
1105	-111.533697347515\\
1106	-120.376634661988\\
1107	-94.8673202068574\\
1108	-79.6202322272308\\
1109	-95.1113171589113\\
1110	-72.3357646675847\\
1111	-67.8747611200713\\
1112	-83.0982847743651\\
1113	-81.8296642417015\\
1114	-64.2415914652973\\
1115	-66.0849351076157\\
1116	-48.3032714333661\\
1117	-50.710627986887\\
1118	-58.4495410441117\\
1119	-44.3351503599275\\
1120	-53.1779904651433\\
1121	-63.3659259311522\\
1122	-98.3248645344234\\
1123	-91.4234314921106\\
1124	-104.602907017345\\
1125	-57.7553561981209\\
1126	-84.0870671275606\\
1127	-127.74626093738\\
1128	-103.125420293371\\
1129	-114.127902552956\\
1130	-108.023708598769\\
1131	-68.5555180707917\\
1132	-44.7648938996969\\
1133	-67.8569555746775\\
1134	-75.562494172471\\
1135	-128.193887605102\\
1136	-153.480878951829\\
1137	-147.864159611061\\
1138	-106.517807957462\\
1139	-116.335229064909\\
1140	-102.681664668402\\
1141	-102.062143063564\\
1142	-105.155565613906\\
1143	-70.2747496388965\\
1144	-66.5956731187714\\
1145	-58.9901253282457\\
1146	-55.1199978064233\\
1147	-65.1694773500746\\
1148	-90.976934965362\\
1149	-127.914551731277\\
1150	-111.906179565242\\
1151	-72.9944822638822\\
1152	-63.877406279917\\
1153	-58.6781479539996\\
1154	-35.8638570350435\\
1155	-26.615723697124\\
1156	-33.2294146046717\\
1157	-56.350024552068\\
1158	-47.3305101171499\\
1159	-53.8164588054263\\
1160	-53.1878705929258\\
1161	-51.2095052174212\\
1162	-36.8132340883974\\
1163	-29.1912002025525\\
1164	-21.5782944211329\\
1165	-32.4735147216088\\
1166	-75.1280341270458\\
1167	-110.07712245996\\
1168	-116.119877772556\\
1169	-78.3148209170806\\
1170	-48.7262086642203\\
1171	-35.3460714439644\\
1172	-24.1732970226011\\
1173	-57.6063661347588\\
1174	-55.4797739215109\\
1175	-80.590745024896\\
1176	-101.086975792973\\
1177	-167.280985211223\\
1178	-194.111359503155\\
1179	-160.480788330829\\
1180	-142.396227707333\\
1181	-94.7776118001291\\
1182	-106.060756036876\\
1183	-120.71542124527\\
1184	-128.484682104171\\
1185	-108.391749143098\\
1186	-90.9301145534042\\
1187	-99.4739601488823\\
1188	-90.7318562779519\\
1189	-101.890289863939\\
1190	-79.7287561707335\\
1191	-123.15784091601\\
1192	-150.711628854222\\
1193	-107.698347133575\\
1194	-72.6267139389616\\
1195	-67.8929230327816\\
1196	-113.050349052799\\
1197	-142.891130681526\\
1198	-163.723596415587\\
1199	-172.263681079774\\
1200	-173.840101506615\\
1201	-133.20820689684\\
1202	-126.158386306094\\
1203	-140.596932159631\\
1204	-163.350577565389\\
1205	-101.403367973126\\
1206	-65.2392949520808\\
1207	-100.68387460091\\
1208	-84.1817914596726\\
1209	-55.9671692099501\\
1210	-75.4744004779003\\
1211	-83.2559668377285\\
1212	-69.6610171896168\\
1213	-61.6788466527582\\
1214	-71.4459611024124\\
1215	-61.9727349395218\\
1216	-74.7903532254806\\
1217	-110.024995386767\\
1218	-95.8659239871189\\
1219	-72.9685075939165\\
1220	-69.0696914560987\\
1221	-98.9451271693646\\
1222	-162.629165963199\\
1223	-131.063899196348\\
1224	-76.2724922482969\\
1225	-61.1108778172842\\
1226	-65.7493220933941\\
1227	-67.4705737104412\\
1228	-49.1843386277305\\
1229	-57.8855767676722\\
1230	-64.8106440694356\\
1231	-80.2591795894181\\
1232	-91.5656549318915\\
1233	-63.9235219651776\\
1234	-97.3874890323415\\
1235	-115.409226571994\\
1236	-104.741097935525\\
1237	-85.7315104688905\\
1238	-89.7696739013183\\
1239	-115.925120880607\\
1240	-88.8138153419355\\
1241	-33.2362300304258\\
1242	-46.5750493613946\\
1243	-54.8037438169258\\
1244	-80.0851583132459\\
1245	-73.1984232012673\\
1246	-56.4009522313722\\
1247	-80.0463834552524\\
1248	-71.205383008152\\
1249	-54.4109778165915\\
1250	-70.8093031764387\\
1251	-59.4727614311239\\
1252	-54.6786802660703\\
1253	-66.854461569196\\
1254	-39.9645973422792\\
1255	-48.7844311097502\\
1256	-34.7756340038671\\
1257	-39.2534968236081\\
1258	-73.0580150509123\\
1259	-74.7945038120957\\
1260	-100.500424007574\\
1261	-130.694666462704\\
1262	-87.8097413715591\\
1263	-51.7335837603496\\
1264	-37.2437682180042\\
1265	-36.428310508251\\
1266	-61.9996042114509\\
1267	-40.6448942373283\\
1268	-46.4610070755822\\
1269	-55.6028988910292\\
1270	-94.090375717925\\
1271	-86.114530760673\\
1272	-119.31709643747\\
1273	-81.2769915459138\\
1274	-66.8838814546598\\
1275	-34.9826116929877\\
1276	-34.4919857949797\\
1277	-23.0451769138577\\
1278	-29.7137694493322\\
1279	-34.8108343173338\\
1280	-59.4231752880484\\
1281	-64.4156261374411\\
1282	-69.059724443756\\
1283	-70.8606888334923\\
1284	-105.817339693161\\
1285	-104.228170369902\\
1286	-73.7341797370997\\
1287	-84.824861781623\\
1288	-91.8181951818021\\
1289	-117.46918168911\\
1290	-96.8506853176641\\
1291	-65.3882236776555\\
1292	-27.4951880581343\\
1293	-19.5861863043322\\
1294	-22.9011529318991\\
1295	-31.5124989894707\\
1296	-36.3970884250702\\
1297	-36.5726632861499\\
1298	-42.3789508126046\\
1299	-68.783187800463\\
1300	-62.3211593829305\\
1301	-60.32698839573\\
1302	-67.4748796805452\\
1303	-41.8625687432025\\
1304	-26.1538921455909\\
1305	-47.4367650642376\\
1306	-73.4421062343776\\
1307	-73.4570296521771\\
1308	-84.7305329671209\\
1309	-66.3032564012494\\
1310	-49.7076284906271\\
1311	-67.6605876331052\\
1312	-92.9698499345051\\
1313	-96.5343510335889\\
1314	-71.0873895620939\\
1315	-111.011894546405\\
1316	-85.6635702392849\\
1317	-80.2077854555219\\
1318	-100.800958842097\\
1319	-94.0505560369689\\
1320	-78.2904403921518\\
1321	-70.7630037974313\\
1322	-108.254692374753\\
1323	-155.321834841644\\
1324	-121.965847484607\\
1325	-71.343011770778\\
1326	-63.3982998103839\\
1327	-72.778109293937\\
1328	-86.1914133309107\\
1329	-100.446544733973\\
1330	-79.1821655615108\\
1331	-86.3853986034396\\
1332	-105.627858356745\\
1333	-108.872804606428\\
1334	-78.5683287718885\\
1335	-68.3450485718065\\
1336	-83.6251762926905\\
1337	-142.326205473331\\
1338	-125.263011179625\\
1339	-121.786263793604\\
1340	-75.6876423896701\\
1341	-63.9987081646332\\
1342	-66.3263266873355\\
1343	-41.4794497797614\\
1344	-26.8360105047331\\
1345	-21.6073916786359\\
1346	-19.6456436916613\\
1347	-45.6765154713787\\
1348	-66.0597599031811\\
1349	-81.9620281487221\\
1350	-61.6583124475608\\
1351	-60.1546623548396\\
1352	-67.4128735260722\\
1353	-41.5370277414002\\
1354	-46.8777031786818\\
1355	-44.6306935609285\\
1356	-34.5442900237571\\
1357	-46.6415533393351\\
1358	-69.8961079556226\\
1359	-90.4880960795012\\
1360	-54.7841381721051\\
1361	-42.5560214758994\\
1362	-35.7677486714521\\
1363	-44.3540701545082\\
1364	-30.8606589557725\\
1365	-71.5954299623379\\
1366	-117.822860522751\\
1367	-96.6475588693031\\
1368	-117.085711409\\
1369	-131.279660868415\\
1370	-134.934856141937\\
1371	-101.027766946531\\
1372	-74.8850441986948\\
1373	-78.6809042827616\\
1374	-77.6424757348684\\
1375	-94.6266808761659\\
1376	-109.768033904751\\
1377	-106.46951158832\\
1378	-140.678913625026\\
1379	-155.272378451781\\
1380	-179.08076231045\\
1381	-130.118086560387\\
1382	-143.109020221207\\
1383	-151.229625718028\\
1384	-127.773357247997\\
1385	-143.473012690893\\
1386	-127.514378969988\\
1387	-74.2187663031873\\
1388	-44.1384664489556\\
1389	-48.6828518279612\\
1390	-75.0991804574362\\
1391	-65.0781572443193\\
1392	-48.0032127184958\\
1393	-64.4303763965601\\
1394	-46.1487843626535\\
1395	-38.6026702441533\\
1396	-46.6237139444333\\
1397	-45.9893329538082\\
1398	-40.5271243060689\\
1399	-64.5951392091237\\
1400	-84.9272482727574\\
1401	-64.7305146287825\\
1402	-107.98313023146\\
1403	-154.572624203191\\
1404	-139.631860241814\\
1405	-169.689711478817\\
1406	-115.742342383262\\
1407	-125.923348926616\\
1408	-86.2480109490696\\
1409	-50.6560807968846\\
1410	-68.6040371798065\\
1411	-57.7797259880561\\
1412	-44.3088859573035\\
1413	-61.054590321897\\
1414	-40.009352551048\\
1415	-24.2626691610132\\
1416	-31.0051335207971\\
1417	-61.1037681846905\\
1418	-84.1613195415776\\
1419	-98.3165509610044\\
1420	-94.8525852816802\\
1421	-83.1199634688726\\
1422	-93.3747601807919\\
1423	-78.6579207117266\\
1424	-67.1800292347031\\
1425	-69.3658128631267\\
1426	-57.0801436611721\\
1427	-88.6938610508641\\
1428	-59.3585937452503\\
1429	-52.3271960956573\\
1430	-77.1927254200006\\
1431	-97.508206263408\\
1432	-78.2429097725323\\
1433	-61.2222073119417\\
1434	-53.7294359328824\\
1435	-60.7583375143265\\
1436	-42.3134806310156\\
1437	-41.9696253740684\\
1438	-56.4873252738479\\
1439	-64.9890454368193\\
1440	-75.05565438552\\
1441	-62.9352855038035\\
1442	-74.2985882418572\\
1443	-66.704466936198\\
1444	-41.9108333070298\\
1445	-45.8928191576249\\
1446	-82.52246281871\\
1447	-122.290622681842\\
1448	-114.611565341893\\
1449	-92.9291254699301\\
1450	-85.7553952851546\\
1451	-71.5009989231504\\
1452	-66.0035738779185\\
1453	-58.9968377406385\\
1454	-74.922577354245\\
1455	-73.6595868927587\\
1456	-46.2221452331677\\
1457	-70.5867893065855\\
1458	-72.0294647986512\\
1459	-86.7304282484407\\
1460	-97.9146955620086\\
1461	-88.8821151742579\\
1462	-130.048630359193\\
1463	-153.802532842985\\
1464	-171.910104120197\\
1465	-119.781244899518\\
1466	-59.5548677112292\\
1467	-34.5461278325035\\
1468	-23.6584511829667\\
1469	-43.9699601180694\\
1470	-41.4966413763577\\
1471	-77.2747034271776\\
1472	-67.5500886488865\\
1473	-61.1455167096611\\
1474	-76.372216177297\\
1475	-82.7227323554793\\
1476	-100.454153366441\\
1477	-104.926944561177\\
1478	-150.565338819325\\
1479	-137.570002559045\\
1480	-103.062302416873\\
1481	-75.9489696351768\\
1482	-69.1771185031479\\
1483	-81.0939605726314\\
1484	-96.977723885695\\
1485	-74.454371713104\\
1486	-74.1612610562207\\
1487	-77.8982401417518\\
1488	-36.462731911008\\
1489	-68.8614352555426\\
1490	-96.6518289382086\\
1491	-71.7691641593888\\
1492	-71.9603646143258\\
1493	-138.91588085194\\
1494	-95.8533543303091\\
1495	-98.000383831908\\
1496	-126.559357084979\\
1497	-118.629086159487\\
1498	-83.1280904899516\\
1499	-51.5169250053964\\
1500	-51.4926867563813\\
};
\addlegendentry{Generated output};

\end{axis}
\end{tikzpicture}%\label{fig:c8all}}
	\caption{Samples of the output obtained experimentally and the output generated by the identified model.}\label{fig:Callout}
\end{figure}
\begin{table}[!h]
	\centering
	\caption{Estimated parameters for the datasets obtained from foam F1.}\label{tab:thetasC15}
	\small
	\begin{tabular}{rrrrrrr}
Step & Terms & C1 & C2 & C4 & C5 & AEER($\%$) \\ 
\hline 
1 & $x_4,x_4$ & -26.04 & -20.99 & -10.69 & -10.96 & 90.313 \\ 
2 & $x_3$ & 75.42 & 59.58 & 33 & 26.06 & 7.772 \\ 
3 & $x_1,x_4$ & 0.62 & 0.76 & 0.32 & 0.48 & 0.072 \\ 
4 & $x_1,x_1$ & 0.01 & -0.19 & -0.15 & -0.22 & 0.035 \\ 
5 & $x_2$ & 0.71 & -0.73 & -2.24 & -0.66 & 0.005 \\ 
6 & $x_4$ & -171.24 & -139.22 & -69.61 & -73.69 & 0.002 \\ 
7 & $c$ & -233.16 & -200.83 & -93.74 & -119.7 & 0.242 \\ 
8 & $x_3,x_4$ & 15.47 & 12.1 & 6.36 & 5.68 & 0.071 \\ 
\hline 
\end{tabular}
\end{table}
\begin{table}[!h]
	\centering
	\caption{Estimated parameters for the datasets obtained from foam F2.}\label{tab:thetasC610}
	\small
	\begin{tabular}{rrrrrrr}
Step & Terms & C6 & C7 & C9 & C10 & AEER($\%$) \\ 
\hline 
1 & $x_4,x_4$ & -199.23 & -163.93 & -90.75 & -71.69 & 88.709 \\ 
2 & $x_3$ & 670.73 & 553.59 & 311.92 & 238.03 & 9.927 \\ 
3 & $x_1,x_4$ & 36.26 & 30.69 & 13.76 & 9.54 & 0.207 \\ 
4 & $x_2$ & -237.96 & -196.97 & -97.99 & -69.39 & 0.066 \\ 
5 & $x_1,x_1$ & 4.36 & 2.85 & 2.01 & 2.31 & 0.048 \\ 
6 & $x_4$ & -1318.98 & -1084.72 & -600.14 & -477.45 & 0.009 \\ 
7 & $c$ & -1665.09 & -1377.67 & -752.29 & -598.01 & 0.374 \\ 
8 & $x_3,x_4$ & 137.39 & 113 & 63.49 & 48.62 & 0.116 \\ 
9 & $x_1$ & 215.63 & 175.27 & 84.61 & 68.21 & 0.003 \\ 
10 & $x_2,x_4$ & -47.82 & -39.33 & -19.6 & -13.79 & 0.006 \\ 
\hline 
\end{tabular}
\end{table}
However, difficulties may arise during the stage of mapping the external parameters onto the internal parameter space. Since only 4 data points are estimated for each foam type, a bilinear model is used for surface fitting
\begin{equation}
\theta_i(L_{cut},D_{rlx}) = \beta_0 + \beta_1 L_{cut} + \beta_2 D_{rlx} + \beta_3 L_{cut} D_{rlx}, \qquad i=1,\dots,N_s.
\end{equation}
Model coefficients are estimated with LS method and are presented in Tables \ref{tab:betasC15}-\ref{tab:betasC610}.
\begin{remark}
	It may be possible to perform the direct estimation of the polynomial coefficients from the time series instead of estimating internal parameters in an intermediate step. This would ensure identifiability of higher order polynomials. However, a batch problem of estimating internal parameters from all datasets must be formulated first in order to avoid rank deficiency in the LS matrices. This will be investigated in the future.
\end{remark}
 \begin{table}[!h]
	\centering
	\caption{Estimated polynomial coefficients for the model describing the dynamics of  foam F1.}\label{tab:betasC15}
	\small
	\begin{tabular}{rrrrr}
Terms & $\beta_0$ & $\beta_1$ & $\beta_2$ & $\beta_3$ \\ 
\hline 
$x_4,x_4$ & 752.37 & -8.64 & -8838.21 & 104.16 \\ 
$x_3$ & 1634.9 & -21.1 & -18933.22 & 256.09 \\ 
$x_1,x_4$ & -200.43 & 2.46 & 2307.82 & -29.14 \\ 
$x_1,x_1$ & 93.51 & -1.17 & -1072.53 & 13.72 \\ 
$x_2$ & -429.54 & 4.88 & 5130.04 & -61.73 \\ 
$x_4$ & 6565.82 & -76.57 & -76864.82 & 921.08 \\ 
$c$ & 22443.12 & -269.32 & -261223.28 & 3228.31 \\ 
$x_3,x_4$ & 50.76 & -0.94 & -534.9 & 10.96 \\ 
\hline 
\end{tabular}
\end{table}
\begin{table}[!h]
	\centering
	\caption{Estimated polynomial coefficients for the model describing the dynamics of  foam F1. foam F2.}\label{tab:betasC610}
	\small
	\begin{tabular}{rrrrr}
Terms & $\beta_0$ & $\beta_1$ & $\beta_2$ & $\beta_3$ \\ 
\hline 
$x_4,x_4$ & 13.44 & 3.89 & -876.45 & -12.5 \\ 
$x_3$ & 923.68 & -25.09 & -3650.85 & 131.7 \\ 
$x_1,x_4$ & -943.69 & 12.18 & 5604.01 & -74.79 \\ 
$x_2$ & 2240.92 & -25.9 & -13596.07 & 162.72 \\ 
$x_1,x_1$ & 355.33 & -5.08 & -1986.17 & 29.06 \\ 
$x_4$ & 144.45 & 25.34 & -6447.46 & -74.73 \\ 
$c$ & 5763.57 & -44.84 & -40741.32 & 368.83 \\ 
$x_3,x_4$ & 306.47 & -6.82 & -1400.59 & 36.47 \\ 
$x_1$ & -1733.21 & 18.12 & 11504.04 & -131.98 \\ 
$x_2,x_4$ & 325.67 & -3.48 & -2002.72 & 22.29 \\ 
\hline 
\end{tabular}
\end{table}

\begin{figure}[!t]
	\centering
	% This file was created by matlab2tikz.
% Minimal pgfplots version: 1.3
%
\definecolor{mycolor1}{rgb}{0.00000,0.44700,0.74100}%
%
\begin{tikzpicture}

\begin{axis}[%
width=5.011742cm,
height=3.595207cm,
at={(0cm,10.936529cm)},
scale only axis,
xmin=55,
xmax=75,
tick align=outside,
xlabel={$L_{cut}$},
xmajorgrids,
ymin=0.09,
ymax=0.13,
ylabel={$D_{rlx}$},
ymajorgrids,
zmin=-3.92997043424299,
zmax=20.7603892264508,
zlabel={$x_1,x_4$},
zmajorgrids,
view={-140}{50},
legend style={at={(1.03,1)},anchor=north west,legend cell align=left,align=left,draw=white!15!black}
]
\addplot3[only marks,mark=*,mark options={},mark size=1.5000pt,color=mycolor1] plot table[row sep=crcr,]{%
74	0.123	0.616147082677301\\
72	0.113	0.756244217181634\\
61	0.095	0.315136730386088\\
56	0.093	0.479225561787365\\
};
\addplot3[only marks,mark=*,mark options={},mark size=1.5000pt,color=black] plot table[row sep=crcr,]{%
69	0.104	0.564359576646865\\
};

\addplot3[%
surf,
opacity=0.7,
shader=interp,
colormap={mymap}{[1pt] rgb(0pt)=(0.0901961,0.239216,0.0745098); rgb(1pt)=(0.0945149,0.242058,0.0739522); rgb(2pt)=(0.0988592,0.244894,0.0733566); rgb(3pt)=(0.103229,0.247724,0.0727241); rgb(4pt)=(0.107623,0.250549,0.0720557); rgb(5pt)=(0.112043,0.253367,0.0713525); rgb(6pt)=(0.116487,0.25618,0.0706154); rgb(7pt)=(0.120956,0.258986,0.0698456); rgb(8pt)=(0.125449,0.261787,0.0690441); rgb(9pt)=(0.129967,0.264581,0.0682118); rgb(10pt)=(0.134508,0.26737,0.06735); rgb(11pt)=(0.139074,0.270152,0.0664596); rgb(12pt)=(0.143663,0.272929,0.0655416); rgb(13pt)=(0.148275,0.275699,0.0645971); rgb(14pt)=(0.152911,0.278463,0.0636271); rgb(15pt)=(0.15757,0.281221,0.0626328); rgb(16pt)=(0.162252,0.283973,0.0616151); rgb(17pt)=(0.166957,0.286719,0.060575); rgb(18pt)=(0.171685,0.289458,0.0595136); rgb(19pt)=(0.176434,0.292191,0.0584321); rgb(20pt)=(0.181207,0.294918,0.0573313); rgb(21pt)=(0.186001,0.297639,0.0562123); rgb(22pt)=(0.190817,0.300353,0.0550763); rgb(23pt)=(0.195655,0.303061,0.0539242); rgb(24pt)=(0.200514,0.305763,0.052757); rgb(25pt)=(0.205395,0.308459,0.0515759); rgb(26pt)=(0.210296,0.311149,0.0503624); rgb(27pt)=(0.215212,0.313846,0.0490067); rgb(28pt)=(0.220142,0.316548,0.0475043); rgb(29pt)=(0.22509,0.319254,0.0458704); rgb(30pt)=(0.230056,0.321962,0.0441205); rgb(31pt)=(0.235042,0.324671,0.04227); rgb(32pt)=(0.240048,0.327379,0.0403343); rgb(33pt)=(0.245078,0.330085,0.0383287); rgb(34pt)=(0.250131,0.332786,0.0362688); rgb(35pt)=(0.25521,0.335482,0.0341698); rgb(36pt)=(0.260317,0.33817,0.0320472); rgb(37pt)=(0.265451,0.340849,0.0299163); rgb(38pt)=(0.270616,0.343517,0.0277927); rgb(39pt)=(0.275813,0.346172,0.0256916); rgb(40pt)=(0.281043,0.348814,0.0236284); rgb(41pt)=(0.286307,0.35144,0.0216186); rgb(42pt)=(0.291607,0.354048,0.0196776); rgb(43pt)=(0.296945,0.356637,0.0178207); rgb(44pt)=(0.302322,0.359206,0.0160634); rgb(45pt)=(0.307739,0.361753,0.0144211); rgb(46pt)=(0.313198,0.364275,0.0129091); rgb(47pt)=(0.318701,0.366772,0.0115428); rgb(48pt)=(0.324249,0.369242,0.0103377); rgb(49pt)=(0.329843,0.371682,0.00930909); rgb(50pt)=(0.335485,0.374093,0.00847245); rgb(51pt)=(0.341176,0.376471,0.00784314); rgb(52pt)=(0.346925,0.378826,0.00732741); rgb(53pt)=(0.352735,0.381168,0.00682184); rgb(54pt)=(0.358605,0.383497,0.00632729); rgb(55pt)=(0.364532,0.385812,0.00584464); rgb(56pt)=(0.370516,0.388113,0.00537476); rgb(57pt)=(0.376552,0.390399,0.00491852); rgb(58pt)=(0.38264,0.39267,0.00447681); rgb(59pt)=(0.388777,0.394925,0.00405048); rgb(60pt)=(0.394962,0.397164,0.00364042); rgb(61pt)=(0.401191,0.399386,0.00324749); rgb(62pt)=(0.407464,0.401592,0.00287258); rgb(63pt)=(0.413777,0.40378,0.00251655); rgb(64pt)=(0.420129,0.40595,0.00218028); rgb(65pt)=(0.426518,0.408102,0.00186463); rgb(66pt)=(0.432942,0.410234,0.00157049); rgb(67pt)=(0.439399,0.412348,0.00129873); rgb(68pt)=(0.445885,0.414441,0.00105022); rgb(69pt)=(0.452401,0.416515,0.000825833); rgb(70pt)=(0.458942,0.418567,0.000626441); rgb(71pt)=(0.465508,0.420599,0.00045292); rgb(72pt)=(0.472096,0.422609,0.000306141); rgb(73pt)=(0.478704,0.424596,0.000186979); rgb(74pt)=(0.485331,0.426562,9.63073e-05); rgb(75pt)=(0.491973,0.428504,3.49981e-05); rgb(76pt)=(0.498628,0.430422,3.92506e-06); rgb(77pt)=(0.505323,0.432315,0); rgb(78pt)=(0.512206,0.434168,0); rgb(79pt)=(0.519282,0.435983,0); rgb(80pt)=(0.526529,0.437764,0); rgb(81pt)=(0.533922,0.439512,0); rgb(82pt)=(0.54144,0.441232,0); rgb(83pt)=(0.549059,0.442927,0); rgb(84pt)=(0.556756,0.444599,0); rgb(85pt)=(0.564508,0.446252,0); rgb(86pt)=(0.572292,0.447889,0); rgb(87pt)=(0.580084,0.449514,0); rgb(88pt)=(0.587863,0.451129,0); rgb(89pt)=(0.595604,0.452737,0); rgb(90pt)=(0.603284,0.454343,0); rgb(91pt)=(0.610882,0.455948,0); rgb(92pt)=(0.618373,0.457556,0); rgb(93pt)=(0.625734,0.459171,0); rgb(94pt)=(0.632943,0.460795,0); rgb(95pt)=(0.639976,0.462432,0); rgb(96pt)=(0.64681,0.464084,0); rgb(97pt)=(0.653423,0.465756,0); rgb(98pt)=(0.659791,0.46745,0); rgb(99pt)=(0.665891,0.469169,0); rgb(100pt)=(0.6717,0.470916,0); rgb(101pt)=(0.677195,0.472696,0); rgb(102pt)=(0.682353,0.47451,0); rgb(103pt)=(0.687242,0.476355,0); rgb(104pt)=(0.691952,0.478225,0); rgb(105pt)=(0.696497,0.480118,0); rgb(106pt)=(0.700887,0.482033,0); rgb(107pt)=(0.705134,0.483968,0); rgb(108pt)=(0.709251,0.485921,0); rgb(109pt)=(0.713249,0.487891,0); rgb(110pt)=(0.71714,0.489876,0); rgb(111pt)=(0.720936,0.491875,0); rgb(112pt)=(0.724649,0.493887,0); rgb(113pt)=(0.72829,0.495909,0); rgb(114pt)=(0.731872,0.49794,0); rgb(115pt)=(0.735406,0.499979,0); rgb(116pt)=(0.738904,0.502025,0); rgb(117pt)=(0.742378,0.504075,0); rgb(118pt)=(0.74584,0.506128,0); rgb(119pt)=(0.749302,0.508182,0); rgb(120pt)=(0.752775,0.510237,0); rgb(121pt)=(0.756272,0.51229,0); rgb(122pt)=(0.759804,0.514339,0); rgb(123pt)=(0.763384,0.516385,0); rgb(124pt)=(0.767022,0.518424,0); rgb(125pt)=(0.770731,0.520455,0); rgb(126pt)=(0.774523,0.522478,0); rgb(127pt)=(0.77841,0.524489,0); rgb(128pt)=(0.782391,0.526491,0); rgb(129pt)=(0.786402,0.528496,0); rgb(130pt)=(0.790431,0.530506,0); rgb(131pt)=(0.794478,0.532521,0); rgb(132pt)=(0.798541,0.534539,0); rgb(133pt)=(0.802619,0.53656,0); rgb(134pt)=(0.806712,0.538584,0); rgb(135pt)=(0.81082,0.540609,0); rgb(136pt)=(0.81494,0.542635,0); rgb(137pt)=(0.819074,0.54466,0); rgb(138pt)=(0.823219,0.546686,0); rgb(139pt)=(0.827374,0.548709,0); rgb(140pt)=(0.831541,0.55073,0); rgb(141pt)=(0.835716,0.552749,0); rgb(142pt)=(0.8399,0.554763,0); rgb(143pt)=(0.844092,0.556774,0); rgb(144pt)=(0.848292,0.558779,0); rgb(145pt)=(0.852497,0.560778,0); rgb(146pt)=(0.856708,0.562771,0); rgb(147pt)=(0.860924,0.564756,0); rgb(148pt)=(0.865143,0.566733,0); rgb(149pt)=(0.869366,0.568701,0); rgb(150pt)=(0.873592,0.57066,0); rgb(151pt)=(0.877819,0.572608,0); rgb(152pt)=(0.882047,0.574545,0); rgb(153pt)=(0.886275,0.576471,0); rgb(154pt)=(0.890659,0.578362,0); rgb(155pt)=(0.895333,0.580203,0); rgb(156pt)=(0.900258,0.581999,0); rgb(157pt)=(0.905397,0.583755,0); rgb(158pt)=(0.910711,0.585479,0); rgb(159pt)=(0.916164,0.587176,0); rgb(160pt)=(0.921717,0.588852,0); rgb(161pt)=(0.927333,0.590513,0); rgb(162pt)=(0.932974,0.592166,0); rgb(163pt)=(0.938602,0.593815,0); rgb(164pt)=(0.94418,0.595468,0); rgb(165pt)=(0.949669,0.59713,0); rgb(166pt)=(0.955033,0.598808,0); rgb(167pt)=(0.960233,0.600507,0); rgb(168pt)=(0.965232,0.602233,0); rgb(169pt)=(0.969992,0.603992,0); rgb(170pt)=(0.974475,0.605791,0); rgb(171pt)=(0.978643,0.607636,0); rgb(172pt)=(0.98246,0.609532,0); rgb(173pt)=(0.985886,0.611486,0); rgb(174pt)=(0.988885,0.613503,0); rgb(175pt)=(0.991419,0.61559,0); rgb(176pt)=(0.99345,0.617753,0); rgb(177pt)=(0.99494,0.619997,0); rgb(178pt)=(0.995851,0.622329,0); rgb(179pt)=(0.996226,0.624763,0); rgb(180pt)=(0.996512,0.627352,0); rgb(181pt)=(0.996788,0.630095,0); rgb(182pt)=(0.997053,0.632982,0); rgb(183pt)=(0.997308,0.636004,0); rgb(184pt)=(0.997552,0.639152,0); rgb(185pt)=(0.997785,0.642416,0); rgb(186pt)=(0.998006,0.645786,0); rgb(187pt)=(0.998217,0.649253,0); rgb(188pt)=(0.998416,0.652807,0); rgb(189pt)=(0.998605,0.656439,0); rgb(190pt)=(0.998781,0.660138,0); rgb(191pt)=(0.998946,0.663897,0); rgb(192pt)=(0.9991,0.667704,0); rgb(193pt)=(0.999242,0.67155,0); rgb(194pt)=(0.999372,0.675427,0); rgb(195pt)=(0.99949,0.679323,0); rgb(196pt)=(0.999596,0.68323,0); rgb(197pt)=(0.99969,0.687139,0); rgb(198pt)=(0.999771,0.691039,0); rgb(199pt)=(0.999841,0.694921,0); rgb(200pt)=(0.999898,0.698775,0); rgb(201pt)=(0.999942,0.702592,0); rgb(202pt)=(0.999974,0.706363,0); rgb(203pt)=(0.999994,0.710077,0); rgb(204pt)=(1,0.713725,0); rgb(205pt)=(1,0.717341,0); rgb(206pt)=(1,0.720963,0); rgb(207pt)=(1,0.724591,0); rgb(208pt)=(1,0.728226,0); rgb(209pt)=(1,0.731867,0); rgb(210pt)=(1,0.735514,0); rgb(211pt)=(1,0.739167,0); rgb(212pt)=(1,0.742827,0); rgb(213pt)=(1,0.746493,0); rgb(214pt)=(1,0.750165,0); rgb(215pt)=(1,0.753843,0); rgb(216pt)=(1,0.757527,0); rgb(217pt)=(1,0.761217,0); rgb(218pt)=(1,0.764913,0); rgb(219pt)=(1,0.768615,0); rgb(220pt)=(1,0.772324,0); rgb(221pt)=(1,0.776038,0); rgb(222pt)=(1,0.779758,0); rgb(223pt)=(1,0.783484,0); rgb(224pt)=(1,0.787215,0); rgb(225pt)=(1,0.790953,0); rgb(226pt)=(1,0.794696,0); rgb(227pt)=(1,0.798445,0); rgb(228pt)=(1,0.8022,0); rgb(229pt)=(1,0.805961,0); rgb(230pt)=(1,0.809727,0); rgb(231pt)=(1,0.8135,0); rgb(232pt)=(1,0.817278,0); rgb(233pt)=(1,0.821063,0); rgb(234pt)=(1,0.824854,0); rgb(235pt)=(1,0.828652,0); rgb(236pt)=(1,0.832455,0); rgb(237pt)=(1,0.836265,0); rgb(238pt)=(1,0.840081,0); rgb(239pt)=(1,0.843903,0); rgb(240pt)=(1,0.847732,0); rgb(241pt)=(1,0.851566,0); rgb(242pt)=(1,0.855406,0); rgb(243pt)=(1,0.859253,0); rgb(244pt)=(1,0.863106,0); rgb(245pt)=(1,0.866964,0); rgb(246pt)=(1,0.870829,0); rgb(247pt)=(1,0.8747,0); rgb(248pt)=(1,0.878577,0); rgb(249pt)=(1,0.88246,0); rgb(250pt)=(1,0.886349,0); rgb(251pt)=(1,0.890243,0); rgb(252pt)=(1,0.894144,0); rgb(253pt)=(1,0.898051,0); rgb(254pt)=(1,0.901964,0); rgb(255pt)=(1,0.905882,0)},
mesh/rows=49]
table[row sep=crcr,header=false] {%
%
56	0.093	0.479225561779344\\
56	0.0936	0.884848835072773\\
56	0.0942	1.29047210836623\\
56	0.0948	1.69609538165963\\
56	0.0954	2.10171865495306\\
56	0.096	2.50734192824649\\
56	0.0966	2.91296520153992\\
56	0.0972	3.31858847483335\\
56	0.0978	3.72421174812678\\
56	0.0984	4.12983502142021\\
56	0.099	4.53545829471364\\
56	0.0996	4.94108156800704\\
56	0.1002	5.34670484130049\\
56	0.1008	5.75232811459392\\
56	0.1014	6.15795138788735\\
56	0.102	6.56357466118075\\
56	0.1026	6.96919793447421\\
56	0.1032	7.37482120776764\\
56	0.1038	7.78044448106104\\
56	0.1044	8.1860677543545\\
56	0.105	8.59169102764793\\
56	0.1056	8.99731430094133\\
56	0.1062	9.40293757423478\\
56	0.1068	9.80856084752821\\
56	0.1074	10.2141841208216\\
56	0.108	10.6198073941151\\
56	0.1086	11.0254306674085\\
56	0.1092	11.4310539407019\\
56	0.1098	11.8366772139953\\
56	0.1104	12.2423004872888\\
56	0.111	12.6479237605822\\
56	0.1116	13.0535470338756\\
56	0.1122	13.459170307169\\
56	0.1128	13.8647935804625\\
56	0.1134	14.2704168537559\\
56	0.114	14.6760401270494\\
56	0.1146	15.0816634003428\\
56	0.1152	15.4872866736362\\
56	0.1158	15.8929099469296\\
56	0.1164	16.298533220223\\
56	0.117	16.7041564935165\\
56	0.1176	17.1097797668099\\
56	0.1182	17.5154030401033\\
56	0.1188	17.9210263133968\\
56	0.1194	18.3266495866902\\
56	0.12	18.7322728599836\\
56	0.1206	19.1378961332771\\
56	0.1212	19.5435194065705\\
56	0.1218	19.9491426798639\\
56	0.1224	20.3547659531573\\
56	0.123	20.7603892264508\\
56.375	0.093	0.387367311862192\\
56.375	0.0936	0.78643431592738\\
56.375	0.0942	1.1855013199926\\
56.375	0.0948	1.58456832405778\\
56.375	0.0954	1.983635328123\\
56.375	0.096	2.38270233218819\\
56.375	0.0966	2.7817693362534\\
56.375	0.0972	3.18083634031859\\
56.375	0.0978	3.57990334438378\\
56.375	0.0984	3.978970348449\\
56.375	0.099	4.37803735251421\\
56.375	0.0996	4.7771043565794\\
56.375	0.1002	5.17617136064462\\
56.375	0.1008	5.5752383647098\\
56.375	0.1014	5.97430536877499\\
56.375	0.102	6.37337237284018\\
56.375	0.1026	6.7724393769054\\
56.375	0.1032	7.17150638097061\\
56.375	0.1038	7.5705733850358\\
56.375	0.1044	7.96964038910102\\
56.375	0.105	8.3687073931662\\
56.375	0.1056	8.76777439723139\\
56.375	0.1062	9.16684140129661\\
56.375	0.1068	9.56590840536182\\
56.375	0.1074	9.96497540942701\\
56.375	0.108	10.3640424134922\\
56.375	0.1086	10.7631094175574\\
56.375	0.1092	11.1621764216226\\
56.375	0.1098	11.5612434256878\\
56.375	0.1104	11.960310429753\\
56.375	0.111	12.3593774338183\\
56.375	0.1116	12.7584444378834\\
56.375	0.1122	13.1575114419486\\
56.375	0.1128	13.5565784460138\\
56.375	0.1134	13.955645450079\\
56.375	0.114	14.3547124541442\\
56.375	0.1146	14.7537794582094\\
56.375	0.1152	15.1528464622747\\
56.375	0.1158	15.5519134663398\\
56.375	0.1164	15.950980470405\\
56.375	0.117	16.3500474744702\\
56.375	0.1176	16.7491144785354\\
56.375	0.1182	17.1481814826006\\
56.375	0.1188	17.5472484866658\\
56.375	0.1194	17.9463154907311\\
56.375	0.12	18.3453824947962\\
56.375	0.1206	18.7444494988614\\
56.375	0.1212	19.1435165029266\\
56.375	0.1218	19.5425835069918\\
56.375	0.1224	19.941650511057\\
56.375	0.123	20.3407175151222\\
56.75	0.093	0.295509061945069\\
56.75	0.0936	0.688019796782044\\
56.75	0.0942	1.08053053161902\\
56.75	0.0948	1.47304126645599\\
56.75	0.0954	1.86555200129294\\
56.75	0.096	2.25806273612992\\
56.75	0.0966	2.65057347096692\\
56.75	0.0972	3.04308420580386\\
56.75	0.0978	3.43559494064084\\
56.75	0.0984	3.82810567547784\\
56.75	0.099	4.22061641031479\\
56.75	0.0996	4.61312714515176\\
56.75	0.1002	5.00563787998874\\
56.75	0.1008	5.39814861482571\\
56.75	0.1014	5.79065934966269\\
56.75	0.102	6.18317008449964\\
56.75	0.1026	6.57568081933664\\
56.75	0.1032	6.96819155417361\\
56.75	0.1038	7.36070228901056\\
56.75	0.1044	7.75321302384756\\
56.75	0.105	8.14572375868451\\
56.75	0.1056	8.53823449352149\\
56.75	0.1062	8.93074522835849\\
56.75	0.1068	9.32325596319544\\
56.75	0.1074	9.71576669803241\\
56.75	0.108	10.1082774328694\\
56.75	0.1086	10.5007881677064\\
56.75	0.1092	10.8932989025433\\
56.75	0.1098	11.2858096373803\\
56.75	0.1104	11.6783203722173\\
56.75	0.111	12.0708311070543\\
56.75	0.1116	12.4633418418912\\
56.75	0.1122	12.8558525767282\\
56.75	0.1128	13.2483633115652\\
56.75	0.1134	13.6408740464022\\
56.75	0.114	14.0333847812391\\
56.75	0.1146	14.4258955160761\\
56.75	0.1152	14.8184062509131\\
56.75	0.1158	15.2109169857501\\
56.75	0.1164	15.603427720587\\
56.75	0.117	15.995938455424\\
56.75	0.1176	16.388449190261\\
56.75	0.1182	16.7809599250979\\
56.75	0.1188	17.1734706599349\\
56.75	0.1194	17.5659813947719\\
56.75	0.12	17.9584921296089\\
56.75	0.1206	18.3510028644458\\
56.75	0.1212	18.7435135992828\\
56.75	0.1218	19.1360243341198\\
56.75	0.1224	19.5285350689568\\
56.75	0.123	19.9210458037937\\
57.125	0.093	0.203650812027973\\
57.125	0.0936	0.589605277636707\\
57.125	0.0942	0.975559743245469\\
57.125	0.0948	1.3615142088542\\
57.125	0.0954	1.74746867446294\\
57.125	0.096	2.13342314007167\\
57.125	0.0966	2.51937760568043\\
57.125	0.0972	2.90533207128917\\
57.125	0.0978	3.2912865368979\\
57.125	0.0984	3.67724100250669\\
57.125	0.099	4.06319546811542\\
57.125	0.0996	4.44914993372416\\
57.125	0.1002	4.83510439933292\\
57.125	0.1008	5.22105886494165\\
57.125	0.1014	5.60701333055039\\
57.125	0.102	5.99296779615912\\
57.125	0.1026	6.37892226176788\\
57.125	0.1032	6.76487672737662\\
57.125	0.1038	7.15083119298538\\
57.125	0.1044	7.53678565859414\\
57.125	0.105	7.92274012420287\\
57.125	0.1056	8.30869458981161\\
57.125	0.1062	8.69464905542037\\
57.125	0.1068	9.0806035210291\\
57.125	0.1074	9.46655798663784\\
57.125	0.108	9.8525124522466\\
57.125	0.1086	10.2384669178553\\
57.125	0.1092	10.6244213834641\\
57.125	0.1098	11.0103758490728\\
57.125	0.1104	11.3963303146816\\
57.125	0.111	11.7822847802904\\
57.125	0.1116	12.1682392458991\\
57.125	0.1122	12.5541937115078\\
57.125	0.1128	12.9401481771166\\
57.125	0.1134	13.3261026427253\\
57.125	0.114	13.7120571083341\\
57.125	0.1146	14.0980115739428\\
57.125	0.1152	14.4839660395516\\
57.125	0.1158	14.8699205051603\\
57.125	0.1164	15.255874970769\\
57.125	0.117	15.6418294363778\\
57.125	0.1176	16.0277839019865\\
57.125	0.1182	16.4137383675953\\
57.125	0.1188	16.799692833204\\
57.125	0.1194	17.1856472988128\\
57.125	0.12	17.5716017644215\\
57.125	0.1206	17.9575562300303\\
57.125	0.1212	18.343510695639\\
57.125	0.1218	18.7294651612478\\
57.125	0.1224	19.1154196268565\\
57.125	0.123	19.5013740924652\\
57.5	0.093	0.11179256211085\\
57.5	0.0936	0.491190758491342\\
57.5	0.0942	0.870588954871891\\
57.5	0.0948	1.24998715125238\\
57.5	0.0954	1.6293853476329\\
57.5	0.096	2.0087835440134\\
57.5	0.0966	2.38818174039395\\
57.5	0.0972	2.76757993677447\\
57.5	0.0978	3.14697813315496\\
57.5	0.0984	3.52637632953551\\
57.5	0.099	3.905774525916\\
57.5	0.0996	4.28517272229652\\
57.5	0.1002	4.66457091867704\\
57.5	0.1008	5.04396911505756\\
57.5	0.1014	5.42336731143808\\
57.5	0.102	5.80276550781858\\
57.5	0.1026	6.18216370419913\\
57.5	0.1032	6.56156190057962\\
57.5	0.1038	6.94096009696014\\
57.5	0.1044	7.32035829334069\\
57.5	0.105	7.69975648972118\\
57.5	0.1056	8.0791546861017\\
57.5	0.1062	8.45855288248222\\
57.5	0.1068	8.83795107886274\\
57.5	0.1074	9.21734927524324\\
57.5	0.108	9.59674747162379\\
57.5	0.1086	9.97614566800431\\
57.5	0.1092	10.3555438643848\\
57.5	0.1098	10.7349420607653\\
57.5	0.1104	11.1143402571458\\
57.5	0.111	11.4937384535264\\
57.5	0.1116	11.8731366499069\\
57.5	0.1122	12.2525348462874\\
57.5	0.1128	12.631933042668\\
57.5	0.1134	13.0113312390484\\
57.5	0.114	13.390729435429\\
57.5	0.1146	13.7701276318095\\
57.5	0.1152	14.14952582819\\
57.5	0.1158	14.5289240245705\\
57.5	0.1164	14.908322220951\\
57.5	0.117	15.2877204173316\\
57.5	0.1176	15.6671186137121\\
57.5	0.1182	16.0465168100926\\
57.5	0.1188	16.4259150064731\\
57.5	0.1194	16.8053132028536\\
57.5	0.12	17.1847113992341\\
57.5	0.1206	17.5641095956146\\
57.5	0.1212	17.9435077919952\\
57.5	0.1218	18.3229059883757\\
57.5	0.1224	18.7023041847562\\
57.5	0.123	19.0817023811367\\
57.875	0.093	0.0199343121936693\\
57.875	0.0936	0.392776239345977\\
57.875	0.0942	0.765618166498285\\
57.875	0.0948	1.13846009365056\\
57.875	0.0954	1.51130202080284\\
57.875	0.096	1.88414394795512\\
57.875	0.0966	2.25698587510743\\
57.875	0.0972	2.62982780225971\\
57.875	0.0978	3.00266972941199\\
57.875	0.0984	3.3755116565643\\
57.875	0.099	3.74835358371658\\
57.875	0.0996	4.12119551086886\\
57.875	0.1002	4.49403743802117\\
57.875	0.1008	4.86687936517345\\
57.875	0.1014	5.23972129232573\\
57.875	0.102	5.61256321947801\\
57.875	0.1026	5.98540514663031\\
57.875	0.1032	6.35824707378259\\
57.875	0.1038	6.7310890009349\\
57.875	0.1044	7.10393092808718\\
57.875	0.105	7.47677285523949\\
57.875	0.1056	7.84961478239177\\
57.875	0.1062	8.22245670954408\\
57.875	0.1068	8.59529863669636\\
57.875	0.1074	8.96814056384864\\
57.875	0.108	9.34098249100094\\
57.875	0.1086	9.71382441815322\\
57.875	0.1092	10.0866663453055\\
57.875	0.1098	10.4595082724578\\
57.875	0.1104	10.8323501996101\\
57.875	0.111	11.2051921267624\\
57.875	0.1116	11.5780340539147\\
57.875	0.1122	11.9508759810669\\
57.875	0.1128	12.3237179082193\\
57.875	0.1134	12.6965598353716\\
57.875	0.114	13.0694017625239\\
57.875	0.1146	13.4422436896761\\
57.875	0.1152	13.8150856168284\\
57.875	0.1158	14.1879275439807\\
57.875	0.1164	14.560769471133\\
57.875	0.117	14.9336113982853\\
57.875	0.1176	15.3064533254376\\
57.875	0.1182	15.6792952525899\\
57.875	0.1188	16.0521371797421\\
57.875	0.1194	16.4249791068945\\
57.875	0.12	16.7978210340467\\
57.875	0.1206	17.170662961199\\
57.875	0.1212	17.5435048883513\\
57.875	0.1218	17.9163468155036\\
57.875	0.1224	18.2891887426559\\
57.875	0.123	18.6620306698082\\
58.25	0.093	-0.0719239377234544\\
58.25	0.0936	0.294361720200612\\
58.25	0.0942	0.660647378124679\\
58.25	0.0948	1.02693303604875\\
58.25	0.0954	1.39321869397278\\
58.25	0.096	1.75950435189685\\
58.25	0.0966	2.12579000982095\\
58.25	0.0972	2.49207566774498\\
58.25	0.0978	2.85836132566905\\
58.25	0.0984	3.22464698359312\\
58.25	0.099	3.59093264151718\\
58.25	0.0996	3.95721829944122\\
58.25	0.1002	4.32350395736532\\
58.25	0.1008	4.68978961528936\\
58.25	0.1014	5.05607527321342\\
58.25	0.102	5.42236093113746\\
58.25	0.1026	5.78864658906156\\
58.25	0.1032	6.1549322469856\\
58.25	0.1038	6.52121790490966\\
58.25	0.1044	6.88750356283373\\
58.25	0.105	7.2537892207578\\
58.25	0.1056	7.62007487868186\\
58.25	0.1062	7.98636053660593\\
58.25	0.1068	8.35264619453\\
58.25	0.1074	8.71893185245403\\
58.25	0.108	9.08521751037813\\
58.25	0.1086	9.45150316830217\\
58.25	0.1092	9.81778882622623\\
58.25	0.1098	10.1840744841503\\
58.25	0.1104	10.5503601420744\\
58.25	0.111	10.9166457999984\\
58.25	0.1116	11.2829314579225\\
58.25	0.1122	11.6492171158465\\
58.25	0.1128	12.0155027737706\\
58.25	0.1134	12.3817884316947\\
58.25	0.114	12.7480740896187\\
58.25	0.1146	13.1143597475428\\
58.25	0.1152	13.4806454054669\\
58.25	0.1158	13.8469310633909\\
58.25	0.1164	14.213216721315\\
58.25	0.117	14.5795023792391\\
58.25	0.1176	14.9457880371631\\
58.25	0.1182	15.3120736950872\\
58.25	0.1188	15.6783593530112\\
58.25	0.1194	16.0446450109353\\
58.25	0.12	16.4109306688594\\
58.25	0.1206	16.7772163267834\\
58.25	0.1212	17.1435019847075\\
58.25	0.1218	17.5097876426316\\
58.25	0.1224	17.8760733005556\\
58.25	0.123	18.2423589584797\\
58.625	0.093	-0.163782187640578\\
58.625	0.0936	0.195947201055247\\
58.625	0.0942	0.555676589751101\\
58.625	0.0948	0.915405978446927\\
58.625	0.0954	1.27513536714275\\
58.625	0.096	1.63486475583858\\
58.625	0.0966	1.99459414453443\\
58.625	0.0972	2.35432353323026\\
58.625	0.0978	2.71405292192608\\
58.625	0.0984	3.07378231062194\\
58.625	0.099	3.43351169931776\\
58.625	0.0996	3.79324108801359\\
58.625	0.1002	4.15297047670944\\
58.625	0.1008	4.51269986540527\\
58.625	0.1014	4.87242925410109\\
58.625	0.102	5.23215864279692\\
58.625	0.1026	5.59188803149277\\
58.625	0.1032	5.9516174201886\\
58.625	0.1038	6.31134680888445\\
58.625	0.1044	6.67107619758028\\
58.625	0.105	7.0308055862761\\
58.625	0.1056	7.39053497497196\\
58.625	0.1062	7.75026436366781\\
58.625	0.1068	8.10999375236364\\
58.625	0.1074	8.46972314105946\\
58.625	0.108	8.82945252975532\\
58.625	0.1086	9.18918191845114\\
58.625	0.1092	9.54891130714697\\
58.625	0.1098	9.90864069584279\\
58.625	0.1104	10.2683700845386\\
58.625	0.111	10.6280994732345\\
58.625	0.1116	10.9878288619303\\
58.625	0.1122	11.3475582506261\\
58.625	0.1128	11.707287639322\\
58.625	0.1134	12.0670170280178\\
58.625	0.114	12.4267464167137\\
58.625	0.1146	12.7864758054095\\
58.625	0.1152	13.1462051941053\\
58.625	0.1158	13.5059345828012\\
58.625	0.1164	13.865663971497\\
58.625	0.117	14.2253933601928\\
58.625	0.1176	14.5851227488887\\
58.625	0.1182	14.9448521375845\\
58.625	0.1188	15.3045815262803\\
58.625	0.1194	15.6643109149762\\
58.625	0.12	16.024040303672\\
58.625	0.1206	16.3837696923678\\
58.625	0.1212	16.7434990810636\\
58.625	0.1218	17.1032284697595\\
58.625	0.1224	17.4629578584553\\
58.625	0.123	17.8226872471512\\
59	0.093	-0.255640437557702\\
59	0.0936	0.0975326819099109\\
59	0.0942	0.450705801377524\\
59	0.0948	0.803878920845136\\
59	0.0954	1.15705204031272\\
59	0.096	1.51022515978033\\
59	0.0966	1.86339827924795\\
59	0.0972	2.21657139871556\\
59	0.0978	2.56974451818314\\
59	0.0984	2.92291763765078\\
59	0.099	3.27609075711837\\
59	0.0996	3.62926387658595\\
59	0.1002	3.98243699605359\\
59	0.1008	4.33561011552121\\
59	0.1014	4.68878323498879\\
59	0.102	5.04195635445637\\
59	0.1026	5.39512947392402\\
59	0.1032	5.7483025933916\\
59	0.1038	6.10147571285921\\
59	0.1044	6.45464883232682\\
59	0.105	6.80782195179444\\
59	0.1056	7.16099507126202\\
59	0.1062	7.51416819072966\\
59	0.1068	7.86734131019728\\
59	0.1074	8.22051442966486\\
59	0.108	8.57368754913247\\
59	0.1086	8.92686066860009\\
59	0.1092	9.28003378806767\\
59	0.1098	9.63320690753528\\
59	0.1104	9.98638002700289\\
59	0.111	10.3395531464705\\
59	0.1116	10.6927262659381\\
59	0.1122	11.0458993854057\\
59	0.1128	11.3990725048733\\
59	0.1134	11.752245624341\\
59	0.114	12.1054187438085\\
59	0.1146	12.4585918632762\\
59	0.1152	12.8117649827438\\
59	0.1158	13.1649381022114\\
59	0.1164	13.518111221679\\
59	0.117	13.8712843411466\\
59	0.1176	14.2244574606142\\
59	0.1182	14.5776305800818\\
59	0.1188	14.9308036995494\\
59	0.1194	15.283976819017\\
59	0.12	15.6371499384846\\
59	0.1206	15.9903230579522\\
59	0.1212	16.3434961774198\\
59	0.1218	16.6966692968874\\
59	0.1224	17.049842416355\\
59	0.123	17.4030155358226\\
59.375	0.093	-0.347498687474854\\
59.375	0.0936	-0.000881837235482408\\
59.375	0.0942	0.345735013003917\\
59.375	0.0948	0.692351863243289\\
59.375	0.0954	1.03896871348266\\
59.375	0.096	1.38558556372203\\
59.375	0.0966	1.73220241396143\\
59.375	0.0972	2.0788192642008\\
59.375	0.0978	2.42543611444017\\
59.375	0.0984	2.77205296467957\\
59.375	0.099	3.11866981491895\\
59.375	0.0996	3.46528666515832\\
59.375	0.1002	3.81190351539772\\
59.375	0.1008	4.15852036563709\\
59.375	0.1014	4.50513721587646\\
59.375	0.102	4.8517540661158\\
59.375	0.1026	5.1983709163552\\
59.375	0.1032	5.54498776659457\\
59.375	0.1038	5.89160461683397\\
59.375	0.1044	6.23822146707334\\
59.375	0.105	6.58483831731272\\
59.375	0.1056	6.93145516755209\\
59.375	0.1062	7.27807201779149\\
59.375	0.1068	7.62468886803086\\
59.375	0.1074	7.97130571827023\\
59.375	0.108	8.31792256850963\\
59.375	0.1086	8.664539418749\\
59.375	0.1092	9.01115626898837\\
59.375	0.1098	9.35777311922774\\
59.375	0.1104	9.70438996946714\\
59.375	0.111	10.0510068197065\\
59.375	0.1116	10.3976236699459\\
59.375	0.1122	10.7442405201853\\
59.375	0.1128	11.0908573704247\\
59.375	0.1134	11.4374742206641\\
59.375	0.114	11.7840910709034\\
59.375	0.1146	12.1307079211428\\
59.375	0.1152	12.4773247713822\\
59.375	0.1158	12.8239416216216\\
59.375	0.1164	13.1705584718609\\
59.375	0.117	13.5171753221003\\
59.375	0.1176	13.8637921723397\\
59.375	0.1182	14.2104090225791\\
59.375	0.1188	14.5570258728185\\
59.375	0.1194	14.9036427230579\\
59.375	0.12	15.2502595732972\\
59.375	0.1206	15.5968764235366\\
59.375	0.1212	15.943493273776\\
59.375	0.1218	16.2901101240153\\
59.375	0.1224	16.6367269742547\\
59.375	0.123	16.9833438244941\\
59.75	0.093	-0.439356937391977\\
59.75	0.0936	-0.0992963563808473\\
59.75	0.0942	0.24076422463034\\
59.75	0.0948	0.58082480564147\\
59.75	0.0954	0.920885386652628\\
59.75	0.096	1.26094596766376\\
59.75	0.0966	1.60100654867495\\
59.75	0.0972	1.94106712968608\\
59.75	0.0978	2.28112771069721\\
59.75	0.0984	2.62118829170839\\
59.75	0.099	2.96124887271952\\
59.75	0.0996	3.30130945373068\\
59.75	0.1002	3.64137003474184\\
59.75	0.1008	3.981430615753\\
59.75	0.1014	4.32149119676413\\
59.75	0.102	4.66155177777529\\
59.75	0.1026	5.00161235878645\\
59.75	0.1032	5.3416729397976\\
59.75	0.1038	5.68173352080873\\
59.75	0.1044	6.02179410181989\\
59.75	0.105	6.36185468283105\\
59.75	0.1056	6.70191526384218\\
59.75	0.1062	7.04197584485337\\
59.75	0.1068	7.3820364258645\\
59.75	0.1074	7.72209700687566\\
59.75	0.108	8.06215758788682\\
59.75	0.1086	8.40221816889797\\
59.75	0.1092	8.7422787499091\\
59.75	0.1098	9.08233933092023\\
59.75	0.1104	9.42239991193142\\
59.75	0.111	9.76246049294258\\
59.75	0.1116	10.1025210739537\\
59.75	0.1122	10.4425816549648\\
59.75	0.1128	10.7826422359761\\
59.75	0.1134	11.1227028169872\\
59.75	0.114	11.4627633979983\\
59.75	0.1146	11.8028239790095\\
59.75	0.1152	12.1428845600206\\
59.75	0.1158	12.4829451410318\\
59.75	0.1164	12.8230057220429\\
59.75	0.117	13.1630663030541\\
59.75	0.1176	13.5031268840652\\
59.75	0.1182	13.8431874650764\\
59.75	0.1188	14.1832480460875\\
59.75	0.1194	14.5233086270987\\
59.75	0.12	14.8633692081098\\
59.75	0.1206	15.203429789121\\
59.75	0.1212	15.5434903701321\\
59.75	0.1218	15.8835509511433\\
59.75	0.1224	16.2236115321545\\
59.75	0.123	16.5636721131656\\
60.125	0.093	-0.531215187309101\\
60.125	0.0936	-0.197710875526184\\
60.125	0.0942	0.135793436256762\\
60.125	0.0948	0.469297748039679\\
60.125	0.0954	0.802802059822596\\
60.125	0.096	1.13630637160549\\
60.125	0.0966	1.46981068338843\\
60.125	0.0972	1.80331499517135\\
60.125	0.0978	2.13681930695427\\
60.125	0.0984	2.47032361873721\\
60.125	0.099	2.80382793052013\\
60.125	0.0996	3.13733224230305\\
60.125	0.1002	3.47083655408599\\
60.125	0.1008	3.80434086586891\\
60.125	0.1014	4.13784517765183\\
60.125	0.102	4.47134948943474\\
60.125	0.1026	4.80485380121769\\
60.125	0.1032	5.13835811300061\\
60.125	0.1038	5.4718624247835\\
60.125	0.1044	5.80536673656644\\
60.125	0.105	6.13887104834936\\
60.125	0.1056	6.47237536013228\\
60.125	0.1062	6.80587967191522\\
60.125	0.1068	7.13938398369814\\
60.125	0.1074	7.47288829548106\\
60.125	0.108	7.806392607264\\
60.125	0.1086	8.13989691904692\\
60.125	0.1092	8.47340123082984\\
60.125	0.1098	8.80690554261275\\
60.125	0.1104	9.1404098543957\\
60.125	0.111	9.47391416617864\\
60.125	0.1116	9.80741847796153\\
60.125	0.1122	10.1409227897444\\
60.125	0.1128	10.4744271015274\\
60.125	0.1134	10.8079314133103\\
60.125	0.114	11.1414357250932\\
60.125	0.1146	11.4749400368761\\
60.125	0.1152	11.8084443486591\\
60.125	0.1158	12.141948660442\\
60.125	0.1164	12.4754529722249\\
60.125	0.117	12.8089572840079\\
60.125	0.1176	13.1424615957908\\
60.125	0.1182	13.4759659075737\\
60.125	0.1188	13.8094702193566\\
60.125	0.1194	14.1429745311395\\
60.125	0.12	14.4764788429225\\
60.125	0.1206	14.8099831547054\\
60.125	0.1212	15.1434874664883\\
60.125	0.1218	15.4769917782712\\
60.125	0.1224	15.8104960900542\\
60.125	0.123	16.1440004018371\\
60.5	0.093	-0.623073437226225\\
60.5	0.0936	-0.296125394671549\\
60.5	0.0942	0.0308226478831841\\
60.5	0.0948	0.35777069043786\\
60.5	0.0954	0.684718732992536\\
60.5	0.096	1.01166677554724\\
60.5	0.0966	1.33861481810195\\
60.5	0.0972	1.66556286065662\\
60.5	0.0978	1.99251090321133\\
60.5	0.0984	2.31945894576603\\
60.5	0.099	2.64640698832073\\
60.5	0.0996	2.97335503087541\\
60.5	0.1002	3.30030307343014\\
60.5	0.1008	3.62725111598482\\
60.5	0.1014	3.9541991585395\\
60.5	0.102	4.2811472010942\\
60.5	0.1026	4.6080952436489\\
60.5	0.1032	4.93504328620361\\
60.5	0.1038	5.26199132875828\\
60.5	0.1044	5.58893937131299\\
60.5	0.105	5.91588741386769\\
60.5	0.1056	6.24283545642237\\
60.5	0.1062	6.5697834989771\\
60.5	0.1068	6.89673154153178\\
60.5	0.1074	7.22367958408645\\
60.5	0.108	7.55062762664119\\
60.5	0.1086	7.87757566919586\\
60.5	0.1092	8.20452371175054\\
60.5	0.1098	8.53147175430524\\
60.5	0.1104	8.85841979685995\\
60.5	0.111	9.18536783941468\\
60.5	0.1116	9.51231588196936\\
60.5	0.1122	9.839263924524\\
60.5	0.1128	10.1662119670788\\
60.5	0.1134	10.4931600096334\\
60.5	0.114	10.8201080521881\\
60.5	0.1146	11.1470560947428\\
60.5	0.1152	11.4740041372975\\
60.5	0.1158	11.8009521798522\\
60.5	0.1164	12.1279002224069\\
60.5	0.117	12.4548482649616\\
60.5	0.1176	12.7817963075163\\
60.5	0.1182	13.108744350071\\
60.5	0.1188	13.4356923926257\\
60.5	0.1194	13.7626404351804\\
60.5	0.12	14.0895884777351\\
60.5	0.1206	14.4165365202898\\
60.5	0.1212	14.7434845628445\\
60.5	0.1218	15.0704326053992\\
60.5	0.1224	15.3973806479539\\
60.5	0.123	15.7243286905086\\
60.875	0.093	-0.714931687143377\\
60.875	0.0936	-0.394539913816914\\
60.875	0.0942	-0.0741481404904505\\
60.875	0.0948	0.246243632836013\\
60.875	0.0954	0.566635406162476\\
60.875	0.096	0.887027179488939\\
60.875	0.0966	1.20741895281543\\
60.875	0.0972	1.52781072614189\\
60.875	0.0978	1.84820249946836\\
60.875	0.0984	2.16859427279485\\
60.875	0.099	2.48898604612128\\
60.875	0.0996	2.80937781944775\\
60.875	0.1002	3.12976959277424\\
60.875	0.1008	3.4501613661007\\
60.875	0.1014	3.77055313942716\\
60.875	0.102	4.09094491275363\\
60.875	0.1026	4.41133668608012\\
60.875	0.1032	4.73172845940658\\
60.875	0.1038	5.05212023273302\\
60.875	0.1044	5.37251200605951\\
60.875	0.105	5.69290377938597\\
60.875	0.1056	6.01329555271244\\
60.875	0.1062	6.33368732603893\\
60.875	0.1068	6.65407909936539\\
60.875	0.1074	6.97447087269185\\
60.875	0.108	7.29486264601834\\
60.875	0.1086	7.61525441934478\\
60.875	0.1092	7.93564619267124\\
60.875	0.1098	8.25603796599771\\
60.875	0.1104	8.5764297393242\\
60.875	0.111	8.89682151265069\\
60.875	0.1116	9.21721328597715\\
60.875	0.1122	9.53760505930359\\
60.875	0.1128	9.85799683263011\\
60.875	0.1134	10.1783886059565\\
60.875	0.114	10.498780379283\\
60.875	0.1146	10.8191721526095\\
60.875	0.1152	11.139563925936\\
60.875	0.1158	11.4599556992624\\
60.875	0.1164	11.7803474725889\\
60.875	0.117	12.1007392459154\\
60.875	0.1176	12.4211310192418\\
60.875	0.1182	12.7415227925683\\
60.875	0.1188	13.0619145658947\\
60.875	0.1194	13.3823063392212\\
60.875	0.12	13.7026981125477\\
60.875	0.1206	14.0230898858742\\
60.875	0.1212	14.3434816592006\\
60.875	0.1218	14.6638734325271\\
60.875	0.1224	14.9842652058535\\
60.875	0.123	15.30465697918\\
61.25	0.093	-0.806789937060501\\
61.25	0.0936	-0.492954432962279\\
61.25	0.0942	-0.179118928864028\\
61.25	0.0948	0.134716575234222\\
61.25	0.0954	0.448552079332444\\
61.25	0.096	0.762387583430666\\
61.25	0.0966	1.07622308752894\\
61.25	0.0972	1.39005859162717\\
61.25	0.0978	1.70389409572539\\
61.25	0.0984	2.01772959982367\\
61.25	0.099	2.33156510392189\\
61.25	0.0996	2.64540060802011\\
61.25	0.1002	2.95923611211839\\
61.25	0.1008	3.27307161621661\\
61.25	0.1014	3.58690712031486\\
61.25	0.102	3.90074262441308\\
61.25	0.1026	4.21457812851133\\
61.25	0.1032	4.52841363260958\\
61.25	0.1038	4.84224913670781\\
61.25	0.1044	5.15608464080606\\
61.25	0.105	5.46992014490431\\
61.25	0.1056	5.78375564900253\\
61.25	0.1062	6.09759115310078\\
61.25	0.1068	6.41142665719903\\
61.25	0.1074	6.72526216129725\\
61.25	0.108	7.0390976653955\\
61.25	0.1086	7.35293316949375\\
61.25	0.1092	7.66676867359197\\
61.25	0.1098	7.9806041776902\\
61.25	0.1104	8.29443968178848\\
61.25	0.111	8.60827518588673\\
61.25	0.1116	8.92211068998498\\
61.25	0.1122	9.23594619408317\\
61.25	0.1128	9.54978169818145\\
61.25	0.1134	9.8636172022797\\
61.25	0.114	10.1774527063779\\
61.25	0.1146	10.4912882104761\\
61.25	0.1152	10.8051237145744\\
61.25	0.1158	11.1189592186726\\
61.25	0.1164	11.4327947227709\\
61.25	0.117	11.7466302268691\\
61.25	0.1176	12.0604657309674\\
61.25	0.1182	12.3743012350656\\
61.25	0.1188	12.6881367391638\\
61.25	0.1194	13.0019722432621\\
61.25	0.12	13.3158077473603\\
61.25	0.1206	13.6296432514586\\
61.25	0.1212	13.9434787555568\\
61.25	0.1218	14.257314259655\\
61.25	0.1224	14.5711497637533\\
61.25	0.123	14.8849852678515\\
61.625	0.093	-0.898648186977653\\
61.625	0.0936	-0.591368952107644\\
61.625	0.0942	-0.284089717237606\\
61.625	0.0948	0.0231895176324031\\
61.625	0.0954	0.330468752502412\\
61.625	0.096	0.637747987372421\\
61.625	0.0966	0.94502722224243\\
61.625	0.0972	1.25230645711244\\
61.625	0.0978	1.55958569198245\\
61.625	0.0984	1.86686492685249\\
61.625	0.099	2.1741441617225\\
61.625	0.0996	2.4814233965925\\
61.625	0.1002	2.78870263146251\\
61.625	0.1008	3.09598186633252\\
61.625	0.1014	3.40326110120253\\
61.625	0.102	3.71054033607254\\
61.625	0.1026	4.01781957094258\\
61.625	0.1032	4.32509880581259\\
61.625	0.1038	4.63237804068257\\
61.625	0.1044	4.93965727555261\\
61.625	0.105	5.24693651042261\\
61.625	0.1056	5.55421574529262\\
61.625	0.1062	5.86149498016266\\
61.625	0.1068	6.16877421503267\\
61.625	0.1074	6.47605344990265\\
61.625	0.108	6.78333268477269\\
61.625	0.1086	7.0906119196427\\
61.625	0.1092	7.39789115451271\\
61.625	0.1098	7.70517038938272\\
61.625	0.1104	8.01244962425272\\
61.625	0.111	8.31972885912276\\
61.625	0.1116	8.62700809399277\\
61.625	0.1122	8.93428732886275\\
61.625	0.1128	9.24156656373282\\
61.625	0.1134	9.54884579860283\\
61.625	0.114	9.85612503347284\\
61.625	0.1146	10.1634042683428\\
61.625	0.1152	10.4706835032129\\
61.625	0.1158	10.7779627380829\\
61.625	0.1164	11.0852419729529\\
61.625	0.117	11.3925212078229\\
61.625	0.1176	11.6998004426929\\
61.625	0.1182	12.0070796775629\\
61.625	0.1188	12.3143589124329\\
61.625	0.1194	12.6216381473029\\
61.625	0.12	12.928917382173\\
61.625	0.1206	13.236196617043\\
61.625	0.1212	13.5434758519129\\
61.625	0.1218	13.850755086783\\
61.625	0.1224	14.158034321653\\
61.625	0.123	14.465313556523\\
62	0.093	-0.990506436894805\\
62	0.0936	-0.689783471253008\\
62	0.0942	-0.389060505611212\\
62	0.0948	-0.0883375399694444\\
62	0.0954	0.212385425672352\\
62	0.096	0.51310839131412\\
62	0.0966	0.813831356955916\\
62	0.0972	1.11455432259768\\
62	0.0978	1.41527728823948\\
62	0.0984	1.71600025388128\\
62	0.099	2.01672321952304\\
62	0.0996	2.31744618516484\\
62	0.1002	2.61816915080664\\
62	0.1008	2.9188921164484\\
62	0.1014	3.2196150820902\\
62	0.102	3.52033804773197\\
62	0.1026	3.82106101337376\\
62	0.1032	4.12178397901556\\
62	0.1038	4.42250694465733\\
62	0.1044	4.72322991029912\\
62	0.105	5.02395287594089\\
62	0.1056	5.32467584158269\\
62	0.1062	5.62539880722449\\
62	0.1068	5.92612177286625\\
62	0.1074	6.22684473850805\\
62	0.108	6.52756770414985\\
62	0.1086	6.82829066979161\\
62	0.1092	7.12901363543341\\
62	0.1098	7.42973660107518\\
62	0.1104	7.73045956671697\\
62	0.111	8.0311825323588\\
62	0.1116	8.33190549800057\\
62	0.1122	8.63262846364231\\
62	0.1128	8.93335142928413\\
62	0.1134	9.23407439492593\\
62	0.114	9.53479736056769\\
62	0.1146	9.83552032620946\\
62	0.1152	10.1362432918513\\
62	0.1158	10.4369662574931\\
62	0.1164	10.7376892231348\\
62	0.117	11.0384121887766\\
62	0.1176	11.3391351544184\\
62	0.1182	11.6398581200602\\
62	0.1188	11.940581085702\\
62	0.1194	12.2413040513438\\
62	0.12	12.5420270169855\\
62	0.1206	12.8427499826273\\
62	0.1212	13.1434729482691\\
62	0.1218	13.4441959139109\\
62	0.1224	13.7449188795527\\
62	0.123	14.0456418451945\\
62.375	0.093	-1.08236468681193\\
62.375	0.0936	-0.788197990398373\\
62.375	0.0942	-0.49403129398479\\
62.375	0.0948	-0.199864597571235\\
62.375	0.0954	0.0943020988422916\\
62.375	0.096	0.388468795255847\\
62.375	0.0966	0.68263549166943\\
62.375	0.0972	0.976802188082985\\
62.375	0.0978	1.27096888449654\\
62.375	0.0984	1.56513558091009\\
62.375	0.099	1.85930227732365\\
62.375	0.0996	2.1534689737372\\
62.375	0.1002	2.44763567015079\\
62.375	0.1008	2.74180236656431\\
62.375	0.1014	3.03596906297787\\
62.375	0.102	3.33013575939142\\
62.375	0.1026	3.62430245580501\\
62.375	0.1032	3.91846915221856\\
62.375	0.1038	4.21263584863209\\
62.375	0.1044	4.50680254504567\\
62.375	0.105	4.80096924145923\\
62.375	0.1056	5.09513593787278\\
62.375	0.1062	5.38930263428637\\
62.375	0.1068	5.68346933069989\\
62.375	0.1074	5.97763602711345\\
62.375	0.108	6.27180272352703\\
62.375	0.1086	6.56596941994059\\
62.375	0.1092	6.86013611635411\\
62.375	0.1098	7.15430281276767\\
62.375	0.1104	7.44846950918125\\
62.375	0.111	7.74263620559483\\
62.375	0.1116	8.03680290200839\\
62.375	0.1122	8.33096959842189\\
62.375	0.1128	8.6251362948355\\
62.375	0.1134	8.91930299124905\\
62.375	0.114	9.21346968766261\\
62.375	0.1146	9.50763638407614\\
62.375	0.1152	9.80180308048972\\
62.375	0.1158	10.0959697769033\\
62.375	0.1164	10.3901364733168\\
62.375	0.117	10.6843031697304\\
62.375	0.1176	10.9784698661439\\
62.375	0.1182	11.2726365625575\\
62.375	0.1188	11.566803258971\\
62.375	0.1194	11.8609699553846\\
62.375	0.12	12.1551366517982\\
62.375	0.1206	12.4493033482117\\
62.375	0.1212	12.7434700446253\\
62.375	0.1218	13.0376367410389\\
62.375	0.1224	13.3318034374524\\
62.375	0.123	13.6259701338659\\
62.75	0.093	-1.17422293672905\\
62.75	0.0936	-0.886612509543738\\
62.75	0.0942	-0.599002082358368\\
62.75	0.0948	-0.311391655173054\\
62.75	0.0954	-0.0237812279877403\\
62.75	0.096	0.263829199197573\\
62.75	0.0966	0.551439626382944\\
62.75	0.0972	0.839050053568258\\
62.75	0.0978	1.12666048075357\\
62.75	0.0984	1.41427090793894\\
62.75	0.099	1.70188133512426\\
62.75	0.0996	1.98949176230957\\
62.75	0.1002	2.27710218949491\\
62.75	0.1008	2.56471261668025\\
62.75	0.1014	2.85232304386557\\
62.75	0.102	3.13993347105088\\
62.75	0.1026	3.42754389823622\\
62.75	0.1032	3.71515432542157\\
62.75	0.1038	4.00276475260688\\
62.75	0.1044	4.29037517979222\\
62.75	0.105	4.57798560697753\\
62.75	0.1056	4.86559603416288\\
62.75	0.1062	5.15320646134822\\
62.75	0.1068	5.44081688853353\\
62.75	0.1074	5.72842731571885\\
62.75	0.108	6.01603774290422\\
62.75	0.1086	6.30364817008953\\
62.75	0.1092	6.59125859727484\\
62.75	0.1098	6.87886902446016\\
62.75	0.1104	7.16647945164553\\
62.75	0.111	7.45408987883087\\
62.75	0.1116	7.74170030601618\\
62.75	0.1122	8.02931073320147\\
62.75	0.1128	8.31692116038687\\
62.75	0.1134	8.60453158757218\\
62.75	0.114	8.8921420147575\\
62.75	0.1146	9.17975244194284\\
62.75	0.1152	9.46736286912818\\
62.75	0.1158	9.75497329631349\\
62.75	0.1164	10.0425837234988\\
62.75	0.117	10.3301941506842\\
62.75	0.1176	10.6178045778695\\
62.75	0.1182	10.9054150050548\\
62.75	0.1188	11.1930254322401\\
62.75	0.1194	11.4806358594255\\
62.75	0.12	11.7682462866108\\
62.75	0.1206	12.0558567137961\\
62.75	0.1212	12.3434671409814\\
62.75	0.1218	12.6310775681668\\
62.75	0.1224	12.9186879953521\\
62.75	0.123	13.2062984225374\\
63.125	0.093	-1.26608118664615\\
63.125	0.0936	-0.985027028689046\\
63.125	0.0942	-0.703972870731945\\
63.125	0.0948	-0.422918712774845\\
63.125	0.0954	-0.141864554817744\\
63.125	0.096	0.139189603139357\\
63.125	0.0966	0.420243761096458\\
63.125	0.0972	0.701297919053559\\
63.125	0.0978	0.98235207701066\\
63.125	0.0984	1.26340623496779\\
63.125	0.099	1.54446039292486\\
63.125	0.0996	1.82551455088196\\
63.125	0.1002	2.10656870883909\\
63.125	0.1008	2.38762286679616\\
63.125	0.1014	2.66867702475326\\
63.125	0.102	2.94973118271037\\
63.125	0.1026	3.2307853406675\\
63.125	0.1032	3.51183949862457\\
63.125	0.1038	3.79289365658167\\
63.125	0.1044	4.0739478145388\\
63.125	0.105	4.3550019724959\\
63.125	0.1056	4.63605613045297\\
63.125	0.1062	4.9171102884101\\
63.125	0.1068	5.1981644463672\\
63.125	0.1074	5.4792186043243\\
63.125	0.108	5.7602727622814\\
63.125	0.1086	6.0413269202385\\
63.125	0.1092	6.3223810781956\\
63.125	0.1098	6.60343523615271\\
63.125	0.1104	6.88448939410981\\
63.125	0.111	7.16554355206694\\
63.125	0.1116	7.44659771002404\\
63.125	0.1122	7.72765186798108\\
63.125	0.1128	8.00870602593824\\
63.125	0.1134	8.28976018389534\\
63.125	0.114	8.57081434185244\\
63.125	0.1146	8.85186849980954\\
63.125	0.1152	9.13292265776664\\
63.125	0.1158	9.41397681572374\\
63.125	0.1164	9.69503097368084\\
63.125	0.117	9.97608513163794\\
63.125	0.1176	10.257139289595\\
63.125	0.1182	10.5381934475521\\
63.125	0.1188	10.8192476055092\\
63.125	0.1194	11.1003017634663\\
63.125	0.12	11.3813559214234\\
63.125	0.1206	11.6624100793805\\
63.125	0.1212	11.9434642373377\\
63.125	0.1218	12.2245183952948\\
63.125	0.1224	12.5055725532519\\
63.125	0.123	12.786626711209\\
63.5	0.093	-1.3579394365633\\
63.5	0.0936	-1.08344154783444\\
63.5	0.0942	-0.808943659105552\\
63.5	0.0948	-0.534445770376692\\
63.5	0.0954	-0.259947881647804\\
63.5	0.096	0.0145500070810556\\
63.5	0.0966	0.289047895809944\\
63.5	0.0972	0.563545784538803\\
63.5	0.0978	0.838043673267663\\
63.5	0.0984	1.11254156199658\\
63.5	0.099	1.38703945072544\\
63.5	0.0996	1.6615373394543\\
63.5	0.1002	1.93603522818319\\
63.5	0.1008	2.21053311691207\\
63.5	0.1014	2.48503100564093\\
63.5	0.102	2.75952889436979\\
63.5	0.1026	3.03402678309868\\
63.5	0.1032	3.30852467182754\\
63.5	0.1038	3.58302256055643\\
63.5	0.1044	3.85752044928532\\
63.5	0.105	4.13201833801418\\
63.5	0.1056	4.40651622674304\\
63.5	0.1062	4.68101411547195\\
63.5	0.1068	4.95551200420081\\
63.5	0.1074	5.23000989292967\\
63.5	0.108	5.50450778165856\\
63.5	0.1086	5.77900567038742\\
63.5	0.1092	6.05350355911631\\
63.5	0.1098	6.32800144784517\\
63.5	0.1104	6.60249933657406\\
63.5	0.111	6.87699722530294\\
63.5	0.1116	7.15149511403183\\
63.5	0.1122	7.42599300276066\\
63.5	0.1128	7.70049089148958\\
63.5	0.1134	7.97498878021844\\
63.5	0.114	8.2494866689473\\
63.5	0.1146	8.52398455767619\\
63.5	0.1152	8.79848244640507\\
63.5	0.1158	9.07298033513393\\
63.5	0.1164	9.34747822386279\\
63.5	0.117	9.62197611259168\\
63.5	0.1176	9.89647400132057\\
63.5	0.1182	10.1709718900494\\
63.5	0.1188	10.4454697787783\\
63.5	0.1194	10.7199676675072\\
63.5	0.12	10.994465556236\\
63.5	0.1206	11.2689634449649\\
63.5	0.1212	11.5434613336938\\
63.5	0.1218	11.8179592224227\\
63.5	0.1224	12.0924571111515\\
63.5	0.123	12.3669549998804\\
63.875	0.093	-1.44979768648042\\
63.875	0.0936	-1.1818560669798\\
63.875	0.0942	-0.913914447479129\\
63.875	0.0948	-0.645972827978483\\
63.875	0.0954	-0.378031208477864\\
63.875	0.096	-0.110089588977218\\
63.875	0.0966	0.157852030523458\\
63.875	0.0972	0.425793650024076\\
63.875	0.0978	0.693735269524723\\
63.875	0.0984	0.961676889025398\\
63.875	0.099	1.22961850852604\\
63.875	0.0996	1.49756012802666\\
63.875	0.1002	1.76550174752734\\
63.875	0.1008	2.03344336702799\\
63.875	0.1014	2.3013849865286\\
63.875	0.102	2.56932660602925\\
63.875	0.1026	2.83726822552993\\
63.875	0.1032	3.10520984503054\\
63.875	0.1038	3.37315146453119\\
63.875	0.1044	3.64109308403187\\
63.875	0.105	3.90903470353248\\
63.875	0.1056	4.17697632303313\\
63.875	0.1062	4.44491794253381\\
63.875	0.1068	4.71285956203445\\
63.875	0.1074	4.98080118153507\\
63.875	0.108	5.24874280103575\\
63.875	0.1086	5.51668442053639\\
63.875	0.1092	5.78462604003701\\
63.875	0.1098	6.05256765953766\\
63.875	0.1104	6.32050927903833\\
63.875	0.111	6.58845089853898\\
63.875	0.1116	6.85639251803963\\
63.875	0.1122	7.12433413754025\\
63.875	0.1128	7.39227575704095\\
63.875	0.1134	7.66021737654157\\
63.875	0.114	7.92815899604221\\
63.875	0.1146	8.19610061554286\\
63.875	0.1152	8.46404223504351\\
63.875	0.1158	8.73198385454415\\
63.875	0.1164	8.9999254740448\\
63.875	0.117	9.26786709354545\\
63.875	0.1176	9.53580871304609\\
63.875	0.1182	9.80375033254674\\
63.875	0.1188	10.0716919520474\\
63.875	0.1194	10.339633571548\\
63.875	0.12	10.6075751910487\\
63.875	0.1206	10.8755168105493\\
63.875	0.1212	11.1434584300499\\
63.875	0.1218	11.4114000495506\\
63.875	0.1224	11.6793416690513\\
63.875	0.123	11.9472832885519\\
64.25	0.093	-1.54165593639755\\
64.25	0.0936	-1.28027058612514\\
64.25	0.0942	-1.01888523585271\\
64.25	0.0948	-0.757499885580302\\
64.25	0.0954	-0.496114535307896\\
64.25	0.096	-0.234729185035491\\
64.25	0.0966	0.0266561652369717\\
64.25	0.0972	0.288041515509377\\
64.25	0.0978	0.549426865781783\\
64.25	0.0984	0.810812216054217\\
64.25	0.099	1.07219756632662\\
64.25	0.0996	1.33358291659903\\
64.25	0.1002	1.59496826687149\\
64.25	0.1008	1.8563536171439\\
64.25	0.1014	2.1177389674163\\
64.25	0.102	2.37912431768871\\
64.25	0.1026	2.64050966796114\\
64.25	0.1032	2.90189501823355\\
64.25	0.1038	3.16328036850598\\
64.25	0.1044	3.42466571877841\\
64.25	0.105	3.68605106905082\\
64.25	0.1056	3.94743641932322\\
64.25	0.1062	4.20882176959566\\
64.25	0.1068	4.47020711986809\\
64.25	0.1074	4.7315924701405\\
64.25	0.108	4.99297782041293\\
64.25	0.1086	5.25436317068534\\
64.25	0.1092	5.51574852095774\\
64.25	0.1098	5.77713387123015\\
64.25	0.1104	6.03851922150258\\
64.25	0.111	6.29990457177504\\
64.25	0.1116	6.56128992204745\\
64.25	0.1122	6.82267527231983\\
64.25	0.1128	7.08406062259229\\
64.25	0.1134	7.3454459728647\\
64.25	0.114	7.60683132313713\\
64.25	0.1146	7.86821667340953\\
64.25	0.1152	8.12960202368197\\
64.25	0.1158	8.39098737395437\\
64.25	0.1164	8.65237272422678\\
64.25	0.117	8.91375807449921\\
64.25	0.1176	9.17514342477165\\
64.25	0.1182	9.43652877504405\\
64.25	0.1188	9.69791412531646\\
64.25	0.1194	9.95929947558889\\
64.25	0.12	10.2206848258613\\
64.25	0.1206	10.4820701761337\\
64.25	0.1212	10.7434555264061\\
64.25	0.1218	11.0048408766786\\
64.25	0.1224	11.266226226951\\
64.25	0.123	11.5276115772234\\
64.625	0.093	-1.6335141863147\\
64.625	0.0936	-1.37868510527051\\
64.625	0.0942	-1.12385602422628\\
64.625	0.0948	-0.869026943182121\\
64.625	0.0954	-0.614197862137928\\
64.625	0.096	-0.359368781093735\\
64.625	0.0966	-0.104539700049543\\
64.625	0.0972	0.15028938099465\\
64.625	0.0978	0.405118462038843\\
64.625	0.0984	0.659947543083035\\
64.625	0.099	0.914776624127228\\
64.625	0.0996	1.16960570517142\\
64.625	0.1002	1.42443478621561\\
64.625	0.1008	1.67926386725981\\
64.625	0.1014	1.934092948304\\
64.625	0.102	2.18892202934816\\
64.625	0.1026	2.44375111039238\\
64.625	0.1032	2.69858019143655\\
64.625	0.1038	2.95340927248074\\
64.625	0.1044	3.20823835352496\\
64.625	0.105	3.46306743456913\\
64.625	0.1056	3.71789651561332\\
64.625	0.1062	3.97272559665754\\
64.625	0.1068	4.2275546777017\\
64.625	0.1074	4.4823837587459\\
64.625	0.108	4.73721283979012\\
64.625	0.1086	4.99204192083428\\
64.625	0.1092	5.24687100187847\\
64.625	0.1098	5.50170008292267\\
64.625	0.1104	5.75652916396686\\
64.625	0.111	6.01135824501108\\
64.625	0.1116	6.26618732605527\\
64.625	0.1122	6.52101640709941\\
64.625	0.1128	6.77584548814366\\
64.625	0.1134	7.03067456918782\\
64.625	0.114	7.28550365023202\\
64.625	0.1146	7.54033273127621\\
64.625	0.1152	7.7951618123204\\
64.625	0.1158	8.04999089336459\\
64.625	0.1164	8.30481997440879\\
64.625	0.117	8.55964905545298\\
64.625	0.1176	8.81447813649717\\
64.625	0.1182	9.06930721754136\\
64.625	0.1188	9.32413629858553\\
64.625	0.1194	9.57896537962975\\
64.625	0.12	9.83379446067394\\
64.625	0.1206	10.0886235417181\\
64.625	0.1212	10.3434526227623\\
64.625	0.1218	10.5982817038065\\
64.625	0.1224	10.8531107848507\\
64.625	0.123	11.1079398658949\\
65	0.093	-1.72537243623185\\
65	0.0936	-1.4770996244159\\
65	0.0942	-1.22882681259989\\
65	0.0948	-0.98055400078394\\
65	0.0954	-0.732281188967988\\
65	0.096	-0.484008377152037\\
65	0.0966	-0.235735565336057\\
65	0.0972	0.0125372464798943\\
65	0.0978	0.260810058295846\\
65	0.0984	0.509082870111826\\
65	0.099	0.757355681927805\\
65	0.0996	1.00562849374376\\
65	0.1002	1.25390130555974\\
65	0.1008	1.50217411737569\\
65	0.1014	1.75044692919164\\
65	0.102	1.99871974100759\\
65	0.1026	2.24699255282357\\
65	0.1032	2.49526536463952\\
65	0.1038	2.7435381764555\\
65	0.1044	2.99181098827148\\
65	0.105	3.24008380008743\\
65	0.1056	3.48835661190338\\
65	0.1062	3.73662942371936\\
65	0.1068	3.98490223553532\\
65	0.1074	4.23317504735127\\
65	0.108	4.48144785916725\\
65	0.1086	4.72972067098323\\
65	0.1092	4.97799348279918\\
65	0.1098	5.22626629461513\\
65	0.1104	5.47453910643111\\
65	0.111	5.72281191824709\\
65	0.1116	5.97108473006304\\
65	0.1122	6.21935754187896\\
65	0.1128	6.46763035369497\\
65	0.1134	6.71590316551095\\
65	0.114	6.9641759773269\\
65	0.1146	7.21244878914285\\
65	0.1152	7.46072160095883\\
65	0.1158	7.70899441277479\\
65	0.1164	7.95726722459074\\
65	0.117	8.20554003640672\\
65	0.1176	8.45381284822267\\
65	0.1182	8.70208566003865\\
65	0.1188	8.9503584718546\\
65	0.1194	9.19863128367058\\
65	0.12	9.44690409548653\\
65	0.1206	9.69517690730248\\
65	0.1212	9.94344971911843\\
65	0.1218	10.1917225309344\\
65	0.1224	10.4399953427504\\
65	0.123	10.6882681545663\\
65.375	0.093	-1.81723068614897\\
65.375	0.0936	-1.57551414356124\\
65.375	0.0942	-1.3337976009735\\
65.375	0.0948	-1.09208105838576\\
65.375	0.0954	-0.85036451579802\\
65.375	0.096	-0.60864797321031\\
65.375	0.0966	-0.366931430622543\\
65.375	0.0972	-0.125214888034833\\
65.375	0.0978	0.116501654552906\\
65.375	0.0984	0.358218197140673\\
65.375	0.099	0.599934739728383\\
65.375	0.0996	0.841651282316121\\
65.375	0.1002	1.08336782490386\\
65.375	0.1008	1.3250843674916\\
65.375	0.1014	1.56680091007934\\
65.375	0.102	1.80851745266705\\
65.375	0.1026	2.05023399525481\\
65.375	0.1032	2.29195053784252\\
65.375	0.1038	2.53366708043026\\
65.375	0.1044	2.77538362301803\\
65.375	0.105	3.01710016560574\\
65.375	0.1056	3.25881670819348\\
65.375	0.1062	3.50053325078122\\
65.375	0.1068	3.74224979336896\\
65.375	0.1074	3.98396633595669\\
65.375	0.108	4.22568287854443\\
65.375	0.1086	4.46739942113217\\
65.375	0.1092	4.70911596371988\\
65.375	0.1098	4.95083250630762\\
65.375	0.1104	5.19254904889539\\
65.375	0.111	5.43426559148313\\
65.375	0.1116	5.67598213407086\\
65.375	0.1122	5.91769867665855\\
65.375	0.1128	6.15941521924634\\
65.375	0.1134	6.40113176183408\\
65.375	0.114	6.64284830442179\\
65.375	0.1146	6.88456484700953\\
65.375	0.1152	7.12628138959727\\
65.375	0.1158	7.36799793218501\\
65.375	0.1164	7.60971447477274\\
65.375	0.117	7.85143101736048\\
65.375	0.1176	8.09314755994822\\
65.375	0.1182	8.33486410253596\\
65.375	0.1188	8.57658064512367\\
65.375	0.1194	8.81829718771144\\
65.375	0.12	9.06001373029915\\
65.375	0.1206	9.30173027288689\\
65.375	0.1212	9.54344681547462\\
65.375	0.1218	9.78516335806236\\
65.375	0.1224	10.0268799006501\\
65.375	0.123	10.2685964432378\\
65.75	0.093	-1.9090889360661\\
65.75	0.0936	-1.6739286627066\\
65.75	0.0942	-1.43876838934708\\
65.75	0.0948	-1.20360811598758\\
65.75	0.0954	-0.96844784262808\\
65.75	0.096	-0.733287569268555\\
65.75	0.0966	-0.498127295909029\\
65.75	0.0972	-0.262967022549532\\
65.75	0.0978	-0.0278067491900345\\
65.75	0.0984	0.207353524169491\\
65.75	0.099	0.442513797528989\\
65.75	0.0996	0.677674070888486\\
65.75	0.1002	0.912834344248012\\
65.75	0.1008	1.14799461760751\\
65.75	0.1014	1.38315489096701\\
65.75	0.102	1.6183151643265\\
65.75	0.1026	1.85347543768603\\
65.75	0.1032	2.08863571104553\\
65.75	0.1038	2.32379598440505\\
65.75	0.1044	2.55895625776458\\
65.75	0.105	2.79411653112408\\
65.75	0.1056	3.02927680448357\\
65.75	0.1062	3.2644370778431\\
65.75	0.1068	3.4995973512026\\
65.75	0.1074	3.73475762456209\\
65.75	0.108	3.96991789792162\\
65.75	0.1086	4.20507817128112\\
65.75	0.1092	4.44023844464061\\
65.75	0.1098	4.67539871800011\\
65.75	0.1104	4.91055899135964\\
65.75	0.111	5.14571926471919\\
65.75	0.1116	5.38087953807869\\
65.75	0.1122	5.61603981143816\\
65.75	0.1128	5.85120008479771\\
65.75	0.1134	6.08636035815721\\
65.75	0.114	6.32152063151671\\
65.75	0.1146	6.5566809048762\\
65.75	0.1152	6.79184117823573\\
65.75	0.1158	7.02700145159523\\
65.75	0.1164	7.26216172495472\\
65.75	0.117	7.49732199831425\\
65.75	0.1176	7.73248227167375\\
65.75	0.1182	7.96764254503324\\
65.75	0.1188	8.20280281839277\\
65.75	0.1194	8.43796309175229\\
65.75	0.12	8.67312336511179\\
65.75	0.1206	8.90828363847129\\
65.75	0.1212	9.14344391183079\\
65.75	0.1218	9.37860418519031\\
65.75	0.1224	9.61376445854981\\
65.75	0.123	9.84892473190931\\
66.125	0.093	-2.00094718598325\\
66.125	0.0936	-1.77234318185197\\
66.125	0.0942	-1.54373917772068\\
66.125	0.0948	-1.3151351735894\\
66.125	0.0954	-1.08653116945814\\
66.125	0.096	-0.857927165326856\\
66.125	0.0966	-0.629323161195572\\
66.125	0.0972	-0.400719157064287\\
66.125	0.0978	-0.172115152933031\\
66.125	0.0984	0.0564888511982815\\
66.125	0.099	0.285092855329566\\
66.125	0.0996	0.513696859460822\\
66.125	0.1002	0.742300863592135\\
66.125	0.1008	0.970904867723391\\
66.125	0.1014	1.19950887185468\\
66.125	0.102	1.42811287598593\\
66.125	0.1026	1.65671688011724\\
66.125	0.1032	1.8853208842485\\
66.125	0.1038	2.11392488837978\\
66.125	0.1044	2.3425288925111\\
66.125	0.105	2.57113289664235\\
66.125	0.1056	2.79973690077364\\
66.125	0.1062	3.02834090490492\\
66.125	0.1068	3.25694490903621\\
66.125	0.1074	3.48554891316746\\
66.125	0.108	3.71415291729878\\
66.125	0.1086	3.94275692143006\\
66.125	0.1092	4.17136092556132\\
66.125	0.1098	4.3999649296926\\
66.125	0.1104	4.62856893382389\\
66.125	0.111	4.8571729379552\\
66.125	0.1116	5.08577694208645\\
66.125	0.1122	5.31438094621771\\
66.125	0.1128	5.54298495034902\\
66.125	0.1134	5.77158895448031\\
66.125	0.114	6.00019295861159\\
66.125	0.1146	6.22879696274285\\
66.125	0.1152	6.45740096687416\\
66.125	0.1158	6.68600497100542\\
66.125	0.1164	6.9146089751367\\
66.125	0.117	7.14321297926799\\
66.125	0.1176	7.37181698339927\\
66.125	0.1182	7.60042098753053\\
66.125	0.1188	7.82902499166181\\
66.125	0.1194	8.0576289957931\\
66.125	0.12	8.28623299992438\\
66.125	0.1206	8.51483700405566\\
66.125	0.1212	8.74344100818692\\
66.125	0.1218	8.97204501231823\\
66.125	0.1224	9.20064901644949\\
66.125	0.123	9.42925302058077\\
66.5	0.093	-2.09280543590037\\
66.5	0.0936	-1.87075770099733\\
66.5	0.0942	-1.64870996609426\\
66.5	0.0948	-1.42666223119122\\
66.5	0.0954	-1.20461449628817\\
66.5	0.096	-0.982566761385129\\
66.5	0.0966	-0.760519026482058\\
66.5	0.0972	-0.538471291579015\\
66.5	0.0978	-0.316423556675971\\
66.5	0.0984	-0.0943758217728998\\
66.5	0.099	0.127671913130143\\
66.5	0.0996	0.349719648033187\\
66.5	0.1002	0.571767382936258\\
66.5	0.1008	0.793815117839301\\
66.5	0.1014	1.01586285274234\\
66.5	0.102	1.23791058764539\\
66.5	0.1026	1.45995832254846\\
66.5	0.1032	1.68200605745153\\
66.5	0.1038	1.90405379235457\\
66.5	0.1044	2.12610152725765\\
66.5	0.105	2.34814926216069\\
66.5	0.1056	2.57019699706373\\
66.5	0.1062	2.7922447319668\\
66.5	0.1068	3.01429246686985\\
66.5	0.1074	3.23634020177289\\
66.5	0.108	3.45838793667596\\
66.5	0.1086	3.680435671579\\
66.5	0.1092	3.90248340648205\\
66.5	0.1098	4.12453114138509\\
66.5	0.1104	4.34657887628816\\
66.5	0.111	4.56862661119123\\
66.5	0.1116	4.79067434609428\\
66.5	0.1122	5.01272208099729\\
66.5	0.1128	5.23476981590039\\
66.5	0.1134	5.45681755080344\\
66.5	0.114	5.67886528570648\\
66.5	0.1146	5.90091302060952\\
66.5	0.1152	6.12296075551259\\
66.5	0.1158	6.34500849041564\\
66.5	0.1164	6.56705622531868\\
66.5	0.117	6.78910396022175\\
66.5	0.1176	7.01115169512479\\
66.5	0.1182	7.23319943002784\\
66.5	0.1188	7.45524716493088\\
66.5	0.1194	7.67729489983395\\
66.5	0.12	7.899342634737\\
66.5	0.1206	8.12139036964004\\
66.5	0.1212	8.34343810454308\\
66.5	0.1218	8.56548583944615\\
66.5	0.1224	8.78753357434923\\
66.5	0.123	9.00958130925227\\
66.875	0.093	-2.1846636858175\\
66.875	0.0936	-1.9691722201427\\
66.875	0.0942	-1.75368075446784\\
66.875	0.0948	-1.53818928879303\\
66.875	0.0954	-1.3226978231182\\
66.875	0.096	-1.1072063574434\\
66.875	0.0966	-0.891714891768544\\
66.875	0.0972	-0.676223426093742\\
66.875	0.0978	-0.460731960418912\\
66.875	0.0984	-0.245240494744081\\
66.875	0.099	-0.0297490290692508\\
66.875	0.0996	0.185742436605551\\
66.875	0.1002	0.40123390228041\\
66.875	0.1008	0.616725367955212\\
66.875	0.1014	0.832216833630042\\
66.875	0.102	1.04770829930484\\
66.875	0.1026	1.2631997649797\\
66.875	0.1032	1.47869123065453\\
66.875	0.1038	1.69418269632934\\
66.875	0.1044	1.90967416200419\\
66.875	0.105	2.125165627679\\
66.875	0.1056	2.34065709335383\\
66.875	0.1062	2.55614855902866\\
66.875	0.1068	2.77164002470349\\
66.875	0.1074	2.98713149037829\\
66.875	0.108	3.20262295605315\\
66.875	0.1086	3.41811442172795\\
66.875	0.1092	3.63360588740278\\
66.875	0.1098	3.84909735307758\\
66.875	0.1104	4.06458881875244\\
66.875	0.111	4.28008028442727\\
66.875	0.1116	4.4955717501021\\
66.875	0.1122	4.71106321577687\\
66.875	0.1128	4.92655468145176\\
66.875	0.1134	5.14204614712656\\
66.875	0.114	5.35753761280139\\
66.875	0.1146	5.5730290784762\\
66.875	0.1152	5.78852054415105\\
66.875	0.1158	6.00401200982586\\
66.875	0.1164	6.21950347550069\\
66.875	0.117	6.43499494117552\\
66.875	0.1176	6.65048640685035\\
66.875	0.1182	6.86597787252515\\
66.875	0.1188	7.08146933819998\\
66.875	0.1194	7.29696080387481\\
66.875	0.12	7.51245226954964\\
66.875	0.1206	7.72794373522444\\
66.875	0.1212	7.94343520089927\\
66.875	0.1218	8.1589266665741\\
66.875	0.1224	8.37441813224893\\
66.875	0.123	8.58990959792374\\
67.25	0.093	-2.27652193573462\\
67.25	0.0936	-2.06758673928803\\
67.25	0.0942	-1.85865154284141\\
67.25	0.0948	-1.64971634639483\\
67.25	0.0954	-1.44078114994824\\
67.25	0.096	-1.23184595350165\\
67.25	0.0966	-1.02291075705503\\
67.25	0.0972	-0.813975560608441\\
67.25	0.0978	-0.605040364161852\\
67.25	0.0984	-0.396105167715234\\
67.25	0.099	-0.187169971268645\\
67.25	0.0996	0.0217652251779441\\
67.25	0.1002	0.230700421624562\\
67.25	0.1008	0.439635618071151\\
67.25	0.1014	0.64857081451774\\
67.25	0.102	0.857506010964329\\
67.25	0.1026	1.06644120741095\\
67.25	0.1032	1.27537640385754\\
67.25	0.1038	1.48431160030412\\
67.25	0.1044	1.69324679675071\\
67.25	0.105	1.9021819931973\\
67.25	0.1056	2.11111718964389\\
67.25	0.1062	2.32005238609051\\
67.25	0.1068	2.5289875825371\\
67.25	0.1074	2.73792277898369\\
67.25	0.108	2.94685797543031\\
67.25	0.1086	3.15579317187689\\
67.25	0.1092	3.36472836832348\\
67.25	0.1098	3.57366356477007\\
67.25	0.1104	3.78259876121669\\
67.25	0.111	3.99153395766331\\
67.25	0.1116	4.2004691541099\\
67.25	0.1122	4.40940435055646\\
67.25	0.1128	4.6183395470031\\
67.25	0.1134	4.82727474344969\\
67.25	0.114	5.03620993989628\\
67.25	0.1146	5.24514513634287\\
67.25	0.1152	5.45408033278949\\
67.25	0.1158	5.66301552923608\\
67.25	0.1164	5.87195072568267\\
67.25	0.117	6.08088592212928\\
67.25	0.1176	6.28982111857587\\
67.25	0.1182	6.49875631502246\\
67.25	0.1188	6.70769151146905\\
67.25	0.1194	6.91662670791567\\
67.25	0.12	7.12556190436226\\
67.25	0.1206	7.33449710080885\\
67.25	0.1212	7.54343229725544\\
67.25	0.1218	7.75236749370205\\
67.25	0.1224	7.96130269014864\\
67.25	0.123	8.17023788659523\\
67.625	0.093	-2.36838018565177\\
67.625	0.0936	-2.16600125843343\\
67.625	0.0942	-1.96362233121502\\
67.625	0.0948	-1.76124340399667\\
67.625	0.0954	-1.5588644767783\\
67.625	0.096	-1.35648554955995\\
67.625	0.0966	-1.15410662234157\\
67.625	0.0972	-0.951727695123196\\
67.625	0.0978	-0.749348767904849\\
67.625	0.0984	-0.546969840686444\\
67.625	0.099	-0.344590913468096\\
67.625	0.0996	-0.14221198624972\\
67.625	0.1002	0.0601669409686565\\
67.625	0.1008	0.262545868187033\\
67.625	0.1014	0.464924795405381\\
67.625	0.102	0.667303722623757\\
67.625	0.1026	0.869682649842133\\
67.625	0.1032	1.07206157706051\\
67.625	0.1038	1.27444050427886\\
67.625	0.1044	1.47681943149723\\
67.625	0.105	1.67919835871561\\
67.625	0.1056	1.88157728593396\\
67.625	0.1062	2.08395621315236\\
67.625	0.1068	2.28633514037071\\
67.625	0.1074	2.48871406758909\\
67.625	0.108	2.69109299480746\\
67.625	0.1086	2.89347192202584\\
67.625	0.1092	3.09585084924419\\
67.625	0.1098	3.29822977646256\\
67.625	0.1104	3.50060870368094\\
67.625	0.111	3.70298763089934\\
67.625	0.1116	3.90536655811769\\
67.625	0.1122	4.10774548533601\\
67.625	0.1128	4.31012441255444\\
67.625	0.1134	4.51250333977279\\
67.625	0.114	4.71488226699117\\
67.625	0.1146	4.91726119420952\\
67.625	0.1152	5.11964012142792\\
67.625	0.1158	5.32201904864627\\
67.625	0.1164	5.52439797586464\\
67.625	0.117	5.72677690308302\\
67.625	0.1176	5.9291558303014\\
67.625	0.1182	6.13153475751974\\
67.625	0.1188	6.33391368473812\\
67.625	0.1194	6.5362926119565\\
67.625	0.12	6.73867153917485\\
67.625	0.1206	6.94105046639322\\
67.625	0.1212	7.14342939361157\\
67.625	0.1218	7.34580832082997\\
67.625	0.1224	7.54818724804832\\
67.625	0.123	7.7505661752667\\
68	0.093	-2.4602384355689\\
68	0.0936	-2.26441577757876\\
68	0.0942	-2.06859311958863\\
68	0.0948	-1.87277046159849\\
68	0.0954	-1.67694780360836\\
68	0.096	-1.48112514561822\\
68	0.0966	-1.28530248762806\\
68	0.0972	-1.08947982963792\\
68	0.0978	-0.893657171647789\\
68	0.0984	-0.697834513657625\\
68	0.099	-0.50201185566749\\
68	0.0996	-0.306189197677355\\
68	0.1002	-0.110366539687192\\
68	0.1008	0.0854561183029432\\
68	0.1014	0.281278776293078\\
68	0.102	0.477101434283213\\
68	0.1026	0.672924092273377\\
68	0.1032	0.868746750263512\\
68	0.1038	1.06456940825362\\
68	0.1044	1.26039206624378\\
68	0.105	1.45621472423392\\
68	0.1056	1.65203738222405\\
68	0.1062	1.84786004021421\\
68	0.1068	2.04368269820435\\
68	0.1074	2.23950535619448\\
68	0.108	2.43532801418465\\
68	0.1086	2.63115067217478\\
68	0.1092	2.82697333016492\\
68	0.1098	3.02279598815505\\
68	0.1104	3.21861864614522\\
68	0.111	3.41444130413538\\
68	0.1116	3.61026396212552\\
68	0.1122	3.80608662011562\\
68	0.1128	4.00190927810581\\
68	0.1134	4.19773193609595\\
68	0.114	4.39355459408605\\
68	0.1146	4.58937725207619\\
68	0.1152	4.78519991006635\\
68	0.1158	4.98102256805649\\
68	0.1164	5.17684522604662\\
68	0.117	5.37266788403679\\
68	0.1176	5.56849054202692\\
68	0.1182	5.76431320001706\\
68	0.1188	5.96013585800719\\
68	0.1194	6.15595851599736\\
68	0.12	6.35178117398749\\
68	0.1206	6.54760383197763\\
68	0.1212	6.74342648996776\\
68	0.1218	6.93924914795792\\
68	0.1224	7.13507180594806\\
68	0.123	7.33089446393819\\
68.375	0.093	-2.55209668548605\\
68.375	0.0936	-2.36283029672413\\
68.375	0.0942	-2.1735639079622\\
68.375	0.0948	-1.98429751920028\\
68.375	0.0954	-1.79503113043839\\
68.375	0.096	-1.60576474167647\\
68.375	0.0966	-1.41649835291454\\
68.375	0.0972	-1.22723196415265\\
68.375	0.0978	-1.03796557539073\\
68.375	0.0984	-0.848699186628807\\
68.375	0.099	-0.659432797866884\\
68.375	0.0996	-0.470166409104991\\
68.375	0.1002	-0.280900020343068\\
68.375	0.1008	-0.0916336315811463\\
68.375	0.1014	0.0976327571807474\\
68.375	0.102	0.28689914594267\\
68.375	0.1026	0.476165534704592\\
68.375	0.1032	0.665431923466514\\
68.375	0.1038	0.854698312228408\\
68.375	0.1044	1.04396470099033\\
68.375	0.105	1.23323108975225\\
68.375	0.1056	1.42249747851415\\
68.375	0.1062	1.6117638672761\\
68.375	0.1068	1.80103025603799\\
68.375	0.1074	1.99029664479988\\
68.375	0.108	2.17956303356183\\
68.375	0.1086	2.36882942232373\\
68.375	0.1092	2.55809581108565\\
68.375	0.1098	2.74736219984754\\
68.375	0.1104	2.93662858860949\\
68.375	0.111	3.12589497737142\\
68.375	0.1116	3.31516136613331\\
68.375	0.1122	3.5044277548952\\
68.375	0.1128	3.69369414365715\\
68.375	0.1134	3.88296053241908\\
68.375	0.114	4.07222692118097\\
68.375	0.1146	4.26149330994286\\
68.375	0.1152	4.45075969870481\\
68.375	0.1158	4.64002608746671\\
68.375	0.1164	4.82929247622863\\
68.375	0.117	5.01855886499055\\
68.375	0.1176	5.20782525375247\\
68.375	0.1182	5.39709164251437\\
68.375	0.1188	5.58635803127626\\
68.375	0.1194	5.77562442003821\\
68.375	0.12	5.96489080880011\\
68.375	0.1206	6.15415719756203\\
68.375	0.1212	6.34342358632392\\
68.375	0.1218	6.53268997508584\\
68.375	0.1224	6.72195636384777\\
68.375	0.123	6.91122275260966\\
68.75	0.093	-2.64395493540317\\
68.75	0.0936	-2.46124481586949\\
68.75	0.0942	-2.27853469633578\\
68.75	0.0948	-2.0958245768021\\
68.75	0.0954	-1.91311445726842\\
68.75	0.096	-1.73040433773474\\
68.75	0.0966	-1.54769421820103\\
68.75	0.0972	-1.36498409866735\\
68.75	0.0978	-1.18227397913367\\
68.75	0.0984	-0.999563859599988\\
68.75	0.099	-0.816853740066307\\
68.75	0.0996	-0.634143620532626\\
68.75	0.1002	-0.451433500998917\\
68.75	0.1008	-0.268723381465236\\
68.75	0.1014	-0.086013261931555\\
68.75	0.102	0.0966968576021259\\
68.75	0.1026	0.279406977135835\\
68.75	0.1032	0.462117096669516\\
68.75	0.1038	0.644827216203169\\
68.75	0.1044	0.827537335736878\\
68.75	0.105	1.01024745527056\\
68.75	0.1056	1.19295757480424\\
68.75	0.1062	1.37566769433795\\
68.75	0.1068	1.55837781387163\\
68.75	0.1074	1.74108793340531\\
68.75	0.108	1.92379805293902\\
68.75	0.1086	2.1065081724727\\
68.75	0.1092	2.28921829200635\\
68.75	0.1098	2.47192841154003\\
68.75	0.1104	2.65463853107374\\
68.75	0.111	2.83734865060745\\
68.75	0.1116	3.02005877014113\\
68.75	0.1122	3.20276888967479\\
68.75	0.1128	3.38547900920852\\
68.75	0.1134	3.5681891287422\\
68.75	0.114	3.75089924827589\\
68.75	0.1146	3.93360936780957\\
68.75	0.1152	4.11631948734325\\
68.75	0.1158	4.29902960687693\\
68.75	0.1164	4.48173972641061\\
68.75	0.117	4.66444984594432\\
68.75	0.1176	4.847159965478\\
68.75	0.1182	5.02987008501168\\
68.75	0.1188	5.21258020454536\\
68.75	0.1194	5.39529032407907\\
68.75	0.12	5.57800044361275\\
68.75	0.1206	5.7607105631464\\
68.75	0.1212	5.94342068268008\\
68.75	0.1218	6.12613080221379\\
68.75	0.1224	6.30884092174747\\
68.75	0.123	6.49155104128116\\
69.125	0.093	-2.73581318532032\\
69.125	0.0936	-2.55965933501486\\
69.125	0.0942	-2.38350548470939\\
69.125	0.0948	-2.20735163440395\\
69.125	0.0954	-2.03119778409848\\
69.125	0.096	-1.85504393379304\\
69.125	0.0966	-1.67889008348754\\
69.125	0.0972	-1.50273623318211\\
69.125	0.0978	-1.32658238287667\\
69.125	0.0984	-1.15042853257117\\
69.125	0.099	-0.97427468226573\\
69.125	0.0996	-0.79812083196029\\
69.125	0.1002	-0.621966981654793\\
69.125	0.1008	-0.445813131349354\\
69.125	0.1014	-0.269659281043914\\
69.125	0.102	-0.0935054307384462\\
69.125	0.1026	0.0826484195670218\\
69.125	0.1032	0.25880226987249\\
69.125	0.1038	0.43495612017793\\
69.125	0.1044	0.611109970483398\\
69.125	0.105	0.787263820788866\\
69.125	0.1056	0.963417671094305\\
69.125	0.1062	1.13957152139977\\
69.125	0.1068	1.31572537170524\\
69.125	0.1074	1.49187922201068\\
69.125	0.108	1.66803307231618\\
69.125	0.1086	1.84418692262162\\
69.125	0.1092	2.02034077292706\\
69.125	0.1098	2.19649462323252\\
69.125	0.1104	2.37264847353799\\
69.125	0.111	2.54880232384346\\
69.125	0.1116	2.72495617414893\\
69.125	0.1122	2.90111002445434\\
69.125	0.1128	3.07726387475984\\
69.125	0.1134	3.2534177250653\\
69.125	0.114	3.42957157537074\\
69.125	0.1146	3.60572542567621\\
69.125	0.1152	3.78187927598168\\
69.125	0.1158	3.95803312628712\\
69.125	0.1164	4.13418697659259\\
69.125	0.117	4.31034082689806\\
69.125	0.1176	4.4864946772035\\
69.125	0.1182	4.66264852750896\\
69.125	0.1188	4.8388023778144\\
69.125	0.1194	5.0149562281199\\
69.125	0.12	5.19111007842534\\
69.125	0.1206	5.36726392873078\\
69.125	0.1212	5.54341777903625\\
69.125	0.1218	5.71957162934171\\
69.125	0.1224	5.89572547964715\\
69.125	0.123	6.07187932995262\\
69.5	0.093	-2.82767143523742\\
69.5	0.0936	-2.65807385416019\\
69.5	0.0942	-2.48847627308294\\
69.5	0.0948	-2.31887869200574\\
69.5	0.0954	-2.14928111092851\\
69.5	0.096	-1.97968352985129\\
69.5	0.0966	-1.81008594877403\\
69.5	0.0972	-1.6404883676968\\
69.5	0.0978	-1.47089078661958\\
69.5	0.0984	-1.30129320554232\\
69.5	0.099	-1.13169562446512\\
69.5	0.0996	-0.962098043387897\\
69.5	0.1002	-0.792500462310642\\
69.5	0.1008	-0.622902881233415\\
69.5	0.1014	-0.453305300156188\\
69.5	0.102	-0.283707719078961\\
69.5	0.1026	-0.114110138001706\\
69.5	0.1032	0.0554874430754921\\
69.5	0.1038	0.225085024152719\\
69.5	0.1044	0.394682605229974\\
69.5	0.105	0.564280186307201\\
69.5	0.1056	0.733877767384428\\
69.5	0.1062	0.903475348461683\\
69.5	0.1068	1.07307292953891\\
69.5	0.1074	1.24267051061611\\
69.5	0.108	1.41226809169336\\
69.5	0.1086	1.58186567277059\\
69.5	0.1092	1.75146325384782\\
69.5	0.1098	1.92106083492504\\
69.5	0.1104	2.0906584160023\\
69.5	0.111	2.26025599707953\\
69.5	0.1116	2.42985357815675\\
69.5	0.1122	2.59945115923395\\
69.5	0.1128	2.76904874031123\\
69.5	0.1134	2.93864632138846\\
69.5	0.114	3.10824390246569\\
69.5	0.1146	3.27784148354291\\
69.5	0.1152	3.44743906462014\\
69.5	0.1158	3.61703664569737\\
69.5	0.1164	3.78663422677459\\
69.5	0.117	3.95623180785185\\
69.5	0.1176	4.12582938892908\\
69.5	0.1182	4.2954269700063\\
69.5	0.1188	4.46502455108353\\
69.5	0.1194	4.63462213216076\\
69.5	0.12	4.80421971323798\\
69.5	0.1206	4.97381729431521\\
69.5	0.1212	5.14341487539244\\
69.5	0.1218	5.31301245646969\\
69.5	0.1224	5.48261003754692\\
69.5	0.123	5.65220761862412\\
69.875	0.093	-2.91952968515454\\
69.875	0.0936	-2.75648837330556\\
69.875	0.0942	-2.59344706145654\\
69.875	0.0948	-2.43040574960753\\
69.875	0.0954	-2.26736443775854\\
69.875	0.096	-2.10432312590956\\
69.875	0.0966	-1.94128181406052\\
69.875	0.0972	-1.77824050221153\\
69.875	0.0978	-1.61519919036255\\
69.875	0.0984	-1.4521578785135\\
69.875	0.099	-1.28911656666452\\
69.875	0.0996	-1.12607525481553\\
69.875	0.1002	-0.96303394296649\\
69.875	0.1008	-0.799992631117505\\
69.875	0.1014	-0.636951319268519\\
69.875	0.102	-0.473910007419505\\
69.875	0.1026	-0.310868695570491\\
69.875	0.1032	-0.147827383721506\\
69.875	0.1038	0.0152139281275083\\
69.875	0.1044	0.178255239976522\\
69.875	0.105	0.341296551825508\\
69.875	0.1056	0.504337863674522\\
69.875	0.1062	0.667379175523536\\
69.875	0.1068	0.830420487372521\\
69.875	0.1074	0.993461799221535\\
69.875	0.108	1.15650311107055\\
69.875	0.1086	1.31954442291953\\
69.875	0.1092	1.48258573476855\\
69.875	0.1098	1.64562704661753\\
69.875	0.1104	1.80866835846655\\
69.875	0.111	1.97170967031559\\
69.875	0.1116	2.13475098216458\\
69.875	0.1122	2.29779229401353\\
69.875	0.1128	2.4608336058626\\
69.875	0.1134	2.62387491771159\\
69.875	0.114	2.78691622956057\\
69.875	0.1146	2.94995754140959\\
69.875	0.1152	3.1129988532586\\
69.875	0.1158	3.27604016510759\\
69.875	0.1164	3.4390814769566\\
69.875	0.117	3.60212278880562\\
69.875	0.1176	3.7651641006546\\
69.875	0.1182	3.92820541250362\\
69.875	0.1188	4.0912467243526\\
69.875	0.1194	4.25428803620161\\
69.875	0.12	4.41732934805063\\
69.875	0.1206	4.58037065989961\\
69.875	0.1212	4.7434119717486\\
69.875	0.1218	4.90645328359764\\
69.875	0.1224	5.06949459544663\\
69.875	0.123	5.23253590729561\\
70.25	0.093	-3.01138793507167\\
70.25	0.0936	-2.85490289245089\\
70.25	0.0942	-2.69841784983012\\
70.25	0.0948	-2.54193280720935\\
70.25	0.0954	-2.38544776458858\\
70.25	0.096	-2.2289627219678\\
70.25	0.0966	-2.072477679347\\
70.25	0.0972	-1.91599263672626\\
70.25	0.0978	-1.75950759410549\\
70.25	0.0984	-1.60302255148468\\
70.25	0.099	-1.44653750886391\\
70.25	0.0996	-1.29005246624314\\
70.25	0.1002	-1.13356742362237\\
70.25	0.1008	-0.977082381001594\\
70.25	0.1014	-0.820597338380821\\
70.25	0.102	-0.664112295760049\\
70.25	0.1026	-0.507627253139248\\
70.25	0.1032	-0.351142210518503\\
70.25	0.1038	-0.194657167897731\\
70.25	0.1044	-0.0381721252769296\\
70.25	0.105	0.118312917343843\\
70.25	0.1056	0.274797959964616\\
70.25	0.1062	0.431283002585388\\
70.25	0.1068	0.587768045206161\\
70.25	0.1074	0.744253087826934\\
70.25	0.108	0.900738130447735\\
70.25	0.1086	1.05722317306851\\
70.25	0.1092	1.21370821568925\\
70.25	0.1098	1.37019325831002\\
70.25	0.1104	1.52667830093083\\
70.25	0.111	1.68316334355163\\
70.25	0.1116	1.8396483861724\\
70.25	0.1122	1.99613342879312\\
70.25	0.1128	2.15261847141394\\
70.25	0.1134	2.30910351403472\\
70.25	0.114	2.46558855665549\\
70.25	0.1146	2.62207359927626\\
70.25	0.1152	2.77855864189706\\
70.25	0.1158	2.93504368451781\\
70.25	0.1164	3.09152872713858\\
70.25	0.117	3.24801376975938\\
70.25	0.1176	3.40449881238015\\
70.25	0.1182	3.56098385500093\\
70.25	0.1188	3.71746889762167\\
70.25	0.1194	3.87395394024247\\
70.25	0.12	4.03043898286325\\
70.25	0.1206	4.18692402548402\\
70.25	0.1212	4.34340906810479\\
70.25	0.1218	4.49989411072556\\
70.25	0.1224	4.65637915334634\\
70.25	0.123	4.81286419596708\\
70.625	0.093	-3.10324618498882\\
70.625	0.0936	-2.95331741159629\\
70.625	0.0942	-2.80338863820373\\
70.625	0.0948	-2.6534598648112\\
70.625	0.0954	-2.50353109141864\\
70.625	0.096	-2.35360231802611\\
70.625	0.0966	-2.20367354463355\\
70.625	0.0972	-2.05374477124099\\
70.625	0.0978	-1.90381599784845\\
70.625	0.0984	-1.75388722445589\\
70.625	0.099	-1.60395845106336\\
70.625	0.0996	-1.4540296776708\\
70.625	0.1002	-1.30410090427824\\
70.625	0.1008	-1.15417213088571\\
70.625	0.1014	-1.00424335749315\\
70.625	0.102	-0.854314584100621\\
70.625	0.1026	-0.704385810708061\\
70.625	0.1032	-0.55445703731553\\
70.625	0.1038	-0.40452826392297\\
70.625	0.1044	-0.25459949053041\\
70.625	0.105	-0.104670717137878\\
70.625	0.1056	0.0452580562546814\\
70.625	0.1062	0.195186829647241\\
70.625	0.1068	0.345115603039773\\
70.625	0.1074	0.495044376432304\\
70.625	0.108	0.644973149824892\\
70.625	0.1086	0.794901923217424\\
70.625	0.1092	0.944830696609955\\
70.625	0.1098	1.09475947000251\\
70.625	0.1104	1.24468824339507\\
70.625	0.111	1.39461701678763\\
70.625	0.1116	1.54454579018017\\
70.625	0.1122	1.6944745635727\\
70.625	0.1128	1.84440333696529\\
70.625	0.1134	1.99433211035782\\
70.625	0.114	2.14426088375038\\
70.625	0.1146	2.29418965714291\\
70.625	0.1152	2.44411843053547\\
70.625	0.1158	2.594047203928\\
70.625	0.1164	2.74397597732056\\
70.625	0.117	2.89390475071312\\
70.625	0.1176	3.04383352410565\\
70.625	0.1182	3.19376229749818\\
70.625	0.1188	3.34369107089074\\
70.625	0.1194	3.4936198442833\\
70.625	0.12	3.64354861767583\\
70.625	0.1206	3.79347739106839\\
70.625	0.1212	3.94340616446092\\
70.625	0.1218	4.09333493785348\\
70.625	0.1224	4.24326371124602\\
70.625	0.123	4.39319248463858\\
71	0.093	-3.19510443490597\\
71	0.0936	-3.05173193074165\\
71	0.0942	-2.90835942657733\\
71	0.0948	-2.76498692241302\\
71	0.0954	-2.6216144182487\\
71	0.096	-2.47824191408438\\
71	0.0966	-2.33486940992006\\
71	0.0972	-2.19149690575574\\
71	0.0978	-2.04812440159142\\
71	0.0984	-1.90475189742708\\
71	0.099	-1.76137939326279\\
71	0.0996	-1.61800688909847\\
71	0.1002	-1.47463438493412\\
71	0.1008	-1.3312618807698\\
71	0.1014	-1.18788937660551\\
71	0.102	-1.04451687244119\\
71	0.1026	-0.901144368276846\\
71	0.1032	-0.757771864112527\\
71	0.1038	-0.614399359948237\\
71	0.1044	-0.47102685578389\\
71	0.105	-0.327654351619572\\
71	0.1056	-0.184281847455253\\
71	0.1062	-0.0409093432909344\\
71	0.1068	0.102463160873384\\
71	0.1074	0.245835665037703\\
71	0.108	0.38920816920205\\
71	0.1086	0.53258067336634\\
71	0.1092	0.675953177530658\\
71	0.1098	0.819325681694977\\
71	0.1104	0.962698185859324\\
71	0.111	1.10607069002364\\
71	0.1116	1.24944319418796\\
71	0.1122	1.39281569835225\\
71	0.1128	1.53618820251663\\
71	0.1134	1.67956070668092\\
71	0.114	1.82293321084524\\
71	0.1146	1.96630571500955\\
71	0.1152	2.1096782191739\\
71	0.1158	2.25305072333819\\
71	0.1164	2.39642322750251\\
71	0.117	2.53979573166686\\
71	0.1176	2.68316823583118\\
71	0.1182	2.82654073999547\\
71	0.1188	2.96991324415978\\
71	0.1194	3.11328574832413\\
71	0.12	3.25665825248845\\
71	0.1206	3.40003075665274\\
71	0.1212	3.54340326081706\\
71	0.1218	3.68677576498141\\
71	0.1224	3.83014826914572\\
71	0.123	3.97352077331001\\
71.375	0.093	-3.2869626848231\\
71.375	0.0936	-3.15014644988699\\
71.375	0.0942	-3.01333021495088\\
71.375	0.0948	-2.87651398001481\\
71.375	0.0954	-2.73969774507873\\
71.375	0.096	-2.60288151014262\\
71.375	0.0966	-2.46606527520652\\
71.375	0.0972	-2.32924904027044\\
71.375	0.0978	-2.19243280533436\\
71.375	0.0984	-2.05561657039823\\
71.375	0.099	-1.91880033546215\\
71.375	0.0996	-1.78198410052607\\
71.375	0.1002	-1.64516786558997\\
71.375	0.1008	-1.50835163065386\\
71.375	0.1014	-1.37153539571779\\
71.375	0.102	-1.23471916078171\\
71.375	0.1026	-1.0979029258456\\
71.375	0.1032	-0.961086690909525\\
71.375	0.1038	-0.824270455973419\\
71.375	0.1044	-0.687454221037314\\
71.375	0.105	-0.550637986101236\\
71.375	0.1056	-0.413821751165159\\
71.375	0.1062	-0.277005516229025\\
71.375	0.1068	-0.140189281292947\\
71.375	0.1074	-0.00337304635687019\\
71.375	0.108	0.133443188579236\\
71.375	0.1086	0.270259423515341\\
71.375	0.1092	0.407075658451419\\
71.375	0.1098	0.543891893387496\\
71.375	0.1104	0.680708128323602\\
71.375	0.111	0.817524363259736\\
71.375	0.1116	0.954340598195813\\
71.375	0.1122	1.09115683313186\\
71.375	0.1128	1.227973068068\\
71.375	0.1134	1.36478930300407\\
71.375	0.114	1.50160553794018\\
71.375	0.1146	1.63842177287626\\
71.375	0.1152	1.77523800781236\\
71.375	0.1158	1.91205424274844\\
71.375	0.1164	2.04887047768455\\
71.375	0.117	2.18568671262065\\
71.375	0.1176	2.32250294755673\\
71.375	0.1182	2.45931918249281\\
71.375	0.1188	2.59613541742891\\
71.375	0.1194	2.73295165236502\\
71.375	0.12	2.86976788730109\\
71.375	0.1206	3.00658412223717\\
71.375	0.1212	3.14340035717328\\
71.375	0.1218	3.28021659210938\\
71.375	0.1224	3.41703282704546\\
71.375	0.123	3.55384906198154\\
71.75	0.093	-3.37882093474025\\
71.75	0.0936	-3.24856096903241\\
71.75	0.0942	-3.11830100332452\\
71.75	0.0948	-2.98804103761665\\
71.75	0.0954	-2.85778107190882\\
71.75	0.096	-2.72752110620095\\
71.75	0.0966	-2.59726114049306\\
71.75	0.0972	-2.4670011747852\\
71.75	0.0978	-2.33674120907736\\
71.75	0.0984	-2.20648124336947\\
71.75	0.099	-2.0762212776616\\
71.75	0.0996	-1.94596131195377\\
71.75	0.1002	-1.81570134624587\\
71.75	0.1008	-1.68544138053801\\
71.75	0.1014	-1.55518141483014\\
71.75	0.102	-1.42492144912231\\
71.75	0.1026	-1.29466148341442\\
71.75	0.1032	-1.16440151770655\\
71.75	0.1038	-1.03414155199872\\
71.75	0.1044	-0.903881586290822\\
71.75	0.105	-0.773621620582958\\
71.75	0.1056	-0.643361654875122\\
71.75	0.1062	-0.513101689167229\\
71.75	0.1068	-0.382841723459364\\
71.75	0.1074	-0.2525817577515\\
71.75	0.108	-0.122321792043635\\
71.75	0.1086	0.007938173664229\\
71.75	0.1092	0.138198139372093\\
71.75	0.1098	0.26845810507993\\
71.75	0.1104	0.398718070787822\\
71.75	0.111	0.528978036495715\\
71.75	0.1116	0.65923800220358\\
71.75	0.1122	0.789497967911387\\
71.75	0.1128	0.919757933619309\\
71.75	0.1134	1.05001789932717\\
71.75	0.114	1.18027786503501\\
71.75	0.1146	1.31053783074287\\
71.75	0.1152	1.44079779645077\\
71.75	0.1158	1.5710577621586\\
71.75	0.1164	1.70131772786647\\
71.75	0.117	1.83157769357436\\
71.75	0.1176	1.96183765928222\\
71.75	0.1182	2.09209762499006\\
71.75	0.1188	2.22235759069792\\
71.75	0.1194	2.35261755640582\\
71.75	0.12	2.48287752211365\\
71.75	0.1206	2.61313748782152\\
71.75	0.1212	2.74339745352941\\
71.75	0.1218	2.8736574192373\\
71.75	0.1224	3.00391738494514\\
71.75	0.123	3.13417735065298\\
72.125	0.093	-3.47067918465737\\
72.125	0.0936	-3.34697548817775\\
72.125	0.0942	-3.22327179169807\\
72.125	0.0948	-3.09956809521844\\
72.125	0.0954	-2.97586439873882\\
72.125	0.096	-2.8521607022592\\
72.125	0.0966	-2.72845700577955\\
72.125	0.0972	-2.60475330929989\\
72.125	0.0978	-2.48104961282027\\
72.125	0.0984	-2.35734591634062\\
72.125	0.099	-2.233642219861\\
72.125	0.0996	-2.10993852338137\\
72.125	0.1002	-1.98623482690169\\
72.125	0.1008	-1.86253113042207\\
72.125	0.1014	-1.73882743394245\\
72.125	0.102	-1.61512373746282\\
72.125	0.1026	-1.49142004098317\\
72.125	0.1032	-1.36771634450355\\
72.125	0.1038	-1.2440126480239\\
72.125	0.1044	-1.12030895154425\\
72.125	0.105	-0.996605255064622\\
72.125	0.1056	-0.872901558584999\\
72.125	0.1062	-0.749197862105348\\
72.125	0.1068	-0.625494165625696\\
72.125	0.1074	-0.501790469146073\\
72.125	0.108	-0.378086772666421\\
72.125	0.1086	-0.254383076186798\\
72.125	0.1092	-0.130679379707175\\
72.125	0.1098	-0.00697568322752318\\
72.125	0.1104	0.116728013252128\\
72.125	0.111	0.24043170973178\\
72.125	0.1116	0.364135406211403\\
72.125	0.1122	0.487839102690998\\
72.125	0.1128	0.611542799170678\\
72.125	0.1134	0.73524649565033\\
72.125	0.114	0.858950192129953\\
72.125	0.1146	0.982653888609576\\
72.125	0.1152	1.10635758508923\\
72.125	0.1158	1.23006128156885\\
72.125	0.1164	1.3537649780485\\
72.125	0.117	1.47746867452815\\
72.125	0.1176	1.60117237100778\\
72.125	0.1182	1.7248760674874\\
72.125	0.1188	1.84857976396702\\
72.125	0.1194	1.9722834604467\\
72.125	0.12	2.09598715692633\\
72.125	0.1206	2.21969085340595\\
72.125	0.1212	2.3433945498856\\
72.125	0.1218	2.46709824636525\\
72.125	0.1224	2.59080194284491\\
72.125	0.123	2.7145056393245\\
72.5	0.093	-3.56253743457447\\
72.5	0.0936	-3.44539000732306\\
72.5	0.0942	-3.32824258007165\\
72.5	0.0948	-3.21109515282023\\
72.5	0.0954	-3.09394772556882\\
72.5	0.096	-2.97680029831744\\
72.5	0.0966	-2.859652871066\\
72.5	0.0972	-2.74250544381459\\
72.5	0.0978	-2.62535801656321\\
72.5	0.0984	-2.50821058931177\\
72.5	0.099	-2.39106316206036\\
72.5	0.0996	-2.27391573480898\\
72.5	0.1002	-2.15676830755754\\
72.5	0.1008	-2.03962088030613\\
72.5	0.1014	-1.92247345305475\\
72.5	0.102	-1.80532602580334\\
72.5	0.1026	-1.6881785985519\\
72.5	0.1032	-1.57103117130052\\
72.5	0.1038	-1.45388374404911\\
72.5	0.1044	-1.33673631679767\\
72.5	0.105	-1.21958888954629\\
72.5	0.1056	-1.10244146229488\\
72.5	0.1062	-0.985294035043438\\
72.5	0.1068	-0.868146607792056\\
72.5	0.1074	-0.750999180540646\\
72.5	0.108	-0.633851753289207\\
72.5	0.1086	-0.516704326037825\\
72.5	0.1092	-0.399556898786415\\
72.5	0.1098	-0.282409471535004\\
72.5	0.1104	-0.165262044283594\\
72.5	0.111	-0.0481146170321551\\
72.5	0.1116	0.0690328102192552\\
72.5	0.1122	0.186180237470609\\
72.5	0.1128	0.303327664722076\\
72.5	0.1134	0.420475091973486\\
72.5	0.114	0.537622519224868\\
72.5	0.1146	0.654769946476279\\
72.5	0.1152	0.771917373727717\\
72.5	0.1158	0.889064800979099\\
72.5	0.1164	1.00621222823051\\
72.5	0.117	1.12335965548195\\
72.5	0.1176	1.24050708273333\\
72.5	0.1182	1.35765450998474\\
72.5	0.1188	1.47480193723615\\
72.5	0.1194	1.59194936448756\\
72.5	0.12	1.70909679173897\\
72.5	0.1206	1.82624421899041\\
72.5	0.1212	1.94339164624182\\
72.5	0.1218	2.06053907349326\\
72.5	0.1224	2.17768650074464\\
72.5	0.123	2.29483392799602\\
72.875	0.093	-3.65439568449162\\
72.875	0.0936	-3.54380452646845\\
72.875	0.0942	-3.43321336844525\\
72.875	0.0948	-3.32262221042205\\
72.875	0.0954	-3.21203105239888\\
72.875	0.096	-3.10143989437572\\
72.875	0.0966	-2.99084873635252\\
72.875	0.0972	-2.88025757832935\\
72.875	0.0978	-2.76966642030618\\
72.875	0.0984	-2.65907526228298\\
72.875	0.099	-2.54848410425981\\
72.875	0.0996	-2.43789294623662\\
72.875	0.1002	-2.32730178821342\\
72.875	0.1008	-2.21671063019025\\
72.875	0.1014	-2.10611947216708\\
72.875	0.102	-1.99552831414391\\
72.875	0.1026	-1.88493715612071\\
72.875	0.1032	-1.77434599809754\\
72.875	0.1038	-1.66375484007435\\
72.875	0.1044	-1.55316368205115\\
72.875	0.105	-1.44257252402798\\
72.875	0.1056	-1.33198136600481\\
72.875	0.1062	-1.22139020798161\\
72.875	0.1068	-1.11079904995844\\
72.875	0.1074	-1.00020789193525\\
72.875	0.108	-0.88961673391205\\
72.875	0.1086	-0.77902557588888\\
72.875	0.1092	-0.668434417865711\\
72.875	0.1098	-0.557843259842542\\
72.875	0.1104	-0.447252101819345\\
72.875	0.111	-0.336660943796147\\
72.875	0.1116	-0.226069785772978\\
72.875	0.1122	-0.115478627749809\\
72.875	0.1128	-0.00488746972658305\\
72.875	0.1134	0.105703688296586\\
72.875	0.114	0.216294846319755\\
72.875	0.1146	0.326886004342924\\
72.875	0.1152	0.437477162366122\\
72.875	0.1158	0.548068320389291\\
72.875	0.1164	0.658659478412488\\
72.875	0.117	0.769250636435686\\
72.875	0.1176	0.879841794458855\\
72.875	0.1182	0.990432952482024\\
72.875	0.1188	1.10102411050519\\
72.875	0.1194	1.21161526852839\\
72.875	0.12	1.32220642655159\\
72.875	0.1206	1.43279758457476\\
72.875	0.1212	1.54338874259793\\
72.875	0.1218	1.65397990062115\\
72.875	0.1224	1.76457105864432\\
72.875	0.123	1.87516221666749\\
73.25	0.093	-3.74625393440874\\
73.25	0.0936	-3.64221904561381\\
73.25	0.0942	-3.53818415681883\\
73.25	0.0948	-3.43414926802387\\
73.25	0.0954	-3.33011437922895\\
73.25	0.096	-3.22607949043399\\
73.25	0.0966	-3.12204460163903\\
73.25	0.0972	-3.01800971284408\\
73.25	0.0978	-2.91397482404912\\
73.25	0.0984	-2.80993993525416\\
73.25	0.099	-2.70590504645921\\
73.25	0.0996	-2.60187015766425\\
73.25	0.1002	-2.4978352688693\\
73.25	0.1008	-2.39380038007434\\
73.25	0.1014	-2.28976549127941\\
73.25	0.102	-2.18573060248445\\
73.25	0.1026	-2.08169571368947\\
73.25	0.1032	-1.97766082489454\\
73.25	0.1038	-1.87362593609959\\
73.25	0.1044	-1.7695910473046\\
73.25	0.105	-1.66555615850967\\
73.25	0.1056	-1.56152126971472\\
73.25	0.1062	-1.45748638091976\\
73.25	0.1068	-1.3534514921248\\
73.25	0.1074	-1.24941660332985\\
73.25	0.108	-1.14538171453489\\
73.25	0.1086	-1.04134682573994\\
73.25	0.1092	-0.93731193694498\\
73.25	0.1098	-0.833277048150052\\
73.25	0.1104	-0.729242159355067\\
73.25	0.111	-0.625207270560111\\
73.25	0.1116	-0.521172381765155\\
73.25	0.1122	-0.417137492970227\\
73.25	0.1128	-0.313102604175242\\
73.25	0.1134	-0.209067715380286\\
73.25	0.114	-0.10503282658533\\
73.25	0.1146	-0.000997937790401693\\
73.25	0.1152	0.103036951004583\\
73.25	0.1158	0.207071839799511\\
73.25	0.1164	0.311106728594467\\
73.25	0.117	0.415141617389452\\
73.25	0.1176	0.51917650618438\\
73.25	0.1182	0.623211394979336\\
73.25	0.1188	0.727246283774292\\
73.25	0.1194	0.831281172569248\\
73.25	0.12	0.935316061364176\\
73.25	0.1206	1.03935095015913\\
73.25	0.1212	1.14338583895409\\
73.25	0.1218	1.24742072774904\\
73.25	0.1224	1.351455616544\\
73.25	0.123	1.45549050533896\\
73.625	0.093	-3.83811218432589\\
73.625	0.0936	-3.74063356475918\\
73.625	0.0942	-3.64315494519244\\
73.625	0.0948	-3.54567632562572\\
73.625	0.0954	-3.44819770605901\\
73.625	0.096	-3.35071908649229\\
73.625	0.0966	-3.25324046692555\\
73.625	0.0972	-3.1557618473588\\
73.625	0.0978	-3.05828322779209\\
73.625	0.0984	-2.96080460822535\\
73.625	0.099	-2.86332598865863\\
73.625	0.0996	-2.76584736909192\\
73.625	0.1002	-2.66836874952517\\
73.625	0.1008	-2.57089012995846\\
73.625	0.1014	-2.47341151039174\\
73.625	0.102	-2.37593289082503\\
73.625	0.1026	-2.27845427125828\\
73.625	0.1032	-2.18097565169154\\
73.625	0.1038	-2.08349703212482\\
73.625	0.1044	-1.98601841255808\\
73.625	0.105	-1.88853979299137\\
73.625	0.1056	-1.79106117342465\\
73.625	0.1062	-1.69358255385791\\
73.625	0.1068	-1.59610393429119\\
73.625	0.1074	-1.49862531472448\\
73.625	0.108	-1.40114669515773\\
73.625	0.1086	-1.30366807559102\\
73.625	0.1092	-1.20618945602428\\
73.625	0.1098	-1.10871083645756\\
73.625	0.1104	-1.01123221689082\\
73.625	0.111	-0.913753597324074\\
73.625	0.1116	-0.816274977757359\\
73.625	0.1122	-0.718796358190673\\
73.625	0.1128	-0.621317738623901\\
73.625	0.1134	-0.523839119057186\\
73.625	0.114	-0.426360499490471\\
73.625	0.1146	-0.328881879923756\\
73.625	0.1152	-0.231403260357013\\
73.625	0.1158	-0.133924640790269\\
73.625	0.1164	-0.0364460212235542\\
73.625	0.117	0.0610325983431892\\
73.625	0.1176	0.158511217909904\\
73.625	0.1182	0.255989837476619\\
73.625	0.1188	0.353468457043334\\
73.625	0.1194	0.450947076610078\\
73.625	0.12	0.548425696176821\\
73.625	0.1206	0.645904315743508\\
73.625	0.1212	0.743382935310251\\
73.625	0.1218	0.840861554876994\\
73.625	0.1224	0.938340174443681\\
73.625	0.123	1.03581879401042\\
74	0.093	-3.92997043424299\\
74	0.0936	-3.83904808390452\\
74	0.0942	-3.74812573356598\\
74	0.0948	-3.65720338322751\\
74	0.0954	-3.56628103288901\\
74	0.096	-3.47535868255054\\
74	0.0966	-3.384436332212\\
74	0.0972	-3.2935139818735\\
74	0.0978	-3.20259163153503\\
74	0.0984	-3.1116692811965\\
74	0.099	-3.02074693085802\\
74	0.0996	-2.92982458051952\\
74	0.1002	-2.83890223018102\\
74	0.1008	-2.74797987984252\\
74	0.1014	-2.65705752950402\\
74	0.102	-2.56613517916554\\
74	0.1026	-2.47521282882701\\
74	0.1032	-2.38429047848854\\
74	0.1038	-2.29336812815004\\
74	0.1044	-2.2024457778115\\
74	0.105	-2.11152342747303\\
74	0.1056	-2.02060107713453\\
74	0.1062	-1.92967872679603\\
74	0.1068	-1.83875637645752\\
74	0.1074	-1.74783402611905\\
74	0.108	-1.65691167578052\\
74	0.1086	-1.56598932544202\\
74	0.1092	-1.47506697510354\\
74	0.1098	-1.38414462476504\\
74	0.1104	-1.29322227442654\\
74	0.111	-1.20229992408801\\
74	0.1116	-1.11137757374954\\
74	0.1122	-1.02045522341106\\
74	0.1128	-0.929532873072503\\
74	0.1134	-0.838610522734029\\
74	0.114	-0.747688172395527\\
74	0.1146	-0.656765822057054\\
74	0.1152	-0.565843471718523\\
74	0.1158	-0.474921121380049\\
74	0.1164	-0.383998771041547\\
74	0.117	-0.293076420703017\\
74	0.1176	-0.202154070364543\\
74	0.1182	-0.111231720026041\\
74	0.1188	-0.0203093696875385\\
74	0.1194	0.0706129806509352\\
74	0.12	0.161535330989466\\
74	0.1206	0.25245768132794\\
74	0.1212	0.343380031666413\\
74	0.1218	0.434302382004944\\
74	0.1224	0.525224732343418\\
74	0.123	0.616147082681948\\
};
\end{axis}

\begin{axis}[%
width=5.011742cm,
height=3.595207cm,
at={(0cm,16.404793cm)},
scale only axis,
xmin=55,
xmax=75,
tick align=outside,
xlabel={$L_{cut}$},
xmajorgrids,
ymin=0.09,
ymax=0.13,
ylabel={$D_{rlx}$},
ymajorgrids,
zmin=-101.122897289483,
zmax=7.88686190848819,
zlabel={$x_4,x_4$},
zmajorgrids,
view={-140}{50},
legend style={at={(1.03,1)},anchor=north west,legend cell align=left,align=left,draw=white!15!black}
]
\addplot3[only marks,mark=*,mark options={},mark size=1.5000pt,color=mycolor1] plot table[row sep=crcr,]{%
74	0.123	-26.0353957891804\\
72	0.113	-20.9879322169279\\
61	0.095	-10.6920630070547\\
56	0.093	-10.9569379219045\\
};
\addplot3[only marks,mark=*,mark options={},mark size=1.5000pt,color=black] plot table[row sep=crcr,]{%
69	0.104	-15.5142132591546\\
};

\addplot3[%
surf,
opacity=0.7,
shader=interp,
colormap={mymap}{[1pt] rgb(0pt)=(0.0901961,0.239216,0.0745098); rgb(1pt)=(0.0945149,0.242058,0.0739522); rgb(2pt)=(0.0988592,0.244894,0.0733566); rgb(3pt)=(0.103229,0.247724,0.0727241); rgb(4pt)=(0.107623,0.250549,0.0720557); rgb(5pt)=(0.112043,0.253367,0.0713525); rgb(6pt)=(0.116487,0.25618,0.0706154); rgb(7pt)=(0.120956,0.258986,0.0698456); rgb(8pt)=(0.125449,0.261787,0.0690441); rgb(9pt)=(0.129967,0.264581,0.0682118); rgb(10pt)=(0.134508,0.26737,0.06735); rgb(11pt)=(0.139074,0.270152,0.0664596); rgb(12pt)=(0.143663,0.272929,0.0655416); rgb(13pt)=(0.148275,0.275699,0.0645971); rgb(14pt)=(0.152911,0.278463,0.0636271); rgb(15pt)=(0.15757,0.281221,0.0626328); rgb(16pt)=(0.162252,0.283973,0.0616151); rgb(17pt)=(0.166957,0.286719,0.060575); rgb(18pt)=(0.171685,0.289458,0.0595136); rgb(19pt)=(0.176434,0.292191,0.0584321); rgb(20pt)=(0.181207,0.294918,0.0573313); rgb(21pt)=(0.186001,0.297639,0.0562123); rgb(22pt)=(0.190817,0.300353,0.0550763); rgb(23pt)=(0.195655,0.303061,0.0539242); rgb(24pt)=(0.200514,0.305763,0.052757); rgb(25pt)=(0.205395,0.308459,0.0515759); rgb(26pt)=(0.210296,0.311149,0.0503624); rgb(27pt)=(0.215212,0.313846,0.0490067); rgb(28pt)=(0.220142,0.316548,0.0475043); rgb(29pt)=(0.22509,0.319254,0.0458704); rgb(30pt)=(0.230056,0.321962,0.0441205); rgb(31pt)=(0.235042,0.324671,0.04227); rgb(32pt)=(0.240048,0.327379,0.0403343); rgb(33pt)=(0.245078,0.330085,0.0383287); rgb(34pt)=(0.250131,0.332786,0.0362688); rgb(35pt)=(0.25521,0.335482,0.0341698); rgb(36pt)=(0.260317,0.33817,0.0320472); rgb(37pt)=(0.265451,0.340849,0.0299163); rgb(38pt)=(0.270616,0.343517,0.0277927); rgb(39pt)=(0.275813,0.346172,0.0256916); rgb(40pt)=(0.281043,0.348814,0.0236284); rgb(41pt)=(0.286307,0.35144,0.0216186); rgb(42pt)=(0.291607,0.354048,0.0196776); rgb(43pt)=(0.296945,0.356637,0.0178207); rgb(44pt)=(0.302322,0.359206,0.0160634); rgb(45pt)=(0.307739,0.361753,0.0144211); rgb(46pt)=(0.313198,0.364275,0.0129091); rgb(47pt)=(0.318701,0.366772,0.0115428); rgb(48pt)=(0.324249,0.369242,0.0103377); rgb(49pt)=(0.329843,0.371682,0.00930909); rgb(50pt)=(0.335485,0.374093,0.00847245); rgb(51pt)=(0.341176,0.376471,0.00784314); rgb(52pt)=(0.346925,0.378826,0.00732741); rgb(53pt)=(0.352735,0.381168,0.00682184); rgb(54pt)=(0.358605,0.383497,0.00632729); rgb(55pt)=(0.364532,0.385812,0.00584464); rgb(56pt)=(0.370516,0.388113,0.00537476); rgb(57pt)=(0.376552,0.390399,0.00491852); rgb(58pt)=(0.38264,0.39267,0.00447681); rgb(59pt)=(0.388777,0.394925,0.00405048); rgb(60pt)=(0.394962,0.397164,0.00364042); rgb(61pt)=(0.401191,0.399386,0.00324749); rgb(62pt)=(0.407464,0.401592,0.00287258); rgb(63pt)=(0.413777,0.40378,0.00251655); rgb(64pt)=(0.420129,0.40595,0.00218028); rgb(65pt)=(0.426518,0.408102,0.00186463); rgb(66pt)=(0.432942,0.410234,0.00157049); rgb(67pt)=(0.439399,0.412348,0.00129873); rgb(68pt)=(0.445885,0.414441,0.00105022); rgb(69pt)=(0.452401,0.416515,0.000825833); rgb(70pt)=(0.458942,0.418567,0.000626441); rgb(71pt)=(0.465508,0.420599,0.00045292); rgb(72pt)=(0.472096,0.422609,0.000306141); rgb(73pt)=(0.478704,0.424596,0.000186979); rgb(74pt)=(0.485331,0.426562,9.63073e-05); rgb(75pt)=(0.491973,0.428504,3.49981e-05); rgb(76pt)=(0.498628,0.430422,3.92506e-06); rgb(77pt)=(0.505323,0.432315,0); rgb(78pt)=(0.512206,0.434168,0); rgb(79pt)=(0.519282,0.435983,0); rgb(80pt)=(0.526529,0.437764,0); rgb(81pt)=(0.533922,0.439512,0); rgb(82pt)=(0.54144,0.441232,0); rgb(83pt)=(0.549059,0.442927,0); rgb(84pt)=(0.556756,0.444599,0); rgb(85pt)=(0.564508,0.446252,0); rgb(86pt)=(0.572292,0.447889,0); rgb(87pt)=(0.580084,0.449514,0); rgb(88pt)=(0.587863,0.451129,0); rgb(89pt)=(0.595604,0.452737,0); rgb(90pt)=(0.603284,0.454343,0); rgb(91pt)=(0.610882,0.455948,0); rgb(92pt)=(0.618373,0.457556,0); rgb(93pt)=(0.625734,0.459171,0); rgb(94pt)=(0.632943,0.460795,0); rgb(95pt)=(0.639976,0.462432,0); rgb(96pt)=(0.64681,0.464084,0); rgb(97pt)=(0.653423,0.465756,0); rgb(98pt)=(0.659791,0.46745,0); rgb(99pt)=(0.665891,0.469169,0); rgb(100pt)=(0.6717,0.470916,0); rgb(101pt)=(0.677195,0.472696,0); rgb(102pt)=(0.682353,0.47451,0); rgb(103pt)=(0.687242,0.476355,0); rgb(104pt)=(0.691952,0.478225,0); rgb(105pt)=(0.696497,0.480118,0); rgb(106pt)=(0.700887,0.482033,0); rgb(107pt)=(0.705134,0.483968,0); rgb(108pt)=(0.709251,0.485921,0); rgb(109pt)=(0.713249,0.487891,0); rgb(110pt)=(0.71714,0.489876,0); rgb(111pt)=(0.720936,0.491875,0); rgb(112pt)=(0.724649,0.493887,0); rgb(113pt)=(0.72829,0.495909,0); rgb(114pt)=(0.731872,0.49794,0); rgb(115pt)=(0.735406,0.499979,0); rgb(116pt)=(0.738904,0.502025,0); rgb(117pt)=(0.742378,0.504075,0); rgb(118pt)=(0.74584,0.506128,0); rgb(119pt)=(0.749302,0.508182,0); rgb(120pt)=(0.752775,0.510237,0); rgb(121pt)=(0.756272,0.51229,0); rgb(122pt)=(0.759804,0.514339,0); rgb(123pt)=(0.763384,0.516385,0); rgb(124pt)=(0.767022,0.518424,0); rgb(125pt)=(0.770731,0.520455,0); rgb(126pt)=(0.774523,0.522478,0); rgb(127pt)=(0.77841,0.524489,0); rgb(128pt)=(0.782391,0.526491,0); rgb(129pt)=(0.786402,0.528496,0); rgb(130pt)=(0.790431,0.530506,0); rgb(131pt)=(0.794478,0.532521,0); rgb(132pt)=(0.798541,0.534539,0); rgb(133pt)=(0.802619,0.53656,0); rgb(134pt)=(0.806712,0.538584,0); rgb(135pt)=(0.81082,0.540609,0); rgb(136pt)=(0.81494,0.542635,0); rgb(137pt)=(0.819074,0.54466,0); rgb(138pt)=(0.823219,0.546686,0); rgb(139pt)=(0.827374,0.548709,0); rgb(140pt)=(0.831541,0.55073,0); rgb(141pt)=(0.835716,0.552749,0); rgb(142pt)=(0.8399,0.554763,0); rgb(143pt)=(0.844092,0.556774,0); rgb(144pt)=(0.848292,0.558779,0); rgb(145pt)=(0.852497,0.560778,0); rgb(146pt)=(0.856708,0.562771,0); rgb(147pt)=(0.860924,0.564756,0); rgb(148pt)=(0.865143,0.566733,0); rgb(149pt)=(0.869366,0.568701,0); rgb(150pt)=(0.873592,0.57066,0); rgb(151pt)=(0.877819,0.572608,0); rgb(152pt)=(0.882047,0.574545,0); rgb(153pt)=(0.886275,0.576471,0); rgb(154pt)=(0.890659,0.578362,0); rgb(155pt)=(0.895333,0.580203,0); rgb(156pt)=(0.900258,0.581999,0); rgb(157pt)=(0.905397,0.583755,0); rgb(158pt)=(0.910711,0.585479,0); rgb(159pt)=(0.916164,0.587176,0); rgb(160pt)=(0.921717,0.588852,0); rgb(161pt)=(0.927333,0.590513,0); rgb(162pt)=(0.932974,0.592166,0); rgb(163pt)=(0.938602,0.593815,0); rgb(164pt)=(0.94418,0.595468,0); rgb(165pt)=(0.949669,0.59713,0); rgb(166pt)=(0.955033,0.598808,0); rgb(167pt)=(0.960233,0.600507,0); rgb(168pt)=(0.965232,0.602233,0); rgb(169pt)=(0.969992,0.603992,0); rgb(170pt)=(0.974475,0.605791,0); rgb(171pt)=(0.978643,0.607636,0); rgb(172pt)=(0.98246,0.609532,0); rgb(173pt)=(0.985886,0.611486,0); rgb(174pt)=(0.988885,0.613503,0); rgb(175pt)=(0.991419,0.61559,0); rgb(176pt)=(0.99345,0.617753,0); rgb(177pt)=(0.99494,0.619997,0); rgb(178pt)=(0.995851,0.622329,0); rgb(179pt)=(0.996226,0.624763,0); rgb(180pt)=(0.996512,0.627352,0); rgb(181pt)=(0.996788,0.630095,0); rgb(182pt)=(0.997053,0.632982,0); rgb(183pt)=(0.997308,0.636004,0); rgb(184pt)=(0.997552,0.639152,0); rgb(185pt)=(0.997785,0.642416,0); rgb(186pt)=(0.998006,0.645786,0); rgb(187pt)=(0.998217,0.649253,0); rgb(188pt)=(0.998416,0.652807,0); rgb(189pt)=(0.998605,0.656439,0); rgb(190pt)=(0.998781,0.660138,0); rgb(191pt)=(0.998946,0.663897,0); rgb(192pt)=(0.9991,0.667704,0); rgb(193pt)=(0.999242,0.67155,0); rgb(194pt)=(0.999372,0.675427,0); rgb(195pt)=(0.99949,0.679323,0); rgb(196pt)=(0.999596,0.68323,0); rgb(197pt)=(0.99969,0.687139,0); rgb(198pt)=(0.999771,0.691039,0); rgb(199pt)=(0.999841,0.694921,0); rgb(200pt)=(0.999898,0.698775,0); rgb(201pt)=(0.999942,0.702592,0); rgb(202pt)=(0.999974,0.706363,0); rgb(203pt)=(0.999994,0.710077,0); rgb(204pt)=(1,0.713725,0); rgb(205pt)=(1,0.717341,0); rgb(206pt)=(1,0.720963,0); rgb(207pt)=(1,0.724591,0); rgb(208pt)=(1,0.728226,0); rgb(209pt)=(1,0.731867,0); rgb(210pt)=(1,0.735514,0); rgb(211pt)=(1,0.739167,0); rgb(212pt)=(1,0.742827,0); rgb(213pt)=(1,0.746493,0); rgb(214pt)=(1,0.750165,0); rgb(215pt)=(1,0.753843,0); rgb(216pt)=(1,0.757527,0); rgb(217pt)=(1,0.761217,0); rgb(218pt)=(1,0.764913,0); rgb(219pt)=(1,0.768615,0); rgb(220pt)=(1,0.772324,0); rgb(221pt)=(1,0.776038,0); rgb(222pt)=(1,0.779758,0); rgb(223pt)=(1,0.783484,0); rgb(224pt)=(1,0.787215,0); rgb(225pt)=(1,0.790953,0); rgb(226pt)=(1,0.794696,0); rgb(227pt)=(1,0.798445,0); rgb(228pt)=(1,0.8022,0); rgb(229pt)=(1,0.805961,0); rgb(230pt)=(1,0.809727,0); rgb(231pt)=(1,0.8135,0); rgb(232pt)=(1,0.817278,0); rgb(233pt)=(1,0.821063,0); rgb(234pt)=(1,0.824854,0); rgb(235pt)=(1,0.828652,0); rgb(236pt)=(1,0.832455,0); rgb(237pt)=(1,0.836265,0); rgb(238pt)=(1,0.840081,0); rgb(239pt)=(1,0.843903,0); rgb(240pt)=(1,0.847732,0); rgb(241pt)=(1,0.851566,0); rgb(242pt)=(1,0.855406,0); rgb(243pt)=(1,0.859253,0); rgb(244pt)=(1,0.863106,0); rgb(245pt)=(1,0.866964,0); rgb(246pt)=(1,0.870829,0); rgb(247pt)=(1,0.8747,0); rgb(248pt)=(1,0.878577,0); rgb(249pt)=(1,0.88246,0); rgb(250pt)=(1,0.886349,0); rgb(251pt)=(1,0.890243,0); rgb(252pt)=(1,0.894144,0); rgb(253pt)=(1,0.898051,0); rgb(254pt)=(1,0.901964,0); rgb(255pt)=(1,0.905882,0)},
mesh/rows=49]
table[row sep=crcr,header=false] {%
%
56	0.093	-10.9569379218993\\
56	0.0936	-12.760257109251\\
56	0.0942	-14.5635762966026\\
56	0.0948	-16.3668954839542\\
56	0.0954	-18.170214671306\\
56	0.096	-19.9735338586578\\
56	0.0966	-21.7768530460093\\
56	0.0972	-23.5801722333612\\
56	0.0978	-25.3834914207127\\
56	0.0984	-27.1868106080644\\
56	0.099	-28.9901297954162\\
56	0.0996	-30.7934489827678\\
56	0.1002	-32.5967681701194\\
56	0.1008	-34.4000873574712\\
56	0.1014	-36.2034065448229\\
56	0.102	-38.0067257321746\\
56	0.1026	-39.810044919526\\
56	0.1032	-41.6133641068778\\
56	0.1038	-43.4166832942296\\
56	0.1044	-45.2200024815812\\
56	0.105	-47.0233216689328\\
56	0.1056	-48.8266408562845\\
56	0.1062	-50.6299600436363\\
56	0.1068	-52.4332792309879\\
56	0.1074	-54.2365984183397\\
56	0.108	-56.0399176056912\\
56	0.1086	-57.843236793043\\
56	0.1092	-59.6465559803946\\
56	0.1098	-61.4498751677464\\
56	0.1104	-63.2531943550979\\
56	0.111	-65.0565135424497\\
56	0.1116	-66.8598327298015\\
56	0.1122	-68.6631519171531\\
56	0.1128	-70.4664711045046\\
56	0.1134	-72.2697902918563\\
56	0.114	-74.0731094792081\\
56	0.1146	-75.8764286665597\\
56	0.1152	-77.6797478539113\\
56	0.1158	-79.4830670412631\\
56	0.1164	-81.2863862286148\\
56	0.117	-83.0897054159665\\
56	0.1176	-84.8930246033182\\
56	0.1182	-86.6963437906697\\
56	0.1188	-88.4996629780215\\
56	0.1194	-90.3029821653731\\
56	0.12	-92.1063013527249\\
56	0.1206	-93.9096205400765\\
56	0.1212	-95.7129397274282\\
56	0.1218	-97.5162589147799\\
56	0.1224	-99.3195781021316\\
56	0.123	-101.122897289483\\
56.375	0.093	-10.5643587587663\\
56.375	0.0936	-12.3442430704221\\
56.375	0.0942	-14.1241273820779\\
56.375	0.0948	-15.9040116937337\\
56.375	0.0954	-17.6838960053897\\
56.375	0.096	-19.4637803170455\\
56.375	0.0966	-21.2436646287013\\
56.375	0.0972	-23.0235489403572\\
56.375	0.0978	-24.803433252013\\
56.375	0.0984	-26.5833175636687\\
56.375	0.099	-28.3632018753246\\
56.375	0.0996	-30.1430861869804\\
56.375	0.1002	-31.9229704986362\\
56.375	0.1008	-33.702854810292\\
56.375	0.1014	-35.482739121948\\
56.375	0.102	-37.2626234336038\\
56.375	0.1026	-39.0425077452596\\
56.375	0.1032	-40.8223920569155\\
56.375	0.1038	-42.6022763685713\\
56.375	0.1044	-44.3821606802271\\
56.375	0.105	-46.1620449918829\\
56.375	0.1056	-47.9419293035388\\
56.375	0.1062	-49.7218136151946\\
56.375	0.1068	-51.5016979268505\\
56.375	0.1074	-53.2815822385064\\
56.375	0.108	-55.0614665501622\\
56.375	0.1086	-56.841350861818\\
56.375	0.1092	-58.6212351734738\\
56.375	0.1098	-60.4011194851297\\
56.375	0.1104	-62.1810037967855\\
56.375	0.111	-63.9608881084413\\
56.375	0.1116	-65.7407724200972\\
56.375	0.1122	-67.520656731753\\
56.375	0.1128	-69.3005410434089\\
56.375	0.1134	-71.0804253550647\\
56.375	0.114	-72.8603096667206\\
56.375	0.1146	-74.6401939783764\\
56.375	0.1152	-76.4200782900322\\
56.375	0.1158	-78.1999626016881\\
56.375	0.1164	-79.9798469133439\\
56.375	0.117	-81.7597312249997\\
56.375	0.1176	-83.5396155366556\\
56.375	0.1182	-85.3194998483112\\
56.375	0.1188	-87.0993841599673\\
56.375	0.1194	-88.879268471623\\
56.375	0.12	-90.659152783279\\
56.375	0.1206	-92.4390370949346\\
56.375	0.1212	-94.2189214065905\\
56.375	0.1218	-95.9988057182463\\
56.375	0.1224	-97.7786900299022\\
56.375	0.123	-99.5585743415581\\
56.75	0.093	-10.1717795956332\\
56.75	0.0936	-11.9282290315932\\
56.75	0.0942	-13.6846784675532\\
56.75	0.0948	-15.4411279035131\\
56.75	0.0954	-17.1975773394732\\
56.75	0.096	-18.9540267754331\\
56.75	0.0966	-20.7104762113931\\
56.75	0.0972	-22.4669256473532\\
56.75	0.0978	-24.2233750833132\\
56.75	0.0984	-25.9798245192732\\
56.75	0.099	-27.7362739552332\\
56.75	0.0996	-29.4927233911932\\
56.75	0.1002	-31.2491728271531\\
56.75	0.1008	-33.0056222631131\\
56.75	0.1014	-34.7620716990732\\
56.75	0.102	-36.5185211350331\\
56.75	0.1026	-38.2749705709931\\
56.75	0.1032	-40.0314200069532\\
56.75	0.1038	-41.7878694429131\\
56.75	0.1044	-43.5443188788731\\
56.75	0.105	-45.300768314833\\
56.75	0.1056	-47.0572177507931\\
56.75	0.1062	-48.8136671867532\\
56.75	0.1068	-50.570116622713\\
56.75	0.1074	-52.3265660586732\\
56.75	0.108	-54.0830154946331\\
56.75	0.1086	-55.8394649305931\\
56.75	0.1092	-57.595914366553\\
56.75	0.1098	-59.3523638025131\\
56.75	0.1104	-61.1088132384731\\
56.75	0.111	-62.8652626744331\\
56.75	0.1116	-64.6217121103931\\
56.75	0.1122	-66.3781615463531\\
56.75	0.1128	-68.134610982313\\
56.75	0.1134	-69.891060418273\\
56.75	0.114	-71.6475098542331\\
56.75	0.1146	-73.403959290193\\
56.75	0.1152	-75.160408726153\\
56.75	0.1158	-76.916858162113\\
56.75	0.1164	-78.6733075980731\\
56.75	0.117	-80.4297570340329\\
56.75	0.1176	-82.1862064699931\\
56.75	0.1182	-83.942655905953\\
56.75	0.1188	-85.699105341913\\
56.75	0.1194	-87.455554777873\\
56.75	0.12	-89.2120042138331\\
56.75	0.1206	-90.9684536497929\\
56.75	0.1212	-92.724903085753\\
56.75	0.1218	-94.4813525217129\\
56.75	0.1224	-96.237801957673\\
56.75	0.123	-97.9942513936331\\
57.125	0.093	-9.77920043250026\\
57.125	0.0936	-11.5122149927643\\
57.125	0.0942	-13.2452295530285\\
57.125	0.0948	-14.9782441132926\\
57.125	0.0954	-16.7112586735568\\
57.125	0.096	-18.4442732338208\\
57.125	0.0966	-20.177287794085\\
57.125	0.0972	-21.9103023543491\\
57.125	0.0978	-23.6433169146134\\
57.125	0.0984	-25.3763314748775\\
57.125	0.099	-27.1093460351417\\
57.125	0.0996	-28.8423605954058\\
57.125	0.1002	-30.57537515567\\
57.125	0.1008	-32.308389715934\\
57.125	0.1014	-34.0414042761983\\
57.125	0.102	-35.7744188364624\\
57.125	0.1026	-37.5074333967266\\
57.125	0.1032	-39.2404479569908\\
57.125	0.1038	-40.9734625172549\\
57.125	0.1044	-42.706477077519\\
57.125	0.105	-44.4394916377832\\
57.125	0.1056	-46.1725061980475\\
57.125	0.1062	-47.9055207583115\\
57.125	0.1068	-49.6385353185757\\
57.125	0.1074	-51.3715498788398\\
57.125	0.108	-53.104564439104\\
57.125	0.1086	-54.837578999368\\
57.125	0.1092	-56.5705935596321\\
57.125	0.1098	-58.3036081198965\\
57.125	0.1104	-60.0366226801607\\
57.125	0.111	-61.7696372404247\\
57.125	0.1116	-63.5026518006889\\
57.125	0.1122	-65.235666360953\\
57.125	0.1128	-66.9686809212172\\
57.125	0.1134	-68.7016954814812\\
57.125	0.114	-70.4347100417455\\
57.125	0.1146	-72.1677246020098\\
57.125	0.1152	-73.9007391622739\\
57.125	0.1158	-75.6337537225381\\
57.125	0.1164	-77.3667682828021\\
57.125	0.117	-79.0997828430662\\
57.125	0.1176	-80.8327974033305\\
57.125	0.1182	-82.5658119635945\\
57.125	0.1188	-84.2988265238587\\
57.125	0.1194	-86.0318410841228\\
57.125	0.12	-87.764855644387\\
57.125	0.1206	-89.4978702046511\\
57.125	0.1212	-91.2308847649153\\
57.125	0.1218	-92.9638993251795\\
57.125	0.1224	-94.6969138854437\\
57.125	0.123	-96.4299284457079\\
57.5	0.093	-9.3866212693672\\
57.5	0.0936	-11.0962009539353\\
57.5	0.0942	-12.8057806385037\\
57.5	0.0948	-14.5153603230719\\
57.5	0.0954	-16.2249400076404\\
57.5	0.096	-17.9345196922086\\
57.5	0.0966	-19.644099376777\\
57.5	0.0972	-21.3536790613452\\
57.5	0.0978	-23.0632587459137\\
57.5	0.0984	-24.7728384304818\\
57.5	0.099	-26.4824181150502\\
57.5	0.0996	-28.1919977996184\\
57.5	0.1002	-29.9015774841869\\
57.5	0.1008	-31.6111571687551\\
57.5	0.1014	-33.3207368533234\\
57.5	0.102	-35.0303165378917\\
57.5	0.1026	-36.7398962224601\\
57.5	0.1032	-38.4494759070285\\
57.5	0.1038	-40.1590555915967\\
57.5	0.1044	-41.8686352761649\\
57.5	0.105	-43.5782149607334\\
57.5	0.1056	-45.2877946453018\\
57.5	0.1062	-46.9973743298699\\
57.5	0.1068	-48.7069540144383\\
57.5	0.1074	-50.4165336990065\\
57.5	0.108	-52.126113383575\\
57.5	0.1086	-53.8356930681432\\
57.5	0.1092	-55.5452727527114\\
57.5	0.1098	-57.2548524372799\\
57.5	0.1104	-58.9644321218482\\
57.5	0.111	-60.6740118064164\\
57.5	0.1116	-62.3835914909848\\
57.5	0.1122	-64.093171175553\\
57.5	0.1128	-65.8027508601215\\
57.5	0.1134	-67.5123305446897\\
57.5	0.114	-69.221910229258\\
57.5	0.1146	-70.9314899138265\\
57.5	0.1152	-72.6410695983947\\
57.5	0.1158	-74.3506492829631\\
57.5	0.1164	-76.0602289675313\\
57.5	0.117	-77.7698086520995\\
57.5	0.1176	-79.479388336668\\
57.5	0.1182	-81.1889680212361\\
57.5	0.1188	-82.8985477058045\\
57.5	0.1194	-84.6081273903727\\
57.5	0.12	-86.3177070749412\\
57.5	0.1206	-88.0272867595094\\
57.5	0.1212	-89.7368664440778\\
57.5	0.1218	-91.446446128646\\
57.5	0.1224	-93.1560258132145\\
57.5	0.123	-94.8656054977828\\
57.875	0.093	-8.99404210623402\\
57.875	0.0936	-10.6801869151063\\
57.875	0.0942	-12.3663317239789\\
57.875	0.0948	-14.0524765328513\\
57.875	0.0954	-15.738621341724\\
57.875	0.096	-17.4247661505963\\
57.875	0.0966	-19.1109109594688\\
57.875	0.0972	-20.7970557683412\\
57.875	0.0978	-22.4832005772138\\
57.875	0.0984	-24.1693453860861\\
57.875	0.099	-25.8554901949586\\
57.875	0.0996	-27.5416350038311\\
57.875	0.1002	-29.2277798127036\\
57.875	0.1008	-30.913924621576\\
57.875	0.1014	-32.6000694304486\\
57.875	0.102	-34.286214239321\\
57.875	0.1026	-35.9723590481935\\
57.875	0.1032	-37.6585038570661\\
57.875	0.1038	-39.3446486659384\\
57.875	0.1044	-41.0307934748108\\
57.875	0.105	-42.7169382836834\\
57.875	0.1056	-44.4030830925559\\
57.875	0.1062	-46.0892279014283\\
57.875	0.1068	-47.7753727103009\\
57.875	0.1074	-49.4615175191733\\
57.875	0.108	-51.1476623280458\\
57.875	0.1086	-52.8338071369182\\
57.875	0.1092	-54.5199519457906\\
57.875	0.1098	-56.2060967546631\\
57.875	0.1104	-57.8922415635357\\
57.875	0.111	-59.578386372408\\
57.875	0.1116	-61.2645311812806\\
57.875	0.1122	-62.9506759901531\\
57.875	0.1128	-64.6368207990256\\
57.875	0.1134	-66.3229656078979\\
57.875	0.114	-68.0091104167705\\
57.875	0.1146	-69.695255225643\\
57.875	0.1152	-71.3814000345154\\
57.875	0.1158	-73.067544843388\\
57.875	0.1164	-74.7536896522603\\
57.875	0.117	-76.4398344611327\\
57.875	0.1176	-78.1259792700054\\
57.875	0.1182	-79.8121240788777\\
57.875	0.1188	-81.4982688877502\\
57.875	0.1194	-83.1844136966226\\
57.875	0.12	-84.8705585054952\\
57.875	0.1206	-86.5567033143675\\
57.875	0.1212	-88.24284812324\\
57.875	0.1218	-89.9289929321125\\
57.875	0.1224	-91.615137740985\\
57.875	0.123	-93.3012825498577\\
58.25	0.093	-8.60146294310084\\
58.25	0.0936	-10.2641728762775\\
58.25	0.0942	-11.9268828094541\\
58.25	0.0948	-13.5895927426307\\
58.25	0.0954	-15.2523026758074\\
58.25	0.096	-16.9150126089841\\
58.25	0.0966	-18.5777225421606\\
58.25	0.0972	-20.2404324753373\\
58.25	0.0978	-21.9031424085139\\
58.25	0.0984	-23.5658523416905\\
58.25	0.099	-25.2285622748673\\
58.25	0.0996	-26.8912722080438\\
58.25	0.1002	-28.5539821412204\\
58.25	0.1008	-30.216692074397\\
58.25	0.1014	-31.8794020075737\\
58.25	0.102	-33.5421119407504\\
58.25	0.1026	-35.2048218739269\\
58.25	0.1032	-36.8675318071037\\
58.25	0.1038	-38.5302417402803\\
58.25	0.1044	-40.1929516734568\\
58.25	0.105	-41.8556616066335\\
58.25	0.1056	-43.5183715398101\\
58.25	0.1062	-45.1810814729868\\
58.25	0.1068	-46.8437914061634\\
58.25	0.1074	-48.50650133934\\
58.25	0.108	-50.1692112725167\\
58.25	0.1086	-51.8319212056932\\
58.25	0.1092	-53.4946311388699\\
58.25	0.1098	-55.1573410720465\\
58.25	0.1104	-56.8200510052231\\
58.25	0.111	-58.4827609383998\\
58.25	0.1116	-60.1454708715764\\
58.25	0.1122	-61.8081808047531\\
58.25	0.1128	-63.4708907379297\\
58.25	0.1134	-65.1336006711063\\
58.25	0.114	-66.796310604283\\
58.25	0.1146	-68.4590205374597\\
58.25	0.1152	-70.1217304706362\\
58.25	0.1158	-71.7844404038129\\
58.25	0.1164	-73.4471503369895\\
58.25	0.117	-75.1098602701661\\
58.25	0.1176	-76.7725702033428\\
58.25	0.1182	-78.4352801365193\\
58.25	0.1188	-80.097990069696\\
58.25	0.1194	-81.7607000028726\\
58.25	0.12	-83.4234099360493\\
58.25	0.1206	-85.0861198692259\\
58.25	0.1212	-86.7488298024025\\
58.25	0.1218	-88.4115397355791\\
58.25	0.1224	-90.0742496687558\\
58.25	0.123	-91.7369596019325\\
58.625	0.093	-8.20888377996789\\
58.625	0.0936	-9.84815883744864\\
58.625	0.0942	-11.4874338949294\\
58.625	0.0948	-13.1267089524101\\
58.625	0.0954	-14.765984009891\\
58.625	0.096	-16.4052590673717\\
58.625	0.0966	-18.0445341248525\\
58.625	0.0972	-19.6838091823333\\
58.625	0.0978	-21.3230842398141\\
58.625	0.0984	-22.9623592972949\\
58.625	0.099	-24.6016343547758\\
58.625	0.0996	-26.2409094122565\\
58.625	0.1002	-27.8801844697373\\
58.625	0.1008	-29.519459527218\\
58.625	0.1014	-31.1587345846989\\
58.625	0.102	-32.7980096421796\\
58.625	0.1026	-34.4372846996604\\
58.625	0.1032	-36.0765597571414\\
58.625	0.1038	-37.7158348146221\\
58.625	0.1044	-39.3551098721028\\
58.625	0.105	-40.9943849295836\\
58.625	0.1056	-42.6336599870644\\
58.625	0.1062	-44.2729350445452\\
58.625	0.1068	-45.9122101020259\\
58.625	0.1074	-47.5514851595068\\
58.625	0.108	-49.1907602169875\\
58.625	0.1086	-50.8300352744684\\
58.625	0.1092	-52.4693103319491\\
58.625	0.1098	-54.10858538943\\
58.625	0.1104	-55.7478604469107\\
58.625	0.111	-57.3871355043915\\
58.625	0.1116	-59.0264105618724\\
58.625	0.1122	-60.6656856193531\\
58.625	0.1128	-62.3049606768338\\
58.625	0.1134	-63.9442357343146\\
58.625	0.114	-65.5835107917956\\
58.625	0.1146	-67.2227858492763\\
58.625	0.1152	-68.8620609067571\\
58.625	0.1158	-70.5013359642379\\
58.625	0.1164	-72.1406110217187\\
58.625	0.117	-73.7798860791994\\
58.625	0.1176	-75.4191611366803\\
58.625	0.1182	-77.0584361941609\\
58.625	0.1188	-78.6977112516419\\
58.625	0.1194	-80.3369863091225\\
58.625	0.12	-81.9762613666035\\
58.625	0.1206	-83.6155364240841\\
58.625	0.1212	-85.254811481565\\
58.625	0.1218	-86.8940865390457\\
58.625	0.1224	-88.5333615965266\\
58.625	0.123	-90.1726366540074\\
59	0.093	-7.81630461683471\\
59	0.0936	-9.43214479861979\\
59	0.0942	-11.0479849804045\\
59	0.0948	-12.6638251621895\\
59	0.0954	-14.2796653439744\\
59	0.096	-15.8955055257595\\
59	0.0966	-17.5113457075444\\
59	0.0972	-19.1271858893294\\
59	0.0978	-20.7430260711143\\
59	0.0984	-22.3588662528994\\
59	0.099	-23.9747064346843\\
59	0.0996	-25.5905466164693\\
59	0.1002	-27.206386798254\\
59	0.1008	-28.8222269800391\\
59	0.1014	-30.4380671618242\\
59	0.102	-32.053907343609\\
59	0.1026	-33.6697475253939\\
59	0.1032	-35.2855877071788\\
59	0.1038	-36.9014278889639\\
59	0.1044	-38.5172680707487\\
59	0.105	-40.1331082525336\\
59	0.1056	-41.7489484343187\\
59	0.1062	-43.3647886161037\\
59	0.1068	-44.9806287978885\\
59	0.1074	-46.5964689796737\\
59	0.108	-48.2123091614584\\
59	0.1086	-49.8281493432435\\
59	0.1092	-51.4439895250283\\
59	0.1098	-53.0598297068134\\
59	0.1104	-54.6756698885982\\
59	0.111	-56.2915100703832\\
59	0.1116	-57.9073502521683\\
59	0.1122	-59.5231904339531\\
59	0.1128	-61.139030615738\\
59	0.1134	-62.754870797523\\
59	0.114	-64.3707109793081\\
59	0.1146	-65.9865511610928\\
59	0.1152	-67.6023913428778\\
59	0.1158	-69.2182315246628\\
59	0.1164	-70.8340717064478\\
59	0.117	-72.4499118882327\\
59	0.1176	-74.0657520700178\\
59	0.1182	-75.6815922518026\\
59	0.1188	-77.2974324335876\\
59	0.1194	-78.9132726153725\\
59	0.12	-80.5291127971575\\
59	0.1206	-82.1449529789423\\
59	0.1212	-83.7607931607274\\
59	0.1218	-85.3766333425123\\
59	0.1224	-86.9924735242973\\
59	0.123	-88.6083137060824\\
59.375	0.093	-7.42372545370176\\
59.375	0.0936	-9.01613075979083\\
59.375	0.0942	-10.6085360658799\\
59.375	0.0948	-12.200941371969\\
59.375	0.0954	-13.7933466780582\\
59.375	0.096	-15.3857519841473\\
59.375	0.0966	-16.9781572902364\\
59.375	0.0972	-18.5705625963255\\
59.375	0.0978	-20.1629679024146\\
59.375	0.0984	-21.7553732085037\\
59.375	0.099	-23.3477785145928\\
59.375	0.0996	-24.9401838206819\\
59.375	0.1002	-26.532589126771\\
59.375	0.1008	-28.1249944328601\\
59.375	0.1014	-29.7173997389493\\
59.375	0.102	-31.3098050450384\\
59.375	0.1026	-32.9022103511275\\
59.375	0.1032	-34.4946156572166\\
59.375	0.1038	-36.0870209633057\\
59.375	0.1044	-37.6794262693948\\
59.375	0.105	-39.2718315754838\\
59.375	0.1056	-40.8642368815731\\
59.375	0.1062	-42.4566421876622\\
59.375	0.1068	-44.0490474937512\\
59.375	0.1074	-45.6414527998404\\
59.375	0.108	-47.2338581059295\\
59.375	0.1086	-48.8262634120185\\
59.375	0.1092	-50.4186687181076\\
59.375	0.1098	-52.0110740241968\\
59.375	0.1104	-53.6034793302858\\
59.375	0.111	-55.195884636375\\
59.375	0.1116	-56.7882899424642\\
59.375	0.1122	-58.3806952485533\\
59.375	0.1128	-59.9731005546423\\
59.375	0.1134	-61.5655058607314\\
59.375	0.114	-63.1579111668206\\
59.375	0.1146	-64.7503164729096\\
59.375	0.1152	-66.3427217789987\\
59.375	0.1158	-67.935127085088\\
59.375	0.1164	-69.527532391177\\
59.375	0.117	-71.1199376972661\\
59.375	0.1176	-72.7123430033553\\
59.375	0.1182	-74.3047483094442\\
59.375	0.1188	-75.8971536155334\\
59.375	0.1194	-77.4895589216225\\
59.375	0.12	-79.0819642277116\\
59.375	0.1206	-80.6743695338006\\
59.375	0.1212	-82.2667748398899\\
59.375	0.1218	-83.8591801459789\\
59.375	0.1224	-85.4515854520681\\
59.375	0.123	-87.0439907581573\\
59.75	0.093	-7.03114629056859\\
59.75	0.0936	-8.60011672096186\\
59.75	0.0942	-10.1690871513551\\
59.75	0.0948	-11.7380575817483\\
59.75	0.0954	-13.3070280121417\\
59.75	0.096	-14.875998442535\\
59.75	0.0966	-16.4449688729281\\
59.75	0.0972	-18.0139393033215\\
59.75	0.0978	-19.5829097337147\\
59.75	0.0984	-21.151880164108\\
59.75	0.099	-22.7208505945014\\
59.75	0.0996	-24.2898210248945\\
59.75	0.1002	-25.8587914552878\\
59.75	0.1008	-27.4277618856811\\
59.75	0.1014	-28.9967323160744\\
59.75	0.102	-30.5657027464677\\
59.75	0.1026	-32.1346731768608\\
59.75	0.1032	-33.7036436072542\\
59.75	0.1038	-35.2726140376475\\
59.75	0.1044	-36.8415844680407\\
59.75	0.105	-38.4105548984339\\
59.75	0.1056	-39.9795253288272\\
59.75	0.1062	-41.5484957592205\\
59.75	0.1068	-43.1174661896138\\
59.75	0.1074	-44.686436620007\\
59.75	0.108	-46.2554070504003\\
59.75	0.1086	-47.8243774807936\\
59.75	0.1092	-49.3933479111868\\
59.75	0.1098	-50.9623183415802\\
59.75	0.1104	-52.5312887719733\\
59.75	0.111	-54.1002592023666\\
59.75	0.1116	-55.66922963276\\
59.75	0.1122	-57.2382000631532\\
59.75	0.1128	-58.8071704935464\\
59.75	0.1134	-60.3761409239396\\
59.75	0.114	-61.945111354333\\
59.75	0.1146	-63.5140817847263\\
59.75	0.1152	-65.0830522151194\\
59.75	0.1158	-66.6520226455128\\
59.75	0.1164	-68.2209930759061\\
59.75	0.117	-69.7899635062993\\
59.75	0.1176	-71.3589339366927\\
59.75	0.1182	-72.9279043670857\\
59.75	0.1188	-74.4968747974791\\
59.75	0.1194	-76.0658452278723\\
59.75	0.12	-77.6348156582657\\
59.75	0.1206	-79.2037860886588\\
59.75	0.1212	-80.7727565190521\\
59.75	0.1218	-82.3417269494454\\
59.75	0.1224	-83.9106973798388\\
59.75	0.123	-85.4796678102321\\
60.125	0.093	-6.63856712743552\\
60.125	0.0936	-8.1841026821329\\
60.125	0.0942	-9.72963823683028\\
60.125	0.0948	-11.2751737915277\\
60.125	0.0954	-12.8207093462252\\
60.125	0.096	-14.3662449009225\\
60.125	0.0966	-15.91178045562\\
60.125	0.0972	-17.4573160103175\\
60.125	0.0978	-19.0028515650149\\
60.125	0.0984	-20.5483871197123\\
60.125	0.099	-22.0939226744098\\
60.125	0.0996	-23.6394582291072\\
60.125	0.1002	-25.1849937838045\\
60.125	0.1008	-26.7305293385019\\
60.125	0.1014	-28.2760648931995\\
60.125	0.102	-29.8216004478968\\
60.125	0.1026	-31.3671360025943\\
60.125	0.1032	-32.9126715572918\\
60.125	0.1038	-34.4582071119892\\
60.125	0.1044	-36.0037426666865\\
60.125	0.105	-37.5492782213839\\
60.125	0.1056	-39.0948137760814\\
60.125	0.1062	-40.6403493307788\\
60.125	0.1068	-42.1858848854763\\
60.125	0.1074	-43.7314204401738\\
60.125	0.108	-45.2769559948712\\
60.125	0.1086	-46.8224915495686\\
60.125	0.1092	-48.3680271042659\\
60.125	0.1098	-49.9135626589634\\
60.125	0.1104	-51.4590982136608\\
60.125	0.111	-53.0046337683582\\
60.125	0.1116	-54.5501693230558\\
60.125	0.1122	-56.0957048777531\\
60.125	0.1128	-57.6412404324506\\
60.125	0.1134	-59.1867759871479\\
60.125	0.114	-60.7323115418454\\
60.125	0.1146	-62.2778470965428\\
60.125	0.1152	-63.8233826512402\\
60.125	0.1158	-65.3689182059377\\
60.125	0.1164	-66.9144537606351\\
60.125	0.117	-68.4599893153325\\
60.125	0.1176	-70.0055248700301\\
60.125	0.1182	-71.5510604247273\\
60.125	0.1188	-73.0965959794248\\
60.125	0.1194	-74.6421315341222\\
60.125	0.12	-76.1876670888197\\
60.125	0.1206	-77.733202643517\\
60.125	0.1212	-79.2787381982145\\
60.125	0.1218	-80.8242737529118\\
60.125	0.1224	-82.3698093076093\\
60.125	0.123	-83.9153448623069\\
60.5	0.093	-6.24598796430246\\
60.5	0.0936	-7.76808864330405\\
60.5	0.0942	-9.29018932230554\\
60.5	0.0948	-10.8122900013071\\
60.5	0.0954	-12.3343906803087\\
60.5	0.096	-13.8564913593103\\
60.5	0.0966	-15.3785920383119\\
60.5	0.0972	-16.9006927173135\\
60.5	0.0978	-18.4227933963151\\
60.5	0.0984	-19.9448940753167\\
60.5	0.099	-21.4669947543183\\
60.5	0.0996	-22.9890954333199\\
60.5	0.1002	-24.5111961123214\\
60.5	0.1008	-26.033296791323\\
60.5	0.1014	-27.5553974703247\\
60.5	0.102	-29.0774981493262\\
60.5	0.1026	-30.5995988283278\\
60.5	0.1032	-32.1216995073295\\
60.5	0.1038	-33.643800186331\\
60.5	0.1044	-35.1659008653326\\
60.5	0.105	-36.688001544334\\
60.5	0.1056	-38.2101022233358\\
60.5	0.1062	-39.7322029023373\\
60.5	0.1068	-41.2543035813388\\
60.5	0.1074	-42.7764042603405\\
60.5	0.108	-44.298504939342\\
60.5	0.1086	-45.8206056183436\\
60.5	0.1092	-47.3427062973452\\
60.5	0.1098	-48.8648069763468\\
60.5	0.1104	-50.3869076553484\\
60.5	0.111	-51.90900833435\\
60.5	0.1116	-53.4311090133516\\
60.5	0.1122	-54.9532096923532\\
60.5	0.1128	-56.4753103713547\\
60.5	0.1134	-57.9974110503563\\
60.5	0.114	-59.519511729358\\
60.5	0.1146	-61.0416124083595\\
60.5	0.1152	-62.5637130873611\\
60.5	0.1158	-64.0858137663628\\
60.5	0.1164	-65.6079144453643\\
60.5	0.117	-67.1300151243659\\
60.5	0.1176	-68.6521158033675\\
60.5	0.1182	-70.1742164823689\\
60.5	0.1188	-71.6963171613706\\
60.5	0.1194	-73.2184178403721\\
60.5	0.12	-74.7405185193738\\
60.5	0.1206	-76.2626191983752\\
60.5	0.1212	-77.7847198773769\\
60.5	0.1218	-79.3068205563785\\
60.5	0.1224	-80.8289212353801\\
60.5	0.123	-82.3510219143818\\
60.875	0.093	-5.85340880116928\\
60.875	0.0936	-7.35207460447509\\
60.875	0.0942	-8.85074040778079\\
60.875	0.0948	-10.3494062110865\\
60.875	0.0954	-11.8480720143923\\
60.875	0.096	-13.346737817698\\
60.875	0.0966	-14.8454036210037\\
60.875	0.0972	-16.3440694243095\\
60.875	0.0978	-17.8427352276152\\
60.875	0.0984	-19.3414010309209\\
60.875	0.099	-20.8400668342268\\
60.875	0.0996	-22.3387326375325\\
60.875	0.1002	-23.8373984408382\\
60.875	0.1008	-25.3360642441439\\
60.875	0.1014	-26.8347300474497\\
60.875	0.102	-28.3333958507554\\
60.875	0.1026	-29.8320616540611\\
60.875	0.1032	-31.3307274573669\\
60.875	0.1038	-32.8293932606728\\
60.875	0.1044	-34.3280590639783\\
60.875	0.105	-35.8267248672842\\
60.875	0.1056	-37.32539067059\\
60.875	0.1062	-38.8240564738957\\
60.875	0.1068	-40.3227222772014\\
60.875	0.1074	-41.8213880805072\\
60.875	0.108	-43.3200538838129\\
60.875	0.1086	-44.8187196871186\\
60.875	0.1092	-46.3173854904243\\
60.875	0.1098	-47.8160512937302\\
60.875	0.1104	-49.3147170970359\\
60.875	0.111	-50.8133829003416\\
60.875	0.1116	-52.3120487036474\\
60.875	0.1122	-53.8107145069531\\
60.875	0.1128	-55.3093803102588\\
60.875	0.1134	-56.8080461135645\\
60.875	0.114	-58.3067119168704\\
60.875	0.1146	-59.8053777201761\\
60.875	0.1152	-61.3040435234818\\
60.875	0.1158	-62.8027093267876\\
60.875	0.1164	-64.3013751300933\\
60.875	0.117	-65.800040933399\\
60.875	0.1176	-67.2987067367048\\
60.875	0.1182	-68.7973725400104\\
60.875	0.1188	-70.2960383433162\\
60.875	0.1194	-71.7947041466219\\
60.875	0.12	-73.2933699499279\\
60.875	0.1206	-74.7920357532334\\
60.875	0.1212	-76.2907015565393\\
60.875	0.1218	-77.789367359845\\
60.875	0.1224	-79.2880331631508\\
60.875	0.123	-80.7866989664566\\
61.25	0.093	-5.46082963803633\\
61.25	0.0936	-6.93606056564624\\
61.25	0.0942	-8.41129149325616\\
61.25	0.0948	-9.88652242086596\\
61.25	0.0954	-11.361753348476\\
61.25	0.096	-12.8369842760859\\
61.25	0.0966	-14.3122152036957\\
61.25	0.0972	-15.7874461313057\\
61.25	0.0978	-17.2626770589155\\
61.25	0.0984	-18.7379079865254\\
61.25	0.099	-20.2131389141355\\
61.25	0.0996	-21.6883698417453\\
61.25	0.1002	-23.1636007693552\\
61.25	0.1008	-24.6388316969651\\
61.25	0.1014	-26.114062624575\\
61.25	0.102	-27.5892935521849\\
61.25	0.1026	-29.0645244797947\\
61.25	0.1032	-30.5397554074048\\
61.25	0.1038	-32.0149863350147\\
61.25	0.1044	-33.4902172626245\\
61.25	0.105	-34.9654481902344\\
61.25	0.1056	-36.4406791178444\\
61.25	0.1062	-37.9159100454542\\
61.25	0.1068	-39.3911409730641\\
61.25	0.1074	-40.866371900674\\
61.25	0.108	-42.341602828284\\
61.25	0.1086	-43.8168337558939\\
61.25	0.1092	-45.2920646835037\\
61.25	0.1098	-46.7672956111137\\
61.25	0.1104	-48.2425265387235\\
61.25	0.111	-49.7177574663334\\
61.25	0.1116	-51.1929883939434\\
61.25	0.1122	-52.6682193215532\\
61.25	0.1128	-54.1434502491632\\
61.25	0.1134	-55.6186811767731\\
61.25	0.114	-57.093912104383\\
61.25	0.1146	-58.5691430319929\\
61.25	0.1152	-60.0443739596027\\
61.25	0.1158	-61.5196048872127\\
61.25	0.1164	-62.9948358148226\\
61.25	0.117	-64.4700667424324\\
61.25	0.1176	-65.9452976700425\\
61.25	0.1182	-67.4205285976523\\
61.25	0.1188	-68.8957595252622\\
61.25	0.1194	-70.3709904528721\\
61.25	0.12	-71.846221380482\\
61.25	0.1206	-73.3214523080918\\
61.25	0.1212	-74.7966832357018\\
61.25	0.1218	-76.2719141633116\\
61.25	0.1224	-77.7471450909217\\
61.25	0.123	-79.2223760185317\\
61.625	0.093	-5.06825047490327\\
61.625	0.0936	-6.52004652681728\\
61.625	0.0942	-7.9718425787313\\
61.625	0.0948	-9.42363863064531\\
61.625	0.0954	-10.8754346825594\\
61.625	0.096	-12.3272307344735\\
61.625	0.0966	-13.7790267863876\\
61.625	0.0972	-15.2308228383017\\
61.625	0.0978	-16.6826188902157\\
61.625	0.0984	-18.1344149421298\\
61.625	0.099	-19.5862109940439\\
61.625	0.0996	-21.0380070459579\\
61.625	0.1002	-22.4898030978719\\
61.625	0.1008	-23.9415991497859\\
61.625	0.1014	-25.3933952017002\\
61.625	0.102	-26.8451912536142\\
61.625	0.1026	-28.2969873055282\\
61.625	0.1032	-29.7487833574423\\
61.625	0.1038	-31.2005794093564\\
61.625	0.1044	-32.6523754612704\\
61.625	0.105	-34.1041715131844\\
61.625	0.1056	-35.5559675650985\\
61.625	0.1062	-37.0077636170126\\
61.625	0.1068	-38.4595596689267\\
61.625	0.1074	-39.9113557208408\\
61.625	0.108	-41.3631517727548\\
61.625	0.1086	-42.8149478246688\\
61.625	0.1092	-44.2667438765828\\
61.625	0.1098	-45.718539928497\\
61.625	0.1104	-47.170335980411\\
61.625	0.111	-48.6221320323251\\
61.625	0.1116	-50.0739280842392\\
61.625	0.1122	-51.5257241361533\\
61.625	0.1128	-52.9775201880673\\
61.625	0.1134	-54.4293162399813\\
61.625	0.114	-55.8811122918954\\
61.625	0.1146	-57.3329083438094\\
61.625	0.1152	-58.7847043957235\\
61.625	0.1158	-60.2365004476376\\
61.625	0.1164	-61.6882964995517\\
61.625	0.117	-63.1400925514657\\
61.625	0.1176	-64.5918886033799\\
61.625	0.1182	-66.0436846552938\\
61.625	0.1188	-67.4954807072079\\
61.625	0.1194	-68.9472767591219\\
61.625	0.12	-70.399072811036\\
61.625	0.1206	-71.8508688629499\\
61.625	0.1212	-73.3026649148642\\
61.625	0.1218	-74.7544609667782\\
61.625	0.1224	-76.2062570186923\\
61.625	0.123	-77.6580530706065\\
62	0.093	-4.67567131177009\\
62	0.0936	-6.10403248798832\\
62	0.0942	-7.53239366420655\\
62	0.0948	-8.96075484042467\\
62	0.0954	-10.389116016643\\
62	0.096	-11.8174771928611\\
62	0.0966	-13.2458383690794\\
62	0.0972	-14.6741995452976\\
62	0.0978	-16.1025607215158\\
62	0.0984	-17.5309218977341\\
62	0.099	-18.9592830739524\\
62	0.0996	-20.3876442501705\\
62	0.1002	-21.8160054263888\\
62	0.1008	-23.2443666026069\\
62	0.1014	-24.6727277788252\\
62	0.102	-26.1010889550433\\
62	0.1026	-27.5294501312616\\
62	0.1032	-28.9578113074799\\
62	0.1038	-30.386172483698\\
62	0.1044	-31.8145336599163\\
62	0.105	-33.2428948361345\\
62	0.1056	-34.6712560123527\\
62	0.1062	-36.099617188571\\
62	0.1068	-37.5279783647892\\
62	0.1074	-38.9563395410074\\
62	0.108	-40.3847007172257\\
62	0.1086	-41.8130618934438\\
62	0.1092	-43.241423069662\\
62	0.1098	-44.6697842458804\\
62	0.1104	-46.0981454220985\\
62	0.111	-47.5265065983167\\
62	0.1116	-48.954867774535\\
62	0.1122	-50.3832289507532\\
62	0.1128	-51.8115901269714\\
62	0.1134	-53.2399513031895\\
62	0.114	-54.6683124794079\\
62	0.1146	-56.0966736556261\\
62	0.1152	-57.5250348318442\\
62	0.1158	-58.9533960080626\\
62	0.1164	-60.3817571842807\\
62	0.117	-61.8101183604989\\
62	0.1176	-63.2384795367173\\
62	0.1182	-64.6668407129353\\
62	0.1188	-66.0952018891536\\
62	0.1194	-67.5235630653717\\
62	0.12	-68.9519242415901\\
62	0.1206	-70.3802854178082\\
62	0.1212	-71.8086465940264\\
62	0.1218	-73.2370077702446\\
62	0.1224	-74.665368946463\\
62	0.123	-76.0937301226812\\
62.375	0.093	-4.28309214863702\\
62.375	0.0936	-5.68801844915947\\
62.375	0.0942	-7.09294474968181\\
62.375	0.0948	-8.49787105020414\\
62.375	0.0954	-9.90279735072659\\
62.375	0.096	-11.3077236512489\\
62.375	0.0966	-12.7126499517713\\
62.375	0.0972	-14.1175762522937\\
62.375	0.0978	-15.522502552816\\
62.375	0.0984	-16.9274288533384\\
62.375	0.099	-18.3323551538609\\
62.375	0.0996	-19.7372814543833\\
62.375	0.1002	-21.1422077549056\\
62.375	0.1008	-22.5471340554279\\
62.375	0.1014	-23.9520603559504\\
62.375	0.102	-25.3569866564727\\
62.375	0.1026	-26.761912956995\\
62.375	0.1032	-28.1668392575175\\
62.375	0.1038	-29.5717655580399\\
62.375	0.1044	-30.9766918585623\\
62.375	0.105	-32.3816181590846\\
62.375	0.1056	-33.7865444596071\\
62.375	0.1062	-35.1914707601294\\
62.375	0.1068	-36.5963970606517\\
62.375	0.1074	-38.0013233611742\\
62.375	0.108	-39.4062496616965\\
62.375	0.1086	-40.811175962219\\
62.375	0.1092	-42.2161022627413\\
62.375	0.1098	-43.6210285632637\\
62.375	0.1104	-45.0259548637861\\
62.375	0.111	-46.4308811643084\\
62.375	0.1116	-47.8358074648309\\
62.375	0.1122	-49.2407337653532\\
62.375	0.1128	-50.6456600658755\\
62.375	0.1134	-52.050586366398\\
62.375	0.114	-53.4555126669204\\
62.375	0.1146	-54.8604389674427\\
62.375	0.1152	-56.2653652679651\\
62.375	0.1158	-57.6702915684875\\
62.375	0.1164	-59.0752178690099\\
62.375	0.117	-60.4801441695322\\
62.375	0.1176	-61.8850704700546\\
62.375	0.1182	-63.289996770577\\
62.375	0.1188	-64.6949230710994\\
62.375	0.1194	-66.0998493716218\\
62.375	0.12	-67.5047756721442\\
62.375	0.1206	-68.9097019726664\\
62.375	0.1212	-70.3146282731889\\
62.375	0.1218	-71.7195545737112\\
62.375	0.1224	-73.1244808742337\\
62.375	0.123	-74.5294071747562\\
62.75	0.093	-3.89051298550407\\
62.75	0.0936	-5.27200441033051\\
62.75	0.0942	-6.65349583515706\\
62.75	0.0948	-8.03498725998361\\
62.75	0.0954	-9.41647868481016\\
62.75	0.096	-10.7979701096367\\
62.75	0.0966	-12.1794615344633\\
62.75	0.0972	-13.5609529592898\\
62.75	0.0978	-14.9424443841164\\
62.75	0.0984	-16.3239358089428\\
62.75	0.099	-17.7054272337695\\
62.75	0.0996	-19.0869186585959\\
62.75	0.1002	-20.4684100834224\\
62.75	0.1008	-21.849901508249\\
62.75	0.1014	-23.2313929330757\\
62.75	0.102	-24.6128843579021\\
62.75	0.1026	-25.9943757827286\\
62.75	0.1032	-27.3758672075552\\
62.75	0.1038	-28.7573586323817\\
62.75	0.1044	-30.1388500572082\\
62.75	0.105	-31.5203414820347\\
62.75	0.1056	-32.9018329068614\\
62.75	0.1062	-34.2833243316878\\
62.75	0.1068	-35.6648157565144\\
62.75	0.1074	-37.046307181341\\
62.75	0.108	-38.4277986061675\\
62.75	0.1086	-39.809290030994\\
62.75	0.1092	-41.1907814558205\\
62.75	0.1098	-42.5722728806471\\
62.75	0.1104	-43.9537643054737\\
62.75	0.111	-45.3352557303001\\
62.75	0.1116	-46.7167471551268\\
62.75	0.1122	-48.0982385799533\\
62.75	0.1128	-49.4797300047798\\
62.75	0.1134	-50.8612214296063\\
62.75	0.114	-52.242712854433\\
62.75	0.1146	-53.6242042792594\\
62.75	0.1152	-55.005695704086\\
62.75	0.1158	-56.3871871289125\\
62.75	0.1164	-57.7686785537391\\
62.75	0.117	-59.1501699785656\\
62.75	0.1176	-60.5316614033921\\
62.75	0.1182	-61.9131528282186\\
62.75	0.1188	-63.2946442530452\\
62.75	0.1194	-64.6761356778717\\
62.75	0.12	-66.0576271026983\\
62.75	0.1206	-67.4391185275247\\
62.75	0.1212	-68.8206099523513\\
62.75	0.1218	-70.2021013771779\\
62.75	0.1224	-71.5835928020044\\
62.75	0.123	-72.9650842268311\\
63.125	0.093	-3.49793382237101\\
63.125	0.0936	-4.85599037150166\\
63.125	0.0942	-6.21404692063231\\
63.125	0.0948	-7.57210346976296\\
63.125	0.0954	-8.93016001889373\\
63.125	0.096	-10.2882165680245\\
63.125	0.0966	-11.6462731171551\\
63.125	0.0972	-13.0043296662859\\
63.125	0.0978	-14.3623862154166\\
63.125	0.0984	-15.7204427645472\\
63.125	0.099	-17.078499313678\\
63.125	0.0996	-18.4365558628086\\
63.125	0.1002	-19.7946124119393\\
63.125	0.1008	-21.15266896107\\
63.125	0.1014	-22.5107255102008\\
63.125	0.102	-23.8687820593315\\
63.125	0.1026	-25.2268386084621\\
63.125	0.1032	-26.5848951575929\\
63.125	0.1038	-27.9429517067235\\
63.125	0.1044	-29.3010082558542\\
63.125	0.105	-30.6590648049848\\
63.125	0.1056	-32.0171213541157\\
63.125	0.1062	-33.3751779032464\\
63.125	0.1068	-34.733234452377\\
63.125	0.1074	-36.0912910015078\\
63.125	0.108	-37.4493475506384\\
63.125	0.1086	-38.8074040997691\\
63.125	0.1092	-40.1654606488997\\
63.125	0.1098	-41.5235171980305\\
63.125	0.1104	-42.8815737471613\\
63.125	0.111	-44.2396302962919\\
63.125	0.1116	-45.5976868454227\\
63.125	0.1122	-46.9557433945533\\
63.125	0.1128	-48.313799943684\\
63.125	0.1134	-49.6718564928146\\
63.125	0.114	-51.0299130419454\\
63.125	0.1146	-52.3879695910762\\
63.125	0.1152	-53.7460261402068\\
63.125	0.1158	-55.1040826893376\\
63.125	0.1164	-56.4621392384682\\
63.125	0.117	-57.8201957875989\\
63.125	0.1176	-59.1782523367297\\
63.125	0.1182	-60.5363088858602\\
63.125	0.1188	-61.894365434991\\
63.125	0.1194	-63.2524219841216\\
63.125	0.12	-64.6104785332525\\
63.125	0.1206	-65.968535082383\\
63.125	0.1212	-67.3265916315138\\
63.125	0.1218	-68.6846481806444\\
63.125	0.1224	-70.0427047297752\\
63.125	0.123	-71.400761278906\\
63.5	0.093	-3.10535465923783\\
63.5	0.0936	-4.4399763326727\\
63.5	0.0942	-5.77459800610757\\
63.5	0.0948	-7.10921967954232\\
63.5	0.0954	-8.4438413529773\\
63.5	0.096	-9.77846302641206\\
63.5	0.0966	-11.1130846998469\\
63.5	0.0972	-12.4477063732819\\
63.5	0.0978	-13.7823280467167\\
63.5	0.0984	-15.1169497201515\\
63.5	0.099	-16.4515713935865\\
63.5	0.0996	-17.7861930670213\\
63.5	0.1002	-19.1208147404561\\
63.5	0.1008	-20.4554364138909\\
63.5	0.1014	-21.7900580873259\\
63.5	0.102	-23.1246797607607\\
63.5	0.1026	-24.4593014341955\\
63.5	0.1032	-25.7939231076305\\
63.5	0.1038	-27.1285447810653\\
63.5	0.1044	-28.4631664545001\\
63.5	0.105	-29.797788127935\\
63.5	0.1056	-31.1324098013698\\
63.5	0.1062	-32.4670314748047\\
63.5	0.1068	-33.8016531482396\\
63.5	0.1074	-35.1362748216744\\
63.5	0.108	-36.4708964951093\\
63.5	0.1086	-37.8055181685442\\
63.5	0.1092	-39.1401398419789\\
63.5	0.1098	-40.4747615154139\\
63.5	0.1104	-41.8093831888486\\
63.5	0.111	-43.1440048622835\\
63.5	0.1116	-44.4786265357185\\
63.5	0.1122	-45.8132482091532\\
63.5	0.1128	-47.1478698825881\\
63.5	0.1134	-48.482491556023\\
63.5	0.114	-49.8171132294578\\
63.5	0.1146	-51.1517349028927\\
63.5	0.1152	-52.4863565763275\\
63.5	0.1158	-53.8209782497624\\
63.5	0.1164	-55.1555999231973\\
63.5	0.117	-56.4902215966321\\
63.5	0.1176	-57.824843270067\\
63.5	0.1182	-59.1594649435018\\
63.5	0.1188	-60.4940866169367\\
63.5	0.1194	-61.8287082903715\\
63.5	0.12	-63.1633299638064\\
63.5	0.1206	-64.4979516372412\\
63.5	0.1212	-65.8325733106761\\
63.5	0.1218	-67.1671949841109\\
63.5	0.1224	-68.5018166575459\\
63.5	0.123	-69.8364383309809\\
63.875	0.093	-2.71277549610477\\
63.875	0.0936	-4.02396229384385\\
63.875	0.0942	-5.33514909158282\\
63.875	0.0948	-6.64633588932179\\
63.875	0.0954	-7.95752268706087\\
63.875	0.096	-9.26870948479984\\
63.875	0.0966	-10.5798962825388\\
63.875	0.0972	-11.8910830802779\\
63.875	0.0978	-13.2022698780169\\
63.875	0.0984	-14.5134566757559\\
63.875	0.099	-15.824643473495\\
63.875	0.0996	-17.135830271234\\
63.875	0.1002	-18.447017068973\\
63.875	0.1008	-19.7582038667119\\
63.875	0.1014	-21.069390664451\\
63.875	0.102	-22.38057746219\\
63.875	0.1026	-23.691764259929\\
63.875	0.1032	-25.0029510576682\\
63.875	0.1038	-26.3141378554071\\
63.875	0.1044	-27.6253246531461\\
63.875	0.105	-28.9365114508851\\
63.875	0.1056	-30.2476982486241\\
63.875	0.1062	-31.5588850463631\\
63.875	0.1068	-32.8700718441021\\
63.875	0.1074	-34.1812586418412\\
63.875	0.108	-35.4924454395803\\
63.875	0.1086	-36.8036322373192\\
63.875	0.1092	-38.1148190350582\\
63.875	0.1098	-39.4260058327973\\
63.875	0.1104	-40.7371926305362\\
63.875	0.111	-42.0483794282752\\
63.875	0.1116	-43.3595662260143\\
63.875	0.1122	-44.6707530237533\\
63.875	0.1128	-45.9819398214923\\
63.875	0.1134	-47.2931266192313\\
63.875	0.114	-48.6043134169704\\
63.875	0.1146	-49.9155002147094\\
63.875	0.1152	-51.2266870124483\\
63.875	0.1158	-52.5378738101874\\
63.875	0.1164	-53.8490606079264\\
63.875	0.117	-55.1602474056654\\
63.875	0.1176	-56.4714342034046\\
63.875	0.1182	-57.7826210011434\\
63.875	0.1188	-59.0938077988825\\
63.875	0.1194	-60.4049945966215\\
63.875	0.12	-61.7161813943605\\
63.875	0.1206	-63.0273681920994\\
63.875	0.1212	-64.3385549898386\\
63.875	0.1218	-65.6497417875776\\
63.875	0.1224	-66.9609285853167\\
63.875	0.123	-68.2721153830557\\
64.25	0.093	-2.3201963329717\\
64.25	0.0936	-3.60794825501478\\
64.25	0.0942	-4.89570017705796\\
64.25	0.0948	-6.18345209910115\\
64.25	0.0954	-7.47120402114433\\
64.25	0.096	-8.75895594318752\\
64.25	0.0966	-10.0467078652307\\
64.25	0.0972	-11.3344597872739\\
64.25	0.0978	-12.6222117093171\\
64.25	0.0984	-13.9099636313601\\
64.25	0.099	-15.1977155534034\\
64.25	0.0996	-16.4854674754466\\
64.25	0.1002	-17.7732193974897\\
64.25	0.1008	-19.0609713195329\\
64.25	0.1014	-20.3487232415762\\
64.25	0.102	-21.6364751636193\\
64.25	0.1026	-22.9242270856624\\
64.25	0.1032	-24.2119790077056\\
64.25	0.1038	-25.4997309297488\\
64.25	0.1044	-26.787482851792\\
64.25	0.105	-28.0752347738351\\
64.25	0.1056	-29.3629866958784\\
64.25	0.1062	-30.6507386179215\\
64.25	0.1068	-31.9384905399646\\
64.25	0.1074	-33.2262424620079\\
64.25	0.108	-34.513994384051\\
64.25	0.1086	-35.8017463060942\\
64.25	0.1092	-37.0894982281374\\
64.25	0.1098	-38.3772501501805\\
64.25	0.1104	-39.6650020722237\\
64.25	0.111	-40.9527539942669\\
64.25	0.1116	-42.2405059163101\\
64.25	0.1122	-43.5282578383533\\
64.25	0.1128	-44.8160097603964\\
64.25	0.1134	-46.1037616824395\\
64.25	0.114	-47.3915136044828\\
64.25	0.1146	-48.679265526526\\
64.25	0.1152	-49.9670174485691\\
64.25	0.1158	-51.2547693706124\\
64.25	0.1164	-52.5425212926555\\
64.25	0.117	-53.8302732146987\\
64.25	0.1176	-55.1180251367419\\
64.25	0.1182	-56.4057770587849\\
64.25	0.1188	-57.6935289808282\\
64.25	0.1194	-58.9812809028713\\
64.25	0.12	-60.2690328249146\\
64.25	0.1206	-61.5567847469576\\
64.25	0.1212	-62.8445366690008\\
64.25	0.1218	-64.132288591044\\
64.25	0.1224	-65.4200405130873\\
64.25	0.123	-66.7077924351305\\
64.625	0.093	-1.92761716983864\\
64.625	0.0936	-3.19193421618593\\
64.625	0.0942	-4.45625126253321\\
64.625	0.0948	-5.7205683088805\\
64.625	0.0954	-6.98488535522802\\
64.625	0.096	-8.2492024015753\\
64.625	0.0966	-9.51351944792259\\
64.625	0.0972	-10.77783649427\\
64.625	0.0978	-12.0421535406173\\
64.625	0.0984	-13.3064705869646\\
64.625	0.099	-14.570787633312\\
64.625	0.0996	-15.8351046796593\\
64.625	0.1002	-17.0994217260067\\
64.625	0.1008	-18.3637387723539\\
64.625	0.1014	-19.6280558187013\\
64.625	0.102	-20.8923728650486\\
64.625	0.1026	-22.1566899113959\\
64.625	0.1032	-23.4210069577433\\
64.625	0.1038	-24.6853240040906\\
64.625	0.1044	-25.9496410504379\\
64.625	0.105	-27.2139580967853\\
64.625	0.1056	-28.4782751431327\\
64.625	0.1062	-29.74259218948\\
64.625	0.1068	-31.0069092358273\\
64.625	0.1074	-32.2712262821747\\
64.625	0.108	-33.535543328522\\
64.625	0.1086	-34.7998603748692\\
64.625	0.1092	-36.0641774212165\\
64.625	0.1098	-37.328494467564\\
64.625	0.1104	-38.5928115139113\\
64.625	0.111	-39.8571285602586\\
64.625	0.1116	-41.121445606606\\
64.625	0.1122	-42.3857626529533\\
64.625	0.1128	-43.6500796993006\\
64.625	0.1134	-44.914396745648\\
64.625	0.114	-46.1787137919954\\
64.625	0.1146	-47.4430308383427\\
64.625	0.1152	-48.70734788469\\
64.625	0.1158	-49.9716649310374\\
64.625	0.1164	-51.2359819773847\\
64.625	0.117	-52.5002990237319\\
64.625	0.1176	-53.7646160700793\\
64.625	0.1182	-55.0289331164266\\
64.625	0.1188	-56.293250162774\\
64.625	0.1194	-57.5575672091213\\
64.625	0.12	-58.8218842554687\\
64.625	0.1206	-60.0862013018159\\
64.625	0.1212	-61.3505183481633\\
64.625	0.1218	-62.6148353945106\\
64.625	0.1224	-63.879152440858\\
64.625	0.123	-65.1434694872055\\
65	0.093	-1.53503800670558\\
65	0.0936	-2.77592017735708\\
65	0.0942	-4.01680234800858\\
65	0.0948	-5.25768451865997\\
65	0.0954	-6.49856668931159\\
65	0.096	-7.73944885996298\\
65	0.0966	-8.98033103061448\\
65	0.0972	-10.2212132012661\\
65	0.0978	-11.4620953719175\\
65	0.0984	-12.702977542569\\
65	0.099	-13.9438597132206\\
65	0.0996	-15.184741883872\\
65	0.1002	-16.4256240545235\\
65	0.1008	-17.6665062251749\\
65	0.1014	-18.9073883958265\\
65	0.102	-20.148270566478\\
65	0.1026	-21.3891527371294\\
65	0.1032	-22.630034907781\\
65	0.1038	-23.8709170784325\\
65	0.1044	-25.1117992490839\\
65	0.105	-26.3526814197354\\
65	0.1056	-27.593563590387\\
65	0.1062	-28.8344457610384\\
65	0.1068	-30.0753279316899\\
65	0.1074	-31.3162101023414\\
65	0.108	-32.5570922729929\\
65	0.1086	-33.7979744436444\\
65	0.1092	-35.0388566142958\\
65	0.1098	-36.2797387849474\\
65	0.1104	-37.5206209555989\\
65	0.111	-38.7615031262503\\
65	0.1116	-40.0023852969019\\
65	0.1122	-41.2432674675533\\
65	0.1128	-42.4841496382048\\
65	0.1134	-43.7250318088563\\
65	0.114	-44.9659139795078\\
65	0.1146	-46.2067961501593\\
65	0.1152	-47.4476783208108\\
65	0.1158	-48.6885604914623\\
65	0.1164	-49.9294426621138\\
65	0.117	-51.1703248327652\\
65	0.1176	-52.4112070034168\\
65	0.1182	-53.6520891740682\\
65	0.1188	-54.8929713447197\\
65	0.1194	-56.1338535153712\\
65	0.12	-57.3747356860229\\
65	0.1206	-58.6156178566741\\
65	0.1212	-59.8565000273258\\
65	0.1218	-61.0973821979773\\
65	0.1224	-62.3382643686288\\
65	0.123	-63.5791465392804\\
65.375	0.093	-1.14245884357251\\
65.375	0.0936	-2.35990613852812\\
65.375	0.0942	-3.57735343348372\\
65.375	0.0948	-4.79480072843933\\
65.375	0.0954	-6.01224802339505\\
65.375	0.096	-7.22969531835065\\
65.375	0.0966	-8.44714261330626\\
65.375	0.0972	-9.66458990826197\\
65.375	0.0978	-10.8820372032177\\
65.375	0.0984	-12.0994844981733\\
65.375	0.099	-13.316931793129\\
65.375	0.0996	-14.5343790880846\\
65.375	0.1002	-15.7518263830402\\
65.375	0.1008	-16.9692736779958\\
65.375	0.1014	-18.1867209729515\\
65.375	0.102	-19.4041682679072\\
65.375	0.1026	-20.6216155628629\\
65.375	0.1032	-21.8390628578186\\
65.375	0.1038	-23.0565101527742\\
65.375	0.1044	-24.2739574477298\\
65.375	0.105	-25.4914047426854\\
65.375	0.1056	-26.7088520376411\\
65.375	0.1062	-27.9262993325968\\
65.375	0.1068	-29.1437466275524\\
65.375	0.1074	-30.3611939225082\\
65.375	0.108	-31.5786412174638\\
65.375	0.1086	-32.7960885124194\\
65.375	0.1092	-34.013535807375\\
65.375	0.1098	-35.2309831023307\\
65.375	0.1104	-36.4484303972863\\
65.375	0.111	-37.665877692242\\
65.375	0.1116	-38.8833249871977\\
65.375	0.1122	-40.1007722821533\\
65.375	0.1128	-41.3182195771089\\
65.375	0.1134	-42.5356668720646\\
65.375	0.114	-43.7531141670203\\
65.375	0.1146	-44.9705614619759\\
65.375	0.1152	-46.1880087569315\\
65.375	0.1158	-47.4054560518873\\
65.375	0.1164	-48.6229033468429\\
65.375	0.117	-49.8403506417985\\
65.375	0.1176	-51.0577979367542\\
65.375	0.1182	-52.2752452317097\\
65.375	0.1188	-53.4926925266655\\
65.375	0.1194	-54.7101398216211\\
65.375	0.12	-55.9275871165769\\
65.375	0.1206	-57.1450344115324\\
65.375	0.1212	-58.3624817064881\\
65.375	0.1218	-59.5799290014437\\
65.375	0.1224	-60.7973762963994\\
65.375	0.123	-62.0148235913551\\
65.75	0.093	-0.749879680439449\\
65.75	0.0936	-1.94389209969916\\
65.75	0.0942	-3.13790451895898\\
65.75	0.0948	-4.3319169382188\\
65.75	0.0954	-5.52592935747862\\
65.75	0.096	-6.71994177673844\\
65.75	0.0966	-7.91395419599826\\
65.75	0.0972	-9.10796661525808\\
65.75	0.0978	-10.3019790345179\\
65.75	0.0984	-11.4959914537776\\
65.75	0.099	-12.6900038730375\\
65.75	0.0996	-13.8840162922974\\
65.75	0.1002	-15.0780287115571\\
65.75	0.1008	-16.2720411308169\\
65.75	0.1014	-17.4660535500768\\
65.75	0.102	-18.6600659693365\\
65.75	0.1026	-19.8540783885963\\
65.75	0.1032	-21.0480908078563\\
65.75	0.1038	-22.242103227116\\
65.75	0.1044	-23.4361156463758\\
65.75	0.105	-24.6301280656355\\
65.75	0.1056	-25.8241404848955\\
65.75	0.1062	-27.0181529041553\\
65.75	0.1068	-28.212165323415\\
65.75	0.1074	-29.4061777426749\\
65.75	0.108	-30.6001901619347\\
65.75	0.1086	-31.7942025811944\\
65.75	0.1092	-32.9882150004543\\
65.75	0.1098	-34.1822274197141\\
65.75	0.1104	-35.3762398389739\\
65.75	0.111	-36.5702522582337\\
65.75	0.1116	-37.7642646774937\\
65.75	0.1122	-38.9582770967534\\
65.75	0.1128	-40.1522895160132\\
65.75	0.1134	-41.3463019352729\\
65.75	0.114	-42.5403143545328\\
65.75	0.1146	-43.7343267737926\\
65.75	0.1152	-44.9283391930524\\
65.75	0.1158	-46.1223516123123\\
65.75	0.1164	-47.3163640315721\\
65.75	0.117	-48.5103764508318\\
65.75	0.1176	-49.7043888700917\\
65.75	0.1182	-50.8984012893513\\
65.75	0.1188	-52.0924137086113\\
65.75	0.1194	-53.2864261278711\\
65.75	0.12	-54.4804385471309\\
65.75	0.1206	-55.6744509663906\\
65.75	0.1212	-56.8684633856506\\
65.75	0.1218	-58.0624758049103\\
65.75	0.1224	-59.2564882241702\\
65.75	0.123	-60.4505006434301\\
66.125	0.093	-0.357300517306385\\
66.125	0.0936	-1.52787806087031\\
66.125	0.0942	-2.69845560443423\\
66.125	0.0948	-3.86903314799815\\
66.125	0.0954	-5.0396106915623\\
66.125	0.096	-6.21018823512622\\
66.125	0.0966	-7.38076577869015\\
66.125	0.0972	-8.55134332225418\\
66.125	0.0978	-9.72192086581811\\
66.125	0.0984	-10.892498409382\\
66.125	0.099	-12.0630759529461\\
66.125	0.0996	-13.2336534965101\\
66.125	0.1002	-14.404231040074\\
66.125	0.1008	-15.5748085836379\\
66.125	0.1014	-16.745386127202\\
66.125	0.102	-17.9159636707659\\
66.125	0.1026	-19.0865412143298\\
66.125	0.1032	-20.2571187578939\\
66.125	0.1038	-21.4276963014579\\
66.125	0.1044	-22.5982738450218\\
66.125	0.105	-23.7688513885857\\
66.125	0.1056	-24.9394289321498\\
66.125	0.1062	-26.1100064757137\\
66.125	0.1068	-27.2805840192776\\
66.125	0.1074	-28.4511615628417\\
66.125	0.108	-29.6217391064056\\
66.125	0.1086	-30.7923166499696\\
66.125	0.1092	-31.9628941935335\\
66.125	0.1098	-33.1334717370976\\
66.125	0.1104	-34.3040492806615\\
66.125	0.111	-35.4746268242254\\
66.125	0.1116	-36.6452043677895\\
66.125	0.1122	-37.8157819113534\\
66.125	0.1128	-38.9863594549174\\
66.125	0.1134	-40.1569369984813\\
66.125	0.114	-41.3275145420454\\
66.125	0.1146	-42.4980920856093\\
66.125	0.1152	-43.6686696291732\\
66.125	0.1158	-44.8392471727373\\
66.125	0.1164	-46.0098247163012\\
66.125	0.117	-47.1804022598651\\
66.125	0.1176	-48.3509798034293\\
66.125	0.1182	-49.5215573469931\\
66.125	0.1188	-50.6921348905571\\
66.125	0.1194	-51.862712434121\\
66.125	0.12	-53.0332899776851\\
66.125	0.1206	-54.2038675212489\\
66.125	0.1212	-55.3744450648129\\
66.125	0.1218	-56.5450226083769\\
66.125	0.1224	-57.715600151941\\
66.125	0.123	-58.886177695505\\
66.5	0.093	0.0352786458267929\\
66.5	0.0936	-1.11186402204135\\
66.5	0.0942	-2.25900668990948\\
66.5	0.0948	-3.40614935777751\\
66.5	0.0954	-4.55329202564576\\
66.5	0.096	-5.7004346935139\\
66.5	0.0966	-6.84757736138192\\
66.5	0.0972	-7.99472002925017\\
66.5	0.0978	-9.14186269711831\\
66.5	0.0984	-10.2890053649863\\
66.5	0.099	-11.4361480328546\\
66.5	0.0996	-12.5832907007226\\
66.5	0.1002	-13.7304333685908\\
66.5	0.1008	-14.8775760364589\\
66.5	0.1014	-16.024718704327\\
66.5	0.102	-17.1718613721952\\
66.5	0.1026	-18.3190040400633\\
66.5	0.1032	-19.4661467079314\\
66.5	0.1038	-20.6132893757996\\
66.5	0.1044	-21.7604320436676\\
66.5	0.105	-22.9075747115357\\
66.5	0.1056	-24.054717379404\\
66.5	0.1062	-25.201860047272\\
66.5	0.1068	-26.3490027151402\\
66.5	0.1074	-27.4961453830084\\
66.5	0.108	-28.6432880508764\\
66.5	0.1086	-29.7904307187446\\
66.5	0.1092	-30.9375733866127\\
66.5	0.1098	-32.0847160544809\\
66.5	0.1104	-33.231858722349\\
66.5	0.111	-34.3790013902171\\
66.5	0.1116	-35.5261440580853\\
66.5	0.1122	-36.6732867259534\\
66.5	0.1128	-37.8204293938214\\
66.5	0.1134	-38.9675720616896\\
66.5	0.114	-40.1147147295578\\
66.5	0.1146	-41.2618573974258\\
66.5	0.1152	-42.409000065294\\
66.5	0.1158	-43.5561427331622\\
66.5	0.1164	-44.7032854010303\\
66.5	0.117	-45.8504280688984\\
66.5	0.1176	-46.9975707367665\\
66.5	0.1182	-48.1447134046346\\
66.5	0.1188	-49.2918560725028\\
66.5	0.1194	-50.4389987403708\\
66.5	0.12	-51.5861414082391\\
66.5	0.1206	-52.7332840761071\\
66.5	0.1212	-53.8804267439752\\
66.5	0.1218	-55.0275694118434\\
66.5	0.1224	-56.1747120797116\\
66.5	0.123	-57.3218547475798\\
66.875	0.093	0.427857808959743\\
66.875	0.0936	-0.695849983212497\\
66.875	0.0942	-1.81955777538474\\
66.875	0.0948	-2.94326556755698\\
66.875	0.0954	-4.06697335972933\\
66.875	0.096	-5.19068115190157\\
66.875	0.0966	-6.31438894407381\\
66.875	0.0972	-7.43809673624617\\
66.875	0.0978	-8.56180452841852\\
66.875	0.0984	-9.68551232059076\\
66.875	0.099	-10.8092201127631\\
66.875	0.0996	-11.9329279049354\\
66.875	0.1002	-13.0566356971076\\
66.875	0.1008	-14.1803434892798\\
66.875	0.1014	-15.3040512814523\\
66.875	0.102	-16.4277590736245\\
66.875	0.1026	-17.5514668657968\\
66.875	0.1032	-18.6751746579691\\
66.875	0.1038	-19.7988824501414\\
66.875	0.1044	-20.9225902423136\\
66.875	0.105	-22.0462980344859\\
66.875	0.1056	-23.1700058266582\\
66.875	0.1062	-24.2937136188306\\
66.875	0.1068	-25.4174214110028\\
66.875	0.1074	-26.5411292031752\\
66.875	0.108	-27.6648369953474\\
66.875	0.1086	-28.7885447875196\\
66.875	0.1092	-29.9122525796919\\
66.875	0.1098	-31.0359603718642\\
66.875	0.1104	-32.1596681640366\\
66.875	0.111	-33.2833759562088\\
66.875	0.1116	-34.4070837483812\\
66.875	0.1122	-35.5307915405534\\
66.875	0.1128	-36.6544993327257\\
66.875	0.1134	-37.7782071248979\\
66.875	0.114	-38.9019149170703\\
66.875	0.1146	-40.0256227092426\\
66.875	0.1152	-41.1493305014149\\
66.875	0.1158	-42.2730382935872\\
66.875	0.1164	-43.3967460857594\\
66.875	0.117	-44.5204538779317\\
66.875	0.1176	-45.644161670104\\
66.875	0.1182	-46.7678694622762\\
66.875	0.1188	-47.8915772544486\\
66.875	0.1194	-49.0152850466209\\
66.875	0.12	-50.1389928387932\\
66.875	0.1206	-51.2627006309654\\
66.875	0.1212	-52.3864084231377\\
66.875	0.1218	-53.5101162153099\\
66.875	0.1224	-54.6338240074823\\
66.875	0.123	-55.7575317996548\\
67.25	0.093	0.820436972092921\\
67.25	0.0936	-0.279835944383535\\
67.25	0.0942	-1.38010886085988\\
67.25	0.0948	-2.48038177733633\\
67.25	0.0954	-3.58065469381279\\
67.25	0.096	-4.68092761028925\\
67.25	0.0966	-5.7812005267657\\
67.25	0.0972	-6.88147344324216\\
67.25	0.0978	-7.98174635971861\\
67.25	0.0984	-9.08201927619507\\
67.25	0.099	-10.1822921926715\\
67.25	0.0996	-11.282565109148\\
67.25	0.1002	-12.3828380256244\\
67.25	0.1008	-13.4831109421008\\
67.25	0.1014	-14.5833838585773\\
67.25	0.102	-15.6836567750537\\
67.25	0.1026	-16.7839296915301\\
67.25	0.1032	-17.8842026080067\\
67.25	0.1038	-18.9844755244831\\
67.25	0.1044	-20.0847484409595\\
67.25	0.105	-21.185021357436\\
67.25	0.1056	-22.2852942739124\\
67.25	0.1062	-23.3855671903889\\
67.25	0.1068	-24.4858401068653\\
67.25	0.1074	-25.5861130233418\\
67.25	0.108	-26.6863859398183\\
67.25	0.1086	-27.7866588562947\\
67.25	0.1092	-28.886931772771\\
67.25	0.1098	-29.9872046892476\\
67.25	0.1104	-31.087477605724\\
67.25	0.111	-32.1877505222004\\
67.25	0.1116	-33.288023438677\\
67.25	0.1122	-34.3882963551533\\
67.25	0.1128	-35.4885692716298\\
67.25	0.1134	-36.5888421881062\\
67.25	0.114	-37.6891151045827\\
67.25	0.1146	-38.7893880210592\\
67.25	0.1152	-39.8896609375356\\
67.25	0.1158	-40.9899338540121\\
67.25	0.1164	-42.0902067704885\\
67.25	0.117	-43.1904796869649\\
67.25	0.1176	-44.2907526034414\\
67.25	0.1182	-45.3910255199178\\
67.25	0.1188	-46.4912984363942\\
67.25	0.1194	-47.5915713528707\\
67.25	0.12	-48.6918442693473\\
67.25	0.1206	-49.7921171858235\\
67.25	0.1212	-50.8923901023001\\
67.25	0.1218	-51.9926630187765\\
67.25	0.1224	-53.092935935253\\
67.25	0.123	-54.1932088517295\\
67.625	0.093	1.21301613522598\\
67.625	0.0936	0.136178094445427\\
67.625	0.0942	-0.940659946335131\\
67.625	0.0948	-2.01749798711569\\
67.625	0.0954	-3.09433602789647\\
67.625	0.096	-4.17117406867703\\
67.625	0.0966	-5.24801210945759\\
67.625	0.0972	-6.32485015023826\\
67.625	0.0978	-7.40168819101882\\
67.625	0.0984	-8.47852623179938\\
67.625	0.099	-9.55536427258016\\
67.625	0.0996	-10.6322023133607\\
67.625	0.1002	-11.7090403541413\\
67.625	0.1008	-12.7858783949218\\
67.625	0.1014	-13.8627164357025\\
67.625	0.102	-14.9395544764831\\
67.625	0.1026	-16.0163925172636\\
67.625	0.1032	-17.0932305580443\\
67.625	0.1038	-18.170068598825\\
67.625	0.1044	-19.2469066396055\\
67.625	0.105	-20.3237446803861\\
67.625	0.1056	-21.4005827211668\\
67.625	0.1062	-22.4774207619473\\
67.625	0.1068	-23.5542588027279\\
67.625	0.1074	-24.6310968435085\\
67.625	0.108	-25.7079348842892\\
67.625	0.1086	-26.7847729250698\\
67.625	0.1092	-27.8616109658503\\
67.625	0.1098	-28.938449006631\\
67.625	0.1104	-30.0152870474116\\
67.625	0.111	-31.0921250881921\\
67.625	0.1116	-32.1689631289729\\
67.625	0.1122	-33.2458011697535\\
67.625	0.1128	-34.322639210534\\
67.625	0.1134	-35.3994772513146\\
67.625	0.114	-36.4763152920953\\
67.625	0.1146	-37.5531533328758\\
67.625	0.1152	-38.6299913736564\\
67.625	0.1158	-39.706829414437\\
67.625	0.1164	-40.7836674552177\\
67.625	0.117	-41.8605054959983\\
67.625	0.1176	-42.9373435367789\\
67.625	0.1182	-44.0141815775594\\
67.625	0.1188	-45.0910196183401\\
67.625	0.1194	-46.1678576591206\\
67.625	0.12	-47.2446956999013\\
67.625	0.1206	-48.3215337406818\\
67.625	0.1212	-49.3983717814625\\
67.625	0.1218	-50.4752098222431\\
67.625	0.1224	-51.5520478630237\\
67.625	0.123	-52.6288859038044\\
68	0.093	1.60559529835905\\
68	0.0936	0.552192133274275\\
68	0.0942	-0.501211031810499\\
68	0.0948	-1.55461419689516\\
68	0.0954	-2.60801736198005\\
68	0.096	-3.66142052706482\\
68	0.0966	-4.71482369214948\\
68	0.0972	-5.76822685723437\\
68	0.0978	-6.82163002231914\\
68	0.0984	-7.8750331874038\\
68	0.099	-8.92843635248869\\
68	0.0996	-9.98183951757335\\
68	0.1002	-11.0352426826581\\
68	0.1008	-12.0886458477429\\
68	0.1014	-13.1420490128277\\
68	0.102	-14.1954521779124\\
68	0.1026	-15.2488553429972\\
68	0.1032	-16.302258508082\\
68	0.1038	-17.3556616731668\\
68	0.1044	-18.4090648382515\\
68	0.105	-19.4624680033362\\
68	0.1056	-20.5158711684211\\
68	0.1062	-21.5692743335059\\
68	0.1068	-22.6226774985905\\
68	0.1074	-23.6760806636754\\
68	0.108	-24.7294838287601\\
68	0.1086	-25.7828869938448\\
68	0.1092	-26.8362901589296\\
68	0.1098	-27.8896933240144\\
68	0.1104	-28.9430964890992\\
68	0.111	-29.9964996541839\\
68	0.1116	-31.0499028192687\\
68	0.1122	-32.1033059843535\\
68	0.1128	-33.1567091494383\\
68	0.1134	-34.2101123145229\\
68	0.114	-35.2635154796078\\
68	0.1146	-36.3169186446926\\
68	0.1152	-37.3703218097772\\
68	0.1158	-38.4237249748621\\
68	0.1164	-39.4771281399469\\
68	0.117	-40.5305313050316\\
68	0.1176	-41.5839344701164\\
68	0.1182	-42.637337635201\\
68	0.1188	-43.6907408002859\\
68	0.1194	-44.7441439653707\\
68	0.12	-45.7975471304554\\
68	0.1206	-46.8509502955401\\
68	0.1212	-47.904353460625\\
68	0.1218	-48.9577566257096\\
68	0.1224	-50.0111597907945\\
68	0.123	-51.0645629558794\\
68.375	0.093	1.99817446149211\\
68.375	0.0936	0.968206172103237\\
68.375	0.0942	-0.0617621172856389\\
68.375	0.0948	-1.09173040667451\\
68.375	0.0954	-2.1216986960635\\
68.375	0.096	-3.15166698545249\\
68.375	0.0966	-4.18163527484137\\
68.375	0.0972	-5.21160356423036\\
68.375	0.0978	-6.24157185361923\\
68.375	0.0984	-7.27154014300811\\
68.375	0.099	-8.3015084323971\\
68.375	0.0996	-9.33147672178598\\
68.375	0.1002	-10.3614450111749\\
68.375	0.1008	-11.3914133005638\\
68.375	0.1014	-12.4213815899528\\
68.375	0.102	-13.4513498793417\\
68.375	0.1026	-14.4813181687306\\
68.375	0.1032	-15.5112864581196\\
68.375	0.1038	-16.5412547475084\\
68.375	0.1044	-17.5712230368973\\
68.375	0.105	-18.6011913262863\\
68.375	0.1056	-19.6311596156753\\
68.375	0.1062	-20.6611279050642\\
68.375	0.1068	-21.6910961944531\\
68.375	0.1074	-22.721064483842\\
68.375	0.108	-23.7510327732309\\
68.375	0.1086	-24.7810010626199\\
68.375	0.1092	-25.8109693520088\\
68.375	0.1098	-26.8409376413978\\
68.375	0.1104	-27.8709059307866\\
68.375	0.111	-28.9008742201755\\
68.375	0.1116	-29.9308425095645\\
68.375	0.1122	-30.9608107989534\\
68.375	0.1128	-31.9907790883423\\
68.375	0.1134	-33.0207473777313\\
68.375	0.114	-34.0507156671202\\
68.375	0.1146	-35.0806839565091\\
68.375	0.1152	-36.110652245898\\
68.375	0.1158	-37.140620535287\\
68.375	0.1164	-38.1705888246759\\
68.375	0.117	-39.2005571140647\\
68.375	0.1176	-40.2305254034538\\
68.375	0.1182	-41.2604936928426\\
68.375	0.1188	-42.2904619822316\\
68.375	0.1194	-43.3204302716205\\
68.375	0.12	-44.3503985610095\\
68.375	0.1206	-45.3803668503982\\
68.375	0.1212	-46.4103351397872\\
68.375	0.1218	-47.4403034291762\\
68.375	0.1224	-48.4702717185652\\
68.375	0.123	-49.5002400079542\\
68.75	0.093	2.39075362462518\\
68.75	0.0936	1.38422021093209\\
68.75	0.0942	0.377686797239107\\
68.75	0.0948	-0.628846616453984\\
68.75	0.0954	-1.63538003014719\\
68.75	0.096	-2.64191344384017\\
68.75	0.0966	-3.64844685753326\\
68.75	0.0972	-4.65498027122635\\
68.75	0.0978	-5.66151368491944\\
68.75	0.0984	-6.66804709861253\\
68.75	0.099	-7.67458051230562\\
68.75	0.0996	-8.68111392599872\\
68.75	0.1002	-9.68764733969181\\
68.75	0.1008	-10.6941807533848\\
68.75	0.1014	-11.700714167078\\
68.75	0.102	-12.7072475807711\\
68.75	0.1026	-13.7137809944641\\
68.75	0.1032	-14.7203144081573\\
68.75	0.1038	-15.7268478218504\\
68.75	0.1044	-16.7333812355433\\
68.75	0.105	-17.7399146492364\\
68.75	0.1056	-18.7464480629295\\
68.75	0.1062	-19.7529814766226\\
68.75	0.1068	-20.7595148903157\\
68.75	0.1074	-21.7660483040088\\
68.75	0.108	-22.7725817177019\\
68.75	0.1086	-23.779115131395\\
68.75	0.1092	-24.785648545088\\
68.75	0.1098	-25.7921819587812\\
68.75	0.1104	-26.7987153724742\\
68.75	0.111	-27.8052487861672\\
68.75	0.1116	-28.8117821998604\\
68.75	0.1122	-29.8183156135535\\
68.75	0.1128	-30.8248490272465\\
68.75	0.1134	-31.8313824409396\\
68.75	0.114	-32.8379158546328\\
68.75	0.1146	-33.8444492683258\\
68.75	0.1152	-34.8509826820189\\
68.75	0.1158	-35.857516095712\\
68.75	0.1164	-36.864049509405\\
68.75	0.117	-37.8705829230981\\
68.75	0.1176	-38.8771163367912\\
68.75	0.1182	-39.8836497504842\\
68.75	0.1188	-40.8901831641774\\
68.75	0.1194	-41.8967165778704\\
68.75	0.12	-42.9032499915636\\
68.75	0.1206	-43.9097834052566\\
68.75	0.1212	-44.9163168189497\\
68.75	0.1218	-45.9228502326428\\
68.75	0.1224	-46.9293836463359\\
68.75	0.123	-47.9359170600291\\
69.125	0.093	2.78333278775835\\
69.125	0.0936	1.80023424976116\\
69.125	0.0942	0.817135711763854\\
69.125	0.0948	-0.16596282623334\\
69.125	0.0954	-1.14906136423065\\
69.125	0.096	-2.13215990222784\\
69.125	0.0966	-3.11525844022503\\
69.125	0.0972	-4.09835697822234\\
69.125	0.0978	-5.08145551621953\\
69.125	0.0984	-6.06455405421684\\
69.125	0.099	-7.04765259221415\\
69.125	0.0996	-8.03075113021134\\
69.125	0.1002	-9.01384966820854\\
69.125	0.1008	-9.99694820620573\\
69.125	0.1014	-10.980046744203\\
69.125	0.102	-11.9631452822002\\
69.125	0.1026	-12.9462438201975\\
69.125	0.1032	-13.9293423581948\\
69.125	0.1038	-14.912440896192\\
69.125	0.1044	-15.8955394341892\\
69.125	0.105	-16.8786379721864\\
69.125	0.1056	-17.8617365101837\\
69.125	0.1062	-18.844835048181\\
69.125	0.1068	-19.8279335861782\\
69.125	0.1074	-20.8110321241755\\
69.125	0.108	-21.7941306621727\\
69.125	0.1086	-22.7772292001699\\
69.125	0.1092	-23.7603277381671\\
69.125	0.1098	-24.7434262761644\\
69.125	0.1104	-25.7265248141616\\
69.125	0.111	-26.7096233521589\\
69.125	0.1116	-27.6927218901562\\
69.125	0.1122	-28.6758204281534\\
69.125	0.1128	-29.6589189661506\\
69.125	0.1134	-30.6420175041478\\
69.125	0.114	-31.6251160421451\\
69.125	0.1146	-32.6082145801424\\
69.125	0.1152	-33.5913131181396\\
69.125	0.1158	-34.5744116561369\\
69.125	0.1164	-35.5575101941341\\
69.125	0.117	-36.5406087321313\\
69.125	0.1176	-37.5237072701286\\
69.125	0.1182	-38.5068058081257\\
69.125	0.1188	-39.4899043461231\\
69.125	0.1194	-40.4730028841203\\
69.125	0.12	-41.4561014221176\\
69.125	0.1206	-42.4391999601147\\
69.125	0.1212	-43.422298498112\\
69.125	0.1218	-44.4053970361092\\
69.125	0.1224	-45.3884955741065\\
69.125	0.123	-46.3715941121039\\
69.5	0.093	3.1759119508913\\
69.5	0.0936	2.2162482885899\\
69.5	0.0942	1.25658462628849\\
69.5	0.0948	0.296920963987191\\
69.5	0.0954	-0.662742698314332\\
69.5	0.096	-1.62240636061574\\
69.5	0.0966	-2.58207002291704\\
69.5	0.0972	-3.54173368521856\\
69.5	0.0978	-4.50139734751997\\
69.5	0.0984	-5.46106100982126\\
69.5	0.099	-6.42072467212279\\
69.5	0.0996	-7.3803883344242\\
69.5	0.1002	-8.34005199672549\\
69.5	0.1008	-9.2997156590269\\
69.5	0.1014	-10.2593793213284\\
69.5	0.102	-11.2190429836297\\
69.5	0.1026	-12.1787066459311\\
69.5	0.1032	-13.1383703082325\\
69.5	0.1038	-14.0980339705339\\
69.5	0.1044	-15.0576976328354\\
69.5	0.105	-16.0173612951367\\
69.5	0.1056	-16.9770249574382\\
69.5	0.1062	-17.9366886197396\\
69.5	0.1068	-18.8963522820409\\
69.5	0.1074	-19.8560159443424\\
69.5	0.108	-20.8156796066438\\
69.5	0.1086	-21.7753432689451\\
69.5	0.1092	-22.7350069312465\\
69.5	0.1098	-23.694670593548\\
69.5	0.1104	-24.6543342558493\\
69.5	0.111	-25.6139979181507\\
69.5	0.1116	-26.5736615804523\\
69.5	0.1122	-27.5333252427536\\
69.5	0.1128	-28.492988905055\\
69.5	0.1134	-29.4526525673563\\
69.5	0.114	-30.4123162296578\\
69.5	0.1146	-31.3719798919592\\
69.5	0.1152	-32.3316435542605\\
69.5	0.1158	-33.291307216562\\
69.5	0.1164	-34.2509708788634\\
69.5	0.117	-35.2106345411647\\
69.5	0.1176	-36.1702982034662\\
69.5	0.1182	-37.1299618657675\\
69.5	0.1188	-38.089625528069\\
69.5	0.1194	-39.0492891903704\\
69.5	0.12	-40.0089528526719\\
69.5	0.1206	-40.9686165149731\\
69.5	0.1212	-41.9282801772746\\
69.5	0.1218	-42.8879438395759\\
69.5	0.1224	-43.8476075018774\\
69.5	0.123	-44.8072711641789\\
69.875	0.093	3.56849111402437\\
69.875	0.0936	2.63226232741886\\
69.875	0.0942	1.69603354081335\\
69.875	0.0948	0.759804754207835\\
69.875	0.0954	-0.17642403239779\\
69.875	0.096	-1.11265281900342\\
69.875	0.0966	-2.04888160560893\\
69.875	0.0972	-2.98511039221455\\
69.875	0.0978	-3.92133917882006\\
69.875	0.0984	-4.85756796542557\\
69.875	0.099	-5.7937967520312\\
69.875	0.0996	-6.73002553863671\\
69.875	0.1002	-7.66625432524233\\
69.875	0.1008	-8.60248311184785\\
69.875	0.1014	-9.53871189845347\\
69.875	0.102	-10.474940685059\\
69.875	0.1026	-11.4111694716645\\
69.875	0.1032	-12.3473982582701\\
69.875	0.1038	-13.2836270448756\\
69.875	0.1044	-14.2198558314813\\
69.875	0.105	-15.1560846180868\\
69.875	0.1056	-16.0923134046924\\
69.875	0.1062	-17.0285421912979\\
69.875	0.1068	-17.9647709779034\\
69.875	0.1074	-18.900999764509\\
69.875	0.108	-19.8372285511145\\
69.875	0.1086	-20.7734573377202\\
69.875	0.1092	-21.7096861243257\\
69.875	0.1098	-22.6459149109313\\
69.875	0.1104	-23.5821436975368\\
69.875	0.111	-24.5183724841423\\
69.875	0.1116	-25.454601270748\\
69.875	0.1122	-26.3908300573535\\
69.875	0.1128	-27.3270588439591\\
69.875	0.1134	-28.2632876305646\\
69.875	0.114	-29.1995164171702\\
69.875	0.1146	-30.1357452037757\\
69.875	0.1152	-31.0719739903813\\
69.875	0.1158	-32.0082027769869\\
69.875	0.1164	-32.9444315635925\\
69.875	0.117	-33.880660350198\\
69.875	0.1176	-34.8168891368036\\
69.875	0.1182	-35.753117923409\\
69.875	0.1188	-36.6893467100147\\
69.875	0.1194	-37.6255754966202\\
69.875	0.12	-38.5618042832258\\
69.875	0.1206	-39.4980330698313\\
69.875	0.1212	-40.4342618564369\\
69.875	0.1218	-41.3704906430424\\
69.875	0.1224	-42.3067194296481\\
69.875	0.123	-43.2429482162537\\
70.25	0.093	3.96107027715743\\
70.25	0.0936	3.04827636624771\\
70.25	0.0942	2.13548245533809\\
70.25	0.0948	1.22268854442837\\
70.25	0.0954	0.309894633518525\\
70.25	0.096	-0.602899277391089\\
70.25	0.0966	-1.51569318830082\\
70.25	0.0972	-2.42848709921054\\
70.25	0.0978	-3.34128101012027\\
70.25	0.0984	-4.25407492103\\
70.25	0.099	-5.16686883193972\\
70.25	0.0996	-6.07966274284945\\
70.25	0.1002	-6.99245665375918\\
70.25	0.1008	-7.90525056466879\\
70.25	0.1014	-8.81804447557863\\
70.25	0.102	-9.73083838648836\\
70.25	0.1026	-10.643632297398\\
70.25	0.1032	-11.5564262083078\\
70.25	0.1038	-12.4692201192175\\
70.25	0.1044	-13.3820140301272\\
70.25	0.105	-14.2948079410369\\
70.25	0.1056	-15.2076018519467\\
70.25	0.1062	-16.1203957628563\\
70.25	0.1068	-17.0331896737661\\
70.25	0.1074	-17.9459835846759\\
70.25	0.108	-18.8587774955855\\
70.25	0.1086	-19.7715714064952\\
70.25	0.1092	-20.6843653174049\\
70.25	0.1098	-21.5971592283147\\
70.25	0.1104	-22.5099531392244\\
70.25	0.111	-23.422747050134\\
70.25	0.1116	-24.3355409610439\\
70.25	0.1122	-25.2483348719536\\
70.25	0.1128	-26.1611287828632\\
70.25	0.1134	-27.0739226937729\\
70.25	0.114	-27.9867166046828\\
70.25	0.1146	-28.8995105155924\\
70.25	0.1152	-29.8123044265021\\
70.25	0.1158	-30.725098337412\\
70.25	0.1164	-31.6378922483216\\
70.25	0.117	-32.5506861592313\\
70.25	0.1176	-33.4634800701411\\
70.25	0.1182	-34.3762739810506\\
70.25	0.1188	-35.2890678919605\\
70.25	0.1194	-36.2018618028702\\
70.25	0.12	-37.1146557137799\\
70.25	0.1206	-38.0274496246896\\
70.25	0.1212	-38.9402435355994\\
70.25	0.1218	-39.853037446509\\
70.25	0.1224	-40.7658313574188\\
70.25	0.123	-41.6786252683286\\
70.625	0.093	4.35364944029061\\
70.625	0.0936	3.46429040507678\\
70.625	0.0942	2.57493136986284\\
70.625	0.0948	1.68557233464901\\
70.625	0.0954	0.796213299435067\\
70.625	0.096	-0.0931457357787622\\
70.625	0.0966	-0.982504770992591\\
70.625	0.0972	-1.87186380620653\\
70.625	0.0978	-2.76122284142048\\
70.625	0.0984	-3.65058187663431\\
70.625	0.099	-4.53994091184825\\
70.625	0.0996	-5.42929994706208\\
70.625	0.1002	-6.31865898227591\\
70.625	0.1008	-7.20801801748974\\
70.625	0.1014	-8.09737705270368\\
70.625	0.102	-8.98673608791762\\
70.625	0.1026	-9.87609512313145\\
70.625	0.1032	-10.7654541583454\\
70.625	0.1038	-11.6548131935592\\
70.625	0.1044	-12.544172228773\\
70.625	0.105	-13.4335312639869\\
70.625	0.1056	-14.3228902992008\\
70.625	0.1062	-15.2122493344148\\
70.625	0.1068	-16.1016083696286\\
70.625	0.1074	-16.9909674048425\\
70.625	0.108	-17.8803264400564\\
70.625	0.1086	-18.7696854752702\\
70.625	0.1092	-19.659044510484\\
70.625	0.1098	-20.548403545698\\
70.625	0.1104	-21.4377625809119\\
70.625	0.111	-22.3271216161257\\
70.625	0.1116	-23.2164806513397\\
70.625	0.1122	-24.1058396865535\\
70.625	0.1128	-24.9951987217673\\
70.625	0.1134	-25.8845577569812\\
70.625	0.114	-26.7739167921952\\
70.625	0.1146	-27.6632758274091\\
70.625	0.1152	-28.5526348626229\\
70.625	0.1158	-29.4419938978368\\
70.625	0.1164	-30.3313529330507\\
70.625	0.117	-31.2207119682645\\
70.625	0.1176	-32.1100710034784\\
70.625	0.1182	-32.9994300386923\\
70.625	0.1188	-33.8887890739062\\
70.625	0.1194	-34.77814810912\\
70.625	0.12	-35.667507144334\\
70.625	0.1206	-36.5568661795477\\
70.625	0.1212	-37.4462252147616\\
70.625	0.1218	-38.3355842499755\\
70.625	0.1224	-39.2249432851895\\
70.625	0.123	-40.1143023204035\\
71	0.093	4.74622860342367\\
71	0.0936	3.88030444390574\\
71	0.0942	3.0143802843877\\
71	0.0948	2.14845612486965\\
71	0.0954	1.28253196535161\\
71	0.096	0.416607805833564\\
71	0.0966	-0.44931635368448\\
71	0.0972	-1.31524051320253\\
71	0.0978	-2.18116467272057\\
71	0.0984	-3.04708883223861\\
71	0.099	-3.91301299175666\\
71	0.0996	-4.7789371512747\\
71	0.1002	-5.64486131079263\\
71	0.1008	-6.51078547031068\\
71	0.1014	-7.37670962982884\\
71	0.102	-8.24263378934677\\
71	0.1026	-9.10855794886481\\
71	0.1032	-9.97448210838297\\
71	0.1038	-10.8404062679009\\
71	0.1044	-11.7063304274189\\
71	0.105	-12.572254586937\\
71	0.1056	-13.438178746455\\
71	0.1062	-14.3041029059731\\
71	0.1068	-15.1700270654911\\
71	0.1074	-16.0359512250092\\
71	0.108	-16.9018753845272\\
71	0.1086	-17.7677995440453\\
71	0.1092	-18.6337237035632\\
71	0.1098	-19.4996478630813\\
71	0.1104	-20.3655720225994\\
71	0.111	-21.2314961821173\\
71	0.1116	-22.0974203416355\\
71	0.1122	-22.9633445011534\\
71	0.1128	-23.8292686606715\\
71	0.1134	-24.6951928201895\\
71	0.114	-25.5611169797075\\
71	0.1146	-26.4270411392256\\
71	0.1152	-27.2929652987436\\
71	0.1158	-28.1588894582617\\
71	0.1164	-29.0248136177797\\
71	0.117	-29.8907377772978\\
71	0.1176	-30.7566619368158\\
71	0.1182	-31.6225860963337\\
71	0.1188	-32.4885102558519\\
71	0.1194	-33.3544344153698\\
71	0.12	-34.220358574888\\
71	0.1206	-35.0862827344059\\
71	0.1212	-35.952206893924\\
71	0.1218	-36.818131053442\\
71	0.1224	-37.6840552129601\\
71	0.123	-38.5499793724782\\
71.375	0.093	5.13880776655662\\
71.375	0.0936	4.29631848273448\\
71.375	0.0942	3.45382919891233\\
71.375	0.0948	2.61133991509018\\
71.375	0.0954	1.76885063126792\\
71.375	0.096	0.926361347445663\\
71.375	0.0966	0.0838720636235166\\
71.375	0.0972	-0.758617220198744\\
71.375	0.0978	-1.60110650402089\\
71.375	0.0984	-2.44359578784304\\
71.375	0.099	-3.2860850716653\\
71.375	0.0996	-4.12857435548744\\
71.375	0.1002	-4.9710636393097\\
71.375	0.1008	-5.81355292313185\\
71.375	0.1014	-6.65604220695411\\
71.375	0.102	-7.49853149077626\\
71.375	0.1026	-8.34102077459841\\
71.375	0.1032	-9.18351005842067\\
71.375	0.1038	-10.0259993422429\\
71.375	0.1044	-10.8684886260651\\
71.375	0.105	-11.7109779098872\\
71.375	0.1056	-12.5534671937095\\
71.375	0.1062	-13.3959564775316\\
71.375	0.1068	-14.2384457613538\\
71.375	0.1074	-15.080935045176\\
71.375	0.108	-15.9234243289983\\
71.375	0.1086	-16.7659136128204\\
71.375	0.1092	-17.6084028966426\\
71.375	0.1098	-18.4508921804648\\
71.375	0.1104	-19.293381464287\\
71.375	0.111	-20.1358707481091\\
71.375	0.1116	-20.9783600319315\\
71.375	0.1122	-21.8208493157537\\
71.375	0.1128	-22.6633385995758\\
71.375	0.1134	-23.505827883398\\
71.375	0.114	-24.3483171672202\\
71.375	0.1146	-25.1908064510424\\
71.375	0.1152	-26.0332957348645\\
71.375	0.1158	-26.8757850186869\\
71.375	0.1164	-27.718274302509\\
71.375	0.117	-28.5607635863312\\
71.375	0.1176	-29.4032528701534\\
71.375	0.1182	-30.2457421539755\\
71.375	0.1188	-31.0882314377977\\
71.375	0.1194	-31.93072072162\\
71.375	0.12	-32.7732100054423\\
71.375	0.1206	-33.6156992892643\\
71.375	0.1212	-34.4581885730865\\
71.375	0.1218	-35.3006778569087\\
71.375	0.1224	-36.143167140731\\
71.375	0.123	-36.9856564245532\\
71.75	0.093	5.5313869296898\\
71.75	0.0936	4.71233252156344\\
71.75	0.0942	3.89327811343708\\
71.75	0.0948	3.07422370531083\\
71.75	0.0954	2.25516929718435\\
71.75	0.096	1.4361148890581\\
71.75	0.0966	0.617060480931741\\
71.75	0.0972	-0.201993927194735\\
71.75	0.0978	-1.02104833532098\\
71.75	0.0984	-1.84010274344735\\
71.75	0.099	-2.65915715157382\\
71.75	0.0996	-3.47821155970007\\
71.75	0.1002	-4.29726596782643\\
71.75	0.1008	-5.1163203759528\\
71.75	0.1014	-5.93537478407916\\
71.75	0.102	-6.75442919220552\\
71.75	0.1026	-7.57348360033188\\
71.75	0.1032	-8.39253800845825\\
71.75	0.1038	-9.21159241658461\\
71.75	0.1044	-10.030646824711\\
71.75	0.105	-10.8497012328372\\
71.75	0.1056	-11.6687556409637\\
71.75	0.1062	-12.4878100490901\\
71.75	0.1068	-13.3068644572163\\
71.75	0.1074	-14.1259188653428\\
71.75	0.108	-14.944973273469\\
71.75	0.1086	-15.7640276815954\\
71.75	0.1092	-16.5830820897218\\
71.75	0.1098	-17.4021364978481\\
71.75	0.1104	-18.2211909059745\\
71.75	0.111	-19.0402453141008\\
71.75	0.1116	-19.8592997222272\\
71.75	0.1122	-20.6783541303536\\
71.75	0.1128	-21.4974085384799\\
71.75	0.1134	-22.3164629466062\\
71.75	0.114	-23.1355173547327\\
71.75	0.1146	-23.954571762859\\
71.75	0.1152	-24.7736261709853\\
71.75	0.1158	-25.5926805791117\\
71.75	0.1164	-26.4117349872381\\
71.75	0.117	-27.2307893953644\\
71.75	0.1176	-28.0498438034908\\
71.75	0.1182	-28.8688982116171\\
71.75	0.1188	-29.6879526197434\\
71.75	0.1194	-30.5070070278698\\
71.75	0.12	-31.3260614359963\\
71.75	0.1206	-32.1451158441224\\
71.75	0.1212	-32.9641702522489\\
71.75	0.1218	-33.7832246603751\\
71.75	0.1224	-34.6022790685016\\
71.75	0.123	-35.4213334766281\\
72.125	0.093	5.92396609282287\\
72.125	0.0936	5.1283465603924\\
72.125	0.0942	4.33272702796182\\
72.125	0.0948	3.53710749553136\\
72.125	0.0954	2.74148796310078\\
72.125	0.096	1.94586843067032\\
72.125	0.0966	1.15024889823985\\
72.125	0.0972	0.354629365809274\\
72.125	0.0978	-0.440990166621305\\
72.125	0.0984	-1.23660969905177\\
72.125	0.099	-2.03222923148235\\
72.125	0.0996	-2.82784876391281\\
72.125	0.1002	-3.62346829634328\\
72.125	0.1008	-4.41908782877374\\
72.125	0.1014	-5.21470736120443\\
72.125	0.102	-6.0103268936349\\
72.125	0.1026	-6.80594642606536\\
72.125	0.1032	-7.60156595849594\\
72.125	0.1038	-8.3971854909264\\
72.125	0.1044	-9.19280502335687\\
72.125	0.105	-9.98842455578733\\
72.125	0.1056	-10.784044088218\\
72.125	0.1062	-11.5796636206485\\
72.125	0.1068	-12.375283153079\\
72.125	0.1074	-13.1709026855095\\
72.125	0.108	-13.96652221794\\
72.125	0.1086	-14.7621417503705\\
72.125	0.1092	-15.5577612828009\\
72.125	0.1098	-16.3533808152316\\
72.125	0.1104	-17.1490003476621\\
72.125	0.111	-17.9446198800925\\
72.125	0.1116	-18.7402394125231\\
72.125	0.1122	-19.5358589449536\\
72.125	0.1128	-20.3314784773841\\
72.125	0.1134	-21.1270980098146\\
72.125	0.114	-21.9227175422452\\
72.125	0.1146	-22.7183370746757\\
72.125	0.1152	-23.5139566071061\\
72.125	0.1158	-24.3095761395367\\
72.125	0.1164	-25.1051956719672\\
72.125	0.117	-25.9008152043978\\
72.125	0.1176	-26.6964347368283\\
72.125	0.1182	-27.4920542692587\\
72.125	0.1188	-28.2876738016893\\
72.125	0.1194	-29.0832933341197\\
72.125	0.12	-29.8789128665503\\
72.125	0.1206	-30.6745323989808\\
72.125	0.1212	-31.4701519314114\\
72.125	0.1218	-32.2657714638418\\
72.125	0.1224	-33.0613909962724\\
72.125	0.123	-33.857010528703\\
72.5	0.093	6.31654525595593\\
72.5	0.0936	5.54436059922136\\
72.5	0.0942	4.77217594248668\\
72.5	0.0948	3.999991285752\\
72.5	0.0954	3.22780662901732\\
72.5	0.096	2.45562197228264\\
72.5	0.0966	1.68343731554796\\
72.5	0.0972	0.911252658813282\\
72.5	0.0978	0.139068002078602\\
72.5	0.0984	-0.633116654656078\\
72.5	0.099	-1.40530131139076\\
72.5	0.0996	-2.17748596812544\\
72.5	0.1002	-2.94967062486012\\
72.5	0.1008	-3.72185528159469\\
72.5	0.1014	-4.49403993832948\\
72.5	0.102	-5.26622459506405\\
72.5	0.1026	-6.03840925179873\\
72.5	0.1032	-6.81059390853352\\
72.5	0.1038	-7.5827785652682\\
72.5	0.1044	-8.35496322200277\\
72.5	0.105	-9.12714787873745\\
72.5	0.1056	-9.89933253547213\\
72.5	0.1062	-10.6715171922068\\
72.5	0.1068	-11.4437018489415\\
72.5	0.1074	-12.2158865056762\\
72.5	0.108	-12.9880711624108\\
72.5	0.1086	-13.7602558191455\\
72.5	0.1092	-14.5324404758801\\
72.5	0.1098	-15.3046251326149\\
72.5	0.1104	-16.0768097893496\\
72.5	0.111	-16.8489944460841\\
72.5	0.1116	-17.6211791028189\\
72.5	0.1122	-18.3933637595536\\
72.5	0.1128	-19.1655484162882\\
72.5	0.1134	-19.9377330730229\\
72.5	0.114	-20.7099177297576\\
72.5	0.1146	-21.4821023864922\\
72.5	0.1152	-22.2542870432269\\
72.5	0.1158	-23.0264716999617\\
72.5	0.1164	-23.7986563566963\\
72.5	0.117	-24.5708410134309\\
72.5	0.1176	-25.3430256701657\\
72.5	0.1182	-26.1152103269002\\
72.5	0.1188	-26.887394983635\\
72.5	0.1194	-27.6595796403697\\
72.5	0.12	-28.4317642971043\\
72.5	0.1206	-29.2039489538389\\
72.5	0.1212	-29.9761336105737\\
72.5	0.1218	-30.7483182673083\\
72.5	0.1224	-31.5205029240431\\
72.5	0.123	-32.2926875807777\\
72.875	0.093	6.70912441908899\\
72.875	0.0936	5.96037463805021\\
72.875	0.0942	5.21162485701143\\
72.875	0.0948	4.46287507597265\\
72.875	0.0954	3.71412529493364\\
72.875	0.096	2.96537551389486\\
72.875	0.0966	2.21662573285607\\
72.875	0.0972	1.46787595181718\\
72.875	0.0978	0.719126170778395\\
72.875	0.0984	-0.0296236102603871\\
72.875	0.099	-0.778373391299397\\
72.875	0.0996	-1.52712317233818\\
72.875	0.1002	-2.27587295337696\\
72.875	0.1008	-3.02462273441574\\
72.875	0.1014	-3.77337251545464\\
72.875	0.102	-4.52212229649342\\
72.875	0.1026	-5.27087207753232\\
72.875	0.1032	-6.01962185857121\\
72.875	0.1038	-6.76837163961\\
72.875	0.1044	-7.51712142064878\\
72.875	0.105	-8.26587120168756\\
72.875	0.1056	-9.01462098272646\\
72.875	0.1062	-9.76337076376524\\
72.875	0.1068	-10.5121205448041\\
72.875	0.1074	-11.260870325843\\
72.875	0.108	-12.0096201068818\\
72.875	0.1086	-12.7583698879206\\
72.875	0.1092	-13.5071196689594\\
72.875	0.1098	-14.2558694499983\\
72.875	0.1104	-15.0046192310372\\
72.875	0.111	-15.753369012076\\
72.875	0.1116	-16.5021187931148\\
72.875	0.1122	-17.2508685741536\\
72.875	0.1128	-17.9996183551924\\
72.875	0.1134	-18.7483681362312\\
72.875	0.114	-19.4971179172702\\
72.875	0.1146	-20.245867698309\\
72.875	0.1152	-20.9946174793478\\
72.875	0.1158	-21.7433672603867\\
72.875	0.1164	-22.4921170414254\\
72.875	0.117	-23.2408668224642\\
72.875	0.1176	-23.9896166035031\\
72.875	0.1182	-24.7383663845419\\
72.875	0.1188	-25.4871161655808\\
72.875	0.1194	-26.2358659466196\\
72.875	0.12	-26.9846157276585\\
72.875	0.1206	-27.7333655086971\\
72.875	0.1212	-28.482115289736\\
72.875	0.1218	-29.2308650707748\\
72.875	0.1224	-29.9796148518138\\
72.875	0.123	-30.7283646328527\\
73.25	0.093	7.10170358222206\\
73.25	0.0936	6.37638867687906\\
73.25	0.0942	5.65107377153606\\
73.25	0.0948	4.92575886619318\\
73.25	0.0954	4.20044396085007\\
73.25	0.096	3.47512905550718\\
73.25	0.0966	2.74981415016418\\
73.25	0.0972	2.02449924482107\\
73.25	0.0978	1.29918433947819\\
73.25	0.0984	0.57386943413519\\
73.25	0.099	-0.151445471207921\\
73.25	0.0996	-0.876760376550806\\
73.25	0.1002	-1.6020752818938\\
73.25	0.1008	-2.3273901872368\\
73.25	0.1014	-3.0527050925798\\
73.25	0.102	-3.7780199979228\\
73.25	0.1026	-4.5033349032658\\
73.25	0.1032	-5.22864980860879\\
73.25	0.1038	-5.95396471395179\\
73.25	0.1044	-6.67927961929479\\
73.25	0.105	-7.40459452463767\\
73.25	0.1056	-8.12990942998078\\
73.25	0.1062	-8.85522433532378\\
73.25	0.1068	-9.58053924066667\\
73.25	0.1074	-10.3058541460098\\
73.25	0.108	-11.0311690513528\\
73.25	0.1086	-11.7564839566957\\
73.25	0.1092	-12.4817988620387\\
73.25	0.1098	-13.2071137673818\\
73.25	0.1104	-13.9324286727247\\
73.25	0.111	-14.6577435780677\\
73.25	0.1116	-15.3830584834108\\
73.25	0.1122	-16.1083733887536\\
73.25	0.1128	-16.8336882940966\\
73.25	0.1134	-17.5590031994396\\
73.25	0.114	-18.2843181047826\\
73.25	0.1146	-19.0096330101256\\
73.25	0.1152	-19.7349479154686\\
73.25	0.1158	-20.4602628208116\\
73.25	0.1164	-21.1855777261546\\
73.25	0.117	-21.9108926314976\\
73.25	0.1176	-22.6362075368406\\
73.25	0.1182	-23.3615224421835\\
73.25	0.1188	-24.0868373475266\\
73.25	0.1194	-24.8121522528695\\
73.25	0.12	-25.5374671582126\\
73.25	0.1206	-26.2627820635554\\
73.25	0.1212	-26.9880969688985\\
73.25	0.1218	-27.7134118742415\\
73.25	0.1224	-28.4387267795845\\
73.25	0.123	-29.1640416849276\\
73.625	0.093	7.49428274535512\\
73.625	0.0936	6.79240271570802\\
73.625	0.0942	6.09052268606092\\
73.625	0.0948	5.38864265641382\\
73.625	0.0954	4.68676262676661\\
73.625	0.096	3.98488259711951\\
73.625	0.0966	3.28300256747229\\
73.625	0.0972	2.58112253782508\\
73.625	0.0978	1.87924250817798\\
73.625	0.0984	1.17736247853088\\
73.625	0.099	0.475482448883668\\
73.625	0.0996	-0.226397580763432\\
73.625	0.1002	-0.928277610410532\\
73.625	0.1008	-1.63015764005775\\
73.625	0.1014	-2.33203766970496\\
73.625	0.102	-3.03391769935206\\
73.625	0.1026	-3.73579772899916\\
73.625	0.1032	-4.43767775864637\\
73.625	0.1038	-5.13955778829347\\
73.625	0.1044	-5.84143781794069\\
73.625	0.105	-6.54331784758779\\
73.625	0.1056	-7.245197877235\\
73.625	0.1062	-7.9470779068821\\
73.625	0.1068	-8.6489579365292\\
73.625	0.1074	-9.35083796617641\\
73.625	0.108	-10.0527179958236\\
73.625	0.1086	-10.7545980254707\\
73.625	0.1092	-11.4564780551178\\
73.625	0.1098	-12.158358084765\\
73.625	0.1104	-12.8602381144121\\
73.625	0.111	-13.5621181440592\\
73.625	0.1116	-14.2639981737065\\
73.625	0.1122	-14.9658782033537\\
73.625	0.1128	-15.6677582330008\\
73.625	0.1134	-16.3696382626479\\
73.625	0.114	-17.0715182922951\\
73.625	0.1146	-17.7733983219422\\
73.625	0.1152	-18.4752783515893\\
73.625	0.1158	-19.1771583812366\\
73.625	0.1164	-19.8790384108837\\
73.625	0.117	-20.5809184405308\\
73.625	0.1176	-21.282798470178\\
73.625	0.1182	-21.984678499825\\
73.625	0.1188	-22.6865585294722\\
73.625	0.1194	-23.3884385591193\\
73.625	0.12	-24.0903185887666\\
73.625	0.1206	-24.7921986184136\\
73.625	0.1212	-25.4940786480608\\
73.625	0.1218	-26.1959586777079\\
73.625	0.1224	-26.8978387073552\\
73.625	0.123	-27.5997187370024\\
74	0.093	7.88686190848819\\
74	0.0936	7.20841675453698\\
74	0.0942	6.52997160058567\\
74	0.0948	5.85152644663435\\
74	0.0954	5.17308129268304\\
74	0.096	4.49463613873172\\
74	0.0966	3.81619098478041\\
74	0.0972	3.13774583082909\\
74	0.0978	2.45930067687777\\
74	0.0984	1.78085552292646\\
74	0.099	1.10241036897514\\
74	0.0996	0.423965215023827\\
74	0.1002	-0.254479938927489\\
74	0.1008	-0.932925092878691\\
74	0.1014	-1.61137024683012\\
74	0.102	-2.28981540078144\\
74	0.1026	-2.96826055473264\\
74	0.1032	-3.64670570868407\\
74	0.1038	-4.32515086263538\\
74	0.1044	-5.00359601658658\\
74	0.105	-5.6820411705379\\
74	0.1056	-6.36048632448933\\
74	0.1062	-7.03893147844053\\
74	0.1068	-7.71737663239185\\
74	0.1074	-8.39582178634328\\
74	0.108	-9.07426694029448\\
74	0.1086	-9.75271209424579\\
74	0.1092	-10.4311572481971\\
74	0.1098	-11.1096024021484\\
74	0.1104	-11.7880475560997\\
74	0.111	-12.4664927100511\\
74	0.1116	-13.1449378640024\\
74	0.1122	-13.8233830179537\\
74	0.1128	-14.501828171905\\
74	0.1134	-15.1802733258562\\
74	0.114	-15.8587184798076\\
74	0.1146	-16.537163633759\\
74	0.1152	-17.2156087877102\\
74	0.1158	-17.8940539416616\\
74	0.1164	-18.5724990956129\\
74	0.117	-19.2509442495641\\
74	0.1176	-19.9293894035155\\
74	0.1182	-20.6078345574666\\
74	0.1188	-21.286279711418\\
74	0.1194	-21.9647248653694\\
74	0.12	-22.6431700193207\\
74	0.1206	-23.3216151732719\\
74	0.1212	-24.0000603272233\\
74	0.1218	-24.6785054811745\\
74	0.1224	-25.3569506351259\\
74	0.123	-26.0353957890774\\
};
\end{axis}

\begin{axis}[%
width=5.011742cm,
height=3.595207cm,
at={(6.594397cm,16.404793cm)},
scale only axis,
xmin=55,
xmax=75,
tick align=outside,
xlabel={$L_{cut}$},
xmajorgrids,
ymin=0.09,
ymax=0.13,
ylabel={$D_{rlx}$},
ymajorgrids,
zmin=-111.704902386548,
zmax=75.4245651997298,
zlabel={$x_3$},
zmajorgrids,
view={-140}{50},
legend style={at={(1.03,1)},anchor=north west,legend cell align=left,align=left,draw=white!15!black}
]
\addplot3[only marks,mark=*,mark options={},mark size=1.5000pt,color=mycolor1] plot table[row sep=crcr,]{%
74	0.123	75.4245651900459\\
72	0.113	59.5782863482935\\
61	0.095	33.0003406266977\\
56	0.093	26.0569470415911\\
};
\addplot3[only marks,mark=*,mark options={},mark size=1.5000pt,color=black] plot table[row sep=crcr,]{%
69	0.104	47.4385858819007\\
};

\addplot3[%
surf,
opacity=0.7,
shader=interp,
colormap={mymap}{[1pt] rgb(0pt)=(0.0901961,0.239216,0.0745098); rgb(1pt)=(0.0945149,0.242058,0.0739522); rgb(2pt)=(0.0988592,0.244894,0.0733566); rgb(3pt)=(0.103229,0.247724,0.0727241); rgb(4pt)=(0.107623,0.250549,0.0720557); rgb(5pt)=(0.112043,0.253367,0.0713525); rgb(6pt)=(0.116487,0.25618,0.0706154); rgb(7pt)=(0.120956,0.258986,0.0698456); rgb(8pt)=(0.125449,0.261787,0.0690441); rgb(9pt)=(0.129967,0.264581,0.0682118); rgb(10pt)=(0.134508,0.26737,0.06735); rgb(11pt)=(0.139074,0.270152,0.0664596); rgb(12pt)=(0.143663,0.272929,0.0655416); rgb(13pt)=(0.148275,0.275699,0.0645971); rgb(14pt)=(0.152911,0.278463,0.0636271); rgb(15pt)=(0.15757,0.281221,0.0626328); rgb(16pt)=(0.162252,0.283973,0.0616151); rgb(17pt)=(0.166957,0.286719,0.060575); rgb(18pt)=(0.171685,0.289458,0.0595136); rgb(19pt)=(0.176434,0.292191,0.0584321); rgb(20pt)=(0.181207,0.294918,0.0573313); rgb(21pt)=(0.186001,0.297639,0.0562123); rgb(22pt)=(0.190817,0.300353,0.0550763); rgb(23pt)=(0.195655,0.303061,0.0539242); rgb(24pt)=(0.200514,0.305763,0.052757); rgb(25pt)=(0.205395,0.308459,0.0515759); rgb(26pt)=(0.210296,0.311149,0.0503624); rgb(27pt)=(0.215212,0.313846,0.0490067); rgb(28pt)=(0.220142,0.316548,0.0475043); rgb(29pt)=(0.22509,0.319254,0.0458704); rgb(30pt)=(0.230056,0.321962,0.0441205); rgb(31pt)=(0.235042,0.324671,0.04227); rgb(32pt)=(0.240048,0.327379,0.0403343); rgb(33pt)=(0.245078,0.330085,0.0383287); rgb(34pt)=(0.250131,0.332786,0.0362688); rgb(35pt)=(0.25521,0.335482,0.0341698); rgb(36pt)=(0.260317,0.33817,0.0320472); rgb(37pt)=(0.265451,0.340849,0.0299163); rgb(38pt)=(0.270616,0.343517,0.0277927); rgb(39pt)=(0.275813,0.346172,0.0256916); rgb(40pt)=(0.281043,0.348814,0.0236284); rgb(41pt)=(0.286307,0.35144,0.0216186); rgb(42pt)=(0.291607,0.354048,0.0196776); rgb(43pt)=(0.296945,0.356637,0.0178207); rgb(44pt)=(0.302322,0.359206,0.0160634); rgb(45pt)=(0.307739,0.361753,0.0144211); rgb(46pt)=(0.313198,0.364275,0.0129091); rgb(47pt)=(0.318701,0.366772,0.0115428); rgb(48pt)=(0.324249,0.369242,0.0103377); rgb(49pt)=(0.329843,0.371682,0.00930909); rgb(50pt)=(0.335485,0.374093,0.00847245); rgb(51pt)=(0.341176,0.376471,0.00784314); rgb(52pt)=(0.346925,0.378826,0.00732741); rgb(53pt)=(0.352735,0.381168,0.00682184); rgb(54pt)=(0.358605,0.383497,0.00632729); rgb(55pt)=(0.364532,0.385812,0.00584464); rgb(56pt)=(0.370516,0.388113,0.00537476); rgb(57pt)=(0.376552,0.390399,0.00491852); rgb(58pt)=(0.38264,0.39267,0.00447681); rgb(59pt)=(0.388777,0.394925,0.00405048); rgb(60pt)=(0.394962,0.397164,0.00364042); rgb(61pt)=(0.401191,0.399386,0.00324749); rgb(62pt)=(0.407464,0.401592,0.00287258); rgb(63pt)=(0.413777,0.40378,0.00251655); rgb(64pt)=(0.420129,0.40595,0.00218028); rgb(65pt)=(0.426518,0.408102,0.00186463); rgb(66pt)=(0.432942,0.410234,0.00157049); rgb(67pt)=(0.439399,0.412348,0.00129873); rgb(68pt)=(0.445885,0.414441,0.00105022); rgb(69pt)=(0.452401,0.416515,0.000825833); rgb(70pt)=(0.458942,0.418567,0.000626441); rgb(71pt)=(0.465508,0.420599,0.00045292); rgb(72pt)=(0.472096,0.422609,0.000306141); rgb(73pt)=(0.478704,0.424596,0.000186979); rgb(74pt)=(0.485331,0.426562,9.63073e-05); rgb(75pt)=(0.491973,0.428504,3.49981e-05); rgb(76pt)=(0.498628,0.430422,3.92506e-06); rgb(77pt)=(0.505323,0.432315,0); rgb(78pt)=(0.512206,0.434168,0); rgb(79pt)=(0.519282,0.435983,0); rgb(80pt)=(0.526529,0.437764,0); rgb(81pt)=(0.533922,0.439512,0); rgb(82pt)=(0.54144,0.441232,0); rgb(83pt)=(0.549059,0.442927,0); rgb(84pt)=(0.556756,0.444599,0); rgb(85pt)=(0.564508,0.446252,0); rgb(86pt)=(0.572292,0.447889,0); rgb(87pt)=(0.580084,0.449514,0); rgb(88pt)=(0.587863,0.451129,0); rgb(89pt)=(0.595604,0.452737,0); rgb(90pt)=(0.603284,0.454343,0); rgb(91pt)=(0.610882,0.455948,0); rgb(92pt)=(0.618373,0.457556,0); rgb(93pt)=(0.625734,0.459171,0); rgb(94pt)=(0.632943,0.460795,0); rgb(95pt)=(0.639976,0.462432,0); rgb(96pt)=(0.64681,0.464084,0); rgb(97pt)=(0.653423,0.465756,0); rgb(98pt)=(0.659791,0.46745,0); rgb(99pt)=(0.665891,0.469169,0); rgb(100pt)=(0.6717,0.470916,0); rgb(101pt)=(0.677195,0.472696,0); rgb(102pt)=(0.682353,0.47451,0); rgb(103pt)=(0.687242,0.476355,0); rgb(104pt)=(0.691952,0.478225,0); rgb(105pt)=(0.696497,0.480118,0); rgb(106pt)=(0.700887,0.482033,0); rgb(107pt)=(0.705134,0.483968,0); rgb(108pt)=(0.709251,0.485921,0); rgb(109pt)=(0.713249,0.487891,0); rgb(110pt)=(0.71714,0.489876,0); rgb(111pt)=(0.720936,0.491875,0); rgb(112pt)=(0.724649,0.493887,0); rgb(113pt)=(0.72829,0.495909,0); rgb(114pt)=(0.731872,0.49794,0); rgb(115pt)=(0.735406,0.499979,0); rgb(116pt)=(0.738904,0.502025,0); rgb(117pt)=(0.742378,0.504075,0); rgb(118pt)=(0.74584,0.506128,0); rgb(119pt)=(0.749302,0.508182,0); rgb(120pt)=(0.752775,0.510237,0); rgb(121pt)=(0.756272,0.51229,0); rgb(122pt)=(0.759804,0.514339,0); rgb(123pt)=(0.763384,0.516385,0); rgb(124pt)=(0.767022,0.518424,0); rgb(125pt)=(0.770731,0.520455,0); rgb(126pt)=(0.774523,0.522478,0); rgb(127pt)=(0.77841,0.524489,0); rgb(128pt)=(0.782391,0.526491,0); rgb(129pt)=(0.786402,0.528496,0); rgb(130pt)=(0.790431,0.530506,0); rgb(131pt)=(0.794478,0.532521,0); rgb(132pt)=(0.798541,0.534539,0); rgb(133pt)=(0.802619,0.53656,0); rgb(134pt)=(0.806712,0.538584,0); rgb(135pt)=(0.81082,0.540609,0); rgb(136pt)=(0.81494,0.542635,0); rgb(137pt)=(0.819074,0.54466,0); rgb(138pt)=(0.823219,0.546686,0); rgb(139pt)=(0.827374,0.548709,0); rgb(140pt)=(0.831541,0.55073,0); rgb(141pt)=(0.835716,0.552749,0); rgb(142pt)=(0.8399,0.554763,0); rgb(143pt)=(0.844092,0.556774,0); rgb(144pt)=(0.848292,0.558779,0); rgb(145pt)=(0.852497,0.560778,0); rgb(146pt)=(0.856708,0.562771,0); rgb(147pt)=(0.860924,0.564756,0); rgb(148pt)=(0.865143,0.566733,0); rgb(149pt)=(0.869366,0.568701,0); rgb(150pt)=(0.873592,0.57066,0); rgb(151pt)=(0.877819,0.572608,0); rgb(152pt)=(0.882047,0.574545,0); rgb(153pt)=(0.886275,0.576471,0); rgb(154pt)=(0.890659,0.578362,0); rgb(155pt)=(0.895333,0.580203,0); rgb(156pt)=(0.900258,0.581999,0); rgb(157pt)=(0.905397,0.583755,0); rgb(158pt)=(0.910711,0.585479,0); rgb(159pt)=(0.916164,0.587176,0); rgb(160pt)=(0.921717,0.588852,0); rgb(161pt)=(0.927333,0.590513,0); rgb(162pt)=(0.932974,0.592166,0); rgb(163pt)=(0.938602,0.593815,0); rgb(164pt)=(0.94418,0.595468,0); rgb(165pt)=(0.949669,0.59713,0); rgb(166pt)=(0.955033,0.598808,0); rgb(167pt)=(0.960233,0.600507,0); rgb(168pt)=(0.965232,0.602233,0); rgb(169pt)=(0.969992,0.603992,0); rgb(170pt)=(0.974475,0.605791,0); rgb(171pt)=(0.978643,0.607636,0); rgb(172pt)=(0.98246,0.609532,0); rgb(173pt)=(0.985886,0.611486,0); rgb(174pt)=(0.988885,0.613503,0); rgb(175pt)=(0.991419,0.61559,0); rgb(176pt)=(0.99345,0.617753,0); rgb(177pt)=(0.99494,0.619997,0); rgb(178pt)=(0.995851,0.622329,0); rgb(179pt)=(0.996226,0.624763,0); rgb(180pt)=(0.996512,0.627352,0); rgb(181pt)=(0.996788,0.630095,0); rgb(182pt)=(0.997053,0.632982,0); rgb(183pt)=(0.997308,0.636004,0); rgb(184pt)=(0.997552,0.639152,0); rgb(185pt)=(0.997785,0.642416,0); rgb(186pt)=(0.998006,0.645786,0); rgb(187pt)=(0.998217,0.649253,0); rgb(188pt)=(0.998416,0.652807,0); rgb(189pt)=(0.998605,0.656439,0); rgb(190pt)=(0.998781,0.660138,0); rgb(191pt)=(0.998946,0.663897,0); rgb(192pt)=(0.9991,0.667704,0); rgb(193pt)=(0.999242,0.67155,0); rgb(194pt)=(0.999372,0.675427,0); rgb(195pt)=(0.99949,0.679323,0); rgb(196pt)=(0.999596,0.68323,0); rgb(197pt)=(0.99969,0.687139,0); rgb(198pt)=(0.999771,0.691039,0); rgb(199pt)=(0.999841,0.694921,0); rgb(200pt)=(0.999898,0.698775,0); rgb(201pt)=(0.999942,0.702592,0); rgb(202pt)=(0.999974,0.706363,0); rgb(203pt)=(0.999994,0.710077,0); rgb(204pt)=(1,0.713725,0); rgb(205pt)=(1,0.717341,0); rgb(206pt)=(1,0.720963,0); rgb(207pt)=(1,0.724591,0); rgb(208pt)=(1,0.728226,0); rgb(209pt)=(1,0.731867,0); rgb(210pt)=(1,0.735514,0); rgb(211pt)=(1,0.739167,0); rgb(212pt)=(1,0.742827,0); rgb(213pt)=(1,0.746493,0); rgb(214pt)=(1,0.750165,0); rgb(215pt)=(1,0.753843,0); rgb(216pt)=(1,0.757527,0); rgb(217pt)=(1,0.761217,0); rgb(218pt)=(1,0.764913,0); rgb(219pt)=(1,0.768615,0); rgb(220pt)=(1,0.772324,0); rgb(221pt)=(1,0.776038,0); rgb(222pt)=(1,0.779758,0); rgb(223pt)=(1,0.783484,0); rgb(224pt)=(1,0.787215,0); rgb(225pt)=(1,0.790953,0); rgb(226pt)=(1,0.794696,0); rgb(227pt)=(1,0.798445,0); rgb(228pt)=(1,0.8022,0); rgb(229pt)=(1,0.805961,0); rgb(230pt)=(1,0.809727,0); rgb(231pt)=(1,0.8135,0); rgb(232pt)=(1,0.817278,0); rgb(233pt)=(1,0.821063,0); rgb(234pt)=(1,0.824854,0); rgb(235pt)=(1,0.828652,0); rgb(236pt)=(1,0.832455,0); rgb(237pt)=(1,0.836265,0); rgb(238pt)=(1,0.840081,0); rgb(239pt)=(1,0.843903,0); rgb(240pt)=(1,0.847732,0); rgb(241pt)=(1,0.851566,0); rgb(242pt)=(1,0.855406,0); rgb(243pt)=(1,0.859253,0); rgb(244pt)=(1,0.863106,0); rgb(245pt)=(1,0.866964,0); rgb(246pt)=(1,0.870829,0); rgb(247pt)=(1,0.8747,0); rgb(248pt)=(1,0.878577,0); rgb(249pt)=(1,0.88246,0); rgb(250pt)=(1,0.886349,0); rgb(251pt)=(1,0.890243,0); rgb(252pt)=(1,0.894144,0); rgb(253pt)=(1,0.898051,0); rgb(254pt)=(1,0.901964,0); rgb(255pt)=(1,0.905882,0)},
mesh/rows=49]
table[row sep=crcr,header=false] {%
%
56	0.093	26.0569470235641\\
56	0.0936	23.301710035362\\
56	0.0942	20.5464730471597\\
56	0.0948	17.7912360589576\\
56	0.0954	15.0359990707552\\
56	0.096	12.2807620825531\\
56	0.0966	9.52552509435077\\
56	0.0972	6.77028810614866\\
56	0.0978	4.01505111794631\\
56	0.0984	1.25981412974397\\
56	0.099	-1.49542285845837\\
56	0.0996	-4.25065984666048\\
56	0.1002	-7.00589683486282\\
56	0.1008	-9.76113382306494\\
56	0.1014	-12.5163708112673\\
56	0.102	-15.2716077994694\\
56	0.1026	-18.0268447876717\\
56	0.1032	-20.7820817758738\\
56	0.1038	-23.5373187640762\\
56	0.1044	-26.2925557522783\\
56	0.105	-29.0477927404806\\
56	0.1056	-31.8030297286828\\
56	0.1062	-34.5582667168851\\
56	0.1068	-37.3135037050872\\
56	0.1074	-40.0687406932896\\
56	0.108	-42.8239776814917\\
56	0.1086	-45.579214669694\\
56	0.1092	-48.3344516578964\\
56	0.1098	-51.0896886460985\\
56	0.1104	-53.8449256343006\\
56	0.111	-56.6001626225027\\
56	0.1116	-59.3553996107053\\
56	0.1122	-62.1106365989071\\
56	0.1128	-64.8658735871097\\
56	0.1134	-67.6211105753118\\
56	0.114	-70.3763475635139\\
56	0.1146	-73.1315845517161\\
56	0.1152	-75.8868215399184\\
56	0.1158	-78.6420585281205\\
56	0.1164	-81.3972955163233\\
56	0.117	-84.152532504525\\
56	0.1176	-86.9077694927278\\
56	0.1182	-89.6630064809299\\
56	0.1188	-92.418243469132\\
56	0.1194	-95.1734804573343\\
56	0.12	-97.9287174455364\\
56	0.1206	-100.683954433739\\
56	0.1212	-103.439191421941\\
56	0.1218	-106.194428410143\\
56	0.1224	-108.949665398346\\
56	0.123	-111.704902386548\\
56.375	0.093	27.0744417352598\\
56.375	0.0936	24.376825464318\\
56.375	0.0942	21.679209193376\\
56.375	0.0948	18.9815929224344\\
56.375	0.0954	16.2839766514924\\
56.375	0.096	13.5863603805506\\
56.375	0.0966	10.8887441096085\\
56.375	0.0972	8.19112783866694\\
56.375	0.0978	5.49351156772491\\
56.375	0.0984	2.7958952967831\\
56.375	0.099	0.098279025841066\\
56.375	0.0996	-2.59933724510051\\
56.375	0.1002	-5.29695351604255\\
56.375	0.1008	-7.99456978698436\\
56.375	0.1014	-10.6921860579264\\
56.375	0.102	-13.3898023288682\\
56.375	0.1026	-16.08741859981\\
56.375	0.1032	-18.7850348707518\\
56.375	0.1038	-21.4826511416939\\
56.375	0.1044	-24.1802674126357\\
56.375	0.105	-26.8778836835775\\
56.375	0.1056	-29.5754999545193\\
56.375	0.1062	-32.2731162254613\\
56.375	0.1068	-34.9707324964031\\
56.375	0.1074	-37.6683487673452\\
56.375	0.108	-40.3659650382867\\
56.375	0.1086	-43.0635813092288\\
56.375	0.1092	-45.7611975801708\\
56.375	0.1098	-48.4588138511126\\
56.375	0.1104	-51.1564301220542\\
56.375	0.111	-53.854046392996\\
56.375	0.1116	-56.5516626639383\\
56.375	0.1122	-59.2492789348798\\
56.375	0.1128	-61.9468952058221\\
56.375	0.1134	-64.6445114767637\\
56.375	0.114	-67.3421277477055\\
56.375	0.1146	-70.0397440186473\\
56.375	0.1152	-72.7373602895893\\
56.375	0.1158	-75.4349765605309\\
56.375	0.1164	-78.1325928314732\\
56.375	0.117	-80.8302091024148\\
56.375	0.1176	-83.527825373357\\
56.375	0.1182	-86.2254416442988\\
56.375	0.1188	-88.9230579152404\\
56.375	0.1194	-91.6206741861824\\
56.375	0.12	-94.3182904571242\\
56.375	0.1206	-97.0159067280661\\
56.375	0.1212	-99.7135229990081\\
56.375	0.1218	-102.41113926995\\
56.375	0.1224	-105.108755540892\\
56.375	0.123	-107.806371811834\\
56.75	0.093	28.0919364469557\\
56.75	0.0936	25.4519408932745\\
56.75	0.0942	22.8119453395927\\
56.75	0.0948	20.1719497859115\\
56.75	0.0954	17.5319542322297\\
56.75	0.096	14.8919586785485\\
56.75	0.0966	12.2519631248667\\
56.75	0.0972	9.61196757118546\\
56.75	0.0978	6.97197201750373\\
56.75	0.0984	4.33197646382223\\
56.75	0.099	1.69198091014073\\
56.75	0.0996	-0.948014643540546\\
56.75	0.1002	-3.58801019722227\\
56.75	0.1008	-6.22800575090378\\
56.75	0.1014	-8.86800130458528\\
56.75	0.102	-11.5079968582668\\
56.75	0.1026	-14.1479924119483\\
56.75	0.1032	-16.7879879656298\\
56.75	0.1038	-19.4279835193113\\
56.75	0.1044	-22.0679790729928\\
56.75	0.105	-24.7079746266743\\
56.75	0.1056	-27.3479701803558\\
56.75	0.1062	-29.9879657340373\\
56.75	0.1068	-32.6279612877188\\
56.75	0.1074	-35.2679568414003\\
56.75	0.108	-37.9079523950818\\
56.75	0.1086	-40.5479479487633\\
56.75	0.1092	-43.187943502445\\
56.75	0.1098	-45.8279390561263\\
56.75	0.1104	-48.4679346098078\\
56.75	0.111	-51.1079301634891\\
56.75	0.1116	-53.747925717171\\
56.75	0.1122	-56.3879212708521\\
56.75	0.1128	-59.027916824534\\
56.75	0.1134	-61.6679123782153\\
56.75	0.114	-64.3079079318968\\
56.75	0.1146	-66.9479034855781\\
56.75	0.1152	-69.5878990392598\\
56.75	0.1158	-72.2278945929411\\
56.75	0.1164	-74.867890146623\\
56.75	0.117	-77.5078857003041\\
56.75	0.1176	-80.147881253986\\
56.75	0.1182	-82.7878768076673\\
56.75	0.1188	-85.4278723613488\\
56.75	0.1194	-88.0678679150303\\
56.75	0.12	-90.7078634687118\\
56.75	0.1206	-93.3478590223933\\
56.75	0.1212	-95.9878545760751\\
56.75	0.1218	-98.6278501297563\\
56.75	0.1224	-101.267845683438\\
56.75	0.123	-103.90784123712\\
57.125	0.093	29.1094311586514\\
57.125	0.0936	26.5270563222305\\
57.125	0.0942	23.9446814858093\\
57.125	0.0948	21.3623066493883\\
57.125	0.0954	18.7799318129669\\
57.125	0.096	16.1975569765459\\
57.125	0.0966	13.6151821401247\\
57.125	0.0972	11.0328073037037\\
57.125	0.0978	8.45043246728233\\
57.125	0.0984	5.86805763086136\\
57.125	0.099	3.28568279444016\\
57.125	0.0996	0.703307958019195\\
57.125	0.1002	-1.87906687840223\\
57.125	0.1008	-4.4614417148232\\
57.125	0.1014	-7.04381655124439\\
57.125	0.102	-9.62619138766536\\
57.125	0.1026	-12.2085662240868\\
57.125	0.1032	-14.7909410605077\\
57.125	0.1038	-17.3733158969289\\
57.125	0.1044	-19.9556907333499\\
57.125	0.105	-22.5380655697713\\
57.125	0.1056	-25.1204404061923\\
57.125	0.1062	-27.7028152426135\\
57.125	0.1068	-30.2851900790347\\
57.125	0.1074	-32.8675649154559\\
57.125	0.108	-35.4499397518769\\
57.125	0.1086	-38.032314588298\\
57.125	0.1092	-40.6146894247195\\
57.125	0.1098	-43.1970642611404\\
57.125	0.1104	-45.7794390975614\\
57.125	0.111	-48.3618139339824\\
57.125	0.1116	-50.944188770404\\
57.125	0.1122	-53.5265636068248\\
57.125	0.1128	-56.1089384432462\\
57.125	0.1134	-58.6913132796672\\
57.125	0.114	-61.2736881160884\\
57.125	0.1146	-63.8560629525093\\
57.125	0.1152	-66.4384377889305\\
57.125	0.1158	-69.0208126253515\\
57.125	0.1164	-71.6031874617731\\
57.125	0.117	-74.1855622981939\\
57.125	0.1176	-76.7679371346153\\
57.125	0.1182	-79.3503119710363\\
57.125	0.1188	-81.9326868074575\\
57.125	0.1194	-84.5150616438787\\
57.125	0.12	-87.0974364802996\\
57.125	0.1206	-89.6798113167206\\
57.125	0.1212	-92.2621861531422\\
57.125	0.1218	-94.844560989563\\
57.125	0.1224	-97.4269358259844\\
57.125	0.123	-100.009310662405\\
57.5	0.093	30.1269258703476\\
57.5	0.0936	27.6021717511867\\
57.5	0.0942	25.0774176320258\\
57.5	0.0948	22.5526635128654\\
57.5	0.0954	20.0279093937045\\
57.5	0.096	17.5031552745436\\
57.5	0.0966	14.9784011553827\\
57.5	0.0972	12.4536470362223\\
57.5	0.0978	9.92889291706138\\
57.5	0.0984	7.40413879790071\\
57.5	0.099	4.8793846787396\\
57.5	0.0996	2.35463055957916\\
57.5	0.1002	-0.170123559581725\\
57.5	0.1008	-2.69487767874239\\
57.5	0.1014	-5.21963179790328\\
57.5	0.102	-7.74438591706394\\
57.5	0.1026	-10.2691400362248\\
57.5	0.1032	-12.7938941553855\\
57.5	0.1038	-15.3186482745464\\
57.5	0.1044	-17.843402393707\\
57.5	0.105	-20.3681565128679\\
57.5	0.1056	-22.8929106320286\\
57.5	0.1062	-25.4176647511895\\
57.5	0.1068	-27.9424188703501\\
57.5	0.1074	-30.467172989511\\
57.5	0.108	-32.9919271086717\\
57.5	0.1086	-35.5166812278326\\
57.5	0.1092	-38.0414353469935\\
57.5	0.1098	-40.5661894661541\\
57.5	0.1104	-43.0909435853148\\
57.5	0.111	-45.6156977044755\\
57.5	0.1116	-48.1404518236366\\
57.5	0.1122	-50.665205942797\\
57.5	0.1128	-53.1899600619581\\
57.5	0.1134	-55.7147141811188\\
57.5	0.114	-58.2394683002794\\
57.5	0.1146	-60.7642224194401\\
57.5	0.1152	-63.288976538601\\
57.5	0.1158	-65.8137306577617\\
57.5	0.1164	-68.3384847769228\\
57.5	0.117	-70.8632388960832\\
57.5	0.1176	-73.3879930152443\\
57.5	0.1182	-75.912747134405\\
57.5	0.1188	-78.4375012535656\\
57.5	0.1194	-80.9622553727265\\
57.5	0.12	-83.4870094918872\\
57.5	0.1206	-86.0117636110479\\
57.5	0.1212	-88.536517730209\\
57.5	0.1218	-91.0612718493694\\
57.5	0.1224	-93.5860259685305\\
57.5	0.123	-96.1107800876912\\
57.875	0.093	31.1444205820433\\
57.875	0.0936	28.6772871801429\\
57.875	0.0942	26.2101537782423\\
57.875	0.0948	23.743020376342\\
57.875	0.0954	21.2758869744416\\
57.875	0.096	18.8087535725413\\
57.875	0.0966	16.3416201706407\\
57.875	0.0972	13.8744867687406\\
57.875	0.0978	11.40735336684\\
57.875	0.0984	8.94021996493962\\
57.875	0.099	6.47308656303903\\
57.875	0.0996	4.0059531611389\\
57.875	0.1002	1.53881975923832\\
57.875	0.1008	-0.928313642662033\\
57.875	0.1014	-3.39544704456262\\
57.875	0.102	-5.86258044646274\\
57.875	0.1026	-8.32971384836333\\
57.875	0.1032	-10.7968472502637\\
57.875	0.1038	-13.2639806521643\\
57.875	0.1044	-15.7311140540644\\
57.875	0.105	-18.198247455965\\
57.875	0.1056	-20.6653808578653\\
57.875	0.1062	-23.1325142597659\\
57.875	0.1068	-25.599647661666\\
57.875	0.1074	-28.0667810635666\\
57.875	0.108	-30.533914465467\\
57.875	0.1086	-33.0010478673676\\
57.875	0.1092	-35.4681812692679\\
57.875	0.1098	-37.9353146711683\\
57.875	0.1104	-40.4024480730686\\
57.875	0.111	-42.8695814749688\\
57.875	0.1116	-45.3367148768696\\
57.875	0.1122	-47.8038482787697\\
57.875	0.1128	-50.2709816806705\\
57.875	0.1134	-52.7381150825706\\
57.875	0.114	-55.205248484471\\
57.875	0.1146	-57.6723818863713\\
57.875	0.1152	-60.1395152882719\\
57.875	0.1158	-62.6066486901721\\
57.875	0.1164	-65.0737820920729\\
57.875	0.117	-67.540915493973\\
57.875	0.1176	-70.0080488958738\\
57.875	0.1182	-72.4751822977739\\
57.875	0.1188	-74.9423156996743\\
57.875	0.1194	-77.4094491015749\\
57.875	0.12	-79.876582503475\\
57.875	0.1206	-82.3437159053753\\
57.875	0.1212	-84.8108493072762\\
57.875	0.1218	-87.2779827091763\\
57.875	0.1224	-89.7451161110769\\
57.875	0.123	-92.2122495129772\\
58.25	0.093	32.1619152937392\\
58.25	0.0936	29.7524026090991\\
58.25	0.0942	27.3428899244591\\
58.25	0.0948	24.933377239819\\
58.25	0.0954	22.523864555179\\
58.25	0.096	20.1143518705389\\
58.25	0.0966	17.7048391858989\\
58.25	0.0972	15.2953265012591\\
58.25	0.0978	12.8858138166188\\
58.25	0.0984	10.476301131979\\
58.25	0.099	8.06678844733869\\
58.25	0.0996	5.65727576269887\\
58.25	0.1002	3.2477630780586\\
58.25	0.1008	0.838250393418775\\
58.25	0.1014	-1.5712622912215\\
58.25	0.102	-3.98077497586132\\
58.25	0.1026	-6.3902876605016\\
58.25	0.1032	-8.79980034514142\\
58.25	0.1038	-11.2093130297817\\
58.25	0.1044	-13.6188257144215\\
58.25	0.105	-16.0283383990618\\
58.25	0.1056	-18.4378510837016\\
58.25	0.1062	-20.8473637683419\\
58.25	0.1068	-23.2568764529817\\
58.25	0.1074	-25.666389137622\\
58.25	0.108	-28.0759018222618\\
58.25	0.1086	-30.4854145069021\\
58.25	0.1092	-32.8949271915421\\
58.25	0.1098	-35.3044398761822\\
58.25	0.1104	-37.713952560822\\
58.25	0.111	-40.1234652454621\\
58.25	0.1116	-42.5329779301023\\
58.25	0.1122	-44.9424906147419\\
58.25	0.1128	-47.3520032993824\\
58.25	0.1134	-49.7615159840223\\
58.25	0.114	-52.1710286686623\\
58.25	0.1146	-54.5805413533021\\
58.25	0.1152	-56.9900540379424\\
58.25	0.1158	-59.3995667225822\\
58.25	0.1164	-61.8090794072227\\
58.25	0.117	-64.2185920918623\\
58.25	0.1176	-66.6281047765028\\
58.25	0.1182	-69.0376174611426\\
58.25	0.1188	-71.4471301457827\\
58.25	0.1194	-73.8566428304227\\
58.25	0.12	-76.2661555150628\\
58.25	0.1206	-78.6756681997026\\
58.25	0.1212	-81.0851808843431\\
58.25	0.1218	-83.4946935689827\\
58.25	0.1224	-85.9042062536232\\
58.25	0.123	-88.313718938263\\
58.625	0.093	33.1794100054349\\
58.625	0.0936	30.8275180380554\\
58.625	0.0942	28.4756260706756\\
58.625	0.0948	26.1237341032959\\
58.625	0.0954	23.7718421359161\\
58.625	0.096	21.4199501685366\\
58.625	0.0966	19.0680582011569\\
58.625	0.0972	16.7161662337774\\
58.625	0.0978	14.3642742663974\\
58.625	0.0984	12.0123822990179\\
58.625	0.099	9.66049033163813\\
58.625	0.0996	7.30859836425861\\
58.625	0.1002	4.95670639687864\\
58.625	0.1008	2.60481442949913\\
58.625	0.1014	0.252922462119386\\
58.625	0.102	-2.09896950526013\\
58.625	0.1026	-4.45086147263987\\
58.625	0.1032	-6.80275344001961\\
58.625	0.1038	-9.15464540739936\\
58.625	0.1044	-11.5065373747789\\
58.625	0.105	-13.8584293421586\\
58.625	0.1056	-16.2103213095384\\
58.625	0.1062	-18.5622132769181\\
58.625	0.1068	-20.9141052442976\\
58.625	0.1074	-23.2659972116774\\
58.625	0.108	-25.6178891790569\\
58.625	0.1086	-27.9697811464368\\
58.625	0.1092	-30.3216731138166\\
58.625	0.1098	-32.6735650811961\\
58.625	0.1104	-35.0254570485756\\
58.625	0.111	-37.3773490159554\\
58.625	0.1116	-39.7292409833353\\
58.625	0.1122	-42.0811329507146\\
58.625	0.1128	-44.4330249180946\\
58.625	0.1134	-46.7849168854741\\
58.625	0.114	-49.1368088528538\\
58.625	0.1146	-51.4887008202334\\
58.625	0.1152	-53.8405927876131\\
58.625	0.1158	-56.1924847549926\\
58.625	0.1164	-58.5443767223728\\
58.625	0.117	-60.8962686897521\\
58.625	0.1176	-63.2481606571321\\
58.625	0.1182	-65.6000526245116\\
58.625	0.1188	-67.9519445918911\\
58.625	0.1194	-70.3038365592711\\
58.625	0.12	-72.6557285266506\\
58.625	0.1206	-75.0076204940301\\
58.625	0.1212	-77.3595124614101\\
58.625	0.1218	-79.7114044287894\\
58.625	0.1224	-82.0632963961696\\
58.625	0.123	-84.4151883635491\\
59	0.093	34.1969047171308\\
59	0.0936	31.9026334670116\\
59	0.0942	29.6083622168921\\
59	0.0948	27.3140909667729\\
59	0.0954	25.0198197166535\\
59	0.096	22.7255484665343\\
59	0.0966	20.4312772164149\\
59	0.0972	18.1370059662959\\
59	0.0978	15.8427347161764\\
59	0.0984	13.5484634660572\\
59	0.099	11.2541922159378\\
59	0.0996	8.95992096581858\\
59	0.1002	6.66564971569915\\
59	0.1008	4.37137846557994\\
59	0.1014	2.0771072154605\\
59	0.102	-0.217164034658708\\
59	0.1026	-2.51143528477814\\
59	0.1032	-4.80570653489735\\
59	0.1038	-7.09997778501679\\
59	0.1044	-9.394249035136\\
59	0.105	-11.6885202852554\\
59	0.1056	-13.9827915353746\\
59	0.1062	-16.2770627854941\\
59	0.1068	-18.5713340356133\\
59	0.1074	-20.8656052857327\\
59	0.108	-23.1598765358519\\
59	0.1086	-25.4541477859714\\
59	0.1092	-27.7484190360908\\
59	0.1098	-30.04269028621\\
59	0.1104	-32.3369615363292\\
59	0.111	-34.6312327864484\\
59	0.1116	-36.9255040365681\\
59	0.1122	-39.2197752866869\\
59	0.1128	-41.5140465368065\\
59	0.1134	-43.8083177869257\\
59	0.114	-46.1025890370449\\
59	0.1146	-48.3968602871641\\
59	0.1152	-50.6911315372836\\
59	0.1158	-52.9854027874028\\
59	0.1164	-55.2796740375225\\
59	0.117	-57.5739452876414\\
59	0.1176	-59.8682165377611\\
59	0.1182	-62.1624877878803\\
59	0.1188	-64.4567590379995\\
59	0.1194	-66.751030288119\\
59	0.12	-69.0453015382382\\
59	0.1206	-71.3395727883574\\
59	0.1212	-73.633844038477\\
59	0.1218	-75.928115288596\\
59	0.1224	-78.2223865387157\\
59	0.123	-80.5166577888349\\
59.375	0.093	35.2143994288265\\
59.375	0.0936	32.9777488959678\\
59.375	0.0942	30.7410983631087\\
59.375	0.0948	28.5044478302498\\
59.375	0.0954	26.2677972973909\\
59.375	0.096	24.031146764532\\
59.375	0.0966	21.7944962316728\\
59.375	0.0972	19.5578456988139\\
59.375	0.0978	17.321195165955\\
59.375	0.0984	15.0845446330961\\
59.375	0.099	12.847894100237\\
59.375	0.0996	10.6112435673783\\
59.375	0.1002	8.37459303451919\\
59.375	0.1008	6.13794250166029\\
59.375	0.1014	3.90129196880139\\
59.375	0.102	1.66464143594249\\
59.375	0.1026	-0.572009096916645\\
59.375	0.1032	-2.80865962977555\\
59.375	0.1038	-5.04531016263445\\
59.375	0.1044	-7.28196069549335\\
59.375	0.105	-9.51861122835248\\
59.375	0.1056	-11.7552617612112\\
59.375	0.1062	-13.9919122940703\\
59.375	0.1068	-16.2285628269292\\
59.375	0.1074	-18.4652133597881\\
59.375	0.108	-20.701863892647\\
59.375	0.1086	-22.9385144255061\\
59.375	0.1092	-25.1751649583653\\
59.375	0.1098	-27.4118154912239\\
59.375	0.1104	-29.6484660240828\\
59.375	0.111	-31.8851165569417\\
59.375	0.1116	-34.1217670898009\\
59.375	0.1122	-36.3584176226595\\
59.375	0.1128	-38.5950681555189\\
59.375	0.1134	-40.8317186883776\\
59.375	0.114	-43.0683692212365\\
59.375	0.1146	-45.3050197540954\\
59.375	0.1152	-47.5416702869545\\
59.375	0.1158	-49.7783208198132\\
59.375	0.1164	-52.0149713526725\\
59.375	0.117	-54.2516218855312\\
59.375	0.1176	-56.4882724183904\\
59.375	0.1182	-58.7249229512493\\
59.375	0.1188	-60.9615734841082\\
59.375	0.1194	-63.1982240169671\\
59.375	0.12	-65.434874549826\\
59.375	0.1206	-67.6715250826849\\
59.375	0.1212	-69.9081756155442\\
59.375	0.1218	-72.1448261484027\\
59.375	0.1224	-74.381476681262\\
59.375	0.123	-76.6181272141209\\
59.75	0.093	36.2318941405226\\
59.75	0.0936	34.052864324924\\
59.75	0.0942	31.8738345093254\\
59.75	0.0948	29.6948046937268\\
59.75	0.0954	27.5157748781282\\
59.75	0.096	25.3367450625296\\
59.75	0.0966	23.157715246931\\
59.75	0.0972	20.9786854313325\\
59.75	0.0978	18.7996556157339\\
59.75	0.0984	16.6206258001355\\
59.75	0.099	14.4415959845367\\
59.75	0.0996	12.2625661689383\\
59.75	0.1002	10.0835363533395\\
59.75	0.1008	7.9045065377411\\
59.75	0.1014	5.72547672214228\\
59.75	0.102	3.54644690654391\\
59.75	0.1026	1.36741709094531\\
59.75	0.1032	-0.811612724653287\\
59.75	0.1038	-2.99064254025188\\
59.75	0.1044	-5.16967235585048\\
59.75	0.105	-7.34870217144908\\
59.75	0.1056	-9.52773198704767\\
59.75	0.1062	-11.7067618026463\\
59.75	0.1068	-13.8857916182449\\
59.75	0.1074	-16.0648214338435\\
59.75	0.108	-18.2438512494418\\
59.75	0.1086	-20.4228810650407\\
59.75	0.1092	-22.6019108806393\\
59.75	0.1098	-24.7809406962378\\
59.75	0.1104	-26.9599705118362\\
59.75	0.111	-29.1390003274348\\
59.75	0.1116	-31.3180301430336\\
59.75	0.1122	-33.497059958632\\
59.75	0.1128	-35.6760897742308\\
59.75	0.1134	-37.8551195898292\\
59.75	0.114	-40.0341494054278\\
59.75	0.1146	-42.2131792210262\\
59.75	0.1152	-44.392209036625\\
59.75	0.1158	-46.5712388522234\\
59.75	0.1164	-48.7502686678224\\
59.75	0.117	-50.9292984834206\\
59.75	0.1176	-53.1083282990196\\
59.75	0.1182	-55.287358114618\\
59.75	0.1188	-57.4663879302163\\
59.75	0.1194	-59.6454177458152\\
59.75	0.12	-61.8244475614135\\
59.75	0.1206	-64.0034773770121\\
59.75	0.1212	-66.182507192611\\
59.75	0.1218	-68.3615370082093\\
59.75	0.1224	-70.5405668238081\\
59.75	0.123	-72.7195966394067\\
60.125	0.093	37.2493888522185\\
60.125	0.0936	35.1279797538805\\
60.125	0.0942	33.006570655542\\
60.125	0.0948	30.8851615572039\\
60.125	0.0954	28.7637524588656\\
60.125	0.096	26.6423433605275\\
60.125	0.0966	24.5209342621893\\
60.125	0.0972	22.399525163851\\
60.125	0.0978	20.2781160655127\\
60.125	0.0984	18.1567069671746\\
60.125	0.099	16.0352978688363\\
60.125	0.0996	13.9138887704983\\
60.125	0.1002	11.79247967216\\
60.125	0.1008	9.67107057382168\\
60.125	0.1014	7.54966147548339\\
60.125	0.102	5.42825237714533\\
60.125	0.1026	3.30684327880704\\
60.125	0.1032	1.18543418046897\\
60.125	0.1038	-0.935974917869544\\
60.125	0.1044	-3.05738401620761\\
60.125	0.105	-5.1787931145459\\
60.125	0.1056	-7.30020221288396\\
60.125	0.1062	-9.42161131122225\\
60.125	0.1068	-11.5430204095603\\
60.125	0.1074	-13.6644295078988\\
60.125	0.108	-15.7858386062369\\
60.125	0.1086	-17.9072477045752\\
60.125	0.1092	-20.0286568029135\\
60.125	0.1098	-22.1500659012515\\
60.125	0.1104	-24.2714749995898\\
60.125	0.111	-26.3928840979279\\
60.125	0.1116	-28.5142931962664\\
60.125	0.1122	-30.6357022946042\\
60.125	0.1128	-32.7571113929428\\
60.125	0.1134	-34.8785204912808\\
60.125	0.114	-36.9999295896191\\
60.125	0.1146	-39.1213386879572\\
60.125	0.1152	-41.2427477862955\\
60.125	0.1158	-43.3641568846335\\
60.125	0.1164	-45.485565982972\\
60.125	0.117	-47.6069750813101\\
60.125	0.1176	-49.7283841796486\\
60.125	0.1182	-51.8497932779867\\
60.125	0.1188	-53.9712023763248\\
60.125	0.1194	-56.092611474663\\
60.125	0.12	-58.2140205730011\\
60.125	0.1206	-60.3354296713394\\
60.125	0.1212	-62.4568387696779\\
60.125	0.1218	-64.5782478680158\\
60.125	0.1224	-66.6996569663543\\
60.125	0.123	-68.8210660646923\\
60.5	0.093	38.2668835639142\\
60.5	0.0936	36.2030951828365\\
60.5	0.0942	34.1393068017585\\
60.5	0.0948	32.0755184206807\\
60.5	0.0954	30.0117300396028\\
60.5	0.096	27.947941658525\\
60.5	0.0966	25.884153277447\\
60.5	0.0972	23.8203648963693\\
60.5	0.0978	21.7565765152915\\
60.5	0.0984	19.6927881342137\\
60.5	0.099	17.6289997531358\\
60.5	0.0996	15.565211372058\\
60.5	0.1002	13.50142299098\\
60.5	0.1008	11.4376346099023\\
60.5	0.1014	9.37384622882428\\
60.5	0.102	7.31005784774652\\
60.5	0.1026	5.24626946666854\\
60.5	0.1032	3.18248108559078\\
60.5	0.1038	1.1186927045128\\
60.5	0.1044	-0.945095676564961\\
60.5	0.105	-3.00888405764294\\
60.5	0.1056	-5.07267243872047\\
60.5	0.1062	-7.13646081979846\\
60.5	0.1068	-9.20024920087621\\
60.5	0.1074	-11.2640375819542\\
60.5	0.108	-13.327825963032\\
60.5	0.1086	-15.3916143441099\\
60.5	0.1092	-17.4554027251879\\
60.5	0.1098	-19.5191911062657\\
60.5	0.1104	-21.5829794873434\\
60.5	0.111	-23.6467678684212\\
60.5	0.1116	-25.7105562494994\\
60.5	0.1122	-27.7743446305769\\
60.5	0.1128	-29.8381330116551\\
60.5	0.1134	-31.9019213927327\\
60.5	0.114	-33.9657097738104\\
60.5	0.1146	-36.0294981548882\\
60.5	0.1152	-38.0932865359662\\
60.5	0.1158	-40.1570749170439\\
60.5	0.1164	-42.2208632981221\\
60.5	0.117	-44.2846516791997\\
60.5	0.1176	-46.3484400602779\\
60.5	0.1182	-48.4122284413556\\
60.5	0.1188	-50.4760168224334\\
60.5	0.1194	-52.5398052035114\\
60.5	0.12	-54.6035935845891\\
60.5	0.1206	-56.6673819656669\\
60.5	0.1212	-58.7311703467449\\
60.5	0.1218	-60.7949587278224\\
60.5	0.1224	-62.8587471089006\\
60.5	0.123	-64.9225354899784\\
60.875	0.093	39.2843782756102\\
60.875	0.0936	37.2782106117927\\
60.875	0.0942	35.2720429479753\\
60.875	0.0948	33.2658752841578\\
60.875	0.0954	31.2597076203401\\
60.875	0.096	29.2535399565229\\
60.875	0.0966	27.2473722927052\\
60.875	0.0972	25.2412046288878\\
60.875	0.0978	23.2350369650703\\
60.875	0.0984	21.2288693012529\\
60.875	0.099	19.2227016374352\\
60.875	0.0996	17.216533973618\\
60.875	0.1002	15.2103663098003\\
60.875	0.1008	13.2041986459828\\
60.875	0.1014	11.1980309821654\\
60.875	0.102	9.19186331834794\\
60.875	0.1026	7.18569565453026\\
60.875	0.1032	5.17952799071304\\
60.875	0.1038	3.17336032689536\\
60.875	0.1044	1.16719266307791\\
60.875	0.105	-0.838975000739538\\
60.875	0.1056	-2.84514266455699\\
60.875	0.1062	-4.85131032837467\\
60.875	0.1068	-6.85747799219189\\
60.875	0.1074	-8.86364565600957\\
60.875	0.108	-10.869813319827\\
60.875	0.1086	-12.8759809836445\\
60.875	0.1092	-14.8821486474621\\
60.875	0.1098	-16.8883163112794\\
60.875	0.1104	-18.8944839750968\\
60.875	0.111	-20.9006516389143\\
60.875	0.1116	-22.9068193027322\\
60.875	0.1122	-24.9129869665492\\
60.875	0.1128	-26.9191546303671\\
60.875	0.1134	-28.9253222941843\\
60.875	0.114	-30.9314899580017\\
60.875	0.1146	-32.9376576218192\\
60.875	0.1152	-34.9438252856366\\
60.875	0.1158	-36.9499929494541\\
60.875	0.1164	-38.956160613272\\
60.875	0.117	-40.962328277089\\
60.875	0.1176	-42.9684959409069\\
60.875	0.1182	-44.9746636047244\\
60.875	0.1188	-46.9808312685416\\
60.875	0.1194	-48.9869989323593\\
60.875	0.12	-50.9931665961767\\
60.875	0.1206	-52.9993342599939\\
60.875	0.1212	-55.0055019238118\\
60.875	0.1218	-57.0116695876291\\
60.875	0.1224	-59.0178372514467\\
60.875	0.123	-61.0240049152642\\
61.25	0.093	40.3018729873058\\
61.25	0.0936	38.3533260407489\\
61.25	0.0942	36.4047790941916\\
61.25	0.0948	34.4562321476346\\
61.25	0.0954	32.5076852010775\\
61.25	0.096	30.5591382545203\\
61.25	0.0966	28.6105913079632\\
61.25	0.0972	26.6620443614061\\
61.25	0.0978	24.7134974148489\\
61.25	0.0984	22.764950468292\\
61.25	0.099	20.8164035217346\\
61.25	0.0996	18.8678565751777\\
61.25	0.1002	16.9193096286203\\
61.25	0.1008	14.9707626820634\\
61.25	0.1014	13.0222157355063\\
61.25	0.102	11.0736687889491\\
61.25	0.1026	9.12512184239199\\
61.25	0.1032	7.17657489583485\\
61.25	0.1038	5.2280279492777\\
61.25	0.1044	3.27948100272079\\
61.25	0.105	1.33093405616341\\
61.25	0.1056	-0.617612890393502\\
61.25	0.1062	-2.56615983695087\\
61.25	0.1068	-4.51470678350779\\
61.25	0.1074	-6.46325373006493\\
61.25	0.108	-8.41180067662208\\
61.25	0.1086	-10.3603476231792\\
61.25	0.1092	-12.3088945697366\\
61.25	0.1098	-14.2574415162935\\
61.25	0.1104	-16.2059884628504\\
61.25	0.111	-18.1545354094076\\
61.25	0.1116	-20.1030823559649\\
61.25	0.1122	-22.0516293025219\\
61.25	0.1128	-24.0001762490792\\
61.25	0.1134	-25.9487231956364\\
61.25	0.114	-27.8972701421933\\
61.25	0.1146	-29.8458170887502\\
61.25	0.1152	-31.7943640353076\\
61.25	0.1158	-33.7429109818645\\
61.25	0.1164	-35.6914579284221\\
61.25	0.117	-37.6400048749788\\
61.25	0.1176	-39.5885518215362\\
61.25	0.1182	-41.5370987680933\\
61.25	0.1188	-43.4856457146502\\
61.25	0.1194	-45.4341926612076\\
61.25	0.12	-47.3827396077645\\
61.25	0.1206	-49.3312865543214\\
61.25	0.1212	-51.279833500879\\
61.25	0.1218	-53.2283804474357\\
61.25	0.1224	-55.1769273939933\\
61.25	0.123	-57.1254743405502\\
61.625	0.093	41.3193676990018\\
61.625	0.0936	39.4284414697051\\
61.625	0.0942	37.5375152404083\\
61.625	0.0948	35.6465890111117\\
61.625	0.0954	33.7556627818149\\
61.625	0.096	31.8647365525183\\
61.625	0.0966	29.9738103232214\\
61.625	0.0972	28.0828840939246\\
61.625	0.0978	26.1919578646277\\
61.625	0.0984	24.3010316353311\\
61.625	0.099	22.4101054060343\\
61.625	0.0996	20.5191791767377\\
61.625	0.1002	18.6282529474408\\
61.625	0.1008	16.7373267181442\\
61.625	0.1014	14.8464004888472\\
61.625	0.102	12.9554742595506\\
61.625	0.1026	11.0645480302537\\
61.625	0.1032	9.17362180095711\\
61.625	0.1038	7.28269557166027\\
61.625	0.1044	5.39176934236366\\
61.625	0.105	3.50084311306682\\
61.625	0.1056	1.60991688376998\\
61.625	0.1062	-0.281009345526854\\
61.625	0.1068	-2.17193557482346\\
61.625	0.1074	-4.0628618041203\\
61.625	0.108	-5.95378803341691\\
61.625	0.1086	-7.84471426271375\\
61.625	0.1092	-9.73564049201059\\
61.625	0.1098	-11.6265667213074\\
61.625	0.1104	-13.517492950604\\
61.625	0.111	-15.4084191799006\\
61.625	0.1116	-17.2993454091977\\
61.625	0.1122	-19.1902716384941\\
61.625	0.1128	-21.0811978677912\\
61.625	0.1134	-22.972124097088\\
61.625	0.114	-24.8630503263846\\
61.625	0.1146	-26.7539765556812\\
61.625	0.1152	-28.6449027849781\\
61.625	0.1158	-30.5358290142747\\
61.625	0.1164	-32.4267552435717\\
61.625	0.117	-34.3176814728681\\
61.625	0.1176	-36.2086077021654\\
61.625	0.1182	-38.099533931462\\
61.625	0.1188	-39.9904601607586\\
61.625	0.1194	-41.8813863900555\\
61.625	0.12	-43.7723126193521\\
61.625	0.1206	-45.6632388486487\\
61.625	0.1212	-47.5541650779458\\
61.625	0.1218	-49.4450913072424\\
61.625	0.1224	-51.3360175365394\\
61.625	0.123	-53.226943765836\\
62	0.093	42.3368624106977\\
62	0.0936	40.5035568986614\\
62	0.0942	38.6702513866248\\
62	0.0948	36.8369458745885\\
62	0.0954	35.003640362552\\
62	0.096	33.1703348505157\\
62	0.0966	31.3370293384792\\
62	0.0972	29.5037238264429\\
62	0.0978	27.6704183144066\\
62	0.0984	25.8371128023703\\
62	0.099	24.0038072903337\\
62	0.0996	22.1705017782974\\
62	0.1002	20.3371962662609\\
62	0.1008	18.5038907542246\\
62	0.1014	16.6705852421881\\
62	0.102	14.8372797301517\\
62	0.1026	13.0039742181152\\
62	0.1032	11.1706687060791\\
62	0.1038	9.33736319404261\\
62	0.1044	7.50405768200631\\
62	0.105	5.67075216996977\\
62	0.1056	3.83744665793347\\
62	0.1062	2.00414114589694\\
62	0.1068	0.170835633860634\\
62	0.1074	-1.6624698781759\\
62	0.108	-3.4957753902122\\
62	0.1086	-5.32908090224851\\
62	0.1092	-7.16238641428504\\
62	0.1098	-8.99569192632134\\
62	0.1104	-10.8289974383576\\
62	0.111	-12.662302950394\\
62	0.1116	-14.4956084624307\\
62	0.1122	-16.3289139744668\\
62	0.1128	-18.1622194865035\\
62	0.1134	-19.9955249985398\\
62	0.114	-21.8288305105759\\
62	0.1146	-23.6621360226122\\
62	0.1152	-25.4954415346488\\
62	0.1158	-27.3287470466851\\
62	0.1164	-29.1620525587218\\
62	0.117	-30.9953580707579\\
62	0.1176	-32.8286635827947\\
62	0.1182	-34.661969094831\\
62	0.1188	-36.4952746068673\\
62	0.1194	-38.3285801189036\\
62	0.12	-40.1618856309399\\
62	0.1206	-41.9951911429762\\
62	0.1212	-43.8284966550129\\
62	0.1218	-45.661802167049\\
62	0.1224	-47.4951076790858\\
62	0.123	-49.3284131911221\\
62.375	0.093	43.3543571223936\\
62.375	0.0936	41.5786723276176\\
62.375	0.0942	39.8029875328416\\
62.375	0.0948	38.0273027380656\\
62.375	0.0954	36.2516179432894\\
62.375	0.096	34.4759331485136\\
62.375	0.0966	32.7002483537374\\
62.375	0.0972	30.9245635589614\\
62.375	0.0978	29.1488787641854\\
62.375	0.0984	27.3731939694094\\
62.375	0.099	25.5975091746334\\
62.375	0.0996	23.8218243798574\\
62.375	0.1002	22.0461395850812\\
62.375	0.1008	20.2704547903054\\
62.375	0.1014	18.4947699955292\\
62.375	0.102	16.7190852007532\\
62.375	0.1026	14.9434004059772\\
62.375	0.1032	13.1677156112012\\
62.375	0.1038	11.3920308164249\\
62.375	0.1044	9.61634602164918\\
62.375	0.105	7.84066122687295\\
62.375	0.1056	6.06497643209718\\
62.375	0.1062	4.28929163732096\\
62.375	0.1068	2.51360684254496\\
62.375	0.1074	0.737922047768961\\
62.375	0.108	-1.03776274700704\\
62.375	0.1086	-2.81344754178326\\
62.375	0.1092	-4.58913233655926\\
62.375	0.1098	-6.36481713133526\\
62.375	0.1104	-8.14050192611103\\
62.375	0.111	-9.91618672088703\\
62.375	0.1116	-11.6918715156635\\
62.375	0.1122	-13.467556310439\\
62.375	0.1128	-15.2432411052155\\
62.375	0.1134	-17.0189258999915\\
62.375	0.114	-18.7946106947672\\
62.375	0.1146	-20.5702954895432\\
62.375	0.1152	-22.3459802843192\\
62.375	0.1158	-24.1216650790952\\
62.375	0.1164	-25.8973498738717\\
62.375	0.117	-27.6730346686472\\
62.375	0.1176	-29.4487194634237\\
62.375	0.1182	-31.2244042581997\\
62.375	0.1188	-33.0000890529755\\
62.375	0.1194	-34.7757738477517\\
62.375	0.12	-36.5514586425275\\
62.375	0.1206	-38.3271434373034\\
62.375	0.1212	-40.1028282320799\\
62.375	0.1218	-41.8785130268554\\
62.375	0.1224	-43.6541978216319\\
62.375	0.123	-45.4298826164079\\
62.75	0.093	44.3718518340893\\
62.75	0.0936	42.6537877565738\\
62.75	0.0942	40.9357236790579\\
62.75	0.0948	39.2176596015424\\
62.75	0.0954	37.4995955240265\\
62.75	0.096	35.7815314465111\\
62.75	0.0966	34.0634673689954\\
62.75	0.0972	32.3454032914797\\
62.75	0.0978	30.627339213964\\
62.75	0.0984	28.9092751364485\\
62.75	0.099	27.1912110589326\\
62.75	0.0996	25.4731469814171\\
62.75	0.1002	23.7550829039014\\
62.75	0.1008	22.0370188263857\\
62.75	0.1014	20.3189547488701\\
62.75	0.102	18.6008906713544\\
62.75	0.1026	16.8828265938387\\
62.75	0.1032	15.1647625163232\\
62.75	0.1038	13.4466984388073\\
62.75	0.1044	11.7286343612918\\
62.75	0.105	10.0105702837761\\
62.75	0.1056	8.29250620626044\\
62.75	0.1062	6.57444212874475\\
62.75	0.1068	4.85637805122906\\
62.75	0.1074	3.13831397371337\\
62.75	0.108	1.4202498961979\\
62.75	0.1086	-0.297814181318017\\
62.75	0.1092	-2.01587825883371\\
62.75	0.1098	-3.73394233634917\\
62.75	0.1104	-5.45200641386486\\
62.75	0.111	-7.17007049138033\\
62.75	0.1116	-8.88813456889625\\
62.75	0.1122	-10.6061986464117\\
62.75	0.1128	-12.3242627239276\\
62.75	0.1134	-14.0423268014433\\
62.75	0.114	-15.7603908789588\\
62.75	0.1146	-17.4784549564743\\
62.75	0.1152	-19.1965190339902\\
62.75	0.1158	-20.9145831115056\\
62.75	0.1164	-22.6326471890216\\
62.75	0.117	-24.350711266537\\
62.75	0.1176	-26.0687753440529\\
62.75	0.1182	-27.7868394215684\\
62.75	0.1188	-29.5049034990841\\
62.75	0.1194	-31.2229675765998\\
62.75	0.12	-32.9410316541155\\
62.75	0.1206	-34.6590957316309\\
62.75	0.1212	-36.3771598091469\\
62.75	0.1218	-38.0952238866623\\
62.75	0.1224	-39.8132879641782\\
62.75	0.123	-41.5313520416937\\
63.125	0.093	45.3893465457852\\
63.125	0.0936	43.72890318553\\
63.125	0.0942	42.0684598252747\\
63.125	0.0948	40.4080164650195\\
63.125	0.0954	38.7475731047641\\
63.125	0.096	37.087129744509\\
63.125	0.0966	35.4266863842533\\
63.125	0.0972	33.7662430239982\\
63.125	0.0978	32.1057996637428\\
63.125	0.0984	30.4453563034876\\
63.125	0.099	28.7849129432323\\
63.125	0.0996	27.1244695829771\\
63.125	0.1002	25.4640262227217\\
63.125	0.1008	23.8035828624666\\
63.125	0.1014	22.1431395022112\\
63.125	0.102	20.482696141956\\
63.125	0.1026	18.8222527817004\\
63.125	0.1032	17.1618094214452\\
63.125	0.1038	15.5013660611899\\
63.125	0.1044	13.8409227009347\\
63.125	0.105	12.1804793406793\\
63.125	0.1056	10.5200359804242\\
63.125	0.1062	8.85959262016877\\
63.125	0.1068	7.19914925991361\\
63.125	0.1074	5.53870589965823\\
63.125	0.108	3.87826253940284\\
63.125	0.1086	2.21781917914745\\
63.125	0.1092	0.557375818892069\\
63.125	0.1098	-1.10306754136309\\
63.125	0.1104	-2.76351090161825\\
63.125	0.111	-4.4239542618734\\
63.125	0.1116	-6.08439762212902\\
63.125	0.1122	-7.74484098238395\\
63.125	0.1128	-9.40528434263956\\
63.125	0.1134	-11.0657277028949\\
63.125	0.114	-12.7261710631501\\
63.125	0.1146	-14.3866144234053\\
63.125	0.1152	-16.0470577836606\\
63.125	0.1158	-17.7075011439158\\
63.125	0.1164	-19.3679445041714\\
63.125	0.117	-21.0283878644263\\
63.125	0.1176	-22.688831224682\\
63.125	0.1182	-24.3492745849371\\
63.125	0.1188	-26.0097179451925\\
63.125	0.1194	-27.6701613054479\\
63.125	0.12	-29.330604665703\\
63.125	0.1206	-30.9910480259582\\
63.125	0.1212	-32.6514913862138\\
63.125	0.1218	-34.3119347464688\\
63.125	0.1224	-35.9723781067244\\
63.125	0.123	-37.6328214669795\\
63.5	0.093	46.4068412574811\\
63.5	0.0936	44.8040186144863\\
63.5	0.0942	43.2011959714914\\
63.5	0.0948	41.5983733284966\\
63.5	0.0954	39.9955506855015\\
63.5	0.096	38.3927280425066\\
63.5	0.0966	36.7899053995116\\
63.5	0.0972	35.1870827565167\\
63.5	0.0978	33.5842601135219\\
63.5	0.0984	31.981437470527\\
63.5	0.099	30.3786148275319\\
63.5	0.0996	28.7757921845371\\
63.5	0.1002	27.172969541542\\
63.5	0.1008	25.5701468985471\\
63.5	0.1014	23.9673242555521\\
63.5	0.102	22.3645016125574\\
63.5	0.1026	20.7616789695624\\
63.5	0.1032	19.1588563265675\\
63.5	0.1038	17.5560336835724\\
63.5	0.1044	15.9532110405776\\
63.5	0.105	14.3503883975825\\
63.5	0.1056	12.7475657545876\\
63.5	0.1062	11.1447431115928\\
63.5	0.1068	9.54192046859794\\
63.5	0.1074	7.93909782560286\\
63.5	0.108	6.33627518260801\\
63.5	0.1086	4.73345253961293\\
63.5	0.1092	3.13062989661785\\
63.5	0.1098	1.52780725362322\\
63.5	0.1104	-0.0750153893716288\\
63.5	0.111	-1.67783803236648\\
63.5	0.1116	-3.28066067536179\\
63.5	0.1122	-4.88348331835641\\
63.5	0.1128	-6.48630596135172\\
63.5	0.1134	-8.08912860434657\\
63.5	0.114	-9.69195124734119\\
63.5	0.1146	-11.294773890336\\
63.5	0.1152	-12.8975965333311\\
63.5	0.1158	-14.500419176326\\
63.5	0.1164	-16.1032418193213\\
63.5	0.117	-17.7060644623159\\
63.5	0.1176	-19.3088871053112\\
63.5	0.1182	-20.9117097483058\\
63.5	0.1188	-22.5145323913007\\
63.5	0.1194	-24.1173550342958\\
63.5	0.12	-25.7201776772906\\
63.5	0.1206	-27.3230003202855\\
63.5	0.1212	-28.9258229632808\\
63.5	0.1218	-30.5286456062752\\
63.5	0.1224	-32.1314682492705\\
63.5	0.123	-33.7342908922653\\
63.875	0.093	47.4243359691768\\
63.875	0.0936	45.8791340434425\\
63.875	0.0942	44.3339321177077\\
63.875	0.0948	42.7887301919734\\
63.875	0.0954	41.2435282662386\\
63.875	0.096	39.6983263405043\\
63.875	0.0966	38.1531244147695\\
63.875	0.0972	36.607922489035\\
63.875	0.0978	35.0627205633004\\
63.875	0.0984	33.5175186375659\\
63.875	0.099	31.9723167118314\\
63.875	0.0996	30.4271147860968\\
63.875	0.1002	28.881912860362\\
63.875	0.1008	27.3367109346277\\
63.875	0.1014	25.7915090088929\\
63.875	0.102	24.2463070831586\\
63.875	0.1026	22.7011051574239\\
63.875	0.1032	21.1559032316893\\
63.875	0.1038	19.6107013059548\\
63.875	0.1044	18.0654993802202\\
63.875	0.105	16.5202974544857\\
63.875	0.1056	14.9750955287511\\
63.875	0.1062	13.4298936030164\\
63.875	0.1068	11.884691677282\\
63.875	0.1074	10.3394897515473\\
63.875	0.108	8.79428782581294\\
63.875	0.1086	7.24908590007817\\
63.875	0.1092	5.70388397434363\\
63.875	0.1098	4.15868204860908\\
63.875	0.1104	2.61348012287453\\
63.875	0.111	1.06827819714022\\
63.875	0.1116	-0.476923728594784\\
63.875	0.1122	-2.02212565432887\\
63.875	0.1128	-3.56732758006387\\
63.875	0.1134	-5.11252950579842\\
63.875	0.114	-6.65773143153274\\
63.875	0.1146	-8.20293335726728\\
63.875	0.1152	-9.74813528300183\\
63.875	0.1158	-11.2933372087364\\
63.875	0.1164	-12.8385391344714\\
63.875	0.117	-14.3837410602055\\
63.875	0.1176	-15.9289429859405\\
63.875	0.1182	-17.4741449116748\\
63.875	0.1188	-19.0193468374093\\
63.875	0.1194	-20.5645487631439\\
63.875	0.12	-22.1097506888784\\
63.875	0.1206	-23.654952614613\\
63.875	0.1212	-25.2001545403477\\
63.875	0.1218	-26.7453564660821\\
63.875	0.1224	-28.2905583918168\\
63.875	0.123	-29.8357603175514\\
64.25	0.093	48.441830680873\\
64.25	0.0936	46.9542494723989\\
64.25	0.0942	45.4666682639247\\
64.25	0.0948	43.9790870554505\\
64.25	0.0954	42.4915058469762\\
64.25	0.096	41.0039246385022\\
64.25	0.0966	39.5163434300277\\
64.25	0.0972	38.0287622215537\\
64.25	0.0978	36.5411810130795\\
64.25	0.0984	35.0535998046053\\
64.25	0.099	33.566018596131\\
64.25	0.0996	32.078437387657\\
64.25	0.1002	30.5908561791825\\
64.25	0.1008	29.1032749707085\\
64.25	0.1014	27.6156937622343\\
64.25	0.102	26.1281125537603\\
64.25	0.1026	24.6405313452858\\
64.25	0.1032	23.1529501368118\\
64.25	0.1038	21.6653689283376\\
64.25	0.1044	20.1777877198633\\
64.25	0.105	18.6902065113891\\
64.25	0.1056	17.2026253029151\\
64.25	0.1062	15.7150440944406\\
64.25	0.1068	14.2274628859666\\
64.25	0.1074	12.7398816774923\\
64.25	0.108	11.2523004690181\\
64.25	0.1086	9.76471926054387\\
64.25	0.1092	8.27713805206963\\
64.25	0.1098	6.78955684359539\\
64.25	0.1104	5.30197563512138\\
64.25	0.111	3.81439442664737\\
64.25	0.1116	2.32681321817267\\
64.25	0.1122	0.83923200969889\\
64.25	0.1128	-0.648349198775577\\
64.25	0.1134	-2.13593040724982\\
64.25	0.114	-3.62351161572383\\
64.25	0.1146	-5.11109282419784\\
64.25	0.1152	-6.59867403267231\\
64.25	0.1158	-8.08625524114632\\
64.25	0.1164	-9.57383644962079\\
64.25	0.117	-11.0614176580948\\
64.25	0.1176	-12.5489988665693\\
64.25	0.1182	-14.0365800750433\\
64.25	0.1188	-15.5241612835175\\
64.25	0.1194	-17.0117424919918\\
64.25	0.12	-18.4993237004658\\
64.25	0.1206	-19.98690490894\\
64.25	0.1212	-21.4744861174145\\
64.25	0.1218	-22.9620673258883\\
64.25	0.1224	-24.449648534363\\
64.25	0.123	-25.937229742837\\
64.625	0.093	49.4593253925689\\
64.625	0.0936	48.0293649013552\\
64.625	0.0942	46.599404410141\\
64.625	0.0948	45.1694439189273\\
64.625	0.0954	43.7394834277134\\
64.625	0.096	42.3095229364997\\
64.625	0.0966	40.8795624452857\\
64.625	0.0972	39.449601954072\\
64.625	0.0978	38.0196414628581\\
64.625	0.0984	36.5896809716444\\
64.625	0.099	35.1597204804305\\
64.625	0.0996	33.7297599892167\\
64.625	0.1002	32.2997994980028\\
64.625	0.1008	30.8698390067891\\
64.625	0.1014	29.4398785155752\\
64.625	0.102	28.0099180243615\\
64.625	0.1026	26.5799575331475\\
64.625	0.1032	25.1499970419336\\
64.625	0.1038	23.7200365507197\\
64.625	0.1044	22.290076059506\\
64.625	0.105	20.860115568292\\
64.625	0.1056	19.4301550770783\\
64.625	0.1062	18.0001945858644\\
64.625	0.1068	16.5702340946507\\
64.625	0.1074	15.1402736034368\\
64.625	0.108	13.710313112223\\
64.625	0.1086	12.2803526210091\\
64.625	0.1092	10.8503921297952\\
64.625	0.1098	9.42043163858148\\
64.625	0.1104	7.99047114736777\\
64.625	0.111	6.56051065615407\\
64.625	0.1116	5.1305501649399\\
64.625	0.1122	3.7005896737262\\
64.625	0.1128	2.27062918251204\\
64.625	0.1134	0.840668691298333\\
64.625	0.114	-0.589291799915372\\
64.625	0.1146	-2.01925229112908\\
64.625	0.1152	-3.44921278234301\\
64.625	0.1158	-4.87917327355672\\
64.625	0.1164	-6.30913376477088\\
64.625	0.117	-7.73909425598436\\
64.625	0.1176	-9.16905474719852\\
64.625	0.1182	-10.5990152384122\\
64.625	0.1188	-12.0289757296259\\
64.625	0.1194	-13.4589362208399\\
64.625	0.12	-14.8888967120536\\
64.625	0.1206	-16.3188572032675\\
64.625	0.1212	-17.7488176944817\\
64.625	0.1218	-19.1787781856951\\
64.625	0.1224	-20.6087386769093\\
64.625	0.123	-22.038699168123\\
65	0.093	50.4768201042648\\
65	0.0936	49.1044803303114\\
65	0.0942	47.7321405563578\\
65	0.0948	46.3598007824044\\
65	0.0954	44.987461008451\\
65	0.096	43.6151212344976\\
65	0.0966	42.2427814605439\\
65	0.0972	40.8704416865905\\
65	0.0978	39.4981019126369\\
65	0.0984	38.1257621386835\\
65	0.099	36.7534223647301\\
65	0.0996	35.3810825907767\\
65	0.1002	34.0087428168231\\
65	0.1008	32.6364030428697\\
65	0.1014	31.2640632689161\\
65	0.102	29.8917234949629\\
65	0.1026	28.5193837210093\\
65	0.1032	27.1470439470559\\
65	0.1038	25.7747041731022\\
65	0.1044	24.4023643991488\\
65	0.105	23.0300246251954\\
65	0.1056	21.657684851242\\
65	0.1062	20.2853450772884\\
65	0.1068	18.913005303335\\
65	0.1074	17.5406655293814\\
65	0.108	16.1683257554282\\
65	0.1086	14.7959859814746\\
65	0.1092	13.423646207521\\
65	0.1098	12.0513064335676\\
65	0.1104	10.6789666596142\\
65	0.111	9.30662688566099\\
65	0.1116	7.93428711170714\\
65	0.1122	6.56194733775396\\
65	0.1128	5.18960756380011\\
65	0.1134	3.81726778984671\\
65	0.114	2.44492801589331\\
65	0.1146	1.07258824194014\\
65	0.1152	-0.299751532013488\\
65	0.1158	-1.67209130596689\\
65	0.1164	-3.04443107992074\\
65	0.117	-4.41677085387391\\
65	0.1176	-5.78911062782754\\
65	0.1182	-7.16145040178094\\
65	0.1188	-8.53379017573434\\
65	0.1194	-9.90612994968797\\
65	0.12	-11.2784697236414\\
65	0.1206	-12.6508094975945\\
65	0.1212	-14.0231492715484\\
65	0.1218	-15.3954890455016\\
65	0.1224	-16.7678288194552\\
65	0.123	-18.140168593409\\
65.375	0.093	51.4943148159605\\
65.375	0.0936	50.1795957592674\\
65.375	0.0942	48.8648767025743\\
65.375	0.0948	47.5501576458812\\
65.375	0.0954	46.2354385891881\\
65.375	0.096	44.920719532495\\
65.375	0.0966	43.6060004758019\\
65.375	0.0972	42.2912814191088\\
65.375	0.0978	40.9765623624157\\
65.375	0.0984	39.6618433057226\\
65.375	0.099	38.3471242490293\\
65.375	0.0996	37.0324051923365\\
65.375	0.1002	35.7176861356431\\
65.375	0.1008	34.4029670789503\\
65.375	0.1014	33.0882480222569\\
65.375	0.102	31.7735289655641\\
65.375	0.1026	30.4588099088708\\
65.375	0.1032	29.1440908521779\\
65.375	0.1038	27.8293717954846\\
65.375	0.1044	26.5146527387915\\
65.375	0.105	25.1999336820984\\
65.375	0.1056	23.8852146254053\\
65.375	0.1062	22.5704955687122\\
65.375	0.1068	21.2557765120191\\
65.375	0.1074	19.941057455326\\
65.375	0.108	18.6263383986329\\
65.375	0.1086	17.3116193419398\\
65.375	0.1092	15.9969002852465\\
65.375	0.1098	14.6821812285536\\
65.375	0.1104	13.3674621718606\\
65.375	0.111	12.0527431151675\\
65.375	0.1116	10.7380240584741\\
65.375	0.1122	9.42330500178127\\
65.375	0.1128	8.10858594508795\\
65.375	0.1134	6.79386688839486\\
65.375	0.114	5.47914783170199\\
65.375	0.1146	4.1644287750089\\
65.375	0.1152	2.84970971831581\\
65.375	0.1158	1.53499066162271\\
65.375	0.1164	0.220271604929394\\
65.375	0.117	-1.09444745176347\\
65.375	0.1176	-2.40916650845702\\
65.375	0.1182	-3.72388556514989\\
65.375	0.1188	-5.03860462184298\\
65.375	0.1194	-6.35332367853607\\
65.375	0.12	-7.66804273522916\\
65.375	0.1206	-8.98276179192203\\
65.375	0.1212	-10.2974808486156\\
65.375	0.1218	-11.6121999053084\\
65.375	0.1224	-12.9269189620018\\
65.375	0.123	-14.2416380186946\\
65.75	0.093	52.5118095276564\\
65.75	0.0936	51.2547111882238\\
65.75	0.0942	49.9976128487911\\
65.75	0.0948	48.7405145093583\\
65.75	0.0954	47.4834161699255\\
65.75	0.096	46.2263178304929\\
65.75	0.0966	44.9692194910599\\
65.75	0.0972	43.7121211516273\\
65.75	0.0978	42.4550228121946\\
65.75	0.0984	41.197924472762\\
65.75	0.099	39.940826133329\\
65.75	0.0996	38.6837277938964\\
65.75	0.1002	37.4266294544636\\
65.75	0.1008	36.1695311150308\\
65.75	0.1014	34.9124327755981\\
65.75	0.102	33.6553344361655\\
65.75	0.1026	32.3982360967325\\
65.75	0.1032	31.1411377572999\\
65.75	0.1038	29.8840394178671\\
65.75	0.1044	28.6269410784346\\
65.75	0.105	27.3698427390016\\
65.75	0.1056	26.112744399569\\
65.75	0.1062	24.8556460601362\\
65.75	0.1068	23.5985477207034\\
65.75	0.1074	22.3414493812706\\
65.75	0.108	21.0843510418381\\
65.75	0.1086	19.8272527024053\\
65.75	0.1092	18.5701543629723\\
65.75	0.1098	17.3130560235397\\
65.75	0.1104	16.0559576841072\\
65.75	0.111	14.7988593446744\\
65.75	0.1116	13.5417610052414\\
65.75	0.1122	12.284662665809\\
65.75	0.1128	11.027564326376\\
65.75	0.1134	9.77046598694324\\
65.75	0.114	8.51336764751068\\
65.75	0.1146	7.25626930807812\\
65.75	0.1152	5.9991709686451\\
65.75	0.1158	4.74207262921254\\
65.75	0.1164	3.48497428977953\\
65.75	0.117	2.2278759503472\\
65.75	0.1176	0.970777610913956\\
65.75	0.1182	-0.286320728518604\\
65.75	0.1188	-1.54341906795116\\
65.75	0.1194	-2.80051740738418\\
65.75	0.12	-4.05761574681674\\
65.75	0.1206	-5.3147140862493\\
65.75	0.1212	-6.57181242568231\\
65.75	0.1218	-7.82891076511487\\
65.75	0.1224	-9.08600910454788\\
65.75	0.123	-10.3431074439804\\
66.125	0.093	53.5293042393521\\
66.125	0.0936	52.3298266171798\\
66.125	0.0942	51.1303489950074\\
66.125	0.0948	49.9308713728351\\
66.125	0.0954	48.7313937506626\\
66.125	0.096	47.5319161284904\\
66.125	0.0966	46.3324385063179\\
66.125	0.0972	45.1329608841456\\
66.125	0.0978	43.9334832619732\\
66.125	0.0984	42.7340056398009\\
66.125	0.099	41.5345280176284\\
66.125	0.0996	40.3350503954562\\
66.125	0.1002	39.1355727732837\\
66.125	0.1008	37.9360951511114\\
66.125	0.1014	36.736617528939\\
66.125	0.102	35.5371399067667\\
66.125	0.1026	34.3376622845942\\
66.125	0.1032	33.138184662422\\
66.125	0.1038	31.9387070402495\\
66.125	0.1044	30.7392294180772\\
66.125	0.105	29.5397517959047\\
66.125	0.1056	28.3402741737325\\
66.125	0.1062	27.14079655156\\
66.125	0.1068	25.9413189293878\\
66.125	0.1074	24.7418413072153\\
66.125	0.108	23.542363685043\\
66.125	0.1086	22.3428860628705\\
66.125	0.1092	21.1434084406978\\
66.125	0.1098	19.9439308185256\\
66.125	0.1104	18.7444531963533\\
66.125	0.111	17.5449755741811\\
66.125	0.1116	16.3454979520084\\
66.125	0.1122	15.1460203298363\\
66.125	0.1128	13.9465427076636\\
66.125	0.1134	12.7470650854914\\
66.125	0.114	11.5475874633191\\
66.125	0.1146	10.3481098411469\\
66.125	0.1152	9.1486322189744\\
66.125	0.1158	7.94915459680215\\
66.125	0.1164	6.74967697462944\\
66.125	0.117	5.55019935245741\\
66.125	0.1176	4.3507217302847\\
66.125	0.1182	3.15124410811245\\
66.125	0.1188	1.9517664859402\\
66.125	0.1194	0.752288863767717\\
66.125	0.12	-0.447188758404536\\
66.125	0.1206	-1.64666638057679\\
66.125	0.1212	-2.8461440027495\\
66.125	0.1218	-4.04562162492175\\
66.125	0.1224	-5.24509924709446\\
66.125	0.123	-6.44457686926671\\
66.5	0.093	54.546798951048\\
66.5	0.0936	53.4049420461363\\
66.5	0.0942	52.2630851412241\\
66.5	0.0948	51.1212282363122\\
66.5	0.0954	49.9793713314\\
66.5	0.096	48.8375144264883\\
66.5	0.0966	47.6956575215761\\
66.5	0.0972	46.5538006166641\\
66.5	0.0978	45.411943711752\\
66.5	0.0984	44.2700868068403\\
66.5	0.099	43.1282299019281\\
66.5	0.0996	41.9863729970161\\
66.5	0.1002	40.844516092104\\
66.5	0.1008	39.702659187192\\
66.5	0.1014	38.5608022822801\\
66.5	0.102	37.4189453773681\\
66.5	0.1026	36.2770884724559\\
66.5	0.1032	35.135231567544\\
66.5	0.1038	33.993374662632\\
66.5	0.1044	32.8515177577201\\
66.5	0.105	31.7096608528079\\
66.5	0.1056	30.567803947896\\
66.5	0.1062	29.425947042984\\
66.5	0.1068	28.2840901380721\\
66.5	0.1074	27.1422332331599\\
66.5	0.108	26.000376328248\\
66.5	0.1086	24.8585194233358\\
66.5	0.1092	23.7166625184238\\
66.5	0.1098	22.5748056135119\\
66.5	0.1104	21.4329487086\\
66.5	0.111	20.291091803688\\
66.5	0.1116	19.1492348987758\\
66.5	0.1122	18.0073779938641\\
66.5	0.1128	16.8655210889517\\
66.5	0.1134	15.7236641840398\\
66.5	0.114	14.5818072791278\\
66.5	0.1146	13.4399503742161\\
66.5	0.1152	12.2980934693039\\
66.5	0.1158	11.156236564392\\
66.5	0.1164	10.0143796594796\\
66.5	0.117	8.87252275456808\\
66.5	0.1176	7.73066584965568\\
66.5	0.1182	6.58880894474373\\
66.5	0.1188	5.44695203983179\\
66.5	0.1194	4.30509513491984\\
66.5	0.12	3.16323823000789\\
66.5	0.1206	2.02138132509572\\
66.5	0.1212	0.879524420183316\\
66.5	0.1218	-0.262332484728176\\
66.5	0.1224	-1.40418938964058\\
66.5	0.123	-2.54604629455207\\
66.875	0.093	55.5642936627441\\
66.875	0.0936	54.4800574750925\\
66.875	0.0942	53.3958212874409\\
66.875	0.0948	52.3115850997892\\
66.875	0.0954	51.2273489121376\\
66.875	0.096	50.1431127244859\\
66.875	0.0966	49.0588765368343\\
66.875	0.0972	47.9746403491827\\
66.875	0.0978	46.890404161531\\
66.875	0.0984	45.8061679738794\\
66.875	0.099	44.7219317862277\\
66.875	0.0996	43.6376955985761\\
66.875	0.1002	42.5534594109242\\
66.875	0.1008	41.4692232232728\\
66.875	0.1014	40.384987035621\\
66.875	0.102	39.3007508479695\\
66.875	0.1026	38.2165146603177\\
66.875	0.1032	37.1322784726663\\
66.875	0.1038	36.0480422850144\\
66.875	0.1044	34.963806097363\\
66.875	0.105	33.8795699097111\\
66.875	0.1056	32.7953337220597\\
66.875	0.1062	31.7110975344078\\
66.875	0.1068	30.6268613467564\\
66.875	0.1074	29.5426251591045\\
66.875	0.108	28.4583889714531\\
66.875	0.1086	27.3741527838013\\
66.875	0.1092	26.2899165961496\\
66.875	0.1098	25.205680408498\\
66.875	0.1104	24.1214442208466\\
66.875	0.111	23.0372080331949\\
66.875	0.1116	21.9529718455431\\
66.875	0.1122	20.8687356578916\\
66.875	0.1128	19.7844994702398\\
66.875	0.1134	18.7002632825881\\
66.875	0.114	17.6160270949367\\
66.875	0.1146	16.5317909072851\\
66.875	0.1152	15.4475547196334\\
66.875	0.1158	14.3633185319818\\
66.875	0.1164	13.2790823443299\\
66.875	0.117	12.1948461566785\\
66.875	0.1176	11.1106099690267\\
66.875	0.1182	10.026373781375\\
66.875	0.1188	8.9421375937236\\
66.875	0.1194	7.85790140607173\\
66.875	0.12	6.77366521842032\\
66.875	0.1206	5.68942903076868\\
66.875	0.1212	4.60519284311658\\
66.875	0.1218	3.5209566554654\\
66.875	0.1224	2.4367204678133\\
66.875	0.123	1.35248428016212\\
67.25	0.093	56.5817883744398\\
67.25	0.0936	55.5551729040487\\
67.25	0.0942	54.5285574336572\\
67.25	0.0948	53.5019419632661\\
67.25	0.0954	52.4753264928747\\
67.25	0.096	51.4487110224836\\
67.25	0.0966	50.4220955520921\\
67.25	0.0972	49.395480081701\\
67.25	0.0978	48.3688646113096\\
67.25	0.0984	47.3422491409185\\
67.25	0.099	46.3156336705269\\
67.25	0.0996	45.2890182001358\\
67.25	0.1002	44.2624027297445\\
67.25	0.1008	43.2357872593534\\
67.25	0.1014	42.2091717889618\\
67.25	0.102	41.1825563185707\\
67.25	0.1026	40.1559408481794\\
67.25	0.1032	39.1293253777883\\
67.25	0.1038	38.1027099073967\\
67.25	0.1044	37.0760944370056\\
67.25	0.105	36.0494789666143\\
67.25	0.1056	35.0228634962232\\
67.25	0.1062	33.9962480258316\\
67.25	0.1068	32.9696325554405\\
67.25	0.1074	31.9430170850492\\
67.25	0.108	30.9164016146578\\
67.25	0.1086	29.8897861442665\\
67.25	0.1092	28.8631706738752\\
67.25	0.1098	27.8365552034841\\
67.25	0.1104	26.8099397330927\\
67.25	0.111	25.7833242627016\\
67.25	0.1116	24.7567087923101\\
67.25	0.1122	23.7300933219192\\
67.25	0.1128	22.7034778515274\\
67.25	0.1134	21.6768623811363\\
67.25	0.114	20.6502469107452\\
67.25	0.1146	19.6236314403541\\
67.25	0.1152	18.5970159699625\\
67.25	0.1158	17.5704004995714\\
67.25	0.1164	16.5437850291798\\
67.25	0.117	15.517169558789\\
67.25	0.1176	14.4905540883972\\
67.25	0.1182	13.4639386180061\\
67.25	0.1188	12.437323147615\\
67.25	0.1194	11.4107076772234\\
67.25	0.12	10.3840922068325\\
67.25	0.1206	9.35747673644119\\
67.25	0.1212	8.3308612660494\\
67.25	0.1218	7.30424579565852\\
67.25	0.1224	6.27763032526718\\
67.25	0.123	5.25101485487585\\
67.625	0.093	57.5992830861358\\
67.625	0.0936	56.630288333005\\
67.625	0.0942	55.6612935798739\\
67.625	0.0948	54.6922988267431\\
67.625	0.0954	53.7233040736121\\
67.625	0.096	52.7543093204813\\
67.625	0.0966	51.7853145673503\\
67.625	0.0972	50.8163198142195\\
67.625	0.0978	49.8473250610884\\
67.625	0.0984	48.8783303079576\\
67.625	0.099	47.9093355548266\\
67.625	0.0996	46.9403408016958\\
67.625	0.1002	45.9713460485648\\
67.625	0.1008	45.002351295434\\
67.625	0.1014	44.033356542303\\
67.625	0.102	43.0643617891722\\
67.625	0.1026	42.0953670360411\\
67.625	0.1032	41.1263722829103\\
67.625	0.1038	40.1573775297793\\
67.625	0.1044	39.1883827766485\\
67.625	0.105	38.2193880235175\\
67.625	0.1056	37.2503932703867\\
67.625	0.1062	36.2813985172556\\
67.625	0.1068	35.3124037641248\\
67.625	0.1074	34.3434090109938\\
67.625	0.108	33.374414257863\\
67.625	0.1086	32.405419504732\\
67.625	0.1092	31.436424751601\\
67.625	0.1098	30.4674299984702\\
67.625	0.1104	29.4984352453394\\
67.625	0.111	28.5294404922086\\
67.625	0.1116	27.5604457390773\\
67.625	0.1122	26.5914509859467\\
67.625	0.1128	25.6224562328155\\
67.625	0.1134	24.6534614796847\\
67.625	0.114	23.6844667265539\\
67.625	0.1146	22.7154719734231\\
67.625	0.1152	21.746477220292\\
67.625	0.1158	20.7774824671612\\
67.625	0.1164	19.80848771403\\
67.625	0.117	18.8394929608994\\
67.625	0.1176	17.8704982077682\\
67.625	0.1182	16.9015034546373\\
67.625	0.1188	15.9325087015065\\
67.625	0.1194	14.9635139483753\\
67.625	0.12	13.9945191952452\\
67.625	0.1206	13.0255244421137\\
67.625	0.1212	12.0565296889827\\
67.625	0.1218	11.0875349358521\\
67.625	0.1224	10.1185401827206\\
67.625	0.123	9.14954542959049\\
68	0.093	58.6167777978314\\
68	0.0936	57.705403761961\\
68	0.0942	56.7940297260905\\
68	0.0948	55.88265569022\\
68	0.0954	54.9712816543492\\
68	0.096	54.059907618479\\
68	0.0966	53.1485335826083\\
68	0.0972	52.2371595467378\\
68	0.0978	51.325785510867\\
68	0.0984	50.4144114749968\\
68	0.099	49.503037439126\\
68	0.0996	48.5916634032556\\
68	0.1002	47.6802893673848\\
68	0.1008	46.7689153315146\\
68	0.1014	45.8575412956438\\
68	0.102	44.9461672597733\\
68	0.1026	44.0347932239026\\
68	0.1032	43.1234191880324\\
68	0.1038	42.2120451521616\\
68	0.1044	41.3006711162911\\
68	0.105	40.3892970804204\\
68	0.1056	39.4779230445502\\
68	0.1062	38.5665490086794\\
68	0.1068	37.6551749728089\\
68	0.1074	36.7438009369384\\
68	0.108	35.8324269010679\\
68	0.1086	34.9210528651972\\
68	0.1092	34.0096788293265\\
68	0.1098	33.0983047934562\\
68	0.1104	32.1869307575857\\
68	0.111	31.2755567217152\\
68	0.1116	30.3641826858443\\
68	0.1122	29.4528086499743\\
68	0.1128	28.5414346141033\\
68	0.1134	27.6300605782328\\
68	0.114	26.7186865423623\\
68	0.1146	25.8073125064921\\
68	0.1152	24.8959384706213\\
68	0.1158	23.9845644347508\\
68	0.1164	23.0731903988801\\
68	0.117	22.1618163630098\\
68	0.1176	21.2504423271389\\
68	0.1182	20.3390682912684\\
68	0.1188	19.4276942553979\\
68	0.1194	18.5163202195276\\
68	0.12	17.6049461836569\\
68	0.1206	16.6935721477867\\
68	0.1212	15.7821981119159\\
68	0.1218	14.8708240760457\\
68	0.1224	13.9594500401749\\
68	0.123	13.0480760043038\\
68.375	0.093	59.6342725095274\\
68.375	0.0936	58.7805191909174\\
68.375	0.0942	57.926765872307\\
68.375	0.0948	57.073012553697\\
68.375	0.0954	56.2192592350866\\
68.375	0.096	55.3655059164766\\
68.375	0.0966	54.5117525978665\\
68.375	0.0972	53.6579992792563\\
68.375	0.0978	52.8042459606461\\
68.375	0.0984	51.9504926420359\\
68.375	0.099	51.0967393234257\\
68.375	0.0996	50.2429860048155\\
68.375	0.1002	49.3892326862053\\
68.375	0.1008	48.5354793675951\\
68.375	0.1014	47.681726048985\\
68.375	0.102	46.8279727303748\\
68.375	0.1026	45.9742194117646\\
68.375	0.1032	45.1204660931544\\
68.375	0.1038	44.2667127745442\\
68.375	0.1044	43.412959455934\\
68.375	0.105	42.5592061373238\\
68.375	0.1056	41.7054528187136\\
68.375	0.1062	40.8516995001035\\
68.375	0.1068	39.9979461814933\\
68.375	0.1074	39.1441928628831\\
68.375	0.108	38.2904395442729\\
68.375	0.1086	37.4366862256627\\
68.375	0.1092	36.5829329070523\\
68.375	0.1098	35.7291795884423\\
68.375	0.1104	34.8754262698324\\
68.375	0.111	34.0216729512222\\
68.375	0.1116	33.1679196326118\\
68.375	0.1122	32.3141663140018\\
68.375	0.1128	31.4604129953914\\
68.375	0.1134	30.6066596767812\\
68.375	0.114	29.7529063581712\\
68.375	0.1146	28.899153039561\\
68.375	0.1152	28.0453997209509\\
68.375	0.1158	27.1916464023407\\
68.375	0.1164	26.3378930837303\\
68.375	0.117	25.4841397651203\\
68.375	0.1176	24.6303864465099\\
68.375	0.1182	23.7766331278999\\
68.375	0.1188	22.9228798092895\\
68.375	0.1194	22.0691264906795\\
68.375	0.12	21.2153731720691\\
68.375	0.1206	20.3616198534592\\
68.375	0.1212	19.5078665348487\\
68.375	0.1218	18.6541132162388\\
68.375	0.1224	17.8003598976284\\
68.375	0.123	16.9466065790184\\
68.75	0.093	60.6517672212233\\
68.75	0.0936	59.8556346198734\\
68.75	0.0942	59.0595020185235\\
68.75	0.0948	58.2633694171739\\
68.75	0.0954	57.467236815824\\
68.75	0.096	56.6711042144741\\
68.75	0.0966	55.8749716131242\\
68.75	0.0972	55.0788390117746\\
68.75	0.0978	54.2827064104247\\
68.75	0.0984	53.486573809075\\
68.75	0.099	52.6904412077249\\
68.75	0.0996	51.8943086063753\\
68.75	0.1002	51.0981760050254\\
68.75	0.1008	50.3020434036757\\
68.75	0.1014	49.5059108023258\\
68.75	0.102	48.709778200976\\
68.75	0.1026	47.9136455996261\\
68.75	0.1032	47.1175129982764\\
68.75	0.1038	46.3213803969265\\
68.75	0.1044	45.5252477955767\\
68.75	0.105	44.7291151942268\\
68.75	0.1056	43.9329825928771\\
68.75	0.1062	43.1368499915272\\
68.75	0.1068	42.3407173901776\\
68.75	0.1074	41.5445847888275\\
68.75	0.108	40.7484521874778\\
68.75	0.1086	39.9523195861279\\
68.75	0.1092	39.1561869847781\\
68.75	0.1098	38.3600543834282\\
68.75	0.1104	37.5639217820785\\
68.75	0.111	36.7677891807289\\
68.75	0.1116	35.9716565793788\\
68.75	0.1122	35.1755239780293\\
68.75	0.1128	34.379391376679\\
68.75	0.1134	33.5832587753293\\
68.75	0.114	32.7871261739797\\
68.75	0.1146	31.99099357263\\
68.75	0.1152	31.1948609712801\\
68.75	0.1158	30.3987283699303\\
68.75	0.1164	29.6025957685802\\
68.75	0.117	28.8064631672305\\
68.75	0.1176	28.0103305658804\\
68.75	0.1182	27.2141979645307\\
68.75	0.1188	26.4180653631811\\
68.75	0.1194	25.621932761831\\
68.75	0.12	24.8258001604813\\
68.75	0.1206	24.0296675591317\\
68.75	0.1212	23.2335349577816\\
68.75	0.1218	22.4374023564324\\
68.75	0.1224	21.6412697550818\\
68.75	0.123	20.8451371537321\\
69.125	0.093	61.6692619329192\\
69.125	0.0936	60.9307500488299\\
69.125	0.0942	60.1922381647403\\
69.125	0.0948	59.4537262806509\\
69.125	0.0954	58.7152143965614\\
69.125	0.096	57.976702512472\\
69.125	0.0966	57.2381906283824\\
69.125	0.0972	56.4996787442931\\
69.125	0.0978	55.7611668602035\\
69.125	0.0984	55.0226549761142\\
69.125	0.099	54.2841430920246\\
69.125	0.0996	53.5456312079352\\
69.125	0.1002	52.8071193238457\\
69.125	0.1008	52.0686074397563\\
69.125	0.1014	51.3300955556667\\
69.125	0.102	50.5915836715774\\
69.125	0.1026	49.8530717874878\\
69.125	0.1032	49.1145599033985\\
69.125	0.1038	48.3760480193089\\
69.125	0.1044	47.6375361352195\\
69.125	0.105	46.8990242511302\\
69.125	0.1056	46.1605123670408\\
69.125	0.1062	45.4220004829513\\
69.125	0.1068	44.6834885988619\\
69.125	0.1074	43.9449767147723\\
69.125	0.108	43.206464830683\\
69.125	0.1086	42.4679529465934\\
69.125	0.1092	41.7294410625038\\
69.125	0.1098	40.9909291784145\\
69.125	0.1104	40.2524172943251\\
69.125	0.111	39.5139054102358\\
69.125	0.1116	38.775393526146\\
69.125	0.1122	38.0368816420569\\
69.125	0.1128	37.2983697579671\\
69.125	0.1134	36.5598578738777\\
69.125	0.114	35.8213459897884\\
69.125	0.1146	35.082834105699\\
69.125	0.1152	34.3443222216094\\
69.125	0.1158	33.6058103375201\\
69.125	0.1164	32.8672984534303\\
69.125	0.117	32.1287865693414\\
69.125	0.1176	31.3902746852514\\
69.125	0.1182	30.6517628011625\\
69.125	0.1188	29.9132509170727\\
69.125	0.1194	29.1747390329833\\
69.125	0.12	28.4362271488935\\
69.125	0.1206	27.6977152648046\\
69.125	0.1212	26.9592033807148\\
69.125	0.1218	26.2206914966259\\
69.125	0.1224	25.4821796125361\\
69.125	0.123	24.7436677284463\\
69.5	0.093	62.6867566446149\\
69.5	0.0936	62.0058654777858\\
69.5	0.0942	61.3249743109568\\
69.5	0.0948	60.6440831441278\\
69.5	0.0954	59.9631919772985\\
69.5	0.096	59.2823008104695\\
69.5	0.0966	58.6014096436404\\
69.5	0.0972	57.9205184768114\\
69.5	0.0978	57.2396273099821\\
69.5	0.0984	56.5587361431533\\
69.5	0.099	55.877844976324\\
69.5	0.0996	55.196953809495\\
69.5	0.1002	54.5160626426657\\
69.5	0.1008	53.8351714758369\\
69.5	0.1014	53.1542803090076\\
69.5	0.102	52.4733891421786\\
69.5	0.1026	51.7924979753495\\
69.5	0.1032	51.1116068085205\\
69.5	0.1038	50.4307156416912\\
69.5	0.1044	49.7498244748624\\
69.5	0.105	49.0689333080331\\
69.5	0.1056	48.3880421412041\\
69.5	0.1062	47.7071509743751\\
69.5	0.1068	47.026259807546\\
69.5	0.1074	46.3453686407167\\
69.5	0.108	45.6644774738877\\
69.5	0.1086	44.9835863070587\\
69.5	0.1092	44.3026951402294\\
69.5	0.1098	43.6218039734003\\
69.5	0.1104	42.9409128065715\\
69.5	0.111	42.2600216397425\\
69.5	0.1116	41.579130472913\\
69.5	0.1122	40.8982393060844\\
69.5	0.1128	40.2173481392549\\
69.5	0.1134	39.5364569724259\\
69.5	0.114	38.8555658055968\\
69.5	0.1146	38.174674638768\\
69.5	0.1152	37.4937834719387\\
69.5	0.1158	36.8128923051097\\
69.5	0.1164	36.1320011382802\\
69.5	0.117	35.4511099714516\\
69.5	0.1176	34.7702188046223\\
69.5	0.1182	34.0893276377928\\
69.5	0.1188	33.4084364709643\\
69.5	0.1194	32.7275453041348\\
69.5	0.12	32.0466541373062\\
69.5	0.1206	31.3657629704767\\
69.5	0.1212	30.6848718036476\\
69.5	0.1218	30.0039806368186\\
69.5	0.1224	29.3230894699891\\
69.5	0.123	28.6421983031605\\
69.875	0.093	63.7042513563108\\
69.875	0.0936	63.0809809067423\\
69.875	0.0942	62.4577104571733\\
69.875	0.0948	61.8344400076048\\
69.875	0.0954	61.2111695580359\\
69.875	0.096	60.5878991084674\\
69.875	0.0966	59.9646286588984\\
69.875	0.0972	59.3413582093299\\
69.875	0.0978	58.7180877597611\\
69.875	0.0984	58.0948173101924\\
69.875	0.099	57.4715468606237\\
69.875	0.0996	56.8482764110549\\
69.875	0.1002	56.2250059614862\\
69.875	0.1008	55.6017355119175\\
69.875	0.1014	54.9784650623487\\
69.875	0.102	54.35519461278\\
69.875	0.1026	53.7319241632113\\
69.875	0.1032	53.1086537136428\\
69.875	0.1038	52.4853832640738\\
69.875	0.1044	51.8621128145053\\
69.875	0.105	51.2388423649363\\
69.875	0.1056	50.6155719153678\\
69.875	0.1062	49.9923014657988\\
69.875	0.1068	49.3690310162303\\
69.875	0.1074	48.7457605666614\\
69.875	0.108	48.1224901170929\\
69.875	0.1086	47.4992196675241\\
69.875	0.1092	46.8759492179552\\
69.875	0.1098	46.2526787683867\\
69.875	0.1104	45.6294083188179\\
69.875	0.111	45.0061378692494\\
69.875	0.1116	44.3828674196802\\
69.875	0.1122	43.7595969701119\\
69.875	0.1128	43.136326520543\\
69.875	0.1134	42.5130560709742\\
69.875	0.114	41.8897856214057\\
69.875	0.1146	41.266515171837\\
69.875	0.1152	40.6432447222683\\
69.875	0.1158	40.0199742726995\\
69.875	0.1164	39.3967038231303\\
69.875	0.117	38.7734333735621\\
69.875	0.1176	38.1501629239933\\
69.875	0.1182	37.5268924744246\\
69.875	0.1188	36.9036220248558\\
69.875	0.1194	36.2803515752871\\
69.875	0.12	35.6570811257184\\
69.875	0.1206	35.0338106761496\\
69.875	0.1212	34.4105402265809\\
69.875	0.1218	33.7872697770122\\
69.875	0.1224	33.1639993274434\\
69.875	0.123	32.5407288778747\\
70.25	0.093	64.7217460680067\\
70.25	0.0936	64.1560963356985\\
70.25	0.0942	63.5904466033901\\
70.25	0.0948	63.0247968710819\\
70.25	0.0954	62.4591471387732\\
70.25	0.096	61.893497406465\\
70.25	0.0966	61.3278476741566\\
70.25	0.0972	60.7621979418484\\
70.25	0.0978	60.19654820954\\
70.25	0.0984	59.6308984772318\\
70.25	0.099	59.0652487449231\\
70.25	0.0996	58.4995990126149\\
70.25	0.1002	57.9339492803065\\
70.25	0.1008	57.3682995479983\\
70.25	0.1014	56.8026498156898\\
70.25	0.102	56.2370000833814\\
70.25	0.1026	55.671350351073\\
70.25	0.1032	55.1057006187648\\
70.25	0.1038	54.5400508864564\\
70.25	0.1044	53.9744011541482\\
70.25	0.105	53.4087514218397\\
70.25	0.1056	52.8431016895313\\
70.25	0.1062	52.2774519572229\\
70.25	0.1068	51.7118022249147\\
70.25	0.1074	51.1461524926062\\
70.25	0.108	50.580502760298\\
70.25	0.1086	50.0148530279896\\
70.25	0.1092	49.4492032956809\\
70.25	0.1098	48.8835535633727\\
70.25	0.1104	48.3179038310645\\
70.25	0.111	47.7522540987563\\
70.25	0.1116	47.1866043664477\\
70.25	0.1122	46.6209546341395\\
70.25	0.1128	46.0553049018308\\
70.25	0.1134	45.4896551695226\\
70.25	0.114	44.9240054372142\\
70.25	0.1146	44.3583557049062\\
70.25	0.1152	43.7927059725978\\
70.25	0.1158	43.2270562402894\\
70.25	0.1164	42.6614065079807\\
70.25	0.117	42.0957567756727\\
70.25	0.1176	41.5301070433638\\
70.25	0.1182	40.9644573110559\\
70.25	0.1188	40.3988075787474\\
70.25	0.1194	39.8331578464395\\
70.25	0.12	39.2675081141306\\
70.25	0.1206	38.7018583818226\\
70.25	0.1212	38.1362086495142\\
70.25	0.1218	37.5705589172062\\
70.25	0.1224	37.0049091848973\\
70.25	0.123	36.4392594525889\\
70.625	0.093	65.7392407797024\\
70.625	0.0936	65.2312117646545\\
70.625	0.0942	64.7231827496066\\
70.625	0.0948	64.2151537345587\\
70.625	0.0954	63.7071247195106\\
70.625	0.096	63.1990957044627\\
70.625	0.0966	62.6910666894146\\
70.625	0.0972	62.1830376743667\\
70.625	0.0978	61.6750086593186\\
70.625	0.0984	61.1669796442707\\
70.625	0.099	60.6589506292225\\
70.625	0.0996	60.1509216141746\\
70.625	0.1002	59.6428925991265\\
70.625	0.1008	59.1348635840786\\
70.625	0.1014	58.6268345690307\\
70.625	0.102	58.1188055539828\\
70.625	0.1026	57.6107765389347\\
70.625	0.1032	57.1027475238868\\
70.625	0.1038	56.5947185088387\\
70.625	0.1044	56.0866894937908\\
70.625	0.105	55.5786604787427\\
70.625	0.1056	55.0706314636948\\
70.625	0.1062	54.5626024486467\\
70.625	0.1068	54.0545734335988\\
70.625	0.1074	53.5465444185506\\
70.625	0.108	53.0385154035027\\
70.625	0.1086	52.5304863884546\\
70.625	0.1092	52.0224573734065\\
70.625	0.1098	51.5144283583588\\
70.625	0.1104	51.0063993433109\\
70.625	0.111	50.498370328263\\
70.625	0.1116	49.9903413132147\\
70.625	0.1122	49.482312298167\\
70.625	0.1128	48.9742832831187\\
70.625	0.1134	48.466254268071\\
70.625	0.114	47.9582252530229\\
70.625	0.1146	47.4501962379752\\
70.625	0.1152	46.9421672229271\\
70.625	0.1158	46.434138207879\\
70.625	0.1164	45.9261091928311\\
70.625	0.117	45.4180801777834\\
70.625	0.1176	44.9100511627348\\
70.625	0.1182	44.4020221476867\\
70.625	0.1188	43.893993132639\\
70.625	0.1194	43.3859641175904\\
70.625	0.12	42.8779351025432\\
70.625	0.1206	42.3699060874951\\
70.625	0.1212	41.8618770724465\\
70.625	0.1218	41.3538480573989\\
70.625	0.1224	40.8458190423507\\
70.625	0.123	40.3377900273031\\
71	0.093	66.7567354913986\\
71	0.0936	66.306327193611\\
71	0.0942	65.8559188958232\\
71	0.0948	65.4055105980358\\
71	0.0954	64.955102300248\\
71	0.096	64.5046940024604\\
71	0.0966	64.0542857046728\\
71	0.0972	63.6038774068852\\
71	0.0978	63.1534691090974\\
71	0.0984	62.70306081131\\
71	0.099	62.2526525135222\\
71	0.0996	61.8022442157346\\
71	0.1002	61.351835917947\\
71	0.1008	60.9014276201594\\
71	0.1014	60.4510193223716\\
71	0.102	60.0006110245843\\
71	0.1026	59.5502027267964\\
71	0.1032	59.0997944290089\\
71	0.1038	58.6493861312213\\
71	0.1044	58.1989778334337\\
71	0.105	57.7485695356459\\
71	0.1056	57.2981612378585\\
71	0.1062	56.8477529400707\\
71	0.1068	56.3973446422831\\
71	0.1074	55.9469363444955\\
71	0.108	55.4965280467079\\
71	0.1086	55.0461197489201\\
71	0.1092	54.5957114511325\\
71	0.1098	54.1453031533449\\
71	0.1104	53.6948948555573\\
71	0.111	53.24448655777\\
71	0.1116	52.7940782599819\\
71	0.1122	52.3436699621946\\
71	0.1128	51.8932616644065\\
71	0.1134	51.4428533666191\\
71	0.114	50.9924450688318\\
71	0.1146	50.542036771044\\
71	0.1152	50.0916284732566\\
71	0.1158	49.6412201754688\\
71	0.1164	49.190811877681\\
71	0.117	48.7404035798936\\
71	0.1176	48.2899952821058\\
71	0.1182	47.839586984318\\
71	0.1188	47.3891786865306\\
71	0.1194	46.9387703887428\\
71	0.12	46.4883620909554\\
71	0.1206	46.0379537931681\\
71	0.1212	45.5875454953798\\
71	0.1218	45.1371371975924\\
71	0.1224	44.6867288998046\\
71	0.123	44.2363206020173\\
71.375	0.093	67.7742302030942\\
71.375	0.0936	67.381442622567\\
71.375	0.0942	66.9886550420397\\
71.375	0.0948	66.5958674615126\\
71.375	0.0954	66.2030798809851\\
71.375	0.096	65.8102923004581\\
71.375	0.0966	65.4175047199305\\
71.375	0.0972	65.0247171394035\\
71.375	0.0978	64.6319295588762\\
71.375	0.0984	64.2391419783489\\
71.375	0.099	63.8463543978216\\
71.375	0.0996	63.4535668172944\\
71.375	0.1002	63.0607792367671\\
71.375	0.1008	62.66799165624\\
71.375	0.1014	62.2752040757125\\
71.375	0.102	61.8824164951855\\
71.375	0.1026	61.4896289146579\\
71.375	0.1032	61.0968413341309\\
71.375	0.1038	60.7040537536036\\
71.375	0.1044	60.3112661730763\\
71.375	0.105	59.918478592549\\
71.375	0.1056	59.5256910120218\\
71.375	0.1062	59.1329034314945\\
71.375	0.1068	58.7401158509674\\
71.375	0.1074	58.3473282704399\\
71.375	0.108	57.9545406899128\\
71.375	0.1086	57.5617531093853\\
71.375	0.1092	57.1689655288581\\
71.375	0.1098	56.7761779483308\\
71.375	0.1104	56.3833903678037\\
71.375	0.111	55.9906027872767\\
71.375	0.1116	55.5978152067489\\
71.375	0.1122	55.2050276262221\\
71.375	0.1128	54.8122400456946\\
71.375	0.1134	54.4194524651675\\
71.375	0.114	54.02666488464\\
71.375	0.1146	53.633877304113\\
71.375	0.1152	53.2410897235854\\
71.375	0.1158	52.8483021430584\\
71.375	0.1164	52.4555145625309\\
71.375	0.117	52.0627269820038\\
71.375	0.1176	51.6699394014763\\
71.375	0.1182	51.2771518209493\\
71.375	0.1188	50.8843642404222\\
71.375	0.1194	50.4915766598947\\
71.375	0.12	50.0987890793676\\
71.375	0.1206	49.7060014988406\\
71.375	0.1212	49.3132139183126\\
71.375	0.1218	48.920426337786\\
71.375	0.1224	48.5276387572585\\
71.375	0.123	48.134851176731\\
71.75	0.093	68.7917249147902\\
71.75	0.0936	68.4565580515234\\
71.75	0.0942	68.1213911882564\\
71.75	0.0948	67.7862243249895\\
71.75	0.0954	67.4510574617225\\
71.75	0.096	67.1158905984557\\
71.75	0.0966	66.7807237351888\\
71.75	0.0972	66.445556871922\\
71.75	0.0978	66.110390008655\\
71.75	0.0984	65.7752231453883\\
71.75	0.099	65.4400562821213\\
71.75	0.0996	65.1048894188543\\
71.75	0.1002	64.7697225555874\\
71.75	0.1008	64.4345556923206\\
71.75	0.1014	64.0993888290536\\
71.75	0.102	63.7642219657869\\
71.75	0.1026	63.4290551025199\\
71.75	0.1032	63.0938882392531\\
71.75	0.1038	62.7587213759859\\
71.75	0.1044	62.4235545127192\\
71.75	0.105	62.0883876494522\\
71.75	0.1056	61.7532207861855\\
71.75	0.1062	61.4180539229185\\
71.75	0.1068	61.0828870596517\\
71.75	0.1074	60.7477201963845\\
71.75	0.108	60.4125533331178\\
71.75	0.1086	60.0773864698508\\
71.75	0.1092	59.7422196065838\\
71.75	0.1098	59.4070527433171\\
71.75	0.1104	59.0718858800503\\
71.75	0.111	58.7367190167836\\
71.75	0.1116	58.4015521535164\\
71.75	0.1122	58.0663852902496\\
71.75	0.1128	57.7312184269824\\
71.75	0.1134	57.3960515637157\\
71.75	0.114	57.0608847004489\\
71.75	0.1146	56.7257178371824\\
71.75	0.1152	56.3905509739147\\
71.75	0.1158	56.0553841106484\\
71.75	0.1164	55.7202172473812\\
71.75	0.117	55.3850503841149\\
71.75	0.1176	55.0498835208477\\
71.75	0.1182	54.7147166575801\\
71.75	0.1188	54.3795497943138\\
71.75	0.1194	54.0443829310466\\
71.75	0.12	53.7092160677803\\
71.75	0.1206	53.3740492045131\\
71.75	0.1212	53.0388823412459\\
71.75	0.1218	52.7037154779791\\
71.75	0.1224	52.3685486147119\\
71.75	0.123	52.0333817514456\\
72.125	0.093	69.8092196264859\\
72.125	0.0936	69.5316734804794\\
72.125	0.0942	69.2541273344727\\
72.125	0.0948	68.9765811884663\\
72.125	0.0954	68.6990350424599\\
72.125	0.096	68.4214888964534\\
72.125	0.0966	68.1439427504467\\
72.125	0.0972	67.8663966044403\\
72.125	0.0978	67.5888504584336\\
72.125	0.0984	67.3113043124272\\
72.125	0.099	67.0337581664205\\
72.125	0.0996	66.7562120204141\\
72.125	0.1002	66.4786658744074\\
72.125	0.1008	66.2011197284012\\
72.125	0.1014	65.9235735823945\\
72.125	0.102	65.6460274363881\\
72.125	0.1026	65.3684812903814\\
72.125	0.1032	65.090935144375\\
72.125	0.1038	64.8133889983683\\
72.125	0.1044	64.5358428523618\\
72.125	0.105	64.2582967063552\\
72.125	0.1056	63.980750560349\\
72.125	0.1062	63.7032044143423\\
72.125	0.1068	63.4256582683358\\
72.125	0.1074	63.1481121223292\\
72.125	0.108	62.8705659763227\\
72.125	0.1086	62.5930198303161\\
72.125	0.1092	62.3154736843094\\
72.125	0.1098	62.0379275383029\\
72.125	0.1104	61.7603813922965\\
72.125	0.111	61.4828352462903\\
72.125	0.1116	61.2052891002834\\
72.125	0.1122	60.9277429542769\\
72.125	0.1128	60.65019680827\\
72.125	0.1134	60.372650662264\\
72.125	0.114	60.0951045162576\\
72.125	0.1146	59.8175583702509\\
72.125	0.1152	59.5400122242445\\
72.125	0.1158	59.2624660782376\\
72.125	0.1164	58.9849199322307\\
72.125	0.117	58.7073737862247\\
72.125	0.1176	58.4298276402178\\
72.125	0.1182	58.1522814942118\\
72.125	0.1188	57.8747353482049\\
72.125	0.1194	57.5971892021989\\
72.125	0.12	57.319643056192\\
72.125	0.1206	57.042096910186\\
72.125	0.1212	56.7645507641791\\
72.125	0.1218	56.4870046181727\\
72.125	0.1224	56.2094584721658\\
72.125	0.123	55.9319123261589\\
72.5	0.093	70.8267143381818\\
72.5	0.0936	70.6067889094359\\
72.5	0.0942	70.3868634806895\\
72.5	0.0948	70.1669380519434\\
72.5	0.0954	69.9470126231972\\
72.5	0.096	69.7270871944511\\
72.5	0.0966	69.507161765705\\
72.5	0.0972	69.2872363369588\\
72.5	0.0978	69.0673109082124\\
72.5	0.0984	68.8473854794665\\
72.5	0.099	68.6274600507202\\
72.5	0.0996	68.407534621974\\
72.5	0.1002	68.1876091932279\\
72.5	0.1008	67.9676837644818\\
72.5	0.1014	67.7477583357356\\
72.5	0.102	67.5278329069895\\
72.5	0.1026	67.3079074782431\\
72.5	0.1032	67.0879820494972\\
72.5	0.1038	66.8680566207508\\
72.5	0.1044	66.6481311920047\\
72.5	0.105	66.4282057632586\\
72.5	0.1056	66.2082803345124\\
72.5	0.1062	65.9883549057663\\
72.5	0.1068	65.7684294770202\\
72.5	0.1074	65.5485040482738\\
72.5	0.108	65.3285786195279\\
72.5	0.1086	65.1086531907815\\
72.5	0.1092	64.8887277620352\\
72.5	0.1098	64.6688023332892\\
72.5	0.1104	64.4488769045433\\
72.5	0.111	64.2289514757972\\
72.5	0.1116	64.0090260470506\\
72.5	0.1122	63.7891006183049\\
72.5	0.1128	63.5691751895583\\
72.5	0.1134	63.3492497608122\\
72.5	0.114	63.1293243320661\\
72.5	0.1146	62.9093989033199\\
72.5	0.1152	62.6894734745738\\
72.5	0.1158	62.4695480458277\\
72.5	0.1164	62.2496226170811\\
72.5	0.117	62.0296971883354\\
72.5	0.1176	61.8097717595888\\
72.5	0.1182	61.5898463308431\\
72.5	0.1188	61.369920902097\\
72.5	0.1194	61.1499954733504\\
72.5	0.12	60.9300700446047\\
72.5	0.1206	60.7101446158586\\
72.5	0.1212	60.490219187112\\
72.5	0.1218	60.2702937583663\\
72.5	0.1224	60.0503683296197\\
72.5	0.123	59.8304429008735\\
72.875	0.093	71.8442090498775\\
72.875	0.0936	71.6819043383919\\
72.875	0.0942	71.519599626906\\
72.875	0.0948	71.3572949154202\\
72.875	0.0954	71.1949902039344\\
72.875	0.096	71.0326854924488\\
72.875	0.0966	70.8703807809627\\
72.875	0.0972	70.7080760694771\\
72.875	0.0978	70.5457713579913\\
72.875	0.0984	70.3834666465054\\
72.875	0.099	70.2211619350196\\
72.875	0.0996	70.0588572235338\\
72.875	0.1002	69.8965525120479\\
72.875	0.1008	69.7342478005623\\
72.875	0.1014	69.5719430890763\\
72.875	0.102	69.4096383775907\\
72.875	0.1026	69.2473336661049\\
72.875	0.1032	69.085028954619\\
72.875	0.1038	68.9227242431332\\
72.875	0.1044	68.7604195316474\\
72.875	0.105	68.5981148201615\\
72.875	0.1056	68.4358101086759\\
72.875	0.1062	68.2735053971899\\
72.875	0.1068	68.1112006857043\\
72.875	0.1074	67.9488959742184\\
72.875	0.108	67.7865912627326\\
72.875	0.1086	67.6242865512468\\
72.875	0.1092	67.4619818397609\\
72.875	0.1098	67.2996771282751\\
72.875	0.1104	67.1373724167893\\
72.875	0.111	66.9750677053039\\
72.875	0.1116	66.8127629938176\\
72.875	0.1122	66.6504582823322\\
72.875	0.1128	66.488153570846\\
72.875	0.1134	66.3258488593606\\
72.875	0.114	66.1635441478747\\
72.875	0.1146	66.0012394363889\\
72.875	0.1152	65.8389347249031\\
72.875	0.1158	65.6766300134173\\
72.875	0.1164	65.5143253019314\\
72.875	0.117	65.3520205904456\\
72.875	0.1176	65.1897158789598\\
72.875	0.1182	65.0274111674739\\
72.875	0.1188	64.8651064559881\\
72.875	0.1194	64.7028017445023\\
72.875	0.12	64.5404970330164\\
72.875	0.1206	64.3781923215311\\
72.875	0.1212	64.2158876100448\\
72.875	0.1218	64.0535828985594\\
72.875	0.1224	63.8912781870731\\
72.875	0.123	63.7289734755877\\
73.25	0.093	72.8617037615736\\
73.25	0.0936	72.7570197673483\\
73.25	0.0942	72.6523357731226\\
73.25	0.0948	72.5476517788973\\
73.25	0.0954	72.4429677846717\\
73.25	0.096	72.3382837904464\\
73.25	0.0966	72.2335997962209\\
73.25	0.0972	72.1289158019956\\
73.25	0.0978	72.0242318077701\\
73.25	0.0984	71.9195478135448\\
73.25	0.099	71.8148638193193\\
73.25	0.0996	71.7101798250937\\
73.25	0.1002	71.6054958308682\\
73.25	0.1008	71.5008118366429\\
73.25	0.1014	71.3961278424174\\
73.25	0.102	71.2914438481921\\
73.25	0.1026	71.1867598539666\\
73.25	0.1032	71.0820758597413\\
73.25	0.1038	70.9773918655158\\
73.25	0.1044	70.8727078712905\\
73.25	0.105	70.7680238770647\\
73.25	0.1056	70.6633398828394\\
73.25	0.1062	70.5586558886139\\
73.25	0.1068	70.4539718943886\\
73.25	0.1074	70.3492879001631\\
73.25	0.108	70.2446039059378\\
73.25	0.1086	70.1399199117122\\
73.25	0.1092	70.0352359174865\\
73.25	0.1098	69.9305519232614\\
73.25	0.1104	69.8258679290359\\
73.25	0.111	69.7211839348108\\
73.25	0.1116	69.6164999405848\\
73.25	0.1122	69.5118159463598\\
73.25	0.1128	69.4071319521343\\
73.25	0.1134	69.3024479579085\\
73.25	0.114	69.1977639636834\\
73.25	0.1146	69.0930799694579\\
73.25	0.1152	68.9883959752328\\
73.25	0.1158	68.8837119810069\\
73.25	0.1164	68.7790279867813\\
73.25	0.117	68.6743439925563\\
73.25	0.1176	68.5696599983303\\
73.25	0.1182	68.4649760041057\\
73.25	0.1188	68.3602920098797\\
73.25	0.1194	68.2556080156546\\
73.25	0.12	68.1509240214286\\
73.25	0.1206	68.046240027204\\
73.25	0.1212	67.941556032978\\
73.25	0.1218	67.836872038753\\
73.25	0.1224	67.7321880445274\\
73.25	0.123	67.6275040503015\\
73.625	0.093	73.8791984732695\\
73.625	0.0936	73.8321351963045\\
73.625	0.0942	73.7850719193393\\
73.625	0.0948	73.7380086423743\\
73.625	0.0954	73.6909453654091\\
73.625	0.096	73.6438820884443\\
73.625	0.0966	73.5968188114791\\
73.625	0.0972	73.5497555345141\\
73.625	0.0978	73.5026922575489\\
73.625	0.0984	73.4556289805839\\
73.625	0.099	73.4085657036187\\
73.625	0.0996	73.3615024266537\\
73.625	0.1002	73.3144391496887\\
73.625	0.1008	73.2673758727237\\
73.625	0.1014	73.2203125957585\\
73.625	0.102	73.1732493187935\\
73.625	0.1026	73.1261860418283\\
73.625	0.1032	73.0791227648633\\
73.625	0.1038	73.0320594878983\\
73.625	0.1044	72.9849962109333\\
73.625	0.105	72.9379329339681\\
73.625	0.1056	72.8908696570031\\
73.625	0.1062	72.8438063800379\\
73.625	0.1068	72.7967431030729\\
73.625	0.1074	72.7496798261077\\
73.625	0.108	72.7026165491429\\
73.625	0.1086	72.6555532721777\\
73.625	0.1092	72.6084899952125\\
73.625	0.1098	72.5614267182477\\
73.625	0.1104	72.5143634412825\\
73.625	0.111	72.4673001643177\\
73.625	0.1116	72.4202368873521\\
73.625	0.1122	72.3731736103873\\
73.625	0.1128	72.3261103334221\\
73.625	0.1134	72.2790470564573\\
73.625	0.114	72.2319837794921\\
73.625	0.1146	72.1849205025273\\
73.625	0.1152	72.1378572255621\\
73.625	0.1158	72.0907939485969\\
73.625	0.1164	72.0437306716317\\
73.625	0.117	71.9966673946669\\
73.625	0.1176	71.9496041177017\\
73.625	0.1182	71.9025408407365\\
73.625	0.1188	71.8554775637717\\
73.625	0.1194	71.8084142868065\\
73.625	0.12	71.7613510098413\\
73.625	0.1206	71.7142877328765\\
73.625	0.1212	71.6672244559113\\
73.625	0.1218	71.6201611789465\\
73.625	0.1224	71.5730979019809\\
73.625	0.123	71.5260346250161\\
74	0.093	74.8966931849652\\
74	0.0936	74.9072506252605\\
74	0.0942	74.9178080655558\\
74	0.0948	74.9283655058512\\
74	0.0954	74.9389229461465\\
74	0.096	74.9494803864418\\
74	0.0966	74.9600378267371\\
74	0.0972	74.9705952670324\\
74	0.0978	74.9811527073275\\
74	0.0984	74.9917101476231\\
74	0.099	75.0022675879181\\
74	0.0996	75.0128250282135\\
74	0.1002	75.0233824685088\\
74	0.1008	75.0339399088041\\
74	0.1014	75.0444973490994\\
74	0.102	75.0550547893947\\
74	0.1026	75.06561222969\\
74	0.1032	75.0761696699853\\
74	0.1038	75.0867271102804\\
74	0.1044	75.097284550576\\
74	0.105	75.1078419908711\\
74	0.1056	75.1183994311666\\
74	0.1062	75.1289568714617\\
74	0.1068	75.1395143117572\\
74	0.1074	75.1500717520523\\
74	0.108	75.1606291923476\\
74	0.1086	75.1711866326427\\
74	0.1092	75.1817440729383\\
74	0.1098	75.1923015132334\\
74	0.1104	75.2028589535289\\
74	0.111	75.2134163938244\\
74	0.1116	75.2239738341191\\
74	0.1122	75.2345312744151\\
74	0.1128	75.2450887147097\\
74	0.1134	75.2556461550053\\
74	0.114	75.2662035953008\\
74	0.1146	75.2767610355959\\
74	0.1152	75.2873184758914\\
74	0.1158	75.2978759161865\\
74	0.1164	75.3084333564816\\
74	0.117	75.3189907967771\\
74	0.1176	75.3295482370722\\
74	0.1182	75.3401056773678\\
74	0.1188	75.3506631176629\\
74	0.1194	75.3612205579584\\
74	0.12	75.3717779982535\\
74	0.1206	75.382335438549\\
74	0.1212	75.3928928788441\\
74	0.1218	75.4034503191397\\
74	0.1224	75.4140077594348\\
74	0.123	75.4245651997298\\
};
\end{axis}

\begin{axis}[%
width=5.011742cm,
height=3.595207cm,
at={(6.594397cm,0cm)},
scale only axis,
xmin=55,
xmax=75,
tick align=outside,
xlabel={$L_{cut}$},
xmajorgrids,
ymin=0.09,
ymax=0.13,
ylabel={$D_{rlx}$},
ymajorgrids,
zmin=5.67589105085895,
zmax=15.4680684467611,
zlabel={$x_3,x_4$},
zmajorgrids,
view={-140}{50},
legend style={at={(1.03,1)},anchor=north west,legend cell align=left,align=left,draw=white!15!black}
]
\addplot3[only marks,mark=*,mark options={},mark size=1.5000pt,color=mycolor1] plot table[row sep=crcr,]{%
74	0.123	15.4680684467597\\
72	0.113	12.0981268654473\\
61	0.095	6.35969255839677\\
56	0.093	5.67589105085895\\
};
\addplot3[only marks,mark=*,mark options={},mark size=1.5000pt,color=black] plot table[row sep=crcr,]{%
69	0.104	9.19562768308651\\
};

\addplot3[%
surf,
opacity=0.7,
shader=interp,
colormap={mymap}{[1pt] rgb(0pt)=(0.0901961,0.239216,0.0745098); rgb(1pt)=(0.0945149,0.242058,0.0739522); rgb(2pt)=(0.0988592,0.244894,0.0733566); rgb(3pt)=(0.103229,0.247724,0.0727241); rgb(4pt)=(0.107623,0.250549,0.0720557); rgb(5pt)=(0.112043,0.253367,0.0713525); rgb(6pt)=(0.116487,0.25618,0.0706154); rgb(7pt)=(0.120956,0.258986,0.0698456); rgb(8pt)=(0.125449,0.261787,0.0690441); rgb(9pt)=(0.129967,0.264581,0.0682118); rgb(10pt)=(0.134508,0.26737,0.06735); rgb(11pt)=(0.139074,0.270152,0.0664596); rgb(12pt)=(0.143663,0.272929,0.0655416); rgb(13pt)=(0.148275,0.275699,0.0645971); rgb(14pt)=(0.152911,0.278463,0.0636271); rgb(15pt)=(0.15757,0.281221,0.0626328); rgb(16pt)=(0.162252,0.283973,0.0616151); rgb(17pt)=(0.166957,0.286719,0.060575); rgb(18pt)=(0.171685,0.289458,0.0595136); rgb(19pt)=(0.176434,0.292191,0.0584321); rgb(20pt)=(0.181207,0.294918,0.0573313); rgb(21pt)=(0.186001,0.297639,0.0562123); rgb(22pt)=(0.190817,0.300353,0.0550763); rgb(23pt)=(0.195655,0.303061,0.0539242); rgb(24pt)=(0.200514,0.305763,0.052757); rgb(25pt)=(0.205395,0.308459,0.0515759); rgb(26pt)=(0.210296,0.311149,0.0503624); rgb(27pt)=(0.215212,0.313846,0.0490067); rgb(28pt)=(0.220142,0.316548,0.0475043); rgb(29pt)=(0.22509,0.319254,0.0458704); rgb(30pt)=(0.230056,0.321962,0.0441205); rgb(31pt)=(0.235042,0.324671,0.04227); rgb(32pt)=(0.240048,0.327379,0.0403343); rgb(33pt)=(0.245078,0.330085,0.0383287); rgb(34pt)=(0.250131,0.332786,0.0362688); rgb(35pt)=(0.25521,0.335482,0.0341698); rgb(36pt)=(0.260317,0.33817,0.0320472); rgb(37pt)=(0.265451,0.340849,0.0299163); rgb(38pt)=(0.270616,0.343517,0.0277927); rgb(39pt)=(0.275813,0.346172,0.0256916); rgb(40pt)=(0.281043,0.348814,0.0236284); rgb(41pt)=(0.286307,0.35144,0.0216186); rgb(42pt)=(0.291607,0.354048,0.0196776); rgb(43pt)=(0.296945,0.356637,0.0178207); rgb(44pt)=(0.302322,0.359206,0.0160634); rgb(45pt)=(0.307739,0.361753,0.0144211); rgb(46pt)=(0.313198,0.364275,0.0129091); rgb(47pt)=(0.318701,0.366772,0.0115428); rgb(48pt)=(0.324249,0.369242,0.0103377); rgb(49pt)=(0.329843,0.371682,0.00930909); rgb(50pt)=(0.335485,0.374093,0.00847245); rgb(51pt)=(0.341176,0.376471,0.00784314); rgb(52pt)=(0.346925,0.378826,0.00732741); rgb(53pt)=(0.352735,0.381168,0.00682184); rgb(54pt)=(0.358605,0.383497,0.00632729); rgb(55pt)=(0.364532,0.385812,0.00584464); rgb(56pt)=(0.370516,0.388113,0.00537476); rgb(57pt)=(0.376552,0.390399,0.00491852); rgb(58pt)=(0.38264,0.39267,0.00447681); rgb(59pt)=(0.388777,0.394925,0.00405048); rgb(60pt)=(0.394962,0.397164,0.00364042); rgb(61pt)=(0.401191,0.399386,0.00324749); rgb(62pt)=(0.407464,0.401592,0.00287258); rgb(63pt)=(0.413777,0.40378,0.00251655); rgb(64pt)=(0.420129,0.40595,0.00218028); rgb(65pt)=(0.426518,0.408102,0.00186463); rgb(66pt)=(0.432942,0.410234,0.00157049); rgb(67pt)=(0.439399,0.412348,0.00129873); rgb(68pt)=(0.445885,0.414441,0.00105022); rgb(69pt)=(0.452401,0.416515,0.000825833); rgb(70pt)=(0.458942,0.418567,0.000626441); rgb(71pt)=(0.465508,0.420599,0.00045292); rgb(72pt)=(0.472096,0.422609,0.000306141); rgb(73pt)=(0.478704,0.424596,0.000186979); rgb(74pt)=(0.485331,0.426562,9.63073e-05); rgb(75pt)=(0.491973,0.428504,3.49981e-05); rgb(76pt)=(0.498628,0.430422,3.92506e-06); rgb(77pt)=(0.505323,0.432315,0); rgb(78pt)=(0.512206,0.434168,0); rgb(79pt)=(0.519282,0.435983,0); rgb(80pt)=(0.526529,0.437764,0); rgb(81pt)=(0.533922,0.439512,0); rgb(82pt)=(0.54144,0.441232,0); rgb(83pt)=(0.549059,0.442927,0); rgb(84pt)=(0.556756,0.444599,0); rgb(85pt)=(0.564508,0.446252,0); rgb(86pt)=(0.572292,0.447889,0); rgb(87pt)=(0.580084,0.449514,0); rgb(88pt)=(0.587863,0.451129,0); rgb(89pt)=(0.595604,0.452737,0); rgb(90pt)=(0.603284,0.454343,0); rgb(91pt)=(0.610882,0.455948,0); rgb(92pt)=(0.618373,0.457556,0); rgb(93pt)=(0.625734,0.459171,0); rgb(94pt)=(0.632943,0.460795,0); rgb(95pt)=(0.639976,0.462432,0); rgb(96pt)=(0.64681,0.464084,0); rgb(97pt)=(0.653423,0.465756,0); rgb(98pt)=(0.659791,0.46745,0); rgb(99pt)=(0.665891,0.469169,0); rgb(100pt)=(0.6717,0.470916,0); rgb(101pt)=(0.677195,0.472696,0); rgb(102pt)=(0.682353,0.47451,0); rgb(103pt)=(0.687242,0.476355,0); rgb(104pt)=(0.691952,0.478225,0); rgb(105pt)=(0.696497,0.480118,0); rgb(106pt)=(0.700887,0.482033,0); rgb(107pt)=(0.705134,0.483968,0); rgb(108pt)=(0.709251,0.485921,0); rgb(109pt)=(0.713249,0.487891,0); rgb(110pt)=(0.71714,0.489876,0); rgb(111pt)=(0.720936,0.491875,0); rgb(112pt)=(0.724649,0.493887,0); rgb(113pt)=(0.72829,0.495909,0); rgb(114pt)=(0.731872,0.49794,0); rgb(115pt)=(0.735406,0.499979,0); rgb(116pt)=(0.738904,0.502025,0); rgb(117pt)=(0.742378,0.504075,0); rgb(118pt)=(0.74584,0.506128,0); rgb(119pt)=(0.749302,0.508182,0); rgb(120pt)=(0.752775,0.510237,0); rgb(121pt)=(0.756272,0.51229,0); rgb(122pt)=(0.759804,0.514339,0); rgb(123pt)=(0.763384,0.516385,0); rgb(124pt)=(0.767022,0.518424,0); rgb(125pt)=(0.770731,0.520455,0); rgb(126pt)=(0.774523,0.522478,0); rgb(127pt)=(0.77841,0.524489,0); rgb(128pt)=(0.782391,0.526491,0); rgb(129pt)=(0.786402,0.528496,0); rgb(130pt)=(0.790431,0.530506,0); rgb(131pt)=(0.794478,0.532521,0); rgb(132pt)=(0.798541,0.534539,0); rgb(133pt)=(0.802619,0.53656,0); rgb(134pt)=(0.806712,0.538584,0); rgb(135pt)=(0.81082,0.540609,0); rgb(136pt)=(0.81494,0.542635,0); rgb(137pt)=(0.819074,0.54466,0); rgb(138pt)=(0.823219,0.546686,0); rgb(139pt)=(0.827374,0.548709,0); rgb(140pt)=(0.831541,0.55073,0); rgb(141pt)=(0.835716,0.552749,0); rgb(142pt)=(0.8399,0.554763,0); rgb(143pt)=(0.844092,0.556774,0); rgb(144pt)=(0.848292,0.558779,0); rgb(145pt)=(0.852497,0.560778,0); rgb(146pt)=(0.856708,0.562771,0); rgb(147pt)=(0.860924,0.564756,0); rgb(148pt)=(0.865143,0.566733,0); rgb(149pt)=(0.869366,0.568701,0); rgb(150pt)=(0.873592,0.57066,0); rgb(151pt)=(0.877819,0.572608,0); rgb(152pt)=(0.882047,0.574545,0); rgb(153pt)=(0.886275,0.576471,0); rgb(154pt)=(0.890659,0.578362,0); rgb(155pt)=(0.895333,0.580203,0); rgb(156pt)=(0.900258,0.581999,0); rgb(157pt)=(0.905397,0.583755,0); rgb(158pt)=(0.910711,0.585479,0); rgb(159pt)=(0.916164,0.587176,0); rgb(160pt)=(0.921717,0.588852,0); rgb(161pt)=(0.927333,0.590513,0); rgb(162pt)=(0.932974,0.592166,0); rgb(163pt)=(0.938602,0.593815,0); rgb(164pt)=(0.94418,0.595468,0); rgb(165pt)=(0.949669,0.59713,0); rgb(166pt)=(0.955033,0.598808,0); rgb(167pt)=(0.960233,0.600507,0); rgb(168pt)=(0.965232,0.602233,0); rgb(169pt)=(0.969992,0.603992,0); rgb(170pt)=(0.974475,0.605791,0); rgb(171pt)=(0.978643,0.607636,0); rgb(172pt)=(0.98246,0.609532,0); rgb(173pt)=(0.985886,0.611486,0); rgb(174pt)=(0.988885,0.613503,0); rgb(175pt)=(0.991419,0.61559,0); rgb(176pt)=(0.99345,0.617753,0); rgb(177pt)=(0.99494,0.619997,0); rgb(178pt)=(0.995851,0.622329,0); rgb(179pt)=(0.996226,0.624763,0); rgb(180pt)=(0.996512,0.627352,0); rgb(181pt)=(0.996788,0.630095,0); rgb(182pt)=(0.997053,0.632982,0); rgb(183pt)=(0.997308,0.636004,0); rgb(184pt)=(0.997552,0.639152,0); rgb(185pt)=(0.997785,0.642416,0); rgb(186pt)=(0.998006,0.645786,0); rgb(187pt)=(0.998217,0.649253,0); rgb(188pt)=(0.998416,0.652807,0); rgb(189pt)=(0.998605,0.656439,0); rgb(190pt)=(0.998781,0.660138,0); rgb(191pt)=(0.998946,0.663897,0); rgb(192pt)=(0.9991,0.667704,0); rgb(193pt)=(0.999242,0.67155,0); rgb(194pt)=(0.999372,0.675427,0); rgb(195pt)=(0.99949,0.679323,0); rgb(196pt)=(0.999596,0.68323,0); rgb(197pt)=(0.99969,0.687139,0); rgb(198pt)=(0.999771,0.691039,0); rgb(199pt)=(0.999841,0.694921,0); rgb(200pt)=(0.999898,0.698775,0); rgb(201pt)=(0.999942,0.702592,0); rgb(202pt)=(0.999974,0.706363,0); rgb(203pt)=(0.999994,0.710077,0); rgb(204pt)=(1,0.713725,0); rgb(205pt)=(1,0.717341,0); rgb(206pt)=(1,0.720963,0); rgb(207pt)=(1,0.724591,0); rgb(208pt)=(1,0.728226,0); rgb(209pt)=(1,0.731867,0); rgb(210pt)=(1,0.735514,0); rgb(211pt)=(1,0.739167,0); rgb(212pt)=(1,0.742827,0); rgb(213pt)=(1,0.746493,0); rgb(214pt)=(1,0.750165,0); rgb(215pt)=(1,0.753843,0); rgb(216pt)=(1,0.757527,0); rgb(217pt)=(1,0.761217,0); rgb(218pt)=(1,0.764913,0); rgb(219pt)=(1,0.768615,0); rgb(220pt)=(1,0.772324,0); rgb(221pt)=(1,0.776038,0); rgb(222pt)=(1,0.779758,0); rgb(223pt)=(1,0.783484,0); rgb(224pt)=(1,0.787215,0); rgb(225pt)=(1,0.790953,0); rgb(226pt)=(1,0.794696,0); rgb(227pt)=(1,0.798445,0); rgb(228pt)=(1,0.8022,0); rgb(229pt)=(1,0.805961,0); rgb(230pt)=(1,0.809727,0); rgb(231pt)=(1,0.8135,0); rgb(232pt)=(1,0.817278,0); rgb(233pt)=(1,0.821063,0); rgb(234pt)=(1,0.824854,0); rgb(235pt)=(1,0.828652,0); rgb(236pt)=(1,0.832455,0); rgb(237pt)=(1,0.836265,0); rgb(238pt)=(1,0.840081,0); rgb(239pt)=(1,0.843903,0); rgb(240pt)=(1,0.847732,0); rgb(241pt)=(1,0.851566,0); rgb(242pt)=(1,0.855406,0); rgb(243pt)=(1,0.859253,0); rgb(244pt)=(1,0.863106,0); rgb(245pt)=(1,0.866964,0); rgb(246pt)=(1,0.870829,0); rgb(247pt)=(1,0.8747,0); rgb(248pt)=(1,0.878577,0); rgb(249pt)=(1,0.88246,0); rgb(250pt)=(1,0.886349,0); rgb(251pt)=(1,0.890243,0); rgb(252pt)=(1,0.894144,0); rgb(253pt)=(1,0.898051,0); rgb(254pt)=(1,0.901964,0); rgb(255pt)=(1,0.905882,0)},
mesh/rows=49]
table[row sep=crcr,header=false] {%
%
56	0.093	5.67589105086124\\
56	0.0936	5.72337040959763\\
56	0.0942	5.77084976833403\\
56	0.0948	5.81832912707042\\
56	0.0954	5.86580848580682\\
56	0.096	5.91328784454321\\
56	0.0966	5.96076720327961\\
56	0.0972	6.008246562016\\
56	0.0978	6.05572592075239\\
56	0.0984	6.10320527948878\\
56	0.099	6.15068463822518\\
56	0.0996	6.19816399696158\\
56	0.1002	6.24564335569796\\
56	0.1008	6.29312271443436\\
56	0.1014	6.34060207317075\\
56	0.102	6.38808143190715\\
56	0.1026	6.43556079064354\\
56	0.1032	6.48304014937994\\
56	0.1038	6.53051950811633\\
56	0.1044	6.57799886685272\\
56	0.105	6.62547822558912\\
56	0.1056	6.6729575843255\\
56	0.1062	6.7204369430619\\
56	0.1068	6.76791630179829\\
56	0.1074	6.8153956605347\\
56	0.108	6.86287501927109\\
56	0.1086	6.91035437800748\\
56	0.1092	6.95783373674387\\
56	0.1098	7.00531309548027\\
56	0.1104	7.05279245421666\\
56	0.111	7.10027181295305\\
56	0.1116	7.14775117168945\\
56	0.1122	7.19523053042584\\
56	0.1128	7.24270988916224\\
56	0.1134	7.29018924789863\\
56	0.114	7.33766860663502\\
56	0.1146	7.38514796537142\\
56	0.1152	7.43262732410781\\
56	0.1158	7.4801066828442\\
56	0.1164	7.5275860415806\\
56	0.117	7.57506540031699\\
56	0.1176	7.62254475905337\\
56	0.1182	7.67002411778978\\
56	0.1188	7.71750347652616\\
56	0.1194	7.76498283526257\\
56	0.12	7.81246219399894\\
56	0.1206	7.85994155273535\\
56	0.1212	7.90742091147173\\
56	0.1218	7.95490027020814\\
56	0.1224	8.00237962894452\\
56	0.123	8.04985898768093\\
56.375	0.093	5.70708269874104\\
56.375	0.0936	5.75702914512779\\
56.375	0.0942	5.80697559151454\\
56.375	0.0948	5.85692203790128\\
56.375	0.0954	5.90686848428803\\
56.375	0.096	5.95681493067478\\
56.375	0.0966	6.00676137706153\\
56.375	0.0972	6.05670782344828\\
56.375	0.0978	6.10665426983502\\
56.375	0.0984	6.15660071622177\\
56.375	0.099	6.20654716260852\\
56.375	0.0996	6.25649360899526\\
56.375	0.1002	6.30644005538201\\
56.375	0.1008	6.35638650176877\\
56.375	0.1014	6.40633294815551\\
56.375	0.102	6.45627939454226\\
56.375	0.1026	6.50622584092901\\
56.375	0.1032	6.55617228731575\\
56.375	0.1038	6.6061187337025\\
56.375	0.1044	6.65606518008924\\
56.375	0.105	6.706011626476\\
56.375	0.1056	6.75595807286274\\
56.375	0.1062	6.80590451924949\\
56.375	0.1068	6.85585096563624\\
56.375	0.1074	6.90579741202298\\
56.375	0.108	6.95574385840973\\
56.375	0.1086	7.00569030479649\\
56.375	0.1092	7.05563675118324\\
56.375	0.1098	7.10558319756998\\
56.375	0.1104	7.15552964395673\\
56.375	0.111	7.20547609034347\\
56.375	0.1116	7.25542253673022\\
56.375	0.1122	7.30536898311697\\
56.375	0.1128	7.35531542950373\\
56.375	0.1134	7.40526187589047\\
56.375	0.114	7.45520832227722\\
56.375	0.1146	7.50515476866396\\
56.375	0.1152	7.5551012150507\\
56.375	0.1158	7.60504766143746\\
56.375	0.1164	7.65499410782419\\
56.375	0.117	7.70494055421095\\
56.375	0.1176	7.75488700059769\\
56.375	0.1182	7.80483344698446\\
56.375	0.1188	7.85477989337119\\
56.375	0.1194	7.90472633975794\\
56.375	0.12	7.95467278614468\\
56.375	0.1206	8.00461923253144\\
56.375	0.1212	8.05456567891818\\
56.375	0.1218	8.10451212530494\\
56.375	0.1224	8.15445857169168\\
56.375	0.123	8.20440501807843\\
56.75	0.093	5.73827434662083\\
56.75	0.0936	5.79068788065794\\
56.75	0.0942	5.84310141469504\\
56.75	0.0948	5.89551494873213\\
56.75	0.0954	5.94792848276924\\
56.75	0.096	6.00034201680634\\
56.75	0.0966	6.05275555084345\\
56.75	0.0972	6.10516908488055\\
56.75	0.0978	6.15758261891765\\
56.75	0.0984	6.20999615295475\\
56.75	0.099	6.26240968699186\\
56.75	0.0996	6.31482322102896\\
56.75	0.1002	6.36723675506606\\
56.75	0.1008	6.41965028910317\\
56.75	0.1014	6.47206382314026\\
56.75	0.102	6.52447735717736\\
56.75	0.1026	6.57689089121447\\
56.75	0.1032	6.62930442525158\\
56.75	0.1038	6.68171795928868\\
56.75	0.1044	6.73413149332578\\
56.75	0.105	6.78654502736288\\
56.75	0.1056	6.83895856139998\\
56.75	0.1062	6.89137209543708\\
56.75	0.1068	6.94378562947418\\
56.75	0.1074	6.99619916351129\\
56.75	0.108	7.04861269754838\\
56.75	0.1086	7.10102623158549\\
56.75	0.1092	7.15343976562259\\
56.75	0.1098	7.20585329965969\\
56.75	0.1104	7.2582668336968\\
56.75	0.111	7.31068036773389\\
56.75	0.1116	7.36309390177099\\
56.75	0.1122	7.4155074358081\\
56.75	0.1128	7.4679209698452\\
56.75	0.1134	7.5203345038823\\
56.75	0.114	7.57274803791941\\
56.75	0.1146	7.62516157195651\\
56.75	0.1152	7.6775751059936\\
56.75	0.1158	7.72998864003073\\
56.75	0.1164	7.78240217406781\\
56.75	0.117	7.83481570810491\\
56.75	0.1176	7.88722924214201\\
56.75	0.1182	7.93964277617913\\
56.75	0.1188	7.99205631021621\\
56.75	0.1194	8.04446984425333\\
56.75	0.12	8.09688337829041\\
56.75	0.1206	8.14929691232753\\
56.75	0.1212	8.20171044636463\\
56.75	0.1218	8.25412398040173\\
56.75	0.1224	8.30653751443882\\
56.75	0.123	8.35895104847594\\
57.125	0.093	5.76946599450064\\
57.125	0.0936	5.82434661618809\\
57.125	0.0942	5.87922723787555\\
57.125	0.0948	5.934107859563\\
57.125	0.0954	5.98898848125045\\
57.125	0.096	6.04386910293791\\
57.125	0.0966	6.09874972462537\\
57.125	0.0972	6.15363034631283\\
57.125	0.0978	6.20851096800028\\
57.125	0.0984	6.26339158968774\\
57.125	0.099	6.3182722113752\\
57.125	0.0996	6.37315283306265\\
57.125	0.1002	6.42803345475011\\
57.125	0.1008	6.48291407643757\\
57.125	0.1014	6.53779469812502\\
57.125	0.102	6.59267531981247\\
57.125	0.1026	6.64755594149993\\
57.125	0.1032	6.70243656318739\\
57.125	0.1038	6.75731718487485\\
57.125	0.1044	6.8121978065623\\
57.125	0.105	6.86707842824976\\
57.125	0.1056	6.92195904993721\\
57.125	0.1062	6.97683967162467\\
57.125	0.1068	7.03172029331213\\
57.125	0.1074	7.08660091499959\\
57.125	0.108	7.14148153668704\\
57.125	0.1086	7.1963621583745\\
57.125	0.1092	7.25124278006195\\
57.125	0.1098	7.30612340174941\\
57.125	0.1104	7.36100402343687\\
57.125	0.111	7.41588464512432\\
57.125	0.1116	7.47076526681177\\
57.125	0.1122	7.52564588849923\\
57.125	0.1128	7.58052651018669\\
57.125	0.1134	7.63540713187414\\
57.125	0.114	7.69028775356159\\
57.125	0.1146	7.74516837524905\\
57.125	0.1152	7.8000489969365\\
57.125	0.1158	7.85492961862397\\
57.125	0.1164	7.90981024031142\\
57.125	0.117	7.96469086199889\\
57.125	0.1176	8.01957148368632\\
57.125	0.1182	8.07445210537379\\
57.125	0.1188	8.12933272706124\\
57.125	0.1194	8.1842133487487\\
57.125	0.12	8.23909397043614\\
57.125	0.1206	8.29397459212362\\
57.125	0.1212	8.34885521381106\\
57.125	0.1218	8.40373583549852\\
57.125	0.1224	8.45861645718597\\
57.125	0.123	8.51349707887344\\
57.5	0.093	5.80065764238043\\
57.5	0.0936	5.85800535171824\\
57.5	0.0942	5.91535306105605\\
57.5	0.0948	5.97270077039386\\
57.5	0.0954	6.03004847973167\\
57.5	0.096	6.08739618906948\\
57.5	0.0966	6.14474389840729\\
57.5	0.0972	6.2020916077451\\
57.5	0.0978	6.25943931708291\\
57.5	0.0984	6.31678702642072\\
57.5	0.099	6.37413473575853\\
57.5	0.0996	6.43148244509634\\
57.5	0.1002	6.48883015443415\\
57.5	0.1008	6.54617786377197\\
57.5	0.1014	6.60352557310977\\
57.5	0.102	6.66087328244758\\
57.5	0.1026	6.7182209917854\\
57.5	0.1032	6.77556870112321\\
57.5	0.1038	6.83291641046102\\
57.5	0.1044	6.89026411979881\\
57.5	0.105	6.94761182913663\\
57.5	0.1056	7.00495953847445\\
57.5	0.1062	7.06230724781226\\
57.5	0.1068	7.11965495715008\\
57.5	0.1074	7.17700266648789\\
57.5	0.108	7.23435037582568\\
57.5	0.1086	7.2916980851635\\
57.5	0.1092	7.34904579450131\\
57.5	0.1098	7.40639350383912\\
57.5	0.1104	7.46374121317694\\
57.5	0.111	7.52108892251474\\
57.5	0.1116	7.57843663185255\\
57.5	0.1122	7.63578434119037\\
57.5	0.1128	7.69313205052817\\
57.5	0.1134	7.75047975986599\\
57.5	0.114	7.80782746920379\\
57.5	0.1146	7.86517517854161\\
57.5	0.1152	7.92252288787941\\
57.5	0.1158	7.97987059721723\\
57.5	0.1164	8.03721830655503\\
57.5	0.117	8.09456601589285\\
57.5	0.1176	8.15191372523064\\
57.5	0.1182	8.20926143456848\\
57.5	0.1188	8.26660914390627\\
57.5	0.1194	8.32395685324408\\
57.5	0.12	8.38130456258189\\
57.5	0.1206	8.43865227191971\\
57.5	0.1212	8.49599998125751\\
57.5	0.1218	8.55334769059533\\
57.5	0.1224	8.61069539993312\\
57.5	0.123	8.66804310927095\\
57.875	0.093	5.83184929026022\\
57.875	0.0936	5.89166408724839\\
57.875	0.0942	5.95147888423656\\
57.875	0.0948	6.01129368122471\\
57.875	0.0954	6.07110847821289\\
57.875	0.096	6.13092327520105\\
57.875	0.0966	6.19073807218921\\
57.875	0.0972	6.25055286917738\\
57.875	0.0978	6.31036766616554\\
57.875	0.0984	6.3701824631537\\
57.875	0.099	6.42999726014187\\
57.875	0.0996	6.48981205713004\\
57.875	0.1002	6.5496268541182\\
57.875	0.1008	6.60944165110637\\
57.875	0.1014	6.66925644809452\\
57.875	0.102	6.72907124508269\\
57.875	0.1026	6.78888604207086\\
57.875	0.1032	6.84870083905902\\
57.875	0.1038	6.90851563604718\\
57.875	0.1044	6.96833043303535\\
57.875	0.105	7.02814523002351\\
57.875	0.1056	7.08796002701168\\
57.875	0.1062	7.14777482399985\\
57.875	0.1068	7.20758962098802\\
57.875	0.1074	7.26740441797617\\
57.875	0.108	7.32721921496433\\
57.875	0.1086	7.38703401195251\\
57.875	0.1092	7.44684880894066\\
57.875	0.1098	7.50666360592883\\
57.875	0.1104	7.56647840291701\\
57.875	0.111	7.62629319990516\\
57.875	0.1116	7.68610799689333\\
57.875	0.1122	7.74592279388148\\
57.875	0.1128	7.80573759086966\\
57.875	0.1134	7.86555238785783\\
57.875	0.114	7.92536718484598\\
57.875	0.1146	7.98518198183415\\
57.875	0.1152	8.04499677882231\\
57.875	0.1158	8.10481157581049\\
57.875	0.1164	8.16462637279864\\
57.875	0.117	8.22444116978681\\
57.875	0.1176	8.28425596677496\\
57.875	0.1182	8.34407076376314\\
57.875	0.1188	8.40388556075129\\
57.875	0.1194	8.46370035773947\\
57.875	0.12	8.52351515472762\\
57.875	0.1206	8.5833299517158\\
57.875	0.1212	8.64314474870395\\
57.875	0.1218	8.70295954569212\\
57.875	0.1224	8.76277434268027\\
57.875	0.123	8.82258913966845\\
58.25	0.093	5.86304093814002\\
58.25	0.0936	5.92532282277854\\
58.25	0.0942	5.98760470741706\\
58.25	0.0948	6.04988659205557\\
58.25	0.0954	6.11216847669409\\
58.25	0.096	6.17445036133262\\
58.25	0.0966	6.23673224597113\\
58.25	0.0972	6.29901413060966\\
58.25	0.0978	6.36129601524816\\
58.25	0.0984	6.42357789988669\\
58.25	0.099	6.48585978452521\\
58.25	0.0996	6.54814166916373\\
58.25	0.1002	6.61042355380225\\
58.25	0.1008	6.67270543844077\\
58.25	0.1014	6.73498732307929\\
58.25	0.102	6.7972692077178\\
58.25	0.1026	6.85955109235633\\
58.25	0.1032	6.92183297699484\\
58.25	0.1038	6.98411486163336\\
58.25	0.1044	7.04639674627187\\
58.25	0.105	7.10867863091039\\
58.25	0.1056	7.17096051554891\\
58.25	0.1062	7.23324240018744\\
58.25	0.1068	7.29552428482596\\
58.25	0.1074	7.35780616946447\\
58.25	0.108	7.42008805410299\\
58.25	0.1086	7.48236993874151\\
58.25	0.1092	7.54465182338003\\
58.25	0.1098	7.60693370801855\\
58.25	0.1104	7.66921559265706\\
58.25	0.111	7.73149747729558\\
58.25	0.1116	7.7937793619341\\
58.25	0.1122	7.85606124657262\\
58.25	0.1128	7.91834313121113\\
58.25	0.1134	7.98062501584967\\
58.25	0.114	8.04290690048818\\
58.25	0.1146	8.1051887851267\\
58.25	0.1152	8.16747066976521\\
58.25	0.1158	8.22975255440375\\
58.25	0.1164	8.29203443904224\\
58.25	0.117	8.35431632368078\\
58.25	0.1176	8.41659820831927\\
58.25	0.1182	8.47888009295781\\
58.25	0.1188	8.54116197759632\\
58.25	0.1194	8.60344386223484\\
58.25	0.12	8.66572574687335\\
58.25	0.1206	8.72800763151189\\
58.25	0.1212	8.79028951615039\\
58.25	0.1218	8.85257140078892\\
58.25	0.1224	8.91485328542743\\
58.25	0.123	8.97713517006596\\
58.625	0.093	5.89423258601983\\
58.625	0.0936	5.9589815583087\\
58.625	0.0942	6.02373053059758\\
58.625	0.0948	6.08847950288644\\
58.625	0.0954	6.15322847517532\\
58.625	0.096	6.21797744746419\\
58.625	0.0966	6.28272641975307\\
58.625	0.0972	6.34747539204194\\
58.625	0.0978	6.4122243643308\\
58.625	0.0984	6.47697333661968\\
58.625	0.099	6.54172230890855\\
58.625	0.0996	6.60647128119742\\
58.625	0.1002	6.6712202534863\\
58.625	0.1008	6.73596922577517\\
58.625	0.1014	6.80071819806405\\
58.625	0.102	6.86546717035292\\
58.625	0.1026	6.93021614264179\\
58.625	0.1032	6.99496511493067\\
58.625	0.1038	7.05971408721954\\
58.625	0.1044	7.12446305950841\\
58.625	0.105	7.18921203179728\\
58.625	0.1056	7.25396100408616\\
58.625	0.1062	7.31870997637503\\
58.625	0.1068	7.38345894866391\\
58.625	0.1074	7.44820792095278\\
58.625	0.108	7.51295689324164\\
58.625	0.1086	7.57770586553053\\
58.625	0.1092	7.64245483781939\\
58.625	0.1098	7.70720381010827\\
58.625	0.1104	7.77195278239714\\
58.625	0.111	7.83670175468601\\
58.625	0.1116	7.90145072697488\\
58.625	0.1122	7.96619969926375\\
58.625	0.1128	8.03094867155264\\
58.625	0.1134	8.0956976438415\\
58.625	0.114	8.16044661613037\\
58.625	0.1146	8.22519558841925\\
58.625	0.1152	8.28994456070812\\
58.625	0.1158	8.35469353299699\\
58.625	0.1164	8.41944250528587\\
58.625	0.117	8.48419147757474\\
58.625	0.1176	8.54894044986361\\
58.625	0.1182	8.61368942215249\\
58.625	0.1188	8.67843839444136\\
58.625	0.1194	8.74318736673023\\
58.625	0.12	8.80793633901909\\
58.625	0.1206	8.87268531130798\\
58.625	0.1212	8.93743428359684\\
58.625	0.1218	9.00218325588573\\
58.625	0.1224	9.06693222817459\\
58.625	0.123	9.13168120046346\\
59	0.093	5.92542423389963\\
59	0.0936	5.99264029383885\\
59	0.0942	6.05985635377808\\
59	0.0948	6.1270724137173\\
59	0.0954	6.19428847365653\\
59	0.096	6.26150453359575\\
59	0.0966	6.32872059353499\\
59	0.0972	6.39593665347422\\
59	0.0978	6.46315271341344\\
59	0.0984	6.53036877335266\\
59	0.099	6.59758483329189\\
59	0.0996	6.66480089323112\\
59	0.1002	6.73201695317035\\
59	0.1008	6.79923301310958\\
59	0.1014	6.8664490730488\\
59	0.102	6.93366513298803\\
59	0.1026	7.00088119292726\\
59	0.1032	7.06809725286649\\
59	0.1038	7.13531331280571\\
59	0.1044	7.20252937274493\\
59	0.105	7.26974543268417\\
59	0.1056	7.3369614926234\\
59	0.1062	7.40417755256262\\
59	0.1068	7.47139361250185\\
59	0.1074	7.53860967244107\\
59	0.108	7.60582573238029\\
59	0.1086	7.67304179231953\\
59	0.1092	7.74025785225875\\
59	0.1098	7.80747391219798\\
59	0.1104	7.8746899721372\\
59	0.111	7.94190603207642\\
59	0.1116	8.00912209201566\\
59	0.1122	8.07633815195489\\
59	0.1128	8.14355421189411\\
59	0.1134	8.21077027183334\\
59	0.114	8.27798633177255\\
59	0.1146	8.34520239171181\\
59	0.1152	8.41241845165102\\
59	0.1158	8.47963451159025\\
59	0.1164	8.54685057152948\\
59	0.117	8.6140666314687\\
59	0.1176	8.68128269140793\\
59	0.1182	8.74849875134716\\
59	0.1188	8.81571481128638\\
59	0.1194	8.88293087122561\\
59	0.12	8.95014693116484\\
59	0.1206	9.01736299110406\\
59	0.1212	9.08457905104329\\
59	0.1218	9.15179511098252\\
59	0.1224	9.21901117092175\\
59	0.123	9.28622723086097\\
59.375	0.093	5.95661588177942\\
59.375	0.0936	6.026299029369\\
59.375	0.0942	6.09598217695859\\
59.375	0.0948	6.16566532454816\\
59.375	0.0954	6.23534847213774\\
59.375	0.096	6.30503161972732\\
59.375	0.0966	6.37471476731691\\
59.375	0.0972	6.44439791490649\\
59.375	0.0978	6.51408106249607\\
59.375	0.0984	6.58376421008565\\
59.375	0.099	6.65344735767523\\
59.375	0.0996	6.72313050526481\\
59.375	0.1002	6.7928136528544\\
59.375	0.1008	6.86249680044399\\
59.375	0.1014	6.93217994803355\\
59.375	0.102	7.00186309562313\\
59.375	0.1026	7.07154624321272\\
59.375	0.1032	7.1412293908023\\
59.375	0.1038	7.21091253839189\\
59.375	0.1044	7.28059568598146\\
59.375	0.105	7.35027883357105\\
59.375	0.1056	7.41996198116062\\
59.375	0.1062	7.4896451287502\\
59.375	0.1068	7.55932827633979\\
59.375	0.1074	7.62901142392937\\
59.375	0.108	7.69869457151895\\
59.375	0.1086	7.76837771910854\\
59.375	0.1092	7.83806086669811\\
59.375	0.1098	7.90774401428769\\
59.375	0.1104	7.97742716187727\\
59.375	0.111	8.04711030946686\\
59.375	0.1116	8.11679345705643\\
59.375	0.1122	8.18647660464602\\
59.375	0.1128	8.2561597522356\\
59.375	0.1134	8.32584289982518\\
59.375	0.114	8.39552604741475\\
59.375	0.1146	8.46520919500435\\
59.375	0.1152	8.53489234259392\\
59.375	0.1158	8.60457549018351\\
59.375	0.1164	8.67425863777309\\
59.375	0.117	8.74394178536267\\
59.375	0.1176	8.81362493295225\\
59.375	0.1182	8.88330808054182\\
59.375	0.1188	8.95299122813141\\
59.375	0.1194	9.02267437572098\\
59.375	0.12	9.09235752331057\\
59.375	0.1206	9.16204067090015\\
59.375	0.1212	9.23172381848974\\
59.375	0.1218	9.30140696607933\\
59.375	0.1224	9.37109011366888\\
59.375	0.123	9.44077326125847\\
59.75	0.093	5.98780752965921\\
59.75	0.0936	6.05995776489915\\
59.75	0.0942	6.13210800013909\\
59.75	0.0948	6.20425823537902\\
59.75	0.0954	6.27640847061895\\
59.75	0.096	6.34855870585889\\
59.75	0.0966	6.42070894109883\\
59.75	0.0972	6.49285917633877\\
59.75	0.0978	6.5650094115787\\
59.75	0.0984	6.63715964681863\\
59.75	0.099	6.70930988205858\\
59.75	0.0996	6.7814601172985\\
59.75	0.1002	6.85361035253845\\
59.75	0.1008	6.92576058777838\\
59.75	0.1014	6.99791082301832\\
59.75	0.102	7.07006105825824\\
59.75	0.1026	7.14221129349819\\
59.75	0.1032	7.21436152873812\\
59.75	0.1038	7.28651176397806\\
59.75	0.1044	7.35866199921798\\
59.75	0.105	7.43081223445792\\
59.75	0.1056	7.50296246969786\\
59.75	0.1062	7.5751127049378\\
59.75	0.1068	7.64726294017773\\
59.75	0.1074	7.71941317541767\\
59.75	0.108	7.7915634106576\\
59.75	0.1086	7.86371364589754\\
59.75	0.1092	7.93586388113746\\
59.75	0.1098	8.00801411637741\\
59.75	0.1104	8.08016435161734\\
59.75	0.111	8.15231458685729\\
59.75	0.1116	8.2244648220972\\
59.75	0.1122	8.29661505733715\\
59.75	0.1128	8.36876529257708\\
59.75	0.1134	8.44091552781703\\
59.75	0.114	8.51306576305694\\
59.75	0.1146	8.58521599829689\\
59.75	0.1152	8.65736623353682\\
59.75	0.1158	8.72951646877677\\
59.75	0.1164	8.8016667040167\\
59.75	0.117	8.87381693925663\\
59.75	0.1176	8.94596717449656\\
59.75	0.1182	9.0181174097365\\
59.75	0.1188	9.09026764497644\\
59.75	0.1194	9.16241788021637\\
59.75	0.12	9.2345681154563\\
59.75	0.1206	9.30671835069624\\
59.75	0.1212	9.37886858593617\\
59.75	0.1218	9.45101882117612\\
59.75	0.1224	9.52316905641604\\
59.75	0.123	9.59531929165598\\
60.125	0.093	6.01899917753902\\
60.125	0.0936	6.09361650042931\\
60.125	0.0942	6.1682338233196\\
60.125	0.0948	6.24285114620988\\
60.125	0.0954	6.31746846910017\\
60.125	0.096	6.39208579199046\\
60.125	0.0966	6.46670311488075\\
60.125	0.0972	6.54132043777104\\
60.125	0.0978	6.61593776066132\\
60.125	0.0984	6.69055508355162\\
60.125	0.099	6.76517240644191\\
60.125	0.0996	6.8397897293322\\
60.125	0.1002	6.91440705222249\\
60.125	0.1008	6.98902437511278\\
60.125	0.1014	7.06364169800307\\
60.125	0.102	7.13825902089335\\
60.125	0.1026	7.21287634378364\\
60.125	0.1032	7.28749366667393\\
60.125	0.1038	7.36211098956422\\
60.125	0.1044	7.43672831245451\\
60.125	0.105	7.5113456353448\\
60.125	0.1056	7.58596295823509\\
60.125	0.1062	7.66058028112538\\
60.125	0.1068	7.73519760401567\\
60.125	0.1074	7.80981492690596\\
60.125	0.108	7.88443224979626\\
60.125	0.1086	7.95904957268655\\
60.125	0.1092	8.03366689557683\\
60.125	0.1098	8.10828421846712\\
60.125	0.1104	8.18290154135741\\
60.125	0.111	8.2575188642477\\
60.125	0.1116	8.33213618713798\\
60.125	0.1122	8.40675351002828\\
60.125	0.1128	8.48137083291857\\
60.125	0.1134	8.55598815580886\\
60.125	0.114	8.63060547869914\\
60.125	0.1146	8.70522280158944\\
60.125	0.1152	8.77984012447972\\
60.125	0.1158	8.85445744737001\\
60.125	0.1164	8.9290747702603\\
60.125	0.117	9.00369209315059\\
60.125	0.1176	9.07830941604088\\
60.125	0.1182	9.15292673893117\\
60.125	0.1188	9.22754406182146\\
60.125	0.1194	9.30216138471174\\
60.125	0.12	9.37677870760204\\
60.125	0.1206	9.45139603049233\\
60.125	0.1212	9.52601335338262\\
60.125	0.1218	9.60063067627291\\
60.125	0.1224	9.67524799916319\\
60.125	0.123	9.74986532205348\\
60.5	0.093	6.0501908254188\\
60.5	0.0936	6.12727523595944\\
60.5	0.0942	6.20435964650009\\
60.5	0.0948	6.28144405704072\\
60.5	0.0954	6.35852846758137\\
60.5	0.096	6.43561287812201\\
60.5	0.0966	6.51269728866266\\
60.5	0.0972	6.58978169920331\\
60.5	0.0978	6.66686610974395\\
60.5	0.0984	6.74395052028459\\
60.5	0.099	6.82103493082523\\
60.5	0.0996	6.89811934136588\\
60.5	0.1002	6.97520375190653\\
60.5	0.1008	7.05228816244718\\
60.5	0.1014	7.1293725729878\\
60.5	0.102	7.20645698352845\\
60.5	0.1026	7.2835413940691\\
60.5	0.1032	7.36062580460975\\
60.5	0.1038	7.43771021515038\\
60.5	0.1044	7.51479462569102\\
60.5	0.105	7.59187903623167\\
60.5	0.1056	7.66896344677232\\
60.5	0.1062	7.74604785731297\\
60.5	0.1068	7.8231322678536\\
60.5	0.1074	7.90021667839424\\
60.5	0.108	7.97730108893489\\
60.5	0.1086	8.05438549947554\\
60.5	0.1092	8.13146991001618\\
60.5	0.1098	8.20855432055681\\
60.5	0.1104	8.28563873109746\\
60.5	0.111	8.36272314163811\\
60.5	0.1116	8.43980755217875\\
60.5	0.1122	8.5168919627194\\
60.5	0.1128	8.59397637326003\\
60.5	0.1134	8.67106078380068\\
60.5	0.114	8.74814519434132\\
60.5	0.1146	8.82522960488197\\
60.5	0.1152	8.90231401542262\\
60.5	0.1158	8.97939842596325\\
60.5	0.1164	9.0564828365039\\
60.5	0.117	9.13356724704454\\
60.5	0.1176	9.21065165758519\\
60.5	0.1182	9.28773606812582\\
60.5	0.1188	9.36482047866647\\
60.5	0.1194	9.44190488920711\\
60.5	0.12	9.51898929974776\\
60.5	0.1206	9.59607371028841\\
60.5	0.1212	9.67315812082904\\
60.5	0.1218	9.75024253136969\\
60.5	0.1224	9.82732694191033\\
60.5	0.123	9.90441135245098\\
60.875	0.093	6.0813824732986\\
60.875	0.0936	6.1609339714896\\
60.875	0.0942	6.24048546968059\\
60.875	0.0948	6.32003696787159\\
60.875	0.0954	6.39958846606258\\
60.875	0.096	6.47913996425359\\
60.875	0.0966	6.55869146244459\\
60.875	0.0972	6.63824296063559\\
60.875	0.0978	6.71779445882657\\
60.875	0.0984	6.79734595701757\\
60.875	0.099	6.87689745520858\\
60.875	0.0996	6.95644895339957\\
60.875	0.1002	7.03600045159057\\
60.875	0.1008	7.11555194978157\\
60.875	0.1014	7.19510344797256\\
60.875	0.102	7.27465494616356\\
60.875	0.1026	7.35420644435457\\
60.875	0.1032	7.43375794254556\\
60.875	0.1038	7.51330944073656\\
60.875	0.1044	7.59286093892755\\
60.875	0.105	7.67241243711855\\
60.875	0.1056	7.75196393530955\\
60.875	0.1062	7.83151543350055\\
60.875	0.1068	7.91106693169154\\
60.875	0.1074	7.99061842988255\\
60.875	0.108	8.07016992807353\\
60.875	0.1086	8.14972142626455\\
60.875	0.1092	8.22927292445553\\
60.875	0.1098	8.30882442264654\\
60.875	0.1104	8.38837592083753\\
60.875	0.111	8.46792741902853\\
60.875	0.1116	8.54747891721952\\
60.875	0.1122	8.62703041541053\\
60.875	0.1128	8.70658191360153\\
60.875	0.1134	8.78613341179252\\
60.875	0.114	8.86568490998351\\
60.875	0.1146	8.94523640817452\\
60.875	0.1152	9.02478790636552\\
60.875	0.1158	9.10433940455651\\
60.875	0.1164	9.18389090274751\\
60.875	0.117	9.2634424009385\\
60.875	0.1176	9.3429938991295\\
60.875	0.1182	9.4225453973205\\
60.875	0.1188	9.5020968955115\\
60.875	0.1194	9.58164839370249\\
60.875	0.12	9.66119989189349\\
60.875	0.1206	9.7407513900845\\
60.875	0.1212	9.82030288827549\\
60.875	0.1218	9.8998543864665\\
60.875	0.1224	9.97940588465748\\
60.875	0.123	10.0589573828485\\
61.25	0.093	6.11257412117839\\
61.25	0.0936	6.19459270701974\\
61.25	0.0942	6.2766112928611\\
61.25	0.0948	6.35862987870244\\
61.25	0.0954	6.4406484645438\\
61.25	0.096	6.52266705038515\\
61.25	0.0966	6.60468563622651\\
61.25	0.0972	6.68670422206785\\
61.25	0.0978	6.7687228079092\\
61.25	0.0984	6.85074139375055\\
61.25	0.099	6.93275997959191\\
61.25	0.0996	7.01477856543327\\
61.25	0.1002	7.09679715127462\\
61.25	0.1008	7.17881573711598\\
61.25	0.1014	7.26083432295732\\
61.25	0.102	7.34285290879867\\
61.25	0.1026	7.42487149464002\\
61.25	0.1032	7.50689008048138\\
61.25	0.1038	7.58890866632273\\
61.25	0.1044	7.67092725216407\\
61.25	0.105	7.75294583800542\\
61.25	0.1056	7.83496442384678\\
61.25	0.1062	7.91698300968813\\
61.25	0.1068	7.99900159552949\\
61.25	0.1074	8.08102018137085\\
61.25	0.108	8.16303876721219\\
61.25	0.1086	8.24505735305355\\
61.25	0.1092	8.32707593889489\\
61.25	0.1098	8.40909452473625\\
61.25	0.1104	8.4911131105776\\
61.25	0.111	8.57313169641895\\
61.25	0.1116	8.65515028226029\\
61.25	0.1122	8.73716886810165\\
61.25	0.1128	8.819187453943\\
61.25	0.1134	8.90120603978436\\
61.25	0.114	8.9832246256257\\
61.25	0.1146	9.06524321146706\\
61.25	0.1152	9.14726179730842\\
61.25	0.1158	9.22928038314977\\
61.25	0.1164	9.31129896899112\\
61.25	0.117	9.39331755483246\\
61.25	0.1176	9.47533614067382\\
61.25	0.1182	9.55735472651517\\
61.25	0.1188	9.63937331235653\\
61.25	0.1194	9.72139189819787\\
61.25	0.12	9.80341048403922\\
61.25	0.1206	9.88542906988057\\
61.25	0.1212	9.96744765572193\\
61.25	0.1218	10.0494662415633\\
61.25	0.1224	10.1314848274046\\
61.25	0.123	10.213503413246\\
61.625	0.093	6.1437657690582\\
61.625	0.0936	6.2282514425499\\
61.625	0.0942	6.31273711604161\\
61.625	0.0948	6.39722278953331\\
61.625	0.0954	6.48170846302502\\
61.625	0.096	6.56619413651672\\
61.625	0.0966	6.65067981000843\\
61.625	0.0972	6.73516548350014\\
61.625	0.0978	6.81965115699185\\
61.625	0.0984	6.90413683048354\\
61.625	0.099	6.98862250397526\\
61.625	0.0996	7.07310817746696\\
61.625	0.1002	7.15759385095868\\
61.625	0.1008	7.24207952445038\\
61.625	0.1014	7.32656519794208\\
61.625	0.102	7.41105087143379\\
61.625	0.1026	7.4955365449255\\
61.625	0.1032	7.5800222184172\\
61.625	0.1038	7.66450789190891\\
61.625	0.1044	7.74899356540061\\
61.625	0.105	7.83347923889231\\
61.625	0.1056	7.91796491238402\\
61.625	0.1062	8.00245058587574\\
61.625	0.1068	8.08693625936743\\
61.625	0.1074	8.17142193285913\\
61.625	0.108	8.25590760635085\\
61.625	0.1086	8.34039327984256\\
61.625	0.1092	8.42487895333427\\
61.625	0.1098	8.50936462682596\\
61.625	0.1104	8.59385030031768\\
61.625	0.111	8.67833597380938\\
61.625	0.1116	8.76282164730108\\
61.625	0.1122	8.84730732079278\\
61.625	0.1128	8.93179299428451\\
61.625	0.1134	9.01627866777621\\
61.625	0.114	9.1007643412679\\
61.625	0.1146	9.18525001475962\\
61.625	0.1152	9.26973568825133\\
61.625	0.1158	9.35422136174303\\
61.625	0.1164	9.43870703523474\\
61.625	0.117	9.52319270872644\\
61.625	0.1176	9.60767838221815\\
61.625	0.1182	9.69216405570984\\
61.625	0.1188	9.77664972920157\\
61.625	0.1194	9.86113540269326\\
61.625	0.12	9.94562107618498\\
61.625	0.1206	10.0301067496767\\
61.625	0.1212	10.1145924231684\\
61.625	0.1218	10.1990780966601\\
61.625	0.1224	10.2835637701518\\
61.625	0.123	10.3680494436435\\
62	0.093	6.17495741693799\\
62	0.0936	6.26191017808006\\
62	0.0942	6.34886293922212\\
62	0.0948	6.43581570036417\\
62	0.0954	6.52276846150624\\
62	0.096	6.6097212226483\\
62	0.0966	6.69667398379035\\
62	0.0972	6.78362674493242\\
62	0.0978	6.87057950607447\\
62	0.0984	6.95753226721653\\
62	0.099	7.0444850283586\\
62	0.0996	7.13143778950065\\
62	0.1002	7.21839055064272\\
62	0.1008	7.30534331178478\\
62	0.1014	7.39229607292683\\
62	0.102	7.4792488340689\\
62	0.1026	7.56620159521096\\
62	0.1032	7.65315435635301\\
62	0.1038	7.74010711749509\\
62	0.1044	7.82705987863713\\
62	0.105	7.9140126397792\\
62	0.1056	8.00096540092126\\
62	0.1062	8.08791816206332\\
62	0.1068	8.17487092320538\\
62	0.1074	8.26182368434743\\
62	0.108	8.34877644548949\\
62	0.1086	8.43572920663156\\
62	0.1092	8.52268196777362\\
62	0.1098	8.60963472891568\\
62	0.1104	8.69658749005775\\
62	0.111	8.7835402511998\\
62	0.1116	8.87049301234187\\
62	0.1122	8.95744577348391\\
62	0.1128	9.04439853462598\\
62	0.1134	9.13135129576804\\
62	0.114	9.2183040569101\\
62	0.1146	9.30525681805216\\
62	0.1152	9.39220957919423\\
62	0.1158	9.47916234033627\\
62	0.1164	9.56611510147836\\
62	0.117	9.6530678626204\\
62	0.1176	9.74002062376246\\
62	0.1182	9.82697338490452\\
62	0.1188	9.91392614604659\\
62	0.1194	10.0008789071886\\
62	0.12	10.0878316683307\\
62	0.1206	10.1747844294728\\
62	0.1212	10.2617371906148\\
62	0.1218	10.3486899517569\\
62	0.1224	10.435642712899\\
62	0.123	10.522595474041\\
62.375	0.093	6.20614906481779\\
62.375	0.0936	6.2955689136102\\
62.375	0.0942	6.38498876240262\\
62.375	0.0948	6.47440861119502\\
62.375	0.0954	6.56382845998744\\
62.375	0.096	6.65324830877986\\
62.375	0.0966	6.74266815757228\\
62.375	0.0972	6.83208800636469\\
62.375	0.0978	6.9215078551571\\
62.375	0.0984	7.01092770394951\\
62.375	0.099	7.10034755274193\\
62.375	0.0996	7.18976740153435\\
62.375	0.1002	7.27918725032677\\
62.375	0.1008	7.36860709911917\\
62.375	0.1014	7.45802694791158\\
62.375	0.102	7.54744679670401\\
62.375	0.1026	7.63686664549643\\
62.375	0.1032	7.72628649428884\\
62.375	0.1038	7.81570634308126\\
62.375	0.1044	7.90512619187366\\
62.375	0.105	7.99454604066608\\
62.375	0.1056	8.0839658894585\\
62.375	0.1062	8.17338573825091\\
62.375	0.1068	8.26280558704333\\
62.375	0.1074	8.35222543583573\\
62.375	0.108	8.44164528462815\\
62.375	0.1086	8.53106513342057\\
62.375	0.1092	8.62048498221299\\
62.375	0.1098	8.70990483100539\\
62.375	0.1104	8.79932467979782\\
62.375	0.111	8.88874452859022\\
62.375	0.1116	8.97816437738264\\
62.375	0.1122	9.06758422617504\\
62.375	0.1128	9.15700407496747\\
62.375	0.1134	9.24642392375988\\
62.375	0.114	9.33584377255229\\
62.375	0.1146	9.42526362134471\\
62.375	0.1152	9.51468347013713\\
62.375	0.1158	9.60410331892953\\
62.375	0.1164	9.69352316772196\\
62.375	0.117	9.78294301651437\\
62.375	0.1176	9.87236286530678\\
62.375	0.1182	9.96178271409919\\
62.375	0.1188	10.0512025628916\\
62.375	0.1194	10.140622411684\\
62.375	0.12	10.2300422604764\\
62.375	0.1206	10.3194621092688\\
62.375	0.1212	10.4088819580613\\
62.375	0.1218	10.4983018068537\\
62.375	0.1224	10.5877216556461\\
62.375	0.123	10.6771415044385\\
62.75	0.093	6.23734071269759\\
62.75	0.0936	6.32922764914036\\
62.75	0.0942	6.42111458558313\\
62.75	0.0948	6.5130015220259\\
62.75	0.0954	6.60488845846866\\
62.75	0.096	6.69677539491143\\
62.75	0.0966	6.7886623313542\\
62.75	0.0972	6.88054926779697\\
62.75	0.0978	6.97243620423973\\
62.75	0.0984	7.0643231406825\\
62.75	0.099	7.15621007712527\\
62.75	0.0996	7.24809701356804\\
62.75	0.1002	7.3399839500108\\
62.75	0.1008	7.43187088645358\\
62.75	0.1014	7.52375782289635\\
62.75	0.102	7.61564475933912\\
62.75	0.1026	7.70753169578188\\
62.75	0.1032	7.79941863222466\\
62.75	0.1038	7.89130556866742\\
62.75	0.1044	7.98319250511018\\
62.75	0.105	8.07507944155296\\
62.75	0.1056	8.16696637799572\\
62.75	0.1062	8.2588533144385\\
62.75	0.1068	8.35074025088127\\
62.75	0.1074	8.44262718732404\\
62.75	0.108	8.5345141237668\\
62.75	0.1086	8.62640106020957\\
62.75	0.1092	8.71828799665235\\
62.75	0.1098	8.8101749330951\\
62.75	0.1104	8.90206186953789\\
62.75	0.111	8.99394880598064\\
62.75	0.1116	9.08583574242341\\
62.75	0.1122	9.17772267886617\\
62.75	0.1128	9.26960961530895\\
62.75	0.1134	9.36149655175171\\
62.75	0.114	9.45338348819449\\
62.75	0.1146	9.54527042463725\\
62.75	0.1152	9.63715736108003\\
62.75	0.1158	9.72904429752279\\
62.75	0.1164	9.82093123396557\\
62.75	0.117	9.91281817040833\\
62.75	0.1176	10.0047051068511\\
62.75	0.1182	10.0965920432939\\
62.75	0.1188	10.1884789797366\\
62.75	0.1194	10.2803659161794\\
62.75	0.12	10.3722528526222\\
62.75	0.1206	10.4641397890649\\
62.75	0.1212	10.5560267255077\\
62.75	0.1218	10.6479136619505\\
62.75	0.1224	10.7398005983932\\
62.75	0.123	10.831687534836\\
63.125	0.093	6.26853236057739\\
63.125	0.0936	6.36288638467051\\
63.125	0.0942	6.45724040876363\\
63.125	0.0948	6.55159443285675\\
63.125	0.0954	6.64594845694987\\
63.125	0.096	6.740302481043\\
63.125	0.0966	6.83465650513612\\
63.125	0.0972	6.92901052922925\\
63.125	0.0978	7.02336455332236\\
63.125	0.0984	7.11771857741548\\
63.125	0.099	7.21207260150862\\
63.125	0.0996	7.30642662560173\\
63.125	0.1002	7.40078064969485\\
63.125	0.1008	7.49513467378799\\
63.125	0.1014	7.5894886978811\\
63.125	0.102	7.68384272197422\\
63.125	0.1026	7.77819674606735\\
63.125	0.1032	7.87255077016047\\
63.125	0.1038	7.96690479425359\\
63.125	0.1044	8.06125881834672\\
63.125	0.105	8.15561284243984\\
63.125	0.1056	8.24996686653296\\
63.125	0.1062	8.34432089062608\\
63.125	0.1068	8.43867491471921\\
63.125	0.1074	8.53302893881232\\
63.125	0.108	8.62738296290544\\
63.125	0.1086	8.72173698699858\\
63.125	0.1092	8.8160910110917\\
63.125	0.1098	8.91044503518482\\
63.125	0.1104	9.00479905927796\\
63.125	0.111	9.09915308337106\\
63.125	0.1116	9.19350710746419\\
63.125	0.1122	9.28786113155731\\
63.125	0.1128	9.38221515565044\\
63.125	0.1134	9.47656917974355\\
63.125	0.114	9.57092320383667\\
63.125	0.1146	9.66527722792981\\
63.125	0.1152	9.75963125202293\\
63.125	0.1158	9.85398527611605\\
63.125	0.1164	9.94833930020918\\
63.125	0.117	10.0426933243023\\
63.125	0.1176	10.1370473483954\\
63.125	0.1182	10.2314013724885\\
63.125	0.1188	10.3257553965817\\
63.125	0.1194	10.4201094206748\\
63.125	0.12	10.5144634447679\\
63.125	0.1206	10.608817468861\\
63.125	0.1212	10.7031714929542\\
63.125	0.1218	10.7975255170473\\
63.125	0.1224	10.8918795411404\\
63.125	0.123	10.9862335652335\\
63.5	0.093	6.29972400845718\\
63.5	0.0936	6.39654512020066\\
63.5	0.0942	6.49336623194414\\
63.5	0.0948	6.59018734368761\\
63.5	0.0954	6.68700845543109\\
63.5	0.096	6.78382956717457\\
63.5	0.0966	6.88065067891804\\
63.5	0.0972	6.97747179066153\\
63.5	0.0978	7.074292902405\\
63.5	0.0984	7.17111401414846\\
63.5	0.099	7.26793512589195\\
63.5	0.0996	7.36475623763542\\
63.5	0.1002	7.4615773493789\\
63.5	0.1008	7.55839846112238\\
63.5	0.1014	7.65521957286585\\
63.5	0.102	7.75204068460933\\
63.5	0.1026	7.84886179635281\\
63.5	0.1032	7.94568290809629\\
63.5	0.1038	8.04250401983977\\
63.5	0.1044	8.13932513158323\\
63.5	0.105	8.23614624332671\\
63.5	0.1056	8.3329673550702\\
63.5	0.1062	8.42978846681368\\
63.5	0.1068	8.52660957855716\\
63.5	0.1074	8.62343069030062\\
63.5	0.108	8.7202518020441\\
63.5	0.1086	8.81707291378758\\
63.5	0.1092	8.91389402553106\\
63.5	0.1098	9.01071513727453\\
63.5	0.1104	9.10753624901803\\
63.5	0.111	9.20435736076149\\
63.5	0.1116	9.30117847250496\\
63.5	0.1122	9.39799958424844\\
63.5	0.1128	9.49482069599193\\
63.5	0.1134	9.59164180773539\\
63.5	0.114	9.68846291947887\\
63.5	0.1146	9.78528403122235\\
63.5	0.1152	9.88210514296583\\
63.5	0.1158	9.97892625470929\\
63.5	0.1164	10.0757473664528\\
63.5	0.117	10.1725684781963\\
63.5	0.1176	10.2693895899397\\
63.5	0.1182	10.3662107016832\\
63.5	0.1188	10.4630318134267\\
63.5	0.1194	10.5598529251702\\
63.5	0.12	10.6566740369136\\
63.5	0.1206	10.7534951486571\\
63.5	0.1212	10.8503162604006\\
63.5	0.1218	10.9471373721441\\
63.5	0.1224	11.0439584838876\\
63.5	0.123	11.140779595631\\
63.875	0.093	6.33091565633698\\
63.875	0.0936	6.43020385573082\\
63.875	0.0942	6.52949205512464\\
63.875	0.0948	6.62878025451847\\
63.875	0.0954	6.72806845391229\\
63.875	0.096	6.82735665330613\\
63.875	0.0966	6.92664485269997\\
63.875	0.0972	7.02593305209379\\
63.875	0.0978	7.12522125148762\\
63.875	0.0984	7.22450945088145\\
63.875	0.099	7.32379765027529\\
63.875	0.0996	7.42308584966912\\
63.875	0.1002	7.52237404906295\\
63.875	0.1008	7.62166224845679\\
63.875	0.1014	7.7209504478506\\
63.875	0.102	7.82023864724444\\
63.875	0.1026	7.91952684663828\\
63.875	0.1032	8.0188150460321\\
63.875	0.1038	8.11810324542594\\
63.875	0.1044	8.21739144481977\\
63.875	0.105	8.31667964421359\\
63.875	0.1056	8.41596784360743\\
63.875	0.1062	8.51525604300126\\
63.875	0.1068	8.6145442423951\\
63.875	0.1074	8.71383244178892\\
63.875	0.108	8.81312064118275\\
63.875	0.1086	8.91240884057659\\
63.875	0.1092	9.01169703997043\\
63.875	0.1098	9.11098523936424\\
63.875	0.1104	9.21027343875809\\
63.875	0.111	9.30956163815191\\
63.875	0.1116	9.40884983754574\\
63.875	0.1122	9.50813803693957\\
63.875	0.1128	9.60742623633341\\
63.875	0.1134	9.70671443572724\\
63.875	0.114	9.80600263512106\\
63.875	0.1146	9.9052908345149\\
63.875	0.1152	10.0045790339087\\
63.875	0.1158	10.1038672333026\\
63.875	0.1164	10.2031554326964\\
63.875	0.117	10.3024436320902\\
63.875	0.1176	10.4017318314841\\
63.875	0.1182	10.5010200308779\\
63.875	0.1188	10.6003082302717\\
63.875	0.1194	10.6995964296655\\
63.875	0.12	10.7988846290594\\
63.875	0.1206	10.8981728284532\\
63.875	0.1212	10.997461027847\\
63.875	0.1218	11.0967492272409\\
63.875	0.1224	11.1960374266347\\
63.875	0.123	11.2953256260285\\
64.25	0.093	6.36210730421677\\
64.25	0.0936	6.46386259126096\\
64.25	0.0942	6.56561787830515\\
64.25	0.0948	6.66737316534932\\
64.25	0.0954	6.76912845239351\\
64.25	0.096	6.87088373943769\\
64.25	0.0966	6.97263902648189\\
64.25	0.0972	7.07439431352607\\
64.25	0.0978	7.17614960057025\\
64.25	0.0984	7.27790488761443\\
64.25	0.099	7.37966017465863\\
64.25	0.0996	7.48141546170281\\
64.25	0.1002	7.58317074874699\\
64.25	0.1008	7.68492603579118\\
64.25	0.1014	7.78668132283537\\
64.25	0.102	7.88843660987955\\
64.25	0.1026	7.99019189692373\\
64.25	0.1032	8.09194718396792\\
64.25	0.1038	8.19370247101212\\
64.25	0.1044	8.29545775805629\\
64.25	0.105	8.39721304510047\\
64.25	0.1056	8.49896833214467\\
64.25	0.1062	8.60072361918886\\
64.25	0.1068	8.70247890623304\\
64.25	0.1074	8.80423419327721\\
64.25	0.108	8.90598948032141\\
64.25	0.1086	9.00774476736559\\
64.25	0.1092	9.10950005440978\\
64.25	0.1098	9.21125534145395\\
64.25	0.1104	9.31301062849816\\
64.25	0.111	9.41476591554233\\
64.25	0.1116	9.51652120258652\\
64.25	0.1122	9.61827648963069\\
64.25	0.1128	9.7200317766749\\
64.25	0.1134	9.82178706371907\\
64.25	0.114	9.92354235076326\\
64.25	0.1146	10.0252976378074\\
64.25	0.1152	10.1270529248516\\
64.25	0.1158	10.2288082118958\\
64.25	0.1164	10.33056349894\\
64.25	0.117	10.4323187859842\\
64.25	0.1176	10.5340740730284\\
64.25	0.1182	10.6358293600725\\
64.25	0.1188	10.7375846471167\\
64.25	0.1194	10.8393399341609\\
64.25	0.12	10.9410952212051\\
64.25	0.1206	11.0428505082493\\
64.25	0.1212	11.1446057952935\\
64.25	0.1218	11.2463610823377\\
64.25	0.1224	11.3481163693819\\
64.25	0.123	11.449871656426\\
64.625	0.093	6.39329895209657\\
64.625	0.0936	6.49752132679112\\
64.625	0.0942	6.60174370148566\\
64.625	0.0948	6.70596607618019\\
64.625	0.0954	6.81018845087473\\
64.625	0.096	6.91441082556928\\
64.625	0.0966	7.01863320026382\\
64.625	0.0972	7.12285557495836\\
64.625	0.0978	7.22707794965289\\
64.625	0.0984	7.33130032434742\\
64.625	0.099	7.43552269904197\\
64.625	0.0996	7.5397450737365\\
64.625	0.1002	7.64396744843105\\
64.625	0.1008	7.74818982312559\\
64.625	0.1014	7.85241219782013\\
64.625	0.102	7.95663457251467\\
64.625	0.1026	8.0608569472092\\
64.625	0.1032	8.16507932190376\\
64.625	0.1038	8.26930169659828\\
64.625	0.1044	8.37352407129283\\
64.625	0.105	8.47774644598735\\
64.625	0.1056	8.58196882068191\\
64.625	0.1062	8.68619119537644\\
64.625	0.1068	8.790413570071\\
64.625	0.1074	8.89463594476553\\
64.625	0.108	8.99885831946007\\
64.625	0.1086	9.1030806941546\\
64.625	0.1092	9.20730306884914\\
64.625	0.1098	9.31152544354369\\
64.625	0.1104	9.41574781823822\\
64.625	0.111	9.51997019293275\\
64.625	0.1116	9.6241925676273\\
64.625	0.1122	9.72841494232183\\
64.625	0.1128	9.83263731701638\\
64.625	0.1134	9.93685969171092\\
64.625	0.114	10.0410820664055\\
64.625	0.1146	10.1453044411\\
64.625	0.1152	10.2495268157945\\
64.625	0.1158	10.3537491904891\\
64.625	0.1164	10.4579715651836\\
64.625	0.117	10.5621939398782\\
64.625	0.1176	10.6664163145727\\
64.625	0.1182	10.7706386892672\\
64.625	0.1188	10.8748610639618\\
64.625	0.1194	10.9790834386563\\
64.625	0.12	11.0833058133508\\
64.625	0.1206	11.1875281880454\\
64.625	0.1212	11.2917505627399\\
64.625	0.1218	11.3959729374345\\
64.625	0.1224	11.500195312129\\
64.625	0.123	11.6044176868235\\
65	0.093	6.42449059997637\\
65	0.0936	6.53118006232125\\
65	0.0942	6.63786952466616\\
65	0.0948	6.74455898701105\\
65	0.0954	6.85124844935594\\
65	0.096	6.95793791170082\\
65	0.0966	7.06462737404573\\
65	0.0972	7.17131683639062\\
65	0.0978	7.27800629873551\\
65	0.0984	7.38469576108041\\
65	0.099	7.4913852234253\\
65	0.0996	7.59807468577019\\
65	0.1002	7.70476414811508\\
65	0.1008	7.81145361045999\\
65	0.1014	7.91814307280487\\
65	0.102	8.02483253514977\\
65	0.1026	8.13152199749466\\
65	0.1032	8.23821145983956\\
65	0.1038	8.34490092218444\\
65	0.1044	8.45159038452934\\
65	0.105	8.55827984687423\\
65	0.1056	8.66496930921913\\
65	0.1062	8.77165877156402\\
65	0.1068	8.87834823390892\\
65	0.1074	8.98503769625381\\
65	0.108	9.0917271585987\\
65	0.1086	9.19841662094359\\
65	0.1092	9.30510608328849\\
65	0.1098	9.41179554563338\\
65	0.1104	9.51848500797828\\
65	0.111	9.62517447032316\\
65	0.1116	9.73186393266806\\
65	0.1122	9.83855339501295\\
65	0.1128	9.94524285735785\\
65	0.1134	10.0519323197027\\
65	0.114	10.1586217820476\\
65	0.1146	10.2653112443925\\
65	0.1152	10.3720007067374\\
65	0.1158	10.4786901690823\\
65	0.1164	10.5853796314272\\
65	0.117	10.6920690937721\\
65	0.1176	10.798758556117\\
65	0.1182	10.9054480184619\\
65	0.1188	11.0121374808068\\
65	0.1194	11.1188269431517\\
65	0.12	11.2255164054966\\
65	0.1206	11.3322058678415\\
65	0.1212	11.4388953301864\\
65	0.1218	11.5455847925313\\
65	0.1224	11.6522742548761\\
65	0.123	11.758963717221\\
65.375	0.093	6.45568224785616\\
65.375	0.0936	6.56483879785141\\
65.375	0.0942	6.67399534784666\\
65.375	0.0948	6.7831518978419\\
65.375	0.0954	6.89230844783715\\
65.375	0.096	7.0014649978324\\
65.375	0.0966	7.11062154782765\\
65.375	0.0972	7.21977809782289\\
65.375	0.0978	7.32893464781814\\
65.375	0.0984	7.43809119781339\\
65.375	0.099	7.54724774780864\\
65.375	0.0996	7.65640429780388\\
65.375	0.1002	7.76556084779912\\
65.375	0.1008	7.87471739779438\\
65.375	0.1014	7.98387394778963\\
65.375	0.102	8.09303049778488\\
65.375	0.1026	8.20218704778011\\
65.375	0.1032	8.31134359777538\\
65.375	0.1038	8.42050014777061\\
65.375	0.1044	8.52965669776587\\
65.375	0.105	8.63881324776111\\
65.375	0.1056	8.74796979775637\\
65.375	0.1062	8.8571263477516\\
65.375	0.1068	8.96628289774686\\
65.375	0.1074	9.07543944774211\\
65.375	0.108	9.18459599773736\\
65.375	0.1086	9.29375254773259\\
65.375	0.1092	9.40290909772784\\
65.375	0.1098	9.51206564772309\\
65.375	0.1104	9.62122219771834\\
65.375	0.111	9.73037874771359\\
65.375	0.1116	9.83953529770885\\
65.375	0.1122	9.94869184770408\\
65.375	0.1128	10.0578483976993\\
65.375	0.1134	10.1670049476946\\
65.375	0.114	10.2761614976898\\
65.375	0.1146	10.3853180476851\\
65.375	0.1152	10.4944745976803\\
65.375	0.1158	10.6036311476756\\
65.375	0.1164	10.7127876976708\\
65.375	0.117	10.8219442476661\\
65.375	0.1176	10.9311007976613\\
65.375	0.1182	11.0402573476566\\
65.375	0.1188	11.1494138976518\\
65.375	0.1194	11.2585704476471\\
65.375	0.12	11.3677269976423\\
65.375	0.1206	11.4768835476376\\
65.375	0.1212	11.5860400976328\\
65.375	0.1218	11.6951966476281\\
65.375	0.1224	11.8043531976233\\
65.375	0.123	11.9135097476185\\
65.75	0.093	6.48687389573595\\
65.75	0.0936	6.59849753338156\\
65.75	0.0942	6.71012117102717\\
65.75	0.0948	6.82174480867276\\
65.75	0.0954	6.93336844631837\\
65.75	0.096	7.04499208396396\\
65.75	0.0966	7.15661572160957\\
65.75	0.0972	7.26823935925517\\
65.75	0.0978	7.37986299690077\\
65.75	0.0984	7.49148663454638\\
65.75	0.099	7.60311027219197\\
65.75	0.0996	7.71473390983759\\
65.75	0.1002	7.82635754748317\\
65.75	0.1008	7.93798118512879\\
65.75	0.1014	8.04960482277438\\
65.75	0.102	8.16122846041999\\
65.75	0.1026	8.27285209806558\\
65.75	0.1032	8.38447573571119\\
65.75	0.1038	8.49609937335678\\
65.75	0.1044	8.60772301100239\\
65.75	0.105	8.71934664864798\\
65.75	0.1056	8.83097028629361\\
65.75	0.1062	8.94259392393919\\
65.75	0.1068	9.05421756158481\\
65.75	0.1074	9.1658411992304\\
65.75	0.108	9.27746483687601\\
65.75	0.1086	9.3890884745216\\
65.75	0.1092	9.50071211216721\\
65.75	0.1098	9.61233574981281\\
65.75	0.1104	9.72395938745841\\
65.75	0.111	9.835583025104\\
65.75	0.1116	9.94720666274962\\
65.75	0.1122	10.0588303003952\\
65.75	0.1128	10.1704539380408\\
65.75	0.1134	10.2820775756864\\
65.75	0.114	10.393701213332\\
65.75	0.1146	10.5053248509776\\
65.75	0.1152	10.6169484886232\\
65.75	0.1158	10.7285721262688\\
65.75	0.1164	10.8401957639144\\
65.75	0.117	10.95181940156\\
65.75	0.1176	11.0634430392056\\
65.75	0.1182	11.1750666768512\\
65.75	0.1188	11.2866903144968\\
65.75	0.1194	11.3983139521424\\
65.75	0.12	11.509937589788\\
65.75	0.1206	11.6215612274336\\
65.75	0.1212	11.7331848650792\\
65.75	0.1218	11.8448085027248\\
65.75	0.1224	11.9564321403704\\
65.75	0.123	12.068055778016\\
66.125	0.093	6.51806554361575\\
66.125	0.0936	6.63215626891171\\
66.125	0.0942	6.74624699420767\\
66.125	0.0948	6.86033771950362\\
66.125	0.0954	6.97442844479957\\
66.125	0.096	7.08851917009554\\
66.125	0.0966	7.20260989539149\\
66.125	0.0972	7.31670062068745\\
66.125	0.0978	7.43079134598339\\
66.125	0.0984	7.54488207127936\\
66.125	0.099	7.65897279657531\\
66.125	0.0996	7.77306352187128\\
66.125	0.1002	7.88715424716722\\
66.125	0.1008	8.0012449724632\\
66.125	0.1014	8.11533569775914\\
66.125	0.102	8.2294264230551\\
66.125	0.1026	8.34351714835104\\
66.125	0.1032	8.45760787364702\\
66.125	0.1038	8.57169859894296\\
66.125	0.1044	8.68578932423893\\
66.125	0.105	8.79988004953486\\
66.125	0.1056	8.91397077483083\\
66.125	0.1062	9.02806150012678\\
66.125	0.1068	9.14215222542275\\
66.125	0.1074	9.2562429507187\\
66.125	0.108	9.37033367601465\\
66.125	0.1086	9.4844244013106\\
66.125	0.1092	9.59851512660657\\
66.125	0.1098	9.71260585190252\\
66.125	0.1104	9.82669657719848\\
66.125	0.111	9.94078730249443\\
66.125	0.1116	10.0548780277904\\
66.125	0.1122	10.1689687530863\\
66.125	0.1128	10.2830594783823\\
66.125	0.1134	10.3971502036783\\
66.125	0.114	10.5112409289742\\
66.125	0.1146	10.6253316542702\\
66.125	0.1152	10.7394223795661\\
66.125	0.1158	10.8535131048621\\
66.125	0.1164	10.967603830158\\
66.125	0.117	11.081694555454\\
66.125	0.1176	11.1957852807499\\
66.125	0.1182	11.3098760060459\\
66.125	0.1188	11.4239667313419\\
66.125	0.1194	11.5380574566378\\
66.125	0.12	11.6521481819338\\
66.125	0.1206	11.7662389072297\\
66.125	0.1212	11.8803296325257\\
66.125	0.1218	11.9944203578217\\
66.125	0.1224	12.1085110831176\\
66.125	0.123	12.2226018084135\\
66.5	0.093	6.54925719149556\\
66.5	0.0936	6.66581500444187\\
66.5	0.0942	6.78237281738818\\
66.5	0.0948	6.89893063033448\\
66.5	0.0954	7.01548844328079\\
66.5	0.096	7.1320462562271\\
66.5	0.0966	7.24860406917341\\
66.5	0.0972	7.36516188211973\\
66.5	0.0978	7.48171969506603\\
66.5	0.0984	7.59827750801234\\
66.5	0.099	7.71483532095866\\
66.5	0.0996	7.83139313390497\\
66.5	0.1002	7.94795094685126\\
66.5	0.1008	8.06450875979759\\
66.5	0.1014	8.18106657274389\\
66.5	0.102	8.29762438569021\\
66.5	0.1026	8.41418219863651\\
66.5	0.1032	8.53074001158284\\
66.5	0.1038	8.64729782452913\\
66.5	0.1044	8.76385563747544\\
66.5	0.105	8.88041345042174\\
66.5	0.1056	8.99697126336807\\
66.5	0.1062	9.11352907631436\\
66.5	0.1068	9.23008688926069\\
66.5	0.1074	9.346644702207\\
66.5	0.108	9.46320251515331\\
66.5	0.1086	9.57976032809961\\
66.5	0.1092	9.69631814104592\\
66.5	0.1098	9.81287595399223\\
66.5	0.1104	9.92943376693854\\
66.5	0.111	10.0459915798848\\
66.5	0.1116	10.1625493928312\\
66.5	0.1122	10.2791072057775\\
66.5	0.1128	10.3956650187238\\
66.5	0.1134	10.5122228316701\\
66.5	0.114	10.6287806446164\\
66.5	0.1146	10.7453384575627\\
66.5	0.1152	10.861896270509\\
66.5	0.1158	10.9784540834553\\
66.5	0.1164	11.0950118964016\\
66.5	0.117	11.211569709348\\
66.5	0.1176	11.3281275222943\\
66.5	0.1182	11.4446853352406\\
66.5	0.1188	11.5612431481869\\
66.5	0.1194	11.6778009611332\\
66.5	0.12	11.7943587740795\\
66.5	0.1206	11.9109165870258\\
66.5	0.1212	12.0274743999721\\
66.5	0.1218	12.1440322129184\\
66.5	0.1224	12.2605900258647\\
66.5	0.123	12.3771478388111\\
66.875	0.093	6.58044883937535\\
66.875	0.0936	6.69947373997202\\
66.875	0.0942	6.81849864056868\\
66.875	0.0948	6.93752354116533\\
66.875	0.0954	7.056548441762\\
66.875	0.096	7.17557334235867\\
66.875	0.0966	7.29459824295533\\
66.875	0.0972	7.41362314355199\\
66.875	0.0978	7.53264804414866\\
66.875	0.0984	7.65167294474533\\
66.875	0.099	7.77069784534199\\
66.875	0.0996	7.88972274593866\\
66.875	0.1002	8.00874764653531\\
66.875	0.1008	8.127772547132\\
66.875	0.1014	8.24679744772865\\
66.875	0.102	8.36582234832532\\
66.875	0.1026	8.48484724892197\\
66.875	0.1032	8.60387214951865\\
66.875	0.1038	8.72289705011531\\
66.875	0.1044	8.84192195071198\\
66.875	0.105	8.96094685130862\\
66.875	0.1056	9.0799717519053\\
66.875	0.1062	9.19899665250196\\
66.875	0.1068	9.31802155309863\\
66.875	0.1074	9.43704645369529\\
66.875	0.108	9.55607135429196\\
66.875	0.1086	9.67509625488862\\
66.875	0.1092	9.79412115548529\\
66.875	0.1098	9.91314605608196\\
66.875	0.1104	10.0321709566786\\
66.875	0.111	10.1511958572753\\
66.875	0.1116	10.270220757872\\
66.875	0.1122	10.3892456584686\\
66.875	0.1128	10.5082705590653\\
66.875	0.1134	10.6272954596619\\
66.875	0.114	10.7463203602586\\
66.875	0.1146	10.8653452608552\\
66.875	0.1152	10.9843701614519\\
66.875	0.1158	11.1033950620486\\
66.875	0.1164	11.2224199626453\\
66.875	0.117	11.3414448632419\\
66.875	0.1176	11.4604697638386\\
66.875	0.1182	11.5794946644352\\
66.875	0.1188	11.6985195650319\\
66.875	0.1194	11.8175444656286\\
66.875	0.12	11.9365693662252\\
66.875	0.1206	12.0555942668219\\
66.875	0.1212	12.1746191674186\\
66.875	0.1218	12.2936440680152\\
66.875	0.1224	12.4126689686119\\
66.875	0.123	12.5316938692086\\
67.25	0.093	6.61164048725514\\
67.25	0.0936	6.73313247550216\\
67.25	0.0942	6.85462446374919\\
67.25	0.0948	6.97611645199619\\
67.25	0.0954	7.09760844024322\\
67.25	0.096	7.21910042849024\\
67.25	0.0966	7.34059241673725\\
67.25	0.0972	7.46208440498428\\
67.25	0.0978	7.58357639323128\\
67.25	0.0984	7.70506838147831\\
67.25	0.099	7.82656036972533\\
67.25	0.0996	7.94805235797236\\
67.25	0.1002	8.06954434621936\\
67.25	0.1008	8.19103633446639\\
67.25	0.1014	8.3125283227134\\
67.25	0.102	8.43402031096042\\
67.25	0.1026	8.55551229920744\\
67.25	0.1032	8.67700428745447\\
67.25	0.1038	8.79849627570147\\
67.25	0.1044	8.9199882639485\\
67.25	0.105	9.0414802521955\\
67.25	0.1056	9.16297224044254\\
67.25	0.1062	9.28446422868954\\
67.25	0.1068	9.40595621693657\\
67.25	0.1074	9.52744820518359\\
67.25	0.108	9.6489401934306\\
67.25	0.1086	9.77043218167762\\
67.25	0.1092	9.89192416992464\\
67.25	0.1098	10.0134161581717\\
67.25	0.1104	10.1349081464187\\
67.25	0.111	10.2564001346657\\
67.25	0.1116	10.3778921229127\\
67.25	0.1122	10.4993841111597\\
67.25	0.1128	10.6208760994068\\
67.25	0.1134	10.7423680876538\\
67.25	0.114	10.8638600759008\\
67.25	0.1146	10.9853520641478\\
67.25	0.1152	11.1068440523948\\
67.25	0.1158	11.2283360406418\\
67.25	0.1164	11.3498280288889\\
67.25	0.117	11.4713200171359\\
67.25	0.1176	11.5928120053829\\
67.25	0.1182	11.7143039936299\\
67.25	0.1188	11.8357959818769\\
67.25	0.1194	11.957287970124\\
67.25	0.12	12.078779958371\\
67.25	0.1206	12.200271946618\\
67.25	0.1212	12.321763934865\\
67.25	0.1218	12.443255923112\\
67.25	0.1224	12.564747911359\\
67.25	0.123	12.6862398996061\\
67.625	0.093	6.64283213513495\\
67.625	0.0936	6.76679121103232\\
67.625	0.0942	6.89075028692969\\
67.625	0.0948	7.01470936282706\\
67.625	0.0954	7.13866843872443\\
67.625	0.096	7.2626275146218\\
67.625	0.0966	7.38658659051917\\
67.625	0.0972	7.51054566641656\\
67.625	0.0978	7.63450474231392\\
67.625	0.0984	7.75846381821131\\
67.625	0.099	7.88242289410867\\
67.625	0.0996	8.00638197000605\\
67.625	0.1002	8.13034104590342\\
67.625	0.1008	8.2543001218008\\
67.625	0.1014	8.37825919769816\\
67.625	0.102	8.50221827359553\\
67.625	0.1026	8.62617734949291\\
67.625	0.1032	8.75013642539028\\
67.625	0.1038	8.87409550128766\\
67.625	0.1044	8.99805457718503\\
67.625	0.105	9.1220136530824\\
67.625	0.1056	9.24597272897978\\
67.625	0.1062	9.36993180487715\\
67.625	0.1068	9.49389088077453\\
67.625	0.1074	9.6178499566719\\
67.625	0.108	9.74180903256926\\
67.625	0.1086	9.86576810846664\\
67.625	0.1092	9.989727184364\\
67.625	0.1098	10.1136862602614\\
67.625	0.1104	10.2376453361587\\
67.625	0.111	10.3616044120561\\
67.625	0.1116	10.4855634879535\\
67.625	0.1122	10.6095225638509\\
67.625	0.1128	10.7334816397482\\
67.625	0.1134	10.8574407156456\\
67.625	0.114	10.981399791543\\
67.625	0.1146	11.1053588674404\\
67.625	0.1152	11.2293179433377\\
67.625	0.1158	11.3532770192351\\
67.625	0.1164	11.4772360951325\\
67.625	0.117	11.6011951710299\\
67.625	0.1176	11.7251542469272\\
67.625	0.1182	11.8491133228246\\
67.625	0.1188	11.973072398722\\
67.625	0.1194	12.0970314746193\\
67.625	0.12	12.2209905505167\\
67.625	0.1206	12.3449496264141\\
67.625	0.1212	12.4689087023115\\
67.625	0.1218	12.5928677782088\\
67.625	0.1224	12.7168268541062\\
67.625	0.123	12.8407859300036\\
68	0.093	6.67402378301474\\
68	0.0936	6.80044994656247\\
68	0.0942	6.92687611011021\\
68	0.0948	7.05330227365793\\
68	0.0954	7.17972843720565\\
68	0.096	7.30615460075339\\
68	0.0966	7.43258076430111\\
68	0.0972	7.55900692784883\\
68	0.0978	7.68543309139655\\
68	0.0984	7.8118592549443\\
68	0.099	7.938285418492\\
68	0.0996	8.06471158203973\\
68	0.1002	8.19113774558747\\
68	0.1008	8.3175639091352\\
68	0.1014	8.44399007268291\\
68	0.102	8.57041623623064\\
68	0.1026	8.69684239977838\\
68	0.1032	8.8232685633261\\
68	0.1038	8.94969472687383\\
68	0.1044	9.07612089042155\\
68	0.105	9.20254705396928\\
68	0.1056	9.328973217517\\
68	0.1062	9.45539938106474\\
68	0.1068	9.58182554461247\\
68	0.1074	9.7082517081602\\
68	0.108	9.83467787170791\\
68	0.1086	9.96110403525564\\
68	0.1092	10.0875301988034\\
68	0.1098	10.2139563623511\\
68	0.1104	10.3403825258988\\
68	0.111	10.4668086894466\\
68	0.1116	10.5932348529943\\
68	0.1122	10.719661016542\\
68	0.1128	10.8460871800897\\
68	0.1134	10.9725133436375\\
68	0.114	11.0989395071852\\
68	0.1146	11.2253656707329\\
68	0.1152	11.3517918342806\\
68	0.1158	11.4782179978284\\
68	0.1164	11.6046441613761\\
68	0.117	11.7310703249238\\
68	0.1176	11.8574964884715\\
68	0.1182	11.9839226520193\\
68	0.1188	12.110348815567\\
68	0.1194	12.2367749791147\\
68	0.12	12.3632011426624\\
68	0.1206	12.4896273062102\\
68	0.1212	12.6160534697579\\
68	0.1218	12.7424796333056\\
68	0.1224	12.8689057968533\\
68	0.123	12.9953319604011\\
68.375	0.093	6.70521543089455\\
68.375	0.0936	6.83410868209263\\
68.375	0.0942	6.96300193329071\\
68.375	0.0948	7.09189518448878\\
68.375	0.0954	7.22078843568686\\
68.375	0.096	7.34968168688495\\
68.375	0.0966	7.47857493808303\\
68.375	0.0972	7.60746818928111\\
68.375	0.0978	7.73636144047917\\
68.375	0.0984	7.86525469167728\\
68.375	0.099	7.99414794287534\\
68.375	0.0996	8.12304119407344\\
68.375	0.1002	8.25193444527152\\
68.375	0.1008	8.38082769646959\\
68.375	0.1014	8.50972094766766\\
68.375	0.102	8.63861419886575\\
68.375	0.1026	8.76750745006383\\
68.375	0.1032	8.89640070126192\\
68.375	0.1038	9.02529395246\\
68.375	0.1044	9.15418720365808\\
68.375	0.105	9.28308045485616\\
68.375	0.1056	9.41197370605424\\
68.375	0.1062	9.54086695725232\\
68.375	0.1068	9.66976020845041\\
68.375	0.1074	9.79865345964849\\
68.375	0.108	9.92754671084657\\
68.375	0.1086	10.0564399620446\\
68.375	0.1092	10.1853332132427\\
68.375	0.1098	10.3142264644408\\
68.375	0.1104	10.4431197156389\\
68.375	0.111	10.572012966837\\
68.375	0.1116	10.7009062180351\\
68.375	0.1122	10.8297994692331\\
68.375	0.1128	10.9586927204312\\
68.375	0.1134	11.0875859716293\\
68.375	0.114	11.2164792228274\\
68.375	0.1146	11.3453724740255\\
68.375	0.1152	11.4742657252235\\
68.375	0.1158	11.6031589764216\\
68.375	0.1164	11.7320522276197\\
68.375	0.117	11.8609454788178\\
68.375	0.1176	11.9898387300158\\
68.375	0.1182	12.118731981214\\
68.375	0.1188	12.247625232412\\
68.375	0.1194	12.3765184836101\\
68.375	0.12	12.5054117348082\\
68.375	0.1206	12.6343049860063\\
68.375	0.1212	12.7631982372043\\
68.375	0.1218	12.8920914884024\\
68.375	0.1224	13.0209847396005\\
68.375	0.123	13.1498779907986\\
68.75	0.093	6.73640707877434\\
68.75	0.0936	6.86776741762278\\
68.75	0.0942	6.99912775647121\\
68.75	0.0948	7.13048809531965\\
68.75	0.0954	7.26184843416807\\
68.75	0.096	7.39320877301653\\
68.75	0.0966	7.52456911186495\\
68.75	0.0972	7.65592945071339\\
68.75	0.0978	7.78728978956181\\
68.75	0.0984	7.91865012841026\\
68.75	0.099	8.05001046725869\\
68.75	0.0996	8.18137080610713\\
68.75	0.1002	8.31273114495556\\
68.75	0.1008	8.444091483804\\
68.75	0.1014	8.57545182265243\\
68.75	0.102	8.70681216150086\\
68.75	0.1026	8.8381725003493\\
68.75	0.1032	8.96953283919774\\
68.75	0.1038	9.10089317804618\\
68.75	0.1044	9.2322535168946\\
68.75	0.105	9.36361385574304\\
68.75	0.1056	9.49497419459148\\
68.75	0.1062	9.62633453343992\\
68.75	0.1068	9.75769487228835\\
68.75	0.1074	9.88905521113679\\
68.75	0.108	10.0204155499852\\
68.75	0.1086	10.1517758888337\\
68.75	0.1092	10.2831362276821\\
68.75	0.1098	10.4144965665305\\
68.75	0.1104	10.545856905379\\
68.75	0.111	10.6772172442274\\
68.75	0.1116	10.8085775830758\\
68.75	0.1122	10.9399379219243\\
68.75	0.1128	11.0712982607727\\
68.75	0.1134	11.2026585996211\\
68.75	0.114	11.3340189384696\\
68.75	0.1146	11.465379277318\\
68.75	0.1152	11.5967396161664\\
68.75	0.1158	11.7280999550149\\
68.75	0.1164	11.8594602938633\\
68.75	0.117	11.9908206327117\\
68.75	0.1176	12.1221809715602\\
68.75	0.1182	12.2535413104086\\
68.75	0.1188	12.384901649257\\
68.75	0.1194	12.5162619881055\\
68.75	0.12	12.6476223269539\\
68.75	0.1206	12.7789826658024\\
68.75	0.1212	12.9103430046508\\
68.75	0.1218	13.0417033434992\\
68.75	0.1224	13.1730636823476\\
68.75	0.123	13.3044240211961\\
69.125	0.093	6.76759872665412\\
69.125	0.0936	6.90142615315293\\
69.125	0.0942	7.03525357965169\\
69.125	0.0948	7.16908100615049\\
69.125	0.0954	7.30290843264928\\
69.125	0.096	7.43673585914807\\
69.125	0.0966	7.57056328564686\\
69.125	0.0972	7.70439071214565\\
69.125	0.0978	7.83821813864442\\
69.125	0.0984	7.97204556514323\\
69.125	0.099	8.10587299164202\\
69.125	0.0996	8.2397004181408\\
69.125	0.1002	8.3735278446396\\
69.125	0.1008	8.50735527113839\\
69.125	0.1014	8.64118269763716\\
69.125	0.102	8.77501012413596\\
69.125	0.1026	8.90883755063476\\
69.125	0.1032	9.04266497713354\\
69.125	0.1038	9.17649240363234\\
69.125	0.1044	9.31031983013112\\
69.125	0.105	9.4441472566299\\
69.125	0.1056	9.5779746831287\\
69.125	0.1062	9.71180210962748\\
69.125	0.1068	9.84562953612628\\
69.125	0.1074	9.97945696262508\\
69.125	0.108	10.1132843891238\\
69.125	0.1086	10.2471118156226\\
69.125	0.1092	10.3809392421214\\
69.125	0.1098	10.5147666686202\\
69.125	0.1104	10.648594095119\\
69.125	0.111	10.7824215216178\\
69.125	0.1116	10.9162489481166\\
69.125	0.1122	11.0500763746154\\
69.125	0.1128	11.1839038011142\\
69.125	0.1134	11.317731227613\\
69.125	0.114	11.4515586541118\\
69.125	0.1146	11.5853860806105\\
69.125	0.1152	11.7192135071093\\
69.125	0.1158	11.8530409336081\\
69.125	0.1164	11.9868683601069\\
69.125	0.117	12.1206957866057\\
69.125	0.1176	12.2545232131045\\
69.125	0.1182	12.3883506396033\\
69.125	0.1188	12.5221780661021\\
69.125	0.1194	12.6560054926009\\
69.125	0.12	12.7898329190996\\
69.125	0.1206	12.9236603455984\\
69.125	0.1212	13.0574877720972\\
69.125	0.1218	13.191315198596\\
69.125	0.1224	13.3251426250948\\
69.125	0.123	13.4589700515936\\
69.5	0.093	6.79879037453391\\
69.5	0.0936	6.93508488868308\\
69.5	0.0942	7.07137940283221\\
69.5	0.0948	7.20767391698135\\
69.5	0.0954	7.34396843113049\\
69.5	0.096	7.48026294527965\\
69.5	0.0966	7.61655745942878\\
69.5	0.0972	7.75285197357793\\
69.5	0.0978	7.88914648772706\\
69.5	0.0984	8.02544100187622\\
69.5	0.099	8.16173551602535\\
69.5	0.0996	8.2980300301745\\
69.5	0.1002	8.43432454432364\\
69.5	0.1008	8.57061905847279\\
69.5	0.1014	8.70691357262191\\
69.5	0.102	8.84320808677107\\
69.5	0.1026	8.97950260092021\\
69.5	0.1032	9.11579711506936\\
69.5	0.1038	9.25209162921851\\
69.5	0.1044	9.38838614336764\\
69.5	0.105	9.52468065751678\\
69.5	0.1056	9.66097517166594\\
69.5	0.1062	9.79726968581508\\
69.5	0.1068	9.93356419996422\\
69.5	0.1074	10.0698587141134\\
69.5	0.108	10.2061532282625\\
69.5	0.1086	10.3424477424117\\
69.5	0.1092	10.4787422565608\\
69.5	0.1098	10.6150367707099\\
69.5	0.1104	10.7513312848591\\
69.5	0.111	10.8876257990082\\
69.5	0.1116	11.0239203131574\\
69.5	0.1122	11.1602148273065\\
69.5	0.1128	11.2965093414556\\
69.5	0.1134	11.4328038556048\\
69.5	0.114	11.5690983697539\\
69.5	0.1146	11.7053928839031\\
69.5	0.1152	11.8416873980522\\
69.5	0.1158	11.9779819122014\\
69.5	0.1164	12.1142764263505\\
69.5	0.117	12.2505709404997\\
69.5	0.1176	12.3868654546488\\
69.5	0.1182	12.523159968798\\
69.5	0.1188	12.6594544829471\\
69.5	0.1194	12.7957489970962\\
69.5	0.12	12.9320435112454\\
69.5	0.1206	13.0683380253945\\
69.5	0.1212	13.2046325395437\\
69.5	0.1218	13.3409270536928\\
69.5	0.1224	13.4772215678419\\
69.5	0.123	13.6135160819911\\
69.875	0.093	6.82998202241372\\
69.875	0.0936	6.96874362421322\\
69.875	0.0942	7.10750522601271\\
69.875	0.0948	7.24626682781221\\
69.875	0.0954	7.38502842961169\\
69.875	0.096	7.52379003141121\\
69.875	0.0966	7.6625516332107\\
69.875	0.0972	7.8013132350102\\
69.875	0.0978	7.94007483680969\\
69.875	0.0984	8.0788364386092\\
69.875	0.099	8.21759804040869\\
69.875	0.0996	8.35635964220819\\
69.875	0.1002	8.49512124400769\\
69.875	0.1008	8.63388284580719\\
69.875	0.1014	8.77264444760668\\
69.875	0.102	8.91140604940618\\
69.875	0.1026	9.05016765120568\\
69.875	0.1032	9.18892925300518\\
69.875	0.1038	9.32769085480467\\
69.875	0.1044	9.46645245660417\\
69.875	0.105	9.60521405840366\\
69.875	0.1056	9.74397566020316\\
69.875	0.1062	9.88273726200266\\
69.875	0.1068	10.0214988638022\\
69.875	0.1074	10.1602604656017\\
69.875	0.108	10.2990220674012\\
69.875	0.1086	10.4377836692007\\
69.875	0.1092	10.5765452710001\\
69.875	0.1098	10.7153068727997\\
69.875	0.1104	10.8540684745991\\
69.875	0.111	10.9928300763986\\
69.875	0.1116	11.1315916781981\\
69.875	0.1122	11.2703532799977\\
69.875	0.1128	11.4091148817971\\
69.875	0.1134	11.5478764835966\\
69.875	0.114	11.6866380853961\\
69.875	0.1146	11.8253996871956\\
69.875	0.1152	11.9641612889951\\
69.875	0.1158	12.1029228907946\\
69.875	0.1164	12.2416844925941\\
69.875	0.117	12.3804460943936\\
69.875	0.1176	12.5192076961931\\
69.875	0.1182	12.6579692979926\\
69.875	0.1188	12.7967308997921\\
69.875	0.1194	12.9354925015916\\
69.875	0.12	13.0742541033911\\
69.875	0.1206	13.2130157051906\\
69.875	0.1212	13.3517773069901\\
69.875	0.1218	13.4905389087896\\
69.875	0.1224	13.6293005105891\\
69.875	0.123	13.7680621123886\\
70.25	0.093	6.86117367029351\\
70.25	0.0936	7.00240235974337\\
70.25	0.0942	7.14363104919322\\
70.25	0.0948	7.28485973864308\\
70.25	0.0954	7.42608842809291\\
70.25	0.096	7.56731711754279\\
70.25	0.0966	7.70854580699262\\
70.25	0.0972	7.84977449644248\\
70.25	0.0978	7.99100318589231\\
70.25	0.0984	8.13223187534219\\
70.25	0.099	8.27346056479203\\
70.25	0.0996	8.41468925424188\\
70.25	0.1002	8.55591794369174\\
70.25	0.1008	8.69714663314159\\
70.25	0.1014	8.83837532259143\\
70.25	0.102	8.97960401204129\\
70.25	0.1026	9.12083270149114\\
70.25	0.1032	9.262061390941\\
70.25	0.1038	9.40329008039085\\
70.25	0.1044	9.54451876984069\\
70.25	0.105	9.68574745929055\\
70.25	0.1056	9.8269761487404\\
70.25	0.1062	9.96820483819026\\
70.25	0.1068	10.1094335276401\\
70.25	0.1074	10.25066221709\\
70.25	0.108	10.3918909065398\\
70.25	0.1086	10.5331195959897\\
70.25	0.1092	10.6743482854395\\
70.25	0.1098	10.8155769748894\\
70.25	0.1104	10.9568056643392\\
70.25	0.111	11.0980343537891\\
70.25	0.1116	11.2392630432389\\
70.25	0.1122	11.3804917326888\\
70.25	0.1128	11.5217204221386\\
70.25	0.1134	11.6629491115885\\
70.25	0.114	11.8041778010383\\
70.25	0.1146	11.9454064904882\\
70.25	0.1152	12.086635179938\\
70.25	0.1158	12.2278638693879\\
70.25	0.1164	12.3690925588377\\
70.25	0.117	12.5103212482876\\
70.25	0.1176	12.6515499377374\\
70.25	0.1182	12.7927786271873\\
70.25	0.1188	12.9340073166371\\
70.25	0.1194	13.075236006087\\
70.25	0.12	13.2164646955368\\
70.25	0.1206	13.3576933849867\\
70.25	0.1212	13.4989220744365\\
70.25	0.1218	13.6401507638864\\
70.25	0.1224	13.7813794533362\\
70.25	0.123	13.9226081427861\\
70.625	0.093	6.8923653181733\\
70.625	0.0936	7.03606109527352\\
70.625	0.0942	7.17975687237372\\
70.625	0.0948	7.32345264947394\\
70.625	0.0954	7.46714842657413\\
70.625	0.096	7.61084420367435\\
70.625	0.0966	7.75453998077454\\
70.625	0.0972	7.89823575787476\\
70.625	0.0978	8.04193153497495\\
70.625	0.0984	8.18562731207517\\
70.625	0.099	8.32932308917536\\
70.625	0.0996	8.47301886627558\\
70.625	0.1002	8.61671464337579\\
70.625	0.1008	8.76041042047599\\
70.625	0.1014	8.90410619757618\\
70.625	0.102	9.0478019746764\\
70.625	0.1026	9.19149775177661\\
70.625	0.1032	9.33519352887681\\
70.625	0.1038	9.47888930597702\\
70.625	0.1044	9.62258508307721\\
70.625	0.105	9.76628086017743\\
70.625	0.1056	9.90997663727764\\
70.625	0.1062	10.0536724143778\\
70.625	0.1068	10.197368191478\\
70.625	0.1074	10.3410639685783\\
70.625	0.108	10.4847597456785\\
70.625	0.1086	10.6284555227787\\
70.625	0.1092	10.7721512998789\\
70.625	0.1098	10.9158470769791\\
70.625	0.1104	11.0595428540793\\
70.625	0.111	11.2032386311795\\
70.625	0.1116	11.3469344082797\\
70.625	0.1122	11.4906301853799\\
70.625	0.1128	11.6343259624801\\
70.625	0.1134	11.7780217395803\\
70.625	0.114	11.9217175166805\\
70.625	0.1146	12.0654132937807\\
70.625	0.1152	12.2091090708809\\
70.625	0.1158	12.3528048479811\\
70.625	0.1164	12.4965006250813\\
70.625	0.117	12.6401964021815\\
70.625	0.1176	12.7838921792818\\
70.625	0.1182	12.927587956382\\
70.625	0.1188	13.0712837334822\\
70.625	0.1194	13.2149795105824\\
70.625	0.12	13.3586752876826\\
70.625	0.1206	13.5023710647828\\
70.625	0.1212	13.646066841883\\
70.625	0.1218	13.7897626189832\\
70.625	0.1224	13.9334583960834\\
70.625	0.123	14.0771541731836\\
71	0.093	6.9235569660531\\
71	0.0936	7.06971983080368\\
71	0.0942	7.21588269555423\\
71	0.0948	7.36204556030479\\
71	0.0954	7.50820842505534\\
71	0.096	7.65437128980592\\
71	0.0966	7.80053415455646\\
71	0.0972	7.94669701930702\\
71	0.0978	8.09285988405757\\
71	0.0984	8.23902274880815\\
71	0.099	8.3851856135587\\
71	0.0996	8.53134847830927\\
71	0.1002	8.67751134305983\\
71	0.1008	8.8236742078104\\
71	0.1014	8.96983707256094\\
71	0.102	9.1159999373115\\
71	0.1026	9.26216280206206\\
71	0.1032	9.40832566681263\\
71	0.1038	9.55448853156319\\
71	0.1044	9.70065139631375\\
71	0.105	9.84681426106431\\
71	0.1056	9.99297712581487\\
71	0.1062	10.1391399905654\\
71	0.1068	10.285302855316\\
71	0.1074	10.4314657200666\\
71	0.108	10.5776285848171\\
71	0.1086	10.7237914495677\\
71	0.1092	10.8699543143182\\
71	0.1098	11.0161171790688\\
71	0.1104	11.1622800438194\\
71	0.111	11.3084429085699\\
71	0.1116	11.4546057733205\\
71	0.1122	11.600768638071\\
71	0.1128	11.7469315028216\\
71	0.1134	11.8930943675722\\
71	0.114	12.0392572323227\\
71	0.1146	12.1854200970733\\
71	0.1152	12.3315829618238\\
71	0.1158	12.4777458265744\\
71	0.1164	12.6239086913249\\
71	0.117	12.7700715560755\\
71	0.1176	12.9162344208261\\
71	0.1182	13.0623972855766\\
71	0.1188	13.2085601503272\\
71	0.1194	13.3547230150778\\
71	0.12	13.5008858798283\\
71	0.1206	13.6470487445789\\
71	0.1212	13.7932116093294\\
71	0.1218	13.93937447408\\
71	0.1224	14.0855373388305\\
71	0.123	14.2317002035811\\
71.375	0.093	6.95474861393289\\
71.375	0.0936	7.10337856633383\\
71.375	0.0942	7.25200851873473\\
71.375	0.0948	7.40063847113565\\
71.375	0.0954	7.54926842353655\\
71.375	0.096	7.69789837593748\\
71.375	0.0966	7.84652832833838\\
71.375	0.0972	7.9951582807393\\
71.375	0.0978	8.14378823314021\\
71.375	0.0984	8.29241818554114\\
71.375	0.099	8.44104813794205\\
71.375	0.0996	8.58967809034296\\
71.375	0.1002	8.73830804274388\\
71.375	0.1008	8.88693799514479\\
71.375	0.1014	9.0355679475457\\
71.375	0.102	9.18419789994661\\
71.375	0.1026	9.33282785234753\\
71.375	0.1032	9.48145780474844\\
71.375	0.1038	9.63008775714937\\
71.375	0.1044	9.77871770955026\\
71.375	0.105	9.92734766195119\\
71.375	0.1056	10.0759776143521\\
71.375	0.1062	10.224607566753\\
71.375	0.1068	10.3732375191539\\
71.375	0.1074	10.5218674715549\\
71.375	0.108	10.6704974239558\\
71.375	0.1086	10.8191273763567\\
71.375	0.1092	10.9677573287576\\
71.375	0.1098	11.1163872811585\\
71.375	0.1104	11.2650172335594\\
71.375	0.111	11.4136471859603\\
71.375	0.1116	11.5622771383613\\
71.375	0.1122	11.7109070907622\\
71.375	0.1128	11.8595370431631\\
71.375	0.1134	12.008166995564\\
71.375	0.114	12.1567969479649\\
71.375	0.1146	12.3054269003658\\
71.375	0.1152	12.4540568527667\\
71.375	0.1158	12.6026868051677\\
71.375	0.1164	12.7513167575686\\
71.375	0.117	12.8999467099695\\
71.375	0.1176	13.0485766623704\\
71.375	0.1182	13.1972066147713\\
71.375	0.1188	13.3458365671722\\
71.375	0.1194	13.4944665195731\\
71.375	0.12	13.643096471974\\
71.375	0.1206	13.791726424375\\
71.375	0.1212	13.9403563767759\\
71.375	0.1218	14.0889863291768\\
71.375	0.1224	14.2376162815777\\
71.375	0.123	14.3862462339786\\
71.75	0.093	6.9859402618127\\
71.75	0.0936	7.13703730186398\\
71.75	0.0942	7.28813434191524\\
71.75	0.0948	7.43923138196651\\
71.75	0.0954	7.59032842201776\\
71.75	0.096	7.74142546206905\\
71.75	0.0966	7.8925225021203\\
71.75	0.0972	8.04361954217158\\
71.75	0.0978	8.19471658222284\\
71.75	0.0984	8.34581362227412\\
71.75	0.099	8.49691066232538\\
71.75	0.0996	8.64800770237666\\
71.75	0.1002	8.79910474242793\\
71.75	0.1008	8.9502017824792\\
71.75	0.1014	9.10129882253045\\
71.75	0.102	9.25239586258172\\
71.75	0.1026	9.40349290263299\\
71.75	0.1032	9.55458994268426\\
71.75	0.1038	9.70568698273554\\
71.75	0.1044	9.8567840227868\\
71.75	0.105	10.0078810628381\\
71.75	0.1056	10.1589781028893\\
71.75	0.1062	10.3100751429406\\
71.75	0.1068	10.4611721829919\\
71.75	0.1074	10.6122692230432\\
71.75	0.108	10.7633662630944\\
71.75	0.1086	10.9144633031457\\
71.75	0.1092	11.0655603431969\\
71.75	0.1098	11.2166573832482\\
71.75	0.1104	11.3677544232995\\
71.75	0.111	11.5188514633508\\
71.75	0.1116	11.669948503402\\
71.75	0.1122	11.8210455434533\\
71.75	0.1128	11.9721425835045\\
71.75	0.1134	12.1232396235558\\
71.75	0.114	12.2743366636071\\
71.75	0.1146	12.4254337036584\\
71.75	0.1152	12.5765307437096\\
71.75	0.1158	12.7276277837609\\
71.75	0.1164	12.8787248238122\\
71.75	0.117	13.0298218638634\\
71.75	0.1176	13.1809189039147\\
71.75	0.1182	13.332015943966\\
71.75	0.1188	13.4831129840172\\
71.75	0.1194	13.6342100240685\\
71.75	0.12	13.7853070641198\\
71.75	0.1206	13.936404104171\\
71.75	0.1212	14.0875011442223\\
71.75	0.1218	14.2385981842736\\
71.75	0.1224	14.3896952243248\\
71.75	0.123	14.5407922643761\\
72.125	0.093	7.01713190969252\\
72.125	0.0936	7.17069603739414\\
72.125	0.0942	7.32426016509577\\
72.125	0.0948	7.47782429279737\\
72.125	0.0954	7.63138842049901\\
72.125	0.096	7.78495254820062\\
72.125	0.0966	7.93851667590226\\
72.125	0.0972	8.09208080360388\\
72.125	0.0978	8.24564493130549\\
72.125	0.0984	8.39920905900711\\
72.125	0.099	8.55277318670875\\
72.125	0.0996	8.70633731441036\\
72.125	0.1002	8.85990144211199\\
72.125	0.1008	9.0134655698136\\
72.125	0.1014	9.16702969751523\\
72.125	0.102	9.32059382521685\\
72.125	0.1026	9.47415795291847\\
72.125	0.1032	9.6277220806201\\
72.125	0.1038	9.78128620832172\\
72.125	0.1044	9.93485033602333\\
72.125	0.105	10.088414463725\\
72.125	0.1056	10.2419785914266\\
72.125	0.1062	10.3955427191282\\
72.125	0.1068	10.5491068468298\\
72.125	0.1074	10.7026709745314\\
72.125	0.108	10.8562351022331\\
72.125	0.1086	11.0097992299347\\
72.125	0.1092	11.1633633576363\\
72.125	0.1098	11.3169274853379\\
72.125	0.1104	11.4704916130396\\
72.125	0.111	11.6240557407412\\
72.125	0.1116	11.7776198684428\\
72.125	0.1122	11.9311839961444\\
72.125	0.1128	12.0847481238461\\
72.125	0.1134	12.2383122515477\\
72.125	0.114	12.3918763792493\\
72.125	0.1146	12.5454405069509\\
72.125	0.1152	12.6990046346526\\
72.125	0.1158	12.8525687623542\\
72.125	0.1164	13.0061328900558\\
72.125	0.117	13.1596970177574\\
72.125	0.1176	13.313261145459\\
72.125	0.1182	13.4668252731607\\
72.125	0.1188	13.6203894008623\\
72.125	0.1194	13.7739535285639\\
72.125	0.12	13.9275176562655\\
72.125	0.1206	14.0810817839671\\
72.125	0.1212	14.2346459116688\\
72.125	0.1218	14.3882100393704\\
72.125	0.1224	14.541774167072\\
72.125	0.123	14.6953382947736\\
72.5	0.093	7.04832355757232\\
72.5	0.0936	7.20435477292428\\
72.5	0.0942	7.36038598827628\\
72.5	0.0948	7.51641720362822\\
72.5	0.0954	7.67244841898022\\
72.5	0.096	7.82847963433218\\
72.5	0.0966	7.98451084968418\\
72.5	0.0972	8.14054206503614\\
72.5	0.0978	8.29657328038813\\
72.5	0.0984	8.45260449574009\\
72.5	0.099	8.60863571109208\\
72.5	0.0996	8.76466692644405\\
72.5	0.1002	8.92069814179604\\
72.5	0.1008	9.07672935714801\\
72.5	0.1014	9.23276057249998\\
72.5	0.102	9.38879178785196\\
72.5	0.1026	9.54482300320393\\
72.5	0.1032	9.70085421855592\\
72.5	0.1038	9.85688543390789\\
72.5	0.1044	10.0129166492599\\
72.5	0.105	10.1689478646118\\
72.5	0.1056	10.3249790799638\\
72.5	0.1062	10.4810102953158\\
72.5	0.1068	10.6370415106678\\
72.5	0.1074	10.7930727260197\\
72.5	0.108	10.9491039413717\\
72.5	0.1086	11.1051351567237\\
72.5	0.1092	11.2611663720757\\
72.5	0.1098	11.4171975874277\\
72.5	0.1104	11.5732288027796\\
72.5	0.111	11.7292600181316\\
72.5	0.1116	11.8852912334836\\
72.5	0.1122	12.0413224488356\\
72.5	0.1128	12.1973536641875\\
72.5	0.1134	12.3533848795395\\
72.5	0.114	12.5094160948915\\
72.5	0.1146	12.6654473102435\\
72.5	0.1152	12.8214785255955\\
72.5	0.1158	12.9775097409474\\
72.5	0.1164	13.1335409562994\\
72.5	0.117	13.2895721716514\\
72.5	0.1176	13.4456033870034\\
72.5	0.1182	13.6016346023553\\
72.5	0.1188	13.7576658177073\\
72.5	0.1194	13.9136970330593\\
72.5	0.12	14.0697282484113\\
72.5	0.1206	14.2257594637632\\
72.5	0.1212	14.3817906791152\\
72.5	0.1218	14.5378218944672\\
72.5	0.1224	14.6938531098192\\
72.5	0.123	14.8498843251711\\
72.875	0.093	7.07951520545211\\
72.875	0.0936	7.23801350845443\\
72.875	0.0942	7.39651181145678\\
72.875	0.0948	7.55501011445909\\
72.875	0.0954	7.71350841746144\\
72.875	0.096	7.87200672046376\\
72.875	0.0966	8.03050502346611\\
72.875	0.0972	8.18900332646842\\
72.875	0.0978	8.34750162947076\\
72.875	0.0984	8.50599993247307\\
72.875	0.099	8.66449823547542\\
72.875	0.0996	8.82299653847774\\
72.875	0.1002	8.98149484148007\\
72.875	0.1008	9.13999314448242\\
72.875	0.1014	9.29849144748475\\
72.875	0.102	9.45698975048707\\
72.875	0.1026	9.6154880534894\\
72.875	0.1032	9.77398635649173\\
72.875	0.1038	9.93248465949407\\
72.875	0.1044	10.0909829624964\\
72.875	0.105	10.2494812654987\\
72.875	0.1056	10.407979568501\\
72.875	0.1062	10.5664778715034\\
72.875	0.1068	10.7249761745057\\
72.875	0.1074	10.883474477508\\
72.875	0.108	11.0419727805104\\
72.875	0.1086	11.2004710835127\\
72.875	0.1092	11.358969386515\\
72.875	0.1098	11.5174676895174\\
72.875	0.1104	11.6759659925197\\
72.875	0.111	11.834464295522\\
72.875	0.1116	11.9929625985244\\
72.875	0.1122	12.1514609015267\\
72.875	0.1128	12.309959204529\\
72.875	0.1134	12.4684575075314\\
72.875	0.114	12.6269558105337\\
72.875	0.1146	12.785454113536\\
72.875	0.1152	12.9439524165384\\
72.875	0.1158	13.1024507195407\\
72.875	0.1164	13.260949022543\\
72.875	0.117	13.4194473255453\\
72.875	0.1176	13.5779456285477\\
72.875	0.1182	13.73644393155\\
72.875	0.1188	13.8949422345523\\
72.875	0.1194	14.0534405375547\\
72.875	0.12	14.211938840557\\
72.875	0.1206	14.3704371435593\\
72.875	0.1212	14.5289354465617\\
72.875	0.1218	14.687433749564\\
72.875	0.1224	14.8459320525663\\
72.875	0.123	15.0044303555686\\
73.25	0.093	7.11070685333191\\
73.25	0.0936	7.27167224398458\\
73.25	0.0942	7.43263763463729\\
73.25	0.0948	7.59360302528995\\
73.25	0.0954	7.75456841594264\\
73.25	0.096	7.91553380659532\\
73.25	0.0966	8.07649919724803\\
73.25	0.0972	8.2374645879007\\
73.25	0.0978	8.39842997855338\\
73.25	0.0984	8.55939536920606\\
73.25	0.099	8.72036075985876\\
73.25	0.0996	8.88132615051144\\
73.25	0.1002	9.04229154116412\\
73.25	0.1008	9.20325693181681\\
73.25	0.1014	9.3642223224695\\
73.25	0.102	9.52518771312218\\
73.25	0.1026	9.68615310377486\\
73.25	0.1032	9.84711849442755\\
73.25	0.1038	10.0080838850802\\
73.25	0.1044	10.1690492757329\\
73.25	0.105	10.3300146663856\\
73.25	0.1056	10.4909800570383\\
73.25	0.1062	10.651945447691\\
73.25	0.1068	10.8129108383437\\
73.25	0.1074	10.9738762289963\\
73.25	0.108	11.134841619649\\
73.25	0.1086	11.2958070103017\\
73.25	0.1092	11.4567724009544\\
73.25	0.1098	11.6177377916071\\
73.25	0.1104	11.7787031822598\\
73.25	0.111	11.9396685729125\\
73.25	0.1116	12.1006339635651\\
73.25	0.1122	12.2615993542178\\
73.25	0.1128	12.4225647448705\\
73.25	0.1134	12.5835301355232\\
73.25	0.114	12.7444955261759\\
73.25	0.1146	12.9054609168286\\
73.25	0.1152	13.0664263074813\\
73.25	0.1158	13.2273916981339\\
73.25	0.1164	13.3883570887866\\
73.25	0.117	13.5493224794393\\
73.25	0.1176	13.710287870092\\
73.25	0.1182	13.8712532607447\\
73.25	0.1188	14.0322186513974\\
73.25	0.1194	14.19318404205\\
73.25	0.12	14.3541494327027\\
73.25	0.1206	14.5151148233554\\
73.25	0.1212	14.6760802140081\\
73.25	0.1218	14.8370456046608\\
73.25	0.1224	14.9980109953135\\
73.25	0.123	15.1589763859661\\
73.625	0.093	7.1418985012117\\
73.625	0.0936	7.30533097951472\\
73.625	0.0942	7.46876345781777\\
73.625	0.0948	7.63219593612079\\
73.625	0.0954	7.79562841442385\\
73.625	0.096	7.95906089272688\\
73.625	0.0966	8.12249337102993\\
73.625	0.0972	8.28592584933297\\
73.625	0.0978	8.44935832763601\\
73.625	0.0984	8.61279080593903\\
73.625	0.099	8.77622328424209\\
73.625	0.0996	8.93965576254512\\
73.625	0.1002	9.10308824084817\\
73.625	0.1008	9.2665207191512\\
73.625	0.1014	9.42995319745424\\
73.625	0.102	9.59338567575728\\
73.625	0.1026	9.75681815406031\\
73.625	0.1032	9.92025063236336\\
73.625	0.1038	10.0836831106664\\
73.625	0.1044	10.2471155889694\\
73.625	0.105	10.4105480672725\\
73.625	0.1056	10.5739805455755\\
73.625	0.1062	10.7374130238785\\
73.625	0.1068	10.9008455021816\\
73.625	0.1074	11.0642779804846\\
73.625	0.108	11.2277104587877\\
73.625	0.1086	11.3911429370907\\
73.625	0.1092	11.5545754153938\\
73.625	0.1098	11.7180078936968\\
73.625	0.1104	11.8814403719998\\
73.625	0.111	12.0448728503029\\
73.625	0.1116	12.2083053286059\\
73.625	0.1122	12.3717378069089\\
73.625	0.1128	12.535170285212\\
73.625	0.1134	12.698602763515\\
73.625	0.114	12.8620352418181\\
73.625	0.1146	13.0254677201211\\
73.625	0.1152	13.1889001984242\\
73.625	0.1158	13.3523326767272\\
73.625	0.1164	13.5157651550302\\
73.625	0.117	13.6791976333333\\
73.625	0.1176	13.8426301116363\\
73.625	0.1182	14.0060625899393\\
73.625	0.1188	14.1694950682424\\
73.625	0.1194	14.3329275465454\\
73.625	0.12	14.4963600248485\\
73.625	0.1206	14.6597925031515\\
73.625	0.1212	14.8232249814545\\
73.625	0.1218	14.9866574597576\\
73.625	0.1224	15.1500899380606\\
73.625	0.123	15.3135224163636\\
74	0.093	7.17309014909149\\
74	0.0936	7.33898971504487\\
74	0.0942	7.50488928099828\\
74	0.0948	7.67078884695165\\
74	0.0954	7.83668841290506\\
74	0.096	8.00258797885844\\
74	0.0966	8.16848754481185\\
74	0.0972	8.33438711076523\\
74	0.0978	8.50028667671863\\
74	0.0984	8.66618624267201\\
74	0.099	8.83208580862542\\
74	0.0996	8.99798537457882\\
74	0.1002	9.16388494053221\\
74	0.1008	9.32978450648561\\
74	0.1014	9.495684072439\\
74	0.102	9.66158363839239\\
74	0.1026	9.82748320434578\\
74	0.1032	9.99338277029918\\
74	0.1038	10.1592823362526\\
74	0.1044	10.325181902206\\
74	0.105	10.4910814681594\\
74	0.1056	10.6569810341127\\
74	0.1062	10.8228806000661\\
74	0.1068	10.9887801660195\\
74	0.1074	11.1546797319729\\
74	0.108	11.3205792979263\\
74	0.1086	11.4864788638797\\
74	0.1092	11.6523784298331\\
74	0.1098	11.8182779957865\\
74	0.1104	11.9841775617399\\
74	0.111	12.1500771276933\\
74	0.1116	12.3159766936467\\
74	0.1122	12.4818762596001\\
74	0.1128	12.6477758255535\\
74	0.1134	12.8136753915068\\
74	0.114	12.9795749574602\\
74	0.1146	13.1454745234136\\
74	0.1152	13.311374089367\\
74	0.1158	13.4772736553204\\
74	0.1164	13.6431732212738\\
74	0.117	13.8090727872272\\
74	0.1176	13.9749723531806\\
74	0.1182	14.140871919134\\
74	0.1188	14.3067714850874\\
74	0.1194	14.4726710510408\\
74	0.12	14.6385706169942\\
74	0.1206	14.8044701829476\\
74	0.1212	14.970369748901\\
74	0.1218	15.1362693148544\\
74	0.1224	15.3021688808078\\
74	0.123	15.4680684467611\\
};
\end{axis}

\begin{axis}[%
width=5.011742cm,
height=3.595207cm,
at={(0cm,5.468264cm)},
scale only axis,
xmin=55,
xmax=75,
tick align=outside,
xlabel={$L_{cut}$},
xmajorgrids,
ymin=0.09,
ymax=0.13,
ylabel={$D_{rlx}$},
ymajorgrids,
zmin=-16.1627687887745,
zmax=49.5376850778156,
zlabel={$x_2$},
zmajorgrids,
view={-140}{50},
legend style={at={(1.03,1)},anchor=north west,legend cell align=left,align=left,draw=white!15!black}
]
\addplot3[only marks,mark=*,mark options={},mark size=1.5000pt,color=mycolor1] plot table[row sep=crcr,]{%
74	0.123	0.706917161850943\\
72	0.113	-0.725188933625476\\
61	0.095	-2.23969084196957\\
56	0.093	-0.663995623424443\\
};
\addplot3[only marks,mark=*,mark options={},mark size=1.5000pt,color=black] plot table[row sep=crcr,]{%
69	0.104	-2.27707711447556\\
};

\addplot3[%
surf,
opacity=0.7,
shader=interp,
colormap={mymap}{[1pt] rgb(0pt)=(0.0901961,0.239216,0.0745098); rgb(1pt)=(0.0945149,0.242058,0.0739522); rgb(2pt)=(0.0988592,0.244894,0.0733566); rgb(3pt)=(0.103229,0.247724,0.0727241); rgb(4pt)=(0.107623,0.250549,0.0720557); rgb(5pt)=(0.112043,0.253367,0.0713525); rgb(6pt)=(0.116487,0.25618,0.0706154); rgb(7pt)=(0.120956,0.258986,0.0698456); rgb(8pt)=(0.125449,0.261787,0.0690441); rgb(9pt)=(0.129967,0.264581,0.0682118); rgb(10pt)=(0.134508,0.26737,0.06735); rgb(11pt)=(0.139074,0.270152,0.0664596); rgb(12pt)=(0.143663,0.272929,0.0655416); rgb(13pt)=(0.148275,0.275699,0.0645971); rgb(14pt)=(0.152911,0.278463,0.0636271); rgb(15pt)=(0.15757,0.281221,0.0626328); rgb(16pt)=(0.162252,0.283973,0.0616151); rgb(17pt)=(0.166957,0.286719,0.060575); rgb(18pt)=(0.171685,0.289458,0.0595136); rgb(19pt)=(0.176434,0.292191,0.0584321); rgb(20pt)=(0.181207,0.294918,0.0573313); rgb(21pt)=(0.186001,0.297639,0.0562123); rgb(22pt)=(0.190817,0.300353,0.0550763); rgb(23pt)=(0.195655,0.303061,0.0539242); rgb(24pt)=(0.200514,0.305763,0.052757); rgb(25pt)=(0.205395,0.308459,0.0515759); rgb(26pt)=(0.210296,0.311149,0.0503624); rgb(27pt)=(0.215212,0.313846,0.0490067); rgb(28pt)=(0.220142,0.316548,0.0475043); rgb(29pt)=(0.22509,0.319254,0.0458704); rgb(30pt)=(0.230056,0.321962,0.0441205); rgb(31pt)=(0.235042,0.324671,0.04227); rgb(32pt)=(0.240048,0.327379,0.0403343); rgb(33pt)=(0.245078,0.330085,0.0383287); rgb(34pt)=(0.250131,0.332786,0.0362688); rgb(35pt)=(0.25521,0.335482,0.0341698); rgb(36pt)=(0.260317,0.33817,0.0320472); rgb(37pt)=(0.265451,0.340849,0.0299163); rgb(38pt)=(0.270616,0.343517,0.0277927); rgb(39pt)=(0.275813,0.346172,0.0256916); rgb(40pt)=(0.281043,0.348814,0.0236284); rgb(41pt)=(0.286307,0.35144,0.0216186); rgb(42pt)=(0.291607,0.354048,0.0196776); rgb(43pt)=(0.296945,0.356637,0.0178207); rgb(44pt)=(0.302322,0.359206,0.0160634); rgb(45pt)=(0.307739,0.361753,0.0144211); rgb(46pt)=(0.313198,0.364275,0.0129091); rgb(47pt)=(0.318701,0.366772,0.0115428); rgb(48pt)=(0.324249,0.369242,0.0103377); rgb(49pt)=(0.329843,0.371682,0.00930909); rgb(50pt)=(0.335485,0.374093,0.00847245); rgb(51pt)=(0.341176,0.376471,0.00784314); rgb(52pt)=(0.346925,0.378826,0.00732741); rgb(53pt)=(0.352735,0.381168,0.00682184); rgb(54pt)=(0.358605,0.383497,0.00632729); rgb(55pt)=(0.364532,0.385812,0.00584464); rgb(56pt)=(0.370516,0.388113,0.00537476); rgb(57pt)=(0.376552,0.390399,0.00491852); rgb(58pt)=(0.38264,0.39267,0.00447681); rgb(59pt)=(0.388777,0.394925,0.00405048); rgb(60pt)=(0.394962,0.397164,0.00364042); rgb(61pt)=(0.401191,0.399386,0.00324749); rgb(62pt)=(0.407464,0.401592,0.00287258); rgb(63pt)=(0.413777,0.40378,0.00251655); rgb(64pt)=(0.420129,0.40595,0.00218028); rgb(65pt)=(0.426518,0.408102,0.00186463); rgb(66pt)=(0.432942,0.410234,0.00157049); rgb(67pt)=(0.439399,0.412348,0.00129873); rgb(68pt)=(0.445885,0.414441,0.00105022); rgb(69pt)=(0.452401,0.416515,0.000825833); rgb(70pt)=(0.458942,0.418567,0.000626441); rgb(71pt)=(0.465508,0.420599,0.00045292); rgb(72pt)=(0.472096,0.422609,0.000306141); rgb(73pt)=(0.478704,0.424596,0.000186979); rgb(74pt)=(0.485331,0.426562,9.63073e-05); rgb(75pt)=(0.491973,0.428504,3.49981e-05); rgb(76pt)=(0.498628,0.430422,3.92506e-06); rgb(77pt)=(0.505323,0.432315,0); rgb(78pt)=(0.512206,0.434168,0); rgb(79pt)=(0.519282,0.435983,0); rgb(80pt)=(0.526529,0.437764,0); rgb(81pt)=(0.533922,0.439512,0); rgb(82pt)=(0.54144,0.441232,0); rgb(83pt)=(0.549059,0.442927,0); rgb(84pt)=(0.556756,0.444599,0); rgb(85pt)=(0.564508,0.446252,0); rgb(86pt)=(0.572292,0.447889,0); rgb(87pt)=(0.580084,0.449514,0); rgb(88pt)=(0.587863,0.451129,0); rgb(89pt)=(0.595604,0.452737,0); rgb(90pt)=(0.603284,0.454343,0); rgb(91pt)=(0.610882,0.455948,0); rgb(92pt)=(0.618373,0.457556,0); rgb(93pt)=(0.625734,0.459171,0); rgb(94pt)=(0.632943,0.460795,0); rgb(95pt)=(0.639976,0.462432,0); rgb(96pt)=(0.64681,0.464084,0); rgb(97pt)=(0.653423,0.465756,0); rgb(98pt)=(0.659791,0.46745,0); rgb(99pt)=(0.665891,0.469169,0); rgb(100pt)=(0.6717,0.470916,0); rgb(101pt)=(0.677195,0.472696,0); rgb(102pt)=(0.682353,0.47451,0); rgb(103pt)=(0.687242,0.476355,0); rgb(104pt)=(0.691952,0.478225,0); rgb(105pt)=(0.696497,0.480118,0); rgb(106pt)=(0.700887,0.482033,0); rgb(107pt)=(0.705134,0.483968,0); rgb(108pt)=(0.709251,0.485921,0); rgb(109pt)=(0.713249,0.487891,0); rgb(110pt)=(0.71714,0.489876,0); rgb(111pt)=(0.720936,0.491875,0); rgb(112pt)=(0.724649,0.493887,0); rgb(113pt)=(0.72829,0.495909,0); rgb(114pt)=(0.731872,0.49794,0); rgb(115pt)=(0.735406,0.499979,0); rgb(116pt)=(0.738904,0.502025,0); rgb(117pt)=(0.742378,0.504075,0); rgb(118pt)=(0.74584,0.506128,0); rgb(119pt)=(0.749302,0.508182,0); rgb(120pt)=(0.752775,0.510237,0); rgb(121pt)=(0.756272,0.51229,0); rgb(122pt)=(0.759804,0.514339,0); rgb(123pt)=(0.763384,0.516385,0); rgb(124pt)=(0.767022,0.518424,0); rgb(125pt)=(0.770731,0.520455,0); rgb(126pt)=(0.774523,0.522478,0); rgb(127pt)=(0.77841,0.524489,0); rgb(128pt)=(0.782391,0.526491,0); rgb(129pt)=(0.786402,0.528496,0); rgb(130pt)=(0.790431,0.530506,0); rgb(131pt)=(0.794478,0.532521,0); rgb(132pt)=(0.798541,0.534539,0); rgb(133pt)=(0.802619,0.53656,0); rgb(134pt)=(0.806712,0.538584,0); rgb(135pt)=(0.81082,0.540609,0); rgb(136pt)=(0.81494,0.542635,0); rgb(137pt)=(0.819074,0.54466,0); rgb(138pt)=(0.823219,0.546686,0); rgb(139pt)=(0.827374,0.548709,0); rgb(140pt)=(0.831541,0.55073,0); rgb(141pt)=(0.835716,0.552749,0); rgb(142pt)=(0.8399,0.554763,0); rgb(143pt)=(0.844092,0.556774,0); rgb(144pt)=(0.848292,0.558779,0); rgb(145pt)=(0.852497,0.560778,0); rgb(146pt)=(0.856708,0.562771,0); rgb(147pt)=(0.860924,0.564756,0); rgb(148pt)=(0.865143,0.566733,0); rgb(149pt)=(0.869366,0.568701,0); rgb(150pt)=(0.873592,0.57066,0); rgb(151pt)=(0.877819,0.572608,0); rgb(152pt)=(0.882047,0.574545,0); rgb(153pt)=(0.886275,0.576471,0); rgb(154pt)=(0.890659,0.578362,0); rgb(155pt)=(0.895333,0.580203,0); rgb(156pt)=(0.900258,0.581999,0); rgb(157pt)=(0.905397,0.583755,0); rgb(158pt)=(0.910711,0.585479,0); rgb(159pt)=(0.916164,0.587176,0); rgb(160pt)=(0.921717,0.588852,0); rgb(161pt)=(0.927333,0.590513,0); rgb(162pt)=(0.932974,0.592166,0); rgb(163pt)=(0.938602,0.593815,0); rgb(164pt)=(0.94418,0.595468,0); rgb(165pt)=(0.949669,0.59713,0); rgb(166pt)=(0.955033,0.598808,0); rgb(167pt)=(0.960233,0.600507,0); rgb(168pt)=(0.965232,0.602233,0); rgb(169pt)=(0.969992,0.603992,0); rgb(170pt)=(0.974475,0.605791,0); rgb(171pt)=(0.978643,0.607636,0); rgb(172pt)=(0.98246,0.609532,0); rgb(173pt)=(0.985886,0.611486,0); rgb(174pt)=(0.988885,0.613503,0); rgb(175pt)=(0.991419,0.61559,0); rgb(176pt)=(0.99345,0.617753,0); rgb(177pt)=(0.99494,0.619997,0); rgb(178pt)=(0.995851,0.622329,0); rgb(179pt)=(0.996226,0.624763,0); rgb(180pt)=(0.996512,0.627352,0); rgb(181pt)=(0.996788,0.630095,0); rgb(182pt)=(0.997053,0.632982,0); rgb(183pt)=(0.997308,0.636004,0); rgb(184pt)=(0.997552,0.639152,0); rgb(185pt)=(0.997785,0.642416,0); rgb(186pt)=(0.998006,0.645786,0); rgb(187pt)=(0.998217,0.649253,0); rgb(188pt)=(0.998416,0.652807,0); rgb(189pt)=(0.998605,0.656439,0); rgb(190pt)=(0.998781,0.660138,0); rgb(191pt)=(0.998946,0.663897,0); rgb(192pt)=(0.9991,0.667704,0); rgb(193pt)=(0.999242,0.67155,0); rgb(194pt)=(0.999372,0.675427,0); rgb(195pt)=(0.99949,0.679323,0); rgb(196pt)=(0.999596,0.68323,0); rgb(197pt)=(0.99969,0.687139,0); rgb(198pt)=(0.999771,0.691039,0); rgb(199pt)=(0.999841,0.694921,0); rgb(200pt)=(0.999898,0.698775,0); rgb(201pt)=(0.999942,0.702592,0); rgb(202pt)=(0.999974,0.706363,0); rgb(203pt)=(0.999994,0.710077,0); rgb(204pt)=(1,0.713725,0); rgb(205pt)=(1,0.717341,0); rgb(206pt)=(1,0.720963,0); rgb(207pt)=(1,0.724591,0); rgb(208pt)=(1,0.728226,0); rgb(209pt)=(1,0.731867,0); rgb(210pt)=(1,0.735514,0); rgb(211pt)=(1,0.739167,0); rgb(212pt)=(1,0.742827,0); rgb(213pt)=(1,0.746493,0); rgb(214pt)=(1,0.750165,0); rgb(215pt)=(1,0.753843,0); rgb(216pt)=(1,0.757527,0); rgb(217pt)=(1,0.761217,0); rgb(218pt)=(1,0.764913,0); rgb(219pt)=(1,0.768615,0); rgb(220pt)=(1,0.772324,0); rgb(221pt)=(1,0.776038,0); rgb(222pt)=(1,0.779758,0); rgb(223pt)=(1,0.783484,0); rgb(224pt)=(1,0.787215,0); rgb(225pt)=(1,0.790953,0); rgb(226pt)=(1,0.794696,0); rgb(227pt)=(1,0.798445,0); rgb(228pt)=(1,0.8022,0); rgb(229pt)=(1,0.805961,0); rgb(230pt)=(1,0.809727,0); rgb(231pt)=(1,0.8135,0); rgb(232pt)=(1,0.817278,0); rgb(233pt)=(1,0.821063,0); rgb(234pt)=(1,0.824854,0); rgb(235pt)=(1,0.828652,0); rgb(236pt)=(1,0.832455,0); rgb(237pt)=(1,0.836265,0); rgb(238pt)=(1,0.840081,0); rgb(239pt)=(1,0.843903,0); rgb(240pt)=(1,0.847732,0); rgb(241pt)=(1,0.851566,0); rgb(242pt)=(1,0.855406,0); rgb(243pt)=(1,0.859253,0); rgb(244pt)=(1,0.863106,0); rgb(245pt)=(1,0.866964,0); rgb(246pt)=(1,0.870829,0); rgb(247pt)=(1,0.8747,0); rgb(248pt)=(1,0.878577,0); rgb(249pt)=(1,0.88246,0); rgb(250pt)=(1,0.886349,0); rgb(251pt)=(1,0.890243,0); rgb(252pt)=(1,0.894144,0); rgb(253pt)=(1,0.898051,0); rgb(254pt)=(1,0.901964,0); rgb(255pt)=(1,0.905882,0)},
mesh/rows=49]
table[row sep=crcr,header=false] {%
%
56	0.093	-0.663995625005896\\
56	0.0936	0.340037989050529\\
56	0.0942	1.34407160310695\\
56	0.0948	2.34810521716338\\
56	0.0954	3.3521388312198\\
56	0.096	4.35617244527623\\
56	0.0966	5.36020605933265\\
56	0.0972	6.36423967338908\\
56	0.0978	7.3682732874455\\
56	0.0984	8.37230690150199\\
56	0.099	9.37634051555841\\
56	0.0996	10.3803741296148\\
56	0.1002	11.3844077436713\\
56	0.1008	12.3884413577277\\
56	0.1014	13.3924749717842\\
56	0.102	14.3965085858405\\
56	0.1026	15.400542199897\\
56	0.1032	16.4045758139534\\
56	0.1038	17.4086094280098\\
56	0.1044	18.4126430420663\\
56	0.105	19.4166766561228\\
56	0.1056	20.4207102701792\\
56	0.1062	21.4247438842356\\
56	0.1068	22.428777498292\\
56	0.1074	23.4328111123484\\
56	0.108	24.4368447264048\\
56	0.1086	25.4408783404613\\
56	0.1092	26.4449119545176\\
56	0.1098	27.4489455685742\\
56	0.1104	28.4529791826307\\
56	0.111	29.4570127966871\\
56	0.1116	30.4610464107435\\
56	0.1122	31.4650800247999\\
56	0.1128	32.4691136388563\\
56	0.1134	33.4731472529127\\
56	0.114	34.4771808669691\\
56	0.1146	35.4812144810256\\
56	0.1152	36.4852480950821\\
56	0.1158	37.4892817091385\\
56	0.1164	38.493315323195\\
56	0.117	39.4973489372513\\
56	0.1176	40.5013825513078\\
56	0.1182	41.5054161653642\\
56	0.1188	42.5094497794206\\
56	0.1194	43.513483393477\\
56	0.12	44.5175170075334\\
56	0.1206	45.5215506215899\\
56	0.1212	46.5255842356464\\
56	0.1218	47.5296178497027\\
56	0.1224	48.5336514637592\\
56	0.123	49.5376850778156\\
56.375	0.093	-0.986886732584423\\
56.375	0.0936	0.00325855032514255\\
56.375	0.0942	0.993403833234709\\
56.375	0.0948	1.98354911614427\\
56.375	0.0954	2.97369439905384\\
56.375	0.096	3.96383968196341\\
56.375	0.0966	4.95398496487297\\
56.375	0.0972	5.94413024778254\\
56.375	0.0978	6.9342755306921\\
56.375	0.0984	7.92442081360167\\
56.375	0.099	8.91456609651124\\
56.375	0.0996	9.9047113794208\\
56.375	0.1002	10.8948566623304\\
56.375	0.1008	11.88500194524\\
56.375	0.1014	12.8751472281496\\
56.375	0.102	13.8652925110591\\
56.375	0.1026	14.8554377939687\\
56.375	0.1032	15.8455830768783\\
56.375	0.1038	16.8357283597878\\
56.375	0.1044	17.8258736426974\\
56.375	0.105	18.816018925607\\
56.375	0.1056	19.8061642085166\\
56.375	0.1062	20.7963094914261\\
56.375	0.1068	21.7864547743357\\
56.375	0.1074	22.7766000572452\\
56.375	0.108	23.7667453401548\\
56.375	0.1086	24.7568906230644\\
56.375	0.1092	25.7470359059739\\
56.375	0.1098	26.7371811888836\\
56.375	0.1104	27.7273264717932\\
56.375	0.111	28.7174717547027\\
56.375	0.1116	29.7076170376123\\
56.375	0.1122	30.6977623205218\\
56.375	0.1128	31.6879076034314\\
56.375	0.1134	32.6780528863409\\
56.375	0.114	33.6681981692505\\
56.375	0.1146	34.6583434521601\\
56.375	0.1152	35.6484887350697\\
56.375	0.1158	36.6386340179793\\
56.375	0.1164	37.6287793008888\\
56.375	0.117	38.6189245837983\\
56.375	0.1176	39.6090698667079\\
56.375	0.1182	40.5992151496175\\
56.375	0.1188	41.5893604325271\\
56.375	0.1194	42.5795057154366\\
56.375	0.12	43.5696509983462\\
56.375	0.1206	44.5597962812558\\
56.375	0.1212	45.5499415641654\\
56.375	0.1218	46.5400868470749\\
56.375	0.1224	47.5302321299845\\
56.375	0.123	48.520377412894\\
56.75	0.093	-1.30977784016295\\
56.75	0.0936	-0.333520888400187\\
56.75	0.0942	0.64273606336252\\
56.75	0.0948	1.61899301512523\\
56.75	0.0954	2.59524996688793\\
56.75	0.096	3.57150691865064\\
56.75	0.0966	4.54776387041335\\
56.75	0.0972	5.52402082217606\\
56.75	0.0978	6.50027777393876\\
56.75	0.0984	7.47653472570147\\
56.75	0.099	8.45279167746418\\
56.75	0.0996	9.42904862922688\\
56.75	0.1002	10.4053055809896\\
56.75	0.1008	11.3815625327524\\
56.75	0.1014	12.3578194845151\\
56.75	0.102	13.3340764362777\\
56.75	0.1026	14.3103333880404\\
56.75	0.1032	15.2865903398031\\
56.75	0.1038	16.2628472915658\\
56.75	0.1044	17.2391042433286\\
56.75	0.105	18.2153611950913\\
56.75	0.1056	19.191618146854\\
56.75	0.1062	20.1678750986167\\
56.75	0.1068	21.1441320503794\\
56.75	0.1074	22.1203890021421\\
56.75	0.108	23.0966459539048\\
56.75	0.1086	24.0729029056675\\
56.75	0.1092	25.0491598574301\\
56.75	0.1098	26.025416809193\\
56.75	0.1104	27.0016737609557\\
56.75	0.111	27.9779307127184\\
56.75	0.1116	28.9541876644811\\
56.75	0.1122	29.9304446162438\\
56.75	0.1128	30.9067015680065\\
56.75	0.1134	31.8829585197692\\
56.75	0.114	32.8592154715319\\
56.75	0.1146	33.8354724232946\\
56.75	0.1152	34.8117293750574\\
56.75	0.1158	35.7879863268201\\
56.75	0.1164	36.7642432785828\\
56.75	0.117	37.7405002303454\\
56.75	0.1176	38.7167571821082\\
56.75	0.1182	39.6930141338709\\
56.75	0.1188	40.6692710856336\\
56.75	0.1194	41.6455280373962\\
56.75	0.12	42.6217849891589\\
56.75	0.1206	43.5980419409217\\
56.75	0.1212	44.5742988926845\\
56.75	0.1218	45.5505558444471\\
56.75	0.1224	46.5268127962098\\
56.75	0.123	47.5030697479726\\
57.125	0.093	-1.63266894774148\\
57.125	0.0936	-0.670300327125631\\
57.125	0.0942	0.292068293490217\\
57.125	0.0948	1.25443691410607\\
57.125	0.0954	2.21680553472191\\
57.125	0.096	3.17917415533776\\
57.125	0.0966	4.14154277595361\\
57.125	0.0972	5.10391139656946\\
57.125	0.0978	6.06628001718531\\
57.125	0.0984	7.02864863780115\\
57.125	0.099	7.991017258417\\
57.125	0.0996	8.95338587903285\\
57.125	0.1002	9.91575449964876\\
57.125	0.1008	10.8781231202646\\
57.125	0.1014	11.8404917408805\\
57.125	0.102	12.8028603614962\\
57.125	0.1026	13.7652289821121\\
57.125	0.1032	14.7275976027279\\
57.125	0.1038	15.6899662233438\\
57.125	0.1044	16.6523348439597\\
57.125	0.105	17.6147034645755\\
57.125	0.1056	18.5770720851914\\
57.125	0.1062	19.5394407058073\\
57.125	0.1068	20.5018093264231\\
57.125	0.1074	21.4641779470389\\
57.125	0.108	22.4265465676548\\
57.125	0.1086	23.3889151882706\\
57.125	0.1092	24.3512838088864\\
57.125	0.1098	25.3136524295024\\
57.125	0.1104	26.2760210501182\\
57.125	0.111	27.2383896707341\\
57.125	0.1116	28.2007582913499\\
57.125	0.1122	29.1631269119657\\
57.125	0.1128	30.1254955325816\\
57.125	0.1134	31.0878641531974\\
57.125	0.114	32.0502327738133\\
57.125	0.1146	33.0126013944291\\
57.125	0.1152	33.974970015045\\
57.125	0.1158	34.9373386356609\\
57.125	0.1164	35.8997072562767\\
57.125	0.117	36.8620758768925\\
57.125	0.1176	37.8244444975084\\
57.125	0.1182	38.7868131181242\\
57.125	0.1188	39.74918173874\\
57.125	0.1194	40.7115503593558\\
57.125	0.12	41.6739189799717\\
57.125	0.1206	42.6362876005877\\
57.125	0.1212	43.5986562212035\\
57.125	0.1218	44.5610248418193\\
57.125	0.1224	45.5233934624351\\
57.125	0.123	46.485762083051\\
57.5	0.093	-1.95556005532001\\
57.5	0.0936	-1.00707976585102\\
57.5	0.0942	-0.0585994763820281\\
57.5	0.0948	0.889880813086961\\
57.5	0.0954	1.83836110255595\\
57.5	0.096	2.78684139202494\\
57.5	0.0966	3.73532168149393\\
57.5	0.0972	4.68380197096292\\
57.5	0.0978	5.63228226043191\\
57.5	0.0984	6.5807625499009\\
57.5	0.099	7.52924283936989\\
57.5	0.0996	8.47772312883887\\
57.5	0.1002	9.42620341830792\\
57.5	0.1008	10.3746837077769\\
57.5	0.1014	11.3231639972459\\
57.5	0.102	12.2716442867148\\
57.5	0.1026	13.2201245761838\\
57.5	0.1032	14.1686048656528\\
57.5	0.1038	15.1170851551218\\
57.5	0.1044	16.0655654445908\\
57.5	0.105	17.0140457340598\\
57.5	0.1056	17.9625260235288\\
57.5	0.1062	18.9110063129978\\
57.5	0.1068	19.8594866024668\\
57.5	0.1074	20.8079668919357\\
57.5	0.108	21.7564471814047\\
57.5	0.1086	22.7049274708737\\
57.5	0.1092	23.6534077603426\\
57.5	0.1098	24.6018880498117\\
57.5	0.1104	25.5503683392807\\
57.5	0.111	26.4988486287497\\
57.5	0.1116	27.4473289182187\\
57.5	0.1122	28.3958092076876\\
57.5	0.1128	29.3442894971566\\
57.5	0.1134	30.2927697866256\\
57.5	0.114	31.2412500760946\\
57.5	0.1146	32.1897303655636\\
57.5	0.1152	33.1382106550327\\
57.5	0.1158	34.0866909445016\\
57.5	0.1164	35.0351712339706\\
57.5	0.117	35.9836515234395\\
57.5	0.1176	36.9321318129086\\
57.5	0.1182	37.8806121023775\\
57.5	0.1188	38.8290923918465\\
57.5	0.1194	39.7775726813154\\
57.5	0.12	40.7260529707844\\
57.5	0.1206	41.6745332602535\\
57.5	0.1212	42.6230135497225\\
57.5	0.1218	43.5714938391914\\
57.5	0.1224	44.5199741286604\\
57.5	0.123	45.4684544181294\\
57.875	0.093	-2.27845116289848\\
57.875	0.0936	-1.34385920457635\\
57.875	0.0942	-0.409267246254217\\
57.875	0.0948	0.525324712067913\\
57.875	0.0954	1.45991667039004\\
57.875	0.096	2.39450862871217\\
57.875	0.0966	3.3291005870343\\
57.875	0.0972	4.26369254535643\\
57.875	0.0978	5.19828450367856\\
57.875	0.0984	6.1328764620007\\
57.875	0.099	7.06746842032283\\
57.875	0.0996	8.00206037864496\\
57.875	0.1002	8.93665233696709\\
57.875	0.1008	9.87124429528922\\
57.875	0.1014	10.8058362536113\\
57.875	0.102	11.7404282119334\\
57.875	0.1026	12.6750201702556\\
57.875	0.1032	13.6096121285777\\
57.875	0.1038	14.5442040868998\\
57.875	0.1044	15.478796045222\\
57.875	0.105	16.4133880035441\\
57.875	0.1056	17.3479799618663\\
57.875	0.1062	18.2825719201884\\
57.875	0.1068	19.2171638785105\\
57.875	0.1074	20.1517558368326\\
57.875	0.108	21.0863477951547\\
57.875	0.1086	22.0209397534769\\
57.875	0.1092	22.9555317117989\\
57.875	0.1098	23.8901236701212\\
57.875	0.1104	24.8247156284433\\
57.875	0.111	25.7593075867654\\
57.875	0.1116	26.6938995450876\\
57.875	0.1122	27.6284915034096\\
57.875	0.1128	28.5630834617318\\
57.875	0.1134	29.4976754200539\\
57.875	0.114	30.432267378376\\
57.875	0.1146	31.3668593366982\\
57.875	0.1152	32.3014512950203\\
57.875	0.1158	33.2360432533424\\
57.875	0.1164	34.1706352116645\\
57.875	0.117	35.1052271699866\\
57.875	0.1176	36.0398191283087\\
57.875	0.1182	36.9744110866309\\
57.875	0.1188	37.909003044953\\
57.875	0.1194	38.8435950032751\\
57.875	0.12	39.7781869615972\\
57.875	0.1206	40.7127789199195\\
57.875	0.1212	41.6473708782416\\
57.875	0.1218	42.5819628365637\\
57.875	0.1224	43.5165547948858\\
57.875	0.123	44.4511467532079\\
58.25	0.093	-2.60134227047701\\
58.25	0.0936	-1.68063864330173\\
58.25	0.0942	-0.759935016126462\\
58.25	0.0948	0.160768611048809\\
58.25	0.0954	1.08147223822408\\
58.25	0.096	2.00217586539935\\
58.25	0.0966	2.92287949257462\\
58.25	0.0972	3.84358311974989\\
58.25	0.0978	4.76428674692511\\
58.25	0.0984	5.68499037410038\\
58.25	0.099	6.60569400127565\\
58.25	0.0996	7.52639762845092\\
58.25	0.1002	8.44710125562625\\
58.25	0.1008	9.36780488280152\\
58.25	0.1014	10.2885085099768\\
58.25	0.102	11.209212137152\\
58.25	0.1026	12.1299157643273\\
58.25	0.1032	13.0506193915026\\
58.25	0.1038	13.9713230186778\\
58.25	0.1044	14.8920266458532\\
58.25	0.105	15.8127302730284\\
58.25	0.1056	16.7334339002036\\
58.25	0.1062	17.6541375273789\\
58.25	0.1068	18.5748411545542\\
58.25	0.1074	19.4955447817294\\
58.25	0.108	20.4162484089047\\
58.25	0.1086	21.3369520360799\\
58.25	0.1092	22.2576556632552\\
58.25	0.1098	23.1783592904305\\
58.25	0.1104	24.0990629176058\\
58.25	0.111	25.0197665447811\\
58.25	0.1116	25.9404701719563\\
58.25	0.1122	26.8611737991316\\
58.25	0.1128	27.7818774263068\\
58.25	0.1134	28.702581053482\\
58.25	0.114	29.6232846806573\\
58.25	0.1146	30.5439883078326\\
58.25	0.1152	31.4646919350079\\
58.25	0.1158	32.3853955621832\\
58.25	0.1164	33.3060991893585\\
58.25	0.117	34.2268028165337\\
58.25	0.1176	35.1475064437089\\
58.25	0.1182	36.0682100708842\\
58.25	0.1188	36.9889136980595\\
58.25	0.1194	37.9096173252347\\
58.25	0.12	38.83032095241\\
58.25	0.1206	39.7510245795854\\
58.25	0.1212	40.6717282067606\\
58.25	0.1218	41.5924318339358\\
58.25	0.1224	42.5131354611111\\
58.25	0.123	43.4338390882863\\
58.625	0.093	-2.92423337805548\\
58.625	0.0936	-2.01741808202706\\
58.625	0.0942	-1.11060278599865\\
58.625	0.0948	-0.203787489970239\\
58.625	0.0954	0.703027806058117\\
58.625	0.096	1.60984310208653\\
58.625	0.0966	2.51665839811494\\
58.625	0.0972	3.42347369414335\\
58.625	0.0978	4.33028899017177\\
58.625	0.0984	5.23710428620018\\
58.625	0.099	6.14391958222859\\
58.625	0.0996	7.050734878257\\
58.625	0.1002	7.95755017428547\\
58.625	0.1008	8.86436547031388\\
58.625	0.1014	9.77118076634224\\
58.625	0.102	10.6779960623706\\
58.625	0.1026	11.584811358399\\
58.625	0.1032	12.4916266544274\\
58.625	0.1038	13.3984419504558\\
58.625	0.1044	14.3052572464843\\
58.625	0.105	15.2120725425127\\
58.625	0.1056	16.1188878385411\\
58.625	0.1062	17.0257031345695\\
58.625	0.1068	17.932518430598\\
58.625	0.1074	18.8393337266263\\
58.625	0.108	19.7461490226547\\
58.625	0.1086	20.6529643186831\\
58.625	0.1092	21.5597796147114\\
58.625	0.1098	22.46659491074\\
58.625	0.1104	23.3734102067684\\
58.625	0.111	24.2802255027968\\
58.625	0.1116	25.1870407988252\\
58.625	0.1122	26.0938560948535\\
58.625	0.1128	27.0006713908819\\
58.625	0.1134	27.9074866869103\\
58.625	0.114	28.8143019829387\\
58.625	0.1146	29.7211172789671\\
58.625	0.1152	30.6279325749956\\
58.625	0.1158	31.534747871024\\
58.625	0.1164	32.4415631670524\\
58.625	0.117	33.3483784630808\\
58.625	0.1176	34.2551937591092\\
58.625	0.1182	35.1620090551376\\
58.625	0.1188	36.068824351166\\
58.625	0.1194	36.9756396471943\\
58.625	0.12	37.8824549432227\\
58.625	0.1206	38.7892702392513\\
58.625	0.1212	39.6960855352797\\
58.625	0.1218	40.602900831308\\
58.625	0.1224	41.5097161273364\\
58.625	0.123	42.4165314233649\\
59	0.093	-3.247124485634\\
59	0.0936	-2.35419752075245\\
59	0.0942	-1.46127055587095\\
59	0.0948	-0.5683435909894\\
59	0.0954	0.324583373892153\\
59	0.096	1.21751033877371\\
59	0.0966	2.11043730365526\\
59	0.0972	3.00336426853681\\
59	0.0978	3.89629123341837\\
59	0.0984	4.78921819829986\\
59	0.099	5.68214516318142\\
59	0.0996	6.57507212806297\\
59	0.1002	7.46799909294458\\
59	0.1008	8.36092605782613\\
59	0.1014	9.25385302270769\\
59	0.102	10.1467799875892\\
59	0.1026	11.0397069524707\\
59	0.1032	11.9326339173522\\
59	0.1038	12.8255608822338\\
59	0.1044	13.7184878471154\\
59	0.105	14.611414811997\\
59	0.1056	15.5043417768785\\
59	0.1062	16.3972687417601\\
59	0.1068	17.2901957066416\\
59	0.1074	18.1831226715231\\
59	0.108	19.0760496364046\\
59	0.1086	19.9689766012862\\
59	0.1092	20.8619035661677\\
59	0.1098	21.7548305310493\\
59	0.1104	22.6477574959309\\
59	0.111	23.5406844608124\\
59	0.1116	24.433611425694\\
59	0.1122	25.3265383905754\\
59	0.1128	26.219465355457\\
59	0.1134	27.1123923203385\\
59	0.114	28.0053192852201\\
59	0.1146	28.8982462501016\\
59	0.1152	29.7911732149832\\
59	0.1158	30.6841001798648\\
59	0.1164	31.5770271447464\\
59	0.117	32.4699541096278\\
59	0.1176	33.3628810745093\\
59	0.1182	34.2558080393909\\
59	0.1188	35.1487350042725\\
59	0.1194	36.0416619691539\\
59	0.12	36.9345889340355\\
59	0.1206	37.8275158989172\\
59	0.1212	38.7204428637987\\
59	0.1218	39.6133698286802\\
59	0.1224	40.5062967935617\\
59	0.123	41.3992237584433\\
59.375	0.093	-3.57001559321253\\
59.375	0.0936	-2.69097695947784\\
59.375	0.0942	-1.8119383257432\\
59.375	0.0948	-0.932899692008505\\
59.375	0.0954	-0.0538610582738102\\
59.375	0.096	0.825177575460884\\
59.375	0.0966	1.70421620919558\\
59.375	0.0972	2.58325484293027\\
59.375	0.0978	3.46229347666491\\
59.375	0.0984	4.34133211039961\\
59.375	0.099	5.2203707441343\\
59.375	0.0996	6.09940937786899\\
59.375	0.1002	6.97844801160375\\
59.375	0.1008	7.85748664533844\\
59.375	0.1014	8.73652527907308\\
59.375	0.102	9.61556391280772\\
59.375	0.1026	10.4946025465424\\
59.375	0.1032	11.3736411802771\\
59.375	0.1038	12.2526798140118\\
59.375	0.1044	13.1317184477466\\
59.375	0.105	14.0107570814812\\
59.375	0.1056	14.8897957152159\\
59.375	0.1062	15.7688343489506\\
59.375	0.1068	16.6478729826853\\
59.375	0.1074	17.5269116164199\\
59.375	0.108	18.4059502501546\\
59.375	0.1086	19.2849888838892\\
59.375	0.1092	20.1640275176239\\
59.375	0.1098	21.0430661513587\\
59.375	0.1104	21.9221047850934\\
59.375	0.111	22.8011434188281\\
59.375	0.1116	23.6801820525628\\
59.375	0.1122	24.5592206862974\\
59.375	0.1128	25.438259320032\\
59.375	0.1134	26.3172979537667\\
59.375	0.114	27.1963365875014\\
59.375	0.1146	28.0753752212361\\
59.375	0.1152	28.9544138549709\\
59.375	0.1158	29.8334524887056\\
59.375	0.1164	30.7124911224402\\
59.375	0.117	31.5915297561749\\
59.375	0.1176	32.4705683899095\\
59.375	0.1182	33.3496070236442\\
59.375	0.1188	34.2286456573789\\
59.375	0.1194	35.1076842911135\\
59.375	0.12	35.9867229248482\\
59.375	0.1206	36.865761558583\\
59.375	0.1212	37.7448001923177\\
59.375	0.1218	38.6238388260523\\
59.375	0.1224	39.502877459787\\
59.375	0.123	40.3819160935217\\
59.75	0.093	-3.89290670079106\\
59.75	0.0936	-3.02775639820322\\
59.75	0.0942	-2.16260609561539\\
59.75	0.0948	-1.29745579302755\\
59.75	0.0954	-0.432305490439717\\
59.75	0.096	0.432844812148119\\
59.75	0.0966	1.2979951147359\\
59.75	0.0972	2.16314541732373\\
59.75	0.0978	3.02829571991157\\
59.75	0.0984	3.8934460224994\\
59.75	0.099	4.75859632508724\\
59.75	0.0996	5.62374662767502\\
59.75	0.1002	6.48889693026291\\
59.75	0.1008	7.35404723285075\\
59.75	0.1014	8.21919753543858\\
59.75	0.102	9.08434783802636\\
59.75	0.1026	9.94949814061414\\
59.75	0.1032	10.814648443202\\
59.75	0.1038	11.6797987457898\\
59.75	0.1044	12.5449490483777\\
59.75	0.105	13.4100993509655\\
59.75	0.1056	14.2752496535533\\
59.75	0.1062	15.1403999561412\\
59.75	0.1068	16.005550258729\\
59.75	0.1074	16.8707005613168\\
59.75	0.108	17.7358508639046\\
59.75	0.1086	18.6010011664924\\
59.75	0.1092	19.4661514690802\\
59.75	0.1098	20.3313017716681\\
59.75	0.1104	21.1964520742559\\
59.75	0.111	22.0616023768438\\
59.75	0.1116	22.9267526794316\\
59.75	0.1122	23.7919029820193\\
59.75	0.1128	24.6570532846072\\
59.75	0.1134	25.522203587195\\
59.75	0.114	26.3873538897828\\
59.75	0.1146	27.2525041923707\\
59.75	0.1152	28.1176544949585\\
59.75	0.1158	28.9828047975464\\
59.75	0.1164	29.8479551001342\\
59.75	0.117	30.713105402722\\
59.75	0.1176	31.5782557053098\\
59.75	0.1182	32.4434060078976\\
59.75	0.1188	33.3085563104854\\
59.75	0.1194	34.1737066130732\\
59.75	0.12	35.038856915661\\
59.75	0.1206	35.904007218249\\
59.75	0.1212	36.7691575208368\\
59.75	0.1218	37.6343078234245\\
59.75	0.1224	38.4994581260124\\
59.75	0.123	39.3646084286002\\
60.125	0.093	-4.21579780836953\\
60.125	0.0936	-3.36453583692855\\
60.125	0.0942	-2.51327386548758\\
60.125	0.0948	-1.6620118940466\\
60.125	0.0954	-0.81074992260568\\
60.125	0.096	0.0405120488352964\\
60.125	0.0966	0.891774020276273\\
60.125	0.0972	1.74303599171725\\
60.125	0.0978	2.59429796315823\\
60.125	0.0984	3.44555993459915\\
60.125	0.099	4.29682190604012\\
60.125	0.0996	5.1480838774811\\
60.125	0.1002	5.99934584892213\\
60.125	0.1008	6.85060782036305\\
60.125	0.1014	7.70186979180403\\
60.125	0.102	8.55313176324495\\
60.125	0.1026	9.40439373468593\\
60.125	0.1032	10.2556557061269\\
60.125	0.1038	11.1069176775678\\
60.125	0.1044	11.9581796490089\\
60.125	0.105	12.8094416204498\\
60.125	0.1056	13.6607035918908\\
60.125	0.1062	14.5119655633318\\
60.125	0.1068	15.3632275347727\\
60.125	0.1074	16.2144895062136\\
60.125	0.108	17.0657514776546\\
60.125	0.1086	17.9170134490956\\
60.125	0.1092	18.7682754205364\\
60.125	0.1098	19.6195373919775\\
60.125	0.1104	20.4707993634185\\
60.125	0.111	21.3220613348595\\
60.125	0.1116	22.1733233063005\\
60.125	0.1122	23.0245852777413\\
60.125	0.1128	23.8758472491823\\
60.125	0.1134	24.7271092206233\\
60.125	0.114	25.5783711920643\\
60.125	0.1146	26.4296331635052\\
60.125	0.1152	27.2808951349462\\
60.125	0.1158	28.1321571063872\\
60.125	0.1164	28.9834190778282\\
60.125	0.117	29.8346810492691\\
60.125	0.1176	30.68594302071\\
60.125	0.1182	31.537204992151\\
60.125	0.1188	32.388466963592\\
60.125	0.1194	33.2397289350329\\
60.125	0.12	34.0909909064738\\
60.125	0.1206	34.9422528779149\\
60.125	0.1212	35.7935148493559\\
60.125	0.1218	36.6447768207968\\
60.125	0.1224	37.4960387922378\\
60.125	0.123	38.3473007636787\\
60.5	0.093	-4.53868891594811\\
60.5	0.0936	-3.701315275654\\
60.5	0.0942	-2.86394163535988\\
60.5	0.0948	-2.02656799506576\\
60.5	0.0954	-1.1891943547717\\
60.5	0.096	-0.351820714477583\\
60.5	0.0966	0.485552925816535\\
60.5	0.0972	1.32292656611065\\
60.5	0.0978	2.16030020640471\\
60.5	0.0984	2.99767384669883\\
60.5	0.099	3.83504748699295\\
60.5	0.0996	4.67242112728707\\
60.5	0.1002	5.50979476758118\\
60.5	0.1008	6.3471684078753\\
60.5	0.1014	7.18454204816942\\
60.5	0.102	8.02191568846348\\
60.5	0.1026	8.85928932875754\\
60.5	0.1032	9.69666296905166\\
60.5	0.1038	10.5340366093458\\
60.5	0.1044	11.37141024964\\
60.5	0.105	12.208783889934\\
60.5	0.1056	13.0461575302281\\
60.5	0.1062	13.8835311705222\\
60.5	0.1068	14.7209048108164\\
60.5	0.1074	15.5582784511104\\
60.5	0.108	16.3956520914045\\
60.5	0.1086	17.2330257316986\\
60.5	0.1092	18.0703993719927\\
60.5	0.1098	18.9077730122868\\
60.5	0.1104	19.745146652581\\
60.5	0.111	20.5825202928751\\
60.5	0.1116	21.4198939331692\\
60.5	0.1122	22.2572675734632\\
60.5	0.1128	23.0946412137573\\
60.5	0.1134	23.9320148540514\\
60.5	0.114	24.7693884943455\\
60.5	0.1146	25.6067621346396\\
60.5	0.1152	26.4441357749338\\
60.5	0.1158	27.2815094152279\\
60.5	0.1164	28.118883055522\\
60.5	0.117	28.956256695816\\
60.5	0.1176	29.7936303361101\\
60.5	0.1182	30.6310039764043\\
60.5	0.1188	31.4683776166984\\
60.5	0.1194	32.3057512569924\\
60.5	0.12	33.1431248972865\\
60.5	0.1206	33.9804985375807\\
60.5	0.1212	34.8178721778748\\
60.5	0.1218	35.6552458181689\\
60.5	0.1224	36.492619458463\\
60.5	0.123	37.3299930987571\\
60.875	0.093	-4.86158002352659\\
60.875	0.0936	-4.03809471437933\\
60.875	0.0942	-3.21460940523207\\
60.875	0.0948	-2.39112409608487\\
60.875	0.0954	-1.56763878693761\\
60.875	0.096	-0.744153477790348\\
60.875	0.0966	0.0793318313568534\\
60.875	0.0972	0.902817140504112\\
60.875	0.0978	1.72630244965137\\
60.875	0.0984	2.54978775879863\\
60.875	0.099	3.37327306794583\\
60.875	0.0996	4.19675837709309\\
60.875	0.1002	5.02024368624041\\
60.875	0.1008	5.84372899538766\\
60.875	0.1014	6.66721430453487\\
60.875	0.102	7.49069961368207\\
60.875	0.1026	8.31418492282933\\
60.875	0.1032	9.13767023197653\\
60.875	0.1038	9.96115554112379\\
60.875	0.1044	10.7846408502711\\
60.875	0.105	11.6081261594184\\
60.875	0.1056	12.4316114685656\\
60.875	0.1062	13.2550967777128\\
60.875	0.1068	14.0785820868601\\
60.875	0.1074	14.9020673960072\\
60.875	0.108	15.7255527051545\\
60.875	0.1086	16.5490380143017\\
60.875	0.1092	17.3725233234489\\
60.875	0.1098	18.1960086325963\\
60.875	0.1104	19.0194939417435\\
60.875	0.111	19.8429792508908\\
60.875	0.1116	20.666464560038\\
60.875	0.1122	21.4899498691852\\
60.875	0.1128	22.3134351783324\\
60.875	0.1134	23.1369204874797\\
60.875	0.114	23.9604057966269\\
60.875	0.1146	24.7838911057742\\
60.875	0.1152	25.6073764149215\\
60.875	0.1158	26.4308617240687\\
60.875	0.1164	27.2543470332159\\
60.875	0.117	28.0778323423631\\
60.875	0.1176	28.9013176515104\\
60.875	0.1182	29.7248029606576\\
60.875	0.1188	30.5482882698049\\
60.875	0.1194	31.3717735789521\\
60.875	0.12	32.1952588880993\\
60.875	0.1206	33.0187441972466\\
60.875	0.1212	33.8422295063939\\
60.875	0.1218	34.6657148155411\\
60.875	0.1224	35.4892001246884\\
60.875	0.123	36.3126854338356\\
61.25	0.093	-5.18447113110511\\
61.25	0.0936	-4.37487415310471\\
61.25	0.0942	-3.56527717510431\\
61.25	0.0948	-2.75568019710397\\
61.25	0.0954	-1.94608321910357\\
61.25	0.096	-1.13648624110317\\
61.25	0.0966	-0.326889263102828\\
61.25	0.0972	0.482707714897572\\
61.25	0.0978	1.29230469289797\\
61.25	0.0984	2.10190167089831\\
61.25	0.099	2.91149864889871\\
61.25	0.0996	3.72109562689911\\
61.25	0.1002	4.53069260489951\\
61.25	0.1008	5.34028958289991\\
61.25	0.1014	6.14988656090031\\
61.25	0.102	6.9594835389006\\
61.25	0.1026	7.769080516901\\
61.25	0.1032	8.5786774949014\\
61.25	0.1038	9.3882744729018\\
61.25	0.1044	10.1978714509022\\
61.25	0.105	11.0074684289026\\
61.25	0.1056	11.817065406903\\
61.25	0.1062	12.6266623849033\\
61.25	0.1068	13.4362593629037\\
61.25	0.1074	14.2458563409041\\
61.25	0.108	15.0554533189044\\
61.25	0.1086	15.8650502969048\\
61.25	0.1092	16.6746472749052\\
61.25	0.1098	17.4842442529056\\
61.25	0.1104	18.293841230906\\
61.25	0.111	19.1034382089064\\
61.25	0.1116	19.9130351869068\\
61.25	0.1122	20.7226321649071\\
61.25	0.1128	21.5322291429075\\
61.25	0.1134	22.3418261209079\\
61.25	0.114	23.1514230989083\\
61.25	0.1146	23.9610200769087\\
61.25	0.1152	24.7706170549091\\
61.25	0.1158	25.5802140329095\\
61.25	0.1164	26.3898110109099\\
61.25	0.117	27.1994079889102\\
61.25	0.1176	28.0090049669105\\
61.25	0.1182	28.8186019449109\\
61.25	0.1188	29.6281989229113\\
61.25	0.1194	30.4377959009117\\
61.25	0.12	31.247392878912\\
61.25	0.1206	32.0569898569125\\
61.25	0.1212	32.8665868349129\\
61.25	0.1218	33.6761838129132\\
61.25	0.1224	34.4857807909136\\
61.25	0.123	35.295377768914\\
61.625	0.093	-5.50736223868358\\
61.625	0.0936	-4.71165359183004\\
61.625	0.0942	-3.91594494497656\\
61.625	0.0948	-3.12023629812302\\
61.625	0.0954	-2.32452765126948\\
61.625	0.096	-1.52881900441599\\
61.625	0.0966	-0.733110357562452\\
61.625	0.0972	0.0625982892910883\\
61.625	0.0978	0.858306936144572\\
61.625	0.0984	1.65401558299811\\
61.625	0.099	2.44972422985165\\
61.625	0.0996	3.24543287670514\\
61.625	0.1002	4.04114152355874\\
61.625	0.1008	4.83685017041228\\
61.625	0.1014	5.63255881726576\\
61.625	0.102	6.42826746411924\\
61.625	0.1026	7.22397611097279\\
61.625	0.1032	8.01968475782627\\
61.625	0.1038	8.81539340467981\\
61.625	0.1044	9.61110205153335\\
61.625	0.105	10.4068106983869\\
61.625	0.1056	11.2025193452404\\
61.625	0.1062	11.998227992094\\
61.625	0.1068	12.7939366389475\\
61.625	0.1074	13.5896452858009\\
61.625	0.108	14.3853539326545\\
61.625	0.1086	15.181062579508\\
61.625	0.1092	15.9767712263614\\
61.625	0.1098	16.772479873215\\
61.625	0.1104	17.5681885200686\\
61.625	0.111	18.3638971669221\\
61.625	0.1116	19.1596058137757\\
61.625	0.1122	19.9553144606291\\
61.625	0.1128	20.7510231074826\\
61.625	0.1134	21.5467317543362\\
61.625	0.114	22.3424404011897\\
61.625	0.1146	23.1381490480432\\
61.625	0.1152	23.9338576948967\\
61.625	0.1158	24.7295663417503\\
61.625	0.1164	25.5252749886038\\
61.625	0.117	26.3209836354573\\
61.625	0.1176	27.1166922823108\\
61.625	0.1182	27.9124009291643\\
61.625	0.1188	28.7081095760178\\
61.625	0.1194	29.5038182228713\\
61.625	0.12	30.2995268697248\\
61.625	0.1206	31.0952355165784\\
61.625	0.1212	31.890944163432\\
61.625	0.1218	32.6866528102855\\
61.625	0.1224	33.4823614571389\\
61.625	0.123	34.2780701039925\\
62	0.093	-5.83025334626205\\
62	0.0936	-5.04843303055543\\
62	0.0942	-4.26661271484875\\
62	0.0948	-3.48479239914207\\
62	0.0954	-2.70297208343544\\
62	0.096	-1.92115176772876\\
62	0.0966	-1.13933145202208\\
62	0.0972	-0.357511136315452\\
62	0.0978	0.42430917939123\\
62	0.0984	1.20612949509785\\
62	0.099	1.98794981080454\\
62	0.0996	2.76977012651122\\
62	0.1002	3.5515904422179\\
62	0.1008	4.33341075792458\\
62	0.1014	5.11523107363126\\
62	0.102	5.89705138933783\\
62	0.1026	6.67887170504451\\
62	0.1032	7.4606920207512\\
62	0.1038	8.24251233645782\\
62	0.1044	9.02433265216456\\
62	0.105	9.80615296787118\\
62	0.1056	10.5879732835779\\
62	0.1062	11.3697935992845\\
62	0.1068	12.1516139149912\\
62	0.1074	12.9334342306978\\
62	0.108	13.7152545464045\\
62	0.1086	14.4970748621111\\
62	0.1092	15.2788951778177\\
62	0.1098	16.0607154935245\\
62	0.1104	16.8425358092312\\
62	0.111	17.6243561249378\\
62	0.1116	18.4061764406445\\
62	0.1122	19.1879967563511\\
62	0.1128	19.9698170720578\\
62	0.1134	20.7516373877644\\
62	0.114	21.5334577034711\\
62	0.1146	22.3152780191778\\
62	0.1152	23.0970983348844\\
62	0.1158	23.8789186505911\\
62	0.1164	24.6607389662978\\
62	0.117	25.4425592820044\\
62	0.1176	26.224379597711\\
62	0.1182	27.0061999134177\\
62	0.1188	27.7880202291244\\
62	0.1194	28.569840544831\\
62	0.12	29.3516608605376\\
62	0.1206	30.1334811762444\\
62	0.1212	30.9153014919511\\
62	0.1218	31.6971218076577\\
62	0.1224	32.4789421233643\\
62	0.123	33.260762439071\\
62.375	0.093	-6.15314445384064\\
62.375	0.0936	-5.38521246928087\\
62.375	0.0942	-4.61728048472105\\
62.375	0.0948	-3.84934850016123\\
62.375	0.0954	-3.08141651560146\\
62.375	0.096	-2.31348453104164\\
62.375	0.0966	-1.54555254648187\\
62.375	0.0972	-0.777620561922049\\
62.375	0.0978	-0.00968857736222617\\
62.375	0.0984	0.75824340719754\\
62.375	0.099	1.52617539175736\\
62.375	0.0996	2.29410737631713\\
62.375	0.1002	3.06203936087701\\
62.375	0.1008	3.82997134543683\\
62.375	0.1014	4.5979033299966\\
62.375	0.102	5.36583531455636\\
62.375	0.1026	6.13376729911613\\
62.375	0.1032	6.90169928367595\\
62.375	0.1038	7.66963126823578\\
62.375	0.1044	8.4375632527956\\
62.375	0.105	9.20549523735542\\
62.375	0.1056	9.97342722191519\\
62.375	0.1062	10.741359206475\\
62.375	0.1068	11.5092911910348\\
62.375	0.1074	12.2772231755945\\
62.375	0.108	13.0451551601544\\
62.375	0.1086	13.8130871447141\\
62.375	0.1092	14.5810191292739\\
62.375	0.1098	15.3489511138338\\
62.375	0.1104	16.1168830983936\\
62.375	0.111	16.8848150829534\\
62.375	0.1116	17.6527470675132\\
62.375	0.1122	18.420679052073\\
62.375	0.1128	19.1886110366328\\
62.375	0.1134	19.9565430211925\\
62.375	0.114	20.7244750057524\\
62.375	0.1146	21.4924069903122\\
62.375	0.1152	22.260338974872\\
62.375	0.1158	23.0282709594318\\
62.375	0.1164	23.7962029439916\\
62.375	0.117	24.5641349285514\\
62.375	0.1176	25.3320669131112\\
62.375	0.1182	26.099998897671\\
62.375	0.1188	26.8679308822308\\
62.375	0.1194	27.6358628667905\\
62.375	0.12	28.4037948513503\\
62.375	0.1206	29.1717268359102\\
62.375	0.1212	29.93965882047\\
62.375	0.1218	30.7075908050298\\
62.375	0.1224	31.4755227895895\\
62.375	0.123	32.2434547741494\\
62.75	0.093	-6.47603556141911\\
62.75	0.0936	-5.7219919080062\\
62.75	0.0942	-4.96794825459324\\
62.75	0.0948	-4.21390460118033\\
62.75	0.0954	-3.45986094776737\\
62.75	0.096	-2.70581729435446\\
62.75	0.0966	-1.9517736409415\\
62.75	0.0972	-1.19772998752853\\
62.75	0.0978	-0.443686334115625\\
62.75	0.0984	0.310357319297339\\
62.75	0.099	1.06440097271025\\
62.75	0.0996	1.81844462612321\\
62.75	0.1002	2.57248827953617\\
62.75	0.1008	3.32653193294914\\
62.75	0.1014	4.0805755863621\\
62.75	0.102	4.83461923977495\\
62.75	0.1026	5.58866289318792\\
62.75	0.1032	6.34270654660082\\
62.75	0.1038	7.09675020001379\\
62.75	0.1044	7.85079385342675\\
62.75	0.105	8.60483750683971\\
62.75	0.1056	9.35888116025268\\
62.75	0.1062	10.1129248136656\\
62.75	0.1068	10.8669684670786\\
62.75	0.1074	11.6210121204914\\
62.75	0.108	12.3750557739044\\
62.75	0.1086	13.1290994273173\\
62.75	0.1092	13.8831430807302\\
62.75	0.1098	14.6371867341433\\
62.75	0.1104	15.3912303875562\\
62.75	0.111	16.1452740409691\\
62.75	0.1116	16.899317694382\\
62.75	0.1122	17.6533613477949\\
62.75	0.1128	18.4074050012079\\
62.75	0.1134	19.1614486546208\\
62.75	0.114	19.9154923080338\\
62.75	0.1146	20.6695359614467\\
62.75	0.1152	21.4235796148597\\
62.75	0.1158	22.1776232682726\\
62.75	0.1164	22.9316669216856\\
62.75	0.117	23.6857105750985\\
62.75	0.1176	24.4397542285114\\
62.75	0.1182	25.1937978819244\\
62.75	0.1188	25.9478415353373\\
62.75	0.1194	26.7018851887502\\
62.75	0.12	27.4559288421631\\
62.75	0.1206	28.2099724955762\\
62.75	0.1212	28.9640161489891\\
62.75	0.1218	29.718059802402\\
62.75	0.1224	30.4721034558149\\
62.75	0.123	31.2261471092278\\
63.125	0.093	-6.79892666899764\\
63.125	0.0936	-6.05877134673153\\
63.125	0.0942	-5.31861602446548\\
63.125	0.0948	-4.57846070219938\\
63.125	0.0954	-3.83830537993327\\
63.125	0.096	-3.09815005766723\\
63.125	0.0966	-2.35799473540112\\
63.125	0.0972	-1.61783941313507\\
63.125	0.0978	-0.877684090868968\\
63.125	0.0984	-0.13752876860292\\
63.125	0.099	0.602626553663185\\
63.125	0.0996	1.34278187592923\\
63.125	0.1002	2.0829371981954\\
63.125	0.1008	2.82309252046144\\
63.125	0.1014	3.56324784272755\\
63.125	0.102	4.3034031649936\\
63.125	0.1026	5.04355848725965\\
63.125	0.1032	5.78371380952575\\
63.125	0.1038	6.5238691317918\\
63.125	0.1044	7.26402445405796\\
63.125	0.105	8.00417977632401\\
63.125	0.1056	8.74433509859011\\
63.125	0.1062	9.48449042085616\\
63.125	0.1068	10.2246457431223\\
63.125	0.1074	10.9648010653883\\
63.125	0.108	11.7049563876544\\
63.125	0.1086	12.4451117099205\\
63.125	0.1092	13.1852670321865\\
63.125	0.1098	13.9254223544527\\
63.125	0.1104	14.6655776767187\\
63.125	0.111	15.4057329989848\\
63.125	0.1116	16.1458883212509\\
63.125	0.1122	16.8860436435169\\
63.125	0.1128	17.626198965783\\
63.125	0.1134	18.3663542880491\\
63.125	0.114	19.1065096103152\\
63.125	0.1146	19.8466649325812\\
63.125	0.1152	20.5868202548474\\
63.125	0.1158	21.3269755771134\\
63.125	0.1164	22.0671308993795\\
63.125	0.117	22.8072862216455\\
63.125	0.1176	23.5474415439116\\
63.125	0.1182	24.2875968661777\\
63.125	0.1188	25.0277521884438\\
63.125	0.1194	25.7679075107098\\
63.125	0.12	26.5080628329759\\
63.125	0.1206	27.2482181552421\\
63.125	0.1212	27.9883734775082\\
63.125	0.1218	28.7285287997742\\
63.125	0.1224	29.4686841220403\\
63.125	0.123	30.2088394443064\\
63.5	0.093	-7.12181777657617\\
63.5	0.0936	-6.39555078545692\\
63.5	0.0942	-5.66928379433773\\
63.5	0.0948	-4.94301680321848\\
63.5	0.0954	-4.21674981209929\\
63.5	0.096	-3.49048282098005\\
63.5	0.0966	-2.76421582986086\\
63.5	0.0972	-2.03794883874161\\
63.5	0.0978	-1.31168184762242\\
63.5	0.0984	-0.585414856503178\\
63.5	0.099	0.140852134616068\\
63.5	0.0996	0.867119125735258\\
63.5	0.1002	1.59338611685456\\
63.5	0.1008	2.31965310797375\\
63.5	0.1014	3.045920099093\\
63.5	0.102	3.77218709021213\\
63.5	0.1026	4.49845408133137\\
63.5	0.1032	5.22472107245056\\
63.5	0.1038	5.95098806356981\\
63.5	0.1044	6.67725505468906\\
63.5	0.105	7.4035220458083\\
63.5	0.1056	8.12978903692749\\
63.5	0.1062	8.85605602804674\\
63.5	0.1068	9.58232301916593\\
63.5	0.1074	10.3085900102851\\
63.5	0.108	11.0348570014043\\
63.5	0.1086	11.7611239925236\\
63.5	0.1092	12.4873909836427\\
63.5	0.1098	13.213657974762\\
63.5	0.1104	13.9399249658812\\
63.5	0.111	14.6661919570005\\
63.5	0.1116	15.3924589481197\\
63.5	0.1122	16.1187259392389\\
63.5	0.1128	16.844992930358\\
63.5	0.1134	17.5712599214773\\
63.5	0.114	18.2975269125965\\
63.5	0.1146	19.0237939037157\\
63.5	0.1152	19.750060894835\\
63.5	0.1158	20.4763278859542\\
63.5	0.1164	21.2025948770735\\
63.5	0.117	21.9288618681926\\
63.5	0.1176	22.6551288593118\\
63.5	0.1182	23.381395850431\\
63.5	0.1188	24.1076628415503\\
63.5	0.1194	24.8339298326694\\
63.5	0.12	25.5601968237887\\
63.5	0.1206	26.286463814908\\
63.5	0.1212	27.0127308060272\\
63.5	0.1218	27.7389977971463\\
63.5	0.1224	28.4652647882656\\
63.5	0.123	29.1915317793848\\
63.875	0.093	-7.44470888415469\\
63.875	0.0936	-6.73233022418231\\
63.875	0.0942	-6.01995156420998\\
63.875	0.0948	-5.30757290423759\\
63.875	0.0954	-4.59519424426526\\
63.875	0.096	-3.88281558429287\\
63.875	0.0966	-3.17043692432054\\
63.875	0.0972	-2.45805826434815\\
63.875	0.0978	-1.74567960437582\\
63.875	0.0984	-1.03330094440344\\
63.875	0.099	-0.320922284431106\\
63.875	0.0996	0.391456375541281\\
63.875	0.1002	1.10383503551367\\
63.875	0.1008	1.81621369548606\\
63.875	0.1014	2.52859235545839\\
63.875	0.102	3.24097101543066\\
63.875	0.1026	3.95334967540305\\
63.875	0.1032	4.66572833537538\\
63.875	0.1038	5.37810699534776\\
63.875	0.1044	6.09048565532015\\
63.875	0.105	6.80286431529254\\
63.875	0.1056	7.51524297526487\\
63.875	0.1062	8.22762163523726\\
63.875	0.1068	8.94000029520959\\
63.875	0.1074	9.65237895518192\\
63.875	0.108	10.3647576151542\\
63.875	0.1086	11.0771362751266\\
63.875	0.1092	11.7895149350989\\
63.875	0.1098	12.5018935950714\\
63.875	0.1104	13.2142722550437\\
63.875	0.111	13.9266509150161\\
63.875	0.1116	14.6390295749885\\
63.875	0.1122	15.3514082349608\\
63.875	0.1128	16.0637868949331\\
63.875	0.1134	16.7761655549055\\
63.875	0.114	17.4885442148778\\
63.875	0.1146	18.2009228748502\\
63.875	0.1152	18.9133015348226\\
63.875	0.1158	19.625680194795\\
63.875	0.1164	20.3380588547673\\
63.875	0.117	21.0504375147397\\
63.875	0.1176	21.762816174712\\
63.875	0.1182	22.4751948346844\\
63.875	0.1188	23.1875734946567\\
63.875	0.1194	23.899952154629\\
63.875	0.12	24.6123308146014\\
63.875	0.1206	25.3247094745739\\
63.875	0.1212	26.0370881345462\\
63.875	0.1218	26.7494667945185\\
63.875	0.1224	27.4618454544909\\
63.875	0.123	28.1742241144632\\
64.25	0.093	-7.76759999173322\\
64.25	0.0936	-7.06910966290769\\
64.25	0.0942	-6.37061933408222\\
64.25	0.0948	-5.67212900525669\\
64.25	0.0954	-4.97363867643122\\
64.25	0.096	-4.27514834760575\\
64.25	0.0966	-3.57665801878022\\
64.25	0.0972	-2.87816768995475\\
64.25	0.0978	-2.17967736112922\\
64.25	0.0984	-1.48118703230375\\
64.25	0.099	-0.782696703478223\\
64.25	0.0996	-0.0842063746527515\\
64.25	0.1002	0.614283954172834\\
64.25	0.1008	1.3127742829983\\
64.25	0.1014	2.01126461182383\\
64.25	0.102	2.70975494064925\\
64.25	0.1026	3.40824526947478\\
64.25	0.1032	4.10673559830025\\
64.25	0.1038	4.80522592712572\\
64.25	0.1044	5.5037162559513\\
64.25	0.105	6.20220658477677\\
64.25	0.1056	6.9006969136023\\
64.25	0.1062	7.59918724242777\\
64.25	0.1068	8.2976775712533\\
64.25	0.1074	8.99616790007872\\
64.25	0.108	9.69465822890425\\
64.25	0.1086	10.3931485577297\\
64.25	0.1092	11.0916388865552\\
64.25	0.1098	11.7901292153808\\
64.25	0.1104	12.4886195442062\\
64.25	0.111	13.1871098730318\\
64.25	0.1116	13.8856002018572\\
64.25	0.1122	14.5840905306827\\
64.25	0.1128	15.2825808595082\\
64.25	0.1134	15.9810711883337\\
64.25	0.114	16.6795615171592\\
64.25	0.1146	17.3780518459847\\
64.25	0.1152	18.0765421748102\\
64.25	0.1158	18.7750325036358\\
64.25	0.1164	19.4735228324612\\
64.25	0.117	20.1720131612867\\
64.25	0.1176	20.8705034901122\\
64.25	0.1182	21.5689938189377\\
64.25	0.1188	22.2674841477632\\
64.25	0.1194	22.9659744765886\\
64.25	0.12	23.6644648054141\\
64.25	0.1206	24.3629551342397\\
64.25	0.1212	25.0614454630652\\
64.25	0.1218	25.7599357918907\\
64.25	0.1224	26.4584261207162\\
64.25	0.123	27.1569164495417\\
64.625	0.093	-8.09049109931169\\
64.625	0.0936	-7.40588910163308\\
64.625	0.0942	-6.72128710395441\\
64.625	0.0948	-6.0366851062758\\
64.625	0.0954	-5.35208310859713\\
64.625	0.096	-4.66748111091852\\
64.625	0.0966	-3.98287911323985\\
64.625	0.0972	-3.29827711556123\\
64.625	0.0978	-2.61367511788262\\
64.625	0.0984	-1.92907312020395\\
64.625	0.099	-1.24447112252534\\
64.625	0.0996	-0.559869124846671\\
64.625	0.1002	0.124732872831999\\
64.625	0.1008	0.809334870510668\\
64.625	0.1014	1.49393686818928\\
64.625	0.102	2.17853886586784\\
64.625	0.1026	2.8631408635465\\
64.625	0.1032	3.54774286122512\\
64.625	0.1038	4.23234485890379\\
64.625	0.1044	4.91694685658246\\
64.625	0.105	5.60154885426113\\
64.625	0.1056	6.28615085193974\\
64.625	0.1062	6.97075284961835\\
64.625	0.1068	7.65535484729702\\
64.625	0.1074	8.33995684497557\\
64.625	0.108	9.02455884265424\\
64.625	0.1086	9.70916084033286\\
64.625	0.1092	10.3937628380115\\
64.625	0.1098	11.0783648356902\\
64.625	0.1104	11.7629668333688\\
64.625	0.111	12.4475688310475\\
64.625	0.1116	13.1321708287261\\
64.625	0.1122	13.8167728264047\\
64.625	0.1128	14.5013748240833\\
64.625	0.1134	15.185976821762\\
64.625	0.114	15.8705788194406\\
64.625	0.1146	16.5551808171193\\
64.625	0.1152	17.2397828147979\\
64.625	0.1158	17.9243848124765\\
64.625	0.1164	18.6089868101552\\
64.625	0.117	19.2935888078338\\
64.625	0.1176	19.9781908055124\\
64.625	0.1182	20.6627928031911\\
64.625	0.1188	21.3473948008697\\
64.625	0.1194	22.0319967985483\\
64.625	0.12	22.7165987962269\\
64.625	0.1206	23.4012007939057\\
64.625	0.1212	24.0858027915843\\
64.625	0.1218	24.7704047892629\\
64.625	0.1224	25.4550067869415\\
64.625	0.123	26.1396087846202\\
65	0.093	-8.41338220689016\\
65	0.0936	-7.74266854035841\\
65	0.0942	-7.0719548738266\\
65	0.0948	-6.40124120729485\\
65	0.0954	-5.73052754076309\\
65	0.096	-5.05981387423128\\
65	0.0966	-4.38910020769953\\
65	0.0972	-3.71838654116772\\
65	0.0978	-3.04767287463596\\
65	0.0984	-2.37695920810421\\
65	0.099	-1.7062455415724\\
65	0.0996	-1.03553187504065\\
65	0.1002	-0.36481820850878\\
65	0.1008	0.305895458022974\\
65	0.1014	0.976609124554727\\
65	0.102	1.64732279108648\\
65	0.1026	2.31803645761823\\
65	0.1032	2.98875012415004\\
65	0.1038	3.6594637906818\\
65	0.1044	4.33017745721361\\
65	0.105	5.00089112374542\\
65	0.1056	5.67160479027717\\
65	0.1062	6.34231845680898\\
65	0.1068	7.01303212334074\\
65	0.1074	7.68374578987243\\
65	0.108	8.35445945640424\\
65	0.1086	9.025173122936\\
65	0.1092	9.69588678946775\\
65	0.1098	10.3666004559996\\
65	0.1104	11.0373141225314\\
65	0.111	11.7080277890632\\
65	0.1116	12.3787414555949\\
65	0.1122	13.0494551221267\\
65	0.1128	13.7201687886584\\
65	0.1134	14.3908824551903\\
65	0.114	15.061596121722\\
65	0.1146	15.7323097882538\\
65	0.1152	16.4030234547856\\
65	0.1158	17.0737371213174\\
65	0.1164	17.7444507878492\\
65	0.117	18.4151644543809\\
65	0.1176	19.0858781209126\\
65	0.1182	19.7565917874444\\
65	0.1188	20.4273054539762\\
65	0.1194	21.098019120508\\
65	0.12	21.7687327870397\\
65	0.1206	22.4394464535716\\
65	0.1212	23.1101601201034\\
65	0.1218	23.7808737866351\\
65	0.1224	24.4515874531669\\
65	0.123	25.1223011196986\\
65.375	0.093	-8.73627331446869\\
65.375	0.0936	-8.0794479790838\\
65.375	0.0942	-7.42262264369884\\
65.375	0.0948	-6.76579730831395\\
65.375	0.0954	-6.10897197292906\\
65.375	0.096	-5.4521466375441\\
65.375	0.0966	-4.79532130215921\\
65.375	0.0972	-4.13849596677431\\
65.375	0.0978	-3.48167063138936\\
65.375	0.0984	-2.82484529600447\\
65.375	0.099	-2.16801996061952\\
65.375	0.0996	-1.51119462523462\\
65.375	0.1002	-0.854369289849672\\
65.375	0.1008	-0.19754395446472\\
65.375	0.1014	0.459281380920174\\
65.375	0.102	1.11610671630501\\
65.375	0.1026	1.77293205168996\\
65.375	0.1032	2.42975738707486\\
65.375	0.1038	3.08658272245975\\
65.375	0.1044	3.74340805784476\\
65.375	0.105	4.40023339322966\\
65.375	0.1056	5.05705872861461\\
65.375	0.1062	5.7138840639995\\
65.375	0.1068	6.3707093993844\\
65.375	0.1074	7.02753473476929\\
65.375	0.108	7.68436007015418\\
65.375	0.1086	8.34118540553908\\
65.375	0.1092	8.99801074092397\\
65.375	0.1098	9.65483607630898\\
65.375	0.1104	10.3116614116939\\
65.375	0.111	10.9684867470788\\
65.375	0.1116	11.6253120824637\\
65.375	0.1122	12.2821374178486\\
65.375	0.1128	12.9389627532335\\
65.375	0.1134	13.5957880886184\\
65.375	0.114	14.2526134240034\\
65.375	0.1146	14.9094387593883\\
65.375	0.1152	15.5662640947733\\
65.375	0.1158	16.2230894301582\\
65.375	0.1164	16.879914765543\\
65.375	0.117	17.5367401009279\\
65.375	0.1176	18.1935654363128\\
65.375	0.1182	18.8503907716977\\
65.375	0.1188	19.5072161070827\\
65.375	0.1194	20.1640414424675\\
65.375	0.12	20.8208667778525\\
65.375	0.1206	21.4776921132375\\
65.375	0.1212	22.1345174486224\\
65.375	0.1218	22.7913427840073\\
65.375	0.1224	23.4481681193922\\
65.375	0.123	24.1049934547771\\
65.75	0.093	-9.05916442204722\\
65.75	0.0936	-8.41622741780918\\
65.75	0.0942	-7.77329041357115\\
65.75	0.0948	-7.13035340933305\\
65.75	0.0954	-6.48741640509502\\
65.75	0.096	-5.84447940085698\\
65.75	0.0966	-5.20154239661889\\
65.75	0.0972	-4.55860539238085\\
65.75	0.0978	-3.91566838814282\\
65.75	0.0984	-3.27273138390473\\
65.75	0.099	-2.62979437966669\\
65.75	0.0996	-1.98685737542866\\
65.75	0.1002	-1.34392037119051\\
65.75	0.1008	-0.700983366952471\\
65.75	0.1014	-0.0580463627144354\\
65.75	0.102	0.5848906415236\\
65.75	0.1026	1.22782764576164\\
65.75	0.1032	1.87076464999967\\
65.75	0.1038	2.51370165423776\\
65.75	0.1044	3.15663865847586\\
65.75	0.105	3.79957566271389\\
65.75	0.1056	4.44251266695198\\
65.75	0.1062	5.08544967119002\\
65.75	0.1068	5.72838667542811\\
65.75	0.1074	6.37132367966609\\
65.75	0.108	7.01426068390413\\
65.75	0.1086	7.65719768814222\\
65.75	0.1092	8.3001346923802\\
65.75	0.1098	8.94307169661835\\
65.75	0.1104	9.58600870085644\\
65.75	0.111	10.2289457050945\\
65.75	0.1116	10.8718827093325\\
65.75	0.1122	11.5148197135705\\
65.75	0.1128	12.1577567178086\\
65.75	0.1134	12.8006937220466\\
65.75	0.114	13.4436307262847\\
65.75	0.1146	14.0865677305227\\
65.75	0.1152	14.7295047347608\\
65.75	0.1158	15.3724417389989\\
65.75	0.1164	16.015378743237\\
65.75	0.117	16.6583157474749\\
65.75	0.1176	17.301252751713\\
65.75	0.1182	17.9441897559511\\
65.75	0.1188	18.5871267601891\\
65.75	0.1194	19.2300637644271\\
65.75	0.12	19.8730007686652\\
65.75	0.1206	20.5159377729033\\
65.75	0.1212	21.1588747771414\\
65.75	0.1218	21.8018117813793\\
65.75	0.1224	22.4447487856175\\
65.75	0.123	23.0876857898555\\
66.125	0.093	-9.38205552962575\\
66.125	0.0936	-8.75300685653451\\
66.125	0.0942	-8.12395818344334\\
66.125	0.0948	-7.49490951035216\\
66.125	0.0954	-6.86586083726093\\
66.125	0.096	-6.23681216416975\\
66.125	0.0966	-5.60776349107857\\
66.125	0.0972	-4.97871481798734\\
66.125	0.0978	-4.34966614489616\\
66.125	0.0984	-3.72061747180499\\
66.125	0.099	-3.09156879871375\\
66.125	0.0996	-2.46252012562258\\
66.125	0.1002	-1.83347145253134\\
66.125	0.1008	-1.20442277944016\\
66.125	0.1014	-0.575374106348931\\
66.125	0.102	0.0536745667421883\\
66.125	0.1026	0.682723239833365\\
66.125	0.1032	1.3117719129246\\
66.125	0.1038	1.94082058601578\\
66.125	0.1044	2.56986925910701\\
66.125	0.105	3.19891793219824\\
66.125	0.1056	3.82796660528942\\
66.125	0.1062	4.4570152783806\\
66.125	0.1068	5.08606395147183\\
66.125	0.1074	5.71511262456295\\
66.125	0.108	6.34416129765413\\
66.125	0.1086	6.97320997074536\\
66.125	0.1092	7.60225864383648\\
66.125	0.1098	8.23130731692777\\
66.125	0.1104	8.860355990019\\
66.125	0.111	9.48940466311018\\
66.125	0.1116	10.1184533362014\\
66.125	0.1122	10.7475020092925\\
66.125	0.1128	11.3765506823837\\
66.125	0.1134	12.0055993554749\\
66.125	0.114	12.6346480285661\\
66.125	0.1146	13.2636967016573\\
66.125	0.1152	13.8927453747485\\
66.125	0.1158	14.5217940478397\\
66.125	0.1164	15.1508427209309\\
66.125	0.117	15.7798913940221\\
66.125	0.1176	16.4089400671132\\
66.125	0.1182	17.0379887402045\\
66.125	0.1188	17.6670374132956\\
66.125	0.1194	18.2960860863868\\
66.125	0.12	18.925134759478\\
66.125	0.1206	19.5541834325693\\
66.125	0.1212	20.1832321056605\\
66.125	0.1218	20.8122807787517\\
66.125	0.1224	21.4413294518428\\
66.125	0.123	22.0703781249341\\
66.5	0.093	-9.70494663720422\\
66.5	0.0936	-9.0897862952599\\
66.5	0.0942	-8.47462595331552\\
66.5	0.0948	-7.85946561137121\\
66.5	0.0954	-7.24430526942689\\
66.5	0.096	-6.62914492748251\\
66.5	0.0966	-6.0139845855382\\
66.5	0.0972	-5.39882424359388\\
66.5	0.0978	-4.78366390164956\\
66.5	0.0984	-4.16850355970519\\
66.5	0.099	-3.55334321776087\\
66.5	0.0996	-2.93818287581655\\
66.5	0.1002	-2.32302253387212\\
66.5	0.1008	-1.7078621919278\\
66.5	0.1014	-1.09270184998348\\
66.5	0.102	-0.477541508039167\\
66.5	0.1026	0.137618833905151\\
66.5	0.1032	0.752779175849469\\
66.5	0.1038	1.36793951779379\\
66.5	0.1044	1.98309985973822\\
66.5	0.105	2.59826020168254\\
66.5	0.1056	3.21342054362685\\
66.5	0.1062	3.82858088557123\\
66.5	0.1068	4.44374122751555\\
66.5	0.1074	5.05890156945981\\
66.5	0.108	5.67406191140412\\
66.5	0.1086	6.2892222533485\\
66.5	0.1092	6.90438259529276\\
66.5	0.1098	7.51954293723719\\
66.5	0.1104	8.13470327918156\\
66.5	0.111	8.74986362112588\\
66.5	0.1116	9.3650239630702\\
66.5	0.1122	9.98018430501446\\
66.5	0.1128	10.5953446469588\\
66.5	0.1134	11.2105049889032\\
66.5	0.114	11.8256653308475\\
66.5	0.1146	12.4408256727918\\
66.5	0.1152	13.0559860147362\\
66.5	0.1158	13.6711463566805\\
66.5	0.1164	14.2863066986249\\
66.5	0.117	14.9014670405692\\
66.5	0.1176	15.5166273825135\\
66.5	0.1182	16.1317877244578\\
66.5	0.1188	16.7469480664022\\
66.5	0.1194	17.3621084083464\\
66.5	0.12	17.9772687502908\\
66.5	0.1206	18.5924290922353\\
66.5	0.1212	19.2075894341795\\
66.5	0.1218	19.8227497761238\\
66.5	0.1224	20.4379101180682\\
66.5	0.123	21.0530704600125\\
66.875	0.093	-10.0278377447827\\
66.875	0.0936	-9.42656573398529\\
66.875	0.0942	-8.82529372318777\\
66.875	0.0948	-8.22402171239031\\
66.875	0.0954	-7.62274970159285\\
66.875	0.096	-7.02147769079539\\
66.875	0.0966	-6.42020567999788\\
66.875	0.0972	-5.81893366920042\\
66.875	0.0978	-5.21766165840296\\
66.875	0.0984	-4.61638964760544\\
66.875	0.099	-4.01511763680799\\
66.875	0.0996	-3.41384562601053\\
66.875	0.1002	-2.81257361521301\\
66.875	0.1008	-2.2113016044155\\
66.875	0.1014	-1.61002959361804\\
66.875	0.102	-1.00875758282064\\
66.875	0.1026	-0.407485572023177\\
66.875	0.1032	0.193786438774339\\
66.875	0.1038	0.795058449571798\\
66.875	0.1044	1.39633046036931\\
66.875	0.105	1.99760247116677\\
66.875	0.1056	2.59887448196429\\
66.875	0.1062	3.20014649276175\\
66.875	0.1068	3.8014185035592\\
66.875	0.1074	4.40269051435661\\
66.875	0.108	5.00396252515412\\
66.875	0.1086	5.60523453595158\\
66.875	0.1092	6.20650654674898\\
66.875	0.1098	6.80777855754656\\
66.875	0.1104	7.40905056834407\\
66.875	0.111	8.01032257914153\\
66.875	0.1116	8.61159458993899\\
66.875	0.1122	9.21286660073639\\
66.875	0.1128	9.81413861153391\\
66.875	0.1134	10.4154106223314\\
66.875	0.114	11.0166826331288\\
66.875	0.1146	11.6179546439263\\
66.875	0.1152	12.2192266547239\\
66.875	0.1158	12.8204986655213\\
66.875	0.1164	13.4217706763188\\
66.875	0.117	14.0230426871162\\
66.875	0.1176	14.6243146979137\\
66.875	0.1182	15.2255867087111\\
66.875	0.1188	15.8268587195086\\
66.875	0.1194	16.4281307303061\\
66.875	0.12	17.0294027411035\\
66.875	0.1206	17.630674751901\\
66.875	0.1212	18.2319467626986\\
66.875	0.1218	18.8332187734961\\
66.875	0.1224	19.4344907842934\\
66.875	0.123	20.035762795091\\
67.25	0.093	-10.3507288523613\\
67.25	0.0936	-9.76334517271073\\
67.25	0.0942	-9.17596149306007\\
67.25	0.0948	-8.58857781340947\\
67.25	0.0954	-8.00119413375887\\
67.25	0.096	-7.41381045410827\\
67.25	0.0966	-6.82642677445762\\
67.25	0.0972	-6.23904309480702\\
67.25	0.0978	-5.65165941515642\\
67.25	0.0984	-5.06427573550582\\
67.25	0.099	-4.47689205585522\\
67.25	0.0996	-3.88950837620456\\
67.25	0.1002	-3.3021246965539\\
67.25	0.1008	-2.7147410169033\\
67.25	0.1014	-2.1273573372527\\
67.25	0.102	-1.5399736576021\\
67.25	0.1026	-0.952589977951504\\
67.25	0.1032	-0.365206298300905\\
67.25	0.1038	0.222177381349695\\
67.25	0.1044	0.809561061000409\\
67.25	0.105	1.39694474065101\\
67.25	0.1056	1.98432842030161\\
67.25	0.1062	2.57171209995221\\
67.25	0.1068	3.15909577960281\\
67.25	0.1074	3.74647945925341\\
67.25	0.108	4.33386313890401\\
67.25	0.1086	4.92124681855461\\
67.25	0.1092	5.50863049820515\\
67.25	0.1098	6.09601417785592\\
67.25	0.1104	6.68339785750652\\
67.25	0.111	7.27078153715712\\
67.25	0.1116	7.85816521680772\\
67.25	0.1122	8.44554889645826\\
67.25	0.1128	9.03293257610892\\
67.25	0.1134	9.62031625575952\\
67.25	0.114	10.2076999354101\\
67.25	0.1146	10.7950836150607\\
67.25	0.1152	11.3824672947114\\
67.25	0.1158	11.969850974362\\
67.25	0.1164	12.5572346540126\\
67.25	0.117	13.1446183336632\\
67.25	0.1176	13.7320020133138\\
67.25	0.1182	14.3193856929644\\
67.25	0.1188	14.906769372615\\
67.25	0.1194	15.4941530522656\\
67.25	0.12	16.0815367319162\\
67.25	0.1206	16.6689204115669\\
67.25	0.1212	17.2563040912175\\
67.25	0.1218	17.8436877708681\\
67.25	0.1224	18.4310714505187\\
67.25	0.123	19.0184551301693\\
67.625	0.093	-10.6736199599397\\
67.625	0.0936	-10.100124611436\\
67.625	0.0942	-9.52662926293226\\
67.625	0.0948	-8.95313391442852\\
67.625	0.0954	-8.37963856592472\\
67.625	0.096	-7.80614321742098\\
67.625	0.0966	-7.23264786891724\\
67.625	0.0972	-6.6591525204135\\
67.625	0.0978	-6.08565717190976\\
67.625	0.0984	-5.51216182340596\\
67.625	0.099	-4.93866647490222\\
67.625	0.0996	-4.36517112639848\\
67.625	0.1002	-3.79167577789468\\
67.625	0.1008	-3.21818042939094\\
67.625	0.1014	-2.64468508088714\\
67.625	0.102	-2.07118973238346\\
67.625	0.1026	-1.49769438387972\\
67.625	0.1032	-0.924199035375977\\
67.625	0.1038	-0.350703686872237\\
67.625	0.1044	0.222791661631618\\
67.625	0.105	0.796287010135359\\
67.625	0.1056	1.3697823586391\\
67.625	0.1062	1.94327770714284\\
67.625	0.1068	2.51677305564658\\
67.625	0.1074	3.09026840415032\\
67.625	0.108	3.66376375265406\\
67.625	0.1086	4.2372591011578\\
67.625	0.1092	4.81075444966149\\
67.625	0.1098	5.38424979816534\\
67.625	0.1104	5.95774514666914\\
67.625	0.111	6.53124049517288\\
67.625	0.1116	7.10473584367662\\
67.625	0.1122	7.67823119218031\\
67.625	0.1128	8.2517265406841\\
67.625	0.1134	8.82522188918784\\
67.625	0.114	9.39871723769159\\
67.625	0.1146	9.97221258619533\\
67.625	0.1152	10.5457079346991\\
67.625	0.1158	11.1192032832029\\
67.625	0.1164	11.6926986317067\\
67.625	0.117	12.2661939802103\\
67.625	0.1176	12.8396893287141\\
67.625	0.1182	13.4131846772178\\
67.625	0.1188	13.9866800257216\\
67.625	0.1194	14.5601753742253\\
67.625	0.12	15.1336707227291\\
67.625	0.1206	15.7071660712329\\
67.625	0.1212	16.2806614197366\\
67.625	0.1218	16.8541567682404\\
67.625	0.1224	17.4276521167441\\
67.625	0.123	18.0011474652479\\
68	0.093	-10.9965110675183\\
68	0.0936	-10.4369040501614\\
68	0.0942	-9.87729703280451\\
68	0.0948	-9.31769001544762\\
68	0.0954	-8.75808299809074\\
68	0.096	-8.1984759807338\\
68	0.0966	-7.63886896337692\\
68	0.0972	-7.07926194602004\\
68	0.0978	-6.51965492866316\\
68	0.0984	-5.96004791130628\\
68	0.099	-5.40044089394934\\
68	0.0996	-4.84083387659246\\
68	0.1002	-4.28122685923552\\
68	0.1008	-3.72161984187863\\
68	0.1014	-3.16201282452175\\
68	0.102	-2.60240580716493\\
68	0.1026	-2.04279878980799\\
68	0.1032	-1.48319177245111\\
68	0.1038	-0.923584755094225\\
68	0.1044	-0.363977737737287\\
68	0.105	0.195629279619595\\
68	0.1056	0.755236296976477\\
68	0.1062	1.31484331433342\\
68	0.1068	1.8744503316903\\
68	0.1074	2.43405734904712\\
68	0.108	2.993664366404\\
68	0.1086	3.55327138376089\\
68	0.1092	4.11287840111771\\
68	0.1098	4.67248541847476\\
68	0.1104	5.23209243583165\\
68	0.111	5.79169945318853\\
68	0.1116	6.35130647054541\\
68	0.1122	6.91091348790223\\
68	0.1128	7.47052050525912\\
68	0.1134	8.03012752261606\\
68	0.114	8.58973453997294\\
68	0.1146	9.14934155732982\\
68	0.1152	9.70894857468676\\
68	0.1158	10.2685555920436\\
68	0.1164	10.8281626094006\\
68	0.117	11.3877696267574\\
68	0.1176	11.9473766441143\\
68	0.1182	12.5069836614712\\
68	0.1188	13.066590678828\\
68	0.1194	13.6261976961849\\
68	0.12	14.1858047135418\\
68	0.1206	14.7454117308988\\
68	0.1212	15.3050187482557\\
68	0.1218	15.8646257656125\\
68	0.1224	16.4242327829694\\
68	0.123	16.9838398003263\\
68.375	0.093	-11.3194021750968\\
68.375	0.0936	-10.7736834888868\\
68.375	0.0942	-10.2279648026768\\
68.375	0.0948	-9.68224611646673\\
68.375	0.0954	-9.13652743025671\\
68.375	0.096	-8.59080874404663\\
68.375	0.0966	-8.0450900578366\\
68.375	0.0972	-7.49937137162658\\
68.375	0.0978	-6.95365268541656\\
68.375	0.0984	-6.40793399920653\\
68.375	0.099	-5.86221531299651\\
68.375	0.0996	-5.31649662678649\\
68.375	0.1002	-4.77077794057635\\
68.375	0.1008	-4.22505925436633\\
68.375	0.1014	-3.67934056815631\\
68.375	0.102	-3.13362188194634\\
68.375	0.1026	-2.58790319573632\\
68.375	0.1032	-2.04218450952629\\
68.375	0.1038	-1.49646582331627\\
68.375	0.1044	-0.950747137106134\\
68.375	0.105	-0.405028450896111\\
68.375	0.1056	0.140690235313912\\
68.375	0.1062	0.686408921523935\\
68.375	0.1068	1.23212760773396\\
68.375	0.1074	1.77784629394392\\
68.375	0.108	2.32356498015395\\
68.375	0.1086	2.86928366636397\\
68.375	0.1092	3.41500235257399\\
68.375	0.1098	3.96072103878413\\
68.375	0.1104	4.50643972499415\\
68.375	0.111	5.05215841120418\\
68.375	0.1116	5.5978770974142\\
68.375	0.1122	6.14359578362416\\
68.375	0.1128	6.68931446983419\\
68.375	0.1134	7.23503315604427\\
68.375	0.114	7.78075184225429\\
68.375	0.1146	8.32647052846431\\
68.375	0.1152	8.87218921467439\\
68.375	0.1158	9.41790790088442\\
68.375	0.1164	9.96362658709444\\
68.375	0.117	10.5093452733044\\
68.375	0.1176	11.0550639595145\\
68.375	0.1182	11.6007826457245\\
68.375	0.1188	12.1465013319345\\
68.375	0.1194	12.6922200181445\\
68.375	0.12	13.2379387043545\\
68.375	0.1206	13.7836573905647\\
68.375	0.1212	14.3293760767747\\
68.375	0.1218	14.8750947629846\\
68.375	0.1224	15.4208134491947\\
68.375	0.123	15.9665321354047\\
68.75	0.093	-11.6422932826754\\
68.75	0.0936	-11.1104629276122\\
68.75	0.0942	-10.5786325725491\\
68.75	0.0948	-10.0468022174859\\
68.75	0.0954	-9.51497186242273\\
68.75	0.096	-8.98314150735956\\
68.75	0.0966	-8.45131115229634\\
68.75	0.0972	-7.91948079723318\\
68.75	0.0978	-7.38765044217001\\
68.75	0.0984	-6.85582008710685\\
68.75	0.099	-6.32398973204369\\
68.75	0.0996	-5.79215937698052\\
68.75	0.1002	-5.2603290219173\\
68.75	0.1008	-4.72849866685414\\
68.75	0.1014	-4.19666831179097\\
68.75	0.102	-3.66483795672781\\
68.75	0.1026	-3.13300760166464\\
68.75	0.1032	-2.60117724660148\\
68.75	0.1038	-2.06934689153832\\
68.75	0.1044	-1.5375165364751\\
68.75	0.105	-1.00568618141193\\
68.75	0.1056	-0.473855826348768\\
68.75	0.1062	0.0579745287143965\\
68.75	0.1068	0.58980488377756\\
68.75	0.1074	1.12163523884067\\
68.75	0.108	1.65346559390389\\
68.75	0.1086	2.18529594896705\\
68.75	0.1092	2.71712630403016\\
68.75	0.1098	3.24895665909344\\
68.75	0.1104	3.7807870141566\\
68.75	0.111	4.31261736921977\\
68.75	0.1116	4.84444772428293\\
68.75	0.1122	5.37627807934604\\
68.75	0.1128	5.9081084344092\\
68.75	0.1134	6.43993878947242\\
68.75	0.114	6.97176914453559\\
68.75	0.1146	7.50359949959875\\
68.75	0.1152	8.03542985466197\\
68.75	0.1158	8.56726020972513\\
68.75	0.1164	9.0990905647883\\
68.75	0.117	9.63092091985141\\
68.75	0.1176	10.1627512749146\\
68.75	0.1182	10.6945816299777\\
68.75	0.1188	11.226411985041\\
68.75	0.1194	11.7582423401041\\
68.75	0.12	12.2900726951672\\
68.75	0.1206	12.8219030502304\\
68.75	0.1212	13.3537334052937\\
68.75	0.1218	13.8855637603568\\
68.75	0.1224	14.4173941154199\\
68.75	0.123	14.9492244704832\\
69.125	0.093	-11.9651843902539\\
69.125	0.0936	-11.4472423663375\\
69.125	0.0942	-10.9293003424212\\
69.125	0.0948	-10.4113583185049\\
69.125	0.0954	-9.89341629458858\\
69.125	0.096	-9.37547427067227\\
69.125	0.0966	-8.85753224675597\\
69.125	0.0972	-8.33959022283966\\
69.125	0.0978	-7.82164819892336\\
69.125	0.0984	-7.30370617500705\\
69.125	0.099	-6.78576415109075\\
69.125	0.0996	-6.26782212717444\\
69.125	0.1002	-5.74988010325808\\
69.125	0.1008	-5.23193807934172\\
69.125	0.1014	-4.71399605542541\\
69.125	0.102	-4.19605403150916\\
69.125	0.1026	-3.67811200759286\\
69.125	0.1032	-3.16016998367655\\
69.125	0.1038	-2.64222795976025\\
69.125	0.1044	-2.12428593584389\\
69.125	0.105	-1.60634391192758\\
69.125	0.1056	-1.08840188801128\\
69.125	0.1062	-0.570459864094971\\
69.125	0.1068	-0.0525178401786661\\
69.125	0.1074	0.465424183737582\\
69.125	0.108	0.983366207653944\\
69.125	0.1086	1.50130823157025\\
69.125	0.1092	2.0192502554865\\
69.125	0.1098	2.53719227940292\\
69.125	0.1104	3.05513430331922\\
69.125	0.111	3.57307632723553\\
69.125	0.1116	4.09101835115183\\
69.125	0.1122	4.60896037506808\\
69.125	0.1128	5.12690239898438\\
69.125	0.1134	5.64484442290069\\
69.125	0.114	6.16278644681699\\
69.125	0.1146	6.6807284707333\\
69.125	0.1152	7.19867049464972\\
69.125	0.1158	7.71661251856602\\
69.125	0.1164	8.23455454248233\\
69.125	0.117	8.75249656639858\\
69.125	0.1176	9.27043859031488\\
69.125	0.1182	9.78838061423124\\
69.125	0.1188	10.3063226381474\\
69.125	0.1194	10.8242646620637\\
69.125	0.12	11.3422066859802\\
69.125	0.1206	11.8601487098965\\
69.125	0.1212	12.3780907338128\\
69.125	0.1218	12.896032757729\\
69.125	0.1224	13.4139747816454\\
69.125	0.123	13.9319168055616\\
69.5	0.093	-12.2880754978324\\
69.5	0.0936	-11.7840218050629\\
69.5	0.0942	-11.2799681122934\\
69.5	0.0948	-10.775914419524\\
69.5	0.0954	-10.2718607267545\\
69.5	0.096	-9.76780703398509\\
69.5	0.0966	-9.26375334121565\\
69.5	0.0972	-8.7596996484462\\
69.5	0.0978	-8.25564595567675\\
69.5	0.0984	-7.75159226290731\\
69.5	0.099	-7.24753857013786\\
69.5	0.0996	-6.74348487736842\\
69.5	0.1002	-6.23943118459891\\
69.5	0.1008	-5.73537749182947\\
69.5	0.1014	-5.23132379906002\\
69.5	0.102	-4.72727010629063\\
69.5	0.1026	-4.22321641352119\\
69.5	0.1032	-3.71916272075174\\
69.5	0.1038	-3.21510902798229\\
69.5	0.1044	-2.71105533521279\\
69.5	0.105	-2.20700164244329\\
69.5	0.1056	-1.70294794967384\\
69.5	0.1062	-1.1988942569044\\
69.5	0.1068	-0.69484056413495\\
69.5	0.1074	-0.19078687136556\\
69.5	0.108	0.313266821403886\\
69.5	0.1086	0.817320514173332\\
69.5	0.1092	1.32137420694272\\
69.5	0.1098	1.82542789971228\\
69.5	0.1104	2.32948159248173\\
69.5	0.111	2.83353528525117\\
69.5	0.1116	3.33758897802062\\
69.5	0.1122	3.84164267079001\\
69.5	0.1128	4.34569636355945\\
69.5	0.1134	4.8497500563289\\
69.5	0.114	5.35380374909835\\
69.5	0.1146	5.85785744186779\\
69.5	0.1152	6.3619111346373\\
69.5	0.1158	6.86596482740674\\
69.5	0.1164	7.37001852017625\\
69.5	0.117	7.87407221294563\\
69.5	0.1176	8.37812590571514\\
69.5	0.1182	8.88217959848447\\
69.5	0.1188	9.38623329125403\\
69.5	0.1194	9.89028698402342\\
69.5	0.12	10.3943406767927\\
69.5	0.1206	10.8983943695623\\
69.5	0.1212	11.4024480623318\\
69.5	0.1218	11.9065017551013\\
69.5	0.1224	12.4105554478706\\
69.5	0.123	12.9146091406402\\
69.875	0.093	-12.6109666054109\\
69.875	0.0936	-12.1208012437883\\
69.875	0.0942	-11.6306358821657\\
69.875	0.0948	-11.1404705205431\\
69.875	0.0954	-10.6503051589205\\
69.875	0.096	-10.1601397972979\\
69.875	0.0966	-9.66997443567533\\
69.875	0.0972	-9.17980907405274\\
69.875	0.0978	-8.68964371243015\\
69.875	0.0984	-8.19947835080757\\
69.875	0.099	-7.70931298918498\\
69.875	0.0996	-7.21914762756239\\
69.875	0.1002	-6.72898226593975\\
69.875	0.1008	-6.23881690431716\\
69.875	0.1014	-5.74865154269457\\
69.875	0.102	-5.25848618107204\\
69.875	0.1026	-4.76832081944946\\
69.875	0.1032	-4.27815545782687\\
69.875	0.1038	-3.78799009620428\\
69.875	0.1044	-3.29782473458164\\
69.875	0.105	-2.80765937295905\\
69.875	0.1056	-2.31749401133646\\
69.875	0.1062	-1.82732864971388\\
69.875	0.1068	-1.33716328809129\\
69.875	0.1074	-0.846997926468759\\
69.875	0.108	-0.356832564846172\\
69.875	0.1086	0.133332796776415\\
69.875	0.1092	0.623498158398945\\
69.875	0.1098	1.11366352002165\\
69.875	0.1104	1.60382888164423\\
69.875	0.111	2.09399424326682\\
69.875	0.1116	2.58415960488941\\
69.875	0.1122	3.07432496651194\\
69.875	0.1128	3.56449032813453\\
69.875	0.1134	4.05465568975711\\
69.875	0.114	4.5448210513797\\
69.875	0.1146	5.03498641300229\\
69.875	0.1152	5.52515177462493\\
69.875	0.1158	6.01531713624752\\
69.875	0.1164	6.5054824978701\\
69.875	0.117	6.99564785949264\\
69.875	0.1176	7.48581322111517\\
69.875	0.1182	7.97597858273787\\
69.875	0.1188	8.4661439443604\\
69.875	0.1194	8.95630930598293\\
69.875	0.12	9.44647466760557\\
69.875	0.1206	9.93664002922833\\
69.875	0.1212	10.4268053908507\\
69.875	0.1218	10.9169707524733\\
69.875	0.1224	11.407136114096\\
69.875	0.123	11.8973014757185\\
70.25	0.093	-12.9338577129893\\
70.25	0.0936	-12.4575806825136\\
70.25	0.0942	-11.9813036520379\\
70.25	0.0948	-11.5050266215621\\
70.25	0.0954	-11.0287495910864\\
70.25	0.096	-10.5524725606107\\
70.25	0.0966	-10.076195530135\\
70.25	0.0972	-9.59991849965922\\
70.25	0.0978	-9.1236414691835\\
70.25	0.0984	-8.64736443870777\\
70.25	0.099	-8.17108740823204\\
70.25	0.0996	-7.69481037775631\\
70.25	0.1002	-7.21853334728053\\
70.25	0.1008	-6.7422563168048\\
70.25	0.1014	-6.26597928632907\\
70.25	0.102	-5.7897022558534\\
70.25	0.1026	-5.31342522537767\\
70.25	0.1032	-4.83714819490194\\
70.25	0.1038	-4.36087116442621\\
70.25	0.1044	-3.88459413395043\\
70.25	0.105	-3.4083171034747\\
70.25	0.1056	-2.93204007299897\\
70.25	0.1062	-2.45576304252324\\
70.25	0.1068	-1.97948601204752\\
70.25	0.1074	-1.50320898157185\\
70.25	0.108	-1.02693195109612\\
70.25	0.1086	-0.550654920620389\\
70.25	0.1092	-0.0743778901447172\\
70.25	0.1098	0.401899140331125\\
70.25	0.1104	0.878176170806853\\
70.25	0.111	1.35445320128258\\
70.25	0.1116	1.83073023175831\\
70.25	0.1122	2.30700726223398\\
70.25	0.1128	2.78328429270971\\
70.25	0.1134	3.25956132318544\\
70.25	0.114	3.73583835366117\\
70.25	0.1146	4.21211538413689\\
70.25	0.1152	4.68839241461268\\
70.25	0.1158	5.16466944508841\\
70.25	0.1164	5.64094647556414\\
70.25	0.117	6.11722350603975\\
70.25	0.1176	6.59350053651559\\
70.25	0.1182	7.06977756699121\\
70.25	0.1188	7.54605459746699\\
70.25	0.1194	8.02233162794278\\
70.25	0.12	8.49860865841833\\
70.25	0.1206	8.97488568889423\\
70.25	0.1212	9.45116271937002\\
70.25	0.1218	9.92743974984569\\
70.25	0.1224	10.4037167803214\\
70.25	0.123	10.8799938107971\\
70.625	0.093	-13.2567488205679\\
70.625	0.0936	-12.794360121239\\
70.625	0.0942	-12.3319714219102\\
70.625	0.0948	-11.8695827225813\\
70.625	0.0954	-11.4071940232524\\
70.625	0.096	-10.9448053239236\\
70.625	0.0966	-10.4824166245947\\
70.625	0.0972	-10.0200279252658\\
70.625	0.0978	-9.55763922593695\\
70.625	0.0984	-9.09525052660808\\
70.625	0.099	-8.63286182727921\\
70.625	0.0996	-8.17047312795034\\
70.625	0.1002	-7.70808442862142\\
70.625	0.1008	-7.24569572929255\\
70.625	0.1014	-6.78330702996368\\
70.625	0.102	-6.32091833063487\\
70.625	0.1026	-5.858529631306\\
70.625	0.1032	-5.39614093197719\\
70.625	0.1038	-4.93375223264832\\
70.625	0.1044	-4.47136353331939\\
70.625	0.105	-4.00897483399052\\
70.625	0.1056	-3.54658613466165\\
70.625	0.1062	-3.08419743533278\\
70.625	0.1068	-2.62180873600391\\
70.625	0.1074	-2.1594200366751\\
70.625	0.108	-1.69703133734623\\
70.625	0.1086	-1.23464263801736\\
70.625	0.1092	-0.77225393868855\\
70.625	0.1098	-0.309865239359567\\
70.625	0.1104	0.152523459969302\\
70.625	0.111	0.614912159298171\\
70.625	0.1116	1.07730085862704\\
70.625	0.1122	1.53968955795585\\
70.625	0.1128	2.00207825728472\\
70.625	0.1134	2.46446695661359\\
70.625	0.114	2.92685565594246\\
70.625	0.1146	3.38924435527133\\
70.625	0.1152	3.85163305460026\\
70.625	0.1158	4.31402175392913\\
70.625	0.1164	4.77641045325799\\
70.625	0.117	5.23879915258681\\
70.625	0.1176	5.70118785191562\\
70.625	0.1182	6.16357655124455\\
70.625	0.1188	6.62596525057336\\
70.625	0.1194	7.08835394990217\\
70.625	0.12	7.5507426492311\\
70.625	0.1206	8.01313134856002\\
70.625	0.1212	8.47552004788895\\
70.625	0.1218	8.93790874721776\\
70.625	0.1224	9.40029744654657\\
70.625	0.123	9.8626861458755\\
71	0.093	-13.5796399281464\\
71	0.0936	-13.1311395599644\\
71	0.0942	-12.6826391917824\\
71	0.0948	-12.2341388236004\\
71	0.0954	-11.7856384554184\\
71	0.096	-11.3371380872364\\
71	0.0966	-10.8886377190544\\
71	0.0972	-10.4401373508724\\
71	0.0978	-9.99163698269035\\
71	0.0984	-9.5431366145084\\
71	0.099	-9.09463624632639\\
71	0.0996	-8.64613587814438\\
71	0.1002	-8.19763550996231\\
71	0.1008	-7.7491351417803\\
71	0.1014	-7.30063477359829\\
71	0.102	-6.85213440541634\\
71	0.1026	-6.40363403723433\\
71	0.1032	-5.95513366905232\\
71	0.1038	-5.50663330087031\\
71	0.1044	-5.05813293268824\\
71	0.105	-4.60963256450623\\
71	0.1056	-4.16113219632422\\
71	0.1062	-3.71263182814221\\
71	0.1068	-3.2641314599602\\
71	0.1074	-2.81563109177824\\
71	0.108	-2.36713072359629\\
71	0.1086	-1.91863035541428\\
71	0.1092	-1.47012998723233\\
71	0.1098	-1.0216296190502\\
71	0.1104	-0.573129250868192\\
71	0.111	-0.124628882686181\\
71	0.1116	0.323871485495829\\
71	0.1122	0.772371853677782\\
71	0.1128	1.22087222185979\\
71	0.1134	1.6693725900418\\
71	0.114	2.11787295822381\\
71	0.1146	2.56637332640582\\
71	0.1152	3.01487369458789\\
71	0.1158	3.4633740627699\\
71	0.1164	3.91187443095191\\
71	0.117	4.36037479913387\\
71	0.1176	4.80887516731582\\
71	0.1182	5.25737553549789\\
71	0.1188	5.70587590367984\\
71	0.1194	6.15437627186179\\
71	0.12	6.60287664004386\\
71	0.1206	7.05137700822593\\
71	0.1212	7.49987737640799\\
71	0.1218	7.94837774458995\\
71	0.1224	8.3968781127719\\
71	0.123	8.84537848095385\\
71.375	0.093	-13.9025310357248\\
71.375	0.0936	-13.4679189986897\\
71.375	0.0942	-13.0333069616546\\
71.375	0.0948	-12.5986949246195\\
71.375	0.0954	-12.1640828875843\\
71.375	0.096	-11.7294708505491\\
71.375	0.0966	-11.294858813514\\
71.375	0.0972	-10.8602467764788\\
71.375	0.0978	-10.4256347394437\\
71.375	0.0984	-9.99102270240854\\
71.375	0.099	-9.55641066537339\\
71.375	0.0996	-9.12179862833824\\
71.375	0.1002	-8.68718659130303\\
71.375	0.1008	-8.25257455426794\\
71.375	0.1014	-7.81796251723279\\
71.375	0.102	-7.38335048019769\\
71.375	0.1026	-6.94873844316254\\
71.375	0.1032	-6.51412640612739\\
71.375	0.1038	-6.07951436909224\\
71.375	0.1044	-5.64490233205703\\
71.375	0.105	-5.21029029502188\\
71.375	0.1056	-4.77567825798673\\
71.375	0.1062	-4.34106622095157\\
71.375	0.1068	-3.90645418391642\\
71.375	0.1074	-3.47184214688133\\
71.375	0.108	-3.03723010984623\\
71.375	0.1086	-2.60261807281108\\
71.375	0.1092	-2.16800603577599\\
71.375	0.1098	-1.73339399874072\\
71.375	0.1104	-1.29878196170557\\
71.375	0.111	-0.864169924670421\\
71.375	0.1116	-0.429557887635269\\
71.375	0.1122	0.00505414939982529\\
71.375	0.1128	0.439666186434977\\
71.375	0.1134	0.874278223470128\\
71.375	0.114	1.30889026050528\\
71.375	0.1146	1.74350229754037\\
71.375	0.1152	2.17811433457558\\
71.375	0.1158	2.61272637161073\\
71.375	0.1164	3.04733840864594\\
71.375	0.117	3.48195044568104\\
71.375	0.1176	3.91656248271613\\
71.375	0.1182	4.35117451975123\\
71.375	0.1188	4.78578655678643\\
71.375	0.1194	5.22039859382153\\
71.375	0.12	5.65501063085662\\
71.375	0.1206	6.08962266789194\\
71.375	0.1212	6.52423470492704\\
71.375	0.1218	6.95884674196213\\
71.375	0.1224	7.39345877899734\\
71.375	0.123	7.82807081603244\\
71.75	0.093	-14.2254221433034\\
71.75	0.0936	-13.8046984374151\\
71.75	0.0942	-13.3839747315268\\
71.75	0.0948	-12.9632510256386\\
71.75	0.0954	-12.5425273197503\\
71.75	0.096	-12.121803613862\\
71.75	0.0966	-11.7010799079737\\
71.75	0.0972	-11.2803562020854\\
71.75	0.0978	-10.8596324961971\\
71.75	0.0984	-10.4389087903088\\
71.75	0.099	-10.0181850844206\\
71.75	0.0996	-9.59746137853227\\
71.75	0.1002	-9.17673767264392\\
71.75	0.1008	-8.75601396675563\\
71.75	0.1014	-8.33529026086734\\
71.75	0.102	-7.9145665549791\\
71.75	0.1026	-7.49384284909081\\
71.75	0.1032	-7.07311914320252\\
71.75	0.1038	-6.65239543731428\\
71.75	0.1044	-6.23167173142593\\
71.75	0.105	-5.81094802553764\\
71.75	0.1056	-5.39022431964935\\
71.75	0.1062	-4.96950061376106\\
71.75	0.1068	-4.54877690787276\\
71.75	0.1074	-4.12805320198453\\
71.75	0.108	-3.70732949609624\\
71.75	0.1086	-3.28660579020794\\
71.75	0.1092	-2.86588208431971\\
71.75	0.1098	-2.44515837843136\\
71.75	0.1104	-2.02443467254307\\
71.75	0.111	-1.60371096665477\\
71.75	0.1116	-1.18298726076648\\
71.75	0.1122	-0.762263554878245\\
71.75	0.1128	-0.341539848989953\\
71.75	0.1134	0.0791838568983394\\
71.75	0.114	0.499907562786575\\
71.75	0.1146	0.920631268674867\\
71.75	0.1152	1.34135497456322\\
71.75	0.1158	1.76207868045151\\
71.75	0.1164	2.18280238633986\\
71.75	0.117	2.60352609222798\\
71.75	0.1176	3.02424979811633\\
71.75	0.1182	3.44497350400457\\
71.75	0.1188	3.86569720989291\\
71.75	0.1194	4.28642091578115\\
71.75	0.12	4.70714462166939\\
71.75	0.1206	5.12786832755785\\
71.75	0.1212	5.54859203344608\\
71.75	0.1218	5.96931573933432\\
71.75	0.1224	6.39003944522256\\
71.75	0.123	6.8107631511109\\
72.125	0.093	-14.548313250882\\
72.125	0.0936	-14.1414778761405\\
72.125	0.0942	-13.7346425013992\\
72.125	0.0948	-13.3278071266577\\
72.125	0.0954	-12.9209717519163\\
72.125	0.096	-12.5141363771749\\
72.125	0.0966	-12.1073010024334\\
72.125	0.0972	-11.700465627692\\
72.125	0.0978	-11.2936302529506\\
72.125	0.0984	-10.8867948782092\\
72.125	0.099	-10.4799595034677\\
72.125	0.0996	-10.0731241287263\\
72.125	0.1002	-9.66628875398482\\
72.125	0.1008	-9.25945337924338\\
72.125	0.1014	-8.85261800450195\\
72.125	0.102	-8.44578262976063\\
72.125	0.1026	-8.0389472550192\\
72.125	0.1032	-7.63211188027776\\
72.125	0.1038	-7.22527650553633\\
72.125	0.1044	-6.81844113079484\\
72.125	0.105	-6.4116057560534\\
72.125	0.1056	-6.00477038131197\\
72.125	0.1062	-5.59793500657059\\
72.125	0.1068	-5.19109963182916\\
72.125	0.1074	-4.78426425708778\\
72.125	0.108	-4.37742888234635\\
72.125	0.1086	-3.97059350760492\\
72.125	0.1092	-3.56375813286354\\
72.125	0.1098	-3.15692275812199\\
72.125	0.1104	-2.75008738338062\\
72.125	0.111	-2.34325200863918\\
72.125	0.1116	-1.93641663389775\\
72.125	0.1122	-1.52958125915637\\
72.125	0.1128	-1.12274588441494\\
72.125	0.1134	-0.715910509673506\\
72.125	0.114	-0.309075134932129\\
72.125	0.1146	0.097760239809304\\
72.125	0.1152	0.504595614550794\\
72.125	0.1158	0.911430989292285\\
72.125	0.1164	1.31826636403355\\
72.125	0.117	1.72510173877492\\
72.125	0.1176	2.13193711351653\\
72.125	0.1182	2.53877248825779\\
72.125	0.1188	2.9456078629994\\
72.125	0.1194	3.35244323774077\\
72.125	0.12	3.75927861248203\\
72.125	0.1206	4.16611398722353\\
72.125	0.1212	4.57294936196513\\
72.125	0.1218	4.97978473670651\\
72.125	0.1224	5.38662011144777\\
72.125	0.123	5.79345548618937\\
72.5	0.093	-14.8712043584604\\
72.5	0.0936	-14.4782573148659\\
72.5	0.0942	-14.0853102712713\\
72.5	0.0948	-13.6923632276767\\
72.5	0.0954	-13.2994161840821\\
72.5	0.096	-12.9064691404876\\
72.5	0.0966	-12.513522096893\\
72.5	0.0972	-12.1205750532985\\
72.5	0.0978	-11.7276280097039\\
72.5	0.0984	-11.3346809661093\\
72.5	0.099	-10.9417339225148\\
72.5	0.0996	-10.5487868789202\\
72.5	0.1002	-10.1558398353256\\
72.5	0.1008	-9.76289279173102\\
72.5	0.1014	-9.36994574813644\\
72.5	0.102	-8.97699870454193\\
72.5	0.1026	-8.58405166094741\\
72.5	0.1032	-8.19110461735283\\
72.5	0.1038	-7.79815757375826\\
72.5	0.1044	-7.40521053016363\\
72.5	0.105	-7.01226348656905\\
72.5	0.1056	-6.61931644297448\\
72.5	0.1062	-6.22636939937996\\
72.5	0.1068	-5.83342235578539\\
72.5	0.1074	-5.44047531219087\\
72.5	0.108	-5.0475282685963\\
72.5	0.1086	-4.65458122500172\\
72.5	0.1092	-4.2616341814072\\
72.5	0.1098	-3.86868713781257\\
72.5	0.1104	-3.475740094218\\
72.5	0.111	-3.08279305062342\\
72.5	0.1116	-2.68984600702885\\
72.5	0.1122	-2.29689896343433\\
72.5	0.1128	-1.90395191983976\\
72.5	0.1134	-1.51100487624524\\
72.5	0.114	-1.11805783265066\\
72.5	0.1146	-0.725110789056203\\
72.5	0.1152	-0.332163745461457\\
72.5	0.1158	0.0607832981330603\\
72.5	0.1164	0.453730341727692\\
72.5	0.117	0.846677385322209\\
72.5	0.1176	1.23962442891673\\
72.5	0.1182	1.63257147251136\\
72.5	0.1188	2.02551851610576\\
72.5	0.1194	2.41846555970028\\
72.5	0.12	2.81141260329503\\
72.5	0.1206	3.20435964688966\\
72.5	0.1212	3.59730669048417\\
72.5	0.1218	3.99025373407858\\
72.5	0.1224	4.38320077767332\\
72.5	0.123	4.77614782126773\\
72.875	0.093	-15.194095466039\\
72.875	0.0936	-14.8150367535912\\
72.875	0.0942	-14.4359780411435\\
72.875	0.0948	-14.0569193286958\\
72.875	0.0954	-13.6778606162482\\
72.875	0.096	-13.2988019038004\\
72.875	0.0966	-12.9197431913527\\
72.875	0.0972	-12.540684478905\\
72.875	0.0978	-12.1616257664573\\
72.875	0.0984	-11.7825670540096\\
72.875	0.099	-11.4035083415619\\
72.875	0.0996	-11.0244496291142\\
72.875	0.1002	-10.6453909166664\\
72.875	0.1008	-10.2663322042187\\
72.875	0.1014	-9.88727349177105\\
72.875	0.102	-9.5082147793234\\
72.875	0.1026	-9.12915606687568\\
72.875	0.1032	-8.75009735442796\\
72.875	0.1038	-8.37103864198025\\
72.875	0.1044	-7.99197992953253\\
72.875	0.105	-7.61292121708482\\
72.875	0.1056	-7.2338625046371\\
72.875	0.1062	-6.85480379218939\\
72.875	0.1068	-6.47574507974167\\
72.875	0.1074	-6.09668636729407\\
72.875	0.108	-5.71762765484635\\
72.875	0.1086	-5.33856894239864\\
72.875	0.1092	-4.95951022995098\\
72.875	0.1098	-4.58045151750315\\
72.875	0.1104	-4.20139280505549\\
72.875	0.111	-3.82233409260778\\
72.875	0.1116	-3.44327538016006\\
72.875	0.1122	-3.0642166677124\\
72.875	0.1128	-2.68515795526469\\
72.875	0.1134	-2.30609924281703\\
72.875	0.114	-1.92704053036931\\
72.875	0.1146	-1.5479818179216\\
72.875	0.1152	-1.16892310547394\\
72.875	0.1158	-0.78986439302605\\
72.875	0.1164	-0.410805680578505\\
72.875	0.117	-0.0317469681308467\\
72.875	0.1176	0.347311744316926\\
72.875	0.1182	0.726370456764585\\
72.875	0.1188	1.10542916921236\\
72.875	0.1194	1.48448788166002\\
72.875	0.12	1.86354659410767\\
72.875	0.1206	2.24260530655545\\
72.875	0.1212	2.62166401900322\\
72.875	0.1218	3.00072273145088\\
72.875	0.1224	3.37978144389854\\
72.875	0.123	3.75884015634631\\
73.25	0.093	-15.5169865736175\\
73.25	0.0936	-15.1518161923166\\
73.25	0.0942	-14.7866458110158\\
73.25	0.0948	-14.4214754297149\\
73.25	0.0954	-14.0563050484141\\
73.25	0.096	-13.6911346671133\\
73.25	0.0966	-13.3259642858124\\
73.25	0.0972	-12.9607939045115\\
73.25	0.0978	-12.5956235232107\\
73.25	0.0984	-12.2304531419099\\
73.25	0.099	-11.865282760609\\
73.25	0.0996	-11.5001123793082\\
73.25	0.1002	-11.1349419980073\\
73.25	0.1008	-10.7697716167065\\
73.25	0.1014	-10.4046012354056\\
73.25	0.102	-10.0394308541048\\
73.25	0.1026	-9.67426047280395\\
73.25	0.1032	-9.30909009150315\\
73.25	0.1038	-8.94391971020229\\
73.25	0.1044	-8.57874932890138\\
73.25	0.105	-8.21357894760052\\
73.25	0.1056	-7.84840856629972\\
73.25	0.1062	-7.48323818499887\\
73.25	0.1068	-7.11806780369801\\
73.25	0.1074	-6.75289742239721\\
73.25	0.108	-6.38772704109641\\
73.25	0.1086	-6.02255665979555\\
73.25	0.1092	-5.65738627849476\\
73.25	0.1098	-5.29221589719378\\
73.25	0.1104	-4.92704551589298\\
73.25	0.111	-4.56187513459213\\
73.25	0.1116	-4.19670475329127\\
73.25	0.1122	-3.83153437199047\\
73.25	0.1128	-3.46636399068962\\
73.25	0.1134	-3.10119360938876\\
73.25	0.114	-2.7360232280879\\
73.25	0.1146	-2.37085284678722\\
73.25	0.1152	-2.00568246548619\\
73.25	0.1158	-1.64051208418539\\
73.25	0.1164	-1.27534170288448\\
73.25	0.117	-0.910171321583675\\
73.25	0.1176	-0.545000940282989\\
73.25	0.1182	-0.179830558981962\\
73.25	0.1188	0.185339822318724\\
73.25	0.1194	0.550510203619524\\
73.25	0.12	0.915680584920437\\
73.25	0.1206	1.28085096622146\\
73.25	0.1212	1.64602134752215\\
73.25	0.1218	2.01119172882295\\
73.25	0.1224	2.37636211012386\\
73.25	0.123	2.74153249142466\\
73.625	0.093	-15.8398776811959\\
73.625	0.0936	-15.4885956310419\\
73.625	0.0942	-15.137313580888\\
73.625	0.0948	-14.786031530734\\
73.625	0.0954	-14.43474948058\\
73.625	0.096	-14.083467430426\\
73.625	0.0966	-13.732185380272\\
73.625	0.0972	-13.380903330118\\
73.625	0.0978	-13.029621279964\\
73.625	0.0984	-12.6783392298101\\
73.625	0.099	-12.3270571796561\\
73.625	0.0996	-11.9757751295021\\
73.625	0.1002	-11.624493079348\\
73.625	0.1008	-11.2732110291941\\
73.625	0.1014	-10.9219289790401\\
73.625	0.102	-10.5706469288862\\
73.625	0.1026	-10.2193648787322\\
73.625	0.1032	-9.86808282857822\\
73.625	0.1038	-9.51680077842423\\
73.625	0.1044	-9.16551872827017\\
73.625	0.105	-8.81423667811617\\
73.625	0.1056	-8.46295462796223\\
73.625	0.1062	-8.11167257780824\\
73.625	0.1068	-7.76039052765424\\
73.625	0.1074	-7.4091084775003\\
73.625	0.108	-7.05782642734636\\
73.625	0.1086	-6.70654437719236\\
73.625	0.1092	-6.35526232703842\\
73.625	0.1098	-6.00398027688431\\
73.625	0.1104	-5.65269822673037\\
73.625	0.111	-5.30141617657637\\
73.625	0.1116	-4.95013412642237\\
73.625	0.1122	-4.59885207626843\\
73.625	0.1128	-4.24757002611449\\
73.625	0.1134	-3.89628797596043\\
73.625	0.114	-3.54500592580655\\
73.625	0.1146	-3.19372387565249\\
73.625	0.1152	-2.84244182549855\\
73.625	0.1158	-2.4911597753445\\
73.625	0.1164	-2.13987772519056\\
73.625	0.117	-1.78859567503662\\
73.625	0.1176	-1.43731362488256\\
73.625	0.1182	-1.08603157472874\\
73.625	0.1188	-0.734749524574568\\
73.625	0.1194	-0.383467474420627\\
73.625	0.12	-0.0321854242667996\\
73.625	0.1206	0.319096625887255\\
73.625	0.1212	0.670378676041423\\
73.625	0.1218	1.02166072619536\\
73.625	0.1224	1.37294277634919\\
73.625	0.123	1.72422482650325\\
74	0.093	-16.1627687887745\\
74	0.0936	-15.8253750697673\\
74	0.0942	-15.4879813507603\\
74	0.0948	-15.1505876317531\\
74	0.0954	-14.813193912746\\
74	0.096	-14.4758001937389\\
74	0.0966	-14.1384064747318\\
74	0.0972	-13.8010127557246\\
74	0.0978	-13.4636190367175\\
74	0.0984	-13.1262253177104\\
74	0.099	-12.7888315987033\\
74	0.0996	-12.4514378796961\\
74	0.1002	-12.114044160689\\
74	0.1008	-11.7766504416819\\
74	0.1014	-11.4392567226747\\
74	0.102	-11.1018630036676\\
74	0.1026	-10.7644692846605\\
74	0.1032	-10.4270755656534\\
74	0.1038	-10.0896818466463\\
74	0.1044	-9.75228812763913\\
74	0.105	-9.41489440863199\\
74	0.1056	-9.07750068962486\\
74	0.1062	-8.74010697061777\\
74	0.1068	-8.40271325161063\\
74	0.1074	-8.06531953260355\\
74	0.108	-7.72792581359641\\
74	0.1086	-7.39053209458933\\
74	0.1092	-7.05313837558225\\
74	0.1098	-6.715744656575\\
74	0.1104	-6.37835093756792\\
74	0.111	-6.04095721856078\\
74	0.1116	-5.70356349955364\\
74	0.1122	-5.36616978054656\\
74	0.1128	-5.02877606153947\\
74	0.1134	-4.69138234253239\\
74	0.114	-4.3539886235252\\
74	0.1146	-4.01659490451812\\
74	0.1152	-3.67920118551092\\
74	0.1158	-3.34180746650384\\
74	0.1164	-3.00441374749664\\
74	0.117	-2.66702002848956\\
74	0.1176	-2.32962630948248\\
74	0.1182	-1.9922325904754\\
74	0.1188	-1.6548388714682\\
74	0.1194	-1.31744515246112\\
74	0.12	-0.980051433454037\\
74	0.1206	-0.642657714446841\\
74	0.1212	-0.305263995439645\\
74	0.1218	0.0321297235673228\\
74	0.1224	0.369523442574518\\
74	0.123	0.7069171615816\\
};
\end{axis}

\begin{axis}[%
width=5.011742cm,
height=3.595207cm,
at={(6.594397cm,5.468264cm)},
scale only axis,
xmin=55,
xmax=75,
tick align=outside,
xlabel={$L_{cut}$},
xmajorgrids,
ymin=0.09,
ymax=0.13,
ylabel={$D_{rlx}$},
ymajorgrids,
zmin=-1000,
zmax=89.8946346873299,
zlabel={$x_4$},
zmajorgrids,
view={-140}{50},
legend style={at={(1.03,1)},anchor=north west,legend cell align=left,align=left,draw=white!15!black}
]
\addplot3[only marks,mark=*,mark options={},mark size=1.5000pt,color=mycolor1] plot table[row sep=crcr,]{%
74	0.123	-171.244880190644\\
72	0.113	-139.217460396701\\
61	0.095	-69.6051966848396\\
56	0.093	-73.6870961160989\\
};
\addplot3[only marks,mark=*,mark options={},mark size=1.5000pt,color=black] plot table[row sep=crcr,]{%
69	0.104	-101.955479221514\\
};

\addplot3[%
surf,
opacity=0.7,
shader=interp,
colormap={mymap}{[1pt] rgb(0pt)=(0.0901961,0.239216,0.0745098); rgb(1pt)=(0.0945149,0.242058,0.0739522); rgb(2pt)=(0.0988592,0.244894,0.0733566); rgb(3pt)=(0.103229,0.247724,0.0727241); rgb(4pt)=(0.107623,0.250549,0.0720557); rgb(5pt)=(0.112043,0.253367,0.0713525); rgb(6pt)=(0.116487,0.25618,0.0706154); rgb(7pt)=(0.120956,0.258986,0.0698456); rgb(8pt)=(0.125449,0.261787,0.0690441); rgb(9pt)=(0.129967,0.264581,0.0682118); rgb(10pt)=(0.134508,0.26737,0.06735); rgb(11pt)=(0.139074,0.270152,0.0664596); rgb(12pt)=(0.143663,0.272929,0.0655416); rgb(13pt)=(0.148275,0.275699,0.0645971); rgb(14pt)=(0.152911,0.278463,0.0636271); rgb(15pt)=(0.15757,0.281221,0.0626328); rgb(16pt)=(0.162252,0.283973,0.0616151); rgb(17pt)=(0.166957,0.286719,0.060575); rgb(18pt)=(0.171685,0.289458,0.0595136); rgb(19pt)=(0.176434,0.292191,0.0584321); rgb(20pt)=(0.181207,0.294918,0.0573313); rgb(21pt)=(0.186001,0.297639,0.0562123); rgb(22pt)=(0.190817,0.300353,0.0550763); rgb(23pt)=(0.195655,0.303061,0.0539242); rgb(24pt)=(0.200514,0.305763,0.052757); rgb(25pt)=(0.205395,0.308459,0.0515759); rgb(26pt)=(0.210296,0.311149,0.0503624); rgb(27pt)=(0.215212,0.313846,0.0490067); rgb(28pt)=(0.220142,0.316548,0.0475043); rgb(29pt)=(0.22509,0.319254,0.0458704); rgb(30pt)=(0.230056,0.321962,0.0441205); rgb(31pt)=(0.235042,0.324671,0.04227); rgb(32pt)=(0.240048,0.327379,0.0403343); rgb(33pt)=(0.245078,0.330085,0.0383287); rgb(34pt)=(0.250131,0.332786,0.0362688); rgb(35pt)=(0.25521,0.335482,0.0341698); rgb(36pt)=(0.260317,0.33817,0.0320472); rgb(37pt)=(0.265451,0.340849,0.0299163); rgb(38pt)=(0.270616,0.343517,0.0277927); rgb(39pt)=(0.275813,0.346172,0.0256916); rgb(40pt)=(0.281043,0.348814,0.0236284); rgb(41pt)=(0.286307,0.35144,0.0216186); rgb(42pt)=(0.291607,0.354048,0.0196776); rgb(43pt)=(0.296945,0.356637,0.0178207); rgb(44pt)=(0.302322,0.359206,0.0160634); rgb(45pt)=(0.307739,0.361753,0.0144211); rgb(46pt)=(0.313198,0.364275,0.0129091); rgb(47pt)=(0.318701,0.366772,0.0115428); rgb(48pt)=(0.324249,0.369242,0.0103377); rgb(49pt)=(0.329843,0.371682,0.00930909); rgb(50pt)=(0.335485,0.374093,0.00847245); rgb(51pt)=(0.341176,0.376471,0.00784314); rgb(52pt)=(0.346925,0.378826,0.00732741); rgb(53pt)=(0.352735,0.381168,0.00682184); rgb(54pt)=(0.358605,0.383497,0.00632729); rgb(55pt)=(0.364532,0.385812,0.00584464); rgb(56pt)=(0.370516,0.388113,0.00537476); rgb(57pt)=(0.376552,0.390399,0.00491852); rgb(58pt)=(0.38264,0.39267,0.00447681); rgb(59pt)=(0.388777,0.394925,0.00405048); rgb(60pt)=(0.394962,0.397164,0.00364042); rgb(61pt)=(0.401191,0.399386,0.00324749); rgb(62pt)=(0.407464,0.401592,0.00287258); rgb(63pt)=(0.413777,0.40378,0.00251655); rgb(64pt)=(0.420129,0.40595,0.00218028); rgb(65pt)=(0.426518,0.408102,0.00186463); rgb(66pt)=(0.432942,0.410234,0.00157049); rgb(67pt)=(0.439399,0.412348,0.00129873); rgb(68pt)=(0.445885,0.414441,0.00105022); rgb(69pt)=(0.452401,0.416515,0.000825833); rgb(70pt)=(0.458942,0.418567,0.000626441); rgb(71pt)=(0.465508,0.420599,0.00045292); rgb(72pt)=(0.472096,0.422609,0.000306141); rgb(73pt)=(0.478704,0.424596,0.000186979); rgb(74pt)=(0.485331,0.426562,9.63073e-05); rgb(75pt)=(0.491973,0.428504,3.49981e-05); rgb(76pt)=(0.498628,0.430422,3.92506e-06); rgb(77pt)=(0.505323,0.432315,0); rgb(78pt)=(0.512206,0.434168,0); rgb(79pt)=(0.519282,0.435983,0); rgb(80pt)=(0.526529,0.437764,0); rgb(81pt)=(0.533922,0.439512,0); rgb(82pt)=(0.54144,0.441232,0); rgb(83pt)=(0.549059,0.442927,0); rgb(84pt)=(0.556756,0.444599,0); rgb(85pt)=(0.564508,0.446252,0); rgb(86pt)=(0.572292,0.447889,0); rgb(87pt)=(0.580084,0.449514,0); rgb(88pt)=(0.587863,0.451129,0); rgb(89pt)=(0.595604,0.452737,0); rgb(90pt)=(0.603284,0.454343,0); rgb(91pt)=(0.610882,0.455948,0); rgb(92pt)=(0.618373,0.457556,0); rgb(93pt)=(0.625734,0.459171,0); rgb(94pt)=(0.632943,0.460795,0); rgb(95pt)=(0.639976,0.462432,0); rgb(96pt)=(0.64681,0.464084,0); rgb(97pt)=(0.653423,0.465756,0); rgb(98pt)=(0.659791,0.46745,0); rgb(99pt)=(0.665891,0.469169,0); rgb(100pt)=(0.6717,0.470916,0); rgb(101pt)=(0.677195,0.472696,0); rgb(102pt)=(0.682353,0.47451,0); rgb(103pt)=(0.687242,0.476355,0); rgb(104pt)=(0.691952,0.478225,0); rgb(105pt)=(0.696497,0.480118,0); rgb(106pt)=(0.700887,0.482033,0); rgb(107pt)=(0.705134,0.483968,0); rgb(108pt)=(0.709251,0.485921,0); rgb(109pt)=(0.713249,0.487891,0); rgb(110pt)=(0.71714,0.489876,0); rgb(111pt)=(0.720936,0.491875,0); rgb(112pt)=(0.724649,0.493887,0); rgb(113pt)=(0.72829,0.495909,0); rgb(114pt)=(0.731872,0.49794,0); rgb(115pt)=(0.735406,0.499979,0); rgb(116pt)=(0.738904,0.502025,0); rgb(117pt)=(0.742378,0.504075,0); rgb(118pt)=(0.74584,0.506128,0); rgb(119pt)=(0.749302,0.508182,0); rgb(120pt)=(0.752775,0.510237,0); rgb(121pt)=(0.756272,0.51229,0); rgb(122pt)=(0.759804,0.514339,0); rgb(123pt)=(0.763384,0.516385,0); rgb(124pt)=(0.767022,0.518424,0); rgb(125pt)=(0.770731,0.520455,0); rgb(126pt)=(0.774523,0.522478,0); rgb(127pt)=(0.77841,0.524489,0); rgb(128pt)=(0.782391,0.526491,0); rgb(129pt)=(0.786402,0.528496,0); rgb(130pt)=(0.790431,0.530506,0); rgb(131pt)=(0.794478,0.532521,0); rgb(132pt)=(0.798541,0.534539,0); rgb(133pt)=(0.802619,0.53656,0); rgb(134pt)=(0.806712,0.538584,0); rgb(135pt)=(0.81082,0.540609,0); rgb(136pt)=(0.81494,0.542635,0); rgb(137pt)=(0.819074,0.54466,0); rgb(138pt)=(0.823219,0.546686,0); rgb(139pt)=(0.827374,0.548709,0); rgb(140pt)=(0.831541,0.55073,0); rgb(141pt)=(0.835716,0.552749,0); rgb(142pt)=(0.8399,0.554763,0); rgb(143pt)=(0.844092,0.556774,0); rgb(144pt)=(0.848292,0.558779,0); rgb(145pt)=(0.852497,0.560778,0); rgb(146pt)=(0.856708,0.562771,0); rgb(147pt)=(0.860924,0.564756,0); rgb(148pt)=(0.865143,0.566733,0); rgb(149pt)=(0.869366,0.568701,0); rgb(150pt)=(0.873592,0.57066,0); rgb(151pt)=(0.877819,0.572608,0); rgb(152pt)=(0.882047,0.574545,0); rgb(153pt)=(0.886275,0.576471,0); rgb(154pt)=(0.890659,0.578362,0); rgb(155pt)=(0.895333,0.580203,0); rgb(156pt)=(0.900258,0.581999,0); rgb(157pt)=(0.905397,0.583755,0); rgb(158pt)=(0.910711,0.585479,0); rgb(159pt)=(0.916164,0.587176,0); rgb(160pt)=(0.921717,0.588852,0); rgb(161pt)=(0.927333,0.590513,0); rgb(162pt)=(0.932974,0.592166,0); rgb(163pt)=(0.938602,0.593815,0); rgb(164pt)=(0.94418,0.595468,0); rgb(165pt)=(0.949669,0.59713,0); rgb(166pt)=(0.955033,0.598808,0); rgb(167pt)=(0.960233,0.600507,0); rgb(168pt)=(0.965232,0.602233,0); rgb(169pt)=(0.969992,0.603992,0); rgb(170pt)=(0.974475,0.605791,0); rgb(171pt)=(0.978643,0.607636,0); rgb(172pt)=(0.98246,0.609532,0); rgb(173pt)=(0.985886,0.611486,0); rgb(174pt)=(0.988885,0.613503,0); rgb(175pt)=(0.991419,0.61559,0); rgb(176pt)=(0.99345,0.617753,0); rgb(177pt)=(0.99494,0.619997,0); rgb(178pt)=(0.995851,0.622329,0); rgb(179pt)=(0.996226,0.624763,0); rgb(180pt)=(0.996512,0.627352,0); rgb(181pt)=(0.996788,0.630095,0); rgb(182pt)=(0.997053,0.632982,0); rgb(183pt)=(0.997308,0.636004,0); rgb(184pt)=(0.997552,0.639152,0); rgb(185pt)=(0.997785,0.642416,0); rgb(186pt)=(0.998006,0.645786,0); rgb(187pt)=(0.998217,0.649253,0); rgb(188pt)=(0.998416,0.652807,0); rgb(189pt)=(0.998605,0.656439,0); rgb(190pt)=(0.998781,0.660138,0); rgb(191pt)=(0.998946,0.663897,0); rgb(192pt)=(0.9991,0.667704,0); rgb(193pt)=(0.999242,0.67155,0); rgb(194pt)=(0.999372,0.675427,0); rgb(195pt)=(0.99949,0.679323,0); rgb(196pt)=(0.999596,0.68323,0); rgb(197pt)=(0.99969,0.687139,0); rgb(198pt)=(0.999771,0.691039,0); rgb(199pt)=(0.999841,0.694921,0); rgb(200pt)=(0.999898,0.698775,0); rgb(201pt)=(0.999942,0.702592,0); rgb(202pt)=(0.999974,0.706363,0); rgb(203pt)=(0.999994,0.710077,0); rgb(204pt)=(1,0.713725,0); rgb(205pt)=(1,0.717341,0); rgb(206pt)=(1,0.720963,0); rgb(207pt)=(1,0.724591,0); rgb(208pt)=(1,0.728226,0); rgb(209pt)=(1,0.731867,0); rgb(210pt)=(1,0.735514,0); rgb(211pt)=(1,0.739167,0); rgb(212pt)=(1,0.742827,0); rgb(213pt)=(1,0.746493,0); rgb(214pt)=(1,0.750165,0); rgb(215pt)=(1,0.753843,0); rgb(216pt)=(1,0.757527,0); rgb(217pt)=(1,0.761217,0); rgb(218pt)=(1,0.764913,0); rgb(219pt)=(1,0.768615,0); rgb(220pt)=(1,0.772324,0); rgb(221pt)=(1,0.776038,0); rgb(222pt)=(1,0.779758,0); rgb(223pt)=(1,0.783484,0); rgb(224pt)=(1,0.787215,0); rgb(225pt)=(1,0.790953,0); rgb(226pt)=(1,0.794696,0); rgb(227pt)=(1,0.798445,0); rgb(228pt)=(1,0.8022,0); rgb(229pt)=(1,0.805961,0); rgb(230pt)=(1,0.809727,0); rgb(231pt)=(1,0.8135,0); rgb(232pt)=(1,0.817278,0); rgb(233pt)=(1,0.821063,0); rgb(234pt)=(1,0.824854,0); rgb(235pt)=(1,0.828652,0); rgb(236pt)=(1,0.832455,0); rgb(237pt)=(1,0.836265,0); rgb(238pt)=(1,0.840081,0); rgb(239pt)=(1,0.843903,0); rgb(240pt)=(1,0.847732,0); rgb(241pt)=(1,0.851566,0); rgb(242pt)=(1,0.855406,0); rgb(243pt)=(1,0.859253,0); rgb(244pt)=(1,0.863106,0); rgb(245pt)=(1,0.866964,0); rgb(246pt)=(1,0.870829,0); rgb(247pt)=(1,0.8747,0); rgb(248pt)=(1,0.878577,0); rgb(249pt)=(1,0.88246,0); rgb(250pt)=(1,0.886349,0); rgb(251pt)=(1,0.890243,0); rgb(252pt)=(1,0.894144,0); rgb(253pt)=(1,0.898051,0); rgb(254pt)=(1,0.901964,0); rgb(255pt)=(1,0.905882,0)},
mesh/rows=49]
table[row sep=crcr,header=false] {%
%
56	0.093	-73.6870961158693\\
56	0.0936	-88.8575874843045\\
56	0.0942	-104.028078852742\\
56	0.0948	-119.198570221177\\
56	0.0954	-134.369061589614\\
56	0.096	-149.53955295805\\
56	0.0966	-164.710044326485\\
56	0.0972	-179.880535694921\\
56	0.0978	-195.051027063358\\
56	0.0984	-210.221518431795\\
56	0.099	-225.39200980023\\
56	0.0996	-240.562501168666\\
56	0.1002	-255.732992537102\\
56	0.1008	-270.903483905538\\
56	0.1014	-286.073975273974\\
56	0.102	-301.244466642411\\
56	0.1026	-316.414958010846\\
56	0.1032	-331.585449379283\\
56	0.1038	-346.755940747718\\
56	0.1044	-361.926432116155\\
56	0.105	-377.096923484591\\
56	0.1056	-392.267414853027\\
56	0.1062	-407.437906221463\\
56	0.1068	-422.608397589899\\
56	0.1074	-437.778888958334\\
56	0.108	-452.949380326771\\
56	0.1086	-468.119871695208\\
56	0.1092	-483.290363063643\\
56	0.1098	-498.46085443208\\
56	0.1104	-513.631345800516\\
56	0.111	-528.801837168951\\
56	0.1116	-543.972328537388\\
56	0.1122	-559.142819905823\\
56	0.1128	-574.31331127426\\
56	0.1134	-589.483802642696\\
56	0.114	-604.654294011133\\
56	0.1146	-619.824785379567\\
56	0.1152	-634.995276748004\\
56	0.1158	-650.16576811644\\
56	0.1164	-665.336259484876\\
56	0.117	-680.506750853311\\
56	0.1176	-695.677242221748\\
56	0.1182	-710.847733590184\\
56	0.1188	-726.018224958621\\
56	0.1194	-741.188716327057\\
56	0.12	-756.359207695493\\
56	0.1206	-771.52969906393\\
56	0.1212	-786.700190432364\\
56	0.1218	-801.870681800801\\
56	0.1224	-817.041173169237\\
56	0.123	-832.211664537673\\
56.375	0.093	-70.2791433908023\\
56.375	0.0936	-85.2423909869267\\
56.375	0.0942	-100.205638583052\\
56.375	0.0948	-115.168886179176\\
56.375	0.0954	-130.132133775302\\
56.375	0.096	-145.095381371427\\
56.375	0.0966	-160.058628967551\\
56.375	0.0972	-175.021876563676\\
56.375	0.0978	-189.985124159801\\
56.375	0.0984	-204.948371755926\\
56.375	0.099	-219.91161935205\\
56.375	0.0996	-234.874866948176\\
56.375	0.1002	-249.8381145443\\
56.375	0.1008	-264.801362140425\\
56.375	0.1014	-279.76460973655\\
56.375	0.102	-294.727857332675\\
56.375	0.1026	-309.691104928799\\
56.375	0.1032	-324.654352524924\\
56.375	0.1038	-339.617600121049\\
56.375	0.1044	-354.580847717174\\
56.375	0.105	-369.544095313298\\
56.375	0.1056	-384.507342909424\\
56.375	0.1062	-399.470590505548\\
56.375	0.1068	-414.433838101673\\
56.375	0.1074	-429.397085697798\\
56.375	0.108	-444.360333293923\\
56.375	0.1086	-459.323580890048\\
56.375	0.1092	-474.286828486172\\
56.375	0.1098	-489.250076082298\\
56.375	0.1104	-504.213323678423\\
56.375	0.111	-519.176571274546\\
56.375	0.1116	-534.139818870672\\
56.375	0.1122	-549.103066466796\\
56.375	0.1128	-564.066314062921\\
56.375	0.1134	-579.029561659047\\
56.375	0.114	-593.992809255172\\
56.375	0.1146	-608.956056851295\\
56.375	0.1152	-623.919304447421\\
56.375	0.1158	-638.882552043545\\
56.375	0.1164	-653.84579963967\\
56.375	0.117	-668.809047235794\\
56.375	0.1176	-683.77229483192\\
56.375	0.1182	-698.735542428045\\
56.375	0.1188	-713.69879002417\\
56.375	0.1194	-728.662037620295\\
56.375	0.12	-743.62528521642\\
56.375	0.1206	-758.588532812545\\
56.375	0.1212	-773.551780408668\\
56.375	0.1218	-788.515028004795\\
56.375	0.1224	-803.478275600918\\
56.375	0.123	-818.441523197043\\
56.75	0.093	-66.8711906657363\\
56.75	0.0936	-81.6271944895498\\
56.75	0.0942	-96.3831983133641\\
56.75	0.0948	-111.139202137177\\
56.75	0.0954	-125.895205960991\\
56.75	0.096	-140.651209784804\\
56.75	0.0966	-155.407213608618\\
56.75	0.0972	-170.163217432431\\
56.75	0.0978	-184.919221256245\\
56.75	0.0984	-199.675225080059\\
56.75	0.099	-214.431228903873\\
56.75	0.0996	-229.187232727686\\
56.75	0.1002	-243.943236551499\\
56.75	0.1008	-258.699240375313\\
56.75	0.1014	-273.455244199126\\
56.75	0.102	-288.211248022941\\
56.75	0.1026	-302.967251846753\\
56.75	0.1032	-317.723255670568\\
56.75	0.1038	-332.479259494381\\
56.75	0.1044	-347.235263318194\\
56.75	0.105	-361.991267142008\\
56.75	0.1056	-376.747270965821\\
56.75	0.1062	-391.503274789635\\
56.75	0.1068	-406.259278613449\\
56.75	0.1074	-421.015282437262\\
56.75	0.108	-435.771286261076\\
56.75	0.1086	-450.52729008489\\
56.75	0.1092	-465.283293908703\\
56.75	0.1098	-480.039297732517\\
56.75	0.1104	-494.795301556331\\
56.75	0.111	-509.551305380143\\
56.75	0.1116	-524.307309203958\\
56.75	0.1122	-539.06331302777\\
56.75	0.1128	-553.819316851585\\
56.75	0.1134	-568.575320675399\\
56.75	0.114	-583.331324499212\\
56.75	0.1146	-598.087328323025\\
56.75	0.1152	-612.843332146839\\
56.75	0.1158	-627.599335970652\\
56.75	0.1164	-642.355339794466\\
56.75	0.117	-657.111343618279\\
56.75	0.1176	-671.867347442093\\
56.75	0.1182	-686.623351265907\\
56.75	0.1188	-701.379355089721\\
56.75	0.1194	-716.135358913534\\
56.75	0.12	-730.891362737348\\
56.75	0.1206	-745.647366561162\\
56.75	0.1212	-760.403370384975\\
56.75	0.1218	-775.159374208789\\
56.75	0.1224	-789.915378032601\\
56.75	0.123	-804.671381856416\\
57.125	0.093	-63.4632379406694\\
57.125	0.0936	-78.011997992171\\
57.125	0.0942	-92.5607580436745\\
57.125	0.0948	-107.109518095176\\
57.125	0.0954	-121.658278146679\\
57.125	0.096	-136.207038198181\\
57.125	0.0966	-150.755798249683\\
57.125	0.0972	-165.304558301185\\
57.125	0.0978	-179.853318352688\\
57.125	0.0984	-194.40207840419\\
57.125	0.099	-208.950838455692\\
57.125	0.0996	-223.499598507196\\
57.125	0.1002	-238.048358558697\\
57.125	0.1008	-252.5971186102\\
57.125	0.1014	-267.145878661701\\
57.125	0.102	-281.694638713205\\
57.125	0.1026	-296.243398764706\\
57.125	0.1032	-310.792158816209\\
57.125	0.1038	-325.340918867711\\
57.125	0.1044	-339.889678919213\\
57.125	0.105	-354.438438970716\\
57.125	0.1056	-368.987199022218\\
57.125	0.1062	-383.53595907372\\
57.125	0.1068	-398.084719125223\\
57.125	0.1074	-412.633479176725\\
57.125	0.108	-427.182239228227\\
57.125	0.1086	-441.73099927973\\
57.125	0.1092	-456.279759331232\\
57.125	0.1098	-470.828519382735\\
57.125	0.1104	-485.377279434238\\
57.125	0.111	-499.926039485738\\
57.125	0.1116	-514.474799537242\\
57.125	0.1122	-529.023559588743\\
57.125	0.1128	-543.572319640246\\
57.125	0.1134	-558.121079691748\\
57.125	0.114	-572.669839743252\\
57.125	0.1146	-587.218599794753\\
57.125	0.1152	-601.767359846256\\
57.125	0.1158	-616.316119897757\\
57.125	0.1164	-630.86487994926\\
57.125	0.117	-645.413640000762\\
57.125	0.1176	-659.962400052264\\
57.125	0.1182	-674.511160103767\\
57.125	0.1188	-689.05992015527\\
57.125	0.1194	-703.608680206772\\
57.125	0.12	-718.157440258275\\
57.125	0.1206	-732.706200309777\\
57.125	0.1212	-747.254960361278\\
57.125	0.1218	-761.803720412782\\
57.125	0.1224	-776.352480464283\\
57.125	0.123	-790.901240515786\\
57.5	0.093	-60.0552852156034\\
57.5	0.0936	-74.3968014947941\\
57.5	0.0942	-88.7383177739848\\
57.5	0.0948	-103.079834053176\\
57.5	0.0954	-117.421350332367\\
57.5	0.096	-131.762866611559\\
57.5	0.0966	-146.10438289075\\
57.5	0.0972	-160.44589916994\\
57.5	0.0978	-174.787415449132\\
57.5	0.0984	-189.128931728323\\
57.5	0.099	-203.470448007513\\
57.5	0.0996	-217.811964286705\\
57.5	0.1002	-232.153480565896\\
57.5	0.1008	-246.494996845087\\
57.5	0.1014	-260.836513124278\\
57.5	0.102	-275.17802940347\\
57.5	0.1026	-289.51954568266\\
57.5	0.1032	-303.861061961851\\
57.5	0.1038	-318.202578241042\\
57.5	0.1044	-332.544094520234\\
57.5	0.105	-346.885610799424\\
57.5	0.1056	-361.227127078616\\
57.5	0.1062	-375.568643357807\\
57.5	0.1068	-389.910159636998\\
57.5	0.1074	-404.251675916188\\
57.5	0.108	-418.59319219538\\
57.5	0.1086	-432.934708474571\\
57.5	0.1092	-447.276224753762\\
57.5	0.1098	-461.617741032954\\
57.5	0.1104	-475.959257312145\\
57.5	0.111	-490.300773591335\\
57.5	0.1116	-504.642289870526\\
57.5	0.1122	-518.983806149717\\
57.5	0.1128	-533.325322428908\\
57.5	0.1134	-547.6668387081\\
57.5	0.114	-562.008354987292\\
57.5	0.1146	-576.349871266481\\
57.5	0.1152	-590.691387545674\\
57.5	0.1158	-605.032903824863\\
57.5	0.1164	-619.374420104054\\
57.5	0.117	-633.715936383245\\
57.5	0.1176	-648.057452662437\\
57.5	0.1182	-662.398968941628\\
57.5	0.1188	-676.74048522082\\
57.5	0.1194	-691.082001500011\\
57.5	0.12	-705.423517779202\\
57.5	0.1206	-719.765034058393\\
57.5	0.1212	-734.106550337583\\
57.5	0.1218	-748.448066616776\\
57.5	0.1224	-762.789582895965\\
57.5	0.123	-777.131099175157\\
57.875	0.093	-56.6473324905355\\
57.875	0.0936	-70.7816049974153\\
57.875	0.0942	-84.9158775042952\\
57.875	0.0948	-99.050150011175\\
57.875	0.0954	-113.184422518055\\
57.875	0.096	-127.318695024936\\
57.875	0.0966	-141.452967531814\\
57.875	0.0972	-155.587240038694\\
57.875	0.0978	-169.721512545574\\
57.875	0.0984	-183.855785052455\\
57.875	0.099	-197.990057559334\\
57.875	0.0996	-212.124330066214\\
57.875	0.1002	-226.258602573093\\
57.875	0.1008	-240.392875079974\\
57.875	0.1014	-254.527147586853\\
57.875	0.102	-268.661420093734\\
57.875	0.1026	-282.795692600613\\
57.875	0.1032	-296.929965107493\\
57.875	0.1038	-311.064237614372\\
57.875	0.1044	-325.198510121252\\
57.875	0.105	-339.332782628132\\
57.875	0.1056	-353.467055135012\\
57.875	0.1062	-367.601327641892\\
57.875	0.1068	-381.735600148772\\
57.875	0.1074	-395.869872655651\\
57.875	0.108	-410.004145162531\\
57.875	0.1086	-424.138417669412\\
57.875	0.1092	-438.272690176291\\
57.875	0.1098	-452.406962683172\\
57.875	0.1104	-466.541235190051\\
57.875	0.111	-480.67550769693\\
57.875	0.1116	-494.80978020381\\
57.875	0.1122	-508.94405271069\\
57.875	0.1128	-523.07832521757\\
57.875	0.1134	-537.21259772445\\
57.875	0.114	-551.34687023133\\
57.875	0.1146	-565.481142738209\\
57.875	0.1152	-579.61541524509\\
57.875	0.1158	-593.749687751969\\
57.875	0.1164	-607.883960258849\\
57.875	0.117	-622.018232765728\\
57.875	0.1176	-636.152505272608\\
57.875	0.1182	-650.286777779489\\
57.875	0.1188	-664.421050286369\\
57.875	0.1194	-678.555322793248\\
57.875	0.12	-692.689595300129\\
57.875	0.1206	-706.823867807008\\
57.875	0.1212	-720.958140313887\\
57.875	0.1218	-735.092412820768\\
57.875	0.1224	-749.226685327647\\
57.875	0.123	-763.360957834527\\
58.25	0.093	-53.2393797654695\\
58.25	0.0936	-67.1664085000375\\
58.25	0.0942	-81.0934372346064\\
58.25	0.0948	-95.0204659691744\\
58.25	0.0954	-108.947494703744\\
58.25	0.096	-122.874523438313\\
58.25	0.0966	-136.801552172881\\
58.25	0.0972	-150.728580907449\\
58.25	0.0978	-164.655609642018\\
58.25	0.0984	-178.582638376587\\
58.25	0.099	-192.509667111155\\
58.25	0.0996	-206.436695845724\\
58.25	0.1002	-220.363724580292\\
58.25	0.1008	-234.290753314861\\
58.25	0.1014	-248.21778204943\\
58.25	0.102	-262.144810783999\\
58.25	0.1026	-276.071839518567\\
58.25	0.1032	-289.998868253136\\
58.25	0.1038	-303.925896987704\\
58.25	0.1044	-317.852925722273\\
58.25	0.105	-331.779954456841\\
58.25	0.1056	-345.70698319141\\
58.25	0.1062	-359.634011925978\\
58.25	0.1068	-373.561040660546\\
58.25	0.1074	-387.488069395115\\
58.25	0.108	-401.415098129684\\
58.25	0.1086	-415.342126864253\\
58.25	0.1092	-429.269155598821\\
58.25	0.1098	-443.19618433339\\
58.25	0.1104	-457.123213067959\\
58.25	0.111	-471.050241802526\\
58.25	0.1116	-484.977270537095\\
58.25	0.1122	-498.904299271663\\
58.25	0.1128	-512.831328006232\\
58.25	0.1134	-526.758356740801\\
58.25	0.114	-540.685385475371\\
58.25	0.1146	-554.612414209938\\
58.25	0.1152	-568.539442944508\\
58.25	0.1158	-582.466471679075\\
58.25	0.1164	-596.393500413644\\
58.25	0.117	-610.320529148212\\
58.25	0.1176	-624.247557882781\\
58.25	0.1182	-638.17458661735\\
58.25	0.1188	-652.101615351919\\
58.25	0.1194	-666.028644086487\\
58.25	0.12	-679.955672821055\\
58.25	0.1206	-693.882701555625\\
58.25	0.1212	-707.809730290192\\
58.25	0.1218	-721.736759024762\\
58.25	0.1224	-735.663787759329\\
58.25	0.123	-749.590816493898\\
58.625	0.093	-49.8314270404035\\
58.625	0.0936	-63.5512120026606\\
58.625	0.0942	-77.2709969649177\\
58.625	0.0948	-90.9907819271748\\
58.625	0.0954	-104.710566889433\\
58.625	0.096	-118.43035185169\\
58.625	0.0966	-132.150136813947\\
58.625	0.0972	-145.869921776204\\
58.625	0.0978	-159.589706738462\\
58.625	0.0984	-173.309491700719\\
58.625	0.099	-187.029276662976\\
58.625	0.0996	-200.749061625234\\
58.625	0.1002	-214.468846587491\\
58.625	0.1008	-228.188631549749\\
58.625	0.1014	-241.908416512006\\
58.625	0.102	-255.628201474264\\
58.625	0.1026	-269.34798643652\\
58.625	0.1032	-283.067771398778\\
58.625	0.1038	-296.787556361035\\
58.625	0.1044	-310.507341323292\\
58.625	0.105	-324.227126285549\\
58.625	0.1056	-337.946911247807\\
58.625	0.1062	-351.666696210064\\
58.625	0.1068	-365.386481172322\\
58.625	0.1074	-379.106266134579\\
58.625	0.108	-392.826051096837\\
58.625	0.1086	-406.545836059094\\
58.625	0.1092	-420.265621021351\\
58.625	0.1098	-433.985405983609\\
58.625	0.1104	-447.705190945867\\
58.625	0.111	-461.424975908122\\
58.625	0.1116	-475.14476087038\\
58.625	0.1122	-488.864545832637\\
58.625	0.1128	-502.584330794894\\
58.625	0.1134	-516.304115757152\\
58.625	0.114	-530.02390071941\\
58.625	0.1146	-543.743685681667\\
58.625	0.1152	-557.463470643925\\
58.625	0.1158	-571.183255606181\\
58.625	0.1164	-584.903040568439\\
58.625	0.117	-598.622825530695\\
58.625	0.1176	-612.342610492953\\
58.625	0.1182	-626.062395455211\\
58.625	0.1188	-639.782180417469\\
58.625	0.1194	-653.501965379725\\
58.625	0.12	-667.221750341983\\
58.625	0.1206	-680.941535304241\\
58.625	0.1212	-694.661320266497\\
58.625	0.1218	-708.381105228756\\
58.625	0.1224	-722.100890191012\\
58.625	0.123	-735.82067515327\\
59	0.093	-46.4234743153365\\
59	0.0936	-59.9360155052818\\
59	0.0942	-73.448556695228\\
59	0.0948	-86.9610978851742\\
59	0.0954	-100.47363907512\\
59	0.096	-113.986180265067\\
59	0.0966	-127.498721455013\\
59	0.0972	-141.011262644958\\
59	0.0978	-154.523803834904\\
59	0.0984	-168.036345024851\\
59	0.099	-181.548886214797\\
59	0.0996	-195.061427404743\\
59	0.1002	-208.573968594689\\
59	0.1008	-222.086509784635\\
59	0.1014	-235.599050974581\\
59	0.102	-249.111592164527\\
59	0.1026	-262.624133354473\\
59	0.1032	-276.136674544419\\
59	0.1038	-289.649215734365\\
59	0.1044	-303.161756924312\\
59	0.105	-316.674298114257\\
59	0.1056	-330.186839304204\\
59	0.1062	-343.699380494149\\
59	0.1068	-357.211921684096\\
59	0.1074	-370.724462874042\\
59	0.108	-384.237004063988\\
59	0.1086	-397.749545253934\\
59	0.1092	-411.26208644388\\
59	0.1098	-424.774627633827\\
59	0.1104	-438.287168823773\\
59	0.111	-451.799710013718\\
59	0.1116	-465.312251203664\\
59	0.1122	-478.82479239361\\
59	0.1128	-492.337333583557\\
59	0.1134	-505.849874773503\\
59	0.114	-519.362415963449\\
59	0.1146	-532.874957153394\\
59	0.1152	-546.387498343342\\
59	0.1158	-559.900039533286\\
59	0.1164	-573.412580723233\\
59	0.117	-586.925121913178\\
59	0.1176	-600.437663103125\\
59	0.1182	-613.950204293072\\
59	0.1188	-627.462745483018\\
59	0.1194	-640.975286672963\\
59	0.12	-654.48782786291\\
59	0.1206	-668.000369052857\\
59	0.1212	-681.512910242801\\
59	0.1218	-695.025451432748\\
59	0.1224	-708.537992622693\\
59	0.123	-722.05053381264\\
59.375	0.093	-43.0155215902696\\
59.375	0.0936	-56.320819007904\\
59.375	0.0942	-69.6261164255393\\
59.375	0.0948	-82.9314138431737\\
59.375	0.0954	-96.236711260809\\
59.375	0.096	-109.542008678444\\
59.375	0.0966	-122.847306096079\\
59.375	0.0972	-136.152603513713\\
59.375	0.0978	-149.457900931348\\
59.375	0.0984	-162.763198348984\\
59.375	0.099	-176.068495766618\\
59.375	0.0996	-189.373793184253\\
59.375	0.1002	-202.679090601888\\
59.375	0.1008	-215.984388019523\\
59.375	0.1014	-229.289685437157\\
59.375	0.102	-242.594982854792\\
59.375	0.1026	-255.900280272426\\
59.375	0.1032	-269.205577690062\\
59.375	0.1038	-282.510875107696\\
59.375	0.1044	-295.816172525331\\
59.375	0.105	-309.121469942966\\
59.375	0.1056	-322.426767360601\\
59.375	0.1062	-335.732064778235\\
59.375	0.1068	-349.037362195871\\
59.375	0.1074	-362.342659613505\\
59.375	0.108	-375.64795703114\\
59.375	0.1086	-388.953254448776\\
59.375	0.1092	-402.25855186641\\
59.375	0.1098	-415.563849284045\\
59.375	0.1104	-428.869146701681\\
59.375	0.111	-442.174444119314\\
59.375	0.1116	-455.479741536949\\
59.375	0.1122	-468.785038954584\\
59.375	0.1128	-482.090336372219\\
59.375	0.1134	-495.395633789854\\
59.375	0.114	-508.70093120749\\
59.375	0.1146	-522.006228625123\\
59.375	0.1152	-535.311526042758\\
59.375	0.1158	-548.616823460392\\
59.375	0.1164	-561.922120878027\\
59.375	0.117	-575.227418295662\\
59.375	0.1176	-588.532715713297\\
59.375	0.1182	-601.838013130932\\
59.375	0.1188	-615.143310548568\\
59.375	0.1194	-628.448607966202\\
59.375	0.12	-641.753905383837\\
59.375	0.1206	-655.059202801473\\
59.375	0.1212	-668.364500219106\\
59.375	0.1218	-681.669797636742\\
59.375	0.1224	-694.975095054376\\
59.375	0.123	-708.280392472011\\
59.75	0.093	-39.6075688652027\\
59.75	0.0936	-52.7056225105262\\
59.75	0.0942	-65.8036761558496\\
59.75	0.0948	-78.9017298011731\\
59.75	0.0954	-91.9997834464966\\
59.75	0.096	-105.097837091821\\
59.75	0.0966	-118.195890737145\\
59.75	0.0972	-131.293944382467\\
59.75	0.0978	-144.391998027791\\
59.75	0.0984	-157.490051673115\\
59.75	0.099	-170.588105318438\\
59.75	0.0996	-183.686158963762\\
59.75	0.1002	-196.784212609085\\
59.75	0.1008	-209.882266254409\\
59.75	0.1014	-222.980319899732\\
59.75	0.102	-236.078373545056\\
59.75	0.1026	-249.176427190379\\
59.75	0.1032	-262.274480835704\\
59.75	0.1038	-275.372534481026\\
59.75	0.1044	-288.470588126351\\
59.75	0.105	-301.568641771673\\
59.75	0.1056	-314.666695416998\\
59.75	0.1062	-327.764749062321\\
59.75	0.1068	-340.862802707645\\
59.75	0.1074	-353.960856352968\\
59.75	0.108	-367.058909998292\\
59.75	0.1086	-380.156963643616\\
59.75	0.1092	-393.255017288939\\
59.75	0.1098	-406.353070934263\\
59.75	0.1104	-419.451124579587\\
59.75	0.111	-432.549178224909\\
59.75	0.1116	-445.647231870234\\
59.75	0.1122	-458.745285515556\\
59.75	0.1128	-471.843339160881\\
59.75	0.1134	-484.941392806204\\
59.75	0.114	-498.039446451528\\
59.75	0.1146	-511.137500096851\\
59.75	0.1152	-524.235553742175\\
59.75	0.1158	-537.333607387498\\
59.75	0.1164	-550.431661032821\\
59.75	0.117	-563.529714678145\\
59.75	0.1176	-576.627768323468\\
59.75	0.1182	-589.725821968793\\
59.75	0.1188	-602.823875614116\\
59.75	0.1194	-615.92192925944\\
59.75	0.12	-629.019982904764\\
59.75	0.1206	-642.118036550088\\
59.75	0.1212	-655.21609019541\\
59.75	0.1218	-668.314143840735\\
59.75	0.1224	-681.412197486057\\
59.75	0.123	-694.510251131381\\
60.125	0.093	-36.1996161401366\\
60.125	0.0936	-49.0904260131483\\
60.125	0.0942	-61.9812358861609\\
60.125	0.0948	-74.8720457591726\\
60.125	0.0954	-87.7628556321852\\
60.125	0.096	-100.653665505199\\
60.125	0.0966	-113.54447537821\\
60.125	0.0972	-126.435285251222\\
60.125	0.0978	-139.326095124235\\
60.125	0.0984	-152.216904997247\\
60.125	0.099	-165.10771487026\\
60.125	0.0996	-177.998524743272\\
60.125	0.1002	-190.889334616284\\
60.125	0.1008	-203.780144489297\\
60.125	0.1014	-216.670954362308\\
60.125	0.102	-229.561764235321\\
60.125	0.1026	-242.452574108333\\
60.125	0.1032	-255.343383981346\\
60.125	0.1038	-268.234193854358\\
60.125	0.1044	-281.12500372737\\
60.125	0.105	-294.015813600382\\
60.125	0.1056	-306.906623473395\\
60.125	0.1062	-319.797433346407\\
60.125	0.1068	-332.68824321942\\
60.125	0.1074	-345.579053092431\\
60.125	0.108	-358.469862965444\\
60.125	0.1086	-371.360672838458\\
60.125	0.1092	-384.251482711469\\
60.125	0.1098	-397.142292584482\\
60.125	0.1104	-410.033102457494\\
60.125	0.111	-422.923912330505\\
60.125	0.1116	-435.814722203519\\
60.125	0.1122	-448.70553207653\\
60.125	0.1128	-461.596341949543\\
60.125	0.1134	-474.487151822555\\
60.125	0.114	-487.377961695568\\
60.125	0.1146	-500.26877156858\\
60.125	0.1152	-513.159581441593\\
60.125	0.1158	-526.050391314604\\
60.125	0.1164	-538.941201187617\\
60.125	0.117	-551.832011060628\\
60.125	0.1176	-564.722820933641\\
60.125	0.1182	-577.613630806654\\
60.125	0.1188	-590.504440679667\\
60.125	0.1194	-603.395250552679\\
60.125	0.12	-616.286060425691\\
60.125	0.1206	-629.176870298704\\
60.125	0.1212	-642.067680171715\\
60.125	0.1218	-654.958490044729\\
60.125	0.1224	-667.84929991774\\
60.125	0.123	-680.740109790752\\
60.5	0.093	-32.7916634150688\\
60.5	0.0936	-45.4752295157696\\
60.5	0.0942	-58.1587956164713\\
60.5	0.0948	-70.842361717172\\
60.5	0.0954	-83.5259278178737\\
60.5	0.096	-96.2094939185754\\
60.5	0.0966	-108.893060019275\\
60.5	0.0972	-121.576626119976\\
60.5	0.0978	-134.260192220678\\
60.5	0.0984	-146.943758321379\\
60.5	0.099	-159.62732442208\\
60.5	0.0996	-172.310890522781\\
60.5	0.1002	-184.994456623482\\
60.5	0.1008	-197.678022724183\\
60.5	0.1014	-210.361588824884\\
60.5	0.102	-223.045154925585\\
60.5	0.1026	-235.728721026286\\
60.5	0.1032	-248.412287126987\\
60.5	0.1038	-261.095853227688\\
60.5	0.1044	-273.77941932839\\
60.5	0.105	-286.462985429091\\
60.5	0.1056	-299.146551529791\\
60.5	0.1062	-311.830117630492\\
60.5	0.1068	-324.513683731194\\
60.5	0.1074	-337.197249831895\\
60.5	0.108	-349.880815932596\\
60.5	0.1086	-362.564382033297\\
60.5	0.1092	-375.247948133998\\
60.5	0.1098	-387.9315142347\\
60.5	0.1104	-400.615080335401\\
60.5	0.111	-413.298646436101\\
60.5	0.1116	-425.982212536802\\
60.5	0.1122	-438.665778637503\\
60.5	0.1128	-451.349344738204\\
60.5	0.1134	-464.032910838906\\
60.5	0.114	-476.716476939608\\
60.5	0.1146	-489.400043040307\\
60.5	0.1152	-502.083609141009\\
60.5	0.1158	-514.767175241709\\
60.5	0.1164	-527.450741342411\\
60.5	0.117	-540.134307443112\\
60.5	0.1176	-552.817873543812\\
60.5	0.1182	-565.501439644514\\
60.5	0.1188	-578.185005745216\\
60.5	0.1194	-590.868571845916\\
60.5	0.12	-603.552137946618\\
60.5	0.1206	-616.235704047319\\
60.5	0.1212	-628.919270148019\\
60.5	0.1218	-641.602836248721\\
60.5	0.1224	-654.286402349421\\
60.5	0.123	-666.969968450123\\
60.875	0.093	-29.3837106900037\\
60.875	0.0936	-41.8600330183926\\
60.875	0.0942	-54.3363553467834\\
60.875	0.0948	-66.8126776751724\\
60.875	0.0954	-79.2890000035632\\
60.875	0.096	-91.765322331953\\
60.875	0.0966	-104.241644660343\\
60.875	0.0972	-116.717966988732\\
60.875	0.0978	-129.194289317122\\
60.875	0.0984	-141.670611645513\\
60.875	0.099	-154.146933973901\\
60.875	0.0996	-166.623256302292\\
60.875	0.1002	-179.099578630681\\
60.875	0.1008	-191.575900959072\\
60.875	0.1014	-204.052223287461\\
60.875	0.102	-216.528545615851\\
60.875	0.1026	-229.004867944241\\
60.875	0.1032	-241.481190272631\\
60.875	0.1038	-253.95751260102\\
60.875	0.1044	-266.43383492941\\
60.875	0.105	-278.910157257799\\
60.875	0.1056	-291.38647958619\\
60.875	0.1062	-303.862801914579\\
60.875	0.1068	-316.33912424297\\
60.875	0.1074	-328.815446571359\\
60.875	0.108	-341.29176889975\\
60.875	0.1086	-353.768091228139\\
60.875	0.1092	-366.244413556528\\
60.875	0.1098	-378.720735884919\\
60.875	0.1104	-391.197058213309\\
60.875	0.111	-403.673380541698\\
60.875	0.1116	-416.149702870088\\
60.875	0.1122	-428.626025198478\\
60.875	0.1128	-441.102347526868\\
60.875	0.1134	-453.578669855257\\
60.875	0.114	-466.054992183648\\
60.875	0.1146	-478.531314512036\\
60.875	0.1152	-491.007636840428\\
60.875	0.1158	-503.483959168816\\
60.875	0.1164	-515.960281497207\\
60.875	0.117	-528.436603825596\\
60.875	0.1176	-540.912926153986\\
60.875	0.1182	-553.389248482376\\
60.875	0.1188	-565.865570810766\\
60.875	0.1194	-578.341893139156\\
60.875	0.12	-590.818215467546\\
60.875	0.1206	-603.294537795937\\
60.875	0.1212	-615.770860124325\\
60.875	0.1218	-628.247182452716\\
60.875	0.1224	-640.723504781105\\
60.875	0.123	-653.199827109494\\
61.25	0.093	-25.9757579649367\\
61.25	0.0936	-38.2448365210148\\
61.25	0.0942	-50.5139150770938\\
61.25	0.0948	-62.7829936331718\\
61.25	0.0954	-75.0520721892508\\
61.25	0.096	-87.3211507453298\\
61.25	0.0966	-99.5902293014078\\
61.25	0.0972	-111.859307857486\\
61.25	0.0978	-124.128386413565\\
61.25	0.0984	-136.397464969644\\
61.25	0.099	-148.666543525722\\
61.25	0.0996	-160.935622081801\\
61.25	0.1002	-173.204700637879\\
61.25	0.1008	-185.473779193958\\
61.25	0.1014	-197.742857750036\\
61.25	0.102	-210.011936306115\\
61.25	0.1026	-222.281014862193\\
61.25	0.1032	-234.550093418272\\
61.25	0.1038	-246.819171974351\\
61.25	0.1044	-259.088250530429\\
61.25	0.105	-271.357329086508\\
61.25	0.1056	-283.626407642587\\
61.25	0.1062	-295.895486198665\\
61.25	0.1068	-308.164564754744\\
61.25	0.1074	-320.433643310822\\
61.25	0.108	-332.702721866901\\
61.25	0.1086	-344.97180042298\\
61.25	0.1092	-357.240878979058\\
61.25	0.1098	-369.509957535137\\
61.25	0.1104	-381.779036091216\\
61.25	0.111	-394.048114647293\\
61.25	0.1116	-406.317193203372\\
61.25	0.1122	-418.58627175945\\
61.25	0.1128	-430.855350315529\\
61.25	0.1134	-443.124428871608\\
61.25	0.114	-455.393507427687\\
61.25	0.1146	-467.662585983764\\
61.25	0.1152	-479.931664539844\\
61.25	0.1158	-492.200743095921\\
61.25	0.1164	-504.469821652\\
61.25	0.117	-516.738900208078\\
61.25	0.1176	-529.007978764157\\
61.25	0.1182	-541.277057320237\\
61.25	0.1188	-553.546135876316\\
61.25	0.1194	-565.815214432394\\
61.25	0.12	-578.084292988473\\
61.25	0.1206	-590.353371544552\\
61.25	0.1212	-602.622450100629\\
61.25	0.1218	-614.891528656709\\
61.25	0.1224	-627.160607212786\\
61.25	0.123	-639.429685768865\\
61.625	0.093	-22.5678052398698\\
61.625	0.0936	-34.629640023637\\
61.625	0.0942	-46.691474807405\\
61.625	0.0948	-58.7533095911713\\
61.625	0.0954	-70.8151443749393\\
61.625	0.096	-82.8769791587074\\
61.625	0.0966	-94.9388139424736\\
61.625	0.0972	-107.000648726241\\
61.625	0.0978	-119.062483510009\\
61.625	0.0984	-131.124318293776\\
61.625	0.099	-143.186153077543\\
61.625	0.0996	-155.247987861311\\
61.625	0.1002	-167.309822645077\\
61.625	0.1008	-179.371657428846\\
61.625	0.1014	-191.433492212613\\
61.625	0.102	-203.49532699638\\
61.625	0.1026	-215.557161780147\\
61.625	0.1032	-227.618996563915\\
61.625	0.1038	-239.680831347681\\
61.625	0.1044	-251.742666131449\\
61.625	0.105	-263.804500915217\\
61.625	0.1056	-275.866335698984\\
61.625	0.1062	-287.928170482751\\
61.625	0.1068	-299.990005266519\\
61.625	0.1074	-312.051840050285\\
61.625	0.108	-324.113674834053\\
61.625	0.1086	-336.175509617821\\
61.625	0.1092	-348.237344401588\\
61.625	0.1098	-360.299179185356\\
61.625	0.1104	-372.361013969124\\
61.625	0.111	-384.422848752889\\
61.625	0.1116	-396.484683536657\\
61.625	0.1122	-408.546518320424\\
61.625	0.1128	-420.608353104191\\
61.625	0.1134	-432.670187887959\\
61.625	0.114	-444.732022671727\\
61.625	0.1146	-456.793857455493\\
61.625	0.1152	-468.855692239262\\
61.625	0.1158	-480.917527023028\\
61.625	0.1164	-492.979361806795\\
61.625	0.117	-505.041196590562\\
61.625	0.1176	-517.103031374329\\
61.625	0.1182	-529.164866158098\\
61.625	0.1188	-541.226700941866\\
61.625	0.1194	-553.288535725632\\
61.625	0.12	-565.3503705094\\
61.625	0.1206	-577.412205293168\\
61.625	0.1212	-589.474040076934\\
61.625	0.1218	-601.535874860702\\
61.625	0.1224	-613.597709644469\\
61.625	0.123	-625.659544428236\\
62	0.093	-19.1598525148029\\
62	0.0936	-31.0144435262582\\
62	0.0942	-42.8690345377154\\
62	0.0948	-54.7236255491707\\
62	0.0954	-66.578216560627\\
62	0.096	-78.4328075720841\\
62	0.0966	-90.2873985835395\\
62	0.0972	-102.141989594995\\
62	0.0978	-113.996580606451\\
62	0.0984	-125.851171617908\\
62	0.099	-137.705762629364\\
62	0.0996	-149.56035364082\\
62	0.1002	-161.414944652275\\
62	0.1008	-173.269535663732\\
62	0.1014	-185.124126675188\\
62	0.102	-196.978717686644\\
62	0.1026	-208.8333086981\\
62	0.1032	-220.687899709556\\
62	0.1038	-232.542490721012\\
62	0.1044	-244.397081732468\\
62	0.105	-256.251672743924\\
62	0.1056	-268.10626375538\\
62	0.1062	-279.960854766836\\
62	0.1068	-291.815445778293\\
62	0.1074	-303.670036789748\\
62	0.108	-315.524627801205\\
62	0.1086	-327.379218812661\\
62	0.1092	-339.233809824117\\
62	0.1098	-351.088400835573\\
62	0.1104	-362.94299184703\\
62	0.111	-374.797582858485\\
62	0.1116	-386.652173869941\\
62	0.1122	-398.506764881397\\
62	0.1128	-410.361355892853\\
62	0.1134	-422.21594690431\\
62	0.114	-434.070537915766\\
62	0.1146	-445.925128927221\\
62	0.1152	-457.779719938678\\
62	0.1158	-469.634310950133\\
62	0.1164	-481.488901961589\\
62	0.117	-493.343492973045\\
62	0.1176	-505.198083984502\\
62	0.1182	-517.052674995958\\
62	0.1188	-528.907266007414\\
62	0.1194	-540.76185701887\\
62	0.12	-552.616448030327\\
62	0.1206	-564.471039041783\\
62	0.1212	-576.325630053238\\
62	0.1218	-588.180221064695\\
62	0.1224	-600.03481207615\\
62	0.123	-611.889403087606\\
62.375	0.093	-15.7518997897369\\
62.375	0.0936	-27.3992470288813\\
62.375	0.0942	-39.0465942680266\\
62.375	0.0948	-50.6939415071711\\
62.375	0.0954	-62.3412887463155\\
62.375	0.096	-73.9886359854609\\
62.375	0.0966	-85.6359832246053\\
62.375	0.0972	-97.2833304637497\\
62.375	0.0978	-108.930677702895\\
62.375	0.0984	-120.57802494204\\
62.375	0.099	-132.225372181185\\
62.375	0.0996	-143.87271942033\\
62.375	0.1002	-155.520066659475\\
62.375	0.1008	-167.16741389862\\
62.375	0.1014	-178.814761137764\\
62.375	0.102	-190.462108376909\\
62.375	0.1026	-202.109455616053\\
62.375	0.1032	-213.756802855199\\
62.375	0.1038	-225.404150094343\\
62.375	0.1044	-237.051497333488\\
62.375	0.105	-248.698844572633\\
62.375	0.1056	-260.346191811778\\
62.375	0.1062	-271.993539050923\\
62.375	0.1068	-283.640886290068\\
62.375	0.1074	-295.288233529212\\
62.375	0.108	-306.935580768357\\
62.375	0.1086	-318.582928007502\\
62.375	0.1092	-330.230275246647\\
62.375	0.1098	-341.877622485792\\
62.375	0.1104	-353.524969724937\\
62.375	0.111	-365.172316964081\\
62.375	0.1116	-376.819664203226\\
62.375	0.1122	-388.467011442371\\
62.375	0.1128	-400.114358681515\\
62.375	0.1134	-411.76170592066\\
62.375	0.114	-423.409053159806\\
62.375	0.1146	-435.056400398949\\
62.375	0.1152	-446.703747638096\\
62.375	0.1158	-458.351094877239\\
62.375	0.1164	-469.998442116384\\
62.375	0.117	-481.645789355529\\
62.375	0.1176	-493.293136594674\\
62.375	0.1182	-504.94048383382\\
62.375	0.1188	-516.587831072964\\
62.375	0.1194	-528.235178312108\\
62.375	0.12	-539.882525551254\\
62.375	0.1206	-551.529872790399\\
62.375	0.1212	-563.177220029543\\
62.375	0.1218	-574.824567268689\\
62.375	0.1224	-586.471914507832\\
62.375	0.123	-598.119261746978\\
62.75	0.093	-12.3439470646699\\
62.75	0.0936	-23.7840505315025\\
62.75	0.0942	-35.224153998337\\
62.75	0.0948	-46.6642574651696\\
62.75	0.0954	-58.1043609320041\\
62.75	0.096	-69.5444643988376\\
62.75	0.0966	-80.9845678656711\\
62.75	0.0972	-92.4246713325037\\
62.75	0.0978	-103.864774799338\\
62.75	0.0984	-115.304878266172\\
62.75	0.099	-126.744981733005\\
62.75	0.0996	-138.185085199839\\
62.75	0.1002	-149.625188666672\\
62.75	0.1008	-161.065292133506\\
62.75	0.1014	-172.505395600339\\
62.75	0.102	-183.945499067173\\
62.75	0.1026	-195.385602534006\\
62.75	0.1032	-206.82570600084\\
62.75	0.1038	-218.265809467674\\
62.75	0.1044	-229.705912934507\\
62.75	0.105	-241.146016401341\\
62.75	0.1056	-252.586119868174\\
62.75	0.1062	-264.026223335008\\
62.75	0.1068	-275.466326801841\\
62.75	0.1074	-286.906430268675\\
62.75	0.108	-298.346533735508\\
62.75	0.1086	-309.786637202343\\
62.75	0.1092	-321.226740669175\\
62.75	0.1098	-332.66684413601\\
62.75	0.1104	-344.106947602843\\
62.75	0.111	-355.547051069676\\
62.75	0.1116	-366.98715453651\\
62.75	0.1122	-378.427258003343\\
62.75	0.1128	-389.867361470177\\
62.75	0.1134	-401.307464937011\\
62.75	0.114	-412.747568403845\\
62.75	0.1146	-424.187671870677\\
62.75	0.1152	-435.627775337513\\
62.75	0.1158	-447.067878804344\\
62.75	0.1164	-458.507982271179\\
62.75	0.117	-469.948085738011\\
62.75	0.1176	-481.388189204846\\
62.75	0.1182	-492.828292671679\\
62.75	0.1188	-504.268396138514\\
62.75	0.1194	-515.708499605346\\
62.75	0.12	-527.148603072181\\
62.75	0.1206	-538.588706539014\\
62.75	0.1212	-550.028810005847\\
62.75	0.1218	-561.468913472681\\
62.75	0.1224	-572.909016939514\\
62.75	0.123	-584.349120406348\\
63.125	0.093	-8.93599433960298\\
63.125	0.0936	-20.1688540341247\\
63.125	0.0942	-31.4017137286482\\
63.125	0.0948	-42.63457342317\\
63.125	0.0954	-53.8674331176926\\
63.125	0.096	-65.1002928122152\\
63.125	0.0966	-76.333152506737\\
63.125	0.0972	-87.5660122012587\\
63.125	0.0978	-98.7988718957813\\
63.125	0.0984	-110.031731590305\\
63.125	0.099	-121.264591284827\\
63.125	0.0996	-132.497450979349\\
63.125	0.1002	-143.730310673871\\
63.125	0.1008	-154.963170368394\\
63.125	0.1014	-166.196030062915\\
63.125	0.102	-177.428889757438\\
63.125	0.1026	-188.66174945196\\
63.125	0.1032	-199.894609146483\\
63.125	0.1038	-211.127468841005\\
63.125	0.1044	-222.360328535528\\
63.125	0.105	-233.593188230049\\
63.125	0.1056	-244.826047924572\\
63.125	0.1062	-256.058907619094\\
63.125	0.1068	-267.291767313616\\
63.125	0.1074	-278.524627008138\\
63.125	0.108	-289.757486702661\\
63.125	0.1086	-300.990346397184\\
63.125	0.1092	-312.223206091706\\
63.125	0.1098	-323.456065786228\\
63.125	0.1104	-334.688925480751\\
63.125	0.111	-345.921785175272\\
63.125	0.1116	-357.154644869795\\
63.125	0.1122	-368.387504564316\\
63.125	0.1128	-379.62036425884\\
63.125	0.1134	-390.853223953362\\
63.125	0.114	-402.086083647885\\
63.125	0.1146	-413.318943342406\\
63.125	0.1152	-424.551803036929\\
63.125	0.1158	-435.78466273145\\
63.125	0.1164	-447.017522425973\\
63.125	0.117	-458.250382120495\\
63.125	0.1176	-469.483241815018\\
63.125	0.1182	-480.716101509541\\
63.125	0.1188	-491.948961204063\\
63.125	0.1194	-503.181820898585\\
63.125	0.12	-514.414680593108\\
63.125	0.1206	-525.64754028763\\
63.125	0.1212	-536.880399982152\\
63.125	0.1218	-548.113259676676\\
63.125	0.1224	-559.346119371196\\
63.125	0.123	-570.578979065719\\
63.5	0.093	-5.52804161453605\\
63.5	0.0936	-16.5536575367469\\
63.5	0.0942	-27.5792734589586\\
63.5	0.0948	-38.6048893811685\\
63.5	0.0954	-49.6305053033802\\
63.5	0.096	-60.656121225592\\
63.5	0.0966	-71.6817371478028\\
63.5	0.0972	-82.7073530700127\\
63.5	0.0978	-93.7329689922244\\
63.5	0.0984	-104.758584914436\\
63.5	0.099	-115.784200836647\\
63.5	0.0996	-126.809816758858\\
63.5	0.1002	-137.835432681069\\
63.5	0.1008	-148.86104860328\\
63.5	0.1014	-159.886664525491\\
63.5	0.102	-170.912280447702\\
63.5	0.1026	-181.937896369913\\
63.5	0.1032	-192.963512292125\\
63.5	0.1038	-203.989128214335\\
63.5	0.1044	-215.014744136546\\
63.5	0.105	-226.040360058757\\
63.5	0.1056	-237.065975980969\\
63.5	0.1062	-248.09159190318\\
63.5	0.1068	-259.11720782539\\
63.5	0.1074	-270.142823747601\\
63.5	0.108	-281.168439669813\\
63.5	0.1086	-292.194055592025\\
63.5	0.1092	-303.219671514235\\
63.5	0.1098	-314.245287436446\\
63.5	0.1104	-325.270903358658\\
63.5	0.111	-336.296519280868\\
63.5	0.1116	-347.322135203079\\
63.5	0.1122	-358.34775112529\\
63.5	0.1128	-369.373367047501\\
63.5	0.1134	-380.398982969713\\
63.5	0.114	-391.424598891924\\
63.5	0.1146	-402.450214814134\\
63.5	0.1152	-413.475830736346\\
63.5	0.1158	-424.501446658556\\
63.5	0.1164	-435.527062580767\\
63.5	0.117	-446.552678502978\\
63.5	0.1176	-457.57829442519\\
63.5	0.1182	-468.603910347401\\
63.5	0.1188	-479.629526269612\\
63.5	0.1194	-490.655142191823\\
63.5	0.12	-501.680758114035\\
63.5	0.1206	-512.706374036246\\
63.5	0.1212	-523.731989958455\\
63.5	0.1218	-534.757605880668\\
63.5	0.1224	-545.783221802878\\
63.5	0.123	-556.808837725089\\
63.875	0.093	-2.12008888947003\\
63.875	0.0936	-12.938461039369\\
63.875	0.0942	-23.7568331892699\\
63.875	0.0948	-34.5752053391689\\
63.875	0.0954	-45.3935774890688\\
63.875	0.096	-56.2119496389696\\
63.875	0.0966	-67.0303217888686\\
63.875	0.0972	-77.8486939387676\\
63.875	0.0978	-88.6670660886684\\
63.875	0.0984	-99.4854382385683\\
63.875	0.099	-110.303810388467\\
63.875	0.0996	-121.122182538368\\
63.875	0.1002	-131.940554688267\\
63.875	0.1008	-142.758926838168\\
63.875	0.1014	-153.577298988067\\
63.875	0.102	-164.395671137967\\
63.875	0.1026	-175.214043287867\\
63.875	0.1032	-186.032415437767\\
63.875	0.1038	-196.850787587666\\
63.875	0.1044	-207.669159737567\\
63.875	0.105	-218.487531887466\\
63.875	0.1056	-229.305904037366\\
63.875	0.1062	-240.124276187265\\
63.875	0.1068	-250.942648337165\\
63.875	0.1074	-261.761020487065\\
63.875	0.108	-272.579392636965\\
63.875	0.1086	-283.397764786865\\
63.875	0.1092	-294.216136936765\\
63.875	0.1098	-305.034509086665\\
63.875	0.1104	-315.852881236565\\
63.875	0.111	-326.671253386464\\
63.875	0.1116	-337.489625536364\\
63.875	0.1122	-348.307997686263\\
63.875	0.1128	-359.126369836164\\
63.875	0.1134	-369.944741986063\\
63.875	0.114	-380.763114135964\\
63.875	0.1146	-391.581486285862\\
63.875	0.1152	-402.399858435763\\
63.875	0.1158	-413.218230585662\\
63.875	0.1164	-424.036602735562\\
63.875	0.117	-434.854974885461\\
63.875	0.1176	-445.673347035362\\
63.875	0.1182	-456.491719185262\\
63.875	0.1188	-467.310091335163\\
63.875	0.1194	-478.128463485062\\
63.875	0.12	-488.946835634962\\
63.875	0.1206	-499.765207784862\\
63.875	0.1212	-510.583579934761\\
63.875	0.1218	-521.401952084661\\
63.875	0.1224	-532.22032423456\\
63.875	0.123	-543.03869638446\\
64.25	0.093	1.28786383559691\\
64.25	0.0936	-9.32326454199119\\
64.25	0.0942	-19.9343929195802\\
64.25	0.0948	-30.5455212971674\\
64.25	0.0954	-41.1566496747564\\
64.25	0.096	-51.7677780523454\\
64.25	0.0966	-62.3789064299335\\
64.25	0.0972	-72.9900348075216\\
64.25	0.0978	-83.6011631851106\\
64.25	0.0984	-94.2122915626996\\
64.25	0.099	-104.823419940288\\
64.25	0.0996	-115.434548317877\\
64.25	0.1002	-126.045676695465\\
64.25	0.1008	-136.656805073054\\
64.25	0.1014	-147.267933450642\\
64.25	0.102	-157.879061828231\\
64.25	0.1026	-168.490190205819\\
64.25	0.1032	-179.101318583408\\
64.25	0.1038	-189.712446960996\\
64.25	0.1044	-200.323575338585\\
64.25	0.105	-210.934703716173\\
64.25	0.1056	-221.545832093762\\
64.25	0.1062	-232.15696047135\\
64.25	0.1068	-242.768088848939\\
64.25	0.1074	-253.379217226528\\
64.25	0.108	-263.990345604117\\
64.25	0.1086	-274.601473981706\\
64.25	0.1092	-285.212602359294\\
64.25	0.1098	-295.823730736883\\
64.25	0.1104	-306.434859114472\\
64.25	0.111	-317.045987492059\\
64.25	0.1116	-327.657115869648\\
64.25	0.1122	-338.268244247236\\
64.25	0.1128	-348.879372624825\\
64.25	0.1134	-359.490501002414\\
64.25	0.114	-370.101629380003\\
64.25	0.1146	-380.71275775759\\
64.25	0.1152	-391.32388613518\\
64.25	0.1158	-401.935014512767\\
64.25	0.1164	-412.546142890356\\
64.25	0.117	-423.157271267944\\
64.25	0.1176	-433.768399645533\\
64.25	0.1182	-444.379528023122\\
64.25	0.1188	-454.990656400711\\
64.25	0.1194	-465.6017847783\\
64.25	0.12	-476.212913155889\\
64.25	0.1206	-486.824041533478\\
64.25	0.1212	-497.435169911065\\
64.25	0.1218	-508.046298288655\\
64.25	0.1224	-518.657426666242\\
64.25	0.123	-529.268555043831\\
64.625	0.093	4.69581656066384\\
64.625	0.0936	-5.70806804461336\\
64.625	0.0942	-16.1119526498906\\
64.625	0.0948	-26.5158372551678\\
64.625	0.0954	-36.9197218604459\\
64.625	0.096	-47.3236064657231\\
64.625	0.0966	-57.7274910710003\\
64.625	0.0972	-68.1313756762765\\
64.625	0.0978	-78.5352602815547\\
64.625	0.0984	-88.9391448868328\\
64.625	0.099	-99.3430294921091\\
64.625	0.0996	-109.746914097387\\
64.625	0.1002	-120.150798702663\\
64.625	0.1008	-130.554683307942\\
64.625	0.1014	-140.958567913219\\
64.625	0.102	-151.362452518496\\
64.625	0.1026	-161.766337123773\\
64.625	0.1032	-172.17022172905\\
64.625	0.1038	-182.574106334328\\
64.625	0.1044	-192.977990939606\\
64.625	0.105	-203.381875544882\\
64.625	0.1056	-213.78576015016\\
64.625	0.1062	-224.189644755437\\
64.625	0.1068	-234.593529360714\\
64.625	0.1074	-244.997413965992\\
64.625	0.108	-255.401298571269\\
64.625	0.1086	-265.805183176547\\
64.625	0.1092	-276.209067781823\\
64.625	0.1098	-286.612952387101\\
64.625	0.1104	-297.016836992379\\
64.625	0.111	-307.420721597655\\
64.625	0.1116	-317.824606202933\\
64.625	0.1122	-328.228490808209\\
64.625	0.1128	-338.632375413487\\
64.625	0.1134	-349.036260018765\\
64.625	0.114	-359.440144624043\\
64.625	0.1146	-369.844029229319\\
64.625	0.1152	-380.247913834597\\
64.625	0.1158	-390.651798439873\\
64.625	0.1164	-401.055683045151\\
64.625	0.117	-411.459567650428\\
64.625	0.1176	-421.863452255706\\
64.625	0.1182	-432.267336860984\\
64.625	0.1188	-442.671221466261\\
64.625	0.1194	-453.075106071538\\
64.625	0.12	-463.478990676816\\
64.625	0.1206	-473.882875282094\\
64.625	0.1212	-484.28675988737\\
64.625	0.1218	-494.690644492648\\
64.625	0.1224	-505.094529097924\\
64.625	0.123	-515.498413703202\\
65	0.093	8.10376928572987\\
65	0.0936	-2.09287154723552\\
65	0.0942	-12.2895123802018\\
65	0.0948	-22.4861532131672\\
65	0.0954	-32.6827940461344\\
65	0.096	-42.8794348791007\\
65	0.0966	-53.0760757120661\\
65	0.0972	-63.2727165450315\\
65	0.0978	-73.4693573779978\\
65	0.0984	-83.665998210965\\
65	0.099	-93.8626390439304\\
65	0.0996	-104.059279876897\\
65	0.1002	-114.255920709862\\
65	0.1008	-124.452561542829\\
65	0.1014	-134.649202375795\\
65	0.102	-144.845843208761\\
65	0.1026	-155.042484041726\\
65	0.1032	-165.239124874694\\
65	0.1038	-175.435765707659\\
65	0.1044	-185.632406540625\\
65	0.105	-195.829047373591\\
65	0.1056	-206.025688206557\\
65	0.1062	-216.222329039523\\
65	0.1068	-226.418969872489\\
65	0.1074	-236.615610705455\\
65	0.108	-246.812251538421\\
65	0.1086	-257.008892371388\\
65	0.1092	-267.205533204354\\
65	0.1098	-277.40217403732\\
65	0.1104	-287.598814870286\\
65	0.111	-297.795455703251\\
65	0.1116	-307.992096536218\\
65	0.1122	-318.188737369183\\
65	0.1128	-328.38537820215\\
65	0.1134	-338.582019035116\\
65	0.114	-348.778659868083\\
65	0.1146	-358.975300701048\\
65	0.1152	-369.171941534015\\
65	0.1158	-379.368582366979\\
65	0.1164	-389.565223199946\\
65	0.117	-399.761864032912\\
65	0.1176	-409.958504865878\\
65	0.1182	-420.155145698845\\
65	0.1188	-430.351786531811\\
65	0.1194	-440.548427364777\\
65	0.12	-450.745068197743\\
65	0.1206	-460.94170903071\\
65	0.1212	-471.138349863674\\
65	0.1218	-481.334990696641\\
65	0.1224	-491.531631529607\\
65	0.123	-501.728272362573\\
65.375	0.093	11.5117220107968\\
65.375	0.0936	1.52232495014232\\
65.375	0.0942	-8.46707211051307\\
65.375	0.0948	-18.4564691711676\\
65.375	0.0954	-28.4458662318229\\
65.375	0.096	-38.4352632924783\\
65.375	0.0966	-48.4246603531328\\
65.375	0.0972	-58.4140574137864\\
65.375	0.0978	-68.4034544744418\\
65.375	0.0984	-78.3928515350972\\
65.375	0.099	-88.3822485957517\\
65.375	0.0996	-98.3716456564071\\
65.375	0.1002	-108.361042717062\\
65.375	0.1008	-118.350439777716\\
65.375	0.1014	-128.339836838371\\
65.375	0.102	-138.329233899026\\
65.375	0.1026	-148.31863095968\\
65.375	0.1032	-158.308028020336\\
65.375	0.1038	-168.29742508099\\
65.375	0.1044	-178.286822141645\\
65.375	0.105	-188.276219202299\\
65.375	0.1056	-198.265616262955\\
65.375	0.1062	-208.255013323609\\
65.375	0.1068	-218.244410384264\\
65.375	0.1074	-228.233807444919\\
65.375	0.108	-238.223204505573\\
65.375	0.1086	-248.212601566229\\
65.375	0.1092	-258.201998626883\\
65.375	0.1098	-268.191395687539\\
65.375	0.1104	-278.180792748194\\
65.375	0.111	-288.170189808848\\
65.375	0.1116	-298.159586869503\\
65.375	0.1122	-308.148983930157\\
65.375	0.1128	-318.138380990812\\
65.375	0.1134	-328.127778051467\\
65.375	0.114	-338.117175112123\\
65.375	0.1146	-348.106572172776\\
65.375	0.1152	-358.095969233433\\
65.375	0.1158	-368.085366294085\\
65.375	0.1164	-378.074763354741\\
65.375	0.117	-388.064160415395\\
65.375	0.1176	-398.053557476051\\
65.375	0.1182	-408.042954536706\\
65.375	0.1188	-418.032351597361\\
65.375	0.1194	-428.021748658015\\
65.375	0.12	-438.01114571867\\
65.375	0.1206	-448.000542779326\\
65.375	0.1212	-457.989939839979\\
65.375	0.1218	-467.979336900636\\
65.375	0.1224	-477.968733961289\\
65.375	0.123	-487.958131021945\\
65.75	0.093	14.9196747358637\\
65.75	0.0936	5.13752144752016\\
65.75	0.0942	-4.64463184082342\\
65.75	0.0948	-14.426785129167\\
65.75	0.0954	-24.2089384175106\\
65.75	0.096	-33.9910917058542\\
65.75	0.0966	-43.7732449941977\\
65.75	0.0972	-53.5553982825404\\
65.75	0.0978	-63.3375515708849\\
65.75	0.0984	-73.1197048592285\\
65.75	0.099	-82.9018581475721\\
65.75	0.0996	-92.6840114359156\\
65.75	0.1002	-102.466164724259\\
65.75	0.1008	-112.248318012603\\
65.75	0.1014	-122.030471300946\\
65.75	0.102	-131.81262458929\\
65.75	0.1026	-141.594777877634\\
65.75	0.1032	-151.376931165977\\
65.75	0.1038	-161.15908445432\\
65.75	0.1044	-170.941237742664\\
65.75	0.105	-180.723391031007\\
65.75	0.1056	-190.505544319351\\
65.75	0.1062	-200.287697607694\\
65.75	0.1068	-210.069850896039\\
65.75	0.1074	-219.852004184381\\
65.75	0.108	-229.634157472726\\
65.75	0.1086	-239.416310761069\\
65.75	0.1092	-249.198464049413\\
65.75	0.1098	-258.980617337756\\
65.75	0.1104	-268.7627706261\\
65.75	0.111	-278.544923914443\\
65.75	0.1116	-288.327077202786\\
65.75	0.1122	-298.10923049113\\
65.75	0.1128	-307.891383779473\\
65.75	0.1134	-317.673537067818\\
65.75	0.114	-327.455690356162\\
65.75	0.1146	-337.237843644504\\
65.75	0.1152	-347.019996932849\\
65.75	0.1158	-356.802150221191\\
65.75	0.1164	-366.584303509535\\
65.75	0.117	-376.366456797879\\
65.75	0.1176	-386.148610086222\\
65.75	0.1182	-395.930763374566\\
65.75	0.1188	-405.71291666291\\
65.75	0.1194	-415.495069951253\\
65.75	0.12	-425.277223239597\\
65.75	0.1206	-435.059376527941\\
65.75	0.1212	-444.841529816284\\
65.75	0.1218	-454.623683104628\\
65.75	0.1224	-464.405836392971\\
65.75	0.123	-474.187989681314\\
66.125	0.093	18.3276274609298\\
66.125	0.0936	8.752717944898\\
66.125	0.0942	-0.822191571134681\\
66.125	0.0948	-10.3971010871664\\
66.125	0.0954	-19.9720106031991\\
66.125	0.096	-29.5469201192318\\
66.125	0.0966	-39.1218296352636\\
66.125	0.0972	-48.6967391512953\\
66.125	0.0978	-58.2716486673289\\
66.125	0.0984	-67.8465581833616\\
66.125	0.099	-77.4214676993934\\
66.125	0.0996	-86.996377215426\\
66.125	0.1002	-96.5712867314578\\
66.125	0.1008	-106.14619624749\\
66.125	0.1014	-115.721105763522\\
66.125	0.102	-125.296015279555\\
66.125	0.1026	-134.870924795587\\
66.125	0.1032	-144.445834311619\\
66.125	0.1038	-154.020743827651\\
66.125	0.1044	-163.595653343684\\
66.125	0.105	-173.170562859716\\
66.125	0.1056	-182.745472375749\\
66.125	0.1062	-192.320381891781\\
66.125	0.1068	-201.895291407814\\
66.125	0.1074	-211.470200923845\\
66.125	0.108	-221.045110439878\\
66.125	0.1086	-230.620019955911\\
66.125	0.1092	-240.194929471942\\
66.125	0.1098	-249.769838987975\\
66.125	0.1104	-259.344748504008\\
66.125	0.111	-268.919658020039\\
66.125	0.1116	-278.494567536071\\
66.125	0.1122	-288.069477052103\\
66.125	0.1128	-297.644386568136\\
66.125	0.1134	-307.219296084169\\
66.125	0.114	-316.794205600202\\
66.125	0.1146	-326.369115116233\\
66.125	0.1152	-335.944024632267\\
66.125	0.1158	-345.518934148297\\
66.125	0.1164	-355.09384366433\\
66.125	0.117	-364.668753180362\\
66.125	0.1176	-374.243662696394\\
66.125	0.1182	-383.818572212427\\
66.125	0.1188	-393.39348172846\\
66.125	0.1194	-402.968391244492\\
66.125	0.12	-412.543300760524\\
66.125	0.1206	-422.118210276557\\
66.125	0.1212	-431.693119792589\\
66.125	0.1218	-441.268029308622\\
66.125	0.1224	-450.842938824653\\
66.125	0.123	-460.417848340686\\
66.5	0.093	21.7355801859967\\
66.5	0.0936	12.3679144422767\\
66.5	0.0942	3.00024869855497\\
66.5	0.0948	-6.36741704516589\\
66.5	0.0954	-15.7350827888868\\
66.5	0.096	-25.1027485326085\\
66.5	0.0966	-34.4704142763294\\
66.5	0.0972	-43.8380800200493\\
66.5	0.0978	-53.2057457637711\\
66.5	0.0984	-62.5734115074929\\
66.5	0.099	-71.9410772512138\\
66.5	0.0996	-81.3087429949346\\
66.5	0.1002	-90.6764087386555\\
66.5	0.1008	-100.044074482377\\
66.5	0.1014	-109.411740226097\\
66.5	0.102	-118.779405969819\\
66.5	0.1026	-128.14707171354\\
66.5	0.1032	-137.514737457261\\
66.5	0.1038	-146.882403200982\\
66.5	0.1044	-156.250068944703\\
66.5	0.105	-165.617734688424\\
66.5	0.1056	-174.985400432145\\
66.5	0.1062	-184.353066175866\\
66.5	0.1068	-193.720731919588\\
66.5	0.1074	-203.088397663308\\
66.5	0.108	-212.456063407029\\
66.5	0.1086	-221.823729150751\\
66.5	0.1092	-231.191394894471\\
66.5	0.1098	-240.559060638193\\
66.5	0.1104	-249.926726381915\\
66.5	0.111	-259.294392125634\\
66.5	0.1116	-268.662057869356\\
66.5	0.1122	-278.029723613076\\
66.5	0.1128	-287.397389356798\\
66.5	0.1134	-296.765055100519\\
66.5	0.114	-306.132720844241\\
66.5	0.1146	-315.500386587961\\
66.5	0.1152	-324.868052331683\\
66.5	0.1158	-334.235718075402\\
66.5	0.1164	-343.603383819124\\
66.5	0.117	-352.971049562844\\
66.5	0.1176	-362.338715306566\\
66.5	0.1182	-371.706381050288\\
66.5	0.1188	-381.07404679401\\
66.5	0.1194	-390.441712537729\\
66.5	0.12	-399.809378281451\\
66.5	0.1206	-409.177044025173\\
66.5	0.1212	-418.544709768892\\
66.5	0.1218	-427.912375512615\\
66.5	0.1224	-437.280041256335\\
66.5	0.123	-446.647707000056\\
66.875	0.093	25.1435329110636\\
66.875	0.0936	15.9831109396537\\
66.875	0.0942	6.82268896824371\\
66.875	0.0948	-2.33773300316534\\
66.875	0.0954	-11.4981549745762\\
66.875	0.096	-20.6585769459862\\
66.875	0.0966	-29.8189989173952\\
66.875	0.0972	-38.9794208888052\\
66.875	0.0978	-48.1398428602151\\
66.875	0.0984	-57.3002648316251\\
66.875	0.099	-66.4606868030351\\
66.875	0.0996	-75.621108774445\\
66.875	0.1002	-84.7815307458541\\
66.875	0.1008	-93.941952717264\\
66.875	0.1014	-103.102374688674\\
66.875	0.102	-112.262796660084\\
66.875	0.1026	-121.423218631493\\
66.875	0.1032	-130.583640602904\\
66.875	0.1038	-139.744062574313\\
66.875	0.1044	-148.904484545723\\
66.875	0.105	-158.064906517133\\
66.875	0.1056	-167.225328488543\\
66.875	0.1062	-176.385750459952\\
66.875	0.1068	-185.546172431363\\
66.875	0.1074	-194.706594402772\\
66.875	0.108	-203.867016374182\\
66.875	0.1086	-213.027438345592\\
66.875	0.1092	-222.187860317002\\
66.875	0.1098	-231.348282288412\\
66.875	0.1104	-240.508704259822\\
66.875	0.111	-249.669126231231\\
66.875	0.1116	-258.829548202641\\
66.875	0.1122	-267.98997017405\\
66.875	0.1128	-277.150392145461\\
66.875	0.1134	-286.31081411687\\
66.875	0.114	-295.47123608828\\
66.875	0.1146	-304.631658059689\\
66.875	0.1152	-313.7920800311\\
66.875	0.1158	-322.952502002508\\
66.875	0.1164	-332.112923973919\\
66.875	0.117	-341.273345945328\\
66.875	0.1176	-350.433767916738\\
66.875	0.1182	-359.594189888148\\
66.875	0.1188	-368.754611859559\\
66.875	0.1194	-377.915033830968\\
66.875	0.12	-387.075455802378\\
66.875	0.1206	-396.235877773789\\
66.875	0.1212	-405.396299745197\\
66.875	0.1218	-414.556721716608\\
66.875	0.1224	-423.717143688017\\
66.875	0.123	-432.877565659427\\
67.25	0.093	28.5514856361297\\
67.25	0.0936	19.5983074370315\\
67.25	0.0942	10.6451292379325\\
67.25	0.0948	1.6919510388343\\
67.25	0.0954	-7.26122716026475\\
67.25	0.096	-16.2144053593638\\
67.25	0.0966	-25.167583558462\\
67.25	0.0972	-34.1207617575601\\
67.25	0.0978	-43.0739399566592\\
67.25	0.0984	-52.0271181557573\\
67.25	0.099	-60.9802963548555\\
67.25	0.0996	-69.9334745539545\\
67.25	0.1002	-78.8866527530527\\
67.25	0.1008	-87.8398309521517\\
67.25	0.1014	-96.7930091512499\\
67.25	0.102	-105.746187350349\\
67.25	0.1026	-114.699365549447\\
67.25	0.1032	-123.652543748546\\
67.25	0.1038	-132.605721947644\\
67.25	0.1044	-141.558900146743\\
67.25	0.105	-150.512078345841\\
67.25	0.1056	-159.465256544941\\
67.25	0.1062	-168.418434744039\\
67.25	0.1068	-177.371612943137\\
67.25	0.1074	-186.324791142235\\
67.25	0.108	-195.277969341334\\
67.25	0.1086	-204.231147540433\\
67.25	0.1092	-213.184325739531\\
67.25	0.1098	-222.13750393863\\
67.25	0.1104	-231.090682137729\\
67.25	0.111	-240.043860336827\\
67.25	0.1116	-248.997038535926\\
67.25	0.1122	-257.950216735024\\
67.25	0.1128	-266.903394934123\\
67.25	0.1134	-275.856573133222\\
67.25	0.114	-284.809751332321\\
67.25	0.1146	-293.762929531418\\
67.25	0.1152	-302.716107730517\\
67.25	0.1158	-311.669285929614\\
67.25	0.1164	-320.622464128714\\
67.25	0.117	-329.575642327812\\
67.25	0.1176	-338.528820526911\\
67.25	0.1182	-347.48199872601\\
67.25	0.1188	-356.435176925109\\
67.25	0.1194	-365.388355124207\\
67.25	0.12	-374.341533323306\\
67.25	0.1206	-383.294711522405\\
67.25	0.1212	-392.247889721502\\
67.25	0.1218	-401.201067920602\\
67.25	0.1224	-410.1542461197\\
67.25	0.123	-419.107424318799\\
67.625	0.093	31.9594383611966\\
67.625	0.0936	23.2135039344103\\
67.625	0.0942	14.4675695076221\\
67.625	0.0948	5.72163508083577\\
67.625	0.0954	-3.02429934595239\\
67.625	0.096	-11.7702337727405\\
67.625	0.0966	-20.5161681995269\\
67.625	0.0972	-29.2621026263141\\
67.625	0.0978	-38.0080370531014\\
67.625	0.0984	-46.7539714798895\\
67.625	0.099	-55.4999059066758\\
67.625	0.0996	-64.245840333464\\
67.625	0.1002	-72.9917747602503\\
67.625	0.1008	-81.7377091870385\\
67.625	0.1014	-90.4836436138248\\
67.625	0.102	-99.229578040613\\
67.625	0.1026	-107.9755124674\\
67.625	0.1032	-116.721446894187\\
67.625	0.1038	-125.467381320975\\
67.625	0.1044	-134.213315747762\\
67.625	0.105	-142.959250174549\\
67.625	0.1056	-151.705184601336\\
67.625	0.1062	-160.451119028124\\
67.625	0.1068	-169.197053454911\\
67.625	0.1074	-177.942987881698\\
67.625	0.108	-186.688922308485\\
67.625	0.1086	-195.434856735274\\
67.625	0.1092	-204.180791162061\\
67.625	0.1098	-212.926725588848\\
67.625	0.1104	-221.672660015636\\
67.625	0.111	-230.418594442422\\
67.625	0.1116	-239.16452886921\\
67.625	0.1122	-247.910463295996\\
67.625	0.1128	-256.656397722784\\
67.625	0.1134	-265.402332149572\\
67.625	0.114	-274.14826657636\\
67.625	0.1146	-282.894201003146\\
67.625	0.1152	-291.640135429934\\
67.625	0.1158	-300.38606985672\\
67.625	0.1164	-309.132004283508\\
67.625	0.117	-317.877938710295\\
67.625	0.1176	-326.623873137082\\
67.625	0.1182	-335.36980756387\\
67.625	0.1188	-344.115741990658\\
67.625	0.1194	-352.861676417445\\
67.625	0.12	-361.607610844232\\
67.625	0.1206	-370.35354527102\\
67.625	0.1212	-379.099479697807\\
67.625	0.1218	-387.845414124595\\
67.625	0.1224	-396.591348551381\\
67.625	0.123	-405.337282978168\\
68	0.093	35.3673910862626\\
68	0.0936	26.8287004317872\\
68	0.0942	18.2900097773108\\
68	0.0948	9.75131912283541\\
68	0.0954	1.21262846835907\\
68	0.096	-7.32606218611727\\
68	0.0966	-15.8647528405936\\
68	0.0972	-24.403443495069\\
68	0.0978	-32.9421341495454\\
68	0.0984	-41.4808248040217\\
68	0.099	-50.0195154584972\\
68	0.0996	-58.5582061129735\\
68	0.1002	-67.0968967674498\\
68	0.1008	-75.6355874219262\\
68	0.1014	-84.1742780764016\\
68	0.102	-92.7129687308779\\
68	0.1026	-101.251659385353\\
68	0.1032	-109.79035003983\\
68	0.1038	-118.329040694306\\
68	0.1044	-126.867731348782\\
68	0.105	-135.406422003258\\
68	0.1056	-143.945112657734\\
68	0.1062	-152.48380331221\\
68	0.1068	-161.022493966686\\
68	0.1074	-169.561184621161\\
68	0.108	-178.099875275639\\
68	0.1086	-186.638565930115\\
68	0.1092	-195.17725658459\\
68	0.1098	-203.715947239067\\
68	0.1104	-212.254637893543\\
68	0.111	-220.793328548018\\
68	0.1116	-229.332019202495\\
68	0.1122	-237.87070985697\\
68	0.1128	-246.409400511447\\
68	0.1134	-254.948091165923\\
68	0.114	-263.486781820399\\
68	0.1146	-272.025472474875\\
68	0.1152	-280.564163129352\\
68	0.1158	-289.102853783827\\
68	0.1164	-297.641544438303\\
68	0.117	-306.180235092778\\
68	0.1176	-314.718925747255\\
68	0.1182	-323.257616401731\\
68	0.1188	-331.796307056208\\
68	0.1194	-340.334997710684\\
68	0.12	-348.87368836516\\
68	0.1206	-357.412379019636\\
68	0.1212	-365.951069674111\\
68	0.1218	-374.489760328588\\
68	0.1224	-383.028450983064\\
68	0.123	-391.56714163754\\
68.375	0.093	38.7753438113305\\
68.375	0.0936	30.4438969291659\\
68.375	0.0942	22.1124500470005\\
68.375	0.0948	13.781003164836\\
68.375	0.0954	5.44955628267144\\
68.375	0.096	-2.881890599494\\
68.375	0.0966	-11.2133374816585\\
68.375	0.0972	-19.5447843638231\\
68.375	0.0978	-27.8762312459885\\
68.375	0.0984	-36.207678128153\\
68.375	0.099	-44.5391250103175\\
68.375	0.0996	-52.870571892483\\
68.375	0.1002	-61.2020187746475\\
68.375	0.1008	-69.533465656812\\
68.375	0.1014	-77.8649125389766\\
68.375	0.102	-86.196359421142\\
68.375	0.1026	-94.5278063033065\\
68.375	0.1032	-102.859253185472\\
68.375	0.1038	-111.190700067636\\
68.375	0.1044	-119.522146949801\\
68.375	0.105	-127.853593831966\\
68.375	0.1056	-136.185040714131\\
68.375	0.1062	-144.516487596295\\
68.375	0.1068	-152.84793447846\\
68.375	0.1074	-161.179381360625\\
68.375	0.108	-169.51082824279\\
68.375	0.1086	-177.842275124955\\
68.375	0.1092	-186.173722007119\\
68.375	0.1098	-194.505168889285\\
68.375	0.1104	-202.83661577145\\
68.375	0.111	-211.168062653614\\
68.375	0.1116	-219.499509535779\\
68.375	0.1122	-227.830956417943\\
68.375	0.1128	-236.162403300108\\
68.375	0.1134	-244.493850182273\\
68.375	0.114	-252.825297064439\\
68.375	0.1146	-261.156743946602\\
68.375	0.1152	-269.488190828768\\
68.375	0.1158	-277.819637710932\\
68.375	0.1164	-286.151084593097\\
68.375	0.117	-294.482531475262\\
68.375	0.1176	-302.813978357426\\
68.375	0.1182	-311.145425239592\\
68.375	0.1188	-319.476872121757\\
68.375	0.1194	-327.808319003921\\
68.375	0.12	-336.139765886086\\
68.375	0.1206	-344.471212768251\\
68.375	0.1212	-352.802659650415\\
68.375	0.1218	-361.134106532581\\
68.375	0.1224	-369.465553414745\\
68.375	0.123	-377.79700029691\\
68.75	0.093	42.1832965363965\\
68.75	0.0936	34.0590934265438\\
68.75	0.0942	25.9348903166892\\
68.75	0.0948	17.8106872068365\\
68.75	0.0954	9.68648409698199\\
68.75	0.096	1.56228098712836\\
68.75	0.0966	-6.56192212272435\\
68.75	0.0972	-14.686125232578\\
68.75	0.0978	-22.8103283424316\\
68.75	0.0984	-30.9345314522861\\
68.75	0.099	-39.0587345621389\\
68.75	0.0996	-47.1829376719925\\
68.75	0.1002	-55.3071407818461\\
68.75	0.1008	-63.4313438916997\\
68.75	0.1014	-71.5555470015534\\
68.75	0.102	-79.679750111407\\
68.75	0.1026	-87.8039532212597\\
68.75	0.1032	-95.9281563311142\\
68.75	0.1038	-104.052359440967\\
68.75	0.1044	-112.176562550821\\
68.75	0.105	-120.300765660674\\
68.75	0.1056	-128.424968770528\\
68.75	0.1062	-136.549171880381\\
68.75	0.1068	-144.673374990235\\
68.75	0.1074	-152.797578100089\\
68.75	0.108	-160.921781209942\\
68.75	0.1086	-169.045984319796\\
68.75	0.1092	-177.17018742965\\
68.75	0.1098	-185.294390539503\\
68.75	0.1104	-193.418593649358\\
68.75	0.111	-201.54279675921\\
68.75	0.1116	-209.666999869063\\
68.75	0.1122	-217.791202978917\\
68.75	0.1128	-225.91540608877\\
68.75	0.1134	-234.039609198625\\
68.75	0.114	-242.163812308479\\
68.75	0.1146	-250.28801541833\\
68.75	0.1152	-258.412218528186\\
68.75	0.1158	-266.536421638038\\
68.75	0.1164	-274.660624747892\\
68.75	0.117	-282.784827857745\\
68.75	0.1176	-290.909030967598\\
68.75	0.1182	-299.033234077453\\
68.75	0.1188	-307.157437187307\\
68.75	0.1194	-315.28164029716\\
68.75	0.12	-323.405843407014\\
68.75	0.1206	-331.530046516868\\
68.75	0.1212	-339.65424962672\\
68.75	0.1218	-347.778452736575\\
68.75	0.1224	-355.902655846427\\
68.75	0.123	-364.026858956281\\
69.125	0.093	45.5912492614634\\
69.125	0.0936	37.6742899239216\\
69.125	0.0942	29.757330586378\\
69.125	0.0948	21.8403712488362\\
69.125	0.0954	13.9234119112934\\
69.125	0.096	6.00645257375072\\
69.125	0.0966	-1.91050676379109\\
69.125	0.0972	-9.8274661013329\\
69.125	0.0978	-17.7444254388756\\
69.125	0.0984	-25.6613847764183\\
69.125	0.099	-33.5783441139602\\
69.125	0.0996	-41.4953034515029\\
69.125	0.1002	-49.4122627890447\\
69.125	0.1008	-57.3292221265874\\
69.125	0.1014	-65.2461814641292\\
69.125	0.102	-73.1631408016719\\
69.125	0.1026	-81.0801001392138\\
69.125	0.1032	-88.9970594767565\\
69.125	0.1038	-96.9140188142983\\
69.125	0.1044	-104.830978151841\\
69.125	0.105	-112.747937489383\\
69.125	0.1056	-120.664896826926\\
69.125	0.1062	-128.581856164467\\
69.125	0.1068	-136.49881550201\\
69.125	0.1074	-144.415774839552\\
69.125	0.108	-152.332734177095\\
69.125	0.1086	-160.249693514637\\
69.125	0.1092	-168.166652852179\\
69.125	0.1098	-176.083612189722\\
69.125	0.1104	-184.000571527265\\
69.125	0.111	-191.917530864805\\
69.125	0.1116	-199.834490202348\\
69.125	0.1122	-207.75144953989\\
69.125	0.1128	-215.668408877433\\
69.125	0.1134	-223.585368214975\\
69.125	0.114	-231.502327552518\\
69.125	0.1146	-239.419286890059\\
69.125	0.1152	-247.336246227603\\
69.125	0.1158	-255.253205565144\\
69.125	0.1164	-263.170164902686\\
69.125	0.117	-271.087124240228\\
69.125	0.1176	-279.004083577771\\
69.125	0.1182	-286.921042915314\\
69.125	0.1188	-294.838002252856\\
69.125	0.1194	-302.754961590398\\
69.125	0.12	-310.671920927941\\
69.125	0.1206	-318.588880265484\\
69.125	0.1212	-326.505839603025\\
69.125	0.1218	-334.422798940569\\
69.125	0.1224	-342.33975827811\\
69.125	0.123	-350.256717615653\\
69.5	0.093	48.9992019865294\\
69.5	0.0936	41.2894864212985\\
69.5	0.0942	33.5797708560676\\
69.5	0.0948	25.8700552908367\\
69.5	0.0954	18.1603397256049\\
69.5	0.096	10.450624160374\\
69.5	0.0966	2.74090859514308\\
69.5	0.0972	-4.96880697008783\\
69.5	0.0978	-12.6785225353187\\
69.5	0.0984	-20.3882381005506\\
69.5	0.099	-28.0979536657815\\
69.5	0.0996	-35.8076692310124\\
69.5	0.1002	-43.5173847962433\\
69.5	0.1008	-51.2271003614751\\
69.5	0.1014	-58.9368159267051\\
69.5	0.102	-66.6465314919369\\
69.5	0.1026	-74.3562470571669\\
69.5	0.1032	-82.0659626223987\\
69.5	0.1038	-89.7756781876296\\
69.5	0.1044	-97.4853937528605\\
69.5	0.105	-105.195109318091\\
69.5	0.1056	-112.904824883323\\
69.5	0.1062	-120.614540448553\\
69.5	0.1068	-128.324256013785\\
69.5	0.1074	-136.033971579016\\
69.5	0.108	-143.743687144247\\
69.5	0.1086	-151.453402709479\\
69.5	0.1092	-159.163118274709\\
69.5	0.1098	-166.872833839941\\
69.5	0.1104	-174.582549405172\\
69.5	0.111	-182.292264970401\\
69.5	0.1116	-190.001980535633\\
69.5	0.1122	-197.711696100864\\
69.5	0.1128	-205.421411666095\\
69.5	0.1134	-213.131127231327\\
69.5	0.114	-220.840842796559\\
69.5	0.1146	-228.550558361788\\
69.5	0.1152	-236.260273927021\\
69.5	0.1158	-243.969989492251\\
69.5	0.1164	-251.679705057481\\
69.5	0.117	-259.389420622712\\
69.5	0.1176	-267.099136187943\\
69.5	0.1182	-274.808851753175\\
69.5	0.1188	-282.518567318407\\
69.5	0.1194	-290.228282883637\\
69.5	0.12	-297.937998448869\\
69.5	0.1206	-305.647714014101\\
69.5	0.1212	-313.35742957933\\
69.5	0.1218	-321.067145144562\\
69.5	0.1224	-328.776860709792\\
69.5	0.123	-336.486576275022\\
69.875	0.093	52.4071547115964\\
69.875	0.0936	44.9046829186773\\
69.875	0.0942	37.4022111257573\\
69.875	0.0948	29.8997393328373\\
69.875	0.0954	22.3972675399173\\
69.875	0.096	14.8947957469973\\
69.875	0.0966	7.39232395407817\\
69.875	0.0972	-0.11014783884184\\
69.875	0.0978	-7.61261963176185\\
69.875	0.0984	-15.1150914246819\\
69.875	0.099	-22.6175632176019\\
69.875	0.0996	-30.1200350105219\\
69.875	0.1002	-37.622506803441\\
69.875	0.1008	-45.124978596361\\
69.875	0.1014	-52.627450389281\\
69.875	0.102	-60.129922182201\\
69.875	0.1026	-67.6323939751201\\
69.875	0.1032	-75.1348657680401\\
69.875	0.1038	-82.6373375609601\\
69.875	0.1044	-90.1398093538801\\
69.875	0.105	-97.6422811467992\\
69.875	0.1056	-105.144752939719\\
69.875	0.1062	-112.647224732639\\
69.875	0.1068	-120.149696525559\\
69.875	0.1074	-127.652168318478\\
69.875	0.108	-135.154640111398\\
69.875	0.1086	-142.657111904319\\
69.875	0.1092	-150.159583697238\\
69.875	0.1098	-157.662055490158\\
69.875	0.1104	-165.164527283078\\
69.875	0.111	-172.666999075997\\
69.875	0.1116	-180.169470868917\\
69.875	0.1122	-187.671942661837\\
69.875	0.1128	-195.174414454757\\
69.875	0.1134	-202.676886247677\\
69.875	0.114	-210.179358040597\\
69.875	0.1146	-217.681829833516\\
69.875	0.1152	-225.184301626437\\
69.875	0.1158	-232.686773419356\\
69.875	0.1164	-240.189245212276\\
69.875	0.117	-247.691717005195\\
69.875	0.1176	-255.194188798115\\
69.875	0.1182	-262.696660591036\\
69.875	0.1188	-270.199132383956\\
69.875	0.1194	-277.701604176875\\
69.875	0.12	-285.204075969796\\
69.875	0.1206	-292.706547762716\\
69.875	0.1212	-300.209019555634\\
69.875	0.1218	-307.711491348555\\
69.875	0.1224	-315.213963141474\\
69.875	0.123	-322.716434934394\\
70.25	0.093	55.8151074366624\\
70.25	0.0936	48.5198794160542\\
70.25	0.0942	41.224651395446\\
70.25	0.0948	33.9294233748378\\
70.25	0.0954	26.6341953542287\\
70.25	0.096	19.3389673336196\\
70.25	0.0966	12.0437393130114\\
70.25	0.0972	4.74851129240324\\
70.25	0.0978	-2.54671672820587\\
70.25	0.0984	-9.84194474881497\\
70.25	0.099	-17.1371727694222\\
70.25	0.0996	-24.4324007900314\\
70.25	0.1002	-31.7276288106395\\
70.25	0.1008	-39.0228568312486\\
70.25	0.1014	-46.3180848518568\\
70.25	0.102	-53.6133128724659\\
70.25	0.1026	-60.9085408930741\\
70.25	0.1032	-68.2037689136832\\
70.25	0.1038	-75.4989969342905\\
70.25	0.1044	-82.7942249548996\\
70.25	0.105	-90.0894529755078\\
70.25	0.1056	-97.3846809961169\\
70.25	0.1062	-104.679909016725\\
70.25	0.1068	-111.975137037334\\
70.25	0.1074	-119.270365057942\\
70.25	0.108	-126.565593078552\\
70.25	0.1086	-133.86082109916\\
70.25	0.1092	-141.156049119768\\
70.25	0.1098	-148.451277140377\\
70.25	0.1104	-155.746505160986\\
70.25	0.111	-163.041733181593\\
70.25	0.1116	-170.336961202202\\
70.25	0.1122	-177.632189222811\\
70.25	0.1128	-184.92741724342\\
70.25	0.1134	-192.222645264028\\
70.25	0.114	-199.517873284637\\
70.25	0.1146	-206.813101305244\\
70.25	0.1152	-214.108329325854\\
70.25	0.1158	-221.403557346462\\
70.25	0.1164	-228.698785367071\\
70.25	0.117	-235.994013387679\\
70.25	0.1176	-243.289241408288\\
70.25	0.1182	-250.584469428896\\
70.25	0.1188	-257.879697449505\\
70.25	0.1194	-265.174925470114\\
70.25	0.12	-272.470153490723\\
70.25	0.1206	-279.765381511332\\
70.25	0.1212	-287.060609531939\\
70.25	0.1218	-294.355837552549\\
70.25	0.1224	-301.651065573156\\
70.25	0.123	-308.946293593764\\
70.625	0.093	59.2230601617302\\
70.625	0.0936	52.135075913433\\
70.625	0.0942	45.0470916651357\\
70.625	0.0948	37.9591074168384\\
70.625	0.0954	30.8711231685411\\
70.625	0.096	23.7831389202429\\
70.625	0.0966	16.6951546719465\\
70.625	0.0972	9.60717042364922\\
70.625	0.0978	2.51918617535193\\
70.625	0.0984	-4.56879807294627\\
70.625	0.099	-11.6567823212426\\
70.625	0.0996	-18.7447665695408\\
70.625	0.1002	-25.8327508178372\\
70.625	0.1008	-32.9207350661354\\
70.625	0.1014	-40.0087193144318\\
70.625	0.102	-47.09670356273\\
70.625	0.1026	-54.1846878110264\\
70.625	0.1032	-61.2726720593246\\
70.625	0.1038	-68.360656307621\\
70.625	0.1044	-75.4486405559192\\
70.625	0.105	-82.5366248042155\\
70.625	0.1056	-89.6246090525137\\
70.625	0.1062	-96.7125933008101\\
70.625	0.1068	-103.800577549108\\
70.625	0.1074	-110.888561797405\\
70.625	0.108	-117.976546045703\\
70.625	0.1086	-125.064530294\\
70.625	0.1092	-132.152514542297\\
70.625	0.1098	-139.240498790595\\
70.625	0.1104	-146.328483038893\\
70.625	0.111	-153.416467287188\\
70.625	0.1116	-160.504451535487\\
70.625	0.1122	-167.592435783783\\
70.625	0.1128	-174.680420032081\\
70.625	0.1134	-181.768404280378\\
70.625	0.114	-188.856388528677\\
70.625	0.1146	-195.944372776972\\
70.625	0.1152	-203.032357025271\\
70.625	0.1158	-210.120341273567\\
70.625	0.1164	-217.208325521865\\
70.625	0.117	-224.296309770161\\
70.625	0.1176	-231.38429401846\\
70.625	0.1182	-238.472278266757\\
70.625	0.1188	-245.560262515055\\
70.625	0.1194	-252.648246763351\\
70.625	0.12	-259.73623101165\\
70.625	0.1206	-266.824215259947\\
70.625	0.1212	-273.912199508243\\
70.625	0.1218	-281.000183756542\\
70.625	0.1224	-288.088168004839\\
70.625	0.123	-295.176152253134\\
71	0.093	62.6310128867963\\
71	0.0936	55.7502724108108\\
71	0.0942	48.8695319348244\\
71	0.0948	41.9887914588389\\
71	0.0954	35.1080509828516\\
71	0.096	28.2273105068653\\
71	0.0966	21.3465700308798\\
71	0.0972	14.4658295548943\\
71	0.0978	7.58508907890791\\
71	0.0984	0.704348602921527\\
71	0.099	-6.17639187306395\\
71	0.0996	-13.0571323490503\\
71	0.1002	-19.9378728250358\\
71	0.1008	-26.8186133010231\\
71	0.1014	-33.6993537770086\\
71	0.102	-40.580094252995\\
71	0.1026	-47.4608347289804\\
71	0.1032	-54.3415752049668\\
71	0.1038	-61.2223156809523\\
71	0.1044	-68.1030561569387\\
71	0.105	-74.9837966329242\\
71	0.1056	-81.8645371089106\\
71	0.1062	-88.7452775848969\\
71	0.1068	-95.6260180608833\\
71	0.1074	-102.506758536869\\
71	0.108	-109.387499012855\\
71	0.1086	-116.268239488842\\
71	0.1092	-123.148979964827\\
71	0.1098	-130.029720440813\\
71	0.1104	-136.9104609168\\
71	0.111	-143.791201392784\\
71	0.1116	-150.671941868772\\
71	0.1122	-157.552682344757\\
71	0.1128	-164.433422820744\\
71	0.1134	-171.31416329673\\
71	0.114	-178.194903772716\\
71	0.1146	-185.075644248701\\
71	0.1152	-191.956384724688\\
71	0.1158	-198.837125200673\\
71	0.1164	-205.717865676659\\
71	0.117	-212.598606152646\\
71	0.1176	-219.479346628632\\
71	0.1182	-226.360087104618\\
71	0.1188	-233.240827580605\\
71	0.1194	-240.12156805659\\
71	0.12	-247.002308532577\\
71	0.1206	-253.883049008563\\
71	0.1212	-260.763789484547\\
71	0.1218	-267.644529960534\\
71	0.1224	-274.525270436519\\
71	0.123	-281.406010912508\\
71.375	0.093	66.0389656118623\\
71.375	0.0936	59.3654689081886\\
71.375	0.0942	52.6919722045131\\
71.375	0.0948	46.0184755008386\\
71.375	0.0954	39.3449787971631\\
71.375	0.096	32.6714820934885\\
71.375	0.0966	25.9979853898139\\
71.375	0.0972	19.3244886861394\\
71.375	0.0978	12.6509919824648\\
71.375	0.0984	5.97749527878932\\
71.375	0.099	-0.696001424885253\\
71.375	0.0996	-7.36949812856074\\
71.375	0.1002	-14.0429948322344\\
71.375	0.1008	-20.7164915359099\\
71.375	0.1014	-27.3899882395845\\
71.375	0.102	-34.0634849432599\\
71.375	0.1026	-40.7369816469336\\
71.375	0.1032	-47.4104783506091\\
71.375	0.1038	-54.0839750542837\\
71.375	0.1044	-60.7574717579591\\
71.375	0.105	-67.4309684616328\\
71.375	0.1056	-74.1044651653083\\
71.375	0.1062	-80.7779618689829\\
71.375	0.1068	-87.4514585726574\\
71.375	0.1074	-94.124955276332\\
71.375	0.108	-100.798451980007\\
71.375	0.1086	-107.471948683683\\
71.375	0.1092	-114.145445387357\\
71.375	0.1098	-120.818942091032\\
71.375	0.1104	-127.492438794708\\
71.375	0.111	-134.165935498381\\
71.375	0.1116	-140.839432202056\\
71.375	0.1122	-147.51292890573\\
71.375	0.1128	-154.186425609406\\
71.375	0.1134	-160.85992231308\\
71.375	0.114	-167.533419016756\\
71.375	0.1146	-174.20691572043\\
71.375	0.1152	-180.880412424106\\
71.375	0.1158	-187.553909127779\\
71.375	0.1164	-194.227405831454\\
71.375	0.117	-200.900902535129\\
71.375	0.1176	-207.574399238804\\
71.375	0.1182	-214.247895942479\\
71.375	0.1188	-220.921392646154\\
71.375	0.1194	-227.594889349829\\
71.375	0.12	-234.268386053504\\
71.375	0.1206	-240.941882757179\\
71.375	0.1212	-247.615379460853\\
71.375	0.1218	-254.288876164529\\
71.375	0.1224	-260.962372868202\\
71.375	0.123	-267.635869571877\\
71.75	0.093	69.4469183369292\\
71.75	0.0936	62.9806654055665\\
71.75	0.0942	56.5144124742028\\
71.75	0.0948	50.0481595428391\\
71.75	0.0954	43.5819066114755\\
71.75	0.096	37.1156536801118\\
71.75	0.0966	30.6494007487481\\
71.75	0.0972	24.1831478173854\\
71.75	0.0978	17.7168948860217\\
71.75	0.0984	11.2506419546571\\
71.75	0.099	4.78438902329435\\
71.75	0.0996	-1.68186390806932\\
71.75	0.1002	-8.14811683943299\\
71.75	0.1008	-14.6143697707967\\
71.75	0.1014	-21.0806227021594\\
71.75	0.102	-27.546875633524\\
71.75	0.1026	-34.0131285648868\\
71.75	0.1032	-40.4793814962504\\
71.75	0.1038	-46.9456344276141\\
71.75	0.1044	-53.4118873589778\\
71.75	0.105	-59.8781402903405\\
71.75	0.1056	-66.3443932217051\\
71.75	0.1062	-72.8106461530679\\
71.75	0.1068	-79.2768990844324\\
71.75	0.1074	-85.7431520157952\\
71.75	0.108	-92.2094049471589\\
71.75	0.1086	-98.6756578785235\\
71.75	0.1092	-105.141910809886\\
71.75	0.1098	-111.60816374125\\
71.75	0.1104	-118.074416672614\\
71.75	0.111	-124.540669603976\\
71.75	0.1116	-131.00692253534\\
71.75	0.1122	-137.473175466704\\
71.75	0.1128	-143.939428398067\\
71.75	0.1134	-150.405681329431\\
71.75	0.114	-156.871934260796\\
71.75	0.1146	-163.338187192157\\
71.75	0.1152	-169.804440123522\\
71.75	0.1158	-176.270693054885\\
71.75	0.1164	-182.736945986248\\
71.75	0.117	-189.203198917611\\
71.75	0.1176	-195.669451848976\\
71.75	0.1182	-202.135704780339\\
71.75	0.1188	-208.601957711703\\
71.75	0.1194	-215.068210643067\\
71.75	0.12	-221.53446357443\\
71.75	0.1206	-228.000716505794\\
71.75	0.1212	-234.466969437157\\
71.75	0.1218	-240.933222368521\\
71.75	0.1224	-247.399475299884\\
71.75	0.123	-253.865728231248\\
72.125	0.093	72.8548710619962\\
72.125	0.0936	66.5958619029443\\
72.125	0.0942	60.3368527438915\\
72.125	0.0948	54.0778435848397\\
72.125	0.0954	47.8188344257869\\
72.125	0.096	41.5598252667342\\
72.125	0.0966	35.3008161076823\\
72.125	0.0972	29.0418069486304\\
72.125	0.0978	22.7827977895777\\
72.125	0.0984	16.5237886305249\\
72.125	0.099	10.264779471473\\
72.125	0.0996	4.00577031242028\\
72.125	0.1002	-2.25323884663158\\
72.125	0.1008	-8.51224800568434\\
72.125	0.1014	-14.7712571647362\\
72.125	0.102	-21.030266323789\\
72.125	0.1026	-27.2892754828408\\
72.125	0.1032	-33.5482846418927\\
72.125	0.1038	-39.8072938009445\\
72.125	0.1044	-46.0663029599973\\
72.125	0.105	-52.3253121190492\\
72.125	0.1056	-58.5843212781019\\
72.125	0.1062	-64.8433304371538\\
72.125	0.1068	-71.1023395962065\\
72.125	0.1074	-77.3613487552584\\
72.125	0.108	-83.6203579143112\\
72.125	0.1086	-89.8793670733639\\
72.125	0.1092	-96.1383762324158\\
72.125	0.1098	-102.397385391469\\
72.125	0.1104	-108.656394550521\\
72.125	0.111	-114.915403709572\\
72.125	0.1116	-121.174412868625\\
72.125	0.1122	-127.433422027677\\
72.125	0.1128	-133.69243118673\\
72.125	0.1134	-139.951440345782\\
72.125	0.114	-146.210449504835\\
72.125	0.1146	-152.469458663886\\
72.125	0.1152	-158.72846782294\\
72.125	0.1158	-164.987476981991\\
72.125	0.1164	-171.246486141044\\
72.125	0.117	-177.505495300095\\
72.125	0.1176	-183.764504459148\\
72.125	0.1182	-190.023513618201\\
72.125	0.1188	-196.282522777254\\
72.125	0.1194	-202.541531936306\\
72.125	0.12	-208.800541095358\\
72.125	0.1206	-215.059550254412\\
72.125	0.1212	-221.318559413463\\
72.125	0.1218	-227.577568572515\\
72.125	0.1224	-233.836577731565\\
72.125	0.123	-240.095586890619\\
72.5	0.093	76.2628237870631\\
72.5	0.0936	70.2110584003221\\
72.5	0.0942	64.1592930135812\\
72.5	0.0948	58.1075276268402\\
72.5	0.0954	52.0557622400993\\
72.5	0.096	46.0039968533574\\
72.5	0.0966	39.9522314666165\\
72.5	0.0972	33.9004660798764\\
72.5	0.0978	27.8487006931346\\
72.5	0.0984	21.7969353063936\\
72.5	0.099	15.7451699196527\\
72.5	0.0996	9.6934045329117\\
72.5	0.1002	3.64163914617075\\
72.5	0.1008	-2.41012624057112\\
72.5	0.1014	-8.46189162731116\\
72.5	0.102	-14.513657014053\\
72.5	0.1026	-20.5654224007931\\
72.5	0.1032	-26.6171877875349\\
72.5	0.1038	-32.668953174275\\
72.5	0.1044	-38.7207185610168\\
72.5	0.105	-44.7724839477578\\
72.5	0.1056	-50.8242493344987\\
72.5	0.1062	-56.8760147212397\\
72.5	0.1068	-62.9277801079807\\
72.5	0.1074	-68.9795454947216\\
72.5	0.108	-75.0313108814635\\
72.5	0.1086	-81.0830762682044\\
72.5	0.1092	-87.1348416549454\\
72.5	0.1098	-93.1866070416863\\
72.5	0.1104	-99.2383724284282\\
72.5	0.111	-105.290137815167\\
72.5	0.1116	-111.341903201909\\
72.5	0.1122	-117.39366858865\\
72.5	0.1128	-123.445433975391\\
72.5	0.1134	-129.497199362133\\
72.5	0.114	-135.548964748874\\
72.5	0.1146	-141.600730135614\\
72.5	0.1152	-147.652495522357\\
72.5	0.1158	-153.704260909096\\
72.5	0.1164	-159.756026295838\\
72.5	0.117	-165.807791682578\\
72.5	0.1176	-171.85955706932\\
72.5	0.1182	-177.911322456061\\
72.5	0.1188	-183.963087842802\\
72.5	0.1194	-190.014853229543\\
72.5	0.12	-196.066618616285\\
72.5	0.1206	-202.118384003025\\
72.5	0.1212	-208.170149389764\\
72.5	0.1218	-214.221914776509\\
72.5	0.1224	-220.273680163248\\
72.5	0.123	-226.325445549988\\
72.875	0.093	79.67077651213\\
72.875	0.0936	73.8262548977\\
72.875	0.0942	67.9817332832699\\
72.875	0.0948	62.1372116688408\\
72.875	0.0954	56.2926900544107\\
72.875	0.096	50.4481684399798\\
72.875	0.0966	44.6036468255506\\
72.875	0.0972	38.7591252111215\\
72.875	0.0978	32.9146035966914\\
72.875	0.0984	27.0700819822614\\
72.875	0.099	21.2255603678313\\
72.875	0.0996	15.3810387534013\\
72.875	0.1002	9.53651713897216\\
72.875	0.1008	3.69199552454211\\
72.875	0.1014	-2.15252608988703\\
72.875	0.102	-7.99704770431799\\
72.875	0.1026	-13.8415693187471\\
72.875	0.1032	-19.6860909331772\\
72.875	0.1038	-25.5306125476063\\
72.875	0.1044	-31.3751341620364\\
72.875	0.105	-37.2196557764664\\
72.875	0.1056	-43.0641773908965\\
72.875	0.1062	-48.9086990053256\\
72.875	0.1068	-54.7532206197557\\
72.875	0.1074	-60.5977422341848\\
72.875	0.108	-66.4422638486158\\
72.875	0.1086	-72.2867854630458\\
72.875	0.1092	-78.131307077475\\
72.875	0.1098	-83.975828691905\\
72.875	0.1104	-89.8203503063351\\
72.875	0.111	-95.6648719207642\\
72.875	0.1116	-101.509393535194\\
72.875	0.1122	-107.353915149623\\
72.875	0.1128	-113.198436764053\\
72.875	0.1134	-119.042958378483\\
72.875	0.114	-124.887479992914\\
72.875	0.1146	-130.732001607343\\
72.875	0.1152	-136.576523221774\\
72.875	0.1158	-142.421044836202\\
72.875	0.1164	-148.265566450632\\
72.875	0.117	-154.110088065062\\
72.875	0.1176	-159.954609679492\\
72.875	0.1182	-165.799131293922\\
72.875	0.1188	-171.643652908352\\
72.875	0.1194	-177.488174522781\\
72.875	0.12	-183.332696137212\\
72.875	0.1206	-189.177217751642\\
72.875	0.1212	-195.021739366071\\
72.875	0.1218	-200.866260980501\\
72.875	0.1224	-206.71078259493\\
72.875	0.123	-212.55530420936\\
73.25	0.093	83.078729237197\\
73.25	0.0936	77.4414513950787\\
73.25	0.0942	71.8041735529596\\
73.25	0.0948	66.1668957108413\\
73.25	0.0954	60.5296178687222\\
73.25	0.096	54.8923400266031\\
73.25	0.0966	49.2550621844857\\
73.25	0.0972	43.6177843423675\\
73.25	0.0978	37.9805065002483\\
73.25	0.0984	32.3432286581292\\
73.25	0.099	26.705950816011\\
73.25	0.0996	21.0686729738927\\
73.25	0.1002	15.4313951317745\\
73.25	0.1008	9.79411728965533\\
73.25	0.1014	4.1568394475371\\
73.25	0.102	-1.48043839458205\\
73.25	0.1026	-7.11771623670029\\
73.25	0.1032	-12.7549940788185\\
73.25	0.1038	-18.3922719209368\\
73.25	0.1044	-24.0295497630559\\
73.25	0.105	-29.6668276051741\\
73.25	0.1056	-35.3041054472933\\
73.25	0.1062	-40.9413832894106\\
73.25	0.1068	-46.5786611315298\\
73.25	0.1074	-52.215938973648\\
73.25	0.108	-57.8532168157672\\
73.25	0.1086	-63.4904946578863\\
73.25	0.1092	-69.1277725000045\\
73.25	0.1098	-74.7650503421228\\
73.25	0.1104	-80.4023281842419\\
73.25	0.111	-86.0396060263593\\
73.25	0.1116	-91.6768838684784\\
73.25	0.1122	-97.3141617105966\\
73.25	0.1128	-102.951439552715\\
73.25	0.1134	-108.588717394834\\
73.25	0.114	-114.225995236953\\
73.25	0.1146	-119.86327307907\\
73.25	0.1152	-125.500550921191\\
73.25	0.1158	-131.137828763307\\
73.25	0.1164	-136.775106605426\\
73.25	0.117	-142.412384447544\\
73.25	0.1176	-148.049662289664\\
73.25	0.1182	-153.686940131783\\
73.25	0.1188	-159.324217973902\\
73.25	0.1194	-164.961495816019\\
73.25	0.12	-170.598773658138\\
73.25	0.1206	-176.236051500257\\
73.25	0.1212	-181.873329342375\\
73.25	0.1218	-187.510607184495\\
73.25	0.1224	-193.147885026612\\
73.25	0.123	-198.785162868731\\
73.625	0.093	86.486681962263\\
73.625	0.0936	81.0566478924566\\
73.625	0.0942	75.6266138226483\\
73.625	0.0948	70.1965797528419\\
73.625	0.0954	64.7665456830337\\
73.625	0.096	59.3365116132263\\
73.625	0.0966	53.906477543419\\
73.625	0.0972	48.4764434736126\\
73.625	0.0978	43.0464094038052\\
73.625	0.0984	37.616375333997\\
73.625	0.099	32.1863412641906\\
73.625	0.0996	26.7563071943823\\
73.625	0.1002	21.3262731245759\\
73.625	0.1008	15.8962390547676\\
73.625	0.1014	10.4662049849612\\
73.625	0.102	5.03617091515298\\
73.625	0.1026	-0.393863154653445\\
73.625	0.1032	-5.82389722446078\\
73.625	0.1038	-11.2539312942681\\
73.625	0.1044	-16.6839653640754\\
73.625	0.105	-22.1139994338828\\
73.625	0.1056	-27.5440335036901\\
73.625	0.1062	-32.9740675734975\\
73.625	0.1068	-38.4041016433048\\
73.625	0.1074	-43.8341357131112\\
73.625	0.108	-49.2641697829195\\
73.625	0.1086	-54.6942038527268\\
73.625	0.1092	-60.1242379225341\\
73.625	0.1098	-65.5542719923415\\
73.625	0.1104	-70.9843060621497\\
73.625	0.111	-76.4143401319552\\
73.625	0.1116	-81.8443742017635\\
73.625	0.1122	-87.2744082715699\\
73.625	0.1128	-92.7044423413772\\
73.625	0.1134	-98.1344764111855\\
73.625	0.114	-103.564510480993\\
73.625	0.1146	-108.994544550799\\
73.625	0.1152	-114.424578620607\\
73.625	0.1158	-119.854612690414\\
73.625	0.1164	-125.284646760221\\
73.625	0.117	-130.714680830029\\
73.625	0.1176	-136.144714899836\\
73.625	0.1182	-141.574748969643\\
73.625	0.1188	-147.004783039451\\
73.625	0.1194	-152.434817109257\\
73.625	0.12	-157.864851179067\\
73.625	0.1206	-163.294885248873\\
73.625	0.1212	-168.724919318678\\
73.625	0.1218	-174.15495338849\\
73.625	0.1224	-179.584987458295\\
73.625	0.123	-185.015021528101\\
74	0.093	89.8946346873299\\
74	0.0936	84.6718443898344\\
74	0.0942	79.449054092338\\
74	0.0948	74.2262637948425\\
74	0.0954	69.003473497346\\
74	0.096	63.7806831998496\\
74	0.0966	58.5578929023541\\
74	0.0972	53.3351026048585\\
74	0.0978	48.1123123073621\\
74	0.0984	42.8895220098657\\
74	0.099	37.6667317123702\\
74	0.0996	32.4439414148728\\
74	0.1002	27.2211511173773\\
74	0.1008	21.9983608198809\\
74	0.1014	16.7755705223854\\
74	0.102	11.5527802248889\\
74	0.1026	6.3299899273934\\
74	0.1032	1.10719962989697\\
74	0.1038	-4.11559066759855\\
74	0.1044	-9.33838096509498\\
74	0.105	-14.5611712625905\\
74	0.1056	-19.7839615600869\\
74	0.1062	-25.0067518575825\\
74	0.1068	-30.2295421550789\\
74	0.1074	-35.4523324525744\\
74	0.108	-40.6751227500708\\
74	0.1086	-45.8979130475673\\
74	0.1092	-51.1207033450628\\
74	0.1098	-56.3434936425592\\
74	0.1104	-61.5662839400557\\
74	0.111	-66.7890742375503\\
74	0.1116	-72.0118645350476\\
74	0.1122	-77.2346548325431\\
74	0.1128	-82.4574451300396\\
74	0.1134	-87.680235427536\\
74	0.114	-92.9030257250324\\
74	0.1146	-98.125816022527\\
74	0.1152	-103.348606320024\\
74	0.1158	-108.571396617519\\
74	0.1164	-113.794186915015\\
74	0.117	-119.016977212511\\
74	0.1176	-124.239767510007\\
74	0.1182	-129.462557807504\\
74	0.1188	-134.685348105001\\
74	0.1194	-139.908138402497\\
74	0.12	-145.130928699991\\
74	0.1206	-150.35371899749\\
74	0.1212	-155.576509294984\\
74	0.1218	-160.79929959248\\
74	0.1224	-166.022089889975\\
74	0.123	-171.244880187473\\
};
\end{axis}

\begin{axis}[%
width=5.011742cm,
height=3.595207cm,
at={(0cm,0cm)},
scale only axis,
xmin=55,
xmax=75,
tick align=outside,
xlabel={$L_{cut}$},
xmajorgrids,
ymin=0.09,
ymax=0.13,
ylabel={$D_{rlx}$},
ymajorgrids,
zmin=-2532.82546903929,
zmax=436.68108977922,
zlabel={$c$},
zmajorgrids,
view={-140}{50},
legend style={at={(1.03,1)},anchor=north west,legend cell align=left,align=left,draw=white!15!black}
]
\addplot3[only marks,mark=*,mark options={},mark size=1.5000pt,color=mycolor1] plot table[row sep=crcr,]{%
74	0.123	-233.157966898601\\
72	0.113	-200.830593420783\\
61	0.095	-93.7391747783605\\
56	0.093	-119.696332564413\\
};
\addplot3[only marks,mark=*,mark options={},mark size=1.5000pt,color=black] plot table[row sep=crcr,]{%
69	0.104	-141.03317126945\\
};

\addplot3[%
surf,
opacity=0.7,
shader=interp,
colormap={mymap}{[1pt] rgb(0pt)=(0.0901961,0.239216,0.0745098); rgb(1pt)=(0.0945149,0.242058,0.0739522); rgb(2pt)=(0.0988592,0.244894,0.0733566); rgb(3pt)=(0.103229,0.247724,0.0727241); rgb(4pt)=(0.107623,0.250549,0.0720557); rgb(5pt)=(0.112043,0.253367,0.0713525); rgb(6pt)=(0.116487,0.25618,0.0706154); rgb(7pt)=(0.120956,0.258986,0.0698456); rgb(8pt)=(0.125449,0.261787,0.0690441); rgb(9pt)=(0.129967,0.264581,0.0682118); rgb(10pt)=(0.134508,0.26737,0.06735); rgb(11pt)=(0.139074,0.270152,0.0664596); rgb(12pt)=(0.143663,0.272929,0.0655416); rgb(13pt)=(0.148275,0.275699,0.0645971); rgb(14pt)=(0.152911,0.278463,0.0636271); rgb(15pt)=(0.15757,0.281221,0.0626328); rgb(16pt)=(0.162252,0.283973,0.0616151); rgb(17pt)=(0.166957,0.286719,0.060575); rgb(18pt)=(0.171685,0.289458,0.0595136); rgb(19pt)=(0.176434,0.292191,0.0584321); rgb(20pt)=(0.181207,0.294918,0.0573313); rgb(21pt)=(0.186001,0.297639,0.0562123); rgb(22pt)=(0.190817,0.300353,0.0550763); rgb(23pt)=(0.195655,0.303061,0.0539242); rgb(24pt)=(0.200514,0.305763,0.052757); rgb(25pt)=(0.205395,0.308459,0.0515759); rgb(26pt)=(0.210296,0.311149,0.0503624); rgb(27pt)=(0.215212,0.313846,0.0490067); rgb(28pt)=(0.220142,0.316548,0.0475043); rgb(29pt)=(0.22509,0.319254,0.0458704); rgb(30pt)=(0.230056,0.321962,0.0441205); rgb(31pt)=(0.235042,0.324671,0.04227); rgb(32pt)=(0.240048,0.327379,0.0403343); rgb(33pt)=(0.245078,0.330085,0.0383287); rgb(34pt)=(0.250131,0.332786,0.0362688); rgb(35pt)=(0.25521,0.335482,0.0341698); rgb(36pt)=(0.260317,0.33817,0.0320472); rgb(37pt)=(0.265451,0.340849,0.0299163); rgb(38pt)=(0.270616,0.343517,0.0277927); rgb(39pt)=(0.275813,0.346172,0.0256916); rgb(40pt)=(0.281043,0.348814,0.0236284); rgb(41pt)=(0.286307,0.35144,0.0216186); rgb(42pt)=(0.291607,0.354048,0.0196776); rgb(43pt)=(0.296945,0.356637,0.0178207); rgb(44pt)=(0.302322,0.359206,0.0160634); rgb(45pt)=(0.307739,0.361753,0.0144211); rgb(46pt)=(0.313198,0.364275,0.0129091); rgb(47pt)=(0.318701,0.366772,0.0115428); rgb(48pt)=(0.324249,0.369242,0.0103377); rgb(49pt)=(0.329843,0.371682,0.00930909); rgb(50pt)=(0.335485,0.374093,0.00847245); rgb(51pt)=(0.341176,0.376471,0.00784314); rgb(52pt)=(0.346925,0.378826,0.00732741); rgb(53pt)=(0.352735,0.381168,0.00682184); rgb(54pt)=(0.358605,0.383497,0.00632729); rgb(55pt)=(0.364532,0.385812,0.00584464); rgb(56pt)=(0.370516,0.388113,0.00537476); rgb(57pt)=(0.376552,0.390399,0.00491852); rgb(58pt)=(0.38264,0.39267,0.00447681); rgb(59pt)=(0.388777,0.394925,0.00405048); rgb(60pt)=(0.394962,0.397164,0.00364042); rgb(61pt)=(0.401191,0.399386,0.00324749); rgb(62pt)=(0.407464,0.401592,0.00287258); rgb(63pt)=(0.413777,0.40378,0.00251655); rgb(64pt)=(0.420129,0.40595,0.00218028); rgb(65pt)=(0.426518,0.408102,0.00186463); rgb(66pt)=(0.432942,0.410234,0.00157049); rgb(67pt)=(0.439399,0.412348,0.00129873); rgb(68pt)=(0.445885,0.414441,0.00105022); rgb(69pt)=(0.452401,0.416515,0.000825833); rgb(70pt)=(0.458942,0.418567,0.000626441); rgb(71pt)=(0.465508,0.420599,0.00045292); rgb(72pt)=(0.472096,0.422609,0.000306141); rgb(73pt)=(0.478704,0.424596,0.000186979); rgb(74pt)=(0.485331,0.426562,9.63073e-05); rgb(75pt)=(0.491973,0.428504,3.49981e-05); rgb(76pt)=(0.498628,0.430422,3.92506e-06); rgb(77pt)=(0.505323,0.432315,0); rgb(78pt)=(0.512206,0.434168,0); rgb(79pt)=(0.519282,0.435983,0); rgb(80pt)=(0.526529,0.437764,0); rgb(81pt)=(0.533922,0.439512,0); rgb(82pt)=(0.54144,0.441232,0); rgb(83pt)=(0.549059,0.442927,0); rgb(84pt)=(0.556756,0.444599,0); rgb(85pt)=(0.564508,0.446252,0); rgb(86pt)=(0.572292,0.447889,0); rgb(87pt)=(0.580084,0.449514,0); rgb(88pt)=(0.587863,0.451129,0); rgb(89pt)=(0.595604,0.452737,0); rgb(90pt)=(0.603284,0.454343,0); rgb(91pt)=(0.610882,0.455948,0); rgb(92pt)=(0.618373,0.457556,0); rgb(93pt)=(0.625734,0.459171,0); rgb(94pt)=(0.632943,0.460795,0); rgb(95pt)=(0.639976,0.462432,0); rgb(96pt)=(0.64681,0.464084,0); rgb(97pt)=(0.653423,0.465756,0); rgb(98pt)=(0.659791,0.46745,0); rgb(99pt)=(0.665891,0.469169,0); rgb(100pt)=(0.6717,0.470916,0); rgb(101pt)=(0.677195,0.472696,0); rgb(102pt)=(0.682353,0.47451,0); rgb(103pt)=(0.687242,0.476355,0); rgb(104pt)=(0.691952,0.478225,0); rgb(105pt)=(0.696497,0.480118,0); rgb(106pt)=(0.700887,0.482033,0); rgb(107pt)=(0.705134,0.483968,0); rgb(108pt)=(0.709251,0.485921,0); rgb(109pt)=(0.713249,0.487891,0); rgb(110pt)=(0.71714,0.489876,0); rgb(111pt)=(0.720936,0.491875,0); rgb(112pt)=(0.724649,0.493887,0); rgb(113pt)=(0.72829,0.495909,0); rgb(114pt)=(0.731872,0.49794,0); rgb(115pt)=(0.735406,0.499979,0); rgb(116pt)=(0.738904,0.502025,0); rgb(117pt)=(0.742378,0.504075,0); rgb(118pt)=(0.74584,0.506128,0); rgb(119pt)=(0.749302,0.508182,0); rgb(120pt)=(0.752775,0.510237,0); rgb(121pt)=(0.756272,0.51229,0); rgb(122pt)=(0.759804,0.514339,0); rgb(123pt)=(0.763384,0.516385,0); rgb(124pt)=(0.767022,0.518424,0); rgb(125pt)=(0.770731,0.520455,0); rgb(126pt)=(0.774523,0.522478,0); rgb(127pt)=(0.77841,0.524489,0); rgb(128pt)=(0.782391,0.526491,0); rgb(129pt)=(0.786402,0.528496,0); rgb(130pt)=(0.790431,0.530506,0); rgb(131pt)=(0.794478,0.532521,0); rgb(132pt)=(0.798541,0.534539,0); rgb(133pt)=(0.802619,0.53656,0); rgb(134pt)=(0.806712,0.538584,0); rgb(135pt)=(0.81082,0.540609,0); rgb(136pt)=(0.81494,0.542635,0); rgb(137pt)=(0.819074,0.54466,0); rgb(138pt)=(0.823219,0.546686,0); rgb(139pt)=(0.827374,0.548709,0); rgb(140pt)=(0.831541,0.55073,0); rgb(141pt)=(0.835716,0.552749,0); rgb(142pt)=(0.8399,0.554763,0); rgb(143pt)=(0.844092,0.556774,0); rgb(144pt)=(0.848292,0.558779,0); rgb(145pt)=(0.852497,0.560778,0); rgb(146pt)=(0.856708,0.562771,0); rgb(147pt)=(0.860924,0.564756,0); rgb(148pt)=(0.865143,0.566733,0); rgb(149pt)=(0.869366,0.568701,0); rgb(150pt)=(0.873592,0.57066,0); rgb(151pt)=(0.877819,0.572608,0); rgb(152pt)=(0.882047,0.574545,0); rgb(153pt)=(0.886275,0.576471,0); rgb(154pt)=(0.890659,0.578362,0); rgb(155pt)=(0.895333,0.580203,0); rgb(156pt)=(0.900258,0.581999,0); rgb(157pt)=(0.905397,0.583755,0); rgb(158pt)=(0.910711,0.585479,0); rgb(159pt)=(0.916164,0.587176,0); rgb(160pt)=(0.921717,0.588852,0); rgb(161pt)=(0.927333,0.590513,0); rgb(162pt)=(0.932974,0.592166,0); rgb(163pt)=(0.938602,0.593815,0); rgb(164pt)=(0.94418,0.595468,0); rgb(165pt)=(0.949669,0.59713,0); rgb(166pt)=(0.955033,0.598808,0); rgb(167pt)=(0.960233,0.600507,0); rgb(168pt)=(0.965232,0.602233,0); rgb(169pt)=(0.969992,0.603992,0); rgb(170pt)=(0.974475,0.605791,0); rgb(171pt)=(0.978643,0.607636,0); rgb(172pt)=(0.98246,0.609532,0); rgb(173pt)=(0.985886,0.611486,0); rgb(174pt)=(0.988885,0.613503,0); rgb(175pt)=(0.991419,0.61559,0); rgb(176pt)=(0.99345,0.617753,0); rgb(177pt)=(0.99494,0.619997,0); rgb(178pt)=(0.995851,0.622329,0); rgb(179pt)=(0.996226,0.624763,0); rgb(180pt)=(0.996512,0.627352,0); rgb(181pt)=(0.996788,0.630095,0); rgb(182pt)=(0.997053,0.632982,0); rgb(183pt)=(0.997308,0.636004,0); rgb(184pt)=(0.997552,0.639152,0); rgb(185pt)=(0.997785,0.642416,0); rgb(186pt)=(0.998006,0.645786,0); rgb(187pt)=(0.998217,0.649253,0); rgb(188pt)=(0.998416,0.652807,0); rgb(189pt)=(0.998605,0.656439,0); rgb(190pt)=(0.998781,0.660138,0); rgb(191pt)=(0.998946,0.663897,0); rgb(192pt)=(0.9991,0.667704,0); rgb(193pt)=(0.999242,0.67155,0); rgb(194pt)=(0.999372,0.675427,0); rgb(195pt)=(0.99949,0.679323,0); rgb(196pt)=(0.999596,0.68323,0); rgb(197pt)=(0.99969,0.687139,0); rgb(198pt)=(0.999771,0.691039,0); rgb(199pt)=(0.999841,0.694921,0); rgb(200pt)=(0.999898,0.698775,0); rgb(201pt)=(0.999942,0.702592,0); rgb(202pt)=(0.999974,0.706363,0); rgb(203pt)=(0.999994,0.710077,0); rgb(204pt)=(1,0.713725,0); rgb(205pt)=(1,0.717341,0); rgb(206pt)=(1,0.720963,0); rgb(207pt)=(1,0.724591,0); rgb(208pt)=(1,0.728226,0); rgb(209pt)=(1,0.731867,0); rgb(210pt)=(1,0.735514,0); rgb(211pt)=(1,0.739167,0); rgb(212pt)=(1,0.742827,0); rgb(213pt)=(1,0.746493,0); rgb(214pt)=(1,0.750165,0); rgb(215pt)=(1,0.753843,0); rgb(216pt)=(1,0.757527,0); rgb(217pt)=(1,0.761217,0); rgb(218pt)=(1,0.764913,0); rgb(219pt)=(1,0.768615,0); rgb(220pt)=(1,0.772324,0); rgb(221pt)=(1,0.776038,0); rgb(222pt)=(1,0.779758,0); rgb(223pt)=(1,0.783484,0); rgb(224pt)=(1,0.787215,0); rgb(225pt)=(1,0.790953,0); rgb(226pt)=(1,0.794696,0); rgb(227pt)=(1,0.798445,0); rgb(228pt)=(1,0.8022,0); rgb(229pt)=(1,0.805961,0); rgb(230pt)=(1,0.809727,0); rgb(231pt)=(1,0.8135,0); rgb(232pt)=(1,0.817278,0); rgb(233pt)=(1,0.821063,0); rgb(234pt)=(1,0.824854,0); rgb(235pt)=(1,0.828652,0); rgb(236pt)=(1,0.832455,0); rgb(237pt)=(1,0.836265,0); rgb(238pt)=(1,0.840081,0); rgb(239pt)=(1,0.843903,0); rgb(240pt)=(1,0.847732,0); rgb(241pt)=(1,0.851566,0); rgb(242pt)=(1,0.855406,0); rgb(243pt)=(1,0.859253,0); rgb(244pt)=(1,0.863106,0); rgb(245pt)=(1,0.866964,0); rgb(246pt)=(1,0.870829,0); rgb(247pt)=(1,0.8747,0); rgb(248pt)=(1,0.878577,0); rgb(249pt)=(1,0.88246,0); rgb(250pt)=(1,0.886349,0); rgb(251pt)=(1,0.890243,0); rgb(252pt)=(1,0.894144,0); rgb(253pt)=(1,0.898051,0); rgb(254pt)=(1,0.901964,0); rgb(255pt)=(1,0.905882,0)},
mesh/rows=49]
table[row sep=crcr,header=false] {%
%
56	0.093	-119.696332566822\\
56	0.0936	-167.958915296273\\
56	0.0942	-216.221498025719\\
56	0.0948	-264.48408075517\\
56	0.0954	-312.746663484621\\
56	0.096	-361.009246214071\\
56	0.0966	-409.271828943518\\
56	0.0972	-457.534411672968\\
56	0.0978	-505.796994402419\\
56	0.0984	-554.059577131866\\
56	0.099	-602.32215986132\\
56	0.0996	-650.584742590767\\
56	0.1002	-698.847325320214\\
56	0.1008	-747.109908049664\\
56	0.1014	-795.372490779115\\
56	0.102	-843.635073508565\\
56	0.1026	-891.897656238012\\
56	0.1032	-940.160238967459\\
56	0.1038	-988.422821696913\\
56	0.1044	-1036.68540442636\\
56	0.105	-1084.94798715581\\
56	0.1056	-1133.21056988526\\
56	0.1062	-1181.47315261471\\
56	0.1068	-1229.73573534416\\
56	0.1074	-1277.99831807361\\
56	0.108	-1326.26090080306\\
56	0.1086	-1374.52348353251\\
56	0.1092	-1422.78606626195\\
56	0.1098	-1471.0486489914\\
56	0.1104	-1519.31123172085\\
56	0.111	-1567.5738144503\\
56	0.1116	-1615.83639717975\\
56	0.1122	-1664.0989799092\\
56	0.1128	-1712.36156263865\\
56	0.1134	-1760.6241453681\\
56	0.114	-1808.88672809755\\
56	0.1146	-1857.149310827\\
56	0.1152	-1905.41189355645\\
56	0.1158	-1953.6744762859\\
56	0.1164	-2001.93705901535\\
56	0.117	-2050.1996417448\\
56	0.1176	-2098.46222447425\\
56	0.1182	-2146.72480720369\\
56	0.1188	-2194.98738993315\\
56	0.1194	-2243.24997266259\\
56	0.12	-2291.51255539204\\
56	0.1206	-2339.77513812149\\
56	0.1212	-2388.03772085094\\
56	0.1218	-2436.30030358039\\
56	0.1224	-2484.56288630984\\
56	0.123	-2532.82546903929\\
56.375	0.093	-108.105136267946\\
56.375	0.0936	-155.641348130826\\
56.375	0.0942	-203.177559993703\\
56.375	0.0948	-250.713771856583\\
56.375	0.0954	-298.249983719459\\
56.375	0.096	-345.786195582339\\
56.375	0.0966	-393.322407445219\\
56.375	0.0972	-440.858619308092\\
56.375	0.0978	-488.394831170976\\
56.375	0.0984	-535.931043033852\\
56.375	0.099	-583.467254896732\\
56.375	0.0996	-631.003466759608\\
56.375	0.1002	-678.539678622485\\
56.375	0.1008	-726.075890485365\\
56.375	0.1014	-773.612102348245\\
56.375	0.102	-821.148314211125\\
56.375	0.1026	-868.684526074001\\
56.375	0.1032	-916.220737936877\\
56.375	0.1038	-963.756949799757\\
56.375	0.1044	-1011.29316166263\\
56.375	0.105	-1058.82937352551\\
56.375	0.1056	-1106.36558538839\\
56.375	0.1062	-1153.90179725127\\
56.375	0.1068	-1201.43800911415\\
56.375	0.1074	-1248.97422097703\\
56.375	0.108	-1296.51043283991\\
56.375	0.1086	-1344.04664470278\\
56.375	0.1092	-1391.58285656566\\
56.375	0.1098	-1439.11906842854\\
56.375	0.1104	-1486.65528029142\\
56.375	0.111	-1534.1914921543\\
56.375	0.1116	-1581.72770401718\\
56.375	0.1122	-1629.26391588005\\
56.375	0.1128	-1676.80012774293\\
56.375	0.1134	-1724.33633960581\\
56.375	0.114	-1771.87255146869\\
56.375	0.1146	-1819.40876333157\\
56.375	0.1152	-1866.94497519444\\
56.375	0.1158	-1914.48118705732\\
56.375	0.1164	-1962.0173989202\\
56.375	0.117	-2009.55361078308\\
56.375	0.1176	-2057.08982264596\\
56.375	0.1182	-2104.62603450883\\
56.375	0.1188	-2152.16224637172\\
56.375	0.1194	-2199.69845823459\\
56.375	0.12	-2247.23467009747\\
56.375	0.1206	-2294.77088196035\\
56.375	0.1212	-2342.30709382323\\
56.375	0.1218	-2389.84330568611\\
56.375	0.1224	-2437.37951754898\\
56.375	0.123	-2484.91572941186\\
56.75	0.093	-96.513939969067\\
56.75	0.0936	-143.32378096538\\
56.75	0.0942	-190.133621961686\\
56.75	0.0948	-236.943462957992\\
56.75	0.0954	-283.753303954298\\
56.75	0.096	-330.563144950611\\
56.75	0.0966	-377.372985946917\\
56.75	0.0972	-424.182826943219\\
56.75	0.0978	-470.992667939528\\
56.75	0.0984	-517.802508935838\\
56.75	0.099	-564.612349932147\\
56.75	0.0996	-611.42219092845\\
56.75	0.1002	-658.232031924756\\
56.75	0.1008	-705.041872921065\\
56.75	0.1014	-751.851713917375\\
56.75	0.102	-798.66155491368\\
56.75	0.1026	-845.471395909986\\
56.75	0.1032	-892.281236906292\\
56.75	0.1038	-939.091077902605\\
56.75	0.1044	-985.900918898908\\
56.75	0.105	-1032.71075989522\\
56.75	0.1056	-1079.52060089152\\
56.75	0.1062	-1126.33044188783\\
56.75	0.1068	-1173.14028288414\\
56.75	0.1074	-1219.95012388044\\
56.75	0.108	-1266.75996487675\\
56.75	0.1086	-1313.56980587306\\
56.75	0.1092	-1360.37964686937\\
56.75	0.1098	-1407.18948786567\\
56.75	0.1104	-1453.99932886198\\
56.75	0.111	-1500.80916985829\\
56.75	0.1116	-1547.6190108546\\
56.75	0.1122	-1594.4288518509\\
56.75	0.1128	-1641.23869284721\\
56.75	0.1134	-1688.04853384352\\
56.75	0.114	-1734.85837483983\\
56.75	0.1146	-1781.66821583614\\
56.75	0.1152	-1828.47805683244\\
56.75	0.1158	-1875.28789782875\\
56.75	0.1164	-1922.09773882505\\
56.75	0.117	-1968.90757982136\\
56.75	0.1176	-2015.71742081767\\
56.75	0.1182	-2062.52726181398\\
56.75	0.1188	-2109.33710281028\\
56.75	0.1194	-2156.14694380659\\
56.75	0.12	-2202.9567848029\\
56.75	0.1206	-2249.76662579921\\
56.75	0.1212	-2296.57646679551\\
56.75	0.1218	-2343.38630779182\\
56.75	0.1224	-2390.19614878813\\
56.75	0.123	-2437.00598978444\\
57.125	0.093	-84.9227436701949\\
57.125	0.0936	-131.006213799934\\
57.125	0.0942	-177.089683929669\\
57.125	0.0948	-223.173154059408\\
57.125	0.0954	-269.256624189144\\
57.125	0.096	-315.340094318883\\
57.125	0.0966	-361.423564448614\\
57.125	0.0972	-407.50703457835\\
57.125	0.0978	-453.590504708089\\
57.125	0.0984	-499.673974837824\\
57.125	0.099	-545.757444967563\\
57.125	0.0996	-591.840915097298\\
57.125	0.1002	-637.92438522703\\
57.125	0.1008	-684.007855356769\\
57.125	0.1014	-730.091325486505\\
57.125	0.102	-776.174795616244\\
57.125	0.1026	-822.258265745979\\
57.125	0.1032	-868.341735875714\\
57.125	0.1038	-914.425206005453\\
57.125	0.1044	-960.508676135185\\
57.125	0.105	-1006.59214626492\\
57.125	0.1056	-1052.67561639466\\
57.125	0.1062	-1098.75908652439\\
57.125	0.1068	-1144.84255665413\\
57.125	0.1074	-1190.92602678387\\
57.125	0.108	-1237.00949691361\\
57.125	0.1086	-1283.09296704334\\
57.125	0.1092	-1329.17643717308\\
57.125	0.1098	-1375.25990730281\\
57.125	0.1104	-1421.34337743255\\
57.125	0.111	-1467.42684756229\\
57.125	0.1116	-1513.51031769202\\
57.125	0.1122	-1559.59378782176\\
57.125	0.1128	-1605.67725795149\\
57.125	0.1134	-1651.76072808123\\
57.125	0.114	-1697.84419821097\\
57.125	0.1146	-1743.9276683407\\
57.125	0.1152	-1790.01113847044\\
57.125	0.1158	-1836.09460860018\\
57.125	0.1164	-1882.17807872991\\
57.125	0.117	-1928.26154885965\\
57.125	0.1176	-1974.34501898938\\
57.125	0.1182	-2020.42848911912\\
57.125	0.1188	-2066.51195924886\\
57.125	0.1194	-2112.59542937859\\
57.125	0.12	-2158.67889950833\\
57.125	0.1206	-2204.76236963807\\
57.125	0.1212	-2250.8458397678\\
57.125	0.1218	-2296.92930989754\\
57.125	0.1224	-2343.01278002727\\
57.125	0.123	-2389.09625015701\\
57.5	0.093	-73.3315473713192\\
57.5	0.0936	-118.688646634488\\
57.5	0.0942	-164.045745897653\\
57.5	0.0948	-209.402845160817\\
57.5	0.0954	-254.759944423982\\
57.5	0.096	-300.117043687151\\
57.5	0.0966	-345.474142950316\\
57.5	0.0972	-390.831242213477\\
57.5	0.0978	-436.188341476645\\
57.5	0.0984	-481.54544073981\\
57.5	0.099	-526.902540002975\\
57.5	0.0996	-572.25963926614\\
57.5	0.1002	-617.616738529305\\
57.5	0.1008	-662.97383779247\\
57.5	0.1014	-708.330937055634\\
57.5	0.102	-753.688036318803\\
57.5	0.1026	-799.045135581968\\
57.5	0.1032	-844.402234845129\\
57.5	0.1038	-889.759334108297\\
57.5	0.1044	-935.116433371462\\
57.5	0.105	-980.473532634627\\
57.5	0.1056	-1025.83063189779\\
57.5	0.1062	-1071.18773116096\\
57.5	0.1068	-1116.54483042413\\
57.5	0.1074	-1161.90192968729\\
57.5	0.108	-1207.25902895046\\
57.5	0.1086	-1252.61612821362\\
57.5	0.1092	-1297.97322747678\\
57.5	0.1098	-1343.33032673995\\
57.5	0.1104	-1388.68742600311\\
57.5	0.111	-1434.04452526628\\
57.5	0.1116	-1479.40162452944\\
57.5	0.1122	-1524.75872379261\\
57.5	0.1128	-1570.11582305578\\
57.5	0.1134	-1615.47292231894\\
57.5	0.114	-1660.83002158211\\
57.5	0.1146	-1706.18712084527\\
57.5	0.1152	-1751.54422010844\\
57.5	0.1158	-1796.9013193716\\
57.5	0.1164	-1842.25841863477\\
57.5	0.117	-1887.61551789794\\
57.5	0.1176	-1932.9726171611\\
57.5	0.1182	-1978.32971642426\\
57.5	0.1188	-2023.68681568743\\
57.5	0.1194	-2069.04391495059\\
57.5	0.12	-2114.40101421376\\
57.5	0.1206	-2159.75811347692\\
57.5	0.1212	-2205.11521274009\\
57.5	0.1218	-2250.47231200325\\
57.5	0.1224	-2295.82941126642\\
57.5	0.123	-2341.18651052959\\
57.875	0.093	-61.7403510724434\\
57.875	0.0936	-106.371079469041\\
57.875	0.0942	-151.001807865632\\
57.875	0.0948	-195.63253626223\\
57.875	0.0954	-240.263264658821\\
57.875	0.096	-284.893993055419\\
57.875	0.0966	-329.524721452013\\
57.875	0.0972	-374.155449848604\\
57.875	0.0978	-418.786178245202\\
57.875	0.0984	-463.416906641793\\
57.875	0.099	-508.047635038391\\
57.875	0.0996	-552.678363434981\\
57.875	0.1002	-597.309091831576\\
57.875	0.1008	-641.939820228174\\
57.875	0.1014	-686.570548624764\\
57.875	0.102	-731.201277021362\\
57.875	0.1026	-775.832005417953\\
57.875	0.1032	-820.462733814547\\
57.875	0.1038	-865.093462211142\\
57.875	0.1044	-909.724190607736\\
57.875	0.105	-954.35491900433\\
57.875	0.1056	-998.985647400925\\
57.875	0.1062	-1043.61637579752\\
57.875	0.1068	-1088.24710419411\\
57.875	0.1074	-1132.87783259071\\
57.875	0.108	-1177.5085609873\\
57.875	0.1086	-1222.1392893839\\
57.875	0.1092	-1266.77001778049\\
57.875	0.1098	-1311.40074617709\\
57.875	0.1104	-1356.03147457368\\
57.875	0.111	-1400.66220297027\\
57.875	0.1116	-1445.29293136687\\
57.875	0.1122	-1489.92365976346\\
57.875	0.1128	-1534.55438816006\\
57.875	0.1134	-1579.18511655665\\
57.875	0.114	-1623.81584495325\\
57.875	0.1146	-1668.44657334984\\
57.875	0.1152	-1713.07730174643\\
57.875	0.1158	-1757.70803014303\\
57.875	0.1164	-1802.33875853962\\
57.875	0.117	-1846.96948693622\\
57.875	0.1176	-1891.60021533281\\
57.875	0.1182	-1936.2309437294\\
57.875	0.1188	-1980.861672126\\
57.875	0.1194	-2025.49240052259\\
57.875	0.12	-2070.12312891919\\
57.875	0.1206	-2114.75385731578\\
57.875	0.1212	-2159.38458571237\\
57.875	0.1218	-2204.01531410897\\
57.875	0.1224	-2248.64604250556\\
57.875	0.123	-2293.27677090216\\
58.25	0.093	-50.1491547735641\\
58.25	0.0936	-94.0535123035879\\
58.25	0.0942	-137.957869833615\\
58.25	0.0948	-181.862227363636\\
58.25	0.0954	-225.766584893663\\
58.25	0.096	-269.670942423687\\
58.25	0.0966	-313.575299953707\\
58.25	0.0972	-357.479657483727\\
58.25	0.0978	-401.384015013751\\
58.25	0.0984	-445.288372543779\\
58.25	0.099	-489.192730073803\\
58.25	0.0996	-533.097087603826\\
58.25	0.1002	-577.001445133843\\
58.25	0.1008	-620.90580266387\\
58.25	0.1014	-664.810160193894\\
58.25	0.102	-708.714517723914\\
58.25	0.1026	-752.618875253942\\
58.25	0.1032	-796.523232783959\\
58.25	0.1038	-840.427590313986\\
58.25	0.1044	-884.331947844006\\
58.25	0.105	-928.236305374034\\
58.25	0.1056	-972.140662904058\\
58.25	0.1062	-1016.04502043408\\
58.25	0.1068	-1059.9493779641\\
58.25	0.1074	-1103.85373549412\\
58.25	0.108	-1147.75809302415\\
58.25	0.1086	-1191.66245055417\\
58.25	0.1092	-1235.56680808419\\
58.25	0.1098	-1279.47116561422\\
58.25	0.1104	-1323.37552314424\\
58.25	0.111	-1367.27988067426\\
58.25	0.1116	-1411.18423820429\\
58.25	0.1122	-1455.08859573431\\
58.25	0.1128	-1498.99295326434\\
58.25	0.1134	-1542.89731079436\\
58.25	0.114	-1586.80166832438\\
58.25	0.1146	-1630.70602585441\\
58.25	0.1152	-1674.61038338443\\
58.25	0.1158	-1718.51474091445\\
58.25	0.1164	-1762.41909844447\\
58.25	0.117	-1806.3234559745\\
58.25	0.1176	-1850.22781350452\\
58.25	0.1182	-1894.13217103454\\
58.25	0.1188	-1938.03652856457\\
58.25	0.1194	-1981.94088609459\\
58.25	0.12	-2025.84524362462\\
58.25	0.1206	-2069.74960115464\\
58.25	0.1212	-2113.65395868466\\
58.25	0.1218	-2157.55831621468\\
58.25	0.1224	-2201.46267374471\\
58.25	0.123	-2245.36703127473\\
58.625	0.093	-38.5579584746883\\
58.625	0.0936	-81.7359451381417\\
58.625	0.0942	-124.913931801599\\
58.625	0.0948	-168.091918465048\\
58.625	0.0954	-211.269905128502\\
58.625	0.096	-254.447891791955\\
58.625	0.0966	-297.625878455405\\
58.625	0.0972	-340.803865118854\\
58.625	0.0978	-383.981851782308\\
58.625	0.0984	-427.159838445765\\
58.625	0.099	-470.337825109218\\
58.625	0.0996	-513.515811772668\\
58.625	0.1002	-556.693798436114\\
58.625	0.1008	-599.871785099571\\
58.625	0.1014	-643.049771763024\\
58.625	0.102	-686.227758426474\\
58.625	0.1026	-729.405745089927\\
58.625	0.1032	-772.583731753377\\
58.625	0.1038	-815.76171841683\\
58.625	0.1044	-858.93970508028\\
58.625	0.105	-902.117691743737\\
58.625	0.1056	-945.29567840719\\
58.625	0.1062	-988.473665070636\\
58.625	0.1068	-1031.65165173409\\
58.625	0.1074	-1074.82963839754\\
58.625	0.108	-1118.007625061\\
58.625	0.1086	-1161.18561172445\\
58.625	0.1092	-1204.3635983879\\
58.625	0.1098	-1247.54158505136\\
58.625	0.1104	-1290.7195717148\\
58.625	0.111	-1333.89755837826\\
58.625	0.1116	-1377.07554504171\\
58.625	0.1122	-1420.25353170516\\
58.625	0.1128	-1463.43151836862\\
58.625	0.1134	-1506.60950503207\\
58.625	0.114	-1549.78749169552\\
58.625	0.1146	-1592.96547835897\\
58.625	0.1152	-1636.14346502243\\
58.625	0.1158	-1679.32145168588\\
58.625	0.1164	-1722.49943834932\\
58.625	0.117	-1765.67742501278\\
58.625	0.1176	-1808.85541167623\\
58.625	0.1182	-1852.03339833968\\
58.625	0.1188	-1895.21138500314\\
58.625	0.1194	-1938.38937166659\\
58.625	0.12	-1981.56735833004\\
58.625	0.1206	-2024.74534499349\\
58.625	0.1212	-2067.92333165695\\
58.625	0.1218	-2111.1013183204\\
58.625	0.1224	-2154.27930498385\\
58.625	0.123	-2197.4572916473\\
59	0.093	-26.9667621758126\\
59	0.0936	-69.4183779726955\\
59	0.0942	-111.869993769578\\
59	0.0948	-154.321609566457\\
59	0.0954	-196.77322536334\\
59	0.096	-239.224841160227\\
59	0.0966	-281.676456957102\\
59	0.0972	-324.128072753982\\
59	0.0978	-366.579688550864\\
59	0.0984	-409.031304347747\\
59	0.099	-451.48292014463\\
59	0.0996	-493.934535941509\\
59	0.1002	-536.386151738388\\
59	0.1008	-578.837767535271\\
59	0.1014	-621.289383332154\\
59	0.102	-663.740999129033\\
59	0.1026	-706.192614925916\\
59	0.1032	-748.644230722792\\
59	0.1038	-791.095846519678\\
59	0.1044	-833.547462316554\\
59	0.105	-875.99907811344\\
59	0.1056	-918.450693910323\\
59	0.1062	-960.902309707199\\
59	0.1068	-1003.35392550408\\
59	0.1074	-1045.80554130096\\
59	0.108	-1088.25715709784\\
59	0.1086	-1130.70877289473\\
59	0.1092	-1173.16038869161\\
59	0.1098	-1215.61200448849\\
59	0.1104	-1258.06362028537\\
59	0.111	-1300.51523608225\\
59	0.1116	-1342.96685187913\\
59	0.1122	-1385.41846767601\\
59	0.1128	-1427.8700834729\\
59	0.1134	-1470.32169926977\\
59	0.114	-1512.77331506666\\
59	0.1146	-1555.22493086354\\
59	0.1152	-1597.67654666042\\
59	0.1158	-1640.1281624573\\
59	0.1164	-1682.57977825418\\
59	0.117	-1725.03139405107\\
59	0.1176	-1767.48300984794\\
59	0.1182	-1809.93462564483\\
59	0.1188	-1852.38624144171\\
59	0.1194	-1894.83785723859\\
59	0.12	-1937.28947303547\\
59	0.1206	-1979.74108883235\\
59	0.1212	-2022.19270462923\\
59	0.1218	-2064.64432042611\\
59	0.1224	-2107.095936223\\
59	0.123	-2149.54755201988\\
59.375	0.093	-15.3755658769369\\
59.375	0.0936	-57.1008108072529\\
59.375	0.0942	-98.8260557375652\\
59.375	0.0948	-140.55130066787\\
59.375	0.0954	-182.276545598186\\
59.375	0.096	-224.001790528499\\
59.375	0.0966	-265.727035458804\\
59.375	0.0972	-307.452280389109\\
59.375	0.0978	-349.177525319425\\
59.375	0.0984	-390.902770249737\\
59.375	0.099	-432.628015180049\\
59.375	0.0996	-474.353260110354\\
59.375	0.1002	-516.078505040663\\
59.375	0.1008	-557.803749970975\\
59.375	0.1014	-599.528994901288\\
59.375	0.102	-641.254239831596\\
59.375	0.1026	-682.979484761909\\
59.375	0.1032	-724.704729692214\\
59.375	0.1038	-766.429974622526\\
59.375	0.1044	-808.155219552835\\
59.375	0.105	-849.880464483147\\
59.375	0.1056	-891.605709413459\\
59.375	0.1062	-933.330954343764\\
59.375	0.1068	-975.056199274077\\
59.375	0.1074	-1016.78144420439\\
59.375	0.108	-1058.5066891347\\
59.375	0.1086	-1100.23193406501\\
59.375	0.1092	-1141.95717899532\\
59.375	0.1098	-1183.68242392563\\
59.375	0.1104	-1225.40766885594\\
59.375	0.111	-1267.13291378625\\
59.375	0.1116	-1308.85815871656\\
59.375	0.1122	-1350.58340364687\\
59.375	0.1128	-1392.30864857718\\
59.375	0.1134	-1434.03389350749\\
59.375	0.114	-1475.7591384378\\
59.375	0.1146	-1517.48438336811\\
59.375	0.1152	-1559.20962829842\\
59.375	0.1158	-1600.93487322873\\
59.375	0.1164	-1642.66011815904\\
59.375	0.117	-1684.38536308935\\
59.375	0.1176	-1726.11060801966\\
59.375	0.1182	-1767.83585294997\\
59.375	0.1188	-1809.56109788028\\
59.375	0.1194	-1851.28634281059\\
59.375	0.12	-1893.0115877409\\
59.375	0.1206	-1934.73683267121\\
59.375	0.1212	-1976.46207760152\\
59.375	0.1218	-2018.18732253183\\
59.375	0.1224	-2059.91256746214\\
59.375	0.123	-2101.63781239246\\
59.75	0.093	-3.78436957806116\\
59.75	0.0936	-44.783243641803\\
59.75	0.0942	-85.7821177055448\\
59.75	0.0948	-126.780991769283\\
59.75	0.0954	-167.779865833025\\
59.75	0.096	-208.778739896767\\
59.75	0.0966	-249.777613960501\\
59.75	0.0972	-290.776488024236\\
59.75	0.0978	-331.775362087978\\
59.75	0.0984	-372.774236151719\\
59.75	0.099	-413.773110215461\\
59.75	0.0996	-454.771984279199\\
59.75	0.1002	-495.770858342934\\
59.75	0.1008	-536.769732406676\\
59.75	0.1014	-577.768606470418\\
59.75	0.102	-618.767480534152\\
59.75	0.1026	-659.766354597894\\
59.75	0.1032	-700.765228661628\\
59.75	0.1038	-741.76410272537\\
59.75	0.1044	-782.762976789109\\
59.75	0.105	-823.76185085285\\
59.75	0.1056	-864.760724916592\\
59.75	0.1062	-905.759598980327\\
59.75	0.1068	-946.758473044065\\
59.75	0.1074	-987.757347107803\\
59.75	0.108	-1028.75622117154\\
59.75	0.1086	-1069.75509523529\\
59.75	0.1092	-1110.75396929902\\
59.75	0.1098	-1151.75284336277\\
59.75	0.1104	-1192.7517174265\\
59.75	0.111	-1233.75059149024\\
59.75	0.1116	-1274.74946555398\\
59.75	0.1122	-1315.74833961772\\
59.75	0.1128	-1356.74721368146\\
59.75	0.1134	-1397.7460877452\\
59.75	0.114	-1438.74496180894\\
59.75	0.1146	-1479.74383587268\\
59.75	0.1152	-1520.74270993641\\
59.75	0.1158	-1561.74158400016\\
59.75	0.1164	-1602.74045806389\\
59.75	0.117	-1643.73933212764\\
59.75	0.1176	-1684.73820619137\\
59.75	0.1182	-1725.73708025511\\
59.75	0.1188	-1766.73595431885\\
59.75	0.1194	-1807.73482838259\\
59.75	0.12	-1848.73370244633\\
59.75	0.1206	-1889.73257651006\\
59.75	0.1212	-1930.73145057381\\
59.75	0.1218	-1971.73032463754\\
59.75	0.1224	-2012.72919870129\\
59.75	0.123	-2053.72807276503\\
60.125	0.093	7.80682672081457\\
60.125	0.0936	-32.4656764763567\\
60.125	0.0942	-72.7381796735281\\
60.125	0.0948	-113.010682870692\\
60.125	0.0954	-153.283186067863\\
60.125	0.096	-193.555689265035\\
60.125	0.0966	-233.828192462199\\
60.125	0.0972	-274.100695659363\\
60.125	0.0978	-314.373198856534\\
60.125	0.0984	-354.645702053705\\
60.125	0.099	-394.918205250877\\
60.125	0.0996	-435.190708448041\\
60.125	0.1002	-475.463211645205\\
60.125	0.1008	-515.735714842376\\
60.125	0.1014	-556.008218039548\\
60.125	0.102	-596.280721236712\\
60.125	0.1026	-636.553224433883\\
60.125	0.1032	-676.825727631047\\
60.125	0.1038	-717.098230828215\\
60.125	0.1044	-757.370734025382\\
60.125	0.105	-797.643237222554\\
60.125	0.1056	-837.915740419725\\
60.125	0.1062	-878.188243616885\\
60.125	0.1068	-918.460746814057\\
60.125	0.1074	-958.733250011224\\
60.125	0.108	-999.005753208392\\
60.125	0.1086	-1039.27825640556\\
60.125	0.1092	-1079.55075960273\\
60.125	0.1098	-1119.8232627999\\
60.125	0.1104	-1160.09576599706\\
60.125	0.111	-1200.36826919423\\
60.125	0.1116	-1240.64077239141\\
60.125	0.1122	-1280.91327558857\\
60.125	0.1128	-1321.18577878574\\
60.125	0.1134	-1361.4582819829\\
60.125	0.114	-1401.73078518008\\
60.125	0.1146	-1442.00328837725\\
60.125	0.1152	-1482.27579157441\\
60.125	0.1158	-1522.54829477158\\
60.125	0.1164	-1562.82079796875\\
60.125	0.117	-1603.09330116592\\
60.125	0.1176	-1643.36580436308\\
60.125	0.1182	-1683.63830756025\\
60.125	0.1188	-1723.91081075742\\
60.125	0.1194	-1764.18331395459\\
60.125	0.12	-1804.45581715176\\
60.125	0.1206	-1844.72832034892\\
60.125	0.1212	-1885.00082354609\\
60.125	0.1218	-1925.27332674326\\
60.125	0.1224	-1965.54582994043\\
60.125	0.123	-2005.8183331376\\
60.5	0.093	19.3980230196903\\
60.5	0.0936	-20.1481093109105\\
60.5	0.0942	-59.6942416415077\\
60.5	0.0948	-99.2403739721049\\
60.5	0.0954	-138.786506302702\\
60.5	0.096	-178.332638633299\\
60.5	0.0966	-217.878770963896\\
60.5	0.0972	-257.42490329449\\
60.5	0.0978	-296.971035625091\\
60.5	0.0984	-336.517167955688\\
60.5	0.099	-376.063300286285\\
60.5	0.0996	-415.609432616882\\
60.5	0.1002	-455.155564947476\\
60.5	0.1008	-494.701697278077\\
60.5	0.1014	-534.24782960867\\
60.5	0.102	-573.793961939271\\
60.5	0.1026	-613.340094269868\\
60.5	0.1032	-652.886226600462\\
60.5	0.1038	-692.432358931062\\
60.5	0.1044	-731.978491261656\\
60.5	0.105	-771.524623592257\\
60.5	0.1056	-811.07075592285\\
60.5	0.1062	-850.616888253448\\
60.5	0.1068	-890.163020584048\\
60.5	0.1074	-929.709152914642\\
60.5	0.108	-969.255285245243\\
60.5	0.1086	-1008.80141757584\\
60.5	0.1092	-1048.34754990643\\
60.5	0.1098	-1087.89368223703\\
60.5	0.1104	-1127.43981456763\\
60.5	0.111	-1166.98594689823\\
60.5	0.1116	-1206.53207922882\\
60.5	0.1122	-1246.07821155942\\
60.5	0.1128	-1285.62434389002\\
60.5	0.1134	-1325.17047622061\\
60.5	0.114	-1364.71660855121\\
60.5	0.1146	-1404.26274088181\\
60.5	0.1152	-1443.80887321241\\
60.5	0.1158	-1483.355005543\\
60.5	0.1164	-1522.9011378736\\
60.5	0.117	-1562.4472702042\\
60.5	0.1176	-1601.99340253479\\
60.5	0.1182	-1641.53953486539\\
60.5	0.1188	-1681.08566719599\\
60.5	0.1194	-1720.63179952659\\
60.5	0.12	-1760.17793185718\\
60.5	0.1206	-1799.72406418778\\
60.5	0.1212	-1839.27019651838\\
60.5	0.1218	-1878.81632884897\\
60.5	0.1224	-1918.36246117957\\
60.5	0.123	-1957.90859351017\\
60.875	0.093	30.989219318566\\
60.875	0.0936	-7.83054214546428\\
60.875	0.0942	-46.6503036094873\\
60.875	0.0948	-85.470065073514\\
60.875	0.0954	-124.289826537541\\
60.875	0.096	-163.109588001567\\
60.875	0.0966	-201.929349465594\\
60.875	0.0972	-240.749110929617\\
60.875	0.0978	-279.568872393647\\
60.875	0.0984	-318.38863385767\\
60.875	0.099	-357.208395321701\\
60.875	0.0996	-396.028156785724\\
60.875	0.1002	-434.847918249747\\
60.875	0.1008	-473.667679713777\\
60.875	0.1014	-512.4874411778\\
60.875	0.102	-551.30720264183\\
60.875	0.1026	-590.126964105853\\
60.875	0.1032	-628.94672556988\\
60.875	0.1038	-667.766487033907\\
60.875	0.1044	-706.58624849793\\
60.875	0.105	-745.40600996196\\
60.875	0.1056	-784.225771425983\\
60.875	0.1062	-823.04553289001\\
60.875	0.1068	-861.865294354036\\
60.875	0.1074	-900.685055818063\\
60.875	0.108	-939.50481728209\\
60.875	0.1086	-978.324578746116\\
60.875	0.1092	-1017.14434021014\\
60.875	0.1098	-1055.96410167417\\
60.875	0.1104	-1094.78386313819\\
60.875	0.111	-1133.60362460222\\
60.875	0.1116	-1172.42338606625\\
60.875	0.1122	-1211.24314753027\\
60.875	0.1128	-1250.0629089943\\
60.875	0.1134	-1288.88267045832\\
60.875	0.114	-1327.70243192235\\
60.875	0.1146	-1366.52219338638\\
60.875	0.1152	-1405.3419548504\\
60.875	0.1158	-1444.16171631443\\
60.875	0.1164	-1482.98147777845\\
60.875	0.117	-1521.80123924248\\
60.875	0.1176	-1560.62100070651\\
60.875	0.1182	-1599.44076217053\\
60.875	0.1188	-1638.26052363456\\
60.875	0.1194	-1677.08028509858\\
60.875	0.12	-1715.90004656261\\
60.875	0.1206	-1754.71980802664\\
60.875	0.1212	-1793.53956949066\\
60.875	0.1218	-1832.35933095469\\
60.875	0.1224	-1871.17909241872\\
60.875	0.123	-1909.99885388274\\
61.25	0.093	42.5804156174418\\
61.25	0.0936	4.48702501998559\\
61.25	0.0942	-33.6063655774706\\
61.25	0.0948	-71.6997561749267\\
61.25	0.0954	-109.793146772379\\
61.25	0.096	-147.886537369839\\
61.25	0.0966	-185.979927967292\\
61.25	0.0972	-224.073318564744\\
61.25	0.0978	-262.1667091622\\
61.25	0.0984	-300.260099759656\\
61.25	0.099	-338.353490357113\\
61.25	0.0996	-376.446880954565\\
61.25	0.1002	-414.540271552021\\
61.25	0.1008	-452.633662149477\\
61.25	0.1014	-490.72705274693\\
61.25	0.102	-528.820443344386\\
61.25	0.1026	-566.913833941842\\
61.25	0.1032	-605.007224539295\\
61.25	0.1038	-643.100615136755\\
61.25	0.1044	-681.194005734207\\
61.25	0.105	-719.287396331663\\
61.25	0.1056	-757.380786929116\\
61.25	0.1062	-795.474177526572\\
61.25	0.1068	-833.567568124028\\
61.25	0.1074	-871.660958721481\\
61.25	0.108	-909.754349318941\\
61.25	0.1086	-947.847739916393\\
61.25	0.1092	-985.941130513846\\
61.25	0.1098	-1024.0345211113\\
61.25	0.1104	-1062.12791170876\\
61.25	0.111	-1100.22130230621\\
61.25	0.1116	-1138.31469290367\\
61.25	0.1122	-1176.40808350112\\
61.25	0.1128	-1214.50147409858\\
61.25	0.1134	-1252.59486469603\\
61.25	0.114	-1290.68825529349\\
61.25	0.1146	-1328.78164589094\\
61.25	0.1152	-1366.8750364884\\
61.25	0.1158	-1404.96842708585\\
61.25	0.1164	-1443.06181768331\\
61.25	0.117	-1481.15520828076\\
61.25	0.1176	-1519.24859887822\\
61.25	0.1182	-1557.34198947567\\
61.25	0.1188	-1595.43538007313\\
61.25	0.1194	-1633.52877067058\\
61.25	0.12	-1671.62216126804\\
61.25	0.1206	-1709.71555186549\\
61.25	0.1212	-1747.80894246295\\
61.25	0.1218	-1785.9023330604\\
61.25	0.1224	-1823.99572365786\\
61.25	0.123	-1862.08911425532\\
61.625	0.093	54.1716119163139\\
61.625	0.0936	16.8045921854282\\
61.625	0.0942	-20.5624275454538\\
61.625	0.0948	-57.9294472763395\\
61.625	0.0954	-95.2964670072215\\
61.625	0.096	-132.663486738111\\
61.625	0.0966	-170.030506468993\\
61.625	0.0972	-207.397526199875\\
61.625	0.0978	-244.764545930761\\
61.625	0.0984	-282.131565661643\\
61.625	0.099	-319.498585392528\\
61.625	0.0996	-356.86560512341\\
61.625	0.1002	-394.232624854296\\
61.625	0.1008	-431.599644585182\\
61.625	0.1014	-468.966664316064\\
61.625	0.102	-506.333684046949\\
61.625	0.1026	-543.700703777831\\
61.625	0.1032	-581.067723508717\\
61.625	0.1038	-618.434743239603\\
61.625	0.1044	-655.801762970485\\
61.625	0.105	-693.16878270137\\
61.625	0.1056	-730.535802432252\\
61.625	0.1062	-767.902822163134\\
61.625	0.1068	-805.269841894024\\
61.625	0.1074	-842.636861624906\\
61.625	0.108	-880.003881355791\\
61.625	0.1086	-917.370901086673\\
61.625	0.1092	-954.737920817555\\
61.625	0.1098	-992.104940548441\\
61.625	0.1104	-1029.47196027932\\
61.625	0.111	-1066.83898001021\\
61.625	0.1116	-1104.20599974109\\
61.625	0.1122	-1141.57301947198\\
61.625	0.1128	-1178.94003920286\\
61.625	0.1134	-1216.30705893374\\
61.625	0.114	-1253.67407866463\\
61.625	0.1146	-1291.04109839552\\
61.625	0.1152	-1328.4081181264\\
61.625	0.1158	-1365.77513785728\\
61.625	0.1164	-1403.14215758816\\
61.625	0.117	-1440.50917731905\\
61.625	0.1176	-1477.87619704993\\
61.625	0.1182	-1515.24321678082\\
61.625	0.1188	-1552.6102365117\\
61.625	0.1194	-1589.97725624259\\
61.625	0.12	-1627.34427597347\\
61.625	0.1206	-1664.71129570435\\
61.625	0.1212	-1702.07831543524\\
61.625	0.1218	-1739.44533516612\\
61.625	0.1224	-1776.81235489701\\
61.625	0.123	-1814.17937462789\\
62	0.093	65.7628082151896\\
62	0.0936	29.1221593508744\\
62	0.0942	-7.51848951343709\\
62	0.0948	-44.1591383777522\\
62	0.0954	-80.7997872420638\\
62	0.096	-117.440436106379\\
62	0.0966	-154.08108497069\\
62	0.0972	-190.721733835002\\
62	0.0978	-227.362382699317\\
62	0.0984	-264.003031563629\\
62	0.099	-300.643680427944\\
62	0.0996	-337.284329292255\\
62	0.1002	-373.924978156567\\
62	0.1008	-410.565627020882\\
62	0.1014	-447.206275885193\\
62	0.102	-483.846924749509\\
62	0.1026	-520.48757361382\\
62	0.1032	-557.128222478132\\
62	0.1038	-593.768871342447\\
62	0.1044	-630.409520206758\\
62	0.105	-667.050169071073\\
62	0.1056	-703.690817935385\\
62	0.1062	-740.331466799696\\
62	0.1068	-776.972115664012\\
62	0.1074	-813.612764528323\\
62	0.108	-850.253413392638\\
62	0.1086	-886.89406225695\\
62	0.1092	-923.534711121261\\
62	0.1098	-960.175359985577\\
62	0.1104	-996.816008849888\\
62	0.111	-1033.4566577142\\
62	0.1116	-1070.09730657851\\
62	0.1122	-1106.73795544283\\
62	0.1128	-1143.37860430714\\
62	0.1134	-1180.01925317145\\
62	0.114	-1216.65990203577\\
62	0.1146	-1253.30055090008\\
62	0.1152	-1289.94119976439\\
62	0.1158	-1326.58184862871\\
62	0.1164	-1363.22249749302\\
62	0.117	-1399.86314635733\\
62	0.1176	-1436.50379522164\\
62	0.1182	-1473.14444408596\\
62	0.1188	-1509.78509295027\\
62	0.1194	-1546.42574181458\\
62	0.12	-1583.0663906789\\
62	0.1206	-1619.70703954321\\
62	0.1212	-1656.34768840752\\
62	0.1218	-1692.98833727184\\
62	0.1224	-1729.62898613615\\
62	0.123	-1766.26963500047\\
62.375	0.093	77.3540045140653\\
62.375	0.0936	41.4397265163207\\
62.375	0.0942	5.52544851857965\\
62.375	0.0948	-30.3888294791614\\
62.375	0.0954	-66.3031074769024\\
62.375	0.096	-102.217385474647\\
62.375	0.0966	-138.131663472388\\
62.375	0.0972	-174.045941470129\\
62.375	0.0978	-209.960219467874\\
62.375	0.0984	-245.874497465615\\
62.375	0.099	-281.788775463356\\
62.375	0.0996	-317.703053461097\\
62.375	0.1002	-353.617331458838\\
62.375	0.1008	-389.531609456582\\
62.375	0.1014	-425.445887454323\\
62.375	0.102	-461.360165452068\\
62.375	0.1026	-497.274443449809\\
62.375	0.1032	-533.18872144755\\
62.375	0.1038	-569.102999445291\\
62.375	0.1044	-605.017277443032\\
62.375	0.105	-640.931555440777\\
62.375	0.1056	-676.845833438518\\
62.375	0.1062	-712.760111436259\\
62.375	0.1068	-748.674389434003\\
62.375	0.1074	-784.588667431744\\
62.375	0.108	-820.502945429489\\
62.375	0.1086	-856.417223427226\\
62.375	0.1092	-892.331501424967\\
62.375	0.1098	-928.245779422712\\
62.375	0.1104	-964.160057420453\\
62.375	0.111	-1000.0743354182\\
62.375	0.1116	-1035.98861341594\\
62.375	0.1122	-1071.90289141368\\
62.375	0.1128	-1107.81716941142\\
62.375	0.1134	-1143.73144740916\\
62.375	0.114	-1179.64572540691\\
62.375	0.1146	-1215.56000340465\\
62.375	0.1152	-1251.47428140239\\
62.375	0.1158	-1287.38855940013\\
62.375	0.1164	-1323.30283739787\\
62.375	0.117	-1359.21711539562\\
62.375	0.1176	-1395.13139339336\\
62.375	0.1182	-1431.0456713911\\
62.375	0.1188	-1466.95994938884\\
62.375	0.1194	-1502.87422738658\\
62.375	0.12	-1538.78850538433\\
62.375	0.1206	-1574.70278338207\\
62.375	0.1212	-1610.61706137981\\
62.375	0.1218	-1646.53133937755\\
62.375	0.1224	-1682.44561737529\\
62.375	0.123	-1718.35989537304\\
62.75	0.093	88.945200812941\\
62.75	0.0936	53.7572936817705\\
62.75	0.0942	18.5693865506\\
62.75	0.0948	-16.6185205805741\\
62.75	0.0954	-51.806427711741\\
62.75	0.096	-86.9943348429151\\
62.75	0.0966	-122.182241974086\\
62.75	0.0972	-157.370149105256\\
62.75	0.0978	-192.558056236427\\
62.75	0.0984	-227.745963367597\\
62.75	0.099	-262.933870498771\\
62.75	0.0996	-298.121777629942\\
62.75	0.1002	-333.309684761109\\
62.75	0.1008	-368.497591892283\\
62.75	0.1014	-403.685499023453\\
62.75	0.102	-438.873406154624\\
62.75	0.1026	-474.061313285794\\
62.75	0.1032	-509.249220416965\\
62.75	0.1038	-544.437127548139\\
62.75	0.1044	-579.625034679306\\
62.75	0.105	-614.81294181048\\
62.75	0.1056	-650.00084894165\\
62.75	0.1062	-685.188756072821\\
62.75	0.1068	-720.376663203995\\
62.75	0.1074	-755.564570335162\\
62.75	0.108	-790.752477466336\\
62.75	0.1086	-825.940384597507\\
62.75	0.1092	-861.128291728674\\
62.75	0.1098	-896.316198859848\\
62.75	0.1104	-931.504105991018\\
62.75	0.111	-966.692013122192\\
62.75	0.1116	-1001.87992025336\\
62.75	0.1122	-1037.06782738453\\
62.75	0.1128	-1072.2557345157\\
62.75	0.1134	-1107.44364164687\\
62.75	0.114	-1142.63154877804\\
62.75	0.1146	-1177.81945590922\\
62.75	0.1152	-1213.00736304039\\
62.75	0.1158	-1248.19527017156\\
62.75	0.1164	-1283.38317730273\\
62.75	0.117	-1318.5710844339\\
62.75	0.1176	-1353.75899156507\\
62.75	0.1182	-1388.94689869624\\
62.75	0.1188	-1424.13480582741\\
62.75	0.1194	-1459.32271295858\\
62.75	0.12	-1494.51062008975\\
62.75	0.1206	-1529.69852722092\\
62.75	0.1212	-1564.88643435209\\
62.75	0.1218	-1600.07434148327\\
62.75	0.1224	-1635.26224861444\\
62.75	0.123	-1670.45015574561\\
63.125	0.093	100.536397111817\\
63.125	0.0936	66.0748608472168\\
63.125	0.0942	31.6133245826168\\
63.125	0.0948	-2.84821168198323\\
63.125	0.0954	-37.3097479465832\\
63.125	0.096	-71.7712842111832\\
63.125	0.0966	-106.232820475783\\
63.125	0.0972	-140.694356740383\\
63.125	0.0978	-175.155893004983\\
63.125	0.0984	-209.617429269583\\
63.125	0.099	-244.078965534183\\
63.125	0.0996	-278.540501798783\\
63.125	0.1002	-313.002038063383\\
63.125	0.1008	-347.463574327983\\
63.125	0.1014	-381.925110592583\\
63.125	0.102	-416.386646857183\\
63.125	0.1026	-450.848183121783\\
63.125	0.1032	-485.309719386383\\
63.125	0.1038	-519.771255650983\\
63.125	0.1044	-554.232791915583\\
63.125	0.105	-588.694328180183\\
63.125	0.1056	-623.155864444783\\
63.125	0.1062	-657.617400709383\\
63.125	0.1068	-692.078936973983\\
63.125	0.1074	-726.540473238583\\
63.125	0.108	-761.002009503183\\
63.125	0.1086	-795.463545767783\\
63.125	0.1092	-829.92508203238\\
63.125	0.1098	-864.386618296983\\
63.125	0.1104	-898.848154561583\\
63.125	0.111	-933.309690826183\\
63.125	0.1116	-967.771227090783\\
63.125	0.1122	-1002.23276335538\\
63.125	0.1128	-1036.69429961998\\
63.125	0.1134	-1071.15583588458\\
63.125	0.114	-1105.61737214918\\
63.125	0.1146	-1140.07890841378\\
63.125	0.1152	-1174.54044467838\\
63.125	0.1158	-1209.00198094298\\
63.125	0.1164	-1243.46351720758\\
63.125	0.117	-1277.92505347218\\
63.125	0.1176	-1312.38658973678\\
63.125	0.1182	-1346.84812600138\\
63.125	0.1188	-1381.30966226598\\
63.125	0.1194	-1415.77119853058\\
63.125	0.12	-1450.23273479518\\
63.125	0.1206	-1484.69427105978\\
63.125	0.1212	-1519.15580732438\\
63.125	0.1218	-1553.61734358898\\
63.125	0.1224	-1588.07887985358\\
63.125	0.123	-1622.54041611818\\
63.5	0.093	112.1275934107\\
63.5	0.0936	78.3924280126666\\
63.5	0.0942	44.6572626146408\\
63.5	0.0948	10.9220972166077\\
63.5	0.0954	-22.8130681814182\\
63.5	0.096	-56.5482335794513\\
63.5	0.0966	-90.2833989774772\\
63.5	0.0972	-124.018564375503\\
63.5	0.0978	-157.753729773536\\
63.5	0.0984	-191.488895171562\\
63.5	0.099	-225.224060569595\\
63.5	0.0996	-258.959225967621\\
63.5	0.1002	-292.694391365651\\
63.5	0.1008	-326.42955676368\\
63.5	0.1014	-360.16472216171\\
63.5	0.102	-393.899887559739\\
63.5	0.1026	-427.635052957768\\
63.5	0.1032	-461.370218355794\\
63.5	0.1038	-495.105383753827\\
63.5	0.1044	-528.840549151853\\
63.5	0.105	-562.575714549883\\
63.5	0.1056	-596.310879947912\\
63.5	0.1062	-630.046045345938\\
63.5	0.1068	-663.781210743971\\
63.5	0.1074	-697.516376141997\\
63.5	0.108	-731.25154154003\\
63.5	0.1086	-764.986706938056\\
63.5	0.1092	-798.721872336082\\
63.5	0.1098	-832.457037734115\\
63.5	0.1104	-866.192203132141\\
63.5	0.111	-899.927368530174\\
63.5	0.1116	-933.6625339282\\
63.5	0.1122	-967.397699326229\\
63.5	0.1128	-1001.13286472426\\
63.5	0.1134	-1034.86803012229\\
63.5	0.114	-1068.60319552032\\
63.5	0.1146	-1102.33836091835\\
63.5	0.1152	-1136.07352631637\\
63.5	0.1158	-1169.8086917144\\
63.5	0.1164	-1203.54385711243\\
63.5	0.117	-1237.27902251046\\
63.5	0.1176	-1271.01418790849\\
63.5	0.1182	-1304.74935330652\\
63.5	0.1188	-1338.48451870455\\
63.5	0.1194	-1372.21968410258\\
63.5	0.12	-1405.95484950061\\
63.5	0.1206	-1439.69001489864\\
63.5	0.1212	-1473.42518029666\\
63.5	0.1218	-1507.16034569469\\
63.5	0.1224	-1540.89551109272\\
63.5	0.123	-1574.63067649075\\
63.875	0.093	123.718789709572\\
63.875	0.0936	90.7099951781129\\
63.875	0.0942	57.7012006466539\\
63.875	0.0948	24.6924061151949\\
63.875	0.0954	-8.31638841626045\\
63.875	0.096	-41.3251829477231\\
63.875	0.0966	-74.3339774791784\\
63.875	0.0972	-107.342772010634\\
63.875	0.0978	-140.351566542096\\
63.875	0.0984	-173.360361073552\\
63.875	0.099	-206.369155605011\\
63.875	0.0996	-239.377950136466\\
63.875	0.1002	-272.386744667925\\
63.875	0.1008	-305.395539199384\\
63.875	0.1014	-338.404333730843\\
63.875	0.102	-371.413128262302\\
63.875	0.1026	-404.421922793757\\
63.875	0.1032	-437.430717325213\\
63.875	0.1038	-470.439511856675\\
63.875	0.1044	-503.448306388131\\
63.875	0.105	-536.45710091959\\
63.875	0.1056	-569.465895451049\\
63.875	0.1062	-602.474689982504\\
63.875	0.1068	-635.483484513963\\
63.875	0.1074	-668.492279045418\\
63.875	0.108	-701.501073576881\\
63.875	0.1086	-734.509868108336\\
63.875	0.1092	-767.518662639792\\
63.875	0.1098	-800.527457171254\\
63.875	0.1104	-833.53625170271\\
63.875	0.111	-866.545046234169\\
63.875	0.1116	-899.553840765628\\
63.875	0.1122	-932.562635297083\\
63.875	0.1128	-965.571429828542\\
63.875	0.1134	-998.580224360001\\
63.875	0.114	-1031.58901889146\\
63.875	0.1146	-1064.59781342292\\
63.875	0.1152	-1097.60660795437\\
63.875	0.1158	-1130.61540248583\\
63.875	0.1164	-1163.62419701729\\
63.875	0.117	-1196.63299154875\\
63.875	0.1176	-1229.64178608021\\
63.875	0.1182	-1262.65058061166\\
63.875	0.1188	-1295.65937514312\\
63.875	0.1194	-1328.66816967458\\
63.875	0.12	-1361.67696420604\\
63.875	0.1206	-1394.68575873749\\
63.875	0.1212	-1427.69455326895\\
63.875	0.1218	-1460.70334780041\\
63.875	0.1224	-1493.71214233187\\
63.875	0.123	-1526.72093686333\\
64.25	0.093	135.309986008448\\
64.25	0.0936	103.027562343559\\
64.25	0.0942	70.7451386786743\\
64.25	0.0948	38.4627150137858\\
64.25	0.0954	6.1802913488973\\
64.25	0.096	-26.1021323159912\\
64.25	0.0966	-58.384555980876\\
64.25	0.0972	-90.6669796457609\\
64.25	0.0978	-122.949403310649\\
64.25	0.0984	-155.231826975538\\
64.25	0.099	-187.514250640426\\
64.25	0.0996	-219.796674305311\\
64.25	0.1002	-252.079097970196\\
64.25	0.1008	-284.361521635085\\
64.25	0.1014	-316.643945299969\\
64.25	0.102	-348.926368964858\\
64.25	0.1026	-381.208792629746\\
64.25	0.1032	-413.491216294631\\
64.25	0.1038	-445.77363995952\\
64.25	0.1044	-478.056063624404\\
64.25	0.105	-510.338487289293\\
64.25	0.1056	-542.620910954181\\
64.25	0.1062	-574.903334619066\\
64.25	0.1068	-607.185758283955\\
64.25	0.1074	-639.46818194884\\
64.25	0.108	-671.750605613728\\
64.25	0.1086	-704.033029278613\\
64.25	0.1092	-736.315452943498\\
64.25	0.1098	-768.59787660839\\
64.25	0.1104	-800.880300273275\\
64.25	0.111	-833.162723938163\\
64.25	0.1116	-865.445147603048\\
64.25	0.1122	-897.727571267933\\
64.25	0.1128	-930.009994932825\\
64.25	0.1134	-962.29241859771\\
64.25	0.114	-994.574842262598\\
64.25	0.1146	-1026.85726592748\\
64.25	0.1152	-1059.13968959237\\
64.25	0.1158	-1091.42211325726\\
64.25	0.1164	-1123.70453692215\\
64.25	0.117	-1155.98696058703\\
64.25	0.1176	-1188.26938425192\\
64.25	0.1182	-1220.5518079168\\
64.25	0.1188	-1252.83423158169\\
64.25	0.1194	-1285.11665524658\\
64.25	0.12	-1317.39907891147\\
64.25	0.1206	-1349.68150257635\\
64.25	0.1212	-1381.96392624124\\
64.25	0.1218	-1414.24634990613\\
64.25	0.1224	-1446.52877357101\\
64.25	0.123	-1478.8111972359\\
64.625	0.093	146.901182307323\\
64.625	0.0936	115.345129509005\\
64.625	0.0942	83.789076710691\\
64.625	0.0948	52.233023912373\\
64.625	0.0954	20.6769711140587\\
64.625	0.096	-10.8790816842593\\
64.625	0.0966	-42.4351344825736\\
64.625	0.0972	-73.991187280888\\
64.625	0.0978	-105.547240079206\\
64.625	0.0984	-137.10329287752\\
64.625	0.099	-168.659345675838\\
64.625	0.0996	-200.215398474153\\
64.625	0.1002	-231.771451272467\\
64.625	0.1008	-263.327504070785\\
64.625	0.1014	-294.883556869099\\
64.625	0.102	-326.439609667417\\
64.625	0.1026	-357.995662465732\\
64.625	0.1032	-389.551715264046\\
64.625	0.1038	-421.107768062364\\
64.625	0.1044	-452.663820860678\\
64.625	0.105	-484.219873658996\\
64.625	0.1056	-515.775926457311\\
64.625	0.1062	-547.331979255629\\
64.625	0.1068	-578.888032053947\\
64.625	0.1074	-610.444084852261\\
64.625	0.108	-642.000137650579\\
64.625	0.1086	-673.556190448893\\
64.625	0.1092	-705.112243247208\\
64.625	0.1098	-736.668296045525\\
64.625	0.1104	-768.22434884384\\
64.625	0.111	-799.780401642158\\
64.625	0.1116	-831.336454440472\\
64.625	0.1122	-862.892507238786\\
64.625	0.1128	-894.448560037104\\
64.625	0.1134	-926.004612835419\\
64.625	0.114	-957.560665633737\\
64.625	0.1146	-989.116718432051\\
64.625	0.1152	-1020.67277123037\\
64.625	0.1158	-1052.22882402868\\
64.625	0.1164	-1083.784876827\\
64.625	0.117	-1115.34092962532\\
64.625	0.1176	-1146.89698242363\\
64.625	0.1182	-1178.45303522194\\
64.625	0.1188	-1210.00908802026\\
64.625	0.1194	-1241.56514081858\\
64.625	0.12	-1273.12119361689\\
64.625	0.1206	-1304.67724641521\\
64.625	0.1212	-1336.23329921352\\
64.625	0.1218	-1367.78935201184\\
64.625	0.1224	-1399.34540481016\\
64.625	0.123	-1430.90145760847\\
65	0.093	158.492378606199\\
65	0.0936	127.662696674452\\
65	0.0942	96.8330147427077\\
65	0.0948	66.0033328109639\\
65	0.0954	35.1736508792201\\
65	0.096	4.3439689474726\\
65	0.0966	-26.4857129842712\\
65	0.0972	-57.3153949160151\\
65	0.0978	-88.1450768477625\\
65	0.0984	-118.974758779506\\
65	0.099	-149.804440711254\\
65	0.0996	-180.634122642998\\
65	0.1002	-211.463804574738\\
65	0.1008	-242.293486506485\\
65	0.1014	-273.123168438229\\
65	0.102	-303.952850369977\\
65	0.1026	-334.78253230172\\
65	0.1032	-365.612214233464\\
65	0.1038	-396.441896165212\\
65	0.1044	-427.271578096956\\
65	0.105	-458.101260028703\\
65	0.1056	-488.930941960443\\
65	0.1062	-519.760623892187\\
65	0.1068	-550.590305823935\\
65	0.1074	-581.419987755678\\
65	0.108	-612.249669687426\\
65	0.1086	-643.07935161917\\
65	0.1092	-673.909033550914\\
65	0.1098	-704.738715482661\\
65	0.1104	-735.568397414401\\
65	0.111	-766.398079346149\\
65	0.1116	-797.227761277893\\
65	0.1122	-828.057443209636\\
65	0.1128	-858.887125141384\\
65	0.1134	-889.716807073128\\
65	0.114	-920.546489004875\\
65	0.1146	-951.376170936619\\
65	0.1152	-982.205852868363\\
65	0.1158	-1013.03553480011\\
65	0.1164	-1043.86521673185\\
65	0.117	-1074.6948986636\\
65	0.1176	-1105.52458059534\\
65	0.1182	-1136.35426252709\\
65	0.1188	-1167.18394445883\\
65	0.1194	-1198.01362639058\\
65	0.12	-1228.84330832232\\
65	0.1206	-1259.67299025407\\
65	0.1212	-1290.50267218581\\
65	0.1218	-1321.33235411756\\
65	0.1224	-1352.1620360493\\
65	0.123	-1382.99171798105\\
65.375	0.093	170.083574905075\\
65.375	0.0936	139.980263839898\\
65.375	0.0942	109.876952774728\\
65.375	0.0948	79.7736417095512\\
65.375	0.0954	49.6703306443778\\
65.375	0.096	19.5670195792009\\
65.375	0.0966	-10.5362914859688\\
65.375	0.0972	-40.6396025511422\\
65.375	0.0978	-70.7429136163191\\
65.375	0.0984	-100.846224681492\\
65.375	0.099	-130.949535746666\\
65.375	0.0996	-161.052846811839\\
65.375	0.1002	-191.156157877012\\
65.375	0.1008	-221.259468942186\\
65.375	0.1014	-251.362780007359\\
65.375	0.102	-281.466091072536\\
65.375	0.1026	-311.569402137709\\
65.375	0.1032	-341.672713202879\\
65.375	0.1038	-371.776024268056\\
65.375	0.1044	-401.879335333229\\
65.375	0.105	-431.982646398406\\
65.375	0.1056	-462.085957463576\\
65.375	0.1062	-492.189268528749\\
65.375	0.1068	-522.292579593926\\
65.375	0.1074	-552.3958906591\\
65.375	0.108	-582.499201724273\\
65.375	0.1086	-612.602512789446\\
65.375	0.1092	-642.70582385462\\
65.375	0.1098	-672.809134919793\\
65.375	0.1104	-702.912445984966\\
65.375	0.111	-733.015757050143\\
65.375	0.1116	-763.119068115317\\
65.375	0.1122	-793.222379180486\\
65.375	0.1128	-823.325690245663\\
65.375	0.1134	-853.429001310837\\
65.375	0.114	-883.532312376014\\
65.375	0.1146	-913.635623441187\\
65.375	0.1152	-943.738934506357\\
65.375	0.1158	-973.842245571534\\
65.375	0.1164	-1003.94555663671\\
65.375	0.117	-1034.04886770188\\
65.375	0.1176	-1064.15217876705\\
65.375	0.1182	-1094.25548983223\\
65.375	0.1188	-1124.3588008974\\
65.375	0.1194	-1154.46211196257\\
65.375	0.12	-1184.56542302775\\
65.375	0.1206	-1214.66873409292\\
65.375	0.1212	-1244.7720451581\\
65.375	0.1218	-1274.87535622327\\
65.375	0.1224	-1304.97866728844\\
65.375	0.123	-1335.08197835362\\
65.75	0.093	181.674771203951\\
65.75	0.0936	152.297831005348\\
65.75	0.0942	122.920890806745\\
65.75	0.0948	93.543950608142\\
65.75	0.0954	64.1670104095392\\
65.75	0.096	34.7900702109328\\
65.75	0.0966	5.41313001233357\\
65.75	0.0972	-23.9638101862693\\
65.75	0.0978	-53.3407503848757\\
65.75	0.0984	-82.7176905834749\\
65.75	0.099	-112.094630782081\\
65.75	0.0996	-141.471570980681\\
65.75	0.1002	-170.848511179283\\
65.75	0.1008	-200.22545137789\\
65.75	0.1014	-229.602391576489\\
65.75	0.102	-258.979331775095\\
65.75	0.1026	-288.356271973695\\
65.75	0.1032	-317.733212172297\\
65.75	0.1038	-347.110152370904\\
65.75	0.1044	-376.487092569503\\
65.75	0.105	-405.86403276811\\
65.75	0.1056	-435.240972966709\\
65.75	0.1062	-464.617913165312\\
65.75	0.1068	-493.994853363918\\
65.75	0.1074	-523.371793562517\\
65.75	0.108	-552.748733761124\\
65.75	0.1086	-582.125673959723\\
65.75	0.1092	-611.502614158326\\
65.75	0.1098	-640.879554356929\\
65.75	0.1104	-670.256494555531\\
65.75	0.111	-699.633434754138\\
65.75	0.1116	-729.010374952737\\
65.75	0.1122	-758.38731515134\\
65.75	0.1128	-787.764255349943\\
65.75	0.1134	-817.141195548546\\
65.75	0.114	-846.518135747152\\
65.75	0.1146	-875.895075945751\\
65.75	0.1152	-905.272016144354\\
65.75	0.1158	-934.648956342957\\
65.75	0.1164	-964.02589654156\\
65.75	0.117	-993.402836740166\\
65.75	0.1176	-1022.77977693877\\
65.75	0.1182	-1052.15671713737\\
65.75	0.1188	-1081.53365733597\\
65.75	0.1194	-1110.91059753457\\
65.75	0.12	-1140.28753773318\\
65.75	0.1206	-1169.66447793178\\
65.75	0.1212	-1199.04141813038\\
65.75	0.1218	-1228.41835832899\\
65.75	0.1224	-1257.79529852759\\
65.75	0.123	-1287.17223872619\\
66.125	0.093	193.265967502826\\
66.125	0.0936	164.61539817079\\
66.125	0.0942	135.964828838762\\
66.125	0.0948	107.314259506726\\
66.125	0.0954	78.663690174697\\
66.125	0.096	50.013120842661\\
66.125	0.0966	21.3625515106323\\
66.125	0.0972	-7.28801782139999\\
66.125	0.0978	-35.9385871534323\\
66.125	0.0984	-64.5891564854646\\
66.125	0.099	-93.2397258174969\\
66.125	0.0996	-121.890295149526\\
66.125	0.1002	-150.540864481558\\
66.125	0.1008	-179.19143381359\\
66.125	0.1014	-207.842003145623\\
66.125	0.102	-236.492572477655\\
66.125	0.1026	-265.143141809687\\
66.125	0.1032	-293.793711141716\\
66.125	0.1038	-322.444280473752\\
66.125	0.1044	-351.094849805781\\
66.125	0.105	-379.745419137816\\
66.125	0.1056	-408.395988469845\\
66.125	0.1062	-437.046557801877\\
66.125	0.1068	-465.69712713391\\
66.125	0.1074	-494.347696465938\\
66.125	0.108	-522.998265797974\\
66.125	0.1086	-551.648835130003\\
66.125	0.1092	-580.299404462035\\
66.125	0.1098	-608.949973794068\\
66.125	0.1104	-637.6005431261\\
66.125	0.111	-666.251112458132\\
66.125	0.1116	-694.901681790165\\
66.125	0.1122	-723.552251122193\\
66.125	0.1128	-752.202820454229\\
66.125	0.1134	-780.853389786258\\
66.125	0.114	-809.503959118294\\
66.125	0.1146	-838.154528450323\\
66.125	0.1152	-866.805097782351\\
66.125	0.1158	-895.455667114387\\
66.125	0.1164	-924.106236446416\\
66.125	0.117	-952.756805778452\\
66.125	0.1176	-981.407375110481\\
66.125	0.1182	-1010.05794444251\\
66.125	0.1188	-1038.70851377455\\
66.125	0.1194	-1067.35908310658\\
66.125	0.12	-1096.00965243861\\
66.125	0.1206	-1124.66022177064\\
66.125	0.1212	-1153.31079110267\\
66.125	0.1218	-1181.9613604347\\
66.125	0.1224	-1210.61192976674\\
66.125	0.123	-1239.26249909877\\
66.5	0.093	204.857163801702\\
66.5	0.0936	176.932965336237\\
66.5	0.0942	149.008766870778\\
66.5	0.0948	121.084568405317\\
66.5	0.0954	93.1603699398547\\
66.5	0.096	65.2361714743929\\
66.5	0.0966	37.3119730089347\\
66.5	0.0972	9.38777454347291\\
66.5	0.0978	-18.5364239219889\\
66.5	0.0984	-46.4606223874471\\
66.5	0.099	-74.3848208529125\\
66.5	0.0996	-102.309019318371\\
66.5	0.1002	-130.233217783829\\
66.5	0.1008	-158.157416249294\\
66.5	0.1014	-186.081614714753\\
66.5	0.102	-214.005813180214\\
66.5	0.1026	-241.930011645672\\
66.5	0.1032	-269.854210111134\\
66.5	0.1038	-297.778408576596\\
66.5	0.1044	-325.702607042054\\
66.5	0.105	-353.62680550752\\
66.5	0.1056	-381.551003972978\\
66.5	0.1062	-409.475202438436\\
66.5	0.1068	-437.399400903902\\
66.5	0.1074	-465.32359936936\\
66.5	0.108	-493.247797834822\\
66.5	0.1086	-521.171996300283\\
66.5	0.1092	-549.096194765742\\
66.5	0.1098	-577.020393231203\\
66.5	0.1104	-604.944591696665\\
66.5	0.111	-632.868790162127\\
66.5	0.1116	-660.792988627585\\
66.5	0.1122	-688.717187093043\\
66.5	0.1128	-716.641385558509\\
66.5	0.1134	-744.565584023967\\
66.5	0.114	-772.489782489432\\
66.5	0.1146	-800.413980954891\\
66.5	0.1152	-828.338179420349\\
66.5	0.1158	-856.262377885811\\
66.5	0.1164	-884.186576351272\\
66.5	0.117	-912.110774816734\\
66.5	0.1176	-940.034973282192\\
66.5	0.1182	-967.959171747654\\
66.5	0.1188	-995.883370213116\\
66.5	0.1194	-1023.80756867857\\
66.5	0.12	-1051.73176714404\\
66.5	0.1206	-1079.6559656095\\
66.5	0.1212	-1107.58016407496\\
66.5	0.1218	-1135.50436254042\\
66.5	0.1224	-1163.42856100588\\
66.5	0.123	-1191.35275947134\\
66.875	0.093	216.448360100578\\
66.875	0.0936	189.250532501686\\
66.875	0.0942	162.052704902795\\
66.875	0.0948	134.854877303904\\
66.875	0.0954	107.657049705016\\
66.875	0.096	80.4592221061248\\
66.875	0.0966	53.2613945072371\\
66.875	0.0972	26.0635669083495\\
66.875	0.0978	-1.13426069054549\\
66.875	0.0984	-28.3320882894332\\
66.875	0.099	-55.5299158883245\\
66.875	0.0996	-82.7277434872121\\
66.875	0.1002	-109.9255710861\\
66.875	0.1008	-137.123398684995\\
66.875	0.1014	-164.321226283882\\
66.875	0.102	-191.519053882774\\
66.875	0.1026	-218.716881481661\\
66.875	0.1032	-245.914709080549\\
66.875	0.1038	-273.11253667944\\
66.875	0.1044	-300.310364278332\\
66.875	0.105	-327.508191877223\\
66.875	0.1056	-354.706019476111\\
66.875	0.1062	-381.903847074998\\
66.875	0.1068	-409.10167467389\\
66.875	0.1074	-436.299502272777\\
66.875	0.108	-463.497329871672\\
66.875	0.1086	-490.69515747056\\
66.875	0.1092	-517.892985069448\\
66.875	0.1098	-545.090812668339\\
66.875	0.1104	-572.288640267227\\
66.875	0.111	-599.486467866122\\
66.875	0.1116	-626.684295465009\\
66.875	0.1122	-653.882123063897\\
66.875	0.1128	-681.079950662788\\
66.875	0.1134	-708.277778261676\\
66.875	0.114	-735.475605860567\\
66.875	0.1146	-762.673433459458\\
66.875	0.1152	-789.871261058346\\
66.875	0.1158	-817.069088657237\\
66.875	0.1164	-844.266916256125\\
66.875	0.117	-871.464743855016\\
66.875	0.1176	-898.662571453904\\
66.875	0.1182	-925.860399052795\\
66.875	0.1188	-953.058226651687\\
66.875	0.1194	-980.256054250574\\
66.875	0.12	-1007.45388184947\\
66.875	0.1206	-1034.65170944835\\
66.875	0.1212	-1061.84953704724\\
66.875	0.1218	-1089.04736464614\\
66.875	0.1224	-1116.24519224502\\
66.875	0.123	-1143.44301984391\\
67.25	0.093	228.039556399453\\
67.25	0.0936	201.568099667133\\
67.25	0.0942	175.096642934815\\
67.25	0.0948	148.625186202495\\
67.25	0.0954	122.153729470177\\
67.25	0.096	95.6822727378567\\
67.25	0.0966	69.2108160055395\\
67.25	0.0972	42.7393592732224\\
67.25	0.0978	16.2679025409016\\
67.25	0.0984	-10.2035541914192\\
67.25	0.099	-36.67501092374\\
67.25	0.0996	-63.1464676560572\\
67.25	0.1002	-89.6179243883744\\
67.25	0.1008	-116.089381120695\\
67.25	0.1014	-142.560837853012\\
67.25	0.102	-169.032294585333\\
67.25	0.1026	-195.50375131765\\
67.25	0.1032	-221.975208049967\\
67.25	0.1038	-248.446664782288\\
67.25	0.1044	-274.918121514605\\
67.25	0.105	-301.389578246926\\
67.25	0.1056	-327.861034979243\\
67.25	0.1062	-354.332491711561\\
67.25	0.1068	-380.803948443881\\
67.25	0.1074	-407.275405176199\\
67.25	0.108	-433.746861908519\\
67.25	0.1086	-460.218318640837\\
67.25	0.1092	-486.689775373154\\
67.25	0.1098	-513.161232105474\\
67.25	0.1104	-539.632688837792\\
67.25	0.111	-566.104145570112\\
67.25	0.1116	-592.57560230243\\
67.25	0.1122	-619.047059034747\\
67.25	0.1128	-645.518515767068\\
67.25	0.1134	-671.989972499385\\
67.25	0.114	-698.461429231706\\
67.25	0.1146	-724.932885964026\\
67.25	0.1152	-751.40434269634\\
67.25	0.1158	-777.875799428661\\
67.25	0.1164	-804.347256160981\\
67.25	0.117	-830.818712893299\\
67.25	0.1176	-857.290169625619\\
67.25	0.1182	-883.761626357937\\
67.25	0.1188	-910.233083090257\\
67.25	0.1194	-936.704539822575\\
67.25	0.12	-963.175996554895\\
67.25	0.1206	-989.647453287213\\
67.25	0.1212	-1016.11891001953\\
67.25	0.1218	-1042.59036675185\\
67.25	0.1224	-1069.06182348417\\
67.25	0.123	-1095.53328021649\\
67.625	0.093	239.630752698333\\
67.625	0.0936	213.885666832583\\
67.625	0.0942	188.140580966836\\
67.625	0.0948	162.395495101086\\
67.625	0.0954	136.650409235339\\
67.625	0.096	110.905323369589\\
67.625	0.0966	85.1602375038419\\
67.625	0.0972	59.4151516380989\\
67.625	0.0978	33.6700657723486\\
67.625	0.0984	7.92497990660195\\
67.625	0.099	-17.8201059591483\\
67.625	0.0996	-43.565191824895\\
67.625	0.1002	-69.3102776906417\\
67.625	0.1008	-95.055363556392\\
67.625	0.1014	-120.800449422139\\
67.625	0.102	-146.545535287885\\
67.625	0.1026	-172.290621153632\\
67.625	0.1032	-198.035707019379\\
67.625	0.1038	-223.780792885129\\
67.625	0.1044	-249.525878750876\\
67.625	0.105	-275.270964616626\\
67.625	0.1056	-301.016050482373\\
67.625	0.1062	-326.761136348119\\
67.625	0.1068	-352.506222213869\\
67.625	0.1074	-378.251308079612\\
67.625	0.108	-403.996393945363\\
67.625	0.1086	-429.741479811109\\
67.625	0.1092	-455.486565676856\\
67.625	0.1098	-481.231651542606\\
67.625	0.1104	-506.976737408353\\
67.625	0.111	-532.721823274103\\
67.625	0.1116	-558.46690913985\\
67.625	0.1122	-584.211995005597\\
67.625	0.1128	-609.957080871347\\
67.625	0.1134	-635.70216673709\\
67.625	0.114	-661.44725260284\\
67.625	0.1146	-687.192338468587\\
67.625	0.1152	-712.937424334334\\
67.625	0.1158	-738.682510200084\\
67.625	0.1164	-764.427596065831\\
67.625	0.117	-790.172681931581\\
67.625	0.1176	-815.917767797328\\
67.625	0.1182	-841.662853663074\\
67.625	0.1188	-867.407939528821\\
67.625	0.1194	-893.153025394568\\
67.625	0.12	-918.898111260318\\
67.625	0.1206	-944.643197126064\\
67.625	0.1212	-970.388282991811\\
67.625	0.1218	-996.133368857561\\
67.625	0.1224	-1021.87845472331\\
67.625	0.123	-1047.62354058906\\
68	0.093	251.221948997209\\
68	0.0936	226.203233998029\\
68	0.0942	201.184518998853\\
68	0.0948	176.165803999676\\
68	0.0954	151.1470890005\\
68	0.096	126.12837400132\\
68	0.0966	101.109659002144\\
68	0.0972	76.0909440029718\\
68	0.0978	51.072229003792\\
68	0.0984	26.0535140046159\\
68	0.099	1.03479900543607\\
68	0.0996	-23.9839159937364\\
68	0.1002	-49.0026309929126\\
68	0.1008	-74.0213459920924\\
68	0.1014	-99.0400609912685\\
68	0.102	-124.058775990445\\
68	0.1026	-149.077490989621\\
68	0.1032	-174.096205988797\\
68	0.1038	-199.114920987977\\
68	0.1044	-224.133635987149\\
68	0.105	-249.152350986329\\
68	0.1056	-274.171065985505\\
68	0.1062	-299.189780984681\\
68	0.1068	-324.208495983858\\
68	0.1074	-349.227210983034\\
68	0.108	-374.245925982214\\
68	0.1086	-399.26464098139\\
68	0.1092	-424.283355980562\\
68	0.1098	-449.302070979742\\
68	0.1104	-474.320785978918\\
68	0.111	-499.339500978098\\
68	0.1116	-524.35821597727\\
68	0.1122	-549.376930976447\\
68	0.1128	-574.395645975626\\
68	0.1134	-599.414360974803\\
68	0.114	-624.433075973979\\
68	0.1146	-649.451790973155\\
68	0.1152	-674.470505972331\\
68	0.1158	-699.489220971511\\
68	0.1164	-724.507935970683\\
68	0.117	-749.526650969863\\
68	0.1176	-774.545365969039\\
68	0.1182	-799.564080968215\\
68	0.1188	-824.582795967392\\
68	0.1194	-849.601510966568\\
68	0.12	-874.620225965748\\
68	0.1206	-899.63894096492\\
68	0.1212	-924.657655964096\\
68	0.1218	-949.676370963276\\
68	0.1224	-974.695085962452\\
68	0.123	-999.713800961628\\
68.375	0.093	262.813145296081\\
68.375	0.0936	238.520801163475\\
68.375	0.0942	214.228457030869\\
68.375	0.0948	189.93611289826\\
68.375	0.0954	165.643768765658\\
68.375	0.096	141.351424633049\\
68.375	0.0966	117.059080500447\\
68.375	0.0972	92.7667363678411\\
68.375	0.0978	68.4743922352318\\
68.375	0.0984	44.1820481026298\\
68.375	0.099	19.8897039700205\\
68.375	0.0996	-4.40264016258152\\
68.375	0.1002	-28.6949842951872\\
68.375	0.1008	-52.9873284277965\\
68.375	0.1014	-77.2796725603985\\
68.375	0.102	-101.572016693008\\
68.375	0.1026	-125.86436082561\\
68.375	0.1032	-150.156704958215\\
68.375	0.1038	-174.449049090825\\
68.375	0.1044	-198.741393223427\\
68.375	0.105	-223.033737356036\\
68.375	0.1056	-247.326081488642\\
68.375	0.1062	-271.618425621244\\
68.375	0.1068	-295.910769753853\\
68.375	0.1074	-320.203113886455\\
68.375	0.108	-344.495458019064\\
68.375	0.1086	-368.78780215167\\
68.375	0.1092	-393.080146284272\\
68.375	0.1098	-417.372490416881\\
68.375	0.1104	-441.664834549483\\
68.375	0.111	-465.957178682092\\
68.375	0.1116	-490.249522814698\\
68.375	0.1122	-514.5418669473\\
68.375	0.1128	-538.834211079909\\
68.375	0.1134	-563.126555212515\\
68.375	0.114	-587.418899345121\\
68.375	0.1146	-611.711243477726\\
68.375	0.1152	-636.003587610328\\
68.375	0.1158	-660.295931742938\\
68.375	0.1164	-684.588275875543\\
68.375	0.117	-708.880620008149\\
68.375	0.1176	-733.172964140755\\
68.375	0.1182	-757.465308273357\\
68.375	0.1188	-781.757652405966\\
68.375	0.1194	-806.049996538572\\
68.375	0.12	-830.342340671177\\
68.375	0.1206	-854.634684803783\\
68.375	0.1212	-878.927028936385\\
68.375	0.1218	-903.219373068994\\
68.375	0.1224	-927.5117172016\\
68.375	0.123	-951.804061334205\\
68.75	0.093	274.404341594956\\
68.75	0.0936	250.838368328921\\
68.75	0.0942	227.272395062886\\
68.75	0.0948	203.706421796851\\
68.75	0.0954	180.140448530816\\
68.75	0.096	156.574475264781\\
68.75	0.0966	133.008501998745\\
68.75	0.0972	109.442528732714\\
68.75	0.0978	85.8765554666788\\
68.75	0.0984	62.3105822006437\\
68.75	0.099	38.7446089346086\\
68.75	0.0996	15.1786356685734\\
68.75	0.1002	-8.3873375974581\\
68.75	0.1008	-31.9533108634969\\
68.75	0.1014	-55.5192841295284\\
68.75	0.102	-79.0852573955672\\
68.75	0.1026	-102.651230661599\\
68.75	0.1032	-126.217203927634\\
68.75	0.1038	-149.783177193669\\
68.75	0.1044	-173.3491504597\\
68.75	0.105	-196.915123725739\\
68.75	0.1056	-220.481096991774\\
68.75	0.1062	-244.047070257806\\
68.75	0.1068	-267.613043523845\\
68.75	0.1074	-291.179016789876\\
68.75	0.108	-314.744990055911\\
68.75	0.1086	-338.310963321946\\
68.75	0.1092	-361.876936587978\\
68.75	0.1098	-385.442909854017\\
68.75	0.1104	-409.008883120048\\
68.75	0.111	-432.574856386087\\
68.75	0.1116	-456.140829652119\\
68.75	0.1122	-479.706802918154\\
68.75	0.1128	-503.272776184189\\
68.75	0.1134	-526.838749450224\\
68.75	0.114	-550.404722716259\\
68.75	0.1146	-573.970695982294\\
68.75	0.1152	-597.536669248326\\
68.75	0.1158	-621.102642514361\\
68.75	0.1164	-644.668615780396\\
68.75	0.117	-668.234589046431\\
68.75	0.1176	-691.800562312466\\
68.75	0.1182	-715.366535578498\\
68.75	0.1188	-738.932508844537\\
68.75	0.1194	-762.498482110568\\
68.75	0.12	-786.064455376607\\
68.75	0.1206	-809.630428642638\\
68.75	0.1212	-833.196401908674\\
68.75	0.1218	-856.762375174709\\
68.75	0.1224	-880.328348440744\\
68.75	0.123	-903.894321706779\\
69.125	0.093	285.995537893832\\
69.125	0.0936	263.155935494367\\
69.125	0.0942	240.316333094906\\
69.125	0.0948	217.476730695438\\
69.125	0.0954	194.637128295977\\
69.125	0.096	171.797525896513\\
69.125	0.0966	148.957923497048\\
69.125	0.0972	126.118321097587\\
69.125	0.0978	103.278718698122\\
69.125	0.0984	80.4391162986576\\
69.125	0.099	57.599513899193\\
69.125	0.0996	34.759911499732\\
69.125	0.1002	11.920309100271\\
69.125	0.1008	-10.9192932991973\\
69.125	0.1014	-33.7588956986583\\
69.125	0.102	-56.5984980981229\\
69.125	0.1026	-79.4381004975876\\
69.125	0.1032	-102.277702897049\\
69.125	0.1038	-125.117305296517\\
69.125	0.1044	-147.956907695978\\
69.125	0.105	-170.796510095442\\
69.125	0.1056	-193.636112494904\\
69.125	0.1062	-216.475714894368\\
69.125	0.1068	-239.315317293833\\
69.125	0.1074	-262.154919693294\\
69.125	0.108	-284.994522092762\\
69.125	0.1086	-307.834124492223\\
69.125	0.1092	-330.673726891684\\
69.125	0.1098	-353.513329291152\\
69.125	0.1104	-376.352931690613\\
69.125	0.111	-399.192534090078\\
69.125	0.1116	-422.032136489543\\
69.125	0.1122	-444.871738889004\\
69.125	0.1128	-467.711341288468\\
69.125	0.1134	-490.550943687933\\
69.125	0.114	-513.390546087398\\
69.125	0.1146	-536.230148486859\\
69.125	0.1152	-559.069750886323\\
69.125	0.1158	-581.909353285788\\
69.125	0.1164	-604.748955685249\\
69.125	0.117	-627.588558084717\\
69.125	0.1176	-650.428160484178\\
69.125	0.1182	-673.267762883639\\
69.125	0.1188	-696.107365283107\\
69.125	0.1194	-718.946967682568\\
69.125	0.12	-741.786570082033\\
69.125	0.1206	-764.626172481498\\
69.125	0.1212	-787.465774880959\\
69.125	0.1218	-810.305377280423\\
69.125	0.1224	-833.144979679888\\
69.125	0.123	-855.984582079353\\
69.5	0.093	297.586734192708\\
69.5	0.0936	275.473502659814\\
69.5	0.0942	253.360271126923\\
69.5	0.0948	231.247039594029\\
69.5	0.0954	209.133808061139\\
69.5	0.096	187.020576528244\\
69.5	0.0966	164.90734499535\\
69.5	0.0972	142.79411346246\\
69.5	0.0978	120.680881929566\\
69.5	0.0984	98.5676503966752\\
69.5	0.099	76.454418863781\\
69.5	0.0996	54.3411873308905\\
69.5	0.1002	32.2279557979964\\
69.5	0.1008	10.1147242651023\\
69.5	0.1014	-11.9985072677882\\
69.5	0.102	-34.1117388006824\\
69.5	0.1026	-56.2249703335729\\
69.5	0.1032	-78.338201866467\\
69.5	0.1038	-100.451433399361\\
69.5	0.1044	-122.564664932252\\
69.5	0.105	-144.677896465146\\
69.5	0.1056	-166.791127998036\\
69.5	0.1062	-188.90435953093\\
69.5	0.1068	-211.017591063825\\
69.5	0.1074	-233.130822596715\\
69.5	0.108	-255.244054129609\\
69.5	0.1086	-277.3572856625\\
69.5	0.1092	-299.47051719539\\
69.5	0.1098	-321.583748728288\\
69.5	0.1104	-343.696980261178\\
69.5	0.111	-365.810211794073\\
69.5	0.1116	-387.923443326963\\
69.5	0.1122	-410.036674859854\\
69.5	0.1128	-432.149906392751\\
69.5	0.1134	-454.263137925642\\
69.5	0.114	-476.376369458536\\
69.5	0.1146	-498.489600991426\\
69.5	0.1152	-520.602832524317\\
69.5	0.1158	-542.716064057215\\
69.5	0.1164	-564.829295590105\\
69.5	0.117	-586.942527122999\\
69.5	0.1176	-609.05575865589\\
69.5	0.1182	-631.16899018878\\
69.5	0.1188	-653.282221721674\\
69.5	0.1194	-675.395453254569\\
69.5	0.12	-697.508684787463\\
69.5	0.1206	-719.621916320353\\
69.5	0.1212	-741.735147853244\\
69.5	0.1218	-763.848379386138\\
69.5	0.1224	-785.961610919028\\
69.5	0.123	-808.074842451926\\
69.875	0.093	309.177930491587\\
69.875	0.0936	287.79106982526\\
69.875	0.0942	266.40420915894\\
69.875	0.0948	245.017348492616\\
69.875	0.0954	223.630487826296\\
69.875	0.096	202.243627159973\\
69.875	0.0966	180.856766493653\\
69.875	0.0972	159.469905827333\\
69.875	0.0978	138.083045161009\\
69.875	0.0984	116.696184494689\\
69.875	0.099	95.3093238283655\\
69.875	0.0996	73.9224631620455\\
69.875	0.1002	52.5356024957255\\
69.875	0.1008	31.1487418294018\\
69.875	0.1014	9.76188116308185\\
69.875	0.102	-11.6249795032418\\
69.875	0.1026	-33.0118401695618\\
69.875	0.1032	-54.3987008358818\\
69.875	0.1038	-75.7855615022054\\
69.875	0.1044	-97.1724221685254\\
69.875	0.105	-118.559282834849\\
69.875	0.1056	-139.946143501169\\
69.875	0.1062	-161.333004167489\\
69.875	0.1068	-182.719864833813\\
69.875	0.1074	-204.106725500133\\
69.875	0.108	-225.493586166456\\
69.875	0.1086	-246.88044683278\\
69.875	0.1092	-268.267307499096\\
69.875	0.1098	-289.654168165423\\
69.875	0.1104	-311.041028831743\\
69.875	0.111	-332.427889498067\\
69.875	0.1116	-353.814750164387\\
69.875	0.1122	-375.201610830707\\
69.875	0.1128	-396.588471497031\\
69.875	0.1134	-417.975332163351\\
69.875	0.114	-439.362192829674\\
69.875	0.1146	-460.749053495994\\
69.875	0.1152	-482.135914162314\\
69.875	0.1158	-503.522774828638\\
69.875	0.1164	-524.909635494958\\
69.875	0.117	-546.296496161282\\
69.875	0.1176	-567.683356827602\\
69.875	0.1182	-589.070217493922\\
69.875	0.1188	-610.457078160245\\
69.875	0.1194	-631.843938826565\\
69.875	0.12	-653.230799492889\\
69.875	0.1206	-674.617660159209\\
69.875	0.1212	-696.004520825529\\
69.875	0.1218	-717.391381491852\\
69.875	0.1224	-738.778242158172\\
69.875	0.123	-760.1651028245\\
70.25	0.093	320.769126790463\\
70.25	0.0936	300.10863699071\\
70.25	0.0942	279.44814719096\\
70.25	0.0948	258.787657391207\\
70.25	0.0954	238.127167591458\\
70.25	0.096	217.466677791705\\
70.25	0.0966	196.806187991955\\
70.25	0.0972	176.145698192206\\
70.25	0.0978	155.485208392452\\
70.25	0.0984	134.824718592707\\
70.25	0.099	114.164228792954\\
70.25	0.0996	93.503738993204\\
70.25	0.1002	72.8432491934545\\
70.25	0.1008	52.1827593937014\\
70.25	0.1014	31.5222695939519\\
70.25	0.102	10.8617797941988\\
70.25	0.1026	-9.79871000555067\\
70.25	0.1032	-30.4591998053002\\
70.25	0.1038	-51.1196896050533\\
70.25	0.1044	-71.7801794047991\\
70.25	0.105	-92.4406692045523\\
70.25	0.1056	-113.101159004302\\
70.25	0.1062	-133.761648804051\\
70.25	0.1068	-154.422138603804\\
70.25	0.1074	-175.082628403554\\
70.25	0.108	-195.743118203307\\
70.25	0.1086	-216.403608003056\\
70.25	0.1092	-237.064097802806\\
70.25	0.1098	-257.724587602559\\
70.25	0.1104	-278.385077402305\\
70.25	0.111	-299.045567202058\\
70.25	0.1116	-319.706057001807\\
70.25	0.1122	-340.366546801557\\
70.25	0.1128	-361.02703660131\\
70.25	0.1134	-381.68752640106\\
70.25	0.114	-402.348016200813\\
70.25	0.1146	-423.008506000562\\
70.25	0.1152	-443.668995800312\\
70.25	0.1158	-464.329485600065\\
70.25	0.1164	-484.989975399811\\
70.25	0.117	-505.650465199564\\
70.25	0.1176	-526.310954999313\\
70.25	0.1182	-546.971444799063\\
70.25	0.1188	-567.631934598816\\
70.25	0.1194	-588.292424398565\\
70.25	0.12	-608.952914198318\\
70.25	0.1206	-629.613403998068\\
70.25	0.1212	-650.273893797817\\
70.25	0.1218	-670.934383597567\\
70.25	0.1224	-691.594873397316\\
70.25	0.123	-712.25536319707\\
70.625	0.093	332.360323089335\\
70.625	0.0936	312.426204156152\\
70.625	0.0942	292.492085222973\\
70.625	0.0948	272.557966289794\\
70.625	0.0954	252.623847356615\\
70.625	0.096	232.689728423433\\
70.625	0.0966	212.755609490254\\
70.625	0.0972	192.821490557079\\
70.625	0.0978	172.887371623896\\
70.625	0.0984	152.953252690717\\
70.625	0.099	133.019133757534\\
70.625	0.0996	113.085014824359\\
70.625	0.1002	93.15089589118\\
70.625	0.1008	73.2167769579974\\
70.625	0.1014	53.2826580248184\\
70.625	0.102	33.3485390916394\\
70.625	0.1026	13.4144201584604\\
70.625	0.1032	-6.51969877471856\\
70.625	0.1038	-26.4538177079012\\
70.625	0.1044	-46.3879366410765\\
70.625	0.105	-66.3220555742591\\
70.625	0.1056	-86.2561745074381\\
70.625	0.1062	-106.190293440617\\
70.625	0.1068	-126.1244123738\\
70.625	0.1074	-146.058531306975\\
70.625	0.108	-165.992650240158\\
70.625	0.1086	-185.926769173337\\
70.625	0.1092	-205.860888106516\\
70.625	0.1098	-225.795007039695\\
70.625	0.1104	-245.729125972874\\
70.625	0.111	-265.663244906056\\
70.625	0.1116	-285.597363839235\\
70.625	0.1122	-305.531482772411\\
70.625	0.1128	-325.465601705593\\
70.625	0.1134	-345.399720638772\\
70.625	0.114	-365.333839571955\\
70.625	0.1146	-385.267958505134\\
70.625	0.1152	-405.202077438309\\
70.625	0.1158	-425.136196371492\\
70.625	0.1164	-445.070315304671\\
70.625	0.117	-465.00443423785\\
70.625	0.1176	-484.938553171029\\
70.625	0.1182	-504.872672104208\\
70.625	0.1188	-524.80679103739\\
70.625	0.1194	-544.740909970566\\
70.625	0.12	-564.675028903748\\
70.625	0.1206	-584.609147836927\\
70.625	0.1212	-604.543266770106\\
70.625	0.1218	-624.477385703289\\
70.625	0.1224	-644.411504636464\\
70.625	0.123	-664.345623569647\\
71	0.093	343.951519388211\\
71	0.0936	324.743771321599\\
71	0.0942	305.536023254994\\
71	0.0948	286.328275188382\\
71	0.0954	267.120527121777\\
71	0.096	247.912779055165\\
71	0.0966	228.705030988556\\
71	0.0972	209.497282921951\\
71	0.0978	190.289534855339\\
71	0.0984	171.081786788731\\
71	0.099	151.874038722122\\
71	0.0996	132.666290655514\\
71	0.1002	113.458542588909\\
71	0.1008	94.2507945222969\\
71	0.1014	75.0430464556885\\
71	0.102	55.83529838908\\
71	0.1026	36.6275503224715\\
71	0.1032	17.419802255863\\
71	0.1038	-1.78794581074544\\
71	0.1044	-20.9956938773539\\
71	0.105	-40.2034419439624\\
71	0.1056	-59.4111900105709\\
71	0.1062	-78.6189380771793\\
71	0.1068	-97.8266861437878\\
71	0.1074	-117.034434210396\\
71	0.108	-136.242182277005\\
71	0.1086	-155.449930343613\\
71	0.1092	-174.657678410222\\
71	0.1098	-193.86542647683\\
71	0.1104	-213.073174543439\\
71	0.111	-232.280922610051\\
71	0.1116	-251.488670676656\\
71	0.1122	-270.696418743264\\
71	0.1128	-289.904166809873\\
71	0.1134	-309.111914876481\\
71	0.114	-328.319662943093\\
71	0.1146	-347.527411009698\\
71	0.1152	-366.735159076306\\
71	0.1158	-385.942907142919\\
71	0.1164	-405.150655209523\\
71	0.117	-424.358403276136\\
71	0.1176	-443.56615134274\\
71	0.1182	-462.773899409349\\
71	0.1188	-481.981647475961\\
71	0.1194	-501.189395542566\\
71	0.12	-520.397143609178\\
71	0.1206	-539.604891675783\\
71	0.1212	-558.812639742391\\
71	0.1218	-578.020387809003\\
71	0.1224	-597.228135875608\\
71	0.123	-616.43588394222\\
71.375	0.093	355.542715687086\\
71.375	0.0936	337.061338487048\\
71.375	0.0942	318.579961287011\\
71.375	0.0948	300.098584086973\\
71.375	0.0954	281.617206886935\\
71.375	0.096	263.135829686897\\
71.375	0.0966	244.654452486859\\
71.375	0.0972	226.173075286824\\
71.375	0.0978	207.691698086783\\
71.375	0.0984	189.210320886748\\
71.375	0.099	170.728943686707\\
71.375	0.0996	152.247566486672\\
71.375	0.1002	133.766189286634\\
71.375	0.1008	115.284812086597\\
71.375	0.1014	96.8034348865585\\
71.375	0.102	78.3220576865206\\
71.375	0.1026	59.8406804864826\\
71.375	0.1032	41.3593032864483\\
71.375	0.1038	22.8779260864067\\
71.375	0.1044	4.39654888637233\\
71.375	0.105	-14.0848283136656\\
71.375	0.1056	-32.5662055137036\\
71.375	0.1062	-51.0475827137379\\
71.375	0.1068	-69.5289599137795\\
71.375	0.1074	-88.0103371138139\\
71.375	0.108	-106.491714313855\\
71.375	0.1086	-124.97309151389\\
71.375	0.1092	-143.454468713928\\
71.375	0.1098	-161.935845913966\\
71.375	0.1104	-180.417223114004\\
71.375	0.111	-198.898600314042\\
71.375	0.1116	-217.37997751408\\
71.375	0.1122	-235.861354714114\\
71.375	0.1128	-254.342731914156\\
71.375	0.1134	-272.82410911419\\
71.375	0.114	-291.305486314232\\
71.375	0.1146	-309.786863514266\\
71.375	0.1152	-328.268240714304\\
71.375	0.1158	-346.749617914342\\
71.375	0.1164	-365.23099511438\\
71.375	0.117	-383.712372314418\\
71.375	0.1176	-402.193749514452\\
71.375	0.1182	-420.67512671449\\
71.375	0.1188	-439.156503914528\\
71.375	0.1194	-457.637881114566\\
71.375	0.12	-476.119258314604\\
71.375	0.1206	-494.600635514642\\
71.375	0.1212	-513.082012714676\\
71.375	0.1218	-531.563389914718\\
71.375	0.1224	-550.044767114752\\
71.375	0.123	-568.526144314794\\
71.75	0.093	367.133911985966\\
71.75	0.0936	349.378905652498\\
71.75	0.0942	331.623899319031\\
71.75	0.0948	313.868892985563\\
71.75	0.0954	296.1138866521\\
71.75	0.096	278.358880318629\\
71.75	0.0966	260.603873985165\\
71.75	0.0972	242.848867651701\\
71.75	0.0978	225.093861318233\\
71.75	0.0984	207.338854984766\\
71.75	0.099	189.583848651298\\
71.75	0.0996	171.828842317835\\
71.75	0.1002	154.073835984367\\
71.75	0.1008	136.3188296509\\
71.75	0.1014	118.563823317432\\
71.75	0.102	100.808816983965\\
71.75	0.1026	83.053810650501\\
71.75	0.1032	65.2988043170335\\
71.75	0.1038	47.543797983566\\
71.75	0.1044	29.7887916501022\\
71.75	0.105	12.0337853166347\\
71.75	0.1056	-5.72122101683271\\
71.75	0.1062	-23.4762273502965\\
71.75	0.1068	-41.2312336837676\\
71.75	0.1074	-58.9862400172315\\
71.75	0.108	-76.7412463506989\\
71.75	0.1086	-94.4962526841664\\
71.75	0.1092	-112.25125901763\\
71.75	0.1098	-130.006265351098\\
71.75	0.1104	-147.761271684565\\
71.75	0.111	-165.516278018033\\
71.75	0.1116	-183.271284351496\\
71.75	0.1122	-201.02629068496\\
71.75	0.1128	-218.781297018431\\
71.75	0.1134	-236.536303351895\\
71.75	0.114	-254.291309685366\\
71.75	0.1146	-272.04631601883\\
71.75	0.1152	-289.801322352294\\
71.75	0.1158	-307.556328685765\\
71.75	0.1164	-325.311335019229\\
71.75	0.117	-343.066341352696\\
71.75	0.1176	-360.821347686164\\
71.75	0.1182	-378.576354019628\\
71.75	0.1188	-396.331360353095\\
71.75	0.1194	-414.086366686559\\
71.75	0.12	-431.84137302003\\
71.75	0.1206	-449.596379353494\\
71.75	0.1212	-467.351385686961\\
71.75	0.1218	-485.106392020429\\
71.75	0.1224	-502.861398353893\\
71.75	0.123	-520.616404687364\\
72.125	0.093	378.725108284842\\
72.125	0.0936	361.696472817945\\
72.125	0.0942	344.667837351051\\
72.125	0.0948	327.639201884154\\
72.125	0.0954	310.610566417257\\
72.125	0.096	293.58193095036\\
72.125	0.0966	276.553295483467\\
72.125	0.0972	259.524660016574\\
72.125	0.0978	242.496024549677\\
72.125	0.0984	225.467389082783\\
72.125	0.099	208.438753615883\\
72.125	0.0996	191.41011814899\\
72.125	0.1002	174.381482682096\\
72.125	0.1008	157.352847215199\\
72.125	0.1014	140.324211748306\\
72.125	0.102	123.295576281405\\
72.125	0.1026	106.266940814512\\
72.125	0.1032	89.2383053476187\\
72.125	0.1038	72.2096698807218\\
72.125	0.1044	55.1810344138285\\
72.125	0.105	38.1523989469279\\
72.125	0.1056	21.1237634800345\\
72.125	0.1062	4.09512801314122\\
72.125	0.1068	-12.9335074537557\\
72.125	0.1074	-29.9621429206491\\
72.125	0.108	-46.9907783875497\\
72.125	0.1086	-64.019413854443\\
72.125	0.1092	-81.0480493213363\\
72.125	0.1098	-98.0766847882333\\
72.125	0.1104	-115.105320255127\\
72.125	0.111	-132.133955722027\\
72.125	0.1116	-149.16259118892\\
72.125	0.1122	-166.191226655814\\
72.125	0.1128	-183.219862122711\\
72.125	0.1134	-200.248497589604\\
72.125	0.114	-217.277133056505\\
72.125	0.1146	-234.305768523398\\
72.125	0.1152	-251.334403990291\\
72.125	0.1158	-268.363039457188\\
72.125	0.1164	-285.391674924082\\
72.125	0.117	-302.420310390979\\
72.125	0.1176	-319.448945857876\\
72.125	0.1182	-336.477581324769\\
72.125	0.1188	-353.506216791666\\
72.125	0.1194	-370.534852258559\\
72.125	0.12	-387.563487725456\\
72.125	0.1206	-404.592123192353\\
72.125	0.1212	-421.620758659246\\
72.125	0.1218	-438.649394126143\\
72.125	0.1224	-455.678029593037\\
72.125	0.123	-472.706665059934\\
72.5	0.093	390.316304583717\\
72.5	0.0936	374.014039983391\\
72.5	0.0942	357.711775383068\\
72.5	0.0948	341.409510782742\\
72.5	0.0954	325.107246182419\\
72.5	0.096	308.804981582092\\
72.5	0.0966	292.502716981769\\
72.5	0.0972	276.200452381447\\
72.5	0.0978	259.89818778112\\
72.5	0.0984	243.595923180797\\
72.5	0.099	227.293658580471\\
72.5	0.0996	210.991393980148\\
72.5	0.1002	194.689129379825\\
72.5	0.1008	178.386864779499\\
72.5	0.1014	162.084600179176\\
72.5	0.102	145.78233557885\\
72.5	0.1026	129.480070978527\\
72.5	0.1032	113.1778063782\\
72.5	0.1038	96.8755417778739\\
72.5	0.1044	80.5732771775511\\
72.5	0.105	64.2710125772246\\
72.5	0.1056	47.9687479769018\\
72.5	0.1062	31.666483376579\\
72.5	0.1068	15.3642187762525\\
72.5	0.1074	-0.938045824070286\\
72.5	0.108	-17.2403104243967\\
72.5	0.1086	-33.5425750247196\\
72.5	0.1092	-49.8448396250424\\
72.5	0.1098	-66.1471042253688\\
72.5	0.1104	-82.4493688256916\\
72.5	0.111	-98.7516334260181\\
72.5	0.1116	-115.053898026341\\
72.5	0.1122	-131.356162626664\\
72.5	0.1128	-147.65842722699\\
72.5	0.1134	-163.960691827313\\
72.5	0.114	-180.262956427639\\
72.5	0.1146	-196.565221027966\\
72.5	0.1152	-212.867485628285\\
72.5	0.1158	-229.169750228615\\
72.5	0.1164	-245.472014828938\\
72.5	0.117	-261.774279429264\\
72.5	0.1176	-278.076544029587\\
72.5	0.1182	-294.37880862991\\
72.5	0.1188	-310.681073230237\\
72.5	0.1194	-326.983337830559\\
72.5	0.12	-343.285602430886\\
72.5	0.1206	-359.587867031209\\
72.5	0.1212	-375.890131631531\\
72.5	0.1218	-392.192396231858\\
72.5	0.1224	-408.494660832181\\
72.5	0.123	-424.796925432507\\
72.875	0.093	401.907500882593\\
72.875	0.0936	386.331607148837\\
72.875	0.0942	370.755713415085\\
72.875	0.0948	355.179819681329\\
72.875	0.0954	339.603925947576\\
72.875	0.096	324.028032213821\\
72.875	0.0966	308.452138480068\\
72.875	0.0972	292.876244746316\\
72.875	0.0978	277.30035101256\\
72.875	0.0984	261.724457278808\\
72.875	0.099	246.148563545052\\
72.875	0.0996	230.572669811303\\
72.875	0.1002	214.996776077551\\
72.875	0.1008	199.420882343795\\
72.875	0.1014	183.844988610042\\
72.875	0.102	168.269094876287\\
72.875	0.1026	152.693201142534\\
72.875	0.1032	137.117307408782\\
72.875	0.1038	121.541413675026\\
72.875	0.1044	105.965519941274\\
72.875	0.105	90.3896262075177\\
72.875	0.1056	74.8137324737654\\
72.875	0.1062	59.2378387400167\\
72.875	0.1068	43.6619450062608\\
72.875	0.1074	28.0860512725085\\
72.875	0.108	12.5101575387525\\
72.875	0.1086	-3.06573619499977\\
72.875	0.1092	-18.6416299287521\\
72.875	0.1098	-34.217523662508\\
72.875	0.1104	-49.7934173962603\\
72.875	0.111	-65.3693111300163\\
72.875	0.1116	-80.9452048637686\\
72.875	0.1122	-96.5210985975209\\
72.875	0.1128	-112.096992331273\\
72.875	0.1134	-127.672886065026\\
72.875	0.114	-143.248779798781\\
72.875	0.1146	-158.824673532534\\
72.875	0.1152	-174.400567266286\\
72.875	0.1158	-189.976461000042\\
72.875	0.1164	-205.552354733794\\
72.875	0.117	-221.12824846755\\
72.875	0.1176	-236.704142201303\\
72.875	0.1182	-252.280035935055\\
72.875	0.1188	-267.855929668811\\
72.875	0.1194	-283.43182340256\\
72.875	0.12	-299.007717136315\\
72.875	0.1206	-314.583610870068\\
72.875	0.1212	-330.15950460382\\
72.875	0.1218	-345.735398337576\\
72.875	0.1224	-361.311292071328\\
72.875	0.123	-376.887185805084\\
73.25	0.093	413.498697181469\\
73.25	0.0936	398.649174314283\\
73.25	0.0942	383.799651447101\\
73.25	0.0948	368.950128579916\\
73.25	0.0954	354.100605712738\\
73.25	0.096	339.251082845552\\
73.25	0.0966	324.401559978371\\
73.25	0.0972	309.552037111189\\
73.25	0.0978	294.702514244007\\
73.25	0.0984	279.852991376825\\
73.25	0.099	265.00346850964\\
73.25	0.0996	250.153945642458\\
73.25	0.1002	235.304422775276\\
73.25	0.1008	220.454899908094\\
73.25	0.1014	205.605377040913\\
73.25	0.102	190.755854173727\\
73.25	0.1026	175.906331306545\\
73.25	0.1032	161.056808439367\\
73.25	0.1038	146.207285572182\\
73.25	0.1044	131.357762705\\
73.25	0.105	116.508239837814\\
73.25	0.1056	101.658716970636\\
73.25	0.1062	86.8091941034545\\
73.25	0.1068	71.9596712362691\\
73.25	0.1074	57.1101483690873\\
73.25	0.108	42.2606255019018\\
73.25	0.1086	27.4111026347236\\
73.25	0.1092	12.5615797675418\\
73.25	0.1098	-2.2879430996436\\
73.25	0.1104	-17.1374659668254\\
73.25	0.111	-31.9869888340072\\
73.25	0.1116	-46.836511701189\\
73.25	0.1122	-61.6860345683708\\
73.25	0.1128	-76.5355574355563\\
73.25	0.1134	-91.3850803027344\\
73.25	0.114	-106.23460316992\\
73.25	0.1146	-121.084126037102\\
73.25	0.1152	-135.933648904283\\
73.25	0.1158	-150.783171771469\\
73.25	0.1164	-165.632694638647\\
73.25	0.117	-180.482217505833\\
73.25	0.1176	-195.331740373014\\
73.25	0.1182	-210.181263240196\\
73.25	0.1188	-225.030786107378\\
73.25	0.1194	-239.88030897456\\
73.25	0.12	-254.729831841745\\
73.25	0.1206	-269.579354708927\\
73.25	0.1212	-284.428877576105\\
73.25	0.1218	-299.278400443291\\
73.25	0.1224	-314.127923310472\\
73.25	0.123	-328.977446177658\\
73.625	0.093	425.089893480344\\
73.625	0.0936	410.96674147973\\
73.625	0.0942	396.843589479118\\
73.625	0.0948	382.720437478507\\
73.625	0.0954	368.597285477896\\
73.625	0.096	354.474133477284\\
73.625	0.0966	340.350981476673\\
73.625	0.0972	326.227829476062\\
73.625	0.0978	312.10467747545\\
73.625	0.0984	297.981525474839\\
73.625	0.099	283.858373474224\\
73.625	0.0996	269.735221473617\\
73.625	0.1002	255.612069473005\\
73.625	0.1008	241.488917472394\\
73.625	0.1014	227.365765471783\\
73.625	0.102	213.242613471168\\
73.625	0.1026	199.11946147056\\
73.625	0.1032	184.996309469949\\
73.625	0.1038	170.873157469334\\
73.625	0.1044	156.750005468726\\
73.625	0.105	142.626853468111\\
73.625	0.1056	128.503701467504\\
73.625	0.1062	114.380549466892\\
73.625	0.1068	100.257397466277\\
73.625	0.1074	86.1342454656697\\
73.625	0.108	72.0110934650547\\
73.625	0.1086	57.8879414644434\\
73.625	0.1092	43.7647894638358\\
73.625	0.1098	29.6416374632208\\
73.625	0.1104	15.5184854626132\\
73.625	0.111	1.39533346199823\\
73.625	0.1116	-12.7278185386131\\
73.625	0.1122	-26.8509705392207\\
73.625	0.1128	-40.9741225398357\\
73.625	0.1134	-55.097274540447\\
73.625	0.114	-69.2204265410583\\
73.625	0.1146	-83.3435785416696\\
73.625	0.1152	-97.4667305422772\\
73.625	0.1158	-111.589882542892\\
73.625	0.1164	-125.713034543503\\
73.625	0.117	-139.836186544115\\
73.625	0.1176	-153.959338544726\\
73.625	0.1182	-168.082490545337\\
73.625	0.1188	-182.205642545949\\
73.625	0.1194	-196.32879454656\\
73.625	0.12	-210.451946547171\\
73.625	0.1206	-224.575098547783\\
73.625	0.1212	-238.698250548394\\
73.625	0.1218	-252.821402549005\\
73.625	0.1224	-266.944554549616\\
73.625	0.123	-281.067706550231\\
74	0.093	436.68108977922\\
74	0.0936	423.284308645176\\
74	0.0942	409.887527511139\\
74	0.0948	396.490746377094\\
74	0.0954	383.093965243057\\
74	0.096	369.697184109013\\
74	0.0966	356.300402974975\\
74	0.0972	342.903621840935\\
74	0.0978	329.506840706894\\
74	0.0984	316.110059572853\\
74	0.099	302.713278438812\\
74	0.0996	289.316497304775\\
74	0.1002	275.919716170734\\
74	0.1008	262.522935036694\\
74	0.1014	249.126153902653\\
74	0.102	235.729372768612\\
74	0.1026	222.332591634571\\
74	0.1032	208.935810500534\\
74	0.1038	195.53902936649\\
74	0.1044	182.142248232452\\
74	0.105	168.745467098408\\
74	0.1056	155.348685964371\\
74	0.1062	141.95190483033\\
74	0.1068	128.555123696289\\
74	0.1074	115.158342562248\\
74	0.108	101.761561428208\\
74	0.1086	88.3647802941668\\
74	0.1092	74.9679991601297\\
74	0.1098	61.5712180260853\\
74	0.1104	48.1744368920481\\
74	0.111	34.7776557580037\\
74	0.1116	21.3808746239665\\
74	0.1122	7.98409348992573\\
74	0.1128	-5.41268764411507\\
74	0.1134	-18.8094687781559\\
74	0.114	-32.2062499121967\\
74	0.1146	-45.6030310462374\\
74	0.1152	-58.9998121802746\\
74	0.1158	-72.396593314319\\
74	0.1164	-85.7933744483562\\
74	0.117	-99.190155582397\\
74	0.1176	-112.586936716438\\
74	0.1182	-125.983717850479\\
74	0.1188	-139.380498984519\\
74	0.1194	-152.777280118557\\
74	0.12	-166.174061252601\\
74	0.1206	-179.570842386638\\
74	0.1212	-192.967623520679\\
74	0.1218	-206.36440465472\\
74	0.1224	-219.76118578876\\
74	0.123	-233.157966922801\\
};
\end{axis}

\begin{axis}[%
width=5.011742cm,
height=3.595207cm,
at={(6.594397cm,10.936529cm)},
scale only axis,
xmin=55,
xmax=75,
tick align=outside,
xlabel={$L_{cut}$},
xmajorgrids,
ymin=0.09,
ymax=0.13,
ylabel={$D_{rlx}$},
ymajorgrids,
zmin=-10,
zmax=1.7167261986551,
zlabel={$x_1,x_1$},
zmajorgrids,
view={-140}{50},
legend style={at={(1.03,1)},anchor=north west,legend cell align=left,align=left,draw=white!15!black}
]
\addplot3[only marks,mark=*,mark options={},mark size=1.5000pt,color=mycolor1] plot table[row sep=crcr,]{%
74	0.123	0.00740370360905991\\
72	0.113	-0.186626587578763\\
61	0.095	-0.150607769785267\\
56	0.093	-0.216975707230133\\
};
\addplot3[only marks,mark=*,mark options={},mark size=1.5000pt,color=black] plot table[row sep=crcr,]{%
69	0.104	-0.201969248627989\\
};

\addplot3[%
surf,
opacity=0.7,
shader=interp,
colormap={mymap}{[1pt] rgb(0pt)=(0.0901961,0.239216,0.0745098); rgb(1pt)=(0.0945149,0.242058,0.0739522); rgb(2pt)=(0.0988592,0.244894,0.0733566); rgb(3pt)=(0.103229,0.247724,0.0727241); rgb(4pt)=(0.107623,0.250549,0.0720557); rgb(5pt)=(0.112043,0.253367,0.0713525); rgb(6pt)=(0.116487,0.25618,0.0706154); rgb(7pt)=(0.120956,0.258986,0.0698456); rgb(8pt)=(0.125449,0.261787,0.0690441); rgb(9pt)=(0.129967,0.264581,0.0682118); rgb(10pt)=(0.134508,0.26737,0.06735); rgb(11pt)=(0.139074,0.270152,0.0664596); rgb(12pt)=(0.143663,0.272929,0.0655416); rgb(13pt)=(0.148275,0.275699,0.0645971); rgb(14pt)=(0.152911,0.278463,0.0636271); rgb(15pt)=(0.15757,0.281221,0.0626328); rgb(16pt)=(0.162252,0.283973,0.0616151); rgb(17pt)=(0.166957,0.286719,0.060575); rgb(18pt)=(0.171685,0.289458,0.0595136); rgb(19pt)=(0.176434,0.292191,0.0584321); rgb(20pt)=(0.181207,0.294918,0.0573313); rgb(21pt)=(0.186001,0.297639,0.0562123); rgb(22pt)=(0.190817,0.300353,0.0550763); rgb(23pt)=(0.195655,0.303061,0.0539242); rgb(24pt)=(0.200514,0.305763,0.052757); rgb(25pt)=(0.205395,0.308459,0.0515759); rgb(26pt)=(0.210296,0.311149,0.0503624); rgb(27pt)=(0.215212,0.313846,0.0490067); rgb(28pt)=(0.220142,0.316548,0.0475043); rgb(29pt)=(0.22509,0.319254,0.0458704); rgb(30pt)=(0.230056,0.321962,0.0441205); rgb(31pt)=(0.235042,0.324671,0.04227); rgb(32pt)=(0.240048,0.327379,0.0403343); rgb(33pt)=(0.245078,0.330085,0.0383287); rgb(34pt)=(0.250131,0.332786,0.0362688); rgb(35pt)=(0.25521,0.335482,0.0341698); rgb(36pt)=(0.260317,0.33817,0.0320472); rgb(37pt)=(0.265451,0.340849,0.0299163); rgb(38pt)=(0.270616,0.343517,0.0277927); rgb(39pt)=(0.275813,0.346172,0.0256916); rgb(40pt)=(0.281043,0.348814,0.0236284); rgb(41pt)=(0.286307,0.35144,0.0216186); rgb(42pt)=(0.291607,0.354048,0.0196776); rgb(43pt)=(0.296945,0.356637,0.0178207); rgb(44pt)=(0.302322,0.359206,0.0160634); rgb(45pt)=(0.307739,0.361753,0.0144211); rgb(46pt)=(0.313198,0.364275,0.0129091); rgb(47pt)=(0.318701,0.366772,0.0115428); rgb(48pt)=(0.324249,0.369242,0.0103377); rgb(49pt)=(0.329843,0.371682,0.00930909); rgb(50pt)=(0.335485,0.374093,0.00847245); rgb(51pt)=(0.341176,0.376471,0.00784314); rgb(52pt)=(0.346925,0.378826,0.00732741); rgb(53pt)=(0.352735,0.381168,0.00682184); rgb(54pt)=(0.358605,0.383497,0.00632729); rgb(55pt)=(0.364532,0.385812,0.00584464); rgb(56pt)=(0.370516,0.388113,0.00537476); rgb(57pt)=(0.376552,0.390399,0.00491852); rgb(58pt)=(0.38264,0.39267,0.00447681); rgb(59pt)=(0.388777,0.394925,0.00405048); rgb(60pt)=(0.394962,0.397164,0.00364042); rgb(61pt)=(0.401191,0.399386,0.00324749); rgb(62pt)=(0.407464,0.401592,0.00287258); rgb(63pt)=(0.413777,0.40378,0.00251655); rgb(64pt)=(0.420129,0.40595,0.00218028); rgb(65pt)=(0.426518,0.408102,0.00186463); rgb(66pt)=(0.432942,0.410234,0.00157049); rgb(67pt)=(0.439399,0.412348,0.00129873); rgb(68pt)=(0.445885,0.414441,0.00105022); rgb(69pt)=(0.452401,0.416515,0.000825833); rgb(70pt)=(0.458942,0.418567,0.000626441); rgb(71pt)=(0.465508,0.420599,0.00045292); rgb(72pt)=(0.472096,0.422609,0.000306141); rgb(73pt)=(0.478704,0.424596,0.000186979); rgb(74pt)=(0.485331,0.426562,9.63073e-05); rgb(75pt)=(0.491973,0.428504,3.49981e-05); rgb(76pt)=(0.498628,0.430422,3.92506e-06); rgb(77pt)=(0.505323,0.432315,0); rgb(78pt)=(0.512206,0.434168,0); rgb(79pt)=(0.519282,0.435983,0); rgb(80pt)=(0.526529,0.437764,0); rgb(81pt)=(0.533922,0.439512,0); rgb(82pt)=(0.54144,0.441232,0); rgb(83pt)=(0.549059,0.442927,0); rgb(84pt)=(0.556756,0.444599,0); rgb(85pt)=(0.564508,0.446252,0); rgb(86pt)=(0.572292,0.447889,0); rgb(87pt)=(0.580084,0.449514,0); rgb(88pt)=(0.587863,0.451129,0); rgb(89pt)=(0.595604,0.452737,0); rgb(90pt)=(0.603284,0.454343,0); rgb(91pt)=(0.610882,0.455948,0); rgb(92pt)=(0.618373,0.457556,0); rgb(93pt)=(0.625734,0.459171,0); rgb(94pt)=(0.632943,0.460795,0); rgb(95pt)=(0.639976,0.462432,0); rgb(96pt)=(0.64681,0.464084,0); rgb(97pt)=(0.653423,0.465756,0); rgb(98pt)=(0.659791,0.46745,0); rgb(99pt)=(0.665891,0.469169,0); rgb(100pt)=(0.6717,0.470916,0); rgb(101pt)=(0.677195,0.472696,0); rgb(102pt)=(0.682353,0.47451,0); rgb(103pt)=(0.687242,0.476355,0); rgb(104pt)=(0.691952,0.478225,0); rgb(105pt)=(0.696497,0.480118,0); rgb(106pt)=(0.700887,0.482033,0); rgb(107pt)=(0.705134,0.483968,0); rgb(108pt)=(0.709251,0.485921,0); rgb(109pt)=(0.713249,0.487891,0); rgb(110pt)=(0.71714,0.489876,0); rgb(111pt)=(0.720936,0.491875,0); rgb(112pt)=(0.724649,0.493887,0); rgb(113pt)=(0.72829,0.495909,0); rgb(114pt)=(0.731872,0.49794,0); rgb(115pt)=(0.735406,0.499979,0); rgb(116pt)=(0.738904,0.502025,0); rgb(117pt)=(0.742378,0.504075,0); rgb(118pt)=(0.74584,0.506128,0); rgb(119pt)=(0.749302,0.508182,0); rgb(120pt)=(0.752775,0.510237,0); rgb(121pt)=(0.756272,0.51229,0); rgb(122pt)=(0.759804,0.514339,0); rgb(123pt)=(0.763384,0.516385,0); rgb(124pt)=(0.767022,0.518424,0); rgb(125pt)=(0.770731,0.520455,0); rgb(126pt)=(0.774523,0.522478,0); rgb(127pt)=(0.77841,0.524489,0); rgb(128pt)=(0.782391,0.526491,0); rgb(129pt)=(0.786402,0.528496,0); rgb(130pt)=(0.790431,0.530506,0); rgb(131pt)=(0.794478,0.532521,0); rgb(132pt)=(0.798541,0.534539,0); rgb(133pt)=(0.802619,0.53656,0); rgb(134pt)=(0.806712,0.538584,0); rgb(135pt)=(0.81082,0.540609,0); rgb(136pt)=(0.81494,0.542635,0); rgb(137pt)=(0.819074,0.54466,0); rgb(138pt)=(0.823219,0.546686,0); rgb(139pt)=(0.827374,0.548709,0); rgb(140pt)=(0.831541,0.55073,0); rgb(141pt)=(0.835716,0.552749,0); rgb(142pt)=(0.8399,0.554763,0); rgb(143pt)=(0.844092,0.556774,0); rgb(144pt)=(0.848292,0.558779,0); rgb(145pt)=(0.852497,0.560778,0); rgb(146pt)=(0.856708,0.562771,0); rgb(147pt)=(0.860924,0.564756,0); rgb(148pt)=(0.865143,0.566733,0); rgb(149pt)=(0.869366,0.568701,0); rgb(150pt)=(0.873592,0.57066,0); rgb(151pt)=(0.877819,0.572608,0); rgb(152pt)=(0.882047,0.574545,0); rgb(153pt)=(0.886275,0.576471,0); rgb(154pt)=(0.890659,0.578362,0); rgb(155pt)=(0.895333,0.580203,0); rgb(156pt)=(0.900258,0.581999,0); rgb(157pt)=(0.905397,0.583755,0); rgb(158pt)=(0.910711,0.585479,0); rgb(159pt)=(0.916164,0.587176,0); rgb(160pt)=(0.921717,0.588852,0); rgb(161pt)=(0.927333,0.590513,0); rgb(162pt)=(0.932974,0.592166,0); rgb(163pt)=(0.938602,0.593815,0); rgb(164pt)=(0.94418,0.595468,0); rgb(165pt)=(0.949669,0.59713,0); rgb(166pt)=(0.955033,0.598808,0); rgb(167pt)=(0.960233,0.600507,0); rgb(168pt)=(0.965232,0.602233,0); rgb(169pt)=(0.969992,0.603992,0); rgb(170pt)=(0.974475,0.605791,0); rgb(171pt)=(0.978643,0.607636,0); rgb(172pt)=(0.98246,0.609532,0); rgb(173pt)=(0.985886,0.611486,0); rgb(174pt)=(0.988885,0.613503,0); rgb(175pt)=(0.991419,0.61559,0); rgb(176pt)=(0.99345,0.617753,0); rgb(177pt)=(0.99494,0.619997,0); rgb(178pt)=(0.995851,0.622329,0); rgb(179pt)=(0.996226,0.624763,0); rgb(180pt)=(0.996512,0.627352,0); rgb(181pt)=(0.996788,0.630095,0); rgb(182pt)=(0.997053,0.632982,0); rgb(183pt)=(0.997308,0.636004,0); rgb(184pt)=(0.997552,0.639152,0); rgb(185pt)=(0.997785,0.642416,0); rgb(186pt)=(0.998006,0.645786,0); rgb(187pt)=(0.998217,0.649253,0); rgb(188pt)=(0.998416,0.652807,0); rgb(189pt)=(0.998605,0.656439,0); rgb(190pt)=(0.998781,0.660138,0); rgb(191pt)=(0.998946,0.663897,0); rgb(192pt)=(0.9991,0.667704,0); rgb(193pt)=(0.999242,0.67155,0); rgb(194pt)=(0.999372,0.675427,0); rgb(195pt)=(0.99949,0.679323,0); rgb(196pt)=(0.999596,0.68323,0); rgb(197pt)=(0.99969,0.687139,0); rgb(198pt)=(0.999771,0.691039,0); rgb(199pt)=(0.999841,0.694921,0); rgb(200pt)=(0.999898,0.698775,0); rgb(201pt)=(0.999942,0.702592,0); rgb(202pt)=(0.999974,0.706363,0); rgb(203pt)=(0.999994,0.710077,0); rgb(204pt)=(1,0.713725,0); rgb(205pt)=(1,0.717341,0); rgb(206pt)=(1,0.720963,0); rgb(207pt)=(1,0.724591,0); rgb(208pt)=(1,0.728226,0); rgb(209pt)=(1,0.731867,0); rgb(210pt)=(1,0.735514,0); rgb(211pt)=(1,0.739167,0); rgb(212pt)=(1,0.742827,0); rgb(213pt)=(1,0.746493,0); rgb(214pt)=(1,0.750165,0); rgb(215pt)=(1,0.753843,0); rgb(216pt)=(1,0.757527,0); rgb(217pt)=(1,0.761217,0); rgb(218pt)=(1,0.764913,0); rgb(219pt)=(1,0.768615,0); rgb(220pt)=(1,0.772324,0); rgb(221pt)=(1,0.776038,0); rgb(222pt)=(1,0.779758,0); rgb(223pt)=(1,0.783484,0); rgb(224pt)=(1,0.787215,0); rgb(225pt)=(1,0.790953,0); rgb(226pt)=(1,0.794696,0); rgb(227pt)=(1,0.798445,0); rgb(228pt)=(1,0.8022,0); rgb(229pt)=(1,0.805961,0); rgb(230pt)=(1,0.809727,0); rgb(231pt)=(1,0.8135,0); rgb(232pt)=(1,0.817278,0); rgb(233pt)=(1,0.821063,0); rgb(234pt)=(1,0.824854,0); rgb(235pt)=(1,0.828652,0); rgb(236pt)=(1,0.832455,0); rgb(237pt)=(1,0.836265,0); rgb(238pt)=(1,0.840081,0); rgb(239pt)=(1,0.843903,0); rgb(240pt)=(1,0.847732,0); rgb(241pt)=(1,0.851566,0); rgb(242pt)=(1,0.855406,0); rgb(243pt)=(1,0.859253,0); rgb(244pt)=(1,0.863106,0); rgb(245pt)=(1,0.866964,0); rgb(246pt)=(1,0.870829,0); rgb(247pt)=(1,0.8747,0); rgb(248pt)=(1,0.878577,0); rgb(249pt)=(1,0.88246,0); rgb(250pt)=(1,0.886349,0); rgb(251pt)=(1,0.890243,0); rgb(252pt)=(1,0.894144,0); rgb(253pt)=(1,0.898051,0); rgb(254pt)=(1,0.901964,0); rgb(255pt)=(1,0.905882,0)},
mesh/rows=49]
table[row sep=crcr,header=false] {%
%
56	0.093	-0.216975707328501\\
56	0.0936	-0.399378303263845\\
56	0.0942	-0.581780899199202\\
56	0.0948	-0.764183495134546\\
56	0.0954	-0.946586091069904\\
56	0.096	-1.12898868700528\\
56	0.0966	-1.31139128294062\\
56	0.0972	-1.49379387887598\\
56	0.0978	-1.67619647481132\\
56	0.0984	-1.85859907074668\\
56	0.099	-2.04100166668202\\
56	0.0996	-2.22340426261738\\
56	0.1002	-2.40580685855272\\
56	0.1008	-2.58820945448808\\
56	0.1014	-2.77061205042344\\
56	0.102	-2.95301464635878\\
56	0.1026	-3.13541724229414\\
56	0.1032	-3.31781983822948\\
56	0.1038	-3.50022243416484\\
56	0.1044	-3.6826250301002\\
56	0.105	-3.86502762603556\\
56	0.1056	-4.04743022197091\\
56	0.1062	-4.22983281790626\\
56	0.1068	-4.41223541384161\\
56	0.1074	-4.59463800977696\\
56	0.108	-4.77704060571232\\
56	0.1086	-4.95944320164766\\
56	0.1092	-5.14184579758302\\
56	0.1098	-5.32424839351836\\
56	0.1104	-5.50665098945372\\
56	0.111	-5.68905358538908\\
56	0.1116	-5.87145618132442\\
56	0.1122	-6.05385877725978\\
56	0.1128	-6.23626137319513\\
56	0.1134	-6.41866396913049\\
56	0.114	-6.60106656506584\\
56	0.1146	-6.78346916100119\\
56	0.1152	-6.96587175693654\\
56	0.1158	-7.14827435287189\\
56	0.1164	-7.33067694880725\\
56	0.117	-7.5130795447426\\
56	0.1176	-7.69548214067795\\
56	0.1182	-7.8778847366133\\
56	0.1188	-8.06028733254865\\
56	0.1194	-8.242689928484\\
56	0.12	-8.42509252441936\\
56	0.1206	-8.60749512035471\\
56	0.1212	-8.78989771629006\\
56	0.1218	-8.97230031222543\\
56	0.1224	-9.15470290816077\\
56	0.123	-9.33710550409613\\
56.375	0.093	-0.176690250953826\\
56.375	0.0936	-0.356005010513528\\
56.375	0.0942	-0.53531977007323\\
56.375	0.0948	-0.714634529632917\\
56.375	0.0954	-0.893949289192619\\
56.375	0.096	-1.07326404875234\\
56.375	0.0966	-1.25257880831202\\
56.375	0.0972	-1.43189356787173\\
56.375	0.0978	-1.61120832743143\\
56.375	0.0984	-1.79052308699112\\
56.375	0.099	-1.96983784655082\\
56.375	0.0996	-2.1491526061105\\
56.375	0.1002	-2.32846736567021\\
56.375	0.1008	-2.50778212522991\\
56.375	0.1014	-2.6870968847896\\
56.375	0.102	-2.8664116443493\\
56.375	0.1026	-3.045726403909\\
56.375	0.1032	-3.22504116346869\\
56.375	0.1038	-3.40435592302839\\
56.375	0.1044	-3.58367068258809\\
56.375	0.105	-3.76298544214779\\
56.375	0.1056	-3.9423002017075\\
56.375	0.1062	-4.12161496126718\\
56.375	0.1068	-4.30092972082689\\
56.375	0.1074	-4.48024448038659\\
56.375	0.108	-4.65955923994628\\
56.375	0.1086	-4.83887399950598\\
56.375	0.1092	-5.01818875906567\\
56.375	0.1098	-5.19750351862537\\
56.375	0.1104	-5.37681827818507\\
56.375	0.111	-5.55613303774477\\
56.375	0.1116	-5.73544779730446\\
56.375	0.1122	-5.91476255686416\\
56.375	0.1128	-6.09407731642386\\
56.375	0.1134	-6.27339207598357\\
56.375	0.114	-6.45270683554327\\
56.375	0.1146	-6.63202159510296\\
56.375	0.1152	-6.81133635466266\\
56.375	0.1158	-6.99065111422236\\
56.375	0.1164	-7.16996587378205\\
56.375	0.117	-7.34928063334175\\
56.375	0.1176	-7.52859539290144\\
56.375	0.1182	-7.70791015246114\\
56.375	0.1188	-7.88722491202084\\
56.375	0.1194	-8.06653967158053\\
56.375	0.12	-8.24585443114023\\
56.375	0.1206	-8.42516919069993\\
56.375	0.1212	-8.60448395025962\\
56.375	0.1218	-8.78379870981934\\
56.375	0.1224	-8.96311346937902\\
56.375	0.123	-9.14242822893873\\
56.75	0.093	-0.136404794579178\\
56.75	0.0936	-0.312631717763225\\
56.75	0.0942	-0.488858640947257\\
56.75	0.0948	-0.665085564131303\\
56.75	0.0954	-0.841312487315349\\
56.75	0.096	-1.0175394104994\\
56.75	0.0966	-1.19376633368344\\
56.75	0.0972	-1.36999325686749\\
56.75	0.0978	-1.54622018005153\\
56.75	0.0984	-1.72244710323557\\
56.75	0.099	-1.89867402641961\\
56.75	0.0996	-2.07490094960366\\
56.75	0.1002	-2.25112787278769\\
56.75	0.1008	-2.42735479597174\\
56.75	0.1014	-2.60358171915578\\
56.75	0.102	-2.77980864233982\\
56.75	0.1026	-2.95603556552386\\
56.75	0.1032	-3.13226248870791\\
56.75	0.1038	-3.30848941189196\\
56.75	0.1044	-3.484716335076\\
56.75	0.105	-3.66094325826005\\
56.75	0.1056	-3.83717018144409\\
56.75	0.1062	-4.01339710462814\\
56.75	0.1068	-4.18962402781217\\
56.75	0.1074	-4.36585095099622\\
56.75	0.108	-4.54207787418026\\
56.75	0.1086	-4.7183047973643\\
56.75	0.1092	-4.89453172054834\\
56.75	0.1098	-5.07075864373239\\
56.75	0.1104	-5.24698556691644\\
56.75	0.111	-5.42321249010047\\
56.75	0.1116	-5.59943941328451\\
56.75	0.1122	-5.77566633646856\\
56.75	0.1128	-5.95189325965261\\
56.75	0.1134	-6.12812018283665\\
56.75	0.114	-6.3043471060207\\
56.75	0.1146	-6.48057402920475\\
56.75	0.1152	-6.65680095238878\\
56.75	0.1158	-6.83302787557282\\
56.75	0.1164	-7.00925479875687\\
56.75	0.117	-7.1854817219409\\
56.75	0.1176	-7.36170864512495\\
56.75	0.1182	-7.537935568309\\
56.75	0.1188	-7.71416249149304\\
56.75	0.1194	-7.89038941467707\\
56.75	0.12	-8.06661633786112\\
56.75	0.1206	-8.24284326104517\\
56.75	0.1212	-8.41907018422921\\
56.75	0.1218	-8.59529710741326\\
56.75	0.1224	-8.77152403059731\\
56.75	0.123	-8.94775095378135\\
57.125	0.093	-0.0961193382045309\\
57.125	0.0936	-0.269258425012907\\
57.125	0.0942	-0.442397511821298\\
57.125	0.0948	-0.615536598629689\\
57.125	0.0954	-0.788675685438079\\
57.125	0.096	-0.96181477224647\\
57.125	0.0966	-1.13495385905486\\
57.125	0.0972	-1.30809294586325\\
57.125	0.0978	-1.48123203267164\\
57.125	0.0984	-1.65437111948003\\
57.125	0.099	-1.82751020628841\\
57.125	0.0996	-2.0006492930968\\
57.125	0.1002	-2.17378837990519\\
57.125	0.1008	-2.34692746671358\\
57.125	0.1014	-2.52006655352197\\
57.125	0.102	-2.69320564033035\\
57.125	0.1026	-2.86634472713874\\
57.125	0.1032	-3.03948381394713\\
57.125	0.1038	-3.21262290075552\\
57.125	0.1044	-3.38576198756391\\
57.125	0.105	-3.5589010743723\\
57.125	0.1056	-3.73204016118069\\
57.125	0.1062	-3.90517924798908\\
57.125	0.1068	-4.07831833479747\\
57.125	0.1074	-4.25145742160585\\
57.125	0.108	-4.42459650841424\\
57.125	0.1086	-4.59773559522263\\
57.125	0.1092	-4.77087468203102\\
57.125	0.1098	-4.94401376883941\\
57.125	0.1104	-5.11715285564779\\
57.125	0.111	-5.29029194245618\\
57.125	0.1116	-5.46343102926457\\
57.125	0.1122	-5.63657011607296\\
57.125	0.1128	-5.80970920288136\\
57.125	0.1134	-5.98284828968974\\
57.125	0.114	-6.15598737649813\\
57.125	0.1146	-6.32912646330652\\
57.125	0.1152	-6.50226555011491\\
57.125	0.1158	-6.6754046369233\\
57.125	0.1164	-6.84854372373168\\
57.125	0.117	-7.02168281054007\\
57.125	0.1176	-7.19482189734846\\
57.125	0.1182	-7.36796098415685\\
57.125	0.1188	-7.54110007096524\\
57.125	0.1194	-7.71423915777362\\
57.125	0.12	-7.88737824458201\\
57.125	0.1206	-8.0605173313904\\
57.125	0.1212	-8.23365641819879\\
57.125	0.1218	-8.40679550500718\\
57.125	0.1224	-8.57993459181557\\
57.125	0.123	-8.75307367862396\\
57.5	0.093	-0.0558338818298552\\
57.5	0.0936	-0.22588513226259\\
57.5	0.0942	-0.395936382695325\\
57.5	0.0948	-0.56598763312806\\
57.5	0.0954	-0.736038883560795\\
57.5	0.096	-0.906090133993544\\
57.5	0.0966	-1.07614138442626\\
57.5	0.0972	-1.246192634859\\
57.5	0.0978	-1.41624388529173\\
57.5	0.0984	-1.58629513572447\\
57.5	0.099	-1.7563463861572\\
57.5	0.0996	-1.92639763658994\\
57.5	0.1002	-2.09644888702267\\
57.5	0.1008	-2.2665001374554\\
57.5	0.1014	-2.43655138788813\\
57.5	0.102	-2.60660263832087\\
57.5	0.1026	-2.7766538887536\\
57.5	0.1032	-2.94670513918634\\
57.5	0.1038	-3.11675638961907\\
57.5	0.1044	-3.2868076400518\\
57.5	0.105	-3.45685889048454\\
57.5	0.1056	-3.62691014091727\\
57.5	0.1062	-3.79696139135001\\
57.5	0.1068	-3.96701264178274\\
57.5	0.1074	-4.13706389221548\\
57.5	0.108	-4.30711514264821\\
57.5	0.1086	-4.47716639308095\\
57.5	0.1092	-4.64721764351367\\
57.5	0.1098	-4.81726889394641\\
57.5	0.1104	-4.98732014437914\\
57.5	0.111	-5.15737139481188\\
57.5	0.1116	-5.32742264524461\\
57.5	0.1122	-5.49747389567735\\
57.5	0.1128	-5.66752514611009\\
57.5	0.1134	-5.83757639654283\\
57.5	0.114	-6.00762764697556\\
57.5	0.1146	-6.17767889740828\\
57.5	0.1152	-6.34773014784102\\
57.5	0.1158	-6.51778139827375\\
57.5	0.1164	-6.68783264870649\\
57.5	0.117	-6.85788389913922\\
57.5	0.1176	-7.02793514957196\\
57.5	0.1182	-7.19798640000469\\
57.5	0.1188	-7.36803765043742\\
57.5	0.1194	-7.53808890087015\\
57.5	0.12	-7.70814015130289\\
57.5	0.1206	-7.87819140173562\\
57.5	0.1212	-8.04824265216835\\
57.5	0.1218	-8.2182939026011\\
57.5	0.1224	-8.38834515303382\\
57.5	0.123	-8.55839640346656\\
57.875	0.093	-0.0155484254552078\\
57.875	0.0936	-0.182511839512287\\
57.875	0.0942	-0.349475253569366\\
57.875	0.0948	-0.516438667626431\\
57.875	0.0954	-0.683402081683511\\
57.875	0.096	-0.850365495740604\\
57.875	0.0966	-1.01732890979768\\
57.875	0.0972	-1.18429232385476\\
57.875	0.0978	-1.35125573791184\\
57.875	0.0984	-1.51821915196892\\
57.875	0.099	-1.685182566026\\
57.875	0.0996	-1.85214598008308\\
57.875	0.1002	-2.01910939414016\\
57.875	0.1008	-2.18607280819724\\
57.875	0.1014	-2.35303622225432\\
57.875	0.102	-2.5199996363114\\
57.875	0.1026	-2.68696305036848\\
57.875	0.1032	-2.85392646442556\\
57.875	0.1038	-3.02088987848263\\
57.875	0.1044	-3.18785329253971\\
57.875	0.105	-3.35481670659679\\
57.875	0.1056	-3.52178012065387\\
57.875	0.1062	-3.68874353471095\\
57.875	0.1068	-3.85570694876803\\
57.875	0.1074	-4.02267036282511\\
57.875	0.108	-4.18963377688219\\
57.875	0.1086	-4.35659719093927\\
57.875	0.1092	-4.52356060499635\\
57.875	0.1098	-4.69052401905343\\
57.875	0.1104	-4.85748743311051\\
57.875	0.111	-5.02445084716759\\
57.875	0.1116	-5.19141426122466\\
57.875	0.1122	-5.35837767528174\\
57.875	0.1128	-5.52534108933884\\
57.875	0.1134	-5.69230450339592\\
57.875	0.114	-5.859267917453\\
57.875	0.1146	-6.02623133151008\\
57.875	0.1152	-6.19319474556715\\
57.875	0.1158	-6.36015815962423\\
57.875	0.1164	-6.52712157368131\\
57.875	0.117	-6.69408498773838\\
57.875	0.1176	-6.86104840179546\\
57.875	0.1182	-7.02801181585254\\
57.875	0.1188	-7.19497522990962\\
57.875	0.1194	-7.3619386439667\\
57.875	0.12	-7.52890205802377\\
57.875	0.1206	-7.69586547208085\\
57.875	0.1212	-7.86282888613793\\
57.875	0.1218	-8.02979230019503\\
57.875	0.1224	-8.19675571425211\\
57.875	0.123	-8.36371912830919\\
58.25	0.093	0.0247370309194537\\
58.25	0.0936	-0.13913854676197\\
58.25	0.0942	-0.303014124443393\\
58.25	0.0948	-0.466889702124817\\
58.25	0.0954	-0.630765279806241\\
58.25	0.096	-0.794640857487678\\
58.25	0.0966	-0.958516435169102\\
58.25	0.0972	-1.12239201285053\\
58.25	0.0978	-1.28626759053195\\
58.25	0.0984	-1.45014316821337\\
58.25	0.099	-1.61401874589481\\
58.25	0.0996	-1.77789432357622\\
58.25	0.1002	-1.94176990125764\\
58.25	0.1008	-2.10564547893908\\
58.25	0.1014	-2.2695210566205\\
58.25	0.102	-2.43339663430191\\
58.25	0.1026	-2.59727221198335\\
58.25	0.1032	-2.76114778966478\\
58.25	0.1038	-2.9250233673462\\
58.25	0.1044	-3.08889894502764\\
58.25	0.105	-3.25277452270906\\
58.25	0.1056	-3.41665010039048\\
58.25	0.1062	-3.58052567807191\\
58.25	0.1068	-3.74440125575333\\
58.25	0.1074	-3.90827683343475\\
58.25	0.108	-4.07215241111618\\
58.25	0.1086	-4.2360279887976\\
58.25	0.1092	-4.39990356647903\\
58.25	0.1098	-4.56377914416045\\
58.25	0.1104	-4.72765472184187\\
58.25	0.111	-4.8915302995233\\
58.25	0.1116	-5.05540587720472\\
58.25	0.1122	-5.21928145488614\\
58.25	0.1128	-5.38315703256758\\
58.25	0.1134	-5.547032610249\\
58.25	0.114	-5.71090818793043\\
58.25	0.1146	-5.87478376561185\\
58.25	0.1152	-6.03865934329328\\
58.25	0.1158	-6.2025349209747\\
58.25	0.1164	-6.36641049865612\\
58.25	0.117	-6.53028607633755\\
58.25	0.1176	-6.69416165401897\\
58.25	0.1182	-6.85803723170039\\
58.25	0.1188	-7.02191280938182\\
58.25	0.1194	-7.18578838706324\\
58.25	0.12	-7.34966396474466\\
58.25	0.1206	-7.51353954242609\\
58.25	0.1212	-7.67741512010751\\
58.25	0.1218	-7.84129069778895\\
58.25	0.1224	-8.00516627547037\\
58.25	0.123	-8.1690418531518\\
58.625	0.093	0.0650224872941152\\
58.625	0.0936	-0.0957652540116527\\
58.625	0.0942	-0.256552995317421\\
58.625	0.0948	-0.417340736623188\\
58.625	0.0954	-0.578128477928956\\
58.625	0.096	-0.738916219234738\\
58.625	0.0966	-0.899703960540521\\
58.625	0.0972	-1.06049170184627\\
58.625	0.0978	-1.22127944315206\\
58.625	0.0984	-1.38206718445782\\
58.625	0.099	-1.54285492576359\\
58.625	0.0996	-1.70364266706936\\
58.625	0.1002	-1.86443040837513\\
58.625	0.1008	-2.0252181496809\\
58.625	0.1014	-2.18600589098666\\
58.625	0.102	-2.34679363229243\\
58.625	0.1026	-2.5075813735982\\
58.625	0.1032	-2.66836911490398\\
58.625	0.1038	-2.82915685620975\\
58.625	0.1044	-2.98994459751553\\
58.625	0.105	-3.1507323388213\\
58.625	0.1056	-3.31152008012707\\
58.625	0.1062	-3.47230782143284\\
58.625	0.1068	-3.6330955627386\\
58.625	0.1074	-3.79388330404437\\
58.625	0.108	-3.95467104535014\\
58.625	0.1086	-4.11545878665591\\
58.625	0.1092	-4.27624652796167\\
58.625	0.1098	-4.43703426926744\\
58.625	0.1104	-4.59782201057322\\
58.625	0.111	-4.75860975187899\\
58.625	0.1116	-4.91939749318476\\
58.625	0.1122	-5.08018523449053\\
58.625	0.1128	-5.24097297579631\\
58.625	0.1134	-5.40176071710208\\
58.625	0.114	-5.56254845840785\\
58.625	0.1146	-5.72333619971361\\
58.625	0.1152	-5.88412394101938\\
58.625	0.1158	-6.04491168232516\\
58.625	0.1164	-6.20569942363093\\
58.625	0.117	-6.3664871649367\\
58.625	0.1176	-6.52727490624247\\
58.625	0.1182	-6.68806264754824\\
58.625	0.1188	-6.848850388854\\
58.625	0.1194	-7.00963813015977\\
58.625	0.12	-7.17042587146554\\
58.625	0.1206	-7.33121361277131\\
58.625	0.1212	-7.49200135407708\\
58.625	0.1218	-7.65278909538286\\
58.625	0.1224	-7.81357683668864\\
58.625	0.123	-7.97436457799441\\
59	0.093	0.105307943668777\\
59	0.0936	-0.0523919612613355\\
59	0.0942	-0.210091866191462\\
59	0.0948	-0.367791771121574\\
59	0.0954	-0.525491676051686\\
59	0.096	-0.683191580981813\\
59	0.0966	-0.840891485911925\\
59	0.0972	-0.998591390842037\\
59	0.0978	-1.15629129577216\\
59	0.0984	-1.31399120070228\\
59	0.099	-1.47169110563239\\
59	0.0996	-1.6293910105625\\
59	0.1002	-1.78709091549261\\
59	0.1008	-1.94479082042274\\
59	0.1014	-2.10249072535285\\
59	0.102	-2.26019063028296\\
59	0.1026	-2.41789053521308\\
59	0.1032	-2.57559044014319\\
59	0.1038	-2.7332903450733\\
59	0.1044	-2.89099025000344\\
59	0.105	-3.04869015493355\\
59	0.1056	-3.20639005986366\\
59	0.1062	-3.36408996479378\\
59	0.1068	-3.52178986972389\\
59	0.1074	-3.679489774654\\
59	0.108	-3.83718967958413\\
59	0.1086	-3.99488958451424\\
59	0.1092	-4.15258948944435\\
59	0.1098	-4.31028939437446\\
59	0.1104	-4.46798929930458\\
59	0.111	-4.6256892042347\\
59	0.1116	-4.78338910916482\\
59	0.1122	-4.94108901409493\\
59	0.1128	-5.09878891902505\\
59	0.1134	-5.25648882395517\\
59	0.114	-5.41418872888529\\
59	0.1146	-5.57188863381541\\
59	0.1152	-5.72958853874552\\
59	0.1158	-5.88728844367563\\
59	0.1164	-6.04498834860574\\
59	0.117	-6.20268825353585\\
59	0.1176	-6.36038815846597\\
59	0.1182	-6.51808806339609\\
59	0.1188	-6.6757879683262\\
59	0.1194	-6.83348787325632\\
59	0.12	-6.99118777818643\\
59	0.1206	-7.14888768311654\\
59	0.1212	-7.30658758804667\\
59	0.1218	-7.46428749297679\\
59	0.1224	-7.62198739790691\\
59	0.123	-7.77968730283702\\
59.375	0.093	0.145593400043424\\
59.375	0.0936	-0.0090186685110325\\
59.375	0.0942	-0.163630737065503\\
59.375	0.0948	-0.31824280561996\\
59.375	0.0954	-0.472854874174416\\
59.375	0.096	-0.627466942728887\\
59.375	0.0966	-0.782079011283358\\
59.375	0.0972	-0.936691079837814\\
59.375	0.0978	-1.09130314839227\\
59.375	0.0984	-1.24591521694673\\
59.375	0.099	-1.4005272855012\\
59.375	0.0996	-1.55513935405565\\
59.375	0.1002	-1.70975142261011\\
59.375	0.1008	-1.86436349116457\\
59.375	0.1014	-2.01897555971904\\
59.375	0.102	-2.17358762827349\\
59.375	0.1026	-2.32819969682795\\
59.375	0.1032	-2.48281176538241\\
59.375	0.1038	-2.63742383393688\\
59.375	0.1044	-2.79203590249135\\
59.375	0.105	-2.94664797104581\\
59.375	0.1056	-3.10126003960026\\
59.375	0.1062	-3.25587210815473\\
59.375	0.1068	-3.41048417670919\\
59.375	0.1074	-3.56509624526365\\
59.375	0.108	-3.7197083138181\\
59.375	0.1086	-3.87432038237256\\
59.375	0.1092	-4.02893245092703\\
59.375	0.1098	-4.18354451948149\\
59.375	0.1104	-4.33815658803594\\
59.375	0.111	-4.4927686565904\\
59.375	0.1116	-4.64738072514487\\
59.375	0.1122	-4.80199279369933\\
59.375	0.1128	-4.9566048622538\\
59.375	0.1134	-5.11121693080825\\
59.375	0.114	-5.26582899936272\\
59.375	0.1146	-5.42044106791718\\
59.375	0.1152	-5.57505313647164\\
59.375	0.1158	-5.72966520502609\\
59.375	0.1164	-5.88427727358057\\
59.375	0.117	-6.03888934213502\\
59.375	0.1176	-6.19350141068948\\
59.375	0.1182	-6.34811347924393\\
59.375	0.1188	-6.50272554779841\\
59.375	0.1194	-6.65733761635286\\
59.375	0.12	-6.81194968490732\\
59.375	0.1206	-6.96656175346178\\
59.375	0.1212	-7.12117382201625\\
59.375	0.1218	-7.27578589057072\\
59.375	0.1224	-7.43039795912517\\
59.375	0.123	-7.58501002767963\\
59.75	0.093	0.185878856418086\\
59.75	0.0936	0.0343546242392847\\
59.75	0.0942	-0.117169607939516\\
59.75	0.0948	-0.268693840118331\\
59.75	0.0954	-0.420218072297132\\
59.75	0.096	-0.571742304475947\\
59.75	0.0966	-0.723266536654762\\
59.75	0.0972	-0.874790768833563\\
59.75	0.0978	-1.02631500101236\\
59.75	0.0984	-1.17783923319118\\
59.75	0.099	-1.32936346536998\\
59.75	0.0996	-1.48088769754878\\
59.75	0.1002	-1.63241192972758\\
59.75	0.1008	-1.7839361619064\\
59.75	0.1014	-1.9354603940852\\
59.75	0.102	-2.086984626264\\
59.75	0.1026	-2.23850885844281\\
59.75	0.1032	-2.39003309062161\\
59.75	0.1038	-2.54155732280043\\
59.75	0.1044	-2.69308155497924\\
59.75	0.105	-2.84460578715805\\
59.75	0.1056	-2.99613001933685\\
59.75	0.1062	-3.14765425151566\\
59.75	0.1068	-3.29917848369446\\
59.75	0.1074	-3.45070271587326\\
59.75	0.108	-3.60222694805208\\
59.75	0.1086	-3.75375118023088\\
59.75	0.1092	-3.90527541240968\\
59.75	0.1098	-4.05679964458849\\
59.75	0.1104	-4.2083238767673\\
59.75	0.111	-4.3598481089461\\
59.75	0.1116	-4.51137234112491\\
59.75	0.1122	-4.66289657330371\\
59.75	0.1128	-4.81442080548253\\
59.75	0.1134	-4.96594503766134\\
59.75	0.114	-5.11746926984014\\
59.75	0.1146	-5.26899350201894\\
59.75	0.1152	-5.42051773419776\\
59.75	0.1158	-5.57204196637656\\
59.75	0.1164	-5.72356619855536\\
59.75	0.117	-5.87509043073418\\
59.75	0.1176	-6.02661466291298\\
59.75	0.1182	-6.17813889509178\\
59.75	0.1188	-6.32966312727059\\
59.75	0.1194	-6.48118735944939\\
59.75	0.12	-6.63271159162819\\
59.75	0.1206	-6.78423582380701\\
59.75	0.1212	-6.93576005598581\\
59.75	0.1218	-7.08728428816462\\
59.75	0.1224	-7.23880852034344\\
59.75	0.123	-7.39033275252224\\
60.125	0.093	0.226164312792747\\
60.125	0.0936	0.0777279169896019\\
60.125	0.0942	-0.0707084788135575\\
60.125	0.0948	-0.219144874616703\\
60.125	0.0954	-0.367581270419862\\
60.125	0.096	-0.516017666223021\\
60.125	0.0966	-0.664454062026181\\
60.125	0.0972	-0.812890457829326\\
60.125	0.0978	-0.961326853632471\\
60.125	0.0984	-1.10976324943563\\
60.125	0.099	-1.25819964523878\\
60.125	0.0996	-1.40663604104192\\
60.125	0.1002	-1.55507243684508\\
60.125	0.1008	-1.70350883264823\\
60.125	0.1014	-1.85194522845138\\
60.125	0.102	-2.00038162425453\\
60.125	0.1026	-2.14881802005767\\
60.125	0.1032	-2.29725441586083\\
60.125	0.1038	-2.44569081166398\\
60.125	0.1044	-2.59412720746715\\
60.125	0.105	-2.7425636032703\\
60.125	0.1056	-2.89099999907346\\
60.125	0.1062	-3.0394363948766\\
60.125	0.1068	-3.18787279067975\\
60.125	0.1074	-3.33630918648291\\
60.125	0.108	-3.48474558228605\\
60.125	0.1086	-3.63318197808921\\
60.125	0.1092	-3.78161837389236\\
60.125	0.1098	-3.9300547696955\\
60.125	0.1104	-4.07849116549866\\
60.125	0.111	-4.22692756130181\\
60.125	0.1116	-4.37536395710497\\
60.125	0.1122	-4.52380035290811\\
60.125	0.1128	-4.67223674871127\\
60.125	0.1134	-4.82067314451443\\
60.125	0.114	-4.96910954031758\\
60.125	0.1146	-5.11754593612073\\
60.125	0.1152	-5.26598233192388\\
60.125	0.1158	-5.41441872772702\\
60.125	0.1164	-5.56285512353018\\
60.125	0.117	-5.71129151933333\\
60.125	0.1176	-5.85972791513649\\
60.125	0.1182	-6.00816431093963\\
60.125	0.1188	-6.15660070674279\\
60.125	0.1194	-6.30503710254594\\
60.125	0.12	-6.45347349834908\\
60.125	0.1206	-6.60190989415224\\
60.125	0.1212	-6.75034628995539\\
60.125	0.1218	-6.89878268575855\\
60.125	0.1224	-7.04721908156171\\
60.125	0.123	-7.19565547736485\\
60.5	0.093	0.266449769167394\\
60.5	0.0936	0.121101209739905\\
60.5	0.0942	-0.0242473496875988\\
60.5	0.0948	-0.169595909115088\\
60.5	0.0954	-0.314944468542592\\
60.5	0.096	-0.460293027970096\\
60.5	0.0966	-0.605641587397599\\
60.5	0.0972	-0.750990146825089\\
60.5	0.0978	-0.896338706252578\\
60.5	0.0984	-1.04168726568008\\
60.5	0.099	-1.18703582510759\\
60.5	0.0996	-1.33238438453508\\
60.5	0.1002	-1.47773294396256\\
60.5	0.1008	-1.62308150339007\\
60.5	0.1014	-1.76843006281756\\
60.5	0.102	-1.91377862224506\\
60.5	0.1026	-2.05912718167255\\
60.5	0.1032	-2.20447574110005\\
60.5	0.1038	-2.34982430052754\\
60.5	0.1044	-2.49517285995506\\
60.5	0.105	-2.64052141938255\\
60.5	0.1056	-2.78586997881006\\
60.5	0.1062	-2.93121853823754\\
60.5	0.1068	-3.07656709766505\\
60.5	0.1074	-3.22191565709254\\
60.5	0.108	-3.36726421652004\\
60.5	0.1086	-3.51261277594753\\
60.5	0.1092	-3.65796133537503\\
60.5	0.1098	-3.80330989480252\\
60.5	0.1104	-3.94865845423003\\
60.5	0.111	-4.09400701365752\\
60.5	0.1116	-4.23935557308502\\
60.5	0.1122	-4.38470413251251\\
60.5	0.1128	-4.53005269194003\\
60.5	0.1134	-4.67540125136752\\
60.5	0.114	-4.82074981079502\\
60.5	0.1146	-4.96609837022251\\
60.5	0.1152	-5.11144692965\\
60.5	0.1158	-5.2567954890775\\
60.5	0.1164	-5.40214404850499\\
60.5	0.117	-5.5474926079325\\
60.5	0.1176	-5.69284116735999\\
60.5	0.1182	-5.83818972678749\\
60.5	0.1188	-5.98353828621498\\
60.5	0.1194	-6.12888684564248\\
60.5	0.12	-6.27423540506997\\
60.5	0.1206	-6.41958396449748\\
60.5	0.1212	-6.56493252392497\\
60.5	0.1218	-6.71028108335248\\
60.5	0.1224	-6.85562964277997\\
60.5	0.123	-7.00097820220748\\
60.875	0.093	0.30673522554207\\
60.875	0.0936	0.164474502490222\\
60.875	0.0942	0.0222137794383883\\
60.875	0.0948	-0.12004694361346\\
60.875	0.0954	-0.262307666665308\\
60.875	0.096	-0.404568389717156\\
60.875	0.0966	-0.546829112769004\\
60.875	0.0972	-0.689089835820837\\
60.875	0.0978	-0.831350558872685\\
60.875	0.0984	-0.973611281924519\\
60.875	0.099	-1.11587200497637\\
60.875	0.0996	-1.2581327280282\\
60.875	0.1002	-1.40039345108005\\
60.875	0.1008	-1.5426541741319\\
60.875	0.1014	-1.68491489718373\\
60.875	0.102	-1.82717562023558\\
60.875	0.1026	-1.96943634328741\\
60.875	0.1032	-2.11169706633926\\
60.875	0.1038	-2.25395778939109\\
60.875	0.1044	-2.39621851244296\\
60.875	0.105	-2.53847923549479\\
60.875	0.1056	-2.68073995854664\\
60.875	0.1062	-2.82300068159849\\
60.875	0.1068	-2.96526140465032\\
60.875	0.1074	-3.10752212770217\\
60.875	0.108	-3.249782850754\\
60.875	0.1086	-3.39204357380585\\
60.875	0.1092	-3.53430429685768\\
60.875	0.1098	-3.67656501990953\\
60.875	0.1104	-3.81882574296137\\
60.875	0.111	-3.96108646601321\\
60.875	0.1116	-4.10334718906506\\
60.875	0.1122	-4.2456079121169\\
60.875	0.1128	-4.38786863516876\\
60.875	0.1134	-4.53012935822059\\
60.875	0.114	-4.67239008127244\\
60.875	0.1146	-4.81465080432427\\
60.875	0.1152	-4.95691152737612\\
60.875	0.1158	-5.09917225042796\\
60.875	0.1164	-5.2414329734798\\
60.875	0.117	-5.38369369653164\\
60.875	0.1176	-5.52595441958348\\
60.875	0.1182	-5.66821514263533\\
60.875	0.1188	-5.81047586568717\\
60.875	0.1194	-5.95273658873901\\
60.875	0.12	-6.09499731179085\\
60.875	0.1206	-6.2372580348427\\
60.875	0.1212	-6.37951875789454\\
60.875	0.1218	-6.52177948094639\\
60.875	0.1224	-6.66404020399824\\
60.875	0.123	-6.80630092705007\\
61.25	0.093	0.347020681916717\\
61.25	0.0936	0.207847795240539\\
61.25	0.0942	0.068674908564347\\
61.25	0.0948	-0.0704979781118453\\
61.25	0.0954	-0.209670864788023\\
61.25	0.096	-0.34884375146423\\
61.25	0.0966	-0.488016638140422\\
61.25	0.0972	-0.6271895248166\\
61.25	0.0978	-0.766362411492793\\
61.25	0.0984	-0.905535298168971\\
61.25	0.099	-1.04470818484516\\
61.25	0.0996	-1.18388107152136\\
61.25	0.1002	-1.32305395819753\\
61.25	0.1008	-1.46222684487373\\
61.25	0.1014	-1.60139973154992\\
61.25	0.102	-1.7405726182261\\
61.25	0.1026	-1.87974550490229\\
61.25	0.1032	-2.01891839157848\\
61.25	0.1038	-2.15809127825466\\
61.25	0.1044	-2.29726416493087\\
61.25	0.105	-2.43643705160704\\
61.25	0.1056	-2.57560993828324\\
61.25	0.1062	-2.71478282495943\\
61.25	0.1068	-2.85395571163561\\
61.25	0.1074	-2.9931285983118\\
61.25	0.108	-3.13230148498799\\
61.25	0.1086	-3.27147437166417\\
61.25	0.1092	-3.41064725834036\\
61.25	0.1098	-3.54982014501654\\
61.25	0.1104	-3.68899303169273\\
61.25	0.111	-3.82816591836892\\
61.25	0.1116	-3.96733880504512\\
61.25	0.1122	-4.10651169172129\\
61.25	0.1128	-4.2456845783975\\
61.25	0.1134	-4.38485746507368\\
61.25	0.114	-4.52403035174987\\
61.25	0.1146	-4.66320323842606\\
61.25	0.1152	-4.80237612510224\\
61.25	0.1158	-4.94154901177843\\
61.25	0.1164	-5.08072189845463\\
61.25	0.117	-5.2198947851308\\
61.25	0.1176	-5.359067671807\\
61.25	0.1182	-5.49824055848318\\
61.25	0.1188	-5.63741344515937\\
61.25	0.1194	-5.77658633183556\\
61.25	0.12	-5.91575921851174\\
61.25	0.1206	-6.05493210518793\\
61.25	0.1212	-6.19410499186412\\
61.25	0.1218	-6.33327787854031\\
61.25	0.1224	-6.47245076521651\\
61.25	0.123	-6.6116236518927\\
61.625	0.093	0.387306138291379\\
61.625	0.0936	0.251221087990842\\
61.625	0.0942	0.115136037690306\\
61.625	0.0948	-0.0209490126102168\\
61.625	0.0954	-0.157034062910753\\
61.625	0.096	-0.293119113211304\\
61.625	0.0966	-0.429204163511841\\
61.625	0.0972	-0.565289213812363\\
61.625	0.0978	-0.7013742641129\\
61.625	0.0984	-0.837459314413437\\
61.625	0.099	-0.973544364713973\\
61.625	0.0996	-1.1096294150145\\
61.625	0.1002	-1.24571446531503\\
61.625	0.1008	-1.38179951561557\\
61.625	0.1014	-1.51788456591609\\
61.625	0.102	-1.65396961621663\\
61.625	0.1026	-1.79005466651716\\
61.625	0.1032	-1.92613971681769\\
61.625	0.1038	-2.06222476711822\\
61.625	0.1044	-2.19830981741877\\
61.625	0.105	-2.33439486771931\\
61.625	0.1056	-2.47047991801983\\
61.625	0.1062	-2.60656496832037\\
61.625	0.1068	-2.74265001862091\\
61.625	0.1074	-2.87873506892143\\
61.625	0.108	-3.01482011922197\\
61.625	0.1086	-3.1509051695225\\
61.625	0.1092	-3.28699021982302\\
61.625	0.1098	-3.42307527012356\\
61.625	0.1104	-3.5591603204241\\
61.625	0.111	-3.69524537072463\\
61.625	0.1116	-3.83133042102516\\
61.625	0.1122	-3.96741547132569\\
61.625	0.1128	-4.10350052162624\\
61.625	0.1134	-4.23958557192678\\
61.625	0.114	-4.3756706222273\\
61.625	0.1146	-4.51175567252784\\
61.625	0.1152	-4.64784072282838\\
61.625	0.1158	-4.7839257731289\\
61.625	0.1164	-4.92001082342944\\
61.625	0.117	-5.05609587372997\\
61.625	0.1176	-5.19218092403051\\
61.625	0.1182	-5.32826597433103\\
61.625	0.1188	-5.46435102463157\\
61.625	0.1194	-5.60043607493211\\
61.625	0.12	-5.73652112523263\\
61.625	0.1206	-5.87260617553316\\
61.625	0.1212	-6.0086912258337\\
61.625	0.1218	-6.14477627613424\\
61.625	0.1224	-6.28086132643477\\
61.625	0.123	-6.41694637673531\\
62	0.093	0.42759159466604\\
62	0.0936	0.29459438074116\\
62	0.0942	0.161597166816279\\
62	0.0948	0.0285999528914118\\
62	0.0954	-0.104397261033469\\
62	0.096	-0.237394474958364\\
62	0.0966	-0.370391688883245\\
62	0.0972	-0.503388902808112\\
62	0.0978	-0.636386116732993\\
62	0.0984	-0.769383330657874\\
62	0.099	-0.902380544582755\\
62	0.0996	-1.03537775850764\\
62	0.1002	-1.1683749724325\\
62	0.1008	-1.30137218635738\\
62	0.1014	-1.43436940028226\\
62	0.102	-1.56736661420715\\
62	0.1026	-1.70036382813203\\
62	0.1032	-1.83336104205689\\
62	0.1038	-1.96635825598177\\
62	0.1044	-2.09935546990667\\
62	0.105	-2.23235268383155\\
62	0.1056	-2.36534989775643\\
62	0.1062	-2.4983471116813\\
62	0.1068	-2.63134432560618\\
62	0.1074	-2.76434153953106\\
62	0.108	-2.89733875345594\\
62	0.1086	-3.03033596738082\\
62	0.1092	-3.16333318130569\\
62	0.1098	-3.29633039523057\\
62	0.1104	-3.42932760915545\\
62	0.111	-3.56232482308033\\
62	0.1116	-3.6953220370052\\
62	0.1122	-3.82831925093008\\
62	0.1128	-3.96131646485497\\
62	0.1134	-4.09431367877986\\
62	0.114	-4.22731089270474\\
62	0.1146	-4.3603081066296\\
62	0.1152	-4.49330532055448\\
62	0.1158	-4.62630253447936\\
62	0.1164	-4.75929974840425\\
62	0.117	-4.89229696232911\\
62	0.1176	-5.02529417625399\\
62	0.1182	-5.15829139017887\\
62	0.1188	-5.29128860410376\\
62	0.1194	-5.42428581802864\\
62	0.12	-5.5572830319535\\
62	0.1206	-5.69028024587838\\
62	0.1212	-5.82327745980326\\
62	0.1218	-5.95627467372816\\
62	0.1224	-6.08927188765304\\
62	0.123	-6.22226910157791\\
62.375	0.093	0.467877051040702\\
62.375	0.0936	0.337967673491477\\
62.375	0.0942	0.208058295942251\\
62.375	0.0948	0.0781489183930262\\
62.375	0.0954	-0.0517604591561991\\
62.375	0.096	-0.181669836705439\\
62.375	0.0966	-0.311579214254664\\
62.375	0.0972	-0.441488591803875\\
62.375	0.0978	-0.5713979693531\\
62.375	0.0984	-0.701307346902325\\
62.375	0.099	-0.831216724451551\\
62.375	0.0996	-0.961126102000776\\
62.375	0.1002	-1.09103547955\\
62.375	0.1008	-1.22094485709923\\
62.375	0.1014	-1.35085423464845\\
62.375	0.102	-1.48076361219766\\
62.375	0.1026	-1.61067298974689\\
62.375	0.1032	-1.74058236729611\\
62.375	0.1038	-1.87049174484534\\
62.375	0.1044	-2.00040112239458\\
62.375	0.105	-2.1303104999438\\
62.375	0.1056	-2.26021987749303\\
62.375	0.1062	-2.39012925504225\\
62.375	0.1068	-2.52003863259146\\
62.375	0.1074	-2.64994801014069\\
62.375	0.108	-2.77985738768992\\
62.375	0.1086	-2.90976676523914\\
62.375	0.1092	-3.03967614278837\\
62.375	0.1098	-3.16958552033759\\
62.375	0.1104	-3.29949489788682\\
62.375	0.111	-3.42940427543603\\
62.375	0.1116	-3.55931365298525\\
62.375	0.1122	-3.68922303053448\\
62.375	0.1128	-3.81913240808372\\
62.375	0.1134	-3.94904178563294\\
62.375	0.114	-4.07895116318217\\
62.375	0.1146	-4.20886054073139\\
62.375	0.1152	-4.3387699182806\\
62.375	0.1158	-4.46867929582983\\
62.375	0.1164	-4.59858867337906\\
62.375	0.117	-4.72849805092828\\
62.375	0.1176	-4.85840742847751\\
62.375	0.1182	-4.98831680602673\\
62.375	0.1188	-5.11822618357596\\
62.375	0.1194	-5.24813556112517\\
62.375	0.12	-5.37804493867439\\
62.375	0.1206	-5.50795431622362\\
62.375	0.1212	-5.63786369377284\\
62.375	0.1218	-5.76777307132208\\
62.375	0.1224	-5.89768244887131\\
62.375	0.123	-6.02759182642053\\
62.75	0.093	0.508162507415349\\
62.75	0.0936	0.38134096624178\\
62.75	0.0942	0.25451942506821\\
62.75	0.0948	0.127697883894641\\
62.75	0.0954	0.000876342721070955\\
62.75	0.096	-0.125945198452513\\
62.75	0.0966	-0.252766739626082\\
62.75	0.0972	-0.379588280799652\\
62.75	0.0978	-0.506409821973222\\
62.75	0.0984	-0.633231363146791\\
62.75	0.099	-0.760052904320361\\
62.75	0.0996	-0.886874445493916\\
62.75	0.1002	-1.01369598666749\\
62.75	0.1008	-1.14051752784106\\
62.75	0.1014	-1.26733906901462\\
62.75	0.102	-1.39416061018819\\
62.75	0.1026	-1.52098215136176\\
62.75	0.1032	-1.64780369253533\\
62.75	0.1038	-1.7746252337089\\
62.75	0.1044	-1.90144677488249\\
62.75	0.105	-2.02826831605606\\
62.75	0.1056	-2.15508985722963\\
62.75	0.1062	-2.2819113984032\\
62.75	0.1068	-2.40873293957677\\
62.75	0.1074	-2.53555448075033\\
62.75	0.108	-2.6623760219239\\
62.75	0.1086	-2.78919756309747\\
62.75	0.1092	-2.91601910427103\\
62.75	0.1098	-3.0428406454446\\
62.75	0.1104	-3.16966218661817\\
62.75	0.111	-3.29648372779174\\
62.75	0.1116	-3.42330526896531\\
62.75	0.1122	-3.55012681013888\\
62.75	0.1128	-3.67694835131246\\
62.75	0.1134	-3.80376989248603\\
62.75	0.114	-3.9305914336596\\
62.75	0.1146	-4.05741297483317\\
62.75	0.1152	-4.18423451600674\\
62.75	0.1158	-4.31105605718031\\
62.75	0.1164	-4.43787759835388\\
62.75	0.117	-4.56469913952745\\
62.75	0.1176	-4.69152068070102\\
62.75	0.1182	-4.81834222187459\\
62.75	0.1188	-4.94516376304816\\
62.75	0.1194	-5.07198530422171\\
62.75	0.12	-5.19880684539528\\
62.75	0.1206	-5.32562838656885\\
62.75	0.1212	-5.45244992774242\\
62.75	0.1218	-5.57927146891601\\
62.75	0.1224	-5.70609301008957\\
62.75	0.123	-5.83291455126314\\
63.125	0.093	0.548447963790011\\
63.125	0.0936	0.424714258992097\\
63.125	0.0942	0.300980554194183\\
63.125	0.0948	0.177246849396269\\
63.125	0.0954	0.0535131445983552\\
63.125	0.096	-0.0702205601995729\\
63.125	0.0966	-0.193954264997487\\
63.125	0.0972	-0.317687969795401\\
63.125	0.0978	-0.441421674593315\\
63.125	0.0984	-0.565155379391229\\
63.125	0.099	-0.688889084189142\\
63.125	0.0996	-0.812622788987056\\
63.125	0.1002	-0.93635649378497\\
63.125	0.1008	-1.06009019858288\\
63.125	0.1014	-1.1838239033808\\
63.125	0.102	-1.30755760817871\\
63.125	0.1026	-1.43129131297663\\
63.125	0.1032	-1.55502501777454\\
63.125	0.1038	-1.67875872257245\\
63.125	0.1044	-1.80249242737038\\
63.125	0.105	-1.9262261321683\\
63.125	0.1056	-2.04995983696621\\
63.125	0.1062	-2.17369354176412\\
63.125	0.1068	-2.29742724656204\\
63.125	0.1074	-2.42116095135995\\
63.125	0.108	-2.54489465615787\\
63.125	0.1086	-2.66862836095578\\
63.125	0.1092	-2.79236206575369\\
63.125	0.1098	-2.91609577055161\\
63.125	0.1104	-3.03982947534952\\
63.125	0.111	-3.16356318014743\\
63.125	0.1116	-3.28729688494535\\
63.125	0.1122	-3.41103058974326\\
63.125	0.1128	-3.53476429454119\\
63.125	0.1134	-3.6584979993391\\
63.125	0.114	-3.78223170413702\\
63.125	0.1146	-3.90596540893493\\
63.125	0.1152	-4.02969911373285\\
63.125	0.1158	-4.15343281853076\\
63.125	0.1164	-4.27716652332867\\
63.125	0.117	-4.40090022812659\\
63.125	0.1176	-4.5246339329245\\
63.125	0.1182	-4.64836763772242\\
63.125	0.1188	-4.77210134252034\\
63.125	0.1194	-4.89583504731824\\
63.125	0.12	-5.01956875211616\\
63.125	0.1206	-5.14330245691409\\
63.125	0.1212	-5.267036161712\\
63.125	0.1218	-5.39076986650991\\
63.125	0.1224	-5.51450357130784\\
63.125	0.123	-5.63823727610576\\
63.5	0.093	0.588733420164672\\
63.5	0.0936	0.468087551742414\\
63.5	0.0942	0.347441683320156\\
63.5	0.0948	0.226795814897898\\
63.5	0.0954	0.106149946475625\\
63.5	0.096	-0.0144959219466472\\
63.5	0.0966	-0.135141790368905\\
63.5	0.0972	-0.255787658791164\\
63.5	0.0978	-0.376433527213422\\
63.5	0.0984	-0.49707939563568\\
63.5	0.099	-0.617725264057938\\
63.5	0.0996	-0.738371132480196\\
63.5	0.1002	-0.859017000902455\\
63.5	0.1008	-0.979662869324713\\
63.5	0.1014	-1.10030873774699\\
63.5	0.102	-1.22095460616924\\
63.5	0.1026	-1.3416004745915\\
63.5	0.1032	-1.46224634301376\\
63.5	0.1038	-1.58289221143602\\
63.5	0.1044	-1.70353807985829\\
63.5	0.105	-1.82418394828055\\
63.5	0.1056	-1.94482981670281\\
63.5	0.1062	-2.06547568512507\\
63.5	0.1068	-2.18612155354732\\
63.5	0.1074	-2.30676742196958\\
63.5	0.108	-2.42741329039185\\
63.5	0.1086	-2.54805915881411\\
63.5	0.1092	-2.66870502723637\\
63.5	0.1098	-2.78935089565863\\
63.5	0.1104	-2.90999676408089\\
63.5	0.111	-3.03064263250315\\
63.5	0.1116	-3.1512885009254\\
63.5	0.1122	-3.27193436934766\\
63.5	0.1128	-3.39258023776993\\
63.5	0.1134	-3.51322610619219\\
63.5	0.114	-3.63387197461446\\
63.5	0.1146	-3.75451784303672\\
63.5	0.1152	-3.87516371145898\\
63.5	0.1158	-3.99580957988124\\
63.5	0.1164	-4.1164554483035\\
63.5	0.117	-4.23710131672576\\
63.5	0.1176	-4.35774718514801\\
63.5	0.1182	-4.47839305357027\\
63.5	0.1188	-4.59903892199253\\
63.5	0.1194	-4.71968479041479\\
63.5	0.12	-4.84033065883705\\
63.5	0.1206	-4.96097652725932\\
63.5	0.1212	-5.08162239568158\\
63.5	0.1218	-5.20226826410385\\
63.5	0.1224	-5.32291413252611\\
63.5	0.123	-5.44356000094837\\
63.875	0.093	0.62901887653932\\
63.875	0.0936	0.511460844492717\\
63.875	0.0942	0.393902812446115\\
63.875	0.0948	0.276344780399512\\
63.875	0.0954	0.15878674835291\\
63.875	0.096	0.0412287163062786\\
63.875	0.0966	-0.076329315740324\\
63.875	0.0972	-0.193887347786927\\
63.875	0.0978	-0.311445379833529\\
63.875	0.0984	-0.429003411880132\\
63.875	0.099	-0.546561443926748\\
63.875	0.0996	-0.664119475973351\\
63.875	0.1002	-0.781677508019953\\
63.875	0.1008	-0.899235540066556\\
63.875	0.1014	-1.01679357211316\\
63.875	0.102	-1.13435160415976\\
63.875	0.1026	-1.25190963620638\\
63.875	0.1032	-1.36946766825298\\
63.875	0.1038	-1.48702570029958\\
63.875	0.1044	-1.6045837323462\\
63.875	0.105	-1.7221417643928\\
63.875	0.1056	-1.8396997964394\\
63.875	0.1062	-1.95725782848602\\
63.875	0.1068	-2.07481586053262\\
63.875	0.1074	-2.19237389257923\\
63.875	0.108	-2.30993192462583\\
63.875	0.1086	-2.42748995667243\\
63.875	0.1092	-2.54504798871903\\
63.875	0.1098	-2.66260602076565\\
63.875	0.1104	-2.78016405281225\\
63.875	0.111	-2.89772208485886\\
63.875	0.1116	-3.01528011690546\\
63.875	0.1122	-3.13283814895206\\
63.875	0.1128	-3.25039618099868\\
63.875	0.1134	-3.36795421304529\\
63.875	0.114	-3.4855122450919\\
63.875	0.1146	-3.6030702771385\\
63.875	0.1152	-3.7206283091851\\
63.875	0.1158	-3.8381863412317\\
63.875	0.1164	-3.95574437327832\\
63.875	0.117	-4.07330240532492\\
63.875	0.1176	-4.19086043737153\\
63.875	0.1182	-4.30841846941813\\
63.875	0.1188	-4.42597650146473\\
63.875	0.1194	-4.54353453351133\\
63.875	0.12	-4.66109256555795\\
63.875	0.1206	-4.77865059760455\\
63.875	0.1212	-4.89620862965116\\
63.875	0.1218	-5.01376666169777\\
63.875	0.1224	-5.13132469374438\\
63.875	0.123	-5.24888272579098\\
64.25	0.093	0.669304332913995\\
64.25	0.0936	0.554834137243034\\
64.25	0.0942	0.440363941572087\\
64.25	0.0948	0.325893745901141\\
64.25	0.0954	0.211423550230194\\
64.25	0.096	0.0969533545592185\\
64.25	0.0966	-0.0175168411117284\\
64.25	0.0972	-0.131987036782675\\
64.25	0.0978	-0.246457232453622\\
64.25	0.0984	-0.360927428124583\\
64.25	0.099	-0.47539762379553\\
64.25	0.0996	-0.589867819466477\\
64.25	0.1002	-0.704338015137424\\
64.25	0.1008	-0.818808210808385\\
64.25	0.1014	-0.933278406479332\\
64.25	0.102	-1.04774860215028\\
64.25	0.1026	-1.16221879782123\\
64.25	0.1032	-1.27668899349219\\
64.25	0.1038	-1.39115918916313\\
64.25	0.1044	-1.50562938483409\\
64.25	0.105	-1.62009958050504\\
64.25	0.1056	-1.734569776176\\
64.25	0.1062	-1.84903997184695\\
64.25	0.1068	-1.9635101675179\\
64.25	0.1074	-2.07798036318884\\
64.25	0.108	-2.1924505588598\\
64.25	0.1086	-2.30692075453075\\
64.25	0.1092	-2.4213909502017\\
64.25	0.1098	-2.53586114587264\\
64.25	0.1104	-2.65033134154361\\
64.25	0.111	-2.76480153721455\\
64.25	0.1116	-2.8792717328855\\
64.25	0.1122	-2.99374192855645\\
64.25	0.1128	-3.10821212422742\\
64.25	0.1134	-3.22268231989837\\
64.25	0.114	-3.33715251556931\\
64.25	0.1146	-3.45162271124026\\
64.25	0.1152	-3.56609290691122\\
64.25	0.1158	-3.68056310258217\\
64.25	0.1164	-3.79503329825312\\
64.25	0.117	-3.90950349392406\\
64.25	0.1176	-4.02397368959502\\
64.25	0.1182	-4.13844388526597\\
64.25	0.1188	-4.25291408093692\\
64.25	0.1194	-4.36738427660786\\
64.25	0.12	-4.48185447227883\\
64.25	0.1206	-4.59632466794977\\
64.25	0.1212	-4.71079486362072\\
64.25	0.1218	-4.82526505929168\\
64.25	0.1224	-4.93973525496264\\
64.25	0.123	-5.05420545063359\\
64.625	0.093	0.709589789288643\\
64.625	0.0936	0.598207429993352\\
64.625	0.0942	0.486825070698046\\
64.625	0.0948	0.375442711402755\\
64.625	0.0954	0.264060352107464\\
64.625	0.096	0.152677992812158\\
64.625	0.0966	0.0412956335168531\\
64.625	0.0972	-0.0700867257784381\\
64.625	0.0978	-0.181469085073729\\
64.625	0.0984	-0.292851444369035\\
64.625	0.099	-0.404233803664326\\
64.625	0.0996	-0.515616162959617\\
64.625	0.1002	-0.626998522254922\\
64.625	0.1008	-0.738380881550214\\
64.625	0.1014	-0.849763240845505\\
64.625	0.102	-0.96114560014081\\
64.625	0.1026	-1.0725279594361\\
64.625	0.1032	-1.18391031873139\\
64.625	0.1038	-1.2952926780267\\
64.625	0.1044	-1.406675037322\\
64.625	0.105	-1.51805739661729\\
64.625	0.1056	-1.6294397559126\\
64.625	0.1062	-1.74082211520789\\
64.625	0.1068	-1.85220447450318\\
64.625	0.1074	-1.96358683379849\\
64.625	0.108	-2.07496919309378\\
64.625	0.1086	-2.18635155238907\\
64.625	0.1092	-2.29773391168438\\
64.625	0.1098	-2.40911627097967\\
64.625	0.1104	-2.52049863027496\\
64.625	0.111	-2.63188098957026\\
64.625	0.1116	-2.74326334886555\\
64.625	0.1122	-2.85464570816085\\
64.625	0.1128	-2.96602806745616\\
64.625	0.1134	-3.07741042675146\\
64.625	0.114	-3.18879278604675\\
64.625	0.1146	-3.30017514534205\\
64.625	0.1152	-3.41155750463734\\
64.625	0.1158	-3.52293986393263\\
64.625	0.1164	-3.63432222322794\\
64.625	0.117	-3.74570458252323\\
64.625	0.1176	-3.85708694181852\\
64.625	0.1182	-3.96846930111383\\
64.625	0.1188	-4.07985166040912\\
64.625	0.1194	-4.19123401970441\\
64.625	0.12	-4.30261637899972\\
64.625	0.1206	-4.41399873829501\\
64.625	0.1212	-4.5253810975903\\
64.625	0.1218	-4.63676345688562\\
64.625	0.1224	-4.74814581618091\\
64.625	0.123	-4.8595281754762\\
65	0.093	0.749875245663304\\
65	0.0936	0.641580722743655\\
65	0.0942	0.533286199824019\\
65	0.0948	0.424991676904369\\
65	0.0954	0.316697153984734\\
65	0.096	0.20840263106507\\
65	0.0966	0.100108108145434\\
65	0.0972	-0.00818641477420101\\
65	0.0978	-0.116480937693851\\
65	0.0984	-0.224775460613486\\
65	0.099	-0.333069983533136\\
65	0.0996	-0.441364506452771\\
65	0.1002	-0.549659029372407\\
65	0.1008	-0.657953552292057\\
65	0.1014	-0.766248075211692\\
65	0.102	-0.874542598131328\\
65	0.1026	-0.982837121050977\\
65	0.1032	-1.09113164397061\\
65	0.1038	-1.19942616689026\\
65	0.1044	-1.30772068980991\\
65	0.105	-1.41601521272956\\
65	0.1056	-1.5243097356492\\
65	0.1062	-1.63260425856883\\
65	0.1068	-1.74089878148848\\
65	0.1074	-1.84919330440812\\
65	0.108	-1.95748782732777\\
65	0.1086	-2.0657823502474\\
65	0.1092	-2.17407687316704\\
65	0.1098	-2.28237139608669\\
65	0.1104	-2.39066591900632\\
65	0.111	-2.49896044192597\\
65	0.1116	-2.60725496484561\\
65	0.1122	-2.71554948776524\\
65	0.1128	-2.82384401068491\\
65	0.1134	-2.93213853360454\\
65	0.114	-3.04043305652419\\
65	0.1146	-3.14872757944383\\
65	0.1152	-3.25702210236346\\
65	0.1158	-3.36531662528311\\
65	0.1164	-3.47361114820275\\
65	0.117	-3.5819056711224\\
65	0.1176	-3.69020019404203\\
65	0.1182	-3.79849471696167\\
65	0.1188	-3.90678923988132\\
65	0.1194	-4.01508376280096\\
65	0.12	-4.12337828572061\\
65	0.1206	-4.23167280864024\\
65	0.1212	-4.33996733155988\\
65	0.1218	-4.44826185447954\\
65	0.1224	-4.55655637739918\\
65	0.123	-4.66485090031883\\
65.375	0.093	0.790160702037966\\
65.375	0.0936	0.684954015493972\\
65.375	0.0942	0.579747328949992\\
65.375	0.0948	0.474540642405998\\
65.375	0.0954	0.369333955862018\\
65.375	0.096	0.26412726931801\\
65.375	0.0966	0.15892058277403\\
65.375	0.0972	0.0537138962300503\\
65.375	0.0978	-0.0514927903139437\\
65.375	0.0984	-0.156699476857924\\
65.375	0.099	-0.261906163401918\\
65.375	0.0996	-0.367112849945897\\
65.375	0.1002	-0.472319536489891\\
65.375	0.1008	-0.577526223033871\\
65.375	0.1014	-0.682732909577865\\
65.375	0.102	-0.787939596121845\\
65.375	0.1026	-0.893146282665839\\
65.375	0.1032	-0.998352969209819\\
65.375	0.1038	-1.10355965575381\\
65.375	0.1044	-1.20876634229781\\
65.375	0.105	-1.3139730288418\\
65.375	0.1056	-1.41917971538578\\
65.375	0.1062	-1.52438640192977\\
65.375	0.1068	-1.62959308847375\\
65.375	0.1074	-1.73479977501775\\
65.375	0.108	-1.84000646156173\\
65.375	0.1086	-1.94521314810572\\
65.375	0.1092	-2.0504198346497\\
65.375	0.1098	-2.1556265211937\\
65.375	0.1104	-2.26083320773768\\
65.375	0.111	-2.36603989428167\\
65.375	0.1116	-2.47124658082565\\
65.375	0.1122	-2.57645326736963\\
65.375	0.1128	-2.68165995391364\\
65.375	0.1134	-2.78686664045762\\
65.375	0.114	-2.89207332700161\\
65.375	0.1146	-2.99728001354559\\
65.375	0.1152	-3.10248670008959\\
65.375	0.1158	-3.20769338663357\\
65.375	0.1164	-3.31290007317756\\
65.375	0.117	-3.41810675972154\\
65.375	0.1176	-3.52331344626553\\
65.375	0.1182	-3.62852013280951\\
65.375	0.1188	-3.73372681935351\\
65.375	0.1194	-3.83893350589749\\
65.375	0.12	-3.94414019244148\\
65.375	0.1206	-4.04934687898546\\
65.375	0.1212	-4.15455356552945\\
65.375	0.1218	-4.25976025207345\\
65.375	0.1224	-4.36496693861744\\
65.375	0.123	-4.47017362516142\\
65.75	0.093	0.830446158412613\\
65.75	0.0936	0.728327308244289\\
65.75	0.0942	0.626208458075951\\
65.75	0.0948	0.524089607907626\\
65.75	0.0954	0.421970757739288\\
65.75	0.096	0.31985190757095\\
65.75	0.0966	0.217733057402612\\
65.75	0.0972	0.115614207234287\\
65.75	0.0978	0.0134953570659491\\
65.75	0.0984	-0.0886234931023893\\
65.75	0.099	-0.190742343270713\\
65.75	0.0996	-0.292861193439052\\
65.75	0.1002	-0.394980043607376\\
65.75	0.1008	-0.497098893775714\\
65.75	0.1014	-0.599217743944038\\
65.75	0.102	-0.701336594112377\\
65.75	0.1026	-0.803455444280701\\
65.75	0.1032	-0.905574294449039\\
65.75	0.1038	-1.00769314461738\\
65.75	0.1044	-1.10981199478572\\
65.75	0.105	-1.21193084495405\\
65.75	0.1056	-1.31404969512238\\
65.75	0.1062	-1.41616854529072\\
65.75	0.1068	-1.51828739545904\\
65.75	0.1074	-1.62040624562738\\
65.75	0.108	-1.72252509579572\\
65.75	0.1086	-1.82464394596404\\
65.75	0.1092	-1.92676279613238\\
65.75	0.1098	-2.0288816463007\\
65.75	0.1104	-2.13100049646904\\
65.75	0.111	-2.23311934663737\\
65.75	0.1116	-2.3352381968057\\
65.75	0.1122	-2.43735704697403\\
65.75	0.1128	-2.53947589714238\\
65.75	0.1134	-2.64159474731072\\
65.75	0.114	-2.74371359747904\\
65.75	0.1146	-2.84583244764738\\
65.75	0.1152	-2.94795129781571\\
65.75	0.1158	-3.05007014798404\\
65.75	0.1164	-3.15218899815237\\
65.75	0.117	-3.25430784832071\\
65.75	0.1176	-3.35642669848903\\
65.75	0.1182	-3.45854554865737\\
65.75	0.1188	-3.56066439882571\\
65.75	0.1194	-3.66278324899403\\
65.75	0.12	-3.76490209916237\\
65.75	0.1206	-3.86702094933069\\
65.75	0.1212	-3.96913979949903\\
65.75	0.1218	-4.07125864966737\\
65.75	0.1224	-4.17337749983571\\
65.75	0.123	-4.27549635000403\\
66.125	0.093	0.870731614787275\\
66.125	0.0936	0.771700600994592\\
66.125	0.0942	0.672669587201909\\
66.125	0.0948	0.573638573409241\\
66.125	0.0954	0.474607559616558\\
66.125	0.096	0.375576545823876\\
66.125	0.0966	0.276545532031193\\
66.125	0.0972	0.17751451823851\\
66.125	0.0978	0.0784835044458418\\
66.125	0.0984	-0.0205475093468408\\
66.125	0.099	-0.119578523139523\\
66.125	0.0996	-0.218609536932192\\
66.125	0.1002	-0.317640550724875\\
66.125	0.1008	-0.416671564517543\\
66.125	0.1014	-0.515702578310226\\
66.125	0.102	-0.614733592102908\\
66.125	0.1026	-0.713764605895577\\
66.125	0.1032	-0.812795619688259\\
66.125	0.1038	-0.911826633480942\\
66.125	0.1044	-1.01085764727362\\
66.125	0.105	-1.10988866106631\\
66.125	0.1056	-1.20891967485899\\
66.125	0.1062	-1.30795068865166\\
66.125	0.1068	-1.40698170244434\\
66.125	0.1074	-1.50601271623701\\
66.125	0.108	-1.60504373002969\\
66.125	0.1086	-1.70407474382237\\
66.125	0.1092	-1.80310575761504\\
66.125	0.1098	-1.90213677140773\\
66.125	0.1104	-2.00116778520041\\
66.125	0.111	-2.10019879899308\\
66.125	0.1116	-2.19922981278576\\
66.125	0.1122	-2.29826082657843\\
66.125	0.1128	-2.39729184037112\\
66.125	0.1134	-2.49632285416381\\
66.125	0.114	-2.59535386795648\\
66.125	0.1146	-2.69438488174916\\
66.125	0.1152	-2.79341589554184\\
66.125	0.1158	-2.89244690933451\\
66.125	0.1164	-2.99147792312719\\
66.125	0.117	-3.09050893691986\\
66.125	0.1176	-3.18953995071254\\
66.125	0.1182	-3.28857096450523\\
66.125	0.1188	-3.38760197829789\\
66.125	0.1194	-3.48663299209058\\
66.125	0.12	-3.58566400588326\\
66.125	0.1206	-3.68469501967593\\
66.125	0.1212	-3.78372603346861\\
66.125	0.1218	-3.88275704726131\\
66.125	0.1224	-3.98178806105398\\
66.125	0.123	-4.08081907484666\\
66.5	0.093	0.911017071161936\\
66.5	0.0936	0.815073893744909\\
66.5	0.0942	0.719130716327896\\
66.5	0.0948	0.623187538910869\\
66.5	0.0954	0.527244361493842\\
66.5	0.096	0.431301184076816\\
66.5	0.0966	0.335358006659789\\
66.5	0.0972	0.239414829242762\\
66.5	0.0978	0.143471651825735\\
66.5	0.0984	0.0475284744087219\\
66.5	0.099	-0.0484147030083051\\
66.5	0.0996	-0.144357880425332\\
66.5	0.1002	-0.240301057842345\\
66.5	0.1008	-0.336244235259372\\
66.5	0.1014	-0.432187412676399\\
66.5	0.102	-0.528130590093411\\
66.5	0.1026	-0.624073767510438\\
66.5	0.1032	-0.720016944927465\\
66.5	0.1038	-0.815960122344492\\
66.5	0.1044	-0.911903299761519\\
66.5	0.105	-1.00784647717855\\
66.5	0.1056	-1.10378965459557\\
66.5	0.1062	-1.19973283201259\\
66.5	0.1068	-1.29567600942961\\
66.5	0.1074	-1.39161918684664\\
66.5	0.108	-1.48756236426367\\
66.5	0.1086	-1.58350554168068\\
66.5	0.1092	-1.67944871909771\\
66.5	0.1098	-1.77539189651473\\
66.5	0.1104	-1.87133507393175\\
66.5	0.111	-1.96727825134877\\
66.5	0.1116	-2.0632214287658\\
66.5	0.1122	-2.15916460618283\\
66.5	0.1128	-2.25510778359985\\
66.5	0.1134	-2.35105096101688\\
66.5	0.114	-2.44699413843391\\
66.5	0.1146	-2.54293731585092\\
66.5	0.1152	-2.63888049326795\\
66.5	0.1158	-2.73482367068497\\
66.5	0.1164	-2.830766848102\\
66.5	0.117	-2.92671002551901\\
66.5	0.1176	-3.02265320293604\\
66.5	0.1182	-3.11859638035307\\
66.5	0.1188	-3.21453955777008\\
66.5	0.1194	-3.31048273518711\\
66.5	0.12	-3.40642591260413\\
66.5	0.1206	-3.50236909002115\\
66.5	0.1212	-3.59831226743817\\
66.5	0.1218	-3.69425544485522\\
66.5	0.1224	-3.79019862227224\\
66.5	0.123	-3.88614179968926\\
66.875	0.093	0.951302527536598\\
66.875	0.0936	0.858447186495226\\
66.875	0.0942	0.765591845453855\\
66.875	0.0948	0.672736504412484\\
66.875	0.0954	0.579881163371127\\
66.875	0.096	0.487025822329741\\
66.875	0.0966	0.39417048128837\\
66.875	0.0972	0.301315140246999\\
66.875	0.0978	0.208459799205627\\
66.875	0.0984	0.11560445816427\\
66.875	0.099	0.0227491171228991\\
66.875	0.0996	-0.0701062239184722\\
66.875	0.1002	-0.162961564959843\\
66.875	0.1008	-0.255816906001215\\
66.875	0.1014	-0.348672247042572\\
66.875	0.102	-0.441527588083943\\
66.875	0.1026	-0.534382929125314\\
66.875	0.1032	-0.627238270166686\\
66.875	0.1038	-0.720093611208043\\
66.875	0.1044	-0.812948952249428\\
66.875	0.105	-0.905804293290799\\
66.875	0.1056	-0.998659634332171\\
66.875	0.1062	-1.09151497537354\\
66.875	0.1068	-1.1843703164149\\
66.875	0.1074	-1.27722565745627\\
66.875	0.108	-1.37008099849764\\
66.875	0.1086	-1.46293633953901\\
66.875	0.1092	-1.55579168058037\\
66.875	0.1098	-1.64864702162174\\
66.875	0.1104	-1.74150236266311\\
66.875	0.111	-1.83435770370448\\
66.875	0.1116	-1.92721304474586\\
66.875	0.1122	-2.02006838578721\\
66.875	0.1128	-2.1129237268286\\
66.875	0.1134	-2.20577906786997\\
66.875	0.114	-2.29863440891134\\
66.875	0.1146	-2.39148974995271\\
66.875	0.1152	-2.48434509099407\\
66.875	0.1158	-2.57720043203544\\
66.875	0.1164	-2.67005577307681\\
66.875	0.117	-2.76291111411818\\
66.875	0.1176	-2.85576645515955\\
66.875	0.1182	-2.94862179620091\\
66.875	0.1188	-3.04147713724228\\
66.875	0.1194	-3.13433247828365\\
66.875	0.12	-3.22718781932502\\
66.875	0.1206	-3.3200431603664\\
66.875	0.1212	-3.41289850140775\\
66.875	0.1218	-3.50575384244914\\
66.875	0.1224	-3.59860918349051\\
66.875	0.123	-3.69146452453188\\
67.25	0.093	0.991587983911245\\
67.25	0.0936	0.901820479245529\\
67.25	0.0942	0.812052974579814\\
67.25	0.0948	0.722285469914112\\
67.25	0.0954	0.632517965248397\\
67.25	0.096	0.542750460582667\\
67.25	0.0966	0.452982955916951\\
67.25	0.0972	0.363215451251236\\
67.25	0.0978	0.27344794658552\\
67.25	0.0984	0.183680441919805\\
67.25	0.099	0.093912937254089\\
67.25	0.0996	0.00414543258838762\\
67.25	0.1002	-0.085622072077328\\
67.25	0.1008	-0.175389576743044\\
67.25	0.1014	-0.265157081408759\\
67.25	0.102	-0.354924586074475\\
67.25	0.1026	-0.44469209074019\\
67.25	0.1032	-0.534459595405906\\
67.25	0.1038	-0.624227100071607\\
67.25	0.1044	-0.713994604737337\\
67.25	0.105	-0.803762109403053\\
67.25	0.1056	-0.893529614068768\\
67.25	0.1062	-0.983297118734484\\
67.25	0.1068	-1.0730646234002\\
67.25	0.1074	-1.16283212806592\\
67.25	0.108	-1.25259963273163\\
67.25	0.1086	-1.34236713739733\\
67.25	0.1092	-1.43213464206305\\
67.25	0.1098	-1.52190214672876\\
67.25	0.1104	-1.61166965139448\\
67.25	0.111	-1.70143715606019\\
67.25	0.1116	-1.79120466072591\\
67.25	0.1122	-1.88097216539163\\
67.25	0.1128	-1.97073967005734\\
67.25	0.1134	-2.06050717472306\\
67.25	0.114	-2.15027467938877\\
67.25	0.1146	-2.24004218405449\\
67.25	0.1152	-2.3298096887202\\
67.25	0.1158	-2.41957719338592\\
67.25	0.1164	-2.50934469805163\\
67.25	0.117	-2.59911220271734\\
67.25	0.1176	-2.68887970738305\\
67.25	0.1182	-2.77864721204877\\
67.25	0.1188	-2.86841471671448\\
67.25	0.1194	-2.9581822213802\\
67.25	0.12	-3.04794972604591\\
67.25	0.1206	-3.13771723071163\\
67.25	0.1212	-3.22748473537735\\
67.25	0.1218	-3.31725224004306\\
67.25	0.1224	-3.40701974470878\\
67.25	0.123	-3.49678724937449\\
67.625	0.093	1.03187344028591\\
67.625	0.0936	0.945193771995847\\
67.625	0.0942	0.858514103705801\\
67.625	0.0948	0.771834435415741\\
67.625	0.0954	0.685154767125681\\
67.625	0.096	0.598475098835607\\
67.625	0.0966	0.511795430545547\\
67.625	0.0972	0.425115762255487\\
67.625	0.0978	0.338436093965427\\
67.625	0.0984	0.251756425675367\\
67.625	0.099	0.165076757385307\\
67.625	0.0996	0.0783970890952475\\
67.625	0.1002	-0.00828257919481246\\
67.625	0.1008	-0.0949622474848724\\
67.625	0.1014	-0.181641915774932\\
67.625	0.102	-0.268321584064992\\
67.625	0.1026	-0.355001252355052\\
67.625	0.1032	-0.441680920645112\\
67.625	0.1038	-0.528360588935172\\
67.625	0.1044	-0.615040257225232\\
67.625	0.105	-0.701719925515292\\
67.625	0.1056	-0.788399593805352\\
67.625	0.1062	-0.875079262095412\\
67.625	0.1068	-0.961758930385471\\
67.625	0.1074	-1.04843859867553\\
67.625	0.108	-1.13511826696559\\
67.625	0.1086	-1.22179793525565\\
67.625	0.1092	-1.30847760354571\\
67.625	0.1098	-1.39515727183577\\
67.625	0.1104	-1.48183694012583\\
67.625	0.111	-1.56851660841589\\
67.625	0.1116	-1.65519627670595\\
67.625	0.1122	-1.74187594499601\\
67.625	0.1128	-1.82855561328608\\
67.625	0.1134	-1.91523528157614\\
67.625	0.114	-2.00191494986619\\
67.625	0.1146	-2.08859461815625\\
67.625	0.1152	-2.17527428644631\\
67.625	0.1158	-2.26195395473637\\
67.625	0.1164	-2.34863362302643\\
67.625	0.117	-2.43531329131649\\
67.625	0.1176	-2.52199295960655\\
67.625	0.1182	-2.60867262789661\\
67.625	0.1188	-2.69535229618667\\
67.625	0.1194	-2.78203196447673\\
67.625	0.12	-2.86871163276679\\
67.625	0.1206	-2.95539130105685\\
67.625	0.1212	-3.04207096934691\\
67.625	0.1218	-3.12875063763698\\
67.625	0.1224	-3.21543030592704\\
67.625	0.123	-3.3021099742171\\
68	0.093	1.07215889666057\\
68	0.0936	0.988567064746164\\
68	0.0942	0.90497523283176\\
68	0.0948	0.821383400917355\\
68	0.0954	0.737791569002951\\
68	0.096	0.654199737088533\\
68	0.0966	0.570607905174128\\
68	0.0972	0.487016073259724\\
68	0.0978	0.40342424134532\\
68	0.0984	0.319832409430916\\
68	0.099	0.236240577516512\\
68	0.0996	0.152648745602107\\
68	0.1002	0.0690569136877031\\
68	0.1008	-0.0145349182267012\\
68	0.1014	-0.0981267501411054\\
68	0.102	-0.18171858205551\\
68	0.1026	-0.265310413969914\\
68	0.1032	-0.348902245884318\\
68	0.1038	-0.432494077798722\\
68	0.1044	-0.516085909713141\\
68	0.105	-0.599677741627545\\
68	0.1056	-0.683269573541949\\
68	0.1062	-0.766861405456353\\
68	0.1068	-0.850453237370758\\
68	0.1074	-0.934045069285162\\
68	0.108	-1.01763690119957\\
68	0.1086	-1.10122873311397\\
68	0.1092	-1.18482056502837\\
68	0.1098	-1.26841239694278\\
68	0.1104	-1.3520042288572\\
68	0.111	-1.4355960607716\\
68	0.1116	-1.51918789268601\\
68	0.1122	-1.60277972460041\\
68	0.1128	-1.68637155651483\\
68	0.1134	-1.76996338842923\\
68	0.114	-1.85355522034364\\
68	0.1146	-1.93714705225804\\
68	0.1152	-2.02073888417245\\
68	0.1158	-2.10433071608685\\
68	0.1164	-2.18792254800125\\
68	0.117	-2.27151437991566\\
68	0.1176	-2.35510621183006\\
68	0.1182	-2.43869804374447\\
68	0.1188	-2.52228987565887\\
68	0.1194	-2.60588170757327\\
68	0.12	-2.68947353948768\\
68	0.1206	-2.77306537140208\\
68	0.1212	-2.85665720331649\\
68	0.1218	-2.94024903523091\\
68	0.1224	-3.02384086714531\\
68	0.123	-3.10743269905971\\
68.375	0.093	1.11244435303522\\
68.375	0.0936	1.03194035749647\\
68.375	0.0942	0.951436361957718\\
68.375	0.0948	0.87093236641897\\
68.375	0.0954	0.790428370880221\\
68.375	0.096	0.709924375341458\\
68.375	0.0966	0.62942037980271\\
68.375	0.0972	0.548916384263961\\
68.375	0.0978	0.468412388725213\\
68.375	0.0984	0.387908393186464\\
68.375	0.099	0.307404397647701\\
68.375	0.0996	0.226900402108953\\
68.375	0.1002	0.146396406570204\\
68.375	0.1008	0.0658924110314558\\
68.375	0.1014	-0.0146115845072927\\
68.375	0.102	-0.0951155800460413\\
68.375	0.1026	-0.17561957558479\\
68.375	0.1032	-0.256123571123538\\
68.375	0.1038	-0.336627566662287\\
68.375	0.1044	-0.41713156220105\\
68.375	0.105	-0.497635557739798\\
68.375	0.1056	-0.578139553278561\\
68.375	0.1062	-0.658643548817309\\
68.375	0.1068	-0.739147544356058\\
68.375	0.1074	-0.819651539894807\\
68.375	0.108	-0.900155535433555\\
68.375	0.1086	-0.980659530972304\\
68.375	0.1092	-1.06116352651105\\
68.375	0.1098	-1.1416675220498\\
68.375	0.1104	-1.22217151758855\\
68.375	0.111	-1.3026755131273\\
68.375	0.1116	-1.38317950866606\\
68.375	0.1122	-1.46368350420481\\
68.375	0.1128	-1.54418749974357\\
68.375	0.1134	-1.62469149528232\\
68.375	0.114	-1.70519549082107\\
68.375	0.1146	-1.78569948635982\\
68.375	0.1152	-1.86620348189857\\
68.375	0.1158	-1.94670747743731\\
68.375	0.1164	-2.02721147297606\\
68.375	0.117	-2.10771546851481\\
68.375	0.1176	-2.18821946405356\\
68.375	0.1182	-2.26872345959232\\
68.375	0.1188	-2.34922745513107\\
68.375	0.1194	-2.42973145066982\\
68.375	0.12	-2.51023544620857\\
68.375	0.1206	-2.59073944174732\\
68.375	0.1212	-2.67124343728607\\
68.375	0.1218	-2.75174743282483\\
68.375	0.1224	-2.83225142836358\\
68.375	0.123	-2.91275542390233\\
68.75	0.093	1.15272980940989\\
68.75	0.0936	1.0753136502468\\
68.75	0.0942	0.997897491083691\\
68.75	0.0948	0.920481331920598\\
68.75	0.0954	0.843065172757505\\
68.75	0.096	0.765649013594398\\
68.75	0.0966	0.688232854431305\\
68.75	0.0972	0.610816695268213\\
68.75	0.0978	0.533400536105106\\
68.75	0.0984	0.455984376942013\\
68.75	0.099	0.37856821777892\\
68.75	0.0996	0.301152058615827\\
68.75	0.1002	0.223735899452734\\
68.75	0.1008	0.146319740289641\\
68.75	0.1014	0.0689035811265342\\
68.75	0.102	-0.00851257803655869\\
68.75	0.1026	-0.0859287371996516\\
68.75	0.1032	-0.163344896362744\\
68.75	0.1038	-0.240761055525837\\
68.75	0.1044	-0.318177214688944\\
68.75	0.105	-0.395593373852051\\
68.75	0.1056	-0.473009533015144\\
68.75	0.1062	-0.550425692178237\\
68.75	0.1068	-0.62784185134133\\
68.75	0.1074	-0.705258010504423\\
68.75	0.108	-0.78267416966753\\
68.75	0.1086	-0.860090328830623\\
68.75	0.1092	-0.937506487993716\\
68.75	0.1098	-1.01492264715681\\
68.75	0.1104	-1.0923388063199\\
68.75	0.111	-1.16975496548299\\
68.75	0.1116	-1.2471711246461\\
68.75	0.1122	-1.32458728380919\\
68.75	0.1128	-1.4020034429723\\
68.75	0.1134	-1.47941960213539\\
68.75	0.114	-1.55683576129849\\
68.75	0.1146	-1.63425192046158\\
68.75	0.1152	-1.71166807962469\\
68.75	0.1158	-1.78908423878778\\
68.75	0.1164	-1.86650039795087\\
68.75	0.117	-1.94391655711397\\
68.75	0.1176	-2.02133271627706\\
68.75	0.1182	-2.09874887544015\\
68.75	0.1188	-2.17616503460326\\
68.75	0.1194	-2.25358119376635\\
68.75	0.12	-2.33099735292944\\
68.75	0.1206	-2.40841351209254\\
68.75	0.1212	-2.48582967125563\\
68.75	0.1218	-2.56324583041874\\
68.75	0.1224	-2.64066198958184\\
68.75	0.123	-2.71807814874494\\
69.125	0.093	1.19301526578455\\
69.125	0.0936	1.11868694299712\\
69.125	0.0942	1.04435862020968\\
69.125	0.0948	0.970030297422227\\
69.125	0.0954	0.89570197463479\\
69.125	0.096	0.821373651847338\\
69.125	0.0966	0.747045329059901\\
69.125	0.0972	0.67271700627245\\
69.125	0.0978	0.598388683485013\\
69.125	0.0984	0.524060360697575\\
69.125	0.099	0.449732037910138\\
69.125	0.0996	0.375403715122687\\
69.125	0.1002	0.30107539233525\\
69.125	0.1008	0.226747069547812\\
69.125	0.1014	0.152418746760375\\
69.125	0.102	0.0780904239729239\\
69.125	0.1026	0.0037621011854867\\
69.125	0.1032	-0.0705662216019505\\
69.125	0.1038	-0.144894544389388\\
69.125	0.1044	-0.219222867176853\\
69.125	0.105	-0.29355118996429\\
69.125	0.1056	-0.367879512751728\\
69.125	0.1062	-0.442207835539165\\
69.125	0.1068	-0.516536158326616\\
69.125	0.1074	-0.590864481114053\\
69.125	0.108	-0.66519280390149\\
69.125	0.1086	-0.739521126688928\\
69.125	0.1092	-0.813849449476379\\
69.125	0.1098	-0.888177772263816\\
69.125	0.1104	-0.962506095051253\\
69.125	0.111	-1.03683441783869\\
69.125	0.1116	-1.11116274062614\\
69.125	0.1122	-1.18549106341358\\
69.125	0.1128	-1.25981938620103\\
69.125	0.1134	-1.33414770898847\\
69.125	0.114	-1.40847603177592\\
69.125	0.1146	-1.48280435456336\\
69.125	0.1152	-1.55713267735079\\
69.125	0.1158	-1.63146100013823\\
69.125	0.1164	-1.70578932292568\\
69.125	0.117	-1.78011764571312\\
69.125	0.1176	-1.85444596850056\\
69.125	0.1182	-1.92877429128799\\
69.125	0.1188	-2.00310261407543\\
69.125	0.1194	-2.07743093686288\\
69.125	0.12	-2.15175925965032\\
69.125	0.1206	-2.22608758243776\\
69.125	0.1212	-2.30041590522519\\
69.125	0.1218	-2.37474422801266\\
69.125	0.1224	-2.4490725508001\\
69.125	0.123	-2.52340087358753\\
69.5	0.093	1.2333007221592\\
69.5	0.0936	1.1620602357474\\
69.5	0.0942	1.09081974933562\\
69.5	0.0948	1.01957926292384\\
69.5	0.0954	0.948338776512045\\
69.5	0.096	0.87709829010025\\
69.5	0.0966	0.805857803688468\\
69.5	0.0972	0.734617317276673\\
69.5	0.0978	0.663376830864891\\
69.5	0.0984	0.59213634445311\\
69.5	0.099	0.520895858041314\\
69.5	0.0996	0.449655371629532\\
69.5	0.1002	0.378414885217751\\
69.5	0.1008	0.307174398805955\\
69.5	0.1014	0.235933912394174\\
69.5	0.102	0.164693425982392\\
69.5	0.1026	0.0934529395705965\\
69.5	0.1032	0.022212453158815\\
69.5	0.1038	-0.0490280332529665\\
69.5	0.1044	-0.120268519664762\\
69.5	0.105	-0.191509006076558\\
69.5	0.1056	-0.262749492488339\\
69.5	0.1062	-0.333989978900135\\
69.5	0.1068	-0.405230465311917\\
69.5	0.1074	-0.476470951723698\\
69.5	0.108	-0.54771143813548\\
69.5	0.1086	-0.618951924547275\\
69.5	0.1092	-0.690192410959057\\
69.5	0.1098	-0.761432897370838\\
69.5	0.1104	-0.832673383782634\\
69.5	0.111	-0.903913870194415\\
69.5	0.1116	-0.975154356606197\\
69.5	0.1122	-1.04639484301799\\
69.5	0.1128	-1.11763532942979\\
69.5	0.1134	-1.18887581584157\\
69.5	0.114	-1.26011630225337\\
69.5	0.1146	-1.33135678866515\\
69.5	0.1152	-1.40259727507693\\
69.5	0.1158	-1.47383776148872\\
69.5	0.1164	-1.54507824790051\\
69.5	0.117	-1.61631873431229\\
69.5	0.1176	-1.68755922072408\\
69.5	0.1182	-1.75879970713586\\
69.5	0.1188	-1.83004019354765\\
69.5	0.1194	-1.90128067995944\\
69.5	0.12	-1.97252116637122\\
69.5	0.1206	-2.043761652783\\
69.5	0.1212	-2.1150021391948\\
69.5	0.1218	-2.1862426256066\\
69.5	0.1224	-2.25748311201838\\
69.5	0.123	-2.32872359843017\\
69.875	0.093	1.27358617853386\\
69.875	0.0936	1.20543352849774\\
69.875	0.0942	1.1372808784616\\
69.875	0.0948	1.06912822842547\\
69.875	0.0954	1.00097557838933\\
69.875	0.096	0.93282292835319\\
69.875	0.0966	0.864670278317064\\
69.875	0.0972	0.796517628280924\\
69.875	0.0978	0.728364978244798\\
69.875	0.0984	0.660212328208658\\
69.875	0.099	0.592059678172532\\
69.875	0.0996	0.523907028136406\\
69.875	0.1002	0.455754378100266\\
69.875	0.1008	0.387601728064141\\
69.875	0.1014	0.319449078028001\\
69.875	0.102	0.251296427991875\\
69.875	0.1026	0.183143777955749\\
69.875	0.1032	0.114991127919609\\
69.875	0.1038	0.0468384778834832\\
69.875	0.1044	-0.0213141721526711\\
69.875	0.105	-0.0894668221887969\\
69.875	0.1056	-0.157619472224923\\
69.875	0.1062	-0.225772122261063\\
69.875	0.1068	-0.293924772297188\\
69.875	0.1074	-0.362077422333329\\
69.875	0.108	-0.430230072369454\\
69.875	0.1086	-0.49838272240558\\
69.875	0.1092	-0.56653537244172\\
69.875	0.1098	-0.634688022477846\\
69.875	0.1104	-0.702840672513986\\
69.875	0.111	-0.770993322550112\\
69.875	0.1116	-0.839145972586238\\
69.875	0.1122	-0.907298622622378\\
69.875	0.1128	-0.975451272658518\\
69.875	0.1134	-1.04360392269466\\
69.875	0.114	-1.11175657273078\\
69.875	0.1146	-1.17990922276691\\
69.875	0.1152	-1.24806187280305\\
69.875	0.1158	-1.31621452283918\\
69.875	0.1164	-1.38436717287532\\
69.875	0.117	-1.45251982291144\\
69.875	0.1176	-1.52067247294757\\
69.875	0.1182	-1.58882512298371\\
69.875	0.1188	-1.65697777301983\\
69.875	0.1194	-1.72513042305597\\
69.875	0.12	-1.7932830730921\\
69.875	0.1206	-1.86143572312822\\
69.875	0.1212	-1.92958837316436\\
69.875	0.1218	-1.9977410232005\\
69.875	0.1224	-2.06589367323664\\
69.875	0.123	-2.13404632327277\\
70.25	0.093	1.31387163490852\\
70.25	0.0936	1.24880682124805\\
70.25	0.0942	1.18374200758757\\
70.25	0.0948	1.1186771939271\\
70.25	0.0954	1.05361238026661\\
70.25	0.096	0.98854756660613\\
70.25	0.0966	0.923482752945645\\
70.25	0.0972	0.858417939285175\\
70.25	0.0978	0.793353125624705\\
70.25	0.0984	0.728288311964221\\
70.25	0.099	0.663223498303751\\
70.25	0.0996	0.598158684643266\\
70.25	0.1002	0.533093870982796\\
70.25	0.1008	0.468029057322312\\
70.25	0.1014	0.402964243661842\\
70.25	0.102	0.337899430001357\\
70.25	0.1026	0.272834616340887\\
70.25	0.1032	0.207769802680403\\
70.25	0.1038	0.142704989019933\\
70.25	0.1044	0.0776401753594342\\
70.25	0.105	0.0125753616989641\\
70.25	0.1056	-0.0524894519615202\\
70.25	0.1062	-0.11755426562199\\
70.25	0.1068	-0.182619079282475\\
70.25	0.1074	-0.247683892942945\\
70.25	0.108	-0.312748706603429\\
70.25	0.1086	-0.377813520263899\\
70.25	0.1092	-0.442878333924369\\
70.25	0.1098	-0.507943147584854\\
70.25	0.1104	-0.573007961245324\\
70.25	0.111	-0.638072774905808\\
70.25	0.1116	-0.703137588566278\\
70.25	0.1122	-0.768202402226763\\
70.25	0.1128	-0.833267215887247\\
70.25	0.1134	-0.898332029547731\\
70.25	0.114	-0.963396843208201\\
70.25	0.1146	-1.02846165686869\\
70.25	0.1152	-1.09352647052916\\
70.25	0.1158	-1.15859128418964\\
70.25	0.1164	-1.22365609785011\\
70.25	0.117	-1.28872091151059\\
70.25	0.1176	-1.35378572517106\\
70.25	0.1182	-1.41885053883155\\
70.25	0.1188	-1.48391535249202\\
70.25	0.1194	-1.5489801661525\\
70.25	0.12	-1.61404497981297\\
70.25	0.1206	-1.67910979347346\\
70.25	0.1212	-1.74417460713393\\
70.25	0.1218	-1.80923942079441\\
70.25	0.1224	-1.8743042344549\\
70.25	0.123	-1.93936904811537\\
70.625	0.093	1.35415709128317\\
70.625	0.0936	1.29218011399834\\
70.625	0.0942	1.23020313671353\\
70.625	0.0948	1.1682261594287\\
70.625	0.0954	1.10624918214388\\
70.625	0.096	1.04427220485904\\
70.625	0.0966	0.982295227574227\\
70.625	0.0972	0.920318250289398\\
70.625	0.0978	0.858341273004584\\
70.625	0.0984	0.796364295719755\\
70.625	0.099	0.734387318434926\\
70.625	0.0996	0.672410341150112\\
70.625	0.1002	0.610433363865283\\
70.625	0.1008	0.548456386580469\\
70.625	0.1014	0.48647940929564\\
70.625	0.102	0.424502432010826\\
70.625	0.1026	0.362525454725997\\
70.625	0.1032	0.300548477441183\\
70.625	0.1038	0.238571500156354\\
70.625	0.1044	0.176594522871511\\
70.625	0.105	0.114617545586697\\
70.625	0.1056	0.052640568301868\\
70.625	0.1062	-0.00933640898294641\\
70.625	0.1068	-0.0713133862677751\\
70.625	0.1074	-0.133290363552589\\
70.625	0.108	-0.195267340837418\\
70.625	0.1086	-0.257244318122233\\
70.625	0.1092	-0.319221295407061\\
70.625	0.1098	-0.38119827269189\\
70.625	0.1104	-0.443175249976704\\
70.625	0.111	-0.505152227261533\\
70.625	0.1116	-0.567129204546347\\
70.625	0.1122	-0.629106181831176\\
70.625	0.1128	-0.691083159116005\\
70.625	0.1134	-0.753060136400833\\
70.625	0.114	-0.815037113685662\\
70.625	0.1146	-0.877014090970476\\
70.625	0.1152	-0.938991068255305\\
70.625	0.1158	-1.00096804554012\\
70.625	0.1164	-1.06294502282495\\
70.625	0.117	-1.12492200010976\\
70.625	0.1176	-1.18689897739459\\
70.625	0.1182	-1.24887595467941\\
70.625	0.1188	-1.31085293196423\\
70.625	0.1194	-1.37282990924905\\
70.625	0.12	-1.43480688653388\\
70.625	0.1206	-1.49678386381871\\
70.625	0.1212	-1.55876084110352\\
70.625	0.1218	-1.62073781838836\\
70.625	0.1224	-1.68271479567318\\
70.625	0.123	-1.74469177295801\\
71	0.093	1.39444254765783\\
71	0.0936	1.33555340674867\\
71	0.0942	1.2766642658395\\
71	0.0948	1.21777512493033\\
71	0.0954	1.15888598402117\\
71	0.096	1.09999684311198\\
71	0.0966	1.04110770220281\\
71	0.0972	0.982218561293649\\
71	0.0978	0.923329420384476\\
71	0.0984	0.864440279475318\\
71	0.099	0.805551138566145\\
71	0.0996	0.746661997656972\\
71	0.1002	0.687772856747813\\
71	0.1008	0.62888371583864\\
71	0.1014	0.569994574929467\\
71	0.102	0.511105434020308\\
71	0.1026	0.452216293111135\\
71	0.1032	0.393327152201977\\
71	0.1038	0.334438011292804\\
71	0.1044	0.275548870383616\\
71	0.105	0.216659729474458\\
71	0.1056	0.157770588565285\\
71	0.1062	0.0988814476561117\\
71	0.1068	0.039992306746953\\
71	0.1074	-0.01889683416222\\
71	0.108	-0.0777859750713787\\
71	0.1086	-0.136675115980552\\
71	0.1092	-0.195564256889725\\
71	0.1098	-0.254453397798883\\
71	0.1104	-0.313342538708056\\
71	0.111	-0.372231679617229\\
71	0.1116	-0.431120820526388\\
71	0.1122	-0.490009961435561\\
71	0.1128	-0.548899102344748\\
71	0.1134	-0.607788243253907\\
71	0.114	-0.66667738416308\\
71	0.1146	-0.725566525072239\\
71	0.1152	-0.784455665981412\\
71	0.1158	-0.843344806890585\\
71	0.1164	-0.902233947799743\\
71	0.117	-0.961123088708916\\
71	0.1176	-1.02001222961809\\
71	0.1182	-1.07890137052725\\
71	0.1188	-1.13779051143642\\
71	0.1194	-1.19667965234558\\
71	0.12	-1.25556879325475\\
71	0.1206	-1.31445793416393\\
71	0.1212	-1.37334707507308\\
71	0.1218	-1.43223621598227\\
71	0.1224	-1.49112535689144\\
71	0.123	-1.5500144978006\\
71.375	0.093	1.43472800403251\\
71.375	0.0936	1.37892669949899\\
71.375	0.0942	1.32312539496547\\
71.375	0.0948	1.26732409043197\\
71.375	0.0954	1.21152278589845\\
71.375	0.096	1.15572148136492\\
71.375	0.0966	1.0999201768314\\
71.375	0.0972	1.0441188722979\\
71.375	0.0978	0.988317567764383\\
71.375	0.0984	0.932516263230866\\
71.375	0.099	0.876714958697363\\
71.375	0.0996	0.820913654163846\\
71.375	0.1002	0.765112349630328\\
71.375	0.1008	0.709311045096811\\
71.375	0.1014	0.653509740563308\\
71.375	0.102	0.597708436029791\\
71.375	0.1026	0.541907131496274\\
71.375	0.1032	0.48610582696277\\
71.375	0.1038	0.430304522429253\\
71.375	0.1044	0.374503217895722\\
71.375	0.105	0.318701913362219\\
71.375	0.1056	0.262900608828701\\
71.375	0.1062	0.207099304295184\\
71.375	0.1068	0.151297999761667\\
71.375	0.1074	0.0954966952281637\\
71.375	0.108	0.0396953906946464\\
71.375	0.1086	-0.0161059138388708\\
71.375	0.1092	-0.0719072183723739\\
71.375	0.1098	-0.127708522905891\\
71.375	0.1104	-0.183509827439408\\
71.375	0.111	-0.239311131972926\\
71.375	0.1116	-0.295112436506429\\
71.375	0.1122	-0.350913741039946\\
71.375	0.1128	-0.406715045573478\\
71.375	0.1134	-0.462516350106981\\
71.375	0.114	-0.518317654640498\\
71.375	0.1146	-0.574118959174015\\
71.375	0.1152	-0.629920263707518\\
71.375	0.1158	-0.685721568241036\\
71.375	0.1164	-0.741522872774553\\
71.375	0.117	-0.79732417730807\\
71.375	0.1176	-0.853125481841573\\
71.375	0.1182	-0.908926786375091\\
71.375	0.1188	-0.964728090908608\\
71.375	0.1194	-1.02052939544211\\
71.375	0.12	-1.07633069997563\\
71.375	0.1206	-1.13213200450915\\
71.375	0.1212	-1.18793330904266\\
71.375	0.1218	-1.24373461357618\\
71.375	0.1224	-1.2995359181097\\
71.375	0.123	-1.35533722264321\\
71.75	0.093	1.47501346040714\\
71.75	0.0936	1.42229999224928\\
71.75	0.0942	1.36958652409143\\
71.75	0.0948	1.31687305593357\\
71.75	0.0954	1.26415958777571\\
71.75	0.096	1.21144611961783\\
71.75	0.0966	1.15873265145999\\
71.75	0.0972	1.10601918330212\\
71.75	0.0978	1.05330571514426\\
71.75	0.0984	1.0005922469864\\
71.75	0.099	0.947878778828539\\
71.75	0.0996	0.895165310670691\\
71.75	0.1002	0.84245184251283\\
71.75	0.1008	0.789738374354968\\
71.75	0.1014	0.737024906197107\\
71.75	0.102	0.684311438039245\\
71.75	0.1026	0.631597969881398\\
71.75	0.1032	0.578884501723536\\
71.75	0.1038	0.526171033565674\\
71.75	0.1044	0.473457565407799\\
71.75	0.105	0.420744097249951\\
71.75	0.1056	0.36803062909209\\
71.75	0.1062	0.315317160934228\\
71.75	0.1068	0.262603692776366\\
71.75	0.1074	0.209890224618505\\
71.75	0.108	0.157176756460657\\
71.75	0.1086	0.104463288302796\\
71.75	0.1092	0.0517498201449342\\
71.75	0.1098	-0.000963648012927365\\
71.75	0.1104	-0.053677116170789\\
71.75	0.111	-0.106390584328636\\
71.75	0.1116	-0.159104052486498\\
71.75	0.1122	-0.21181752064436\\
71.75	0.1128	-0.264530988802235\\
71.75	0.1134	-0.317244456960083\\
71.75	0.114	-0.369957925117944\\
71.75	0.1146	-0.422671393275806\\
71.75	0.1152	-0.475384861433668\\
71.75	0.1158	-0.528098329591529\\
71.75	0.1164	-0.580811797749377\\
71.75	0.117	-0.633525265907238\\
71.75	0.1176	-0.6862387340651\\
71.75	0.1182	-0.738952202222961\\
71.75	0.1188	-0.791665670380823\\
71.75	0.1194	-0.84437913853867\\
71.75	0.12	-0.897092606696532\\
71.75	0.1206	-0.949806074854394\\
71.75	0.1212	-1.00251954301226\\
71.75	0.1218	-1.05523301117012\\
71.75	0.1224	-1.10794647932798\\
71.75	0.123	-1.16065994748584\\
72.125	0.093	1.51529891678182\\
72.125	0.0936	1.46567328499961\\
72.125	0.0942	1.4160476532174\\
72.125	0.0948	1.3664220214352\\
72.125	0.0954	1.31679638965299\\
72.125	0.096	1.26717075787077\\
72.125	0.0966	1.21754512608857\\
72.125	0.0972	1.16791949430637\\
72.125	0.0978	1.11829386252417\\
72.125	0.0984	1.06866823074196\\
72.125	0.099	1.01904259895976\\
72.125	0.0996	0.969416967177551\\
72.125	0.1002	0.919791335395345\\
72.125	0.1008	0.870165703613139\\
72.125	0.1014	0.820540071830948\\
72.125	0.102	0.770914440048742\\
72.125	0.1026	0.721288808266536\\
72.125	0.1032	0.67166317648433\\
72.125	0.1038	0.622037544702124\\
72.125	0.1044	0.572411912919904\\
72.125	0.105	0.522786281137698\\
72.125	0.1056	0.473160649355506\\
72.125	0.1062	0.4235350175733\\
72.125	0.1068	0.373909385791094\\
72.125	0.1074	0.324283754008889\\
72.125	0.108	0.274658122226683\\
72.125	0.1086	0.225032490444477\\
72.125	0.1092	0.175406858662271\\
72.125	0.1098	0.125781226880065\\
72.125	0.1104	0.0761555950978732\\
72.125	0.111	0.0265299633156673\\
72.125	0.1116	-0.0230956684665387\\
72.125	0.1122	-0.0727213002487446\\
72.125	0.1128	-0.122346932030965\\
72.125	0.1134	-0.171972563813171\\
72.125	0.114	-0.221598195595377\\
72.125	0.1146	-0.271223827377568\\
72.125	0.1152	-0.320849459159774\\
72.125	0.1158	-0.37047509094198\\
72.125	0.1164	-0.420100722724186\\
72.125	0.117	-0.469726354506392\\
72.125	0.1176	-0.519351986288598\\
72.125	0.1182	-0.568977618070804\\
72.125	0.1188	-0.618603249852995\\
72.125	0.1194	-0.668228881635201\\
72.125	0.12	-0.717854513417407\\
72.125	0.1206	-0.767480145199613\\
72.125	0.1212	-0.817105776981819\\
72.125	0.1218	-0.866731408764039\\
72.125	0.1224	-0.916357040546245\\
72.125	0.123	-0.965982672328437\\
72.5	0.093	1.55558437315648\\
72.5	0.0936	1.50904657774993\\
72.5	0.0942	1.46250878234338\\
72.5	0.0948	1.41597098693683\\
72.5	0.0954	1.36943319153028\\
72.5	0.096	1.32289539612371\\
72.5	0.0966	1.27635760071716\\
72.5	0.0972	1.22981980531061\\
72.5	0.0978	1.18328200990406\\
72.5	0.0984	1.13674421449751\\
72.5	0.099	1.09020641909098\\
72.5	0.0996	1.04366862368443\\
72.5	0.1002	0.997130828277875\\
72.5	0.1008	0.950593032871325\\
72.5	0.1014	0.904055237464775\\
72.5	0.102	0.857517442058224\\
72.5	0.1026	0.810979646651674\\
72.5	0.1032	0.764441851245124\\
72.5	0.1038	0.717904055838574\\
72.5	0.1044	0.671366260432009\\
72.5	0.105	0.624828465025459\\
72.5	0.1056	0.578290669618909\\
72.5	0.1062	0.531752874212359\\
72.5	0.1068	0.485215078805808\\
72.5	0.1074	0.438677283399258\\
72.5	0.108	0.392139487992722\\
72.5	0.1086	0.345601692586172\\
72.5	0.1092	0.299063897179622\\
72.5	0.1098	0.252526101773071\\
72.5	0.1104	0.205988306366521\\
72.5	0.111	0.159450510959971\\
72.5	0.1116	0.112912715553421\\
72.5	0.1122	0.0663749201468704\\
72.5	0.1128	0.019837124740306\\
72.5	0.1134	-0.0267006706662443\\
72.5	0.114	-0.0732384660727945\\
72.5	0.1146	-0.119776261479345\\
72.5	0.1152	-0.166314056885895\\
72.5	0.1158	-0.212851852292445\\
72.5	0.1164	-0.259389647698995\\
72.5	0.117	-0.305927443105546\\
72.5	0.1176	-0.352465238512082\\
72.5	0.1182	-0.399003033918632\\
72.5	0.1188	-0.445540829325182\\
72.5	0.1194	-0.492078624731732\\
72.5	0.12	-0.538616420138283\\
72.5	0.1206	-0.585154215544833\\
72.5	0.1212	-0.631692010951383\\
72.5	0.1218	-0.678229806357947\\
72.5	0.1224	-0.724767601764498\\
72.5	0.123	-0.771305397171048\\
72.875	0.093	1.59586982953113\\
72.875	0.0936	1.55241987050022\\
72.875	0.0942	1.50896991146932\\
72.875	0.0948	1.46551995243843\\
72.875	0.0954	1.42206999340753\\
72.875	0.096	1.37862003437662\\
72.875	0.0966	1.33517007534573\\
72.875	0.0972	1.29172011631483\\
72.875	0.0978	1.24827015728394\\
72.875	0.0984	1.20482019825305\\
72.875	0.099	1.16137023922215\\
72.875	0.0996	1.11792028019126\\
72.875	0.1002	1.07447032116036\\
72.875	0.1008	1.03102036212947\\
72.875	0.1014	0.987570403098573\\
72.875	0.102	0.944120444067678\\
72.875	0.1026	0.900670485036784\\
72.875	0.1032	0.857220526005889\\
72.875	0.1038	0.813770566974995\\
72.875	0.1044	0.770320607944086\\
72.875	0.105	0.726870648913192\\
72.875	0.1056	0.683420689882297\\
72.875	0.1062	0.639970730851402\\
72.875	0.1068	0.596520771820508\\
72.875	0.1074	0.553070812789613\\
72.875	0.108	0.509620853758719\\
72.875	0.1086	0.466170894727824\\
72.875	0.1092	0.42272093569693\\
72.875	0.1098	0.379270976666035\\
72.875	0.1104	0.335821017635141\\
72.875	0.111	0.292371058604246\\
72.875	0.1116	0.248921099573352\\
72.875	0.1122	0.205471140542457\\
72.875	0.1128	0.162021181511548\\
72.875	0.1134	0.118571222480654\\
72.875	0.114	0.0751212634497591\\
72.875	0.1146	0.0316713044188646\\
72.875	0.1152	-0.01177865461203\\
72.875	0.1158	-0.0552286136429245\\
72.875	0.1164	-0.0986785726738191\\
72.875	0.117	-0.142128531704714\\
72.875	0.1176	-0.185578490735608\\
72.875	0.1182	-0.229028449766503\\
72.875	0.1188	-0.272478408797397\\
72.875	0.1194	-0.315928367828292\\
72.875	0.12	-0.359378326859186\\
72.875	0.1206	-0.402828285890081\\
72.875	0.1212	-0.446278244920975\\
72.875	0.1218	-0.489728203951884\\
72.875	0.1224	-0.533178162982779\\
72.875	0.123	-0.576628122013673\\
73.25	0.093	1.63615528590579\\
73.25	0.0936	1.59579316325055\\
73.25	0.0942	1.55543104059531\\
73.25	0.0948	1.51506891794007\\
73.25	0.0954	1.47470679528483\\
73.25	0.096	1.43434467262956\\
73.25	0.0966	1.39398254997433\\
73.25	0.0972	1.35362042731909\\
73.25	0.0978	1.31325830466385\\
73.25	0.0984	1.27289618200861\\
73.25	0.099	1.23253405935337\\
73.25	0.0996	1.19217193669813\\
73.25	0.1002	1.15180981404289\\
73.25	0.1008	1.11144769138765\\
73.25	0.1014	1.07108556873241\\
73.25	0.102	1.03072344607718\\
73.25	0.1026	0.990361323421922\\
73.25	0.1032	0.949999200766683\\
73.25	0.1038	0.909637078111444\\
73.25	0.1044	0.869274955456191\\
73.25	0.105	0.828912832800953\\
73.25	0.1056	0.788550710145714\\
73.25	0.1062	0.748188587490475\\
73.25	0.1068	0.707826464835236\\
73.25	0.1074	0.667464342179997\\
73.25	0.108	0.627102219524758\\
73.25	0.1086	0.586740096869505\\
73.25	0.1092	0.546377974214266\\
73.25	0.1098	0.506015851559027\\
73.25	0.1104	0.465653728903789\\
73.25	0.111	0.42529160624855\\
73.25	0.1116	0.384929483593311\\
73.25	0.1122	0.344567360938072\\
73.25	0.1128	0.304205238282819\\
73.25	0.1134	0.26384311562758\\
73.25	0.114	0.223480992972341\\
73.25	0.1146	0.183118870317102\\
73.25	0.1152	0.142756747661849\\
73.25	0.1158	0.10239462500661\\
73.25	0.1164	0.0620325023513715\\
73.25	0.117	0.0216703796961326\\
73.25	0.1176	-0.0186917429591063\\
73.25	0.1182	-0.0590538656143451\\
73.25	0.1188	-0.099415988269584\\
73.25	0.1194	-0.139778110924823\\
73.25	0.12	-0.180140233580062\\
73.25	0.1206	-0.220502356235301\\
73.25	0.1212	-0.260864478890539\\
73.25	0.1218	-0.301226601545807\\
73.25	0.1224	-0.341588724201046\\
73.25	0.123	-0.381950846856284\\
73.625	0.093	1.67644074228045\\
73.625	0.0936	1.63916645600086\\
73.625	0.0942	1.60189216972128\\
73.625	0.0948	1.5646178834417\\
73.625	0.0954	1.52734359716212\\
73.625	0.096	1.4900693108825\\
73.625	0.0966	1.45279502460292\\
73.625	0.0972	1.41552073832334\\
73.625	0.0978	1.37824645204375\\
73.625	0.0984	1.34097216576417\\
73.625	0.099	1.30369787948459\\
73.625	0.0996	1.26642359320499\\
73.625	0.1002	1.22914930692541\\
73.625	0.1008	1.19187502064582\\
73.625	0.1014	1.15460073436624\\
73.625	0.102	1.11732644808666\\
73.625	0.1026	1.08005216180707\\
73.625	0.1032	1.04277787552748\\
73.625	0.1038	1.00550358924789\\
73.625	0.1044	0.968229302968297\\
73.625	0.105	0.930955016688713\\
73.625	0.1056	0.89368073040913\\
73.625	0.1062	0.856406444129547\\
73.625	0.1068	0.81913215784995\\
73.625	0.1074	0.781857871570367\\
73.625	0.108	0.744583585290783\\
73.625	0.1086	0.7073092990112\\
73.625	0.1092	0.670035012731617\\
73.625	0.1098	0.632760726452034\\
73.625	0.1104	0.595486440172436\\
73.625	0.111	0.558212153892853\\
73.625	0.1116	0.52093786761327\\
73.625	0.1122	0.483663581333687\\
73.625	0.1128	0.44638929505409\\
73.625	0.1134	0.409115008774506\\
73.625	0.114	0.371840722494909\\
73.625	0.1146	0.334566436215326\\
73.625	0.1152	0.297292149935743\\
73.625	0.1158	0.260017863656159\\
73.625	0.1164	0.222743577376576\\
73.625	0.117	0.185469291096993\\
73.625	0.1176	0.148195004817396\\
73.625	0.1182	0.110920718537812\\
73.625	0.1188	0.0736464322582293\\
73.625	0.1194	0.0363721459786461\\
73.625	0.12	-0.000902140300937049\\
73.625	0.1206	-0.0381764265805202\\
73.625	0.1212	-0.0754507128601176\\
73.625	0.1218	-0.112724999139715\\
73.625	0.1224	-0.149999285419298\\
73.625	0.123	-0.187273571698881\\
74	0.093	1.7167261986551\\
74	0.0936	1.68253974875117\\
74	0.0942	1.64835329884723\\
74	0.0948	1.6141668489433\\
74	0.0954	1.57998039903937\\
74	0.096	1.54579394913543\\
74	0.0966	1.51160749923149\\
74	0.0972	1.47742104932756\\
74	0.0978	1.44323459942363\\
74	0.0984	1.40904814951971\\
74	0.099	1.37486169961576\\
74	0.0996	1.34067524971184\\
74	0.1002	1.30648879980791\\
74	0.1008	1.27230234990398\\
74	0.1014	1.23811590000004\\
74	0.102	1.20392945009611\\
74	0.1026	1.16974300019218\\
74	0.1032	1.13555655028826\\
74	0.1038	1.10137010038432\\
74	0.1044	1.06718365048037\\
74	0.105	1.03299720057645\\
74	0.1056	0.998810750672519\\
74	0.1062	0.964624300768577\\
74	0.1068	0.930437850864649\\
74	0.1074	0.896251400960722\\
74	0.108	0.862064951056794\\
74	0.1086	0.827878501152853\\
74	0.1092	0.793692051248925\\
74	0.1098	0.759505601344998\\
74	0.1104	0.72531915144107\\
74	0.111	0.691132701537128\\
74	0.1116	0.656946251633201\\
74	0.1122	0.622759801729273\\
74	0.1128	0.588573351825332\\
74	0.1134	0.55438690192139\\
74	0.114	0.520200452017463\\
74	0.1146	0.486014002113535\\
74	0.1152	0.451827552209608\\
74	0.1158	0.417641102305666\\
74	0.1164	0.383454652401738\\
74	0.117	0.349268202497811\\
74	0.1176	0.315081752593883\\
74	0.1182	0.280895302689942\\
74	0.1188	0.246708852786014\\
74	0.1194	0.212522402882087\\
74	0.12	0.178335952978159\\
74	0.1206	0.144149503074217\\
74	0.1212	0.10996305317029\\
74	0.1218	0.0757766032663483\\
74	0.1224	0.0415901533624208\\
74	0.123	0.00740370345847907\\
};
\end{axis}
\end{tikzpicture}%
	\caption{The  surfaces (\ref{tikz:theta_surf}) fitted to the internal parameter values estimated for the sets C1-C5 (\ref{tikz:thetas1}) are  used to compute the internal model parameters corresponding to the settings of choice (\ref{tikz:thetaidentified}).}\label{fig:surfaces_C15}
\end{figure}

\begin{figure}[!h]
	\centering
	\definecolor{mycolor1}{rgb}{0.00000,0.44700,0.74100}%
	\definecolor{mycolor2}{rgb}{0.85000,0.32500,0.09800}%
	% This file was created by matlab2tikz.
% Minimal pgfplots version: 1.3
%
\definecolor{mycolor1}{rgb}{0.85000,0.32500,0.09800}%
%
\begin{tikzpicture}

\begin{axis}[%
width=4.927496cm,
height=3.050847cm,
at={(0cm,8.474576cm)},
scale only axis,
xmin=57,
xmax=67,
tick align=outside,
xlabel={$L_{cut}$},
xmajorgrids,
ymin=0.193,
ymax=0.276,
ylabel={$D_{rlx}$},
ymajorgrids,
zmin=-25.0714095538756,
zmax=7.62200035910757,
zlabel={$x_1,x_1$},
zmajorgrids,
view={-140}{50},
legend style={at={(1.03,1)},anchor=north west,legend cell align=left,align=left,draw=white!15!black}
]
\addplot3[only marks,mark=*,mark options={},mark size=1.5000pt,color=mycolor1] plot table[row sep=crcr,]{%
67	0.276	4.36348336800989\\
66	0.255	2.85466243105591\\
62	0.209	2.01093650050702\\
57	0.193	2.30549632833024\\
};
\addplot3[only marks,mark=*,mark options={},mark size=1.5000pt,color=black] plot table[row sep=crcr,]{%
64	0.23	1.348985952742\\
};

\addplot3[%
surf,
opacity=0.7,
shader=interp,
colormap={mymap}{[1pt] rgb(0pt)=(0.0901961,0.239216,0.0745098); rgb(1pt)=(0.0945149,0.242058,0.0739522); rgb(2pt)=(0.0988592,0.244894,0.0733566); rgb(3pt)=(0.103229,0.247724,0.0727241); rgb(4pt)=(0.107623,0.250549,0.0720557); rgb(5pt)=(0.112043,0.253367,0.0713525); rgb(6pt)=(0.116487,0.25618,0.0706154); rgb(7pt)=(0.120956,0.258986,0.0698456); rgb(8pt)=(0.125449,0.261787,0.0690441); rgb(9pt)=(0.129967,0.264581,0.0682118); rgb(10pt)=(0.134508,0.26737,0.06735); rgb(11pt)=(0.139074,0.270152,0.0664596); rgb(12pt)=(0.143663,0.272929,0.0655416); rgb(13pt)=(0.148275,0.275699,0.0645971); rgb(14pt)=(0.152911,0.278463,0.0636271); rgb(15pt)=(0.15757,0.281221,0.0626328); rgb(16pt)=(0.162252,0.283973,0.0616151); rgb(17pt)=(0.166957,0.286719,0.060575); rgb(18pt)=(0.171685,0.289458,0.0595136); rgb(19pt)=(0.176434,0.292191,0.0584321); rgb(20pt)=(0.181207,0.294918,0.0573313); rgb(21pt)=(0.186001,0.297639,0.0562123); rgb(22pt)=(0.190817,0.300353,0.0550763); rgb(23pt)=(0.195655,0.303061,0.0539242); rgb(24pt)=(0.200514,0.305763,0.052757); rgb(25pt)=(0.205395,0.308459,0.0515759); rgb(26pt)=(0.210296,0.311149,0.0503624); rgb(27pt)=(0.215212,0.313846,0.0490067); rgb(28pt)=(0.220142,0.316548,0.0475043); rgb(29pt)=(0.22509,0.319254,0.0458704); rgb(30pt)=(0.230056,0.321962,0.0441205); rgb(31pt)=(0.235042,0.324671,0.04227); rgb(32pt)=(0.240048,0.327379,0.0403343); rgb(33pt)=(0.245078,0.330085,0.0383287); rgb(34pt)=(0.250131,0.332786,0.0362688); rgb(35pt)=(0.25521,0.335482,0.0341698); rgb(36pt)=(0.260317,0.33817,0.0320472); rgb(37pt)=(0.265451,0.340849,0.0299163); rgb(38pt)=(0.270616,0.343517,0.0277927); rgb(39pt)=(0.275813,0.346172,0.0256916); rgb(40pt)=(0.281043,0.348814,0.0236284); rgb(41pt)=(0.286307,0.35144,0.0216186); rgb(42pt)=(0.291607,0.354048,0.0196776); rgb(43pt)=(0.296945,0.356637,0.0178207); rgb(44pt)=(0.302322,0.359206,0.0160634); rgb(45pt)=(0.307739,0.361753,0.0144211); rgb(46pt)=(0.313198,0.364275,0.0129091); rgb(47pt)=(0.318701,0.366772,0.0115428); rgb(48pt)=(0.324249,0.369242,0.0103377); rgb(49pt)=(0.329843,0.371682,0.00930909); rgb(50pt)=(0.335485,0.374093,0.00847245); rgb(51pt)=(0.341176,0.376471,0.00784314); rgb(52pt)=(0.346925,0.378826,0.00732741); rgb(53pt)=(0.352735,0.381168,0.00682184); rgb(54pt)=(0.358605,0.383497,0.00632729); rgb(55pt)=(0.364532,0.385812,0.00584464); rgb(56pt)=(0.370516,0.388113,0.00537476); rgb(57pt)=(0.376552,0.390399,0.00491852); rgb(58pt)=(0.38264,0.39267,0.00447681); rgb(59pt)=(0.388777,0.394925,0.00405048); rgb(60pt)=(0.394962,0.397164,0.00364042); rgb(61pt)=(0.401191,0.399386,0.00324749); rgb(62pt)=(0.407464,0.401592,0.00287258); rgb(63pt)=(0.413777,0.40378,0.00251655); rgb(64pt)=(0.420129,0.40595,0.00218028); rgb(65pt)=(0.426518,0.408102,0.00186463); rgb(66pt)=(0.432942,0.410234,0.00157049); rgb(67pt)=(0.439399,0.412348,0.00129873); rgb(68pt)=(0.445885,0.414441,0.00105022); rgb(69pt)=(0.452401,0.416515,0.000825833); rgb(70pt)=(0.458942,0.418567,0.000626441); rgb(71pt)=(0.465508,0.420599,0.00045292); rgb(72pt)=(0.472096,0.422609,0.000306141); rgb(73pt)=(0.478704,0.424596,0.000186979); rgb(74pt)=(0.485331,0.426562,9.63073e-05); rgb(75pt)=(0.491973,0.428504,3.49981e-05); rgb(76pt)=(0.498628,0.430422,3.92506e-06); rgb(77pt)=(0.505323,0.432315,0); rgb(78pt)=(0.512206,0.434168,0); rgb(79pt)=(0.519282,0.435983,0); rgb(80pt)=(0.526529,0.437764,0); rgb(81pt)=(0.533922,0.439512,0); rgb(82pt)=(0.54144,0.441232,0); rgb(83pt)=(0.549059,0.442927,0); rgb(84pt)=(0.556756,0.444599,0); rgb(85pt)=(0.564508,0.446252,0); rgb(86pt)=(0.572292,0.447889,0); rgb(87pt)=(0.580084,0.449514,0); rgb(88pt)=(0.587863,0.451129,0); rgb(89pt)=(0.595604,0.452737,0); rgb(90pt)=(0.603284,0.454343,0); rgb(91pt)=(0.610882,0.455948,0); rgb(92pt)=(0.618373,0.457556,0); rgb(93pt)=(0.625734,0.459171,0); rgb(94pt)=(0.632943,0.460795,0); rgb(95pt)=(0.639976,0.462432,0); rgb(96pt)=(0.64681,0.464084,0); rgb(97pt)=(0.653423,0.465756,0); rgb(98pt)=(0.659791,0.46745,0); rgb(99pt)=(0.665891,0.469169,0); rgb(100pt)=(0.6717,0.470916,0); rgb(101pt)=(0.677195,0.472696,0); rgb(102pt)=(0.682353,0.47451,0); rgb(103pt)=(0.687242,0.476355,0); rgb(104pt)=(0.691952,0.478225,0); rgb(105pt)=(0.696497,0.480118,0); rgb(106pt)=(0.700887,0.482033,0); rgb(107pt)=(0.705134,0.483968,0); rgb(108pt)=(0.709251,0.485921,0); rgb(109pt)=(0.713249,0.487891,0); rgb(110pt)=(0.71714,0.489876,0); rgb(111pt)=(0.720936,0.491875,0); rgb(112pt)=(0.724649,0.493887,0); rgb(113pt)=(0.72829,0.495909,0); rgb(114pt)=(0.731872,0.49794,0); rgb(115pt)=(0.735406,0.499979,0); rgb(116pt)=(0.738904,0.502025,0); rgb(117pt)=(0.742378,0.504075,0); rgb(118pt)=(0.74584,0.506128,0); rgb(119pt)=(0.749302,0.508182,0); rgb(120pt)=(0.752775,0.510237,0); rgb(121pt)=(0.756272,0.51229,0); rgb(122pt)=(0.759804,0.514339,0); rgb(123pt)=(0.763384,0.516385,0); rgb(124pt)=(0.767022,0.518424,0); rgb(125pt)=(0.770731,0.520455,0); rgb(126pt)=(0.774523,0.522478,0); rgb(127pt)=(0.77841,0.524489,0); rgb(128pt)=(0.782391,0.526491,0); rgb(129pt)=(0.786402,0.528496,0); rgb(130pt)=(0.790431,0.530506,0); rgb(131pt)=(0.794478,0.532521,0); rgb(132pt)=(0.798541,0.534539,0); rgb(133pt)=(0.802619,0.53656,0); rgb(134pt)=(0.806712,0.538584,0); rgb(135pt)=(0.81082,0.540609,0); rgb(136pt)=(0.81494,0.542635,0); rgb(137pt)=(0.819074,0.54466,0); rgb(138pt)=(0.823219,0.546686,0); rgb(139pt)=(0.827374,0.548709,0); rgb(140pt)=(0.831541,0.55073,0); rgb(141pt)=(0.835716,0.552749,0); rgb(142pt)=(0.8399,0.554763,0); rgb(143pt)=(0.844092,0.556774,0); rgb(144pt)=(0.848292,0.558779,0); rgb(145pt)=(0.852497,0.560778,0); rgb(146pt)=(0.856708,0.562771,0); rgb(147pt)=(0.860924,0.564756,0); rgb(148pt)=(0.865143,0.566733,0); rgb(149pt)=(0.869366,0.568701,0); rgb(150pt)=(0.873592,0.57066,0); rgb(151pt)=(0.877819,0.572608,0); rgb(152pt)=(0.882047,0.574545,0); rgb(153pt)=(0.886275,0.576471,0); rgb(154pt)=(0.890659,0.578362,0); rgb(155pt)=(0.895333,0.580203,0); rgb(156pt)=(0.900258,0.581999,0); rgb(157pt)=(0.905397,0.583755,0); rgb(158pt)=(0.910711,0.585479,0); rgb(159pt)=(0.916164,0.587176,0); rgb(160pt)=(0.921717,0.588852,0); rgb(161pt)=(0.927333,0.590513,0); rgb(162pt)=(0.932974,0.592166,0); rgb(163pt)=(0.938602,0.593815,0); rgb(164pt)=(0.94418,0.595468,0); rgb(165pt)=(0.949669,0.59713,0); rgb(166pt)=(0.955033,0.598808,0); rgb(167pt)=(0.960233,0.600507,0); rgb(168pt)=(0.965232,0.602233,0); rgb(169pt)=(0.969992,0.603992,0); rgb(170pt)=(0.974475,0.605791,0); rgb(171pt)=(0.978643,0.607636,0); rgb(172pt)=(0.98246,0.609532,0); rgb(173pt)=(0.985886,0.611486,0); rgb(174pt)=(0.988885,0.613503,0); rgb(175pt)=(0.991419,0.61559,0); rgb(176pt)=(0.99345,0.617753,0); rgb(177pt)=(0.99494,0.619997,0); rgb(178pt)=(0.995851,0.622329,0); rgb(179pt)=(0.996226,0.624763,0); rgb(180pt)=(0.996512,0.627352,0); rgb(181pt)=(0.996788,0.630095,0); rgb(182pt)=(0.997053,0.632982,0); rgb(183pt)=(0.997308,0.636004,0); rgb(184pt)=(0.997552,0.639152,0); rgb(185pt)=(0.997785,0.642416,0); rgb(186pt)=(0.998006,0.645786,0); rgb(187pt)=(0.998217,0.649253,0); rgb(188pt)=(0.998416,0.652807,0); rgb(189pt)=(0.998605,0.656439,0); rgb(190pt)=(0.998781,0.660138,0); rgb(191pt)=(0.998946,0.663897,0); rgb(192pt)=(0.9991,0.667704,0); rgb(193pt)=(0.999242,0.67155,0); rgb(194pt)=(0.999372,0.675427,0); rgb(195pt)=(0.99949,0.679323,0); rgb(196pt)=(0.999596,0.68323,0); rgb(197pt)=(0.99969,0.687139,0); rgb(198pt)=(0.999771,0.691039,0); rgb(199pt)=(0.999841,0.694921,0); rgb(200pt)=(0.999898,0.698775,0); rgb(201pt)=(0.999942,0.702592,0); rgb(202pt)=(0.999974,0.706363,0); rgb(203pt)=(0.999994,0.710077,0); rgb(204pt)=(1,0.713725,0); rgb(205pt)=(1,0.717341,0); rgb(206pt)=(1,0.720963,0); rgb(207pt)=(1,0.724591,0); rgb(208pt)=(1,0.728226,0); rgb(209pt)=(1,0.731867,0); rgb(210pt)=(1,0.735514,0); rgb(211pt)=(1,0.739167,0); rgb(212pt)=(1,0.742827,0); rgb(213pt)=(1,0.746493,0); rgb(214pt)=(1,0.750165,0); rgb(215pt)=(1,0.753843,0); rgb(216pt)=(1,0.757527,0); rgb(217pt)=(1,0.761217,0); rgb(218pt)=(1,0.764913,0); rgb(219pt)=(1,0.768615,0); rgb(220pt)=(1,0.772324,0); rgb(221pt)=(1,0.776038,0); rgb(222pt)=(1,0.779758,0); rgb(223pt)=(1,0.783484,0); rgb(224pt)=(1,0.787215,0); rgb(225pt)=(1,0.790953,0); rgb(226pt)=(1,0.794696,0); rgb(227pt)=(1,0.798445,0); rgb(228pt)=(1,0.8022,0); rgb(229pt)=(1,0.805961,0); rgb(230pt)=(1,0.809727,0); rgb(231pt)=(1,0.8135,0); rgb(232pt)=(1,0.817278,0); rgb(233pt)=(1,0.821063,0); rgb(234pt)=(1,0.824854,0); rgb(235pt)=(1,0.828652,0); rgb(236pt)=(1,0.832455,0); rgb(237pt)=(1,0.836265,0); rgb(238pt)=(1,0.840081,0); rgb(239pt)=(1,0.843903,0); rgb(240pt)=(1,0.847732,0); rgb(241pt)=(1,0.851566,0); rgb(242pt)=(1,0.855406,0); rgb(243pt)=(1,0.859253,0); rgb(244pt)=(1,0.863106,0); rgb(245pt)=(1,0.866964,0); rgb(246pt)=(1,0.870829,0); rgb(247pt)=(1,0.8747,0); rgb(248pt)=(1,0.878577,0); rgb(249pt)=(1,0.88246,0); rgb(250pt)=(1,0.886349,0); rgb(251pt)=(1,0.890243,0); rgb(252pt)=(1,0.894144,0); rgb(253pt)=(1,0.898051,0); rgb(254pt)=(1,0.901964,0); rgb(255pt)=(1,0.905882,0)},
mesh/rows=49]
table[row sep=crcr,header=false] {%
%
57	0.193	2.30549632832259\\
57	0.19466	1.75795821067862\\
57	0.19632	1.21042009303466\\
57	0.19798	0.66288197539069\\
57	0.19964	0.115343857746723\\
57	0.2013	-0.432194259897244\\
57	0.20296	-0.979732377541211\\
57	0.20462	-1.52727049518512\\
57	0.20628	-2.07480861282914\\
57	0.20794	-2.62234673047311\\
57	0.2096	-3.16988484811702\\
57	0.21126	-3.71742296576105\\
57	0.21292	-4.26496108340496\\
57	0.21458	-4.81249920104892\\
57	0.21624	-5.36003731869295\\
57	0.2179	-5.90757543633686\\
57	0.21956	-6.45511355398082\\
57	0.22122	-7.00265167162485\\
57	0.22288	-7.55018978926876\\
57	0.22454	-8.09772790691272\\
57	0.2262	-8.64526602455669\\
57	0.22786	-9.19280414220066\\
57	0.22952	-9.74034225984462\\
57	0.23118	-10.2878803774886\\
57	0.23284	-10.8354184951326\\
57	0.2345	-11.3829566127765\\
57	0.23616	-11.9304947304205\\
57	0.23782	-12.4780328480645\\
57	0.23948	-13.0255709657084\\
57	0.24114	-13.5731090833524\\
57	0.2428	-14.1206472009964\\
57	0.24446	-14.6681853186403\\
57	0.24612	-15.2157234362843\\
57	0.24778	-15.7632615539283\\
57	0.24944	-16.3107996715722\\
57	0.2511	-16.8583377892161\\
57	0.25276	-17.4058759068601\\
57	0.25442	-17.9534140245041\\
57	0.25608	-18.500952142148\\
57	0.25774	-19.048490259792\\
57	0.2594	-19.596028377436\\
57	0.26106	-20.1435664950799\\
57	0.26272	-20.6911046127239\\
57	0.26438	-21.2386427303679\\
57	0.26604	-21.7861808480118\\
57	0.2677	-22.3337189656558\\
57	0.26936	-22.8812570832997\\
57	0.27102	-23.4287952009437\\
57	0.27268	-23.9763333185877\\
57	0.27434	-24.5238714362316\\
57	0.276	-25.0714095538756\\
57.2083333333333	0.193	2.41625682896392\\
57.2083333333333	0.19466	1.87876804002462\\
57.2083333333333	0.19632	1.34127925108527\\
57.2083333333333	0.19798	0.803790462145912\\
57.2083333333333	0.19964	0.266301673206613\\
57.2083333333333	0.2013	-0.271187115732744\\
57.2083333333333	0.20296	-0.8086759046721\\
57.2083333333333	0.20462	-1.3461646936114\\
57.2083333333333	0.20628	-1.88365348255081\\
57.2083333333333	0.20794	-2.42114227149011\\
57.2083333333333	0.2096	-2.95863106042941\\
57.2083333333333	0.21126	-3.49611984936877\\
57.2083333333333	0.21292	-4.03360863830812\\
57.2083333333333	0.21458	-4.57109742724742\\
57.2083333333333	0.21624	-5.10858621618684\\
57.2083333333333	0.2179	-5.64607500512614\\
57.2083333333333	0.21956	-6.18356379406544\\
57.2083333333333	0.22122	-6.72105258300485\\
57.2083333333333	0.22288	-7.25854137194415\\
57.2083333333333	0.22454	-7.79603016088345\\
57.2083333333333	0.2262	-8.33351894982286\\
57.2083333333333	0.22786	-8.87100773876216\\
57.2083333333333	0.22952	-9.40849652770146\\
57.2083333333333	0.23118	-9.94598531664087\\
57.2083333333333	0.23284	-10.4834741055802\\
57.2083333333333	0.2345	-11.0209628945195\\
57.2083333333333	0.23616	-11.5584516834589\\
57.2083333333333	0.23782	-12.0959404723982\\
57.2083333333333	0.23948	-12.6334292613375\\
57.2083333333333	0.24114	-13.1709180502769\\
57.2083333333333	0.2428	-13.7084068392162\\
57.2083333333333	0.24446	-14.2458956281555\\
57.2083333333333	0.24612	-14.7833844170949\\
57.2083333333333	0.24778	-15.3208732060342\\
57.2083333333333	0.24944	-15.8583619949736\\
57.2083333333333	0.2511	-16.3958507839129\\
57.2083333333333	0.25276	-16.9333395728522\\
57.2083333333333	0.25442	-17.4708283617916\\
57.2083333333333	0.25608	-18.0083171507309\\
57.2083333333333	0.25774	-18.5458059396702\\
57.2083333333333	0.2594	-19.0832947286096\\
57.2083333333333	0.26106	-19.6207835175489\\
57.2083333333333	0.26272	-20.1582723064882\\
57.2083333333333	0.26438	-20.6957610954277\\
57.2083333333333	0.26604	-21.2332498843669\\
57.2083333333333	0.2677	-21.7707386733063\\
57.2083333333333	0.26936	-22.3082274622456\\
57.2083333333333	0.27102	-22.8457162511849\\
57.2083333333333	0.27268	-23.3832050401243\\
57.2083333333333	0.27434	-23.9206938290636\\
57.2083333333333	0.276	-24.4581826180029\\
57.4166666666667	0.193	2.52701732960526\\
57.4166666666667	0.19466	1.99957786937057\\
57.4166666666667	0.19632	1.47213840913582\\
57.4166666666667	0.19798	0.944698948901134\\
57.4166666666667	0.19964	0.417259488666446\\
57.4166666666667	0.2013	-0.1101799715683\\
57.4166666666667	0.20296	-0.637619431802989\\
57.4166666666667	0.20462	-1.16505889203768\\
57.4166666666667	0.20628	-1.69249835227248\\
57.4166666666667	0.20794	-2.21993781250717\\
57.4166666666667	0.2096	-2.74737727274186\\
57.4166666666667	0.21126	-3.2748167329766\\
57.4166666666667	0.21292	-3.80225619321129\\
57.4166666666667	0.21458	-4.32969565344598\\
57.4166666666667	0.21624	-4.85713511368073\\
57.4166666666667	0.2179	-5.38457457391542\\
57.4166666666667	0.21956	-5.9120140341501\\
57.4166666666667	0.22122	-6.43945349438485\\
57.4166666666667	0.22288	-6.9668929546196\\
57.4166666666667	0.22454	-7.49433241485428\\
57.4166666666667	0.2262	-8.02177187508903\\
57.4166666666667	0.22786	-8.54921133532372\\
57.4166666666667	0.22952	-9.07665079555841\\
57.4166666666667	0.23118	-9.60409025579315\\
57.4166666666667	0.23284	-10.1315297160278\\
57.4166666666667	0.2345	-10.6589691762625\\
57.4166666666667	0.23616	-11.1864086364973\\
57.4166666666667	0.23782	-11.713848096732\\
57.4166666666667	0.23948	-12.2412875569667\\
57.4166666666667	0.24114	-12.7687270172015\\
57.4166666666667	0.2428	-13.2961664774361\\
57.4166666666667	0.24446	-13.8236059376708\\
57.4166666666667	0.24612	-14.3510453979056\\
57.4166666666667	0.24778	-14.8784848581403\\
57.4166666666667	0.24944	-15.405924318375\\
57.4166666666667	0.2511	-15.9333637786096\\
57.4166666666667	0.25276	-16.4608032388444\\
57.4166666666667	0.25442	-16.9882426990791\\
57.4166666666667	0.25608	-17.5156821593138\\
57.4166666666667	0.25774	-18.0431216195485\\
57.4166666666667	0.2594	-18.5705610797833\\
57.4166666666667	0.26106	-19.0980005400179\\
57.4166666666667	0.26272	-19.6254400002527\\
57.4166666666667	0.26438	-20.1528794604874\\
57.4166666666667	0.26604	-20.6803189207221\\
57.4166666666667	0.2677	-21.2077583809568\\
57.4166666666667	0.26936	-21.7351978411915\\
57.4166666666667	0.27102	-22.2626373014262\\
57.4166666666667	0.27268	-22.7900767616609\\
57.4166666666667	0.27434	-23.3175162218956\\
57.4166666666667	0.276	-23.8449556821303\\
57.625	0.193	2.63777783024665\\
57.625	0.19466	2.12038769871657\\
57.625	0.19632	1.60299756718643\\
57.625	0.19798	1.08560743565636\\
57.625	0.19964	0.568217304126279\\
57.625	0.2013	0.0508271725962004\\
57.625	0.20296	-0.466562958933878\\
57.625	0.20462	-0.983953090463956\\
57.625	0.20628	-1.50134322199409\\
57.625	0.20794	-2.01873335352417\\
57.625	0.2096	-2.53612348505425\\
57.625	0.21126	-3.05351361658433\\
57.625	0.21292	-3.5709037481144\\
57.625	0.21458	-4.08829387964448\\
57.625	0.21624	-4.60568401117462\\
57.625	0.2179	-5.1230741427047\\
57.625	0.21956	-5.64046427423472\\
57.625	0.22122	-6.15785440576485\\
57.625	0.22288	-6.67524453729493\\
57.625	0.22454	-7.19263466882501\\
57.625	0.2262	-7.71002480035514\\
57.625	0.22786	-8.22741493188522\\
57.625	0.22952	-8.74480506341524\\
57.625	0.23118	-9.26219519494538\\
57.625	0.23284	-9.77958532647546\\
57.625	0.2345	-10.2969754580055\\
57.625	0.23616	-10.8143655895357\\
57.625	0.23782	-11.3317557210657\\
57.625	0.23948	-11.8491458525958\\
57.625	0.24114	-12.3665359841259\\
57.625	0.2428	-12.883926115656\\
57.625	0.24446	-13.4013162471861\\
57.625	0.24612	-13.9187063787162\\
57.625	0.24778	-14.4360965102463\\
57.625	0.24944	-14.9534866417763\\
57.625	0.2511	-15.4708767733064\\
57.625	0.25276	-15.9882669048365\\
57.625	0.25442	-16.5056570363666\\
57.625	0.25608	-17.0230471678967\\
57.625	0.25774	-17.5404372994268\\
57.625	0.2594	-18.0578274309569\\
57.625	0.26106	-18.5752175624869\\
57.625	0.26272	-19.092607694017\\
57.625	0.26438	-19.6099978255472\\
57.625	0.26604	-20.1273879570772\\
57.625	0.2677	-20.6447780886073\\
57.625	0.26936	-21.1621682201373\\
57.625	0.27102	-21.6795583516674\\
57.625	0.27268	-22.1969484831976\\
57.625	0.27434	-22.7143386147276\\
57.625	0.276	-23.2317287462577\\
57.8333333333333	0.193	2.74853833088804\\
57.8333333333333	0.19466	2.24119752806257\\
57.8333333333333	0.19632	1.7338567252371\\
57.8333333333333	0.19798	1.22651592241164\\
57.8333333333333	0.19964	0.719175119586225\\
57.8333333333333	0.2013	0.211834316760701\\
57.8333333333333	0.20296	-0.29550648606471\\
57.8333333333333	0.20462	-0.802847288890177\\
57.8333333333333	0.20628	-1.31018809171564\\
57.8333333333333	0.20794	-1.81752889454111\\
57.8333333333333	0.2096	-2.32486969736658\\
57.8333333333333	0.21126	-2.83221050019205\\
57.8333333333333	0.21292	-3.33955130301752\\
57.8333333333333	0.21458	-3.84689210584293\\
57.8333333333333	0.21624	-4.35423290866845\\
57.8333333333333	0.2179	-4.86157371149386\\
57.8333333333333	0.21956	-5.36891451431933\\
57.8333333333333	0.22122	-5.8762553171448\\
57.8333333333333	0.22288	-6.38359611997026\\
57.8333333333333	0.22454	-6.89093692279573\\
57.8333333333333	0.2262	-7.3982777256212\\
57.8333333333333	0.22786	-7.90561852844667\\
57.8333333333333	0.22952	-8.41295933127208\\
57.8333333333333	0.23118	-8.9203001340976\\
57.8333333333333	0.23284	-9.42764093692301\\
57.8333333333333	0.2345	-9.93498173974848\\
57.8333333333333	0.23616	-10.4423225425739\\
57.8333333333333	0.23782	-10.9496633453994\\
57.8333333333333	0.23948	-11.4570041482249\\
57.8333333333333	0.24114	-11.9643449510504\\
57.8333333333333	0.2428	-12.4716857538758\\
57.8333333333333	0.24446	-12.9790265567012\\
57.8333333333333	0.24612	-13.4863673595268\\
57.8333333333333	0.24778	-13.9937081623522\\
57.8333333333333	0.24944	-14.5010489651776\\
57.8333333333333	0.2511	-15.008389768003\\
57.8333333333333	0.25276	-15.5157305708286\\
57.8333333333333	0.25442	-16.023071373654\\
57.8333333333333	0.25608	-16.5304121764794\\
57.8333333333333	0.25774	-17.037752979305\\
57.8333333333333	0.2594	-17.5450937821304\\
57.8333333333333	0.26106	-18.0524345849558\\
57.8333333333333	0.26272	-18.5597753877813\\
57.8333333333333	0.26438	-19.0671161906068\\
57.8333333333333	0.26604	-19.5744569934322\\
57.8333333333333	0.2677	-20.0817977962577\\
57.8333333333333	0.26936	-20.5891385990831\\
57.8333333333333	0.27102	-21.0964794019086\\
57.8333333333333	0.27268	-21.6038202047341\\
57.8333333333333	0.27434	-22.1111610075595\\
57.8333333333333	0.276	-22.618501810385\\
58.0416666666667	0.193	2.85929883152937\\
58.0416666666667	0.19466	2.36200735740857\\
58.0416666666667	0.19632	1.86471588328766\\
58.0416666666667	0.19798	1.36742440916686\\
58.0416666666667	0.19964	0.870132935046058\\
58.0416666666667	0.2013	0.372841460925144\\
58.0416666666667	0.20296	-0.124450013195656\\
58.0416666666667	0.20462	-0.621741487316456\\
58.0416666666667	0.20628	-1.11903296143737\\
58.0416666666667	0.20794	-1.61632443555817\\
58.0416666666667	0.2096	-2.11361590967897\\
58.0416666666667	0.21126	-2.61090738379983\\
58.0416666666667	0.21292	-3.10819885792068\\
58.0416666666667	0.21458	-3.60549033204148\\
58.0416666666667	0.21624	-4.10278180616234\\
58.0416666666667	0.2179	-4.6000732802832\\
58.0416666666667	0.21956	-5.097364754404\\
58.0416666666667	0.22122	-5.59465622852485\\
58.0416666666667	0.22288	-6.09194770264571\\
58.0416666666667	0.22454	-6.58923917676651\\
58.0416666666667	0.2262	-7.08653065088737\\
58.0416666666667	0.22786	-7.58382212500817\\
58.0416666666667	0.22952	-8.08111359912903\\
58.0416666666667	0.23118	-8.57840507324988\\
58.0416666666667	0.23284	-9.07569654737068\\
58.0416666666667	0.2345	-9.57298802149154\\
58.0416666666667	0.23616	-10.0702794956124\\
58.0416666666667	0.23782	-10.5675709697332\\
58.0416666666667	0.23948	-11.064862443854\\
58.0416666666667	0.24114	-11.5621539179749\\
58.0416666666667	0.2428	-12.0594453920957\\
58.0416666666667	0.24446	-12.5567368662165\\
58.0416666666667	0.24612	-13.0540283403374\\
58.0416666666667	0.24778	-13.5513198144582\\
58.0416666666667	0.24944	-14.048611288579\\
58.0416666666667	0.2511	-14.5459027626998\\
58.0416666666667	0.25276	-15.0431942368207\\
58.0416666666667	0.25442	-15.5404857109415\\
58.0416666666667	0.25608	-16.0377771850623\\
58.0416666666667	0.25774	-16.5350686591833\\
58.0416666666667	0.2594	-17.0323601333041\\
58.0416666666667	0.26106	-17.5296516074249\\
58.0416666666667	0.26272	-18.0269430815457\\
58.0416666666667	0.26438	-18.5242345556666\\
58.0416666666667	0.26604	-19.0215260297874\\
58.0416666666667	0.2677	-19.5188175039083\\
58.0416666666667	0.26936	-20.016108978029\\
58.0416666666667	0.27102	-20.5134004521499\\
58.0416666666667	0.27268	-21.0106919262707\\
58.0416666666667	0.27434	-21.5079834003915\\
58.0416666666667	0.276	-22.0052748745124\\
58.25	0.193	2.97005933217071\\
58.25	0.19466	2.48281718675452\\
58.25	0.19632	1.99557504133827\\
58.25	0.19798	1.50833289592208\\
58.25	0.19964	1.02109075050589\\
58.25	0.2013	0.533848605089645\\
58.25	0.20296	0.0466064596734554\\
58.25	0.20462	-0.440635685742734\\
58.25	0.20628	-0.92787783115898\\
58.25	0.20794	-1.41511997657517\\
58.25	0.2096	-1.90236212199136\\
58.25	0.21126	-2.38960426740761\\
58.25	0.21292	-2.8768464128238\\
58.25	0.21458	-3.36408855823998\\
58.25	0.21624	-3.85133070365623\\
58.25	0.2179	-4.33857284907242\\
58.25	0.21956	-4.82581499448861\\
58.25	0.22122	-5.31305713990486\\
58.25	0.22288	-5.80029928532105\\
58.25	0.22454	-6.28754143073724\\
58.25	0.2262	-6.77478357615348\\
58.25	0.22786	-7.26202572156967\\
58.25	0.22952	-7.74926786698586\\
58.25	0.23118	-8.23651001240211\\
58.25	0.23284	-8.7237521578183\\
58.25	0.2345	-9.21099430323449\\
58.25	0.23616	-9.69823644865073\\
58.25	0.23782	-10.185478594067\\
58.25	0.23948	-10.6727207394832\\
58.25	0.24114	-11.1599628848994\\
58.25	0.2428	-11.6472050303155\\
58.25	0.24446	-12.1344471757317\\
58.25	0.24612	-12.621689321148\\
58.25	0.24778	-13.1089314665642\\
58.25	0.24944	-13.5961736119804\\
58.25	0.2511	-14.0834157573966\\
58.25	0.25276	-14.5706579028128\\
58.25	0.25442	-15.057900048229\\
58.25	0.25608	-15.5451421936452\\
58.25	0.25774	-16.0323843390615\\
58.25	0.2594	-16.5196264844777\\
58.25	0.26106	-17.0068686298939\\
58.25	0.26272	-17.4941107753101\\
58.25	0.26438	-17.9813529207264\\
58.25	0.26604	-18.4685950661425\\
58.25	0.2677	-18.9558372115587\\
58.25	0.26936	-19.4430793569749\\
58.25	0.27102	-19.9303215023911\\
58.25	0.27268	-20.4175636478074\\
58.25	0.27434	-20.9048057932235\\
58.25	0.276	-21.3920479386397\\
58.4583333333333	0.193	3.0808198328121\\
58.4583333333333	0.19466	2.60362701610052\\
58.4583333333333	0.19632	2.12643419938888\\
58.4583333333333	0.19798	1.6492413826773\\
58.4583333333333	0.19964	1.17204856596578\\
58.4583333333333	0.2013	0.694855749254145\\
58.4583333333333	0.20296	0.217662932542567\\
58.4583333333333	0.20462	-0.259529884169012\\
58.4583333333333	0.20628	-0.736722700880591\\
58.4583333333333	0.20794	-1.21391551759217\\
58.4583333333333	0.2096	-1.69110833430375\\
58.4583333333333	0.21126	-2.16830115101538\\
58.4583333333333	0.21292	-2.64549396772691\\
58.4583333333333	0.21458	-3.12268678443849\\
58.4583333333333	0.21624	-3.59987960115012\\
58.4583333333333	0.2179	-4.0770724178617\\
58.4583333333333	0.21956	-4.55426523457322\\
58.4583333333333	0.22122	-5.03145805128486\\
58.4583333333333	0.22288	-5.50865086799644\\
58.4583333333333	0.22454	-5.98584368470802\\
58.4583333333333	0.2262	-6.46303650141959\\
58.4583333333333	0.22786	-6.94022931813117\\
58.4583333333333	0.22952	-7.41742213484275\\
58.4583333333333	0.23118	-7.89461495155439\\
58.4583333333333	0.23284	-8.37180776826591\\
58.4583333333333	0.2345	-8.84900058497749\\
58.4583333333333	0.23616	-9.32619340168912\\
58.4583333333333	0.23782	-9.8033862184007\\
58.4583333333333	0.23948	-10.2805790351123\\
58.4583333333333	0.24114	-10.7577718518239\\
58.4583333333333	0.2428	-11.2349646685354\\
58.4583333333333	0.24446	-11.712157485247\\
58.4583333333333	0.24612	-12.1893503019586\\
58.4583333333333	0.24778	-12.6665431186702\\
58.4583333333333	0.24944	-13.1437359353818\\
58.4583333333333	0.2511	-13.6209287520933\\
58.4583333333333	0.25276	-14.0981215688049\\
58.4583333333333	0.25442	-14.5753143855165\\
58.4583333333333	0.25608	-15.0525072022281\\
58.4583333333333	0.25774	-15.5297000189397\\
58.4583333333333	0.2594	-16.0068928356513\\
58.4583333333333	0.26106	-16.4840856523628\\
58.4583333333333	0.26272	-16.9612784690744\\
58.4583333333333	0.26438	-17.4384712857861\\
58.4583333333333	0.26604	-17.9156641024976\\
58.4583333333333	0.2677	-18.3928569192092\\
58.4583333333333	0.26936	-18.8700497359207\\
58.4583333333333	0.27102	-19.3472425526323\\
58.4583333333333	0.27268	-19.824435369344\\
58.4583333333333	0.27434	-20.3016281860554\\
58.4583333333333	0.276	-20.7788210027671\\
58.6666666666667	0.193	3.19158033345337\\
58.6666666666667	0.19466	2.72443684544646\\
58.6666666666667	0.19632	2.25729335743949\\
58.6666666666667	0.19798	1.79014986943253\\
58.6666666666667	0.19964	1.32300638142556\\
58.6666666666667	0.2013	0.855862893418589\\
58.6666666666667	0.20296	0.388719405411678\\
58.6666666666667	0.20462	-0.0784240825952907\\
58.6666666666667	0.20628	-0.545567570602316\\
58.6666666666667	0.20794	-1.01271105860923\\
58.6666666666667	0.2096	-1.47985454661614\\
58.6666666666667	0.21126	-1.94699803462316\\
58.6666666666667	0.21292	-2.41414152263008\\
58.6666666666667	0.21458	-2.88128501063704\\
58.6666666666667	0.21624	-3.34842849864401\\
58.6666666666667	0.2179	-3.81557198665098\\
58.6666666666667	0.21956	-4.28271547465789\\
58.6666666666667	0.22122	-4.74985896266492\\
58.6666666666667	0.22288	-5.21700245067188\\
58.6666666666667	0.22454	-5.6841459386788\\
58.6666666666667	0.2262	-6.15128942668576\\
58.6666666666667	0.22786	-6.61843291469273\\
58.6666666666667	0.22952	-7.08557640269964\\
58.6666666666667	0.23118	-7.55271989070667\\
58.6666666666667	0.23284	-8.01986337871358\\
58.6666666666667	0.2345	-8.48700686672055\\
58.6666666666667	0.23616	-8.95415035472752\\
58.6666666666667	0.23782	-9.42129384273449\\
58.6666666666667	0.23948	-9.8884373307414\\
58.6666666666667	0.24114	-10.3555808187484\\
58.6666666666667	0.2428	-10.8227243067553\\
58.6666666666667	0.24446	-11.2898677947623\\
58.6666666666667	0.24612	-11.7570112827693\\
58.6666666666667	0.24778	-12.2241547707762\\
58.6666666666667	0.24944	-12.6912982587831\\
58.6666666666667	0.2511	-13.1584417467901\\
58.6666666666667	0.25276	-13.6255852347971\\
58.6666666666667	0.25442	-14.0927287228041\\
58.6666666666667	0.25608	-14.559872210811\\
58.6666666666667	0.25774	-15.027015698818\\
58.6666666666667	0.2594	-15.494159186825\\
58.6666666666667	0.26106	-15.9613026748319\\
58.6666666666667	0.26272	-16.4284461628388\\
58.6666666666667	0.26438	-16.8955896508459\\
58.6666666666667	0.26604	-17.3627331388528\\
58.6666666666667	0.2677	-17.8298766268597\\
58.6666666666667	0.26936	-18.2970201148666\\
58.6666666666667	0.27102	-18.7641636028736\\
58.6666666666667	0.27268	-19.2313070908806\\
58.6666666666667	0.27434	-19.6984505788875\\
58.6666666666667	0.276	-20.1655940668945\\
58.875	0.193	3.30234083409482\\
58.875	0.19466	2.84524667479252\\
58.875	0.19632	2.38815251549011\\
58.875	0.19798	1.9310583561878\\
58.875	0.19964	1.4739641968855\\
58.875	0.2013	1.01687003758315\\
58.875	0.20296	0.559775878280846\\
58.875	0.20462	0.102681718978488\\
58.875	0.20628	-0.35441244032387\\
58.875	0.20794	-0.811506599626171\\
58.875	0.2096	-1.26860075892847\\
58.875	0.21126	-1.72569491823089\\
58.875	0.21292	-2.18278907753319\\
58.875	0.21458	-2.63988323683549\\
58.875	0.21624	-3.09697739613785\\
58.875	0.2179	-3.5540715554402\\
58.875	0.21956	-4.0111657147425\\
58.875	0.22122	-4.46825987404486\\
58.875	0.22288	-4.92535403334716\\
58.875	0.22454	-5.38244819264946\\
58.875	0.2262	-5.83954235195188\\
58.875	0.22786	-6.29663651125418\\
58.875	0.22952	-6.75373067055648\\
58.875	0.23118	-7.21082482985884\\
58.875	0.23284	-7.66791898916119\\
58.875	0.2345	-8.12501314846349\\
58.875	0.23616	-8.58210730776585\\
58.875	0.23782	-9.03920146706815\\
58.875	0.23948	-9.49629562637045\\
58.875	0.24114	-9.95338978567287\\
58.875	0.2428	-10.4104839449752\\
58.875	0.24446	-10.8675781042775\\
58.875	0.24612	-11.3246722635798\\
58.875	0.24778	-11.7817664228822\\
58.875	0.24944	-12.2388605821845\\
58.875	0.2511	-12.6959547414868\\
58.875	0.25276	-13.1530489007891\\
58.875	0.25442	-13.6101430600915\\
58.875	0.25608	-14.0672372193938\\
58.875	0.25774	-14.5243313786962\\
58.875	0.2594	-14.9814255379985\\
58.875	0.26106	-15.4385196973008\\
58.875	0.26272	-15.8956138566031\\
58.875	0.26438	-16.3527080159055\\
58.875	0.26604	-16.8098021752078\\
58.875	0.2677	-17.2668963345101\\
58.875	0.26936	-17.7239904938124\\
58.875	0.27102	-18.1810846531148\\
58.875	0.27268	-18.6381788124172\\
58.875	0.27434	-19.0952729717194\\
58.875	0.276	-19.5523671310218\\
59.0833333333333	0.193	3.4131013347361\\
59.0833333333333	0.19466	2.96605650413841\\
59.0833333333333	0.19632	2.51901167354072\\
59.0833333333333	0.19798	2.07196684294297\\
59.0833333333333	0.19964	1.62492201234534\\
59.0833333333333	0.2013	1.17787718174759\\
59.0833333333333	0.20296	0.7308323511499\\
59.0833333333333	0.20462	0.28378752055221\\
59.0833333333333	0.20628	-0.163257310045537\\
59.0833333333333	0.20794	-0.610302140643228\\
59.0833333333333	0.2096	-1.05734697124092\\
59.0833333333333	0.21126	-1.50439180183866\\
59.0833333333333	0.21292	-1.95143663243636\\
59.0833333333333	0.21458	-2.39848146303405\\
59.0833333333333	0.21624	-2.84552629363179\\
59.0833333333333	0.2179	-3.29257112422948\\
59.0833333333333	0.21956	-3.73961595482717\\
59.0833333333333	0.22122	-4.18666078542492\\
59.0833333333333	0.22288	-4.63370561602261\\
59.0833333333333	0.22454	-5.0807504466203\\
59.0833333333333	0.2262	-5.52779527721805\\
59.0833333333333	0.22786	-5.97484010781574\\
59.0833333333333	0.22952	-6.42188493841343\\
59.0833333333333	0.23118	-6.86892976901117\\
59.0833333333333	0.23284	-7.31597459960886\\
59.0833333333333	0.2345	-7.76301943020655\\
59.0833333333333	0.23616	-8.21006426080424\\
59.0833333333333	0.23782	-8.65710909140199\\
59.0833333333333	0.23948	-9.10415392199963\\
59.0833333333333	0.24114	-9.55119875259737\\
59.0833333333333	0.2428	-9.99824358319506\\
59.0833333333333	0.24446	-10.4452884137928\\
59.0833333333333	0.24612	-10.8923332443905\\
59.0833333333333	0.24778	-11.3393780749882\\
59.0833333333333	0.24944	-11.7864229055859\\
59.0833333333333	0.2511	-12.2334677361836\\
59.0833333333333	0.25276	-12.6805125667813\\
59.0833333333333	0.25442	-13.127557397379\\
59.0833333333333	0.25608	-13.5746022279767\\
59.0833333333333	0.25774	-14.0216470585744\\
59.0833333333333	0.2594	-14.4686918891722\\
59.0833333333333	0.26106	-14.9157367197698\\
59.0833333333333	0.26272	-15.3627815503676\\
59.0833333333333	0.26438	-15.8098263809653\\
59.0833333333333	0.26604	-16.256871211563\\
59.0833333333333	0.2677	-16.7039160421607\\
59.0833333333333	0.26936	-17.1509608727583\\
59.0833333333333	0.27102	-17.5980057033561\\
59.0833333333333	0.27268	-18.0450505339538\\
59.0833333333333	0.27434	-18.4920953645515\\
59.0833333333333	0.276	-18.9391401951492\\
59.2916666666667	0.193	3.52386183537743\\
59.2916666666667	0.19466	3.08686633348441\\
59.2916666666667	0.19632	2.64987083159127\\
59.2916666666667	0.19798	2.21287532969819\\
59.2916666666667	0.19964	1.77587982780511\\
59.2916666666667	0.2013	1.33888432591203\\
59.2916666666667	0.20296	0.901888824018954\\
59.2916666666667	0.20462	0.464893322125874\\
59.2916666666667	0.20628	0.0278978202327949\\
59.2916666666667	0.20794	-0.409097681660285\\
59.2916666666667	0.2096	-0.846093183553364\\
59.2916666666667	0.21126	-1.28308868544644\\
59.2916666666667	0.21292	-1.72008418733952\\
59.2916666666667	0.21458	-2.1570796892326\\
59.2916666666667	0.21624	-2.59407519112574\\
59.2916666666667	0.2179	-3.03107069301876\\
59.2916666666667	0.21956	-3.46806619491184\\
59.2916666666667	0.22122	-3.90506169680498\\
59.2916666666667	0.22288	-4.342057198698\\
59.2916666666667	0.22454	-4.77905270059108\\
59.2916666666667	0.2262	-5.21604820248422\\
59.2916666666667	0.22786	-5.65304370437724\\
59.2916666666667	0.22952	-6.09003920627032\\
59.2916666666667	0.23118	-6.52703470816346\\
59.2916666666667	0.23284	-6.96403021005653\\
59.2916666666667	0.2345	-7.40102571194956\\
59.2916666666667	0.23616	-7.83802121384269\\
59.2916666666667	0.23782	-8.27501671573577\\
59.2916666666667	0.23948	-8.7120122176288\\
59.2916666666667	0.24114	-9.14900771952193\\
59.2916666666667	0.2428	-9.58600322141501\\
59.2916666666667	0.24446	-10.022998723308\\
59.2916666666667	0.24612	-10.4599942252012\\
59.2916666666667	0.24778	-10.8969897270943\\
59.2916666666667	0.24944	-11.3339852289873\\
59.2916666666667	0.2511	-11.7709807308804\\
59.2916666666667	0.25276	-12.2079762327735\\
59.2916666666667	0.25442	-12.6449717346666\\
59.2916666666667	0.25608	-13.0819672365596\\
59.2916666666667	0.25774	-13.5189627384527\\
59.2916666666667	0.2594	-13.9559582403459\\
59.2916666666667	0.26106	-14.3929537422388\\
59.2916666666667	0.26272	-14.829949244132\\
59.2916666666667	0.26438	-15.2669447460251\\
59.2916666666667	0.26604	-15.7039402479181\\
59.2916666666667	0.2677	-16.1409357498112\\
59.2916666666667	0.26936	-16.5779312517042\\
59.2916666666667	0.27102	-17.0149267535974\\
59.2916666666667	0.27268	-17.4519222554904\\
59.2916666666667	0.27434	-17.8889177573835\\
59.2916666666667	0.276	-18.3259132592766\\
59.5	0.193	3.63462233601888\\
59.5	0.19466	3.20767616283041\\
59.5	0.19632	2.78072998964194\\
59.5	0.19798	2.35378381645347\\
59.5	0.19964	1.92683764326506\\
59.5	0.2013	1.49989147007653\\
59.5	0.20296	1.07294529688812\\
59.5	0.20462	0.64599912369971\\
59.5	0.20628	0.219052950511184\\
59.5	0.20794	-0.207893222677228\\
59.5	0.2096	-0.634839395865697\\
59.5	0.21126	-1.06178556905417\\
59.5	0.21292	-1.48873174224263\\
59.5	0.21458	-1.91567791543105\\
59.5	0.21624	-2.34262408861952\\
59.5	0.2179	-2.76957026180798\\
59.5	0.21956	-3.1965164349964\\
59.5	0.22122	-3.62346260818492\\
59.5	0.22288	-4.05040878137333\\
59.5	0.22454	-4.4773549545618\\
59.5	0.2262	-4.90430112775027\\
59.5	0.22786	-5.33124730093868\\
59.5	0.22952	-5.75819347412715\\
59.5	0.23118	-6.18513964731562\\
59.5	0.23284	-6.61208582050409\\
59.5	0.2345	-7.0390319936925\\
59.5	0.23616	-7.46597816688103\\
59.5	0.23782	-7.89292434006944\\
59.5	0.23948	-8.31987051325785\\
59.5	0.24114	-8.74681668644638\\
59.5	0.2428	-9.17376285963479\\
59.5	0.24446	-9.60070903282326\\
59.5	0.24612	-10.0276552060117\\
59.5	0.24778	-10.4546013792002\\
59.5	0.24944	-10.8815475523886\\
59.5	0.2511	-11.308493725577\\
59.5	0.25276	-11.7354398987655\\
59.5	0.25442	-12.162386071954\\
59.5	0.25608	-12.5893322451424\\
59.5	0.25774	-13.0162784183309\\
59.5	0.2594	-13.4432245915194\\
59.5	0.26106	-13.8701707647078\\
59.5	0.26272	-14.2971169378962\\
59.5	0.26438	-14.7240631110848\\
59.5	0.26604	-15.1510092842731\\
59.5	0.2677	-15.5779554574617\\
59.5	0.26936	-16.00490163065\\
59.5	0.27102	-16.4318478038385\\
59.5	0.27268	-16.858793977027\\
59.5	0.27434	-17.2857401502154\\
59.5	0.276	-17.7126863234039\\
59.7083333333333	0.193	3.74538283666016\\
59.7083333333333	0.19466	3.32848599217635\\
59.7083333333333	0.19632	2.9115891476925\\
59.7083333333333	0.19798	2.49469230320869\\
59.7083333333333	0.19964	2.07779545872484\\
59.7083333333333	0.2013	1.66089861424098\\
59.7083333333333	0.20296	1.24400176975718\\
59.7083333333333	0.20462	0.827104925273375\\
59.7083333333333	0.20628	0.410208080789516\\
59.7083333333333	0.20794	-0.00668876369428517\\
59.7083333333333	0.2096	-0.423585608178144\\
59.7083333333333	0.21126	-0.840482452662002\\
59.7083333333333	0.21292	-1.2573792971458\\
59.7083333333333	0.21458	-1.6742761416296\\
59.7083333333333	0.21624	-2.09117298611346\\
59.7083333333333	0.2179	-2.50806983059726\\
59.7083333333333	0.21956	-2.92496667508107\\
59.7083333333333	0.22122	-3.34186351956498\\
59.7083333333333	0.22288	-3.75876036404878\\
59.7083333333333	0.22454	-4.17565720853258\\
59.7083333333333	0.2262	-4.59255405301644\\
59.7083333333333	0.22786	-5.00945089750024\\
59.7083333333333	0.22952	-5.42634774198405\\
59.7083333333333	0.23118	-5.84324458646796\\
59.7083333333333	0.23284	-6.26014143095176\\
59.7083333333333	0.2345	-6.67703827543556\\
59.7083333333333	0.23616	-7.09393511991942\\
59.7083333333333	0.23782	-7.51083196440322\\
59.7083333333333	0.23948	-7.92772880888703\\
59.7083333333333	0.24114	-8.34462565337094\\
59.7083333333333	0.2428	-8.76152249785474\\
59.7083333333333	0.24446	-9.17841934233854\\
59.7083333333333	0.24612	-9.5953161868224\\
59.7083333333333	0.24778	-10.0122130313062\\
59.7083333333333	0.24944	-10.42910987579\\
59.7083333333333	0.2511	-10.8460067202738\\
59.7083333333333	0.25276	-11.2629035647577\\
59.7083333333333	0.25442	-11.6798004092415\\
59.7083333333333	0.25608	-12.0966972537253\\
59.7083333333333	0.25774	-12.5135940982092\\
59.7083333333333	0.2594	-12.930490942693\\
59.7083333333333	0.26106	-13.3473877871768\\
59.7083333333333	0.26272	-13.7642846316607\\
59.7083333333333	0.26438	-14.1811814761446\\
59.7083333333333	0.26604	-14.5980783206283\\
59.7083333333333	0.2677	-15.0149751651122\\
59.7083333333333	0.26936	-15.4318720095959\\
59.7083333333333	0.27102	-15.8487688540798\\
59.7083333333333	0.27268	-16.2656656985637\\
59.7083333333333	0.27434	-16.6825625430474\\
59.7083333333333	0.276	-17.0994593875313\\
59.9166666666667	0.193	3.85614333730155\\
59.9166666666667	0.19466	3.44929582152236\\
59.9166666666667	0.19632	3.04244830574311\\
59.9166666666667	0.19798	2.63560078996392\\
59.9166666666667	0.19964	2.22875327418473\\
59.9166666666667	0.2013	1.82190575840548\\
59.9166666666667	0.20296	1.41505824262629\\
59.9166666666667	0.20462	1.0082107268471\\
59.9166666666667	0.20628	0.601363211067849\\
59.9166666666667	0.20794	0.194515695288715\\
59.9166666666667	0.2096	-0.212331820490476\\
59.9166666666667	0.21126	-0.619179336269724\\
59.9166666666667	0.21292	-1.02602685204891\\
59.9166666666667	0.21458	-1.43287436782811\\
59.9166666666667	0.21624	-1.83972188360735\\
59.9166666666667	0.2179	-2.24656939938654\\
59.9166666666667	0.21956	-2.65341691516574\\
59.9166666666667	0.22122	-3.06026443094498\\
59.9166666666667	0.22288	-3.46711194672417\\
59.9166666666667	0.22454	-3.87395946250331\\
59.9166666666667	0.2262	-4.28080697828256\\
59.9166666666667	0.22786	-4.68765449406175\\
59.9166666666667	0.22952	-5.09450200984094\\
59.9166666666667	0.23118	-5.50134952562018\\
59.9166666666667	0.23284	-5.90819704139938\\
59.9166666666667	0.2345	-6.31504455717857\\
59.9166666666667	0.23616	-6.72189207295781\\
59.9166666666667	0.23782	-7.12873958873701\\
59.9166666666667	0.23948	-7.53558710451614\\
59.9166666666667	0.24114	-7.94243462029539\\
59.9166666666667	0.2428	-8.34928213607458\\
59.9166666666667	0.24446	-8.75612965185377\\
59.9166666666667	0.24612	-9.16297716763302\\
59.9166666666667	0.24778	-9.56982468341221\\
59.9166666666667	0.24944	-9.9766721991914\\
59.9166666666667	0.2511	-10.3835197149706\\
59.9166666666667	0.25276	-10.7903672307498\\
59.9166666666667	0.25442	-11.197214746529\\
59.9166666666667	0.25608	-11.6040622623082\\
59.9166666666667	0.25774	-12.0109097780874\\
59.9166666666667	0.2594	-12.4177572938667\\
59.9166666666667	0.26106	-12.8246048096458\\
59.9166666666667	0.26272	-13.231452325425\\
59.9166666666667	0.26438	-13.6382998412043\\
59.9166666666667	0.26604	-14.0451473569834\\
59.9166666666667	0.2677	-14.4519948727627\\
59.9166666666667	0.26936	-14.8588423885418\\
59.9166666666667	0.27102	-15.265689904321\\
59.9166666666667	0.27268	-15.6725374201002\\
59.9166666666667	0.27434	-16.0793849358794\\
59.9166666666667	0.276	-16.4862324516586\\
60.125	0.193	3.96690383794288\\
60.125	0.19466	3.57010565086836\\
60.125	0.19632	3.17330746379372\\
60.125	0.19798	2.77650927671914\\
60.125	0.19964	2.37971108964462\\
60.125	0.2013	1.98291290256998\\
60.125	0.20296	1.5861147154954\\
60.125	0.20462	1.18931652842087\\
60.125	0.20628	0.792518341346238\\
60.125	0.20794	0.395720154271714\\
60.125	0.2096	-0.00107803280286589\\
60.125	0.21126	-0.397876219877503\\
60.125	0.21292	-0.794674406952026\\
60.125	0.21458	-1.19147259402661\\
60.125	0.21624	-1.58827078110124\\
60.125	0.2179	-1.98506896817577\\
60.125	0.21956	-2.38186715525035\\
60.125	0.22122	-2.77866534232493\\
60.125	0.22288	-3.17546352939951\\
60.125	0.22454	-3.57226171647409\\
60.125	0.2262	-3.96905990354867\\
60.125	0.22786	-4.36585809062325\\
60.125	0.22952	-4.76265627769783\\
60.125	0.23118	-5.15945446477241\\
60.125	0.23284	-5.55625265184699\\
60.125	0.2345	-5.95305083892157\\
60.125	0.23616	-6.34984902599615\\
60.125	0.23782	-6.74664721307073\\
60.125	0.23948	-7.14344540014525\\
60.125	0.24114	-7.54024358721989\\
60.125	0.2428	-7.93704177429447\\
60.125	0.24446	-8.33383996136899\\
60.125	0.24612	-8.73063814844363\\
60.125	0.24778	-9.12743633551821\\
60.125	0.24944	-9.52423452259274\\
60.125	0.2511	-9.92103270966732\\
60.125	0.25276	-10.3178308967419\\
60.125	0.25442	-10.7146290838165\\
60.125	0.25608	-11.1114272708911\\
60.125	0.25774	-11.5082254579656\\
60.125	0.2594	-11.9050236450403\\
60.125	0.26106	-12.3018218321148\\
60.125	0.26272	-12.6986200191894\\
60.125	0.26438	-13.095418206264\\
60.125	0.26604	-13.4922163933385\\
60.125	0.2677	-13.8890145804131\\
60.125	0.26936	-14.2858127674876\\
60.125	0.27102	-14.6826109545622\\
60.125	0.27268	-15.0794091416369\\
60.125	0.27434	-15.4762073287114\\
60.125	0.276	-15.873005515786\\
60.3333333333333	0.193	4.07766433858427\\
60.3333333333333	0.19466	3.6909154802143\\
60.3333333333333	0.19632	3.30416662184433\\
60.3333333333333	0.19798	2.91741776347442\\
60.3333333333333	0.19964	2.53066890510445\\
60.3333333333333	0.2013	2.14392004673448\\
60.3333333333333	0.20296	1.75717118836457\\
60.3333333333333	0.20462	1.3704223299946\\
60.3333333333333	0.20628	0.983673471624627\\
60.3333333333333	0.20794	0.596924613254657\\
60.3333333333333	0.2096	0.210175754884744\\
60.3333333333333	0.21126	-0.176573103485225\\
60.3333333333333	0.21292	-0.563321961855195\\
60.3333333333333	0.21458	-0.950070820225108\\
60.3333333333333	0.21624	-1.33681967859508\\
60.3333333333333	0.2179	-1.72356853696505\\
60.3333333333333	0.21956	-2.11031739533496\\
60.3333333333333	0.22122	-2.49706625370493\\
60.3333333333333	0.22288	-2.8838151120749\\
60.3333333333333	0.22454	-3.27056397044481\\
60.3333333333333	0.2262	-3.65731282881478\\
60.3333333333333	0.22786	-4.04406168718475\\
60.3333333333333	0.22952	-4.43081054555466\\
60.3333333333333	0.23118	-4.81755940392469\\
60.3333333333333	0.23284	-5.2043082622946\\
60.3333333333333	0.2345	-5.59105712066452\\
60.3333333333333	0.23616	-5.97780597903454\\
60.3333333333333	0.23782	-6.36455483740446\\
60.3333333333333	0.23948	-6.75130369577437\\
60.3333333333333	0.24114	-7.13805255414439\\
60.3333333333333	0.2428	-7.52480141251431\\
60.3333333333333	0.24446	-7.91155027088422\\
60.3333333333333	0.24612	-8.29829912925425\\
60.3333333333333	0.24778	-8.68504798762416\\
60.3333333333333	0.24944	-9.07179684599413\\
60.3333333333333	0.2511	-9.45854570436404\\
60.3333333333333	0.25276	-9.84529456273401\\
60.3333333333333	0.25442	-10.232043421104\\
60.3333333333333	0.25608	-10.6187922794739\\
60.3333333333333	0.25774	-11.0055411378439\\
60.3333333333333	0.2594	-11.3922899962139\\
60.3333333333333	0.26106	-11.7790388545837\\
60.3333333333333	0.26272	-12.1657877129537\\
60.3333333333333	0.26438	-12.5525365713237\\
60.3333333333333	0.26604	-12.9392854296936\\
60.3333333333333	0.2677	-13.3260342880636\\
60.3333333333333	0.26936	-13.7127831464335\\
60.3333333333333	0.27102	-14.0995320048035\\
60.3333333333333	0.27268	-14.4862808631735\\
60.3333333333333	0.27434	-14.8730297215433\\
60.3333333333333	0.276	-15.2597785799133\\
60.5416666666667	0.193	4.1884248392256\\
60.5416666666667	0.19466	3.8117253095603\\
60.5416666666667	0.19632	3.43502577989494\\
60.5416666666667	0.19798	3.05832625022964\\
60.5416666666667	0.19964	2.68162672056434\\
60.5416666666667	0.2013	2.30492719089898\\
60.5416666666667	0.20296	1.92822766123368\\
60.5416666666667	0.20462	1.55152813156838\\
60.5416666666667	0.20628	1.17482860190296\\
60.5416666666667	0.20794	0.798129072237657\\
60.5416666666667	0.2096	0.421429542572355\\
60.5416666666667	0.21126	0.0447300129069959\\
60.5416666666667	0.21292	-0.331969516758306\\
60.5416666666667	0.21458	-0.708669046423609\\
60.5416666666667	0.21624	-1.08536857608897\\
60.5416666666667	0.2179	-1.46206810575427\\
60.5416666666667	0.21956	-1.83876763541957\\
60.5416666666667	0.22122	-2.21546716508493\\
60.5416666666667	0.22288	-2.59216669475023\\
60.5416666666667	0.22454	-2.96886622441559\\
60.5416666666667	0.2262	-3.34556575408095\\
60.5416666666667	0.22786	-3.72226528374625\\
60.5416666666667	0.22952	-4.09896481341156\\
60.5416666666667	0.23118	-4.47566434307691\\
60.5416666666667	0.23284	-4.85236387274222\\
60.5416666666667	0.2345	-5.22906340240752\\
60.5416666666667	0.23616	-5.60576293207288\\
60.5416666666667	0.23782	-5.98246246173818\\
60.5416666666667	0.23948	-6.35916199140348\\
60.5416666666667	0.24114	-6.7358615210689\\
60.5416666666667	0.2428	-7.1125610507342\\
60.5416666666667	0.24446	-7.4892605803995\\
60.5416666666667	0.24612	-7.86596011006486\\
60.5416666666667	0.24778	-8.24265963973016\\
60.5416666666667	0.24944	-8.61935916939547\\
60.5416666666667	0.2511	-8.99605869906077\\
60.5416666666667	0.25276	-9.37275822872613\\
60.5416666666667	0.25442	-9.74945775839143\\
60.5416666666667	0.25608	-10.1261572880567\\
60.5416666666667	0.25774	-10.5028568177221\\
60.5416666666667	0.2594	-10.8795563473875\\
60.5416666666667	0.26106	-11.2562558770528\\
60.5416666666667	0.26272	-11.6329554067181\\
60.5416666666667	0.26438	-12.0096549363835\\
60.5416666666667	0.26604	-12.3863544660487\\
60.5416666666667	0.2677	-12.7630539957141\\
60.5416666666667	0.26936	-13.1397535253793\\
60.5416666666667	0.27102	-13.5164530550447\\
60.5416666666667	0.27268	-13.89315258471\\
60.5416666666667	0.27434	-14.2698521143753\\
60.5416666666667	0.276	-14.6465516440407\\
60.75	0.193	4.29918533986694\\
60.75	0.19466	3.93253513890625\\
60.75	0.19632	3.5658849379455\\
60.75	0.19798	3.19923473698481\\
60.75	0.19964	2.83258453602417\\
60.75	0.2013	2.46593433506342\\
60.75	0.20296	2.09928413410273\\
60.75	0.20462	1.73263393314204\\
60.75	0.20628	1.36598373218129\\
60.75	0.20794	0.9993335312206\\
60.75	0.2096	0.632683330259965\\
60.75	0.21126	0.266033129299217\\
60.75	0.21292	-0.100617071661475\\
60.75	0.21458	-0.467267272622166\\
60.75	0.21624	-0.833917473582915\\
60.75	0.2179	-1.20056767454361\\
60.75	0.21956	-1.56721787550424\\
60.75	0.22122	-1.93386807646499\\
60.75	0.22288	-2.30051827742568\\
60.75	0.22454	-2.66716847838637\\
60.75	0.2262	-3.03381867934712\\
60.75	0.22786	-3.40046888030781\\
60.75	0.22952	-3.76711908126845\\
60.75	0.23118	-4.1337692822292\\
60.75	0.23284	-4.50041948318989\\
60.75	0.2345	-4.86706968415058\\
60.75	0.23616	-5.23371988511133\\
60.75	0.23782	-5.60037008607202\\
60.75	0.23948	-5.96702028703265\\
60.75	0.24114	-6.3336704879934\\
60.75	0.2428	-6.70032068895409\\
60.75	0.24446	-7.06697088991478\\
60.75	0.24612	-7.43362109087553\\
60.75	0.24778	-7.80027129183622\\
60.75	0.24944	-8.16692149279686\\
60.75	0.2511	-8.53357169375755\\
60.75	0.25276	-8.9002218947183\\
60.75	0.25442	-9.26687209567899\\
60.75	0.25608	-9.63352229663968\\
60.75	0.25774	-10.0001724976004\\
60.75	0.2594	-10.3668226985611\\
60.75	0.26106	-10.7334728995218\\
60.75	0.26272	-11.1001231004825\\
60.75	0.26438	-11.4667733014433\\
60.75	0.26604	-11.8334235024039\\
60.75	0.2677	-12.2000737033646\\
60.75	0.26936	-12.5667239043252\\
60.75	0.27102	-12.933374105286\\
60.75	0.27268	-13.3000243062467\\
60.75	0.27434	-13.6666745072073\\
60.75	0.276	-14.0333247081681\\
60.9583333333333	0.193	4.40994584050827\\
60.9583333333333	0.19466	4.05334496825225\\
60.9583333333333	0.19632	3.69674409599611\\
60.9583333333333	0.19798	3.34014322374009\\
60.9583333333333	0.19964	2.983542351484\\
60.9583333333333	0.2013	2.62694147922792\\
60.9583333333333	0.20296	2.27034060697184\\
60.9583333333333	0.20462	1.91373973471582\\
60.9583333333333	0.20628	1.55713886245968\\
60.9583333333333	0.20794	1.2005379902036\\
60.9583333333333	0.2096	0.843937117947576\\
60.9583333333333	0.21126	0.487336245691438\\
60.9583333333333	0.21292	0.130735373435414\\
60.9583333333333	0.21458	-0.225865498820667\\
60.9583333333333	0.21624	-0.582466371076748\\
60.9583333333333	0.2179	-0.939067243332829\\
60.9583333333333	0.21956	-1.29566811558891\\
60.9583333333333	0.22122	-1.65226898784499\\
60.9583333333333	0.22288	-2.00886986010107\\
60.9583333333333	0.22454	-2.3654707323571\\
60.9583333333333	0.2262	-2.72207160461323\\
60.9583333333333	0.22786	-3.07867247686926\\
60.9583333333333	0.22952	-3.43527334912534\\
60.9583333333333	0.23118	-3.79187422138148\\
60.9583333333333	0.23284	-4.1484750936375\\
60.9583333333333	0.2345	-4.50507596589358\\
60.9583333333333	0.23616	-4.86167683814966\\
60.9583333333333	0.23782	-5.21827771040574\\
60.9583333333333	0.23948	-5.57487858266177\\
60.9583333333333	0.24114	-5.93147945491791\\
60.9583333333333	0.2428	-6.28808032717399\\
60.9583333333333	0.24446	-6.64468119943001\\
60.9583333333333	0.24612	-7.00128207168615\\
60.9583333333333	0.24778	-7.35788294394217\\
60.9583333333333	0.24944	-7.71448381619825\\
60.9583333333333	0.2511	-8.07108468845428\\
60.9583333333333	0.25276	-8.42768556071042\\
60.9583333333333	0.25442	-8.7842864329665\\
60.9583333333333	0.25608	-9.14088730522252\\
60.9583333333333	0.25774	-9.49748817747866\\
60.9583333333333	0.2594	-9.85408904973474\\
60.9583333333333	0.26106	-10.2106899219908\\
60.9583333333333	0.26272	-10.5672907942468\\
60.9583333333333	0.26438	-10.923891666503\\
60.9583333333333	0.26604	-11.280492538759\\
60.9583333333333	0.2677	-11.6370934110151\\
60.9583333333333	0.26936	-11.9936942832711\\
60.9583333333333	0.27102	-12.3502951555272\\
60.9583333333333	0.27268	-12.7068960277833\\
60.9583333333333	0.27434	-13.0634969000393\\
60.9583333333333	0.276	-13.4200977722954\\
61.1666666666667	0.193	4.52070634114966\\
61.1666666666667	0.19466	4.17415479759825\\
61.1666666666667	0.19632	3.82760325404672\\
61.1666666666667	0.19798	3.48105171049531\\
61.1666666666667	0.19964	3.13450016694389\\
61.1666666666667	0.2013	2.78794862339242\\
61.1666666666667	0.20296	2.44139707984095\\
61.1666666666667	0.20462	2.09484553628954\\
61.1666666666667	0.20628	1.74829399273807\\
61.1666666666667	0.20794	1.4017424491866\\
61.1666666666667	0.2096	1.05519090563519\\
61.1666666666667	0.21126	0.708639362083716\\
61.1666666666667	0.21292	0.362087818532245\\
61.1666666666667	0.21458	0.0155362749808319\\
61.1666666666667	0.21624	-0.331015268570638\\
61.1666666666667	0.2179	-0.677566812122052\\
61.1666666666667	0.21956	-1.02411835567352\\
61.1666666666667	0.22122	-1.37066989922499\\
61.1666666666667	0.22288	-1.71722144277641\\
61.1666666666667	0.22454	-2.06377298632788\\
61.1666666666667	0.2262	-2.41032452987935\\
61.1666666666667	0.22786	-2.75687607343076\\
61.1666666666667	0.22952	-3.10342761698223\\
61.1666666666667	0.23118	-3.4499791605337\\
61.1666666666667	0.23284	-3.79653070408511\\
61.1666666666667	0.2345	-4.14308224763653\\
61.1666666666667	0.23616	-4.48963379118806\\
61.1666666666667	0.23782	-4.83618533473947\\
61.1666666666667	0.23948	-5.18273687829088\\
61.1666666666667	0.24114	-5.52928842184241\\
61.1666666666667	0.2428	-5.87583996539382\\
61.1666666666667	0.24446	-6.22239150894524\\
61.1666666666667	0.24612	-6.56894305249676\\
61.1666666666667	0.24778	-6.91549459604818\\
61.1666666666667	0.24944	-7.26204613959959\\
61.1666666666667	0.2511	-7.608597683151\\
61.1666666666667	0.25276	-7.95514922670253\\
61.1666666666667	0.25442	-8.30170077025394\\
61.1666666666667	0.25608	-8.64825231380536\\
61.1666666666667	0.25774	-8.99480385735689\\
61.1666666666667	0.2594	-9.34135540090836\\
61.1666666666667	0.26106	-9.68790694445971\\
61.1666666666667	0.26272	-10.0344584880112\\
61.1666666666667	0.26438	-10.3810100315627\\
61.1666666666667	0.26604	-10.7275615751141\\
61.1666666666667	0.2677	-11.0741131186656\\
61.1666666666667	0.26936	-11.420664662217\\
61.1666666666667	0.27102	-11.7672162057684\\
61.1666666666667	0.27268	-12.1137677493199\\
61.1666666666667	0.27434	-12.4603192928713\\
61.1666666666667	0.276	-12.8068708364228\\
61.375	0.193	4.63146684179105\\
61.375	0.19466	4.29496462694425\\
61.375	0.19632	3.95846241209739\\
61.375	0.19798	3.62196019725053\\
61.375	0.19964	3.28545798240373\\
61.375	0.2013	2.94895576755687\\
61.375	0.20296	2.61245355271006\\
61.375	0.20462	2.27595133786326\\
61.375	0.20628	1.9394491230164\\
61.375	0.20794	1.6029469081696\\
61.375	0.2096	1.2664446933228\\
61.375	0.21126	0.929942478475937\\
61.375	0.21292	0.593440263629134\\
61.375	0.21458	0.256938048782331\\
61.375	0.21624	-0.0795641660645288\\
61.375	0.2179	-0.416066380911332\\
61.375	0.21956	-0.752568595758135\\
61.375	0.22122	-1.08907081060499\\
61.375	0.22288	-1.4255730254518\\
61.375	0.22454	-1.7620752402986\\
61.375	0.2262	-2.09857745514546\\
61.375	0.22786	-2.43507966999226\\
61.375	0.22952	-2.77158188483907\\
61.375	0.23118	-3.10808409968593\\
61.375	0.23284	-3.44458631453273\\
61.375	0.2345	-3.78108852937953\\
61.375	0.23616	-4.11759074422639\\
61.375	0.23782	-4.45409295907319\\
61.375	0.23948	-4.79059517392\\
61.375	0.24114	-5.12709738876686\\
61.375	0.2428	-5.46359960361366\\
61.375	0.24446	-5.80010181846046\\
61.375	0.24612	-6.13660403330732\\
61.375	0.24778	-6.47310624815418\\
61.375	0.24944	-6.80960846300098\\
61.375	0.2511	-7.14611067784773\\
61.375	0.25276	-7.48261289269459\\
61.375	0.25442	-7.81911510754145\\
61.375	0.25608	-8.15561732238825\\
61.375	0.25774	-8.49211953723511\\
61.375	0.2594	-8.82862175208197\\
61.375	0.26106	-9.16512396692872\\
61.375	0.26272	-9.50162618177558\\
61.375	0.26438	-9.83812839662244\\
61.375	0.26604	-10.1746306114692\\
61.375	0.2677	-10.511132826316\\
61.375	0.26936	-10.8476350411628\\
61.375	0.27102	-11.1841372560096\\
61.375	0.27268	-11.5206394708565\\
61.375	0.27434	-11.8571416857033\\
61.375	0.276	-12.1936439005501\\
61.5833333333333	0.193	4.74222734243239\\
61.5833333333333	0.19466	4.41577445629019\\
61.5833333333333	0.19632	4.08932157014794\\
61.5833333333333	0.19798	3.76286868400581\\
61.5833333333333	0.19964	3.43641579786362\\
61.5833333333333	0.2013	3.10996291172137\\
61.5833333333333	0.20296	2.78351002557918\\
61.5833333333333	0.20462	2.45705713943704\\
61.5833333333333	0.20628	2.13060425329479\\
61.5833333333333	0.20794	1.8041513671526\\
61.5833333333333	0.2096	1.47769848101041\\
61.5833333333333	0.21126	1.15124559486821\\
61.5833333333333	0.21292	0.824792708726022\\
61.5833333333333	0.21458	0.49833982258383\\
61.5833333333333	0.21624	0.171886936441581\\
61.5833333333333	0.2179	-0.154565949700611\\
61.5833333333333	0.21956	-0.481018835842747\\
61.5833333333333	0.22122	-0.807471721984996\\
61.5833333333333	0.22288	-1.13392460812719\\
61.5833333333333	0.22454	-1.46037749426938\\
61.5833333333333	0.2262	-1.78683038041157\\
61.5833333333333	0.22786	-2.11328326655376\\
61.5833333333333	0.22952	-2.43973615269596\\
61.5833333333333	0.23118	-2.76618903883821\\
61.5833333333333	0.23284	-3.09264192498034\\
61.5833333333333	0.2345	-3.41909481112253\\
61.5833333333333	0.23616	-3.74554769726478\\
61.5833333333333	0.23782	-4.07200058340698\\
61.5833333333333	0.23948	-4.39845346954911\\
61.5833333333333	0.24114	-4.72490635569136\\
61.5833333333333	0.2428	-5.05135924183355\\
61.5833333333333	0.24446	-5.37781212797574\\
61.5833333333333	0.24612	-5.70426501411794\\
61.5833333333333	0.24778	-6.03071790026013\\
61.5833333333333	0.24944	-6.35717078640232\\
61.5833333333333	0.2511	-6.68362367254451\\
61.5833333333333	0.25276	-7.01007655868671\\
61.5833333333333	0.25442	-7.3365294448289\\
61.5833333333333	0.25608	-7.66298233097109\\
61.5833333333333	0.25774	-7.98943521711334\\
61.5833333333333	0.2594	-8.31588810325559\\
61.5833333333333	0.26106	-8.64234098939767\\
61.5833333333333	0.26272	-8.96879387553992\\
61.5833333333333	0.26438	-9.29524676168217\\
61.5833333333333	0.26604	-9.6216996478243\\
61.5833333333333	0.2677	-9.94815253396649\\
61.5833333333333	0.26936	-10.2746054201086\\
61.5833333333333	0.27102	-10.6010583062509\\
61.5833333333333	0.27268	-10.9275111923931\\
61.5833333333333	0.27434	-11.2539640785352\\
61.5833333333333	0.276	-11.5804169646775\\
61.7916666666667	0.193	4.85298784307372\\
61.7916666666667	0.19466	4.53658428563614\\
61.7916666666667	0.19632	4.22018072819856\\
61.7916666666667	0.19798	3.90377717076097\\
61.7916666666667	0.19964	3.58737361332345\\
61.7916666666667	0.2013	3.27097005588581\\
61.7916666666667	0.20296	2.95456649844829\\
61.7916666666667	0.20462	2.63816294101071\\
61.7916666666667	0.20628	2.32175938357312\\
61.7916666666667	0.20794	2.00535582613554\\
61.7916666666667	0.2096	1.68895226869802\\
61.7916666666667	0.21126	1.37254871126038\\
61.7916666666667	0.21292	1.05614515382285\\
61.7916666666667	0.21458	0.739741596385272\\
61.7916666666667	0.21624	0.423338038947691\\
61.7916666666667	0.2179	0.106934481510109\\
61.7916666666667	0.21956	-0.209469075927416\\
61.7916666666667	0.22122	-0.525872633365054\\
61.7916666666667	0.22288	-0.842276190802579\\
61.7916666666667	0.22454	-1.15867974824016\\
61.7916666666667	0.2262	-1.47508330567774\\
61.7916666666667	0.22786	-1.79148686311532\\
61.7916666666667	0.22952	-2.10789042055285\\
61.7916666666667	0.23118	-2.42429397799049\\
61.7916666666667	0.23284	-2.74069753542801\\
61.7916666666667	0.2345	-3.05710109286559\\
61.7916666666667	0.23616	-3.37350465030318\\
61.7916666666667	0.23782	-3.68990820774076\\
61.7916666666667	0.23948	-4.00631176517828\\
61.7916666666667	0.24114	-4.32271532261592\\
61.7916666666667	0.2428	-4.63911888005345\\
61.7916666666667	0.24446	-4.95552243749103\\
61.7916666666667	0.24612	-5.27192599492861\\
61.7916666666667	0.24778	-5.58832955236619\\
61.7916666666667	0.24944	-5.90473310980371\\
61.7916666666667	0.2511	-6.2211366672413\\
61.7916666666667	0.25276	-6.53754022467888\\
61.7916666666667	0.25442	-6.85394378211646\\
61.7916666666667	0.25608	-7.17034733955398\\
61.7916666666667	0.25774	-7.48675089699162\\
61.7916666666667	0.2594	-7.8031544544292\\
61.7916666666667	0.26106	-8.11955801186673\\
61.7916666666667	0.26272	-8.43596156930431\\
61.7916666666667	0.26438	-8.75236512674195\\
61.7916666666667	0.26604	-9.06876868417947\\
61.7916666666667	0.2677	-9.38517224161706\\
61.7916666666667	0.26936	-9.70157579905458\\
61.7916666666667	0.27102	-10.0179793564922\\
61.7916666666667	0.27268	-10.3343829139297\\
61.7916666666667	0.27434	-10.6507864713673\\
61.7916666666667	0.276	-10.9671900288049\\
62	0.193	4.96374834371505\\
62	0.19466	4.65739411498214\\
62	0.19632	4.35103988624917\\
62	0.19798	4.04468565751625\\
62	0.19964	3.73833142878328\\
62	0.2013	3.43197720005031\\
62	0.20296	3.1256229713174\\
62	0.20462	2.81926874258448\\
62	0.20628	2.51291451385146\\
62	0.20794	2.20656028511854\\
62	0.2096	1.90020605638563\\
62	0.21126	1.59385182765266\\
62	0.21292	1.28749759891969\\
62	0.21458	0.981143370186771\\
62	0.21624	0.6747891414538\\
62	0.2179	0.368434912720886\\
62	0.21956	0.062080683987972\\
62	0.22122	-0.244273544745056\\
62	0.22288	-0.55062777347797\\
62	0.22454	-0.856982002210884\\
62	0.2262	-1.16333623094386\\
62	0.22786	-1.46969045967683\\
62	0.22952	-1.77604468840974\\
62	0.23118	-2.08239891714271\\
62	0.23284	-2.38875314587563\\
62	0.2345	-2.6951073746086\\
62	0.23616	-3.00146160334157\\
62	0.23782	-3.30781583207448\\
62	0.23948	-3.6141700608074\\
62	0.24114	-3.92052428954042\\
62	0.2428	-4.22687851827334\\
62	0.24446	-4.53323274700625\\
62	0.24612	-4.83958697573922\\
62	0.24778	-5.14594120447219\\
62	0.24944	-5.45229543320511\\
62	0.2511	-5.75864966193802\\
62	0.25276	-6.06500389067099\\
62	0.25442	-6.37135811940396\\
62	0.25608	-6.67771234813688\\
62	0.25774	-6.98406657686985\\
62	0.2594	-7.29042080560282\\
62	0.26106	-7.59677503433568\\
62	0.26272	-7.90312926306871\\
62	0.26438	-8.20948349180168\\
62	0.26604	-8.51583772053453\\
62	0.2677	-8.82219194926756\\
62	0.26936	-9.12854617800042\\
62	0.27102	-9.43490040673339\\
62	0.27268	-9.74125463546636\\
62	0.27434	-10.0476088641992\\
62	0.276	-10.3539630929322\\
62.2083333333333	0.193	5.07450884435644\\
62.2083333333333	0.19466	4.77820394432814\\
62.2083333333333	0.19632	4.48189904429978\\
62.2083333333333	0.19798	4.18559414427148\\
62.2083333333333	0.19964	3.88928924424317\\
62.2083333333333	0.2013	3.59298434421481\\
62.2083333333333	0.20296	3.29667944418651\\
62.2083333333333	0.20462	3.00037454415821\\
62.2083333333333	0.20628	2.70406964412985\\
62.2083333333333	0.20794	2.40776474410154\\
62.2083333333333	0.2096	2.11145984407324\\
62.2083333333333	0.21126	1.81515494404488\\
62.2083333333333	0.21292	1.51885004401657\\
62.2083333333333	0.21458	1.22254514398827\\
62.2083333333333	0.21624	0.92624024395991\\
62.2083333333333	0.2179	0.629935343931606\\
62.2083333333333	0.21956	0.333630443903303\\
62.2083333333333	0.22122	0.0373255438749425\\
62.2083333333333	0.22288	-0.258979356153361\\
62.2083333333333	0.22454	-0.555284256181665\\
62.2083333333333	0.2262	-0.851589156210025\\
62.2083333333333	0.22786	-1.14789405623833\\
62.2083333333333	0.22952	-1.44419895626663\\
62.2083333333333	0.23118	-1.74050385629499\\
62.2083333333333	0.23284	-2.0368087563233\\
62.2083333333333	0.2345	-2.33311365635154\\
62.2083333333333	0.23616	-2.6294185563799\\
62.2083333333333	0.23782	-2.92572345640821\\
62.2083333333333	0.23948	-3.22202835643651\\
62.2083333333333	0.24114	-3.51833325646487\\
62.2083333333333	0.2428	-3.81463815649317\\
62.2083333333333	0.24446	-4.11094305652148\\
62.2083333333333	0.24612	-4.40724795654984\\
62.2083333333333	0.24778	-4.70355285657814\\
62.2083333333333	0.24944	-4.99985775660645\\
62.2083333333333	0.2511	-5.29616265663475\\
62.2083333333333	0.25276	-5.59246755666311\\
62.2083333333333	0.25442	-5.88877245669141\\
62.2083333333333	0.25608	-6.18507735671972\\
62.2083333333333	0.25774	-6.48138225674808\\
62.2083333333333	0.2594	-6.77768715677644\\
62.2083333333333	0.26106	-7.07399205680468\\
62.2083333333333	0.26272	-7.37029695683304\\
62.2083333333333	0.26438	-7.6666018568614\\
62.2083333333333	0.26604	-7.96290675688965\\
62.2083333333333	0.2677	-8.25921165691801\\
62.2083333333333	0.26936	-8.55551655694626\\
62.2083333333333	0.27102	-8.85182145697462\\
62.2083333333333	0.27268	-9.14812635700298\\
62.2083333333333	0.27434	-9.44443125703123\\
62.2083333333333	0.276	-9.74073615705959\\
62.4166666666667	0.193	5.18526934499772\\
62.4166666666667	0.19466	4.89901377367408\\
62.4166666666667	0.19632	4.61275820235034\\
62.4166666666667	0.19798	4.32650263102664\\
62.4166666666667	0.19964	4.04024705970301\\
62.4166666666667	0.2013	3.75399148837926\\
62.4166666666667	0.20296	3.46773591705556\\
62.4166666666667	0.20462	3.18148034573193\\
62.4166666666667	0.20628	2.89522477440818\\
62.4166666666667	0.20794	2.60896920308448\\
62.4166666666667	0.2096	2.32271363176079\\
62.4166666666667	0.21126	2.0364580604371\\
62.4166666666667	0.21292	1.75020248911341\\
62.4166666666667	0.21458	1.46394691778971\\
62.4166666666667	0.21624	1.17769134646602\\
62.4166666666667	0.2179	0.891435775142327\\
62.4166666666667	0.21956	0.605180203818634\\
62.4166666666667	0.22122	0.318924632494884\\
62.4166666666667	0.22288	0.0326690611712479\\
62.4166666666667	0.22454	-0.253586510152445\\
62.4166666666667	0.2262	-0.539842081476195\\
62.4166666666667	0.22786	-0.826097652799831\\
62.4166666666667	0.22952	-1.11235322412352\\
62.4166666666667	0.23118	-1.39860879544727\\
62.4166666666667	0.23284	-1.68486436677097\\
62.4166666666667	0.2345	-1.9711199380946\\
62.4166666666667	0.23616	-2.25737550941835\\
62.4166666666667	0.23782	-2.54363108074205\\
62.4166666666667	0.23948	-2.82988665206568\\
62.4166666666667	0.24114	-3.11614222338943\\
62.4166666666667	0.2428	-3.40239779471312\\
62.4166666666667	0.24446	-3.68865336603676\\
62.4166666666667	0.24612	-3.97490893736051\\
62.4166666666667	0.24778	-4.2611645086842\\
62.4166666666667	0.24944	-4.5474200800079\\
62.4166666666667	0.2511	-4.83367565133153\\
62.4166666666667	0.25276	-5.11993122265528\\
62.4166666666667	0.25442	-5.40618679397897\\
62.4166666666667	0.25608	-5.69244236530261\\
62.4166666666667	0.25774	-5.97869793662636\\
62.4166666666667	0.2594	-6.26495350795011\\
62.4166666666667	0.26106	-6.55120907927369\\
62.4166666666667	0.26272	-6.83746465059744\\
62.4166666666667	0.26438	-7.12372022192119\\
62.4166666666667	0.26604	-7.40997579324483\\
62.4166666666667	0.2677	-7.69623136456852\\
62.4166666666667	0.26936	-7.98248693589215\\
62.4166666666667	0.27102	-8.2687425072159\\
62.4166666666667	0.27268	-8.55499807853965\\
62.4166666666667	0.27434	-8.84125364986323\\
62.4166666666667	0.276	-9.12750922118698\\
62.625	0.193	5.29602984563917\\
62.625	0.19466	5.01982360302009\\
62.625	0.19632	4.743617360401\\
62.625	0.19798	4.46741111778192\\
62.625	0.19964	4.1912048751629\\
62.625	0.2013	3.91499863254381\\
62.625	0.20296	3.63879238992473\\
62.625	0.20462	3.36258614730571\\
62.625	0.20628	3.08637990468657\\
62.625	0.20794	2.81017366206754\\
62.625	0.2096	2.53396741944846\\
62.625	0.21126	2.25776117682938\\
62.625	0.21292	1.98155493421035\\
62.625	0.21458	1.70534869159127\\
62.625	0.21624	1.42914244897219\\
62.625	0.2179	1.1529362063531\\
62.625	0.21956	0.876729963734078\\
62.625	0.22122	0.600523721114939\\
62.625	0.22288	0.324317478495914\\
62.625	0.22454	0.0481112358768314\\
62.625	0.2262	-0.228095006742251\\
62.625	0.22786	-0.504301249361276\\
62.625	0.22952	-0.780507491980359\\
62.625	0.23118	-1.05671373459944\\
62.625	0.23284	-1.33291997721852\\
62.625	0.2345	-1.60912621983755\\
62.625	0.23616	-1.88533246245663\\
62.625	0.23782	-2.16153870507571\\
62.625	0.23948	-2.43774494769474\\
62.625	0.24114	-2.71395119031388\\
62.625	0.2428	-2.9901574329329\\
62.625	0.24446	-3.26636367555199\\
62.625	0.24612	-3.54256991817107\\
62.625	0.24778	-3.81877616079015\\
62.625	0.24944	-4.09498240340918\\
62.625	0.2511	-4.3711886460282\\
62.625	0.25276	-4.64739488864734\\
62.625	0.25442	-4.92360113126637\\
62.625	0.25608	-5.19980737388545\\
62.625	0.25774	-5.47601361650453\\
62.625	0.2594	-5.75221985912367\\
62.625	0.26106	-6.02842610174264\\
62.625	0.26272	-6.30463234436172\\
62.625	0.26438	-6.58083858698086\\
62.625	0.26604	-6.85704482959983\\
62.625	0.2677	-7.13325107221897\\
62.625	0.26936	-7.40945731483794\\
62.625	0.27102	-7.68566355745708\\
62.625	0.27268	-7.96186980007616\\
62.625	0.27434	-8.23807604269518\\
62.625	0.276	-8.51428228531427\\
62.8333333333333	0.193	5.4067903462805\\
62.8333333333333	0.19466	5.14063343236609\\
62.8333333333333	0.19632	4.87447651845162\\
62.8333333333333	0.19798	4.6083196045372\\
62.8333333333333	0.19964	4.34216269062279\\
62.8333333333333	0.2013	4.07600577670826\\
62.8333333333333	0.20296	3.80984886279384\\
62.8333333333333	0.20462	3.54369194887943\\
62.8333333333333	0.20628	3.27753503496496\\
62.8333333333333	0.20794	3.01137812105054\\
62.8333333333333	0.2096	2.74522120713613\\
62.8333333333333	0.21126	2.4790642932216\\
62.8333333333333	0.21292	2.21290737930718\\
62.8333333333333	0.21458	1.94675046539277\\
62.8333333333333	0.21624	1.6805935514783\\
62.8333333333333	0.2179	1.41443663756388\\
62.8333333333333	0.21956	1.14827972364947\\
62.8333333333333	0.22122	0.882122809734994\\
62.8333333333333	0.22288	0.615965895820523\\
62.8333333333333	0.22454	0.349808981906108\\
62.8333333333333	0.2262	0.0836520679916362\\
62.8333333333333	0.22786	-0.182504845922779\\
62.8333333333333	0.22952	-0.448661759837194\\
62.8333333333333	0.23118	-0.714818673751722\\
62.8333333333333	0.23284	-0.980975587666137\\
62.8333333333333	0.2345	-1.24713250158055\\
62.8333333333333	0.23616	-1.51328941549502\\
62.8333333333333	0.23782	-1.77944632940944\\
62.8333333333333	0.23948	-2.04560324332385\\
62.8333333333333	0.24114	-2.31176015723838\\
62.8333333333333	0.2428	-2.5779170711528\\
62.8333333333333	0.24446	-2.84407398506721\\
62.8333333333333	0.24612	-3.11023089898168\\
62.8333333333333	0.24778	-3.3763878128961\\
62.8333333333333	0.24944	-3.64254472681051\\
62.8333333333333	0.2511	-3.90870164072493\\
62.8333333333333	0.25276	-4.17485855463946\\
62.8333333333333	0.25442	-4.44101546855387\\
62.8333333333333	0.25608	-4.70717238246829\\
62.8333333333333	0.25774	-4.97332929638276\\
62.8333333333333	0.2594	-5.23948621029723\\
62.8333333333333	0.26106	-5.50564312421164\\
62.8333333333333	0.26272	-5.77180003812612\\
62.8333333333333	0.26438	-6.03795695204059\\
62.8333333333333	0.26604	-6.30411386595495\\
62.8333333333333	0.2677	-6.57027077986942\\
62.8333333333333	0.26936	-6.83642769378378\\
62.8333333333333	0.27102	-7.1025846076983\\
62.8333333333333	0.27268	-7.36874152161278\\
62.8333333333333	0.27434	-7.63489843552713\\
62.8333333333333	0.276	-7.90105534944161\\
63.0416666666667	0.193	5.51755084692184\\
63.0416666666667	0.19466	5.26144326171203\\
63.0416666666667	0.19632	5.00533567650217\\
63.0416666666667	0.19798	4.74922809129237\\
63.0416666666667	0.19964	4.49312050608256\\
63.0416666666667	0.2013	4.23701292087276\\
63.0416666666667	0.20296	3.98090533566295\\
63.0416666666667	0.20462	3.72479775045315\\
63.0416666666667	0.20628	3.46869016524329\\
63.0416666666667	0.20794	3.21258258003348\\
63.0416666666667	0.2096	2.95647499482368\\
63.0416666666667	0.21126	2.70036740961382\\
63.0416666666667	0.21292	2.44425982440401\\
63.0416666666667	0.21458	2.18815223919421\\
63.0416666666667	0.21624	1.93204465398435\\
63.0416666666667	0.2179	1.67593706877454\\
63.0416666666667	0.21956	1.4198294835648\\
63.0416666666667	0.22122	1.16372189835494\\
63.0416666666667	0.22288	0.907614313145132\\
63.0416666666667	0.22454	0.651506727935327\\
63.0416666666667	0.2262	0.395399142725466\\
63.0416666666667	0.22786	0.139291557515662\\
63.0416666666667	0.22952	-0.116816027694142\\
63.0416666666667	0.23118	-0.372923612904003\\
63.0416666666667	0.23284	-0.629031198113807\\
63.0416666666667	0.2345	-0.885138783323612\\
63.0416666666667	0.23616	-1.14124636853342\\
63.0416666666667	0.23782	-1.39735395374322\\
63.0416666666667	0.23948	-1.65346153895302\\
63.0416666666667	0.24114	-1.90956912416289\\
63.0416666666667	0.2428	-2.16567670937269\\
63.0416666666667	0.24446	-2.42178429458249\\
63.0416666666667	0.24612	-2.67789187979236\\
63.0416666666667	0.24778	-2.93399946500216\\
63.0416666666667	0.24944	-3.19010705021196\\
63.0416666666667	0.2511	-3.44621463542177\\
63.0416666666667	0.25276	-3.70232222063157\\
63.0416666666667	0.25442	-3.95842980584143\\
63.0416666666667	0.25608	-4.21453739105118\\
63.0416666666667	0.25774	-4.47064497626104\\
63.0416666666667	0.2594	-4.7267525614709\\
63.0416666666667	0.26106	-4.98286014668065\\
63.0416666666667	0.26272	-5.23896773189051\\
63.0416666666667	0.26438	-5.49507531710037\\
63.0416666666667	0.26604	-5.75118290231012\\
63.0416666666667	0.2677	-6.00729048751998\\
63.0416666666667	0.26936	-6.26339807272973\\
63.0416666666667	0.27102	-6.51950565793959\\
63.0416666666667	0.27268	-6.77561324314945\\
63.0416666666667	0.27434	-7.03172082835914\\
63.0416666666667	0.276	-7.287828413569\\
63.25	0.193	5.62831134756317\\
63.25	0.19466	5.38225309105803\\
63.25	0.19632	5.13619483455278\\
63.25	0.19798	4.89013657804759\\
63.25	0.19964	4.64407832154245\\
63.25	0.2013	4.3980200650372\\
63.25	0.20296	4.15196180853206\\
63.25	0.20462	3.90590355202687\\
63.25	0.20628	3.65984529552162\\
63.25	0.20794	3.41378703901648\\
63.25	0.2096	3.16772878251129\\
63.25	0.21126	2.92167052600604\\
63.25	0.21292	2.6756122695009\\
63.25	0.21458	2.42955401299571\\
63.25	0.21624	2.18349575649052\\
63.25	0.2179	1.93743749998532\\
63.25	0.21956	1.69137924348013\\
63.25	0.22122	1.44532098697493\\
63.25	0.22288	1.19926273046974\\
63.25	0.22454	0.953204473964547\\
63.25	0.2262	0.707146217459353\\
63.25	0.22786	0.46108796095416\\
63.25	0.22952	0.215029704448966\\
63.25	0.23118	-0.0310285520562275\\
63.25	0.23284	-0.277086808561421\\
63.25	0.2345	-0.523145065066558\\
63.25	0.23616	-0.769203321571808\\
63.25	0.23782	-1.015261578077\\
63.25	0.23948	-1.26131983458214\\
63.25	0.24114	-1.50737809108739\\
63.25	0.2428	-1.75343634759258\\
63.25	0.24446	-1.99949460409772\\
63.25	0.24612	-2.24555286060297\\
63.25	0.24778	-2.49161111710816\\
63.25	0.24944	-2.7376693736133\\
63.25	0.2511	-2.98372763011849\\
63.25	0.25276	-3.22978588662369\\
63.25	0.25442	-3.47584414312888\\
63.25	0.25608	-3.72190239963408\\
63.25	0.25774	-3.96796065613927\\
63.25	0.2594	-4.21401891264452\\
63.25	0.26106	-4.46007716914966\\
63.25	0.26272	-4.70613542565485\\
63.25	0.26438	-4.9521936821601\\
63.25	0.26604	-5.19825193866524\\
63.25	0.2677	-5.44431019517043\\
63.25	0.26936	-5.69036845167557\\
63.25	0.27102	-5.93642670818082\\
63.25	0.27268	-6.18248496468601\\
63.25	0.27434	-6.42854322119115\\
63.25	0.276	-6.67460147769629\\
63.4583333333333	0.193	5.73907184820456\\
63.4583333333333	0.19466	5.50306292040403\\
63.4583333333333	0.19632	5.26705399260339\\
63.4583333333333	0.19798	5.03104506480287\\
63.4583333333333	0.19964	4.79503613700234\\
63.4583333333333	0.2013	4.5590272092017\\
63.4583333333333	0.20296	4.32301828140118\\
63.4583333333333	0.20462	4.08700935360059\\
63.4583333333333	0.20628	3.85100042580001\\
63.4583333333333	0.20794	3.61499149799948\\
63.4583333333333	0.2096	3.3789825701989\\
63.4583333333333	0.21126	3.14297364239832\\
63.4583333333333	0.21292	2.90696471459779\\
63.4583333333333	0.21458	2.67095578679721\\
63.4583333333333	0.21624	2.43494685899662\\
63.4583333333333	0.2179	2.19893793119604\\
63.4583333333333	0.21956	1.96292900339552\\
63.4583333333333	0.22122	1.72692007559493\\
63.4583333333333	0.22288	1.49091114779435\\
63.4583333333333	0.22454	1.25490221999382\\
63.4583333333333	0.2262	1.01889329219324\\
63.4583333333333	0.22786	0.782884364392658\\
63.4583333333333	0.22952	0.546875436592131\\
63.4583333333333	0.23118	0.310866508791491\\
63.4583333333333	0.23284	0.0748575809909653\\
63.4583333333333	0.2345	-0.161151346809561\\
63.4583333333333	0.23616	-0.397160274610201\\
63.4583333333333	0.23782	-0.633169202410727\\
63.4583333333333	0.23948	-0.869178130211253\\
63.4583333333333	0.24114	-1.10518705801189\\
63.4583333333333	0.2428	-1.34119598581242\\
63.4583333333333	0.24446	-1.57720491361295\\
63.4583333333333	0.24612	-1.81321384141359\\
63.4583333333333	0.24778	-2.04922276921411\\
63.4583333333333	0.24944	-2.28523169701469\\
63.4583333333333	0.2511	-2.52124062481522\\
63.4583333333333	0.25276	-2.7572495526158\\
63.4583333333333	0.25442	-2.99325848041639\\
63.4583333333333	0.25608	-3.22926740821691\\
63.4583333333333	0.25774	-3.4652763360175\\
63.4583333333333	0.2594	-3.70128526381814\\
63.4583333333333	0.26106	-3.93729419161861\\
63.4583333333333	0.26272	-4.17330311941924\\
63.4583333333333	0.26438	-4.40931204721983\\
63.4583333333333	0.26604	-4.6453209750203\\
63.4583333333333	0.2677	-4.88132990282094\\
63.4583333333333	0.26936	-5.11733883062141\\
63.4583333333333	0.27102	-5.35334775842199\\
63.4583333333333	0.27268	-5.58935668622263\\
63.4583333333333	0.27434	-5.8253656140231\\
63.4583333333333	0.276	-6.0613745418238\\
63.6666666666667	0.193	5.84983234884589\\
63.6666666666667	0.19466	5.62387274974998\\
63.6666666666667	0.19632	5.39791315065401\\
63.6666666666667	0.19798	5.17195355155803\\
63.6666666666667	0.19964	4.94599395246212\\
63.6666666666667	0.2013	4.72003435336615\\
63.6666666666667	0.20296	4.49407475427023\\
63.6666666666667	0.20462	4.26811515517431\\
63.6666666666667	0.20628	4.04215555607834\\
63.6666666666667	0.20794	3.81619595698243\\
63.6666666666667	0.2096	3.59023635788651\\
63.6666666666667	0.21126	3.36427675879054\\
63.6666666666667	0.21292	3.13831715969462\\
63.6666666666667	0.21458	2.91235756059865\\
63.6666666666667	0.21624	2.68639796150268\\
63.6666666666667	0.2179	2.46043836240676\\
63.6666666666667	0.21956	2.23447876331085\\
63.6666666666667	0.22122	2.00851916421487\\
63.6666666666667	0.22288	1.78255956511896\\
63.6666666666667	0.22454	1.55659996602304\\
63.6666666666667	0.2262	1.33064036692707\\
63.6666666666667	0.22786	1.1046807678311\\
63.6666666666667	0.22952	0.878721168735183\\
63.6666666666667	0.23118	0.65276156963921\\
63.6666666666667	0.23284	0.426801970543295\\
63.6666666666667	0.2345	0.200842371447379\\
63.6666666666667	0.23616	-0.0251172276485931\\
63.6666666666667	0.23782	-0.251076826744509\\
63.6666666666667	0.23948	-0.477036425840424\\
63.6666666666667	0.24114	-0.702996024936397\\
63.6666666666667	0.2428	-0.928955624032369\\
63.6666666666667	0.24446	-1.15491522312828\\
63.6666666666667	0.24612	-1.38087482222426\\
63.6666666666667	0.24778	-1.60683442132017\\
63.6666666666667	0.24944	-1.83279402041609\\
63.6666666666667	0.2511	-2.058753619512\\
63.6666666666667	0.25276	-2.28471321860798\\
63.6666666666667	0.25442	-2.51067281770389\\
63.6666666666667	0.25608	-2.73663241679981\\
63.6666666666667	0.25774	-2.96259201589584\\
63.6666666666667	0.2594	-3.18855161499181\\
63.6666666666667	0.26106	-3.41451121408767\\
63.6666666666667	0.26272	-3.64047081318364\\
63.6666666666667	0.26438	-3.86643041227961\\
63.6666666666667	0.26604	-4.09239001137547\\
63.6666666666667	0.2677	-4.31834961047144\\
63.6666666666667	0.26936	-4.5443092095673\\
63.6666666666667	0.27102	-4.77026880866327\\
63.6666666666667	0.27268	-4.9962284077593\\
63.6666666666667	0.27434	-5.22218800685511\\
63.6666666666667	0.276	-5.44814760595108\\
63.875	0.193	5.96059284948728\\
63.875	0.19466	5.74468257909598\\
63.875	0.19632	5.52877230870462\\
63.875	0.19798	5.31286203831337\\
63.875	0.19964	5.09695176792206\\
63.875	0.2013	4.8810414975307\\
63.875	0.20296	4.6651312271394\\
63.875	0.20462	4.44922095674809\\
63.875	0.20628	4.23331068635673\\
63.875	0.20794	4.01740041596548\\
63.875	0.2096	3.80149014557418\\
63.875	0.21126	3.58557987518282\\
63.875	0.21292	3.36966960479151\\
63.875	0.21458	3.15375933440021\\
63.875	0.21624	2.93784906400884\\
63.875	0.2179	2.7219387936176\\
63.875	0.21956	2.50602852322629\\
63.875	0.22122	2.29011825283493\\
63.875	0.22288	2.07420798244362\\
63.875	0.22454	1.85829771205232\\
63.875	0.2262	1.64238744166096\\
63.875	0.22786	1.42647717126971\\
63.875	0.22952	1.2105669008784\\
63.875	0.23118	0.994656630487043\\
63.875	0.23284	0.778746360095738\\
63.875	0.2345	0.562836089704433\\
63.875	0.23616	0.346925819313071\\
63.875	0.23782	0.131015548921766\\
63.875	0.23948	-0.0848947214694817\\
63.875	0.24114	-0.300804991860844\\
63.875	0.2428	-0.516715262252148\\
63.875	0.24446	-0.732625532643453\\
63.875	0.24612	-0.948535803034815\\
63.875	0.24778	-1.16444607342612\\
63.875	0.24944	-1.38035634381737\\
63.875	0.2511	-1.59626661420867\\
63.875	0.25276	-1.81217688460004\\
63.875	0.25442	-2.02808715499134\\
63.875	0.25608	-2.24399742538264\\
63.875	0.25774	-2.45990769577401\\
63.875	0.2594	-2.67581796616531\\
63.875	0.26106	-2.89172823655656\\
63.875	0.26272	-3.10763850694792\\
63.875	0.26438	-3.32354877733928\\
63.875	0.26604	-3.53945904773053\\
63.875	0.2677	-3.75536931812189\\
63.875	0.26936	-3.97127958851314\\
63.875	0.27102	-4.18718985890445\\
63.875	0.27268	-4.40310012929581\\
63.875	0.27434	-4.61901039968711\\
63.875	0.276	-4.83492067007842\\
64.0833333333333	0.193	6.07135335012862\\
64.0833333333333	0.19466	5.86549240844192\\
64.0833333333333	0.19632	5.65963146675523\\
64.0833333333333	0.19798	5.45377052506853\\
64.0833333333333	0.19964	5.2479095833819\\
64.0833333333333	0.2013	5.04204864169515\\
64.0833333333333	0.20296	4.83618770000845\\
64.0833333333333	0.20462	4.63032675832181\\
64.0833333333333	0.20628	4.42446581663506\\
64.0833333333333	0.20794	4.21860487494843\\
64.0833333333333	0.2096	4.01274393326173\\
64.0833333333333	0.21126	3.80688299157504\\
64.0833333333333	0.21292	3.60102204988834\\
64.0833333333333	0.21458	3.39516110820165\\
64.0833333333333	0.21624	3.18930016651495\\
64.0833333333333	0.2179	2.98343922482826\\
64.0833333333333	0.21956	2.77757828314162\\
64.0833333333333	0.22122	2.57171734145487\\
64.0833333333333	0.22288	2.36585639976818\\
64.0833333333333	0.22454	2.15999545808154\\
64.0833333333333	0.2262	1.95413451639479\\
64.0833333333333	0.22786	1.74827357470815\\
64.0833333333333	0.22952	1.54241263302146\\
64.0833333333333	0.23118	1.33655169133476\\
64.0833333333333	0.23284	1.13069074964807\\
64.0833333333333	0.2345	0.924829807961373\\
64.0833333333333	0.23616	0.718968866274679\\
64.0833333333333	0.23782	0.513107924587985\\
64.0833333333333	0.23948	0.307246982901347\\
64.0833333333333	0.24114	0.101386041214596\\
64.0833333333333	0.2428	-0.104474900472042\\
64.0833333333333	0.24446	-0.310335842158736\\
64.0833333333333	0.24612	-0.516196783845487\\
64.0833333333333	0.24778	-0.722057725532125\\
64.0833333333333	0.24944	-0.927918667218819\\
64.0833333333333	0.2511	-1.13377960890546\\
64.0833333333333	0.25276	-1.33964055059221\\
64.0833333333333	0.25442	-1.5455014922789\\
64.0833333333333	0.25608	-1.75136243396554\\
64.0833333333333	0.25774	-1.95722337565229\\
64.0833333333333	0.2594	-2.16308431733898\\
64.0833333333333	0.26106	-2.36894525902562\\
64.0833333333333	0.26272	-2.57480620071232\\
64.0833333333333	0.26438	-2.78066714239907\\
64.0833333333333	0.26604	-2.98652808408571\\
64.0833333333333	0.2677	-3.1923890257724\\
64.0833333333333	0.26936	-3.39824996745904\\
64.0833333333333	0.27102	-3.60411090914573\\
64.0833333333333	0.27268	-3.80997185083248\\
64.0833333333333	0.27434	-4.01583279251906\\
64.0833333333333	0.276	-4.22169373420581\\
64.2916666666667	0.193	6.18211385076995\\
64.2916666666667	0.19466	5.98630223778792\\
64.2916666666667	0.19632	5.79049062480584\\
64.2916666666667	0.19798	5.59467901182376\\
64.2916666666667	0.19964	5.39886739884173\\
64.2916666666667	0.2013	5.20305578585965\\
64.2916666666667	0.20296	5.00724417287756\\
64.2916666666667	0.20462	4.81143255989554\\
64.2916666666667	0.20628	4.61562094691345\\
64.2916666666667	0.20794	4.41980933393137\\
64.2916666666667	0.2096	4.22399772094934\\
64.2916666666667	0.21126	4.02818610796726\\
64.2916666666667	0.21292	3.83237449498523\\
64.2916666666667	0.21458	3.63656288200315\\
64.2916666666667	0.21624	3.44075126902106\\
64.2916666666667	0.2179	3.24493965603904\\
64.2916666666667	0.21956	3.04912804305695\\
64.2916666666667	0.22122	2.85331643007487\\
64.2916666666667	0.22288	2.65750481709284\\
64.2916666666667	0.22454	2.46169320411076\\
64.2916666666667	0.2262	2.26588159112868\\
64.2916666666667	0.22786	2.07006997814665\\
64.2916666666667	0.22952	1.87425836516462\\
64.2916666666667	0.23118	1.67844675218248\\
64.2916666666667	0.23284	1.48263513920045\\
64.2916666666667	0.2345	1.28682352621843\\
64.2916666666667	0.23616	1.09101191323629\\
64.2916666666667	0.23782	0.89520030025426\\
64.2916666666667	0.23948	0.699388687272233\\
64.2916666666667	0.24114	0.503577074290092\\
64.2916666666667	0.2428	0.307765461308065\\
64.2916666666667	0.24446	0.111953848326038\\
64.2916666666667	0.24612	-0.0838577646561021\\
64.2916666666667	0.24778	-0.279669377638129\\
64.2916666666667	0.24944	-0.475480990620156\\
64.2916666666667	0.2511	-0.671292603602183\\
64.2916666666667	0.25276	-0.867104216584323\\
64.2916666666667	0.25442	-1.06291582956635\\
64.2916666666667	0.25608	-1.25872744254838\\
64.2916666666667	0.25774	-1.45453905553052\\
64.2916666666667	0.2594	-1.6503506685126\\
64.2916666666667	0.26106	-1.84616228149457\\
64.2916666666667	0.26272	-2.04197389447671\\
64.2916666666667	0.26438	-2.2377855074588\\
64.2916666666667	0.26604	-2.43359712044077\\
64.2916666666667	0.2677	-2.62940873342291\\
64.2916666666667	0.26936	-2.82522034640488\\
64.2916666666667	0.27102	-3.02103195938696\\
64.2916666666667	0.27268	-3.2168435723691\\
64.2916666666667	0.27434	-3.41265518535101\\
64.2916666666667	0.276	-3.60846679833321\\
64.5	0.193	6.29287435141134\\
64.5	0.19466	6.10711206713393\\
64.5	0.19632	5.92134978285645\\
64.5	0.19798	5.73558749857904\\
64.5	0.19964	5.54982521430162\\
64.5	0.2013	5.36406293002415\\
64.5	0.20296	5.17830064574673\\
64.5	0.20462	4.99253836146931\\
64.5	0.20628	4.80677607719178\\
64.5	0.20794	4.62101379291437\\
64.5	0.2096	4.43525150863695\\
64.5	0.21126	4.24948922435948\\
64.5	0.21292	4.06372694008206\\
64.5	0.21458	3.87796465580465\\
64.5	0.21624	3.69220237152717\\
64.5	0.2179	3.50644008724976\\
64.5	0.21956	3.32067780297234\\
64.5	0.22122	3.13491551869487\\
64.5	0.22288	2.94915323441745\\
64.5	0.22454	2.76339095014004\\
64.5	0.2262	2.57762866586256\\
64.5	0.22786	2.39186638158515\\
64.5	0.22952	2.20610409730773\\
64.5	0.23118	2.02034181303026\\
64.5	0.23284	1.83457952875284\\
64.5	0.2345	1.64881724447542\\
64.5	0.23616	1.46305496019795\\
64.5	0.23782	1.27729267592053\\
64.5	0.23948	1.09153039164312\\
64.5	0.24114	0.905768107365645\\
64.5	0.2428	0.720005823088229\\
64.5	0.24446	0.534243538810813\\
64.5	0.24612	0.34848125453334\\
64.5	0.24778	0.162718970255867\\
64.5	0.24944	-0.0230433140215496\\
64.5	0.2511	-0.208805598298966\\
64.5	0.25276	-0.394567882576439\\
64.5	0.25442	-0.580330166853855\\
64.5	0.25608	-0.766092451131271\\
64.5	0.25774	-0.951854735408745\\
64.5	0.2594	-1.13761701968622\\
64.5	0.26106	-1.32337930396358\\
64.5	0.26272	-1.50914158824105\\
64.5	0.26438	-1.69490387251852\\
64.5	0.26604	-1.88066615679588\\
64.5	0.2677	-2.06642844107336\\
64.5	0.26936	-2.25219072535072\\
64.5	0.27102	-2.43795300962819\\
64.5	0.27268	-2.62371529390566\\
64.5	0.27434	-2.80947757818308\\
64.5	0.276	-2.99523986246049\\
64.7083333333333	0.193	6.40363485205268\\
64.7083333333333	0.19466	6.22792189647987\\
64.7083333333333	0.19632	6.05220894090701\\
64.7083333333333	0.19798	5.8764959853342\\
64.7083333333333	0.19964	5.7007830297614\\
64.7083333333333	0.2013	5.52507007418859\\
64.7083333333333	0.20296	5.34935711861579\\
64.7083333333333	0.20462	5.17364416304298\\
64.7083333333333	0.20628	4.99793120747012\\
64.7083333333333	0.20794	4.82221825189737\\
64.7083333333333	0.2096	4.64650529632456\\
64.7083333333333	0.21126	4.4707923407517\\
64.7083333333333	0.21292	4.29507938517889\\
64.7083333333333	0.21458	4.11936642960609\\
64.7083333333333	0.21624	3.94365347403328\\
64.7083333333333	0.2179	3.76794051846048\\
64.7083333333333	0.21956	3.59222756288767\\
64.7083333333333	0.22122	3.41651460731481\\
64.7083333333333	0.22288	3.240801651742\\
64.7083333333333	0.22454	3.06508869616925\\
64.7083333333333	0.2262	2.88937574059639\\
64.7083333333333	0.22786	2.71366278502359\\
64.7083333333333	0.22952	2.53794982945078\\
64.7083333333333	0.23118	2.36223687387798\\
64.7083333333333	0.23284	2.18652391830517\\
64.7083333333333	0.2345	2.01081096273236\\
64.7083333333333	0.23616	1.8350980071595\\
64.7083333333333	0.23782	1.6593850515867\\
64.7083333333333	0.23948	1.48367209601395\\
64.7083333333333	0.24114	1.30795914044108\\
64.7083333333333	0.2428	1.13224618486828\\
64.7083333333333	0.24446	0.956533229295474\\
64.7083333333333	0.24612	0.780820273722668\\
64.7083333333333	0.24778	0.605107318149862\\
64.7083333333333	0.24944	0.429394362577057\\
64.7083333333333	0.2511	0.253681407004251\\
64.7083333333333	0.25276	0.0779684514314454\\
64.7083333333333	0.25442	-0.0977445041413603\\
64.7083333333333	0.25608	-0.273457459714166\\
64.7083333333333	0.25774	-0.449170415287028\\
64.7083333333333	0.2594	-0.624883370859891\\
64.7083333333333	0.26106	-0.800596326432583\\
64.7083333333333	0.26272	-0.976309282005445\\
64.7083333333333	0.26438	-1.15202223757831\\
64.7083333333333	0.26604	-1.32773519315106\\
64.7083333333333	0.2677	-1.50344814872392\\
64.7083333333333	0.26936	-1.67916110429661\\
64.7083333333333	0.27102	-1.85487405986947\\
64.7083333333333	0.27268	-2.03058701544228\\
64.7083333333333	0.27434	-2.20629997101503\\
64.7083333333333	0.276	-2.38201292658789\\
64.9166666666667	0.193	6.51439535269407\\
64.9166666666667	0.19466	6.34873172582587\\
64.9166666666667	0.19632	6.18306809895768\\
64.9166666666667	0.19798	6.01740447208948\\
64.9166666666667	0.19964	5.85174084522134\\
64.9166666666667	0.2013	5.68607721835309\\
64.9166666666667	0.20296	5.52041359148495\\
64.9166666666667	0.20462	5.35474996461676\\
64.9166666666667	0.20628	5.18908633774856\\
64.9166666666667	0.20794	5.02342271088037\\
64.9166666666667	0.2096	4.85775908401223\\
64.9166666666667	0.21126	4.69209545714398\\
64.9166666666667	0.21292	4.52643183027584\\
64.9166666666667	0.21458	4.36076820340764\\
64.9166666666667	0.21624	4.19510457653945\\
64.9166666666667	0.2179	4.02944094967125\\
64.9166666666667	0.21956	3.86377732280312\\
64.9166666666667	0.22122	3.69811369593486\\
64.9166666666667	0.22288	3.53245006906673\\
64.9166666666667	0.22454	3.36678644219853\\
64.9166666666667	0.2262	3.20112281533034\\
64.9166666666667	0.22786	3.03545918846214\\
64.9166666666667	0.22952	2.869795561594\\
64.9166666666667	0.23118	2.70413193472575\\
64.9166666666667	0.23284	2.53846830785761\\
64.9166666666667	0.2345	2.37280468098942\\
64.9166666666667	0.23616	2.20714105412122\\
64.9166666666667	0.23782	2.04147742725303\\
64.9166666666667	0.23948	1.87581380038489\\
64.9166666666667	0.24114	1.71015017351664\\
64.9166666666667	0.2428	1.5444865466485\\
64.9166666666667	0.24446	1.3788229197803\\
64.9166666666667	0.24612	1.21315929291211\\
64.9166666666667	0.24778	1.04749566604391\\
64.9166666666667	0.24944	0.881832039175777\\
64.9166666666667	0.2511	0.716168412307582\\
64.9166666666667	0.25276	0.550504785439387\\
64.9166666666667	0.25442	0.384841158571191\\
64.9166666666667	0.25608	0.219177531703053\\
64.9166666666667	0.25774	0.0535139048348015\\
64.9166666666667	0.2594	-0.112149722033394\\
64.9166666666667	0.26106	-0.277813348901532\\
64.9166666666667	0.26272	-0.443476975769727\\
64.9166666666667	0.26438	-0.609140602637979\\
64.9166666666667	0.26604	-0.774804229506117\\
64.9166666666667	0.2677	-0.940467856374312\\
64.9166666666667	0.26936	-1.10613148324245\\
64.9166666666667	0.27102	-1.27179511011059\\
64.9166666666667	0.27268	-1.43745873697878\\
64.9166666666667	0.27434	-1.60312236384698\\
64.9166666666667	0.276	-1.76878599071529\\
65.125	0.193	6.6251558533354\\
65.125	0.19466	6.46954155517182\\
65.125	0.19632	6.31392725700823\\
65.125	0.19798	6.1583129588447\\
65.125	0.19964	6.00269866068118\\
65.125	0.2013	5.84708436251753\\
65.125	0.20296	5.69147006435401\\
65.125	0.20462	5.53585576619048\\
65.125	0.20628	5.3802414680269\\
65.125	0.20794	5.22462716986331\\
65.125	0.2096	5.06901287169978\\
65.125	0.21126	4.9133985735362\\
65.125	0.21292	4.75778427537267\\
65.125	0.21458	4.60216997720909\\
65.125	0.21624	4.4465556790455\\
65.125	0.2179	4.29094138088197\\
65.125	0.21956	4.13532708271845\\
65.125	0.22122	3.97971278455481\\
65.125	0.22288	3.82409848639128\\
65.125	0.22454	3.66848418822775\\
65.125	0.2262	3.51286989006417\\
65.125	0.22786	3.35725559190058\\
65.125	0.22952	3.20164129373705\\
65.125	0.23118	3.04602699557347\\
65.125	0.23284	2.89041269740994\\
65.125	0.2345	2.73479839924636\\
65.125	0.23616	2.57918410108277\\
65.125	0.23782	2.42356980291925\\
65.125	0.23948	2.26795550475572\\
65.125	0.24114	2.11234120659208\\
65.125	0.2428	1.95672690842855\\
65.125	0.24446	1.80111261026502\\
65.125	0.24612	1.64549831210144\\
65.125	0.24778	1.48988401393785\\
65.125	0.24944	1.33426971577433\\
65.125	0.2511	1.1786554176108\\
65.125	0.25276	1.02304111944721\\
65.125	0.25442	0.86742682128363\\
65.125	0.25608	0.711812523120102\\
65.125	0.25774	0.556198224956518\\
65.125	0.2594	0.400583926792933\\
65.125	0.26106	0.244969628629462\\
65.125	0.26272	0.0893553304658212\\
65.125	0.26438	-0.0662589676977632\\
65.125	0.26604	-0.221873265861234\\
65.125	0.2677	-0.377487564024875\\
65.125	0.26936	-0.533101862188346\\
65.125	0.27102	-0.688716160351987\\
65.125	0.27268	-0.844330458515628\\
65.125	0.27434	-0.999944756678929\\
65.125	0.276	-1.15555905484257\\
65.3333333333333	0.193	6.73591635397668\\
65.3333333333333	0.19466	6.59035138451776\\
65.3333333333333	0.19632	6.44478641505879\\
65.3333333333333	0.19798	6.29922144559987\\
65.3333333333333	0.19964	6.15365647614095\\
65.3333333333333	0.2013	6.00809150668198\\
65.3333333333333	0.20296	5.86252653722306\\
65.3333333333333	0.20462	5.71696156776414\\
65.3333333333333	0.20628	5.57139659830517\\
65.3333333333333	0.20794	5.42583162884625\\
65.3333333333333	0.2096	5.28026665938734\\
65.3333333333333	0.21126	5.13470168992836\\
65.3333333333333	0.21292	4.9891367204695\\
65.3333333333333	0.21458	4.84357175101059\\
65.3333333333333	0.21624	4.69800678155161\\
65.3333333333333	0.2179	4.5524418120927\\
65.3333333333333	0.21956	4.40687684263378\\
65.3333333333333	0.22122	4.2613118731748\\
65.3333333333333	0.22288	4.11574690371589\\
65.3333333333333	0.22454	3.97018193425697\\
65.3333333333333	0.2262	3.824616964798\\
65.3333333333333	0.22786	3.67905199533908\\
65.3333333333333	0.22952	3.53348702588016\\
65.3333333333333	0.23118	3.38792205642119\\
65.3333333333333	0.23284	3.24235708696227\\
65.3333333333333	0.2345	3.09679211750336\\
65.3333333333333	0.23616	2.95122714804438\\
65.3333333333333	0.23782	2.80566217858546\\
65.3333333333333	0.23948	2.66009720912655\\
65.3333333333333	0.24114	2.51453223966757\\
65.3333333333333	0.2428	2.36896727020866\\
65.3333333333333	0.24446	2.22340230074974\\
65.3333333333333	0.24612	2.07783733129077\\
65.3333333333333	0.24778	1.93227236183185\\
65.3333333333333	0.24944	1.78670739237293\\
65.3333333333333	0.2511	1.64114242291402\\
65.3333333333333	0.25276	1.49557745345504\\
65.3333333333333	0.25442	1.35001248399612\\
65.3333333333333	0.25608	1.20444751453721\\
65.3333333333333	0.25774	1.05888254507823\\
65.3333333333333	0.2594	0.91331757561926\\
65.3333333333333	0.26106	0.7677526061604\\
65.3333333333333	0.26272	0.622187636701426\\
65.3333333333333	0.26438	0.476622667242452\\
65.3333333333333	0.26604	0.331057697783592\\
65.3333333333333	0.2677	0.185492728324618\\
65.3333333333333	0.26936	0.0399277588657583\\
65.3333333333333	0.27102	-0.105637210593159\\
65.3333333333333	0.27268	-0.251202180052132\\
65.3333333333333	0.27434	-0.396767149511106\\
65.3333333333333	0.276	-0.54233211897008\\
65.5416666666667	0.193	6.84667685461807\\
65.5416666666667	0.19466	6.71116121386382\\
65.5416666666667	0.19632	6.57564557310945\\
65.5416666666667	0.19798	6.44012993235515\\
65.5416666666667	0.19964	6.3046142916009\\
65.5416666666667	0.2013	6.16909865084654\\
65.5416666666667	0.20296	6.03358301009223\\
65.5416666666667	0.20462	5.89806736933798\\
65.5416666666667	0.20628	5.76255172858362\\
65.5416666666667	0.20794	5.62703608782931\\
65.5416666666667	0.2096	5.491520447075\\
65.5416666666667	0.21126	5.3560048063207\\
65.5416666666667	0.21292	5.22048916556639\\
65.5416666666667	0.21458	5.08497352481209\\
65.5416666666667	0.21624	4.94945788405778\\
65.5416666666667	0.2179	4.81394224330347\\
65.5416666666667	0.21956	4.67842660254917\\
65.5416666666667	0.22122	4.5429109617948\\
65.5416666666667	0.22288	4.40739532104055\\
65.5416666666667	0.22454	4.27187968028625\\
65.5416666666667	0.2262	4.13636403953188\\
65.5416666666667	0.22786	4.00084839877763\\
65.5416666666667	0.22952	3.86533275802333\\
65.5416666666667	0.23118	3.72981711726896\\
65.5416666666667	0.23284	3.59430147651466\\
65.5416666666667	0.2345	3.45878583576041\\
65.5416666666667	0.23616	3.32327019500605\\
65.5416666666667	0.23782	3.18775455425174\\
65.5416666666667	0.23948	3.05223891349749\\
65.5416666666667	0.24114	2.91672327274313\\
65.5416666666667	0.2428	2.78120763198882\\
65.5416666666667	0.24446	2.64569199123457\\
65.5416666666667	0.24612	2.51017635048021\\
65.5416666666667	0.24778	2.3746607097259\\
65.5416666666667	0.24944	2.2391450689716\\
65.5416666666667	0.2511	2.10362942821735\\
65.5416666666667	0.25276	1.96811378746298\\
65.5416666666667	0.25442	1.83259814670868\\
65.5416666666667	0.25608	1.69708250595443\\
65.5416666666667	0.25774	1.56156686520006\\
65.5416666666667	0.2594	1.4260512244457\\
65.5416666666667	0.26106	1.29053558369145\\
65.5416666666667	0.26272	1.15501994293714\\
65.5416666666667	0.26438	1.01950430218278\\
65.5416666666667	0.26604	0.883988661428532\\
65.5416666666667	0.2677	0.748473020674226\\
65.5416666666667	0.26936	0.612957379919976\\
65.5416666666667	0.27102	0.47744173916567\\
65.5416666666667	0.27268	0.34192609841125\\
65.5416666666667	0.27434	0.206410457657057\\
65.5416666666667	0.276	0.0708948169027508\\
65.75	0.193	6.9574373552594\\
65.75	0.19466	6.83197104320976\\
65.75	0.19632	6.70650473116001\\
65.75	0.19798	6.58103841911037\\
65.75	0.19964	6.45557210706068\\
65.75	0.2013	6.33010579501098\\
65.75	0.20296	6.20463948296134\\
65.75	0.20462	6.07917317091164\\
65.75	0.20628	5.95370685886195\\
65.75	0.20794	5.82824054681225\\
65.75	0.2096	5.70277423476261\\
65.75	0.21126	5.57730792271286\\
65.75	0.21292	5.45184161066322\\
65.75	0.21458	5.32637529861353\\
65.75	0.21624	5.20090898656383\\
65.75	0.2179	5.07544267451419\\
65.75	0.21956	4.9499763624645\\
65.75	0.22122	4.8245100504148\\
65.75	0.22288	4.69904373836511\\
65.75	0.22454	4.57357742631547\\
65.75	0.2262	4.44811111426571\\
65.75	0.22786	4.32264480221608\\
65.75	0.22952	4.19717849016638\\
65.75	0.23118	4.07171217811668\\
65.75	0.23284	3.94624586606699\\
65.75	0.2345	3.82077955401735\\
65.75	0.23616	3.69531324196765\\
65.75	0.23782	3.56984692991796\\
65.75	0.23948	3.44438061786832\\
65.75	0.24114	3.31891430581857\\
65.75	0.2428	3.19344799376893\\
65.75	0.24446	3.06798168171923\\
65.75	0.24612	2.94251536966954\\
65.75	0.24778	2.81704905761984\\
65.75	0.24944	2.6915827455702\\
65.75	0.2511	2.56611643352056\\
65.75	0.25276	2.44065012147081\\
65.75	0.25442	2.31518380942117\\
65.75	0.25608	2.18971749737148\\
65.75	0.25774	2.06425118532178\\
65.75	0.2594	1.93878487327203\\
65.75	0.26106	1.81331856122245\\
65.75	0.26272	1.68785224917269\\
65.75	0.26438	1.562385937123\\
65.75	0.26604	1.43691962507336\\
65.75	0.2677	1.31145331302366\\
65.75	0.26936	1.18598700097402\\
65.75	0.27102	1.06052068892438\\
65.75	0.27268	0.935054376874632\\
65.75	0.27434	0.809588064824993\\
65.75	0.276	0.684121752775241\\
65.9583333333333	0.193	7.06819785590079\\
65.9583333333333	0.19466	6.95278087255576\\
65.9583333333333	0.19632	6.83736388921062\\
65.9583333333333	0.19798	6.72194690586559\\
65.9583333333333	0.19964	6.60652992252056\\
65.9583333333333	0.2013	6.49111293917548\\
65.9583333333333	0.20296	6.37569595583045\\
65.9583333333333	0.20462	6.26027897248542\\
65.9583333333333	0.20628	6.14486198914028\\
65.9583333333333	0.20794	6.02944500579525\\
65.9583333333333	0.2096	5.91402802245022\\
65.9583333333333	0.21126	5.79861103910514\\
65.9583333333333	0.21292	5.68319405576011\\
65.9583333333333	0.21458	5.56777707241503\\
65.9583333333333	0.21624	5.45236008906994\\
65.9583333333333	0.2179	5.33694310572491\\
65.9583333333333	0.21956	5.22152612237988\\
65.9583333333333	0.22122	5.1061091390348\\
65.9583333333333	0.22288	4.99069215568971\\
65.9583333333333	0.22454	4.87527517234469\\
65.9583333333333	0.2262	4.7598581889996\\
65.9583333333333	0.22786	4.64444120565457\\
65.9583333333333	0.22952	4.52902422230954\\
65.9583333333333	0.23118	4.41360723896446\\
65.9583333333333	0.23284	4.29819025561937\\
65.9583333333333	0.2345	4.18277327227435\\
65.9583333333333	0.23616	4.06735628892926\\
65.9583333333333	0.23782	3.95193930558423\\
65.9583333333333	0.23948	3.8365223222392\\
65.9583333333333	0.24114	3.72110533889406\\
65.9583333333333	0.2428	3.60568835554903\\
65.9583333333333	0.24446	3.49027137220401\\
65.9583333333333	0.24612	3.37485438885892\\
65.9583333333333	0.24778	3.25943740551389\\
65.9583333333333	0.24944	3.14402042216886\\
65.9583333333333	0.2511	3.02860343882378\\
65.9583333333333	0.25276	2.91318645547869\\
65.9583333333333	0.25442	2.79776947213367\\
65.9583333333333	0.25608	2.68235248878864\\
65.9583333333333	0.25774	2.56693550544355\\
65.9583333333333	0.2594	2.45151852209847\\
65.9583333333333	0.26106	2.33610153875344\\
65.9583333333333	0.26272	2.22068455540835\\
65.9583333333333	0.26438	2.10526757206327\\
65.9583333333333	0.26604	1.9898505887183\\
65.9583333333333	0.2677	1.87443360537316\\
65.9583333333333	0.26936	1.75901662202818\\
65.9583333333333	0.27102	1.6435996386831\\
65.9583333333333	0.27268	1.52818265533801\\
65.9583333333333	0.27434	1.41276567199304\\
65.9583333333333	0.276	1.29734868864796\\
66.1666666666667	0.193	7.17895835654213\\
66.1666666666667	0.19466	7.07359070190171\\
66.1666666666667	0.19632	6.96822304726123\\
66.1666666666667	0.19798	6.86285539262082\\
66.1666666666667	0.19964	6.7574877379804\\
66.1666666666667	0.2013	6.65212008333998\\
66.1666666666667	0.20296	6.54675242869956\\
66.1666666666667	0.20462	6.44138477405915\\
66.1666666666667	0.20628	6.33601711941867\\
66.1666666666667	0.20794	6.23064946477825\\
66.1666666666667	0.2096	6.12528181013784\\
66.1666666666667	0.21126	6.01991415549736\\
66.1666666666667	0.21292	5.91454650085694\\
66.1666666666667	0.21458	5.80917884621653\\
66.1666666666667	0.21624	5.70381119157605\\
66.1666666666667	0.2179	5.59844353693563\\
66.1666666666667	0.21956	5.49307588229527\\
66.1666666666667	0.22122	5.3877082276548\\
66.1666666666667	0.22288	5.28234057301438\\
66.1666666666667	0.22454	5.17697291837396\\
66.1666666666667	0.2262	5.07160526373349\\
66.1666666666667	0.22786	4.96623760909307\\
66.1666666666667	0.22952	4.86086995445265\\
66.1666666666667	0.23118	4.75550229981218\\
66.1666666666667	0.23284	4.65013464517176\\
66.1666666666667	0.2345	4.54476699053134\\
66.1666666666667	0.23616	4.43939933589087\\
66.1666666666667	0.23782	4.33403168125051\\
66.1666666666667	0.23948	4.22866402661009\\
66.1666666666667	0.24114	4.12329637196962\\
66.1666666666667	0.2428	4.0179287173292\\
66.1666666666667	0.24446	3.91256106268878\\
66.1666666666667	0.24612	3.80719340804831\\
66.1666666666667	0.24778	3.70182575340789\\
66.1666666666667	0.24944	3.59645809876747\\
66.1666666666667	0.2511	3.49109044412705\\
66.1666666666667	0.25276	3.38572278948658\\
66.1666666666667	0.25442	3.28035513484616\\
66.1666666666667	0.25608	3.1749874802058\\
66.1666666666667	0.25774	3.06961982556533\\
66.1666666666667	0.2594	2.96425217092485\\
66.1666666666667	0.26106	2.85888451628449\\
66.1666666666667	0.26272	2.75351686164402\\
66.1666666666667	0.26438	2.64814920700354\\
66.1666666666667	0.26604	2.54278155236318\\
66.1666666666667	0.2677	2.43741389772276\\
66.1666666666667	0.26936	2.33204624308235\\
66.1666666666667	0.27102	2.22667858844181\\
66.1666666666667	0.27268	2.1213109338014\\
66.1666666666667	0.27434	2.01594327916109\\
66.1666666666667	0.276	1.91057562452067\\
66.375	0.193	7.28971885718352\\
66.375	0.19466	7.19440053124771\\
66.375	0.19632	7.0990822053119\\
66.375	0.19798	7.00376387937609\\
66.375	0.19964	6.90844555344029\\
66.375	0.2013	6.81312722750448\\
66.375	0.20296	6.71780890156867\\
66.375	0.20462	6.62249057563287\\
66.375	0.20628	6.52717224969706\\
66.375	0.20794	6.43185392376125\\
66.375	0.2096	6.33653559782545\\
66.375	0.21126	6.24121727188964\\
66.375	0.21292	6.14589894595383\\
66.375	0.21458	6.05058062001802\\
66.375	0.21624	5.95526229408222\\
66.375	0.2179	5.85994396814641\\
66.375	0.21956	5.76462564221066\\
66.375	0.22122	5.6693073162748\\
66.375	0.22288	5.57398899033899\\
66.375	0.22454	5.47867066440324\\
66.375	0.2262	5.38335233846738\\
66.375	0.22786	5.28803401253157\\
66.375	0.22952	5.19271568659582\\
66.375	0.23118	5.09739736065995\\
66.375	0.23284	5.00207903472415\\
66.375	0.2345	4.9067607087884\\
66.375	0.23616	4.81144238285253\\
66.375	0.23782	4.71612405691673\\
66.375	0.23948	4.62080573098098\\
66.375	0.24114	4.52548740504511\\
66.375	0.2428	4.43016907910931\\
66.375	0.24446	4.33485075317355\\
66.375	0.24612	4.23953242723769\\
66.375	0.24778	4.14421410130188\\
66.375	0.24944	4.04889577536613\\
66.375	0.2511	3.95357744943033\\
66.375	0.25276	3.85825912349452\\
66.375	0.25442	3.76294079755871\\
66.375	0.25608	3.66762247162291\\
66.375	0.25774	3.5723041456871\\
66.375	0.2594	3.47698581975123\\
66.375	0.26106	3.38166749381548\\
66.375	0.26272	3.28634916787968\\
66.375	0.26438	3.19103084194381\\
66.375	0.26604	3.09571251600812\\
66.375	0.2677	3.00039419007226\\
66.375	0.26936	2.90507586413651\\
66.375	0.27102	2.80975753820064\\
66.375	0.27268	2.71443921226478\\
66.375	0.27434	2.61912088632914\\
66.375	0.276	2.52380256039328\\
66.5833333333333	0.193	7.40047935782485\\
66.5833333333333	0.19466	7.31521036059371\\
66.5833333333333	0.19632	7.22994136336251\\
66.5833333333333	0.19798	7.14467236613132\\
66.5833333333333	0.19964	7.05940336890018\\
66.5833333333333	0.2013	6.97413437166898\\
66.5833333333333	0.20296	6.88886537443778\\
66.5833333333333	0.20462	6.80359637720665\\
66.5833333333333	0.20628	6.71832737997539\\
66.5833333333333	0.20794	6.63305838274425\\
66.5833333333333	0.2096	6.54778938551311\\
66.5833333333333	0.21126	6.46252038828186\\
66.5833333333333	0.21292	6.37725139105072\\
66.5833333333333	0.21458	6.29198239381958\\
66.5833333333333	0.21624	6.20671339658833\\
66.5833333333333	0.2179	6.12144439935719\\
66.5833333333333	0.21956	6.03617540212599\\
66.5833333333333	0.22122	5.95090640489479\\
66.5833333333333	0.22288	5.86563740766366\\
66.5833333333333	0.22454	5.78036841043246\\
66.5833333333333	0.2262	5.69509941320126\\
66.5833333333333	0.22786	5.60983041597007\\
66.5833333333333	0.22952	5.52456141873893\\
66.5833333333333	0.23118	5.43929242150773\\
66.5833333333333	0.23284	5.35402342427653\\
66.5833333333333	0.2345	5.26875442704539\\
66.5833333333333	0.23616	5.18348542981414\\
66.5833333333333	0.23782	5.098216432583\\
66.5833333333333	0.23948	5.01294743535186\\
66.5833333333333	0.24114	4.92767843812061\\
66.5833333333333	0.2428	4.84240944088947\\
66.5833333333333	0.24446	4.75714044365833\\
66.5833333333333	0.24612	4.67187144642708\\
66.5833333333333	0.24778	4.58660244919594\\
66.5833333333333	0.24944	4.50133345196474\\
66.5833333333333	0.2511	4.4160644547336\\
66.5833333333333	0.25276	4.3307954575024\\
66.5833333333333	0.25442	4.24552646027121\\
66.5833333333333	0.25608	4.16025746304007\\
66.5833333333333	0.25774	4.07498846580881\\
66.5833333333333	0.2594	3.98971946857762\\
66.5833333333333	0.26106	3.90445047134654\\
66.5833333333333	0.26272	3.81918147411528\\
66.5833333333333	0.26438	3.73391247688409\\
66.5833333333333	0.26604	3.64864347965295\\
66.5833333333333	0.2677	3.56337448242175\\
66.5833333333333	0.26936	3.47810548519067\\
66.5833333333333	0.27102	3.39283648795947\\
66.5833333333333	0.27268	3.30756749072816\\
66.5833333333333	0.27434	3.22229849349708\\
66.5833333333333	0.276	3.13702949626588\\
66.7916666666667	0.193	7.51123985846624\\
66.7916666666667	0.19466	7.43602018993971\\
66.7916666666667	0.19632	7.36080052141313\\
66.7916666666667	0.19798	7.28558085288654\\
66.7916666666667	0.19964	7.21036118436001\\
66.7916666666667	0.2013	7.13514151583342\\
66.7916666666667	0.20296	7.0599218473069\\
66.7916666666667	0.20462	6.98470217878037\\
66.7916666666667	0.20628	6.90948251025378\\
66.7916666666667	0.20794	6.83426284172725\\
66.7916666666667	0.2096	6.75904317320072\\
66.7916666666667	0.21126	6.68382350467414\\
66.7916666666667	0.21292	6.60860383614755\\
66.7916666666667	0.21458	6.53338416762102\\
66.7916666666667	0.21624	6.45816449909444\\
66.7916666666667	0.2179	6.38294483056791\\
66.7916666666667	0.21956	6.30772516204138\\
66.7916666666667	0.22122	6.23250549351479\\
66.7916666666667	0.22288	6.15728582498826\\
66.7916666666667	0.22454	6.08206615646174\\
66.7916666666667	0.2262	6.00684648793515\\
66.7916666666667	0.22786	5.93162681940856\\
66.7916666666667	0.22952	5.85640715088203\\
66.7916666666667	0.23118	5.78118748235545\\
66.7916666666667	0.23284	5.70596781382892\\
66.7916666666667	0.2345	5.63074814530239\\
66.7916666666667	0.23616	5.55552847677581\\
66.7916666666667	0.23782	5.48030880824928\\
66.7916666666667	0.23948	5.40508913972275\\
66.7916666666667	0.24114	5.32986947119616\\
66.7916666666667	0.2428	5.25464980266958\\
66.7916666666667	0.24446	5.17943013414305\\
66.7916666666667	0.24612	5.10421046561646\\
66.7916666666667	0.24778	5.02899079708993\\
66.7916666666667	0.24944	4.9537711285634\\
66.7916666666667	0.2511	4.87855146003687\\
66.7916666666667	0.25276	4.80333179151029\\
66.7916666666667	0.25442	4.72811212298376\\
66.7916666666667	0.25608	4.65289245445723\\
66.7916666666667	0.25774	4.57767278593059\\
66.7916666666667	0.2594	4.502453117404\\
66.7916666666667	0.26106	4.42723344887753\\
66.7916666666667	0.26272	4.35201378035094\\
66.7916666666667	0.26438	4.27679411182436\\
66.7916666666667	0.26604	4.20157444329777\\
66.7916666666667	0.2677	4.12635477477124\\
66.7916666666667	0.26936	4.05113510624483\\
66.7916666666667	0.27102	3.9759154377183\\
66.7916666666667	0.27268	3.90069576919166\\
66.7916666666667	0.27434	3.82547610066513\\
66.7916666666667	0.276	3.75025643213849\\
67	0.193	7.62200035910757\\
67	0.19466	7.55683001928566\\
67	0.19632	7.49165967946368\\
67	0.19798	7.42648933964176\\
67	0.19964	7.36131899981984\\
67	0.2013	7.29614865999787\\
67	0.20296	7.23097832017595\\
67	0.20462	7.16580798035409\\
67	0.20628	7.10063764053211\\
67	0.20794	7.0354673007102\\
67	0.2096	6.97029696088828\\
67	0.21126	6.9051266210663\\
67	0.21292	6.83995628124438\\
67	0.21458	6.77478594142252\\
67	0.21624	6.70961560160055\\
67	0.2179	6.64444526177863\\
67	0.21956	6.57927492195671\\
67	0.22122	6.51410458213473\\
67	0.22288	6.44893424231282\\
67	0.22454	6.3837639024909\\
67	0.2262	6.31859356266898\\
67	0.22786	6.25342322284706\\
67	0.22952	6.18825288302514\\
67	0.23118	6.12308254320317\\
67	0.23284	6.05791220338125\\
67	0.2345	5.99274186355933\\
67	0.23616	5.92757152373736\\
67	0.23782	5.86240118391549\\
67	0.23948	5.79723084409358\\
67	0.24114	5.7320605042716\\
67	0.2428	5.66689016444968\\
67	0.24446	5.60171982462776\\
67	0.24612	5.53654948480579\\
67	0.24778	5.47137914498387\\
67	0.24944	5.40620880516201\\
67	0.2511	5.34103846534009\\
67	0.25276	5.27586812551812\\
67	0.25442	5.2106977856962\\
67	0.25608	5.14552744587428\\
67	0.25774	5.0803571060523\\
67	0.2594	5.01518676623033\\
67	0.26106	4.95001642640852\\
67	0.26272	4.88484608658655\\
67	0.26438	4.81967574676452\\
67	0.26604	4.75450540694283\\
67	0.2677	4.68933506712085\\
67	0.26936	4.62416472729888\\
67	0.27102	4.5589943874769\\
67	0.27268	4.49382404765493\\
67	0.27434	4.42865370783318\\
67	0.276	4.3634833680112\\
};
\end{axis}

\begin{axis}[%
width=4.927496cm,
height=3.050847cm,
at={(6.483547cm,8.474576cm)},
scale only axis,
xmin=57,
xmax=67,
tick align=outside,
xlabel={$L_{cut}$},
xmajorgrids,
ymin=0.193,
ymax=0.276,
ylabel={$D_{rlx}$},
ymajorgrids,
zmin=-1366.1541381888,
zmax=-368.246171544014,
zlabel={$x_4$},
zmajorgrids,
view={-140}{50},
legend style={at={(1.03,1)},anchor=north west,legend cell align=left,align=left,draw=white!15!black}
]
\addplot3[only marks,mark=*,mark options={},mark size=1.5000pt,color=mycolor1] plot table[row sep=crcr,]{%
67	0.276	-1318.97892531254\\
66	0.255	-1084.71901141138\\
62	0.209	-600.143326233093\\
57	0.193	-477.450293212086\\
};
\addplot3[only marks,mark=*,mark options={},mark size=1.5000pt,color=black] plot table[row sep=crcr,]{%
64	0.23	-816.532600837712\\
};

\addplot3[%
surf,
opacity=0.7,
shader=interp,
colormap={mymap}{[1pt] rgb(0pt)=(0.0901961,0.239216,0.0745098); rgb(1pt)=(0.0945149,0.242058,0.0739522); rgb(2pt)=(0.0988592,0.244894,0.0733566); rgb(3pt)=(0.103229,0.247724,0.0727241); rgb(4pt)=(0.107623,0.250549,0.0720557); rgb(5pt)=(0.112043,0.253367,0.0713525); rgb(6pt)=(0.116487,0.25618,0.0706154); rgb(7pt)=(0.120956,0.258986,0.0698456); rgb(8pt)=(0.125449,0.261787,0.0690441); rgb(9pt)=(0.129967,0.264581,0.0682118); rgb(10pt)=(0.134508,0.26737,0.06735); rgb(11pt)=(0.139074,0.270152,0.0664596); rgb(12pt)=(0.143663,0.272929,0.0655416); rgb(13pt)=(0.148275,0.275699,0.0645971); rgb(14pt)=(0.152911,0.278463,0.0636271); rgb(15pt)=(0.15757,0.281221,0.0626328); rgb(16pt)=(0.162252,0.283973,0.0616151); rgb(17pt)=(0.166957,0.286719,0.060575); rgb(18pt)=(0.171685,0.289458,0.0595136); rgb(19pt)=(0.176434,0.292191,0.0584321); rgb(20pt)=(0.181207,0.294918,0.0573313); rgb(21pt)=(0.186001,0.297639,0.0562123); rgb(22pt)=(0.190817,0.300353,0.0550763); rgb(23pt)=(0.195655,0.303061,0.0539242); rgb(24pt)=(0.200514,0.305763,0.052757); rgb(25pt)=(0.205395,0.308459,0.0515759); rgb(26pt)=(0.210296,0.311149,0.0503624); rgb(27pt)=(0.215212,0.313846,0.0490067); rgb(28pt)=(0.220142,0.316548,0.0475043); rgb(29pt)=(0.22509,0.319254,0.0458704); rgb(30pt)=(0.230056,0.321962,0.0441205); rgb(31pt)=(0.235042,0.324671,0.04227); rgb(32pt)=(0.240048,0.327379,0.0403343); rgb(33pt)=(0.245078,0.330085,0.0383287); rgb(34pt)=(0.250131,0.332786,0.0362688); rgb(35pt)=(0.25521,0.335482,0.0341698); rgb(36pt)=(0.260317,0.33817,0.0320472); rgb(37pt)=(0.265451,0.340849,0.0299163); rgb(38pt)=(0.270616,0.343517,0.0277927); rgb(39pt)=(0.275813,0.346172,0.0256916); rgb(40pt)=(0.281043,0.348814,0.0236284); rgb(41pt)=(0.286307,0.35144,0.0216186); rgb(42pt)=(0.291607,0.354048,0.0196776); rgb(43pt)=(0.296945,0.356637,0.0178207); rgb(44pt)=(0.302322,0.359206,0.0160634); rgb(45pt)=(0.307739,0.361753,0.0144211); rgb(46pt)=(0.313198,0.364275,0.0129091); rgb(47pt)=(0.318701,0.366772,0.0115428); rgb(48pt)=(0.324249,0.369242,0.0103377); rgb(49pt)=(0.329843,0.371682,0.00930909); rgb(50pt)=(0.335485,0.374093,0.00847245); rgb(51pt)=(0.341176,0.376471,0.00784314); rgb(52pt)=(0.346925,0.378826,0.00732741); rgb(53pt)=(0.352735,0.381168,0.00682184); rgb(54pt)=(0.358605,0.383497,0.00632729); rgb(55pt)=(0.364532,0.385812,0.00584464); rgb(56pt)=(0.370516,0.388113,0.00537476); rgb(57pt)=(0.376552,0.390399,0.00491852); rgb(58pt)=(0.38264,0.39267,0.00447681); rgb(59pt)=(0.388777,0.394925,0.00405048); rgb(60pt)=(0.394962,0.397164,0.00364042); rgb(61pt)=(0.401191,0.399386,0.00324749); rgb(62pt)=(0.407464,0.401592,0.00287258); rgb(63pt)=(0.413777,0.40378,0.00251655); rgb(64pt)=(0.420129,0.40595,0.00218028); rgb(65pt)=(0.426518,0.408102,0.00186463); rgb(66pt)=(0.432942,0.410234,0.00157049); rgb(67pt)=(0.439399,0.412348,0.00129873); rgb(68pt)=(0.445885,0.414441,0.00105022); rgb(69pt)=(0.452401,0.416515,0.000825833); rgb(70pt)=(0.458942,0.418567,0.000626441); rgb(71pt)=(0.465508,0.420599,0.00045292); rgb(72pt)=(0.472096,0.422609,0.000306141); rgb(73pt)=(0.478704,0.424596,0.000186979); rgb(74pt)=(0.485331,0.426562,9.63073e-05); rgb(75pt)=(0.491973,0.428504,3.49981e-05); rgb(76pt)=(0.498628,0.430422,3.92506e-06); rgb(77pt)=(0.505323,0.432315,0); rgb(78pt)=(0.512206,0.434168,0); rgb(79pt)=(0.519282,0.435983,0); rgb(80pt)=(0.526529,0.437764,0); rgb(81pt)=(0.533922,0.439512,0); rgb(82pt)=(0.54144,0.441232,0); rgb(83pt)=(0.549059,0.442927,0); rgb(84pt)=(0.556756,0.444599,0); rgb(85pt)=(0.564508,0.446252,0); rgb(86pt)=(0.572292,0.447889,0); rgb(87pt)=(0.580084,0.449514,0); rgb(88pt)=(0.587863,0.451129,0); rgb(89pt)=(0.595604,0.452737,0); rgb(90pt)=(0.603284,0.454343,0); rgb(91pt)=(0.610882,0.455948,0); rgb(92pt)=(0.618373,0.457556,0); rgb(93pt)=(0.625734,0.459171,0); rgb(94pt)=(0.632943,0.460795,0); rgb(95pt)=(0.639976,0.462432,0); rgb(96pt)=(0.64681,0.464084,0); rgb(97pt)=(0.653423,0.465756,0); rgb(98pt)=(0.659791,0.46745,0); rgb(99pt)=(0.665891,0.469169,0); rgb(100pt)=(0.6717,0.470916,0); rgb(101pt)=(0.677195,0.472696,0); rgb(102pt)=(0.682353,0.47451,0); rgb(103pt)=(0.687242,0.476355,0); rgb(104pt)=(0.691952,0.478225,0); rgb(105pt)=(0.696497,0.480118,0); rgb(106pt)=(0.700887,0.482033,0); rgb(107pt)=(0.705134,0.483968,0); rgb(108pt)=(0.709251,0.485921,0); rgb(109pt)=(0.713249,0.487891,0); rgb(110pt)=(0.71714,0.489876,0); rgb(111pt)=(0.720936,0.491875,0); rgb(112pt)=(0.724649,0.493887,0); rgb(113pt)=(0.72829,0.495909,0); rgb(114pt)=(0.731872,0.49794,0); rgb(115pt)=(0.735406,0.499979,0); rgb(116pt)=(0.738904,0.502025,0); rgb(117pt)=(0.742378,0.504075,0); rgb(118pt)=(0.74584,0.506128,0); rgb(119pt)=(0.749302,0.508182,0); rgb(120pt)=(0.752775,0.510237,0); rgb(121pt)=(0.756272,0.51229,0); rgb(122pt)=(0.759804,0.514339,0); rgb(123pt)=(0.763384,0.516385,0); rgb(124pt)=(0.767022,0.518424,0); rgb(125pt)=(0.770731,0.520455,0); rgb(126pt)=(0.774523,0.522478,0); rgb(127pt)=(0.77841,0.524489,0); rgb(128pt)=(0.782391,0.526491,0); rgb(129pt)=(0.786402,0.528496,0); rgb(130pt)=(0.790431,0.530506,0); rgb(131pt)=(0.794478,0.532521,0); rgb(132pt)=(0.798541,0.534539,0); rgb(133pt)=(0.802619,0.53656,0); rgb(134pt)=(0.806712,0.538584,0); rgb(135pt)=(0.81082,0.540609,0); rgb(136pt)=(0.81494,0.542635,0); rgb(137pt)=(0.819074,0.54466,0); rgb(138pt)=(0.823219,0.546686,0); rgb(139pt)=(0.827374,0.548709,0); rgb(140pt)=(0.831541,0.55073,0); rgb(141pt)=(0.835716,0.552749,0); rgb(142pt)=(0.8399,0.554763,0); rgb(143pt)=(0.844092,0.556774,0); rgb(144pt)=(0.848292,0.558779,0); rgb(145pt)=(0.852497,0.560778,0); rgb(146pt)=(0.856708,0.562771,0); rgb(147pt)=(0.860924,0.564756,0); rgb(148pt)=(0.865143,0.566733,0); rgb(149pt)=(0.869366,0.568701,0); rgb(150pt)=(0.873592,0.57066,0); rgb(151pt)=(0.877819,0.572608,0); rgb(152pt)=(0.882047,0.574545,0); rgb(153pt)=(0.886275,0.576471,0); rgb(154pt)=(0.890659,0.578362,0); rgb(155pt)=(0.895333,0.580203,0); rgb(156pt)=(0.900258,0.581999,0); rgb(157pt)=(0.905397,0.583755,0); rgb(158pt)=(0.910711,0.585479,0); rgb(159pt)=(0.916164,0.587176,0); rgb(160pt)=(0.921717,0.588852,0); rgb(161pt)=(0.927333,0.590513,0); rgb(162pt)=(0.932974,0.592166,0); rgb(163pt)=(0.938602,0.593815,0); rgb(164pt)=(0.94418,0.595468,0); rgb(165pt)=(0.949669,0.59713,0); rgb(166pt)=(0.955033,0.598808,0); rgb(167pt)=(0.960233,0.600507,0); rgb(168pt)=(0.965232,0.602233,0); rgb(169pt)=(0.969992,0.603992,0); rgb(170pt)=(0.974475,0.605791,0); rgb(171pt)=(0.978643,0.607636,0); rgb(172pt)=(0.98246,0.609532,0); rgb(173pt)=(0.985886,0.611486,0); rgb(174pt)=(0.988885,0.613503,0); rgb(175pt)=(0.991419,0.61559,0); rgb(176pt)=(0.99345,0.617753,0); rgb(177pt)=(0.99494,0.619997,0); rgb(178pt)=(0.995851,0.622329,0); rgb(179pt)=(0.996226,0.624763,0); rgb(180pt)=(0.996512,0.627352,0); rgb(181pt)=(0.996788,0.630095,0); rgb(182pt)=(0.997053,0.632982,0); rgb(183pt)=(0.997308,0.636004,0); rgb(184pt)=(0.997552,0.639152,0); rgb(185pt)=(0.997785,0.642416,0); rgb(186pt)=(0.998006,0.645786,0); rgb(187pt)=(0.998217,0.649253,0); rgb(188pt)=(0.998416,0.652807,0); rgb(189pt)=(0.998605,0.656439,0); rgb(190pt)=(0.998781,0.660138,0); rgb(191pt)=(0.998946,0.663897,0); rgb(192pt)=(0.9991,0.667704,0); rgb(193pt)=(0.999242,0.67155,0); rgb(194pt)=(0.999372,0.675427,0); rgb(195pt)=(0.99949,0.679323,0); rgb(196pt)=(0.999596,0.68323,0); rgb(197pt)=(0.99969,0.687139,0); rgb(198pt)=(0.999771,0.691039,0); rgb(199pt)=(0.999841,0.694921,0); rgb(200pt)=(0.999898,0.698775,0); rgb(201pt)=(0.999942,0.702592,0); rgb(202pt)=(0.999974,0.706363,0); rgb(203pt)=(0.999994,0.710077,0); rgb(204pt)=(1,0.713725,0); rgb(205pt)=(1,0.717341,0); rgb(206pt)=(1,0.720963,0); rgb(207pt)=(1,0.724591,0); rgb(208pt)=(1,0.728226,0); rgb(209pt)=(1,0.731867,0); rgb(210pt)=(1,0.735514,0); rgb(211pt)=(1,0.739167,0); rgb(212pt)=(1,0.742827,0); rgb(213pt)=(1,0.746493,0); rgb(214pt)=(1,0.750165,0); rgb(215pt)=(1,0.753843,0); rgb(216pt)=(1,0.757527,0); rgb(217pt)=(1,0.761217,0); rgb(218pt)=(1,0.764913,0); rgb(219pt)=(1,0.768615,0); rgb(220pt)=(1,0.772324,0); rgb(221pt)=(1,0.776038,0); rgb(222pt)=(1,0.779758,0); rgb(223pt)=(1,0.783484,0); rgb(224pt)=(1,0.787215,0); rgb(225pt)=(1,0.790953,0); rgb(226pt)=(1,0.794696,0); rgb(227pt)=(1,0.798445,0); rgb(228pt)=(1,0.8022,0); rgb(229pt)=(1,0.805961,0); rgb(230pt)=(1,0.809727,0); rgb(231pt)=(1,0.8135,0); rgb(232pt)=(1,0.817278,0); rgb(233pt)=(1,0.821063,0); rgb(234pt)=(1,0.824854,0); rgb(235pt)=(1,0.828652,0); rgb(236pt)=(1,0.832455,0); rgb(237pt)=(1,0.836265,0); rgb(238pt)=(1,0.840081,0); rgb(239pt)=(1,0.843903,0); rgb(240pt)=(1,0.847732,0); rgb(241pt)=(1,0.851566,0); rgb(242pt)=(1,0.855406,0); rgb(243pt)=(1,0.859253,0); rgb(244pt)=(1,0.863106,0); rgb(245pt)=(1,0.866964,0); rgb(246pt)=(1,0.870829,0); rgb(247pt)=(1,0.8747,0); rgb(248pt)=(1,0.878577,0); rgb(249pt)=(1,0.88246,0); rgb(250pt)=(1,0.886349,0); rgb(251pt)=(1,0.890243,0); rgb(252pt)=(1,0.894144,0); rgb(253pt)=(1,0.898051,0); rgb(254pt)=(1,0.901964,0); rgb(255pt)=(1,0.905882,0)},
mesh/rows=49]
table[row sep=crcr,header=false] {%
%
57	0.193	-477.450293211973\\
57	0.19466	-495.224370111509\\
57	0.19632	-512.998447011046\\
57	0.19798	-530.772523910582\\
57	0.19964	-548.546600810119\\
57	0.2013	-566.320677709655\\
57	0.20296	-584.094754609192\\
57	0.20462	-601.868831508728\\
57	0.20628	-619.642908408265\\
57	0.20794	-637.416985307801\\
57	0.2096	-655.191062207338\\
57	0.21126	-672.965139106874\\
57	0.21292	-690.739216006411\\
57	0.21458	-708.513292905947\\
57	0.21624	-726.287369805484\\
57	0.2179	-744.06144670502\\
57	0.21956	-761.835523604557\\
57	0.22122	-779.609600504093\\
57	0.22288	-797.38367740363\\
57	0.22454	-815.157754303166\\
57	0.2262	-832.931831202703\\
57	0.22786	-850.705908102239\\
57	0.22952	-868.479985001776\\
57	0.23118	-886.254061901312\\
57	0.23284	-904.028138800849\\
57	0.2345	-921.802215700385\\
57	0.23616	-939.576292599922\\
57	0.23782	-957.350369499459\\
57	0.23948	-975.124446398995\\
57	0.24114	-992.898523298531\\
57	0.2428	-1010.67260019807\\
57	0.24446	-1028.4466770976\\
57	0.24612	-1046.22075399714\\
57	0.24778	-1063.99483089668\\
57	0.24944	-1081.76890779621\\
57	0.2511	-1099.54298469575\\
57	0.25276	-1117.31706159529\\
57	0.25442	-1135.09113849482\\
57	0.25608	-1152.86521539436\\
57	0.25774	-1170.6392922939\\
57	0.2594	-1188.41336919343\\
57	0.26106	-1206.18744609297\\
57	0.26272	-1223.96152299251\\
57	0.26438	-1241.73559989204\\
57	0.26604	-1259.50967679158\\
57	0.2677	-1277.28375369112\\
57	0.26936	-1295.05783059065\\
57	0.27102	-1312.83190749019\\
57	0.27268	-1330.60598438972\\
57	0.27434	-1348.38006128926\\
57	0.276	-1366.1541381888\\
57.2083333333333	0.193	-475.17520734389\\
57.2083333333333	0.19466	-492.97512962209\\
57.2083333333333	0.19632	-510.77505190029\\
57.2083333333333	0.19798	-528.574974178489\\
57.2083333333333	0.19964	-546.374896456689\\
57.2083333333333	0.2013	-564.174818734889\\
57.2083333333333	0.20296	-581.974741013088\\
57.2083333333333	0.20462	-599.774663291288\\
57.2083333333333	0.20628	-617.574585569488\\
57.2083333333333	0.20794	-635.374507847687\\
57.2083333333333	0.2096	-653.174430125887\\
57.2083333333333	0.21126	-670.974352404087\\
57.2083333333333	0.21292	-688.774274682286\\
57.2083333333333	0.21458	-706.574196960486\\
57.2083333333333	0.21624	-724.374119238686\\
57.2083333333333	0.2179	-742.174041516885\\
57.2083333333333	0.21956	-759.973963795085\\
57.2083333333333	0.22122	-777.773886073285\\
57.2083333333333	0.22288	-795.573808351484\\
57.2083333333333	0.22454	-813.373730629684\\
57.2083333333333	0.2262	-831.173652907883\\
57.2083333333333	0.22786	-848.973575186083\\
57.2083333333333	0.22952	-866.773497464283\\
57.2083333333333	0.23118	-884.573419742483\\
57.2083333333333	0.23284	-902.373342020682\\
57.2083333333333	0.2345	-920.173264298882\\
57.2083333333333	0.23616	-937.973186577081\\
57.2083333333333	0.23782	-955.773108855282\\
57.2083333333333	0.23948	-973.573031133481\\
57.2083333333333	0.24114	-991.372953411681\\
57.2083333333333	0.2428	-1009.17287568988\\
57.2083333333333	0.24446	-1026.97279796808\\
57.2083333333333	0.24612	-1044.77272024628\\
57.2083333333333	0.24778	-1062.57264252448\\
57.2083333333333	0.24944	-1080.37256480268\\
57.2083333333333	0.2511	-1098.17248708088\\
57.2083333333333	0.25276	-1115.97240935908\\
57.2083333333333	0.25442	-1133.77233163728\\
57.2083333333333	0.25608	-1151.57225391548\\
57.2083333333333	0.25774	-1169.37217619368\\
57.2083333333333	0.2594	-1187.17209847188\\
57.2083333333333	0.26106	-1204.97202075008\\
57.2083333333333	0.26272	-1222.77194302828\\
57.2083333333333	0.26438	-1240.57186530648\\
57.2083333333333	0.26604	-1258.37178758468\\
57.2083333333333	0.2677	-1276.17170986288\\
57.2083333333333	0.26936	-1293.97163214108\\
57.2083333333333	0.27102	-1311.77155441927\\
57.2083333333333	0.27268	-1329.57147669747\\
57.2083333333333	0.27434	-1347.37139897567\\
57.2083333333333	0.276	-1365.17132125387\\
57.4166666666667	0.193	-472.900121475808\\
57.4166666666667	0.19466	-490.725889132671\\
57.4166666666667	0.19632	-508.551656789534\\
57.4166666666667	0.19798	-526.377424446397\\
57.4166666666667	0.19964	-544.203192103259\\
57.4166666666667	0.2013	-562.028959760122\\
57.4166666666667	0.20296	-579.854727416985\\
57.4166666666667	0.20462	-597.680495073848\\
57.4166666666667	0.20628	-615.506262730711\\
57.4166666666667	0.20794	-633.332030387573\\
57.4166666666667	0.2096	-651.157798044436\\
57.4166666666667	0.21126	-668.983565701299\\
57.4166666666667	0.21292	-686.809333358162\\
57.4166666666667	0.21458	-704.635101015025\\
57.4166666666667	0.21624	-722.460868671888\\
57.4166666666667	0.2179	-740.28663632875\\
57.4166666666667	0.21956	-758.112403985613\\
57.4166666666667	0.22122	-775.938171642476\\
57.4166666666667	0.22288	-793.763939299339\\
57.4166666666667	0.22454	-811.589706956202\\
57.4166666666667	0.2262	-829.415474613064\\
57.4166666666667	0.22786	-847.241242269927\\
57.4166666666667	0.22952	-865.06700992679\\
57.4166666666667	0.23118	-882.892777583653\\
57.4166666666667	0.23284	-900.718545240516\\
57.4166666666667	0.2345	-918.544312897379\\
57.4166666666667	0.23616	-936.370080554241\\
57.4166666666667	0.23782	-954.195848211105\\
57.4166666666667	0.23948	-972.021615867967\\
57.4166666666667	0.24114	-989.84738352483\\
57.4166666666667	0.2428	-1007.67315118169\\
57.4166666666667	0.24446	-1025.49891883856\\
57.4166666666667	0.24612	-1043.32468649542\\
57.4166666666667	0.24778	-1061.15045415228\\
57.4166666666667	0.24944	-1078.97622180914\\
57.4166666666667	0.2511	-1096.80198946601\\
57.4166666666667	0.25276	-1114.62775712287\\
57.4166666666667	0.25442	-1132.45352477973\\
57.4166666666667	0.25608	-1150.2792924366\\
57.4166666666667	0.25774	-1168.10506009346\\
57.4166666666667	0.2594	-1185.93082775032\\
57.4166666666667	0.26106	-1203.75659540718\\
57.4166666666667	0.26272	-1221.58236306405\\
57.4166666666667	0.26438	-1239.40813072091\\
57.4166666666667	0.26604	-1257.23389837777\\
57.4166666666667	0.2677	-1275.05966603464\\
57.4166666666667	0.26936	-1292.8854336915\\
57.4166666666667	0.27102	-1310.71120134836\\
57.4166666666667	0.27268	-1328.53696900522\\
57.4166666666667	0.27434	-1346.36273666209\\
57.4166666666667	0.276	-1364.18850431895\\
57.625	0.193	-470.625035607726\\
57.625	0.19466	-488.476648643251\\
57.625	0.19632	-506.328261678777\\
57.625	0.19798	-524.179874714303\\
57.625	0.19964	-542.031487749829\\
57.625	0.2013	-559.883100785356\\
57.625	0.20296	-577.734713820881\\
57.625	0.20462	-595.586326856407\\
57.625	0.20628	-613.437939891934\\
57.625	0.20794	-631.289552927459\\
57.625	0.2096	-649.141165962985\\
57.625	0.21126	-666.992778998511\\
57.625	0.21292	-684.844392034037\\
57.625	0.21458	-702.696005069563\\
57.625	0.21624	-720.547618105089\\
57.625	0.2179	-738.399231140615\\
57.625	0.21956	-756.250844176141\\
57.625	0.22122	-774.102457211667\\
57.625	0.22288	-791.954070247193\\
57.625	0.22454	-809.805683282719\\
57.625	0.2262	-827.657296318245\\
57.625	0.22786	-845.508909353771\\
57.625	0.22952	-863.360522389297\\
57.625	0.23118	-881.212135424823\\
57.625	0.23284	-899.063748460349\\
57.625	0.2345	-916.915361495875\\
57.625	0.23616	-934.766974531401\\
57.625	0.23782	-952.618587566928\\
57.625	0.23948	-970.470200602453\\
57.625	0.24114	-988.321813637979\\
57.625	0.2428	-1006.17342667351\\
57.625	0.24446	-1024.02503970903\\
57.625	0.24612	-1041.87665274456\\
57.625	0.24778	-1059.72826578008\\
57.625	0.24944	-1077.57987881561\\
57.625	0.2511	-1095.43149185113\\
57.625	0.25276	-1113.28310488666\\
57.625	0.25442	-1131.13471792219\\
57.625	0.25608	-1148.98633095771\\
57.625	0.25774	-1166.83794399324\\
57.625	0.2594	-1184.68955702877\\
57.625	0.26106	-1202.54117006429\\
57.625	0.26272	-1220.39278309982\\
57.625	0.26438	-1238.24439613534\\
57.625	0.26604	-1256.09600917087\\
57.625	0.2677	-1273.9476222064\\
57.625	0.26936	-1291.79923524192\\
57.625	0.27102	-1309.65084827745\\
57.625	0.27268	-1327.50246131297\\
57.625	0.27434	-1345.3540743485\\
57.625	0.276	-1363.20568738402\\
57.8333333333333	0.193	-468.349949739643\\
57.8333333333333	0.19466	-486.227408153832\\
57.8333333333333	0.19632	-504.104866568021\\
57.8333333333333	0.19798	-521.98232498221\\
57.8333333333333	0.19964	-539.859783396399\\
57.8333333333333	0.2013	-557.737241810589\\
57.8333333333333	0.20296	-575.614700224777\\
57.8333333333333	0.20462	-593.492158638967\\
57.8333333333333	0.20628	-611.369617053156\\
57.8333333333333	0.20794	-629.247075467345\\
57.8333333333333	0.2096	-647.124533881534\\
57.8333333333333	0.21126	-665.001992295723\\
57.8333333333333	0.21292	-682.879450709912\\
57.8333333333333	0.21458	-700.756909124102\\
57.8333333333333	0.21624	-718.634367538291\\
57.8333333333333	0.2179	-736.51182595248\\
57.8333333333333	0.21956	-754.389284366669\\
57.8333333333333	0.22122	-772.266742780858\\
57.8333333333333	0.22288	-790.144201195048\\
57.8333333333333	0.22454	-808.021659609237\\
57.8333333333333	0.2262	-825.899118023426\\
57.8333333333333	0.22786	-843.776576437615\\
57.8333333333333	0.22952	-861.654034851804\\
57.8333333333333	0.23118	-879.531493265993\\
57.8333333333333	0.23284	-897.408951680183\\
57.8333333333333	0.2345	-915.286410094371\\
57.8333333333333	0.23616	-933.163868508561\\
57.8333333333333	0.23782	-951.04132692275\\
57.8333333333333	0.23948	-968.918785336939\\
57.8333333333333	0.24114	-986.796243751128\\
57.8333333333333	0.2428	-1004.67370216532\\
57.8333333333333	0.24446	-1022.55116057951\\
57.8333333333333	0.24612	-1040.4286189937\\
57.8333333333333	0.24778	-1058.30607740788\\
57.8333333333333	0.24944	-1076.18353582207\\
57.8333333333333	0.2511	-1094.06099423626\\
57.8333333333333	0.25276	-1111.93845265045\\
57.8333333333333	0.25442	-1129.81591106464\\
57.8333333333333	0.25608	-1147.69336947883\\
57.8333333333333	0.25774	-1165.57082789302\\
57.8333333333333	0.2594	-1183.44828630721\\
57.8333333333333	0.26106	-1201.3257447214\\
57.8333333333333	0.26272	-1219.20320313559\\
57.8333333333333	0.26438	-1237.08066154978\\
57.8333333333333	0.26604	-1254.95811996397\\
57.8333333333333	0.2677	-1272.83557837815\\
57.8333333333333	0.26936	-1290.71303679234\\
57.8333333333333	0.27102	-1308.59049520653\\
57.8333333333333	0.27268	-1326.46795362072\\
57.8333333333333	0.27434	-1344.34541203491\\
57.8333333333333	0.276	-1362.2228704491\\
58.0416666666667	0.193	-466.074863871561\\
58.0416666666667	0.19466	-483.978167664413\\
58.0416666666667	0.19632	-501.881471457265\\
58.0416666666667	0.19798	-519.784775250117\\
58.0416666666667	0.19964	-537.68807904297\\
58.0416666666667	0.2013	-555.591382835822\\
58.0416666666667	0.20296	-573.494686628674\\
58.0416666666667	0.20462	-591.397990421527\\
58.0416666666667	0.20628	-609.301294214379\\
58.0416666666667	0.20794	-627.204598007231\\
58.0416666666667	0.2096	-645.107901800084\\
58.0416666666667	0.21126	-663.011205592936\\
58.0416666666667	0.21292	-680.914509385788\\
58.0416666666667	0.21458	-698.81781317864\\
58.0416666666667	0.21624	-716.721116971493\\
58.0416666666667	0.2179	-734.624420764345\\
58.0416666666667	0.21956	-752.527724557198\\
58.0416666666667	0.22122	-770.43102835005\\
58.0416666666667	0.22288	-788.334332142902\\
58.0416666666667	0.22454	-806.237635935755\\
58.0416666666667	0.2262	-824.140939728607\\
58.0416666666667	0.22786	-842.044243521459\\
58.0416666666667	0.22952	-859.947547314311\\
58.0416666666667	0.23118	-877.850851107164\\
58.0416666666667	0.23284	-895.754154900016\\
58.0416666666667	0.2345	-913.657458692868\\
58.0416666666667	0.23616	-931.560762485721\\
58.0416666666667	0.23782	-949.464066278573\\
58.0416666666667	0.23948	-967.367370071425\\
58.0416666666667	0.24114	-985.270673864278\\
58.0416666666667	0.2428	-1003.17397765713\\
58.0416666666667	0.24446	-1021.07728144998\\
58.0416666666667	0.24612	-1038.98058524283\\
58.0416666666667	0.24778	-1056.88388903569\\
58.0416666666667	0.24944	-1074.78719282854\\
58.0416666666667	0.2511	-1092.69049662139\\
58.0416666666667	0.25276	-1110.59380041424\\
58.0416666666667	0.25442	-1128.4971042071\\
58.0416666666667	0.25608	-1146.40040799995\\
58.0416666666667	0.25774	-1164.3037117928\\
58.0416666666667	0.2594	-1182.20701558565\\
58.0416666666667	0.26106	-1200.11031937851\\
58.0416666666667	0.26272	-1218.01362317136\\
58.0416666666667	0.26438	-1235.91692696421\\
58.0416666666667	0.26604	-1253.82023075706\\
58.0416666666667	0.2677	-1271.72353454992\\
58.0416666666667	0.26936	-1289.62683834277\\
58.0416666666667	0.27102	-1307.53014213562\\
58.0416666666667	0.27268	-1325.43344592847\\
58.0416666666667	0.27434	-1343.33674972132\\
58.0416666666667	0.276	-1361.24005351418\\
58.25	0.193	-463.799778003478\\
58.25	0.19466	-481.728927174993\\
58.25	0.19632	-499.658076346509\\
58.25	0.19798	-517.587225518024\\
58.25	0.19964	-535.51637468954\\
58.25	0.2013	-553.445523861055\\
58.25	0.20296	-571.374673032571\\
58.25	0.20462	-589.303822204086\\
58.25	0.20628	-607.232971375602\\
58.25	0.20794	-625.162120547117\\
58.25	0.2096	-643.091269718633\\
58.25	0.21126	-661.020418890148\\
58.25	0.21292	-678.949568061664\\
58.25	0.21458	-696.878717233179\\
58.25	0.21624	-714.807866404695\\
58.25	0.2179	-732.73701557621\\
58.25	0.21956	-750.666164747726\\
58.25	0.22122	-768.595313919241\\
58.25	0.22288	-786.524463090757\\
58.25	0.22454	-804.453612262272\\
58.25	0.2262	-822.382761433787\\
58.25	0.22786	-840.311910605303\\
58.25	0.22952	-858.241059776818\\
58.25	0.23118	-876.170208948334\\
58.25	0.23284	-894.09935811985\\
58.25	0.2345	-912.028507291365\\
58.25	0.23616	-929.95765646288\\
58.25	0.23782	-947.886805634396\\
58.25	0.23948	-965.815954805911\\
58.25	0.24114	-983.745103977427\\
58.25	0.2428	-1001.67425314894\\
58.25	0.24446	-1019.60340232046\\
58.25	0.24612	-1037.53255149197\\
58.25	0.24778	-1055.46170066349\\
58.25	0.24944	-1073.390849835\\
58.25	0.2511	-1091.31999900652\\
58.25	0.25276	-1109.24914817804\\
58.25	0.25442	-1127.17829734955\\
58.25	0.25608	-1145.10744652107\\
58.25	0.25774	-1163.03659569258\\
58.25	0.2594	-1180.9657448641\\
58.25	0.26106	-1198.89489403561\\
58.25	0.26272	-1216.82404320713\\
58.25	0.26438	-1234.75319237864\\
58.25	0.26604	-1252.68234155016\\
58.25	0.2677	-1270.61149072167\\
58.25	0.26936	-1288.54063989319\\
58.25	0.27102	-1306.46978906471\\
58.25	0.27268	-1324.39893823622\\
58.25	0.27434	-1342.32808740774\\
58.25	0.276	-1360.25723657925\\
58.4583333333333	0.193	-461.524692135395\\
58.4583333333333	0.19466	-479.479686685574\\
58.4583333333333	0.19632	-497.434681235753\\
58.4583333333333	0.19798	-515.389675785931\\
58.4583333333333	0.19964	-533.34467033611\\
58.4583333333333	0.2013	-551.299664886289\\
58.4583333333333	0.20296	-569.254659436467\\
58.4583333333333	0.20462	-587.209653986646\\
58.4583333333333	0.20628	-605.164648536825\\
58.4583333333333	0.20794	-623.119643087003\\
58.4583333333333	0.2096	-641.074637637182\\
58.4583333333333	0.21126	-659.02963218736\\
58.4583333333333	0.21292	-676.984626737539\\
58.4583333333333	0.21458	-694.939621287718\\
58.4583333333333	0.21624	-712.894615837896\\
58.4583333333333	0.2179	-730.849610388075\\
58.4583333333333	0.21956	-748.804604938254\\
58.4583333333333	0.22122	-766.759599488432\\
58.4583333333333	0.22288	-784.714594038611\\
58.4583333333333	0.22454	-802.66958858879\\
58.4583333333333	0.2262	-820.624583138968\\
58.4583333333333	0.22786	-838.579577689147\\
58.4583333333333	0.22952	-856.534572239325\\
58.4583333333333	0.23118	-874.489566789504\\
58.4583333333333	0.23284	-892.444561339683\\
58.4583333333333	0.2345	-910.399555889861\\
58.4583333333333	0.23616	-928.35455044004\\
58.4583333333333	0.23782	-946.309544990219\\
58.4583333333333	0.23948	-964.264539540397\\
58.4583333333333	0.24114	-982.219534090576\\
58.4583333333333	0.2428	-1000.17452864075\\
58.4583333333333	0.24446	-1018.12952319093\\
58.4583333333333	0.24612	-1036.08451774111\\
58.4583333333333	0.24778	-1054.03951229129\\
58.4583333333333	0.24944	-1071.99450684147\\
58.4583333333333	0.2511	-1089.94950139165\\
58.4583333333333	0.25276	-1107.90449594183\\
58.4583333333333	0.25442	-1125.85949049201\\
58.4583333333333	0.25608	-1143.81448504218\\
58.4583333333333	0.25774	-1161.76947959236\\
58.4583333333333	0.2594	-1179.72447414254\\
58.4583333333333	0.26106	-1197.67946869272\\
58.4583333333333	0.26272	-1215.6344632429\\
58.4583333333333	0.26438	-1233.58945779308\\
58.4583333333333	0.26604	-1251.54445234326\\
58.4583333333333	0.2677	-1269.49944689343\\
58.4583333333333	0.26936	-1287.45444144361\\
58.4583333333333	0.27102	-1305.40943599379\\
58.4583333333333	0.27268	-1323.36443054397\\
58.4583333333333	0.27434	-1341.31942509415\\
58.4583333333333	0.276	-1359.27441964433\\
58.6666666666667	0.193	-459.249606267313\\
58.6666666666667	0.19466	-477.230446196155\\
58.6666666666667	0.19632	-495.211286124997\\
58.6666666666667	0.19798	-513.192126053838\\
58.6666666666667	0.19964	-531.17296598268\\
58.6666666666667	0.2013	-549.153805911522\\
58.6666666666667	0.20296	-567.134645840364\\
58.6666666666667	0.20462	-585.115485769206\\
58.6666666666667	0.20628	-603.096325698048\\
58.6666666666667	0.20794	-621.077165626889\\
58.6666666666667	0.2096	-639.058005555731\\
58.6666666666667	0.21126	-657.038845484573\\
58.6666666666667	0.21292	-675.019685413415\\
58.6666666666667	0.21458	-693.000525342256\\
58.6666666666667	0.21624	-710.981365271098\\
58.6666666666667	0.2179	-728.96220519994\\
58.6666666666667	0.21956	-746.943045128782\\
58.6666666666667	0.22122	-764.923885057624\\
58.6666666666667	0.22288	-782.904724986466\\
58.6666666666667	0.22454	-800.885564915308\\
58.6666666666667	0.2262	-818.866404844149\\
58.6666666666667	0.22786	-836.847244772991\\
58.6666666666667	0.22952	-854.828084701833\\
58.6666666666667	0.23118	-872.808924630675\\
58.6666666666667	0.23284	-890.789764559517\\
58.6666666666667	0.2345	-908.770604488358\\
58.6666666666667	0.23616	-926.7514444172\\
58.6666666666667	0.23782	-944.732284346042\\
58.6666666666667	0.23948	-962.713124274884\\
58.6666666666667	0.24114	-980.693964203725\\
58.6666666666667	0.2428	-998.674804132567\\
58.6666666666667	0.24446	-1016.65564406141\\
58.6666666666667	0.24612	-1034.63648399025\\
58.6666666666667	0.24778	-1052.61732391909\\
58.6666666666667	0.24944	-1070.59816384793\\
58.6666666666667	0.2511	-1088.57900377678\\
58.6666666666667	0.25276	-1106.55984370562\\
58.6666666666667	0.25442	-1124.54068363446\\
58.6666666666667	0.25608	-1142.5215235633\\
58.6666666666667	0.25774	-1160.50236349214\\
58.6666666666667	0.2594	-1178.48320342099\\
58.6666666666667	0.26106	-1196.46404334983\\
58.6666666666667	0.26272	-1214.44488327867\\
58.6666666666667	0.26438	-1232.42572320751\\
58.6666666666667	0.26604	-1250.40656313635\\
58.6666666666667	0.2677	-1268.38740306519\\
58.6666666666667	0.26936	-1286.36824299404\\
58.6666666666667	0.27102	-1304.34908292288\\
58.6666666666667	0.27268	-1322.32992285172\\
58.6666666666667	0.27434	-1340.31076278056\\
58.6666666666667	0.276	-1358.2916027094\\
58.875	0.193	-456.97452039923\\
58.875	0.19466	-474.981205706735\\
58.875	0.19632	-492.98789101424\\
58.875	0.19798	-510.994576321745\\
58.875	0.19964	-529.00126162925\\
58.875	0.2013	-547.007946936755\\
58.875	0.20296	-565.01463224426\\
58.875	0.20462	-583.021317551765\\
58.875	0.20628	-601.02800285927\\
58.875	0.20794	-619.034688166775\\
58.875	0.2096	-637.04137347428\\
58.875	0.21126	-655.048058781785\\
58.875	0.21292	-673.05474408929\\
58.875	0.21458	-691.061429396795\\
58.875	0.21624	-709.0681147043\\
58.875	0.2179	-727.074800011805\\
58.875	0.21956	-745.08148531931\\
58.875	0.22122	-763.088170626815\\
58.875	0.22288	-781.09485593432\\
58.875	0.22454	-799.101541241825\\
58.875	0.2262	-817.10822654933\\
58.875	0.22786	-835.114911856835\\
58.875	0.22952	-853.12159716434\\
58.875	0.23118	-871.128282471845\\
58.875	0.23284	-889.13496777935\\
58.875	0.2345	-907.141653086855\\
58.875	0.23616	-925.14833839436\\
58.875	0.23782	-943.155023701865\\
58.875	0.23948	-961.16170900937\\
58.875	0.24114	-979.168394316875\\
58.875	0.2428	-997.17507962438\\
58.875	0.24446	-1015.18176493188\\
58.875	0.24612	-1033.18845023939\\
58.875	0.24778	-1051.19513554689\\
58.875	0.24944	-1069.2018208544\\
58.875	0.2511	-1087.2085061619\\
58.875	0.25276	-1105.21519146941\\
58.875	0.25442	-1123.22187677691\\
58.875	0.25608	-1141.22856208442\\
58.875	0.25774	-1159.23524739192\\
58.875	0.2594	-1177.24193269943\\
58.875	0.26106	-1195.24861800693\\
58.875	0.26272	-1213.25530331444\\
58.875	0.26438	-1231.26198862194\\
58.875	0.26604	-1249.26867392945\\
58.875	0.2677	-1267.27535923695\\
58.875	0.26936	-1285.28204454446\\
58.875	0.27102	-1303.28872985196\\
58.875	0.27268	-1321.29541515947\\
58.875	0.27434	-1339.30210046697\\
58.875	0.276	-1357.30878577448\\
59.0833333333333	0.193	-454.699434531148\\
59.0833333333333	0.19466	-472.731965217316\\
59.0833333333333	0.19632	-490.764495903484\\
59.0833333333333	0.19798	-508.797026589652\\
59.0833333333333	0.19964	-526.82955727582\\
59.0833333333333	0.2013	-544.862087961989\\
59.0833333333333	0.20296	-562.894618648156\\
59.0833333333333	0.20462	-580.927149334325\\
59.0833333333333	0.20628	-598.959680020493\\
59.0833333333333	0.20794	-616.992210706661\\
59.0833333333333	0.2096	-635.024741392829\\
59.0833333333333	0.21126	-653.057272078997\\
59.0833333333333	0.21292	-671.089802765165\\
59.0833333333333	0.21458	-689.122333451333\\
59.0833333333333	0.21624	-707.154864137502\\
59.0833333333333	0.2179	-725.18739482367\\
59.0833333333333	0.21956	-743.219925509838\\
59.0833333333333	0.22122	-761.252456196006\\
59.0833333333333	0.22288	-779.284986882174\\
59.0833333333333	0.22454	-797.317517568343\\
59.0833333333333	0.2262	-815.35004825451\\
59.0833333333333	0.22786	-833.382578940679\\
59.0833333333333	0.22952	-851.415109626847\\
59.0833333333333	0.23118	-869.447640313015\\
59.0833333333333	0.23284	-887.480170999183\\
59.0833333333333	0.2345	-905.512701685351\\
59.0833333333333	0.23616	-923.545232371519\\
59.0833333333333	0.23782	-941.577763057688\\
59.0833333333333	0.23948	-959.610293743856\\
59.0833333333333	0.24114	-977.642824430024\\
59.0833333333333	0.2428	-995.675355116192\\
59.0833333333333	0.24446	-1013.70788580236\\
59.0833333333333	0.24612	-1031.74041648853\\
59.0833333333333	0.24778	-1049.7729471747\\
59.0833333333333	0.24944	-1067.80547786086\\
59.0833333333333	0.2511	-1085.83800854703\\
59.0833333333333	0.25276	-1103.8705392332\\
59.0833333333333	0.25442	-1121.90306991937\\
59.0833333333333	0.25608	-1139.93560060554\\
59.0833333333333	0.25774	-1157.96813129171\\
59.0833333333333	0.2594	-1176.00066197787\\
59.0833333333333	0.26106	-1194.03319266404\\
59.0833333333333	0.26272	-1212.06572335021\\
59.0833333333333	0.26438	-1230.09825403638\\
59.0833333333333	0.26604	-1248.13078472255\\
59.0833333333333	0.2677	-1266.16331540871\\
59.0833333333333	0.26936	-1284.19584609488\\
59.0833333333333	0.27102	-1302.22837678105\\
59.0833333333333	0.27268	-1320.26090746722\\
59.0833333333333	0.27434	-1338.29343815339\\
59.0833333333333	0.276	-1356.32596883955\\
59.2916666666667	0.193	-452.424348663066\\
59.2916666666667	0.19466	-470.482724727897\\
59.2916666666667	0.19632	-488.541100792728\\
59.2916666666667	0.19798	-506.599476857559\\
59.2916666666667	0.19964	-524.657852922391\\
59.2916666666667	0.2013	-542.716228987222\\
59.2916666666667	0.20296	-560.774605052053\\
59.2916666666667	0.20462	-578.832981116884\\
59.2916666666667	0.20628	-596.891357181716\\
59.2916666666667	0.20794	-614.949733246547\\
59.2916666666667	0.2096	-633.008109311378\\
59.2916666666667	0.21126	-651.06648537621\\
59.2916666666667	0.21292	-669.124861441041\\
59.2916666666667	0.21458	-687.183237505872\\
59.2916666666667	0.21624	-705.241613570704\\
59.2916666666667	0.2179	-723.299989635535\\
59.2916666666667	0.21956	-741.358365700366\\
59.2916666666667	0.22122	-759.416741765198\\
59.2916666666667	0.22288	-777.475117830029\\
59.2916666666667	0.22454	-795.53349389486\\
59.2916666666667	0.2262	-813.591869959691\\
59.2916666666667	0.22786	-831.650246024523\\
59.2916666666667	0.22952	-849.708622089354\\
59.2916666666667	0.23118	-867.766998154186\\
59.2916666666667	0.23284	-885.825374219017\\
59.2916666666667	0.2345	-903.883750283848\\
59.2916666666667	0.23616	-921.942126348679\\
59.2916666666667	0.23782	-940.000502413511\\
59.2916666666667	0.23948	-958.058878478342\\
59.2916666666667	0.24114	-976.117254543173\\
59.2916666666667	0.2428	-994.175630608005\\
59.2916666666667	0.24446	-1012.23400667284\\
59.2916666666667	0.24612	-1030.29238273767\\
59.2916666666667	0.24778	-1048.3507588025\\
59.2916666666667	0.24944	-1066.40913486733\\
59.2916666666667	0.2511	-1084.46751093216\\
59.2916666666667	0.25276	-1102.52588699699\\
59.2916666666667	0.25442	-1120.58426306182\\
59.2916666666667	0.25608	-1138.64263912666\\
59.2916666666667	0.25774	-1156.70101519149\\
59.2916666666667	0.2594	-1174.75939125632\\
59.2916666666667	0.26106	-1192.81776732115\\
59.2916666666667	0.26272	-1210.87614338598\\
59.2916666666667	0.26438	-1228.93451945081\\
59.2916666666667	0.26604	-1246.99289551564\\
59.2916666666667	0.2677	-1265.05127158047\\
59.2916666666667	0.26936	-1283.10964764531\\
59.2916666666667	0.27102	-1301.16802371014\\
59.2916666666667	0.27268	-1319.22639977497\\
59.2916666666667	0.27434	-1337.2847758398\\
59.2916666666667	0.276	-1355.34315190463\\
59.5	0.193	-450.149262794983\\
59.5	0.19466	-468.233484238477\\
59.5	0.19632	-486.317705681972\\
59.5	0.19798	-504.401927125466\\
59.5	0.19964	-522.486148568961\\
59.5	0.2013	-540.570370012455\\
59.5	0.20296	-558.65459145595\\
59.5	0.20462	-576.738812899444\\
59.5	0.20628	-594.823034342939\\
59.5	0.20794	-612.907255786433\\
59.5	0.2096	-630.991477229928\\
59.5	0.21126	-649.075698673422\\
59.5	0.21292	-667.159920116916\\
59.5	0.21458	-685.244141560411\\
59.5	0.21624	-703.328363003905\\
59.5	0.2179	-721.4125844474\\
59.5	0.21956	-739.496805890894\\
59.5	0.22122	-757.581027334389\\
59.5	0.22288	-775.665248777883\\
59.5	0.22454	-793.749470221378\\
59.5	0.2262	-811.833691664872\\
59.5	0.22786	-829.917913108367\\
59.5	0.22952	-848.002134551861\\
59.5	0.23118	-866.086355995356\\
59.5	0.23284	-884.17057743885\\
59.5	0.2345	-902.254798882344\\
59.5	0.23616	-920.339020325839\\
59.5	0.23782	-938.423241769334\\
59.5	0.23948	-956.507463212828\\
59.5	0.24114	-974.591684656323\\
59.5	0.2428	-992.675906099817\\
59.5	0.24446	-1010.76012754331\\
59.5	0.24612	-1028.84434898681\\
59.5	0.24778	-1046.9285704303\\
59.5	0.24944	-1065.01279187379\\
59.5	0.2511	-1083.09701331729\\
59.5	0.25276	-1101.18123476078\\
59.5	0.25442	-1119.26545620428\\
59.5	0.25608	-1137.34967764777\\
59.5	0.25774	-1155.43389909127\\
59.5	0.2594	-1173.51812053476\\
59.5	0.26106	-1191.60234197826\\
59.5	0.26272	-1209.68656342175\\
59.5	0.26438	-1227.77078486524\\
59.5	0.26604	-1245.85500630874\\
59.5	0.2677	-1263.93922775223\\
59.5	0.26936	-1282.02344919573\\
59.5	0.27102	-1300.10767063922\\
59.5	0.27268	-1318.19189208272\\
59.5	0.27434	-1336.27611352621\\
59.5	0.276	-1354.36033496971\\
59.7083333333333	0.193	-447.874176926901\\
59.7083333333333	0.19466	-465.984243749058\\
59.7083333333333	0.19632	-484.094310571216\\
59.7083333333333	0.19798	-502.204377393374\\
59.7083333333333	0.19964	-520.314444215531\\
59.7083333333333	0.2013	-538.424511037689\\
59.7083333333333	0.20296	-556.534577859846\\
59.7083333333333	0.20462	-574.644644682004\\
59.7083333333333	0.20628	-592.754711504162\\
59.7083333333333	0.20794	-610.864778326319\\
59.7083333333333	0.2096	-628.974845148477\\
59.7083333333333	0.21126	-647.084911970635\\
59.7083333333333	0.21292	-665.194978792792\\
59.7083333333333	0.21458	-683.30504561495\\
59.7083333333333	0.21624	-701.415112437107\\
59.7083333333333	0.2179	-719.525179259265\\
59.7083333333333	0.21956	-737.635246081423\\
59.7083333333333	0.22122	-755.74531290358\\
59.7083333333333	0.22288	-773.855379725738\\
59.7083333333333	0.22454	-791.965446547896\\
59.7083333333333	0.2262	-810.075513370053\\
59.7083333333333	0.22786	-828.185580192211\\
59.7083333333333	0.22952	-846.295647014368\\
59.7083333333333	0.23118	-864.405713836526\\
59.7083333333333	0.23284	-882.515780658684\\
59.7083333333333	0.2345	-900.625847480841\\
59.7083333333333	0.23616	-918.735914302999\\
59.7083333333333	0.23782	-936.845981125157\\
59.7083333333333	0.23948	-954.956047947314\\
59.7083333333333	0.24114	-973.066114769472\\
59.7083333333333	0.2428	-991.17618159163\\
59.7083333333333	0.24446	-1009.28624841379\\
59.7083333333333	0.24612	-1027.39631523594\\
59.7083333333333	0.24778	-1045.5063820581\\
59.7083333333333	0.24944	-1063.61644888026\\
59.7083333333333	0.2511	-1081.72651570242\\
59.7083333333333	0.25276	-1099.83658252458\\
59.7083333333333	0.25442	-1117.94664934673\\
59.7083333333333	0.25608	-1136.05671616889\\
59.7083333333333	0.25774	-1154.16678299105\\
59.7083333333333	0.2594	-1172.27684981321\\
59.7083333333333	0.26106	-1190.38691663536\\
59.7083333333333	0.26272	-1208.49698345752\\
59.7083333333333	0.26438	-1226.60705027968\\
59.7083333333333	0.26604	-1244.71711710184\\
59.7083333333333	0.2677	-1262.82718392399\\
59.7083333333333	0.26936	-1280.93725074615\\
59.7083333333333	0.27102	-1299.04731756831\\
59.7083333333333	0.27268	-1317.15738439047\\
59.7083333333333	0.27434	-1335.26745121262\\
59.7083333333333	0.276	-1353.37751803478\\
59.9166666666667	0.193	-445.599091058818\\
59.9166666666667	0.19466	-463.735003259639\\
59.9166666666667	0.19632	-481.87091546046\\
59.9166666666667	0.19798	-500.00682766128\\
59.9166666666667	0.19964	-518.142739862101\\
59.9166666666667	0.2013	-536.278652062922\\
59.9166666666667	0.20296	-554.414564263743\\
59.9166666666667	0.20462	-572.550476464564\\
59.9166666666667	0.20628	-590.686388665385\\
59.9166666666667	0.20794	-608.822300866205\\
59.9166666666667	0.2096	-626.958213067026\\
59.9166666666667	0.21126	-645.094125267847\\
59.9166666666667	0.21292	-663.230037468667\\
59.9166666666667	0.21458	-681.365949669488\\
59.9166666666667	0.21624	-699.501861870309\\
59.9166666666667	0.2179	-717.63777407113\\
59.9166666666667	0.21956	-735.773686271951\\
59.9166666666667	0.22122	-753.909598472772\\
59.9166666666667	0.22288	-772.045510673593\\
59.9166666666667	0.22454	-790.181422874413\\
59.9166666666667	0.2262	-808.317335075234\\
59.9166666666667	0.22786	-826.453247276055\\
59.9166666666667	0.22952	-844.589159476875\\
59.9166666666667	0.23118	-862.725071677696\\
59.9166666666667	0.23284	-880.860983878517\\
59.9166666666667	0.2345	-898.996896079338\\
59.9166666666667	0.23616	-917.132808280159\\
59.9166666666667	0.23782	-935.26872048098\\
59.9166666666667	0.23948	-953.4046326818\\
59.9166666666667	0.24114	-971.540544882621\\
59.9166666666667	0.2428	-989.676457083442\\
59.9166666666667	0.24446	-1007.81236928426\\
59.9166666666667	0.24612	-1025.94828148508\\
59.9166666666667	0.24778	-1044.0841936859\\
59.9166666666667	0.24944	-1062.22010588673\\
59.9166666666667	0.2511	-1080.35601808755\\
59.9166666666667	0.25276	-1098.49193028837\\
59.9166666666667	0.25442	-1116.62784248919\\
59.9166666666667	0.25608	-1134.76375469001\\
59.9166666666667	0.25774	-1152.89966689083\\
59.9166666666667	0.2594	-1171.03557909165\\
59.9166666666667	0.26106	-1189.17149129247\\
59.9166666666667	0.26272	-1207.30740349329\\
59.9166666666667	0.26438	-1225.44331569411\\
59.9166666666667	0.26604	-1243.57922789493\\
59.9166666666667	0.2677	-1261.71514009575\\
59.9166666666667	0.26936	-1279.85105229657\\
59.9166666666667	0.27102	-1297.9869644974\\
59.9166666666667	0.27268	-1316.12287669822\\
59.9166666666667	0.27434	-1334.25878889904\\
59.9166666666667	0.276	-1352.39470109986\\
60.125	0.193	-443.324005190736\\
60.125	0.19466	-461.48576277022\\
60.125	0.19632	-479.647520349704\\
60.125	0.19798	-497.809277929188\\
60.125	0.19964	-515.971035508672\\
60.125	0.2013	-534.132793088156\\
60.125	0.20296	-552.294550667639\\
60.125	0.20462	-570.456308247123\\
60.125	0.20628	-588.618065826608\\
60.125	0.20794	-606.779823406091\\
60.125	0.2096	-624.941580985575\\
60.125	0.21126	-643.103338565059\\
60.125	0.21292	-661.265096144543\\
60.125	0.21458	-679.426853724027\\
60.125	0.21624	-697.588611303511\\
60.125	0.2179	-715.750368882995\\
60.125	0.21956	-733.912126462479\\
60.125	0.22122	-752.073884041963\\
60.125	0.22288	-770.235641621447\\
60.125	0.22454	-788.397399200931\\
60.125	0.2262	-806.559156780415\\
60.125	0.22786	-824.720914359899\\
60.125	0.22952	-842.882671939383\\
60.125	0.23118	-861.044429518867\\
60.125	0.23284	-879.206187098351\\
60.125	0.2345	-897.367944677835\\
60.125	0.23616	-915.529702257319\\
60.125	0.23782	-933.691459836803\\
60.125	0.23948	-951.853217416287\\
60.125	0.24114	-970.014974995771\\
60.125	0.2428	-988.176732575255\\
60.125	0.24446	-1006.33849015474\\
60.125	0.24612	-1024.50024773422\\
60.125	0.24778	-1042.66200531371\\
60.125	0.24944	-1060.82376289319\\
60.125	0.2511	-1078.98552047267\\
60.125	0.25276	-1097.14727805216\\
60.125	0.25442	-1115.30903563164\\
60.125	0.25608	-1133.47079321113\\
60.125	0.25774	-1151.63255079061\\
60.125	0.2594	-1169.79430837009\\
60.125	0.26106	-1187.95606594958\\
60.125	0.26272	-1206.11782352906\\
60.125	0.26438	-1224.27958110855\\
60.125	0.26604	-1242.44133868803\\
60.125	0.2677	-1260.60309626751\\
60.125	0.26936	-1278.764853847\\
60.125	0.27102	-1296.92661142648\\
60.125	0.27268	-1315.08836900597\\
60.125	0.27434	-1333.25012658545\\
60.125	0.276	-1351.41188416493\\
60.3333333333333	0.193	-441.048919322653\\
60.3333333333333	0.19466	-459.2365222808\\
60.3333333333333	0.19632	-477.424125238947\\
60.3333333333333	0.19798	-495.611728197095\\
60.3333333333333	0.19964	-513.799331155242\\
60.3333333333333	0.2013	-531.986934113389\\
60.3333333333333	0.20296	-550.174537071536\\
60.3333333333333	0.20462	-568.362140029683\\
60.3333333333333	0.20628	-586.54974298783\\
60.3333333333333	0.20794	-604.737345945977\\
60.3333333333333	0.2096	-622.924948904124\\
60.3333333333333	0.21126	-641.112551862272\\
60.3333333333333	0.21292	-659.300154820418\\
60.3333333333333	0.21458	-677.487757778566\\
60.3333333333333	0.21624	-695.675360736713\\
60.3333333333333	0.2179	-713.86296369486\\
60.3333333333333	0.21956	-732.050566653007\\
60.3333333333333	0.22122	-750.238169611154\\
60.3333333333333	0.22288	-768.425772569302\\
60.3333333333333	0.22454	-786.613375527449\\
60.3333333333333	0.2262	-804.800978485596\\
60.3333333333333	0.22786	-822.988581443743\\
60.3333333333333	0.22952	-841.17618440189\\
60.3333333333333	0.23118	-859.363787360037\\
60.3333333333333	0.23284	-877.551390318184\\
60.3333333333333	0.2345	-895.738993276331\\
60.3333333333333	0.23616	-913.926596234478\\
60.3333333333333	0.23782	-932.114199192626\\
60.3333333333333	0.23948	-950.301802150773\\
60.3333333333333	0.24114	-968.48940510892\\
60.3333333333333	0.2428	-986.677008067067\\
60.3333333333333	0.24446	-1004.86461102521\\
60.3333333333333	0.24612	-1023.05221398336\\
60.3333333333333	0.24778	-1041.23981694151\\
60.3333333333333	0.24944	-1059.42741989966\\
60.3333333333333	0.2511	-1077.6150228578\\
60.3333333333333	0.25276	-1095.80262581595\\
60.3333333333333	0.25442	-1113.9902287741\\
60.3333333333333	0.25608	-1132.17783173224\\
60.3333333333333	0.25774	-1150.36543469039\\
60.3333333333333	0.2594	-1168.55303764854\\
60.3333333333333	0.26106	-1186.74064060669\\
60.3333333333333	0.26272	-1204.92824356483\\
60.3333333333333	0.26438	-1223.11584652298\\
60.3333333333333	0.26604	-1241.30344948113\\
60.3333333333333	0.2677	-1259.49105243927\\
60.3333333333333	0.26936	-1277.67865539742\\
60.3333333333333	0.27102	-1295.86625835557\\
60.3333333333333	0.27268	-1314.05386131372\\
60.3333333333333	0.27434	-1332.24146427186\\
60.3333333333333	0.276	-1350.42906723001\\
60.5416666666667	0.193	-438.773833454571\\
60.5416666666667	0.19466	-456.987281791381\\
60.5416666666667	0.19632	-475.200730128191\\
60.5416666666667	0.19798	-493.414178465001\\
60.5416666666667	0.19964	-511.627626801812\\
60.5416666666667	0.2013	-529.841075138622\\
60.5416666666667	0.20296	-548.054523475432\\
60.5416666666667	0.20462	-566.267971812242\\
60.5416666666667	0.20628	-584.481420149053\\
60.5416666666667	0.20794	-602.694868485863\\
60.5416666666667	0.2096	-620.908316822673\\
60.5416666666667	0.21126	-639.121765159484\\
60.5416666666667	0.21292	-657.335213496294\\
60.5416666666667	0.21458	-675.548661833104\\
60.5416666666667	0.21624	-693.762110169914\\
60.5416666666667	0.2179	-711.975558506725\\
60.5416666666667	0.21956	-730.189006843535\\
60.5416666666667	0.22122	-748.402455180345\\
60.5416666666667	0.22288	-766.615903517156\\
60.5416666666667	0.22454	-784.829351853966\\
60.5416666666667	0.2262	-803.042800190776\\
60.5416666666667	0.22786	-821.256248527586\\
60.5416666666667	0.22952	-839.469696864397\\
60.5416666666667	0.23118	-857.683145201207\\
60.5416666666667	0.23284	-875.896593538017\\
60.5416666666667	0.2345	-894.110041874828\\
60.5416666666667	0.23616	-912.323490211638\\
60.5416666666667	0.23782	-930.536938548448\\
60.5416666666667	0.23948	-948.750386885258\\
60.5416666666667	0.24114	-966.963835222069\\
60.5416666666667	0.2428	-985.177283558879\\
60.5416666666667	0.24446	-1003.39073189569\\
60.5416666666667	0.24612	-1021.6041802325\\
60.5416666666667	0.24778	-1039.81762856931\\
60.5416666666667	0.24944	-1058.03107690612\\
60.5416666666667	0.2511	-1076.24452524293\\
60.5416666666667	0.25276	-1094.45797357974\\
60.5416666666667	0.25442	-1112.67142191655\\
60.5416666666667	0.25608	-1130.88487025336\\
60.5416666666667	0.25774	-1149.09831859017\\
60.5416666666667	0.2594	-1167.31176692698\\
60.5416666666667	0.26106	-1185.52521526379\\
60.5416666666667	0.26272	-1203.7386636006\\
60.5416666666667	0.26438	-1221.95211193741\\
60.5416666666667	0.26604	-1240.16556027422\\
60.5416666666667	0.2677	-1258.37900861103\\
60.5416666666667	0.26936	-1276.59245694784\\
60.5416666666667	0.27102	-1294.80590528465\\
60.5416666666667	0.27268	-1313.01935362146\\
60.5416666666667	0.27434	-1331.23280195827\\
60.5416666666667	0.276	-1349.44625029508\\
60.75	0.193	-436.498747586489\\
60.75	0.19466	-454.738041301962\\
60.75	0.19632	-472.977335017435\\
60.75	0.19798	-491.216628732909\\
60.75	0.19964	-509.455922448382\\
60.75	0.2013	-527.695216163856\\
60.75	0.20296	-545.934509879329\\
60.75	0.20462	-564.173803594802\\
60.75	0.20628	-582.413097310276\\
60.75	0.20794	-600.652391025749\\
60.75	0.2096	-618.891684741223\\
60.75	0.21126	-637.130978456696\\
60.75	0.21292	-655.37027217217\\
60.75	0.21458	-673.609565887643\\
60.75	0.21624	-691.848859603117\\
60.75	0.2179	-710.08815331859\\
60.75	0.21956	-728.327447034064\\
60.75	0.22122	-746.566740749537\\
60.75	0.22288	-764.806034465011\\
60.75	0.22454	-783.045328180484\\
60.75	0.2262	-801.284621895957\\
60.75	0.22786	-819.523915611431\\
60.75	0.22952	-837.763209326904\\
60.75	0.23118	-856.002503042378\\
60.75	0.23284	-874.241796757851\\
60.75	0.2345	-892.481090473324\\
60.75	0.23616	-910.720384188798\\
60.75	0.23782	-928.959677904272\\
60.75	0.23948	-947.198971619745\\
60.75	0.24114	-965.438265335219\\
60.75	0.2428	-983.677559050692\\
60.75	0.24446	-1001.91685276617\\
60.75	0.24612	-1020.15614648164\\
60.75	0.24778	-1038.39544019711\\
60.75	0.24944	-1056.63473391259\\
60.75	0.2511	-1074.87402762806\\
60.75	0.25276	-1093.11332134353\\
60.75	0.25442	-1111.35261505901\\
60.75	0.25608	-1129.59190877448\\
60.75	0.25774	-1147.83120248995\\
60.75	0.2594	-1166.07049620543\\
60.75	0.26106	-1184.3097899209\\
60.75	0.26272	-1202.54908363637\\
60.75	0.26438	-1220.78837735185\\
60.75	0.26604	-1239.02767106732\\
60.75	0.2677	-1257.26696478279\\
60.75	0.26936	-1275.50625849827\\
60.75	0.27102	-1293.74555221374\\
60.75	0.27268	-1311.98484592921\\
60.75	0.27434	-1330.22413964469\\
60.75	0.276	-1348.46343336016\\
60.9583333333333	0.193	-434.223661718406\\
60.9583333333333	0.19466	-452.488800812542\\
60.9583333333333	0.19632	-470.753939906679\\
60.9583333333333	0.19798	-489.019079000816\\
60.9583333333333	0.19964	-507.284218094952\\
60.9583333333333	0.2013	-525.549357189089\\
60.9583333333333	0.20296	-543.814496283225\\
60.9583333333333	0.20462	-562.079635377362\\
60.9583333333333	0.20628	-580.344774471499\\
60.9583333333333	0.20794	-598.609913565635\\
60.9583333333333	0.2096	-616.875052659772\\
60.9583333333333	0.21126	-635.140191753908\\
60.9583333333333	0.21292	-653.405330848045\\
60.9583333333333	0.21458	-671.670469942182\\
60.9583333333333	0.21624	-689.935609036318\\
60.9583333333333	0.2179	-708.200748130455\\
60.9583333333333	0.21956	-726.465887224591\\
60.9583333333333	0.22122	-744.731026318728\\
60.9583333333333	0.22288	-762.996165412865\\
60.9583333333333	0.22454	-781.261304507001\\
60.9583333333333	0.2262	-799.526443601138\\
60.9583333333333	0.22786	-817.791582695274\\
60.9583333333333	0.22952	-836.056721789411\\
60.9583333333333	0.23118	-854.321860883548\\
60.9583333333333	0.23284	-872.586999977684\\
60.9583333333333	0.2345	-890.852139071821\\
60.9583333333333	0.23616	-909.117278165957\\
60.9583333333333	0.23782	-927.382417260094\\
60.9583333333333	0.23948	-945.647556354231\\
60.9583333333333	0.24114	-963.912695448367\\
60.9583333333333	0.2428	-982.177834542504\\
60.9583333333333	0.24446	-1000.44297363664\\
60.9583333333333	0.24612	-1018.70811273078\\
60.9583333333333	0.24778	-1036.97325182491\\
60.9583333333333	0.24944	-1055.23839091905\\
60.9583333333333	0.2511	-1073.50353001319\\
60.9583333333333	0.25276	-1091.76866910732\\
60.9583333333333	0.25442	-1110.03380820146\\
60.9583333333333	0.25608	-1128.2989472956\\
60.9583333333333	0.25774	-1146.56408638973\\
60.9583333333333	0.2594	-1164.82922548387\\
60.9583333333333	0.26106	-1183.09436457801\\
60.9583333333333	0.26272	-1201.35950367214\\
60.9583333333333	0.26438	-1219.62464276628\\
60.9583333333333	0.26604	-1237.88978186042\\
60.9583333333333	0.2677	-1256.15492095455\\
60.9583333333333	0.26936	-1274.42006004869\\
60.9583333333333	0.27102	-1292.68519914283\\
60.9583333333333	0.27268	-1310.95033823696\\
60.9583333333333	0.27434	-1329.2154773311\\
60.9583333333333	0.276	-1347.48061642524\\
61.1666666666667	0.193	-431.948575850324\\
61.1666666666667	0.19466	-450.239560323123\\
61.1666666666667	0.19632	-468.530544795923\\
61.1666666666667	0.19798	-486.821529268723\\
61.1666666666667	0.19964	-505.112513741523\\
61.1666666666667	0.2013	-523.403498214322\\
61.1666666666667	0.20296	-541.694482687122\\
61.1666666666667	0.20462	-559.985467159922\\
61.1666666666667	0.20628	-578.276451632722\\
61.1666666666667	0.20794	-596.567436105521\\
61.1666666666667	0.2096	-614.858420578321\\
61.1666666666667	0.21126	-633.149405051121\\
61.1666666666667	0.21292	-651.440389523921\\
61.1666666666667	0.21458	-669.73137399672\\
61.1666666666667	0.21624	-688.02235846952\\
61.1666666666667	0.2179	-706.31334294232\\
61.1666666666667	0.21956	-724.60432741512\\
61.1666666666667	0.22122	-742.89531188792\\
61.1666666666667	0.22288	-761.18629636072\\
61.1666666666667	0.22454	-779.477280833519\\
61.1666666666667	0.2262	-797.768265306319\\
61.1666666666667	0.22786	-816.059249779119\\
61.1666666666667	0.22952	-834.350234251918\\
61.1666666666667	0.23118	-852.641218724718\\
61.1666666666667	0.23284	-870.932203197518\\
61.1666666666667	0.2345	-889.223187670318\\
61.1666666666667	0.23616	-907.514172143118\\
61.1666666666667	0.23782	-925.805156615918\\
61.1666666666667	0.23948	-944.096141088717\\
61.1666666666667	0.24114	-962.387125561517\\
61.1666666666667	0.2428	-980.678110034317\\
61.1666666666667	0.24446	-998.969094507116\\
61.1666666666667	0.24612	-1017.26007897992\\
61.1666666666667	0.24778	-1035.55106345272\\
61.1666666666667	0.24944	-1053.84204792552\\
61.1666666666667	0.2511	-1072.13303239832\\
61.1666666666667	0.25276	-1090.42401687112\\
61.1666666666667	0.25442	-1108.71500134392\\
61.1666666666667	0.25608	-1127.00598581672\\
61.1666666666667	0.25774	-1145.29697028951\\
61.1666666666667	0.2594	-1163.58795476231\\
61.1666666666667	0.26106	-1181.87893923511\\
61.1666666666667	0.26272	-1200.16992370791\\
61.1666666666667	0.26438	-1218.46090818071\\
61.1666666666667	0.26604	-1236.75189265351\\
61.1666666666667	0.2677	-1255.04287712631\\
61.1666666666667	0.26936	-1273.33386159911\\
61.1666666666667	0.27102	-1291.62484607191\\
61.1666666666667	0.27268	-1309.91583054471\\
61.1666666666667	0.27434	-1328.20681501751\\
61.1666666666667	0.276	-1346.49779949031\\
61.375	0.193	-429.673489982241\\
61.375	0.19466	-447.990319833704\\
61.375	0.19632	-466.307149685167\\
61.375	0.19798	-484.62397953663\\
61.375	0.19964	-502.940809388093\\
61.375	0.2013	-521.257639239556\\
61.375	0.20296	-539.574469091018\\
61.375	0.20462	-557.891298942481\\
61.375	0.20628	-576.208128793945\\
61.375	0.20794	-594.524958645407\\
61.375	0.2096	-612.84178849687\\
61.375	0.21126	-631.158618348333\\
61.375	0.21292	-649.475448199796\\
61.375	0.21458	-667.792278051259\\
61.375	0.21624	-686.109107902722\\
61.375	0.2179	-704.425937754185\\
61.375	0.21956	-722.742767605648\\
61.375	0.22122	-741.059597457111\\
61.375	0.22288	-759.376427308574\\
61.375	0.22454	-777.693257160037\\
61.375	0.2262	-796.0100870115\\
61.375	0.22786	-814.326916862962\\
61.375	0.22952	-832.643746714426\\
61.375	0.23118	-850.960576565888\\
61.375	0.23284	-869.277406417352\\
61.375	0.2345	-887.594236268814\\
61.375	0.23616	-905.911066120277\\
61.375	0.23782	-924.227895971741\\
61.375	0.23948	-942.544725823203\\
61.375	0.24114	-960.861555674666\\
61.375	0.2428	-979.178385526129\\
61.375	0.24446	-997.495215377592\\
61.375	0.24612	-1015.81204522905\\
61.375	0.24778	-1034.12887508052\\
61.375	0.24944	-1052.44570493198\\
61.375	0.2511	-1070.76253478344\\
61.375	0.25276	-1089.07936463491\\
61.375	0.25442	-1107.39619448637\\
61.375	0.25608	-1125.71302433783\\
61.375	0.25774	-1144.0298541893\\
61.375	0.2594	-1162.34668404076\\
61.375	0.26106	-1180.66351389222\\
61.375	0.26272	-1198.98034374368\\
61.375	0.26438	-1217.29717359515\\
61.375	0.26604	-1235.61400344661\\
61.375	0.2677	-1253.93083329807\\
61.375	0.26936	-1272.24766314954\\
61.375	0.27102	-1290.564493001\\
61.375	0.27268	-1308.88132285246\\
61.375	0.27434	-1327.19815270392\\
61.375	0.276	-1345.51498255539\\
61.5833333333333	0.193	-427.398404114158\\
61.5833333333333	0.19466	-445.741079344284\\
61.5833333333333	0.19632	-464.08375457441\\
61.5833333333333	0.19798	-482.426429804537\\
61.5833333333333	0.19964	-500.769105034663\\
61.5833333333333	0.2013	-519.111780264789\\
61.5833333333333	0.20296	-537.454455494915\\
61.5833333333333	0.20462	-555.797130725041\\
61.5833333333333	0.20628	-574.139805955167\\
61.5833333333333	0.20794	-592.482481185293\\
61.5833333333333	0.2096	-610.825156415419\\
61.5833333333333	0.21126	-629.167831645545\\
61.5833333333333	0.21292	-647.510506875671\\
61.5833333333333	0.21458	-665.853182105797\\
61.5833333333333	0.21624	-684.195857335924\\
61.5833333333333	0.2179	-702.53853256605\\
61.5833333333333	0.21956	-720.881207796176\\
61.5833333333333	0.22122	-739.223883026302\\
61.5833333333333	0.22288	-757.566558256428\\
61.5833333333333	0.22454	-775.909233486555\\
61.5833333333333	0.2262	-794.25190871668\\
61.5833333333333	0.22786	-812.594583946806\\
61.5833333333333	0.22952	-830.937259176932\\
61.5833333333333	0.23118	-849.279934407059\\
61.5833333333333	0.23284	-867.622609637185\\
61.5833333333333	0.2345	-885.965284867311\\
61.5833333333333	0.23616	-904.307960097437\\
61.5833333333333	0.23782	-922.650635327563\\
61.5833333333333	0.23948	-940.993310557689\\
61.5833333333333	0.24114	-959.335985787815\\
61.5833333333333	0.2428	-977.678661017942\\
61.5833333333333	0.24446	-996.021336248067\\
61.5833333333333	0.24612	-1014.36401147819\\
61.5833333333333	0.24778	-1032.70668670832\\
61.5833333333333	0.24944	-1051.04936193845\\
61.5833333333333	0.2511	-1069.39203716857\\
61.5833333333333	0.25276	-1087.7347123987\\
61.5833333333333	0.25442	-1106.07738762882\\
61.5833333333333	0.25608	-1124.42006285895\\
61.5833333333333	0.25774	-1142.76273808908\\
61.5833333333333	0.2594	-1161.1054133192\\
61.5833333333333	0.26106	-1179.44808854933\\
61.5833333333333	0.26272	-1197.79076377945\\
61.5833333333333	0.26438	-1216.13343900958\\
61.5833333333333	0.26604	-1234.47611423971\\
61.5833333333333	0.2677	-1252.81878946983\\
61.5833333333333	0.26936	-1271.16146469996\\
61.5833333333333	0.27102	-1289.50413993009\\
61.5833333333333	0.27268	-1307.84681516021\\
61.5833333333333	0.27434	-1326.18949039034\\
61.5833333333333	0.276	-1344.53216562046\\
61.7916666666667	0.193	-425.123318246076\\
61.7916666666667	0.19466	-443.491838854865\\
61.7916666666667	0.19632	-461.860359463655\\
61.7916666666667	0.19798	-480.228880072444\\
61.7916666666667	0.19964	-498.597400681233\\
61.7916666666667	0.2013	-516.965921290022\\
61.7916666666667	0.20296	-535.334441898812\\
61.7916666666667	0.20462	-553.702962507601\\
61.7916666666667	0.20628	-572.07148311639\\
61.7916666666667	0.20794	-590.440003725179\\
61.7916666666667	0.2096	-608.808524333969\\
61.7916666666667	0.21126	-627.177044942758\\
61.7916666666667	0.21292	-645.545565551547\\
61.7916666666667	0.21458	-663.914086160336\\
61.7916666666667	0.21624	-682.282606769126\\
61.7916666666667	0.2179	-700.651127377915\\
61.7916666666667	0.21956	-719.019647986704\\
61.7916666666667	0.22122	-737.388168595494\\
61.7916666666667	0.22288	-755.756689204283\\
61.7916666666667	0.22454	-774.125209813072\\
61.7916666666667	0.2262	-792.493730421861\\
61.7916666666667	0.22786	-810.862251030651\\
61.7916666666667	0.22952	-829.23077163944\\
61.7916666666667	0.23118	-847.599292248229\\
61.7916666666667	0.23284	-865.967812857019\\
61.7916666666667	0.2345	-884.336333465808\\
61.7916666666667	0.23616	-902.704854074597\\
61.7916666666667	0.23782	-921.073374683386\\
61.7916666666667	0.23948	-939.441895292176\\
61.7916666666667	0.24114	-957.810415900965\\
61.7916666666667	0.2428	-976.178936509754\\
61.7916666666667	0.24446	-994.547457118543\\
61.7916666666667	0.24612	-1012.91597772733\\
61.7916666666667	0.24778	-1031.28449833612\\
61.7916666666667	0.24944	-1049.65301894491\\
61.7916666666667	0.2511	-1068.0215395537\\
61.7916666666667	0.25276	-1086.39006016249\\
61.7916666666667	0.25442	-1104.75858077128\\
61.7916666666667	0.25608	-1123.12710138007\\
61.7916666666667	0.25774	-1141.49562198886\\
61.7916666666667	0.2594	-1159.86414259765\\
61.7916666666667	0.26106	-1178.23266320644\\
61.7916666666667	0.26272	-1196.60118381523\\
61.7916666666667	0.26438	-1214.96970442401\\
61.7916666666667	0.26604	-1233.3382250328\\
61.7916666666667	0.2677	-1251.70674564159\\
61.7916666666667	0.26936	-1270.07526625038\\
61.7916666666667	0.27102	-1288.44378685917\\
61.7916666666667	0.27268	-1306.81230746796\\
61.7916666666667	0.27434	-1325.18082807675\\
61.7916666666667	0.276	-1343.54934868554\\
62	0.193	-422.848232377993\\
62	0.19466	-441.242598365446\\
62	0.19632	-459.636964352898\\
62	0.19798	-478.031330340351\\
62	0.19964	-496.425696327803\\
62	0.2013	-514.820062315256\\
62	0.20296	-533.214428302708\\
62	0.20462	-551.60879429016\\
62	0.20628	-570.003160277613\\
62	0.20794	-588.397526265065\\
62	0.2096	-606.791892252518\\
62	0.21126	-625.18625823997\\
62	0.21292	-643.580624227422\\
62	0.21458	-661.974990214875\\
62	0.21624	-680.369356202327\\
62	0.2179	-698.76372218978\\
62	0.21956	-717.158088177232\\
62	0.22122	-735.552454164685\\
62	0.22288	-753.946820152137\\
62	0.22454	-772.34118613959\\
62	0.2262	-790.735552127042\\
62	0.22786	-809.129918114494\\
62	0.22952	-827.524284101947\\
62	0.23118	-845.918650089399\\
62	0.23284	-864.313016076852\\
62	0.2345	-882.707382064304\\
62	0.23616	-901.101748051756\\
62	0.23782	-919.496114039209\\
62	0.23948	-937.890480026661\\
62	0.24114	-956.284846014114\\
62	0.2428	-974.679212001566\\
62	0.24446	-993.073577989019\\
62	0.24612	-1011.46794397647\\
62	0.24778	-1029.86230996392\\
62	0.24944	-1048.25667595138\\
62	0.2511	-1066.65104193883\\
62	0.25276	-1085.04540792628\\
62	0.25442	-1103.43977391373\\
62	0.25608	-1121.83413990119\\
62	0.25774	-1140.22850588864\\
62	0.2594	-1158.62287187609\\
62	0.26106	-1177.01723786354\\
62	0.26272	-1195.411603851\\
62	0.26438	-1213.80596983845\\
62	0.26604	-1232.2003358259\\
62	0.2677	-1250.59470181335\\
62	0.26936	-1268.98906780081\\
62	0.27102	-1287.38343378826\\
62	0.27268	-1305.77779977571\\
62	0.27434	-1324.17216576316\\
62	0.276	-1342.56653175061\\
62.2083333333333	0.193	-420.573146509911\\
62.2083333333333	0.19466	-438.993357876026\\
62.2083333333333	0.19632	-457.413569242142\\
62.2083333333333	0.19798	-475.833780608258\\
62.2083333333333	0.19964	-494.253991974373\\
62.2083333333333	0.2013	-512.674203340489\\
62.2083333333333	0.20296	-531.094414706604\\
62.2083333333333	0.20462	-549.51462607272\\
62.2083333333333	0.20628	-567.934837438836\\
62.2083333333333	0.20794	-586.355048804951\\
62.2083333333333	0.2096	-604.775260171067\\
62.2083333333333	0.21126	-623.195471537182\\
62.2083333333333	0.21292	-641.615682903298\\
62.2083333333333	0.21458	-660.035894269413\\
62.2083333333333	0.21624	-678.456105635529\\
62.2083333333333	0.2179	-696.876317001645\\
62.2083333333333	0.21956	-715.29652836776\\
62.2083333333333	0.22122	-733.716739733876\\
62.2083333333333	0.22288	-752.136951099992\\
62.2083333333333	0.22454	-770.557162466107\\
62.2083333333333	0.2262	-788.977373832223\\
62.2083333333333	0.22786	-807.397585198338\\
62.2083333333333	0.22952	-825.817796564454\\
62.2083333333333	0.23118	-844.238007930569\\
62.2083333333333	0.23284	-862.658219296685\\
62.2083333333333	0.2345	-881.0784306628\\
62.2083333333333	0.23616	-899.498642028916\\
62.2083333333333	0.23782	-917.918853395032\\
62.2083333333333	0.23948	-936.339064761147\\
62.2083333333333	0.24114	-954.759276127263\\
62.2083333333333	0.2428	-973.179487493379\\
62.2083333333333	0.24446	-991.599698859494\\
62.2083333333333	0.24612	-1010.01991022561\\
62.2083333333333	0.24778	-1028.44012159173\\
62.2083333333333	0.24944	-1046.86033295784\\
62.2083333333333	0.2511	-1065.28054432396\\
62.2083333333333	0.25276	-1083.70075569007\\
62.2083333333333	0.25442	-1102.12096705619\\
62.2083333333333	0.25608	-1120.5411784223\\
62.2083333333333	0.25774	-1138.96138978842\\
62.2083333333333	0.2594	-1157.38160115453\\
62.2083333333333	0.26106	-1175.80181252065\\
62.2083333333333	0.26272	-1194.22202388677\\
62.2083333333333	0.26438	-1212.64223525288\\
62.2083333333333	0.26604	-1231.062446619\\
62.2083333333333	0.2677	-1249.48265798511\\
62.2083333333333	0.26936	-1267.90286935123\\
62.2083333333333	0.27102	-1286.32308071734\\
62.2083333333333	0.27268	-1304.74329208346\\
62.2083333333333	0.27434	-1323.16350344958\\
62.2083333333333	0.276	-1341.58371481569\\
62.4166666666667	0.193	-418.298060641829\\
62.4166666666667	0.19466	-436.744117386607\\
62.4166666666667	0.19632	-455.190174131386\\
62.4166666666667	0.19798	-473.636230876165\\
62.4166666666667	0.19964	-492.082287620943\\
62.4166666666667	0.2013	-510.528344365722\\
62.4166666666667	0.20296	-528.974401110501\\
62.4166666666667	0.20462	-547.42045785528\\
62.4166666666667	0.20628	-565.866514600059\\
62.4166666666667	0.20794	-584.312571344837\\
62.4166666666667	0.2096	-602.758628089616\\
62.4166666666667	0.21126	-621.204684834395\\
62.4166666666667	0.21292	-639.650741579173\\
62.4166666666667	0.21458	-658.096798323952\\
62.4166666666667	0.21624	-676.542855068731\\
62.4166666666667	0.2179	-694.98891181351\\
62.4166666666667	0.21956	-713.434968558289\\
62.4166666666667	0.22122	-731.881025303067\\
62.4166666666667	0.22288	-750.327082047846\\
62.4166666666667	0.22454	-768.773138792625\\
62.4166666666667	0.2262	-787.219195537403\\
62.4166666666667	0.22786	-805.665252282182\\
62.4166666666667	0.22952	-824.111309026961\\
62.4166666666667	0.23118	-842.55736577174\\
62.4166666666667	0.23284	-861.003422516519\\
62.4166666666667	0.2345	-879.449479261297\\
62.4166666666667	0.23616	-897.895536006076\\
62.4166666666667	0.23782	-916.341592750855\\
62.4166666666667	0.23948	-934.787649495634\\
62.4166666666667	0.24114	-953.233706240412\\
62.4166666666667	0.2428	-971.679762985191\\
62.4166666666667	0.24446	-990.12581972997\\
62.4166666666667	0.24612	-1008.57187647475\\
62.4166666666667	0.24778	-1027.01793321953\\
62.4166666666667	0.24944	-1045.46398996431\\
62.4166666666667	0.2511	-1063.91004670908\\
62.4166666666667	0.25276	-1082.35610345386\\
62.4166666666667	0.25442	-1100.80216019864\\
62.4166666666667	0.25608	-1119.24821694342\\
62.4166666666667	0.25774	-1137.6942736882\\
62.4166666666667	0.2594	-1156.14033043298\\
62.4166666666667	0.26106	-1174.58638717776\\
62.4166666666667	0.26272	-1193.03244392254\\
62.4166666666667	0.26438	-1211.47850066731\\
62.4166666666667	0.26604	-1229.92455741209\\
62.4166666666667	0.2677	-1248.37061415687\\
62.4166666666667	0.26936	-1266.81667090165\\
62.4166666666667	0.27102	-1285.26272764643\\
62.4166666666667	0.27268	-1303.70878439121\\
62.4166666666667	0.27434	-1322.15484113599\\
62.4166666666667	0.276	-1340.60089788077\\
62.625	0.193	-416.022974773746\\
62.625	0.19466	-434.494876897188\\
62.625	0.19632	-452.96677902063\\
62.625	0.19798	-471.438681144072\\
62.625	0.19964	-489.910583267514\\
62.625	0.2013	-508.382485390956\\
62.625	0.20296	-526.854387514397\\
62.625	0.20462	-545.326289637839\\
62.625	0.20628	-563.798191761281\\
62.625	0.20794	-582.270093884723\\
62.625	0.2096	-600.741996008165\\
62.625	0.21126	-619.213898131607\\
62.625	0.21292	-637.685800255049\\
62.625	0.21458	-656.157702378491\\
62.625	0.21624	-674.629604501933\\
62.625	0.2179	-693.101506625375\\
62.625	0.21956	-711.573408748817\\
62.625	0.22122	-730.045310872259\\
62.625	0.22288	-748.517212995701\\
62.625	0.22454	-766.989115119143\\
62.625	0.2262	-785.461017242584\\
62.625	0.22786	-803.932919366026\\
62.625	0.22952	-822.404821489468\\
62.625	0.23118	-840.87672361291\\
62.625	0.23284	-859.348625736352\\
62.625	0.2345	-877.820527859794\\
62.625	0.23616	-896.292429983236\\
62.625	0.23782	-914.764332106678\\
62.625	0.23948	-933.23623423012\\
62.625	0.24114	-951.708136353562\\
62.625	0.2428	-970.180038477004\\
62.625	0.24446	-988.651940600445\\
62.625	0.24612	-1007.12384272389\\
62.625	0.24778	-1025.59574484733\\
62.625	0.24944	-1044.06764697077\\
62.625	0.2511	-1062.53954909421\\
62.625	0.25276	-1081.01145121765\\
62.625	0.25442	-1099.4833533411\\
62.625	0.25608	-1117.95525546454\\
62.625	0.25774	-1136.42715758798\\
62.625	0.2594	-1154.89905971142\\
62.625	0.26106	-1173.37096183486\\
62.625	0.26272	-1191.84286395831\\
62.625	0.26438	-1210.31476608175\\
62.625	0.26604	-1228.78666820519\\
62.625	0.2677	-1247.25857032863\\
62.625	0.26936	-1265.73047245207\\
62.625	0.27102	-1284.20237457552\\
62.625	0.27268	-1302.67427669896\\
62.625	0.27434	-1321.1461788224\\
62.625	0.276	-1339.61808094584\\
62.8333333333333	0.193	-413.747888905664\\
62.8333333333333	0.19466	-432.245636407769\\
62.8333333333333	0.19632	-450.743383909874\\
62.8333333333333	0.19798	-469.241131411979\\
62.8333333333333	0.19964	-487.738878914084\\
62.8333333333333	0.2013	-506.236626416189\\
62.8333333333333	0.20296	-524.734373918294\\
62.8333333333333	0.20462	-543.232121420399\\
62.8333333333333	0.20628	-561.729868922505\\
62.8333333333333	0.20794	-580.22761642461\\
62.8333333333333	0.2096	-598.725363926715\\
62.8333333333333	0.21126	-617.22311142882\\
62.8333333333333	0.21292	-635.720858930925\\
62.8333333333333	0.21458	-654.21860643303\\
62.8333333333333	0.21624	-672.716353935135\\
62.8333333333333	0.2179	-691.21410143724\\
62.8333333333333	0.21956	-709.711848939345\\
62.8333333333333	0.22122	-728.20959644145\\
62.8333333333333	0.22288	-746.707343943556\\
62.8333333333333	0.22454	-765.205091445661\\
62.8333333333333	0.2262	-783.702838947765\\
62.8333333333333	0.22786	-802.200586449871\\
62.8333333333333	0.22952	-820.698333951976\\
62.8333333333333	0.23118	-839.196081454081\\
62.8333333333333	0.23284	-857.693828956186\\
62.8333333333333	0.2345	-876.191576458291\\
62.8333333333333	0.23616	-894.689323960396\\
62.8333333333333	0.23782	-913.187071462502\\
62.8333333333333	0.23948	-931.684818964606\\
62.8333333333333	0.24114	-950.182566466711\\
62.8333333333333	0.2428	-968.680313968817\\
62.8333333333333	0.24446	-987.178061470921\\
62.8333333333333	0.24612	-1005.67580897303\\
62.8333333333333	0.24778	-1024.17355647513\\
62.8333333333333	0.24944	-1042.67130397724\\
62.8333333333333	0.2511	-1061.16905147934\\
62.8333333333333	0.25276	-1079.66679898145\\
62.8333333333333	0.25442	-1098.16454648355\\
62.8333333333333	0.25608	-1116.66229398566\\
62.8333333333333	0.25774	-1135.16004148776\\
62.8333333333333	0.2594	-1153.65778898987\\
62.8333333333333	0.26106	-1172.15553649197\\
62.8333333333333	0.26272	-1190.65328399408\\
62.8333333333333	0.26438	-1209.15103149618\\
62.8333333333333	0.26604	-1227.64877899829\\
62.8333333333333	0.2677	-1246.14652650039\\
62.8333333333333	0.26936	-1264.6442740025\\
62.8333333333333	0.27102	-1283.1420215046\\
62.8333333333333	0.27268	-1301.63976900671\\
62.8333333333333	0.27434	-1320.13751650881\\
62.8333333333333	0.276	-1338.63526401092\\
63.0416666666667	0.193	-411.472803037581\\
63.0416666666667	0.19466	-429.996395918349\\
63.0416666666667	0.19632	-448.519988799117\\
63.0416666666667	0.19798	-467.043581679886\\
63.0416666666667	0.19964	-485.567174560654\\
63.0416666666667	0.2013	-504.090767441422\\
63.0416666666667	0.20296	-522.61436032219\\
63.0416666666667	0.20462	-541.137953202959\\
63.0416666666667	0.20628	-559.661546083727\\
63.0416666666667	0.20794	-578.185138964495\\
63.0416666666667	0.2096	-596.708731845264\\
63.0416666666667	0.21126	-615.232324726032\\
63.0416666666667	0.21292	-633.7559176068\\
63.0416666666667	0.21458	-652.279510487568\\
63.0416666666667	0.21624	-670.803103368337\\
63.0416666666667	0.2179	-689.326696249105\\
63.0416666666667	0.21956	-707.850289129873\\
63.0416666666667	0.22122	-726.373882010641\\
63.0416666666667	0.22288	-744.89747489141\\
63.0416666666667	0.22454	-763.421067772178\\
63.0416666666667	0.2262	-781.944660652946\\
63.0416666666667	0.22786	-800.468253533714\\
63.0416666666667	0.22952	-818.991846414483\\
63.0416666666667	0.23118	-837.515439295251\\
63.0416666666667	0.23284	-856.039032176019\\
63.0416666666667	0.2345	-874.562625056787\\
63.0416666666667	0.23616	-893.086217937555\\
63.0416666666667	0.23782	-911.609810818324\\
63.0416666666667	0.23948	-930.133403699092\\
63.0416666666667	0.24114	-948.65699657986\\
63.0416666666667	0.2428	-967.180589460629\\
63.0416666666667	0.24446	-985.704182341397\\
63.0416666666667	0.24612	-1004.22777522217\\
63.0416666666667	0.24778	-1022.75136810293\\
63.0416666666667	0.24944	-1041.2749609837\\
63.0416666666667	0.2511	-1059.79855386447\\
63.0416666666667	0.25276	-1078.32214674524\\
63.0416666666667	0.25442	-1096.84573962601\\
63.0416666666667	0.25608	-1115.36933250677\\
63.0416666666667	0.25774	-1133.89292538754\\
63.0416666666667	0.2594	-1152.41651826831\\
63.0416666666667	0.26106	-1170.94011114908\\
63.0416666666667	0.26272	-1189.46370402985\\
63.0416666666667	0.26438	-1207.98729691062\\
63.0416666666667	0.26604	-1226.51088979138\\
63.0416666666667	0.2677	-1245.03448267215\\
63.0416666666667	0.26936	-1263.55807555292\\
63.0416666666667	0.27102	-1282.08166843369\\
63.0416666666667	0.27268	-1300.60526131446\\
63.0416666666667	0.27434	-1319.12885419523\\
63.0416666666667	0.276	-1337.65244707599\\
63.25	0.193	-409.197717169498\\
63.25	0.19466	-427.74715542893\\
63.25	0.19632	-446.296593688361\\
63.25	0.19798	-464.846031947793\\
63.25	0.19964	-483.395470207224\\
63.25	0.2013	-501.944908466655\\
63.25	0.20296	-520.494346726087\\
63.25	0.20462	-539.043784985518\\
63.25	0.20628	-557.59322324495\\
63.25	0.20794	-576.142661504381\\
63.25	0.2096	-594.692099763813\\
63.25	0.21126	-613.241538023244\\
63.25	0.21292	-631.790976282675\\
63.25	0.21458	-650.340414542106\\
63.25	0.21624	-668.889852801538\\
63.25	0.2179	-687.439291060969\\
63.25	0.21956	-705.988729320401\\
63.25	0.22122	-724.538167579832\\
63.25	0.22288	-743.087605839264\\
63.25	0.22454	-761.637044098695\\
63.25	0.2262	-780.186482358127\\
63.25	0.22786	-798.735920617558\\
63.25	0.22952	-817.285358876989\\
63.25	0.23118	-835.834797136421\\
63.25	0.23284	-854.384235395852\\
63.25	0.2345	-872.933673655284\\
63.25	0.23616	-891.483111914715\\
63.25	0.23782	-910.032550174147\\
63.25	0.23948	-928.581988433578\\
63.25	0.24114	-947.131426693009\\
63.25	0.2428	-965.680864952441\\
63.25	0.24446	-984.230303211872\\
63.25	0.24612	-1002.7797414713\\
63.25	0.24778	-1021.32917973073\\
63.25	0.24944	-1039.87861799017\\
63.25	0.2511	-1058.4280562496\\
63.25	0.25276	-1076.97749450903\\
63.25	0.25442	-1095.52693276846\\
63.25	0.25608	-1114.07637102789\\
63.25	0.25774	-1132.62580928732\\
63.25	0.2594	-1151.17524754675\\
63.25	0.26106	-1169.72468580619\\
63.25	0.26272	-1188.27412406562\\
63.25	0.26438	-1206.82356232505\\
63.25	0.26604	-1225.37300058448\\
63.25	0.2677	-1243.92243884391\\
63.25	0.26936	-1262.47187710334\\
63.25	0.27102	-1281.02131536277\\
63.25	0.27268	-1299.57075362221\\
63.25	0.27434	-1318.12019188164\\
63.25	0.276	-1336.66963014107\\
63.4583333333333	0.193	-406.922631301416\\
63.4583333333333	0.19466	-425.497914939511\\
63.4583333333333	0.19632	-444.073198577605\\
63.4583333333333	0.19798	-462.6484822157\\
63.4583333333333	0.19964	-481.223765853795\\
63.4583333333333	0.2013	-499.799049491889\\
63.4583333333333	0.20296	-518.374333129983\\
63.4583333333333	0.20462	-536.949616768078\\
63.4583333333333	0.20628	-555.524900406173\\
63.4583333333333	0.20794	-574.100184044267\\
63.4583333333333	0.2096	-592.675467682362\\
63.4583333333333	0.21126	-611.250751320457\\
63.4583333333333	0.21292	-629.826034958551\\
63.4583333333333	0.21458	-648.401318596646\\
63.4583333333333	0.21624	-666.97660223474\\
63.4583333333333	0.2179	-685.551885872835\\
63.4583333333333	0.21956	-704.127169510929\\
63.4583333333333	0.22122	-722.702453149024\\
63.4583333333333	0.22288	-741.277736787119\\
63.4583333333333	0.22454	-759.853020425213\\
63.4583333333333	0.2262	-778.428304063308\\
63.4583333333333	0.22786	-797.003587701402\\
63.4583333333333	0.22952	-815.578871339497\\
63.4583333333333	0.23118	-834.154154977592\\
63.4583333333333	0.23284	-852.729438615686\\
63.4583333333333	0.2345	-871.30472225378\\
63.4583333333333	0.23616	-889.880005891875\\
63.4583333333333	0.23782	-908.45528952997\\
63.4583333333333	0.23948	-927.030573168064\\
63.4583333333333	0.24114	-945.605856806159\\
63.4583333333333	0.2428	-964.181140444254\\
63.4583333333333	0.24446	-982.756424082348\\
63.4583333333333	0.24612	-1001.33170772044\\
63.4583333333333	0.24778	-1019.90699135854\\
63.4583333333333	0.24944	-1038.48227499663\\
63.4583333333333	0.2511	-1057.05755863473\\
63.4583333333333	0.25276	-1075.63284227282\\
63.4583333333333	0.25442	-1094.20812591092\\
63.4583333333333	0.25608	-1112.78340954901\\
63.4583333333333	0.25774	-1131.3586931871\\
63.4583333333333	0.2594	-1149.9339768252\\
63.4583333333333	0.26106	-1168.50926046329\\
63.4583333333333	0.26272	-1187.08454410139\\
63.4583333333333	0.26438	-1205.65982773948\\
63.4583333333333	0.26604	-1224.23511137758\\
63.4583333333333	0.2677	-1242.81039501567\\
63.4583333333333	0.26936	-1261.38567865377\\
63.4583333333333	0.27102	-1279.96096229186\\
63.4583333333333	0.27268	-1298.53624592996\\
63.4583333333333	0.27434	-1317.11152956805\\
63.4583333333333	0.276	-1335.68681320614\\
63.6666666666667	0.193	-404.647545433334\\
63.6666666666667	0.19466	-423.248674450091\\
63.6666666666667	0.19632	-441.849803466849\\
63.6666666666667	0.19798	-460.450932483607\\
63.6666666666667	0.19964	-479.052061500365\\
63.6666666666667	0.2013	-497.653190517122\\
63.6666666666667	0.20296	-516.25431953388\\
63.6666666666667	0.20462	-534.855448550638\\
63.6666666666667	0.20628	-553.456577567396\\
63.6666666666667	0.20794	-572.057706584153\\
63.6666666666667	0.2096	-590.658835600911\\
63.6666666666667	0.21126	-609.259964617669\\
63.6666666666667	0.21292	-627.861093634426\\
63.6666666666667	0.21458	-646.462222651184\\
63.6666666666667	0.21624	-665.063351667942\\
63.6666666666667	0.2179	-683.6644806847\\
63.6666666666667	0.21956	-702.265609701457\\
63.6666666666667	0.22122	-720.866738718215\\
63.6666666666667	0.22288	-739.467867734973\\
63.6666666666667	0.22454	-758.068996751731\\
63.6666666666667	0.2262	-776.670125768488\\
63.6666666666667	0.22786	-795.271254785246\\
63.6666666666667	0.22952	-813.872383802004\\
63.6666666666667	0.23118	-832.473512818762\\
63.6666666666667	0.23284	-851.074641835519\\
63.6666666666667	0.2345	-869.675770852277\\
63.6666666666667	0.23616	-888.276899869035\\
63.6666666666667	0.23782	-906.878028885793\\
63.6666666666667	0.23948	-925.47915790255\\
63.6666666666667	0.24114	-944.080286919308\\
63.6666666666667	0.2428	-962.681415936066\\
63.6666666666667	0.24446	-981.282544952823\\
63.6666666666667	0.24612	-999.883673969581\\
63.6666666666667	0.24778	-1018.48480298634\\
63.6666666666667	0.24944	-1037.0859320031\\
63.6666666666667	0.2511	-1055.68706101985\\
63.6666666666667	0.25276	-1074.28819003661\\
63.6666666666667	0.25442	-1092.88931905337\\
63.6666666666667	0.25608	-1111.49044807013\\
63.6666666666667	0.25774	-1130.09157708689\\
63.6666666666667	0.2594	-1148.69270610364\\
63.6666666666667	0.26106	-1167.2938351204\\
63.6666666666667	0.26272	-1185.89496413716\\
63.6666666666667	0.26438	-1204.49609315392\\
63.6666666666667	0.26604	-1223.09722217067\\
63.6666666666667	0.2677	-1241.69835118743\\
63.6666666666667	0.26936	-1260.29948020419\\
63.6666666666667	0.27102	-1278.90060922095\\
63.6666666666667	0.27268	-1297.50173823771\\
63.6666666666667	0.27434	-1316.10286725446\\
63.6666666666667	0.276	-1334.70399627122\\
63.875	0.193	-402.372459565251\\
63.875	0.19466	-420.999433960672\\
63.875	0.19632	-439.626408356093\\
63.875	0.19798	-458.253382751514\\
63.875	0.19964	-476.880357146935\\
63.875	0.2013	-495.507331542356\\
63.875	0.20296	-514.134305937777\\
63.875	0.20462	-532.761280333198\\
63.875	0.20628	-551.388254728619\\
63.875	0.20794	-570.01522912404\\
63.875	0.2096	-588.642203519461\\
63.875	0.21126	-607.269177914881\\
63.875	0.21292	-625.896152310302\\
63.875	0.21458	-644.523126705723\\
63.875	0.21624	-663.150101101144\\
63.875	0.2179	-681.777075496565\\
63.875	0.21956	-700.404049891986\\
63.875	0.22122	-719.031024287407\\
63.875	0.22288	-737.657998682828\\
63.875	0.22454	-756.284973078249\\
63.875	0.2262	-774.911947473669\\
63.875	0.22786	-793.53892186909\\
63.875	0.22952	-812.165896264511\\
63.875	0.23118	-830.792870659932\\
63.875	0.23284	-849.419845055353\\
63.875	0.2345	-868.046819450774\\
63.875	0.23616	-886.673793846195\\
63.875	0.23782	-905.300768241616\\
63.875	0.23948	-923.927742637037\\
63.875	0.24114	-942.554717032458\\
63.875	0.2428	-961.181691427879\\
63.875	0.24446	-979.808665823299\\
63.875	0.24612	-998.43564021872\\
63.875	0.24778	-1017.06261461414\\
63.875	0.24944	-1035.68958900956\\
63.875	0.2511	-1054.31656340498\\
63.875	0.25276	-1072.9435378004\\
63.875	0.25442	-1091.57051219583\\
63.875	0.25608	-1110.19748659125\\
63.875	0.25774	-1128.82446098667\\
63.875	0.2594	-1147.45143538209\\
63.875	0.26106	-1166.07840977751\\
63.875	0.26272	-1184.70538417293\\
63.875	0.26438	-1203.33235856835\\
63.875	0.26604	-1221.95933296377\\
63.875	0.2677	-1240.58630735919\\
63.875	0.26936	-1259.21328175461\\
63.875	0.27102	-1277.84025615003\\
63.875	0.27268	-1296.46723054546\\
63.875	0.27434	-1315.09420494088\\
63.875	0.276	-1333.7211793363\\
64.0833333333333	0.193	-400.097373697169\\
64.0833333333333	0.19466	-418.750193471253\\
64.0833333333333	0.19632	-437.403013245337\\
64.0833333333333	0.19798	-456.055833019421\\
64.0833333333333	0.19964	-474.708652793505\\
64.0833333333333	0.2013	-493.361472567589\\
64.0833333333333	0.20296	-512.014292341673\\
64.0833333333333	0.20462	-530.667112115757\\
64.0833333333333	0.20628	-549.319931889841\\
64.0833333333333	0.20794	-567.972751663925\\
64.0833333333333	0.2096	-586.625571438009\\
64.0833333333333	0.21126	-605.278391212093\\
64.0833333333333	0.21292	-623.931210986177\\
64.0833333333333	0.21458	-642.584030760261\\
64.0833333333333	0.21624	-661.236850534346\\
64.0833333333333	0.2179	-679.88967030843\\
64.0833333333333	0.21956	-698.542490082514\\
64.0833333333333	0.22122	-717.195309856598\\
64.0833333333333	0.22288	-735.848129630682\\
64.0833333333333	0.22454	-754.500949404766\\
64.0833333333333	0.2262	-773.15376917885\\
64.0833333333333	0.22786	-791.806588952934\\
64.0833333333333	0.22952	-810.459408727018\\
64.0833333333333	0.23118	-829.112228501102\\
64.0833333333333	0.23284	-847.765048275186\\
64.0833333333333	0.2345	-866.41786804927\\
64.0833333333333	0.23616	-885.070687823354\\
64.0833333333333	0.23782	-903.723507597439\\
64.0833333333333	0.23948	-922.376327371523\\
64.0833333333333	0.24114	-941.029147145607\\
64.0833333333333	0.2428	-959.681966919691\\
64.0833333333333	0.24446	-978.334786693775\\
64.0833333333333	0.24612	-996.987606467859\\
64.0833333333333	0.24778	-1015.64042624194\\
64.0833333333333	0.24944	-1034.29324601603\\
64.0833333333333	0.2511	-1052.94606579011\\
64.0833333333333	0.25276	-1071.59888556419\\
64.0833333333333	0.25442	-1090.25170533828\\
64.0833333333333	0.25608	-1108.90452511236\\
64.0833333333333	0.25774	-1127.55734488645\\
64.0833333333333	0.2594	-1146.21016466053\\
64.0833333333333	0.26106	-1164.86298443462\\
64.0833333333333	0.26272	-1183.5158042087\\
64.0833333333333	0.26438	-1202.16862398278\\
64.0833333333333	0.26604	-1220.82144375687\\
64.0833333333333	0.2677	-1239.47426353095\\
64.0833333333333	0.26936	-1258.12708330504\\
64.0833333333333	0.27102	-1276.77990307912\\
64.0833333333333	0.27268	-1295.4327228532\\
64.0833333333333	0.27434	-1314.08554262729\\
64.0833333333333	0.276	-1332.73836240137\\
64.2916666666667	0.193	-397.822287829086\\
64.2916666666667	0.19466	-416.500952981833\\
64.2916666666667	0.19632	-435.17961813458\\
64.2916666666667	0.19798	-453.858283287328\\
64.2916666666667	0.19964	-472.536948440075\\
64.2916666666667	0.2013	-491.215613592822\\
64.2916666666667	0.20296	-509.894278745569\\
64.2916666666667	0.20462	-528.572943898317\\
64.2916666666667	0.20628	-547.251609051064\\
64.2916666666667	0.20794	-565.930274203811\\
64.2916666666667	0.2096	-584.608939356558\\
64.2916666666667	0.21126	-603.287604509306\\
64.2916666666667	0.21292	-621.966269662053\\
64.2916666666667	0.21458	-640.6449348148\\
64.2916666666667	0.21624	-659.323599967547\\
64.2916666666667	0.2179	-678.002265120295\\
64.2916666666667	0.21956	-696.680930273042\\
64.2916666666667	0.22122	-715.359595425789\\
64.2916666666667	0.22288	-734.038260578536\\
64.2916666666667	0.22454	-752.716925731284\\
64.2916666666667	0.2262	-771.395590884031\\
64.2916666666667	0.22786	-790.074256036778\\
64.2916666666667	0.22952	-808.752921189525\\
64.2916666666667	0.23118	-827.431586342273\\
64.2916666666667	0.23284	-846.11025149502\\
64.2916666666667	0.2345	-864.788916647767\\
64.2916666666667	0.23616	-883.467581800514\\
64.2916666666667	0.23782	-902.146246953262\\
64.2916666666667	0.23948	-920.824912106009\\
64.2916666666667	0.24114	-939.503577258756\\
64.2916666666667	0.2428	-958.182242411503\\
64.2916666666667	0.24446	-976.86090756425\\
64.2916666666667	0.24612	-995.539572716998\\
64.2916666666667	0.24778	-1014.21823786974\\
64.2916666666667	0.24944	-1032.89690302249\\
64.2916666666667	0.2511	-1051.57556817524\\
64.2916666666667	0.25276	-1070.25423332799\\
64.2916666666667	0.25442	-1088.93289848073\\
64.2916666666667	0.25608	-1107.61156363348\\
64.2916666666667	0.25774	-1126.29022878623\\
64.2916666666667	0.2594	-1144.96889393898\\
64.2916666666667	0.26106	-1163.64755909172\\
64.2916666666667	0.26272	-1182.32622424447\\
64.2916666666667	0.26438	-1201.00488939722\\
64.2916666666667	0.26604	-1219.68355454996\\
64.2916666666667	0.2677	-1238.36221970271\\
64.2916666666667	0.26936	-1257.04088485546\\
64.2916666666667	0.27102	-1275.71955000821\\
64.2916666666667	0.27268	-1294.39821516095\\
64.2916666666667	0.27434	-1313.0768803137\\
64.2916666666667	0.276	-1331.75554546645\\
64.5	0.193	-395.547201961004\\
64.5	0.19466	-414.251712492414\\
64.5	0.19632	-432.956223023824\\
64.5	0.19798	-451.660733555235\\
64.5	0.19964	-470.365244086645\\
64.5	0.2013	-489.069754618056\\
64.5	0.20296	-507.774265149466\\
64.5	0.20462	-526.478775680876\\
64.5	0.20628	-545.183286212287\\
64.5	0.20794	-563.887796743697\\
64.5	0.2096	-582.592307275108\\
64.5	0.21126	-601.296817806518\\
64.5	0.21292	-620.001328337928\\
64.5	0.21458	-638.705838869339\\
64.5	0.21624	-657.410349400749\\
64.5	0.2179	-676.11485993216\\
64.5	0.21956	-694.81937046357\\
64.5	0.22122	-713.523880994981\\
64.5	0.22288	-732.228391526391\\
64.5	0.22454	-750.932902057802\\
64.5	0.2262	-769.637412589212\\
64.5	0.22786	-788.341923120622\\
64.5	0.22952	-807.046433652032\\
64.5	0.23118	-825.750944183443\\
64.5	0.23284	-844.455454714853\\
64.5	0.2345	-863.159965246264\\
64.5	0.23616	-881.864475777674\\
64.5	0.23782	-900.568986309085\\
64.5	0.23948	-919.273496840495\\
64.5	0.24114	-937.978007371906\\
64.5	0.2428	-956.682517903316\\
64.5	0.24446	-975.387028434726\\
64.5	0.24612	-994.091538966136\\
64.5	0.24778	-1012.79604949755\\
64.5	0.24944	-1031.50056002896\\
64.5	0.2511	-1050.20507056037\\
64.5	0.25276	-1068.90958109178\\
64.5	0.25442	-1087.61409162319\\
64.5	0.25608	-1106.3186021546\\
64.5	0.25774	-1125.02311268601\\
64.5	0.2594	-1143.72762321742\\
64.5	0.26106	-1162.43213374883\\
64.5	0.26272	-1181.13664428024\\
64.5	0.26438	-1199.84115481165\\
64.5	0.26604	-1218.54566534306\\
64.5	0.2677	-1237.25017587447\\
64.5	0.26936	-1255.95468640588\\
64.5	0.27102	-1274.65919693729\\
64.5	0.27268	-1293.3637074687\\
64.5	0.27434	-1312.06821800011\\
64.5	0.276	-1330.77272853152\\
64.7083333333333	0.193	-393.272116092921\\
64.7083333333333	0.19466	-412.002472002995\\
64.7083333333333	0.19632	-430.732827913068\\
64.7083333333333	0.19798	-449.463183823142\\
64.7083333333333	0.19964	-468.193539733215\\
64.7083333333333	0.2013	-486.923895643289\\
64.7083333333333	0.20296	-505.654251553362\\
64.7083333333333	0.20462	-524.384607463436\\
64.7083333333333	0.20628	-543.11496337351\\
64.7083333333333	0.20794	-561.845319283583\\
64.7083333333333	0.2096	-580.575675193657\\
64.7083333333333	0.21126	-599.30603110373\\
64.7083333333333	0.21292	-618.036387013804\\
64.7083333333333	0.21458	-636.766742923877\\
64.7083333333333	0.21624	-655.497098833951\\
64.7083333333333	0.2179	-674.227454744024\\
64.7083333333333	0.21956	-692.957810654098\\
64.7083333333333	0.22122	-711.688166564172\\
64.7083333333333	0.22288	-730.418522474245\\
64.7083333333333	0.22454	-749.148878384319\\
64.7083333333333	0.2262	-767.879234294392\\
64.7083333333333	0.22786	-786.609590204466\\
64.7083333333333	0.22952	-805.339946114539\\
64.7083333333333	0.23118	-824.070302024613\\
64.7083333333333	0.23284	-842.800657934687\\
64.7083333333333	0.2345	-861.53101384476\\
64.7083333333333	0.23616	-880.261369754834\\
64.7083333333333	0.23782	-898.991725664908\\
64.7083333333333	0.23948	-917.722081574981\\
64.7083333333333	0.24114	-936.452437485054\\
64.7083333333333	0.2428	-955.182793395128\\
64.7083333333333	0.24446	-973.913149305201\\
64.7083333333333	0.24612	-992.643505215275\\
64.7083333333333	0.24778	-1011.37386112535\\
64.7083333333333	0.24944	-1030.10421703542\\
64.7083333333333	0.2511	-1048.8345729455\\
64.7083333333333	0.25276	-1067.56492885557\\
64.7083333333333	0.25442	-1086.29528476564\\
64.7083333333333	0.25608	-1105.02564067572\\
64.7083333333333	0.25774	-1123.75599658579\\
64.7083333333333	0.2594	-1142.48635249586\\
64.7083333333333	0.26106	-1161.21670840594\\
64.7083333333333	0.26272	-1179.94706431601\\
64.7083333333333	0.26438	-1198.67742022608\\
64.7083333333333	0.26604	-1217.40777613616\\
64.7083333333333	0.2677	-1236.13813204623\\
64.7083333333333	0.26936	-1254.86848795631\\
64.7083333333333	0.27102	-1273.59884386638\\
64.7083333333333	0.27268	-1292.32919977645\\
64.7083333333333	0.27434	-1311.05955568653\\
64.7083333333333	0.276	-1329.7899115966\\
64.9166666666667	0.193	-390.997030224839\\
64.9166666666667	0.19466	-409.753231513575\\
64.9166666666667	0.19632	-428.509432802312\\
64.9166666666667	0.19798	-447.265634091049\\
64.9166666666667	0.19964	-466.021835379785\\
64.9166666666667	0.2013	-484.778036668522\\
64.9166666666667	0.20296	-503.534237957259\\
64.9166666666667	0.20462	-522.290439245995\\
64.9166666666667	0.20628	-541.046640534733\\
64.9166666666667	0.20794	-559.802841823469\\
64.9166666666667	0.2096	-578.559043112206\\
64.9166666666667	0.21126	-597.315244400942\\
64.9166666666667	0.21292	-616.071445689679\\
64.9166666666667	0.21458	-634.827646978416\\
64.9166666666667	0.21624	-653.583848267153\\
64.9166666666667	0.2179	-672.340049555889\\
64.9166666666667	0.21956	-691.096250844626\\
64.9166666666667	0.22122	-709.852452133363\\
64.9166666666667	0.22288	-728.6086534221\\
64.9166666666667	0.22454	-747.364854710836\\
64.9166666666667	0.2262	-766.121055999573\\
64.9166666666667	0.22786	-784.87725728831\\
64.9166666666667	0.22952	-803.633458577046\\
64.9166666666667	0.23118	-822.389659865783\\
64.9166666666667	0.23284	-841.14586115452\\
64.9166666666667	0.2345	-859.902062443256\\
64.9166666666667	0.23616	-878.658263731993\\
64.9166666666667	0.23782	-897.41446502073\\
64.9166666666667	0.23948	-916.170666309467\\
64.9166666666667	0.24114	-934.926867598204\\
64.9166666666667	0.2428	-953.68306888694\\
64.9166666666667	0.24446	-972.439270175677\\
64.9166666666667	0.24612	-991.195471464414\\
64.9166666666667	0.24778	-1009.95167275315\\
64.9166666666667	0.24944	-1028.70787404189\\
64.9166666666667	0.2511	-1047.46407533062\\
64.9166666666667	0.25276	-1066.22027661936\\
64.9166666666667	0.25442	-1084.9764779081\\
64.9166666666667	0.25608	-1103.73267919683\\
64.9166666666667	0.25774	-1122.48888048557\\
64.9166666666667	0.2594	-1141.24508177431\\
64.9166666666667	0.26106	-1160.00128306304\\
64.9166666666667	0.26272	-1178.75748435178\\
64.9166666666667	0.26438	-1197.51368564052\\
64.9166666666667	0.26604	-1216.26988692925\\
64.9166666666667	0.2677	-1235.02608821799\\
64.9166666666667	0.26936	-1253.78228950673\\
64.9166666666667	0.27102	-1272.53849079546\\
64.9166666666667	0.27268	-1291.2946920842\\
64.9166666666667	0.27434	-1310.05089337294\\
64.9166666666667	0.276	-1328.80709466167\\
65.125	0.193	-388.721944356756\\
65.125	0.19466	-407.503991024156\\
65.125	0.19632	-426.286037691556\\
65.125	0.19798	-445.068084358956\\
65.125	0.19964	-463.850131026356\\
65.125	0.2013	-482.632177693756\\
65.125	0.20296	-501.414224361156\\
65.125	0.20462	-520.196271028556\\
65.125	0.20628	-538.978317695956\\
65.125	0.20794	-557.760364363355\\
65.125	0.2096	-576.542411030755\\
65.125	0.21126	-595.324457698155\\
65.125	0.21292	-614.106504365555\\
65.125	0.21458	-632.888551032955\\
65.125	0.21624	-651.670597700355\\
65.125	0.2179	-670.452644367755\\
65.125	0.21956	-689.234691035155\\
65.125	0.22122	-708.016737702555\\
65.125	0.22288	-726.798784369955\\
65.125	0.22454	-745.580831037355\\
65.125	0.2262	-764.362877704754\\
65.125	0.22786	-783.144924372154\\
65.125	0.22952	-801.926971039554\\
65.125	0.23118	-820.709017706954\\
65.125	0.23284	-839.491064374354\\
65.125	0.2345	-858.273111041754\\
65.125	0.23616	-877.055157709153\\
65.125	0.23782	-895.837204376554\\
65.125	0.23948	-914.619251043953\\
65.125	0.24114	-933.401297711353\\
65.125	0.2428	-952.183344378753\\
65.125	0.24446	-970.965391046153\\
65.125	0.24612	-989.747437713553\\
65.125	0.24778	-1008.52948438095\\
65.125	0.24944	-1027.31153104835\\
65.125	0.2511	-1046.09357771575\\
65.125	0.25276	-1064.87562438315\\
65.125	0.25442	-1083.65767105055\\
65.125	0.25608	-1102.43971771795\\
65.125	0.25774	-1121.22176438535\\
65.125	0.2594	-1140.00381105275\\
65.125	0.26106	-1158.78585772015\\
65.125	0.26272	-1177.56790438755\\
65.125	0.26438	-1196.34995105495\\
65.125	0.26604	-1215.13199772235\\
65.125	0.2677	-1233.91404438975\\
65.125	0.26936	-1252.69609105715\\
65.125	0.27102	-1271.47813772455\\
65.125	0.27268	-1290.26018439195\\
65.125	0.27434	-1309.04223105935\\
65.125	0.276	-1327.82427772675\\
65.3333333333333	0.193	-386.446858488674\\
65.3333333333333	0.19466	-405.254750534737\\
65.3333333333333	0.19632	-424.0626425808\\
65.3333333333333	0.19798	-442.870534626863\\
65.3333333333333	0.19964	-461.678426672926\\
65.3333333333333	0.2013	-480.486318718989\\
65.3333333333333	0.20296	-499.294210765052\\
65.3333333333333	0.20462	-518.102102811115\\
65.3333333333333	0.20628	-536.909994857179\\
65.3333333333333	0.20794	-555.717886903242\\
65.3333333333333	0.2096	-574.525778949305\\
65.3333333333333	0.21126	-593.333670995368\\
65.3333333333333	0.21292	-612.141563041431\\
65.3333333333333	0.21458	-630.949455087494\\
65.3333333333333	0.21624	-649.757347133557\\
65.3333333333333	0.2179	-668.56523917962\\
65.3333333333333	0.21956	-687.373131225683\\
65.3333333333333	0.22122	-706.181023271746\\
65.3333333333333	0.22288	-724.988915317809\\
65.3333333333333	0.22454	-743.796807363872\\
65.3333333333333	0.2262	-762.604699409935\\
65.3333333333333	0.22786	-781.412591455998\\
65.3333333333333	0.22952	-800.220483502061\\
65.3333333333333	0.23118	-819.028375548124\\
65.3333333333333	0.23284	-837.836267594188\\
65.3333333333333	0.2345	-856.64415964025\\
65.3333333333333	0.23616	-875.452051686313\\
65.3333333333333	0.23782	-894.259943732377\\
65.3333333333333	0.23948	-913.06783577844\\
65.3333333333333	0.24114	-931.875727824503\\
65.3333333333333	0.2428	-950.683619870566\\
65.3333333333333	0.24446	-969.491511916629\\
65.3333333333333	0.24612	-988.299403962692\\
65.3333333333333	0.24778	-1007.10729600875\\
65.3333333333333	0.24944	-1025.91518805482\\
65.3333333333333	0.2511	-1044.72308010088\\
65.3333333333333	0.25276	-1063.53097214694\\
65.3333333333333	0.25442	-1082.33886419301\\
65.3333333333333	0.25608	-1101.14675623907\\
65.3333333333333	0.25774	-1119.95464828513\\
65.3333333333333	0.2594	-1138.7625403312\\
65.3333333333333	0.26106	-1157.57043237726\\
65.3333333333333	0.26272	-1176.37832442332\\
65.3333333333333	0.26438	-1195.18621646939\\
65.3333333333333	0.26604	-1213.99410851545\\
65.3333333333333	0.2677	-1232.80200056151\\
65.3333333333333	0.26936	-1251.60989260757\\
65.3333333333333	0.27102	-1270.41778465364\\
65.3333333333333	0.27268	-1289.2256766997\\
65.3333333333333	0.27434	-1308.03356874576\\
65.3333333333333	0.276	-1326.84146079183\\
65.5416666666667	0.193	-384.171772620591\\
65.5416666666667	0.19466	-403.005510045317\\
65.5416666666667	0.19632	-421.839247470043\\
65.5416666666667	0.19798	-440.67298489477\\
65.5416666666667	0.19964	-459.506722319496\\
65.5416666666667	0.2013	-478.340459744222\\
65.5416666666667	0.20296	-497.174197168948\\
65.5416666666667	0.20462	-516.007934593674\\
65.5416666666667	0.20628	-534.841672018401\\
65.5416666666667	0.20794	-553.675409443127\\
65.5416666666667	0.2096	-572.509146867853\\
65.5416666666667	0.21126	-591.34288429258\\
65.5416666666667	0.21292	-610.176621717306\\
65.5416666666667	0.21458	-629.010359142032\\
65.5416666666667	0.21624	-647.844096566758\\
65.5416666666667	0.2179	-666.677833991484\\
65.5416666666667	0.21956	-685.511571416211\\
65.5416666666667	0.22122	-704.345308840937\\
65.5416666666667	0.22288	-723.179046265663\\
65.5416666666667	0.22454	-742.012783690389\\
65.5416666666667	0.2262	-760.846521115115\\
65.5416666666667	0.22786	-779.680258539842\\
65.5416666666667	0.22952	-798.513995964568\\
65.5416666666667	0.23118	-817.347733389294\\
65.5416666666667	0.23284	-836.18147081402\\
65.5416666666667	0.2345	-855.015208238746\\
65.5416666666667	0.23616	-873.848945663473\\
65.5416666666667	0.23782	-892.682683088199\\
65.5416666666667	0.23948	-911.516420512925\\
65.5416666666667	0.24114	-930.350157937652\\
65.5416666666667	0.2428	-949.183895362378\\
65.5416666666667	0.24446	-968.017632787104\\
65.5416666666667	0.24612	-986.85137021183\\
65.5416666666667	0.24778	-1005.68510763656\\
65.5416666666667	0.24944	-1024.51884506128\\
65.5416666666667	0.2511	-1043.35258248601\\
65.5416666666667	0.25276	-1062.18631991073\\
65.5416666666667	0.25442	-1081.02005733546\\
65.5416666666667	0.25608	-1099.85379476019\\
65.5416666666667	0.25774	-1118.68753218491\\
65.5416666666667	0.2594	-1137.52126960964\\
65.5416666666667	0.26106	-1156.35500703437\\
65.5416666666667	0.26272	-1175.18874445909\\
65.5416666666667	0.26438	-1194.02248188382\\
65.5416666666667	0.26604	-1212.85621930854\\
65.5416666666667	0.2677	-1231.68995673327\\
65.5416666666667	0.26936	-1250.523694158\\
65.5416666666667	0.27102	-1269.35743158272\\
65.5416666666667	0.27268	-1288.19116900745\\
65.5416666666667	0.27434	-1307.02490643218\\
65.5416666666667	0.276	-1325.8586438569\\
65.75	0.193	-381.896686752509\\
65.75	0.19466	-400.756269555898\\
65.75	0.19632	-419.615852359287\\
65.75	0.19798	-438.475435162677\\
65.75	0.19964	-457.335017966066\\
65.75	0.2013	-476.194600769456\\
65.75	0.20296	-495.054183572845\\
65.75	0.20462	-513.913766376234\\
65.75	0.20628	-532.773349179624\\
65.75	0.20794	-551.632931983013\\
65.75	0.2096	-570.492514786403\\
65.75	0.21126	-589.352097589792\\
65.75	0.21292	-608.211680393181\\
65.75	0.21458	-627.071263196571\\
65.75	0.21624	-645.93084599996\\
65.75	0.2179	-664.79042880335\\
65.75	0.21956	-683.650011606739\\
65.75	0.22122	-702.509594410128\\
65.75	0.22288	-721.369177213518\\
65.75	0.22454	-740.228760016907\\
65.75	0.2262	-759.088342820296\\
65.75	0.22786	-777.947925623686\\
65.75	0.22952	-796.807508427075\\
65.75	0.23118	-815.667091230465\\
65.75	0.23284	-834.526674033854\\
65.75	0.2345	-853.386256837243\\
65.75	0.23616	-872.245839640633\\
65.75	0.23782	-891.105422444022\\
65.75	0.23948	-909.965005247412\\
65.75	0.24114	-928.824588050801\\
65.75	0.2428	-947.68417085419\\
65.75	0.24446	-966.54375365758\\
65.75	0.24612	-985.403336460969\\
65.75	0.24778	-1004.26291926436\\
65.75	0.24944	-1023.12250206775\\
65.75	0.2511	-1041.98208487114\\
65.75	0.25276	-1060.84166767453\\
65.75	0.25442	-1079.70125047792\\
65.75	0.25608	-1098.56083328131\\
65.75	0.25774	-1117.42041608469\\
65.75	0.2594	-1136.27999888808\\
65.75	0.26106	-1155.13958169147\\
65.75	0.26272	-1173.99916449486\\
65.75	0.26438	-1192.85874729825\\
65.75	0.26604	-1211.71833010164\\
65.75	0.2677	-1230.57791290503\\
65.75	0.26936	-1249.43749570842\\
65.75	0.27102	-1268.29707851181\\
65.75	0.27268	-1287.1566613152\\
65.75	0.27434	-1306.01624411859\\
65.75	0.276	-1324.87582692198\\
65.9583333333333	0.193	-379.621600884427\\
65.9583333333333	0.19466	-398.507029066479\\
65.9583333333333	0.19632	-417.392457248532\\
65.9583333333333	0.19798	-436.277885430584\\
65.9583333333333	0.19964	-455.163313612637\\
65.9583333333333	0.2013	-474.048741794689\\
65.9583333333333	0.20296	-492.934169976742\\
65.9583333333333	0.20462	-511.819598158794\\
65.9583333333333	0.20628	-530.705026340847\\
65.9583333333333	0.20794	-549.5904545229\\
65.9583333333333	0.2096	-568.475882704952\\
65.9583333333333	0.21126	-587.361310887004\\
65.9583333333333	0.21292	-606.246739069057\\
65.9583333333333	0.21458	-625.132167251109\\
65.9583333333333	0.21624	-644.017595433162\\
65.9583333333333	0.2179	-662.903023615215\\
65.9583333333333	0.21956	-681.788451797267\\
65.9583333333333	0.22122	-700.67387997932\\
65.9583333333333	0.22288	-719.559308161372\\
65.9583333333333	0.22454	-738.444736343425\\
65.9583333333333	0.2262	-757.330164525477\\
65.9583333333333	0.22786	-776.21559270753\\
65.9583333333333	0.22952	-795.101020889583\\
65.9583333333333	0.23118	-813.986449071635\\
65.9583333333333	0.23284	-832.871877253688\\
65.9583333333333	0.2345	-851.75730543574\\
65.9583333333333	0.23616	-870.642733617793\\
65.9583333333333	0.23782	-889.528161799846\\
65.9583333333333	0.23948	-908.413589981898\\
65.9583333333333	0.24114	-927.29901816395\\
65.9583333333333	0.2428	-946.184446346003\\
65.9583333333333	0.24446	-965.069874528055\\
65.9583333333333	0.24612	-983.955302710108\\
65.9583333333333	0.24778	-1002.84073089216\\
65.9583333333333	0.24944	-1021.72615907421\\
65.9583333333333	0.2511	-1040.61158725627\\
65.9583333333333	0.25276	-1059.49701543832\\
65.9583333333333	0.25442	-1078.38244362037\\
65.9583333333333	0.25608	-1097.26787180242\\
65.9583333333333	0.25774	-1116.15329998448\\
65.9583333333333	0.2594	-1135.03872816653\\
65.9583333333333	0.26106	-1153.92415634858\\
65.9583333333333	0.26272	-1172.80958453063\\
65.9583333333333	0.26438	-1191.69501271269\\
65.9583333333333	0.26604	-1210.58044089474\\
65.9583333333333	0.2677	-1229.46586907679\\
65.9583333333333	0.26936	-1248.35129725884\\
65.9583333333333	0.27102	-1267.2367254409\\
65.9583333333333	0.27268	-1286.12215362295\\
65.9583333333333	0.27434	-1305.007581805\\
65.9583333333333	0.276	-1323.89300998705\\
66.1666666666667	0.193	-377.346515016344\\
66.1666666666667	0.19466	-396.25778857706\\
66.1666666666667	0.19632	-415.169062137775\\
66.1666666666667	0.19798	-434.080335698491\\
66.1666666666667	0.19964	-452.991609259207\\
66.1666666666667	0.2013	-471.902882819923\\
66.1666666666667	0.20296	-490.814156380638\\
66.1666666666667	0.20462	-509.725429941354\\
66.1666666666667	0.20628	-528.63670350207\\
66.1666666666667	0.20794	-547.547977062785\\
66.1666666666667	0.2096	-566.459250623501\\
66.1666666666667	0.21126	-585.370524184217\\
66.1666666666667	0.21292	-604.281797744932\\
66.1666666666667	0.21458	-623.193071305648\\
66.1666666666667	0.21624	-642.104344866364\\
66.1666666666667	0.2179	-661.01561842708\\
66.1666666666667	0.21956	-679.926891987795\\
66.1666666666667	0.22122	-698.838165548511\\
66.1666666666667	0.22288	-717.749439109227\\
66.1666666666667	0.22454	-736.660712669943\\
66.1666666666667	0.2262	-755.571986230658\\
66.1666666666667	0.22786	-774.483259791374\\
66.1666666666667	0.22952	-793.39453335209\\
66.1666666666667	0.23118	-812.305806912805\\
66.1666666666667	0.23284	-831.217080473521\\
66.1666666666667	0.2345	-850.128354034237\\
66.1666666666667	0.23616	-869.039627594952\\
66.1666666666667	0.23782	-887.950901155668\\
66.1666666666667	0.23948	-906.862174716384\\
66.1666666666667	0.24114	-925.7734482771\\
66.1666666666667	0.2428	-944.684721837815\\
66.1666666666667	0.24446	-963.595995398531\\
66.1666666666667	0.24612	-982.507268959247\\
66.1666666666667	0.24778	-1001.41854251996\\
66.1666666666667	0.24944	-1020.32981608068\\
66.1666666666667	0.2511	-1039.24108964139\\
66.1666666666667	0.25276	-1058.15236320211\\
66.1666666666667	0.25442	-1077.06363676283\\
66.1666666666667	0.25608	-1095.97491032354\\
66.1666666666667	0.25774	-1114.88618388426\\
66.1666666666667	0.2594	-1133.79745744497\\
66.1666666666667	0.26106	-1152.70873100569\\
66.1666666666667	0.26272	-1171.6200045664\\
66.1666666666667	0.26438	-1190.53127812712\\
66.1666666666667	0.26604	-1209.44255168784\\
66.1666666666667	0.2677	-1228.35382524855\\
66.1666666666667	0.26936	-1247.26509880927\\
66.1666666666667	0.27102	-1266.17637236998\\
66.1666666666667	0.27268	-1285.0876459307\\
66.1666666666667	0.27434	-1303.99891949141\\
66.1666666666667	0.276	-1322.91019305213\\
66.375	0.193	-375.071429148261\\
66.375	0.19466	-394.00854808764\\
66.375	0.19632	-412.945667027019\\
66.375	0.19798	-431.882785966398\\
66.375	0.19964	-450.819904905777\\
66.375	0.2013	-469.757023845156\\
66.375	0.20296	-488.694142784534\\
66.375	0.20462	-507.631261723913\\
66.375	0.20628	-526.568380663292\\
66.375	0.20794	-545.505499602671\\
66.375	0.2096	-564.44261854205\\
66.375	0.21126	-583.379737481429\\
66.375	0.21292	-602.316856420808\\
66.375	0.21458	-621.253975360186\\
66.375	0.21624	-640.191094299566\\
66.375	0.2179	-659.128213238944\\
66.375	0.21956	-678.065332178323\\
66.375	0.22122	-697.002451117702\\
66.375	0.22288	-715.939570057081\\
66.375	0.22454	-734.87668899646\\
66.375	0.2262	-753.813807935839\\
66.375	0.22786	-772.750926875218\\
66.375	0.22952	-791.688045814596\\
66.375	0.23118	-810.625164753975\\
66.375	0.23284	-829.562283693354\\
66.375	0.2345	-848.499402632733\\
66.375	0.23616	-867.436521572112\\
66.375	0.23782	-886.373640511491\\
66.375	0.23948	-905.31075945087\\
66.375	0.24114	-924.247878390249\\
66.375	0.2428	-943.184997329628\\
66.375	0.24446	-962.122116269006\\
66.375	0.24612	-981.059235208385\\
66.375	0.24778	-999.996354147764\\
66.375	0.24944	-1018.93347308714\\
66.375	0.2511	-1037.87059202652\\
66.375	0.25276	-1056.8077109659\\
66.375	0.25442	-1075.74482990528\\
66.375	0.25608	-1094.68194884466\\
66.375	0.25774	-1113.61906778404\\
66.375	0.2594	-1132.55618672342\\
66.375	0.26106	-1151.4933056628\\
66.375	0.26272	-1170.43042460217\\
66.375	0.26438	-1189.36754354155\\
66.375	0.26604	-1208.30466248093\\
66.375	0.2677	-1227.24178142031\\
66.375	0.26936	-1246.17890035969\\
66.375	0.27102	-1265.11601929907\\
66.375	0.27268	-1284.05313823845\\
66.375	0.27434	-1302.99025717783\\
66.375	0.276	-1321.9273761172\\
66.5833333333333	0.193	-372.796343280179\\
66.5833333333333	0.19466	-391.759307598221\\
66.5833333333333	0.19632	-410.722271916263\\
66.5833333333333	0.19798	-429.685236234305\\
66.5833333333333	0.19964	-448.648200552347\\
66.5833333333333	0.2013	-467.611164870389\\
66.5833333333333	0.20296	-486.574129188431\\
66.5833333333333	0.20462	-505.537093506473\\
66.5833333333333	0.20628	-524.500057824516\\
66.5833333333333	0.20794	-543.463022142557\\
66.5833333333333	0.2096	-562.425986460599\\
66.5833333333333	0.21126	-581.388950778641\\
66.5833333333333	0.21292	-600.351915096683\\
66.5833333333333	0.21458	-619.314879414725\\
66.5833333333333	0.21624	-638.277843732767\\
66.5833333333333	0.2179	-657.240808050809\\
66.5833333333333	0.21956	-676.203772368852\\
66.5833333333333	0.22122	-695.166736686894\\
66.5833333333333	0.22288	-714.129701004936\\
66.5833333333333	0.22454	-733.092665322978\\
66.5833333333333	0.2262	-752.05562964102\\
66.5833333333333	0.22786	-771.018593959062\\
66.5833333333333	0.22952	-789.981558277104\\
66.5833333333333	0.23118	-808.944522595146\\
66.5833333333333	0.23284	-827.907486913188\\
66.5833333333333	0.2345	-846.87045123123\\
66.5833333333333	0.23616	-865.833415549272\\
66.5833333333333	0.23782	-884.796379867314\\
66.5833333333333	0.23948	-903.759344185356\\
66.5833333333333	0.24114	-922.722308503398\\
66.5833333333333	0.2428	-941.68527282144\\
66.5833333333333	0.24446	-960.648237139482\\
66.5833333333333	0.24612	-979.611201457524\\
66.5833333333333	0.24778	-998.574165775566\\
66.5833333333333	0.24944	-1017.53713009361\\
66.5833333333333	0.2511	-1036.50009441165\\
66.5833333333333	0.25276	-1055.46305872969\\
66.5833333333333	0.25442	-1074.42602304773\\
66.5833333333333	0.25608	-1093.38898736578\\
66.5833333333333	0.25774	-1112.35195168382\\
66.5833333333333	0.2594	-1131.31491600186\\
66.5833333333333	0.26106	-1150.2778803199\\
66.5833333333333	0.26272	-1169.24084463794\\
66.5833333333333	0.26438	-1188.20380895599\\
66.5833333333333	0.26604	-1207.16677327403\\
66.5833333333333	0.2677	-1226.12973759207\\
66.5833333333333	0.26936	-1245.09270191011\\
66.5833333333333	0.27102	-1264.05566622815\\
66.5833333333333	0.27268	-1283.0186305462\\
66.5833333333333	0.27434	-1301.98159486424\\
66.5833333333333	0.276	-1320.94455918228\\
66.7916666666667	0.193	-370.521257412097\\
66.7916666666667	0.19466	-389.510067108802\\
66.7916666666667	0.19632	-408.498876805507\\
66.7916666666667	0.19798	-427.487686502212\\
66.7916666666667	0.19964	-446.476496198918\\
66.7916666666667	0.2013	-465.465305895623\\
66.7916666666667	0.20296	-484.454115592328\\
66.7916666666667	0.20462	-503.442925289033\\
66.7916666666667	0.20628	-522.431734985739\\
66.7916666666667	0.20794	-541.420544682444\\
66.7916666666667	0.2096	-560.409354379149\\
66.7916666666667	0.21126	-579.398164075854\\
66.7916666666667	0.21292	-598.386973772559\\
66.7916666666667	0.21458	-617.375783469264\\
66.7916666666667	0.21624	-636.364593165969\\
66.7916666666667	0.2179	-655.353402862675\\
66.7916666666667	0.21956	-674.34221255938\\
66.7916666666667	0.22122	-693.331022256085\\
66.7916666666667	0.22288	-712.31983195279\\
66.7916666666667	0.22454	-731.308641649495\\
66.7916666666667	0.2262	-750.297451346201\\
66.7916666666667	0.22786	-769.286261042906\\
66.7916666666667	0.22952	-788.275070739611\\
66.7916666666667	0.23118	-807.263880436316\\
66.7916666666667	0.23284	-826.252690133022\\
66.7916666666667	0.2345	-845.241499829727\\
66.7916666666667	0.23616	-864.230309526432\\
66.7916666666667	0.23782	-883.219119223137\\
66.7916666666667	0.23948	-902.207928919842\\
66.7916666666667	0.24114	-921.196738616547\\
66.7916666666667	0.2428	-940.185548313253\\
66.7916666666667	0.24446	-959.174358009958\\
66.7916666666667	0.24612	-978.163167706663\\
66.7916666666667	0.24778	-997.151977403368\\
66.7916666666667	0.24944	-1016.14078710007\\
66.7916666666667	0.2511	-1035.12959679678\\
66.7916666666667	0.25276	-1054.11840649348\\
66.7916666666667	0.25442	-1073.10721619019\\
66.7916666666667	0.25608	-1092.09602588689\\
66.7916666666667	0.25774	-1111.0848355836\\
66.7916666666667	0.2594	-1130.0736452803\\
66.7916666666667	0.26106	-1149.06245497701\\
66.7916666666667	0.26272	-1168.05126467371\\
66.7916666666667	0.26438	-1187.04007437042\\
66.7916666666667	0.26604	-1206.02888406713\\
66.7916666666667	0.2677	-1225.01769376383\\
66.7916666666667	0.26936	-1244.00650346054\\
66.7916666666667	0.27102	-1262.99531315724\\
66.7916666666667	0.27268	-1281.98412285395\\
66.7916666666667	0.27434	-1300.97293255065\\
66.7916666666667	0.276	-1319.96174224736\\
67	0.193	-368.246171544014\\
67	0.19466	-387.260826619382\\
67	0.19632	-406.275481694751\\
67	0.19798	-425.290136770119\\
67	0.19964	-444.304791845487\\
67	0.2013	-463.319446920856\\
67	0.20296	-482.334101996224\\
67	0.20462	-501.348757071592\\
67	0.20628	-520.363412146961\\
67	0.20794	-539.378067222329\\
67	0.2096	-558.392722297697\\
67	0.21126	-577.407377373066\\
67	0.21292	-596.422032448434\\
67	0.21458	-615.436687523802\\
67	0.21624	-634.451342599171\\
67	0.2179	-653.465997674539\\
67	0.21956	-672.480652749908\\
67	0.22122	-691.495307825276\\
67	0.22288	-710.509962900644\\
67	0.22454	-729.524617976013\\
67	0.2262	-748.539273051381\\
67	0.22786	-767.553928126749\\
67	0.22952	-786.568583202118\\
67	0.23118	-805.583238277486\\
67	0.23284	-824.597893352855\\
67	0.2345	-843.612548428223\\
67	0.23616	-862.627203503591\\
67	0.23782	-881.64185857896\\
67	0.23948	-900.656513654328\\
67	0.24114	-919.671168729697\\
67	0.2428	-938.685823805065\\
67	0.24446	-957.700478880433\\
67	0.24612	-976.715133955801\\
67	0.24778	-995.72978903117\\
67	0.24944	-1014.74444410654\\
67	0.2511	-1033.75909918191\\
67	0.25276	-1052.77375425727\\
67	0.25442	-1071.78840933264\\
67	0.25608	-1090.80306440801\\
67	0.25774	-1109.81771948338\\
67	0.2594	-1128.83237455875\\
67	0.26106	-1147.84702963412\\
67	0.26272	-1166.86168470948\\
67	0.26438	-1185.87633978485\\
67	0.26604	-1204.89099486022\\
67	0.2677	-1223.90564993559\\
67	0.26936	-1242.92030501096\\
67	0.27102	-1261.93496008633\\
67	0.27268	-1280.9496151617\\
67	0.27434	-1299.96427023706\\
67	0.276	-1318.97892531243\\
};
\end{axis}

\begin{axis}[%
width=4.927496cm,
height=3.050847cm,
at={(0cm,0cm)},
scale only axis,
xmin=57,
xmax=67,
tick align=outside,
xlabel={$L_{cut}$},
xmajorgrids,
ymin=0.193,
ymax=0.276,
ylabel={$D_{rlx}$},
ymajorgrids,
zmin=-5.27956478209876,
zmax=398.652222015539,
zlabel={$x_1$},
zmajorgrids,
view={-140}{50},
legend style={at={(1.03,1)},anchor=north west,legend cell align=left,align=left,draw=white!15!black}
]
\addplot3[only marks,mark=*,mark options={},mark size=1.5000pt,color=mycolor1] plot table[row sep=crcr,]{%
67	0.276	215.625873181055\\
66	0.255	175.265231898652\\
62	0.209	84.6051468899632\\
57	0.193	68.205010462529\\
};
\addplot3[only marks,mark=*,mark options={},mark size=1.5000pt,color=black] plot table[row sep=crcr,]{%
64	0.23	129.891288411582\\
};

\addplot3[%
surf,
opacity=0.7,
shader=interp,
colormap={mymap}{[1pt] rgb(0pt)=(0.0901961,0.239216,0.0745098); rgb(1pt)=(0.0945149,0.242058,0.0739522); rgb(2pt)=(0.0988592,0.244894,0.0733566); rgb(3pt)=(0.103229,0.247724,0.0727241); rgb(4pt)=(0.107623,0.250549,0.0720557); rgb(5pt)=(0.112043,0.253367,0.0713525); rgb(6pt)=(0.116487,0.25618,0.0706154); rgb(7pt)=(0.120956,0.258986,0.0698456); rgb(8pt)=(0.125449,0.261787,0.0690441); rgb(9pt)=(0.129967,0.264581,0.0682118); rgb(10pt)=(0.134508,0.26737,0.06735); rgb(11pt)=(0.139074,0.270152,0.0664596); rgb(12pt)=(0.143663,0.272929,0.0655416); rgb(13pt)=(0.148275,0.275699,0.0645971); rgb(14pt)=(0.152911,0.278463,0.0636271); rgb(15pt)=(0.15757,0.281221,0.0626328); rgb(16pt)=(0.162252,0.283973,0.0616151); rgb(17pt)=(0.166957,0.286719,0.060575); rgb(18pt)=(0.171685,0.289458,0.0595136); rgb(19pt)=(0.176434,0.292191,0.0584321); rgb(20pt)=(0.181207,0.294918,0.0573313); rgb(21pt)=(0.186001,0.297639,0.0562123); rgb(22pt)=(0.190817,0.300353,0.0550763); rgb(23pt)=(0.195655,0.303061,0.0539242); rgb(24pt)=(0.200514,0.305763,0.052757); rgb(25pt)=(0.205395,0.308459,0.0515759); rgb(26pt)=(0.210296,0.311149,0.0503624); rgb(27pt)=(0.215212,0.313846,0.0490067); rgb(28pt)=(0.220142,0.316548,0.0475043); rgb(29pt)=(0.22509,0.319254,0.0458704); rgb(30pt)=(0.230056,0.321962,0.0441205); rgb(31pt)=(0.235042,0.324671,0.04227); rgb(32pt)=(0.240048,0.327379,0.0403343); rgb(33pt)=(0.245078,0.330085,0.0383287); rgb(34pt)=(0.250131,0.332786,0.0362688); rgb(35pt)=(0.25521,0.335482,0.0341698); rgb(36pt)=(0.260317,0.33817,0.0320472); rgb(37pt)=(0.265451,0.340849,0.0299163); rgb(38pt)=(0.270616,0.343517,0.0277927); rgb(39pt)=(0.275813,0.346172,0.0256916); rgb(40pt)=(0.281043,0.348814,0.0236284); rgb(41pt)=(0.286307,0.35144,0.0216186); rgb(42pt)=(0.291607,0.354048,0.0196776); rgb(43pt)=(0.296945,0.356637,0.0178207); rgb(44pt)=(0.302322,0.359206,0.0160634); rgb(45pt)=(0.307739,0.361753,0.0144211); rgb(46pt)=(0.313198,0.364275,0.0129091); rgb(47pt)=(0.318701,0.366772,0.0115428); rgb(48pt)=(0.324249,0.369242,0.0103377); rgb(49pt)=(0.329843,0.371682,0.00930909); rgb(50pt)=(0.335485,0.374093,0.00847245); rgb(51pt)=(0.341176,0.376471,0.00784314); rgb(52pt)=(0.346925,0.378826,0.00732741); rgb(53pt)=(0.352735,0.381168,0.00682184); rgb(54pt)=(0.358605,0.383497,0.00632729); rgb(55pt)=(0.364532,0.385812,0.00584464); rgb(56pt)=(0.370516,0.388113,0.00537476); rgb(57pt)=(0.376552,0.390399,0.00491852); rgb(58pt)=(0.38264,0.39267,0.00447681); rgb(59pt)=(0.388777,0.394925,0.00405048); rgb(60pt)=(0.394962,0.397164,0.00364042); rgb(61pt)=(0.401191,0.399386,0.00324749); rgb(62pt)=(0.407464,0.401592,0.00287258); rgb(63pt)=(0.413777,0.40378,0.00251655); rgb(64pt)=(0.420129,0.40595,0.00218028); rgb(65pt)=(0.426518,0.408102,0.00186463); rgb(66pt)=(0.432942,0.410234,0.00157049); rgb(67pt)=(0.439399,0.412348,0.00129873); rgb(68pt)=(0.445885,0.414441,0.00105022); rgb(69pt)=(0.452401,0.416515,0.000825833); rgb(70pt)=(0.458942,0.418567,0.000626441); rgb(71pt)=(0.465508,0.420599,0.00045292); rgb(72pt)=(0.472096,0.422609,0.000306141); rgb(73pt)=(0.478704,0.424596,0.000186979); rgb(74pt)=(0.485331,0.426562,9.63073e-05); rgb(75pt)=(0.491973,0.428504,3.49981e-05); rgb(76pt)=(0.498628,0.430422,3.92506e-06); rgb(77pt)=(0.505323,0.432315,0); rgb(78pt)=(0.512206,0.434168,0); rgb(79pt)=(0.519282,0.435983,0); rgb(80pt)=(0.526529,0.437764,0); rgb(81pt)=(0.533922,0.439512,0); rgb(82pt)=(0.54144,0.441232,0); rgb(83pt)=(0.549059,0.442927,0); rgb(84pt)=(0.556756,0.444599,0); rgb(85pt)=(0.564508,0.446252,0); rgb(86pt)=(0.572292,0.447889,0); rgb(87pt)=(0.580084,0.449514,0); rgb(88pt)=(0.587863,0.451129,0); rgb(89pt)=(0.595604,0.452737,0); rgb(90pt)=(0.603284,0.454343,0); rgb(91pt)=(0.610882,0.455948,0); rgb(92pt)=(0.618373,0.457556,0); rgb(93pt)=(0.625734,0.459171,0); rgb(94pt)=(0.632943,0.460795,0); rgb(95pt)=(0.639976,0.462432,0); rgb(96pt)=(0.64681,0.464084,0); rgb(97pt)=(0.653423,0.465756,0); rgb(98pt)=(0.659791,0.46745,0); rgb(99pt)=(0.665891,0.469169,0); rgb(100pt)=(0.6717,0.470916,0); rgb(101pt)=(0.677195,0.472696,0); rgb(102pt)=(0.682353,0.47451,0); rgb(103pt)=(0.687242,0.476355,0); rgb(104pt)=(0.691952,0.478225,0); rgb(105pt)=(0.696497,0.480118,0); rgb(106pt)=(0.700887,0.482033,0); rgb(107pt)=(0.705134,0.483968,0); rgb(108pt)=(0.709251,0.485921,0); rgb(109pt)=(0.713249,0.487891,0); rgb(110pt)=(0.71714,0.489876,0); rgb(111pt)=(0.720936,0.491875,0); rgb(112pt)=(0.724649,0.493887,0); rgb(113pt)=(0.72829,0.495909,0); rgb(114pt)=(0.731872,0.49794,0); rgb(115pt)=(0.735406,0.499979,0); rgb(116pt)=(0.738904,0.502025,0); rgb(117pt)=(0.742378,0.504075,0); rgb(118pt)=(0.74584,0.506128,0); rgb(119pt)=(0.749302,0.508182,0); rgb(120pt)=(0.752775,0.510237,0); rgb(121pt)=(0.756272,0.51229,0); rgb(122pt)=(0.759804,0.514339,0); rgb(123pt)=(0.763384,0.516385,0); rgb(124pt)=(0.767022,0.518424,0); rgb(125pt)=(0.770731,0.520455,0); rgb(126pt)=(0.774523,0.522478,0); rgb(127pt)=(0.77841,0.524489,0); rgb(128pt)=(0.782391,0.526491,0); rgb(129pt)=(0.786402,0.528496,0); rgb(130pt)=(0.790431,0.530506,0); rgb(131pt)=(0.794478,0.532521,0); rgb(132pt)=(0.798541,0.534539,0); rgb(133pt)=(0.802619,0.53656,0); rgb(134pt)=(0.806712,0.538584,0); rgb(135pt)=(0.81082,0.540609,0); rgb(136pt)=(0.81494,0.542635,0); rgb(137pt)=(0.819074,0.54466,0); rgb(138pt)=(0.823219,0.546686,0); rgb(139pt)=(0.827374,0.548709,0); rgb(140pt)=(0.831541,0.55073,0); rgb(141pt)=(0.835716,0.552749,0); rgb(142pt)=(0.8399,0.554763,0); rgb(143pt)=(0.844092,0.556774,0); rgb(144pt)=(0.848292,0.558779,0); rgb(145pt)=(0.852497,0.560778,0); rgb(146pt)=(0.856708,0.562771,0); rgb(147pt)=(0.860924,0.564756,0); rgb(148pt)=(0.865143,0.566733,0); rgb(149pt)=(0.869366,0.568701,0); rgb(150pt)=(0.873592,0.57066,0); rgb(151pt)=(0.877819,0.572608,0); rgb(152pt)=(0.882047,0.574545,0); rgb(153pt)=(0.886275,0.576471,0); rgb(154pt)=(0.890659,0.578362,0); rgb(155pt)=(0.895333,0.580203,0); rgb(156pt)=(0.900258,0.581999,0); rgb(157pt)=(0.905397,0.583755,0); rgb(158pt)=(0.910711,0.585479,0); rgb(159pt)=(0.916164,0.587176,0); rgb(160pt)=(0.921717,0.588852,0); rgb(161pt)=(0.927333,0.590513,0); rgb(162pt)=(0.932974,0.592166,0); rgb(163pt)=(0.938602,0.593815,0); rgb(164pt)=(0.94418,0.595468,0); rgb(165pt)=(0.949669,0.59713,0); rgb(166pt)=(0.955033,0.598808,0); rgb(167pt)=(0.960233,0.600507,0); rgb(168pt)=(0.965232,0.602233,0); rgb(169pt)=(0.969992,0.603992,0); rgb(170pt)=(0.974475,0.605791,0); rgb(171pt)=(0.978643,0.607636,0); rgb(172pt)=(0.98246,0.609532,0); rgb(173pt)=(0.985886,0.611486,0); rgb(174pt)=(0.988885,0.613503,0); rgb(175pt)=(0.991419,0.61559,0); rgb(176pt)=(0.99345,0.617753,0); rgb(177pt)=(0.99494,0.619997,0); rgb(178pt)=(0.995851,0.622329,0); rgb(179pt)=(0.996226,0.624763,0); rgb(180pt)=(0.996512,0.627352,0); rgb(181pt)=(0.996788,0.630095,0); rgb(182pt)=(0.997053,0.632982,0); rgb(183pt)=(0.997308,0.636004,0); rgb(184pt)=(0.997552,0.639152,0); rgb(185pt)=(0.997785,0.642416,0); rgb(186pt)=(0.998006,0.645786,0); rgb(187pt)=(0.998217,0.649253,0); rgb(188pt)=(0.998416,0.652807,0); rgb(189pt)=(0.998605,0.656439,0); rgb(190pt)=(0.998781,0.660138,0); rgb(191pt)=(0.998946,0.663897,0); rgb(192pt)=(0.9991,0.667704,0); rgb(193pt)=(0.999242,0.67155,0); rgb(194pt)=(0.999372,0.675427,0); rgb(195pt)=(0.99949,0.679323,0); rgb(196pt)=(0.999596,0.68323,0); rgb(197pt)=(0.99969,0.687139,0); rgb(198pt)=(0.999771,0.691039,0); rgb(199pt)=(0.999841,0.694921,0); rgb(200pt)=(0.999898,0.698775,0); rgb(201pt)=(0.999942,0.702592,0); rgb(202pt)=(0.999974,0.706363,0); rgb(203pt)=(0.999994,0.710077,0); rgb(204pt)=(1,0.713725,0); rgb(205pt)=(1,0.717341,0); rgb(206pt)=(1,0.720963,0); rgb(207pt)=(1,0.724591,0); rgb(208pt)=(1,0.728226,0); rgb(209pt)=(1,0.731867,0); rgb(210pt)=(1,0.735514,0); rgb(211pt)=(1,0.739167,0); rgb(212pt)=(1,0.742827,0); rgb(213pt)=(1,0.746493,0); rgb(214pt)=(1,0.750165,0); rgb(215pt)=(1,0.753843,0); rgb(216pt)=(1,0.757527,0); rgb(217pt)=(1,0.761217,0); rgb(218pt)=(1,0.764913,0); rgb(219pt)=(1,0.768615,0); rgb(220pt)=(1,0.772324,0); rgb(221pt)=(1,0.776038,0); rgb(222pt)=(1,0.779758,0); rgb(223pt)=(1,0.783484,0); rgb(224pt)=(1,0.787215,0); rgb(225pt)=(1,0.790953,0); rgb(226pt)=(1,0.794696,0); rgb(227pt)=(1,0.798445,0); rgb(228pt)=(1,0.8022,0); rgb(229pt)=(1,0.805961,0); rgb(230pt)=(1,0.809727,0); rgb(231pt)=(1,0.8135,0); rgb(232pt)=(1,0.817278,0); rgb(233pt)=(1,0.821063,0); rgb(234pt)=(1,0.824854,0); rgb(235pt)=(1,0.828652,0); rgb(236pt)=(1,0.832455,0); rgb(237pt)=(1,0.836265,0); rgb(238pt)=(1,0.840081,0); rgb(239pt)=(1,0.843903,0); rgb(240pt)=(1,0.847732,0); rgb(241pt)=(1,0.851566,0); rgb(242pt)=(1,0.855406,0); rgb(243pt)=(1,0.859253,0); rgb(244pt)=(1,0.863106,0); rgb(245pt)=(1,0.866964,0); rgb(246pt)=(1,0.870829,0); rgb(247pt)=(1,0.8747,0); rgb(248pt)=(1,0.878577,0); rgb(249pt)=(1,0.88246,0); rgb(250pt)=(1,0.886349,0); rgb(251pt)=(1,0.890243,0); rgb(252pt)=(1,0.894144,0); rgb(253pt)=(1,0.898051,0); rgb(254pt)=(1,0.901964,0); rgb(255pt)=(1,0.905882,0)},
mesh/rows=49]
table[row sep=crcr,header=false] {%
%
57	0.193	68.2050104625255\\
57	0.19466	74.8139546935856\\
57	0.19632	81.4228989246462\\
57	0.19798	88.0318431557062\\
57	0.19964	94.6407873867668\\
57	0.2013	101.249731617827\\
57	0.20296	107.858675848887\\
57	0.20462	114.467620079947\\
57	0.20628	121.076564311008\\
57	0.20794	127.685508542068\\
57	0.2096	134.294452773128\\
57	0.21126	140.903397004188\\
57	0.21292	147.512341235249\\
57	0.21458	154.121285466309\\
57	0.21624	160.730229697369\\
57	0.2179	167.33917392843\\
57	0.21956	173.94811815949\\
57	0.22122	180.55706239055\\
57	0.22288	187.16600662161\\
57	0.22454	193.774950852671\\
57	0.2262	200.383895083731\\
57	0.22786	206.992839314791\\
57	0.22952	213.601783545852\\
57	0.23118	220.210727776912\\
57	0.23284	226.819672007972\\
57	0.2345	233.428616239032\\
57	0.23616	240.037560470093\\
57	0.23782	246.646504701153\\
57	0.23948	253.255448932213\\
57	0.24114	259.864393163273\\
57	0.2428	266.473337394334\\
57	0.24446	273.082281625394\\
57	0.24612	279.691225856455\\
57	0.24778	286.300170087515\\
57	0.24944	292.909114318575\\
57	0.2511	299.518058549635\\
57	0.25276	306.127002780695\\
57	0.25442	312.735947011756\\
57	0.25608	319.344891242816\\
57	0.25774	325.953835473876\\
57	0.2594	332.562779704937\\
57	0.26106	339.171723935997\\
57	0.26272	345.780668167057\\
57	0.26438	352.389612398117\\
57	0.26604	358.998556629178\\
57	0.2677	365.607500860238\\
57	0.26936	372.216445091298\\
57	0.27102	378.825389322358\\
57	0.27268	385.434333553419\\
57	0.27434	392.043277784479\\
57	0.276	398.652222015539\\
57.2083333333333	0.193	66.6740818115957\\
57.2083333333333	0.19466	73.2373836369934\\
57.2083333333333	0.19632	79.8006854623918\\
57.2083333333333	0.19798	86.3639872877893\\
57.2083333333333	0.19964	92.9272891131875\\
57.2083333333333	0.2013	99.4905909385852\\
57.2083333333333	0.20296	106.053892763983\\
57.2083333333333	0.20462	112.617194589381\\
57.2083333333333	0.20628	119.180496414779\\
57.2083333333333	0.20794	125.743798240177\\
57.2083333333333	0.2096	132.307100065574\\
57.2083333333333	0.21126	138.870401890972\\
57.2083333333333	0.21292	145.43370371637\\
57.2083333333333	0.21458	151.997005541768\\
57.2083333333333	0.21624	158.560307367165\\
57.2083333333333	0.2179	165.123609192563\\
57.2083333333333	0.21956	171.686911017961\\
57.2083333333333	0.22122	178.250212843359\\
57.2083333333333	0.22288	184.813514668757\\
57.2083333333333	0.22454	191.376816494155\\
57.2083333333333	0.2262	197.940118319553\\
57.2083333333333	0.22786	204.50342014495\\
57.2083333333333	0.22952	211.066721970348\\
57.2083333333333	0.23118	217.630023795746\\
57.2083333333333	0.23284	224.193325621144\\
57.2083333333333	0.2345	230.756627446542\\
57.2083333333333	0.23616	237.31992927194\\
57.2083333333333	0.23782	243.883231097338\\
57.2083333333333	0.23948	250.446532922735\\
57.2083333333333	0.24114	257.009834748133\\
57.2083333333333	0.2428	263.573136573531\\
57.2083333333333	0.24446	270.136438398929\\
57.2083333333333	0.24612	276.699740224327\\
57.2083333333333	0.24778	283.263042049724\\
57.2083333333333	0.24944	289.826343875123\\
57.2083333333333	0.2511	296.38964570052\\
57.2083333333333	0.25276	302.952947525918\\
57.2083333333333	0.25442	309.516249351316\\
57.2083333333333	0.25608	316.079551176714\\
57.2083333333333	0.25774	322.642853002112\\
57.2083333333333	0.2594	329.206154827509\\
57.2083333333333	0.26106	335.769456652907\\
57.2083333333333	0.26272	342.332758478305\\
57.2083333333333	0.26438	348.896060303703\\
57.2083333333333	0.26604	355.459362129101\\
57.2083333333333	0.2677	362.022663954499\\
57.2083333333333	0.26936	368.585965779896\\
57.2083333333333	0.27102	375.149267605294\\
57.2083333333333	0.27268	381.712569430692\\
57.2083333333333	0.27434	388.27587125609\\
57.2083333333333	0.276	394.839173081488\\
57.4166666666667	0.193	65.1431531606661\\
57.4166666666667	0.19466	71.6608125804014\\
57.4166666666667	0.19632	78.1784720001372\\
57.4166666666667	0.19798	84.6961314198722\\
57.4166666666667	0.19964	91.213790839608\\
57.4166666666667	0.2013	97.7314502593433\\
57.4166666666667	0.20296	104.249109679079\\
57.4166666666667	0.20462	110.766769098814\\
57.4166666666667	0.20628	117.284428518549\\
57.4166666666667	0.20794	123.802087938285\\
57.4166666666667	0.2096	130.31974735802\\
57.4166666666667	0.21126	136.837406777755\\
57.4166666666667	0.21292	143.355066197491\\
57.4166666666667	0.21458	149.872725617226\\
57.4166666666667	0.21624	156.390385036962\\
57.4166666666667	0.2179	162.908044456697\\
57.4166666666667	0.21956	169.425703876432\\
57.4166666666667	0.22122	175.943363296168\\
57.4166666666667	0.22288	182.461022715903\\
57.4166666666667	0.22454	188.978682135639\\
57.4166666666667	0.2262	195.496341555374\\
57.4166666666667	0.22786	202.01400097511\\
57.4166666666667	0.22952	208.531660394845\\
57.4166666666667	0.23118	215.04931981458\\
57.4166666666667	0.23284	221.566979234316\\
57.4166666666667	0.2345	228.084638654051\\
57.4166666666667	0.23616	234.602298073786\\
57.4166666666667	0.23782	241.119957493522\\
57.4166666666667	0.23948	247.637616913257\\
57.4166666666667	0.24114	254.155276332993\\
57.4166666666667	0.2428	260.672935752728\\
57.4166666666667	0.24446	267.190595172463\\
57.4166666666667	0.24612	273.708254592199\\
57.4166666666667	0.24778	280.225914011934\\
57.4166666666667	0.24944	286.74357343167\\
57.4166666666667	0.2511	293.261232851405\\
57.4166666666667	0.25276	299.77889227114\\
57.4166666666667	0.25442	306.296551690876\\
57.4166666666667	0.25608	312.814211110611\\
57.4166666666667	0.25774	319.331870530347\\
57.4166666666667	0.2594	325.849529950082\\
57.4166666666667	0.26106	332.367189369817\\
57.4166666666667	0.26272	338.884848789553\\
57.4166666666667	0.26438	345.402508209288\\
57.4166666666667	0.26604	351.920167629024\\
57.4166666666667	0.2677	358.437827048759\\
57.4166666666667	0.26936	364.955486468495\\
57.4166666666667	0.27102	371.47314588823\\
57.4166666666667	0.27268	377.990805307965\\
57.4166666666667	0.27434	384.508464727701\\
57.4166666666667	0.276	391.026124147436\\
57.625	0.193	63.6122245097363\\
57.625	0.19466	70.0842415238092\\
57.625	0.19632	76.5562585378825\\
57.625	0.19798	83.0282755519552\\
57.625	0.19964	89.5002925660285\\
57.625	0.2013	95.9723095801014\\
57.625	0.20296	102.444326594174\\
57.625	0.20462	108.916343608247\\
57.625	0.20628	115.38836062232\\
57.625	0.20794	121.860377636393\\
57.625	0.2096	128.332394650466\\
57.625	0.21126	134.804411664539\\
57.625	0.21292	141.276428678612\\
57.625	0.21458	147.748445692685\\
57.625	0.21624	154.220462706758\\
57.625	0.2179	160.692479720831\\
57.625	0.21956	167.164496734904\\
57.625	0.22122	173.636513748977\\
57.625	0.22288	180.10853076305\\
57.625	0.22454	186.580547777123\\
57.625	0.2262	193.052564791196\\
57.625	0.22786	199.524581805269\\
57.625	0.22952	205.996598819342\\
57.625	0.23118	212.468615833415\\
57.625	0.23284	218.940632847488\\
57.625	0.2345	225.412649861561\\
57.625	0.23616	231.884666875633\\
57.625	0.23782	238.356683889707\\
57.625	0.23948	244.82870090378\\
57.625	0.24114	251.300717917853\\
57.625	0.2428	257.772734931925\\
57.625	0.24446	264.244751945998\\
57.625	0.24612	270.716768960071\\
57.625	0.24778	277.188785974144\\
57.625	0.24944	283.660802988218\\
57.625	0.2511	290.13282000229\\
57.625	0.25276	296.604837016363\\
57.625	0.25442	303.076854030436\\
57.625	0.25608	309.548871044509\\
57.625	0.25774	316.020888058582\\
57.625	0.2594	322.492905072655\\
57.625	0.26106	328.964922086728\\
57.625	0.26272	335.436939100801\\
57.625	0.26438	341.908956114873\\
57.625	0.26604	348.380973128947\\
57.625	0.2677	354.85299014302\\
57.625	0.26936	361.325007157092\\
57.625	0.27102	367.797024171166\\
57.625	0.27268	374.269041185238\\
57.625	0.27434	380.741058199311\\
57.625	0.276	387.213075213385\\
57.8333333333333	0.193	62.0812958588067\\
57.8333333333333	0.19466	68.5076704672172\\
57.8333333333333	0.19632	74.9340450756281\\
57.8333333333333	0.19798	81.3604196840383\\
57.8333333333333	0.19964	87.786794292449\\
57.8333333333333	0.2013	94.2131689008595\\
57.8333333333333	0.20296	100.63954350927\\
57.8333333333333	0.20462	107.06591811768\\
57.8333333333333	0.20628	113.492292726091\\
57.8333333333333	0.20794	119.918667334502\\
57.8333333333333	0.2096	126.345041942912\\
57.8333333333333	0.21126	132.771416551323\\
57.8333333333333	0.21292	139.197791159733\\
57.8333333333333	0.21458	145.624165768143\\
57.8333333333333	0.21624	152.050540376554\\
57.8333333333333	0.2179	158.476914984964\\
57.8333333333333	0.21956	164.903289593375\\
57.8333333333333	0.22122	171.329664201786\\
57.8333333333333	0.22288	177.756038810196\\
57.8333333333333	0.22454	184.182413418607\\
57.8333333333333	0.2262	190.608788027017\\
57.8333333333333	0.22786	197.035162635428\\
57.8333333333333	0.22952	203.461537243838\\
57.8333333333333	0.23118	209.887911852249\\
57.8333333333333	0.23284	216.31428646066\\
57.8333333333333	0.2345	222.74066106907\\
57.8333333333333	0.23616	229.16703567748\\
57.8333333333333	0.23782	235.593410285891\\
57.8333333333333	0.23948	242.019784894301\\
57.8333333333333	0.24114	248.446159502712\\
57.8333333333333	0.2428	254.872534111122\\
57.8333333333333	0.24446	261.298908719533\\
57.8333333333333	0.24612	267.725283327944\\
57.8333333333333	0.24778	274.151657936354\\
57.8333333333333	0.24944	280.578032544765\\
57.8333333333333	0.2511	287.004407153175\\
57.8333333333333	0.25276	293.430781761585\\
57.8333333333333	0.25442	299.857156369996\\
57.8333333333333	0.25608	306.283530978407\\
57.8333333333333	0.25774	312.709905586817\\
57.8333333333333	0.2594	319.136280195228\\
57.8333333333333	0.26106	325.562654803638\\
57.8333333333333	0.26272	331.989029412049\\
57.8333333333333	0.26438	338.415404020459\\
57.8333333333333	0.26604	344.84177862887\\
57.8333333333333	0.2677	351.268153237281\\
57.8333333333333	0.26936	357.694527845691\\
57.8333333333333	0.27102	364.120902454101\\
57.8333333333333	0.27268	370.547277062512\\
57.8333333333333	0.27434	376.973651670922\\
57.8333333333333	0.276	383.400026279333\\
58.0416666666667	0.193	60.5503672078771\\
58.0416666666667	0.19466	66.9310994106252\\
58.0416666666667	0.19632	73.3118316133737\\
58.0416666666667	0.19798	79.6925638161215\\
58.0416666666667	0.19964	86.07329601887\\
58.0416666666667	0.2013	92.4540282216178\\
58.0416666666667	0.20296	98.8347604243659\\
58.0416666666667	0.20462	105.215492627114\\
58.0416666666667	0.20628	111.596224829862\\
58.0416666666667	0.20794	117.97695703261\\
58.0416666666667	0.2096	124.357689235358\\
58.0416666666667	0.21126	130.738421438106\\
58.0416666666667	0.21292	137.119153640854\\
58.0416666666667	0.21458	143.499885843602\\
58.0416666666667	0.21624	149.88061804635\\
58.0416666666667	0.2179	156.261350249099\\
58.0416666666667	0.21956	162.642082451847\\
58.0416666666667	0.22122	169.022814654595\\
58.0416666666667	0.22288	175.403546857343\\
58.0416666666667	0.22454	181.784279060091\\
58.0416666666667	0.2262	188.165011262839\\
58.0416666666667	0.22786	194.545743465587\\
58.0416666666667	0.22952	200.926475668335\\
58.0416666666667	0.23118	207.307207871083\\
58.0416666666667	0.23284	213.687940073831\\
58.0416666666667	0.2345	220.068672276579\\
58.0416666666667	0.23616	226.449404479328\\
58.0416666666667	0.23782	232.830136682076\\
58.0416666666667	0.23948	239.210868884824\\
58.0416666666667	0.24114	245.591601087571\\
58.0416666666667	0.2428	251.97233329032\\
58.0416666666667	0.24446	258.353065493068\\
58.0416666666667	0.24612	264.733797695817\\
58.0416666666667	0.24778	271.114529898564\\
58.0416666666667	0.24944	277.495262101312\\
58.0416666666667	0.2511	283.87599430406\\
58.0416666666667	0.25276	290.256726506808\\
58.0416666666667	0.25442	296.637458709557\\
58.0416666666667	0.25608	303.018190912305\\
58.0416666666667	0.25774	309.398923115053\\
58.0416666666667	0.2594	315.779655317801\\
58.0416666666667	0.26106	322.160387520549\\
58.0416666666667	0.26272	328.541119723297\\
58.0416666666667	0.26438	334.921851926045\\
58.0416666666667	0.26604	341.302584128794\\
58.0416666666667	0.2677	347.683316331541\\
58.0416666666667	0.26936	354.06404853429\\
58.0416666666667	0.27102	360.444780737037\\
58.0416666666667	0.27268	366.825512939786\\
58.0416666666667	0.27434	373.206245142534\\
58.0416666666667	0.276	379.586977345281\\
58.25	0.193	59.0194385569473\\
58.25	0.19466	65.3545283540329\\
58.25	0.19632	71.6896181511188\\
58.25	0.19798	78.0247079482042\\
58.25	0.19964	84.3597977452903\\
58.25	0.2013	90.6948875423759\\
58.25	0.20296	97.0299773394613\\
58.25	0.20462	103.365067136547\\
58.25	0.20628	109.700156933633\\
58.25	0.20794	116.035246730718\\
58.25	0.2096	122.370336527804\\
58.25	0.21126	128.705426324889\\
58.25	0.21292	135.040516121975\\
58.25	0.21458	141.375605919061\\
58.25	0.21624	147.710695716146\\
58.25	0.2179	154.045785513232\\
58.25	0.21956	160.380875310318\\
58.25	0.22122	166.715965107404\\
58.25	0.22288	173.051054904489\\
58.25	0.22454	179.386144701575\\
58.25	0.2262	185.721234498661\\
58.25	0.22786	192.056324295746\\
58.25	0.22952	198.391414092832\\
58.25	0.23118	204.726503889917\\
58.25	0.23284	211.061593687003\\
58.25	0.2345	217.396683484089\\
58.25	0.23616	223.731773281174\\
58.25	0.23782	230.06686307826\\
58.25	0.23948	236.401952875346\\
58.25	0.24114	242.737042672431\\
58.25	0.2428	249.072132469517\\
58.25	0.24446	255.407222266602\\
58.25	0.24612	261.742312063688\\
58.25	0.24778	268.077401860774\\
58.25	0.24944	274.41249165786\\
58.25	0.2511	280.747581454945\\
58.25	0.25276	287.082671252031\\
58.25	0.25442	293.417761049117\\
58.25	0.25608	299.752850846202\\
58.25	0.25774	306.087940643288\\
58.25	0.2594	312.423030440373\\
58.25	0.26106	318.758120237459\\
58.25	0.26272	325.093210034544\\
58.25	0.26438	331.42829983163\\
58.25	0.26604	337.763389628716\\
58.25	0.2677	344.098479425802\\
58.25	0.26936	350.433569222887\\
58.25	0.27102	356.768659019973\\
58.25	0.27268	363.103748817058\\
58.25	0.27434	369.438838614144\\
58.25	0.276	375.77392841123\\
58.4583333333333	0.193	57.4885099060177\\
58.4583333333333	0.19466	63.7779572974407\\
58.4583333333333	0.19632	70.0674046888644\\
58.4583333333333	0.19798	76.3568520802874\\
58.4583333333333	0.19964	82.6462994717108\\
58.4583333333333	0.2013	88.935746863134\\
58.4583333333333	0.20296	95.2251942545572\\
58.4583333333333	0.20462	101.51464164598\\
58.4583333333333	0.20628	107.804089037404\\
58.4583333333333	0.20794	114.093536428827\\
58.4583333333333	0.2096	120.38298382025\\
58.4583333333333	0.21126	126.672431211673\\
58.4583333333333	0.21292	132.961878603096\\
58.4583333333333	0.21458	139.251325994519\\
58.4583333333333	0.21624	145.540773385943\\
58.4583333333333	0.2179	151.830220777366\\
58.4583333333333	0.21956	158.119668168789\\
58.4583333333333	0.22122	164.409115560213\\
58.4583333333333	0.22288	170.698562951635\\
58.4583333333333	0.22454	176.988010343059\\
58.4583333333333	0.2262	183.277457734482\\
58.4583333333333	0.22786	189.566905125905\\
58.4583333333333	0.22952	195.856352517328\\
58.4583333333333	0.23118	202.145799908752\\
58.4583333333333	0.23284	208.435247300175\\
58.4583333333333	0.2345	214.724694691598\\
58.4583333333333	0.23616	221.014142083021\\
58.4583333333333	0.23782	227.303589474445\\
58.4583333333333	0.23948	233.593036865868\\
58.4583333333333	0.24114	239.882484257291\\
58.4583333333333	0.2428	246.171931648714\\
58.4583333333333	0.24446	252.461379040137\\
58.4583333333333	0.24612	258.75082643156\\
58.4583333333333	0.24778	265.040273822983\\
58.4583333333333	0.24944	271.329721214407\\
58.4583333333333	0.2511	277.61916860583\\
58.4583333333333	0.25276	283.908615997253\\
58.4583333333333	0.25442	290.198063388676\\
58.4583333333333	0.25608	296.4875107801\\
58.4583333333333	0.25774	302.776958171523\\
58.4583333333333	0.2594	309.066405562946\\
58.4583333333333	0.26106	315.355852954369\\
58.4583333333333	0.26272	321.645300345792\\
58.4583333333333	0.26438	327.934747737215\\
58.4583333333333	0.26604	334.224195128639\\
58.4583333333333	0.2677	340.513642520062\\
58.4583333333333	0.26936	346.803089911485\\
58.4583333333333	0.27102	353.092537302908\\
58.4583333333333	0.27268	359.381984694332\\
58.4583333333333	0.27434	365.671432085755\\
58.4583333333333	0.276	371.960879477178\\
58.6666666666667	0.193	55.9575812550881\\
58.6666666666667	0.19466	62.2013862408489\\
58.6666666666667	0.19632	68.4451912266099\\
58.6666666666667	0.19798	74.6889962123705\\
58.6666666666667	0.19964	80.9328011981315\\
58.6666666666667	0.2013	87.1766061838923\\
58.6666666666667	0.20296	93.4204111696529\\
58.6666666666667	0.20462	99.6642161554137\\
58.6666666666667	0.20628	105.908021141175\\
58.6666666666667	0.20794	112.151826126935\\
58.6666666666667	0.2096	118.395631112696\\
58.6666666666667	0.21126	124.639436098457\\
58.6666666666667	0.21292	130.883241084217\\
58.6666666666667	0.21458	137.127046069978\\
58.6666666666667	0.21624	143.370851055739\\
58.6666666666667	0.2179	149.6146560415\\
58.6666666666667	0.21956	155.85846102726\\
58.6666666666667	0.22122	162.102266013022\\
58.6666666666667	0.22288	168.346070998782\\
58.6666666666667	0.22454	174.589875984543\\
58.6666666666667	0.2262	180.833680970304\\
58.6666666666667	0.22786	187.077485956064\\
58.6666666666667	0.22952	193.321290941825\\
58.6666666666667	0.23118	199.565095927586\\
58.6666666666667	0.23284	205.808900913347\\
58.6666666666667	0.2345	212.052705899108\\
58.6666666666667	0.23616	218.296510884868\\
58.6666666666667	0.23782	224.540315870629\\
58.6666666666667	0.23948	230.78412085639\\
58.6666666666667	0.24114	237.027925842151\\
58.6666666666667	0.2428	243.271730827911\\
58.6666666666667	0.24446	249.515535813672\\
58.6666666666667	0.24612	255.759340799433\\
58.6666666666667	0.24778	262.003145785194\\
58.6666666666667	0.24944	268.246950770955\\
58.6666666666667	0.2511	274.490755756715\\
58.6666666666667	0.25276	280.734560742476\\
58.6666666666667	0.25442	286.978365728237\\
58.6666666666667	0.25608	293.222170713998\\
58.6666666666667	0.25774	299.465975699758\\
58.6666666666667	0.2594	305.709780685519\\
58.6666666666667	0.26106	311.95358567128\\
58.6666666666667	0.26272	318.197390657041\\
58.6666666666667	0.26438	324.441195642801\\
58.6666666666667	0.26604	330.685000628563\\
58.6666666666667	0.2677	336.928805614323\\
58.6666666666667	0.26936	343.172610600084\\
58.6666666666667	0.27102	349.416415585844\\
58.6666666666667	0.27268	355.660220571605\\
58.6666666666667	0.27434	361.904025557366\\
58.6666666666667	0.276	368.147830543127\\
58.875	0.193	54.4266526041586\\
58.875	0.19466	60.6248151842567\\
58.875	0.19632	66.8229777643555\\
58.875	0.19798	73.0211403444534\\
58.875	0.19964	79.2193029245523\\
58.875	0.2013	85.4174655046504\\
58.875	0.20296	91.6156280847488\\
58.875	0.20462	97.8137906648469\\
58.875	0.20628	104.011953244946\\
58.875	0.20794	110.210115825044\\
58.875	0.2096	116.408278405142\\
58.875	0.21126	122.60644098524\\
58.875	0.21292	128.804603565339\\
58.875	0.21458	135.002766145437\\
58.875	0.21624	141.200928725535\\
58.875	0.2179	147.399091305633\\
58.875	0.21956	153.597253885732\\
58.875	0.22122	159.79541646583\\
58.875	0.22288	165.993579045929\\
58.875	0.22454	172.191741626027\\
58.875	0.2262	178.389904206125\\
58.875	0.22786	184.588066786224\\
58.875	0.22952	190.786229366322\\
58.875	0.23118	196.98439194642\\
58.875	0.23284	203.182554526519\\
58.875	0.2345	209.380717106617\\
58.875	0.23616	215.578879686715\\
58.875	0.23782	221.777042266813\\
58.875	0.23948	227.975204846912\\
58.875	0.24114	234.17336742701\\
58.875	0.2428	240.371530007108\\
58.875	0.24446	246.569692587206\\
58.875	0.24612	252.767855167306\\
58.875	0.24778	258.966017747404\\
58.875	0.24944	265.164180327502\\
58.875	0.2511	271.3623429076\\
58.875	0.25276	277.560505487699\\
58.875	0.25442	283.758668067797\\
58.875	0.25608	289.956830647896\\
58.875	0.25774	296.154993227994\\
58.875	0.2594	302.353155808092\\
58.875	0.26106	308.55131838819\\
58.875	0.26272	314.749480968289\\
58.875	0.26438	320.947643548387\\
58.875	0.26604	327.145806128486\\
58.875	0.2677	333.343968708584\\
58.875	0.26936	339.542131288682\\
58.875	0.27102	345.740293868781\\
58.875	0.27268	351.938456448879\\
58.875	0.27434	358.136619028977\\
58.875	0.276	364.334781609075\\
59.0833333333333	0.193	52.8957239532288\\
59.0833333333333	0.19466	59.0482441276647\\
59.0833333333333	0.19632	65.2007643021009\\
59.0833333333333	0.19798	71.3532844765364\\
59.0833333333333	0.19964	77.5058046509728\\
59.0833333333333	0.2013	83.6583248254085\\
59.0833333333333	0.20296	89.8108449998444\\
59.0833333333333	0.20462	95.9633651742802\\
59.0833333333333	0.20628	102.115885348716\\
59.0833333333333	0.20794	108.268405523152\\
59.0833333333333	0.2096	114.420925697588\\
59.0833333333333	0.21126	120.573445872024\\
59.0833333333333	0.21292	126.725966046459\\
59.0833333333333	0.21458	132.878486220895\\
59.0833333333333	0.21624	139.031006395331\\
59.0833333333333	0.2179	145.183526569767\\
59.0833333333333	0.21956	151.336046744203\\
59.0833333333333	0.22122	157.488566918639\\
59.0833333333333	0.22288	163.641087093075\\
59.0833333333333	0.22454	169.793607267511\\
59.0833333333333	0.2262	175.946127441947\\
59.0833333333333	0.22786	182.098647616383\\
59.0833333333333	0.22952	188.251167790819\\
59.0833333333333	0.23118	194.403687965255\\
59.0833333333333	0.23284	200.55620813969\\
59.0833333333333	0.2345	206.708728314126\\
59.0833333333333	0.23616	212.861248488562\\
59.0833333333333	0.23782	219.013768662998\\
59.0833333333333	0.23948	225.166288837434\\
59.0833333333333	0.24114	231.31880901187\\
59.0833333333333	0.2428	237.471329186306\\
59.0833333333333	0.24446	243.623849360741\\
59.0833333333333	0.24612	249.776369535178\\
59.0833333333333	0.24778	255.928889709613\\
59.0833333333333	0.24944	262.08140988405\\
59.0833333333333	0.2511	268.233930058485\\
59.0833333333333	0.25276	274.386450232921\\
59.0833333333333	0.25442	280.538970407357\\
59.0833333333333	0.25608	286.691490581793\\
59.0833333333333	0.25774	292.844010756229\\
59.0833333333333	0.2594	298.996530930665\\
59.0833333333333	0.26106	305.149051105101\\
59.0833333333333	0.26272	311.301571279536\\
59.0833333333333	0.26438	317.454091453972\\
59.0833333333333	0.26604	323.606611628409\\
59.0833333333333	0.2677	329.759131802844\\
59.0833333333333	0.26936	335.91165197728\\
59.0833333333333	0.27102	342.064172151716\\
59.0833333333333	0.27268	348.216692326152\\
59.0833333333333	0.27434	354.369212500588\\
59.0833333333333	0.276	360.521732675024\\
59.2916666666667	0.193	51.3647953022989\\
59.2916666666667	0.19466	57.4716730710722\\
59.2916666666667	0.19632	63.578550839846\\
59.2916666666667	0.19798	69.6854286086193\\
59.2916666666667	0.19964	75.7923063773931\\
59.2916666666667	0.2013	81.8991841461664\\
59.2916666666667	0.20296	88.0060619149399\\
59.2916666666667	0.20462	94.1129396837132\\
59.2916666666667	0.20628	100.219817452487\\
59.2916666666667	0.20794	106.32669522126\\
59.2916666666667	0.2096	112.433572990034\\
59.2916666666667	0.21126	118.540450758807\\
59.2916666666667	0.21292	124.64732852758\\
59.2916666666667	0.21458	130.754206296354\\
59.2916666666667	0.21624	136.861084065127\\
59.2916666666667	0.2179	142.967961833901\\
59.2916666666667	0.21956	149.074839602674\\
59.2916666666667	0.22122	155.181717371448\\
59.2916666666667	0.22288	161.288595140221\\
59.2916666666667	0.22454	167.395472908995\\
59.2916666666667	0.2262	173.502350677768\\
59.2916666666667	0.22786	179.609228446542\\
59.2916666666667	0.22952	185.716106215315\\
59.2916666666667	0.23118	191.822983984089\\
59.2916666666667	0.23284	197.929861752862\\
59.2916666666667	0.2345	204.036739521635\\
59.2916666666667	0.23616	210.143617290409\\
59.2916666666667	0.23782	216.250495059182\\
59.2916666666667	0.23948	222.357372827956\\
59.2916666666667	0.24114	228.464250596729\\
59.2916666666667	0.2428	234.571128365503\\
59.2916666666667	0.24446	240.678006134276\\
59.2916666666667	0.24612	246.78488390305\\
59.2916666666667	0.24778	252.891761671823\\
59.2916666666667	0.24944	258.998639440597\\
59.2916666666667	0.2511	265.10551720937\\
59.2916666666667	0.25276	271.212394978143\\
59.2916666666667	0.25442	277.319272746917\\
59.2916666666667	0.25608	283.426150515691\\
59.2916666666667	0.25774	289.533028284464\\
59.2916666666667	0.2594	295.639906053237\\
59.2916666666667	0.26106	301.746783822011\\
59.2916666666667	0.26272	307.853661590784\\
59.2916666666667	0.26438	313.960539359558\\
59.2916666666667	0.26604	320.067417128331\\
59.2916666666667	0.2677	326.174294897105\\
59.2916666666667	0.26936	332.281172665878\\
59.2916666666667	0.27102	338.388050434652\\
59.2916666666667	0.27268	344.494928203425\\
59.2916666666667	0.27434	350.601805972198\\
59.2916666666667	0.276	356.708683740972\\
59.5	0.193	49.8338666513694\\
59.5	0.19466	55.8951020144805\\
59.5	0.19632	61.9563373775918\\
59.5	0.19798	68.0175727407025\\
59.5	0.19964	74.0788081038138\\
59.5	0.2013	80.1400434669249\\
59.5	0.20296	86.2012788300358\\
59.5	0.20462	92.2625141931467\\
59.5	0.20628	98.3237495562578\\
59.5	0.20794	104.384984919369\\
59.5	0.2096	110.44622028248\\
59.5	0.21126	116.507455645591\\
59.5	0.21292	122.568691008702\\
59.5	0.21458	128.629926371813\\
59.5	0.21624	134.691161734924\\
59.5	0.2179	140.752397098035\\
59.5	0.21956	146.813632461146\\
59.5	0.22122	152.874867824257\\
59.5	0.22288	158.936103187368\\
59.5	0.22454	164.997338550479\\
59.5	0.2262	171.05857391359\\
59.5	0.22786	177.119809276701\\
59.5	0.22952	183.181044639812\\
59.5	0.23118	189.242280002923\\
59.5	0.23284	195.303515366034\\
59.5	0.2345	201.364750729145\\
59.5	0.23616	207.425986092256\\
59.5	0.23782	213.487221455367\\
59.5	0.23948	219.548456818478\\
59.5	0.24114	225.609692181589\\
59.5	0.2428	231.6709275447\\
59.5	0.24446	237.732162907811\\
59.5	0.24612	243.793398270922\\
59.5	0.24778	249.854633634033\\
59.5	0.24944	255.915868997144\\
59.5	0.2511	261.977104360255\\
59.5	0.25276	268.038339723366\\
59.5	0.25442	274.099575086477\\
59.5	0.25608	280.160810449589\\
59.5	0.25774	286.222045812699\\
59.5	0.2594	292.28328117581\\
59.5	0.26106	298.344516538921\\
59.5	0.26272	304.405751902032\\
59.5	0.26438	310.466987265143\\
59.5	0.26604	316.528222628255\\
59.5	0.2677	322.589457991366\\
59.5	0.26936	328.650693354477\\
59.5	0.27102	334.711928717587\\
59.5	0.27268	340.773164080698\\
59.5	0.27434	346.834399443809\\
59.5	0.276	352.89563480692\\
59.7083333333333	0.193	48.3029380004398\\
59.7083333333333	0.19466	54.3185309578882\\
59.7083333333333	0.19632	60.3341239153372\\
59.7083333333333	0.19798	66.3497168727854\\
59.7083333333333	0.19964	72.3653098302345\\
59.7083333333333	0.2013	78.380902787683\\
59.7083333333333	0.20296	84.3964957451315\\
59.7083333333333	0.20462	90.4120887025799\\
59.7083333333333	0.20628	96.4276816600286\\
59.7083333333333	0.20794	102.443274617477\\
59.7083333333333	0.2096	108.458867574926\\
59.7083333333333	0.21126	114.474460532374\\
59.7083333333333	0.21292	120.490053489823\\
59.7083333333333	0.21458	126.505646447271\\
59.7083333333333	0.21624	132.52123940472\\
59.7083333333333	0.2179	138.536832362169\\
59.7083333333333	0.21956	144.552425319617\\
59.7083333333333	0.22122	150.568018277066\\
59.7083333333333	0.22288	156.583611234514\\
59.7083333333333	0.22454	162.599204191963\\
59.7083333333333	0.2262	168.614797149412\\
59.7083333333333	0.22786	174.63039010686\\
59.7083333333333	0.22952	180.645983064309\\
59.7083333333333	0.23118	186.661576021757\\
59.7083333333333	0.23284	192.677168979206\\
59.7083333333333	0.2345	198.692761936655\\
59.7083333333333	0.23616	204.708354894103\\
59.7083333333333	0.23782	210.723947851551\\
59.7083333333333	0.23948	216.739540809\\
59.7083333333333	0.24114	222.755133766448\\
59.7083333333333	0.2428	228.770726723897\\
59.7083333333333	0.24446	234.786319681346\\
59.7083333333333	0.24612	240.801912638795\\
59.7083333333333	0.24778	246.817505596243\\
59.7083333333333	0.24944	252.833098553692\\
59.7083333333333	0.2511	258.84869151114\\
59.7083333333333	0.25276	264.864284468589\\
59.7083333333333	0.25442	270.879877426038\\
59.7083333333333	0.25608	276.895470383486\\
59.7083333333333	0.25774	282.911063340935\\
59.7083333333333	0.2594	288.926656298383\\
59.7083333333333	0.26106	294.942249255832\\
59.7083333333333	0.26272	300.95784221328\\
59.7083333333333	0.26438	306.973435170729\\
59.7083333333333	0.26604	312.989028128178\\
59.7083333333333	0.2677	319.004621085626\\
59.7083333333333	0.26936	325.020214043075\\
59.7083333333333	0.27102	331.035807000523\\
59.7083333333333	0.27268	337.051399957972\\
59.7083333333333	0.27434	343.066992915421\\
59.7083333333333	0.276	349.082585872869\\
59.9166666666667	0.193	46.77200934951\\
59.9166666666667	0.19466	52.741959901296\\
59.9166666666667	0.19632	58.7119104530825\\
59.9166666666667	0.19798	64.6818610048686\\
59.9166666666667	0.19964	70.651811556655\\
59.9166666666667	0.2013	76.6217621084411\\
59.9166666666667	0.20296	82.5917126602271\\
59.9166666666667	0.20462	88.5616632120132\\
59.9166666666667	0.20628	94.5316137637994\\
59.9166666666667	0.20794	100.501564315585\\
59.9166666666667	0.2096	106.471514867372\\
59.9166666666667	0.21126	112.441465419158\\
59.9166666666667	0.21292	118.411415970944\\
59.9166666666667	0.21458	124.38136652273\\
59.9166666666667	0.21624	130.351317074516\\
59.9166666666667	0.2179	136.321267626302\\
59.9166666666667	0.21956	142.291218178088\\
59.9166666666667	0.22122	148.261168729875\\
59.9166666666667	0.22288	154.231119281661\\
59.9166666666667	0.22454	160.201069833447\\
59.9166666666667	0.2262	166.171020385233\\
59.9166666666667	0.22786	172.140970937019\\
59.9166666666667	0.22952	178.110921488805\\
59.9166666666667	0.23118	184.080872040592\\
59.9166666666667	0.23284	190.050822592378\\
59.9166666666667	0.2345	196.020773144164\\
59.9166666666667	0.23616	201.99072369595\\
59.9166666666667	0.23782	207.960674247736\\
59.9166666666667	0.23948	213.930624799522\\
59.9166666666667	0.24114	219.900575351308\\
59.9166666666667	0.2428	225.870525903094\\
59.9166666666667	0.24446	231.84047645488\\
59.9166666666667	0.24612	237.810427006667\\
59.9166666666667	0.24778	243.780377558453\\
59.9166666666667	0.24944	249.750328110239\\
59.9166666666667	0.2511	255.720278662025\\
59.9166666666667	0.25276	261.690229213811\\
59.9166666666667	0.25442	267.660179765598\\
59.9166666666667	0.25608	273.630130317384\\
59.9166666666667	0.25774	279.60008086917\\
59.9166666666667	0.2594	285.570031420956\\
59.9166666666667	0.26106	291.539981972742\\
59.9166666666667	0.26272	297.509932524528\\
59.9166666666667	0.26438	303.479883076314\\
59.9166666666667	0.26604	309.449833628101\\
59.9166666666667	0.2677	315.419784179887\\
59.9166666666667	0.26936	321.389734731673\\
59.9166666666667	0.27102	327.359685283459\\
59.9166666666667	0.27268	333.329635835245\\
59.9166666666667	0.27434	339.299586387031\\
59.9166666666667	0.276	345.269536938818\\
60.125	0.193	45.2410806985804\\
60.125	0.19466	51.165388844704\\
60.125	0.19632	57.0896969908281\\
60.125	0.19798	63.0140051369515\\
60.125	0.19964	68.9383132830756\\
60.125	0.2013	74.8626214291992\\
60.125	0.20296	80.7869295753228\\
60.125	0.20462	86.7112377214464\\
60.125	0.20628	92.6355458675703\\
60.125	0.20794	98.5598540136939\\
60.125	0.2096	104.484162159818\\
60.125	0.21126	110.408470305941\\
60.125	0.21292	116.332778452065\\
60.125	0.21458	122.257086598188\\
60.125	0.21624	128.181394744312\\
60.125	0.2179	134.105702890436\\
60.125	0.21956	140.030011036559\\
60.125	0.22122	145.954319182684\\
60.125	0.22288	151.878627328807\\
60.125	0.22454	157.802935474931\\
60.125	0.2262	163.727243621055\\
60.125	0.22786	169.651551767178\\
60.125	0.22952	175.575859913302\\
60.125	0.23118	181.500168059426\\
60.125	0.23284	187.42447620555\\
60.125	0.2345	193.348784351673\\
60.125	0.23616	199.273092497797\\
60.125	0.23782	205.197400643921\\
60.125	0.23948	211.121708790044\\
60.125	0.24114	217.046016936168\\
60.125	0.2428	222.970325082292\\
60.125	0.24446	228.894633228415\\
60.125	0.24612	234.818941374539\\
60.125	0.24778	240.743249520663\\
60.125	0.24944	246.667557666787\\
60.125	0.2511	252.59186581291\\
60.125	0.25276	258.516173959034\\
60.125	0.25442	264.440482105158\\
60.125	0.25608	270.364790251281\\
60.125	0.25774	276.289098397405\\
60.125	0.2594	282.213406543529\\
60.125	0.26106	288.137714689653\\
60.125	0.26272	294.062022835776\\
60.125	0.26438	299.9863309819\\
60.125	0.26604	305.910639128024\\
60.125	0.2677	311.834947274147\\
60.125	0.26936	317.759255420271\\
60.125	0.27102	323.683563566395\\
60.125	0.27268	329.607871712518\\
60.125	0.27434	335.532179858642\\
60.125	0.276	341.456488004766\\
60.3333333333333	0.193	43.7101520476506\\
60.3333333333333	0.19466	49.5888177881118\\
60.3333333333333	0.19632	55.4674835285732\\
60.3333333333333	0.19798	61.3461492690342\\
60.3333333333333	0.19964	67.2248150094958\\
60.3333333333333	0.2013	73.103480749957\\
60.3333333333333	0.20296	78.9821464904182\\
60.3333333333333	0.20462	84.8608122308794\\
60.3333333333333	0.20628	90.7394779713409\\
60.3333333333333	0.20794	96.6181437118021\\
60.3333333333333	0.2096	102.496809452263\\
60.3333333333333	0.21126	108.375475192724\\
60.3333333333333	0.21292	114.254140933186\\
60.3333333333333	0.21458	120.132806673647\\
60.3333333333333	0.21624	126.011472414108\\
60.3333333333333	0.2179	131.89013815457\\
60.3333333333333	0.21956	137.768803895031\\
60.3333333333333	0.22122	143.647469635492\\
60.3333333333333	0.22288	149.526135375953\\
60.3333333333333	0.22454	155.404801116415\\
60.3333333333333	0.2262	161.283466856876\\
60.3333333333333	0.22786	167.162132597337\\
60.3333333333333	0.22952	173.040798337798\\
60.3333333333333	0.23118	178.91946407826\\
60.3333333333333	0.23284	184.798129818721\\
60.3333333333333	0.2345	190.676795559182\\
60.3333333333333	0.23616	196.555461299643\\
60.3333333333333	0.23782	202.434127040105\\
60.3333333333333	0.23948	208.312792780566\\
60.3333333333333	0.24114	214.191458521027\\
60.3333333333333	0.2428	220.070124261489\\
60.3333333333333	0.24446	225.94879000195\\
60.3333333333333	0.24612	231.827455742411\\
60.3333333333333	0.24778	237.706121482872\\
60.3333333333333	0.24944	243.584787223334\\
60.3333333333333	0.2511	249.463452963795\\
60.3333333333333	0.25276	255.342118704256\\
60.3333333333333	0.25442	261.220784444718\\
60.3333333333333	0.25608	267.099450185179\\
60.3333333333333	0.25774	272.97811592564\\
60.3333333333333	0.2594	278.856781666101\\
60.3333333333333	0.26106	284.735447406562\\
60.3333333333333	0.26272	290.614113147024\\
60.3333333333333	0.26438	296.492778887485\\
60.3333333333333	0.26604	302.371444627946\\
60.3333333333333	0.2677	308.250110368408\\
60.3333333333333	0.26936	314.128776108869\\
60.3333333333333	0.27102	320.00744184933\\
60.3333333333333	0.27268	325.886107589791\\
60.3333333333333	0.27434	331.764773330252\\
60.3333333333333	0.276	337.643439070715\\
60.5416666666667	0.193	42.179223396721\\
60.5416666666667	0.19466	48.0122467315198\\
60.5416666666667	0.19632	53.845270066319\\
60.5416666666667	0.19798	59.6782934011176\\
60.5416666666667	0.19964	65.5113167359166\\
60.5416666666667	0.2013	71.3443400707154\\
60.5416666666667	0.20296	77.1773634055141\\
60.5416666666667	0.20462	83.0103867403129\\
60.5416666666667	0.20628	88.8434100751119\\
60.5416666666667	0.20794	94.6764334099107\\
60.5416666666667	0.2096	100.509456744709\\
60.5416666666667	0.21126	106.342480079508\\
60.5416666666667	0.21292	112.175503414307\\
60.5416666666667	0.21458	118.008526749106\\
60.5416666666667	0.21624	123.841550083905\\
60.5416666666667	0.2179	129.674573418703\\
60.5416666666667	0.21956	135.507596753502\\
60.5416666666667	0.22122	141.340620088301\\
60.5416666666667	0.22288	147.1736434231\\
60.5416666666667	0.22454	153.006666757899\\
60.5416666666667	0.2262	158.839690092698\\
60.5416666666667	0.22786	164.672713427497\\
60.5416666666667	0.22952	170.505736762295\\
60.5416666666667	0.23118	176.338760097094\\
60.5416666666667	0.23284	182.171783431893\\
60.5416666666667	0.2345	188.004806766692\\
60.5416666666667	0.23616	193.837830101491\\
60.5416666666667	0.23782	199.67085343629\\
60.5416666666667	0.23948	205.503876771088\\
60.5416666666667	0.24114	211.336900105887\\
60.5416666666667	0.2428	217.169923440686\\
60.5416666666667	0.24446	223.002946775485\\
60.5416666666667	0.24612	228.835970110284\\
60.5416666666667	0.24778	234.668993445082\\
60.5416666666667	0.24944	240.502016779882\\
60.5416666666667	0.2511	246.33504011468\\
60.5416666666667	0.25276	252.168063449479\\
60.5416666666667	0.25442	258.001086784278\\
60.5416666666667	0.25608	263.834110119077\\
60.5416666666667	0.25774	269.667133453876\\
60.5416666666667	0.2594	275.500156788674\\
60.5416666666667	0.26106	281.333180123473\\
60.5416666666667	0.26272	287.166203458272\\
60.5416666666667	0.26438	292.999226793071\\
60.5416666666667	0.26604	298.83225012787\\
60.5416666666667	0.2677	304.665273462669\\
60.5416666666667	0.26936	310.498296797467\\
60.5416666666667	0.27102	316.331320132266\\
60.5416666666667	0.27268	322.164343467065\\
60.5416666666667	0.27434	327.997366801863\\
60.5416666666667	0.276	333.830390136663\\
60.75	0.193	40.6482947457912\\
60.75	0.19466	46.4356756749276\\
60.75	0.19632	52.2230566040644\\
60.75	0.19798	58.0104375332005\\
60.75	0.19964	63.7978184623373\\
60.75	0.2013	69.5851993914735\\
60.75	0.20296	75.3725803206098\\
60.75	0.20462	81.1599612497462\\
60.75	0.20628	86.9473421788828\\
60.75	0.20794	92.7347231080191\\
60.75	0.2096	98.5221040371553\\
60.75	0.21126	104.309484966292\\
60.75	0.21292	110.096865895428\\
60.75	0.21458	115.884246824564\\
60.75	0.21624	121.671627753701\\
60.75	0.2179	127.459008682837\\
60.75	0.21956	133.246389611973\\
60.75	0.22122	139.03377054111\\
60.75	0.22288	144.821151470246\\
60.75	0.22454	150.608532399383\\
60.75	0.2262	156.395913328519\\
60.75	0.22786	162.183294257656\\
60.75	0.22952	167.970675186792\\
60.75	0.23118	173.758056115929\\
60.75	0.23284	179.545437045065\\
60.75	0.2345	185.332817974201\\
60.75	0.23616	191.120198903338\\
60.75	0.23782	196.907579832474\\
60.75	0.23948	202.69496076161\\
60.75	0.24114	208.482341690747\\
60.75	0.2428	214.269722619883\\
60.75	0.24446	220.057103549019\\
60.75	0.24612	225.844484478156\\
60.75	0.24778	231.631865407292\\
60.75	0.24944	237.419246336429\\
60.75	0.2511	243.206627265565\\
60.75	0.25276	248.994008194702\\
60.75	0.25442	254.781389123838\\
60.75	0.25608	260.568770052975\\
60.75	0.25774	266.356150982111\\
60.75	0.2594	272.143531911247\\
60.75	0.26106	277.930912840384\\
60.75	0.26272	283.71829376952\\
60.75	0.26438	289.505674698656\\
60.75	0.26604	295.293055627793\\
60.75	0.2677	301.080436556929\\
60.75	0.26936	306.867817486066\\
60.75	0.27102	312.655198415202\\
60.75	0.27268	318.442579344338\\
60.75	0.27434	324.229960273475\\
60.75	0.276	330.017341202611\\
60.9583333333333	0.193	39.1173660948616\\
60.9583333333333	0.19466	44.8591046183356\\
60.9583333333333	0.19632	50.6008431418097\\
60.9583333333333	0.19798	56.3425816652834\\
60.9583333333333	0.19964	62.0843201887578\\
60.9583333333333	0.2013	67.8260587122318\\
60.9583333333333	0.20296	73.5677972357055\\
60.9583333333333	0.20462	79.3095357591794\\
60.9583333333333	0.20628	85.0512742826536\\
60.9583333333333	0.20794	90.7930128061275\\
60.9583333333333	0.2096	96.5347513296012\\
60.9583333333333	0.21126	102.276489853075\\
60.9583333333333	0.21292	108.018228376549\\
60.9583333333333	0.21458	113.759966900023\\
60.9583333333333	0.21624	119.501705423497\\
60.9583333333333	0.2179	125.243443946971\\
60.9583333333333	0.21956	130.985182470445\\
60.9583333333333	0.22122	136.726920993919\\
60.9583333333333	0.22288	142.468659517393\\
60.9583333333333	0.22454	148.210398040867\\
60.9583333333333	0.2262	153.952136564341\\
60.9583333333333	0.22786	159.693875087815\\
60.9583333333333	0.22952	165.435613611289\\
60.9583333333333	0.23118	171.177352134763\\
60.9583333333333	0.23284	176.919090658237\\
60.9583333333333	0.2345	182.660829181711\\
60.9583333333333	0.23616	188.402567705185\\
60.9583333333333	0.23782	194.144306228658\\
60.9583333333333	0.23948	199.886044752132\\
60.9583333333333	0.24114	205.627783275606\\
60.9583333333333	0.2428	211.36952179908\\
60.9583333333333	0.24446	217.111260322554\\
60.9583333333333	0.24612	222.852998846028\\
60.9583333333333	0.24778	228.594737369502\\
60.9583333333333	0.24944	234.336475892976\\
60.9583333333333	0.2511	240.07821441645\\
60.9583333333333	0.25276	245.819952939924\\
60.9583333333333	0.25442	251.561691463398\\
60.9583333333333	0.25608	257.303429986872\\
60.9583333333333	0.25774	263.045168510346\\
60.9583333333333	0.2594	268.78690703382\\
60.9583333333333	0.26106	274.528645557294\\
60.9583333333333	0.26272	280.270384080768\\
60.9583333333333	0.26438	286.012122604242\\
60.9583333333333	0.26604	291.753861127716\\
60.9583333333333	0.2677	297.49559965119\\
60.9583333333333	0.26936	303.237338174663\\
60.9583333333333	0.27102	308.979076698137\\
60.9583333333333	0.27268	314.720815221611\\
60.9583333333333	0.27434	320.462553745085\\
60.9583333333333	0.276	326.20429226856\\
61.1666666666667	0.193	37.586437443932\\
61.1666666666667	0.19466	43.2825335617435\\
61.1666666666667	0.19632	48.9786296795555\\
61.1666666666667	0.19798	54.6747257973666\\
61.1666666666667	0.19964	60.3708219151786\\
61.1666666666667	0.2013	66.0669180329901\\
61.1666666666667	0.20296	71.7630141508014\\
61.1666666666667	0.20462	77.4591102686129\\
61.1666666666667	0.20628	83.1552063864247\\
61.1666666666667	0.20794	88.8513025042359\\
61.1666666666667	0.2096	94.5473986220475\\
61.1666666666667	0.21126	100.243494739859\\
61.1666666666667	0.21292	105.93959085767\\
61.1666666666667	0.21458	111.635686975482\\
61.1666666666667	0.21624	117.331783093293\\
61.1666666666667	0.2179	123.027879211105\\
61.1666666666667	0.21956	128.723975328916\\
61.1666666666667	0.22122	134.420071446728\\
61.1666666666667	0.22288	140.116167564539\\
61.1666666666667	0.22454	145.812263682351\\
61.1666666666667	0.2262	151.508359800163\\
61.1666666666667	0.22786	157.204455917974\\
61.1666666666667	0.22952	162.900552035786\\
61.1666666666667	0.23118	168.596648153597\\
61.1666666666667	0.23284	174.292744271409\\
61.1666666666667	0.2345	179.98884038922\\
61.1666666666667	0.23616	185.684936507031\\
61.1666666666667	0.23782	191.381032624844\\
61.1666666666667	0.23948	197.077128742655\\
61.1666666666667	0.24114	202.773224860466\\
61.1666666666667	0.2428	208.469320978278\\
61.1666666666667	0.24446	214.165417096089\\
61.1666666666667	0.24612	219.861513213901\\
61.1666666666667	0.24778	225.557609331712\\
61.1666666666667	0.24944	231.253705449524\\
61.1666666666667	0.2511	236.949801567335\\
61.1666666666667	0.25276	242.645897685146\\
61.1666666666667	0.25442	248.341993802958\\
61.1666666666667	0.25608	254.03808992077\\
61.1666666666667	0.25774	259.734186038581\\
61.1666666666667	0.2594	265.430282156393\\
61.1666666666667	0.26106	271.126378274204\\
61.1666666666667	0.26272	276.822474392015\\
61.1666666666667	0.26438	282.518570509827\\
61.1666666666667	0.26604	288.214666627639\\
61.1666666666667	0.2677	293.91076274545\\
61.1666666666667	0.26936	299.606858863262\\
61.1666666666667	0.27102	305.302954981073\\
61.1666666666667	0.27268	310.999051098885\\
61.1666666666667	0.27434	316.695147216696\\
61.1666666666667	0.276	322.391243334509\\
61.375	0.193	36.0555087930022\\
61.375	0.19466	41.7059625051513\\
61.375	0.19632	47.3564162173006\\
61.375	0.19798	53.0068699294495\\
61.375	0.19964	58.6573236415989\\
61.375	0.2013	64.307777353748\\
61.375	0.20296	69.9582310658971\\
61.375	0.20462	75.6086847780459\\
61.375	0.20628	81.2591384901953\\
61.375	0.20794	86.9095922023441\\
61.375	0.2096	92.5600459144932\\
61.375	0.21126	98.2104996266421\\
61.375	0.21292	103.860953338791\\
61.375	0.21458	109.51140705094\\
61.375	0.21624	115.161860763089\\
61.375	0.2179	120.812314475239\\
61.375	0.21956	126.462768187387\\
61.375	0.22122	132.113221899537\\
61.375	0.22288	137.763675611686\\
61.375	0.22454	143.414129323835\\
61.375	0.2262	149.064583035984\\
61.375	0.22786	154.715036748133\\
61.375	0.22952	160.365490460282\\
61.375	0.23118	166.015944172431\\
61.375	0.23284	171.66639788458\\
61.375	0.2345	177.316851596729\\
61.375	0.23616	182.967305308878\\
61.375	0.23782	188.617759021027\\
61.375	0.23948	194.268212733177\\
61.375	0.24114	199.918666445325\\
61.375	0.2428	205.569120157475\\
61.375	0.24446	211.219573869623\\
61.375	0.24612	216.870027581773\\
61.375	0.24778	222.520481293922\\
61.375	0.24944	228.170935006071\\
61.375	0.2511	233.82138871822\\
61.375	0.25276	239.471842430369\\
61.375	0.25442	245.122296142518\\
61.375	0.25608	250.772749854667\\
61.375	0.25774	256.423203566816\\
61.375	0.2594	262.073657278966\\
61.375	0.26106	267.724110991114\\
61.375	0.26272	273.374564703263\\
61.375	0.26438	279.025018415412\\
61.375	0.26604	284.675472127562\\
61.375	0.2677	290.325925839711\\
61.375	0.26936	295.97637955186\\
61.375	0.27102	301.626833264009\\
61.375	0.27268	307.277286976158\\
61.375	0.27434	312.927740688307\\
61.375	0.276	318.578194400456\\
61.5833333333333	0.193	34.5245801420726\\
61.5833333333333	0.19466	40.1293914485591\\
61.5833333333333	0.19632	45.7342027550462\\
61.5833333333333	0.19798	51.3390140615325\\
61.5833333333333	0.19964	56.9438253680196\\
61.5833333333333	0.2013	62.548636674506\\
61.5833333333333	0.20296	68.1534479809927\\
61.5833333333333	0.20462	73.7582592874792\\
61.5833333333333	0.20628	79.3630705939661\\
61.5833333333333	0.20794	84.9678819004525\\
61.5833333333333	0.2096	90.5726932069392\\
61.5833333333333	0.21126	96.1775045134257\\
61.5833333333333	0.21292	101.782315819912\\
61.5833333333333	0.21458	107.387127126399\\
61.5833333333333	0.21624	112.991938432885\\
61.5833333333333	0.2179	118.596749739372\\
61.5833333333333	0.21956	124.201561045859\\
61.5833333333333	0.22122	129.806372352346\\
61.5833333333333	0.22288	135.411183658832\\
61.5833333333333	0.22454	141.015994965319\\
61.5833333333333	0.2262	146.620806271806\\
61.5833333333333	0.22786	152.225617578292\\
61.5833333333333	0.22952	157.830428884779\\
61.5833333333333	0.23118	163.435240191266\\
61.5833333333333	0.23284	169.040051497752\\
61.5833333333333	0.2345	174.644862804239\\
61.5833333333333	0.23616	180.249674110725\\
61.5833333333333	0.23782	185.854485417212\\
61.5833333333333	0.23948	191.459296723698\\
61.5833333333333	0.24114	197.064108030185\\
61.5833333333333	0.2428	202.668919336671\\
61.5833333333333	0.24446	208.273730643158\\
61.5833333333333	0.24612	213.878541949646\\
61.5833333333333	0.24778	219.483353256132\\
61.5833333333333	0.24944	225.088164562618\\
61.5833333333333	0.2511	230.692975869105\\
61.5833333333333	0.25276	236.297787175592\\
61.5833333333333	0.25442	241.902598482079\\
61.5833333333333	0.25608	247.507409788565\\
61.5833333333333	0.25774	253.112221095052\\
61.5833333333333	0.2594	258.717032401538\\
61.5833333333333	0.26106	264.321843708025\\
61.5833333333333	0.26272	269.926655014512\\
61.5833333333333	0.26438	275.531466320998\\
61.5833333333333	0.26604	281.136277627485\\
61.5833333333333	0.2677	286.741088933972\\
61.5833333333333	0.26936	292.345900240458\\
61.5833333333333	0.27102	297.950711546945\\
61.5833333333333	0.27268	303.555522853431\\
61.5833333333333	0.27434	309.160334159918\\
61.5833333333333	0.276	314.765145466405\\
61.7916666666667	0.193	32.9936514911428\\
61.7916666666667	0.19466	38.5528203919671\\
61.7916666666667	0.19632	44.1119892927916\\
61.7916666666667	0.19798	49.6711581936154\\
61.7916666666667	0.19964	55.2303270944401\\
61.7916666666667	0.2013	60.7894959952641\\
61.7916666666667	0.20296	66.3486648960884\\
61.7916666666667	0.20462	71.9078337969124\\
61.7916666666667	0.20628	77.4670026977367\\
61.7916666666667	0.20794	83.0261715985609\\
61.7916666666667	0.2096	88.585340499385\\
61.7916666666667	0.21126	94.1445094002092\\
61.7916666666667	0.21292	99.7036783010333\\
61.7916666666667	0.21458	105.262847201857\\
61.7916666666667	0.21624	110.822016102682\\
61.7916666666667	0.2179	116.381185003506\\
61.7916666666667	0.21956	121.94035390433\\
61.7916666666667	0.22122	127.499522805155\\
61.7916666666667	0.22288	133.058691705978\\
61.7916666666667	0.22454	138.617860606803\\
61.7916666666667	0.2262	144.177029507627\\
61.7916666666667	0.22786	149.736198408451\\
61.7916666666667	0.22952	155.295367309275\\
61.7916666666667	0.23118	160.8545362101\\
61.7916666666667	0.23284	166.413705110924\\
61.7916666666667	0.2345	171.972874011748\\
61.7916666666667	0.23616	177.532042912572\\
61.7916666666667	0.23782	183.091211813396\\
61.7916666666667	0.23948	188.65038071422\\
61.7916666666667	0.24114	194.209549615045\\
61.7916666666667	0.2428	199.768718515869\\
61.7916666666667	0.24446	205.327887416693\\
61.7916666666667	0.24612	210.887056317518\\
61.7916666666667	0.24778	216.446225218342\\
61.7916666666667	0.24944	222.005394119166\\
61.7916666666667	0.2511	227.56456301999\\
61.7916666666667	0.25276	233.123731920814\\
61.7916666666667	0.25442	238.682900821639\\
61.7916666666667	0.25608	244.242069722463\\
61.7916666666667	0.25774	249.801238623287\\
61.7916666666667	0.2594	255.360407524111\\
61.7916666666667	0.26106	260.919576424935\\
61.7916666666667	0.26272	266.47874532576\\
61.7916666666667	0.26438	272.037914226584\\
61.7916666666667	0.26604	277.597083127408\\
61.7916666666667	0.2677	283.156252028232\\
61.7916666666667	0.26936	288.715420929057\\
61.7916666666667	0.27102	294.27458982988\\
61.7916666666667	0.27268	299.833758730705\\
61.7916666666667	0.27434	305.392927631529\\
61.7916666666667	0.276	310.952096532353\\
62	0.193	31.4627228402135\\
62	0.19466	36.9762493353751\\
62	0.19632	42.4897758305372\\
62	0.19798	48.0033023256988\\
62	0.19964	53.5168288208608\\
62	0.2013	59.0303553160225\\
62	0.20296	64.5438818111843\\
62	0.20462	70.0574083063459\\
62	0.20628	75.5709348015077\\
62	0.20794	81.0844612966696\\
62	0.2096	86.5979877918312\\
62	0.21126	92.1115142869928\\
62	0.21292	97.6250407821547\\
62	0.21458	103.138567277316\\
62	0.21624	108.652093772478\\
62	0.2179	114.16562026764\\
62	0.21956	119.679146762802\\
62	0.22122	125.192673257964\\
62	0.22288	130.706199753125\\
62	0.22454	136.219726248287\\
62	0.2262	141.733252743449\\
62	0.22786	147.246779238611\\
62	0.22952	152.760305733772\\
62	0.23118	158.273832228934\\
62	0.23284	163.787358724096\\
62	0.2345	169.300885219257\\
62	0.23616	174.814411714419\\
62	0.23782	180.327938209581\\
62	0.23948	185.841464704743\\
62	0.24114	191.354991199905\\
62	0.2428	196.868517695066\\
62	0.24446	202.382044190228\\
62	0.24612	207.89557068539\\
62	0.24778	213.409097180551\\
62	0.24944	218.922623675714\\
62	0.2511	224.436150170875\\
62	0.25276	229.949676666037\\
62	0.25442	235.463203161199\\
62	0.25608	240.97672965636\\
62	0.25774	246.490256151522\\
62	0.2594	252.003782646684\\
62	0.26106	257.517309141846\\
62	0.26272	263.030835637007\\
62	0.26438	268.544362132169\\
62	0.26604	274.057888627331\\
62	0.2677	279.571415122493\\
62	0.26936	285.084941617654\\
62	0.27102	290.598468112816\\
62	0.27268	296.111994607978\\
62	0.27434	301.625521103139\\
62	0.276	307.139047598302\\
62.2083333333333	0.193	29.9317941892837\\
62.2083333333333	0.19466	35.3996782787831\\
62.2083333333333	0.19632	40.8675623682827\\
62.2083333333333	0.19798	46.3354464577817\\
62.2083333333333	0.19964	51.8033305472813\\
62.2083333333333	0.2013	57.2712146367808\\
62.2083333333333	0.20296	62.73909872628\\
62.2083333333333	0.20462	68.2069828157792\\
62.2083333333333	0.20628	73.6748669052786\\
62.2083333333333	0.20794	79.1427509947778\\
62.2083333333333	0.2096	84.6106350842772\\
62.2083333333333	0.21126	90.0785191737764\\
62.2083333333333	0.21292	95.5464032632756\\
62.2083333333333	0.21458	101.014287352775\\
62.2083333333333	0.21624	106.482171442274\\
62.2083333333333	0.2179	111.950055531774\\
62.2083333333333	0.21956	117.417939621273\\
62.2083333333333	0.22122	122.885823710772\\
62.2083333333333	0.22288	128.353707800272\\
62.2083333333333	0.22454	133.821591889771\\
62.2083333333333	0.2262	139.28947597927\\
62.2083333333333	0.22786	144.75736006877\\
62.2083333333333	0.22952	150.225244158269\\
62.2083333333333	0.23118	155.693128247769\\
62.2083333333333	0.23284	161.161012337268\\
62.2083333333333	0.2345	166.628896426767\\
62.2083333333333	0.23616	172.096780516266\\
62.2083333333333	0.23782	177.564664605766\\
62.2083333333333	0.23948	183.032548695265\\
62.2083333333333	0.24114	188.500432784764\\
62.2083333333333	0.2428	193.968316874263\\
62.2083333333333	0.24446	199.436200963763\\
62.2083333333333	0.24612	204.904085053262\\
62.2083333333333	0.24778	210.371969142761\\
62.2083333333333	0.24944	215.839853232261\\
62.2083333333333	0.2511	221.30773732176\\
62.2083333333333	0.25276	226.775621411259\\
62.2083333333333	0.25442	232.243505500759\\
62.2083333333333	0.25608	237.711389590258\\
62.2083333333333	0.25774	243.179273679757\\
62.2083333333333	0.2594	248.647157769257\\
62.2083333333333	0.26106	254.115041858756\\
62.2083333333333	0.26272	259.582925948255\\
62.2083333333333	0.26438	265.050810037755\\
62.2083333333333	0.26604	270.518694127254\\
62.2083333333333	0.2677	275.986578216754\\
62.2083333333333	0.26936	281.454462306253\\
62.2083333333333	0.27102	286.922346395752\\
62.2083333333333	0.27268	292.390230485251\\
62.2083333333333	0.27434	297.85811457475\\
62.2083333333333	0.276	303.32599866425\\
62.4166666666667	0.193	28.4008655383541\\
62.4166666666667	0.19466	33.8231072221909\\
62.4166666666667	0.19632	39.2453489060281\\
62.4166666666667	0.19798	44.6675905898646\\
62.4166666666667	0.19964	50.0898322737019\\
62.4166666666667	0.2013	55.5120739575386\\
62.4166666666667	0.20296	60.9343156413756\\
62.4166666666667	0.20462	66.3565573252124\\
62.4166666666667	0.20628	71.7787990090494\\
62.4166666666667	0.20794	77.2010406928862\\
62.4166666666667	0.2096	82.623282376723\\
62.4166666666667	0.21126	88.0455240605597\\
62.4166666666667	0.21292	93.4677657443967\\
62.4166666666667	0.21458	98.8900074282335\\
62.4166666666667	0.21624	104.312249112071\\
62.4166666666667	0.2179	109.734490795907\\
62.4166666666667	0.21956	115.156732479744\\
62.4166666666667	0.22122	120.578974163581\\
62.4166666666667	0.22288	126.001215847418\\
62.4166666666667	0.22454	131.423457531255\\
62.4166666666667	0.2262	136.845699215092\\
62.4166666666667	0.22786	142.267940898929\\
62.4166666666667	0.22952	147.690182582766\\
62.4166666666667	0.23118	153.112424266603\\
62.4166666666667	0.23284	158.53466595044\\
62.4166666666667	0.2345	163.956907634276\\
62.4166666666667	0.23616	169.379149318113\\
62.4166666666667	0.23782	174.80139100195\\
62.4166666666667	0.23948	180.223632685787\\
62.4166666666667	0.24114	185.645874369624\\
62.4166666666667	0.2428	191.068116053461\\
62.4166666666667	0.24446	196.490357737297\\
62.4166666666667	0.24612	201.912599421135\\
62.4166666666667	0.24778	207.334841104971\\
62.4166666666667	0.24944	212.757082788808\\
62.4166666666667	0.2511	218.179324472645\\
62.4166666666667	0.25276	223.601566156482\\
62.4166666666667	0.25442	229.023807840319\\
62.4166666666667	0.25608	234.446049524156\\
62.4166666666667	0.25774	239.868291207993\\
62.4166666666667	0.2594	245.29053289183\\
62.4166666666667	0.26106	250.712774575667\\
62.4166666666667	0.26272	256.135016259503\\
62.4166666666667	0.26438	261.55725794334\\
62.4166666666667	0.26604	266.979499627177\\
62.4166666666667	0.2677	272.401741311014\\
62.4166666666667	0.26936	277.823982994851\\
62.4166666666667	0.27102	283.246224678688\\
62.4166666666667	0.27268	288.668466362525\\
62.4166666666667	0.27434	294.090708046361\\
62.4166666666667	0.276	299.512949730199\\
62.625	0.193	26.869936887424\\
62.625	0.19466	32.2465361655986\\
62.625	0.19632	37.6231354437734\\
62.625	0.19798	42.9997347219476\\
62.625	0.19964	48.3763340001224\\
62.625	0.2013	53.7529332782967\\
62.625	0.20296	59.1295325564711\\
62.625	0.20462	64.5061318346454\\
62.625	0.20628	69.88273111282\\
62.625	0.20794	75.2593303909944\\
62.625	0.2096	80.6359296691687\\
62.625	0.21126	86.0125289473431\\
62.625	0.21292	91.3891282255174\\
62.625	0.21458	96.7657275036918\\
62.625	0.21624	102.142326781867\\
62.625	0.2179	107.518926060041\\
62.625	0.21956	112.895525338215\\
62.625	0.22122	118.27212461639\\
62.625	0.22288	123.648723894564\\
62.625	0.22454	129.025323172739\\
62.625	0.2262	134.401922450913\\
62.625	0.22786	139.778521729088\\
62.625	0.22952	145.155121007262\\
62.625	0.23118	150.531720285437\\
62.625	0.23284	155.908319563611\\
62.625	0.2345	161.284918841786\\
62.625	0.23616	166.66151811996\\
62.625	0.23782	172.038117398134\\
62.625	0.23948	177.414716676309\\
62.625	0.24114	182.791315954483\\
62.625	0.2428	188.167915232657\\
62.625	0.24446	193.544514510832\\
62.625	0.24612	198.921113789007\\
62.625	0.24778	204.297713067181\\
62.625	0.24944	209.674312345356\\
62.625	0.2511	215.05091162353\\
62.625	0.25276	220.427510901704\\
62.625	0.25442	225.804110179879\\
62.625	0.25608	231.180709458054\\
62.625	0.25774	236.557308736228\\
62.625	0.2594	241.933908014402\\
62.625	0.26106	247.310507292577\\
62.625	0.26272	252.687106570751\\
62.625	0.26438	258.063705848926\\
62.625	0.26604	263.4403051271\\
62.625	0.2677	268.816904405275\\
62.625	0.26936	274.193503683449\\
62.625	0.27102	279.570102961623\\
62.625	0.27268	284.946702239798\\
62.625	0.27434	290.323301517972\\
62.625	0.276	295.699900796147\\
62.8333333333333	0.193	25.3390082364942\\
62.8333333333333	0.19466	30.6699651090062\\
62.8333333333333	0.19632	36.0009219815186\\
62.8333333333333	0.19798	41.3318788540303\\
62.8333333333333	0.19964	46.6628357265427\\
62.8333333333333	0.2013	51.9937925990546\\
62.8333333333333	0.20296	57.3247494715665\\
62.8333333333333	0.20462	62.6557063440785\\
62.8333333333333	0.20628	67.9866632165906\\
62.8333333333333	0.20794	73.3176200891025\\
62.8333333333333	0.2096	78.6485769616145\\
62.8333333333333	0.21126	83.9795338341264\\
62.8333333333333	0.21292	89.3104907066383\\
62.8333333333333	0.21458	94.6414475791503\\
62.8333333333333	0.21624	99.9724044516624\\
62.8333333333333	0.2179	105.303361324174\\
62.8333333333333	0.21956	110.634318196686\\
62.8333333333333	0.22122	115.965275069199\\
62.8333333333333	0.22288	121.29623194171\\
62.8333333333333	0.22454	126.627188814223\\
62.8333333333333	0.2262	131.958145686735\\
62.8333333333333	0.22786	137.289102559247\\
62.8333333333333	0.22952	142.620059431759\\
62.8333333333333	0.23118	147.951016304271\\
62.8333333333333	0.23284	153.281973176783\\
62.8333333333333	0.2345	158.612930049295\\
62.8333333333333	0.23616	163.943886921807\\
62.8333333333333	0.23782	169.274843794319\\
62.8333333333333	0.23948	174.605800666831\\
62.8333333333333	0.24114	179.936757539343\\
62.8333333333333	0.2428	185.267714411855\\
62.8333333333333	0.24446	190.598671284366\\
62.8333333333333	0.24612	195.929628156879\\
62.8333333333333	0.24778	201.26058502939\\
62.8333333333333	0.24944	206.591541901903\\
62.8333333333333	0.2511	211.922498774415\\
62.8333333333333	0.25276	217.253455646926\\
62.8333333333333	0.25442	222.584412519439\\
62.8333333333333	0.25608	227.915369391951\\
62.8333333333333	0.25774	233.246326264463\\
62.8333333333333	0.2594	238.577283136975\\
62.8333333333333	0.26106	243.908240009487\\
62.8333333333333	0.26272	249.239196881998\\
62.8333333333333	0.26438	254.570153754511\\
62.8333333333333	0.26604	259.901110627023\\
62.8333333333333	0.2677	265.232067499535\\
62.8333333333333	0.26936	270.563024372047\\
62.8333333333333	0.27102	275.893981244559\\
62.8333333333333	0.27268	281.22493811707\\
62.8333333333333	0.27434	286.555894989583\\
62.8333333333333	0.276	291.886851862095\\
63.0416666666667	0.193	23.8080795855649\\
63.0416666666667	0.19466	29.0933940524144\\
63.0416666666667	0.19632	34.3787085192641\\
63.0416666666667	0.19798	39.6640229861134\\
63.0416666666667	0.19964	44.9493374529634\\
63.0416666666667	0.2013	50.2346519198129\\
63.0416666666667	0.20296	55.5199663866624\\
63.0416666666667	0.20462	60.8052808535119\\
63.0416666666667	0.20628	66.0905953203617\\
63.0416666666667	0.20794	71.3759097872112\\
63.0416666666667	0.2096	76.6612242540607\\
63.0416666666667	0.21126	81.9465387209102\\
63.0416666666667	0.21292	87.2318531877597\\
63.0416666666667	0.21458	92.5171676546092\\
63.0416666666667	0.21624	97.8024821214588\\
63.0416666666667	0.2179	103.087796588308\\
63.0416666666667	0.21956	108.373111055158\\
63.0416666666667	0.22122	113.658425522008\\
63.0416666666667	0.22288	118.943739988857\\
63.0416666666667	0.22454	124.229054455707\\
63.0416666666667	0.2262	129.514368922557\\
63.0416666666667	0.22786	134.799683389406\\
63.0416666666667	0.22952	140.084997856256\\
63.0416666666667	0.23118	145.370312323105\\
63.0416666666667	0.23284	150.655626789955\\
63.0416666666667	0.2345	155.940941256804\\
63.0416666666667	0.23616	161.226255723653\\
63.0416666666667	0.23782	166.511570190504\\
63.0416666666667	0.23948	171.796884657353\\
63.0416666666667	0.24114	177.082199124203\\
63.0416666666667	0.2428	182.367513591052\\
63.0416666666667	0.24446	187.652828057902\\
63.0416666666667	0.24612	192.938142524751\\
63.0416666666667	0.24778	198.2234569916\\
63.0416666666667	0.24944	203.508771458451\\
63.0416666666667	0.2511	208.794085925299\\
63.0416666666667	0.25276	214.079400392149\\
63.0416666666667	0.25442	219.364714858999\\
63.0416666666667	0.25608	224.650029325849\\
63.0416666666667	0.25774	229.935343792698\\
63.0416666666667	0.2594	235.220658259548\\
63.0416666666667	0.26106	240.505972726397\\
63.0416666666667	0.26272	245.791287193247\\
63.0416666666667	0.26438	251.076601660096\\
63.0416666666667	0.26604	256.361916126946\\
63.0416666666667	0.2677	261.647230593795\\
63.0416666666667	0.26936	266.932545060645\\
63.0416666666667	0.27102	272.217859527495\\
63.0416666666667	0.27268	277.503173994344\\
63.0416666666667	0.27434	282.788488461194\\
63.0416666666667	0.276	288.073802928044\\
63.25	0.193	22.2771509346353\\
63.25	0.19466	27.5168229958224\\
63.25	0.19632	32.7564950570099\\
63.25	0.19798	37.9961671181968\\
63.25	0.19964	43.2358391793841\\
63.25	0.2013	48.4755112405712\\
63.25	0.20296	53.7151833017583\\
63.25	0.20462	58.9548553629454\\
63.25	0.20628	64.1945274241327\\
63.25	0.20794	69.4341994853198\\
63.25	0.2096	74.6738715465067\\
63.25	0.21126	79.9135436076938\\
63.25	0.21292	85.1532156688809\\
63.25	0.21458	90.392887730068\\
63.25	0.21624	95.6325597912553\\
63.25	0.2179	100.872231852442\\
63.25	0.21956	106.111903913629\\
63.25	0.22122	111.351575974817\\
63.25	0.22288	116.591248036004\\
63.25	0.22454	121.830920097191\\
63.25	0.2262	127.070592158378\\
63.25	0.22786	132.310264219565\\
63.25	0.22952	137.549936280752\\
63.25	0.23118	142.78960834194\\
63.25	0.23284	148.029280403127\\
63.25	0.2345	153.268952464314\\
63.25	0.23616	158.508624525501\\
63.25	0.23782	163.748296586688\\
63.25	0.23948	168.987968647875\\
63.25	0.24114	174.227640709062\\
63.25	0.2428	179.467312770249\\
63.25	0.24446	184.706984831436\\
63.25	0.24612	189.946656892624\\
63.25	0.24778	195.186328953811\\
63.25	0.24944	200.426001014998\\
63.25	0.2511	205.665673076185\\
63.25	0.25276	210.905345137372\\
63.25	0.25442	216.14501719856\\
63.25	0.25608	221.384689259747\\
63.25	0.25774	226.624361320934\\
63.25	0.2594	231.864033382121\\
63.25	0.26106	237.103705443308\\
63.25	0.26272	242.343377504495\\
63.25	0.26438	247.583049565682\\
63.25	0.26604	252.82272162687\\
63.25	0.2677	258.062393688057\\
63.25	0.26936	263.302065749243\\
63.25	0.27102	268.541737810431\\
63.25	0.27268	273.781409871618\\
63.25	0.27434	279.021081932805\\
63.25	0.276	284.260753993992\\
63.4583333333333	0.193	20.7462222837055\\
63.4583333333333	0.19466	25.9402519392302\\
63.4583333333333	0.19632	31.1342815947551\\
63.4583333333333	0.19798	36.3283112502795\\
63.4583333333333	0.19964	41.5223409058046\\
63.4583333333333	0.2013	46.7163705613293\\
63.4583333333333	0.20296	51.9104002168538\\
63.4583333333333	0.20462	57.1044298723784\\
63.4583333333333	0.20628	62.2984595279033\\
63.4583333333333	0.20794	67.492489183428\\
63.4583333333333	0.2096	72.6865188389525\\
63.4583333333333	0.21126	77.8805484944771\\
63.4583333333333	0.21292	83.0745781500018\\
63.4583333333333	0.21458	88.2686078055265\\
63.4583333333333	0.21624	93.4626374610511\\
63.4583333333333	0.2179	98.6566671165758\\
63.4583333333333	0.21956	103.8506967721\\
63.4583333333333	0.22122	109.044726427626\\
63.4583333333333	0.22288	114.23875608315\\
63.4583333333333	0.22454	119.432785738675\\
63.4583333333333	0.2262	124.6268153942\\
63.4583333333333	0.22786	129.820845049724\\
63.4583333333333	0.22952	135.014874705249\\
63.4583333333333	0.23118	140.208904360774\\
63.4583333333333	0.23284	145.402934016299\\
63.4583333333333	0.2345	150.596963671823\\
63.4583333333333	0.23616	155.790993327348\\
63.4583333333333	0.23782	160.985022982872\\
63.4583333333333	0.23948	166.179052638397\\
63.4583333333333	0.24114	171.373082293921\\
63.4583333333333	0.2428	176.567111949446\\
63.4583333333333	0.24446	181.761141604971\\
63.4583333333333	0.24612	186.955171260496\\
63.4583333333333	0.24778	192.149200916021\\
63.4583333333333	0.24944	197.343230571545\\
63.4583333333333	0.2511	202.53726022707\\
63.4583333333333	0.25276	207.731289882595\\
63.4583333333333	0.25442	212.92531953812\\
63.4583333333333	0.25608	218.119349193644\\
63.4583333333333	0.25774	223.313378849169\\
63.4583333333333	0.2594	228.507408504694\\
63.4583333333333	0.26106	233.701438160218\\
63.4583333333333	0.26272	238.895467815743\\
63.4583333333333	0.26438	244.089497471267\\
63.4583333333333	0.26604	249.283527126793\\
63.4583333333333	0.2677	254.477556782317\\
63.4583333333333	0.26936	259.671586437842\\
63.4583333333333	0.27102	264.865616093366\\
63.4583333333333	0.27268	270.059645748891\\
63.4583333333333	0.27434	275.253675404415\\
63.4583333333333	0.276	280.44770505994\\
63.6666666666667	0.193	19.2152936327759\\
63.6666666666667	0.19466	24.3636808826382\\
63.6666666666667	0.19632	29.5120681325009\\
63.6666666666667	0.19798	34.6604553823627\\
63.6666666666667	0.19964	39.8088426322254\\
63.6666666666667	0.2013	44.9572298820876\\
63.6666666666667	0.20296	50.1056171319497\\
63.6666666666667	0.20462	55.2540043818119\\
63.6666666666667	0.20628	60.4023916316744\\
63.6666666666667	0.20794	65.5507788815364\\
63.6666666666667	0.2096	70.6991661313987\\
63.6666666666667	0.21126	75.8475533812609\\
63.6666666666667	0.21292	80.9959406311229\\
63.6666666666667	0.21458	86.1443278809852\\
63.6666666666667	0.21624	91.2927151308477\\
63.6666666666667	0.2179	96.4411023807097\\
63.6666666666667	0.21956	101.589489630572\\
63.6666666666667	0.22122	106.737876880435\\
63.6666666666667	0.22288	111.886264130296\\
63.6666666666667	0.22454	117.034651380159\\
63.6666666666667	0.2262	122.183038630021\\
63.6666666666667	0.22786	127.331425879883\\
63.6666666666667	0.22952	132.479813129746\\
63.6666666666667	0.23118	137.628200379608\\
63.6666666666667	0.23284	142.77658762947\\
63.6666666666667	0.2345	147.924974879332\\
63.6666666666667	0.23616	153.073362129195\\
63.6666666666667	0.23782	158.221749379057\\
63.6666666666667	0.23948	163.370136628919\\
63.6666666666667	0.24114	168.518523878781\\
63.6666666666667	0.2428	173.666911128644\\
63.6666666666667	0.24446	178.815298378506\\
63.6666666666667	0.24612	183.963685628368\\
63.6666666666667	0.24778	189.11207287823\\
63.6666666666667	0.24944	194.260460128093\\
63.6666666666667	0.2511	199.408847377955\\
63.6666666666667	0.25276	204.557234627817\\
63.6666666666667	0.25442	209.70562187768\\
63.6666666666667	0.25608	214.854009127542\\
63.6666666666667	0.25774	220.002396377404\\
63.6666666666667	0.2594	225.150783627266\\
63.6666666666667	0.26106	230.299170877129\\
63.6666666666667	0.26272	235.447558126991\\
63.6666666666667	0.26438	240.595945376853\\
63.6666666666667	0.26604	245.744332626716\\
63.6666666666667	0.2677	250.892719876578\\
63.6666666666667	0.26936	256.04110712644\\
63.6666666666667	0.27102	261.189494376302\\
63.6666666666667	0.27268	266.337881626164\\
63.6666666666667	0.27434	271.486268876026\\
63.6666666666667	0.276	276.634656125889\\
63.875	0.193	17.6843649818461\\
63.875	0.19466	22.7871098260459\\
63.875	0.19632	27.889854670246\\
63.875	0.19798	32.9925995144456\\
63.875	0.19964	38.0953443586457\\
63.875	0.2013	43.1980892028455\\
63.875	0.20296	48.3008340470453\\
63.875	0.20462	53.4035788912449\\
63.875	0.20628	58.506323735445\\
63.875	0.20794	63.6090685796446\\
63.875	0.2096	68.7118134238444\\
63.875	0.21126	73.8145582680443\\
63.875	0.21292	78.9173031122439\\
63.875	0.21458	84.0200479564437\\
63.875	0.21624	89.1227928006435\\
63.875	0.2179	94.2255376448434\\
63.875	0.21956	99.328282489043\\
63.875	0.22122	104.431027333243\\
63.875	0.22288	109.533772177443\\
63.875	0.22454	114.636517021643\\
63.875	0.2262	119.739261865843\\
63.875	0.22786	124.842006710042\\
63.875	0.22952	129.944751554242\\
63.875	0.23118	135.047496398442\\
63.875	0.23284	140.150241242642\\
63.875	0.2345	145.252986086841\\
63.875	0.23616	150.355730931041\\
63.875	0.23782	155.458475775242\\
63.875	0.23948	160.561220619441\\
63.875	0.24114	165.663965463641\\
63.875	0.2428	170.766710307841\\
63.875	0.24446	175.869455152041\\
63.875	0.24612	180.97219999624\\
63.875	0.24778	186.07494484044\\
63.875	0.24944	191.17768968464\\
63.875	0.2511	196.28043452884\\
63.875	0.25276	201.383179373039\\
63.875	0.25442	206.485924217239\\
63.875	0.25608	211.588669061439\\
63.875	0.25774	216.691413905639\\
63.875	0.2594	221.794158749839\\
63.875	0.26106	226.896903594039\\
63.875	0.26272	231.999648438238\\
63.875	0.26438	237.102393282438\\
63.875	0.26604	242.205138126638\\
63.875	0.2677	247.307882970838\\
63.875	0.26936	252.410627815038\\
63.875	0.27102	257.513372659237\\
63.875	0.27268	262.616117503437\\
63.875	0.27434	267.718862347637\\
63.875	0.276	272.821607191837\\
64.0833333333333	0.193	16.1534363309163\\
64.0833333333333	0.19466	21.2105387694537\\
64.0833333333333	0.19632	26.2676412079913\\
64.0833333333333	0.19798	31.3247436465285\\
64.0833333333333	0.19964	36.3818460850662\\
64.0833333333333	0.2013	41.4389485236036\\
64.0833333333333	0.20296	46.4960509621408\\
64.0833333333333	0.20462	51.5531534006782\\
64.0833333333333	0.20628	56.6102558392156\\
64.0833333333333	0.20794	61.667358277753\\
64.0833333333333	0.2096	66.7244607162904\\
64.0833333333333	0.21126	71.7815631548276\\
64.0833333333333	0.21292	76.838665593365\\
64.0833333333333	0.21458	81.8957680319022\\
64.0833333333333	0.21624	86.9528704704398\\
64.0833333333333	0.2179	92.009972908977\\
64.0833333333333	0.21956	97.0670753475144\\
64.0833333333333	0.22122	102.124177786052\\
64.0833333333333	0.22288	107.181280224589\\
64.0833333333333	0.22454	112.238382663127\\
64.0833333333333	0.2262	117.295485101664\\
64.0833333333333	0.22786	122.352587540202\\
64.0833333333333	0.22952	127.409689978739\\
64.0833333333333	0.23118	132.466792417276\\
64.0833333333333	0.23284	137.523894855814\\
64.0833333333333	0.2345	142.580997294351\\
64.0833333333333	0.23616	147.638099732888\\
64.0833333333333	0.23782	152.695202171426\\
64.0833333333333	0.23948	157.752304609963\\
64.0833333333333	0.24114	162.809407048501\\
64.0833333333333	0.2428	167.866509487038\\
64.0833333333333	0.24446	172.923611925575\\
64.0833333333333	0.24612	177.980714364112\\
64.0833333333333	0.24778	183.03781680265\\
64.0833333333333	0.24944	188.094919241188\\
64.0833333333333	0.2511	193.152021679724\\
64.0833333333333	0.25276	198.209124118262\\
64.0833333333333	0.25442	203.2662265568\\
64.0833333333333	0.25608	208.323328995337\\
64.0833333333333	0.25774	213.380431433874\\
64.0833333333333	0.2594	218.437533872412\\
64.0833333333333	0.26106	223.494636310949\\
64.0833333333333	0.26272	228.551738749486\\
64.0833333333333	0.26438	233.608841188023\\
64.0833333333333	0.26604	238.665943626561\\
64.0833333333333	0.2677	243.723046065099\\
64.0833333333333	0.26936	248.780148503636\\
64.0833333333333	0.27102	253.837250942173\\
64.0833333333333	0.27268	258.894353380711\\
64.0833333333333	0.27434	263.951455819248\\
64.0833333333333	0.276	269.008558257786\\
64.2916666666667	0.193	14.6225076799867\\
64.2916666666667	0.19466	19.6339677128617\\
64.2916666666667	0.19632	24.6454277457369\\
64.2916666666667	0.19798	29.6568877786115\\
64.2916666666667	0.19964	34.6683478114869\\
64.2916666666667	0.2013	39.6798078443617\\
64.2916666666667	0.20296	44.6912678772367\\
64.2916666666667	0.20462	49.7027279101114\\
64.2916666666667	0.20628	54.7141879429867\\
64.2916666666667	0.20794	59.7256479758614\\
64.2916666666667	0.2096	64.7371080087362\\
64.2916666666667	0.21126	69.7485680416112\\
64.2916666666667	0.21292	74.7600280744859\\
64.2916666666667	0.21458	79.7714881073609\\
64.2916666666667	0.21624	84.7829481402359\\
64.2916666666667	0.2179	89.7944081731109\\
64.2916666666667	0.21956	94.8058682059857\\
64.2916666666667	0.22122	99.8173282388611\\
64.2916666666667	0.22288	104.828788271736\\
64.2916666666667	0.22454	109.840248304611\\
64.2916666666667	0.2262	114.851708337486\\
64.2916666666667	0.22786	119.86316837036\\
64.2916666666667	0.22952	124.874628403235\\
64.2916666666667	0.23118	129.886088436111\\
64.2916666666667	0.23284	134.897548468986\\
64.2916666666667	0.2345	139.909008501861\\
64.2916666666667	0.23616	144.920468534735\\
64.2916666666667	0.23782	149.93192856761\\
64.2916666666667	0.23948	154.943388600485\\
64.2916666666667	0.24114	159.95484863336\\
64.2916666666667	0.2428	164.966308666235\\
64.2916666666667	0.24446	169.97776869911\\
64.2916666666667	0.24612	174.989228731985\\
64.2916666666667	0.24778	180.00068876486\\
64.2916666666667	0.24944	185.012148797735\\
64.2916666666667	0.2511	190.02360883061\\
64.2916666666667	0.25276	195.035068863485\\
64.2916666666667	0.25442	200.04652889636\\
64.2916666666667	0.25608	205.057988929235\\
64.2916666666667	0.25774	210.06944896211\\
64.2916666666667	0.2594	215.080908994985\\
64.2916666666667	0.26106	220.09236902786\\
64.2916666666667	0.26272	225.103829060734\\
64.2916666666667	0.26438	230.115289093609\\
64.2916666666667	0.26604	235.126749126485\\
64.2916666666667	0.2677	240.13820915936\\
64.2916666666667	0.26936	245.149669192234\\
64.2916666666667	0.27102	250.161129225109\\
64.2916666666667	0.27268	255.172589257984\\
64.2916666666667	0.27434	260.184049290859\\
64.2916666666667	0.276	265.195509323734\\
64.5	0.193	13.0915790290571\\
64.5	0.19466	18.0573966562697\\
64.5	0.19632	23.0232142834825\\
64.5	0.19798	27.9890319106948\\
64.5	0.19964	32.9548495379076\\
64.5	0.2013	37.92066716512\\
64.5	0.20296	42.8864847923326\\
64.5	0.20462	47.8523024195449\\
64.5	0.20628	52.8181200467575\\
64.5	0.20794	57.7839376739701\\
64.5	0.2096	62.7497553011824\\
64.5	0.21126	67.7155729283947\\
64.5	0.21292	72.6813905556073\\
64.5	0.21458	77.6472081828197\\
64.5	0.21624	82.6130258100322\\
64.5	0.2179	87.5788434372448\\
64.5	0.21956	92.5446610644572\\
64.5	0.22122	97.5104786916702\\
64.5	0.22288	102.476296318882\\
64.5	0.22454	107.442113946095\\
64.5	0.2262	112.407931573308\\
64.5	0.22786	117.37374920052\\
64.5	0.22952	122.339566827732\\
64.5	0.23118	127.305384454945\\
64.5	0.23284	132.271202082158\\
64.5	0.2345	137.23701970937\\
64.5	0.23616	142.202837336582\\
64.5	0.23782	147.168654963795\\
64.5	0.23948	152.134472591008\\
64.5	0.24114	157.10029021822\\
64.5	0.2428	162.066107845432\\
64.5	0.24446	167.031925472645\\
64.5	0.24612	171.997743099858\\
64.5	0.24778	176.96356072707\\
64.5	0.24944	181.929378354283\\
64.5	0.2511	186.895195981495\\
64.5	0.25276	191.861013608707\\
64.5	0.25442	196.82683123592\\
64.5	0.25608	201.792648863133\\
64.5	0.25774	206.758466490345\\
64.5	0.2594	211.724284117558\\
64.5	0.26106	216.69010174477\\
64.5	0.26272	221.655919371982\\
64.5	0.26438	226.621736999195\\
64.5	0.26604	231.587554626408\\
64.5	0.2677	236.55337225362\\
64.5	0.26936	241.519189880833\\
64.5	0.27102	246.485007508045\\
64.5	0.27268	251.450825135257\\
64.5	0.27434	256.41664276247\\
64.5	0.276	261.382460389683\\
64.7083333333333	0.193	11.5606503781275\\
64.7083333333333	0.19466	16.4808255996775\\
64.7083333333333	0.19632	21.4010008212279\\
64.7083333333333	0.19798	26.3211760427778\\
64.7083333333333	0.19964	31.2413512643282\\
64.7083333333333	0.2013	36.1615264858781\\
64.7083333333333	0.20296	41.0817017074282\\
64.7083333333333	0.20462	46.0018769289782\\
64.7083333333333	0.20628	50.9220521505283\\
64.7083333333333	0.20794	55.8422273720782\\
64.7083333333333	0.2096	60.7624025936284\\
64.7083333333333	0.21126	65.6825778151783\\
64.7083333333333	0.21292	70.6027530367282\\
64.7083333333333	0.21458	75.5229282582784\\
64.7083333333333	0.21624	80.4431034798286\\
64.7083333333333	0.2179	85.3632787013785\\
64.7083333333333	0.21956	90.2834539229284\\
64.7083333333333	0.22122	95.203629144479\\
64.7083333333333	0.22288	100.123804366029\\
64.7083333333333	0.22454	105.043979587579\\
64.7083333333333	0.2262	109.964154809129\\
64.7083333333333	0.22786	114.884330030679\\
64.7083333333333	0.22952	119.804505252229\\
64.7083333333333	0.23118	124.724680473779\\
64.7083333333333	0.23284	129.644855695329\\
64.7083333333333	0.2345	134.565030916879\\
64.7083333333333	0.23616	139.485206138429\\
64.7083333333333	0.23782	144.40538135998\\
64.7083333333333	0.23948	149.32555658153\\
64.7083333333333	0.24114	154.24573180308\\
64.7083333333333	0.2428	159.16590702463\\
64.7083333333333	0.24446	164.08608224618\\
64.7083333333333	0.24612	169.00625746773\\
64.7083333333333	0.24778	173.92643268928\\
64.7083333333333	0.24944	178.84660791083\\
64.7083333333333	0.2511	183.76678313238\\
64.7083333333333	0.25276	188.68695835393\\
64.7083333333333	0.25442	193.60713357548\\
64.7083333333333	0.25608	198.52730879703\\
64.7083333333333	0.25774	203.44748401858\\
64.7083333333333	0.2594	208.36765924013\\
64.7083333333333	0.26106	213.28783446168\\
64.7083333333333	0.26272	218.20800968323\\
64.7083333333333	0.26438	223.12818490478\\
64.7083333333333	0.26604	228.048360126331\\
64.7083333333333	0.2677	232.968535347881\\
64.7083333333333	0.26936	237.888710569431\\
64.7083333333333	0.27102	242.808885790981\\
64.7083333333333	0.27268	247.729061012531\\
64.7083333333333	0.27434	252.64923623408\\
64.7083333333333	0.276	257.569411455631\\
64.9166666666667	0.193	10.029721727198\\
64.9166666666667	0.19466	14.9042545430855\\
64.9166666666667	0.19632	19.7787873589734\\
64.9166666666667	0.19798	24.6533201748607\\
64.9166666666667	0.19964	29.5278529907487\\
64.9166666666667	0.2013	34.4023858066364\\
64.9166666666667	0.20296	39.2769186225239\\
64.9166666666667	0.20462	44.1514514384114\\
64.9166666666667	0.20628	49.0259842542991\\
64.9166666666667	0.20794	53.9005170701867\\
64.9166666666667	0.2096	58.7750498860744\\
64.9166666666667	0.21126	63.6495827019619\\
64.9166666666667	0.21292	68.5241155178494\\
64.9166666666667	0.21458	73.3986483337369\\
64.9166666666667	0.21624	78.2731811496246\\
64.9166666666667	0.2179	83.1477139655124\\
64.9166666666667	0.21956	88.0222467813999\\
64.9166666666667	0.22122	92.8967795972878\\
64.9166666666667	0.22288	97.7713124131751\\
64.9166666666667	0.22454	102.645845229063\\
64.9166666666667	0.2262	107.520378044951\\
64.9166666666667	0.22786	112.394910860838\\
64.9166666666667	0.22952	117.269443676726\\
64.9166666666667	0.23118	122.143976492614\\
64.9166666666667	0.23284	127.018509308501\\
64.9166666666667	0.2345	131.893042124389\\
64.9166666666667	0.23616	136.767574940276\\
64.9166666666667	0.23782	141.642107756164\\
64.9166666666667	0.23948	146.516640572052\\
64.9166666666667	0.24114	151.391173387939\\
64.9166666666667	0.2428	156.265706203827\\
64.9166666666667	0.24446	161.140239019714\\
64.9166666666667	0.24612	166.014771835602\\
64.9166666666667	0.24778	170.88930465149\\
64.9166666666667	0.24944	175.763837467377\\
64.9166666666667	0.2511	180.638370283265\\
64.9166666666667	0.25276	185.512903099152\\
64.9166666666667	0.25442	190.38743591504\\
64.9166666666667	0.25608	195.261968730928\\
64.9166666666667	0.25774	200.136501546815\\
64.9166666666667	0.2594	205.011034362703\\
64.9166666666667	0.26106	209.88556717859\\
64.9166666666667	0.26272	214.760099994478\\
64.9166666666667	0.26438	219.634632810366\\
64.9166666666667	0.26604	224.509165626254\\
64.9166666666667	0.2677	229.383698442141\\
64.9166666666667	0.26936	234.258231258029\\
64.9166666666667	0.27102	239.132764073916\\
64.9166666666667	0.27268	244.007296889804\\
64.9166666666667	0.27434	248.881829705691\\
64.9166666666667	0.276	253.756362521579\\
65.125	0.193	8.49879307626816\\
65.125	0.19466	13.3276834864932\\
65.125	0.19632	18.1565738967188\\
65.125	0.19798	22.9854643069436\\
65.125	0.19964	27.8143547171692\\
65.125	0.2013	32.6432451273945\\
65.125	0.20296	37.4721355376196\\
65.125	0.20462	42.3010259478447\\
65.125	0.20628	47.12991635807\\
65.125	0.20794	51.9588067682951\\
65.125	0.2096	56.7876971785201\\
65.125	0.21126	61.6165875887452\\
65.125	0.21292	66.4454779989705\\
65.125	0.21458	71.2743684091956\\
65.125	0.21624	76.1032588194209\\
65.125	0.2179	80.932149229646\\
65.125	0.21956	85.7610396398711\\
65.125	0.22122	90.5899300500967\\
65.125	0.22288	95.4188204603215\\
65.125	0.22454	100.247710870547\\
65.125	0.2262	105.076601280772\\
65.125	0.22786	109.905491690997\\
65.125	0.22952	114.734382101223\\
65.125	0.23118	119.563272511448\\
65.125	0.23284	124.392162921673\\
65.125	0.2345	129.221053331898\\
65.125	0.23616	134.049943742123\\
65.125	0.23782	138.878834152348\\
65.125	0.23948	143.707724562574\\
65.125	0.24114	148.536614972799\\
65.125	0.2428	153.365505383024\\
65.125	0.24446	158.194395793249\\
65.125	0.24612	163.023286203475\\
65.125	0.24778	167.8521766137\\
65.125	0.24944	172.681067023925\\
65.125	0.2511	177.50995743415\\
65.125	0.25276	182.338847844375\\
65.125	0.25442	187.1677382546\\
65.125	0.25608	191.996628664825\\
65.125	0.25774	196.825519075051\\
65.125	0.2594	201.654409485276\\
65.125	0.26106	206.483299895501\\
65.125	0.26272	211.312190305726\\
65.125	0.26438	216.141080715951\\
65.125	0.26604	220.969971126177\\
65.125	0.2677	225.798861536402\\
65.125	0.26936	230.627751946627\\
65.125	0.27102	235.456642356852\\
65.125	0.27268	240.285532767077\\
65.125	0.27434	245.114423177302\\
65.125	0.276	249.943313587528\\
65.3333333333333	0.193	6.96786442533858\\
65.3333333333333	0.19466	11.7511124299012\\
65.3333333333333	0.19632	16.5343604344644\\
65.3333333333333	0.19798	21.3176084390268\\
65.3333333333333	0.19964	26.1008564435901\\
65.3333333333333	0.2013	30.8841044481528\\
65.3333333333333	0.20296	35.6673524527155\\
65.3333333333333	0.20462	40.4506004572781\\
65.3333333333333	0.20628	45.233848461841\\
65.3333333333333	0.20794	50.0170964664037\\
65.3333333333333	0.2096	54.8003444709664\\
65.3333333333333	0.21126	59.583592475529\\
65.3333333333333	0.21292	64.3668404800917\\
65.3333333333333	0.21458	69.1500884846544\\
65.3333333333333	0.21624	73.9333364892173\\
65.3333333333333	0.2179	78.7165844937799\\
65.3333333333333	0.21956	83.4998324983426\\
65.3333333333333	0.22122	88.2830805029057\\
65.3333333333333	0.22288	93.0663285074681\\
65.3333333333333	0.22454	97.8495765120313\\
65.3333333333333	0.2262	102.632824516594\\
65.3333333333333	0.22786	107.416072521156\\
65.3333333333333	0.22952	112.199320525719\\
65.3333333333333	0.23118	116.982568530282\\
65.3333333333333	0.23284	121.765816534845\\
65.3333333333333	0.2345	126.549064539408\\
65.3333333333333	0.23616	131.332312543971\\
65.3333333333333	0.23782	136.115560548533\\
65.3333333333333	0.23948	140.898808553095\\
65.3333333333333	0.24114	145.682056557658\\
65.3333333333333	0.2428	150.465304562221\\
65.3333333333333	0.24446	155.248552566783\\
65.3333333333333	0.24612	160.031800571347\\
65.3333333333333	0.24778	164.81504857591\\
65.3333333333333	0.24944	169.598296580472\\
65.3333333333333	0.2511	174.381544585035\\
65.3333333333333	0.25276	179.164792589598\\
65.3333333333333	0.25442	183.948040594161\\
65.3333333333333	0.25608	188.731288598724\\
65.3333333333333	0.25774	193.514536603287\\
65.3333333333333	0.2594	198.297784607849\\
65.3333333333333	0.26106	203.081032612412\\
65.3333333333333	0.26272	207.864280616975\\
65.3333333333333	0.26438	212.647528621537\\
65.3333333333333	0.26604	217.4307766261\\
65.3333333333333	0.2677	222.214024630663\\
65.3333333333333	0.26936	226.997272635226\\
65.3333333333333	0.27102	231.780520639788\\
65.3333333333333	0.27268	236.563768644351\\
65.3333333333333	0.27434	241.347016648914\\
65.3333333333333	0.276	246.130264653476\\
65.5416666666667	0.193	5.43693577440899\\
65.5416666666667	0.19466	10.1745413733092\\
65.5416666666667	0.19632	14.9121469722099\\
65.5416666666667	0.19798	19.64975257111\\
65.5416666666667	0.19964	24.3873581700107\\
65.5416666666667	0.2013	29.1249637689109\\
65.5416666666667	0.20296	33.8625693678111\\
65.5416666666667	0.20462	38.6001749667114\\
65.5416666666667	0.20628	43.3377805656119\\
65.5416666666667	0.20794	48.0753861645121\\
65.5416666666667	0.2096	52.8129917634124\\
65.5416666666667	0.21126	57.5505973623126\\
65.5416666666667	0.21292	62.2882029612128\\
65.5416666666667	0.21458	67.0258085601131\\
65.5416666666667	0.21624	71.7634141590136\\
65.5416666666667	0.2179	76.5010197579138\\
65.5416666666667	0.21956	81.2386253568141\\
65.5416666666667	0.22122	85.9762309557148\\
65.5416666666667	0.22288	90.7138365546145\\
65.5416666666667	0.22454	95.4514421535152\\
65.5416666666667	0.2262	100.189047752416\\
65.5416666666667	0.22786	104.926653351316\\
65.5416666666667	0.22952	109.664258950216\\
65.5416666666667	0.23118	114.401864549116\\
65.5416666666667	0.23284	119.139470148016\\
65.5416666666667	0.2345	123.877075746917\\
65.5416666666667	0.23616	128.614681345817\\
65.5416666666667	0.23782	133.352286944718\\
65.5416666666667	0.23948	138.089892543618\\
65.5416666666667	0.24114	142.827498142518\\
65.5416666666667	0.2428	147.565103741419\\
65.5416666666667	0.24446	152.302709340319\\
65.5416666666667	0.24612	157.040314939219\\
65.5416666666667	0.24778	161.777920538119\\
65.5416666666667	0.24944	166.51552613702\\
65.5416666666667	0.2511	171.25313173592\\
65.5416666666667	0.25276	175.99073733482\\
65.5416666666667	0.25442	180.728342933721\\
65.5416666666667	0.25608	185.465948532621\\
65.5416666666667	0.25774	190.203554131521\\
65.5416666666667	0.2594	194.941159730422\\
65.5416666666667	0.26106	199.678765329322\\
65.5416666666667	0.26272	204.416370928222\\
65.5416666666667	0.26438	209.153976527122\\
65.5416666666667	0.26604	213.891582126023\\
65.5416666666667	0.2677	218.629187724923\\
65.5416666666667	0.26936	223.366793323823\\
65.5416666666667	0.27102	228.104398922724\\
65.5416666666667	0.27268	232.842004521624\\
65.5416666666667	0.27434	237.579610120524\\
65.5416666666667	0.276	242.317215719425\\
65.75	0.193	3.90600712347918\\
65.75	0.19466	8.59797031671701\\
65.75	0.19632	13.2899335099553\\
65.75	0.19798	17.9818967031929\\
65.75	0.19964	22.6738598964312\\
65.75	0.2013	27.365823089669\\
65.75	0.20296	32.0577862829068\\
65.75	0.20462	36.7497494761446\\
65.75	0.20628	41.4417126693827\\
65.75	0.20794	46.1336758626203\\
65.75	0.2096	50.8256390558581\\
65.75	0.21126	55.5176022490959\\
65.75	0.21292	60.2095654423338\\
65.75	0.21458	64.9015286355716\\
65.75	0.21624	69.5934918288096\\
65.75	0.2179	74.2854550220475\\
65.75	0.21956	78.9774182152853\\
65.75	0.22122	83.6693814085236\\
65.75	0.22288	88.3613446017609\\
65.75	0.22454	93.0533077949992\\
65.75	0.2262	97.7452709882373\\
65.75	0.22786	102.437234181475\\
65.75	0.22952	107.129197374713\\
65.75	0.23118	111.821160567951\\
65.75	0.23284	116.513123761188\\
65.75	0.2345	121.205086954426\\
65.75	0.23616	125.897050147664\\
65.75	0.23782	130.589013340903\\
65.75	0.23948	135.28097653414\\
65.75	0.24114	139.972939727378\\
65.75	0.2428	144.664902920616\\
65.75	0.24446	149.356866113854\\
65.75	0.24612	154.048829307092\\
65.75	0.24778	158.740792500329\\
65.75	0.24944	163.432755693568\\
65.75	0.2511	168.124718886805\\
65.75	0.25276	172.816682080042\\
65.75	0.25442	177.508645273281\\
65.75	0.25608	182.200608466519\\
65.75	0.25774	186.892571659756\\
65.75	0.2594	191.584534852994\\
65.75	0.26106	196.276498046232\\
65.75	0.26272	200.96846123947\\
65.75	0.26438	205.660424432708\\
65.75	0.26604	210.352387625946\\
65.75	0.2677	215.044350819184\\
65.75	0.26936	219.736314012422\\
65.75	0.27102	224.428277205659\\
65.75	0.27268	229.120240398897\\
65.75	0.27434	233.812203592135\\
65.75	0.276	238.504166785374\\
65.9583333333333	0.193	2.37507847254938\\
65.9583333333333	0.19466	7.02139926012478\\
65.9583333333333	0.19632	11.6677200477006\\
65.9583333333333	0.19798	16.3140408352758\\
65.9583333333333	0.19964	20.9603616228515\\
65.9583333333333	0.2013	25.6066824104269\\
65.9583333333333	0.20296	30.2530031980023\\
65.9583333333333	0.20462	34.8993239855777\\
65.9583333333333	0.20628	39.5456447731533\\
65.9583333333333	0.20794	44.1919655607287\\
65.9583333333333	0.2096	48.8382863483039\\
65.9583333333333	0.21126	53.4846071358793\\
65.9583333333333	0.21292	58.1309279234547\\
65.9583333333333	0.21458	62.7772487110301\\
65.9583333333333	0.21624	67.4235694986057\\
65.9583333333333	0.2179	72.0698902861809\\
65.9583333333333	0.21956	76.7162110737563\\
65.9583333333333	0.22122	81.3625318613322\\
65.9583333333333	0.22288	86.0088526489074\\
65.9583333333333	0.22454	90.6551734364832\\
65.9583333333333	0.2262	95.3014942240584\\
65.9583333333333	0.22786	99.9478150116338\\
65.9583333333333	0.22952	104.594135799209\\
65.9583333333333	0.23118	109.240456586785\\
65.9583333333333	0.23284	113.88677737436\\
65.9583333333333	0.2345	118.533098161935\\
65.9583333333333	0.23616	123.179418949511\\
65.9583333333333	0.23782	127.825739737086\\
65.9583333333333	0.23948	132.472060524662\\
65.9583333333333	0.24114	137.118381312237\\
65.9583333333333	0.2428	141.764702099812\\
65.9583333333333	0.24446	146.411022887388\\
65.9583333333333	0.24612	151.057343674964\\
65.9583333333333	0.24778	155.703664462539\\
65.9583333333333	0.24944	160.349985250115\\
65.9583333333333	0.2511	164.99630603769\\
65.9583333333333	0.25276	169.642626825265\\
65.9583333333333	0.25442	174.288947612841\\
65.9583333333333	0.25608	178.935268400416\\
65.9583333333333	0.25774	183.581589187992\\
65.9583333333333	0.2594	188.227909975567\\
65.9583333333333	0.26106	192.874230763142\\
65.9583333333333	0.26272	197.520551550718\\
65.9583333333333	0.26438	202.166872338293\\
65.9583333333333	0.26604	206.813193125869\\
65.9583333333333	0.2677	211.459513913444\\
65.9583333333333	0.26936	216.10583470102\\
65.9583333333333	0.27102	220.752155488595\\
65.9583333333333	0.27268	225.39847627617\\
65.9583333333333	0.27434	230.044797063746\\
65.9583333333333	0.276	234.691117851321\\
66.1666666666667	0.193	0.844149821619567\\
66.1666666666667	0.19466	5.44482820353255\\
66.1666666666667	0.19632	10.045506585446\\
66.1666666666667	0.19798	14.6461849673585\\
66.1666666666667	0.19964	19.246863349272\\
66.1666666666667	0.2013	23.8475417311849\\
66.1666666666667	0.20296	28.4482201130977\\
66.1666666666667	0.20462	33.0488984950107\\
66.1666666666667	0.20628	37.6495768769239\\
66.1666666666667	0.20794	42.2502552588369\\
66.1666666666667	0.2096	46.8509336407496\\
66.1666666666667	0.21126	51.4516120226626\\
66.1666666666667	0.21292	56.0522904045756\\
66.1666666666667	0.21458	60.6529687864884\\
66.1666666666667	0.21624	65.2536471684016\\
66.1666666666667	0.2179	69.8543255503146\\
66.1666666666667	0.21956	74.4550039322276\\
66.1666666666667	0.22122	79.0556823141408\\
66.1666666666667	0.22288	83.6563606960535\\
66.1666666666667	0.22454	88.2570390779667\\
66.1666666666667	0.2262	92.8577174598795\\
66.1666666666667	0.22786	97.4583958417925\\
66.1666666666667	0.22952	102.059074223705\\
66.1666666666667	0.23118	106.659752605619\\
66.1666666666667	0.23284	111.260430987532\\
66.1666666666667	0.2345	115.861109369445\\
66.1666666666667	0.23616	120.461787751358\\
66.1666666666667	0.23782	125.06246613327\\
66.1666666666667	0.23948	129.663144515183\\
66.1666666666667	0.24114	134.263822897096\\
66.1666666666667	0.2428	138.864501279009\\
66.1666666666667	0.24446	143.465179660922\\
66.1666666666667	0.24612	148.065858042836\\
66.1666666666667	0.24778	152.666536424749\\
66.1666666666667	0.24944	157.267214806662\\
66.1666666666667	0.2511	161.867893188575\\
66.1666666666667	0.25276	166.468571570488\\
66.1666666666667	0.25442	171.069249952401\\
66.1666666666667	0.25608	175.669928334314\\
66.1666666666667	0.25774	180.270606716227\\
66.1666666666667	0.2594	184.87128509814\\
66.1666666666667	0.26106	189.471963480053\\
66.1666666666667	0.26272	194.072641861966\\
66.1666666666667	0.26438	198.673320243879\\
66.1666666666667	0.26604	203.273998625792\\
66.1666666666667	0.2677	207.874677007705\\
66.1666666666667	0.26936	212.475355389618\\
66.1666666666667	0.27102	217.076033771531\\
66.1666666666667	0.27268	221.676712153444\\
66.1666666666667	0.27434	226.277390535357\\
66.1666666666667	0.276	230.87806891727\\
66.375	0.193	-0.686778829310015\\
66.375	0.19466	3.86825714694055\\
66.375	0.19632	8.42329312319157\\
66.375	0.19798	12.9783290994417\\
66.375	0.19964	17.5333650756927\\
66.375	0.2013	22.0884010519433\\
66.375	0.20296	26.6434370281936\\
66.375	0.20462	31.1984730044442\\
66.375	0.20628	35.753508980695\\
66.375	0.20794	40.3085449569453\\
66.375	0.2096	44.8635809331959\\
66.375	0.21126	49.4186169094464\\
66.375	0.21292	53.9736528856968\\
66.375	0.21458	58.5286888619473\\
66.375	0.21624	63.0837248381979\\
66.375	0.2179	67.6387608144485\\
66.375	0.21956	72.193796790699\\
66.375	0.22122	76.7488327669498\\
66.375	0.22288	81.3038687432002\\
66.375	0.22454	85.8589047194512\\
66.375	0.2262	90.4139406957015\\
66.375	0.22786	94.9689766719521\\
66.375	0.22952	99.5240126482026\\
66.375	0.23118	104.079048624453\\
66.375	0.23284	108.634084600704\\
66.375	0.2345	113.189120576954\\
66.375	0.23616	117.744156553205\\
66.375	0.23782	122.299192529455\\
66.375	0.23948	126.854228505706\\
66.375	0.24114	131.409264481957\\
66.375	0.2428	135.964300458207\\
66.375	0.24446	140.519336434457\\
66.375	0.24612	145.074372410708\\
66.375	0.24778	149.629408386958\\
66.375	0.24944	154.184444363209\\
66.375	0.2511	158.73948033946\\
66.375	0.25276	163.29451631571\\
66.375	0.25442	167.849552291961\\
66.375	0.25608	172.404588268212\\
66.375	0.25774	176.959624244462\\
66.375	0.2594	181.514660220713\\
66.375	0.26106	186.069696196963\\
66.375	0.26272	190.624732173213\\
66.375	0.26438	195.179768149464\\
66.375	0.26604	199.734804125715\\
66.375	0.2677	204.289840101966\\
66.375	0.26936	208.844876078216\\
66.375	0.27102	213.399912054466\\
66.375	0.27268	217.954948030717\\
66.375	0.27434	222.509984006967\\
66.375	0.276	227.065019983218\\
66.5833333333333	0.193	-2.2177074802396\\
66.5833333333333	0.19466	2.29168609034832\\
66.5833333333333	0.19632	6.80107966093692\\
66.5833333333333	0.19798	11.3104732315246\\
66.5833333333333	0.19964	15.8198668021132\\
66.5833333333333	0.2013	20.3292603727014\\
66.5833333333333	0.20296	24.8386539432893\\
66.5833333333333	0.20462	29.3480475138774\\
66.5833333333333	0.20628	33.8574410844656\\
66.5833333333333	0.20794	38.3668346550537\\
66.5833333333333	0.2096	42.8762282256416\\
66.5833333333333	0.21126	47.3856217962298\\
66.5833333333333	0.21292	51.8950153668179\\
66.5833333333333	0.21458	56.4044089374058\\
66.5833333333333	0.21624	60.9138025079942\\
66.5833333333333	0.2179	65.4231960785821\\
66.5833333333333	0.21956	69.9325896491703\\
66.5833333333333	0.22122	74.4419832197589\\
66.5833333333333	0.22288	78.9513767903466\\
66.5833333333333	0.22454	83.4607703609352\\
66.5833333333333	0.2262	87.9701639315231\\
66.5833333333333	0.22786	92.4795575021112\\
66.5833333333333	0.22952	96.9889510726991\\
66.5833333333333	0.23118	101.498344643288\\
66.5833333333333	0.23284	106.007738213875\\
66.5833333333333	0.2345	110.517131784464\\
66.5833333333333	0.23616	115.026525355051\\
66.5833333333333	0.23782	119.53591892564\\
66.5833333333333	0.23948	124.045312496228\\
66.5833333333333	0.24114	128.554706066816\\
66.5833333333333	0.2428	133.064099637404\\
66.5833333333333	0.24446	137.573493207992\\
66.5833333333333	0.24612	142.082886778581\\
66.5833333333333	0.24778	146.592280349169\\
66.5833333333333	0.24944	151.101673919757\\
66.5833333333333	0.2511	155.611067490345\\
66.5833333333333	0.25276	160.120461060933\\
66.5833333333333	0.25442	164.629854631521\\
66.5833333333333	0.25608	169.139248202109\\
66.5833333333333	0.25774	173.648641772697\\
66.5833333333333	0.2594	178.158035343286\\
66.5833333333333	0.26106	182.667428913874\\
66.5833333333333	0.26272	187.176822484461\\
66.5833333333333	0.26438	191.686216055049\\
66.5833333333333	0.26604	196.195609625638\\
66.5833333333333	0.2677	200.705003196226\\
66.5833333333333	0.26936	205.214396766814\\
66.5833333333333	0.27102	209.723790337402\\
66.5833333333333	0.27268	214.23318390799\\
66.5833333333333	0.27434	218.742577478578\\
66.5833333333333	0.276	223.251971049167\\
66.7916666666667	0.193	-3.74863613116918\\
66.7916666666667	0.19466	0.715115033756319\\
66.7916666666667	0.19632	5.1788661986825\\
66.7916666666667	0.19798	9.64261736360777\\
66.7916666666667	0.19964	14.1063685285339\\
66.7916666666667	0.2013	18.5701196934594\\
66.7916666666667	0.20296	23.0338708583852\\
66.7916666666667	0.20462	27.4976220233107\\
66.7916666666667	0.20628	31.9613731882364\\
66.7916666666667	0.20794	36.4251243531621\\
66.7916666666667	0.2096	40.8888755180876\\
66.7916666666667	0.21126	45.3526266830133\\
66.7916666666667	0.21292	49.8163778479388\\
66.7916666666667	0.21458	54.2801290128646\\
66.7916666666667	0.21624	58.7438801777903\\
66.7916666666667	0.2179	63.207631342716\\
66.7916666666667	0.21956	67.6713825076415\\
66.7916666666667	0.22122	72.1351336725677\\
66.7916666666667	0.22288	76.598884837493\\
66.7916666666667	0.22454	81.0626360024191\\
66.7916666666667	0.2262	85.5263871673446\\
66.7916666666667	0.22786	89.9901383322704\\
66.7916666666667	0.22952	94.4538894971959\\
66.7916666666667	0.23118	98.9176406621218\\
66.7916666666667	0.23284	103.381391827048\\
66.7916666666667	0.2345	107.845142991973\\
66.7916666666667	0.23616	112.308894156899\\
66.7916666666667	0.23782	116.772645321824\\
66.7916666666667	0.23948	121.23639648675\\
66.7916666666667	0.24114	125.700147651676\\
66.7916666666667	0.2428	130.163898816601\\
66.7916666666667	0.24446	134.627649981527\\
66.7916666666667	0.24612	139.091401146453\\
66.7916666666667	0.24778	143.555152311378\\
66.7916666666667	0.24944	148.018903476304\\
66.7916666666667	0.2511	152.48265464123\\
66.7916666666667	0.25276	156.946405806155\\
66.7916666666667	0.25442	161.410156971082\\
66.7916666666667	0.25608	165.873908136007\\
66.7916666666667	0.25774	170.337659300933\\
66.7916666666667	0.2594	174.801410465858\\
66.7916666666667	0.26106	179.265161630784\\
66.7916666666667	0.26272	183.728912795709\\
66.7916666666667	0.26438	188.192663960635\\
66.7916666666667	0.26604	192.656415125561\\
66.7916666666667	0.2677	197.120166290487\\
66.7916666666667	0.26936	201.583917455412\\
66.7916666666667	0.27102	206.047668620338\\
66.7916666666667	0.27268	210.511419785264\\
66.7916666666667	0.27434	214.975170950189\\
66.7916666666667	0.276	219.438922115115\\
67	0.193	-5.27956478209876\\
67	0.19466	-0.861456022835682\\
67	0.19632	3.55665273642808\\
67	0.19798	7.97476149569093\\
67	0.19964	12.3928702549547\\
67	0.2013	16.8109790142178\\
67	0.20296	21.2290877734808\\
67	0.20462	25.6471965327441\\
67	0.20628	30.0653052920075\\
67	0.20794	34.4834140512708\\
67	0.2096	38.9015228105338\\
67	0.21126	43.3196315697969\\
67	0.21292	47.7377403290602\\
67	0.21458	52.1558490883233\\
67	0.21624	56.5739578475868\\
67	0.2179	60.9920666068499\\
67	0.21956	65.4101753661132\\
67	0.22122	69.8282841253767\\
67	0.22288	74.2463928846396\\
67	0.22454	78.6645016439031\\
67	0.2262	83.0826104031662\\
67	0.22786	87.5007191624295\\
67	0.22952	91.9188279216926\\
67	0.23118	96.3369366809563\\
67	0.23284	100.755045440219\\
67	0.2345	105.173154199483\\
67	0.23616	109.591262958746\\
67	0.23782	114.009371718009\\
67	0.23948	118.427480477272\\
67	0.24114	122.845589236535\\
67	0.2428	127.263697995798\\
67	0.24446	131.681806755062\\
67	0.24612	136.099915514326\\
67	0.24778	140.518024273589\\
67	0.24944	144.936133032852\\
67	0.2511	149.354241792115\\
67	0.25276	153.772350551378\\
67	0.25442	158.190459310642\\
67	0.25608	162.608568069905\\
67	0.25774	167.026676829168\\
67	0.2594	171.444785588431\\
67	0.26106	175.862894347695\\
67	0.26272	180.281003106958\\
67	0.26438	184.699111866221\\
67	0.26604	189.117220625485\\
67	0.2677	193.535329384748\\
67	0.26936	197.953438144011\\
67	0.27102	202.371546903274\\
67	0.27268	206.789655662537\\
67	0.27434	211.2077644218\\
67	0.276	215.625873181064\\
};
\end{axis}

\begin{axis}[%
width=4.927496cm,
height=3.050847cm,
at={(6.483547cm,0cm)},
scale only axis,
xmin=57,
xmax=67,
tick align=outside,
xlabel={$L_{cut}$},
xmajorgrids,
ymin=0.193,
ymax=0.276,
ylabel={$D_{rlx}$},
ymajorgrids,
zmin=-74.5762065540795,
zmax=0,
zlabel={$x_2,x_4$},
zmajorgrids,
view={-140}{50},
legend style={at={(1.03,1)},anchor=north west,legend cell align=left,align=left,draw=white!15!black}
]
\addplot3[only marks,mark=*,mark options={},mark size=1.5000pt,color=mycolor1] plot table[row sep=crcr,]{%
67	0.276	-47.8219770062711\\
66	0.255	-39.3304930498637\\
62	0.209	-19.5983605432748\\
57	0.193	-13.7917672765835\\
};
\addplot3[only marks,mark=*,mark options={},mark size=1.5000pt,color=black] plot table[row sep=crcr,]{%
64	0.23	-29.3369759790979\\
};

\addplot3[%
surf,
opacity=0.7,
shader=interp,
colormap={mymap}{[1pt] rgb(0pt)=(0.0901961,0.239216,0.0745098); rgb(1pt)=(0.0945149,0.242058,0.0739522); rgb(2pt)=(0.0988592,0.244894,0.0733566); rgb(3pt)=(0.103229,0.247724,0.0727241); rgb(4pt)=(0.107623,0.250549,0.0720557); rgb(5pt)=(0.112043,0.253367,0.0713525); rgb(6pt)=(0.116487,0.25618,0.0706154); rgb(7pt)=(0.120956,0.258986,0.0698456); rgb(8pt)=(0.125449,0.261787,0.0690441); rgb(9pt)=(0.129967,0.264581,0.0682118); rgb(10pt)=(0.134508,0.26737,0.06735); rgb(11pt)=(0.139074,0.270152,0.0664596); rgb(12pt)=(0.143663,0.272929,0.0655416); rgb(13pt)=(0.148275,0.275699,0.0645971); rgb(14pt)=(0.152911,0.278463,0.0636271); rgb(15pt)=(0.15757,0.281221,0.0626328); rgb(16pt)=(0.162252,0.283973,0.0616151); rgb(17pt)=(0.166957,0.286719,0.060575); rgb(18pt)=(0.171685,0.289458,0.0595136); rgb(19pt)=(0.176434,0.292191,0.0584321); rgb(20pt)=(0.181207,0.294918,0.0573313); rgb(21pt)=(0.186001,0.297639,0.0562123); rgb(22pt)=(0.190817,0.300353,0.0550763); rgb(23pt)=(0.195655,0.303061,0.0539242); rgb(24pt)=(0.200514,0.305763,0.052757); rgb(25pt)=(0.205395,0.308459,0.0515759); rgb(26pt)=(0.210296,0.311149,0.0503624); rgb(27pt)=(0.215212,0.313846,0.0490067); rgb(28pt)=(0.220142,0.316548,0.0475043); rgb(29pt)=(0.22509,0.319254,0.0458704); rgb(30pt)=(0.230056,0.321962,0.0441205); rgb(31pt)=(0.235042,0.324671,0.04227); rgb(32pt)=(0.240048,0.327379,0.0403343); rgb(33pt)=(0.245078,0.330085,0.0383287); rgb(34pt)=(0.250131,0.332786,0.0362688); rgb(35pt)=(0.25521,0.335482,0.0341698); rgb(36pt)=(0.260317,0.33817,0.0320472); rgb(37pt)=(0.265451,0.340849,0.0299163); rgb(38pt)=(0.270616,0.343517,0.0277927); rgb(39pt)=(0.275813,0.346172,0.0256916); rgb(40pt)=(0.281043,0.348814,0.0236284); rgb(41pt)=(0.286307,0.35144,0.0216186); rgb(42pt)=(0.291607,0.354048,0.0196776); rgb(43pt)=(0.296945,0.356637,0.0178207); rgb(44pt)=(0.302322,0.359206,0.0160634); rgb(45pt)=(0.307739,0.361753,0.0144211); rgb(46pt)=(0.313198,0.364275,0.0129091); rgb(47pt)=(0.318701,0.366772,0.0115428); rgb(48pt)=(0.324249,0.369242,0.0103377); rgb(49pt)=(0.329843,0.371682,0.00930909); rgb(50pt)=(0.335485,0.374093,0.00847245); rgb(51pt)=(0.341176,0.376471,0.00784314); rgb(52pt)=(0.346925,0.378826,0.00732741); rgb(53pt)=(0.352735,0.381168,0.00682184); rgb(54pt)=(0.358605,0.383497,0.00632729); rgb(55pt)=(0.364532,0.385812,0.00584464); rgb(56pt)=(0.370516,0.388113,0.00537476); rgb(57pt)=(0.376552,0.390399,0.00491852); rgb(58pt)=(0.38264,0.39267,0.00447681); rgb(59pt)=(0.388777,0.394925,0.00405048); rgb(60pt)=(0.394962,0.397164,0.00364042); rgb(61pt)=(0.401191,0.399386,0.00324749); rgb(62pt)=(0.407464,0.401592,0.00287258); rgb(63pt)=(0.413777,0.40378,0.00251655); rgb(64pt)=(0.420129,0.40595,0.00218028); rgb(65pt)=(0.426518,0.408102,0.00186463); rgb(66pt)=(0.432942,0.410234,0.00157049); rgb(67pt)=(0.439399,0.412348,0.00129873); rgb(68pt)=(0.445885,0.414441,0.00105022); rgb(69pt)=(0.452401,0.416515,0.000825833); rgb(70pt)=(0.458942,0.418567,0.000626441); rgb(71pt)=(0.465508,0.420599,0.00045292); rgb(72pt)=(0.472096,0.422609,0.000306141); rgb(73pt)=(0.478704,0.424596,0.000186979); rgb(74pt)=(0.485331,0.426562,9.63073e-05); rgb(75pt)=(0.491973,0.428504,3.49981e-05); rgb(76pt)=(0.498628,0.430422,3.92506e-06); rgb(77pt)=(0.505323,0.432315,0); rgb(78pt)=(0.512206,0.434168,0); rgb(79pt)=(0.519282,0.435983,0); rgb(80pt)=(0.526529,0.437764,0); rgb(81pt)=(0.533922,0.439512,0); rgb(82pt)=(0.54144,0.441232,0); rgb(83pt)=(0.549059,0.442927,0); rgb(84pt)=(0.556756,0.444599,0); rgb(85pt)=(0.564508,0.446252,0); rgb(86pt)=(0.572292,0.447889,0); rgb(87pt)=(0.580084,0.449514,0); rgb(88pt)=(0.587863,0.451129,0); rgb(89pt)=(0.595604,0.452737,0); rgb(90pt)=(0.603284,0.454343,0); rgb(91pt)=(0.610882,0.455948,0); rgb(92pt)=(0.618373,0.457556,0); rgb(93pt)=(0.625734,0.459171,0); rgb(94pt)=(0.632943,0.460795,0); rgb(95pt)=(0.639976,0.462432,0); rgb(96pt)=(0.64681,0.464084,0); rgb(97pt)=(0.653423,0.465756,0); rgb(98pt)=(0.659791,0.46745,0); rgb(99pt)=(0.665891,0.469169,0); rgb(100pt)=(0.6717,0.470916,0); rgb(101pt)=(0.677195,0.472696,0); rgb(102pt)=(0.682353,0.47451,0); rgb(103pt)=(0.687242,0.476355,0); rgb(104pt)=(0.691952,0.478225,0); rgb(105pt)=(0.696497,0.480118,0); rgb(106pt)=(0.700887,0.482033,0); rgb(107pt)=(0.705134,0.483968,0); rgb(108pt)=(0.709251,0.485921,0); rgb(109pt)=(0.713249,0.487891,0); rgb(110pt)=(0.71714,0.489876,0); rgb(111pt)=(0.720936,0.491875,0); rgb(112pt)=(0.724649,0.493887,0); rgb(113pt)=(0.72829,0.495909,0); rgb(114pt)=(0.731872,0.49794,0); rgb(115pt)=(0.735406,0.499979,0); rgb(116pt)=(0.738904,0.502025,0); rgb(117pt)=(0.742378,0.504075,0); rgb(118pt)=(0.74584,0.506128,0); rgb(119pt)=(0.749302,0.508182,0); rgb(120pt)=(0.752775,0.510237,0); rgb(121pt)=(0.756272,0.51229,0); rgb(122pt)=(0.759804,0.514339,0); rgb(123pt)=(0.763384,0.516385,0); rgb(124pt)=(0.767022,0.518424,0); rgb(125pt)=(0.770731,0.520455,0); rgb(126pt)=(0.774523,0.522478,0); rgb(127pt)=(0.77841,0.524489,0); rgb(128pt)=(0.782391,0.526491,0); rgb(129pt)=(0.786402,0.528496,0); rgb(130pt)=(0.790431,0.530506,0); rgb(131pt)=(0.794478,0.532521,0); rgb(132pt)=(0.798541,0.534539,0); rgb(133pt)=(0.802619,0.53656,0); rgb(134pt)=(0.806712,0.538584,0); rgb(135pt)=(0.81082,0.540609,0); rgb(136pt)=(0.81494,0.542635,0); rgb(137pt)=(0.819074,0.54466,0); rgb(138pt)=(0.823219,0.546686,0); rgb(139pt)=(0.827374,0.548709,0); rgb(140pt)=(0.831541,0.55073,0); rgb(141pt)=(0.835716,0.552749,0); rgb(142pt)=(0.8399,0.554763,0); rgb(143pt)=(0.844092,0.556774,0); rgb(144pt)=(0.848292,0.558779,0); rgb(145pt)=(0.852497,0.560778,0); rgb(146pt)=(0.856708,0.562771,0); rgb(147pt)=(0.860924,0.564756,0); rgb(148pt)=(0.865143,0.566733,0); rgb(149pt)=(0.869366,0.568701,0); rgb(150pt)=(0.873592,0.57066,0); rgb(151pt)=(0.877819,0.572608,0); rgb(152pt)=(0.882047,0.574545,0); rgb(153pt)=(0.886275,0.576471,0); rgb(154pt)=(0.890659,0.578362,0); rgb(155pt)=(0.895333,0.580203,0); rgb(156pt)=(0.900258,0.581999,0); rgb(157pt)=(0.905397,0.583755,0); rgb(158pt)=(0.910711,0.585479,0); rgb(159pt)=(0.916164,0.587176,0); rgb(160pt)=(0.921717,0.588852,0); rgb(161pt)=(0.927333,0.590513,0); rgb(162pt)=(0.932974,0.592166,0); rgb(163pt)=(0.938602,0.593815,0); rgb(164pt)=(0.94418,0.595468,0); rgb(165pt)=(0.949669,0.59713,0); rgb(166pt)=(0.955033,0.598808,0); rgb(167pt)=(0.960233,0.600507,0); rgb(168pt)=(0.965232,0.602233,0); rgb(169pt)=(0.969992,0.603992,0); rgb(170pt)=(0.974475,0.605791,0); rgb(171pt)=(0.978643,0.607636,0); rgb(172pt)=(0.98246,0.609532,0); rgb(173pt)=(0.985886,0.611486,0); rgb(174pt)=(0.988885,0.613503,0); rgb(175pt)=(0.991419,0.61559,0); rgb(176pt)=(0.99345,0.617753,0); rgb(177pt)=(0.99494,0.619997,0); rgb(178pt)=(0.995851,0.622329,0); rgb(179pt)=(0.996226,0.624763,0); rgb(180pt)=(0.996512,0.627352,0); rgb(181pt)=(0.996788,0.630095,0); rgb(182pt)=(0.997053,0.632982,0); rgb(183pt)=(0.997308,0.636004,0); rgb(184pt)=(0.997552,0.639152,0); rgb(185pt)=(0.997785,0.642416,0); rgb(186pt)=(0.998006,0.645786,0); rgb(187pt)=(0.998217,0.649253,0); rgb(188pt)=(0.998416,0.652807,0); rgb(189pt)=(0.998605,0.656439,0); rgb(190pt)=(0.998781,0.660138,0); rgb(191pt)=(0.998946,0.663897,0); rgb(192pt)=(0.9991,0.667704,0); rgb(193pt)=(0.999242,0.67155,0); rgb(194pt)=(0.999372,0.675427,0); rgb(195pt)=(0.99949,0.679323,0); rgb(196pt)=(0.999596,0.68323,0); rgb(197pt)=(0.99969,0.687139,0); rgb(198pt)=(0.999771,0.691039,0); rgb(199pt)=(0.999841,0.694921,0); rgb(200pt)=(0.999898,0.698775,0); rgb(201pt)=(0.999942,0.702592,0); rgb(202pt)=(0.999974,0.706363,0); rgb(203pt)=(0.999994,0.710077,0); rgb(204pt)=(1,0.713725,0); rgb(205pt)=(1,0.717341,0); rgb(206pt)=(1,0.720963,0); rgb(207pt)=(1,0.724591,0); rgb(208pt)=(1,0.728226,0); rgb(209pt)=(1,0.731867,0); rgb(210pt)=(1,0.735514,0); rgb(211pt)=(1,0.739167,0); rgb(212pt)=(1,0.742827,0); rgb(213pt)=(1,0.746493,0); rgb(214pt)=(1,0.750165,0); rgb(215pt)=(1,0.753843,0); rgb(216pt)=(1,0.757527,0); rgb(217pt)=(1,0.761217,0); rgb(218pt)=(1,0.764913,0); rgb(219pt)=(1,0.768615,0); rgb(220pt)=(1,0.772324,0); rgb(221pt)=(1,0.776038,0); rgb(222pt)=(1,0.779758,0); rgb(223pt)=(1,0.783484,0); rgb(224pt)=(1,0.787215,0); rgb(225pt)=(1,0.790953,0); rgb(226pt)=(1,0.794696,0); rgb(227pt)=(1,0.798445,0); rgb(228pt)=(1,0.8022,0); rgb(229pt)=(1,0.805961,0); rgb(230pt)=(1,0.809727,0); rgb(231pt)=(1,0.8135,0); rgb(232pt)=(1,0.817278,0); rgb(233pt)=(1,0.821063,0); rgb(234pt)=(1,0.824854,0); rgb(235pt)=(1,0.828652,0); rgb(236pt)=(1,0.832455,0); rgb(237pt)=(1,0.836265,0); rgb(238pt)=(1,0.840081,0); rgb(239pt)=(1,0.843903,0); rgb(240pt)=(1,0.847732,0); rgb(241pt)=(1,0.851566,0); rgb(242pt)=(1,0.855406,0); rgb(243pt)=(1,0.859253,0); rgb(244pt)=(1,0.863106,0); rgb(245pt)=(1,0.866964,0); rgb(246pt)=(1,0.870829,0); rgb(247pt)=(1,0.8747,0); rgb(248pt)=(1,0.878577,0); rgb(249pt)=(1,0.88246,0); rgb(250pt)=(1,0.886349,0); rgb(251pt)=(1,0.890243,0); rgb(252pt)=(1,0.894144,0); rgb(253pt)=(1,0.898051,0); rgb(254pt)=(1,0.901964,0); rgb(255pt)=(1,0.905882,0)},
mesh/rows=49]
table[row sep=crcr,header=false] {%
%
57	0.193	-13.7917672765565\\
57	0.19466	-15.007456062107\\
57	0.19632	-16.2231448476574\\
57	0.19798	-17.4388336332079\\
57	0.19964	-18.6545224187583\\
57	0.2013	-19.8702112043089\\
57	0.20296	-21.0858999898593\\
57	0.20462	-22.3015887754098\\
57	0.20628	-23.5172775609602\\
57	0.20794	-24.7329663465107\\
57	0.2096	-25.9486551320611\\
57	0.21126	-27.1643439176116\\
57	0.21292	-28.380032703162\\
57	0.21458	-29.5957214887125\\
57	0.21624	-30.811410274263\\
57	0.2179	-32.0270990598135\\
57	0.21956	-33.2427878453639\\
57	0.22122	-34.4584766309143\\
57	0.22288	-35.6741654164648\\
57	0.22454	-36.8898542020153\\
57	0.2262	-38.1055429875657\\
57	0.22786	-39.3212317731162\\
57	0.22952	-40.5369205586667\\
57	0.23118	-41.7526093442171\\
57	0.23284	-42.9682981297676\\
57	0.2345	-44.183986915318\\
57	0.23616	-45.3996757008685\\
57	0.23782	-46.615364486419\\
57	0.23948	-47.8310532719694\\
57	0.24114	-49.04674205752\\
57	0.2428	-50.2624308430704\\
57	0.24446	-51.4781196286208\\
57	0.24612	-52.6938084141713\\
57	0.24778	-53.9094971997218\\
57	0.24944	-55.1251859852722\\
57	0.2511	-56.3408747708227\\
57	0.25276	-57.5565635563731\\
57	0.25442	-58.7722523419237\\
57	0.25608	-59.9879411274741\\
57	0.25774	-61.2036299130245\\
57	0.2594	-62.4193186985749\\
57	0.26106	-63.6350074841254\\
57	0.26272	-64.8506962696758\\
57	0.26438	-66.0663850552264\\
57	0.26604	-67.2820738407768\\
57	0.2677	-68.4977626263272\\
57	0.26936	-69.7134514118777\\
57	0.27102	-70.9291401974281\\
57	0.27268	-72.1448289829787\\
57	0.27434	-73.3605177685291\\
57	0.276	-74.5762065540795\\
57.2083333333333	0.193	-13.6197710384197\\
57.2083333333333	0.19466	-14.827752153088\\
57.2083333333333	0.19632	-16.0357332677561\\
57.2083333333333	0.19798	-17.2437143824244\\
57.2083333333333	0.19964	-18.4516954970925\\
57.2083333333333	0.2013	-19.6596766117608\\
57.2083333333333	0.20296	-20.8676577264289\\
57.2083333333333	0.20462	-22.0756388410972\\
57.2083333333333	0.20628	-23.2836199557653\\
57.2083333333333	0.20794	-24.4916010704335\\
57.2083333333333	0.2096	-25.6995821851017\\
57.2083333333333	0.21126	-26.9075632997699\\
57.2083333333333	0.21292	-28.1155444144381\\
57.2083333333333	0.21458	-29.3235255291063\\
57.2083333333333	0.21624	-30.5315066437745\\
57.2083333333333	0.2179	-31.7394877584427\\
57.2083333333333	0.21956	-32.9474688731109\\
57.2083333333333	0.22122	-34.155449987779\\
57.2083333333333	0.22288	-35.3634311024472\\
57.2083333333333	0.22454	-36.5714122171156\\
57.2083333333333	0.2262	-37.7793933317837\\
57.2083333333333	0.22786	-38.9873744464518\\
57.2083333333333	0.22952	-40.1953555611201\\
57.2083333333333	0.23118	-41.4033366757883\\
57.2083333333333	0.23284	-42.6113177904564\\
57.2083333333333	0.2345	-43.8192989051246\\
57.2083333333333	0.23616	-45.0272800197928\\
57.2083333333333	0.23782	-46.2352611344611\\
57.2083333333333	0.23948	-47.4432422491292\\
57.2083333333333	0.24114	-48.6512233637975\\
57.2083333333333	0.2428	-49.8592044784656\\
57.2083333333333	0.24446	-51.0671855931338\\
57.2083333333333	0.24612	-52.275166707802\\
57.2083333333333	0.24778	-53.4831478224702\\
57.2083333333333	0.24944	-54.6911289371384\\
57.2083333333333	0.2511	-55.8991100518066\\
57.2083333333333	0.25276	-57.1070911664747\\
57.2083333333333	0.25442	-58.315072281143\\
57.2083333333333	0.25608	-59.5230533958112\\
57.2083333333333	0.25774	-60.7310345104793\\
57.2083333333333	0.2594	-61.9390156251475\\
57.2083333333333	0.26106	-63.1469967398157\\
57.2083333333333	0.26272	-64.3549778544839\\
57.2083333333333	0.26438	-65.5629589691521\\
57.2083333333333	0.26604	-66.7709400838203\\
57.2083333333333	0.2677	-67.9789211984885\\
57.2083333333333	0.26936	-69.1869023131567\\
57.2083333333333	0.27102	-70.3948834278249\\
57.2083333333333	0.27268	-71.6028645424931\\
57.2083333333333	0.27434	-72.8108456571613\\
57.2083333333333	0.276	-74.0188267718294\\
57.4166666666667	0.193	-13.447774800283\\
57.4166666666667	0.19466	-14.648048244069\\
57.4166666666667	0.19632	-15.8483216878549\\
57.4166666666667	0.19798	-17.0485951316408\\
57.4166666666667	0.19964	-18.2488685754267\\
57.4166666666667	0.2013	-19.4491420192127\\
57.4166666666667	0.20296	-20.6494154629986\\
57.4166666666667	0.20462	-21.8496889067846\\
57.4166666666667	0.20628	-23.0499623505704\\
57.4166666666667	0.20794	-24.2502357943564\\
57.4166666666667	0.2096	-25.4505092381423\\
57.4166666666667	0.21126	-26.6507826819283\\
57.4166666666667	0.21292	-27.8510561257142\\
57.4166666666667	0.21458	-29.0513295695001\\
57.4166666666667	0.21624	-30.251603013286\\
57.4166666666667	0.2179	-31.451876457072\\
57.4166666666667	0.21956	-32.6521499008579\\
57.4166666666667	0.22122	-33.8524233446438\\
57.4166666666667	0.22288	-35.0526967884297\\
57.4166666666667	0.22454	-36.2529702322157\\
57.4166666666667	0.2262	-37.4532436760016\\
57.4166666666667	0.22786	-38.6535171197875\\
57.4166666666667	0.22952	-39.8537905635735\\
57.4166666666667	0.23118	-41.0540640073594\\
57.4166666666667	0.23284	-42.2543374511453\\
57.4166666666667	0.2345	-43.4546108949312\\
57.4166666666667	0.23616	-44.6548843387172\\
57.4166666666667	0.23782	-45.8551577825032\\
57.4166666666667	0.23948	-47.055431226289\\
57.4166666666667	0.24114	-48.255704670075\\
57.4166666666667	0.2428	-49.4559781138609\\
57.4166666666667	0.24446	-50.6562515576468\\
57.4166666666667	0.24612	-51.8565250014328\\
57.4166666666667	0.24778	-53.0567984452188\\
57.4166666666667	0.24944	-54.2570718890046\\
57.4166666666667	0.2511	-55.4573453327905\\
57.4166666666667	0.25276	-56.6576187765764\\
57.4166666666667	0.25442	-57.8578922203624\\
57.4166666666667	0.25608	-59.0581656641483\\
57.4166666666667	0.25774	-60.2584391079342\\
57.4166666666667	0.2594	-61.4587125517201\\
57.4166666666667	0.26106	-62.6589859955061\\
57.4166666666667	0.26272	-63.859259439292\\
57.4166666666667	0.26438	-65.059532883078\\
57.4166666666667	0.26604	-66.2598063268639\\
57.4166666666667	0.2677	-67.4600797706498\\
57.4166666666667	0.26936	-68.6603532144358\\
57.4166666666667	0.27102	-69.8606266582216\\
57.4166666666667	0.27268	-71.0609001020076\\
57.4166666666667	0.27434	-72.2611735457935\\
57.4166666666667	0.276	-73.4614469895794\\
57.625	0.193	-13.2757785621463\\
57.625	0.19466	-14.46834433505\\
57.625	0.19632	-15.6609101079536\\
57.625	0.19798	-16.8534758808573\\
57.625	0.19964	-18.0460416537609\\
57.625	0.2013	-19.2386074266647\\
57.625	0.20296	-20.4311731995683\\
57.625	0.20462	-21.623738972472\\
57.625	0.20628	-22.8163047453755\\
57.625	0.20794	-24.0088705182793\\
57.625	0.2096	-25.2014362911829\\
57.625	0.21126	-26.3940020640866\\
57.625	0.21292	-27.5865678369902\\
57.625	0.21458	-28.7791336098938\\
57.625	0.21624	-29.9716993827976\\
57.625	0.2179	-31.1642651557013\\
57.625	0.21956	-32.3568309286049\\
57.625	0.22122	-33.5493967015085\\
57.625	0.22288	-34.7419624744122\\
57.625	0.22454	-35.9345282473159\\
57.625	0.2262	-37.1270940202195\\
57.625	0.22786	-38.3196597931232\\
57.625	0.22952	-39.5122255660269\\
57.625	0.23118	-40.7047913389306\\
57.625	0.23284	-41.8973571118342\\
57.625	0.2345	-43.0899228847378\\
57.625	0.23616	-44.2824886576415\\
57.625	0.23782	-45.4750544305452\\
57.625	0.23948	-46.6676202034488\\
57.625	0.24114	-47.8601859763526\\
57.625	0.2428	-49.0527517492562\\
57.625	0.24446	-50.2453175221598\\
57.625	0.24612	-51.4378832950635\\
57.625	0.24778	-52.6304490679672\\
57.625	0.24944	-53.8230148408708\\
57.625	0.2511	-55.0155806137745\\
57.625	0.25276	-56.2081463866781\\
57.625	0.25442	-57.4007121595819\\
57.625	0.25608	-58.5932779324855\\
57.625	0.25774	-59.7858437053891\\
57.625	0.2594	-60.9784094782927\\
57.625	0.26106	-62.1709752511964\\
57.625	0.26272	-63.3635410241\\
57.625	0.26438	-64.5561067970038\\
57.625	0.26604	-65.7486725699074\\
57.625	0.2677	-66.9412383428111\\
57.625	0.26936	-68.1338041157148\\
57.625	0.27102	-69.3263698886184\\
57.625	0.27268	-70.5189356615221\\
57.625	0.27434	-71.7115014344257\\
57.625	0.276	-72.9040672073293\\
57.8333333333333	0.193	-13.1037823240095\\
57.8333333333333	0.19466	-14.288640426031\\
57.8333333333333	0.19632	-15.4734985280523\\
57.8333333333333	0.19798	-16.6583566300737\\
57.8333333333333	0.19964	-17.8432147320951\\
57.8333333333333	0.2013	-19.0280728341165\\
57.8333333333333	0.20296	-20.2129309361379\\
57.8333333333333	0.20462	-21.3977890381594\\
57.8333333333333	0.20628	-22.5826471401807\\
57.8333333333333	0.20794	-23.7675052422022\\
57.8333333333333	0.2096	-24.9523633442235\\
57.8333333333333	0.21126	-26.137221446245\\
57.8333333333333	0.21292	-27.3220795482663\\
57.8333333333333	0.21458	-28.5069376502876\\
57.8333333333333	0.21624	-29.6917957523091\\
57.8333333333333	0.2179	-30.8766538543305\\
57.8333333333333	0.21956	-32.0615119563519\\
57.8333333333333	0.22122	-33.2463700583732\\
57.8333333333333	0.22288	-34.4312281603947\\
57.8333333333333	0.22454	-35.6160862624161\\
57.8333333333333	0.2262	-36.8009443644374\\
57.8333333333333	0.22786	-37.9858024664588\\
57.8333333333333	0.22952	-39.1706605684803\\
57.8333333333333	0.23118	-40.3555186705017\\
57.8333333333333	0.23284	-41.540376772523\\
57.8333333333333	0.2345	-42.7252348745444\\
57.8333333333333	0.23616	-43.9100929765659\\
57.8333333333333	0.23782	-45.0949510785873\\
57.8333333333333	0.23948	-46.2798091806087\\
57.8333333333333	0.24114	-47.4646672826301\\
57.8333333333333	0.2428	-48.6495253846514\\
57.8333333333333	0.24446	-49.8343834866728\\
57.8333333333333	0.24612	-51.0192415886942\\
57.8333333333333	0.24778	-52.2040996907157\\
57.8333333333333	0.24944	-53.388957792737\\
57.8333333333333	0.2511	-54.5738158947584\\
57.8333333333333	0.25276	-55.7586739967797\\
57.8333333333333	0.25442	-56.9435320988013\\
57.8333333333333	0.25608	-58.1283902008226\\
57.8333333333333	0.25774	-59.3132483028439\\
57.8333333333333	0.2594	-60.4981064048653\\
57.8333333333333	0.26106	-61.6829645068868\\
57.8333333333333	0.26272	-62.8678226089081\\
57.8333333333333	0.26438	-64.0526807109296\\
57.8333333333333	0.26604	-65.237538812951\\
57.8333333333333	0.2677	-66.4223969149723\\
57.8333333333333	0.26936	-67.6072550169937\\
57.8333333333333	0.27102	-68.7921131190151\\
57.8333333333333	0.27268	-69.9769712210366\\
57.8333333333333	0.27434	-71.1618293230579\\
57.8333333333333	0.276	-72.3466874250793\\
58.0416666666667	0.193	-12.9317860858728\\
58.0416666666667	0.19466	-14.108936517012\\
58.0416666666667	0.19632	-15.2860869481511\\
58.0416666666667	0.19798	-16.4632373792903\\
58.0416666666667	0.19964	-17.6403878104293\\
58.0416666666667	0.2013	-18.8175382415685\\
58.0416666666667	0.20296	-19.9946886727076\\
58.0416666666667	0.20462	-21.1718391038468\\
58.0416666666667	0.20628	-22.3489895349858\\
58.0416666666667	0.20794	-23.526139966125\\
58.0416666666667	0.2096	-24.7032903972641\\
58.0416666666667	0.21126	-25.8804408284033\\
58.0416666666667	0.21292	-27.0575912595424\\
58.0416666666667	0.21458	-28.2347416906815\\
58.0416666666667	0.21624	-29.4118921218206\\
58.0416666666667	0.2179	-30.5890425529598\\
58.0416666666667	0.21956	-31.766192984099\\
58.0416666666667	0.22122	-32.943343415238\\
58.0416666666667	0.22288	-34.1204938463771\\
58.0416666666667	0.22454	-35.2976442775163\\
58.0416666666667	0.2262	-36.4747947086555\\
58.0416666666667	0.22786	-37.6519451397945\\
58.0416666666667	0.22952	-38.8290955709336\\
58.0416666666667	0.23118	-40.0062460020728\\
58.0416666666667	0.23284	-41.183396433212\\
58.0416666666667	0.2345	-42.360546864351\\
58.0416666666667	0.23616	-43.5376972954902\\
58.0416666666667	0.23782	-44.7148477266293\\
58.0416666666667	0.23948	-45.8919981577685\\
58.0416666666667	0.24114	-47.0691485889076\\
58.0416666666667	0.2428	-48.2462990200467\\
58.0416666666667	0.24446	-49.4234494511858\\
58.0416666666667	0.24612	-50.600599882325\\
58.0416666666667	0.24778	-51.7777503134641\\
58.0416666666667	0.24944	-52.9549007446032\\
58.0416666666667	0.2511	-54.1320511757423\\
58.0416666666667	0.25276	-55.3092016068815\\
58.0416666666667	0.25442	-56.4863520380206\\
58.0416666666667	0.25608	-57.6635024691598\\
58.0416666666667	0.25774	-58.8406529002989\\
58.0416666666667	0.2594	-60.017803331438\\
58.0416666666667	0.26106	-61.1949537625771\\
58.0416666666667	0.26272	-62.3721041937163\\
58.0416666666667	0.26438	-63.5492546248555\\
58.0416666666667	0.26604	-64.7264050559946\\
58.0416666666667	0.2677	-65.9035554871336\\
58.0416666666667	0.26936	-67.0807059182728\\
58.0416666666667	0.27102	-68.257856349412\\
58.0416666666667	0.27268	-69.4350067805511\\
58.0416666666667	0.27434	-70.6121572116902\\
58.0416666666667	0.276	-71.7893076428293\\
58.25	0.193	-12.759789847736\\
58.25	0.19466	-13.9292326079929\\
58.25	0.19632	-15.0986753682498\\
58.25	0.19798	-16.2681181285067\\
58.25	0.19964	-17.4375608887635\\
58.25	0.2013	-18.6070036490204\\
58.25	0.20296	-19.7764464092772\\
58.25	0.20462	-20.9458891695342\\
58.25	0.20628	-22.115331929791\\
58.25	0.20794	-23.2847746900479\\
58.25	0.2096	-24.4542174503047\\
58.25	0.21126	-25.6236602105616\\
58.25	0.21292	-26.7931029708184\\
58.25	0.21458	-27.9625457310753\\
58.25	0.21624	-29.1319884913322\\
58.25	0.2179	-30.3014312515891\\
58.25	0.21956	-31.4708740118459\\
58.25	0.22122	-32.6403167721027\\
58.25	0.22288	-33.8097595323596\\
58.25	0.22454	-34.9792022926165\\
58.25	0.2262	-36.1486450528733\\
58.25	0.22786	-37.3180878131302\\
58.25	0.22952	-38.4875305733871\\
58.25	0.23118	-39.656973333644\\
58.25	0.23284	-40.8264160939008\\
58.25	0.2345	-41.9958588541576\\
58.25	0.23616	-43.1653016144145\\
58.25	0.23782	-44.3347443746714\\
58.25	0.23948	-45.5041871349283\\
58.25	0.24114	-46.6736298951852\\
58.25	0.2428	-47.843072655442\\
58.25	0.24446	-49.0125154156988\\
58.25	0.24612	-50.1819581759557\\
58.25	0.24778	-51.3514009362126\\
58.25	0.24944	-52.5208436964695\\
58.25	0.2511	-53.6902864567263\\
58.25	0.25276	-54.8597292169832\\
58.25	0.25442	-56.0291719772401\\
58.25	0.25608	-57.1986147374969\\
58.25	0.25774	-58.3680574977537\\
58.25	0.2594	-59.5375002580106\\
58.25	0.26106	-60.7069430182675\\
58.25	0.26272	-61.8763857785243\\
58.25	0.26438	-63.0458285387813\\
58.25	0.26604	-64.2152712990381\\
58.25	0.2677	-65.3847140592949\\
58.25	0.26936	-66.5541568195518\\
58.25	0.27102	-67.7235995798086\\
58.25	0.27268	-68.8930423400656\\
58.25	0.27434	-70.0624851003224\\
58.25	0.276	-71.2319278605792\\
58.4583333333333	0.193	-12.5877936095993\\
58.4583333333333	0.19466	-13.7495286989739\\
58.4583333333333	0.19632	-14.9112637883485\\
58.4583333333333	0.19798	-16.0729988777231\\
58.4583333333333	0.19964	-17.2347339670977\\
58.4583333333333	0.2013	-18.3964690564723\\
58.4583333333333	0.20296	-19.5582041458469\\
58.4583333333333	0.20462	-20.7199392352215\\
58.4583333333333	0.20628	-21.8816743245961\\
58.4583333333333	0.20794	-23.0434094139707\\
58.4583333333333	0.2096	-24.2051445033453\\
58.4583333333333	0.21126	-25.3668795927199\\
58.4583333333333	0.21292	-26.5286146820945\\
58.4583333333333	0.21458	-27.6903497714691\\
58.4583333333333	0.21624	-28.8520848608437\\
58.4583333333333	0.2179	-30.0138199502183\\
58.4583333333333	0.21956	-31.1755550395929\\
58.4583333333333	0.22122	-32.3372901289675\\
58.4583333333333	0.22288	-33.4990252183421\\
58.4583333333333	0.22454	-34.6607603077167\\
58.4583333333333	0.2262	-35.8224953970913\\
58.4583333333333	0.22786	-36.9842304864658\\
58.4583333333333	0.22952	-38.1459655758405\\
58.4583333333333	0.23118	-39.3077006652151\\
58.4583333333333	0.23284	-40.4694357545897\\
58.4583333333333	0.2345	-41.6311708439642\\
58.4583333333333	0.23616	-42.7929059333388\\
58.4583333333333	0.23782	-43.9546410227135\\
58.4583333333333	0.23948	-45.1163761120881\\
58.4583333333333	0.24114	-46.2781112014627\\
58.4583333333333	0.2428	-47.4398462908372\\
58.4583333333333	0.24446	-48.6015813802118\\
58.4583333333333	0.24612	-49.7633164695865\\
58.4583333333333	0.24778	-50.9250515589611\\
58.4583333333333	0.24944	-52.0867866483356\\
58.4583333333333	0.2511	-53.2485217377102\\
58.4583333333333	0.25276	-54.4102568270848\\
58.4583333333333	0.25442	-55.5719919164595\\
58.4583333333333	0.25608	-56.7337270058341\\
58.4583333333333	0.25774	-57.8954620952086\\
58.4583333333333	0.2594	-59.0571971845832\\
58.4583333333333	0.26106	-60.2189322739579\\
58.4583333333333	0.26272	-61.3806673633324\\
58.4583333333333	0.26438	-62.5424024527071\\
58.4583333333333	0.26604	-63.7041375420816\\
58.4583333333333	0.2677	-64.8658726314562\\
58.4583333333333	0.26936	-66.0276077208309\\
58.4583333333333	0.27102	-67.1893428102054\\
58.4583333333333	0.27268	-68.3510778995801\\
58.4583333333333	0.27434	-69.5128129889546\\
58.4583333333333	0.276	-70.6745480783292\\
58.6666666666667	0.193	-12.4157973714626\\
58.6666666666667	0.19466	-13.5698247899549\\
58.6666666666667	0.19632	-14.7238522084472\\
58.6666666666667	0.19798	-15.8778796269396\\
58.6666666666667	0.19964	-17.0319070454319\\
58.6666666666667	0.2013	-18.1859344639242\\
58.6666666666667	0.20296	-19.3399618824166\\
58.6666666666667	0.20462	-20.4939893009089\\
58.6666666666667	0.20628	-21.6480167194012\\
58.6666666666667	0.20794	-22.8020441378936\\
58.6666666666667	0.2096	-23.9560715563859\\
58.6666666666667	0.21126	-25.1100989748782\\
58.6666666666667	0.21292	-26.2641263933706\\
58.6666666666667	0.21458	-27.4181538118629\\
58.6666666666667	0.21624	-28.5721812303552\\
58.6666666666667	0.2179	-29.7262086488475\\
58.6666666666667	0.21956	-30.8802360673399\\
58.6666666666667	0.22122	-32.0342634858322\\
58.6666666666667	0.22288	-33.1882909043246\\
58.6666666666667	0.22454	-34.3423183228169\\
58.6666666666667	0.2262	-35.4963457413093\\
58.6666666666667	0.22786	-36.6503731598015\\
58.6666666666667	0.22952	-37.8044005782938\\
58.6666666666667	0.23118	-38.9584279967862\\
58.6666666666667	0.23284	-40.1124554152786\\
58.6666666666667	0.2345	-41.2664828337709\\
58.6666666666667	0.23616	-42.4205102522632\\
58.6666666666667	0.23782	-43.5745376707555\\
58.6666666666667	0.23948	-44.7285650892479\\
58.6666666666667	0.24114	-45.8825925077402\\
58.6666666666667	0.2428	-47.0366199262325\\
58.6666666666667	0.24446	-48.1906473447249\\
58.6666666666667	0.24612	-49.3446747632172\\
58.6666666666667	0.24778	-50.4987021817096\\
58.6666666666667	0.24944	-51.6527296002018\\
58.6666666666667	0.2511	-52.8067570186942\\
58.6666666666667	0.25276	-53.9607844371865\\
58.6666666666667	0.25442	-55.1148118556789\\
58.6666666666667	0.25608	-56.2688392741712\\
58.6666666666667	0.25774	-57.4228666926635\\
58.6666666666667	0.2594	-58.5768941111558\\
58.6666666666667	0.26106	-59.7309215296482\\
58.6666666666667	0.26272	-60.8849489481404\\
58.6666666666667	0.26438	-62.0389763666329\\
58.6666666666667	0.26604	-63.1930037851252\\
58.6666666666667	0.2677	-64.3470312036175\\
58.6666666666667	0.26936	-65.5010586221099\\
58.6666666666667	0.27102	-66.6550860406022\\
58.6666666666667	0.27268	-67.8091134590945\\
58.6666666666667	0.27434	-68.9631408775869\\
58.6666666666667	0.276	-70.1171682960791\\
58.875	0.193	-12.2438011333258\\
58.875	0.19466	-13.3901208809359\\
58.875	0.19632	-14.536440628546\\
58.875	0.19798	-15.6827603761561\\
58.875	0.19964	-16.8290801237661\\
58.875	0.2013	-17.9753998713762\\
58.875	0.20296	-19.1217196189862\\
58.875	0.20462	-20.2680393665963\\
58.875	0.20628	-21.4143591142063\\
58.875	0.20794	-22.5606788618164\\
58.875	0.2096	-23.7069986094265\\
58.875	0.21126	-24.8533183570366\\
58.875	0.21292	-25.9996381046466\\
58.875	0.21458	-27.1459578522567\\
58.875	0.21624	-28.2922775998667\\
58.875	0.2179	-29.4385973474768\\
58.875	0.21956	-30.5849170950869\\
58.875	0.22122	-31.7312368426969\\
58.875	0.22288	-32.877556590307\\
58.875	0.22454	-34.0238763379171\\
58.875	0.2262	-35.1701960855272\\
58.875	0.22786	-36.3165158331372\\
58.875	0.22952	-37.4628355807473\\
58.875	0.23118	-38.6091553283574\\
58.875	0.23284	-39.7554750759674\\
58.875	0.2345	-40.9017948235775\\
58.875	0.23616	-42.0481145711875\\
58.875	0.23782	-43.1944343187976\\
58.875	0.23948	-44.3407540664077\\
58.875	0.24114	-45.4870738140178\\
58.875	0.2428	-46.6333935616278\\
58.875	0.24446	-47.7797133092379\\
58.875	0.24612	-48.9260330568479\\
58.875	0.24778	-50.072352804458\\
58.875	0.24944	-51.218672552068\\
58.875	0.2511	-52.3649922996781\\
58.875	0.25276	-53.5113120472882\\
58.875	0.25442	-54.6576317948983\\
58.875	0.25608	-55.8039515425083\\
58.875	0.25774	-56.9502712901184\\
58.875	0.2594	-58.0965910377284\\
58.875	0.26106	-59.2429107853386\\
58.875	0.26272	-60.3892305329485\\
58.875	0.26438	-61.5355502805587\\
58.875	0.26604	-62.6818700281688\\
58.875	0.2677	-63.8281897757788\\
58.875	0.26936	-64.9745095233889\\
58.875	0.27102	-66.1208292709989\\
58.875	0.27268	-67.267149018609\\
58.875	0.27434	-68.4134687662191\\
58.875	0.276	-69.5597885138291\\
59.0833333333333	0.193	-12.0718048951891\\
59.0833333333333	0.19466	-13.2104169719169\\
59.0833333333333	0.19632	-14.3490290486447\\
59.0833333333333	0.19798	-15.4876411253725\\
59.0833333333333	0.19964	-16.6262532021003\\
59.0833333333333	0.2013	-17.7648652788281\\
59.0833333333333	0.20296	-18.9034773555559\\
59.0833333333333	0.20462	-20.0420894322837\\
59.0833333333333	0.20628	-21.1807015090114\\
59.0833333333333	0.20794	-22.3193135857393\\
59.0833333333333	0.2096	-23.4579256624671\\
59.0833333333333	0.21126	-24.5965377391949\\
59.0833333333333	0.21292	-25.7351498159227\\
59.0833333333333	0.21458	-26.8737618926505\\
59.0833333333333	0.21624	-28.0123739693783\\
59.0833333333333	0.2179	-29.1509860461061\\
59.0833333333333	0.21956	-30.2895981228339\\
59.0833333333333	0.22122	-31.4282101995616\\
59.0833333333333	0.22288	-32.5668222762895\\
59.0833333333333	0.22454	-33.7054343530173\\
59.0833333333333	0.2262	-34.8440464297451\\
59.0833333333333	0.22786	-35.9826585064728\\
59.0833333333333	0.22952	-37.1212705832007\\
59.0833333333333	0.23118	-38.2598826599285\\
59.0833333333333	0.23284	-39.3984947366563\\
59.0833333333333	0.2345	-40.5371068133841\\
59.0833333333333	0.23616	-41.6757188901119\\
59.0833333333333	0.23782	-42.8143309668397\\
59.0833333333333	0.23948	-43.9529430435675\\
59.0833333333333	0.24114	-45.0915551202953\\
59.0833333333333	0.2428	-46.230167197023\\
59.0833333333333	0.24446	-47.3687792737509\\
59.0833333333333	0.24612	-48.5073913504787\\
59.0833333333333	0.24778	-49.6460034272065\\
59.0833333333333	0.24944	-50.7846155039343\\
59.0833333333333	0.2511	-51.9232275806621\\
59.0833333333333	0.25276	-53.0618396573898\\
59.0833333333333	0.25442	-54.2004517341177\\
59.0833333333333	0.25608	-55.3390638108455\\
59.0833333333333	0.25774	-56.4776758875732\\
59.0833333333333	0.2594	-57.616287964301\\
59.0833333333333	0.26106	-58.7549000410289\\
59.0833333333333	0.26272	-59.8935121177566\\
59.0833333333333	0.26438	-61.0321241944845\\
59.0833333333333	0.26604	-62.1707362712123\\
59.0833333333333	0.2677	-63.30934834794\\
59.0833333333333	0.26936	-64.4479604246679\\
59.0833333333333	0.27102	-65.5865725013957\\
59.0833333333333	0.27268	-66.7251845781235\\
59.0833333333333	0.27434	-67.8637966548513\\
59.0833333333333	0.276	-69.002408731579\\
59.2916666666667	0.193	-11.8998086570524\\
59.2916666666667	0.19466	-13.0307130628979\\
59.2916666666667	0.19632	-14.1616174687434\\
59.2916666666667	0.19798	-15.292521874589\\
59.2916666666667	0.19964	-16.4234262804345\\
59.2916666666667	0.2013	-17.55433068628\\
59.2916666666667	0.20296	-18.6852350921256\\
59.2916666666667	0.20462	-19.8161394979711\\
59.2916666666667	0.20628	-20.9470439038166\\
59.2916666666667	0.20794	-22.0779483096621\\
59.2916666666667	0.2096	-23.2088527155077\\
59.2916666666667	0.21126	-24.3397571213532\\
59.2916666666667	0.21292	-25.4706615271988\\
59.2916666666667	0.21458	-26.6015659330443\\
59.2916666666667	0.21624	-27.7324703388899\\
59.2916666666667	0.2179	-28.8633747447353\\
59.2916666666667	0.21956	-29.994279150581\\
59.2916666666667	0.22122	-31.1251835564264\\
59.2916666666667	0.22288	-32.256087962272\\
59.2916666666667	0.22454	-33.3869923681175\\
59.2916666666667	0.2262	-34.5178967739631\\
59.2916666666667	0.22786	-35.6488011798086\\
59.2916666666667	0.22952	-36.779705585654\\
59.2916666666667	0.23118	-37.9106099914996\\
59.2916666666667	0.23284	-39.0415143973452\\
59.2916666666667	0.2345	-40.1724188031907\\
59.2916666666667	0.23616	-41.3033232090362\\
59.2916666666667	0.23782	-42.4342276148818\\
59.2916666666667	0.23948	-43.5651320207273\\
59.2916666666667	0.24114	-44.6960364265728\\
59.2916666666667	0.2428	-45.8269408324183\\
59.2916666666667	0.24446	-46.9578452382639\\
59.2916666666667	0.24612	-48.0887496441094\\
59.2916666666667	0.24778	-49.219654049955\\
59.2916666666667	0.24944	-50.3505584558005\\
59.2916666666667	0.2511	-51.481462861646\\
59.2916666666667	0.25276	-52.6123672674915\\
59.2916666666667	0.25442	-53.7432716733371\\
59.2916666666667	0.25608	-54.8741760791827\\
59.2916666666667	0.25774	-56.0050804850282\\
59.2916666666667	0.2594	-57.1359848908737\\
59.2916666666667	0.26106	-58.2668892967193\\
59.2916666666667	0.26272	-59.3977937025647\\
59.2916666666667	0.26438	-60.5286981084104\\
59.2916666666667	0.26604	-61.6596025142559\\
59.2916666666667	0.2677	-62.7905069201014\\
59.2916666666667	0.26936	-63.921411325947\\
59.2916666666667	0.27102	-65.0523157317925\\
59.2916666666667	0.27268	-66.183220137638\\
59.2916666666667	0.27434	-67.3141245434836\\
59.2916666666667	0.276	-68.445028949329\\
59.5	0.193	-11.7278124189156\\
59.5	0.19466	-12.8510091538789\\
59.5	0.19632	-13.9742058888422\\
59.5	0.19798	-15.0974026238055\\
59.5	0.19964	-16.2205993587687\\
59.5	0.2013	-17.3437960937319\\
59.5	0.20296	-18.4669928286953\\
59.5	0.20462	-19.5901895636584\\
59.5	0.20628	-20.7133862986217\\
59.5	0.20794	-21.836583033585\\
59.5	0.2096	-22.9597797685483\\
59.5	0.21126	-24.0829765035115\\
59.5	0.21292	-25.2061732384749\\
59.5	0.21458	-26.3293699734381\\
59.5	0.21624	-27.4525667084014\\
59.5	0.2179	-28.5757634433646\\
59.5	0.21956	-29.6989601783279\\
59.5	0.22122	-30.8221569132912\\
59.5	0.22288	-31.9453536482545\\
59.5	0.22454	-33.0685503832177\\
59.5	0.2262	-34.191747118181\\
59.5	0.22786	-35.3149438531442\\
59.5	0.22952	-36.4381405881074\\
59.5	0.23118	-37.5613373230707\\
59.5	0.23284	-38.6845340580341\\
59.5	0.2345	-39.8077307929973\\
59.5	0.23616	-40.9309275279605\\
59.5	0.23782	-42.0541242629238\\
59.5	0.23948	-43.1773209978871\\
59.5	0.24114	-44.3005177328504\\
59.5	0.2428	-45.4237144678136\\
59.5	0.24446	-46.5469112027769\\
59.5	0.24612	-47.6701079377401\\
59.5	0.24778	-48.7933046727034\\
59.5	0.24944	-49.9165014076667\\
59.5	0.2511	-51.0396981426299\\
59.5	0.25276	-52.1628948775933\\
59.5	0.25442	-53.2860916125565\\
59.5	0.25608	-54.4092883475199\\
59.5	0.25774	-55.532485082483\\
59.5	0.2594	-56.6556818174463\\
59.5	0.26106	-57.7788785524096\\
59.5	0.26272	-58.9020752873728\\
59.5	0.26438	-60.0252720223362\\
59.5	0.26604	-61.1484687572994\\
59.5	0.2677	-62.2716654922626\\
59.5	0.26936	-63.394862227226\\
59.5	0.27102	-64.5180589621892\\
59.5	0.27268	-65.6412556971525\\
59.5	0.27434	-66.7644524321158\\
59.5	0.276	-67.887649167079\\
59.7083333333333	0.193	-11.5558161807788\\
59.7083333333333	0.19466	-12.6713052448598\\
59.7083333333333	0.19632	-13.7867943089408\\
59.7083333333333	0.19798	-14.9022833730219\\
59.7083333333333	0.19964	-16.0177724371028\\
59.7083333333333	0.2013	-17.1332615011838\\
59.7083333333333	0.20296	-18.2487505652649\\
59.7083333333333	0.20462	-19.3642396293458\\
59.7083333333333	0.20628	-20.4797286934269\\
59.7083333333333	0.20794	-21.5952177575078\\
59.7083333333333	0.2096	-22.7107068215889\\
59.7083333333333	0.21126	-23.8261958856698\\
59.7083333333333	0.21292	-24.9416849497509\\
59.7083333333333	0.21458	-26.0571740138319\\
59.7083333333333	0.21624	-27.1726630779129\\
59.7083333333333	0.2179	-28.2881521419938\\
59.7083333333333	0.21956	-29.4036412060749\\
59.7083333333333	0.22122	-30.5191302701558\\
59.7083333333333	0.22288	-31.6346193342368\\
59.7083333333333	0.22454	-32.7501083983178\\
59.7083333333333	0.2262	-33.8655974623989\\
59.7083333333333	0.22786	-34.9810865264798\\
59.7083333333333	0.22952	-36.0965755905608\\
59.7083333333333	0.23118	-37.2120646546418\\
59.7083333333333	0.23284	-38.3275537187229\\
59.7083333333333	0.2345	-39.4430427828038\\
59.7083333333333	0.23616	-40.5585318468848\\
59.7083333333333	0.23782	-41.6740209109659\\
59.7083333333333	0.23948	-42.7895099750469\\
59.7083333333333	0.24114	-43.9049990391279\\
59.7083333333333	0.2428	-45.0204881032088\\
59.7083333333333	0.24446	-46.1359771672899\\
59.7083333333333	0.24612	-47.2514662313708\\
59.7083333333333	0.24778	-48.3669552954519\\
59.7083333333333	0.24944	-49.4824443595328\\
59.7083333333333	0.2511	-50.5979334236138\\
59.7083333333333	0.25276	-51.7134224876949\\
59.7083333333333	0.25442	-52.8289115517758\\
59.7083333333333	0.25608	-53.9444006158569\\
59.7083333333333	0.25774	-55.0598896799379\\
59.7083333333333	0.2594	-56.1753787440188\\
59.7083333333333	0.26106	-57.2908678080999\\
59.7083333333333	0.26272	-58.4063568721808\\
59.7083333333333	0.26438	-59.5218459362619\\
59.7083333333333	0.26604	-60.6373350003429\\
59.7083333333333	0.2677	-61.7528240644239\\
59.7083333333333	0.26936	-62.868313128505\\
59.7083333333333	0.27102	-63.9838021925859\\
59.7083333333333	0.27268	-65.099291256667\\
59.7083333333333	0.27434	-66.2147803207479\\
59.7083333333333	0.276	-67.3302693848289\\
59.9166666666667	0.193	-11.3838199426421\\
59.9166666666667	0.19466	-12.4916013358408\\
59.9166666666667	0.19632	-13.5993827290396\\
59.9166666666667	0.19798	-14.7071641222384\\
59.9166666666667	0.19964	-15.814945515437\\
59.9166666666667	0.2013	-16.9227269086358\\
59.9166666666667	0.20296	-18.0305083018346\\
59.9166666666667	0.20462	-19.1382896950332\\
59.9166666666667	0.20628	-20.246071088232\\
59.9166666666667	0.20794	-21.3538524814307\\
59.9166666666667	0.2096	-22.4616338746295\\
59.9166666666667	0.21126	-23.5694152678282\\
59.9166666666667	0.21292	-24.677196661027\\
59.9166666666667	0.21458	-25.7849780542257\\
59.9166666666667	0.21624	-26.8927594474245\\
59.9166666666667	0.2179	-28.0005408406231\\
59.9166666666667	0.21956	-29.1083222338219\\
59.9166666666667	0.22122	-30.2161036270206\\
59.9166666666667	0.22288	-31.3238850202194\\
59.9166666666667	0.22454	-32.431666413418\\
59.9166666666667	0.2262	-33.5394478066169\\
59.9166666666667	0.22786	-34.6472291998156\\
59.9166666666667	0.22952	-35.7550105930142\\
59.9166666666667	0.23118	-36.862791986213\\
59.9166666666667	0.23284	-37.9705733794118\\
59.9166666666667	0.2345	-39.0783547726105\\
59.9166666666667	0.23616	-40.1861361658092\\
59.9166666666667	0.23782	-41.2939175590079\\
59.9166666666667	0.23948	-42.4016989522067\\
59.9166666666667	0.24114	-43.5094803454054\\
59.9166666666667	0.2428	-44.6172617386041\\
59.9166666666667	0.24446	-45.7250431318029\\
59.9166666666667	0.24612	-46.8328245250016\\
59.9166666666667	0.24778	-47.9406059182004\\
59.9166666666667	0.24944	-49.048387311399\\
59.9166666666667	0.2511	-50.1561687045978\\
59.9166666666667	0.25276	-51.2639500977966\\
59.9166666666667	0.25442	-52.3717314909953\\
59.9166666666667	0.25608	-53.4795128841941\\
59.9166666666667	0.25774	-54.5872942773928\\
59.9166666666667	0.2594	-55.6950756705915\\
59.9166666666667	0.26106	-56.8028570637903\\
59.9166666666667	0.26272	-57.910638456989\\
59.9166666666667	0.26438	-59.0184198501878\\
59.9166666666667	0.26604	-60.1262012433865\\
59.9166666666667	0.2677	-61.2339826365852\\
59.9166666666667	0.26936	-62.341764029784\\
59.9166666666667	0.27102	-63.4495454229827\\
59.9166666666667	0.27268	-64.5573268161815\\
59.9166666666667	0.27434	-65.6651082093802\\
59.9166666666667	0.276	-66.7728896025789\\
60.125	0.193	-11.2118237045054\\
60.125	0.19466	-12.3118974268218\\
60.125	0.19632	-13.4119711491383\\
60.125	0.19798	-14.5120448714548\\
60.125	0.19964	-15.6121185937712\\
60.125	0.2013	-16.7121923160877\\
60.125	0.20296	-17.8122660384042\\
60.125	0.20462	-18.9123397607206\\
60.125	0.20628	-20.0124134830371\\
60.125	0.20794	-21.1124872053535\\
60.125	0.2096	-22.2125609276701\\
60.125	0.21126	-23.3126346499865\\
60.125	0.21292	-24.412708372303\\
60.125	0.21458	-25.5127820946195\\
60.125	0.21624	-26.612855816936\\
60.125	0.2179	-27.7129295392524\\
60.125	0.21956	-28.8130032615689\\
60.125	0.22122	-29.9130769838853\\
60.125	0.22288	-31.0131507062018\\
60.125	0.22454	-32.1132244285183\\
60.125	0.2262	-33.2132981508348\\
60.125	0.22786	-34.3133718731512\\
60.125	0.22952	-35.4134455954676\\
60.125	0.23118	-36.5135193177841\\
60.125	0.23284	-37.6135930401007\\
60.125	0.2345	-38.7136667624171\\
60.125	0.23616	-39.8137404847335\\
60.125	0.23782	-40.91381420705\\
60.125	0.23948	-42.0138879293665\\
60.125	0.24114	-43.113961651683\\
60.125	0.2428	-44.2140353739994\\
60.125	0.24446	-45.3141090963159\\
60.125	0.24612	-46.4141828186324\\
60.125	0.24778	-47.5142565409488\\
60.125	0.24944	-48.6143302632653\\
60.125	0.2511	-49.7144039855817\\
60.125	0.25276	-50.8144777078983\\
60.125	0.25442	-51.9145514302147\\
60.125	0.25608	-53.0146251525313\\
60.125	0.25774	-54.1146988748477\\
60.125	0.2594	-55.2147725971641\\
60.125	0.26106	-56.3148463194806\\
60.125	0.26272	-57.4149200417971\\
60.125	0.26438	-58.5149937641136\\
60.125	0.26604	-59.6150674864301\\
60.125	0.2677	-60.7151412087465\\
60.125	0.26936	-61.815214931063\\
60.125	0.27102	-62.9152886533794\\
60.125	0.27268	-64.015362375696\\
60.125	0.27434	-65.1154360980124\\
60.125	0.276	-66.2155098203288\\
60.3333333333333	0.193	-11.0398274663686\\
60.3333333333333	0.19466	-12.1321935178028\\
60.3333333333333	0.19632	-13.224559569237\\
60.3333333333333	0.19798	-14.3169256206713\\
60.3333333333333	0.19964	-15.4092916721054\\
60.3333333333333	0.2013	-16.5016577235396\\
60.3333333333333	0.20296	-17.5940237749739\\
60.3333333333333	0.20462	-18.686389826408\\
60.3333333333333	0.20628	-19.7787558778422\\
60.3333333333333	0.20794	-20.8711219292764\\
60.3333333333333	0.2096	-21.9634879807107\\
60.3333333333333	0.21126	-23.0558540321449\\
60.3333333333333	0.21292	-24.1482200835791\\
60.3333333333333	0.21458	-25.2405861350132\\
60.3333333333333	0.21624	-26.3329521864475\\
60.3333333333333	0.2179	-27.4253182378816\\
60.3333333333333	0.21956	-28.5176842893159\\
60.3333333333333	0.22122	-29.6100503407501\\
60.3333333333333	0.22288	-30.7024163921843\\
60.3333333333333	0.22454	-31.7947824436184\\
60.3333333333333	0.2262	-32.8871484950527\\
60.3333333333333	0.22786	-33.9795145464869\\
60.3333333333333	0.22952	-35.0718805979211\\
60.3333333333333	0.23118	-36.1642466493553\\
60.3333333333333	0.23284	-37.2566127007896\\
60.3333333333333	0.2345	-38.3489787522237\\
60.3333333333333	0.23616	-39.4413448036578\\
60.3333333333333	0.23782	-40.5337108550921\\
60.3333333333333	0.23948	-41.6260769065264\\
60.3333333333333	0.24114	-42.7184429579605\\
60.3333333333333	0.2428	-43.8108090093947\\
60.3333333333333	0.24446	-44.9031750608289\\
60.3333333333333	0.24612	-45.9955411122631\\
60.3333333333333	0.24778	-47.0879071636973\\
60.3333333333333	0.24944	-48.1802732151315\\
60.3333333333333	0.2511	-49.2726392665656\\
60.3333333333333	0.25276	-50.3650053179999\\
60.3333333333333	0.25442	-51.4573713694342\\
60.3333333333333	0.25608	-52.5497374208684\\
60.3333333333333	0.25774	-53.6421034723026\\
60.3333333333333	0.2594	-54.7344695237367\\
60.3333333333333	0.26106	-55.826835575171\\
60.3333333333333	0.26272	-56.9192016266051\\
60.3333333333333	0.26438	-58.0115676780394\\
60.3333333333333	0.26604	-59.1039337294736\\
60.3333333333333	0.2677	-60.1962997809077\\
60.3333333333333	0.26936	-61.288665832342\\
60.3333333333333	0.27102	-62.3810318837762\\
60.3333333333333	0.27268	-63.4733979352105\\
60.3333333333333	0.27434	-64.5657639866446\\
60.3333333333333	0.276	-65.6581300380788\\
60.5416666666667	0.193	-10.8678312282318\\
60.5416666666667	0.19466	-11.9524896087838\\
60.5416666666667	0.19632	-13.0371479893357\\
60.5416666666667	0.19798	-14.1218063698877\\
60.5416666666667	0.19964	-15.2064647504396\\
60.5416666666667	0.2013	-16.2911231309915\\
60.5416666666667	0.20296	-17.3757815115435\\
60.5416666666667	0.20462	-18.4604398920954\\
60.5416666666667	0.20628	-19.5450982726473\\
60.5416666666667	0.20794	-20.6297566531993\\
60.5416666666667	0.2096	-21.7144150337512\\
60.5416666666667	0.21126	-22.7990734143032\\
60.5416666666667	0.21292	-23.8837317948551\\
60.5416666666667	0.21458	-24.968390175407\\
60.5416666666667	0.21624	-26.053048555959\\
60.5416666666667	0.2179	-27.1377069365109\\
60.5416666666667	0.21956	-28.2223653170628\\
60.5416666666667	0.22122	-29.3070236976147\\
60.5416666666667	0.22288	-30.3916820781667\\
60.5416666666667	0.22454	-31.4763404587187\\
60.5416666666667	0.2262	-32.5609988392706\\
60.5416666666667	0.22786	-33.6456572198225\\
60.5416666666667	0.22952	-34.7303156003744\\
60.5416666666667	0.23118	-35.8149739809264\\
60.5416666666667	0.23284	-36.8996323614783\\
60.5416666666667	0.2345	-37.9842907420302\\
60.5416666666667	0.23616	-39.0689491225822\\
60.5416666666667	0.23782	-40.1536075031341\\
60.5416666666667	0.23948	-41.2382658836861\\
60.5416666666667	0.24114	-42.3229242642381\\
60.5416666666667	0.2428	-43.40758264479\\
60.5416666666667	0.24446	-44.4922410253419\\
60.5416666666667	0.24612	-45.5768994058938\\
60.5416666666667	0.24778	-46.6615577864458\\
60.5416666666667	0.24944	-47.7462161669977\\
60.5416666666667	0.2511	-48.8308745475496\\
60.5416666666667	0.25276	-49.9155329281015\\
60.5416666666667	0.25442	-51.0001913086535\\
60.5416666666667	0.25608	-52.0848496892055\\
60.5416666666667	0.25774	-53.1695080697574\\
60.5416666666667	0.2594	-54.2541664503092\\
60.5416666666667	0.26106	-55.3388248308613\\
60.5416666666667	0.26272	-56.4234832114132\\
60.5416666666667	0.26438	-57.5081415919652\\
60.5416666666667	0.26604	-58.5927999725171\\
60.5416666666667	0.2677	-59.677458353069\\
60.5416666666667	0.26936	-60.762116733621\\
60.5416666666667	0.27102	-61.8467751141729\\
60.5416666666667	0.27268	-62.9314334947248\\
60.5416666666667	0.27434	-64.0160918752767\\
60.5416666666667	0.276	-65.1007502558286\\
60.75	0.193	-10.6958349900951\\
60.75	0.19466	-11.7727856997648\\
60.75	0.19632	-12.8497364094345\\
60.75	0.19798	-13.9266871191041\\
60.75	0.19964	-15.0036378287738\\
60.75	0.2013	-16.0805885384435\\
60.75	0.20296	-17.1575392481131\\
60.75	0.20462	-18.2344899577828\\
60.75	0.20628	-19.3114406674525\\
60.75	0.20794	-20.3883913771221\\
60.75	0.2096	-21.4653420867918\\
60.75	0.21126	-22.5422927964615\\
60.75	0.21292	-23.6192435061312\\
60.75	0.21458	-24.6961942158008\\
60.75	0.21624	-25.7731449254705\\
60.75	0.2179	-26.8500956351402\\
60.75	0.21956	-27.9270463448099\\
60.75	0.22122	-29.0039970544795\\
60.75	0.22288	-30.0809477641492\\
60.75	0.22454	-31.1578984738189\\
60.75	0.2262	-32.2348491834885\\
60.75	0.22786	-33.3117998931581\\
60.75	0.22952	-34.3887506028278\\
60.75	0.23118	-35.4657013124975\\
60.75	0.23284	-36.5426520221672\\
60.75	0.2345	-37.6196027318368\\
60.75	0.23616	-38.6965534415065\\
60.75	0.23782	-39.7735041511762\\
60.75	0.23948	-40.8504548608459\\
60.75	0.24114	-41.9274055705156\\
60.75	0.2428	-43.0043562801852\\
60.75	0.24446	-44.0813069898549\\
60.75	0.24612	-45.1582576995245\\
60.75	0.24778	-46.2352084091943\\
60.75	0.24944	-47.3121591188639\\
60.75	0.2511	-48.3891098285335\\
60.75	0.25276	-49.4660605382032\\
60.75	0.25442	-50.5430112478729\\
60.75	0.25608	-51.6199619575426\\
60.75	0.25774	-52.6969126672122\\
60.75	0.2594	-53.7738633768819\\
60.75	0.26106	-54.8508140865516\\
60.75	0.26272	-55.9277647962213\\
60.75	0.26438	-57.0047155058909\\
60.75	0.26604	-58.0816662155606\\
60.75	0.2677	-59.1586169252303\\
60.75	0.26936	-60.2355676349\\
60.75	0.27102	-61.3125183445696\\
60.75	0.27268	-62.3894690542393\\
60.75	0.27434	-63.466419763909\\
60.75	0.276	-64.5433704735786\\
60.9583333333333	0.193	-10.5238387519584\\
60.9583333333333	0.19466	-11.5930817907458\\
60.9583333333333	0.19632	-12.6623248295331\\
60.9583333333333	0.19798	-13.7315678683206\\
60.9583333333333	0.19964	-14.800810907108\\
60.9583333333333	0.2013	-15.8700539458954\\
60.9583333333333	0.20296	-16.9392969846828\\
60.9583333333333	0.20462	-18.0085400234702\\
60.9583333333333	0.20628	-19.0777830622576\\
60.9583333333333	0.20794	-20.147026101045\\
60.9583333333333	0.2096	-21.2162691398324\\
60.9583333333333	0.21126	-22.2855121786198\\
60.9583333333333	0.21292	-23.3547552174072\\
60.9583333333333	0.21458	-24.4239982561946\\
60.9583333333333	0.21624	-25.493241294982\\
60.9583333333333	0.2179	-26.5624843337694\\
60.9583333333333	0.21956	-27.6317273725568\\
60.9583333333333	0.22122	-28.7009704113442\\
60.9583333333333	0.22288	-29.7702134501316\\
60.9583333333333	0.22454	-30.839456488919\\
60.9583333333333	0.2262	-31.9086995277065\\
60.9583333333333	0.22786	-32.9779425664938\\
60.9583333333333	0.22952	-34.0471856052812\\
60.9583333333333	0.23118	-35.1164286440687\\
60.9583333333333	0.23284	-36.1856716828561\\
60.9583333333333	0.2345	-37.2549147216434\\
60.9583333333333	0.23616	-38.3241577604309\\
60.9583333333333	0.23782	-39.3934007992183\\
60.9583333333333	0.23948	-40.4626438380057\\
60.9583333333333	0.24114	-41.5318868767931\\
60.9583333333333	0.2428	-42.6011299155805\\
60.9583333333333	0.24446	-43.6703729543679\\
60.9583333333333	0.24612	-44.7396159931553\\
60.9583333333333	0.24778	-45.8088590319427\\
60.9583333333333	0.24944	-46.8781020707301\\
60.9583333333333	0.2511	-47.9473451095175\\
60.9583333333333	0.25276	-49.0165881483049\\
60.9583333333333	0.25442	-50.0858311870924\\
60.9583333333333	0.25608	-51.1550742258798\\
60.9583333333333	0.25774	-52.2243172646671\\
60.9583333333333	0.2594	-53.2935603034545\\
60.9583333333333	0.26106	-54.3628033422419\\
60.9583333333333	0.26272	-55.4320463810293\\
60.9583333333333	0.26438	-56.5012894198168\\
60.9583333333333	0.26604	-57.5705324586042\\
60.9583333333333	0.2677	-58.6397754973916\\
60.9583333333333	0.26936	-59.709018536179\\
60.9583333333333	0.27102	-60.7782615749663\\
60.9583333333333	0.27268	-61.8475046137538\\
60.9583333333333	0.27434	-62.9167476525412\\
60.9583333333333	0.276	-63.9859906913285\\
61.1666666666667	0.193	-10.3518425138216\\
61.1666666666667	0.19466	-11.4133778817268\\
61.1666666666667	0.19632	-12.4749132496319\\
61.1666666666667	0.19798	-13.5364486175371\\
61.1666666666667	0.19964	-14.5979839854422\\
61.1666666666667	0.2013	-15.6595193533473\\
61.1666666666667	0.20296	-16.7210547212525\\
61.1666666666667	0.20462	-17.7825900891576\\
61.1666666666667	0.20628	-18.8441254570627\\
61.1666666666667	0.20794	-19.9056608249679\\
61.1666666666667	0.2096	-20.967196192873\\
61.1666666666667	0.21126	-22.0287315607782\\
61.1666666666667	0.21292	-23.0902669286833\\
61.1666666666667	0.21458	-24.1518022965884\\
61.1666666666667	0.21624	-25.2133376644936\\
61.1666666666667	0.2179	-26.2748730323987\\
61.1666666666667	0.21956	-27.3364084003039\\
61.1666666666667	0.22122	-28.397943768209\\
61.1666666666667	0.22288	-29.4594791361142\\
61.1666666666667	0.22454	-30.5210145040193\\
61.1666666666667	0.2262	-31.5825498719245\\
61.1666666666667	0.22786	-32.6440852398295\\
61.1666666666667	0.22952	-33.7056206077347\\
61.1666666666667	0.23118	-34.7671559756399\\
61.1666666666667	0.23284	-35.828691343545\\
61.1666666666667	0.2345	-36.8902267114501\\
61.1666666666667	0.23616	-37.9517620793553\\
61.1666666666667	0.23782	-39.0132974472604\\
61.1666666666667	0.23948	-40.0748328151656\\
61.1666666666667	0.24114	-41.1363681830707\\
61.1666666666667	0.2428	-42.1979035509758\\
61.1666666666667	0.24446	-43.259438918881\\
61.1666666666667	0.24612	-44.3209742867861\\
61.1666666666667	0.24778	-45.3825096546913\\
61.1666666666667	0.24944	-46.4440450225964\\
61.1666666666667	0.2511	-47.5055803905014\\
61.1666666666667	0.25276	-48.5671157584066\\
61.1666666666667	0.25442	-49.6286511263118\\
61.1666666666667	0.25608	-50.690186494217\\
61.1666666666667	0.25774	-51.7517218621221\\
61.1666666666667	0.2594	-52.8132572300271\\
61.1666666666667	0.26106	-53.8747925979324\\
61.1666666666667	0.26272	-54.9363279658374\\
61.1666666666667	0.26438	-55.9978633337427\\
61.1666666666667	0.26604	-57.0593987016478\\
61.1666666666667	0.2677	-58.1209340695528\\
61.1666666666667	0.26936	-59.1824694374581\\
61.1666666666667	0.27102	-60.2440048053631\\
61.1666666666667	0.27268	-61.3055401732684\\
61.1666666666667	0.27434	-62.3670755411734\\
61.1666666666667	0.276	-63.4286109090785\\
61.375	0.193	-10.1798462756849\\
61.375	0.19466	-11.2336739727078\\
61.375	0.19632	-12.2875016697307\\
61.375	0.19798	-13.3413293667535\\
61.375	0.19964	-14.3951570637764\\
61.375	0.2013	-15.4489847607992\\
61.375	0.20296	-16.5028124578221\\
61.375	0.20462	-17.556640154845\\
61.375	0.20628	-18.6104678518678\\
61.375	0.20794	-19.6642955488907\\
61.375	0.2096	-20.7181232459136\\
61.375	0.21126	-21.7719509429365\\
61.375	0.21292	-22.8257786399594\\
61.375	0.21458	-23.8796063369822\\
61.375	0.21624	-24.9334340340051\\
61.375	0.2179	-25.987261731028\\
61.375	0.21956	-27.0410894280509\\
61.375	0.22122	-28.0949171250737\\
61.375	0.22288	-29.1487448220966\\
61.375	0.22454	-30.2025725191195\\
61.375	0.2262	-31.2564002161424\\
61.375	0.22786	-32.3102279131652\\
61.375	0.22952	-33.3640556101881\\
61.375	0.23118	-34.417883307211\\
61.375	0.23284	-35.4717110042338\\
61.375	0.2345	-36.5255387012567\\
61.375	0.23616	-37.5793663982795\\
61.375	0.23782	-38.6331940953025\\
61.375	0.23948	-39.6870217923254\\
61.375	0.24114	-40.7408494893482\\
61.375	0.2428	-41.7946771863711\\
61.375	0.24446	-42.848504883394\\
61.375	0.24612	-43.9023325804168\\
61.375	0.24778	-44.9561602774397\\
61.375	0.24944	-46.0099879744625\\
61.375	0.2511	-47.0638156714854\\
61.375	0.25276	-48.1176433685083\\
61.375	0.25442	-49.1714710655312\\
61.375	0.25608	-50.2252987625541\\
61.375	0.25774	-51.2791264595769\\
61.375	0.2594	-52.3329541565997\\
61.375	0.26106	-53.3867818536227\\
61.375	0.26272	-54.4406095506455\\
61.375	0.26438	-55.4944372476684\\
61.375	0.26604	-56.5482649446913\\
61.375	0.2677	-57.6020926417141\\
61.375	0.26936	-58.6559203387371\\
61.375	0.27102	-59.7097480357599\\
61.375	0.27268	-60.7635757327828\\
61.375	0.27434	-61.8174034298057\\
61.375	0.276	-62.8712311268285\\
61.5833333333333	0.193	-10.0078500375481\\
61.5833333333333	0.19466	-11.0539700636888\\
61.5833333333333	0.19632	-12.1000900898294\\
61.5833333333333	0.19798	-13.14621011597\\
61.5833333333333	0.19964	-14.1923301421106\\
61.5833333333333	0.2013	-15.2384501682512\\
61.5833333333333	0.20296	-16.2845701943918\\
61.5833333333333	0.20462	-17.3306902205324\\
61.5833333333333	0.20628	-18.376810246673\\
61.5833333333333	0.20794	-19.4229302728136\\
61.5833333333333	0.2096	-20.4690502989542\\
61.5833333333333	0.21126	-21.5151703250949\\
61.5833333333333	0.21292	-22.5612903512355\\
61.5833333333333	0.21458	-23.607410377376\\
61.5833333333333	0.21624	-24.6535304035166\\
61.5833333333333	0.2179	-25.6996504296573\\
61.5833333333333	0.21956	-26.7457704557979\\
61.5833333333333	0.22122	-27.7918904819384\\
61.5833333333333	0.22288	-28.8380105080791\\
61.5833333333333	0.22454	-29.8841305342197\\
61.5833333333333	0.2262	-30.9302505603603\\
61.5833333333333	0.22786	-31.9763705865009\\
61.5833333333333	0.22952	-33.0224906126415\\
61.5833333333333	0.23118	-34.0686106387821\\
61.5833333333333	0.23284	-35.1147306649227\\
61.5833333333333	0.2345	-36.1608506910633\\
61.5833333333333	0.23616	-37.2069707172039\\
61.5833333333333	0.23782	-38.2530907433446\\
61.5833333333333	0.23948	-39.2992107694852\\
61.5833333333333	0.24114	-40.3453307956258\\
61.5833333333333	0.2428	-41.3914508217663\\
61.5833333333333	0.24446	-42.437570847907\\
61.5833333333333	0.24612	-43.4836908740476\\
61.5833333333333	0.24778	-44.5298109001882\\
61.5833333333333	0.24944	-45.5759309263287\\
61.5833333333333	0.2511	-46.6220509524694\\
61.5833333333333	0.25276	-47.66817097861\\
61.5833333333333	0.25442	-48.7142910047506\\
61.5833333333333	0.25608	-49.7604110308912\\
61.5833333333333	0.25774	-50.8065310570318\\
61.5833333333333	0.2594	-51.8526510831724\\
61.5833333333333	0.26106	-52.898771109313\\
61.5833333333333	0.26272	-53.9448911354536\\
61.5833333333333	0.26438	-54.9910111615943\\
61.5833333333333	0.26604	-56.0371311877349\\
61.5833333333333	0.2677	-57.0832512138754\\
61.5833333333333	0.26936	-58.1293712400161\\
61.5833333333333	0.27102	-59.1754912661567\\
61.5833333333333	0.27268	-60.2216112922973\\
61.5833333333333	0.27434	-61.2677313184379\\
61.5833333333333	0.276	-62.3138513445784\\
61.7916666666667	0.193	-9.83585379941132\\
61.7916666666667	0.19466	-10.8742661546697\\
61.7916666666667	0.19632	-11.912678509928\\
61.7916666666667	0.19798	-12.9510908651864\\
61.7916666666667	0.19964	-13.9895032204447\\
61.7916666666667	0.2013	-15.0279155757031\\
61.7916666666667	0.20296	-16.0663279309614\\
61.7916666666667	0.20462	-17.1047402862198\\
61.7916666666667	0.20628	-18.1431526414781\\
61.7916666666667	0.20794	-19.1815649967365\\
61.7916666666667	0.2096	-20.2199773519948\\
61.7916666666667	0.21126	-21.2583897072532\\
61.7916666666667	0.21292	-22.2968020625115\\
61.7916666666667	0.21458	-23.3352144177697\\
61.7916666666667	0.21624	-24.3736267730281\\
61.7916666666667	0.2179	-25.4120391282866\\
61.7916666666667	0.21956	-26.4504514835448\\
61.7916666666667	0.22122	-27.4888638388031\\
61.7916666666667	0.22288	-28.5272761940615\\
61.7916666666667	0.22454	-29.5656885493199\\
61.7916666666667	0.2262	-30.6041009045782\\
61.7916666666667	0.22786	-31.6425132598364\\
61.7916666666667	0.22952	-32.6809256150948\\
61.7916666666667	0.23118	-33.7193379703533\\
61.7916666666667	0.23284	-34.7577503256115\\
61.7916666666667	0.2345	-35.7961626808698\\
61.7916666666667	0.23616	-36.8345750361282\\
61.7916666666667	0.23782	-37.8729873913866\\
61.7916666666667	0.23948	-38.9113997466449\\
61.7916666666667	0.24114	-39.9498121019033\\
61.7916666666667	0.2428	-40.9882244571616\\
61.7916666666667	0.24446	-42.0266368124199\\
61.7916666666667	0.24612	-43.0650491676783\\
61.7916666666667	0.24778	-44.1034615229366\\
61.7916666666667	0.24944	-45.141873878195\\
61.7916666666667	0.2511	-46.1802862334533\\
61.7916666666667	0.25276	-47.2186985887116\\
61.7916666666667	0.25442	-48.25711094397\\
61.7916666666667	0.25608	-49.2955232992283\\
61.7916666666667	0.25774	-50.3339356544866\\
61.7916666666667	0.2594	-51.3723480097449\\
61.7916666666667	0.26106	-52.4107603650033\\
61.7916666666667	0.26272	-53.4491727202616\\
61.7916666666667	0.26438	-54.48758507552\\
61.7916666666667	0.26604	-55.5259974307783\\
61.7916666666667	0.2677	-56.5644097860367\\
61.7916666666667	0.26936	-57.6028221412951\\
61.7916666666667	0.27102	-58.6412344965534\\
61.7916666666667	0.27268	-59.6796468518118\\
61.7916666666667	0.27434	-60.71805920707\\
61.7916666666667	0.276	-61.7564715623283\\
62	0.193	-9.6638575612746\\
62	0.19466	-10.6945622456507\\
62	0.19632	-11.7252669300267\\
62	0.19798	-12.7559716144029\\
62	0.19964	-13.7866762987789\\
62	0.2013	-14.817380983155\\
62	0.20296	-15.8480856675311\\
62	0.20462	-16.8787903519072\\
62	0.20628	-17.9094950362832\\
62	0.20794	-18.9401997206593\\
62	0.2096	-19.9709044050354\\
62	0.21126	-21.0016090894115\\
62	0.21292	-22.0323137737875\\
62	0.21458	-23.0630184581635\\
62	0.21624	-24.0937231425397\\
62	0.2179	-25.1244278269158\\
62	0.21956	-26.1551325112918\\
62	0.22122	-27.1858371956678\\
62	0.22288	-28.216541880044\\
62	0.22454	-29.2472465644201\\
62	0.2262	-30.2779512487961\\
62	0.22786	-31.3086559331721\\
62	0.22952	-32.3393606175483\\
62	0.23118	-33.3700653019244\\
62	0.23284	-34.4007699863004\\
62	0.2345	-35.4314746706764\\
62	0.23616	-36.4621793550526\\
62	0.23782	-37.4928840394287\\
62	0.23948	-38.5235887238047\\
62	0.24114	-39.5542934081809\\
62	0.2428	-40.5849980925569\\
62	0.24446	-41.6157027769329\\
62	0.24612	-42.646407461309\\
62	0.24778	-43.6771121456852\\
62	0.24944	-44.7078168300612\\
62	0.2511	-45.7385215144373\\
62	0.25276	-46.7692261988133\\
62	0.25442	-47.7999308831895\\
62	0.25608	-48.8306355675655\\
62	0.25774	-49.8613402519415\\
62	0.2594	-50.8920449363175\\
62	0.26106	-51.9227496206937\\
62	0.26272	-52.9534543050697\\
62	0.26438	-53.9841589894458\\
62	0.26604	-55.0148636738219\\
62	0.2677	-56.0455683581979\\
62	0.26936	-57.076273042574\\
62	0.27102	-58.1069777269501\\
62	0.27268	-59.1376824113262\\
62	0.27434	-60.1683870957023\\
62	0.276	-61.1990917800783\\
62.2083333333333	0.193	-9.49186132313781\\
62.2083333333333	0.19466	-10.5148583366317\\
62.2083333333333	0.19632	-11.5378553501255\\
62.2083333333333	0.19798	-12.5608523636193\\
62.2083333333333	0.19964	-13.5838493771131\\
62.2083333333333	0.2013	-14.606846390607\\
62.2083333333333	0.20296	-15.6298434041007\\
62.2083333333333	0.20462	-16.6528404175946\\
62.2083333333333	0.20628	-17.6758374310883\\
62.2083333333333	0.20794	-18.6988344445822\\
62.2083333333333	0.2096	-19.7218314580759\\
62.2083333333333	0.21126	-20.7448284715698\\
62.2083333333333	0.21292	-21.7678254850636\\
62.2083333333333	0.21458	-22.7908224985574\\
62.2083333333333	0.21624	-23.8138195120512\\
62.2083333333333	0.2179	-24.836816525545\\
62.2083333333333	0.21956	-25.8598135390388\\
62.2083333333333	0.22122	-26.8828105525326\\
62.2083333333333	0.22288	-27.9058075660264\\
62.2083333333333	0.22454	-28.9288045795203\\
62.2083333333333	0.2262	-29.951801593014\\
62.2083333333333	0.22786	-30.9747986065078\\
62.2083333333333	0.22952	-31.9977956200017\\
62.2083333333333	0.23118	-33.0207926334955\\
62.2083333333333	0.23284	-34.0437896469893\\
62.2083333333333	0.2345	-35.066786660483\\
62.2083333333333	0.23616	-36.0897836739769\\
62.2083333333333	0.23782	-37.1127806874707\\
62.2083333333333	0.23948	-38.1357777009645\\
62.2083333333333	0.24114	-39.1587747144584\\
62.2083333333333	0.2428	-40.1817717279521\\
62.2083333333333	0.24446	-41.2047687414459\\
62.2083333333333	0.24612	-42.2277657549398\\
62.2083333333333	0.24778	-43.2507627684336\\
62.2083333333333	0.24944	-44.2737597819274\\
62.2083333333333	0.2511	-45.2967567954212\\
62.2083333333333	0.25276	-46.3197538089149\\
62.2083333333333	0.25442	-47.3427508224088\\
62.2083333333333	0.25608	-48.3657478359026\\
62.2083333333333	0.25774	-49.3887448493963\\
62.2083333333333	0.2594	-50.4117418628901\\
62.2083333333333	0.26106	-51.434738876384\\
62.2083333333333	0.26272	-52.4577358898778\\
62.2083333333333	0.26438	-53.4807329033716\\
62.2083333333333	0.26604	-54.5037299168654\\
62.2083333333333	0.2677	-55.5267269303592\\
62.2083333333333	0.26936	-56.5497239438531\\
62.2083333333333	0.27102	-57.5727209573468\\
62.2083333333333	0.27268	-58.5957179708407\\
62.2083333333333	0.27434	-59.6187149843345\\
62.2083333333333	0.276	-60.6417119978282\\
62.4166666666667	0.193	-9.31986508500114\\
62.4166666666667	0.19466	-10.3351544276127\\
62.4166666666667	0.19632	-11.3504437702242\\
62.4166666666667	0.19798	-12.3657331128358\\
62.4166666666667	0.19964	-13.3810224554473\\
62.4166666666667	0.2013	-14.3963117980588\\
62.4166666666667	0.20296	-15.4116011406704\\
62.4166666666667	0.20462	-16.426890483282\\
62.4166666666667	0.20628	-17.4421798258934\\
62.4166666666667	0.20794	-18.457469168505\\
62.4166666666667	0.2096	-19.4727585111166\\
62.4166666666667	0.21126	-20.4880478537281\\
62.4166666666667	0.21292	-21.5033371963397\\
62.4166666666667	0.21458	-22.5186265389512\\
62.4166666666667	0.21624	-23.5339158815627\\
62.4166666666667	0.2179	-24.5492052241743\\
62.4166666666667	0.21956	-25.5644945667859\\
62.4166666666667	0.22122	-26.5797839093973\\
62.4166666666667	0.22288	-27.5950732520089\\
62.4166666666667	0.22454	-28.6103625946205\\
62.4166666666667	0.2262	-29.625651937232\\
62.4166666666667	0.22786	-30.6409412798435\\
62.4166666666667	0.22952	-31.656230622455\\
62.4166666666667	0.23118	-32.6715199650666\\
62.4166666666667	0.23284	-33.6868093076782\\
62.4166666666667	0.2345	-34.7020986502897\\
62.4166666666667	0.23616	-35.7173879929013\\
62.4166666666667	0.23782	-36.7326773355128\\
62.4166666666667	0.23948	-37.7479666781244\\
62.4166666666667	0.24114	-38.7632560207359\\
62.4166666666667	0.2428	-39.7785453633474\\
62.4166666666667	0.24446	-40.7938347059589\\
62.4166666666667	0.24612	-41.8091240485705\\
62.4166666666667	0.24778	-42.8244133911821\\
62.4166666666667	0.24944	-43.8397027337936\\
62.4166666666667	0.2511	-44.8549920764051\\
62.4166666666667	0.25276	-45.8702814190166\\
62.4166666666667	0.25442	-46.8855707616282\\
62.4166666666667	0.25608	-47.9008601042398\\
62.4166666666667	0.25774	-48.9161494468513\\
62.4166666666667	0.2594	-49.9314387894628\\
62.4166666666667	0.26106	-50.9467281320744\\
62.4166666666667	0.26272	-51.9620174746859\\
62.4166666666667	0.26438	-52.9773068172975\\
62.4166666666667	0.26604	-53.9925961599091\\
62.4166666666667	0.2677	-55.0078855025205\\
62.4166666666667	0.26936	-56.0231748451321\\
62.4166666666667	0.27102	-57.0384641877436\\
62.4166666666667	0.27268	-58.0537535303553\\
62.4166666666667	0.27434	-59.0690428729667\\
62.4166666666667	0.276	-60.0843322155782\\
62.625	0.193	-9.14786884686441\\
62.625	0.19466	-10.1554505185937\\
62.625	0.19632	-11.163032190323\\
62.625	0.19798	-12.1706138620523\\
62.625	0.19964	-13.1781955337815\\
62.625	0.2013	-14.1857772055108\\
62.625	0.20296	-15.1933588772401\\
62.625	0.20462	-16.2009405489694\\
62.625	0.20628	-17.2085222206986\\
62.625	0.20794	-18.216103892428\\
62.625	0.2096	-19.2236855641572\\
62.625	0.21126	-20.2312672358865\\
62.625	0.21292	-21.2388489076158\\
62.625	0.21458	-22.246430579345\\
62.625	0.21624	-23.2540122510743\\
62.625	0.2179	-24.2615939228036\\
62.625	0.21956	-25.2691755945328\\
62.625	0.22122	-26.2767572662621\\
62.625	0.22288	-27.2843389379914\\
62.625	0.22454	-28.2919206097207\\
62.625	0.2262	-29.29950228145\\
62.625	0.22786	-30.3070839531792\\
62.625	0.22952	-31.3146656249085\\
62.625	0.23118	-32.3222472966378\\
62.625	0.23284	-33.3298289683671\\
62.625	0.2345	-34.3374106400963\\
62.625	0.23616	-35.3449923118256\\
62.625	0.23782	-36.3525739835549\\
62.625	0.23948	-37.3601556552841\\
62.625	0.24114	-38.3677373270135\\
62.625	0.2428	-39.3753189987427\\
62.625	0.24446	-40.3829006704719\\
62.625	0.24612	-41.3904823422013\\
62.625	0.24778	-42.3980640139306\\
62.625	0.24944	-43.4056456856599\\
62.625	0.2511	-44.4132273573891\\
62.625	0.25276	-45.4208090291183\\
62.625	0.25442	-46.4283907008477\\
62.625	0.25608	-47.4359723725769\\
62.625	0.25774	-48.4435540443062\\
62.625	0.2594	-49.4511357160354\\
62.625	0.26106	-50.4587173877648\\
62.625	0.26272	-51.466299059494\\
62.625	0.26438	-52.4738807312233\\
62.625	0.26604	-53.4814624029526\\
62.625	0.2677	-54.4890440746818\\
62.625	0.26936	-55.4966257464112\\
62.625	0.27102	-56.5042074181404\\
62.625	0.27268	-57.5117890898697\\
62.625	0.27434	-58.519370761599\\
62.625	0.276	-59.5269524333282\\
62.8333333333333	0.193	-8.97587260872763\\
62.8333333333333	0.19466	-9.97574660957474\\
62.8333333333333	0.19632	-10.9756206104217\\
62.8333333333333	0.19798	-11.9754946112687\\
62.8333333333333	0.19964	-12.9753686121157\\
62.8333333333333	0.2013	-13.9752426129627\\
62.8333333333333	0.20296	-14.9751166138097\\
62.8333333333333	0.20462	-15.9749906146568\\
62.8333333333333	0.20628	-16.9748646155037\\
62.8333333333333	0.20794	-17.9747386163508\\
62.8333333333333	0.2096	-18.9746126171977\\
62.8333333333333	0.21126	-19.9744866180449\\
62.8333333333333	0.21292	-20.9743606188918\\
62.8333333333333	0.21458	-21.9742346197388\\
62.8333333333333	0.21624	-22.9741086205858\\
62.8333333333333	0.2179	-23.9739826214329\\
62.8333333333333	0.21956	-24.9738566222799\\
62.8333333333333	0.22122	-25.9737306231268\\
62.8333333333333	0.22288	-26.9736046239738\\
62.8333333333333	0.22454	-27.9734786248209\\
62.8333333333333	0.2262	-28.9733526256679\\
62.8333333333333	0.22786	-29.9732266265148\\
62.8333333333333	0.22952	-30.9731006273619\\
62.8333333333333	0.23118	-31.9729746282089\\
62.8333333333333	0.23284	-32.9728486290559\\
62.8333333333333	0.2345	-33.9727226299029\\
62.8333333333333	0.23616	-34.9725966307499\\
62.8333333333333	0.23782	-35.972470631597\\
62.8333333333333	0.23948	-36.9723446324439\\
62.8333333333333	0.24114	-37.972218633291\\
62.8333333333333	0.2428	-38.972092634138\\
62.8333333333333	0.24446	-39.9719666349849\\
62.8333333333333	0.24612	-40.9718406358321\\
62.8333333333333	0.24778	-41.9717146366791\\
62.8333333333333	0.24944	-42.9715886375261\\
62.8333333333333	0.2511	-43.9714626383731\\
62.8333333333333	0.25276	-44.97133663922\\
62.8333333333333	0.25442	-45.9712106400671\\
62.8333333333333	0.25608	-46.9710846409141\\
62.8333333333333	0.25774	-47.9709586417611\\
62.8333333333333	0.2594	-48.970832642608\\
62.8333333333333	0.26106	-49.9707066434551\\
62.8333333333333	0.26272	-50.9705806443021\\
62.8333333333333	0.26438	-51.9704546451491\\
62.8333333333333	0.26604	-52.9703286459961\\
62.8333333333333	0.2677	-53.9702026468431\\
62.8333333333333	0.26936	-54.9700766476902\\
62.8333333333333	0.27102	-55.9699506485371\\
62.8333333333333	0.27268	-56.9698246493842\\
62.8333333333333	0.27434	-57.9696986502312\\
62.8333333333333	0.276	-58.9695726510781\\
63.0416666666667	0.193	-8.8038763705909\\
63.0416666666667	0.19466	-9.79604270055563\\
63.0416666666667	0.19632	-10.7882090305204\\
63.0416666666667	0.19798	-11.7803753604852\\
63.0416666666667	0.19964	-12.7725416904499\\
63.0416666666667	0.2013	-13.7647080204146\\
63.0416666666667	0.20296	-14.7568743503793\\
63.0416666666667	0.20462	-15.7490406803441\\
63.0416666666667	0.20628	-16.7412070103089\\
63.0416666666667	0.20794	-17.7333733402736\\
63.0416666666667	0.2096	-18.7255396702383\\
63.0416666666667	0.21126	-19.7177060002031\\
63.0416666666667	0.21292	-20.7098723301679\\
63.0416666666667	0.21458	-21.7020386601325\\
63.0416666666667	0.21624	-22.6942049900973\\
63.0416666666667	0.2179	-23.6863713200621\\
63.0416666666667	0.21956	-24.6785376500268\\
63.0416666666667	0.22122	-25.6707039799915\\
63.0416666666667	0.22288	-26.6628703099563\\
63.0416666666667	0.22454	-27.655036639921\\
63.0416666666667	0.2262	-28.6472029698858\\
63.0416666666667	0.22786	-29.6393692998505\\
63.0416666666667	0.22952	-30.6315356298152\\
63.0416666666667	0.23118	-31.62370195978\\
63.0416666666667	0.23284	-32.6158682897448\\
63.0416666666667	0.2345	-33.6080346197095\\
63.0416666666667	0.23616	-34.6002009496742\\
63.0416666666667	0.23782	-35.592367279639\\
63.0416666666667	0.23948	-36.5845336096037\\
63.0416666666667	0.24114	-37.5766999395685\\
63.0416666666667	0.2428	-38.5688662695332\\
63.0416666666667	0.24446	-39.561032599498\\
63.0416666666667	0.24612	-40.5531989294627\\
63.0416666666667	0.24778	-41.5453652594275\\
63.0416666666667	0.24944	-42.5375315893922\\
63.0416666666667	0.2511	-43.5296979193569\\
63.0416666666667	0.25276	-44.5218642493217\\
63.0416666666667	0.25442	-45.5140305792864\\
63.0416666666667	0.25608	-46.5061969092512\\
63.0416666666667	0.25774	-47.4983632392159\\
63.0416666666667	0.2594	-48.4905295691806\\
63.0416666666667	0.26106	-49.4826958991454\\
63.0416666666667	0.26272	-50.4748622291101\\
63.0416666666667	0.26438	-51.4670285590749\\
63.0416666666667	0.26604	-52.4591948890397\\
63.0416666666667	0.2677	-53.4513612190044\\
63.0416666666667	0.26936	-54.4435275489691\\
63.0416666666667	0.27102	-55.4356938789339\\
63.0416666666667	0.27268	-56.4278602088987\\
63.0416666666667	0.27434	-57.4200265388634\\
63.0416666666667	0.276	-58.4121928688281\\
63.25	0.193	-8.63188013245417\\
63.25	0.19466	-9.61633879153663\\
63.25	0.19632	-10.6007974506191\\
63.25	0.19798	-11.5852561097016\\
63.25	0.19964	-12.5697147687841\\
63.25	0.2013	-13.5541734278665\\
63.25	0.20296	-14.538632086949\\
63.25	0.20462	-15.5230907460315\\
63.25	0.20628	-16.507549405114\\
63.25	0.20794	-17.4920080641965\\
63.25	0.2096	-18.4764667232789\\
63.25	0.21126	-19.4609253823614\\
63.25	0.21292	-20.4453840414439\\
63.25	0.21458	-21.4298427005264\\
63.25	0.21624	-22.4143013596088\\
63.25	0.2179	-23.3987600186913\\
63.25	0.21956	-24.3832186777738\\
63.25	0.22122	-25.3676773368563\\
63.25	0.22288	-26.3521359959387\\
63.25	0.22454	-27.3365946550213\\
63.25	0.2262	-28.3210533141037\\
63.25	0.22786	-29.3055119731862\\
63.25	0.22952	-30.2899706322686\\
63.25	0.23118	-31.2744292913512\\
63.25	0.23284	-32.2588879504336\\
63.25	0.2345	-33.2433466095161\\
63.25	0.23616	-34.2278052685986\\
63.25	0.23782	-35.2122639276811\\
63.25	0.23948	-36.1967225867635\\
63.25	0.24114	-37.1811812458461\\
63.25	0.2428	-38.1656399049285\\
63.25	0.24446	-39.150098564011\\
63.25	0.24612	-40.1345572230935\\
63.25	0.24778	-41.119015882176\\
63.25	0.24944	-42.1034745412584\\
63.25	0.2511	-43.0879332003408\\
63.25	0.25276	-44.0723918594234\\
63.25	0.25442	-45.0568505185059\\
63.25	0.25608	-46.0413091775883\\
63.25	0.25774	-47.0257678366708\\
63.25	0.2594	-48.0102264957532\\
63.25	0.26106	-48.9946851548358\\
63.25	0.26272	-49.9791438139182\\
63.25	0.26438	-50.9636024730007\\
63.25	0.26604	-51.9480611320832\\
63.25	0.2677	-52.9325197911656\\
63.25	0.26936	-53.9169784502482\\
63.25	0.27102	-54.9014371093306\\
63.25	0.27268	-55.8858957684131\\
63.25	0.27434	-56.8703544274956\\
63.25	0.276	-57.854813086578\\
63.4583333333333	0.193	-8.45988389431739\\
63.4583333333333	0.19466	-9.43663488251764\\
63.4583333333333	0.19632	-10.4133858707178\\
63.4583333333333	0.19798	-11.3901368589181\\
63.4583333333333	0.19964	-12.3668878471182\\
63.4583333333333	0.2013	-13.3436388353185\\
63.4583333333333	0.20296	-14.3203898235187\\
63.4583333333333	0.20462	-15.2971408117189\\
63.4583333333333	0.20628	-16.2738917999191\\
63.4583333333333	0.20794	-17.2506427881193\\
63.4583333333333	0.2096	-18.2273937763196\\
63.4583333333333	0.21126	-19.2041447645198\\
63.4583333333333	0.21292	-20.18089575272\\
63.4583333333333	0.21458	-21.1576467409201\\
63.4583333333333	0.21624	-22.1343977291204\\
63.4583333333333	0.2179	-23.1111487173206\\
63.4583333333333	0.21956	-24.0878997055208\\
63.4583333333333	0.22122	-25.064650693721\\
63.4583333333333	0.22288	-26.0414016819212\\
63.4583333333333	0.22454	-27.0181526701214\\
63.4583333333333	0.2262	-27.9949036583217\\
63.4583333333333	0.22786	-28.9716546465218\\
63.4583333333333	0.22952	-29.9484056347221\\
63.4583333333333	0.23118	-30.9251566229223\\
63.4583333333333	0.23284	-31.9019076111225\\
63.4583333333333	0.2345	-32.8786585993227\\
63.4583333333333	0.23616	-33.8554095875229\\
63.4583333333333	0.23782	-34.8321605757232\\
63.4583333333333	0.23948	-35.8089115639233\\
63.4583333333333	0.24114	-36.7856625521236\\
63.4583333333333	0.2428	-37.7624135403237\\
63.4583333333333	0.24446	-38.739164528524\\
63.4583333333333	0.24612	-39.7159155167242\\
63.4583333333333	0.24778	-40.6926665049244\\
63.4583333333333	0.24944	-41.6694174931246\\
63.4583333333333	0.2511	-42.6461684813248\\
63.4583333333333	0.25276	-43.622919469525\\
63.4583333333333	0.25442	-44.5996704577253\\
63.4583333333333	0.25608	-45.5764214459255\\
63.4583333333333	0.25774	-46.5531724341257\\
63.4583333333333	0.2594	-47.5299234223259\\
63.4583333333333	0.26106	-48.5066744105261\\
63.4583333333333	0.26272	-49.4834253987263\\
63.4583333333333	0.26438	-50.4601763869265\\
63.4583333333333	0.26604	-51.4369273751267\\
63.4583333333333	0.2677	-52.4136783633269\\
63.4583333333333	0.26936	-53.3904293515272\\
63.4583333333333	0.27102	-54.3671803397274\\
63.4583333333333	0.27268	-55.3439313279276\\
63.4583333333333	0.27434	-56.3206823161278\\
63.4583333333333	0.276	-57.297433304328\\
63.6666666666667	0.193	-8.28788765618071\\
63.6666666666667	0.19466	-9.25693097349858\\
63.6666666666667	0.19632	-10.2259742908166\\
63.6666666666667	0.19798	-11.1950176081346\\
63.6666666666667	0.19964	-12.1640609254525\\
63.6666666666667	0.2013	-13.1331042427704\\
63.6666666666667	0.20296	-14.1021475600884\\
63.6666666666667	0.20462	-15.0711908774063\\
63.6666666666667	0.20628	-16.0402341947242\\
63.6666666666667	0.20794	-17.0092775120422\\
63.6666666666667	0.2096	-17.9783208293602\\
63.6666666666667	0.21126	-18.9473641466781\\
63.6666666666667	0.21292	-19.9164074639961\\
63.6666666666667	0.21458	-20.885450781314\\
63.6666666666667	0.21624	-21.854494098632\\
63.6666666666667	0.2179	-22.8235374159498\\
63.6666666666667	0.21956	-23.7925807332679\\
63.6666666666667	0.22122	-24.7616240505858\\
63.6666666666667	0.22288	-25.7306673679037\\
63.6666666666667	0.22454	-26.6997106852216\\
63.6666666666667	0.2262	-27.6687540025396\\
63.6666666666667	0.22786	-28.6377973198576\\
63.6666666666667	0.22952	-29.6068406371754\\
63.6666666666667	0.23118	-30.5758839544934\\
63.6666666666667	0.23284	-31.5449272718114\\
63.6666666666667	0.2345	-32.5139705891293\\
63.6666666666667	0.23616	-33.4830139064472\\
63.6666666666667	0.23782	-34.4520572237652\\
63.6666666666667	0.23948	-35.4211005410832\\
63.6666666666667	0.24114	-36.3901438584011\\
63.6666666666667	0.2428	-37.359187175719\\
63.6666666666667	0.24446	-38.328230493037\\
63.6666666666667	0.24612	-39.2972738103549\\
63.6666666666667	0.24778	-40.2663171276729\\
63.6666666666667	0.24944	-41.2353604449908\\
63.6666666666667	0.2511	-42.2044037623087\\
63.6666666666667	0.25276	-43.1734470796267\\
63.6666666666667	0.25442	-44.1424903969447\\
63.6666666666667	0.25608	-45.1115337142627\\
63.6666666666667	0.25774	-46.0805770315806\\
63.6666666666667	0.2594	-47.0496203488985\\
63.6666666666667	0.26106	-48.0186636662165\\
63.6666666666667	0.26272	-48.9877069835344\\
63.6666666666667	0.26438	-49.9567503008524\\
63.6666666666667	0.26604	-50.9257936181704\\
63.6666666666667	0.2677	-51.8948369354882\\
63.6666666666667	0.26936	-52.8638802528063\\
63.6666666666667	0.27102	-53.8329235701241\\
63.6666666666667	0.27268	-54.8019668874422\\
63.6666666666667	0.27434	-55.77101020476\\
63.6666666666667	0.276	-56.740053522078\\
63.875	0.193	-8.11589141804393\\
63.875	0.19466	-9.07722706447964\\
63.875	0.19632	-10.0385627109153\\
63.875	0.19798	-10.999898357351\\
63.875	0.19964	-11.9612340037867\\
63.875	0.2013	-12.9225696502224\\
63.875	0.20296	-13.883905296658\\
63.875	0.20462	-14.8452409430938\\
63.875	0.20628	-15.8065765895294\\
63.875	0.20794	-16.7679122359651\\
63.875	0.2096	-17.7292478824008\\
63.875	0.21126	-18.6905835288364\\
63.875	0.21292	-19.6519191752722\\
63.875	0.21458	-20.6132548217078\\
63.875	0.21624	-21.5745904681435\\
63.875	0.2179	-22.5359261145792\\
63.875	0.21956	-23.4972617610148\\
63.875	0.22122	-24.4585974074505\\
63.875	0.22288	-25.4199330538862\\
63.875	0.22454	-26.3812687003219\\
63.875	0.2262	-27.3426043467576\\
63.875	0.22786	-28.3039399931932\\
63.875	0.22952	-29.2652756396289\\
63.875	0.23118	-30.2266112860646\\
63.875	0.23284	-31.1879469325003\\
63.875	0.2345	-32.1492825789359\\
63.875	0.23616	-33.1106182253716\\
63.875	0.23782	-34.0719538718074\\
63.875	0.23948	-35.033289518243\\
63.875	0.24114	-35.9946251646787\\
63.875	0.2428	-36.9559608111143\\
63.875	0.24446	-37.91729645755\\
63.875	0.24612	-38.8786321039857\\
63.875	0.24778	-39.8399677504214\\
63.875	0.24944	-40.8013033968571\\
63.875	0.2511	-41.7626390432927\\
63.875	0.25276	-42.7239746897284\\
63.875	0.25442	-43.6853103361642\\
63.875	0.25608	-44.6466459825999\\
63.875	0.25774	-45.6079816290355\\
63.875	0.2594	-46.5693172754711\\
63.875	0.26106	-47.5306529219068\\
63.875	0.26272	-48.4919885683425\\
63.875	0.26438	-49.4533242147782\\
63.875	0.26604	-50.4146598612139\\
63.875	0.2677	-51.3759955076495\\
63.875	0.26936	-52.3373311540853\\
63.875	0.27102	-53.2986668005209\\
63.875	0.27268	-54.2600024469567\\
63.875	0.27434	-55.2213380933923\\
63.875	0.276	-56.1826737398279\\
64.0833333333333	0.193	-7.9438951799072\\
64.0833333333333	0.19466	-8.89752315546059\\
64.0833333333333	0.19632	-9.85115113101403\\
64.0833333333333	0.19798	-10.8047791065675\\
64.0833333333333	0.19964	-11.7584070821209\\
64.0833333333333	0.2013	-12.7120350576742\\
64.0833333333333	0.20296	-13.6656630332277\\
64.0833333333333	0.20462	-14.6192910087811\\
64.0833333333333	0.20628	-15.5729189843345\\
64.0833333333333	0.20794	-16.5265469598879\\
64.0833333333333	0.2096	-17.4801749354414\\
64.0833333333333	0.21126	-18.4338029109947\\
64.0833333333333	0.21292	-19.3874308865482\\
64.0833333333333	0.21458	-20.3410588621016\\
64.0833333333333	0.21624	-21.294686837655\\
64.0833333333333	0.2179	-22.2483148132084\\
64.0833333333333	0.21956	-23.2019427887619\\
64.0833333333333	0.22122	-24.1555707643152\\
64.0833333333333	0.22288	-25.1091987398686\\
64.0833333333333	0.22454	-26.062826715422\\
64.0833333333333	0.2262	-27.0164546909755\\
64.0833333333333	0.22786	-27.9700826665289\\
64.0833333333333	0.22952	-28.9237106420823\\
64.0833333333333	0.23118	-29.8773386176357\\
64.0833333333333	0.23284	-30.8309665931891\\
64.0833333333333	0.2345	-31.7845945687425\\
64.0833333333333	0.23616	-32.7382225442959\\
64.0833333333333	0.23782	-33.6918505198494\\
64.0833333333333	0.23948	-34.6454784954028\\
64.0833333333333	0.24114	-35.5991064709562\\
64.0833333333333	0.2428	-36.5527344465095\\
64.0833333333333	0.24446	-37.506362422063\\
64.0833333333333	0.24612	-38.4599903976164\\
64.0833333333333	0.24778	-39.4136183731699\\
64.0833333333333	0.24944	-40.3672463487232\\
64.0833333333333	0.2511	-41.3208743242766\\
64.0833333333333	0.25276	-42.2745022998301\\
64.0833333333333	0.25442	-43.2281302753835\\
64.0833333333333	0.25608	-44.181758250937\\
64.0833333333333	0.25774	-45.1353862264904\\
64.0833333333333	0.2594	-46.0890142020437\\
64.0833333333333	0.26106	-47.0426421775972\\
64.0833333333333	0.26272	-47.9962701531505\\
64.0833333333333	0.26438	-48.949898128704\\
64.0833333333333	0.26604	-49.9035261042574\\
64.0833333333333	0.2677	-50.8571540798108\\
64.0833333333333	0.26936	-51.8107820553643\\
64.0833333333333	0.27102	-52.7644100309176\\
64.0833333333333	0.27268	-53.7180380064711\\
64.0833333333333	0.27434	-54.6716659820245\\
64.0833333333333	0.276	-55.6252939575779\\
64.2916666666667	0.193	-7.77189894177042\\
64.2916666666667	0.19466	-8.71781924644159\\
64.2916666666667	0.19632	-9.66373955111271\\
64.2916666666667	0.19798	-10.6096598557839\\
64.2916666666667	0.19964	-11.555580160455\\
64.2916666666667	0.2013	-12.5015004651262\\
64.2916666666667	0.20296	-13.4474207697973\\
64.2916666666667	0.20462	-14.3933410744685\\
64.2916666666667	0.20628	-15.3392613791396\\
64.2916666666667	0.20794	-16.2851816838107\\
64.2916666666667	0.2096	-17.2311019884819\\
64.2916666666667	0.21126	-18.1770222931531\\
64.2916666666667	0.21292	-19.1229425978242\\
64.2916666666667	0.21458	-20.0688629024953\\
64.2916666666667	0.21624	-21.0147832071665\\
64.2916666666667	0.2179	-21.9607035118377\\
64.2916666666667	0.21956	-22.9066238165088\\
64.2916666666667	0.22122	-23.8525441211799\\
64.2916666666667	0.22288	-24.7984644258511\\
64.2916666666667	0.22454	-25.7443847305223\\
64.2916666666667	0.2262	-26.6903050351934\\
64.2916666666667	0.22786	-27.6362253398645\\
64.2916666666667	0.22952	-28.5821456445357\\
64.2916666666667	0.23118	-29.5280659492068\\
64.2916666666667	0.23284	-30.473986253878\\
64.2916666666667	0.2345	-31.4199065585491\\
64.2916666666667	0.23616	-32.3658268632203\\
64.2916666666667	0.23782	-33.3117471678914\\
64.2916666666667	0.23948	-34.2576674725626\\
64.2916666666667	0.24114	-35.2035877772337\\
64.2916666666667	0.2428	-36.1495080819048\\
64.2916666666667	0.24446	-37.095428386576\\
64.2916666666667	0.24612	-38.0413486912472\\
64.2916666666667	0.24778	-38.9872689959183\\
64.2916666666667	0.24944	-39.9331893005894\\
64.2916666666667	0.2511	-40.8791096052606\\
64.2916666666667	0.25276	-41.8250299099317\\
64.2916666666667	0.25442	-42.7709502146029\\
64.2916666666667	0.25608	-43.716870519274\\
64.2916666666667	0.25774	-44.6627908239452\\
64.2916666666667	0.2594	-45.6087111286163\\
64.2916666666667	0.26106	-46.5546314332875\\
64.2916666666667	0.26272	-47.5005517379586\\
64.2916666666667	0.26438	-48.4464720426298\\
64.2916666666667	0.26604	-49.3923923473009\\
64.2916666666667	0.2677	-50.338312651972\\
64.2916666666667	0.26936	-51.2842329566432\\
64.2916666666667	0.27102	-52.2301532613143\\
64.2916666666667	0.27268	-53.1760735659856\\
64.2916666666667	0.27434	-54.1219938706566\\
64.2916666666667	0.276	-55.0679141753278\\
64.5	0.193	-7.59990270363369\\
64.5	0.19466	-8.53811533742254\\
64.5	0.19632	-9.4763279712115\\
64.5	0.19798	-10.4145406050004\\
64.5	0.19964	-11.3527532387892\\
64.5	0.2013	-12.2909658725781\\
64.5	0.20296	-13.229178506367\\
64.5	0.20462	-14.1673911401559\\
64.5	0.20628	-15.1056037739448\\
64.5	0.20794	-16.0438164077336\\
64.5	0.2096	-16.9820290415225\\
64.5	0.21126	-17.9202416753114\\
64.5	0.21292	-18.8584543091003\\
64.5	0.21458	-19.7966669428891\\
64.5	0.21624	-20.7348795766781\\
64.5	0.2179	-21.6730922104669\\
64.5	0.21956	-22.6113048442559\\
64.5	0.22122	-23.5495174780447\\
64.5	0.22288	-24.4877301118336\\
64.5	0.22454	-25.4259427456224\\
64.5	0.2262	-26.3641553794114\\
64.5	0.22786	-27.3023680132002\\
64.5	0.22952	-28.240580646989\\
64.5	0.23118	-29.1787932807779\\
64.5	0.23284	-30.1170059145669\\
64.5	0.2345	-31.0552185483558\\
64.5	0.23616	-31.9934311821445\\
64.5	0.23782	-32.9316438159335\\
64.5	0.23948	-33.8698564497224\\
64.5	0.24114	-34.8080690835113\\
64.5	0.2428	-35.7462817173001\\
64.5	0.24446	-36.684494351089\\
64.5	0.24612	-37.6227069848779\\
64.5	0.24778	-38.5609196186668\\
64.5	0.24944	-39.4991322524556\\
64.5	0.2511	-40.4373448862445\\
64.5	0.25276	-41.3755575200335\\
64.5	0.25442	-42.3137701538223\\
64.5	0.25608	-43.2519827876113\\
64.5	0.25774	-44.1901954214001\\
64.5	0.2594	-45.1284080551889\\
64.5	0.26106	-46.0666206889779\\
64.5	0.26272	-47.0048333227667\\
64.5	0.26438	-47.9430459565556\\
64.5	0.26604	-48.8812585903445\\
64.5	0.2677	-49.8194712241333\\
64.5	0.26936	-50.7576838579223\\
64.5	0.27102	-51.6958964917112\\
64.5	0.27268	-52.6341091255001\\
64.5	0.27434	-53.5723217592889\\
64.5	0.276	-54.5105343930778\\
64.7083333333333	0.193	-7.42790646549696\\
64.7083333333333	0.19466	-8.3584114284036\\
64.7083333333333	0.19632	-9.28891639131018\\
64.7083333333333	0.19798	-10.2194213542168\\
64.7083333333333	0.19964	-11.1499263171234\\
64.7083333333333	0.2013	-12.08043128003\\
64.7083333333333	0.20296	-13.0109362429367\\
64.7083333333333	0.20462	-13.9414412058433\\
64.7083333333333	0.20628	-14.8719461687498\\
64.7083333333333	0.20794	-15.8024511316565\\
64.7083333333333	0.2096	-16.7329560945631\\
64.7083333333333	0.21126	-17.6634610574698\\
64.7083333333333	0.21292	-18.5939660203764\\
64.7083333333333	0.21458	-19.5244709832829\\
64.7083333333333	0.21624	-20.4549759461896\\
64.7083333333333	0.2179	-21.3854809090962\\
64.7083333333333	0.21956	-22.3159858720028\\
64.7083333333333	0.22122	-23.2464908349094\\
64.7083333333333	0.22288	-24.1769957978161\\
64.7083333333333	0.22454	-25.1075007607227\\
64.7083333333333	0.2262	-26.0380057236293\\
64.7083333333333	0.22786	-26.9685106865359\\
64.7083333333333	0.22952	-27.8990156494425\\
64.7083333333333	0.23118	-28.8295206123491\\
64.7083333333333	0.23284	-29.7600255752558\\
64.7083333333333	0.2345	-30.6905305381623\\
64.7083333333333	0.23616	-31.6210355010689\\
64.7083333333333	0.23782	-32.5515404639756\\
64.7083333333333	0.23948	-33.4820454268822\\
64.7083333333333	0.24114	-34.4125503897889\\
64.7083333333333	0.2428	-35.3430553526954\\
64.7083333333333	0.24446	-36.273560315602\\
64.7083333333333	0.24612	-37.2040652785087\\
64.7083333333333	0.24778	-38.1345702414153\\
64.7083333333333	0.24944	-39.0650752043219\\
64.7083333333333	0.2511	-39.9955801672285\\
64.7083333333333	0.25276	-40.9260851301351\\
64.7083333333333	0.25442	-41.8565900930417\\
64.7083333333333	0.25608	-42.7870950559484\\
64.7083333333333	0.25774	-43.717600018855\\
64.7083333333333	0.2594	-44.6481049817615\\
64.7083333333333	0.26106	-45.5786099446682\\
64.7083333333333	0.26272	-46.5091149075748\\
64.7083333333333	0.26438	-47.4396198704815\\
64.7083333333333	0.26604	-48.370124833388\\
64.7083333333333	0.2677	-49.3006297962946\\
64.7083333333333	0.26936	-50.2311347592013\\
64.7083333333333	0.27102	-51.1616397221079\\
64.7083333333333	0.27268	-52.0921446850145\\
64.7083333333333	0.27434	-53.0226496479211\\
64.7083333333333	0.276	-53.9531546108277\\
64.9166666666667	0.193	-7.25591022736023\\
64.9166666666667	0.19466	-8.17870751938449\\
64.9166666666667	0.19632	-9.10150481140892\\
64.9166666666667	0.19798	-10.0243021034333\\
64.9166666666667	0.19964	-10.9470993954576\\
64.9166666666667	0.2013	-11.8698966874819\\
64.9166666666667	0.20296	-12.7926939795063\\
64.9166666666667	0.20462	-13.7154912715307\\
64.9166666666667	0.20628	-14.638288563555\\
64.9166666666667	0.20794	-15.5610858555793\\
64.9166666666667	0.2096	-16.4838831476037\\
64.9166666666667	0.21126	-17.406680439628\\
64.9166666666667	0.21292	-18.3294777316524\\
64.9166666666667	0.21458	-19.2522750236768\\
64.9166666666667	0.21624	-20.1750723157011\\
64.9166666666667	0.2179	-21.0978696077254\\
64.9166666666667	0.21956	-22.0206668997498\\
64.9166666666667	0.22122	-22.9434641917741\\
64.9166666666667	0.22288	-23.8662614837985\\
64.9166666666667	0.22454	-24.7890587758228\\
64.9166666666667	0.2262	-25.7118560678472\\
64.9166666666667	0.22786	-26.6346533598715\\
64.9166666666667	0.22952	-27.5574506518959\\
64.9166666666667	0.23118	-28.4802479439202\\
64.9166666666667	0.23284	-29.4030452359446\\
64.9166666666667	0.2345	-30.3258425279689\\
64.9166666666667	0.23616	-31.2486398199932\\
64.9166666666667	0.23782	-32.1714371120176\\
64.9166666666667	0.23948	-33.094234404042\\
64.9166666666667	0.24114	-34.0170316960663\\
64.9166666666667	0.2428	-34.9398289880907\\
64.9166666666667	0.24446	-35.862626280115\\
64.9166666666667	0.24612	-36.7854235721393\\
64.9166666666667	0.24778	-37.7082208641637\\
64.9166666666667	0.24944	-38.631018156188\\
64.9166666666667	0.2511	-39.5538154482124\\
64.9166666666667	0.25276	-40.4766127402368\\
64.9166666666667	0.25442	-41.3994100322611\\
64.9166666666667	0.25608	-42.3222073242856\\
64.9166666666667	0.25774	-43.2450046163098\\
64.9166666666667	0.2594	-44.1678019083341\\
64.9166666666667	0.26106	-45.0905992003586\\
64.9166666666667	0.26272	-46.0133964923829\\
64.9166666666667	0.26438	-46.9361937844073\\
64.9166666666667	0.26604	-47.8589910764316\\
64.9166666666667	0.2677	-48.7817883684559\\
64.9166666666667	0.26936	-49.7045856604803\\
64.9166666666667	0.27102	-50.6273829525046\\
64.9166666666667	0.27268	-51.550180244529\\
64.9166666666667	0.27434	-52.4729775365533\\
64.9166666666667	0.276	-53.3957748285777\\
65.125	0.193	-7.08391398922339\\
65.125	0.19466	-7.99900361036549\\
65.125	0.19632	-8.91409323150759\\
65.125	0.19798	-9.82918285264969\\
65.125	0.19964	-10.7442724737917\\
65.125	0.2013	-11.6593620949338\\
65.125	0.20296	-12.5744517160759\\
65.125	0.20462	-13.489541337218\\
65.125	0.20628	-14.4046309583601\\
65.125	0.20794	-15.3197205795022\\
65.125	0.2096	-16.2348102006442\\
65.125	0.21126	-17.1498998217863\\
65.125	0.21292	-18.0649894429284\\
65.125	0.21458	-18.9800790640705\\
65.125	0.21624	-19.8951686852126\\
65.125	0.2179	-20.8102583063547\\
65.125	0.21956	-21.7253479274968\\
65.125	0.22122	-22.6404375486388\\
65.125	0.22288	-23.5555271697809\\
65.125	0.22454	-24.470616790923\\
65.125	0.2262	-25.3857064120651\\
65.125	0.22786	-26.3007960332071\\
65.125	0.22952	-27.2158856543492\\
65.125	0.23118	-28.1309752754913\\
65.125	0.23284	-29.0460648966334\\
65.125	0.2345	-29.9611545177755\\
65.125	0.23616	-30.8762441389176\\
65.125	0.23782	-31.7913337600597\\
65.125	0.23948	-32.7064233812018\\
65.125	0.24114	-33.6215130023439\\
65.125	0.2428	-34.5366026234859\\
65.125	0.24446	-35.451692244628\\
65.125	0.24612	-36.3667818657701\\
65.125	0.24778	-37.2818714869122\\
65.125	0.24944	-38.1969611080542\\
65.125	0.2511	-39.1120507291963\\
65.125	0.25276	-40.0271403503384\\
65.125	0.25442	-40.9422299714805\\
65.125	0.25608	-41.8573195926226\\
65.125	0.25774	-42.7724092137647\\
65.125	0.2594	-43.6874988349067\\
65.125	0.26106	-44.6025884560489\\
65.125	0.26272	-45.5176780771909\\
65.125	0.26438	-46.432767698333\\
65.125	0.26604	-47.3478573194751\\
65.125	0.2677	-48.2629469406172\\
65.125	0.26936	-49.1780365617593\\
65.125	0.27102	-50.0931261829013\\
65.125	0.27268	-51.0082158040435\\
65.125	0.27434	-51.9233054251855\\
65.125	0.276	-52.8383950463275\\
65.3333333333333	0.193	-6.91191775108672\\
65.3333333333333	0.19466	-7.81929970134655\\
65.3333333333333	0.19632	-8.72668165160638\\
65.3333333333333	0.19798	-9.63406360186622\\
65.3333333333333	0.19964	-10.541445552126\\
65.3333333333333	0.2013	-11.4488275023858\\
65.3333333333333	0.20296	-12.3562094526457\\
65.3333333333333	0.20462	-13.2635914029054\\
65.3333333333333	0.20628	-14.1709733531653\\
65.3333333333333	0.20794	-15.0783553034251\\
65.3333333333333	0.2096	-15.9857372536849\\
65.3333333333333	0.21126	-16.8931192039447\\
65.3333333333333	0.21292	-17.8005011542045\\
65.3333333333333	0.21458	-18.7078831044643\\
65.3333333333333	0.21624	-19.6152650547241\\
65.3333333333333	0.2179	-20.522647004984\\
65.3333333333333	0.21956	-21.4300289552438\\
65.3333333333333	0.22122	-22.3374109055036\\
65.3333333333333	0.22288	-23.2447928557634\\
65.3333333333333	0.22454	-24.1521748060233\\
65.3333333333333	0.2262	-25.0595567562831\\
65.3333333333333	0.22786	-25.9669387065429\\
65.3333333333333	0.22952	-26.8743206568027\\
65.3333333333333	0.23118	-27.7817026070625\\
65.3333333333333	0.23284	-28.6890845573224\\
65.3333333333333	0.2345	-29.5964665075821\\
65.3333333333333	0.23616	-30.503848457842\\
65.3333333333333	0.23782	-31.4112304081018\\
65.3333333333333	0.23948	-32.3186123583616\\
65.3333333333333	0.24114	-33.2259943086215\\
65.3333333333333	0.2428	-34.1333762588812\\
65.3333333333333	0.24446	-35.0407582091411\\
65.3333333333333	0.24612	-35.9481401594009\\
65.3333333333333	0.24778	-36.8555221096607\\
65.3333333333333	0.24944	-37.7629040599205\\
65.3333333333333	0.2511	-38.6702860101803\\
65.3333333333333	0.25276	-39.5776679604401\\
65.3333333333333	0.25442	-40.4850499107\\
65.3333333333333	0.25608	-41.3924318609598\\
65.3333333333333	0.25774	-42.2998138112196\\
65.3333333333333	0.2594	-43.2071957614793\\
65.3333333333333	0.26106	-44.1145777117392\\
65.3333333333333	0.26272	-45.021959661999\\
65.3333333333333	0.26438	-45.9293416122589\\
65.3333333333333	0.26604	-46.8367235625187\\
65.3333333333333	0.2677	-47.7441055127784\\
65.3333333333333	0.26936	-48.6514874630383\\
65.3333333333333	0.27102	-49.5588694132981\\
65.3333333333333	0.27268	-50.466251363558\\
65.3333333333333	0.27434	-51.3736333138178\\
65.3333333333333	0.276	-52.2810152640776\\
65.5416666666667	0.193	-6.73992151294999\\
65.5416666666667	0.19466	-7.6395957923275\\
65.5416666666667	0.19632	-8.53927007170512\\
65.5416666666667	0.19798	-9.43894435108268\\
65.5416666666667	0.19964	-10.3386186304602\\
65.5416666666667	0.2013	-11.2382929098377\\
65.5416666666667	0.20296	-12.1379671892153\\
65.5416666666667	0.20462	-13.0376414685928\\
65.5416666666667	0.20628	-13.9373157479704\\
65.5416666666667	0.20794	-14.8369900273479\\
65.5416666666667	0.2096	-15.7366643067255\\
65.5416666666667	0.21126	-16.636338586103\\
65.5416666666667	0.21292	-17.5360128654806\\
65.5416666666667	0.21458	-18.4356871448582\\
65.5416666666667	0.21624	-19.3353614242357\\
65.5416666666667	0.2179	-20.2350357036132\\
65.5416666666667	0.21956	-21.1347099829908\\
65.5416666666667	0.22122	-22.0343842623683\\
65.5416666666667	0.22288	-22.9340585417459\\
65.5416666666667	0.22454	-23.8337328211234\\
65.5416666666667	0.2262	-24.733407100501\\
65.5416666666667	0.22786	-25.6330813798785\\
65.5416666666667	0.22952	-26.5327556592561\\
65.5416666666667	0.23118	-27.4324299386336\\
65.5416666666667	0.23284	-28.3321042180112\\
65.5416666666667	0.2345	-29.2317784973887\\
65.5416666666667	0.23616	-30.1314527767663\\
65.5416666666667	0.23782	-31.0311270561438\\
65.5416666666667	0.23948	-31.9308013355214\\
65.5416666666667	0.24114	-32.8304756148989\\
65.5416666666667	0.2428	-33.7301498942765\\
65.5416666666667	0.24446	-34.6298241736541\\
65.5416666666667	0.24612	-35.5294984530316\\
65.5416666666667	0.24778	-36.4291727324091\\
65.5416666666667	0.24944	-37.3288470117866\\
65.5416666666667	0.2511	-38.2285212911642\\
65.5416666666667	0.25276	-39.1281955705418\\
65.5416666666667	0.25442	-40.0278698499193\\
65.5416666666667	0.25608	-40.927544129297\\
65.5416666666667	0.25774	-41.8272184086745\\
65.5416666666667	0.2594	-42.726892688052\\
65.5416666666667	0.26106	-43.6265669674296\\
65.5416666666667	0.26272	-44.5262412468071\\
65.5416666666667	0.26438	-45.4259155261847\\
65.5416666666667	0.26604	-46.3255898055622\\
65.5416666666667	0.2677	-47.2252640849397\\
65.5416666666667	0.26936	-48.1249383643174\\
65.5416666666667	0.27102	-49.0246126436949\\
65.5416666666667	0.27268	-49.9242869230725\\
65.5416666666667	0.27434	-50.82396120245\\
65.5416666666667	0.276	-51.7236354818275\\
65.75	0.193	-6.56792527481321\\
65.75	0.19466	-7.4598918833085\\
65.75	0.19632	-8.3518584918038\\
65.75	0.19798	-9.24382510029909\\
65.75	0.19964	-10.1357917087943\\
65.75	0.2013	-11.0277583172896\\
65.75	0.20296	-11.9197249257849\\
65.75	0.20462	-12.8116915342802\\
65.75	0.20628	-13.7036581427755\\
65.75	0.20794	-14.5956247512708\\
65.75	0.2096	-15.4875913597661\\
65.75	0.21126	-16.3795579682614\\
65.75	0.21292	-17.2715245767567\\
65.75	0.21458	-18.1634911852519\\
65.75	0.21624	-19.0554577937472\\
65.75	0.2179	-19.9474244022425\\
65.75	0.21956	-20.8393910107378\\
65.75	0.22122	-21.7313576192331\\
65.75	0.22288	-22.6233242277283\\
65.75	0.22454	-23.5152908362236\\
65.75	0.2262	-24.4072574447189\\
65.75	0.22786	-25.2992240532142\\
65.75	0.22952	-26.1911906617095\\
65.75	0.23118	-27.0831572702048\\
65.75	0.23284	-27.9751238787001\\
65.75	0.2345	-28.8670904871953\\
65.75	0.23616	-29.7590570956906\\
65.75	0.23782	-30.6510237041859\\
65.75	0.23948	-31.5429903126812\\
65.75	0.24114	-32.4349569211765\\
65.75	0.2428	-33.3269235296718\\
65.75	0.24446	-34.2188901381671\\
65.75	0.24612	-35.1108567466624\\
65.75	0.24778	-36.0028233551577\\
65.75	0.24944	-36.8947899636529\\
65.75	0.2511	-37.7867565721482\\
65.75	0.25276	-38.6787231806435\\
65.75	0.25442	-39.5706897891388\\
65.75	0.25608	-40.4626563976341\\
65.75	0.25774	-41.3546230061293\\
65.75	0.2594	-42.2465896146246\\
65.75	0.26106	-43.1385562231199\\
65.75	0.26272	-44.0305228316151\\
65.75	0.26438	-44.9224894401105\\
65.75	0.26604	-45.8144560486058\\
65.75	0.2677	-46.706422657101\\
65.75	0.26936	-47.5983892655964\\
65.75	0.27102	-48.4903558740916\\
65.75	0.27268	-49.382322482587\\
65.75	0.27434	-50.2742890910822\\
65.75	0.276	-51.1662556995774\\
65.9583333333333	0.193	-6.39592903667642\\
65.9583333333333	0.19466	-7.2801879742895\\
65.9583333333333	0.19632	-8.16444691190247\\
65.9583333333333	0.19798	-9.0487058495155\\
65.9583333333333	0.19964	-9.93296478712847\\
65.9583333333333	0.2013	-10.8172237247416\\
65.9583333333333	0.20296	-11.7014826623545\\
65.9583333333333	0.20462	-12.5857415999676\\
65.9583333333333	0.20628	-13.4700005375806\\
65.9583333333333	0.20794	-14.3542594751937\\
65.9583333333333	0.2096	-15.2385184128066\\
65.9583333333333	0.21126	-16.1227773504197\\
65.9583333333333	0.21292	-17.0070362880327\\
65.9583333333333	0.21458	-17.8912952256456\\
65.9583333333333	0.21624	-18.7755541632587\\
65.9583333333333	0.2179	-19.6598131008718\\
65.9583333333333	0.21956	-20.5440720384847\\
65.9583333333333	0.22122	-21.4283309760977\\
65.9583333333333	0.22288	-22.3125899137108\\
65.9583333333333	0.22454	-23.1968488513239\\
65.9583333333333	0.2262	-24.0811077889368\\
65.9583333333333	0.22786	-24.9653667265498\\
65.9583333333333	0.22952	-25.8496256641629\\
65.9583333333333	0.23118	-26.7338846017759\\
65.9583333333333	0.23284	-27.6181435393889\\
65.9583333333333	0.2345	-28.5024024770019\\
65.9583333333333	0.23616	-29.3866614146149\\
65.9583333333333	0.23782	-30.270920352228\\
65.9583333333333	0.23948	-31.155179289841\\
65.9583333333333	0.24114	-32.0394382274541\\
65.9583333333333	0.2428	-32.923697165067\\
65.9583333333333	0.24446	-33.80795610268\\
65.9583333333333	0.24612	-34.6922150402931\\
65.9583333333333	0.24778	-35.5764739779061\\
65.9583333333333	0.24944	-36.4607329155191\\
65.9583333333333	0.2511	-37.3449918531321\\
65.9583333333333	0.25276	-38.2292507907451\\
65.9583333333333	0.25442	-39.1135097283582\\
65.9583333333333	0.25608	-39.9977686659712\\
65.9583333333333	0.25774	-40.8820276035842\\
65.9583333333333	0.2594	-41.7662865411971\\
65.9583333333333	0.26106	-42.6505454788102\\
65.9583333333333	0.26272	-43.5348044164232\\
65.9583333333333	0.26438	-44.4190633540363\\
65.9583333333333	0.26604	-45.3033222916493\\
65.9583333333333	0.2677	-46.1875812292623\\
65.9583333333333	0.26936	-47.0718401668753\\
65.9583333333333	0.27102	-47.9560991044883\\
65.9583333333333	0.27268	-48.8403580421014\\
65.9583333333333	0.27434	-49.7246169797144\\
65.9583333333333	0.276	-50.6088759173273\\
66.1666666666667	0.193	-6.22393279853969\\
66.1666666666667	0.19466	-7.10048406527051\\
66.1666666666667	0.19632	-7.97703533200126\\
66.1666666666667	0.19798	-8.85358659873202\\
66.1666666666667	0.19964	-9.73013786546272\\
66.1666666666667	0.2013	-10.6066891321935\\
66.1666666666667	0.20296	-11.4832403989242\\
66.1666666666667	0.20462	-12.359791665655\\
66.1666666666667	0.20628	-13.2363429323858\\
66.1666666666667	0.20794	-14.1128941991165\\
66.1666666666667	0.2096	-14.9894454658473\\
66.1666666666667	0.21126	-15.865996732578\\
66.1666666666667	0.21292	-16.7425479993088\\
66.1666666666667	0.21458	-17.6190992660395\\
66.1666666666667	0.21624	-18.4956505327703\\
66.1666666666667	0.2179	-19.3722017995011\\
66.1666666666667	0.21956	-20.2487530662318\\
66.1666666666667	0.22122	-21.1253043329625\\
66.1666666666667	0.22288	-22.0018555996933\\
66.1666666666667	0.22454	-22.878406866424\\
66.1666666666667	0.2262	-23.7549581331548\\
66.1666666666667	0.22786	-24.6315093998855\\
66.1666666666667	0.22952	-25.5080606666163\\
66.1666666666667	0.23118	-26.3846119333471\\
66.1666666666667	0.23284	-27.2611632000778\\
66.1666666666667	0.2345	-28.1377144668085\\
66.1666666666667	0.23616	-29.0142657335393\\
66.1666666666667	0.23782	-29.8908170002701\\
66.1666666666667	0.23948	-30.7673682670008\\
66.1666666666667	0.24114	-31.6439195337316\\
66.1666666666667	0.2428	-32.5204708004623\\
66.1666666666667	0.24446	-33.3970220671931\\
66.1666666666667	0.24612	-34.2735733339238\\
66.1666666666667	0.24778	-35.1501246006546\\
66.1666666666667	0.24944	-36.0266758673853\\
66.1666666666667	0.2511	-36.903227134116\\
66.1666666666667	0.25276	-37.7797784008468\\
66.1666666666667	0.25442	-38.6563296675776\\
66.1666666666667	0.25608	-39.5328809343084\\
66.1666666666667	0.25774	-40.4094322010391\\
66.1666666666667	0.2594	-41.2859834677698\\
66.1666666666667	0.26106	-42.1625347345006\\
66.1666666666667	0.26272	-43.0390860012313\\
66.1666666666667	0.26438	-43.9156372679622\\
66.1666666666667	0.26604	-44.7921885346929\\
66.1666666666667	0.2677	-45.6687398014236\\
66.1666666666667	0.26936	-46.5452910681544\\
66.1666666666667	0.27102	-47.4218423348851\\
66.1666666666667	0.27268	-48.2983936016159\\
66.1666666666667	0.27434	-49.1749448683466\\
66.1666666666667	0.276	-50.0514961350773\\
66.375	0.193	-6.05193656040291\\
66.375	0.19466	-6.92078015625145\\
66.375	0.19632	-7.78962375209989\\
66.375	0.19798	-8.65846734794843\\
66.375	0.19964	-9.52731094379686\\
66.375	0.2013	-10.3961545396454\\
66.375	0.20296	-11.2649981354938\\
66.375	0.20462	-12.1338417313424\\
66.375	0.20628	-13.0026853271908\\
66.375	0.20794	-13.8715289230394\\
66.375	0.2096	-14.7403725188878\\
66.375	0.21126	-15.6092161147363\\
66.375	0.21292	-16.4780597105848\\
66.375	0.21458	-17.3469033064332\\
66.375	0.21624	-18.2157469022818\\
66.375	0.2179	-19.0845904981303\\
66.375	0.21956	-19.9534340939787\\
66.375	0.22122	-20.8222776898272\\
66.375	0.22288	-21.6911212856757\\
66.375	0.22454	-22.5599648815243\\
66.375	0.2262	-23.4288084773727\\
66.375	0.22786	-24.2976520732211\\
66.375	0.22952	-25.1664956690697\\
66.375	0.23118	-26.0353392649182\\
66.375	0.23284	-26.9041828607666\\
66.375	0.2345	-27.7730264566151\\
66.375	0.23616	-28.6418700524636\\
66.375	0.23782	-29.5107136483122\\
66.375	0.23948	-30.3795572441606\\
66.375	0.24114	-31.2484008400091\\
66.375	0.2428	-32.1172444358576\\
66.375	0.24446	-32.986088031706\\
66.375	0.24612	-33.8549316275545\\
66.375	0.24778	-34.7237752234031\\
66.375	0.24944	-35.5926188192515\\
66.375	0.2511	-36.4614624151\\
66.375	0.25276	-37.3303060109484\\
66.375	0.25442	-38.199149606797\\
66.375	0.25608	-39.0679932026455\\
66.375	0.25774	-39.9368367984939\\
66.375	0.2594	-40.8056803943423\\
66.375	0.26106	-41.6745239901909\\
66.375	0.26272	-42.5433675860393\\
66.375	0.26438	-43.4122111818879\\
66.375	0.26604	-44.2810547777364\\
66.375	0.2677	-45.1498983735848\\
66.375	0.26936	-46.0187419694333\\
66.375	0.27102	-46.8875855652818\\
66.375	0.27268	-47.7564291611304\\
66.375	0.27434	-48.6252727569788\\
66.375	0.276	-49.4941163528272\\
66.5833333333333	0.193	-5.87994032226624\\
66.5833333333333	0.19466	-6.74107624723251\\
66.5833333333333	0.19632	-7.60221217219868\\
66.5833333333333	0.19798	-8.4633480971649\\
66.5833333333333	0.19964	-9.32448402213112\\
66.5833333333333	0.2013	-10.1856199470974\\
66.5833333333333	0.20296	-11.0467558720636\\
66.5833333333333	0.20462	-11.9078917970298\\
66.5833333333333	0.20628	-12.769027721996\\
66.5833333333333	0.20794	-13.6301636469623\\
66.5833333333333	0.2096	-14.4912995719284\\
66.5833333333333	0.21126	-15.3524354968947\\
66.5833333333333	0.21292	-16.2135714218609\\
66.5833333333333	0.21458	-17.0747073468271\\
66.5833333333333	0.21624	-17.9358432717933\\
66.5833333333333	0.2179	-18.7969791967596\\
66.5833333333333	0.21956	-19.6581151217258\\
66.5833333333333	0.22122	-20.519251046692\\
66.5833333333333	0.22288	-21.3803869716582\\
66.5833333333333	0.22454	-22.2415228966245\\
66.5833333333333	0.2262	-23.1026588215906\\
66.5833333333333	0.22786	-23.9637947465569\\
66.5833333333333	0.22952	-24.8249306715231\\
66.5833333333333	0.23118	-25.6860665964894\\
66.5833333333333	0.23284	-26.5472025214556\\
66.5833333333333	0.2345	-27.4083384464217\\
66.5833333333333	0.23616	-28.269474371388\\
66.5833333333333	0.23782	-29.1306102963542\\
66.5833333333333	0.23948	-29.9917462213205\\
66.5833333333333	0.24114	-30.8528821462867\\
66.5833333333333	0.2428	-31.7140180712529\\
66.5833333333333	0.24446	-32.5751539962191\\
66.5833333333333	0.24612	-33.4362899211853\\
66.5833333333333	0.24778	-34.2974258461516\\
66.5833333333333	0.24944	-35.1585617711178\\
66.5833333333333	0.2511	-36.019697696084\\
66.5833333333333	0.25276	-36.8808336210502\\
66.5833333333333	0.25442	-37.7419695460165\\
66.5833333333333	0.25608	-38.6031054709827\\
66.5833333333333	0.25774	-39.4642413959488\\
66.5833333333333	0.2594	-40.325377320915\\
66.5833333333333	0.26106	-41.1865132458813\\
66.5833333333333	0.26272	-42.0476491708475\\
66.5833333333333	0.26438	-42.9087850958138\\
66.5833333333333	0.26604	-43.76992102078\\
66.5833333333333	0.2677	-44.6310569457461\\
66.5833333333333	0.26936	-45.4921928707124\\
66.5833333333333	0.27102	-46.3533287956786\\
66.5833333333333	0.27268	-47.2144647206449\\
66.5833333333333	0.27434	-48.0756006456111\\
66.5833333333333	0.276	-48.9367365705772\\
66.7916666666667	0.193	-5.70794408412945\\
66.7916666666667	0.19466	-6.56137233821346\\
66.7916666666667	0.19632	-7.41480059229741\\
66.7916666666667	0.19798	-8.26822884638136\\
66.7916666666667	0.19964	-9.12165710046531\\
66.7916666666667	0.2013	-9.97508535454926\\
66.7916666666667	0.20296	-10.8285136086332\\
66.7916666666667	0.20462	-11.6819418627172\\
66.7916666666667	0.20628	-12.5353701168011\\
66.7916666666667	0.20794	-13.3887983708851\\
66.7916666666667	0.2096	-14.242226624969\\
66.7916666666667	0.21126	-15.095654879053\\
66.7916666666667	0.21292	-15.949083133137\\
66.7916666666667	0.21458	-16.8025113872209\\
66.7916666666667	0.21624	-17.6559396413049\\
66.7916666666667	0.2179	-18.5093678953888\\
66.7916666666667	0.21956	-19.3627961494728\\
66.7916666666667	0.22122	-20.2162244035567\\
66.7916666666667	0.22288	-21.0696526576407\\
66.7916666666667	0.22454	-21.9230809117246\\
66.7916666666667	0.2262	-22.7765091658086\\
66.7916666666667	0.22786	-23.6299374198925\\
66.7916666666667	0.22952	-24.4833656739765\\
66.7916666666667	0.23118	-25.3367939280604\\
66.7916666666667	0.23284	-26.1902221821444\\
66.7916666666667	0.2345	-27.0436504362283\\
66.7916666666667	0.23616	-27.8970786903123\\
66.7916666666667	0.23782	-28.7505069443962\\
66.7916666666667	0.23948	-29.6039351984803\\
66.7916666666667	0.24114	-30.4573634525642\\
66.7916666666667	0.2428	-31.3107917066481\\
66.7916666666667	0.24446	-32.1642199607321\\
66.7916666666667	0.24612	-33.017648214816\\
66.7916666666667	0.24778	-33.8710764689\\
66.7916666666667	0.24944	-34.7245047229839\\
66.7916666666667	0.2511	-35.5779329770679\\
66.7916666666667	0.25276	-36.4313612311518\\
66.7916666666667	0.25442	-37.2847894852358\\
66.7916666666667	0.25608	-38.1382177393198\\
66.7916666666667	0.25774	-38.9916459934037\\
66.7916666666667	0.2594	-39.8450742474876\\
66.7916666666667	0.26106	-40.6985025015716\\
66.7916666666667	0.26272	-41.5519307556555\\
66.7916666666667	0.26438	-42.4053590097396\\
66.7916666666667	0.26604	-43.2587872638235\\
66.7916666666667	0.2677	-44.1122155179074\\
66.7916666666667	0.26936	-44.9656437719914\\
66.7916666666667	0.27102	-45.8190720260753\\
66.7916666666667	0.27268	-46.6725002801593\\
66.7916666666667	0.27434	-47.5259285342433\\
66.7916666666667	0.276	-48.3793567883272\\
67	0.193	-5.53594784599272\\
67	0.19466	-6.38166842919446\\
67	0.19632	-7.22738901239609\\
67	0.19798	-8.07310959559783\\
67	0.19964	-8.91883017879945\\
67	0.2013	-9.76455076200125\\
67	0.20296	-10.6102713452029\\
67	0.20462	-11.4559919284046\\
67	0.20628	-12.3017125116062\\
67	0.20794	-13.147433094808\\
67	0.2096	-13.9931536780096\\
67	0.21126	-14.8388742612114\\
67	0.21292	-15.684594844413\\
67	0.21458	-16.5303154276147\\
67	0.21624	-17.3760360108164\\
67	0.2179	-18.2217565940181\\
67	0.21956	-19.0674771772198\\
67	0.22122	-19.9131977604214\\
67	0.22288	-20.7589183436231\\
67	0.22454	-21.6046389268249\\
67	0.2262	-22.4503595100265\\
67	0.22786	-23.2960800932282\\
67	0.22952	-24.1418006764299\\
67	0.23118	-24.9875212596316\\
67	0.23284	-25.8332418428333\\
67	0.2345	-26.6789624260349\\
67	0.23616	-27.5246830092367\\
67	0.23782	-28.3704035924384\\
67	0.23948	-29.21612417564\\
67	0.24114	-30.0618447588418\\
67	0.2428	-30.9075653420434\\
67	0.24446	-31.7532859252451\\
67	0.24612	-32.5990065084468\\
67	0.24778	-33.4447270916485\\
67	0.24944	-34.2904476748502\\
67	0.2511	-35.1361682580518\\
67	0.25276	-35.9818888412535\\
67	0.25442	-36.8276094244553\\
67	0.25608	-37.6733300076569\\
67	0.25774	-38.5190505908586\\
67	0.2594	-39.3647711740602\\
67	0.26106	-40.210491757262\\
67	0.26272	-41.0562123404636\\
67	0.26438	-41.9019329236654\\
67	0.26604	-42.747653506867\\
67	0.2677	-43.5933740900687\\
67	0.26936	-44.4390946732705\\
67	0.27102	-45.2848152564721\\
67	0.27268	-46.1305358396738\\
67	0.27434	-46.9762564228755\\
67	0.276	-47.8219770060771\\
};
\end{axis}

\begin{axis}[%
width=4.927496cm,
height=3.050847cm,
at={(0cm,4.237288cm)},
scale only axis,
xmin=57,
xmax=67,
tick align=outside,
xlabel={$L_{cut}$},
xmajorgrids,
ymin=0.193,
ymax=0.276,
ylabel={$D_{rlx}$},
ymajorgrids,
zmin=-2234.6303598581,
zmax=0,
zlabel={$c$},
zmajorgrids,
view={-140}{50},
legend style={at={(1.03,1)},anchor=north west,legend cell align=left,align=left,draw=white!15!black}
]
\addplot3[only marks,mark=*,mark options={},mark size=1.5000pt,color=mycolor1] plot table[row sep=crcr,]{%
67	0.276	-1665.08665361604\\
66	0.255	-1377.66527094523\\
62	0.209	-752.290671277159\\
57	0.193	-598.013655078842\\
};
\addplot3[only marks,mark=*,mark options={},mark size=1.5000pt,color=black] plot table[row sep=crcr,]{%
64	0.23	-1047.67094505203\\
};

\addplot3[%
surf,
opacity=0.7,
shader=interp,
colormap={mymap}{[1pt] rgb(0pt)=(0.0901961,0.239216,0.0745098); rgb(1pt)=(0.0945149,0.242058,0.0739522); rgb(2pt)=(0.0988592,0.244894,0.0733566); rgb(3pt)=(0.103229,0.247724,0.0727241); rgb(4pt)=(0.107623,0.250549,0.0720557); rgb(5pt)=(0.112043,0.253367,0.0713525); rgb(6pt)=(0.116487,0.25618,0.0706154); rgb(7pt)=(0.120956,0.258986,0.0698456); rgb(8pt)=(0.125449,0.261787,0.0690441); rgb(9pt)=(0.129967,0.264581,0.0682118); rgb(10pt)=(0.134508,0.26737,0.06735); rgb(11pt)=(0.139074,0.270152,0.0664596); rgb(12pt)=(0.143663,0.272929,0.0655416); rgb(13pt)=(0.148275,0.275699,0.0645971); rgb(14pt)=(0.152911,0.278463,0.0636271); rgb(15pt)=(0.15757,0.281221,0.0626328); rgb(16pt)=(0.162252,0.283973,0.0616151); rgb(17pt)=(0.166957,0.286719,0.060575); rgb(18pt)=(0.171685,0.289458,0.0595136); rgb(19pt)=(0.176434,0.292191,0.0584321); rgb(20pt)=(0.181207,0.294918,0.0573313); rgb(21pt)=(0.186001,0.297639,0.0562123); rgb(22pt)=(0.190817,0.300353,0.0550763); rgb(23pt)=(0.195655,0.303061,0.0539242); rgb(24pt)=(0.200514,0.305763,0.052757); rgb(25pt)=(0.205395,0.308459,0.0515759); rgb(26pt)=(0.210296,0.311149,0.0503624); rgb(27pt)=(0.215212,0.313846,0.0490067); rgb(28pt)=(0.220142,0.316548,0.0475043); rgb(29pt)=(0.22509,0.319254,0.0458704); rgb(30pt)=(0.230056,0.321962,0.0441205); rgb(31pt)=(0.235042,0.324671,0.04227); rgb(32pt)=(0.240048,0.327379,0.0403343); rgb(33pt)=(0.245078,0.330085,0.0383287); rgb(34pt)=(0.250131,0.332786,0.0362688); rgb(35pt)=(0.25521,0.335482,0.0341698); rgb(36pt)=(0.260317,0.33817,0.0320472); rgb(37pt)=(0.265451,0.340849,0.0299163); rgb(38pt)=(0.270616,0.343517,0.0277927); rgb(39pt)=(0.275813,0.346172,0.0256916); rgb(40pt)=(0.281043,0.348814,0.0236284); rgb(41pt)=(0.286307,0.35144,0.0216186); rgb(42pt)=(0.291607,0.354048,0.0196776); rgb(43pt)=(0.296945,0.356637,0.0178207); rgb(44pt)=(0.302322,0.359206,0.0160634); rgb(45pt)=(0.307739,0.361753,0.0144211); rgb(46pt)=(0.313198,0.364275,0.0129091); rgb(47pt)=(0.318701,0.366772,0.0115428); rgb(48pt)=(0.324249,0.369242,0.0103377); rgb(49pt)=(0.329843,0.371682,0.00930909); rgb(50pt)=(0.335485,0.374093,0.00847245); rgb(51pt)=(0.341176,0.376471,0.00784314); rgb(52pt)=(0.346925,0.378826,0.00732741); rgb(53pt)=(0.352735,0.381168,0.00682184); rgb(54pt)=(0.358605,0.383497,0.00632729); rgb(55pt)=(0.364532,0.385812,0.00584464); rgb(56pt)=(0.370516,0.388113,0.00537476); rgb(57pt)=(0.376552,0.390399,0.00491852); rgb(58pt)=(0.38264,0.39267,0.00447681); rgb(59pt)=(0.388777,0.394925,0.00405048); rgb(60pt)=(0.394962,0.397164,0.00364042); rgb(61pt)=(0.401191,0.399386,0.00324749); rgb(62pt)=(0.407464,0.401592,0.00287258); rgb(63pt)=(0.413777,0.40378,0.00251655); rgb(64pt)=(0.420129,0.40595,0.00218028); rgb(65pt)=(0.426518,0.408102,0.00186463); rgb(66pt)=(0.432942,0.410234,0.00157049); rgb(67pt)=(0.439399,0.412348,0.00129873); rgb(68pt)=(0.445885,0.414441,0.00105022); rgb(69pt)=(0.452401,0.416515,0.000825833); rgb(70pt)=(0.458942,0.418567,0.000626441); rgb(71pt)=(0.465508,0.420599,0.00045292); rgb(72pt)=(0.472096,0.422609,0.000306141); rgb(73pt)=(0.478704,0.424596,0.000186979); rgb(74pt)=(0.485331,0.426562,9.63073e-05); rgb(75pt)=(0.491973,0.428504,3.49981e-05); rgb(76pt)=(0.498628,0.430422,3.92506e-06); rgb(77pt)=(0.505323,0.432315,0); rgb(78pt)=(0.512206,0.434168,0); rgb(79pt)=(0.519282,0.435983,0); rgb(80pt)=(0.526529,0.437764,0); rgb(81pt)=(0.533922,0.439512,0); rgb(82pt)=(0.54144,0.441232,0); rgb(83pt)=(0.549059,0.442927,0); rgb(84pt)=(0.556756,0.444599,0); rgb(85pt)=(0.564508,0.446252,0); rgb(86pt)=(0.572292,0.447889,0); rgb(87pt)=(0.580084,0.449514,0); rgb(88pt)=(0.587863,0.451129,0); rgb(89pt)=(0.595604,0.452737,0); rgb(90pt)=(0.603284,0.454343,0); rgb(91pt)=(0.610882,0.455948,0); rgb(92pt)=(0.618373,0.457556,0); rgb(93pt)=(0.625734,0.459171,0); rgb(94pt)=(0.632943,0.460795,0); rgb(95pt)=(0.639976,0.462432,0); rgb(96pt)=(0.64681,0.464084,0); rgb(97pt)=(0.653423,0.465756,0); rgb(98pt)=(0.659791,0.46745,0); rgb(99pt)=(0.665891,0.469169,0); rgb(100pt)=(0.6717,0.470916,0); rgb(101pt)=(0.677195,0.472696,0); rgb(102pt)=(0.682353,0.47451,0); rgb(103pt)=(0.687242,0.476355,0); rgb(104pt)=(0.691952,0.478225,0); rgb(105pt)=(0.696497,0.480118,0); rgb(106pt)=(0.700887,0.482033,0); rgb(107pt)=(0.705134,0.483968,0); rgb(108pt)=(0.709251,0.485921,0); rgb(109pt)=(0.713249,0.487891,0); rgb(110pt)=(0.71714,0.489876,0); rgb(111pt)=(0.720936,0.491875,0); rgb(112pt)=(0.724649,0.493887,0); rgb(113pt)=(0.72829,0.495909,0); rgb(114pt)=(0.731872,0.49794,0); rgb(115pt)=(0.735406,0.499979,0); rgb(116pt)=(0.738904,0.502025,0); rgb(117pt)=(0.742378,0.504075,0); rgb(118pt)=(0.74584,0.506128,0); rgb(119pt)=(0.749302,0.508182,0); rgb(120pt)=(0.752775,0.510237,0); rgb(121pt)=(0.756272,0.51229,0); rgb(122pt)=(0.759804,0.514339,0); rgb(123pt)=(0.763384,0.516385,0); rgb(124pt)=(0.767022,0.518424,0); rgb(125pt)=(0.770731,0.520455,0); rgb(126pt)=(0.774523,0.522478,0); rgb(127pt)=(0.77841,0.524489,0); rgb(128pt)=(0.782391,0.526491,0); rgb(129pt)=(0.786402,0.528496,0); rgb(130pt)=(0.790431,0.530506,0); rgb(131pt)=(0.794478,0.532521,0); rgb(132pt)=(0.798541,0.534539,0); rgb(133pt)=(0.802619,0.53656,0); rgb(134pt)=(0.806712,0.538584,0); rgb(135pt)=(0.81082,0.540609,0); rgb(136pt)=(0.81494,0.542635,0); rgb(137pt)=(0.819074,0.54466,0); rgb(138pt)=(0.823219,0.546686,0); rgb(139pt)=(0.827374,0.548709,0); rgb(140pt)=(0.831541,0.55073,0); rgb(141pt)=(0.835716,0.552749,0); rgb(142pt)=(0.8399,0.554763,0); rgb(143pt)=(0.844092,0.556774,0); rgb(144pt)=(0.848292,0.558779,0); rgb(145pt)=(0.852497,0.560778,0); rgb(146pt)=(0.856708,0.562771,0); rgb(147pt)=(0.860924,0.564756,0); rgb(148pt)=(0.865143,0.566733,0); rgb(149pt)=(0.869366,0.568701,0); rgb(150pt)=(0.873592,0.57066,0); rgb(151pt)=(0.877819,0.572608,0); rgb(152pt)=(0.882047,0.574545,0); rgb(153pt)=(0.886275,0.576471,0); rgb(154pt)=(0.890659,0.578362,0); rgb(155pt)=(0.895333,0.580203,0); rgb(156pt)=(0.900258,0.581999,0); rgb(157pt)=(0.905397,0.583755,0); rgb(158pt)=(0.910711,0.585479,0); rgb(159pt)=(0.916164,0.587176,0); rgb(160pt)=(0.921717,0.588852,0); rgb(161pt)=(0.927333,0.590513,0); rgb(162pt)=(0.932974,0.592166,0); rgb(163pt)=(0.938602,0.593815,0); rgb(164pt)=(0.94418,0.595468,0); rgb(165pt)=(0.949669,0.59713,0); rgb(166pt)=(0.955033,0.598808,0); rgb(167pt)=(0.960233,0.600507,0); rgb(168pt)=(0.965232,0.602233,0); rgb(169pt)=(0.969992,0.603992,0); rgb(170pt)=(0.974475,0.605791,0); rgb(171pt)=(0.978643,0.607636,0); rgb(172pt)=(0.98246,0.609532,0); rgb(173pt)=(0.985886,0.611486,0); rgb(174pt)=(0.988885,0.613503,0); rgb(175pt)=(0.991419,0.61559,0); rgb(176pt)=(0.99345,0.617753,0); rgb(177pt)=(0.99494,0.619997,0); rgb(178pt)=(0.995851,0.622329,0); rgb(179pt)=(0.996226,0.624763,0); rgb(180pt)=(0.996512,0.627352,0); rgb(181pt)=(0.996788,0.630095,0); rgb(182pt)=(0.997053,0.632982,0); rgb(183pt)=(0.997308,0.636004,0); rgb(184pt)=(0.997552,0.639152,0); rgb(185pt)=(0.997785,0.642416,0); rgb(186pt)=(0.998006,0.645786,0); rgb(187pt)=(0.998217,0.649253,0); rgb(188pt)=(0.998416,0.652807,0); rgb(189pt)=(0.998605,0.656439,0); rgb(190pt)=(0.998781,0.660138,0); rgb(191pt)=(0.998946,0.663897,0); rgb(192pt)=(0.9991,0.667704,0); rgb(193pt)=(0.999242,0.67155,0); rgb(194pt)=(0.999372,0.675427,0); rgb(195pt)=(0.99949,0.679323,0); rgb(196pt)=(0.999596,0.68323,0); rgb(197pt)=(0.99969,0.687139,0); rgb(198pt)=(0.999771,0.691039,0); rgb(199pt)=(0.999841,0.694921,0); rgb(200pt)=(0.999898,0.698775,0); rgb(201pt)=(0.999942,0.702592,0); rgb(202pt)=(0.999974,0.706363,0); rgb(203pt)=(0.999994,0.710077,0); rgb(204pt)=(1,0.713725,0); rgb(205pt)=(1,0.717341,0); rgb(206pt)=(1,0.720963,0); rgb(207pt)=(1,0.724591,0); rgb(208pt)=(1,0.728226,0); rgb(209pt)=(1,0.731867,0); rgb(210pt)=(1,0.735514,0); rgb(211pt)=(1,0.739167,0); rgb(212pt)=(1,0.742827,0); rgb(213pt)=(1,0.746493,0); rgb(214pt)=(1,0.750165,0); rgb(215pt)=(1,0.753843,0); rgb(216pt)=(1,0.757527,0); rgb(217pt)=(1,0.761217,0); rgb(218pt)=(1,0.764913,0); rgb(219pt)=(1,0.768615,0); rgb(220pt)=(1,0.772324,0); rgb(221pt)=(1,0.776038,0); rgb(222pt)=(1,0.779758,0); rgb(223pt)=(1,0.783484,0); rgb(224pt)=(1,0.787215,0); rgb(225pt)=(1,0.790953,0); rgb(226pt)=(1,0.794696,0); rgb(227pt)=(1,0.798445,0); rgb(228pt)=(1,0.8022,0); rgb(229pt)=(1,0.805961,0); rgb(230pt)=(1,0.809727,0); rgb(231pt)=(1,0.8135,0); rgb(232pt)=(1,0.817278,0); rgb(233pt)=(1,0.821063,0); rgb(234pt)=(1,0.824854,0); rgb(235pt)=(1,0.828652,0); rgb(236pt)=(1,0.832455,0); rgb(237pt)=(1,0.836265,0); rgb(238pt)=(1,0.840081,0); rgb(239pt)=(1,0.843903,0); rgb(240pt)=(1,0.847732,0); rgb(241pt)=(1,0.851566,0); rgb(242pt)=(1,0.855406,0); rgb(243pt)=(1,0.859253,0); rgb(244pt)=(1,0.863106,0); rgb(245pt)=(1,0.866964,0); rgb(246pt)=(1,0.870829,0); rgb(247pt)=(1,0.8747,0); rgb(248pt)=(1,0.878577,0); rgb(249pt)=(1,0.88246,0); rgb(250pt)=(1,0.886349,0); rgb(251pt)=(1,0.890243,0); rgb(252pt)=(1,0.894144,0); rgb(253pt)=(1,0.898051,0); rgb(254pt)=(1,0.901964,0); rgb(255pt)=(1,0.905882,0)},
mesh/rows=49]
table[row sep=crcr,header=false] {%
%
57	0.193	-598.013655079526\\
57	0.19466	-630.745989175097\\
57	0.19632	-663.478323270668\\
57	0.19798	-696.21065736624\\
57	0.19964	-728.942991461811\\
57	0.2013	-761.675325557384\\
57	0.20296	-794.407659652955\\
57	0.20462	-827.139993748526\\
57	0.20628	-859.872327844098\\
57	0.20794	-892.604661939668\\
57	0.2096	-925.336996035241\\
57	0.21126	-958.069330130812\\
57	0.21292	-990.801664226383\\
57	0.21458	-1023.53399832195\\
57	0.21624	-1056.26633241753\\
57	0.2179	-1088.9986665131\\
57	0.21956	-1121.73100060867\\
57	0.22122	-1154.46333470424\\
57	0.22288	-1187.19566879981\\
57	0.22454	-1219.92800289538\\
57	0.2262	-1252.66033699095\\
57	0.22786	-1285.39267108653\\
57	0.22952	-1318.1250051821\\
57	0.23118	-1350.85733927767\\
57	0.23284	-1383.58967337324\\
57	0.2345	-1416.32200746881\\
57	0.23616	-1449.05434156438\\
57	0.23782	-1481.78667565996\\
57	0.23948	-1514.51900975553\\
57	0.24114	-1547.2513438511\\
57	0.2428	-1579.98367794667\\
57	0.24446	-1612.71601204224\\
57	0.24612	-1645.44834613781\\
57	0.24778	-1678.18068023338\\
57	0.24944	-1710.91301432895\\
57	0.2511	-1743.64534842452\\
57	0.25276	-1776.3776825201\\
57	0.25442	-1809.11001661567\\
57	0.25608	-1841.84235071124\\
57	0.25774	-1874.57468480681\\
57	0.2594	-1907.30701890238\\
57	0.26106	-1940.03935299795\\
57	0.26272	-1972.77168709352\\
57	0.26438	-2005.50402118909\\
57	0.26604	-2038.23635528467\\
57	0.2677	-2070.96868938024\\
57	0.26936	-2103.70102347581\\
57	0.27102	-2136.43335757138\\
57	0.27268	-2169.16569166695\\
57	0.27434	-2201.89802576252\\
57	0.276	-2234.6303598581\\
57.2083333333333	0.193	-592.525766222501\\
57.2083333333333	0.19466	-625.130548217612\\
57.2083333333333	0.19632	-657.735330212722\\
57.2083333333333	0.19798	-690.340112207834\\
57.2083333333333	0.19964	-722.944894202945\\
57.2083333333333	0.2013	-755.549676198056\\
57.2083333333333	0.20296	-788.154458193167\\
57.2083333333333	0.20462	-820.759240188278\\
57.2083333333333	0.20628	-853.364022183388\\
57.2083333333333	0.20794	-885.968804178499\\
57.2083333333333	0.2096	-918.573586173612\\
57.2083333333333	0.21126	-951.178368168722\\
57.2083333333333	0.21292	-983.783150163833\\
57.2083333333333	0.21458	-1016.38793215894\\
57.2083333333333	0.21624	-1048.99271415405\\
57.2083333333333	0.2179	-1081.59749614917\\
57.2083333333333	0.21956	-1114.20227814428\\
57.2083333333333	0.22122	-1146.80706013939\\
57.2083333333333	0.22288	-1179.4118421345\\
57.2083333333333	0.22454	-1212.01662412961\\
57.2083333333333	0.2262	-1244.62140612472\\
57.2083333333333	0.22786	-1277.22618811983\\
57.2083333333333	0.22952	-1309.83097011494\\
57.2083333333333	0.23118	-1342.43575211005\\
57.2083333333333	0.23284	-1375.04053410516\\
57.2083333333333	0.2345	-1407.64531610028\\
57.2083333333333	0.23616	-1440.25009809539\\
57.2083333333333	0.23782	-1472.8548800905\\
57.2083333333333	0.23948	-1505.45966208561\\
57.2083333333333	0.24114	-1538.06444408072\\
57.2083333333333	0.2428	-1570.66922607583\\
57.2083333333333	0.24446	-1603.27400807094\\
57.2083333333333	0.24612	-1635.87879006605\\
57.2083333333333	0.24778	-1668.48357206116\\
57.2083333333333	0.24944	-1701.08835405627\\
57.2083333333333	0.2511	-1733.69313605138\\
57.2083333333333	0.25276	-1766.2979180465\\
57.2083333333333	0.25442	-1798.90270004161\\
57.2083333333333	0.25608	-1831.50748203672\\
57.2083333333333	0.25774	-1864.11226403183\\
57.2083333333333	0.2594	-1896.71704602694\\
57.2083333333333	0.26106	-1929.32182802205\\
57.2083333333333	0.26272	-1961.92661001716\\
57.2083333333333	0.26438	-1994.53139201227\\
57.2083333333333	0.26604	-2027.13617400739\\
57.2083333333333	0.2677	-2059.7409560025\\
57.2083333333333	0.26936	-2092.34573799761\\
57.2083333333333	0.27102	-2124.95051999272\\
57.2083333333333	0.27268	-2157.55530198783\\
57.2083333333333	0.27434	-2190.16008398294\\
57.2083333333333	0.276	-2222.76486597805\\
57.4166666666667	0.193	-587.037877365476\\
57.4166666666667	0.19466	-619.515107260127\\
57.4166666666667	0.19632	-651.992337154777\\
57.4166666666667	0.19798	-684.469567049428\\
57.4166666666667	0.19964	-716.946796944078\\
57.4166666666667	0.2013	-749.424026838729\\
57.4166666666667	0.20296	-781.901256733379\\
57.4166666666667	0.20462	-814.378486628029\\
57.4166666666667	0.20628	-846.855716522681\\
57.4166666666667	0.20794	-879.332946417331\\
57.4166666666667	0.2096	-911.810176311983\\
57.4166666666667	0.21126	-944.287406206633\\
57.4166666666667	0.21292	-976.764636101283\\
57.4166666666667	0.21458	-1009.24186599593\\
57.4166666666667	0.21624	-1041.71909589058\\
57.4166666666667	0.2179	-1074.19632578523\\
57.4166666666667	0.21956	-1106.67355567988\\
57.4166666666667	0.22122	-1139.15078557454\\
57.4166666666667	0.22288	-1171.62801546919\\
57.4166666666667	0.22454	-1204.10524536384\\
57.4166666666667	0.2262	-1236.58247525849\\
57.4166666666667	0.22786	-1269.05970515314\\
57.4166666666667	0.22952	-1301.53693504779\\
57.4166666666667	0.23118	-1334.01416494244\\
57.4166666666667	0.23284	-1366.49139483709\\
57.4166666666667	0.2345	-1398.96862473174\\
57.4166666666667	0.23616	-1431.44585462639\\
57.4166666666667	0.23782	-1463.92308452104\\
57.4166666666667	0.23948	-1496.40031441569\\
57.4166666666667	0.24114	-1528.87754431034\\
57.4166666666667	0.2428	-1561.35477420499\\
57.4166666666667	0.24446	-1593.83200409965\\
57.4166666666667	0.24612	-1626.3092339943\\
57.4166666666667	0.24778	-1658.78646388895\\
57.4166666666667	0.24944	-1691.2636937836\\
57.4166666666667	0.2511	-1723.74092367825\\
57.4166666666667	0.25276	-1756.2181535729\\
57.4166666666667	0.25442	-1788.69538346755\\
57.4166666666667	0.25608	-1821.1726133622\\
57.4166666666667	0.25774	-1853.64984325685\\
57.4166666666667	0.2594	-1886.1270731515\\
57.4166666666667	0.26106	-1918.60430304615\\
57.4166666666667	0.26272	-1951.0815329408\\
57.4166666666667	0.26438	-1983.55876283545\\
57.4166666666667	0.26604	-2016.0359927301\\
57.4166666666667	0.2677	-2048.51322262475\\
57.4166666666667	0.26936	-2080.9904525194\\
57.4166666666667	0.27102	-2113.46768241406\\
57.4166666666667	0.27268	-2145.94491230871\\
57.4166666666667	0.27434	-2178.42214220336\\
57.4166666666667	0.276	-2210.89937209801\\
57.625	0.193	-581.549988508451\\
57.625	0.19466	-613.899666302642\\
57.625	0.19632	-646.249344096831\\
57.625	0.19798	-678.599021891022\\
57.625	0.19964	-710.948699685212\\
57.625	0.2013	-743.298377479402\\
57.625	0.20296	-775.648055273592\\
57.625	0.20462	-807.997733067782\\
57.625	0.20628	-840.347410861973\\
57.625	0.20794	-872.697088656162\\
57.625	0.2096	-905.046766450354\\
57.625	0.21126	-937.396444244544\\
57.625	0.21292	-969.746122038733\\
57.625	0.21458	-1002.09579983292\\
57.625	0.21624	-1034.44547762711\\
57.625	0.2179	-1066.7951554213\\
57.625	0.21956	-1099.14483321549\\
57.625	0.22122	-1131.49451100969\\
57.625	0.22288	-1163.84418880388\\
57.625	0.22454	-1196.19386659807\\
57.625	0.2262	-1228.54354439226\\
57.625	0.22786	-1260.89322218645\\
57.625	0.22952	-1293.24289998064\\
57.625	0.23118	-1325.59257777483\\
57.625	0.23284	-1357.94225556902\\
57.625	0.2345	-1390.29193336321\\
57.625	0.23616	-1422.6416111574\\
57.625	0.23782	-1454.99128895159\\
57.625	0.23948	-1487.34096674578\\
57.625	0.24114	-1519.69064453997\\
57.625	0.2428	-1552.04032233416\\
57.625	0.24446	-1584.39000012835\\
57.625	0.24612	-1616.73967792254\\
57.625	0.24778	-1649.08935571673\\
57.625	0.24944	-1681.43903351092\\
57.625	0.2511	-1713.78871130511\\
57.625	0.25276	-1746.1383890993\\
57.625	0.25442	-1778.48806689349\\
57.625	0.25608	-1810.83774468768\\
57.625	0.25774	-1843.18742248187\\
57.625	0.2594	-1875.53710027606\\
57.625	0.26106	-1907.88677807025\\
57.625	0.26272	-1940.23645586444\\
57.625	0.26438	-1972.58613365863\\
57.625	0.26604	-2004.93581145282\\
57.625	0.2677	-2037.28548924701\\
57.625	0.26936	-2069.6351670412\\
57.625	0.27102	-2101.98484483539\\
57.625	0.27268	-2134.33452262958\\
57.625	0.27434	-2166.68420042377\\
57.625	0.276	-2199.03387821796\\
57.8333333333333	0.193	-576.062099651426\\
57.8333333333333	0.19466	-608.284225345155\\
57.8333333333333	0.19632	-640.506351038885\\
57.8333333333333	0.19798	-672.728476732615\\
57.8333333333333	0.19964	-704.950602426345\\
57.8333333333333	0.2013	-737.172728120076\\
57.8333333333333	0.20296	-769.394853813805\\
57.8333333333333	0.20462	-801.616979507535\\
57.8333333333333	0.20628	-833.839105201265\\
57.8333333333333	0.20794	-866.061230894994\\
57.8333333333333	0.2096	-898.283356588725\\
57.8333333333333	0.21126	-930.505482282455\\
57.8333333333333	0.21292	-962.727607976184\\
57.8333333333333	0.21458	-994.949733669913\\
57.8333333333333	0.21624	-1027.17185936364\\
57.8333333333333	0.2179	-1059.39398505737\\
57.8333333333333	0.21956	-1091.6161107511\\
57.8333333333333	0.22122	-1123.83823644483\\
57.8333333333333	0.22288	-1156.06036213856\\
57.8333333333333	0.22454	-1188.28248783229\\
57.8333333333333	0.2262	-1220.50461352602\\
57.8333333333333	0.22786	-1252.72673921975\\
57.8333333333333	0.22952	-1284.94886491348\\
57.8333333333333	0.23118	-1317.17099060721\\
57.8333333333333	0.23284	-1349.39311630094\\
57.8333333333333	0.2345	-1381.61524199467\\
57.8333333333333	0.23616	-1413.8373676884\\
57.8333333333333	0.23782	-1446.05949338213\\
57.8333333333333	0.23948	-1478.28161907586\\
57.8333333333333	0.24114	-1510.50374476959\\
57.8333333333333	0.2428	-1542.72587046332\\
57.8333333333333	0.24446	-1574.94799615705\\
57.8333333333333	0.24612	-1607.17012185078\\
57.8333333333333	0.24778	-1639.39224754451\\
57.8333333333333	0.24944	-1671.61437323824\\
57.8333333333333	0.2511	-1703.83649893197\\
57.8333333333333	0.25276	-1736.0586246257\\
57.8333333333333	0.25442	-1768.28075031943\\
57.8333333333333	0.25608	-1800.50287601316\\
57.8333333333333	0.25774	-1832.72500170689\\
57.8333333333333	0.2594	-1864.94712740062\\
57.8333333333333	0.26106	-1897.16925309435\\
57.8333333333333	0.26272	-1929.39137878808\\
57.8333333333333	0.26438	-1961.61350448181\\
57.8333333333333	0.26604	-1993.83563017554\\
57.8333333333333	0.2677	-2026.05775586927\\
57.8333333333333	0.26936	-2058.279881563\\
57.8333333333333	0.27102	-2090.50200725673\\
57.8333333333333	0.27268	-2122.72413295046\\
57.8333333333333	0.27434	-2154.94625864419\\
57.8333333333333	0.276	-2187.16838433792\\
58.0416666666667	0.193	-570.574210794401\\
58.0416666666667	0.19466	-602.66878438767\\
58.0416666666667	0.19632	-634.763357980939\\
58.0416666666667	0.19798	-666.857931574209\\
58.0416666666667	0.19964	-698.952505167478\\
58.0416666666667	0.2013	-731.047078760749\\
58.0416666666667	0.20296	-763.141652354017\\
58.0416666666667	0.20462	-795.236225947287\\
58.0416666666667	0.20628	-827.330799540557\\
58.0416666666667	0.20794	-859.425373133826\\
58.0416666666667	0.2096	-891.519946727097\\
58.0416666666667	0.21126	-923.614520320365\\
58.0416666666667	0.21292	-955.709093913634\\
58.0416666666667	0.21458	-987.803667506903\\
58.0416666666667	0.21624	-1019.89824110017\\
58.0416666666667	0.2179	-1051.99281469344\\
58.0416666666667	0.21956	-1084.08738828671\\
58.0416666666667	0.22122	-1116.18196187998\\
58.0416666666667	0.22288	-1148.27653547325\\
58.0416666666667	0.22454	-1180.37110906652\\
58.0416666666667	0.2262	-1212.46568265979\\
58.0416666666667	0.22786	-1244.56025625306\\
58.0416666666667	0.22952	-1276.65482984633\\
58.0416666666667	0.23118	-1308.7494034396\\
58.0416666666667	0.23284	-1340.84397703287\\
58.0416666666667	0.2345	-1372.93855062614\\
58.0416666666667	0.23616	-1405.03312421941\\
58.0416666666667	0.23782	-1437.12769781268\\
58.0416666666667	0.23948	-1469.22227140595\\
58.0416666666667	0.24114	-1501.31684499922\\
58.0416666666667	0.2428	-1533.41141859248\\
58.0416666666667	0.24446	-1565.50599218575\\
58.0416666666667	0.24612	-1597.60056577902\\
58.0416666666667	0.24778	-1629.69513937229\\
58.0416666666667	0.24944	-1661.78971296556\\
58.0416666666667	0.2511	-1693.88428655883\\
58.0416666666667	0.25276	-1725.9788601521\\
58.0416666666667	0.25442	-1758.07343374537\\
58.0416666666667	0.25608	-1790.16800733864\\
58.0416666666667	0.25774	-1822.26258093191\\
58.0416666666667	0.2594	-1854.35715452518\\
58.0416666666667	0.26106	-1886.45172811845\\
58.0416666666667	0.26272	-1918.54630171172\\
58.0416666666667	0.26438	-1950.64087530499\\
58.0416666666667	0.26604	-1982.73544889826\\
58.0416666666667	0.2677	-2014.83002249153\\
58.0416666666667	0.26936	-2046.9245960848\\
58.0416666666667	0.27102	-2079.01916967807\\
58.0416666666667	0.27268	-2111.11374327134\\
58.0416666666667	0.27434	-2143.2083168646\\
58.0416666666667	0.276	-2175.30289045787\\
58.25	0.193	-565.086321937375\\
58.25	0.19466	-597.053343430185\\
58.25	0.19632	-629.020364922993\\
58.25	0.19798	-660.987386415803\\
58.25	0.19964	-692.954407908612\\
58.25	0.2013	-724.921429401421\\
58.25	0.20296	-756.888450894229\\
58.25	0.20462	-788.855472387038\\
58.25	0.20628	-820.822493879848\\
58.25	0.20794	-852.789515372657\\
58.25	0.2096	-884.756536865467\\
58.25	0.21126	-916.723558358275\\
58.25	0.21292	-948.690579851084\\
58.25	0.21458	-980.657601343893\\
58.25	0.21624	-1012.6246228367\\
58.25	0.2179	-1044.59164432951\\
58.25	0.21956	-1076.55866582232\\
58.25	0.22122	-1108.52568731513\\
58.25	0.22288	-1140.49270880794\\
58.25	0.22454	-1172.45973030075\\
58.25	0.2262	-1204.42675179356\\
58.25	0.22786	-1236.39377328637\\
58.25	0.22952	-1268.36079477917\\
58.25	0.23118	-1300.32781627198\\
58.25	0.23284	-1332.29483776479\\
58.25	0.2345	-1364.2618592576\\
58.25	0.23616	-1396.22888075041\\
58.25	0.23782	-1428.19590224322\\
58.25	0.23948	-1460.16292373603\\
58.25	0.24114	-1492.12994522884\\
58.25	0.2428	-1524.09696672165\\
58.25	0.24446	-1556.06398821446\\
58.25	0.24612	-1588.03100970727\\
58.25	0.24778	-1619.99803120008\\
58.25	0.24944	-1651.96505269288\\
58.25	0.2511	-1683.93207418569\\
58.25	0.25276	-1715.8990956785\\
58.25	0.25442	-1747.86611717131\\
58.25	0.25608	-1779.83313866412\\
58.25	0.25774	-1811.80016015693\\
58.25	0.2594	-1843.76718164974\\
58.25	0.26106	-1875.73420314255\\
58.25	0.26272	-1907.70122463536\\
58.25	0.26438	-1939.66824612816\\
58.25	0.26604	-1971.63526762098\\
58.25	0.2677	-2003.60228911378\\
58.25	0.26936	-2035.56931060659\\
58.25	0.27102	-2067.5363320994\\
58.25	0.27268	-2099.50335359221\\
58.25	0.27434	-2131.47037508502\\
58.25	0.276	-2163.43739657783\\
58.4583333333333	0.193	-559.598433080349\\
58.4583333333333	0.19466	-591.4379024727\\
58.4583333333333	0.19632	-623.277371865048\\
58.4583333333333	0.19798	-655.116841257397\\
58.4583333333333	0.19964	-686.956310649745\\
58.4583333333333	0.2013	-718.795780042094\\
58.4583333333333	0.20296	-750.635249434442\\
58.4583333333333	0.20462	-782.474718826789\\
58.4583333333333	0.20628	-814.314188219139\\
58.4583333333333	0.20794	-846.153657611488\\
58.4583333333333	0.2096	-877.993127003837\\
58.4583333333333	0.21126	-909.832596396185\\
58.4583333333333	0.21292	-941.672065788533\\
58.4583333333333	0.21458	-973.511535180882\\
58.4583333333333	0.21624	-1005.35100457323\\
58.4583333333333	0.2179	-1037.19047396558\\
58.4583333333333	0.21956	-1069.02994335793\\
58.4583333333333	0.22122	-1100.86941275028\\
58.4583333333333	0.22288	-1132.70888214263\\
58.4583333333333	0.22454	-1164.54835153498\\
58.4583333333333	0.2262	-1196.38782092732\\
58.4583333333333	0.22786	-1228.22729031967\\
58.4583333333333	0.22952	-1260.06675971202\\
58.4583333333333	0.23118	-1291.90622910437\\
58.4583333333333	0.23284	-1323.74569849672\\
58.4583333333333	0.2345	-1355.58516788907\\
58.4583333333333	0.23616	-1387.42463728142\\
58.4583333333333	0.23782	-1419.26410667377\\
58.4583333333333	0.23948	-1451.10357606611\\
58.4583333333333	0.24114	-1482.94304545846\\
58.4583333333333	0.2428	-1514.78251485081\\
58.4583333333333	0.24446	-1546.62198424316\\
58.4583333333333	0.24612	-1578.46145363551\\
58.4583333333333	0.24778	-1610.30092302786\\
58.4583333333333	0.24944	-1642.1403924202\\
58.4583333333333	0.2511	-1673.97986181255\\
58.4583333333333	0.25276	-1705.8193312049\\
58.4583333333333	0.25442	-1737.65880059725\\
58.4583333333333	0.25608	-1769.4982699896\\
58.4583333333333	0.25774	-1801.33773938195\\
58.4583333333333	0.2594	-1833.1772087743\\
58.4583333333333	0.26106	-1865.01667816664\\
58.4583333333333	0.26272	-1896.85614755899\\
58.4583333333333	0.26438	-1928.69561695134\\
58.4583333333333	0.26604	-1960.53508634369\\
58.4583333333333	0.2677	-1992.37455573604\\
58.4583333333333	0.26936	-2024.21402512839\\
58.4583333333333	0.27102	-2056.05349452074\\
58.4583333333333	0.27268	-2087.89296391309\\
58.4583333333333	0.27434	-2119.73243330543\\
58.4583333333333	0.276	-2151.57190269778\\
58.6666666666667	0.193	-554.110544223326\\
58.6666666666667	0.19466	-585.822461515214\\
58.6666666666667	0.19632	-617.534378807102\\
58.6666666666667	0.19798	-649.246296098991\\
58.6666666666667	0.19964	-680.958213390879\\
58.6666666666667	0.2013	-712.670130682767\\
58.6666666666667	0.20296	-744.382047974656\\
58.6666666666667	0.20462	-776.093965266543\\
58.6666666666667	0.20628	-807.805882558432\\
58.6666666666667	0.20794	-839.51779985032\\
58.6666666666667	0.2096	-871.22971714221\\
58.6666666666667	0.21126	-902.941634434097\\
58.6666666666667	0.21292	-934.653551725984\\
58.6666666666667	0.21458	-966.365469017873\\
58.6666666666667	0.21624	-998.077386309762\\
58.6666666666667	0.2179	-1029.78930360165\\
58.6666666666667	0.21956	-1061.50122089354\\
58.6666666666667	0.22122	-1093.21313818543\\
58.6666666666667	0.22288	-1124.92505547732\\
58.6666666666667	0.22454	-1156.6369727692\\
58.6666666666667	0.2262	-1188.34889006109\\
58.6666666666667	0.22786	-1220.06080735298\\
58.6666666666667	0.22952	-1251.77272464487\\
58.6666666666667	0.23118	-1283.48464193676\\
58.6666666666667	0.23284	-1315.19655922864\\
58.6666666666667	0.2345	-1346.90847652053\\
58.6666666666667	0.23616	-1378.62039381242\\
58.6666666666667	0.23782	-1410.33231110431\\
58.6666666666667	0.23948	-1442.0442283962\\
58.6666666666667	0.24114	-1473.75614568809\\
58.6666666666667	0.2428	-1505.46806297997\\
58.6666666666667	0.24446	-1537.17998027186\\
58.6666666666667	0.24612	-1568.89189756375\\
58.6666666666667	0.24778	-1600.60381485564\\
58.6666666666667	0.24944	-1632.31573214753\\
58.6666666666667	0.2511	-1664.02764943941\\
58.6666666666667	0.25276	-1695.7395667313\\
58.6666666666667	0.25442	-1727.45148402319\\
58.6666666666667	0.25608	-1759.16340131508\\
58.6666666666667	0.25774	-1790.87531860697\\
58.6666666666667	0.2594	-1822.58723589886\\
58.6666666666667	0.26106	-1854.29915319075\\
58.6666666666667	0.26272	-1886.01107048263\\
58.6666666666667	0.26438	-1917.72298777452\\
58.6666666666667	0.26604	-1949.43490506641\\
58.6666666666667	0.2677	-1981.1468223583\\
58.6666666666667	0.26936	-2012.85873965019\\
58.6666666666667	0.27102	-2044.57065694208\\
58.6666666666667	0.27268	-2076.28257423396\\
58.6666666666667	0.27434	-2107.99449152585\\
58.6666666666667	0.276	-2139.70640881774\\
58.875	0.193	-548.622655366301\\
58.875	0.19466	-580.207020557728\\
58.875	0.19632	-611.791385749156\\
58.875	0.19798	-643.375750940584\\
58.875	0.19964	-674.960116132012\\
58.875	0.2013	-706.544481323441\\
58.875	0.20296	-738.128846514868\\
58.875	0.20462	-769.713211706296\\
58.875	0.20628	-801.297576897724\\
58.875	0.20794	-832.881942089151\\
58.875	0.2096	-864.466307280581\\
58.875	0.21126	-896.050672472008\\
58.875	0.21292	-927.635037663435\\
58.875	0.21458	-959.219402854863\\
58.875	0.21624	-990.803768046291\\
58.875	0.2179	-1022.38813323772\\
58.875	0.21956	-1053.97249842915\\
58.875	0.22122	-1085.55686362058\\
58.875	0.22288	-1117.141228812\\
58.875	0.22454	-1148.72559400343\\
58.875	0.2262	-1180.30995919486\\
58.875	0.22786	-1211.89432438629\\
58.875	0.22952	-1243.47868957771\\
58.875	0.23118	-1275.06305476914\\
58.875	0.23284	-1306.64741996057\\
58.875	0.2345	-1338.231785152\\
58.875	0.23616	-1369.81615034343\\
58.875	0.23782	-1401.40051553486\\
58.875	0.23948	-1432.98488072628\\
58.875	0.24114	-1464.56924591771\\
58.875	0.2428	-1496.15361110914\\
58.875	0.24446	-1527.73797630057\\
58.875	0.24612	-1559.32234149199\\
58.875	0.24778	-1590.90670668342\\
58.875	0.24944	-1622.49107187485\\
58.875	0.2511	-1654.07543706628\\
58.875	0.25276	-1685.65980225771\\
58.875	0.25442	-1717.24416744913\\
58.875	0.25608	-1748.82853264056\\
58.875	0.25774	-1780.41289783199\\
58.875	0.2594	-1811.99726302342\\
58.875	0.26106	-1843.58162821484\\
58.875	0.26272	-1875.16599340627\\
58.875	0.26438	-1906.7503585977\\
58.875	0.26604	-1938.33472378913\\
58.875	0.2677	-1969.91908898056\\
58.875	0.26936	-2001.50345417198\\
58.875	0.27102	-2033.08781936341\\
58.875	0.27268	-2064.67218455484\\
58.875	0.27434	-2096.25654974627\\
58.875	0.276	-2127.8409149377\\
59.0833333333333	0.193	-543.134766509275\\
59.0833333333333	0.19466	-574.591579600244\\
59.0833333333333	0.19632	-606.04839269121\\
59.0833333333333	0.19798	-637.505205782179\\
59.0833333333333	0.19964	-668.962018873146\\
59.0833333333333	0.2013	-700.418831964113\\
59.0833333333333	0.20296	-731.875645055081\\
59.0833333333333	0.20462	-763.332458146047\\
59.0833333333333	0.20628	-794.789271237016\\
59.0833333333333	0.20794	-826.246084327983\\
59.0833333333333	0.2096	-857.702897418952\\
59.0833333333333	0.21126	-889.159710509919\\
59.0833333333333	0.21292	-920.616523600886\\
59.0833333333333	0.21458	-952.073336691853\\
59.0833333333333	0.21624	-983.530149782821\\
59.0833333333333	0.2179	-1014.98696287379\\
59.0833333333333	0.21956	-1046.44377596476\\
59.0833333333333	0.22122	-1077.90058905572\\
59.0833333333333	0.22288	-1109.35740214669\\
59.0833333333333	0.22454	-1140.81421523766\\
59.0833333333333	0.2262	-1172.27102832863\\
59.0833333333333	0.22786	-1203.72784141959\\
59.0833333333333	0.22952	-1235.18465451056\\
59.0833333333333	0.23118	-1266.64146760153\\
59.0833333333333	0.23284	-1298.0982806925\\
59.0833333333333	0.2345	-1329.55509378346\\
59.0833333333333	0.23616	-1361.01190687443\\
59.0833333333333	0.23782	-1392.4687199654\\
59.0833333333333	0.23948	-1423.92553305637\\
59.0833333333333	0.24114	-1455.38234614733\\
59.0833333333333	0.2428	-1486.8391592383\\
59.0833333333333	0.24446	-1518.29597232927\\
59.0833333333333	0.24612	-1549.75278542024\\
59.0833333333333	0.24778	-1581.2095985112\\
59.0833333333333	0.24944	-1612.66641160217\\
59.0833333333333	0.2511	-1644.12322469314\\
59.0833333333333	0.25276	-1675.58003778411\\
59.0833333333333	0.25442	-1707.03685087507\\
59.0833333333333	0.25608	-1738.49366396604\\
59.0833333333333	0.25774	-1769.95047705701\\
59.0833333333333	0.2594	-1801.40729014798\\
59.0833333333333	0.26106	-1832.86410323894\\
59.0833333333333	0.26272	-1864.32091632991\\
59.0833333333333	0.26438	-1895.77772942088\\
59.0833333333333	0.26604	-1927.23454251185\\
59.0833333333333	0.2677	-1958.69135560281\\
59.0833333333333	0.26936	-1990.14816869378\\
59.0833333333333	0.27102	-2021.60498178475\\
59.0833333333333	0.27268	-2053.06179487572\\
59.0833333333333	0.27434	-2084.51860796668\\
59.0833333333333	0.276	-2115.97542105765\\
59.2916666666667	0.193	-537.64687765225\\
59.2916666666667	0.19466	-568.976138642758\\
59.2916666666667	0.19632	-600.305399633265\\
59.2916666666667	0.19798	-631.634660623773\\
59.2916666666667	0.19964	-662.96392161428\\
59.2916666666667	0.2013	-694.293182604786\\
59.2916666666667	0.20296	-725.622443595293\\
59.2916666666667	0.20462	-756.951704585799\\
59.2916666666667	0.20628	-788.280965576307\\
59.2916666666667	0.20794	-819.610226566813\\
59.2916666666667	0.2096	-850.939487557323\\
59.2916666666667	0.21126	-882.268748547829\\
59.2916666666667	0.21292	-913.598009538336\\
59.2916666666667	0.21458	-944.927270528842\\
59.2916666666667	0.21624	-976.25653151935\\
59.2916666666667	0.2179	-1007.58579250986\\
59.2916666666667	0.21956	-1038.91505350036\\
59.2916666666667	0.22122	-1070.24431449087\\
59.2916666666667	0.22288	-1101.57357548138\\
59.2916666666667	0.22454	-1132.90283647189\\
59.2916666666667	0.2262	-1164.23209746239\\
59.2916666666667	0.22786	-1195.5613584529\\
59.2916666666667	0.22952	-1226.89061944341\\
59.2916666666667	0.23118	-1258.21988043391\\
59.2916666666667	0.23284	-1289.54914142442\\
59.2916666666667	0.2345	-1320.87840241493\\
59.2916666666667	0.23616	-1352.20766340544\\
59.2916666666667	0.23782	-1383.53692439594\\
59.2916666666667	0.23948	-1414.86618538645\\
59.2916666666667	0.24114	-1446.19544637696\\
59.2916666666667	0.2428	-1477.52470736746\\
59.2916666666667	0.24446	-1508.85396835797\\
59.2916666666667	0.24612	-1540.18322934848\\
59.2916666666667	0.24778	-1571.51249033899\\
59.2916666666667	0.24944	-1602.84175132949\\
59.2916666666667	0.2511	-1634.17101232\\
59.2916666666667	0.25276	-1665.50027331051\\
59.2916666666667	0.25442	-1696.82953430101\\
59.2916666666667	0.25608	-1728.15879529152\\
59.2916666666667	0.25774	-1759.48805628203\\
59.2916666666667	0.2594	-1790.81731727254\\
59.2916666666667	0.26106	-1822.14657826304\\
59.2916666666667	0.26272	-1853.47583925355\\
59.2916666666667	0.26438	-1884.80510024406\\
59.2916666666667	0.26604	-1916.13436123456\\
59.2916666666667	0.2677	-1947.46362222507\\
59.2916666666667	0.26936	-1978.79288321558\\
59.2916666666667	0.27102	-2010.12214420609\\
59.2916666666667	0.27268	-2041.45140519659\\
59.2916666666667	0.27434	-2072.7806661871\\
59.2916666666667	0.276	-2104.10992717761\\
59.5	0.193	-532.158988795226\\
59.5	0.19466	-563.360697685272\\
59.5	0.19632	-594.562406575319\\
59.5	0.19798	-625.764115465367\\
59.5	0.19964	-656.965824355413\\
59.5	0.2013	-688.16753324546\\
59.5	0.20296	-719.369242135506\\
59.5	0.20462	-750.570951025553\\
59.5	0.20628	-781.7726599156\\
59.5	0.20794	-812.974368805646\\
59.5	0.2096	-844.176077695694\\
59.5	0.21126	-875.37778658574\\
59.5	0.21292	-906.579495475787\\
59.5	0.21458	-937.781204365833\\
59.5	0.21624	-968.982913255881\\
59.5	0.2179	-1000.18462214593\\
59.5	0.21956	-1031.38633103597\\
59.5	0.22122	-1062.58803992602\\
59.5	0.22288	-1093.78974881607\\
59.5	0.22454	-1124.99145770611\\
59.5	0.2262	-1156.19316659616\\
59.5	0.22786	-1187.39487548621\\
59.5	0.22952	-1218.59658437625\\
59.5	0.23118	-1249.7982932663\\
59.5	0.23284	-1281.00000215635\\
59.5	0.2345	-1312.20171104639\\
59.5	0.23616	-1343.40341993644\\
59.5	0.23782	-1374.60512882649\\
59.5	0.23948	-1405.80683771654\\
59.5	0.24114	-1437.00854660658\\
59.5	0.2428	-1468.21025549663\\
59.5	0.24446	-1499.41196438668\\
59.5	0.24612	-1530.61367327672\\
59.5	0.24778	-1561.81538216677\\
59.5	0.24944	-1593.01709105682\\
59.5	0.2511	-1624.21879994686\\
59.5	0.25276	-1655.42050883691\\
59.5	0.25442	-1686.62221772696\\
59.5	0.25608	-1717.823926617\\
59.5	0.25774	-1749.02563550705\\
59.5	0.2594	-1780.2273443971\\
59.5	0.26106	-1811.42905328714\\
59.5	0.26272	-1842.63076217719\\
59.5	0.26438	-1873.83247106724\\
59.5	0.26604	-1905.03417995728\\
59.5	0.2677	-1936.23588884733\\
59.5	0.26936	-1967.43759773738\\
59.5	0.27102	-1998.63930662742\\
59.5	0.27268	-2029.84101551747\\
59.5	0.27434	-2061.04272440752\\
59.5	0.276	-2092.24443329756\\
59.7083333333333	0.193	-526.671099938201\\
59.7083333333333	0.19466	-557.745256727787\\
59.7083333333333	0.19632	-588.819413517373\\
59.7083333333333	0.19798	-619.89357030696\\
59.7083333333333	0.19964	-650.967727096546\\
59.7083333333333	0.2013	-682.041883886133\\
59.7083333333333	0.20296	-713.116040675719\\
59.7083333333333	0.20462	-744.190197465305\\
59.7083333333333	0.20628	-775.264354254892\\
59.7083333333333	0.20794	-806.338511044478\\
59.7083333333333	0.2096	-837.412667834065\\
59.7083333333333	0.21126	-868.486824623651\\
59.7083333333333	0.21292	-899.560981413238\\
59.7083333333333	0.21458	-930.635138202823\\
59.7083333333333	0.21624	-961.70929499241\\
59.7083333333333	0.2179	-992.783451781996\\
59.7083333333333	0.21956	-1023.85760857158\\
59.7083333333333	0.22122	-1054.93176536117\\
59.7083333333333	0.22288	-1086.00592215076\\
59.7083333333333	0.22454	-1117.08007894034\\
59.7083333333333	0.2262	-1148.15423572993\\
59.7083333333333	0.22786	-1179.22839251951\\
59.7083333333333	0.22952	-1210.3025493091\\
59.7083333333333	0.23118	-1241.37670609869\\
59.7083333333333	0.23284	-1272.45086288827\\
59.7083333333333	0.2345	-1303.52501967786\\
59.7083333333333	0.23616	-1334.59917646745\\
59.7083333333333	0.23782	-1365.67333325703\\
59.7083333333333	0.23948	-1396.74749004662\\
59.7083333333333	0.24114	-1427.82164683621\\
59.7083333333333	0.2428	-1458.89580362579\\
59.7083333333333	0.24446	-1489.96996041538\\
59.7083333333333	0.24612	-1521.04411720497\\
59.7083333333333	0.24778	-1552.11827399455\\
59.7083333333333	0.24944	-1583.19243078414\\
59.7083333333333	0.2511	-1614.26658757372\\
59.7083333333333	0.25276	-1645.34074436331\\
59.7083333333333	0.25442	-1676.4149011529\\
59.7083333333333	0.25608	-1707.48905794248\\
59.7083333333333	0.25774	-1738.56321473207\\
59.7083333333333	0.2594	-1769.63737152166\\
59.7083333333333	0.26106	-1800.71152831124\\
59.7083333333333	0.26272	-1831.78568510083\\
59.7083333333333	0.26438	-1862.85984189041\\
59.7083333333333	0.26604	-1893.93399868\\
59.7083333333333	0.2677	-1925.00815546959\\
59.7083333333333	0.26936	-1956.08231225917\\
59.7083333333333	0.27102	-1987.15646904876\\
59.7083333333333	0.27268	-2018.23062583835\\
59.7083333333333	0.27434	-2049.30478262793\\
59.7083333333333	0.276	-2080.37893941752\\
59.9166666666667	0.193	-521.183211081176\\
59.9166666666667	0.19466	-552.129815770301\\
59.9166666666667	0.19632	-583.076420459427\\
59.9166666666667	0.19798	-614.023025148554\\
59.9166666666667	0.19964	-644.969629837679\\
59.9166666666667	0.2013	-675.916234526806\\
59.9166666666667	0.20296	-706.862839215932\\
59.9166666666667	0.20462	-737.809443905057\\
59.9166666666667	0.20628	-768.756048594184\\
59.9166666666667	0.20794	-799.70265328331\\
59.9166666666667	0.2096	-830.649257972436\\
59.9166666666667	0.21126	-861.595862661562\\
59.9166666666667	0.21292	-892.542467350688\\
59.9166666666667	0.21458	-923.489072039813\\
59.9166666666667	0.21624	-954.43567672894\\
59.9166666666667	0.2179	-985.382281418065\\
59.9166666666667	0.21956	-1016.32888610719\\
59.9166666666667	0.22122	-1047.27549079632\\
59.9166666666667	0.22288	-1078.22209548544\\
59.9166666666667	0.22454	-1109.16870017457\\
59.9166666666667	0.2262	-1140.1153048637\\
59.9166666666667	0.22786	-1171.06190955282\\
59.9166666666667	0.22952	-1202.00851424195\\
59.9166666666667	0.23118	-1232.95511893107\\
59.9166666666667	0.23284	-1263.9017236202\\
59.9166666666667	0.2345	-1294.84832830933\\
59.9166666666667	0.23616	-1325.79493299845\\
59.9166666666667	0.23782	-1356.74153768758\\
59.9166666666667	0.23948	-1387.6881423767\\
59.9166666666667	0.24114	-1418.63474706583\\
59.9166666666667	0.2428	-1449.58135175495\\
59.9166666666667	0.24446	-1480.52795644408\\
59.9166666666667	0.24612	-1511.47456113321\\
59.9166666666667	0.24778	-1542.42116582233\\
59.9166666666667	0.24944	-1573.36777051146\\
59.9166666666667	0.2511	-1604.31437520058\\
59.9166666666667	0.25276	-1635.26097988971\\
59.9166666666667	0.25442	-1666.20758457884\\
59.9166666666667	0.25608	-1697.15418926796\\
59.9166666666667	0.25774	-1728.10079395709\\
59.9166666666667	0.2594	-1759.04739864622\\
59.9166666666667	0.26106	-1789.99400333534\\
59.9166666666667	0.26272	-1820.94060802447\\
59.9166666666667	0.26438	-1851.88721271359\\
59.9166666666667	0.26604	-1882.83381740272\\
59.9166666666667	0.2677	-1913.78042209185\\
59.9166666666667	0.26936	-1944.72702678097\\
59.9166666666667	0.27102	-1975.6736314701\\
59.9166666666667	0.27268	-2006.62023615922\\
59.9166666666667	0.27434	-2037.56684084835\\
59.9166666666667	0.276	-2068.51344553747\\
60.125	0.193	-515.69532222415\\
60.125	0.19466	-546.514374812816\\
60.125	0.19632	-577.333427401481\\
60.125	0.19798	-608.152479990147\\
60.125	0.19964	-638.971532578813\\
60.125	0.2013	-669.790585167479\\
60.125	0.20296	-700.609637756143\\
60.125	0.20462	-731.428690344809\\
60.125	0.20628	-762.247742933475\\
60.125	0.20794	-793.06679552214\\
60.125	0.2096	-823.885848110807\\
60.125	0.21126	-854.704900699472\\
60.125	0.21292	-885.523953288137\\
60.125	0.21458	-916.343005876803\\
60.125	0.21624	-947.162058465468\\
60.125	0.2179	-977.981111054134\\
60.125	0.21956	-1008.8001636428\\
60.125	0.22122	-1039.61921623147\\
60.125	0.22288	-1070.43826882013\\
60.125	0.22454	-1101.2573214088\\
60.125	0.2262	-1132.07637399746\\
60.125	0.22786	-1162.89542658613\\
60.125	0.22952	-1193.71447917479\\
60.125	0.23118	-1224.53353176346\\
60.125	0.23284	-1255.35258435212\\
60.125	0.2345	-1286.17163694079\\
60.125	0.23616	-1316.99068952946\\
60.125	0.23782	-1347.80974211812\\
60.125	0.23948	-1378.62879470679\\
60.125	0.24114	-1409.44784729545\\
60.125	0.2428	-1440.26689988412\\
60.125	0.24446	-1471.08595247278\\
60.125	0.24612	-1501.90500506145\\
60.125	0.24778	-1532.72405765012\\
60.125	0.24944	-1563.54311023878\\
60.125	0.2511	-1594.36216282744\\
60.125	0.25276	-1625.18121541611\\
60.125	0.25442	-1656.00026800478\\
60.125	0.25608	-1686.81932059344\\
60.125	0.25774	-1717.63837318211\\
60.125	0.2594	-1748.45742577077\\
60.125	0.26106	-1779.27647835944\\
60.125	0.26272	-1810.0955309481\\
60.125	0.26438	-1840.91458353677\\
60.125	0.26604	-1871.73363612544\\
60.125	0.2677	-1902.5526887141\\
60.125	0.26936	-1933.37174130277\\
60.125	0.27102	-1964.19079389143\\
60.125	0.27268	-1995.0098464801\\
60.125	0.27434	-2025.82889906876\\
60.125	0.276	-2056.64795165743\\
60.3333333333333	0.193	-510.207433367125\\
60.3333333333333	0.19466	-540.898933855331\\
60.3333333333333	0.19632	-571.590434343536\\
60.3333333333333	0.19798	-602.281934831742\\
60.3333333333333	0.19964	-632.973435319946\\
60.3333333333333	0.2013	-663.664935808152\\
60.3333333333333	0.20296	-694.356436296356\\
60.3333333333333	0.20462	-725.047936784561\\
60.3333333333333	0.20628	-755.739437272767\\
60.3333333333333	0.20794	-786.430937760972\\
60.3333333333333	0.2096	-817.122438249178\\
60.3333333333333	0.21126	-847.813938737383\\
60.3333333333333	0.21292	-878.505439225588\\
60.3333333333333	0.21458	-909.196939713793\\
60.3333333333333	0.21624	-939.888440201998\\
60.3333333333333	0.2179	-970.579940690203\\
60.3333333333333	0.21956	-1001.27144117841\\
60.3333333333333	0.22122	-1031.96294166661\\
60.3333333333333	0.22288	-1062.65444215482\\
60.3333333333333	0.22454	-1093.34594264302\\
60.3333333333333	0.2262	-1124.03744313123\\
60.3333333333333	0.22786	-1154.72894361943\\
60.3333333333333	0.22952	-1185.42044410764\\
60.3333333333333	0.23118	-1216.11194459584\\
60.3333333333333	0.23284	-1246.80344508405\\
60.3333333333333	0.2345	-1277.49494557226\\
60.3333333333333	0.23616	-1308.18644606046\\
60.3333333333333	0.23782	-1338.87794654867\\
60.3333333333333	0.23948	-1369.56944703687\\
60.3333333333333	0.24114	-1400.26094752508\\
60.3333333333333	0.2428	-1430.95244801328\\
60.3333333333333	0.24446	-1461.64394850149\\
60.3333333333333	0.24612	-1492.33544898969\\
60.3333333333333	0.24778	-1523.0269494779\\
60.3333333333333	0.24944	-1553.7184499661\\
60.3333333333333	0.2511	-1584.40995045431\\
60.3333333333333	0.25276	-1615.10145094251\\
60.3333333333333	0.25442	-1645.79295143072\\
60.3333333333333	0.25608	-1676.48445191892\\
60.3333333333333	0.25774	-1707.17595240713\\
60.3333333333333	0.2594	-1737.86745289533\\
60.3333333333333	0.26106	-1768.55895338354\\
60.3333333333333	0.26272	-1799.25045387174\\
60.3333333333333	0.26438	-1829.94195435995\\
60.3333333333333	0.26604	-1860.63345484815\\
60.3333333333333	0.2677	-1891.32495533636\\
60.3333333333333	0.26936	-1922.01645582456\\
60.3333333333333	0.27102	-1952.70795631277\\
60.3333333333333	0.27268	-1983.39945680098\\
60.3333333333333	0.27434	-2014.09095728918\\
60.3333333333333	0.276	-2044.78245777739\\
60.5416666666667	0.193	-504.719544510101\\
60.5416666666667	0.19466	-535.283492897845\\
60.5416666666667	0.19632	-565.847441285589\\
60.5416666666667	0.19798	-596.411389673335\\
60.5416666666667	0.19964	-626.97533806108\\
60.5416666666667	0.2013	-657.539286448825\\
60.5416666666667	0.20296	-688.103234836569\\
60.5416666666667	0.20462	-718.667183224314\\
60.5416666666667	0.20628	-749.231131612059\\
60.5416666666667	0.20794	-779.795079999803\\
60.5416666666667	0.2096	-810.359028387549\\
60.5416666666667	0.21126	-840.922976775294\\
60.5416666666667	0.21292	-871.486925163038\\
60.5416666666667	0.21458	-902.050873550783\\
60.5416666666667	0.21624	-932.614821938528\\
60.5416666666667	0.2179	-963.178770326273\\
60.5416666666667	0.21956	-993.742718714017\\
60.5416666666667	0.22122	-1024.30666710176\\
60.5416666666667	0.22288	-1054.87061548951\\
60.5416666666667	0.22454	-1085.43456387725\\
60.5416666666667	0.2262	-1115.998512265\\
60.5416666666667	0.22786	-1146.56246065274\\
60.5416666666667	0.22952	-1177.12640904049\\
60.5416666666667	0.23118	-1207.69035742823\\
60.5416666666667	0.23284	-1238.25430581598\\
60.5416666666667	0.2345	-1268.81825420372\\
60.5416666666667	0.23616	-1299.38220259147\\
60.5416666666667	0.23782	-1329.94615097921\\
60.5416666666667	0.23948	-1360.51009936696\\
60.5416666666667	0.24114	-1391.0740477547\\
60.5416666666667	0.2428	-1421.63799614244\\
60.5416666666667	0.24446	-1452.20194453019\\
60.5416666666667	0.24612	-1482.76589291793\\
60.5416666666667	0.24778	-1513.32984130568\\
60.5416666666667	0.24944	-1543.89378969342\\
60.5416666666667	0.2511	-1574.45773808117\\
60.5416666666667	0.25276	-1605.02168646891\\
60.5416666666667	0.25442	-1635.58563485666\\
60.5416666666667	0.25608	-1666.1495832444\\
60.5416666666667	0.25774	-1696.71353163215\\
60.5416666666667	0.2594	-1727.27748001989\\
60.5416666666667	0.26106	-1757.84142840764\\
60.5416666666667	0.26272	-1788.40537679538\\
60.5416666666667	0.26438	-1818.96932518313\\
60.5416666666667	0.26604	-1849.53327357087\\
60.5416666666667	0.2677	-1880.09722195862\\
60.5416666666667	0.26936	-1910.66117034636\\
60.5416666666667	0.27102	-1941.22511873411\\
60.5416666666667	0.27268	-1971.78906712185\\
60.5416666666667	0.27434	-2002.3530155096\\
60.5416666666667	0.276	-2032.91696389734\\
60.75	0.193	-499.231655653075\\
60.75	0.19466	-529.668051940359\\
60.75	0.19632	-560.104448227643\\
60.75	0.19798	-590.540844514929\\
60.75	0.19964	-620.977240802213\\
60.75	0.2013	-651.413637089498\\
60.75	0.20296	-681.850033376782\\
60.75	0.20462	-712.286429664066\\
60.75	0.20628	-742.722825951351\\
60.75	0.20794	-773.159222238635\\
60.75	0.2096	-803.595618525921\\
60.75	0.21126	-834.032014813205\\
60.75	0.21292	-864.468411100489\\
60.75	0.21458	-894.904807387773\\
60.75	0.21624	-925.341203675058\\
60.75	0.2179	-955.777599962342\\
60.75	0.21956	-986.213996249626\\
60.75	0.22122	-1016.65039253691\\
60.75	0.22288	-1047.0867888242\\
60.75	0.22454	-1077.52318511148\\
60.75	0.2262	-1107.95958139876\\
60.75	0.22786	-1138.39597768605\\
60.75	0.22952	-1168.83237397333\\
60.75	0.23118	-1199.26877026062\\
60.75	0.23284	-1229.7051665479\\
60.75	0.2345	-1260.14156283519\\
60.75	0.23616	-1290.57795912247\\
60.75	0.23782	-1321.01435540976\\
60.75	0.23948	-1351.45075169704\\
60.75	0.24114	-1381.88714798432\\
60.75	0.2428	-1412.32354427161\\
60.75	0.24446	-1442.75994055889\\
60.75	0.24612	-1473.19633684618\\
60.75	0.24778	-1503.63273313346\\
60.75	0.24944	-1534.06912942075\\
60.75	0.2511	-1564.50552570803\\
60.75	0.25276	-1594.94192199532\\
60.75	0.25442	-1625.3783182826\\
60.75	0.25608	-1655.81471456988\\
60.75	0.25774	-1686.25111085717\\
60.75	0.2594	-1716.68750714445\\
60.75	0.26106	-1747.12390343174\\
60.75	0.26272	-1777.56029971902\\
60.75	0.26438	-1807.9966960063\\
60.75	0.26604	-1838.43309229359\\
60.75	0.2677	-1868.86948858088\\
60.75	0.26936	-1899.30588486816\\
60.75	0.27102	-1929.74228115545\\
60.75	0.27268	-1960.17867744273\\
60.75	0.27434	-1990.61507373001\\
60.75	0.276	-2021.0514700173\\
60.9583333333333	0.193	-493.743766796049\\
60.9583333333333	0.19466	-524.052610982874\\
60.9583333333333	0.19632	-554.361455169697\\
60.9583333333333	0.19798	-584.670299356522\\
60.9583333333333	0.19964	-614.979143543346\\
60.9583333333333	0.2013	-645.28798773017\\
60.9583333333333	0.20296	-675.596831916994\\
60.9583333333333	0.20462	-705.905676103817\\
60.9583333333333	0.20628	-736.214520290642\\
60.9583333333333	0.20794	-766.523364477466\\
60.9583333333333	0.2096	-796.832208664291\\
60.9583333333333	0.21126	-827.141052851115\\
60.9583333333333	0.21292	-857.449897037938\\
60.9583333333333	0.21458	-887.758741224762\\
60.9583333333333	0.21624	-918.067585411586\\
60.9583333333333	0.2179	-948.37642959841\\
60.9583333333333	0.21956	-978.685273785233\\
60.9583333333333	0.22122	-1008.99411797206\\
60.9583333333333	0.22288	-1039.30296215888\\
60.9583333333333	0.22454	-1069.61180634571\\
60.9583333333333	0.2262	-1099.92065053253\\
60.9583333333333	0.22786	-1130.22949471935\\
60.9583333333333	0.22952	-1160.53833890618\\
60.9583333333333	0.23118	-1190.847183093\\
60.9583333333333	0.23284	-1221.15602727983\\
60.9583333333333	0.2345	-1251.46487146665\\
60.9583333333333	0.23616	-1281.77371565347\\
60.9583333333333	0.23782	-1312.0825598403\\
60.9583333333333	0.23948	-1342.39140402712\\
60.9583333333333	0.24114	-1372.70024821395\\
60.9583333333333	0.2428	-1403.00909240077\\
60.9583333333333	0.24446	-1433.3179365876\\
60.9583333333333	0.24612	-1463.62678077442\\
60.9583333333333	0.24778	-1493.93562496124\\
60.9583333333333	0.24944	-1524.24446914807\\
60.9583333333333	0.2511	-1554.55331333489\\
60.9583333333333	0.25276	-1584.86215752171\\
60.9583333333333	0.25442	-1615.17100170854\\
60.9583333333333	0.25608	-1645.47984589536\\
60.9583333333333	0.25774	-1675.78869008219\\
60.9583333333333	0.2594	-1706.09753426901\\
60.9583333333333	0.26106	-1736.40637845584\\
60.9583333333333	0.26272	-1766.71522264266\\
60.9583333333333	0.26438	-1797.02406682948\\
60.9583333333333	0.26604	-1827.33291101631\\
60.9583333333333	0.2677	-1857.64175520313\\
60.9583333333333	0.26936	-1887.95059938996\\
60.9583333333333	0.27102	-1918.25944357678\\
60.9583333333333	0.27268	-1948.5682877636\\
60.9583333333333	0.27434	-1978.87713195043\\
60.9583333333333	0.276	-2009.18597613725\\
61.1666666666667	0.193	-488.255877939025\\
61.1666666666667	0.19466	-518.43717002539\\
61.1666666666667	0.19632	-548.618462111753\\
61.1666666666667	0.19798	-578.799754198117\\
61.1666666666667	0.19964	-608.981046284481\\
61.1666666666667	0.2013	-639.162338370844\\
61.1666666666667	0.20296	-669.343630457207\\
61.1666666666667	0.20462	-699.52492254357\\
61.1666666666667	0.20628	-729.706214629934\\
61.1666666666667	0.20794	-759.887506716297\\
61.1666666666667	0.2096	-790.068798802662\\
61.1666666666667	0.21126	-820.250090889025\\
61.1666666666667	0.21292	-850.431382975388\\
61.1666666666667	0.21458	-880.612675061751\\
61.1666666666667	0.21624	-910.793967148116\\
61.1666666666667	0.2179	-940.975259234479\\
61.1666666666667	0.21956	-971.156551320842\\
61.1666666666667	0.22122	-1001.33784340721\\
61.1666666666667	0.22288	-1031.51913549357\\
61.1666666666667	0.22454	-1061.70042757993\\
61.1666666666667	0.2262	-1091.8817196663\\
61.1666666666667	0.22786	-1122.06301175266\\
61.1666666666667	0.22952	-1152.24430383902\\
61.1666666666667	0.23118	-1182.42559592539\\
61.1666666666667	0.23284	-1212.60688801175\\
61.1666666666667	0.2345	-1242.78818009812\\
61.1666666666667	0.23616	-1272.96947218448\\
61.1666666666667	0.23782	-1303.15076427084\\
61.1666666666667	0.23948	-1333.33205635721\\
61.1666666666667	0.24114	-1363.51334844357\\
61.1666666666667	0.2428	-1393.69464052993\\
61.1666666666667	0.24446	-1423.8759326163\\
61.1666666666667	0.24612	-1454.05722470266\\
61.1666666666667	0.24778	-1484.23851678903\\
61.1666666666667	0.24944	-1514.41980887539\\
61.1666666666667	0.2511	-1544.60110096175\\
61.1666666666667	0.25276	-1574.78239304812\\
61.1666666666667	0.25442	-1604.96368513448\\
61.1666666666667	0.25608	-1635.14497722084\\
61.1666666666667	0.25774	-1665.32626930721\\
61.1666666666667	0.2594	-1695.50756139357\\
61.1666666666667	0.26106	-1725.68885347993\\
61.1666666666667	0.26272	-1755.8701455663\\
61.1666666666667	0.26438	-1786.05143765266\\
61.1666666666667	0.26604	-1816.23272973903\\
61.1666666666667	0.2677	-1846.41402182539\\
61.1666666666667	0.26936	-1876.59531391175\\
61.1666666666667	0.27102	-1906.77660599812\\
61.1666666666667	0.27268	-1936.95789808448\\
61.1666666666667	0.27434	-1967.13919017084\\
61.1666666666667	0.276	-1997.32048225721\\
61.375	0.193	-482.767989082\\
61.375	0.19466	-512.821729067903\\
61.375	0.19632	-542.875469053806\\
61.375	0.19798	-572.92920903971\\
61.375	0.19964	-602.982949025613\\
61.375	0.2013	-633.036689011517\\
61.375	0.20296	-663.09042899742\\
61.375	0.20462	-693.144168983322\\
61.375	0.20628	-723.197908969226\\
61.375	0.20794	-753.251648955129\\
61.375	0.2096	-783.305388941034\\
61.375	0.21126	-813.359128926936\\
61.375	0.21292	-843.412868912839\\
61.375	0.21458	-873.466608898741\\
61.375	0.21624	-903.520348884646\\
61.375	0.2179	-933.574088870549\\
61.375	0.21956	-963.627828856452\\
61.375	0.22122	-993.681568842356\\
61.375	0.22288	-1023.73530882826\\
61.375	0.22454	-1053.78904881416\\
61.375	0.2262	-1083.84278880007\\
61.375	0.22786	-1113.89652878597\\
61.375	0.22952	-1143.95026877187\\
61.375	0.23118	-1174.00400875778\\
61.375	0.23284	-1204.05774874368\\
61.375	0.2345	-1234.11148872958\\
61.375	0.23616	-1264.16522871548\\
61.375	0.23782	-1294.21896870139\\
61.375	0.23948	-1324.27270868729\\
61.375	0.24114	-1354.3264486732\\
61.375	0.2428	-1384.3801886591\\
61.375	0.24446	-1414.433928645\\
61.375	0.24612	-1444.4876686309\\
61.375	0.24778	-1474.54140861681\\
61.375	0.24944	-1504.59514860271\\
61.375	0.2511	-1534.64888858861\\
61.375	0.25276	-1564.70262857452\\
61.375	0.25442	-1594.75636856042\\
61.375	0.25608	-1624.81010854632\\
61.375	0.25774	-1654.86384853223\\
61.375	0.2594	-1684.91758851813\\
61.375	0.26106	-1714.97132850403\\
61.375	0.26272	-1745.02506848994\\
61.375	0.26438	-1775.07880847584\\
61.375	0.26604	-1805.13254846174\\
61.375	0.2677	-1835.18628844765\\
61.375	0.26936	-1865.24002843355\\
61.375	0.27102	-1895.29376841945\\
61.375	0.27268	-1925.34750840536\\
61.375	0.27434	-1955.40124839126\\
61.375	0.276	-1985.45498837716\\
61.5833333333333	0.193	-477.280100224976\\
61.5833333333333	0.19466	-507.206288110418\\
61.5833333333333	0.19632	-537.132475995861\\
61.5833333333333	0.19798	-567.058663881304\\
61.5833333333333	0.19964	-596.984851766746\\
61.5833333333333	0.2013	-626.911039652191\\
61.5833333333333	0.20296	-656.837227537633\\
61.5833333333333	0.20462	-686.763415423075\\
61.5833333333333	0.20628	-716.689603308519\\
61.5833333333333	0.20794	-746.615791193962\\
61.5833333333333	0.2096	-776.541979079406\\
61.5833333333333	0.21126	-806.468166964848\\
61.5833333333333	0.21292	-836.39435485029\\
61.5833333333333	0.21458	-866.320542735732\\
61.5833333333333	0.21624	-896.246730621177\\
61.5833333333333	0.2179	-926.172918506619\\
61.5833333333333	0.21956	-956.099106392061\\
61.5833333333333	0.22122	-986.025294277505\\
61.5833333333333	0.22288	-1015.95148216295\\
61.5833333333333	0.22454	-1045.87767004839\\
61.5833333333333	0.2262	-1075.80385793383\\
61.5833333333333	0.22786	-1105.73004581928\\
61.5833333333333	0.22952	-1135.65623370472\\
61.5833333333333	0.23118	-1165.58242159016\\
61.5833333333333	0.23284	-1195.5086094756\\
61.5833333333333	0.2345	-1225.43479736105\\
61.5833333333333	0.23616	-1255.36098524649\\
61.5833333333333	0.23782	-1285.28717313193\\
61.5833333333333	0.23948	-1315.21336101738\\
61.5833333333333	0.24114	-1345.13954890282\\
61.5833333333333	0.2428	-1375.06573678826\\
61.5833333333333	0.24446	-1404.99192467371\\
61.5833333333333	0.24612	-1434.91811255915\\
61.5833333333333	0.24778	-1464.84430044459\\
61.5833333333333	0.24944	-1494.77048833003\\
61.5833333333333	0.2511	-1524.69667621548\\
61.5833333333333	0.25276	-1554.62286410092\\
61.5833333333333	0.25442	-1584.54905198636\\
61.5833333333333	0.25608	-1614.4752398718\\
61.5833333333333	0.25774	-1644.40142775725\\
61.5833333333333	0.2594	-1674.32761564269\\
61.5833333333333	0.26106	-1704.25380352813\\
61.5833333333333	0.26272	-1734.17999141358\\
61.5833333333333	0.26438	-1764.10617929902\\
61.5833333333333	0.26604	-1794.03236718446\\
61.5833333333333	0.2677	-1823.95855506991\\
61.5833333333333	0.26936	-1853.88474295535\\
61.5833333333333	0.27102	-1883.81093084079\\
61.5833333333333	0.27268	-1913.73711872624\\
61.5833333333333	0.27434	-1943.66330661168\\
61.5833333333333	0.276	-1973.58949449712\\
61.7916666666667	0.193	-471.79221136795\\
61.7916666666667	0.19466	-501.590847152933\\
61.7916666666667	0.19632	-531.389482937915\\
61.7916666666667	0.19798	-561.188118722898\\
61.7916666666667	0.19964	-590.986754507881\\
61.7916666666667	0.2013	-620.785390292863\\
61.7916666666667	0.20296	-650.584026077845\\
61.7916666666667	0.20462	-680.382661862827\\
61.7916666666667	0.20628	-710.18129764781\\
61.7916666666667	0.20794	-739.979933432792\\
61.7916666666667	0.2096	-769.778569217776\\
61.7916666666667	0.21126	-799.577205002758\\
61.7916666666667	0.21292	-829.37584078774\\
61.7916666666667	0.21458	-859.174476572722\\
61.7916666666667	0.21624	-888.973112357705\\
61.7916666666667	0.2179	-918.771748142687\\
61.7916666666667	0.21956	-948.570383927669\\
61.7916666666667	0.22122	-978.369019712653\\
61.7916666666667	0.22288	-1008.16765549764\\
61.7916666666667	0.22454	-1037.96629128262\\
61.7916666666667	0.2262	-1067.7649270676\\
61.7916666666667	0.22786	-1097.56356285258\\
61.7916666666667	0.22952	-1127.36219863756\\
61.7916666666667	0.23118	-1157.16083442255\\
61.7916666666667	0.23284	-1186.95947020753\\
61.7916666666667	0.2345	-1216.75810599251\\
61.7916666666667	0.23616	-1246.5567417775\\
61.7916666666667	0.23782	-1276.35537756248\\
61.7916666666667	0.23948	-1306.15401334746\\
61.7916666666667	0.24114	-1335.95264913244\\
61.7916666666667	0.2428	-1365.75128491742\\
61.7916666666667	0.24446	-1395.54992070241\\
61.7916666666667	0.24612	-1425.34855648739\\
61.7916666666667	0.24778	-1455.14719227237\\
61.7916666666667	0.24944	-1484.94582805736\\
61.7916666666667	0.2511	-1514.74446384234\\
61.7916666666667	0.25276	-1544.54309962732\\
61.7916666666667	0.25442	-1574.3417354123\\
61.7916666666667	0.25608	-1604.14037119728\\
61.7916666666667	0.25774	-1633.93900698227\\
61.7916666666667	0.2594	-1663.73764276725\\
61.7916666666667	0.26106	-1693.53627855223\\
61.7916666666667	0.26272	-1723.33491433721\\
61.7916666666667	0.26438	-1753.1335501222\\
61.7916666666667	0.26604	-1782.93218590718\\
61.7916666666667	0.2677	-1812.73082169216\\
61.7916666666667	0.26936	-1842.52945747715\\
61.7916666666667	0.27102	-1872.32809326213\\
61.7916666666667	0.27268	-1902.12672904711\\
61.7916666666667	0.27434	-1931.92536483209\\
61.7916666666667	0.276	-1961.72400061707\\
62	0.193	-466.304322510925\\
62	0.19466	-495.975406195448\\
62	0.19632	-525.64648987997\\
62	0.19798	-555.317573564493\\
62	0.19964	-584.988657249014\\
62	0.2013	-614.659740933536\\
62	0.20296	-644.330824618058\\
62	0.20462	-674.001908302579\\
62	0.20628	-703.672991987102\\
62	0.20794	-733.344075671624\\
62	0.2096	-763.015159356147\\
62	0.21126	-792.686243040669\\
62	0.21292	-822.35732672519\\
62	0.21458	-852.028410409712\\
62	0.21624	-881.699494094235\\
62	0.2179	-911.370577778756\\
62	0.21956	-941.041661463278\\
62	0.22122	-970.712745147802\\
62	0.22288	-1000.38382883232\\
62	0.22454	-1030.05491251685\\
62	0.2262	-1059.72599620137\\
62	0.22786	-1089.39707988589\\
62	0.22952	-1119.06816357041\\
62	0.23118	-1148.73924725493\\
62	0.23284	-1178.41033093945\\
62	0.2345	-1208.08141462398\\
62	0.23616	-1237.7524983085\\
62	0.23782	-1267.42358199302\\
62	0.23948	-1297.09466567754\\
62	0.24114	-1326.76574936207\\
62	0.2428	-1356.43683304659\\
62	0.24446	-1386.10791673111\\
62	0.24612	-1415.77900041563\\
62	0.24778	-1445.45008410016\\
62	0.24944	-1475.12116778468\\
62	0.2511	-1504.7922514692\\
62	0.25276	-1534.46333515372\\
62	0.25442	-1564.13441883824\\
62	0.25608	-1593.80550252276\\
62	0.25774	-1623.47658620729\\
62	0.2594	-1653.14766989181\\
62	0.26106	-1682.81875357633\\
62	0.26272	-1712.48983726085\\
62	0.26438	-1742.16092094537\\
62	0.26604	-1771.8320046299\\
62	0.2677	-1801.50308831442\\
62	0.26936	-1831.17417199894\\
62	0.27102	-1860.84525568347\\
62	0.27268	-1890.51633936799\\
62	0.27434	-1920.18742305251\\
62	0.276	-1949.85850673703\\
62.2083333333333	0.193	-460.8164336539\\
62.2083333333333	0.19466	-490.359965237963\\
62.2083333333333	0.19632	-519.903496822024\\
62.2083333333333	0.19798	-549.447028406086\\
62.2083333333333	0.19964	-578.990559990147\\
62.2083333333333	0.2013	-608.534091574209\\
62.2083333333333	0.20296	-638.07762315827\\
62.2083333333333	0.20462	-667.621154742332\\
62.2083333333333	0.20628	-697.164686326394\\
62.2083333333333	0.20794	-726.708217910455\\
62.2083333333333	0.2096	-756.251749494519\\
62.2083333333333	0.21126	-785.79528107858\\
62.2083333333333	0.21292	-815.338812662641\\
62.2083333333333	0.21458	-844.882344246702\\
62.2083333333333	0.21624	-874.425875830764\\
62.2083333333333	0.2179	-903.969407414826\\
62.2083333333333	0.21956	-933.512938998887\\
62.2083333333333	0.22122	-963.05647058295\\
62.2083333333333	0.22288	-992.600002167012\\
62.2083333333333	0.22454	-1022.14353375107\\
62.2083333333333	0.2262	-1051.68706533514\\
62.2083333333333	0.22786	-1081.2305969192\\
62.2083333333333	0.22952	-1110.77412850326\\
62.2083333333333	0.23118	-1140.31766008732\\
62.2083333333333	0.23284	-1169.86119167138\\
62.2083333333333	0.2345	-1199.40472325544\\
62.2083333333333	0.23616	-1228.94825483951\\
62.2083333333333	0.23782	-1258.49178642357\\
62.2083333333333	0.23948	-1288.03531800763\\
62.2083333333333	0.24114	-1317.57884959169\\
62.2083333333333	0.2428	-1347.12238117575\\
62.2083333333333	0.24446	-1376.66591275981\\
62.2083333333333	0.24612	-1406.20944434388\\
62.2083333333333	0.24778	-1435.75297592794\\
62.2083333333333	0.24944	-1465.296507512\\
62.2083333333333	0.2511	-1494.84003909606\\
62.2083333333333	0.25276	-1524.38357068012\\
62.2083333333333	0.25442	-1553.92710226418\\
62.2083333333333	0.25608	-1583.47063384825\\
62.2083333333333	0.25774	-1613.01416543231\\
62.2083333333333	0.2594	-1642.55769701637\\
62.2083333333333	0.26106	-1672.10122860043\\
62.2083333333333	0.26272	-1701.64476018449\\
62.2083333333333	0.26438	-1731.18829176855\\
62.2083333333333	0.26604	-1760.73182335262\\
62.2083333333333	0.2677	-1790.27535493668\\
62.2083333333333	0.26936	-1819.81888652074\\
62.2083333333333	0.27102	-1849.3624181048\\
62.2083333333333	0.27268	-1878.90594968886\\
62.2083333333333	0.27434	-1908.44948127292\\
62.2083333333333	0.276	-1937.99301285698\\
62.4166666666667	0.193	-455.328544796876\\
62.4166666666667	0.19466	-484.744524280477\\
62.4166666666667	0.19632	-514.160503764077\\
62.4166666666667	0.19798	-543.57648324768\\
62.4166666666667	0.19964	-572.99246273128\\
62.4166666666667	0.2013	-602.408442214883\\
62.4166666666667	0.20296	-631.824421698483\\
62.4166666666667	0.20462	-661.240401182084\\
62.4166666666667	0.20628	-690.656380665686\\
62.4166666666667	0.20794	-720.072360149287\\
62.4166666666667	0.2096	-749.48833963289\\
62.4166666666667	0.21126	-778.90431911649\\
62.4166666666667	0.21292	-808.320298600092\\
62.4166666666667	0.21458	-837.736278083692\\
62.4166666666667	0.21624	-867.152257567294\\
62.4166666666667	0.2179	-896.568237050895\\
62.4166666666667	0.21956	-925.984216534496\\
62.4166666666667	0.22122	-955.400196018099\\
62.4166666666667	0.22288	-984.816175501701\\
62.4166666666667	0.22454	-1014.2321549853\\
62.4166666666667	0.2262	-1043.6481344689\\
62.4166666666667	0.22786	-1073.0641139525\\
62.4166666666667	0.22952	-1102.4800934361\\
62.4166666666667	0.23118	-1131.89607291971\\
62.4166666666667	0.23284	-1161.31205240331\\
62.4166666666667	0.2345	-1190.72803188691\\
62.4166666666667	0.23616	-1220.14401137051\\
62.4166666666667	0.23782	-1249.55999085411\\
62.4166666666667	0.23948	-1278.97597033771\\
62.4166666666667	0.24114	-1308.39194982131\\
62.4166666666667	0.2428	-1337.80792930491\\
62.4166666666667	0.24446	-1367.22390878852\\
62.4166666666667	0.24612	-1396.63988827212\\
62.4166666666667	0.24778	-1426.05586775572\\
62.4166666666667	0.24944	-1455.47184723932\\
62.4166666666667	0.2511	-1484.88782672292\\
62.4166666666667	0.25276	-1514.30380620652\\
62.4166666666667	0.25442	-1543.71978569012\\
62.4166666666667	0.25608	-1573.13576517373\\
62.4166666666667	0.25774	-1602.55174465733\\
62.4166666666667	0.2594	-1631.96772414093\\
62.4166666666667	0.26106	-1661.38370362453\\
62.4166666666667	0.26272	-1690.79968310813\\
62.4166666666667	0.26438	-1720.21566259173\\
62.4166666666667	0.26604	-1749.63164207534\\
62.4166666666667	0.2677	-1779.04762155894\\
62.4166666666667	0.26936	-1808.46360104254\\
62.4166666666667	0.27102	-1837.87958052614\\
62.4166666666667	0.27268	-1867.29556000974\\
62.4166666666667	0.27434	-1896.71153949334\\
62.4166666666667	0.276	-1926.12751897694\\
62.625	0.193	-449.84065593985\\
62.625	0.19466	-479.12908332299\\
62.625	0.19632	-508.417510706131\\
62.625	0.19798	-537.705938089272\\
62.625	0.19964	-566.994365472413\\
62.625	0.2013	-596.282792855555\\
62.625	0.20296	-625.571220238696\\
62.625	0.20462	-654.859647621836\\
62.625	0.20628	-684.148075004978\\
62.625	0.20794	-713.436502388118\\
62.625	0.2096	-742.724929771261\\
62.625	0.21126	-772.013357154401\\
62.625	0.21292	-801.301784537542\\
62.625	0.21458	-830.590211920682\\
62.625	0.21624	-859.878639303824\\
62.625	0.2179	-889.167066686964\\
62.625	0.21956	-918.455494070105\\
62.625	0.22122	-947.743921453247\\
62.625	0.22288	-977.032348836388\\
62.625	0.22454	-1006.32077621953\\
62.625	0.2262	-1035.60920360267\\
62.625	0.22786	-1064.89763098581\\
62.625	0.22952	-1094.18605836895\\
62.625	0.23118	-1123.47448575209\\
62.625	0.23284	-1152.76291313523\\
62.625	0.2345	-1182.05134051837\\
62.625	0.23616	-1211.33976790152\\
62.625	0.23782	-1240.62819528466\\
62.625	0.23948	-1269.9166226678\\
62.625	0.24114	-1299.20505005094\\
62.625	0.2428	-1328.49347743408\\
62.625	0.24446	-1357.78190481722\\
62.625	0.24612	-1387.07033220036\\
62.625	0.24778	-1416.3587595835\\
62.625	0.24944	-1445.64718696664\\
62.625	0.2511	-1474.93561434978\\
62.625	0.25276	-1504.22404173292\\
62.625	0.25442	-1533.51246911607\\
62.625	0.25608	-1562.80089649921\\
62.625	0.25774	-1592.08932388235\\
62.625	0.2594	-1621.37775126549\\
62.625	0.26106	-1650.66617864863\\
62.625	0.26272	-1679.95460603177\\
62.625	0.26438	-1709.24303341491\\
62.625	0.26604	-1738.53146079805\\
62.625	0.2677	-1767.81988818119\\
62.625	0.26936	-1797.10831556433\\
62.625	0.27102	-1826.39674294748\\
62.625	0.27268	-1855.68517033062\\
62.625	0.27434	-1884.97359771376\\
62.625	0.276	-1914.2620250969\\
62.8333333333333	0.193	-444.352767082824\\
62.8333333333333	0.19466	-473.513642365507\\
62.8333333333333	0.19632	-502.674517648186\\
62.8333333333333	0.19798	-531.835392930868\\
62.8333333333333	0.19964	-560.996268213547\\
62.8333333333333	0.2013	-590.157143496228\\
62.8333333333333	0.20296	-619.318018778908\\
62.8333333333333	0.20462	-648.478894061588\\
62.8333333333333	0.20628	-677.63976934427\\
62.8333333333333	0.20794	-706.800644626949\\
62.8333333333333	0.2096	-735.961519909632\\
62.8333333333333	0.21126	-765.122395192311\\
62.8333333333333	0.21292	-794.283270474992\\
62.8333333333333	0.21458	-823.444145757671\\
62.8333333333333	0.21624	-852.605021040353\\
62.8333333333333	0.2179	-881.765896323032\\
62.8333333333333	0.21956	-910.926771605713\\
62.8333333333333	0.22122	-940.087646888394\\
62.8333333333333	0.22288	-969.248522171076\\
62.8333333333333	0.22454	-998.409397453756\\
62.8333333333333	0.2262	-1027.57027273644\\
62.8333333333333	0.22786	-1056.73114801912\\
62.8333333333333	0.22952	-1085.8920233018\\
62.8333333333333	0.23118	-1115.05289858448\\
62.8333333333333	0.23284	-1144.21377386716\\
62.8333333333333	0.2345	-1173.37464914984\\
62.8333333333333	0.23616	-1202.53552443252\\
62.8333333333333	0.23782	-1231.6963997152\\
62.8333333333333	0.23948	-1260.85727499788\\
62.8333333333333	0.24114	-1290.01815028056\\
62.8333333333333	0.2428	-1319.17902556324\\
62.8333333333333	0.24446	-1348.33990084592\\
62.8333333333333	0.24612	-1377.5007761286\\
62.8333333333333	0.24778	-1406.66165141128\\
62.8333333333333	0.24944	-1435.82252669396\\
62.8333333333333	0.2511	-1464.98340197664\\
62.8333333333333	0.25276	-1494.14427725933\\
62.8333333333333	0.25442	-1523.30515254201\\
62.8333333333333	0.25608	-1552.46602782469\\
62.8333333333333	0.25774	-1581.62690310737\\
62.8333333333333	0.2594	-1610.78777839005\\
62.8333333333333	0.26106	-1639.94865367273\\
62.8333333333333	0.26272	-1669.10952895541\\
62.8333333333333	0.26438	-1698.27040423809\\
62.8333333333333	0.26604	-1727.43127952077\\
62.8333333333333	0.2677	-1756.59215480345\\
62.8333333333333	0.26936	-1785.75303008613\\
62.8333333333333	0.27102	-1814.91390536881\\
62.8333333333333	0.27268	-1844.07478065149\\
62.8333333333333	0.27434	-1873.23565593417\\
62.8333333333333	0.276	-1902.39653121685\\
63.0416666666667	0.193	-438.8648782258\\
63.0416666666667	0.19466	-467.898201408021\\
63.0416666666667	0.19632	-496.931524590241\\
63.0416666666667	0.19798	-525.964847772461\\
63.0416666666667	0.19964	-554.998170954681\\
63.0416666666667	0.2013	-584.031494136901\\
63.0416666666667	0.20296	-613.06481731912\\
63.0416666666667	0.20462	-642.098140501341\\
63.0416666666667	0.20628	-671.131463683561\\
63.0416666666667	0.20794	-700.164786865781\\
63.0416666666667	0.2096	-729.198110048002\\
63.0416666666667	0.21126	-758.231433230222\\
63.0416666666667	0.21292	-787.264756412442\\
63.0416666666667	0.21458	-816.298079594661\\
63.0416666666667	0.21624	-845.331402776882\\
63.0416666666667	0.2179	-874.364725959102\\
63.0416666666667	0.21956	-903.398049141321\\
63.0416666666667	0.22122	-932.431372323544\\
63.0416666666667	0.22288	-961.464695505764\\
63.0416666666667	0.22454	-990.498018687984\\
63.0416666666667	0.2262	-1019.5313418702\\
63.0416666666667	0.22786	-1048.56466505242\\
63.0416666666667	0.22952	-1077.59798823464\\
63.0416666666667	0.23118	-1106.63131141686\\
63.0416666666667	0.23284	-1135.66463459908\\
63.0416666666667	0.2345	-1164.6979577813\\
63.0416666666667	0.23616	-1193.73128096353\\
63.0416666666667	0.23782	-1222.76460414575\\
63.0416666666667	0.23948	-1251.79792732797\\
63.0416666666667	0.24114	-1280.83125051018\\
63.0416666666667	0.2428	-1309.8645736924\\
63.0416666666667	0.24446	-1338.89789687463\\
63.0416666666667	0.24612	-1367.93122005685\\
63.0416666666667	0.24778	-1396.96454323907\\
63.0416666666667	0.24944	-1425.99786642129\\
63.0416666666667	0.2511	-1455.0311896035\\
63.0416666666667	0.25276	-1484.06451278573\\
63.0416666666667	0.25442	-1513.09783596795\\
63.0416666666667	0.25608	-1542.13115915017\\
63.0416666666667	0.25774	-1571.16448233239\\
63.0416666666667	0.2594	-1600.19780551461\\
63.0416666666667	0.26106	-1629.23112869683\\
63.0416666666667	0.26272	-1658.26445187905\\
63.0416666666667	0.26438	-1687.29777506127\\
63.0416666666667	0.26604	-1716.33109824349\\
63.0416666666667	0.2677	-1745.36442142571\\
63.0416666666667	0.26936	-1774.39774460793\\
63.0416666666667	0.27102	-1803.43106779015\\
63.0416666666667	0.27268	-1832.46439097237\\
63.0416666666667	0.27434	-1861.49771415459\\
63.0416666666667	0.276	-1890.53103733681\\
63.25	0.193	-433.376989368775\\
63.25	0.19466	-462.282760450535\\
63.25	0.19632	-491.188531532294\\
63.25	0.19798	-520.094302614055\\
63.25	0.19964	-549.000073695814\\
63.25	0.2013	-577.905844777574\\
63.25	0.20296	-606.811615859334\\
63.25	0.20462	-635.717386941093\\
63.25	0.20628	-664.623158022853\\
63.25	0.20794	-693.528929104613\\
63.25	0.2096	-722.434700186373\\
63.25	0.21126	-751.340471268133\\
63.25	0.21292	-780.246242349892\\
63.25	0.21458	-809.152013431652\\
63.25	0.21624	-838.057784513412\\
63.25	0.2179	-866.963555595172\\
63.25	0.21956	-895.869326676931\\
63.25	0.22122	-924.775097758692\\
63.25	0.22288	-953.680868840453\\
63.25	0.22454	-982.586639922211\\
63.25	0.2262	-1011.49241100397\\
63.25	0.22786	-1040.39818208573\\
63.25	0.22952	-1069.30395316749\\
63.25	0.23118	-1098.20972424925\\
63.25	0.23284	-1127.11549533101\\
63.25	0.2345	-1156.02126641277\\
63.25	0.23616	-1184.92703749453\\
63.25	0.23782	-1213.83280857629\\
63.25	0.23948	-1242.73857965805\\
63.25	0.24114	-1271.64435073981\\
63.25	0.2428	-1300.55012182157\\
63.25	0.24446	-1329.45589290333\\
63.25	0.24612	-1358.36166398509\\
63.25	0.24778	-1387.26743506685\\
63.25	0.24944	-1416.17320614861\\
63.25	0.2511	-1445.07897723037\\
63.25	0.25276	-1473.98474831213\\
63.25	0.25442	-1502.89051939389\\
63.25	0.25608	-1531.79629047565\\
63.25	0.25774	-1560.70206155741\\
63.25	0.2594	-1589.60783263917\\
63.25	0.26106	-1618.51360372093\\
63.25	0.26272	-1647.41937480269\\
63.25	0.26438	-1676.32514588445\\
63.25	0.26604	-1705.23091696621\\
63.25	0.2677	-1734.13668804797\\
63.25	0.26936	-1763.04245912973\\
63.25	0.27102	-1791.94823021149\\
63.25	0.27268	-1820.85400129325\\
63.25	0.27434	-1849.759772375\\
63.25	0.276	-1878.66554345676\\
63.4583333333333	0.193	-427.88910051175\\
63.4583333333333	0.19466	-456.667319493049\\
63.4583333333333	0.19632	-485.445538474348\\
63.4583333333333	0.19798	-514.223757455648\\
63.4583333333333	0.19964	-543.001976436947\\
63.4583333333333	0.2013	-571.780195418248\\
63.4583333333333	0.20296	-600.558414399547\\
63.4583333333333	0.20462	-629.336633380845\\
63.4583333333333	0.20628	-658.114852362145\\
63.4583333333333	0.20794	-686.893071343445\\
63.4583333333333	0.2096	-715.671290324745\\
63.4583333333333	0.21126	-744.449509306044\\
63.4583333333333	0.21292	-773.227728287343\\
63.4583333333333	0.21458	-802.005947268642\\
63.4583333333333	0.21624	-830.784166249942\\
63.4583333333333	0.2179	-859.562385231241\\
63.4583333333333	0.21956	-888.34060421254\\
63.4583333333333	0.22122	-917.11882319384\\
63.4583333333333	0.22288	-945.89704217514\\
63.4583333333333	0.22454	-974.67526115644\\
63.4583333333333	0.2262	-1003.45348013774\\
63.4583333333333	0.22786	-1032.23169911904\\
63.4583333333333	0.22952	-1061.00991810034\\
63.4583333333333	0.23118	-1089.78813708164\\
63.4583333333333	0.23284	-1118.56635606294\\
63.4583333333333	0.2345	-1147.34457504424\\
63.4583333333333	0.23616	-1176.12279402553\\
63.4583333333333	0.23782	-1204.90101300683\\
63.4583333333333	0.23948	-1233.67923198813\\
63.4583333333333	0.24114	-1262.45745096943\\
63.4583333333333	0.2428	-1291.23566995073\\
63.4583333333333	0.24446	-1320.01388893203\\
63.4583333333333	0.24612	-1348.79210791333\\
63.4583333333333	0.24778	-1377.57032689463\\
63.4583333333333	0.24944	-1406.34854587593\\
63.4583333333333	0.2511	-1435.12676485723\\
63.4583333333333	0.25276	-1463.90498383853\\
63.4583333333333	0.25442	-1492.68320281983\\
63.4583333333333	0.25608	-1521.46142180113\\
63.4583333333333	0.25774	-1550.23964078243\\
63.4583333333333	0.2594	-1579.01785976373\\
63.4583333333333	0.26106	-1607.79607874503\\
63.4583333333333	0.26272	-1636.57429772632\\
63.4583333333333	0.26438	-1665.35251670762\\
63.4583333333333	0.26604	-1694.13073568892\\
63.4583333333333	0.2677	-1722.90895467022\\
63.4583333333333	0.26936	-1751.68717365152\\
63.4583333333333	0.27102	-1780.46539263282\\
63.4583333333333	0.27268	-1809.24361161412\\
63.4583333333333	0.27434	-1838.02183059542\\
63.4583333333333	0.276	-1866.80004957672\\
63.6666666666667	0.193	-422.401211654725\\
63.6666666666667	0.19466	-451.051878535564\\
63.6666666666667	0.19632	-479.702545416403\\
63.6666666666667	0.19798	-508.353212297243\\
63.6666666666667	0.19964	-537.003879178081\\
63.6666666666667	0.2013	-565.65454605892\\
63.6666666666667	0.20296	-594.305212939759\\
63.6666666666667	0.20462	-622.955879820597\\
63.6666666666667	0.20628	-651.606546701437\\
63.6666666666667	0.20794	-680.257213582276\\
63.6666666666667	0.2096	-708.907880463116\\
63.6666666666667	0.21126	-737.558547343955\\
63.6666666666667	0.21292	-766.209214224793\\
63.6666666666667	0.21458	-794.859881105632\\
63.6666666666667	0.21624	-823.510547986471\\
63.6666666666667	0.2179	-852.16121486731\\
63.6666666666667	0.21956	-880.811881748148\\
63.6666666666667	0.22122	-909.462548628989\\
63.6666666666667	0.22288	-938.113215509828\\
63.6666666666667	0.22454	-966.763882390667\\
63.6666666666667	0.2262	-995.414549271505\\
63.6666666666667	0.22786	-1024.06521615234\\
63.6666666666667	0.22952	-1052.71588303318\\
63.6666666666667	0.23118	-1081.36654991402\\
63.6666666666667	0.23284	-1110.01721679486\\
63.6666666666667	0.2345	-1138.6678836757\\
63.6666666666667	0.23616	-1167.31855055654\\
63.6666666666667	0.23782	-1195.96921743738\\
63.6666666666667	0.23948	-1224.61988431822\\
63.6666666666667	0.24114	-1253.27055119906\\
63.6666666666667	0.2428	-1281.92121807989\\
63.6666666666667	0.24446	-1310.57188496074\\
63.6666666666667	0.24612	-1339.22255184157\\
63.6666666666667	0.24778	-1367.87321872241\\
63.6666666666667	0.24944	-1396.52388560325\\
63.6666666666667	0.2511	-1425.17455248409\\
63.6666666666667	0.25276	-1453.82521936493\\
63.6666666666667	0.25442	-1482.47588624577\\
63.6666666666667	0.25608	-1511.12655312661\\
63.6666666666667	0.25774	-1539.77722000745\\
63.6666666666667	0.2594	-1568.42788688829\\
63.6666666666667	0.26106	-1597.07855376912\\
63.6666666666667	0.26272	-1625.72922064996\\
63.6666666666667	0.26438	-1654.3798875308\\
63.6666666666667	0.26604	-1683.03055441164\\
63.6666666666667	0.2677	-1711.68122129248\\
63.6666666666667	0.26936	-1740.33188817332\\
63.6666666666667	0.27102	-1768.98255505416\\
63.6666666666667	0.27268	-1797.633221935\\
63.6666666666667	0.27434	-1826.28388881584\\
63.6666666666667	0.276	-1854.93455569668\\
63.875	0.193	-416.913322797699\\
63.875	0.19466	-445.436437578079\\
63.875	0.19632	-473.959552358458\\
63.875	0.19798	-502.482667138836\\
63.875	0.19964	-531.005781919214\\
63.875	0.2013	-559.528896699593\\
63.875	0.20296	-588.052011479971\\
63.875	0.20462	-616.575126260349\\
63.875	0.20628	-645.098241040729\\
63.875	0.20794	-673.621355821107\\
63.875	0.2096	-702.144470601486\\
63.875	0.21126	-730.667585381864\\
63.875	0.21292	-759.190700162243\\
63.875	0.21458	-787.713814942621\\
63.875	0.21624	-816.236929723001\\
63.875	0.2179	-844.760044503379\\
63.875	0.21956	-873.283159283757\\
63.875	0.22122	-901.806274064136\\
63.875	0.22288	-930.329388844516\\
63.875	0.22454	-958.852503624894\\
63.875	0.2262	-987.375618405272\\
63.875	0.22786	-1015.89873318565\\
63.875	0.22952	-1044.42184796603\\
63.875	0.23118	-1072.94496274641\\
63.875	0.23284	-1101.46807752679\\
63.875	0.2345	-1129.99119230717\\
63.875	0.23616	-1158.51430708754\\
63.875	0.23782	-1187.03742186792\\
63.875	0.23948	-1215.5605366483\\
63.875	0.24114	-1244.08365142868\\
63.875	0.2428	-1272.60676620906\\
63.875	0.24446	-1301.12988098944\\
63.875	0.24612	-1329.65299576982\\
63.875	0.24778	-1358.17611055019\\
63.875	0.24944	-1386.69922533057\\
63.875	0.2511	-1415.22234011095\\
63.875	0.25276	-1443.74545489133\\
63.875	0.25442	-1472.26856967171\\
63.875	0.25608	-1500.79168445209\\
63.875	0.25774	-1529.31479923247\\
63.875	0.2594	-1557.83791401285\\
63.875	0.26106	-1586.36102879322\\
63.875	0.26272	-1614.8841435736\\
63.875	0.26438	-1643.40725835398\\
63.875	0.26604	-1671.93037313436\\
63.875	0.2677	-1700.45348791474\\
63.875	0.26936	-1728.97660269512\\
63.875	0.27102	-1757.4997174755\\
63.875	0.27268	-1786.02283225587\\
63.875	0.27434	-1814.54594703625\\
63.875	0.276	-1843.06906181663\\
64.0833333333333	0.193	-411.425433940674\\
64.0833333333333	0.19466	-439.820996620594\\
64.0833333333333	0.19632	-468.216559300511\\
64.0833333333333	0.19798	-496.61212198043\\
64.0833333333333	0.19964	-525.007684660348\\
64.0833333333333	0.2013	-553.403247340266\\
64.0833333333333	0.20296	-581.798810020184\\
64.0833333333333	0.20462	-610.194372700102\\
64.0833333333333	0.20628	-638.589935380021\\
64.0833333333333	0.20794	-666.985498059938\\
64.0833333333333	0.2096	-695.381060739857\\
64.0833333333333	0.21126	-723.776623419775\\
64.0833333333333	0.21292	-752.172186099693\\
64.0833333333333	0.21458	-780.567748779611\\
64.0833333333333	0.21624	-808.96331145953\\
64.0833333333333	0.2179	-837.358874139448\\
64.0833333333333	0.21956	-865.754436819366\\
64.0833333333333	0.22122	-894.149999499285\\
64.0833333333333	0.22288	-922.545562179203\\
64.0833333333333	0.22454	-950.941124859121\\
64.0833333333333	0.2262	-979.336687539039\\
64.0833333333333	0.22786	-1007.73225021896\\
64.0833333333333	0.22952	-1036.12781289888\\
64.0833333333333	0.23118	-1064.52337557879\\
64.0833333333333	0.23284	-1092.91893825871\\
64.0833333333333	0.2345	-1121.31450093863\\
64.0833333333333	0.23616	-1149.71006361855\\
64.0833333333333	0.23782	-1178.10562629847\\
64.0833333333333	0.23948	-1206.50118897839\\
64.0833333333333	0.24114	-1234.8967516583\\
64.0833333333333	0.2428	-1263.29231433822\\
64.0833333333333	0.24446	-1291.68787701814\\
64.0833333333333	0.24612	-1320.08343969806\\
64.0833333333333	0.24778	-1348.47900237798\\
64.0833333333333	0.24944	-1376.87456505789\\
64.0833333333333	0.2511	-1405.27012773781\\
64.0833333333333	0.25276	-1433.66569041773\\
64.0833333333333	0.25442	-1462.06125309765\\
64.0833333333333	0.25608	-1490.45681577757\\
64.0833333333333	0.25774	-1518.85237845749\\
64.0833333333333	0.2594	-1547.2479411374\\
64.0833333333333	0.26106	-1575.64350381732\\
64.0833333333333	0.26272	-1604.03906649724\\
64.0833333333333	0.26438	-1632.43462917716\\
64.0833333333333	0.26604	-1660.83019185708\\
64.0833333333333	0.2677	-1689.225754537\\
64.0833333333333	0.26936	-1717.62131721691\\
64.0833333333333	0.27102	-1746.01687989683\\
64.0833333333333	0.27268	-1774.41244257675\\
64.0833333333333	0.27434	-1802.80800525667\\
64.0833333333333	0.276	-1831.20356793659\\
64.2916666666667	0.193	-405.937545083651\\
64.2916666666667	0.19466	-434.205555663108\\
64.2916666666667	0.19632	-462.473566242565\\
64.2916666666667	0.19798	-490.741576822023\\
64.2916666666667	0.19964	-519.009587401481\\
64.2916666666667	0.2013	-547.27759798094\\
64.2916666666667	0.20296	-575.545608560397\\
64.2916666666667	0.20462	-603.813619139854\\
64.2916666666667	0.20628	-632.081629719313\\
64.2916666666667	0.20794	-660.34964029877\\
64.2916666666667	0.2096	-688.61765087823\\
64.2916666666667	0.21126	-716.885661457687\\
64.2916666666667	0.21292	-745.153672037145\\
64.2916666666667	0.21458	-773.421682616602\\
64.2916666666667	0.21624	-801.689693196061\\
64.2916666666667	0.2179	-829.957703775518\\
64.2916666666667	0.21956	-858.225714354975\\
64.2916666666667	0.22122	-886.493724934435\\
64.2916666666667	0.22288	-914.761735513893\\
64.2916666666667	0.22454	-943.029746093351\\
64.2916666666667	0.2262	-971.297756672808\\
64.2916666666667	0.22786	-999.565767252265\\
64.2916666666667	0.22952	-1027.83377783172\\
64.2916666666667	0.23118	-1056.10178841118\\
64.2916666666667	0.23284	-1084.36979899064\\
64.2916666666667	0.2345	-1112.6378095701\\
64.2916666666667	0.23616	-1140.90582014955\\
64.2916666666667	0.23782	-1169.17383072901\\
64.2916666666667	0.23948	-1197.44184130847\\
64.2916666666667	0.24114	-1225.70985188793\\
64.2916666666667	0.2428	-1253.97786246739\\
64.2916666666667	0.24446	-1282.24587304685\\
64.2916666666667	0.24612	-1310.5138836263\\
64.2916666666667	0.24778	-1338.78189420576\\
64.2916666666667	0.24944	-1367.04990478522\\
64.2916666666667	0.2511	-1395.31791536467\\
64.2916666666667	0.25276	-1423.58592594413\\
64.2916666666667	0.25442	-1451.85393652359\\
64.2916666666667	0.25608	-1480.12194710305\\
64.2916666666667	0.25774	-1508.38995768251\\
64.2916666666667	0.2594	-1536.65796826196\\
64.2916666666667	0.26106	-1564.92597884142\\
64.2916666666667	0.26272	-1593.19398942088\\
64.2916666666667	0.26438	-1621.46200000034\\
64.2916666666667	0.26604	-1649.7300105798\\
64.2916666666667	0.2677	-1677.99802115925\\
64.2916666666667	0.26936	-1706.26603173871\\
64.2916666666667	0.27102	-1734.53404231817\\
64.2916666666667	0.27268	-1762.80205289763\\
64.2916666666667	0.27434	-1791.07006347709\\
64.2916666666667	0.276	-1819.33807405654\\
64.5	0.193	-400.449656226625\\
64.5	0.19466	-428.590114705623\\
64.5	0.19632	-456.73057318462\\
64.5	0.19798	-484.871031663618\\
64.5	0.19964	-513.011490142615\\
64.5	0.2013	-541.151948621613\\
64.5	0.20296	-569.29240710061\\
64.5	0.20462	-597.432865579607\\
64.5	0.20628	-625.573324058605\\
64.5	0.20794	-653.713782537602\\
64.5	0.2096	-681.854241016601\\
64.5	0.21126	-709.994699495598\\
64.5	0.21292	-738.135157974595\\
64.5	0.21458	-766.275616453592\\
64.5	0.21624	-794.41607493259\\
64.5	0.2179	-822.556533411587\\
64.5	0.21956	-850.696991890584\\
64.5	0.22122	-878.837450369583\\
64.5	0.22288	-906.977908848581\\
64.5	0.22454	-935.118367327578\\
64.5	0.2262	-963.258825806575\\
64.5	0.22786	-991.399284285572\\
64.5	0.22952	-1019.53974276457\\
64.5	0.23118	-1047.68020124357\\
64.5	0.23284	-1075.82065972256\\
64.5	0.2345	-1103.96111820156\\
64.5	0.23616	-1132.10157668056\\
64.5	0.23782	-1160.24203515956\\
64.5	0.23948	-1188.38249363856\\
64.5	0.24114	-1216.52295211755\\
64.5	0.2428	-1244.66341059655\\
64.5	0.24446	-1272.80386907555\\
64.5	0.24612	-1300.94432755454\\
64.5	0.24778	-1329.08478603354\\
64.5	0.24944	-1357.22524451254\\
64.5	0.2511	-1385.36570299154\\
64.5	0.25276	-1413.50616147053\\
64.5	0.25442	-1441.64661994953\\
64.5	0.25608	-1469.78707842853\\
64.5	0.25774	-1497.92753690753\\
64.5	0.2594	-1526.06799538652\\
64.5	0.26106	-1554.20845386552\\
64.5	0.26272	-1582.34891234452\\
64.5	0.26438	-1610.48937082352\\
64.5	0.26604	-1638.62982930251\\
64.5	0.2677	-1666.77028778151\\
64.5	0.26936	-1694.91074626051\\
64.5	0.27102	-1723.05120473951\\
64.5	0.27268	-1751.1916632185\\
64.5	0.27434	-1779.3321216975\\
64.5	0.276	-1807.4725801765\\
64.7083333333333	0.193	-394.961767369599\\
64.7083333333333	0.19466	-422.974673748137\\
64.7083333333333	0.19632	-450.987580126673\\
64.7083333333333	0.19798	-479.000486505211\\
64.7083333333333	0.19964	-507.013392883748\\
64.7083333333333	0.2013	-535.026299262286\\
64.7083333333333	0.20296	-563.039205640822\\
64.7083333333333	0.20462	-591.052112019359\\
64.7083333333333	0.20628	-619.065018397896\\
64.7083333333333	0.20794	-647.077924776433\\
64.7083333333333	0.2096	-675.090831154971\\
64.7083333333333	0.21126	-703.103737533508\\
64.7083333333333	0.21292	-731.116643912044\\
64.7083333333333	0.21458	-759.129550290581\\
64.7083333333333	0.21624	-787.142456669119\\
64.7083333333333	0.2179	-815.155363047656\\
64.7083333333333	0.21956	-843.168269426192\\
64.7083333333333	0.22122	-871.181175804731\\
64.7083333333333	0.22288	-899.194082183269\\
64.7083333333333	0.22454	-927.206988561805\\
64.7083333333333	0.2262	-955.219894940342\\
64.7083333333333	0.22786	-983.232801318878\\
64.7083333333333	0.22952	-1011.24570769742\\
64.7083333333333	0.23118	-1039.25861407595\\
64.7083333333333	0.23284	-1067.27152045449\\
64.7083333333333	0.2345	-1095.28442683303\\
64.7083333333333	0.23616	-1123.29733321156\\
64.7083333333333	0.23782	-1151.3102395901\\
64.7083333333333	0.23948	-1179.32314596864\\
64.7083333333333	0.24114	-1207.33605234717\\
64.7083333333333	0.2428	-1235.34895872571\\
64.7083333333333	0.24446	-1263.36186510425\\
64.7083333333333	0.24612	-1291.37477148279\\
64.7083333333333	0.24778	-1319.38767786132\\
64.7083333333333	0.24944	-1347.40058423986\\
64.7083333333333	0.2511	-1375.4134906184\\
64.7083333333333	0.25276	-1403.42639699694\\
64.7083333333333	0.25442	-1431.43930337547\\
64.7083333333333	0.25608	-1459.45220975401\\
64.7083333333333	0.25774	-1487.46511613255\\
64.7083333333333	0.2594	-1515.47802251108\\
64.7083333333333	0.26106	-1543.49092888962\\
64.7083333333333	0.26272	-1571.50383526816\\
64.7083333333333	0.26438	-1599.51674164669\\
64.7083333333333	0.26604	-1627.52964802523\\
64.7083333333333	0.2677	-1655.54255440377\\
64.7083333333333	0.26936	-1683.55546078231\\
64.7083333333333	0.27102	-1711.56836716084\\
64.7083333333333	0.27268	-1739.58127353938\\
64.7083333333333	0.27434	-1767.59417991792\\
64.7083333333333	0.276	-1795.60708629645\\
64.9166666666667	0.193	-389.473878512575\\
64.9166666666667	0.19466	-417.359232790652\\
64.9166666666667	0.19632	-445.244587068728\\
64.9166666666667	0.19798	-473.129941346805\\
64.9166666666667	0.19964	-501.015295624881\\
64.9166666666667	0.2013	-528.900649902958\\
64.9166666666667	0.20296	-556.786004181035\\
64.9166666666667	0.20462	-584.671358459111\\
64.9166666666667	0.20628	-612.556712737189\\
64.9166666666667	0.20794	-640.442067015265\\
64.9166666666667	0.2096	-668.327421293343\\
64.9166666666667	0.21126	-696.212775571419\\
64.9166666666667	0.21292	-724.098129849495\\
64.9166666666667	0.21458	-751.983484127572\\
64.9166666666667	0.21624	-779.868838405649\\
64.9166666666667	0.2179	-807.754192683725\\
64.9166666666667	0.21956	-835.639546961801\\
64.9166666666667	0.22122	-863.524901239879\\
64.9166666666667	0.22288	-891.410255517956\\
64.9166666666667	0.22454	-919.295609796032\\
64.9166666666667	0.2262	-947.180964074109\\
64.9166666666667	0.22786	-975.066318352186\\
64.9166666666667	0.22952	-1002.95167263026\\
64.9166666666667	0.23118	-1030.83702690834\\
64.9166666666667	0.23284	-1058.72238118642\\
64.9166666666667	0.2345	-1086.60773546449\\
64.9166666666667	0.23616	-1114.49308974257\\
64.9166666666667	0.23782	-1142.37844402065\\
64.9166666666667	0.23948	-1170.26379829872\\
64.9166666666667	0.24114	-1198.1491525768\\
64.9166666666667	0.2428	-1226.03450685487\\
64.9166666666667	0.24446	-1253.91986113295\\
64.9166666666667	0.24612	-1281.80521541103\\
64.9166666666667	0.24778	-1309.69056968911\\
64.9166666666667	0.24944	-1337.57592396718\\
64.9166666666667	0.2511	-1365.46127824526\\
64.9166666666667	0.25276	-1393.34663252334\\
64.9166666666667	0.25442	-1421.23198680141\\
64.9166666666667	0.25608	-1449.11734107949\\
64.9166666666667	0.25774	-1477.00269535757\\
64.9166666666667	0.2594	-1504.88804963564\\
64.9166666666667	0.26106	-1532.77340391372\\
64.9166666666667	0.26272	-1560.6587581918\\
64.9166666666667	0.26438	-1588.54411246987\\
64.9166666666667	0.26604	-1616.42946674795\\
64.9166666666667	0.2677	-1644.31482102603\\
64.9166666666667	0.26936	-1672.2001753041\\
64.9166666666667	0.27102	-1700.08552958218\\
64.9166666666667	0.27268	-1727.97088386026\\
64.9166666666667	0.27434	-1755.85623813833\\
64.9166666666667	0.276	-1783.74159241641\\
65.125	0.193	-383.98598965555\\
65.125	0.19466	-411.743791833166\\
65.125	0.19632	-439.501594010782\\
65.125	0.19798	-467.259396188399\\
65.125	0.19964	-495.017198366015\\
65.125	0.2013	-522.775000543632\\
65.125	0.20296	-550.532802721247\\
65.125	0.20462	-578.290604898863\\
65.125	0.20628	-606.048407076481\\
65.125	0.20794	-633.806209254096\\
65.125	0.2096	-661.564011431714\\
65.125	0.21126	-689.32181360933\\
65.125	0.21292	-717.079615786945\\
65.125	0.21458	-744.837417964562\\
65.125	0.21624	-772.595220142179\\
65.125	0.2179	-800.353022319794\\
65.125	0.21956	-828.11082449741\\
65.125	0.22122	-855.868626675027\\
65.125	0.22288	-883.626428852644\\
65.125	0.22454	-911.384231030261\\
65.125	0.2262	-939.142033207876\\
65.125	0.22786	-966.899835385492\\
65.125	0.22952	-994.657637563108\\
65.125	0.23118	-1022.41543974072\\
65.125	0.23284	-1050.17324191834\\
65.125	0.2345	-1077.93104409596\\
65.125	0.23616	-1105.68884627357\\
65.125	0.23782	-1133.44664845119\\
65.125	0.23948	-1161.20445062881\\
65.125	0.24114	-1188.96225280642\\
65.125	0.2428	-1216.72005498404\\
65.125	0.24446	-1244.47785716166\\
65.125	0.24612	-1272.23565933927\\
65.125	0.24778	-1299.99346151689\\
65.125	0.24944	-1327.7512636945\\
65.125	0.2511	-1355.50906587212\\
65.125	0.25276	-1383.26686804974\\
65.125	0.25442	-1411.02467022735\\
65.125	0.25608	-1438.78247240497\\
65.125	0.25774	-1466.54027458259\\
65.125	0.2594	-1494.2980767602\\
65.125	0.26106	-1522.05587893782\\
65.125	0.26272	-1549.81368111543\\
65.125	0.26438	-1577.57148329305\\
65.125	0.26604	-1605.32928547067\\
65.125	0.2677	-1633.08708764828\\
65.125	0.26936	-1660.8448898259\\
65.125	0.27102	-1688.60269200352\\
65.125	0.27268	-1716.36049418113\\
65.125	0.27434	-1744.11829635875\\
65.125	0.276	-1771.87609853637\\
65.3333333333333	0.193	-378.498100798525\\
65.3333333333333	0.19466	-406.128350875681\\
65.3333333333333	0.19632	-433.758600952837\\
65.3333333333333	0.19798	-461.388851029993\\
65.3333333333333	0.19964	-489.019101107148\\
65.3333333333333	0.2013	-516.649351184305\\
65.3333333333333	0.20296	-544.279601261461\\
65.3333333333333	0.20462	-571.909851338617\\
65.3333333333333	0.20628	-599.540101415773\\
65.3333333333333	0.20794	-627.170351492929\\
65.3333333333333	0.2096	-654.800601570086\\
65.3333333333333	0.21126	-682.430851647241\\
65.3333333333333	0.21292	-710.061101724396\\
65.3333333333333	0.21458	-737.691351801553\\
65.3333333333333	0.21624	-765.321601878709\\
65.3333333333333	0.2179	-792.951851955864\\
65.3333333333333	0.21956	-820.58210203302\\
65.3333333333333	0.22122	-848.212352110177\\
65.3333333333333	0.22288	-875.842602187334\\
65.3333333333333	0.22454	-903.472852264489\\
65.3333333333333	0.2262	-931.103102341644\\
65.3333333333333	0.22786	-958.7333524188\\
65.3333333333333	0.22952	-986.363602495955\\
65.3333333333333	0.23118	-1013.99385257311\\
65.3333333333333	0.23284	-1041.62410265027\\
65.3333333333333	0.2345	-1069.25435272742\\
65.3333333333333	0.23616	-1096.88460280458\\
65.3333333333333	0.23782	-1124.51485288174\\
65.3333333333333	0.23948	-1152.14510295889\\
65.3333333333333	0.24114	-1179.77535303605\\
65.3333333333333	0.2428	-1207.4056031132\\
65.3333333333333	0.24446	-1235.03585319036\\
65.3333333333333	0.24612	-1262.66610326752\\
65.3333333333333	0.24778	-1290.29635334467\\
65.3333333333333	0.24944	-1317.92660342183\\
65.3333333333333	0.2511	-1345.55685349898\\
65.3333333333333	0.25276	-1373.18710357614\\
65.3333333333333	0.25442	-1400.81735365329\\
65.3333333333333	0.25608	-1428.44760373045\\
65.3333333333333	0.25774	-1456.07785380761\\
65.3333333333333	0.2594	-1483.70810388476\\
65.3333333333333	0.26106	-1511.33835396192\\
65.3333333333333	0.26272	-1538.96860403907\\
65.3333333333333	0.26438	-1566.59885411623\\
65.3333333333333	0.26604	-1594.22910419339\\
65.3333333333333	0.2677	-1621.85935427054\\
65.3333333333333	0.26936	-1649.4896043477\\
65.3333333333333	0.27102	-1677.11985442486\\
65.3333333333333	0.27268	-1704.75010450201\\
65.3333333333333	0.27434	-1732.38035457917\\
65.3333333333333	0.276	-1760.01060465632\\
65.5416666666667	0.193	-373.010211941499\\
65.5416666666667	0.19466	-400.512909918196\\
65.5416666666667	0.19632	-428.015607894891\\
65.5416666666667	0.19798	-455.518305871587\\
65.5416666666667	0.19964	-483.021003848282\\
65.5416666666667	0.2013	-510.523701824977\\
65.5416666666667	0.20296	-538.026399801673\\
65.5416666666667	0.20462	-565.529097778367\\
65.5416666666667	0.20628	-593.031795755063\\
65.5416666666667	0.20794	-620.534493731759\\
65.5416666666667	0.2096	-648.037191708456\\
65.5416666666667	0.21126	-675.53988968515\\
65.5416666666667	0.21292	-703.042587661846\\
65.5416666666667	0.21458	-730.545285638541\\
65.5416666666667	0.21624	-758.047983615237\\
65.5416666666667	0.2179	-785.550681591932\\
65.5416666666667	0.21956	-813.053379568627\\
65.5416666666667	0.22122	-840.556077545324\\
65.5416666666667	0.22288	-868.05877552202\\
65.5416666666667	0.22454	-895.561473498715\\
65.5416666666667	0.2262	-923.06417147541\\
65.5416666666667	0.22786	-950.566869452105\\
65.5416666666667	0.22952	-978.0695674288\\
65.5416666666667	0.23118	-1005.5722654055\\
65.5416666666667	0.23284	-1033.07496338219\\
65.5416666666667	0.2345	-1060.57766135889\\
65.5416666666667	0.23616	-1088.08035933558\\
65.5416666666667	0.23782	-1115.58305731228\\
65.5416666666667	0.23948	-1143.08575528897\\
65.5416666666667	0.24114	-1170.58845326567\\
65.5416666666667	0.2428	-1198.09115124236\\
65.5416666666667	0.24446	-1225.59384921906\\
65.5416666666667	0.24612	-1253.09654719576\\
65.5416666666667	0.24778	-1280.59924517245\\
65.5416666666667	0.24944	-1308.10194314915\\
65.5416666666667	0.2511	-1335.60464112584\\
65.5416666666667	0.25276	-1363.10733910254\\
65.5416666666667	0.25442	-1390.61003707923\\
65.5416666666667	0.25608	-1418.11273505593\\
65.5416666666667	0.25774	-1445.61543303263\\
65.5416666666667	0.2594	-1473.11813100932\\
65.5416666666667	0.26106	-1500.62082898602\\
65.5416666666667	0.26272	-1528.12352696271\\
65.5416666666667	0.26438	-1555.62622493941\\
65.5416666666667	0.26604	-1583.1289229161\\
65.5416666666667	0.2677	-1610.6316208928\\
65.5416666666667	0.26936	-1638.13431886949\\
65.5416666666667	0.27102	-1665.63701684619\\
65.5416666666667	0.27268	-1693.13971482289\\
65.5416666666667	0.27434	-1720.64241279958\\
65.5416666666667	0.276	-1748.14511077628\\
65.75	0.193	-367.522323084474\\
65.75	0.19466	-394.89746896071\\
65.75	0.19632	-422.272614836946\\
65.75	0.19798	-449.647760713181\\
65.75	0.19964	-477.022906589415\\
65.75	0.2013	-504.39805246565\\
65.75	0.20296	-531.773198341885\\
65.75	0.20462	-559.14834421812\\
65.75	0.20628	-586.523490094355\\
65.75	0.20794	-613.89863597059\\
65.75	0.2096	-641.273781846827\\
65.75	0.21126	-668.648927723061\\
65.75	0.21292	-696.024073599296\\
65.75	0.21458	-723.399219475531\\
65.75	0.21624	-750.774365351766\\
65.75	0.2179	-778.149511228001\\
65.75	0.21956	-805.524657104236\\
65.75	0.22122	-832.899802980472\\
65.75	0.22288	-860.274948856708\\
65.75	0.22454	-887.650094732943\\
65.75	0.2262	-915.025240609178\\
65.75	0.22786	-942.400386485412\\
65.75	0.22952	-969.775532361647\\
65.75	0.23118	-997.150678237883\\
65.75	0.23284	-1024.52582411412\\
65.75	0.2345	-1051.90096999035\\
65.75	0.23616	-1079.27611586659\\
65.75	0.23782	-1106.65126174282\\
65.75	0.23948	-1134.02640761906\\
65.75	0.24114	-1161.40155349529\\
65.75	0.2428	-1188.77669937153\\
65.75	0.24446	-1216.15184524776\\
65.75	0.24612	-1243.526991124\\
65.75	0.24778	-1270.90213700024\\
65.75	0.24944	-1298.27728287647\\
65.75	0.2511	-1325.6524287527\\
65.75	0.25276	-1353.02757462894\\
65.75	0.25442	-1380.40272050518\\
65.75	0.25608	-1407.77786638141\\
65.75	0.25774	-1435.15301225765\\
65.75	0.2594	-1462.52815813388\\
65.75	0.26106	-1489.90330401012\\
65.75	0.26272	-1517.27844988635\\
65.75	0.26438	-1544.65359576258\\
65.75	0.26604	-1572.02874163882\\
65.75	0.2677	-1599.40388751506\\
65.75	0.26936	-1626.77903339129\\
65.75	0.27102	-1654.15417926753\\
65.75	0.27268	-1681.52932514376\\
65.75	0.27434	-1708.90447102\\
65.75	0.276	-1736.27961689623\\
65.9583333333333	0.193	-362.034434227449\\
65.9583333333333	0.19466	-389.282028003225\\
65.9583333333333	0.19632	-416.529621778999\\
65.9583333333333	0.19798	-443.777215554775\\
65.9583333333333	0.19964	-471.024809330549\\
65.9583333333333	0.2013	-498.272403106324\\
65.9583333333333	0.20296	-525.519996882098\\
65.9583333333333	0.20462	-552.767590657872\\
65.9583333333333	0.20628	-580.015184433648\\
65.9583333333333	0.20794	-607.262778209421\\
65.9583333333333	0.2096	-634.510371985198\\
65.9583333333333	0.21126	-661.757965760972\\
65.9583333333333	0.21292	-689.005559536747\\
65.9583333333333	0.21458	-716.253153312521\\
65.9583333333333	0.21624	-743.500747088297\\
65.9583333333333	0.2179	-770.74834086407\\
65.9583333333333	0.21956	-797.995934639845\\
65.9583333333333	0.22122	-825.243528415621\\
65.9583333333333	0.22288	-852.491122191396\\
65.9583333333333	0.22454	-879.738715967171\\
65.9583333333333	0.2262	-906.986309742945\\
65.9583333333333	0.22786	-934.233903518719\\
65.9583333333333	0.22952	-961.481497294493\\
65.9583333333333	0.23118	-988.729091070269\\
65.9583333333333	0.23284	-1015.97668484604\\
65.9583333333333	0.2345	-1043.22427862182\\
65.9583333333333	0.23616	-1070.47187239759\\
65.9583333333333	0.23782	-1097.71946617337\\
65.9583333333333	0.23948	-1124.96705994914\\
65.9583333333333	0.24114	-1152.21465372492\\
65.9583333333333	0.2428	-1179.46224750069\\
65.9583333333333	0.24446	-1206.70984127647\\
65.9583333333333	0.24612	-1233.95743505224\\
65.9583333333333	0.24778	-1261.20502882802\\
65.9583333333333	0.24944	-1288.45262260379\\
65.9583333333333	0.2511	-1315.70021637957\\
65.9583333333333	0.25276	-1342.94781015534\\
65.9583333333333	0.25442	-1370.19540393112\\
65.9583333333333	0.25608	-1397.44299770689\\
65.9583333333333	0.25774	-1424.69059148267\\
65.9583333333333	0.2594	-1451.93818525844\\
65.9583333333333	0.26106	-1479.18577903421\\
65.9583333333333	0.26272	-1506.43337280999\\
65.9583333333333	0.26438	-1533.68096658576\\
65.9583333333333	0.26604	-1560.92856036154\\
65.9583333333333	0.2677	-1588.17615413731\\
65.9583333333333	0.26936	-1615.42374791309\\
65.9583333333333	0.27102	-1642.67134168887\\
65.9583333333333	0.27268	-1669.91893546464\\
65.9583333333333	0.27434	-1697.16652924041\\
65.9583333333333	0.276	-1724.41412301619\\
66.1666666666667	0.193	-356.546545370425\\
66.1666666666667	0.19466	-383.666587045738\\
66.1666666666667	0.19632	-410.786628721052\\
66.1666666666667	0.19798	-437.906670396367\\
66.1666666666667	0.19964	-465.026712071681\\
66.1666666666667	0.2013	-492.146753746997\\
66.1666666666667	0.20296	-519.266795422311\\
66.1666666666667	0.20462	-546.386837097624\\
66.1666666666667	0.20628	-573.50687877294\\
66.1666666666667	0.20794	-600.626920448253\\
66.1666666666667	0.2096	-627.74696212357\\
66.1666666666667	0.21126	-654.867003798883\\
66.1666666666667	0.21292	-681.987045474197\\
66.1666666666667	0.21458	-709.107087149511\\
66.1666666666667	0.21624	-736.227128824826\\
66.1666666666667	0.2179	-763.34717050014\\
66.1666666666667	0.21956	-790.467212175454\\
66.1666666666667	0.22122	-817.587253850769\\
66.1666666666667	0.22288	-844.707295526085\\
66.1666666666667	0.22454	-871.827337201398\\
66.1666666666667	0.2262	-898.947378876713\\
66.1666666666667	0.22786	-926.067420552026\\
66.1666666666667	0.22952	-953.18746222734\\
66.1666666666667	0.23118	-980.307503902655\\
66.1666666666667	0.23284	-1007.42754557797\\
66.1666666666667	0.2345	-1034.54758725328\\
66.1666666666667	0.23616	-1061.6676289286\\
66.1666666666667	0.23782	-1088.78767060391\\
66.1666666666667	0.23948	-1115.90771227923\\
66.1666666666667	0.24114	-1143.02775395454\\
66.1666666666667	0.2428	-1170.14779562985\\
66.1666666666667	0.24446	-1197.26783730517\\
66.1666666666667	0.24612	-1224.38787898048\\
66.1666666666667	0.24778	-1251.5079206558\\
66.1666666666667	0.24944	-1278.62796233111\\
66.1666666666667	0.2511	-1305.74800400643\\
66.1666666666667	0.25276	-1332.86804568174\\
66.1666666666667	0.25442	-1359.98808735706\\
66.1666666666667	0.25608	-1387.10812903237\\
66.1666666666667	0.25774	-1414.22817070769\\
66.1666666666667	0.2594	-1441.348212383\\
66.1666666666667	0.26106	-1468.46825405831\\
66.1666666666667	0.26272	-1495.58829573363\\
66.1666666666667	0.26438	-1522.70833740894\\
66.1666666666667	0.26604	-1549.82837908426\\
66.1666666666667	0.2677	-1576.94842075957\\
66.1666666666667	0.26936	-1604.06846243489\\
66.1666666666667	0.27102	-1631.1885041102\\
66.1666666666667	0.27268	-1658.30854578552\\
66.1666666666667	0.27434	-1685.42858746083\\
66.1666666666667	0.276	-1712.54862913614\\
66.375	0.193	-351.058656513399\\
66.375	0.19466	-378.051146088254\\
66.375	0.19632	-405.043635663107\\
66.375	0.19798	-432.036125237962\\
66.375	0.19964	-459.028614812815\\
66.375	0.2013	-486.02110438767\\
66.375	0.20296	-513.013593962523\\
66.375	0.20462	-540.006083537377\\
66.375	0.20628	-566.998573112231\\
66.375	0.20794	-593.991062687085\\
66.375	0.2096	-620.983552261941\\
66.375	0.21126	-647.976041836794\\
66.375	0.21292	-674.968531411648\\
66.375	0.21458	-701.961020986501\\
66.375	0.21624	-728.953510561356\\
66.375	0.2179	-755.946000136209\\
66.375	0.21956	-782.938489711062\\
66.375	0.22122	-809.930979285918\\
66.375	0.22288	-836.923468860773\\
66.375	0.22454	-863.915958435626\\
66.375	0.2262	-890.908448010479\\
66.375	0.22786	-917.900937585333\\
66.375	0.22952	-944.893427160187\\
66.375	0.23118	-971.885916735041\\
66.375	0.23284	-998.878406309895\\
66.375	0.2345	-1025.87089588475\\
66.375	0.23616	-1052.8633854596\\
66.375	0.23782	-1079.85587503446\\
66.375	0.23948	-1106.84836460931\\
66.375	0.24114	-1133.84085418417\\
66.375	0.2428	-1160.83334375902\\
66.375	0.24446	-1187.82583333387\\
66.375	0.24612	-1214.81832290873\\
66.375	0.24778	-1241.81081248358\\
66.375	0.24944	-1268.80330205844\\
66.375	0.2511	-1295.79579163329\\
66.375	0.25276	-1322.78828120814\\
66.375	0.25442	-1349.780770783\\
66.375	0.25608	-1376.77326035785\\
66.375	0.25774	-1403.76574993271\\
66.375	0.2594	-1430.75823950756\\
66.375	0.26106	-1457.75072908241\\
66.375	0.26272	-1484.74321865727\\
66.375	0.26438	-1511.73570823212\\
66.375	0.26604	-1538.72819780698\\
66.375	0.2677	-1565.72068738183\\
66.375	0.26936	-1592.71317695668\\
66.375	0.27102	-1619.70566653154\\
66.375	0.27268	-1646.69815610639\\
66.375	0.27434	-1673.69064568125\\
66.375	0.276	-1700.6831352561\\
66.5833333333333	0.193	-345.570767656374\\
66.5833333333333	0.19466	-372.435705130768\\
66.5833333333333	0.19632	-399.300642605162\\
66.5833333333333	0.19798	-426.165580079556\\
66.5833333333333	0.19964	-453.030517553949\\
66.5833333333333	0.2013	-479.895455028343\\
66.5833333333333	0.20296	-506.760392502736\\
66.5833333333333	0.20462	-533.625329977129\\
66.5833333333333	0.20628	-560.490267451523\\
66.5833333333333	0.20794	-587.355204925917\\
66.5833333333333	0.2096	-614.220142400311\\
66.5833333333333	0.21126	-641.085079874705\\
66.5833333333333	0.21292	-667.950017349098\\
66.5833333333333	0.21458	-694.814954823491\\
66.5833333333333	0.21624	-721.679892297885\\
66.5833333333333	0.2179	-748.544829772279\\
66.5833333333333	0.21956	-775.409767246671\\
66.5833333333333	0.22122	-802.274704721066\\
66.5833333333333	0.22288	-829.139642195461\\
66.5833333333333	0.22454	-856.004579669854\\
66.5833333333333	0.2262	-882.869517144247\\
66.5833333333333	0.22786	-909.73445461864\\
66.5833333333333	0.22952	-936.599392093033\\
66.5833333333333	0.23118	-963.464329567428\\
66.5833333333333	0.23284	-990.32926704182\\
66.5833333333333	0.2345	-1017.19420451622\\
66.5833333333333	0.23616	-1044.05914199061\\
66.5833333333333	0.23782	-1070.924079465\\
66.5833333333333	0.23948	-1097.7890169394\\
66.5833333333333	0.24114	-1124.65395441379\\
66.5833333333333	0.2428	-1151.51889188818\\
66.5833333333333	0.24446	-1178.38382936258\\
66.5833333333333	0.24612	-1205.24876683697\\
66.5833333333333	0.24778	-1232.11370431136\\
66.5833333333333	0.24944	-1258.97864178576\\
66.5833333333333	0.2511	-1285.84357926015\\
66.5833333333333	0.25276	-1312.70851673454\\
66.5833333333333	0.25442	-1339.57345420894\\
66.5833333333333	0.25608	-1366.43839168333\\
66.5833333333333	0.25774	-1393.30332915773\\
66.5833333333333	0.2594	-1420.16826663212\\
66.5833333333333	0.26106	-1447.03320410651\\
66.5833333333333	0.26272	-1473.89814158091\\
66.5833333333333	0.26438	-1500.7630790553\\
66.5833333333333	0.26604	-1527.62801652969\\
66.5833333333333	0.2677	-1554.49295400409\\
66.5833333333333	0.26936	-1581.35789147848\\
66.5833333333333	0.27102	-1608.22282895288\\
66.5833333333333	0.27268	-1635.08776642727\\
66.5833333333333	0.27434	-1661.95270390166\\
66.5833333333333	0.276	-1688.81764137605\\
66.7916666666667	0.193	-340.08287879935\\
66.7916666666667	0.19466	-366.820264173283\\
66.7916666666667	0.19632	-393.557649547216\\
66.7916666666667	0.19798	-420.29503492115\\
66.7916666666667	0.19964	-447.032420295082\\
66.7916666666667	0.2013	-473.769805669016\\
66.7916666666667	0.20296	-500.507191042949\\
66.7916666666667	0.20462	-527.244576416882\\
66.7916666666667	0.20628	-553.981961790815\\
66.7916666666667	0.20794	-580.719347164748\\
66.7916666666667	0.2096	-607.456732538682\\
66.7916666666667	0.21126	-634.194117912615\\
66.7916666666667	0.21292	-660.931503286548\\
66.7916666666667	0.21458	-687.668888660481\\
66.7916666666667	0.21624	-714.406274034415\\
66.7916666666667	0.2179	-741.143659408348\\
66.7916666666667	0.21956	-767.881044782281\\
66.7916666666667	0.22122	-794.618430156215\\
66.7916666666667	0.22288	-821.355815530149\\
66.7916666666667	0.22454	-848.093200904082\\
66.7916666666667	0.2262	-874.830586278014\\
66.7916666666667	0.22786	-901.567971651947\\
66.7916666666667	0.22952	-928.30535702588\\
66.7916666666667	0.23118	-955.042742399814\\
66.7916666666667	0.23284	-981.780127773746\\
66.7916666666667	0.2345	-1008.51751314768\\
66.7916666666667	0.23616	-1035.25489852161\\
66.7916666666667	0.23782	-1061.99228389555\\
66.7916666666667	0.23948	-1088.72966926948\\
66.7916666666667	0.24114	-1115.46705464341\\
66.7916666666667	0.2428	-1142.20444001735\\
66.7916666666667	0.24446	-1168.94182539128\\
66.7916666666667	0.24612	-1195.67921076521\\
66.7916666666667	0.24778	-1222.41659613915\\
66.7916666666667	0.24944	-1249.15398151308\\
66.7916666666667	0.2511	-1275.89136688701\\
66.7916666666667	0.25276	-1302.62875226095\\
66.7916666666667	0.25442	-1329.36613763488\\
66.7916666666667	0.25608	-1356.10352300881\\
66.7916666666667	0.25774	-1382.84090838275\\
66.7916666666667	0.2594	-1409.57829375668\\
66.7916666666667	0.26106	-1436.31567913061\\
66.7916666666667	0.26272	-1463.05306450454\\
66.7916666666667	0.26438	-1489.79044987848\\
66.7916666666667	0.26604	-1516.52783525241\\
66.7916666666667	0.2677	-1543.26522062634\\
66.7916666666667	0.26936	-1570.00260600028\\
66.7916666666667	0.27102	-1596.73999137421\\
66.7916666666667	0.27268	-1623.47737674815\\
66.7916666666667	0.27434	-1650.21476212208\\
66.7916666666667	0.276	-1676.95214749601\\
67	0.193	-334.594989942326\\
67	0.19466	-361.204823215798\\
67	0.19632	-387.81465648927\\
67	0.19798	-414.424489762743\\
67	0.19964	-441.034323036216\\
67	0.2013	-467.644156309691\\
67	0.20296	-494.253989583162\\
67	0.20462	-520.863822856635\\
67	0.20628	-547.473656130109\\
67	0.20794	-574.083489403581\\
67	0.2096	-600.693322677055\\
67	0.21126	-627.303155950527\\
67	0.21292	-653.912989224\\
67	0.21458	-680.522822497472\\
67	0.21624	-707.132655770945\\
67	0.2179	-733.742489044418\\
67	0.21956	-760.35232231789\\
67	0.22122	-786.962155591365\\
67	0.22288	-813.571988864838\\
67	0.22454	-840.18182213831\\
67	0.2262	-866.791655411783\\
67	0.22786	-893.401488685255\\
67	0.22952	-920.011321958727\\
67	0.23118	-946.621155232201\\
67	0.23284	-973.230988505673\\
67	0.2345	-999.840821779148\\
67	0.23616	-1026.45065505262\\
67	0.23782	-1053.06048832609\\
67	0.23948	-1079.67032159957\\
67	0.24114	-1106.28015487304\\
67	0.2428	-1132.88998814651\\
67	0.24446	-1159.49982141998\\
67	0.24612	-1186.10965469346\\
67	0.24778	-1212.71948796693\\
67	0.24944	-1239.3293212404\\
67	0.2511	-1265.93915451387\\
67	0.25276	-1292.54898778735\\
67	0.25442	-1319.15882106082\\
67	0.25608	-1345.76865433429\\
67	0.25774	-1372.37848760777\\
67	0.2594	-1398.98832088124\\
67	0.26106	-1425.59815415471\\
67	0.26272	-1452.20798742818\\
67	0.26438	-1478.81782070166\\
67	0.26604	-1505.42765397513\\
67	0.2677	-1532.0374872486\\
67	0.26936	-1558.64732052208\\
67	0.27102	-1585.25715379555\\
67	0.27268	-1611.86698706902\\
67	0.27434	-1638.47682034249\\
67	0.276	-1665.08665361597\\
};
\end{axis}

\begin{axis}[%
width=4.927496cm,
height=3.050847cm,
at={(6.483547cm,4.237288cm)},
scale only axis,
xmin=57,
xmax=67,
tick align=outside,
xlabel={$L_{cut}$},
xmajorgrids,
ymin=0.193,
ymax=0.276,
ylabel={$D_{rlx}$},
ymajorgrids,
zmin=48.6240364128161,
zmax=137.389406622426,
zlabel={$x_3,x_4$},
zmajorgrids,
view={-140}{50},
legend style={at={(1.03,1)},anchor=north west,legend cell align=left,align=left,draw=white!15!black}
]
\addplot3[only marks,mark=*,mark options={},mark size=1.5000pt,color=mycolor1] plot table[row sep=crcr,]{%
67	0.276	137.389406622421\\
66	0.255	113.003944476115\\
62	0.209	63.4893800679793\\
57	0.193	48.6240364128161\\
};
\addplot3[only marks,mark=*,mark options={},mark size=1.5000pt,color=black] plot table[row sep=crcr,]{%
64	0.23	84.7015003408395\\
};

\addplot3[%
surf,
opacity=0.7,
shader=interp,
colormap={mymap}{[1pt] rgb(0pt)=(0.0901961,0.239216,0.0745098); rgb(1pt)=(0.0945149,0.242058,0.0739522); rgb(2pt)=(0.0988592,0.244894,0.0733566); rgb(3pt)=(0.103229,0.247724,0.0727241); rgb(4pt)=(0.107623,0.250549,0.0720557); rgb(5pt)=(0.112043,0.253367,0.0713525); rgb(6pt)=(0.116487,0.25618,0.0706154); rgb(7pt)=(0.120956,0.258986,0.0698456); rgb(8pt)=(0.125449,0.261787,0.0690441); rgb(9pt)=(0.129967,0.264581,0.0682118); rgb(10pt)=(0.134508,0.26737,0.06735); rgb(11pt)=(0.139074,0.270152,0.0664596); rgb(12pt)=(0.143663,0.272929,0.0655416); rgb(13pt)=(0.148275,0.275699,0.0645971); rgb(14pt)=(0.152911,0.278463,0.0636271); rgb(15pt)=(0.15757,0.281221,0.0626328); rgb(16pt)=(0.162252,0.283973,0.0616151); rgb(17pt)=(0.166957,0.286719,0.060575); rgb(18pt)=(0.171685,0.289458,0.0595136); rgb(19pt)=(0.176434,0.292191,0.0584321); rgb(20pt)=(0.181207,0.294918,0.0573313); rgb(21pt)=(0.186001,0.297639,0.0562123); rgb(22pt)=(0.190817,0.300353,0.0550763); rgb(23pt)=(0.195655,0.303061,0.0539242); rgb(24pt)=(0.200514,0.305763,0.052757); rgb(25pt)=(0.205395,0.308459,0.0515759); rgb(26pt)=(0.210296,0.311149,0.0503624); rgb(27pt)=(0.215212,0.313846,0.0490067); rgb(28pt)=(0.220142,0.316548,0.0475043); rgb(29pt)=(0.22509,0.319254,0.0458704); rgb(30pt)=(0.230056,0.321962,0.0441205); rgb(31pt)=(0.235042,0.324671,0.04227); rgb(32pt)=(0.240048,0.327379,0.0403343); rgb(33pt)=(0.245078,0.330085,0.0383287); rgb(34pt)=(0.250131,0.332786,0.0362688); rgb(35pt)=(0.25521,0.335482,0.0341698); rgb(36pt)=(0.260317,0.33817,0.0320472); rgb(37pt)=(0.265451,0.340849,0.0299163); rgb(38pt)=(0.270616,0.343517,0.0277927); rgb(39pt)=(0.275813,0.346172,0.0256916); rgb(40pt)=(0.281043,0.348814,0.0236284); rgb(41pt)=(0.286307,0.35144,0.0216186); rgb(42pt)=(0.291607,0.354048,0.0196776); rgb(43pt)=(0.296945,0.356637,0.0178207); rgb(44pt)=(0.302322,0.359206,0.0160634); rgb(45pt)=(0.307739,0.361753,0.0144211); rgb(46pt)=(0.313198,0.364275,0.0129091); rgb(47pt)=(0.318701,0.366772,0.0115428); rgb(48pt)=(0.324249,0.369242,0.0103377); rgb(49pt)=(0.329843,0.371682,0.00930909); rgb(50pt)=(0.335485,0.374093,0.00847245); rgb(51pt)=(0.341176,0.376471,0.00784314); rgb(52pt)=(0.346925,0.378826,0.00732741); rgb(53pt)=(0.352735,0.381168,0.00682184); rgb(54pt)=(0.358605,0.383497,0.00632729); rgb(55pt)=(0.364532,0.385812,0.00584464); rgb(56pt)=(0.370516,0.388113,0.00537476); rgb(57pt)=(0.376552,0.390399,0.00491852); rgb(58pt)=(0.38264,0.39267,0.00447681); rgb(59pt)=(0.388777,0.394925,0.00405048); rgb(60pt)=(0.394962,0.397164,0.00364042); rgb(61pt)=(0.401191,0.399386,0.00324749); rgb(62pt)=(0.407464,0.401592,0.00287258); rgb(63pt)=(0.413777,0.40378,0.00251655); rgb(64pt)=(0.420129,0.40595,0.00218028); rgb(65pt)=(0.426518,0.408102,0.00186463); rgb(66pt)=(0.432942,0.410234,0.00157049); rgb(67pt)=(0.439399,0.412348,0.00129873); rgb(68pt)=(0.445885,0.414441,0.00105022); rgb(69pt)=(0.452401,0.416515,0.000825833); rgb(70pt)=(0.458942,0.418567,0.000626441); rgb(71pt)=(0.465508,0.420599,0.00045292); rgb(72pt)=(0.472096,0.422609,0.000306141); rgb(73pt)=(0.478704,0.424596,0.000186979); rgb(74pt)=(0.485331,0.426562,9.63073e-05); rgb(75pt)=(0.491973,0.428504,3.49981e-05); rgb(76pt)=(0.498628,0.430422,3.92506e-06); rgb(77pt)=(0.505323,0.432315,0); rgb(78pt)=(0.512206,0.434168,0); rgb(79pt)=(0.519282,0.435983,0); rgb(80pt)=(0.526529,0.437764,0); rgb(81pt)=(0.533922,0.439512,0); rgb(82pt)=(0.54144,0.441232,0); rgb(83pt)=(0.549059,0.442927,0); rgb(84pt)=(0.556756,0.444599,0); rgb(85pt)=(0.564508,0.446252,0); rgb(86pt)=(0.572292,0.447889,0); rgb(87pt)=(0.580084,0.449514,0); rgb(88pt)=(0.587863,0.451129,0); rgb(89pt)=(0.595604,0.452737,0); rgb(90pt)=(0.603284,0.454343,0); rgb(91pt)=(0.610882,0.455948,0); rgb(92pt)=(0.618373,0.457556,0); rgb(93pt)=(0.625734,0.459171,0); rgb(94pt)=(0.632943,0.460795,0); rgb(95pt)=(0.639976,0.462432,0); rgb(96pt)=(0.64681,0.464084,0); rgb(97pt)=(0.653423,0.465756,0); rgb(98pt)=(0.659791,0.46745,0); rgb(99pt)=(0.665891,0.469169,0); rgb(100pt)=(0.6717,0.470916,0); rgb(101pt)=(0.677195,0.472696,0); rgb(102pt)=(0.682353,0.47451,0); rgb(103pt)=(0.687242,0.476355,0); rgb(104pt)=(0.691952,0.478225,0); rgb(105pt)=(0.696497,0.480118,0); rgb(106pt)=(0.700887,0.482033,0); rgb(107pt)=(0.705134,0.483968,0); rgb(108pt)=(0.709251,0.485921,0); rgb(109pt)=(0.713249,0.487891,0); rgb(110pt)=(0.71714,0.489876,0); rgb(111pt)=(0.720936,0.491875,0); rgb(112pt)=(0.724649,0.493887,0); rgb(113pt)=(0.72829,0.495909,0); rgb(114pt)=(0.731872,0.49794,0); rgb(115pt)=(0.735406,0.499979,0); rgb(116pt)=(0.738904,0.502025,0); rgb(117pt)=(0.742378,0.504075,0); rgb(118pt)=(0.74584,0.506128,0); rgb(119pt)=(0.749302,0.508182,0); rgb(120pt)=(0.752775,0.510237,0); rgb(121pt)=(0.756272,0.51229,0); rgb(122pt)=(0.759804,0.514339,0); rgb(123pt)=(0.763384,0.516385,0); rgb(124pt)=(0.767022,0.518424,0); rgb(125pt)=(0.770731,0.520455,0); rgb(126pt)=(0.774523,0.522478,0); rgb(127pt)=(0.77841,0.524489,0); rgb(128pt)=(0.782391,0.526491,0); rgb(129pt)=(0.786402,0.528496,0); rgb(130pt)=(0.790431,0.530506,0); rgb(131pt)=(0.794478,0.532521,0); rgb(132pt)=(0.798541,0.534539,0); rgb(133pt)=(0.802619,0.53656,0); rgb(134pt)=(0.806712,0.538584,0); rgb(135pt)=(0.81082,0.540609,0); rgb(136pt)=(0.81494,0.542635,0); rgb(137pt)=(0.819074,0.54466,0); rgb(138pt)=(0.823219,0.546686,0); rgb(139pt)=(0.827374,0.548709,0); rgb(140pt)=(0.831541,0.55073,0); rgb(141pt)=(0.835716,0.552749,0); rgb(142pt)=(0.8399,0.554763,0); rgb(143pt)=(0.844092,0.556774,0); rgb(144pt)=(0.848292,0.558779,0); rgb(145pt)=(0.852497,0.560778,0); rgb(146pt)=(0.856708,0.562771,0); rgb(147pt)=(0.860924,0.564756,0); rgb(148pt)=(0.865143,0.566733,0); rgb(149pt)=(0.869366,0.568701,0); rgb(150pt)=(0.873592,0.57066,0); rgb(151pt)=(0.877819,0.572608,0); rgb(152pt)=(0.882047,0.574545,0); rgb(153pt)=(0.886275,0.576471,0); rgb(154pt)=(0.890659,0.578362,0); rgb(155pt)=(0.895333,0.580203,0); rgb(156pt)=(0.900258,0.581999,0); rgb(157pt)=(0.905397,0.583755,0); rgb(158pt)=(0.910711,0.585479,0); rgb(159pt)=(0.916164,0.587176,0); rgb(160pt)=(0.921717,0.588852,0); rgb(161pt)=(0.927333,0.590513,0); rgb(162pt)=(0.932974,0.592166,0); rgb(163pt)=(0.938602,0.593815,0); rgb(164pt)=(0.94418,0.595468,0); rgb(165pt)=(0.949669,0.59713,0); rgb(166pt)=(0.955033,0.598808,0); rgb(167pt)=(0.960233,0.600507,0); rgb(168pt)=(0.965232,0.602233,0); rgb(169pt)=(0.969992,0.603992,0); rgb(170pt)=(0.974475,0.605791,0); rgb(171pt)=(0.978643,0.607636,0); rgb(172pt)=(0.98246,0.609532,0); rgb(173pt)=(0.985886,0.611486,0); rgb(174pt)=(0.988885,0.613503,0); rgb(175pt)=(0.991419,0.61559,0); rgb(176pt)=(0.99345,0.617753,0); rgb(177pt)=(0.99494,0.619997,0); rgb(178pt)=(0.995851,0.622329,0); rgb(179pt)=(0.996226,0.624763,0); rgb(180pt)=(0.996512,0.627352,0); rgb(181pt)=(0.996788,0.630095,0); rgb(182pt)=(0.997053,0.632982,0); rgb(183pt)=(0.997308,0.636004,0); rgb(184pt)=(0.997552,0.639152,0); rgb(185pt)=(0.997785,0.642416,0); rgb(186pt)=(0.998006,0.645786,0); rgb(187pt)=(0.998217,0.649253,0); rgb(188pt)=(0.998416,0.652807,0); rgb(189pt)=(0.998605,0.656439,0); rgb(190pt)=(0.998781,0.660138,0); rgb(191pt)=(0.998946,0.663897,0); rgb(192pt)=(0.9991,0.667704,0); rgb(193pt)=(0.999242,0.67155,0); rgb(194pt)=(0.999372,0.675427,0); rgb(195pt)=(0.99949,0.679323,0); rgb(196pt)=(0.999596,0.68323,0); rgb(197pt)=(0.99969,0.687139,0); rgb(198pt)=(0.999771,0.691039,0); rgb(199pt)=(0.999841,0.694921,0); rgb(200pt)=(0.999898,0.698775,0); rgb(201pt)=(0.999942,0.702592,0); rgb(202pt)=(0.999974,0.706363,0); rgb(203pt)=(0.999994,0.710077,0); rgb(204pt)=(1,0.713725,0); rgb(205pt)=(1,0.717341,0); rgb(206pt)=(1,0.720963,0); rgb(207pt)=(1,0.724591,0); rgb(208pt)=(1,0.728226,0); rgb(209pt)=(1,0.731867,0); rgb(210pt)=(1,0.735514,0); rgb(211pt)=(1,0.739167,0); rgb(212pt)=(1,0.742827,0); rgb(213pt)=(1,0.746493,0); rgb(214pt)=(1,0.750165,0); rgb(215pt)=(1,0.753843,0); rgb(216pt)=(1,0.757527,0); rgb(217pt)=(1,0.761217,0); rgb(218pt)=(1,0.764913,0); rgb(219pt)=(1,0.768615,0); rgb(220pt)=(1,0.772324,0); rgb(221pt)=(1,0.776038,0); rgb(222pt)=(1,0.779758,0); rgb(223pt)=(1,0.783484,0); rgb(224pt)=(1,0.787215,0); rgb(225pt)=(1,0.790953,0); rgb(226pt)=(1,0.794696,0); rgb(227pt)=(1,0.798445,0); rgb(228pt)=(1,0.8022,0); rgb(229pt)=(1,0.805961,0); rgb(230pt)=(1,0.809727,0); rgb(231pt)=(1,0.8135,0); rgb(232pt)=(1,0.817278,0); rgb(233pt)=(1,0.821063,0); rgb(234pt)=(1,0.824854,0); rgb(235pt)=(1,0.828652,0); rgb(236pt)=(1,0.832455,0); rgb(237pt)=(1,0.836265,0); rgb(238pt)=(1,0.840081,0); rgb(239pt)=(1,0.843903,0); rgb(240pt)=(1,0.847732,0); rgb(241pt)=(1,0.851566,0); rgb(242pt)=(1,0.855406,0); rgb(243pt)=(1,0.859253,0); rgb(244pt)=(1,0.863106,0); rgb(245pt)=(1,0.866964,0); rgb(246pt)=(1,0.870829,0); rgb(247pt)=(1,0.8747,0); rgb(248pt)=(1,0.878577,0); rgb(249pt)=(1,0.88246,0); rgb(250pt)=(1,0.886349,0); rgb(251pt)=(1,0.890243,0); rgb(252pt)=(1,0.894144,0); rgb(253pt)=(1,0.898051,0); rgb(254pt)=(1,0.901964,0); rgb(255pt)=(1,0.905882,0)},
mesh/rows=49]
table[row sep=crcr,header=false] {%
%
57	0.193	48.6240364128203\\
57	0.19466	49.7501557830979\\
57	0.19632	50.8762751533755\\
57	0.19798	52.0023945236532\\
57	0.19964	53.1285138939308\\
57	0.2013	54.2546332642085\\
57	0.20296	55.380752634486\\
57	0.20462	56.5068720047637\\
57	0.20628	57.6329913750413\\
57	0.20794	58.7591107453189\\
57	0.2096	59.8852301155966\\
57	0.21126	61.0113494858742\\
57	0.21292	62.1374688561519\\
57	0.21458	63.2635882264295\\
57	0.21624	64.3897075967071\\
57	0.2179	65.5158269669848\\
57	0.21956	66.6419463372624\\
57	0.22122	67.76806570754\\
57	0.22288	68.8941850778177\\
57	0.22454	70.0203044480953\\
57	0.2262	71.146423818373\\
57	0.22786	72.2725431886506\\
57	0.22952	73.3986625589282\\
57	0.23118	74.5247819292059\\
57	0.23284	75.6509012994835\\
57	0.2345	76.7770206697611\\
57	0.23616	77.9031400400387\\
57	0.23782	79.0292594103164\\
57	0.23948	80.155378780594\\
57	0.24114	81.2814981508716\\
57	0.2428	82.4076175211493\\
57	0.24446	83.5337368914269\\
57	0.24612	84.6598562617045\\
57	0.24778	85.7859756319822\\
57	0.24944	86.9120950022598\\
57	0.2511	88.0382143725375\\
57	0.25276	89.1643337428151\\
57	0.25442	90.2904531130927\\
57	0.25608	91.4165724833704\\
57	0.25774	92.542691853648\\
57	0.2594	93.6688112239256\\
57	0.26106	94.7949305942033\\
57	0.26272	95.9210499644809\\
57	0.26438	97.0471693347585\\
57	0.26604	98.1732887050362\\
57	0.2677	99.2994080753137\\
57	0.26936	100.425527445592\\
57	0.27102	101.551646815869\\
57	0.27268	102.677766186147\\
57	0.27434	103.803885556424\\
57	0.276	104.930004926702\\
57.2083333333333	0.193	48.6695916451524\\
57.2083333333333	0.19466	49.80832466149\\
57.2083333333333	0.19632	50.9470576778274\\
57.2083333333333	0.19798	52.085790694165\\
57.2083333333333	0.19964	53.2245237105026\\
57.2083333333333	0.2013	54.3632567268401\\
57.2083333333333	0.20296	55.5019897431777\\
57.2083333333333	0.20462	56.6407227595152\\
57.2083333333333	0.20628	57.7794557758527\\
57.2083333333333	0.20794	58.9181887921903\\
57.2083333333333	0.2096	60.0569218085278\\
57.2083333333333	0.21126	61.1956548248654\\
57.2083333333333	0.21292	62.3343878412029\\
57.2083333333333	0.21458	63.4731208575404\\
57.2083333333333	0.21624	64.611853873878\\
57.2083333333333	0.2179	65.7505868902155\\
57.2083333333333	0.21956	66.8893199065531\\
57.2083333333333	0.22122	68.0280529228906\\
57.2083333333333	0.22288	69.1667859392281\\
57.2083333333333	0.22454	70.3055189555657\\
57.2083333333333	0.2262	71.4442519719032\\
57.2083333333333	0.22786	72.5829849882408\\
57.2083333333333	0.22952	73.7217180045783\\
57.2083333333333	0.23118	74.8604510209158\\
57.2083333333333	0.23284	75.9991840372534\\
57.2083333333333	0.2345	77.137917053591\\
57.2083333333333	0.23616	78.2766500699285\\
57.2083333333333	0.23782	79.415383086266\\
57.2083333333333	0.23948	80.5541161026035\\
57.2083333333333	0.24114	81.6928491189411\\
57.2083333333333	0.2428	82.8315821352787\\
57.2083333333333	0.24446	83.9703151516161\\
57.2083333333333	0.24612	85.1090481679537\\
57.2083333333333	0.24778	86.2477811842913\\
57.2083333333333	0.24944	87.3865142006288\\
57.2083333333333	0.2511	88.5252472169663\\
57.2083333333333	0.25276	89.6639802333038\\
57.2083333333333	0.25442	90.8027132496414\\
57.2083333333333	0.25608	91.941446265979\\
57.2083333333333	0.25774	93.0801792823165\\
57.2083333333333	0.2594	94.2189122986541\\
57.2083333333333	0.26106	95.3576453149916\\
57.2083333333333	0.26272	96.4963783313291\\
57.2083333333333	0.26438	97.6351113476667\\
57.2083333333333	0.26604	98.7738443640042\\
57.2083333333333	0.2677	99.9125773803417\\
57.2083333333333	0.26936	101.051310396679\\
57.2083333333333	0.27102	102.190043413017\\
57.2083333333333	0.27268	103.328776429354\\
57.2083333333333	0.27434	104.467509445692\\
57.2083333333333	0.276	105.606242462029\\
57.4166666666667	0.193	48.7151468774847\\
57.4166666666667	0.19466	49.8664935398821\\
57.4166666666667	0.19632	51.0178402022796\\
57.4166666666667	0.19798	52.169186864677\\
57.4166666666667	0.19964	53.3205335270745\\
57.4166666666667	0.2013	54.471880189472\\
57.4166666666667	0.20296	55.6232268518694\\
57.4166666666667	0.20462	56.7745735142668\\
57.4166666666667	0.20628	57.9259201766643\\
57.4166666666667	0.20794	59.0772668390617\\
57.4166666666667	0.2096	60.2286135014591\\
57.4166666666667	0.21126	61.3799601638566\\
57.4166666666667	0.21292	62.531306826254\\
57.4166666666667	0.21458	63.6826534886515\\
57.4166666666667	0.21624	64.8340001510489\\
57.4166666666667	0.2179	65.9853468134464\\
57.4166666666667	0.21956	67.1366934758439\\
57.4166666666667	0.22122	68.2880401382413\\
57.4166666666667	0.22288	69.4393868006387\\
57.4166666666667	0.22454	70.5907334630362\\
57.4166666666667	0.2262	71.7420801254336\\
57.4166666666667	0.22786	72.8934267878311\\
57.4166666666667	0.22952	74.0447734502285\\
57.4166666666667	0.23118	75.196120112626\\
57.4166666666667	0.23284	76.3474667750235\\
57.4166666666667	0.2345	77.4988134374209\\
57.4166666666667	0.23616	78.6501600998183\\
57.4166666666667	0.23782	79.8015067622158\\
57.4166666666667	0.23948	80.9528534246132\\
57.4166666666667	0.24114	82.1042000870106\\
57.4166666666667	0.2428	83.2555467494081\\
57.4166666666667	0.24446	84.4068934118055\\
57.4166666666667	0.24612	85.558240074203\\
57.4166666666667	0.24778	86.7095867366004\\
57.4166666666667	0.24944	87.8609333989979\\
57.4166666666667	0.2511	89.0122800613954\\
57.4166666666667	0.25276	90.1636267237927\\
57.4166666666667	0.25442	91.3149733861903\\
57.4166666666667	0.25608	92.4663200485877\\
57.4166666666667	0.25774	93.6176667109851\\
57.4166666666667	0.2594	94.7690133733826\\
57.4166666666667	0.26106	95.92036003578\\
57.4166666666667	0.26272	97.0717066981775\\
57.4166666666667	0.26438	98.2230533605749\\
57.4166666666667	0.26604	99.3744000229724\\
57.4166666666667	0.2677	100.52574668537\\
57.4166666666667	0.26936	101.677093347767\\
57.4166666666667	0.27102	102.828440010165\\
57.4166666666667	0.27268	103.979786672562\\
57.4166666666667	0.27434	105.13113333496\\
57.4166666666667	0.276	106.282479997357\\
57.625	0.193	48.760702109817\\
57.625	0.19466	49.9246624182744\\
57.625	0.19632	51.0886227267317\\
57.625	0.19798	52.252583035189\\
57.625	0.19964	53.4165433436464\\
57.625	0.2013	54.5805036521038\\
57.625	0.20296	55.7444639605611\\
57.625	0.20462	56.9084242690184\\
57.625	0.20628	58.0723845774758\\
57.625	0.20794	59.2363448859332\\
57.625	0.2096	60.4003051943905\\
57.625	0.21126	61.5642655028479\\
57.625	0.21292	62.7282258113052\\
57.625	0.21458	63.8921861197626\\
57.625	0.21624	65.05614642822\\
57.625	0.2179	66.2201067366773\\
57.625	0.21956	67.3840670451347\\
57.625	0.22122	68.548027353592\\
57.625	0.22288	69.7119876620494\\
57.625	0.22454	70.8759479705067\\
57.625	0.2262	72.039908278964\\
57.625	0.22786	73.2038685874214\\
57.625	0.22952	74.3678288958788\\
57.625	0.23118	75.5317892043361\\
57.625	0.23284	76.6957495127935\\
57.625	0.2345	77.8597098212509\\
57.625	0.23616	79.0236701297082\\
57.625	0.23782	80.1876304381656\\
57.625	0.23948	81.3515907466229\\
57.625	0.24114	82.5155510550803\\
57.625	0.2428	83.6795113635376\\
57.625	0.24446	84.843471671995\\
57.625	0.24612	86.0074319804523\\
57.625	0.24778	87.1713922889097\\
57.625	0.24944	88.335352597367\\
57.625	0.2511	89.4993129058244\\
57.625	0.25276	90.6632732142817\\
57.625	0.25442	91.827233522739\\
57.625	0.25608	92.9911938311965\\
57.625	0.25774	94.1551541396539\\
57.625	0.2594	95.3191144481111\\
57.625	0.26106	96.4830747565686\\
57.625	0.26272	97.6470350650258\\
57.625	0.26438	98.8109953734832\\
57.625	0.26604	99.9749556819406\\
57.625	0.2677	101.138915990398\\
57.625	0.26936	102.302876298855\\
57.625	0.27102	103.466836607313\\
57.625	0.27268	104.63079691577\\
57.625	0.27434	105.794757224227\\
57.625	0.276	106.958717532685\\
57.8333333333333	0.193	48.8062573421493\\
57.8333333333333	0.19466	49.9828312966665\\
57.8333333333333	0.19632	51.1594052511837\\
57.8333333333333	0.19798	52.3359792057011\\
57.8333333333333	0.19964	53.5125531602183\\
57.8333333333333	0.2013	54.6891271147356\\
57.8333333333333	0.20296	55.8657010692528\\
57.8333333333333	0.20462	57.0422750237701\\
57.8333333333333	0.20628	58.2188489782873\\
57.8333333333333	0.20794	59.3954229328046\\
57.8333333333333	0.2096	60.5719968873218\\
57.8333333333333	0.21126	61.7485708418391\\
57.8333333333333	0.21292	62.9251447963563\\
57.8333333333333	0.21458	64.1017187508736\\
57.8333333333333	0.21624	65.2782927053909\\
57.8333333333333	0.2179	66.4548666599081\\
57.8333333333333	0.21956	67.6314406144255\\
57.8333333333333	0.22122	68.8080145689427\\
57.8333333333333	0.22288	69.9845885234599\\
57.8333333333333	0.22454	71.1611624779772\\
57.8333333333333	0.2262	72.3377364324945\\
57.8333333333333	0.22786	73.5143103870117\\
57.8333333333333	0.22952	74.690884341529\\
57.8333333333333	0.23118	75.8674582960463\\
57.8333333333333	0.23284	77.0440322505635\\
57.8333333333333	0.2345	78.2206062050808\\
57.8333333333333	0.23616	79.397180159598\\
57.8333333333333	0.23782	80.5737541141153\\
57.8333333333333	0.23948	81.7503280686325\\
57.8333333333333	0.24114	82.9269020231498\\
57.8333333333333	0.2428	84.1034759776671\\
57.8333333333333	0.24446	85.2800499321843\\
57.8333333333333	0.24612	86.4566238867015\\
57.8333333333333	0.24778	87.6331978412189\\
57.8333333333333	0.24944	88.8097717957361\\
57.8333333333333	0.2511	89.9863457502534\\
57.8333333333333	0.25276	91.1629197047706\\
57.8333333333333	0.25442	92.3394936592879\\
57.8333333333333	0.25608	93.5160676138052\\
57.8333333333333	0.25774	94.6926415683225\\
57.8333333333333	0.2594	95.8692155228396\\
57.8333333333333	0.26106	97.0457894773569\\
57.8333333333333	0.26272	98.2223634318742\\
57.8333333333333	0.26438	99.3989373863915\\
57.8333333333333	0.26604	100.575511340909\\
57.8333333333333	0.2677	101.752085295426\\
57.8333333333333	0.26936	102.928659249943\\
57.8333333333333	0.27102	104.105233204461\\
57.8333333333333	0.27268	105.281807158978\\
57.8333333333333	0.27434	106.458381113495\\
57.8333333333333	0.276	107.634955068012\\
58.0416666666667	0.193	48.8518125744815\\
58.0416666666667	0.19466	50.0410001750587\\
58.0416666666667	0.19632	51.2301877756358\\
58.0416666666667	0.19798	52.419375376213\\
58.0416666666667	0.19964	53.6085629767902\\
58.0416666666667	0.2013	54.7977505773674\\
58.0416666666667	0.20296	55.9869381779445\\
58.0416666666667	0.20462	57.1761257785217\\
58.0416666666667	0.20628	58.3653133790989\\
58.0416666666667	0.20794	59.554500979676\\
58.0416666666667	0.2096	60.7436885802531\\
58.0416666666667	0.21126	61.9328761808304\\
58.0416666666667	0.21292	63.1220637814075\\
58.0416666666667	0.21458	64.3112513819847\\
58.0416666666667	0.21624	65.5004389825619\\
58.0416666666667	0.2179	66.689626583139\\
58.0416666666667	0.21956	67.8788141837162\\
58.0416666666667	0.22122	69.0680017842934\\
58.0416666666667	0.22288	70.2571893848705\\
58.0416666666667	0.22454	71.4463769854477\\
58.0416666666667	0.2262	72.6355645860249\\
58.0416666666667	0.22786	73.824752186602\\
58.0416666666667	0.22952	75.0139397871792\\
58.0416666666667	0.23118	76.2031273877564\\
58.0416666666667	0.23284	77.3923149883336\\
58.0416666666667	0.2345	78.5815025889107\\
58.0416666666667	0.23616	79.7706901894879\\
58.0416666666667	0.23782	80.959877790065\\
58.0416666666667	0.23948	82.1490653906422\\
58.0416666666667	0.24114	83.3382529912193\\
58.0416666666667	0.2428	84.5274405917965\\
58.0416666666667	0.24446	85.7166281923737\\
58.0416666666667	0.24612	86.9058157929508\\
58.0416666666667	0.24778	88.0950033935281\\
58.0416666666667	0.24944	89.2841909941052\\
58.0416666666667	0.2511	90.4733785946823\\
58.0416666666667	0.25276	91.6625661952595\\
58.0416666666667	0.25442	92.8517537958367\\
58.0416666666667	0.25608	94.0409413964139\\
58.0416666666667	0.25774	95.2301289969911\\
58.0416666666667	0.2594	96.4193165975682\\
58.0416666666667	0.26106	97.6085041981455\\
58.0416666666667	0.26272	98.7976917987225\\
58.0416666666667	0.26438	99.9868793992997\\
58.0416666666667	0.26604	101.176066999877\\
58.0416666666667	0.2677	102.365254600454\\
58.0416666666667	0.26936	103.554442201031\\
58.0416666666667	0.27102	104.743629801608\\
58.0416666666667	0.27268	105.932817402186\\
58.0416666666667	0.27434	107.122005002763\\
58.0416666666667	0.276	108.31119260334\\
58.25	0.193	48.8973678068137\\
58.25	0.19466	50.0991690534508\\
58.25	0.19632	51.3009703000878\\
58.25	0.19798	52.5027715467249\\
58.25	0.19964	53.704572793362\\
58.25	0.2013	54.9063740399991\\
58.25	0.20296	56.1081752866361\\
58.25	0.20462	57.3099765332732\\
58.25	0.20628	58.5117777799103\\
58.25	0.20794	59.7135790265474\\
58.25	0.2096	60.9153802731844\\
58.25	0.21126	62.1171815198215\\
58.25	0.21292	63.3189827664586\\
58.25	0.21458	64.5207840130956\\
58.25	0.21624	65.7225852597327\\
58.25	0.2179	66.9243865063698\\
58.25	0.21956	68.1261877530069\\
58.25	0.22122	69.327988999644\\
58.25	0.22288	70.529790246281\\
58.25	0.22454	71.7315914929181\\
58.25	0.2262	72.9333927395552\\
58.25	0.22786	74.1351939861922\\
58.25	0.22952	75.3369952328293\\
58.25	0.23118	76.5387964794664\\
58.25	0.23284	77.7405977261035\\
58.25	0.2345	78.9423989727406\\
58.25	0.23616	80.1442002193776\\
58.25	0.23782	81.3460014660147\\
58.25	0.23948	82.5478027126518\\
58.25	0.24114	83.7496039592888\\
58.25	0.2428	84.9514052059259\\
58.25	0.24446	86.153206452563\\
58.25	0.24612	87.3550076992\\
58.25	0.24778	88.5568089458372\\
58.25	0.24944	89.7586101924742\\
58.25	0.2511	90.9604114391113\\
58.25	0.25276	92.1622126857484\\
58.25	0.25442	93.3640139323855\\
58.25	0.25608	94.5658151790226\\
58.25	0.25774	95.7676164256596\\
58.25	0.2594	96.9694176722967\\
58.25	0.26106	98.1712189189338\\
58.25	0.26272	99.3730201655708\\
58.25	0.26438	100.574821412208\\
58.25	0.26604	101.776622658845\\
58.25	0.2677	102.978423905482\\
58.25	0.26936	104.180225152119\\
58.25	0.27102	105.382026398756\\
58.25	0.27268	106.583827645393\\
58.25	0.27434	107.78562889203\\
58.25	0.276	108.987430138667\\
58.4583333333333	0.193	48.942923039146\\
58.4583333333333	0.19466	50.157337931843\\
58.4583333333333	0.19632	51.3717528245399\\
58.4583333333333	0.19798	52.5861677172369\\
58.4583333333333	0.19964	53.8005826099339\\
58.4583333333333	0.2013	55.0149975026309\\
58.4583333333333	0.20296	56.2294123953279\\
58.4583333333333	0.20462	57.4438272880248\\
58.4583333333333	0.20628	58.6582421807219\\
58.4583333333333	0.20794	59.8726570734188\\
58.4583333333333	0.2096	61.0870719661157\\
58.4583333333333	0.21126	62.3014868588128\\
58.4583333333333	0.21292	63.5159017515097\\
58.4583333333333	0.21458	64.7303166442067\\
58.4583333333333	0.21624	65.9447315369037\\
58.4583333333333	0.2179	67.1591464296007\\
58.4583333333333	0.21956	68.3735613222977\\
58.4583333333333	0.22122	69.5879762149946\\
58.4583333333333	0.22288	70.8023911076916\\
58.4583333333333	0.22454	72.0168060003886\\
58.4583333333333	0.2262	73.2312208930856\\
58.4583333333333	0.22786	74.4456357857825\\
58.4583333333333	0.22952	75.6600506784796\\
58.4583333333333	0.23118	76.8744655711765\\
58.4583333333333	0.23284	78.0888804638735\\
58.4583333333333	0.2345	79.3032953565705\\
58.4583333333333	0.23616	80.5177102492675\\
58.4583333333333	0.23782	81.7321251419645\\
58.4583333333333	0.23948	82.9465400346614\\
58.4583333333333	0.24114	84.1609549273584\\
58.4583333333333	0.2428	85.3753698200554\\
58.4583333333333	0.24446	86.5897847127524\\
58.4583333333333	0.24612	87.8041996054493\\
58.4583333333333	0.24778	89.0186144981463\\
58.4583333333333	0.24944	90.2330293908433\\
58.4583333333333	0.2511	91.4474442835403\\
58.4583333333333	0.25276	92.6618591762372\\
58.4583333333333	0.25442	93.8762740689342\\
58.4583333333333	0.25608	95.0906889616313\\
58.4583333333333	0.25774	96.3051038543283\\
58.4583333333333	0.2594	97.5195187470252\\
58.4583333333333	0.26106	98.7339336397222\\
58.4583333333333	0.26272	99.9483485324192\\
58.4583333333333	0.26438	101.162763425116\\
58.4583333333333	0.26604	102.377178317813\\
58.4583333333333	0.2677	103.59159321051\\
58.4583333333333	0.26936	104.806008103207\\
58.4583333333333	0.27102	106.020422995904\\
58.4583333333333	0.27268	107.234837888601\\
58.4583333333333	0.27434	108.449252781298\\
58.4583333333333	0.276	109.663667673995\\
58.6666666666667	0.193	48.9884782714782\\
58.6666666666667	0.19466	50.2155068102351\\
58.6666666666667	0.19632	51.442535348992\\
58.6666666666667	0.19798	52.6695638877489\\
58.6666666666667	0.19964	53.8965924265058\\
58.6666666666667	0.2013	55.1236209652627\\
58.6666666666667	0.20296	56.3506495040195\\
58.6666666666667	0.20462	57.5776780427764\\
58.6666666666667	0.20628	58.8047065815334\\
58.6666666666667	0.20794	60.0317351202903\\
58.6666666666667	0.2096	61.2587636590471\\
58.6666666666667	0.21126	62.485792197804\\
58.6666666666667	0.21292	63.7128207365609\\
58.6666666666667	0.21458	64.9398492753177\\
58.6666666666667	0.21624	66.1668778140747\\
58.6666666666667	0.2179	67.3939063528316\\
58.6666666666667	0.21956	68.6209348915885\\
58.6666666666667	0.22122	69.8479634303453\\
58.6666666666667	0.22288	71.0749919691022\\
58.6666666666667	0.22454	72.3020205078591\\
58.6666666666667	0.2262	73.529049046616\\
58.6666666666667	0.22786	74.7560775853729\\
58.6666666666667	0.22952	75.9831061241298\\
58.6666666666667	0.23118	77.2101346628867\\
58.6666666666667	0.23284	78.4371632016436\\
58.6666666666667	0.2345	79.6641917404004\\
58.6666666666667	0.23616	80.8912202791573\\
58.6666666666667	0.23782	82.1182488179142\\
58.6666666666667	0.23948	83.3452773566711\\
58.6666666666667	0.24114	84.5723058954279\\
58.6666666666667	0.2428	85.7993344341849\\
58.6666666666667	0.24446	87.0263629729417\\
58.6666666666667	0.24612	88.2533915116986\\
58.6666666666667	0.24778	89.4804200504555\\
58.6666666666667	0.24944	90.7074485892124\\
58.6666666666667	0.2511	91.9344771279693\\
58.6666666666667	0.25276	93.1615056667262\\
58.6666666666667	0.25442	94.388534205483\\
58.6666666666667	0.25608	95.61556274424\\
58.6666666666667	0.25774	96.8425912829969\\
58.6666666666667	0.2594	98.0696198217537\\
58.6666666666667	0.26106	99.2966483605107\\
58.6666666666667	0.26272	100.523676899268\\
58.6666666666667	0.26438	101.750705438024\\
58.6666666666667	0.26604	102.977733976781\\
58.6666666666667	0.2677	104.204762515538\\
58.6666666666667	0.26936	105.431791054295\\
58.6666666666667	0.27102	106.658819593052\\
58.6666666666667	0.27268	107.885848131809\\
58.6666666666667	0.27434	109.112876670566\\
58.6666666666667	0.276	110.339905209323\\
58.875	0.193	49.0340335038105\\
58.875	0.19466	50.2736756886272\\
58.875	0.19632	51.513317873444\\
58.875	0.19798	52.7529600582608\\
58.875	0.19964	53.9926022430776\\
58.875	0.2013	55.2322444278944\\
58.875	0.20296	56.4718866127112\\
58.875	0.20462	57.711528797528\\
58.875	0.20628	58.9511709823448\\
58.875	0.20794	60.1908131671616\\
58.875	0.2096	61.4304553519784\\
58.875	0.21126	62.6700975367952\\
58.875	0.21292	63.909739721612\\
58.875	0.21458	65.1493819064287\\
58.875	0.21624	66.3890240912456\\
58.875	0.2179	67.6286662760623\\
58.875	0.21956	68.8683084608792\\
58.875	0.22122	70.107950645696\\
58.875	0.22288	71.3475928305127\\
58.875	0.22454	72.5872350153296\\
58.875	0.2262	73.8268772001463\\
58.875	0.22786	75.0665193849631\\
58.875	0.22952	76.3061615697799\\
58.875	0.23118	77.5458037545967\\
58.875	0.23284	78.7854459394135\\
58.875	0.2345	80.0250881242303\\
58.875	0.23616	81.2647303090471\\
58.875	0.23782	82.5043724938639\\
58.875	0.23948	83.7440146786807\\
58.875	0.24114	84.9836568634975\\
58.875	0.2428	86.2232990483143\\
58.875	0.24446	87.4629412331311\\
58.875	0.24612	88.7025834179479\\
58.875	0.24778	89.9422256027647\\
58.875	0.24944	91.1818677875814\\
58.875	0.2511	92.4215099723982\\
58.875	0.25276	93.6611521572149\\
58.875	0.25442	94.9007943420318\\
58.875	0.25608	96.1404365268487\\
58.875	0.25774	97.3800787116654\\
58.875	0.2594	98.6197208964822\\
58.875	0.26106	99.859363081299\\
58.875	0.26272	101.099005266116\\
58.875	0.26438	102.338647450933\\
58.875	0.26604	103.578289635749\\
58.875	0.2677	104.817931820566\\
58.875	0.26936	106.057574005383\\
58.875	0.27102	107.2972161902\\
58.875	0.27268	108.536858375017\\
58.875	0.27434	109.776500559833\\
58.875	0.276	111.01614274465\\
59.0833333333333	0.193	49.0795887361427\\
59.0833333333333	0.19466	50.3318445670194\\
59.0833333333333	0.19632	51.5841003978961\\
59.0833333333333	0.19798	52.8363562287728\\
59.0833333333333	0.19964	54.0886120596495\\
59.0833333333333	0.2013	55.3408678905262\\
59.0833333333333	0.20296	56.5931237214029\\
59.0833333333333	0.20462	57.8453795522796\\
59.0833333333333	0.20628	59.0976353831563\\
59.0833333333333	0.20794	60.349891214033\\
59.0833333333333	0.2096	61.6021470449097\\
59.0833333333333	0.21126	62.8544028757864\\
59.0833333333333	0.21292	64.1066587066631\\
59.0833333333333	0.21458	65.3589145375398\\
59.0833333333333	0.21624	66.6111703684165\\
59.0833333333333	0.2179	67.8634261992932\\
59.0833333333333	0.21956	69.1156820301699\\
59.0833333333333	0.22122	70.3679378610466\\
59.0833333333333	0.22288	71.6201936919233\\
59.0833333333333	0.22454	72.8724495228\\
59.0833333333333	0.2262	74.1247053536767\\
59.0833333333333	0.22786	75.3769611845534\\
59.0833333333333	0.22952	76.6292170154301\\
59.0833333333333	0.23118	77.8814728463068\\
59.0833333333333	0.23284	79.1337286771836\\
59.0833333333333	0.2345	80.3859845080602\\
59.0833333333333	0.23616	81.6382403389369\\
59.0833333333333	0.23782	82.8904961698136\\
59.0833333333333	0.23948	84.1427520006903\\
59.0833333333333	0.24114	85.3950078315671\\
59.0833333333333	0.2428	86.6472636624437\\
59.0833333333333	0.24446	87.8995194933204\\
59.0833333333333	0.24612	89.1517753241972\\
59.0833333333333	0.24778	90.4040311550739\\
59.0833333333333	0.24944	91.6562869859505\\
59.0833333333333	0.2511	92.9085428168272\\
59.0833333333333	0.25276	94.1607986477039\\
59.0833333333333	0.25442	95.4130544785806\\
59.0833333333333	0.25608	96.6653103094573\\
59.0833333333333	0.25774	97.917566140334\\
59.0833333333333	0.2594	99.1698219712108\\
59.0833333333333	0.26106	100.422077802087\\
59.0833333333333	0.26272	101.674333632964\\
59.0833333333333	0.26438	102.926589463841\\
59.0833333333333	0.26604	104.178845294718\\
59.0833333333333	0.2677	105.431101125594\\
59.0833333333333	0.26936	106.683356956471\\
59.0833333333333	0.27102	107.935612787348\\
59.0833333333333	0.27268	109.187868618224\\
59.0833333333333	0.27434	110.440124449101\\
59.0833333333333	0.276	111.692380279978\\
59.2916666666667	0.193	49.125143968475\\
59.2916666666667	0.19466	50.3900134454116\\
59.2916666666667	0.19632	51.6548829223482\\
59.2916666666667	0.19798	52.9197523992848\\
59.2916666666667	0.19964	54.1846218762214\\
59.2916666666667	0.2013	55.449491353158\\
59.2916666666667	0.20296	56.7143608300946\\
59.2916666666667	0.20462	57.9792303070312\\
59.2916666666667	0.20628	59.2440997839678\\
59.2916666666667	0.20794	60.5089692609045\\
59.2916666666667	0.2096	61.773838737841\\
59.2916666666667	0.21126	63.0387082147777\\
59.2916666666667	0.21292	64.3035776917142\\
59.2916666666667	0.21458	65.5684471686508\\
59.2916666666667	0.21624	66.8333166455875\\
59.2916666666667	0.2179	68.0981861225241\\
59.2916666666667	0.21956	69.3630555994607\\
59.2916666666667	0.22122	70.6279250763973\\
59.2916666666667	0.22288	71.8927945533339\\
59.2916666666667	0.22454	73.1576640302706\\
59.2916666666667	0.2262	74.4225335072071\\
59.2916666666667	0.22786	75.6874029841437\\
59.2916666666667	0.22952	76.9522724610803\\
59.2916666666667	0.23118	78.2171419380169\\
59.2916666666667	0.23284	79.4820114149536\\
59.2916666666667	0.2345	80.7468808918902\\
59.2916666666667	0.23616	82.0117503688268\\
59.2916666666667	0.23782	83.2766198457634\\
59.2916666666667	0.23948	84.5414893226999\\
59.2916666666667	0.24114	85.8063587996365\\
59.2916666666667	0.2428	87.0712282765732\\
59.2916666666667	0.24446	88.3360977535098\\
59.2916666666667	0.24612	89.6009672304464\\
59.2916666666667	0.24778	90.865836707383\\
59.2916666666667	0.24944	92.1307061843196\\
59.2916666666667	0.2511	93.3955756612562\\
59.2916666666667	0.25276	94.6604451381928\\
59.2916666666667	0.25442	95.9253146151295\\
59.2916666666667	0.25608	97.190184092066\\
59.2916666666667	0.25774	98.4550535690026\\
59.2916666666667	0.2594	99.7199230459393\\
59.2916666666667	0.26106	100.984792522876\\
59.2916666666667	0.26272	102.249661999813\\
59.2916666666667	0.26438	103.514531476749\\
59.2916666666667	0.26604	104.779400953686\\
59.2916666666667	0.2677	106.044270430622\\
59.2916666666667	0.26936	107.309139907559\\
59.2916666666667	0.27102	108.574009384496\\
59.2916666666667	0.27268	109.838878861432\\
59.2916666666667	0.27434	111.103748338369\\
59.2916666666667	0.276	112.368617815305\\
59.5	0.193	49.1706992008073\\
59.5	0.19466	50.4481823238038\\
59.5	0.19632	51.7256654468003\\
59.5	0.19798	53.0031485697968\\
59.5	0.19964	54.2806316927933\\
59.5	0.2013	55.5581148157898\\
59.5	0.20296	56.8355979387863\\
59.5	0.20462	58.1130810617829\\
59.5	0.20628	59.3905641847794\\
59.5	0.20794	60.6680473077759\\
59.5	0.2096	61.9455304307724\\
59.5	0.21126	63.223013553769\\
59.5	0.21292	64.5004966767655\\
59.5	0.21458	65.7779797997619\\
59.5	0.21624	67.0554629227585\\
59.5	0.2179	68.332946045755\\
59.5	0.21956	69.6104291687515\\
59.5	0.22122	70.887912291748\\
59.5	0.22288	72.1653954147445\\
59.5	0.22454	73.4428785377411\\
59.5	0.2262	74.7203616607376\\
59.5	0.22786	75.9978447837341\\
59.5	0.22952	77.2753279067306\\
59.5	0.23118	78.5528110297271\\
59.5	0.23284	79.8302941527236\\
59.5	0.2345	81.1077772757201\\
59.5	0.23616	82.3852603987167\\
59.5	0.23782	83.6627435217132\\
59.5	0.23948	84.9402266447097\\
59.5	0.24114	86.2177097677062\\
59.5	0.2428	87.4951928907028\\
59.5	0.24446	88.7726760136993\\
59.5	0.24612	90.0501591366958\\
59.5	0.24778	91.3276422596923\\
59.5	0.24944	92.6051253826888\\
59.5	0.2511	93.8826085056852\\
59.5	0.25276	95.1600916286817\\
59.5	0.25442	96.4375747516783\\
59.5	0.25608	97.7150578746748\\
59.5	0.25774	98.9925409976713\\
59.5	0.2594	100.270024120668\\
59.5	0.26106	101.547507243664\\
59.5	0.26272	102.824990366661\\
59.5	0.26438	104.102473489657\\
59.5	0.26604	105.379956612654\\
59.5	0.2677	106.65743973565\\
59.5	0.26936	107.934922858647\\
59.5	0.27102	109.212405981643\\
59.5	0.27268	110.48988910464\\
59.5	0.27434	111.767372227637\\
59.5	0.276	113.044855350633\\
59.7083333333333	0.193	49.2162544331395\\
59.7083333333333	0.19466	50.5063512021958\\
59.7083333333333	0.19632	51.7964479712522\\
59.7083333333333	0.19798	53.0865447403087\\
59.7083333333333	0.19964	54.3766415093651\\
59.7083333333333	0.2013	55.6667382784216\\
59.7083333333333	0.20296	56.956835047478\\
59.7083333333333	0.20462	58.2469318165344\\
59.7083333333333	0.20628	59.5370285855908\\
59.7083333333333	0.20794	60.8271253546472\\
59.7083333333333	0.2096	62.1172221237036\\
59.7083333333333	0.21126	63.4073188927601\\
59.7083333333333	0.21292	64.6974156618165\\
59.7083333333333	0.21458	65.9875124308729\\
59.7083333333333	0.21624	67.2776091999293\\
59.7083333333333	0.2179	68.5677059689858\\
59.7083333333333	0.21956	69.8578027380422\\
59.7083333333333	0.22122	71.1478995070986\\
59.7083333333333	0.22288	72.437996276155\\
59.7083333333333	0.22454	73.7280930452114\\
59.7083333333333	0.2262	75.0181898142678\\
59.7083333333333	0.22786	76.3082865833243\\
59.7083333333333	0.22952	77.5983833523807\\
59.7083333333333	0.23118	78.8884801214371\\
59.7083333333333	0.23284	80.1785768904936\\
59.7083333333333	0.2345	81.4686736595499\\
59.7083333333333	0.23616	82.7587704286064\\
59.7083333333333	0.23782	84.0488671976628\\
59.7083333333333	0.23948	85.3389639667192\\
59.7083333333333	0.24114	86.6290607357756\\
59.7083333333333	0.2428	87.9191575048321\\
59.7083333333333	0.24446	89.2092542738885\\
59.7083333333333	0.24612	90.4993510429449\\
59.7083333333333	0.24778	91.7894478120013\\
59.7083333333333	0.24944	93.0795445810577\\
59.7083333333333	0.2511	94.3696413501141\\
59.7083333333333	0.25276	95.6597381191705\\
59.7083333333333	0.25442	96.949834888227\\
59.7083333333333	0.25608	98.2399316572834\\
59.7083333333333	0.25774	99.5300284263398\\
59.7083333333333	0.2594	100.820125195396\\
59.7083333333333	0.26106	102.110221964453\\
59.7083333333333	0.26272	103.400318733509\\
59.7083333333333	0.26438	104.690415502566\\
59.7083333333333	0.26604	105.980512271622\\
59.7083333333333	0.2677	107.270609040678\\
59.7083333333333	0.26936	108.560705809735\\
59.7083333333333	0.27102	109.850802578791\\
59.7083333333333	0.27268	111.140899347848\\
59.7083333333333	0.27434	112.430996116904\\
59.7083333333333	0.276	113.72109288596\\
59.9166666666667	0.193	49.2618096654717\\
59.9166666666667	0.19466	50.564520080588\\
59.9166666666667	0.19632	51.8672304957043\\
59.9166666666667	0.19798	53.1699409108207\\
59.9166666666667	0.19964	54.472651325937\\
59.9166666666667	0.2013	55.7753617410534\\
59.9166666666667	0.20296	57.0780721561697\\
59.9166666666667	0.20462	58.380782571286\\
59.9166666666667	0.20628	59.6834929864024\\
59.9166666666667	0.20794	60.9862034015187\\
59.9166666666667	0.2096	62.2889138166349\\
59.9166666666667	0.21126	63.5916242317513\\
59.9166666666667	0.21292	64.8943346468676\\
59.9166666666667	0.21458	66.1970450619839\\
59.9166666666667	0.21624	67.4997554771003\\
59.9166666666667	0.2179	68.8024658922166\\
59.9166666666667	0.21956	70.105176307333\\
59.9166666666667	0.22122	71.4078867224493\\
59.9166666666667	0.22288	72.7105971375656\\
59.9166666666667	0.22454	74.013307552682\\
59.9166666666667	0.2262	75.3160179677982\\
59.9166666666667	0.22786	76.6187283829146\\
59.9166666666667	0.22952	77.9214387980309\\
59.9166666666667	0.23118	79.2241492131472\\
59.9166666666667	0.23284	80.5268596282636\\
59.9166666666667	0.2345	81.8295700433799\\
59.9166666666667	0.23616	83.1322804584962\\
59.9166666666667	0.23782	84.4349908736126\\
59.9166666666667	0.23948	85.7377012887289\\
59.9166666666667	0.24114	87.0404117038452\\
59.9166666666667	0.2428	88.3431221189615\\
59.9166666666667	0.24446	89.6458325340779\\
59.9166666666667	0.24612	90.9485429491942\\
59.9166666666667	0.24778	92.2512533643106\\
59.9166666666667	0.24944	93.5539637794269\\
59.9166666666667	0.2511	94.8566741945431\\
59.9166666666667	0.25276	96.1593846096595\\
59.9166666666667	0.25442	97.4620950247758\\
59.9166666666667	0.25608	98.7648054398921\\
59.9166666666667	0.25774	100.067515855008\\
59.9166666666667	0.2594	101.370226270125\\
59.9166666666667	0.26106	102.672936685241\\
59.9166666666667	0.26272	103.975647100357\\
59.9166666666667	0.26438	105.278357515474\\
59.9166666666667	0.26604	106.58106793059\\
59.9166666666667	0.2677	107.883778345706\\
59.9166666666667	0.26936	109.186488760823\\
59.9166666666667	0.27102	110.489199175939\\
59.9166666666667	0.27268	111.791909591055\\
59.9166666666667	0.27434	113.094620006172\\
59.9166666666667	0.276	114.397330421288\\
60.125	0.193	49.307364897804\\
60.125	0.19466	50.6226889589802\\
60.125	0.19632	51.9380130201565\\
60.125	0.19798	53.2533370813327\\
60.125	0.19964	54.5686611425089\\
60.125	0.2013	55.8839852036852\\
60.125	0.20296	57.1993092648614\\
60.125	0.20462	58.5146333260377\\
60.125	0.20628	59.8299573872139\\
60.125	0.20794	61.1452814483901\\
60.125	0.2096	62.4606055095663\\
60.125	0.21126	63.7759295707426\\
60.125	0.21292	65.0912536319188\\
60.125	0.21458	66.406577693095\\
60.125	0.21624	67.7219017542713\\
60.125	0.2179	69.0372258154475\\
60.125	0.21956	70.3525498766238\\
60.125	0.22122	71.6678739378\\
60.125	0.22288	72.9831979989762\\
60.125	0.22454	74.2985220601525\\
60.125	0.2262	75.6138461213287\\
60.125	0.22786	76.9291701825049\\
60.125	0.22952	78.2444942436812\\
60.125	0.23118	79.5598183048574\\
60.125	0.23284	80.8751423660337\\
60.125	0.2345	82.1904664272099\\
60.125	0.23616	83.5057904883861\\
60.125	0.23782	84.8211145495624\\
60.125	0.23948	86.1364386107385\\
60.125	0.24114	87.4517626719148\\
60.125	0.2428	88.767086733091\\
60.125	0.24446	90.0824107942672\\
60.125	0.24612	91.3977348554435\\
60.125	0.24778	92.7130589166198\\
60.125	0.24944	94.028382977796\\
60.125	0.2511	95.3437070389722\\
60.125	0.25276	96.6590311001484\\
60.125	0.25442	97.9743551613247\\
60.125	0.25608	99.2896792225009\\
60.125	0.25774	100.605003283677\\
60.125	0.2594	101.920327344853\\
60.125	0.26106	103.23565140603\\
60.125	0.26272	104.550975467206\\
60.125	0.26438	105.866299528382\\
60.125	0.26604	107.181623589558\\
60.125	0.2677	108.496947650735\\
60.125	0.26936	109.812271711911\\
60.125	0.27102	111.127595773087\\
60.125	0.27268	112.442919834263\\
60.125	0.27434	113.758243895439\\
60.125	0.276	115.073567956616\\
60.3333333333333	0.193	49.3529201301363\\
60.3333333333333	0.19466	50.6808578373724\\
60.3333333333333	0.19632	52.0087955446085\\
60.3333333333333	0.19798	53.3367332518447\\
60.3333333333333	0.19964	54.6646709590808\\
60.3333333333333	0.2013	55.992608666317\\
60.3333333333333	0.20296	57.3205463735531\\
60.3333333333333	0.20462	58.6484840807892\\
60.3333333333333	0.20628	59.9764217880254\\
60.3333333333333	0.20794	61.3043594952615\\
60.3333333333333	0.2096	62.6322972024976\\
60.3333333333333	0.21126	63.9602349097338\\
60.3333333333333	0.21292	65.2881726169699\\
60.3333333333333	0.21458	66.6161103242061\\
60.3333333333333	0.21624	67.9440480314423\\
60.3333333333333	0.2179	69.2719857386784\\
60.3333333333333	0.21956	70.5999234459146\\
60.3333333333333	0.22122	71.9278611531507\\
60.3333333333333	0.22288	73.2557988603868\\
60.3333333333333	0.22454	74.583736567623\\
60.3333333333333	0.2262	75.9116742748591\\
60.3333333333333	0.22786	77.2396119820952\\
60.3333333333333	0.22952	78.5675496893314\\
60.3333333333333	0.23118	79.8954873965675\\
60.3333333333333	0.23284	81.2234251038037\\
60.3333333333333	0.2345	82.5513628110398\\
60.3333333333333	0.23616	83.8793005182759\\
60.3333333333333	0.23782	85.2072382255121\\
60.3333333333333	0.23948	86.5351759327482\\
60.3333333333333	0.24114	87.8631136399843\\
60.3333333333333	0.2428	89.1910513472205\\
60.3333333333333	0.24446	90.5189890544566\\
60.3333333333333	0.24612	91.8469267616928\\
60.3333333333333	0.24778	93.1748644689289\\
60.3333333333333	0.24944	94.502802176165\\
60.3333333333333	0.2511	95.8307398834011\\
60.3333333333333	0.25276	97.1586775906373\\
60.3333333333333	0.25442	98.4866152978735\\
60.3333333333333	0.25608	99.8145530051096\\
60.3333333333333	0.25774	101.142490712346\\
60.3333333333333	0.2594	102.470428419582\\
60.3333333333333	0.26106	103.798366126818\\
60.3333333333333	0.26272	105.126303834054\\
60.3333333333333	0.26438	106.45424154129\\
60.3333333333333	0.26604	107.782179248526\\
60.3333333333333	0.2677	109.110116955763\\
60.3333333333333	0.26936	110.438054662999\\
60.3333333333333	0.27102	111.765992370235\\
60.3333333333333	0.27268	113.093930077471\\
60.3333333333333	0.27434	114.421867784707\\
60.3333333333333	0.276	115.749805491943\\
60.5416666666667	0.193	49.3984753624686\\
60.5416666666667	0.19466	50.7390267157646\\
60.5416666666667	0.19632	52.0795780690606\\
60.5416666666667	0.19798	53.4201294223567\\
60.5416666666667	0.19964	54.7606807756527\\
60.5416666666667	0.2013	56.1012321289488\\
60.5416666666667	0.20296	57.4417834822448\\
60.5416666666667	0.20462	58.7823348355408\\
60.5416666666667	0.20628	60.122886188837\\
60.5416666666667	0.20794	61.463437542133\\
60.5416666666667	0.2096	62.803988895429\\
60.5416666666667	0.21126	64.1445402487251\\
60.5416666666667	0.21292	65.4850916020211\\
60.5416666666667	0.21458	66.8256429553171\\
60.5416666666667	0.21624	68.1661943086132\\
60.5416666666667	0.2179	69.5067456619092\\
60.5416666666667	0.21956	70.8472970152053\\
60.5416666666667	0.22122	72.1878483685014\\
60.5416666666667	0.22288	73.5283997217974\\
60.5416666666667	0.22454	74.8689510750934\\
60.5416666666667	0.2262	76.2095024283895\\
60.5416666666667	0.22786	77.5500537816855\\
60.5416666666667	0.22952	78.8906051349816\\
60.5416666666667	0.23118	80.2311564882776\\
60.5416666666667	0.23284	81.5717078415737\\
60.5416666666667	0.2345	82.9122591948698\\
60.5416666666667	0.23616	84.2528105481657\\
60.5416666666667	0.23782	85.5933619014619\\
60.5416666666667	0.23948	86.9339132547578\\
60.5416666666667	0.24114	88.2744646080539\\
60.5416666666667	0.2428	89.61501596135\\
60.5416666666667	0.24446	90.955567314646\\
60.5416666666667	0.24612	92.296118667942\\
60.5416666666667	0.24778	93.6366700212381\\
60.5416666666667	0.24944	94.9772213745342\\
60.5416666666667	0.2511	96.3177727278302\\
60.5416666666667	0.25276	97.6583240811262\\
60.5416666666667	0.25442	98.9988754344222\\
60.5416666666667	0.25608	100.339426787718\\
60.5416666666667	0.25774	101.679978141014\\
60.5416666666667	0.2594	103.02052949431\\
60.5416666666667	0.26106	104.361080847607\\
60.5416666666667	0.26272	105.701632200903\\
60.5416666666667	0.26438	107.042183554199\\
60.5416666666667	0.26604	108.382734907495\\
60.5416666666667	0.2677	109.723286260791\\
60.5416666666667	0.26936	111.063837614087\\
60.5416666666667	0.27102	112.404388967383\\
60.5416666666667	0.27268	113.744940320679\\
60.5416666666667	0.27434	115.085491673975\\
60.5416666666667	0.276	116.426043027271\\
60.75	0.193	49.4440305948008\\
60.75	0.19466	50.7971955941567\\
60.75	0.19632	52.1503605935126\\
60.75	0.19798	53.5035255928686\\
60.75	0.19964	54.8566905922245\\
60.75	0.2013	56.2098555915805\\
60.75	0.20296	57.5630205909364\\
60.75	0.20462	58.9161855902924\\
60.75	0.20628	60.2693505896484\\
60.75	0.20794	61.6225155890043\\
60.75	0.2096	62.9756805883603\\
60.75	0.21126	64.3288455877163\\
60.75	0.21292	65.6820105870722\\
60.75	0.21458	67.0351755864281\\
60.75	0.21624	68.3883405857841\\
60.75	0.2179	69.74150558514\\
60.75	0.21956	71.094670584496\\
60.75	0.22122	72.4478355838519\\
60.75	0.22288	73.8010005832079\\
60.75	0.22454	75.1541655825639\\
60.75	0.2262	76.5073305819198\\
60.75	0.22786	77.8604955812758\\
60.75	0.22952	79.2136605806318\\
60.75	0.23118	80.5668255799877\\
60.75	0.23284	81.9199905793437\\
60.75	0.2345	83.2731555786996\\
60.75	0.23616	84.6263205780556\\
60.75	0.23782	85.9794855774115\\
60.75	0.23948	87.3326505767675\\
60.75	0.24114	88.6858155761234\\
60.75	0.2428	90.0389805754793\\
60.75	0.24446	91.3921455748353\\
60.75	0.24612	92.7453105741913\\
60.75	0.24778	94.0984755735472\\
60.75	0.24944	95.4516405729032\\
60.75	0.2511	96.8048055722591\\
60.75	0.25276	98.157970571615\\
60.75	0.25442	99.511135570971\\
60.75	0.25608	100.864300570327\\
60.75	0.25774	102.217465569683\\
60.75	0.2594	103.570630569039\\
60.75	0.26106	104.923795568395\\
60.75	0.26272	106.276960567751\\
60.75	0.26438	107.630125567107\\
60.75	0.26604	108.983290566463\\
60.75	0.2677	110.336455565819\\
60.75	0.26936	111.689620565175\\
60.75	0.27102	113.042785564531\\
60.75	0.27268	114.395950563887\\
60.75	0.27434	115.749115563243\\
60.75	0.276	117.102280562598\\
60.9583333333333	0.193	49.489585827133\\
60.9583333333333	0.19466	50.8553644725488\\
60.9583333333333	0.19632	52.2211431179647\\
60.9583333333333	0.19798	53.5869217633806\\
60.9583333333333	0.19964	54.9527004087964\\
60.9583333333333	0.2013	56.3184790542123\\
60.9583333333333	0.20296	57.6842576996282\\
60.9583333333333	0.20462	59.050036345044\\
60.9583333333333	0.20628	60.4158149904599\\
60.9583333333333	0.20794	61.7815936358758\\
60.9583333333333	0.2096	63.1473722812916\\
60.9583333333333	0.21126	64.5131509267075\\
60.9583333333333	0.21292	65.8789295721233\\
60.9583333333333	0.21458	67.2447082175391\\
60.9583333333333	0.21624	68.610486862955\\
60.9583333333333	0.2179	69.9762655083709\\
60.9583333333333	0.21956	71.3420441537868\\
60.9583333333333	0.22122	72.7078227992026\\
60.9583333333333	0.22288	74.0736014446185\\
60.9583333333333	0.22454	75.4393800900344\\
60.9583333333333	0.2262	76.8051587354502\\
60.9583333333333	0.22786	78.1709373808661\\
60.9583333333333	0.22952	79.536716026282\\
60.9583333333333	0.23118	80.9024946716978\\
60.9583333333333	0.23284	82.2682733171137\\
60.9583333333333	0.2345	83.6340519625296\\
60.9583333333333	0.23616	84.9998306079454\\
60.9583333333333	0.23782	86.3656092533612\\
60.9583333333333	0.23948	87.7313878987771\\
60.9583333333333	0.24114	89.0971665441929\\
60.9583333333333	0.2428	90.4629451896088\\
60.9583333333333	0.24446	91.8287238350247\\
60.9583333333333	0.24612	93.1945024804405\\
60.9583333333333	0.24778	94.5602811258565\\
60.9583333333333	0.24944	95.9260597712723\\
60.9583333333333	0.2511	97.2918384166881\\
60.9583333333333	0.25276	98.657617062104\\
60.9583333333333	0.25442	100.02339570752\\
60.9583333333333	0.25608	101.389174352936\\
60.9583333333333	0.25774	102.754952998352\\
60.9583333333333	0.2594	104.120731643767\\
60.9583333333333	0.26106	105.486510289183\\
60.9583333333333	0.26272	106.852288934599\\
60.9583333333333	0.26438	108.218067580015\\
60.9583333333333	0.26604	109.583846225431\\
60.9583333333333	0.2677	110.949624870847\\
60.9583333333333	0.26936	112.315403516263\\
60.9583333333333	0.27102	113.681182161679\\
60.9583333333333	0.27268	115.046960807094\\
60.9583333333333	0.27434	116.41273945251\\
60.9583333333333	0.276	117.778518097926\\
61.1666666666667	0.193	49.5351410594653\\
61.1666666666667	0.19466	50.913533350941\\
61.1666666666667	0.19632	52.2919256424167\\
61.1666666666667	0.19798	53.6703179338926\\
61.1666666666667	0.19964	55.0487102253683\\
61.1666666666667	0.2013	56.4271025168441\\
61.1666666666667	0.20296	57.8054948083198\\
61.1666666666667	0.20462	59.1838870997956\\
61.1666666666667	0.20628	60.5622793912714\\
61.1666666666667	0.20794	61.9406716827472\\
61.1666666666667	0.2096	63.3190639742229\\
61.1666666666667	0.21126	64.6974562656987\\
61.1666666666667	0.21292	66.0758485571745\\
61.1666666666667	0.21458	67.4542408486502\\
61.1666666666667	0.21624	68.832633140126\\
61.1666666666667	0.2179	70.2110254316017\\
61.1666666666667	0.21956	71.5894177230776\\
61.1666666666667	0.22122	72.9678100145533\\
61.1666666666667	0.22288	74.3462023060291\\
61.1666666666667	0.22454	75.7245945975048\\
61.1666666666667	0.2262	77.1029868889806\\
61.1666666666667	0.22786	78.4813791804564\\
61.1666666666667	0.22952	79.8597714719322\\
61.1666666666667	0.23118	81.238163763408\\
61.1666666666667	0.23284	82.6165560548837\\
61.1666666666667	0.2345	83.9949483463595\\
61.1666666666667	0.23616	85.3733406378352\\
61.1666666666667	0.23782	86.7517329293111\\
61.1666666666667	0.23948	88.1301252207868\\
61.1666666666667	0.24114	89.5085175122625\\
61.1666666666667	0.2428	90.8869098037383\\
61.1666666666667	0.24446	92.2653020952141\\
61.1666666666667	0.24612	93.6436943866898\\
61.1666666666667	0.24778	95.0220866781656\\
61.1666666666667	0.24944	96.4004789696414\\
61.1666666666667	0.2511	97.778871261117\\
61.1666666666667	0.25276	99.1572635525929\\
61.1666666666667	0.25442	100.535655844069\\
61.1666666666667	0.25608	101.914048135544\\
61.1666666666667	0.25774	103.29244042702\\
61.1666666666667	0.2594	104.670832718496\\
61.1666666666667	0.26106	106.049225009972\\
61.1666666666667	0.26272	107.427617301447\\
61.1666666666667	0.26438	108.806009592923\\
61.1666666666667	0.26604	110.184401884399\\
61.1666666666667	0.2677	111.562794175875\\
61.1666666666667	0.26936	112.941186467351\\
61.1666666666667	0.27102	114.319578758826\\
61.1666666666667	0.27268	115.697971050302\\
61.1666666666667	0.27434	117.076363341778\\
61.1666666666667	0.276	118.454755633254\\
61.375	0.193	49.5806962917975\\
61.375	0.19466	50.9717022293331\\
61.375	0.19632	52.3627081668688\\
61.375	0.19798	53.7537141044045\\
61.375	0.19964	55.1447200419402\\
61.375	0.2013	56.5357259794759\\
61.375	0.20296	57.9267319170115\\
61.375	0.20462	59.3177378545471\\
61.375	0.20628	60.7087437920829\\
61.375	0.20794	62.0997497296186\\
61.375	0.2096	63.4907556671542\\
61.375	0.21126	64.8817616046899\\
61.375	0.21292	66.2727675422256\\
61.375	0.21458	67.6637734797612\\
61.375	0.21624	69.0547794172969\\
61.375	0.2179	70.4457853548326\\
61.375	0.21956	71.8367912923683\\
61.375	0.22122	73.227797229904\\
61.375	0.22288	74.6188031674396\\
61.375	0.22454	76.0098091049753\\
61.375	0.2262	77.4008150425109\\
61.375	0.22786	78.7918209800466\\
61.375	0.22952	80.1828269175824\\
61.375	0.23118	81.5738328551179\\
61.375	0.23284	82.9648387926537\\
61.375	0.2345	84.3558447301893\\
61.375	0.23616	85.746850667725\\
61.375	0.23782	87.1378566052607\\
61.375	0.23948	88.5288625427963\\
61.375	0.24114	89.919868480332\\
61.375	0.2428	91.3108744178678\\
61.375	0.24446	92.7018803554034\\
61.375	0.24612	94.092886292939\\
61.375	0.24778	95.4838922304747\\
61.375	0.24944	96.8748981680104\\
61.375	0.2511	98.265904105546\\
61.375	0.25276	99.6569100430817\\
61.375	0.25442	101.047915980617\\
61.375	0.25608	102.438921918153\\
61.375	0.25774	103.829927855689\\
61.375	0.2594	105.220933793224\\
61.375	0.26106	106.61193973076\\
61.375	0.26272	108.002945668296\\
61.375	0.26438	109.393951605831\\
61.375	0.26604	110.784957543367\\
61.375	0.2677	112.175963480903\\
61.375	0.26936	113.566969418439\\
61.375	0.27102	114.957975355974\\
61.375	0.27268	116.34898129351\\
61.375	0.27434	117.739987231046\\
61.375	0.276	119.130993168581\\
61.5833333333333	0.193	49.6262515241297\\
61.5833333333333	0.19466	51.0298711077253\\
61.5833333333333	0.19632	52.4334906913209\\
61.5833333333333	0.19798	53.8371102749165\\
61.5833333333333	0.19964	55.2407298585121\\
61.5833333333333	0.2013	56.6443494421076\\
61.5833333333333	0.20296	58.0479690257032\\
61.5833333333333	0.20462	59.4515886092988\\
61.5833333333333	0.20628	60.8552081928944\\
61.5833333333333	0.20794	62.25882777649\\
61.5833333333333	0.2096	63.6624473600855\\
61.5833333333333	0.21126	65.0660669436811\\
61.5833333333333	0.21292	66.4696865272767\\
61.5833333333333	0.21458	67.8733061108722\\
61.5833333333333	0.21624	69.2769256944679\\
61.5833333333333	0.2179	70.6805452780634\\
61.5833333333333	0.21956	72.084164861659\\
61.5833333333333	0.22122	73.4877844452546\\
61.5833333333333	0.22288	74.8914040288502\\
61.5833333333333	0.22454	76.2950236124458\\
61.5833333333333	0.2262	77.6986431960414\\
61.5833333333333	0.22786	79.1022627796369\\
61.5833333333333	0.22952	80.5058823632325\\
61.5833333333333	0.23118	81.9095019468281\\
61.5833333333333	0.23284	83.3131215304236\\
61.5833333333333	0.2345	84.7167411140193\\
61.5833333333333	0.23616	86.1203606976148\\
61.5833333333333	0.23782	87.5239802812104\\
61.5833333333333	0.23948	88.927599864806\\
61.5833333333333	0.24114	90.3312194484016\\
61.5833333333333	0.2428	91.7348390319971\\
61.5833333333333	0.24446	93.1384586155928\\
61.5833333333333	0.24612	94.5420781991883\\
61.5833333333333	0.24778	95.9456977827839\\
61.5833333333333	0.24944	97.3493173663795\\
61.5833333333333	0.2511	98.752936949975\\
61.5833333333333	0.25276	100.156556533571\\
61.5833333333333	0.25442	101.560176117166\\
61.5833333333333	0.25608	102.963795700762\\
61.5833333333333	0.25774	104.367415284357\\
61.5833333333333	0.2594	105.771034867953\\
61.5833333333333	0.26106	107.174654451549\\
61.5833333333333	0.26272	108.578274035144\\
61.5833333333333	0.26438	109.98189361874\\
61.5833333333333	0.26604	111.385513202335\\
61.5833333333333	0.2677	112.789132785931\\
61.5833333333333	0.26936	114.192752369526\\
61.5833333333333	0.27102	115.596371953122\\
61.5833333333333	0.27268	116.999991536718\\
61.5833333333333	0.27434	118.403611120313\\
61.5833333333333	0.276	119.807230703909\\
61.7916666666667	0.193	49.671806756462\\
61.7916666666667	0.19466	51.0880399861175\\
61.7916666666667	0.19632	52.5042732157729\\
61.7916666666667	0.19798	53.9205064454285\\
61.7916666666667	0.19964	55.3367396750839\\
61.7916666666667	0.2013	56.7529729047395\\
61.7916666666667	0.20296	58.1692061343949\\
61.7916666666667	0.20462	59.5854393640504\\
61.7916666666667	0.20628	61.0016725937059\\
61.7916666666667	0.20794	62.4179058233614\\
61.7916666666667	0.2096	63.8341390530168\\
61.7916666666667	0.21126	65.2503722826723\\
61.7916666666667	0.21292	66.6666055123279\\
61.7916666666667	0.21458	68.0828387419833\\
61.7916666666667	0.21624	69.4990719716388\\
61.7916666666667	0.2179	70.9153052012943\\
61.7916666666667	0.21956	72.3315384309498\\
61.7916666666667	0.22122	73.7477716606053\\
61.7916666666667	0.22288	75.1640048902607\\
61.7916666666667	0.22454	76.5802381199163\\
61.7916666666667	0.2262	77.9964713495717\\
61.7916666666667	0.22786	79.4127045792272\\
61.7916666666667	0.22952	80.8289378088828\\
61.7916666666667	0.23118	82.2451710385382\\
61.7916666666667	0.23284	83.6614042681937\\
61.7916666666667	0.2345	85.0776374978492\\
61.7916666666667	0.23616	86.4938707275046\\
61.7916666666667	0.23782	87.9101039571602\\
61.7916666666667	0.23948	89.3263371868156\\
61.7916666666667	0.24114	90.7425704164712\\
61.7916666666667	0.2428	92.1588036461266\\
61.7916666666667	0.24446	93.5750368757821\\
61.7916666666667	0.24612	94.9912701054375\\
61.7916666666667	0.24778	96.4075033350931\\
61.7916666666667	0.24944	97.8237365647486\\
61.7916666666667	0.2511	99.239969794404\\
61.7916666666667	0.25276	100.656203024059\\
61.7916666666667	0.25442	102.072436253715\\
61.7916666666667	0.25608	103.488669483371\\
61.7916666666667	0.25774	104.904902713026\\
61.7916666666667	0.2594	106.321135942682\\
61.7916666666667	0.26106	107.737369172337\\
61.7916666666667	0.26272	109.153602401992\\
61.7916666666667	0.26438	110.569835631648\\
61.7916666666667	0.26604	111.986068861303\\
61.7916666666667	0.2677	113.402302090959\\
61.7916666666667	0.26936	114.818535320614\\
61.7916666666667	0.27102	116.23476855027\\
61.7916666666667	0.27268	117.651001779925\\
61.7916666666667	0.27434	119.067235009581\\
61.7916666666667	0.276	120.483468239236\\
62	0.193	49.7173619887943\\
62	0.19466	51.1462088645097\\
62	0.19632	52.5750557402251\\
62	0.19798	54.0039026159405\\
62	0.19964	55.4327494916558\\
62	0.2013	56.8615963673713\\
62	0.20296	58.2904432430867\\
62	0.20462	59.7192901188021\\
62	0.20628	61.1481369945175\\
62	0.20794	62.5769838702328\\
62	0.2096	64.0058307459482\\
62	0.21126	65.4346776216636\\
62	0.21292	66.863524497379\\
62	0.21458	68.2923713730944\\
62	0.21624	69.7212182488098\\
62	0.2179	71.1500651245252\\
62	0.21956	72.5789120002406\\
62	0.22122	74.007758875956\\
62	0.22288	75.4366057516714\\
62	0.22454	76.8654526273868\\
62	0.2262	78.2942995031022\\
62	0.22786	79.7231463788176\\
62	0.22952	81.151993254533\\
62	0.23118	82.5808401302484\\
62	0.23284	84.0096870059638\\
62	0.2345	85.4385338816792\\
62	0.23616	86.8673807573946\\
62	0.23782	88.29622763311\\
62	0.23948	89.7250745088254\\
62	0.24114	91.1539213845407\\
62	0.2428	92.5827682602562\\
62	0.24446	94.0116151359716\\
62	0.24612	95.4404620116869\\
62	0.24778	96.8693088874024\\
62	0.24944	98.2981557631177\\
62	0.2511	99.727002638833\\
62	0.25276	101.155849514548\\
62	0.25442	102.584696390264\\
62	0.25608	104.013543265979\\
62	0.25774	105.442390141695\\
62	0.2594	106.87123701741\\
62	0.26106	108.300083893126\\
62	0.26272	109.728930768841\\
62	0.26438	111.157777644556\\
62	0.26604	112.586624520272\\
62	0.2677	114.015471395987\\
62	0.26936	115.444318271702\\
62	0.27102	116.873165147418\\
62	0.27268	118.302012023133\\
62	0.27434	119.730858898849\\
62	0.276	121.159705774564\\
62.2083333333333	0.193	49.7629172211265\\
62.2083333333333	0.19466	51.2043777429017\\
62.2083333333333	0.19632	52.645838264677\\
62.2083333333333	0.19798	54.0872987864524\\
62.2083333333333	0.19964	55.5287593082277\\
62.2083333333333	0.2013	56.970219830003\\
62.2083333333333	0.20296	58.4116803517783\\
62.2083333333333	0.20462	59.8531408735536\\
62.2083333333333	0.20628	61.2946013953289\\
62.2083333333333	0.20794	62.7360619171042\\
62.2083333333333	0.2096	64.1775224388795\\
62.2083333333333	0.21126	65.6189829606548\\
62.2083333333333	0.21292	67.0604434824301\\
62.2083333333333	0.21458	68.5019040042054\\
62.2083333333333	0.21624	69.9433645259807\\
62.2083333333333	0.2179	71.384825047756\\
62.2083333333333	0.21956	72.8262855695313\\
62.2083333333333	0.22122	74.2677460913065\\
62.2083333333333	0.22288	75.7092066130818\\
62.2083333333333	0.22454	77.1506671348572\\
62.2083333333333	0.2262	78.5921276566324\\
62.2083333333333	0.22786	80.0335881784077\\
62.2083333333333	0.22952	81.4750487001831\\
62.2083333333333	0.23118	82.9165092219583\\
62.2083333333333	0.23284	84.3579697437337\\
62.2083333333333	0.2345	85.799430265509\\
62.2083333333333	0.23616	87.2408907872843\\
62.2083333333333	0.23782	88.6823513090596\\
62.2083333333333	0.23948	90.1238118308349\\
62.2083333333333	0.24114	91.5652723526102\\
62.2083333333333	0.2428	93.0067328743855\\
62.2083333333333	0.24446	94.4481933961608\\
62.2083333333333	0.24612	95.889653917936\\
62.2083333333333	0.24778	97.3311144397114\\
62.2083333333333	0.24944	98.7725749614867\\
62.2083333333333	0.2511	100.214035483262\\
62.2083333333333	0.25276	101.655496005037\\
62.2083333333333	0.25442	103.096956526813\\
62.2083333333333	0.25608	104.538417048588\\
62.2083333333333	0.25774	105.979877570363\\
62.2083333333333	0.2594	107.421338092138\\
62.2083333333333	0.26106	108.862798613914\\
62.2083333333333	0.26272	110.304259135689\\
62.2083333333333	0.26438	111.745719657464\\
62.2083333333333	0.26604	113.18718017924\\
62.2083333333333	0.2677	114.628640701015\\
62.2083333333333	0.26936	116.07010122279\\
62.2083333333333	0.27102	117.511561744566\\
62.2083333333333	0.27268	118.953022266341\\
62.2083333333333	0.27434	120.394482788116\\
62.2083333333333	0.276	121.835943309892\\
62.4166666666667	0.193	49.8084724534588\\
62.4166666666667	0.19466	51.262546621294\\
62.4166666666667	0.19632	52.7166207891291\\
62.4166666666667	0.19798	54.1706949569644\\
62.4166666666667	0.19964	55.6247691247996\\
62.4166666666667	0.2013	57.0788432926348\\
62.4166666666667	0.20296	58.53291746047\\
62.4166666666667	0.20462	59.9869916283051\\
62.4166666666667	0.20628	61.4410657961404\\
62.4166666666667	0.20794	62.8951399639756\\
62.4166666666667	0.2096	64.3492141318108\\
62.4166666666667	0.21126	65.803288299646\\
62.4166666666667	0.21292	67.2573624674812\\
62.4166666666667	0.21458	68.7114366353164\\
62.4166666666667	0.21624	70.1655108031516\\
62.4166666666667	0.2179	71.6195849709868\\
62.4166666666667	0.21956	73.0736591388221\\
62.4166666666667	0.22122	74.5277333066572\\
62.4166666666667	0.22288	75.9818074744924\\
62.4166666666667	0.22454	77.4358816423277\\
62.4166666666667	0.2262	78.8899558101629\\
62.4166666666667	0.22786	80.3440299779981\\
62.4166666666667	0.22952	81.7981041458333\\
62.4166666666667	0.23118	83.2521783136685\\
62.4166666666667	0.23284	84.7062524815037\\
62.4166666666667	0.2345	86.1603266493389\\
62.4166666666667	0.23616	87.6144008171741\\
62.4166666666667	0.23782	89.0684749850094\\
62.4166666666667	0.23948	90.5225491528446\\
62.4166666666667	0.24114	91.9766233206798\\
62.4166666666667	0.2428	93.430697488515\\
62.4166666666667	0.24446	94.8847716563501\\
62.4166666666667	0.24612	96.3388458241853\\
62.4166666666667	0.24778	97.7929199920206\\
62.4166666666667	0.24944	99.2469941598558\\
62.4166666666667	0.2511	100.701068327691\\
62.4166666666667	0.25276	102.155142495526\\
62.4166666666667	0.25442	103.609216663361\\
62.4166666666667	0.25608	105.063290831197\\
62.4166666666667	0.25774	106.517364999032\\
62.4166666666667	0.2594	107.971439166867\\
62.4166666666667	0.26106	109.425513334702\\
62.4166666666667	0.26272	110.879587502537\\
62.4166666666667	0.26438	112.333661670373\\
62.4166666666667	0.26604	113.787735838208\\
62.4166666666667	0.2677	115.241810006043\\
62.4166666666667	0.26936	116.695884173878\\
62.4166666666667	0.27102	118.149958341713\\
62.4166666666667	0.27268	119.604032509549\\
62.4166666666667	0.27434	121.058106677384\\
62.4166666666667	0.276	122.512180845219\\
62.625	0.193	49.854027685791\\
62.625	0.19466	51.3207154996861\\
62.625	0.19632	52.7874033135813\\
62.625	0.19798	54.2540911274764\\
62.625	0.19964	55.7207789413715\\
62.625	0.2013	57.1874667552667\\
62.625	0.20296	58.6541545691617\\
62.625	0.20462	60.1208423830568\\
62.625	0.20628	61.5875301969519\\
62.625	0.20794	63.0542180108471\\
62.625	0.2096	64.5209058247422\\
62.625	0.21126	65.9875936386373\\
62.625	0.21292	67.4542814525324\\
62.625	0.21458	68.9209692664275\\
62.625	0.21624	70.3876570803226\\
62.625	0.2179	71.8543448942177\\
62.625	0.21956	73.3210327081129\\
62.625	0.22122	74.787720522008\\
62.625	0.22288	76.2544083359031\\
62.625	0.22454	77.7210961497983\\
62.625	0.2262	79.1877839636933\\
62.625	0.22786	80.6544717775884\\
62.625	0.22952	82.1211595914835\\
62.625	0.23118	83.5878474053787\\
62.625	0.23284	85.0545352192738\\
62.625	0.2345	86.5212230331689\\
62.625	0.23616	87.987910847064\\
62.625	0.23782	89.4545986609592\\
62.625	0.23948	90.9212864748542\\
62.625	0.24114	92.3879742887493\\
62.625	0.2428	93.8546621026445\\
62.625	0.24446	95.3213499165395\\
62.625	0.24612	96.7880377304347\\
62.625	0.24778	98.2547255443298\\
62.625	0.24944	99.7214133582249\\
62.625	0.2511	101.18810117212\\
62.625	0.25276	102.654788986015\\
62.625	0.25442	104.12147679991\\
62.625	0.25608	105.588164613805\\
62.625	0.25774	107.054852427701\\
62.625	0.2594	108.521540241596\\
62.625	0.26106	109.988228055491\\
62.625	0.26272	111.454915869386\\
62.625	0.26438	112.921603683281\\
62.625	0.26604	114.388291497176\\
62.625	0.2677	115.854979311071\\
62.625	0.26936	117.321667124966\\
62.625	0.27102	118.788354938861\\
62.625	0.27268	120.255042752757\\
62.625	0.27434	121.721730566652\\
62.625	0.276	123.188418380547\\
62.8333333333333	0.193	49.8995829181233\\
62.8333333333333	0.19466	51.3788843780783\\
62.8333333333333	0.19632	52.8581858380333\\
62.8333333333333	0.19798	54.3374872979883\\
62.8333333333333	0.19964	55.8167887579434\\
62.8333333333333	0.2013	57.2960902178984\\
62.8333333333333	0.20296	58.7753916778534\\
62.8333333333333	0.20462	60.2546931378084\\
62.8333333333333	0.20628	61.7339945977635\\
62.8333333333333	0.20794	63.2132960577185\\
62.8333333333333	0.2096	64.6925975176734\\
62.8333333333333	0.21126	66.1718989776285\\
62.8333333333333	0.21292	67.6512004375836\\
62.8333333333333	0.21458	69.1305018975385\\
62.8333333333333	0.21624	70.6098033574936\\
62.8333333333333	0.2179	72.0891048174486\\
62.8333333333333	0.21956	73.5684062774037\\
62.8333333333333	0.22122	75.0477077373586\\
62.8333333333333	0.22288	76.5270091973136\\
62.8333333333333	0.22454	78.0063106572687\\
62.8333333333333	0.2262	79.4856121172237\\
62.8333333333333	0.22786	80.9649135771787\\
62.8333333333333	0.22952	82.4442150371338\\
62.8333333333333	0.23118	83.9235164970887\\
62.8333333333333	0.23284	85.4028179570439\\
62.8333333333333	0.2345	86.8821194169988\\
62.8333333333333	0.23616	88.3614208769538\\
62.8333333333333	0.23782	89.8407223369089\\
62.8333333333333	0.23948	91.3200237968638\\
62.8333333333333	0.24114	92.7993252568189\\
62.8333333333333	0.2428	94.2786267167739\\
62.8333333333333	0.24446	95.7579281767289\\
62.8333333333333	0.24612	97.237229636684\\
62.8333333333333	0.24778	98.716531096639\\
62.8333333333333	0.24944	100.195832556594\\
62.8333333333333	0.2511	101.675134016549\\
62.8333333333333	0.25276	103.154435476504\\
62.8333333333333	0.25442	104.633736936459\\
62.8333333333333	0.25608	106.113038396414\\
62.8333333333333	0.25774	107.592339856369\\
62.8333333333333	0.2594	109.071641316324\\
62.8333333333333	0.26106	110.550942776279\\
62.8333333333333	0.26272	112.030244236234\\
62.8333333333333	0.26438	113.509545696189\\
62.8333333333333	0.26604	114.988847156144\\
62.8333333333333	0.2677	116.468148616099\\
62.8333333333333	0.26936	117.947450076054\\
62.8333333333333	0.27102	119.426751536009\\
62.8333333333333	0.27268	120.906052995964\\
62.8333333333333	0.27434	122.385354455919\\
62.8333333333333	0.276	123.864655915874\\
63.0416666666667	0.193	49.9451381504556\\
63.0416666666667	0.19466	51.4370532564704\\
63.0416666666667	0.19632	52.9289683624854\\
63.0416666666667	0.19798	54.4208834685003\\
63.0416666666667	0.19964	55.9127985745153\\
63.0416666666667	0.2013	57.4047136805302\\
63.0416666666667	0.20296	58.8966287865451\\
63.0416666666667	0.20462	60.3885438925601\\
63.0416666666667	0.20628	61.880458998575\\
63.0416666666667	0.20794	63.3723741045899\\
63.0416666666667	0.2096	64.8642892106048\\
63.0416666666667	0.21126	66.3562043166198\\
63.0416666666667	0.21292	67.8481194226347\\
63.0416666666667	0.21458	69.3400345286496\\
63.0416666666667	0.21624	70.8319496346646\\
63.0416666666667	0.2179	72.3238647406795\\
63.0416666666667	0.21956	73.8157798466944\\
63.0416666666667	0.22122	75.3076949527093\\
63.0416666666667	0.22288	76.7996100587243\\
63.0416666666667	0.22454	78.2915251647392\\
63.0416666666667	0.2262	79.7834402707541\\
63.0416666666667	0.22786	81.275355376769\\
63.0416666666667	0.22952	82.7672704827839\\
63.0416666666667	0.23118	84.2591855887989\\
63.0416666666667	0.23284	85.7511006948139\\
63.0416666666667	0.2345	87.2430158008287\\
63.0416666666667	0.23616	88.7349309068437\\
63.0416666666667	0.23782	90.2268460128587\\
63.0416666666667	0.23948	91.7187611188735\\
63.0416666666667	0.24114	93.2106762248884\\
63.0416666666667	0.2428	94.7025913309034\\
63.0416666666667	0.24446	96.1945064369183\\
63.0416666666667	0.24612	97.6864215429333\\
63.0416666666667	0.24778	99.1783366489482\\
63.0416666666667	0.24944	100.670251754963\\
63.0416666666667	0.2511	102.162166860978\\
63.0416666666667	0.25276	103.654081966993\\
63.0416666666667	0.25442	105.145997073008\\
63.0416666666667	0.25608	106.637912179023\\
63.0416666666667	0.25774	108.129827285038\\
63.0416666666667	0.2594	109.621742391053\\
63.0416666666667	0.26106	111.113657497068\\
63.0416666666667	0.26272	112.605572603083\\
63.0416666666667	0.26438	114.097487709097\\
63.0416666666667	0.26604	115.589402815112\\
63.0416666666667	0.2677	117.081317921127\\
63.0416666666667	0.26936	118.573233027142\\
63.0416666666667	0.27102	120.065148133157\\
63.0416666666667	0.27268	121.557063239172\\
63.0416666666667	0.27434	123.048978345187\\
63.0416666666667	0.276	124.540893451202\\
63.25	0.193	49.9906933827878\\
63.25	0.19466	51.4952221348626\\
63.25	0.19632	52.9997508869374\\
63.25	0.19798	54.5042796390123\\
63.25	0.19964	56.0088083910871\\
63.25	0.2013	57.5133371431619\\
63.25	0.20296	59.0178658952368\\
63.25	0.20462	60.5223946473116\\
63.25	0.20628	62.0269233993865\\
63.25	0.20794	63.5314521514613\\
63.25	0.2096	65.0359809035361\\
63.25	0.21126	66.5405096556109\\
63.25	0.21292	68.0450384076858\\
63.25	0.21458	69.5495671597606\\
63.25	0.21624	71.0540959118355\\
63.25	0.2179	72.5586246639103\\
63.25	0.21956	74.0631534159851\\
63.25	0.22122	75.5676821680599\\
63.25	0.22288	77.0722109201348\\
63.25	0.22454	78.5767396722096\\
63.25	0.2262	80.0812684242845\\
63.25	0.22786	81.5857971763592\\
63.25	0.22952	83.0903259284341\\
63.25	0.23118	84.5948546805089\\
63.25	0.23284	86.0993834325839\\
63.25	0.2345	87.6039121846586\\
63.25	0.23616	89.1084409367335\\
63.25	0.23782	90.6129696888083\\
63.25	0.23948	92.1174984408831\\
63.25	0.24114	93.6220271929579\\
63.25	0.2428	95.1265559450329\\
63.25	0.24446	96.6310846971076\\
63.25	0.24612	98.1356134491825\\
63.25	0.24778	99.6401422012573\\
63.25	0.24944	101.144670953332\\
63.25	0.2511	102.649199705407\\
63.25	0.25276	104.153728457482\\
63.25	0.25442	105.658257209557\\
63.25	0.25608	107.162785961631\\
63.25	0.25774	108.667314713706\\
63.25	0.2594	110.171843465781\\
63.25	0.26106	111.676372217856\\
63.25	0.26272	113.180900969931\\
63.25	0.26438	114.685429722006\\
63.25	0.26604	116.189958474081\\
63.25	0.2677	117.694487226155\\
63.25	0.26936	119.19901597823\\
63.25	0.27102	120.703544730305\\
63.25	0.27268	122.20807348238\\
63.25	0.27434	123.712602234455\\
63.25	0.276	125.217130986529\\
63.4583333333333	0.193	50.03624861512\\
63.4583333333333	0.19466	51.5533910132547\\
63.4583333333333	0.19632	53.0705334113895\\
63.4583333333333	0.19798	54.5876758095242\\
63.4583333333333	0.19964	56.104818207659\\
63.4583333333333	0.2013	57.6219606057937\\
63.4583333333333	0.20296	59.1391030039285\\
63.4583333333333	0.20462	60.6562454020632\\
63.4583333333333	0.20628	62.173387800198\\
63.4583333333333	0.20794	63.6905301983327\\
63.4583333333333	0.2096	65.2076725964674\\
63.4583333333333	0.21126	66.7248149946022\\
63.4583333333333	0.21292	68.2419573927369\\
63.4583333333333	0.21458	69.7590997908716\\
63.4583333333333	0.21624	71.2762421890064\\
63.4583333333333	0.2179	72.7933845871411\\
63.4583333333333	0.21956	74.3105269852759\\
63.4583333333333	0.22122	75.8276693834106\\
63.4583333333333	0.22288	77.3448117815454\\
63.4583333333333	0.22454	78.8619541796801\\
63.4583333333333	0.2262	80.3790965778148\\
63.4583333333333	0.22786	81.8962389759496\\
63.4583333333333	0.22952	83.4133813740843\\
63.4583333333333	0.23118	84.9305237722191\\
63.4583333333333	0.23284	86.4476661703538\\
63.4583333333333	0.2345	87.9648085684885\\
63.4583333333333	0.23616	89.4819509666233\\
63.4583333333333	0.23782	90.999093364758\\
63.4583333333333	0.23948	92.5162357628928\\
63.4583333333333	0.24114	94.0333781610275\\
63.4583333333333	0.2428	95.5505205591622\\
63.4583333333333	0.24446	97.067662957297\\
63.4583333333333	0.24612	98.5848053554317\\
63.4583333333333	0.24778	100.101947753566\\
63.4583333333333	0.24944	101.619090151701\\
63.4583333333333	0.2511	103.136232549836\\
63.4583333333333	0.25276	104.653374947971\\
63.4583333333333	0.25442	106.170517346105\\
63.4583333333333	0.25608	107.68765974424\\
63.4583333333333	0.25774	109.204802142375\\
63.4583333333333	0.2594	110.72194454051\\
63.4583333333333	0.26106	112.239086938644\\
63.4583333333333	0.26272	113.756229336779\\
63.4583333333333	0.26438	115.273371734914\\
63.4583333333333	0.26604	116.790514133049\\
63.4583333333333	0.2677	118.307656531183\\
63.4583333333333	0.26936	119.824798929318\\
63.4583333333333	0.27102	121.341941327453\\
63.4583333333333	0.27268	122.859083725588\\
63.4583333333333	0.27434	124.376226123722\\
63.4583333333333	0.276	125.893368521857\\
63.6666666666667	0.193	50.0818038474523\\
63.6666666666667	0.19466	51.6115598916469\\
63.6666666666667	0.19632	53.1413159358415\\
63.6666666666667	0.19798	54.6710719800362\\
63.6666666666667	0.19964	56.2008280242309\\
63.6666666666667	0.2013	57.7305840684256\\
63.6666666666667	0.20296	59.2603401126202\\
63.6666666666667	0.20462	60.7900961568148\\
63.6666666666667	0.20628	62.3198522010095\\
63.6666666666667	0.20794	63.8496082452041\\
63.6666666666667	0.2096	65.3793642893987\\
63.6666666666667	0.21126	66.9091203335934\\
63.6666666666667	0.21292	68.4388763777881\\
63.6666666666667	0.21458	69.9686324219826\\
63.6666666666667	0.21624	71.4983884661773\\
63.6666666666667	0.2179	73.028144510372\\
63.6666666666667	0.21956	74.5579005545667\\
63.6666666666667	0.22122	76.0876565987613\\
63.6666666666667	0.22288	77.617412642956\\
63.6666666666667	0.22454	79.1471686871506\\
63.6666666666667	0.2262	80.6769247313452\\
63.6666666666667	0.22786	82.2066807755398\\
63.6666666666667	0.22952	83.7364368197345\\
63.6666666666667	0.23118	85.2661928639291\\
63.6666666666667	0.23284	86.7959489081238\\
63.6666666666667	0.2345	88.3257049523185\\
63.6666666666667	0.23616	89.8554609965131\\
63.6666666666667	0.23782	91.3852170407077\\
63.6666666666667	0.23948	92.9149730849024\\
63.6666666666667	0.24114	94.444729129097\\
63.6666666666667	0.2428	95.9744851732917\\
63.6666666666667	0.24446	97.5042412174864\\
63.6666666666667	0.24612	99.033997261681\\
63.6666666666667	0.24778	100.563753305876\\
63.6666666666667	0.24944	102.09350935007\\
63.6666666666667	0.2511	103.623265394265\\
63.6666666666667	0.25276	105.15302143846\\
63.6666666666667	0.25442	106.682777482654\\
63.6666666666667	0.25608	108.212533526849\\
63.6666666666667	0.25774	109.742289571044\\
63.6666666666667	0.2594	111.272045615238\\
63.6666666666667	0.26106	112.801801659433\\
63.6666666666667	0.26272	114.331557703627\\
63.6666666666667	0.26438	115.861313747822\\
63.6666666666667	0.26604	117.391069792017\\
63.6666666666667	0.2677	118.920825836211\\
63.6666666666667	0.26936	120.450581880406\\
63.6666666666667	0.27102	121.980337924601\\
63.6666666666667	0.27268	123.510093968795\\
63.6666666666667	0.27434	125.03985001299\\
63.6666666666667	0.276	126.569606057185\\
63.875	0.193	50.1273590797846\\
63.875	0.19466	51.6697287700391\\
63.875	0.19632	53.2120984602936\\
63.875	0.19798	54.7544681505483\\
63.875	0.19964	56.2968378408028\\
63.875	0.2013	57.8392075310574\\
63.875	0.20296	59.3815772213119\\
63.875	0.20462	60.9239469115665\\
63.875	0.20628	62.4663166018211\\
63.875	0.20794	64.0086862920756\\
63.875	0.2096	65.5510559823301\\
63.875	0.21126	67.0934256725847\\
63.875	0.21292	68.6357953628392\\
63.875	0.21458	70.1781650530938\\
63.875	0.21624	71.7205347433484\\
63.875	0.2179	73.2629044336029\\
63.875	0.21956	74.8052741238575\\
63.875	0.22122	76.347643814112\\
63.875	0.22288	77.8900135043666\\
63.875	0.22454	79.4323831946212\\
63.875	0.2262	80.9747528848757\\
63.875	0.22786	82.5171225751303\\
63.875	0.22952	84.0594922653848\\
63.875	0.23118	85.6018619556393\\
63.875	0.23284	87.1442316458939\\
63.875	0.2345	88.6866013361484\\
63.875	0.23616	90.228971026403\\
63.875	0.23782	91.7713407166576\\
63.875	0.23948	93.3137104069122\\
63.875	0.24114	94.8560800971667\\
63.875	0.2428	96.3984497874213\\
63.875	0.24446	97.9408194776758\\
63.875	0.24612	99.4831891679303\\
63.875	0.24778	101.025558858185\\
63.875	0.24944	102.567928548439\\
63.875	0.2511	104.110298238694\\
63.875	0.25276	105.652667928949\\
63.875	0.25442	107.195037619203\\
63.875	0.25608	108.737407309458\\
63.875	0.25774	110.279776999712\\
63.875	0.2594	111.822146689967\\
63.875	0.26106	113.364516380221\\
63.875	0.26272	114.906886070476\\
63.875	0.26438	116.44925576073\\
63.875	0.26604	117.991625450985\\
63.875	0.2677	119.53399514124\\
63.875	0.26936	121.076364831494\\
63.875	0.27102	122.618734521749\\
63.875	0.27268	124.161104212003\\
63.875	0.27434	125.703473902258\\
63.875	0.276	127.245843592512\\
64.0833333333333	0.193	50.1729143121168\\
64.0833333333333	0.19466	51.7278976484313\\
64.0833333333333	0.19632	53.2828809847457\\
64.0833333333333	0.19798	54.8378643210602\\
64.0833333333333	0.19964	56.3928476573747\\
64.0833333333333	0.2013	57.9478309936891\\
64.0833333333333	0.20296	59.5028143300036\\
64.0833333333333	0.20462	61.057797666318\\
64.0833333333333	0.20628	62.6127810026325\\
64.0833333333333	0.20794	64.1677643389469\\
64.0833333333333	0.2096	65.7227476752614\\
64.0833333333333	0.21126	67.2777310115759\\
64.0833333333333	0.21292	68.8327143478903\\
64.0833333333333	0.21458	70.3876976842047\\
64.0833333333333	0.21624	71.9426810205193\\
64.0833333333333	0.2179	73.4976643568337\\
64.0833333333333	0.21956	75.0526476931482\\
64.0833333333333	0.22122	76.6076310294627\\
64.0833333333333	0.22288	78.1626143657771\\
64.0833333333333	0.22454	79.7175977020916\\
64.0833333333333	0.2262	81.272581038406\\
64.0833333333333	0.22786	82.8275643747205\\
64.0833333333333	0.22952	84.3825477110349\\
64.0833333333333	0.23118	85.9375310473494\\
64.0833333333333	0.23284	87.4925143836639\\
64.0833333333333	0.2345	89.0474977199783\\
64.0833333333333	0.23616	90.6024810562927\\
64.0833333333333	0.23782	92.1574643926073\\
64.0833333333333	0.23948	93.7124477289217\\
64.0833333333333	0.24114	95.2674310652361\\
64.0833333333333	0.2428	96.8224144015507\\
64.0833333333333	0.24446	98.3773977378651\\
64.0833333333333	0.24612	99.9323810741795\\
64.0833333333333	0.24778	101.487364410494\\
64.0833333333333	0.24944	103.042347746809\\
64.0833333333333	0.2511	104.597331083123\\
64.0833333333333	0.25276	106.152314419437\\
64.0833333333333	0.25442	107.707297755752\\
64.0833333333333	0.25608	109.262281092066\\
64.0833333333333	0.25774	110.817264428381\\
64.0833333333333	0.2594	112.372247764695\\
64.0833333333333	0.26106	113.92723110101\\
64.0833333333333	0.26272	115.482214437324\\
64.0833333333333	0.26438	117.037197773639\\
64.0833333333333	0.26604	118.592181109953\\
64.0833333333333	0.2677	120.147164446268\\
64.0833333333333	0.26936	121.702147782582\\
64.0833333333333	0.27102	123.257131118896\\
64.0833333333333	0.27268	124.812114455211\\
64.0833333333333	0.27434	126.367097791525\\
64.0833333333333	0.276	127.92208112784\\
64.2916666666667	0.193	50.2184695444491\\
64.2916666666667	0.19466	51.7860665268234\\
64.2916666666667	0.19632	53.3536635091978\\
64.2916666666667	0.19798	54.9212604915722\\
64.2916666666667	0.19964	56.4888574739465\\
64.2916666666667	0.2013	58.0564544563209\\
64.2916666666667	0.20296	59.6240514386952\\
64.2916666666667	0.20462	61.1916484210696\\
64.2916666666667	0.20628	62.759245403444\\
64.2916666666667	0.20794	64.3268423858183\\
64.2916666666667	0.2096	65.8944393681927\\
64.2916666666667	0.21126	67.4620363505671\\
64.2916666666667	0.21292	69.0296333329414\\
64.2916666666667	0.21458	70.5972303153158\\
64.2916666666667	0.21624	72.1648272976902\\
64.2916666666667	0.2179	73.7324242800645\\
64.2916666666667	0.21956	75.3000212624389\\
64.2916666666667	0.22122	76.8676182448133\\
64.2916666666667	0.22288	78.4352152271877\\
64.2916666666667	0.22454	80.0028122095621\\
64.2916666666667	0.2262	81.5704091919364\\
64.2916666666667	0.22786	83.1380061743107\\
64.2916666666667	0.22952	84.7056031566852\\
64.2916666666667	0.23118	86.2732001390596\\
64.2916666666667	0.23284	87.8407971214339\\
64.2916666666667	0.2345	89.4083941038082\\
64.2916666666667	0.23616	90.9759910861827\\
64.2916666666667	0.23782	92.543588068557\\
64.2916666666667	0.23948	94.1111850509314\\
64.2916666666667	0.24114	95.6787820333057\\
64.2916666666667	0.2428	97.2463790156801\\
64.2916666666667	0.24446	98.8139759980545\\
64.2916666666667	0.24612	100.381572980429\\
64.2916666666667	0.24778	101.949169962803\\
64.2916666666667	0.24944	103.516766945178\\
64.2916666666667	0.2511	105.084363927552\\
64.2916666666667	0.25276	106.651960909926\\
64.2916666666667	0.25442	108.219557892301\\
64.2916666666667	0.25608	109.787154874675\\
64.2916666666667	0.25774	111.354751857049\\
64.2916666666667	0.2594	112.922348839424\\
64.2916666666667	0.26106	114.489945821798\\
64.2916666666667	0.26272	116.057542804172\\
64.2916666666667	0.26438	117.625139786547\\
64.2916666666667	0.26604	119.192736768921\\
64.2916666666667	0.2677	120.760333751296\\
64.2916666666667	0.26936	122.32793073367\\
64.2916666666667	0.27102	123.895527716044\\
64.2916666666667	0.27268	125.463124698419\\
64.2916666666667	0.27434	127.030721680793\\
64.2916666666667	0.276	128.598318663168\\
64.5	0.193	50.2640247767814\\
64.5	0.19466	51.8442354052156\\
64.5	0.19632	53.4244460336498\\
64.5	0.19798	55.0046566620841\\
64.5	0.19964	56.5848672905184\\
64.5	0.2013	58.1650779189527\\
64.5	0.20296	59.745288547387\\
64.5	0.20462	61.3254991758212\\
64.5	0.20628	62.9057098042555\\
64.5	0.20794	64.4859204326898\\
64.5	0.2096	66.0661310611241\\
64.5	0.21126	67.6463416895584\\
64.5	0.21292	69.2265523179926\\
64.5	0.21458	70.8067629464269\\
64.5	0.21624	72.3869735748612\\
64.5	0.2179	73.9671842032955\\
64.5	0.21956	75.5473948317297\\
64.5	0.22122	77.127605460164\\
64.5	0.22288	78.7078160885983\\
64.5	0.22454	80.2880267170326\\
64.5	0.2262	81.8682373454669\\
64.5	0.22786	83.4484479739011\\
64.5	0.22952	85.0286586023353\\
64.5	0.23118	86.6088692307696\\
64.5	0.23284	88.189079859204\\
64.5	0.2345	89.7692904876382\\
64.5	0.23616	91.3495011160725\\
64.5	0.23782	92.9297117445068\\
64.5	0.23948	94.509922372941\\
64.5	0.24114	96.0901330013753\\
64.5	0.2428	97.6703436298096\\
64.5	0.24446	99.2505542582439\\
64.5	0.24612	100.830764886678\\
64.5	0.24778	102.410975515112\\
64.5	0.24944	103.991186143547\\
64.5	0.2511	105.571396771981\\
64.5	0.25276	107.151607400415\\
64.5	0.25442	108.731818028849\\
64.5	0.25608	110.312028657284\\
64.5	0.25774	111.892239285718\\
64.5	0.2594	113.472449914152\\
64.5	0.26106	115.052660542587\\
64.5	0.26272	116.632871171021\\
64.5	0.26438	118.213081799455\\
64.5	0.26604	119.793292427889\\
64.5	0.2677	121.373503056324\\
64.5	0.26936	122.953713684758\\
64.5	0.27102	124.533924313192\\
64.5	0.27268	126.114134941627\\
64.5	0.27434	127.694345570061\\
64.5	0.276	129.274556198495\\
64.7083333333333	0.193	50.3095800091136\\
64.7083333333333	0.19466	51.9024042836077\\
64.7083333333333	0.19632	53.4952285581019\\
64.7083333333333	0.19798	55.0880528325961\\
64.7083333333333	0.19964	56.6808771070902\\
64.7083333333333	0.2013	58.2737013815845\\
64.7083333333333	0.20296	59.8665256560786\\
64.7083333333333	0.20462	61.4593499305728\\
64.7083333333333	0.20628	63.052174205067\\
64.7083333333333	0.20794	64.6449984795611\\
64.7083333333333	0.2096	66.2378227540553\\
64.7083333333333	0.21126	67.8306470285495\\
64.7083333333333	0.21292	69.4234713030437\\
64.7083333333333	0.21458	71.0162955775378\\
64.7083333333333	0.21624	72.6091198520321\\
64.7083333333333	0.2179	74.2019441265262\\
64.7083333333333	0.21956	75.7947684010205\\
64.7083333333333	0.22122	77.3875926755147\\
64.7083333333333	0.22288	78.9804169500088\\
64.7083333333333	0.22454	80.573241224503\\
64.7083333333333	0.2262	82.1660654989972\\
64.7083333333333	0.22786	83.7588897734913\\
64.7083333333333	0.22952	85.3517140479855\\
64.7083333333333	0.23118	86.9445383224797\\
64.7083333333333	0.23284	88.5373625969739\\
64.7083333333333	0.2345	90.130186871468\\
64.7083333333333	0.23616	91.7230111459622\\
64.7083333333333	0.23782	93.3158354204564\\
64.7083333333333	0.23948	94.9086596949506\\
64.7083333333333	0.24114	96.5014839694447\\
64.7083333333333	0.2428	98.094308243939\\
64.7083333333333	0.24446	99.6871325184331\\
64.7083333333333	0.24612	101.279956792927\\
64.7083333333333	0.24778	102.872781067421\\
64.7083333333333	0.24944	104.465605341916\\
64.7083333333333	0.2511	106.05842961641\\
64.7083333333333	0.25276	107.651253890904\\
64.7083333333333	0.25442	109.244078165398\\
64.7083333333333	0.25608	110.836902439892\\
64.7083333333333	0.25774	112.429726714387\\
64.7083333333333	0.2594	114.022550988881\\
64.7083333333333	0.26106	115.615375263375\\
64.7083333333333	0.26272	117.208199537869\\
64.7083333333333	0.26438	118.801023812363\\
64.7083333333333	0.26604	120.393848086857\\
64.7083333333333	0.2677	121.986672361352\\
64.7083333333333	0.26936	123.579496635846\\
64.7083333333333	0.27102	125.17232091034\\
64.7083333333333	0.27268	126.765145184834\\
64.7083333333333	0.27434	128.357969459328\\
64.7083333333333	0.276	129.950793733823\\
64.9166666666667	0.193	50.3551352414458\\
64.9166666666667	0.19466	51.9605731619999\\
64.9166666666667	0.19632	53.5660110825539\\
64.9166666666667	0.19798	55.1714490031081\\
64.9166666666667	0.19964	56.7768869236621\\
64.9166666666667	0.2013	58.3823248442163\\
64.9166666666667	0.20296	59.9877627647703\\
64.9166666666667	0.20462	61.5932006853244\\
64.9166666666667	0.20628	63.1986386058785\\
64.9166666666667	0.20794	64.8040765264326\\
64.9166666666667	0.2096	66.4095144469866\\
64.9166666666667	0.21126	68.0149523675408\\
64.9166666666667	0.21292	69.6203902880948\\
64.9166666666667	0.21458	71.2258282086489\\
64.9166666666667	0.21624	72.831266129203\\
64.9166666666667	0.2179	74.4367040497571\\
64.9166666666667	0.21956	76.0421419703113\\
64.9166666666667	0.22122	77.6475798908652\\
64.9166666666667	0.22288	79.2530178114194\\
64.9166666666667	0.22454	80.8584557319734\\
64.9166666666667	0.2262	82.4638936525275\\
64.9166666666667	0.22786	84.0693315730816\\
64.9166666666667	0.22952	85.6747694936358\\
64.9166666666667	0.23118	87.2802074141898\\
64.9166666666667	0.23284	88.885645334744\\
64.9166666666667	0.2345	90.491083255298\\
64.9166666666667	0.23616	92.096521175852\\
64.9166666666667	0.23782	93.7019590964061\\
64.9166666666667	0.23948	95.3073970169602\\
64.9166666666667	0.24114	96.9128349375143\\
64.9166666666667	0.2428	98.5182728580685\\
64.9166666666667	0.24446	100.123710778622\\
64.9166666666667	0.24612	101.729148699177\\
64.9166666666667	0.24778	103.334586619731\\
64.9166666666667	0.24944	104.940024540285\\
64.9166666666667	0.2511	106.545462460839\\
64.9166666666667	0.25276	108.150900381393\\
64.9166666666667	0.25442	109.756338301947\\
64.9166666666667	0.25608	111.361776222501\\
64.9166666666667	0.25774	112.967214143055\\
64.9166666666667	0.2594	114.572652063609\\
64.9166666666667	0.26106	116.178089984163\\
64.9166666666667	0.26272	117.783527904718\\
64.9166666666667	0.26438	119.388965825271\\
64.9166666666667	0.26604	120.994403745826\\
64.9166666666667	0.2677	122.59984166638\\
64.9166666666667	0.26936	124.205279586934\\
64.9166666666667	0.27102	125.810717507488\\
64.9166666666667	0.27268	127.416155428042\\
64.9166666666667	0.27434	129.021593348596\\
64.9166666666667	0.276	130.62703126915\\
65.125	0.193	50.4006904737781\\
65.125	0.19466	52.0187420403921\\
65.125	0.19632	53.636793607006\\
65.125	0.19798	55.25484517362\\
65.125	0.19964	56.872896740234\\
65.125	0.2013	58.4909483068481\\
65.125	0.20296	60.108999873462\\
65.125	0.20462	61.727051440076\\
65.125	0.20628	63.34510300669\\
65.125	0.20794	64.963154573304\\
65.125	0.2096	66.581206139918\\
65.125	0.21126	68.199257706532\\
65.125	0.21292	69.817309273146\\
65.125	0.21458	71.4353608397599\\
65.125	0.21624	73.053412406374\\
65.125	0.2179	74.6714639729879\\
65.125	0.21956	76.289515539602\\
65.125	0.22122	77.9075671062159\\
65.125	0.22288	79.52561867283\\
65.125	0.22454	81.143670239444\\
65.125	0.2262	82.7617218060579\\
65.125	0.22786	84.3797733726719\\
65.125	0.22952	85.9978249392859\\
65.125	0.23118	87.6158765059\\
65.125	0.23284	89.233928072514\\
65.125	0.2345	90.8519796391279\\
65.125	0.23616	92.4700312057419\\
65.125	0.23782	94.088082772356\\
65.125	0.23948	95.7061343389699\\
65.125	0.24114	97.3241859055839\\
65.125	0.2428	98.9422374721979\\
65.125	0.24446	100.560289038812\\
65.125	0.24612	102.178340605426\\
65.125	0.24778	103.79639217204\\
65.125	0.24944	105.414443738654\\
65.125	0.2511	107.032495305268\\
65.125	0.25276	108.650546871882\\
65.125	0.25442	110.268598438496\\
65.125	0.25608	111.88665000511\\
65.125	0.25774	113.504701571724\\
65.125	0.2594	115.122753138338\\
65.125	0.26106	116.740804704952\\
65.125	0.26272	118.358856271566\\
65.125	0.26438	119.97690783818\\
65.125	0.26604	121.594959404794\\
65.125	0.2677	123.213010971408\\
65.125	0.26936	124.831062538022\\
65.125	0.27102	126.449114104636\\
65.125	0.27268	128.06716567125\\
65.125	0.27434	129.685217237864\\
65.125	0.276	131.303268804478\\
65.3333333333333	0.193	50.4462457061103\\
65.3333333333333	0.19466	52.0769109187842\\
65.3333333333333	0.19632	53.707576131458\\
65.3333333333333	0.19798	55.338241344132\\
65.3333333333333	0.19964	56.9689065568058\\
65.3333333333333	0.2013	58.5995717694798\\
65.3333333333333	0.20296	60.2302369821537\\
65.3333333333333	0.20462	61.8609021948275\\
65.3333333333333	0.20628	63.4915674075015\\
65.3333333333333	0.20794	65.1222326201754\\
65.3333333333333	0.2096	66.7528978328493\\
65.3333333333333	0.21126	68.3835630455232\\
65.3333333333333	0.21292	70.0142282581971\\
65.3333333333333	0.21458	71.644893470871\\
65.3333333333333	0.21624	73.2755586835449\\
65.3333333333333	0.2179	74.9062238962188\\
65.3333333333333	0.21956	76.5368891088927\\
65.3333333333333	0.22122	78.1675543215666\\
65.3333333333333	0.22288	79.7982195342405\\
65.3333333333333	0.22454	81.4288847469144\\
65.3333333333333	0.2262	83.0595499595883\\
65.3333333333333	0.22786	84.6902151722622\\
65.3333333333333	0.22952	86.3208803849361\\
65.3333333333333	0.23118	87.9515455976099\\
65.3333333333333	0.23284	89.5822108102839\\
65.3333333333333	0.2345	91.2128760229577\\
65.3333333333333	0.23616	92.8435412356317\\
65.3333333333333	0.23782	94.4742064483056\\
65.3333333333333	0.23948	96.1048716609795\\
65.3333333333333	0.24114	97.7355368736534\\
65.3333333333333	0.2428	99.3662020863273\\
65.3333333333333	0.24446	100.996867299001\\
65.3333333333333	0.24612	102.627532511675\\
65.3333333333333	0.24778	104.258197724349\\
65.3333333333333	0.24944	105.888862937023\\
65.3333333333333	0.2511	107.519528149697\\
65.3333333333333	0.25276	109.150193362371\\
65.3333333333333	0.25442	110.780858575045\\
65.3333333333333	0.25608	112.411523787719\\
65.3333333333333	0.25774	114.042189000392\\
65.3333333333333	0.2594	115.672854213066\\
65.3333333333333	0.26106	117.30351942574\\
65.3333333333333	0.26272	118.934184638414\\
65.3333333333333	0.26438	120.564849851088\\
65.3333333333333	0.26604	122.195515063762\\
65.3333333333333	0.2677	123.826180276436\\
65.3333333333333	0.26936	125.45684548911\\
65.3333333333333	0.27102	127.087510701784\\
65.3333333333333	0.27268	128.718175914458\\
65.3333333333333	0.27434	130.348841127131\\
65.3333333333333	0.276	131.979506339805\\
65.5416666666667	0.193	50.4918009384425\\
65.5416666666667	0.19466	52.1350797971763\\
65.5416666666667	0.19632	53.7783586559101\\
65.5416666666667	0.19798	55.421637514644\\
65.5416666666667	0.19964	57.0649163733777\\
65.5416666666667	0.2013	58.7081952321116\\
65.5416666666667	0.20296	60.3514740908454\\
65.5416666666667	0.20462	61.9947529495792\\
65.5416666666667	0.20628	63.638031808313\\
65.5416666666667	0.20794	65.2813106670468\\
65.5416666666667	0.2096	66.9245895257806\\
65.5416666666667	0.21126	68.5678683845144\\
65.5416666666667	0.21292	70.2111472432482\\
65.5416666666667	0.21458	71.854426101982\\
65.5416666666667	0.21624	73.4977049607158\\
65.5416666666667	0.2179	75.1409838194497\\
65.5416666666667	0.21956	76.7842626781835\\
65.5416666666667	0.22122	78.4275415369173\\
65.5416666666667	0.22288	80.0708203956511\\
65.5416666666667	0.22454	81.7140992543849\\
65.5416666666667	0.2262	83.3573781131187\\
65.5416666666667	0.22786	85.0006569718524\\
65.5416666666667	0.22952	86.6439358305863\\
65.5416666666667	0.23118	88.2872146893201\\
65.5416666666667	0.23284	89.9304935480539\\
65.5416666666667	0.2345	91.5737724067877\\
65.5416666666667	0.23616	93.2170512655215\\
65.5416666666667	0.23782	94.8603301242553\\
65.5416666666667	0.23948	96.5036089829892\\
65.5416666666667	0.24114	98.146887841723\\
65.5416666666667	0.2428	99.7901667004568\\
65.5416666666667	0.24446	101.433445559191\\
65.5416666666667	0.24612	103.076724417924\\
65.5416666666667	0.24778	104.720003276658\\
65.5416666666667	0.24944	106.363282135392\\
65.5416666666667	0.2511	108.006560994126\\
65.5416666666667	0.25276	109.64983985286\\
65.5416666666667	0.25442	111.293118711593\\
65.5416666666667	0.25608	112.936397570327\\
65.5416666666667	0.25774	114.579676429061\\
65.5416666666667	0.2594	116.222955287795\\
65.5416666666667	0.26106	117.866234146529\\
65.5416666666667	0.26272	119.509513005262\\
65.5416666666667	0.26438	121.152791863996\\
65.5416666666667	0.26604	122.79607072273\\
65.5416666666667	0.2677	124.439349581464\\
65.5416666666667	0.26936	126.082628440198\\
65.5416666666667	0.27102	127.725907298931\\
65.5416666666667	0.27268	129.369186157665\\
65.5416666666667	0.27434	131.012465016399\\
65.5416666666667	0.276	132.655743875133\\
65.75	0.193	50.5373561707748\\
65.75	0.19466	52.1932486755685\\
65.75	0.19632	53.8491411803622\\
65.75	0.19798	55.5050336851559\\
65.75	0.19964	57.1609261899496\\
65.75	0.2013	58.8168186947434\\
65.75	0.20296	60.4727111995371\\
65.75	0.20462	62.1286037043308\\
65.75	0.20628	63.7844962091245\\
65.75	0.20794	65.4403887139182\\
65.75	0.2096	67.0962812187119\\
65.75	0.21126	68.7521737235057\\
65.75	0.21292	70.4080662282993\\
65.75	0.21458	72.063958733093\\
65.75	0.21624	73.7198512378868\\
65.75	0.2179	75.3757437426805\\
65.75	0.21956	77.0316362474743\\
65.75	0.22122	78.687528752268\\
65.75	0.22288	80.3434212570617\\
65.75	0.22454	81.9993137618554\\
65.75	0.2262	83.655206266649\\
65.75	0.22786	85.3110987714428\\
65.75	0.22952	86.9669912762365\\
65.75	0.23118	88.6228837810302\\
65.75	0.23284	90.2787762858239\\
65.75	0.2345	91.9346687906177\\
65.75	0.23616	93.5905612954114\\
65.75	0.23782	95.2464538002051\\
65.75	0.23948	96.9023463049988\\
65.75	0.24114	98.5582388097924\\
65.75	0.2428	100.214131314586\\
65.75	0.24446	101.87002381938\\
65.75	0.24612	103.525916324174\\
65.75	0.24778	105.181808828967\\
65.75	0.24944	106.837701333761\\
65.75	0.2511	108.493593838555\\
65.75	0.25276	110.149486343348\\
65.75	0.25442	111.805378848142\\
65.75	0.25608	113.461271352936\\
65.75	0.25774	115.11716385773\\
65.75	0.2594	116.773056362523\\
65.75	0.26106	118.428948867317\\
65.75	0.26272	120.084841372111\\
65.75	0.26438	121.740733876904\\
65.75	0.26604	123.396626381698\\
65.75	0.2677	125.052518886492\\
65.75	0.26936	126.708411391286\\
65.75	0.27102	128.364303896079\\
65.75	0.27268	130.020196400873\\
65.75	0.27434	131.676088905667\\
65.75	0.276	133.33198141046\\
65.9583333333333	0.193	50.582911403107\\
65.9583333333333	0.19466	52.2514175539606\\
65.9583333333333	0.19632	53.9199237048142\\
65.9583333333333	0.19798	55.5884298556679\\
65.9583333333333	0.19964	57.2569360065215\\
65.9583333333333	0.2013	58.9254421573751\\
65.9583333333333	0.20296	60.5939483082287\\
65.9583333333333	0.20462	62.2624544590823\\
65.9583333333333	0.20628	63.930960609936\\
65.9583333333333	0.20794	65.5994667607896\\
65.9583333333333	0.2096	67.2679729116432\\
65.9583333333333	0.21126	68.9364790624969\\
65.9583333333333	0.21292	70.6049852133505\\
65.9583333333333	0.21458	72.273491364204\\
65.9583333333333	0.21624	73.9419975150577\\
65.9583333333333	0.2179	75.6105036659113\\
65.9583333333333	0.21956	77.2790098167649\\
65.9583333333333	0.22122	78.9475159676186\\
65.9583333333333	0.22288	80.6160221184722\\
65.9583333333333	0.22454	82.2845282693258\\
65.9583333333333	0.2262	83.9530344201794\\
65.9583333333333	0.22786	85.621540571033\\
65.9583333333333	0.22952	87.2900467218867\\
65.9583333333333	0.23118	88.9585528727403\\
65.9583333333333	0.23284	90.6270590235939\\
65.9583333333333	0.2345	92.2955651744476\\
65.9583333333333	0.23616	93.9640713253011\\
65.9583333333333	0.23782	95.6325774761548\\
65.9583333333333	0.23948	97.3010836270084\\
65.9583333333333	0.24114	98.969589777862\\
65.9583333333333	0.2428	100.638095928716\\
65.9583333333333	0.24446	102.306602079569\\
65.9583333333333	0.24612	103.975108230423\\
65.9583333333333	0.24778	105.643614381276\\
65.9583333333333	0.24944	107.31212053213\\
65.9583333333333	0.2511	108.980626682984\\
65.9583333333333	0.25276	110.649132833837\\
65.9583333333333	0.25442	112.317638984691\\
65.9583333333333	0.25608	113.986145135545\\
65.9583333333333	0.25774	115.654651286398\\
65.9583333333333	0.2594	117.323157437252\\
65.9583333333333	0.26106	118.991663588105\\
65.9583333333333	0.26272	120.660169738959\\
65.9583333333333	0.26438	122.328675889813\\
65.9583333333333	0.26604	123.997182040666\\
65.9583333333333	0.2677	125.66568819152\\
65.9583333333333	0.26936	127.334194342374\\
65.9583333333333	0.27102	129.002700493227\\
65.9583333333333	0.27268	130.671206644081\\
65.9583333333333	0.27434	132.339712794934\\
65.9583333333333	0.276	134.008218945788\\
66.1666666666667	0.193	50.6284666354393\\
66.1666666666667	0.19466	52.3095864323529\\
66.1666666666667	0.19632	53.9907062292664\\
66.1666666666667	0.19798	55.6718260261799\\
66.1666666666667	0.19964	57.3529458230934\\
66.1666666666667	0.2013	59.034065620007\\
66.1666666666667	0.20296	60.7151854169205\\
66.1666666666667	0.20462	62.396305213834\\
66.1666666666667	0.20628	64.0774250107476\\
66.1666666666667	0.20794	65.7585448076611\\
66.1666666666667	0.2096	67.4396646045746\\
66.1666666666667	0.21126	69.1207844014882\\
66.1666666666667	0.21292	70.8019041984017\\
66.1666666666667	0.21458	72.4830239953151\\
66.1666666666667	0.21624	74.1641437922288\\
66.1666666666667	0.2179	75.8452635891422\\
66.1666666666667	0.21956	77.5263833860558\\
66.1666666666667	0.22122	79.2075031829693\\
66.1666666666667	0.22288	80.8886229798829\\
66.1666666666667	0.22454	82.5697427767964\\
66.1666666666667	0.2262	84.2508625737099\\
66.1666666666667	0.22786	85.9319823706234\\
66.1666666666667	0.22952	87.613102167537\\
66.1666666666667	0.23118	89.2942219644505\\
66.1666666666667	0.23284	90.9753417613641\\
66.1666666666667	0.2345	92.6564615582776\\
66.1666666666667	0.23616	94.337581355191\\
66.1666666666667	0.23782	96.0187011521046\\
66.1666666666667	0.23948	97.6998209490181\\
66.1666666666667	0.24114	99.3809407459317\\
66.1666666666667	0.2428	101.062060542845\\
66.1666666666667	0.24446	102.743180339759\\
66.1666666666667	0.24612	104.424300136672\\
66.1666666666667	0.24778	106.105419933586\\
66.1666666666667	0.24944	107.786539730499\\
66.1666666666667	0.2511	109.467659527413\\
66.1666666666667	0.25276	111.148779324326\\
66.1666666666667	0.25442	112.82989912124\\
66.1666666666667	0.25608	114.511018918153\\
66.1666666666667	0.25774	116.192138715067\\
66.1666666666667	0.2594	117.873258511981\\
66.1666666666667	0.26106	119.554378308894\\
66.1666666666667	0.26272	121.235498105808\\
66.1666666666667	0.26438	122.916617902721\\
66.1666666666667	0.26604	124.597737699635\\
66.1666666666667	0.2677	126.278857496548\\
66.1666666666667	0.26936	127.959977293462\\
66.1666666666667	0.27102	129.641097090375\\
66.1666666666667	0.27268	131.322216887289\\
66.1666666666667	0.27434	133.003336684202\\
66.1666666666667	0.276	134.684456481116\\
66.375	0.193	50.6740218677716\\
66.375	0.19466	52.3677553107451\\
66.375	0.19632	54.0614887537184\\
66.375	0.19798	55.7552221966919\\
66.375	0.19964	57.4489556396653\\
66.375	0.2013	59.1426890826388\\
66.375	0.20296	60.8364225256122\\
66.375	0.20462	62.5301559685856\\
66.375	0.20628	64.2238894115591\\
66.375	0.20794	65.9176228545325\\
66.375	0.2096	67.6113562975059\\
66.375	0.21126	69.3050897404794\\
66.375	0.21292	70.9988231834528\\
66.375	0.21458	72.6925566264263\\
66.375	0.21624	74.3862900693997\\
66.375	0.2179	76.0800235123731\\
66.375	0.21956	77.7737569553466\\
66.375	0.22122	79.46749039832\\
66.375	0.22288	81.1612238412935\\
66.375	0.22454	82.8549572842668\\
66.375	0.2262	84.5486907272403\\
66.375	0.22786	86.2424241702137\\
66.375	0.22952	87.9361576131872\\
66.375	0.23118	89.6298910561606\\
66.375	0.23284	91.3236244991341\\
66.375	0.2345	93.0173579421075\\
66.375	0.23616	94.7110913850809\\
66.375	0.23782	96.4048248280544\\
66.375	0.23948	98.0985582710277\\
66.375	0.24114	99.7922917140012\\
66.375	0.2428	101.486025156975\\
66.375	0.24446	103.179758599948\\
66.375	0.24612	104.873492042922\\
66.375	0.24778	106.567225485895\\
66.375	0.24944	108.260958928868\\
66.375	0.2511	109.954692371842\\
66.375	0.25276	111.648425814815\\
66.375	0.25442	113.342159257789\\
66.375	0.25608	115.035892700762\\
66.375	0.25774	116.729626143736\\
66.375	0.2594	118.423359586709\\
66.375	0.26106	120.117093029683\\
66.375	0.26272	121.810826472656\\
66.375	0.26438	123.504559915629\\
66.375	0.26604	125.198293358603\\
66.375	0.2677	126.892026801576\\
66.375	0.26936	128.58576024455\\
66.375	0.27102	130.279493687523\\
66.375	0.27268	131.973227130497\\
66.375	0.27434	133.66696057347\\
66.375	0.276	135.360694016443\\
66.5833333333333	0.193	50.7195771001038\\
66.5833333333333	0.19466	52.4259241891372\\
66.5833333333333	0.19632	54.1322712781705\\
66.5833333333333	0.19798	55.8386183672039\\
66.5833333333333	0.19964	57.5449654562372\\
66.5833333333333	0.2013	59.2513125452705\\
66.5833333333333	0.20296	60.9576596343039\\
66.5833333333333	0.20462	62.6640067233372\\
66.5833333333333	0.20628	64.3703538123706\\
66.5833333333333	0.20794	66.0767009014039\\
66.5833333333333	0.2096	67.7830479904372\\
66.5833333333333	0.21126	69.4893950794706\\
66.5833333333333	0.21292	71.1957421685039\\
66.5833333333333	0.21458	72.9020892575372\\
66.5833333333333	0.21624	74.6084363465707\\
66.5833333333333	0.2179	76.3147834356039\\
66.5833333333333	0.21956	78.0211305246373\\
66.5833333333333	0.22122	79.7274776136706\\
66.5833333333333	0.22288	81.433824702704\\
66.5833333333333	0.22454	83.1401717917374\\
66.5833333333333	0.2262	84.8465188807707\\
66.5833333333333	0.22786	86.5528659698039\\
66.5833333333333	0.22952	88.2592130588373\\
66.5833333333333	0.23118	89.9655601478707\\
66.5833333333333	0.23284	91.6719072369041\\
66.5833333333333	0.2345	93.3782543259374\\
66.5833333333333	0.23616	95.0846014149707\\
66.5833333333333	0.23782	96.7909485040041\\
66.5833333333333	0.23948	98.4972955930374\\
66.5833333333333	0.24114	100.203642682071\\
66.5833333333333	0.2428	101.909989771104\\
66.5833333333333	0.24446	103.616336860137\\
66.5833333333333	0.24612	105.322683949171\\
66.5833333333333	0.24778	107.029031038204\\
66.5833333333333	0.24944	108.735378127237\\
66.5833333333333	0.2511	110.441725216271\\
66.5833333333333	0.25276	112.148072305304\\
66.5833333333333	0.25442	113.854419394337\\
66.5833333333333	0.25608	115.560766483371\\
66.5833333333333	0.25774	117.267113572404\\
66.5833333333333	0.2594	118.973460661437\\
66.5833333333333	0.26106	120.679807750471\\
66.5833333333333	0.26272	122.386154839504\\
66.5833333333333	0.26438	124.092501928537\\
66.5833333333333	0.26604	125.798849017571\\
66.5833333333333	0.2677	127.505196106604\\
66.5833333333333	0.26936	129.211543195638\\
66.5833333333333	0.27102	130.917890284671\\
66.5833333333333	0.27268	132.624237373704\\
66.5833333333333	0.27434	134.330584462738\\
66.5833333333333	0.276	136.036931551771\\
66.7916666666667	0.193	50.7651323324361\\
66.7916666666667	0.19466	52.4840930675293\\
66.7916666666667	0.19632	54.2030538026225\\
66.7916666666667	0.19798	55.9220145377158\\
66.7916666666667	0.19964	57.640975272809\\
66.7916666666667	0.2013	59.3599360079023\\
66.7916666666667	0.20296	61.0788967429955\\
66.7916666666667	0.20462	62.7978574780888\\
66.7916666666667	0.20628	64.5168182131821\\
66.7916666666667	0.20794	66.2357789482753\\
66.7916666666667	0.2096	67.9547396833685\\
66.7916666666667	0.21126	69.6737004184618\\
66.7916666666667	0.21292	71.392661153555\\
66.7916666666667	0.21458	73.1116218886482\\
66.7916666666667	0.21624	74.8305826237416\\
66.7916666666667	0.2179	76.5495433588348\\
66.7916666666667	0.21956	78.2685040939281\\
66.7916666666667	0.22122	79.9874648290213\\
66.7916666666667	0.22288	81.7064255641146\\
66.7916666666667	0.22454	83.4253862992078\\
66.7916666666667	0.2262	85.1443470343011\\
66.7916666666667	0.22786	86.8633077693943\\
66.7916666666667	0.22952	88.5822685044876\\
66.7916666666667	0.23118	90.3012292395808\\
66.7916666666667	0.23284	92.0201899746741\\
66.7916666666667	0.2345	93.7391507097673\\
66.7916666666667	0.23616	95.4581114448605\\
66.7916666666667	0.23782	97.1770721799538\\
66.7916666666667	0.23948	98.896032915047\\
66.7916666666667	0.24114	100.61499365014\\
66.7916666666667	0.2428	102.333954385234\\
66.7916666666667	0.24446	104.052915120327\\
66.7916666666667	0.24612	105.77187585542\\
66.7916666666667	0.24778	107.490836590513\\
66.7916666666667	0.24944	109.209797325606\\
66.7916666666667	0.2511	110.9287580607\\
66.7916666666667	0.25276	112.647718795793\\
66.7916666666667	0.25442	114.366679530886\\
66.7916666666667	0.25608	116.08564026598\\
66.7916666666667	0.25774	117.804601001073\\
66.7916666666667	0.2594	119.523561736166\\
66.7916666666667	0.26106	121.242522471259\\
66.7916666666667	0.26272	122.961483206352\\
66.7916666666667	0.26438	124.680443941446\\
66.7916666666667	0.26604	126.399404676539\\
66.7916666666667	0.2677	128.118365411632\\
66.7916666666667	0.26936	129.837326146726\\
66.7916666666667	0.27102	131.556286881819\\
66.7916666666667	0.27268	133.275247616912\\
66.7916666666667	0.27434	134.994208352005\\
66.7916666666667	0.276	136.713169087099\\
67	0.193	50.8106875647684\\
67	0.19466	52.5422619459215\\
67	0.19632	54.2738363270746\\
67	0.19798	56.0054107082278\\
67	0.19964	57.7369850893809\\
67	0.2013	59.4685594705342\\
67	0.20296	61.2001338516873\\
67	0.20462	62.9317082328404\\
67	0.20628	64.6632826139936\\
67	0.20794	66.3948569951467\\
67	0.2096	68.1264313762999\\
67	0.21126	69.858005757453\\
67	0.21292	71.5895801386062\\
67	0.21458	73.3211545197594\\
67	0.21624	75.0527289009125\\
67	0.2179	76.7843032820657\\
67	0.21956	78.5158776632189\\
67	0.22122	80.247452044372\\
67	0.22288	81.9790264255251\\
67	0.22454	83.7106008066783\\
67	0.2262	85.4421751878314\\
67	0.22786	87.1737495689845\\
67	0.22952	88.9053239501378\\
67	0.23118	90.6368983312909\\
67	0.23284	92.368472712444\\
67	0.2345	94.1000470935973\\
67	0.23616	95.8316214747504\\
67	0.23782	97.5631958559036\\
67	0.23948	99.2947702370567\\
67	0.24114	101.02634461821\\
67	0.2428	102.757918999363\\
67	0.24446	104.489493380516\\
67	0.24612	106.221067761669\\
67	0.24778	107.952642142822\\
67	0.24944	109.684216523976\\
67	0.2511	111.415790905129\\
67	0.25276	113.147365286282\\
67	0.25442	114.878939667435\\
67	0.25608	116.610514048588\\
67	0.25774	118.342088429741\\
67	0.2594	120.073662810894\\
67	0.26106	121.805237192048\\
67	0.26272	123.536811573201\\
67	0.26438	125.268385954354\\
67	0.26604	126.999960335507\\
67	0.2677	128.73153471666\\
67	0.26936	130.463109097814\\
67	0.27102	132.194683478967\\
67	0.27268	133.92625786012\\
67	0.27434	135.657832241273\\
67	0.276	137.389406622426\\
};
\end{axis}

\begin{axis}[%
width=4.927496cm,
height=3.050847cm,
at={(6.483547cm,16.949153cm)},
scale only axis,
xmin=57,
xmax=67,
tick align=outside,
xlabel={$L_{cut}$},
xmajorgrids,
ymin=0.193,
ymax=0.276,
ylabel={$D_{rlx}$},
ymajorgrids,
zmin=238.026507324664,
zmax=670.729371707935,
zlabel={$x_3$},
zmajorgrids,
view={-140}{50},
legend style={at={(1.03,1)},anchor=north west,legend cell align=left,align=left,draw=white!15!black}
]
\addplot3[only marks,mark=*,mark options={},mark size=1.5000pt,color=mycolor1] plot table[row sep=crcr,]{%
67	0.276	670.729371707823\\
66	0.255	553.593679878151\\
62	0.209	311.924716952636\\
57	0.193	238.026507324664\\
};
\addplot3[only marks,mark=*,mark options={},mark size=1.5000pt,color=black] plot table[row sep=crcr,]{%
64	0.23	417.144665367083\\
};

\addplot3[%
surf,
opacity=0.7,
shader=interp,
colormap={mymap}{[1pt] rgb(0pt)=(0.0901961,0.239216,0.0745098); rgb(1pt)=(0.0945149,0.242058,0.0739522); rgb(2pt)=(0.0988592,0.244894,0.0733566); rgb(3pt)=(0.103229,0.247724,0.0727241); rgb(4pt)=(0.107623,0.250549,0.0720557); rgb(5pt)=(0.112043,0.253367,0.0713525); rgb(6pt)=(0.116487,0.25618,0.0706154); rgb(7pt)=(0.120956,0.258986,0.0698456); rgb(8pt)=(0.125449,0.261787,0.0690441); rgb(9pt)=(0.129967,0.264581,0.0682118); rgb(10pt)=(0.134508,0.26737,0.06735); rgb(11pt)=(0.139074,0.270152,0.0664596); rgb(12pt)=(0.143663,0.272929,0.0655416); rgb(13pt)=(0.148275,0.275699,0.0645971); rgb(14pt)=(0.152911,0.278463,0.0636271); rgb(15pt)=(0.15757,0.281221,0.0626328); rgb(16pt)=(0.162252,0.283973,0.0616151); rgb(17pt)=(0.166957,0.286719,0.060575); rgb(18pt)=(0.171685,0.289458,0.0595136); rgb(19pt)=(0.176434,0.292191,0.0584321); rgb(20pt)=(0.181207,0.294918,0.0573313); rgb(21pt)=(0.186001,0.297639,0.0562123); rgb(22pt)=(0.190817,0.300353,0.0550763); rgb(23pt)=(0.195655,0.303061,0.0539242); rgb(24pt)=(0.200514,0.305763,0.052757); rgb(25pt)=(0.205395,0.308459,0.0515759); rgb(26pt)=(0.210296,0.311149,0.0503624); rgb(27pt)=(0.215212,0.313846,0.0490067); rgb(28pt)=(0.220142,0.316548,0.0475043); rgb(29pt)=(0.22509,0.319254,0.0458704); rgb(30pt)=(0.230056,0.321962,0.0441205); rgb(31pt)=(0.235042,0.324671,0.04227); rgb(32pt)=(0.240048,0.327379,0.0403343); rgb(33pt)=(0.245078,0.330085,0.0383287); rgb(34pt)=(0.250131,0.332786,0.0362688); rgb(35pt)=(0.25521,0.335482,0.0341698); rgb(36pt)=(0.260317,0.33817,0.0320472); rgb(37pt)=(0.265451,0.340849,0.0299163); rgb(38pt)=(0.270616,0.343517,0.0277927); rgb(39pt)=(0.275813,0.346172,0.0256916); rgb(40pt)=(0.281043,0.348814,0.0236284); rgb(41pt)=(0.286307,0.35144,0.0216186); rgb(42pt)=(0.291607,0.354048,0.0196776); rgb(43pt)=(0.296945,0.356637,0.0178207); rgb(44pt)=(0.302322,0.359206,0.0160634); rgb(45pt)=(0.307739,0.361753,0.0144211); rgb(46pt)=(0.313198,0.364275,0.0129091); rgb(47pt)=(0.318701,0.366772,0.0115428); rgb(48pt)=(0.324249,0.369242,0.0103377); rgb(49pt)=(0.329843,0.371682,0.00930909); rgb(50pt)=(0.335485,0.374093,0.00847245); rgb(51pt)=(0.341176,0.376471,0.00784314); rgb(52pt)=(0.346925,0.378826,0.00732741); rgb(53pt)=(0.352735,0.381168,0.00682184); rgb(54pt)=(0.358605,0.383497,0.00632729); rgb(55pt)=(0.364532,0.385812,0.00584464); rgb(56pt)=(0.370516,0.388113,0.00537476); rgb(57pt)=(0.376552,0.390399,0.00491852); rgb(58pt)=(0.38264,0.39267,0.00447681); rgb(59pt)=(0.388777,0.394925,0.00405048); rgb(60pt)=(0.394962,0.397164,0.00364042); rgb(61pt)=(0.401191,0.399386,0.00324749); rgb(62pt)=(0.407464,0.401592,0.00287258); rgb(63pt)=(0.413777,0.40378,0.00251655); rgb(64pt)=(0.420129,0.40595,0.00218028); rgb(65pt)=(0.426518,0.408102,0.00186463); rgb(66pt)=(0.432942,0.410234,0.00157049); rgb(67pt)=(0.439399,0.412348,0.00129873); rgb(68pt)=(0.445885,0.414441,0.00105022); rgb(69pt)=(0.452401,0.416515,0.000825833); rgb(70pt)=(0.458942,0.418567,0.000626441); rgb(71pt)=(0.465508,0.420599,0.00045292); rgb(72pt)=(0.472096,0.422609,0.000306141); rgb(73pt)=(0.478704,0.424596,0.000186979); rgb(74pt)=(0.485331,0.426562,9.63073e-05); rgb(75pt)=(0.491973,0.428504,3.49981e-05); rgb(76pt)=(0.498628,0.430422,3.92506e-06); rgb(77pt)=(0.505323,0.432315,0); rgb(78pt)=(0.512206,0.434168,0); rgb(79pt)=(0.519282,0.435983,0); rgb(80pt)=(0.526529,0.437764,0); rgb(81pt)=(0.533922,0.439512,0); rgb(82pt)=(0.54144,0.441232,0); rgb(83pt)=(0.549059,0.442927,0); rgb(84pt)=(0.556756,0.444599,0); rgb(85pt)=(0.564508,0.446252,0); rgb(86pt)=(0.572292,0.447889,0); rgb(87pt)=(0.580084,0.449514,0); rgb(88pt)=(0.587863,0.451129,0); rgb(89pt)=(0.595604,0.452737,0); rgb(90pt)=(0.603284,0.454343,0); rgb(91pt)=(0.610882,0.455948,0); rgb(92pt)=(0.618373,0.457556,0); rgb(93pt)=(0.625734,0.459171,0); rgb(94pt)=(0.632943,0.460795,0); rgb(95pt)=(0.639976,0.462432,0); rgb(96pt)=(0.64681,0.464084,0); rgb(97pt)=(0.653423,0.465756,0); rgb(98pt)=(0.659791,0.46745,0); rgb(99pt)=(0.665891,0.469169,0); rgb(100pt)=(0.6717,0.470916,0); rgb(101pt)=(0.677195,0.472696,0); rgb(102pt)=(0.682353,0.47451,0); rgb(103pt)=(0.687242,0.476355,0); rgb(104pt)=(0.691952,0.478225,0); rgb(105pt)=(0.696497,0.480118,0); rgb(106pt)=(0.700887,0.482033,0); rgb(107pt)=(0.705134,0.483968,0); rgb(108pt)=(0.709251,0.485921,0); rgb(109pt)=(0.713249,0.487891,0); rgb(110pt)=(0.71714,0.489876,0); rgb(111pt)=(0.720936,0.491875,0); rgb(112pt)=(0.724649,0.493887,0); rgb(113pt)=(0.72829,0.495909,0); rgb(114pt)=(0.731872,0.49794,0); rgb(115pt)=(0.735406,0.499979,0); rgb(116pt)=(0.738904,0.502025,0); rgb(117pt)=(0.742378,0.504075,0); rgb(118pt)=(0.74584,0.506128,0); rgb(119pt)=(0.749302,0.508182,0); rgb(120pt)=(0.752775,0.510237,0); rgb(121pt)=(0.756272,0.51229,0); rgb(122pt)=(0.759804,0.514339,0); rgb(123pt)=(0.763384,0.516385,0); rgb(124pt)=(0.767022,0.518424,0); rgb(125pt)=(0.770731,0.520455,0); rgb(126pt)=(0.774523,0.522478,0); rgb(127pt)=(0.77841,0.524489,0); rgb(128pt)=(0.782391,0.526491,0); rgb(129pt)=(0.786402,0.528496,0); rgb(130pt)=(0.790431,0.530506,0); rgb(131pt)=(0.794478,0.532521,0); rgb(132pt)=(0.798541,0.534539,0); rgb(133pt)=(0.802619,0.53656,0); rgb(134pt)=(0.806712,0.538584,0); rgb(135pt)=(0.81082,0.540609,0); rgb(136pt)=(0.81494,0.542635,0); rgb(137pt)=(0.819074,0.54466,0); rgb(138pt)=(0.823219,0.546686,0); rgb(139pt)=(0.827374,0.548709,0); rgb(140pt)=(0.831541,0.55073,0); rgb(141pt)=(0.835716,0.552749,0); rgb(142pt)=(0.8399,0.554763,0); rgb(143pt)=(0.844092,0.556774,0); rgb(144pt)=(0.848292,0.558779,0); rgb(145pt)=(0.852497,0.560778,0); rgb(146pt)=(0.856708,0.562771,0); rgb(147pt)=(0.860924,0.564756,0); rgb(148pt)=(0.865143,0.566733,0); rgb(149pt)=(0.869366,0.568701,0); rgb(150pt)=(0.873592,0.57066,0); rgb(151pt)=(0.877819,0.572608,0); rgb(152pt)=(0.882047,0.574545,0); rgb(153pt)=(0.886275,0.576471,0); rgb(154pt)=(0.890659,0.578362,0); rgb(155pt)=(0.895333,0.580203,0); rgb(156pt)=(0.900258,0.581999,0); rgb(157pt)=(0.905397,0.583755,0); rgb(158pt)=(0.910711,0.585479,0); rgb(159pt)=(0.916164,0.587176,0); rgb(160pt)=(0.921717,0.588852,0); rgb(161pt)=(0.927333,0.590513,0); rgb(162pt)=(0.932974,0.592166,0); rgb(163pt)=(0.938602,0.593815,0); rgb(164pt)=(0.94418,0.595468,0); rgb(165pt)=(0.949669,0.59713,0); rgb(166pt)=(0.955033,0.598808,0); rgb(167pt)=(0.960233,0.600507,0); rgb(168pt)=(0.965232,0.602233,0); rgb(169pt)=(0.969992,0.603992,0); rgb(170pt)=(0.974475,0.605791,0); rgb(171pt)=(0.978643,0.607636,0); rgb(172pt)=(0.98246,0.609532,0); rgb(173pt)=(0.985886,0.611486,0); rgb(174pt)=(0.988885,0.613503,0); rgb(175pt)=(0.991419,0.61559,0); rgb(176pt)=(0.99345,0.617753,0); rgb(177pt)=(0.99494,0.619997,0); rgb(178pt)=(0.995851,0.622329,0); rgb(179pt)=(0.996226,0.624763,0); rgb(180pt)=(0.996512,0.627352,0); rgb(181pt)=(0.996788,0.630095,0); rgb(182pt)=(0.997053,0.632982,0); rgb(183pt)=(0.997308,0.636004,0); rgb(184pt)=(0.997552,0.639152,0); rgb(185pt)=(0.997785,0.642416,0); rgb(186pt)=(0.998006,0.645786,0); rgb(187pt)=(0.998217,0.649253,0); rgb(188pt)=(0.998416,0.652807,0); rgb(189pt)=(0.998605,0.656439,0); rgb(190pt)=(0.998781,0.660138,0); rgb(191pt)=(0.998946,0.663897,0); rgb(192pt)=(0.9991,0.667704,0); rgb(193pt)=(0.999242,0.67155,0); rgb(194pt)=(0.999372,0.675427,0); rgb(195pt)=(0.99949,0.679323,0); rgb(196pt)=(0.999596,0.68323,0); rgb(197pt)=(0.99969,0.687139,0); rgb(198pt)=(0.999771,0.691039,0); rgb(199pt)=(0.999841,0.694921,0); rgb(200pt)=(0.999898,0.698775,0); rgb(201pt)=(0.999942,0.702592,0); rgb(202pt)=(0.999974,0.706363,0); rgb(203pt)=(0.999994,0.710077,0); rgb(204pt)=(1,0.713725,0); rgb(205pt)=(1,0.717341,0); rgb(206pt)=(1,0.720963,0); rgb(207pt)=(1,0.724591,0); rgb(208pt)=(1,0.728226,0); rgb(209pt)=(1,0.731867,0); rgb(210pt)=(1,0.735514,0); rgb(211pt)=(1,0.739167,0); rgb(212pt)=(1,0.742827,0); rgb(213pt)=(1,0.746493,0); rgb(214pt)=(1,0.750165,0); rgb(215pt)=(1,0.753843,0); rgb(216pt)=(1,0.757527,0); rgb(217pt)=(1,0.761217,0); rgb(218pt)=(1,0.764913,0); rgb(219pt)=(1,0.768615,0); rgb(220pt)=(1,0.772324,0); rgb(221pt)=(1,0.776038,0); rgb(222pt)=(1,0.779758,0); rgb(223pt)=(1,0.783484,0); rgb(224pt)=(1,0.787215,0); rgb(225pt)=(1,0.790953,0); rgb(226pt)=(1,0.794696,0); rgb(227pt)=(1,0.798445,0); rgb(228pt)=(1,0.8022,0); rgb(229pt)=(1,0.805961,0); rgb(230pt)=(1,0.809727,0); rgb(231pt)=(1,0.8135,0); rgb(232pt)=(1,0.817278,0); rgb(233pt)=(1,0.821063,0); rgb(234pt)=(1,0.824854,0); rgb(235pt)=(1,0.828652,0); rgb(236pt)=(1,0.832455,0); rgb(237pt)=(1,0.836265,0); rgb(238pt)=(1,0.840081,0); rgb(239pt)=(1,0.843903,0); rgb(240pt)=(1,0.847732,0); rgb(241pt)=(1,0.851566,0); rgb(242pt)=(1,0.855406,0); rgb(243pt)=(1,0.859253,0); rgb(244pt)=(1,0.863106,0); rgb(245pt)=(1,0.866964,0); rgb(246pt)=(1,0.870829,0); rgb(247pt)=(1,0.8747,0); rgb(248pt)=(1,0.878577,0); rgb(249pt)=(1,0.88246,0); rgb(250pt)=(1,0.886349,0); rgb(251pt)=(1,0.890243,0); rgb(252pt)=(1,0.894144,0); rgb(253pt)=(1,0.898051,0); rgb(254pt)=(1,0.901964,0); rgb(255pt)=(1,0.905882,0)},
mesh/rows=49]
table[row sep=crcr,header=false] {%
%
57	0.193	238.026507324708\\
57	0.19466	244.42778539353\\
57	0.19632	250.829063462352\\
57	0.19798	257.230341531175\\
57	0.19964	263.631619599997\\
57	0.2013	270.032897668819\\
57	0.20296	276.434175737642\\
57	0.20462	282.835453806464\\
57	0.20628	289.236731875286\\
57	0.20794	295.638009944109\\
57	0.2096	302.039288012931\\
57	0.21126	308.440566081753\\
57	0.21292	314.841844150576\\
57	0.21458	321.243122219398\\
57	0.21624	327.64440028822\\
57	0.2179	334.045678357042\\
57	0.21956	340.446956425865\\
57	0.22122	346.848234494687\\
57	0.22288	353.249512563509\\
57	0.22454	359.650790632332\\
57	0.2262	366.052068701154\\
57	0.22786	372.453346769976\\
57	0.22952	378.854624838799\\
57	0.23118	385.255902907621\\
57	0.23284	391.657180976443\\
57	0.2345	398.058459045265\\
57	0.23616	404.459737114088\\
57	0.23782	410.86101518291\\
57	0.23948	417.262293251732\\
57	0.24114	423.663571320555\\
57	0.2428	430.064849389377\\
57	0.24446	436.466127458199\\
57	0.24612	442.867405527022\\
57	0.24778	449.268683595844\\
57	0.24944	455.669961664666\\
57	0.2511	462.071239733488\\
57	0.25276	468.472517802311\\
57	0.25442	474.873795871133\\
57	0.25608	481.275073939956\\
57	0.25774	487.676352008778\\
57	0.2594	494.0776300776\\
57	0.26106	500.478908146422\\
57	0.26272	506.880186215245\\
57	0.26438	513.281464284067\\
57	0.26604	519.68274235289\\
57	0.2677	526.084020421712\\
57	0.26936	532.485298490534\\
57	0.27102	538.886576559356\\
57	0.27268	545.287854628179\\
57	0.27434	551.689132697001\\
57	0.276	558.090410765823\\
57.2083333333333	0.193	238.095796663343\\
57.2083333333333	0.19466	244.542621845786\\
57.2083333333333	0.19632	250.989447028228\\
57.2083333333333	0.19798	257.43627221067\\
57.2083333333333	0.19964	263.883097393112\\
57.2083333333333	0.2013	270.329922575554\\
57.2083333333333	0.20296	276.776747757996\\
57.2083333333333	0.20462	283.223572940438\\
57.2083333333333	0.20628	289.670398122881\\
57.2083333333333	0.20794	296.117223305323\\
57.2083333333333	0.2096	302.564048487765\\
57.2083333333333	0.21126	309.010873670207\\
57.2083333333333	0.21292	315.457698852649\\
57.2083333333333	0.21458	321.904524035091\\
57.2083333333333	0.21624	328.351349217533\\
57.2083333333333	0.2179	334.798174399976\\
57.2083333333333	0.21956	341.244999582418\\
57.2083333333333	0.22122	347.69182476486\\
57.2083333333333	0.22288	354.138649947302\\
57.2083333333333	0.22454	360.585475129744\\
57.2083333333333	0.2262	367.032300312186\\
57.2083333333333	0.22786	373.479125494629\\
57.2083333333333	0.22952	379.925950677071\\
57.2083333333333	0.23118	386.372775859513\\
57.2083333333333	0.23284	392.819601041955\\
57.2083333333333	0.2345	399.266426224397\\
57.2083333333333	0.23616	405.713251406839\\
57.2083333333333	0.23782	412.160076589281\\
57.2083333333333	0.23948	418.606901771723\\
57.2083333333333	0.24114	425.053726954166\\
57.2083333333333	0.2428	431.500552136608\\
57.2083333333333	0.24446	437.94737731905\\
57.2083333333333	0.24612	444.394202501492\\
57.2083333333333	0.24778	450.841027683934\\
57.2083333333333	0.24944	457.287852866376\\
57.2083333333333	0.2511	463.734678048819\\
57.2083333333333	0.25276	470.18150323126\\
57.2083333333333	0.25442	476.628328413703\\
57.2083333333333	0.25608	483.075153596145\\
57.2083333333333	0.25774	489.521978778587\\
57.2083333333333	0.2594	495.968803961029\\
57.2083333333333	0.26106	502.415629143471\\
57.2083333333333	0.26272	508.862454325913\\
57.2083333333333	0.26438	515.309279508356\\
57.2083333333333	0.26604	521.756104690798\\
57.2083333333333	0.2677	528.20292987324\\
57.2083333333333	0.26936	534.649755055682\\
57.2083333333333	0.27102	541.096580238124\\
57.2083333333333	0.27268	547.543405420566\\
57.2083333333333	0.27434	553.990230603009\\
57.2083333333333	0.276	560.43705578545\\
57.4166666666667	0.193	238.165086001979\\
57.4166666666667	0.19466	244.657458298041\\
57.4166666666667	0.19632	251.149830594103\\
57.4166666666667	0.19798	257.642202890165\\
57.4166666666667	0.19964	264.134575186227\\
57.4166666666667	0.2013	270.626947482289\\
57.4166666666667	0.20296	277.119319778351\\
57.4166666666667	0.20462	283.611692074413\\
57.4166666666667	0.20628	290.104064370475\\
57.4166666666667	0.20794	296.596436666537\\
57.4166666666667	0.2096	303.088808962599\\
57.4166666666667	0.21126	309.581181258661\\
57.4166666666667	0.21292	316.073553554723\\
57.4166666666667	0.21458	322.565925850785\\
57.4166666666667	0.21624	329.058298146847\\
57.4166666666667	0.2179	335.550670442909\\
57.4166666666667	0.21956	342.043042738971\\
57.4166666666667	0.22122	348.535415035033\\
57.4166666666667	0.22288	355.027787331094\\
57.4166666666667	0.22454	361.520159627157\\
57.4166666666667	0.2262	368.012531923218\\
57.4166666666667	0.22786	374.50490421928\\
57.4166666666667	0.22952	380.997276515342\\
57.4166666666667	0.23118	387.489648811405\\
57.4166666666667	0.23284	393.982021107466\\
57.4166666666667	0.2345	400.474393403528\\
57.4166666666667	0.23616	406.96676569959\\
57.4166666666667	0.23782	413.459137995652\\
57.4166666666667	0.23948	419.951510291714\\
57.4166666666667	0.24114	426.443882587776\\
57.4166666666667	0.2428	432.936254883838\\
57.4166666666667	0.24446	439.4286271799\\
57.4166666666667	0.24612	445.920999475962\\
57.4166666666667	0.24778	452.413371772024\\
57.4166666666667	0.24944	458.905744068086\\
57.4166666666667	0.2511	465.398116364148\\
57.4166666666667	0.25276	471.89048866021\\
57.4166666666667	0.25442	478.382860956272\\
57.4166666666667	0.25608	484.875233252334\\
57.4166666666667	0.25774	491.367605548396\\
57.4166666666667	0.2594	497.859977844458\\
57.4166666666667	0.26106	504.35235014052\\
57.4166666666667	0.26272	510.844722436582\\
57.4166666666667	0.26438	517.337094732644\\
57.4166666666667	0.26604	523.829467028706\\
57.4166666666667	0.2677	530.321839324768\\
57.4166666666667	0.26936	536.81421162083\\
57.4166666666667	0.27102	543.306583916892\\
57.4166666666667	0.27268	549.798956212954\\
57.4166666666667	0.27434	556.291328509016\\
57.4166666666667	0.276	562.783700805078\\
57.625	0.193	238.234375340615\\
57.625	0.19466	244.772294750297\\
57.625	0.19632	251.310214159978\\
57.625	0.19798	257.84813356966\\
57.625	0.19964	264.386052979342\\
57.625	0.2013	270.923972389024\\
57.625	0.20296	277.461891798705\\
57.625	0.20462	283.999811208387\\
57.625	0.20628	290.537730618069\\
57.625	0.20794	297.075650027751\\
57.625	0.2096	303.613569437433\\
57.625	0.21126	310.151488847115\\
57.625	0.21292	316.689408256796\\
57.625	0.21458	323.227327666478\\
57.625	0.21624	329.76524707616\\
57.625	0.2179	336.303166485842\\
57.625	0.21956	342.841085895524\\
57.625	0.22122	349.379005305205\\
57.625	0.22288	355.916924714887\\
57.625	0.22454	362.454844124569\\
57.625	0.2262	368.992763534251\\
57.625	0.22786	375.530682943933\\
57.625	0.22952	382.068602353615\\
57.625	0.23118	388.606521763296\\
57.625	0.23284	395.144441172978\\
57.625	0.2345	401.68236058266\\
57.625	0.23616	408.220279992342\\
57.625	0.23782	414.758199402024\\
57.625	0.23948	421.296118811705\\
57.625	0.24114	427.834038221387\\
57.625	0.2428	434.371957631069\\
57.625	0.24446	440.909877040751\\
57.625	0.24612	447.447796450433\\
57.625	0.24778	453.985715860115\\
57.625	0.24944	460.523635269796\\
57.625	0.2511	467.061554679478\\
57.625	0.25276	473.59947408916\\
57.625	0.25442	480.137393498842\\
57.625	0.25608	486.675312908523\\
57.625	0.25774	493.213232318205\\
57.625	0.2594	499.751151727887\\
57.625	0.26106	506.289071137569\\
57.625	0.26272	512.826990547251\\
57.625	0.26438	519.364909956932\\
57.625	0.26604	525.902829366614\\
57.625	0.2677	532.440748776296\\
57.625	0.26936	538.978668185978\\
57.625	0.27102	545.51658759566\\
57.625	0.27268	552.054507005341\\
57.625	0.27434	558.592426415024\\
57.625	0.276	565.130345824705\\
57.8333333333333	0.193	238.30366467925\\
57.8333333333333	0.19466	244.887131202552\\
57.8333333333333	0.19632	251.470597725854\\
57.8333333333333	0.19798	258.054064249156\\
57.8333333333333	0.19964	264.637530772457\\
57.8333333333333	0.2013	271.220997295759\\
57.8333333333333	0.20296	277.80446381906\\
57.8333333333333	0.20462	284.387930342362\\
57.8333333333333	0.20628	290.971396865664\\
57.8333333333333	0.20794	297.554863388965\\
57.8333333333333	0.2096	304.138329912267\\
57.8333333333333	0.21126	310.721796435568\\
57.8333333333333	0.21292	317.30526295887\\
57.8333333333333	0.21458	323.888729482172\\
57.8333333333333	0.21624	330.472196005474\\
57.8333333333333	0.2179	337.055662528775\\
57.8333333333333	0.21956	343.639129052077\\
57.8333333333333	0.22122	350.222595575378\\
57.8333333333333	0.22288	356.80606209868\\
57.8333333333333	0.22454	363.389528621982\\
57.8333333333333	0.2262	369.972995145283\\
57.8333333333333	0.22786	376.556461668585\\
57.8333333333333	0.22952	383.139928191887\\
57.8333333333333	0.23118	389.723394715188\\
57.8333333333333	0.23284	396.30686123849\\
57.8333333333333	0.2345	402.890327761791\\
57.8333333333333	0.23616	409.473794285093\\
57.8333333333333	0.23782	416.057260808395\\
57.8333333333333	0.23948	422.640727331696\\
57.8333333333333	0.24114	429.224193854998\\
57.8333333333333	0.2428	435.8076603783\\
57.8333333333333	0.24446	442.391126901601\\
57.8333333333333	0.24612	448.974593424903\\
57.8333333333333	0.24778	455.558059948205\\
57.8333333333333	0.24944	462.141526471506\\
57.8333333333333	0.2511	468.724992994808\\
57.8333333333333	0.25276	475.308459518109\\
57.8333333333333	0.25442	481.891926041411\\
57.8333333333333	0.25608	488.475392564713\\
57.8333333333333	0.25774	495.058859088015\\
57.8333333333333	0.2594	501.642325611316\\
57.8333333333333	0.26106	508.225792134618\\
57.8333333333333	0.26272	514.809258657919\\
57.8333333333333	0.26438	521.392725181221\\
57.8333333333333	0.26604	527.976191704523\\
57.8333333333333	0.2677	534.559658227824\\
57.8333333333333	0.26936	541.143124751126\\
57.8333333333333	0.27102	547.726591274428\\
57.8333333333333	0.27268	554.310057797729\\
57.8333333333333	0.27434	560.893524321031\\
57.8333333333333	0.276	567.476990844333\\
58.0416666666667	0.193	238.372954017886\\
58.0416666666667	0.19466	245.001967654807\\
58.0416666666667	0.19632	251.630981291729\\
58.0416666666667	0.19798	258.25999492865\\
58.0416666666667	0.19964	264.889008565572\\
58.0416666666667	0.2013	271.518022202493\\
58.0416666666667	0.20296	278.147035839415\\
58.0416666666667	0.20462	284.776049476336\\
58.0416666666667	0.20628	291.405063113258\\
58.0416666666667	0.20794	298.034076750179\\
58.0416666666667	0.2096	304.663090387101\\
58.0416666666667	0.21126	311.292104024022\\
58.0416666666667	0.21292	317.921117660944\\
58.0416666666667	0.21458	324.550131297865\\
58.0416666666667	0.21624	331.179144934787\\
58.0416666666667	0.2179	337.808158571708\\
58.0416666666667	0.21956	344.43717220863\\
58.0416666666667	0.22122	351.066185845551\\
58.0416666666667	0.22288	357.695199482473\\
58.0416666666667	0.22454	364.324213119394\\
58.0416666666667	0.2262	370.953226756315\\
58.0416666666667	0.22786	377.582240393237\\
58.0416666666667	0.22952	384.211254030158\\
58.0416666666667	0.23118	390.84026766708\\
58.0416666666667	0.23284	397.469281304001\\
58.0416666666667	0.2345	404.098294940923\\
58.0416666666667	0.23616	410.727308577844\\
58.0416666666667	0.23782	417.356322214766\\
58.0416666666667	0.23948	423.985335851687\\
58.0416666666667	0.24114	430.614349488609\\
58.0416666666667	0.2428	437.24336312553\\
58.0416666666667	0.24446	443.872376762451\\
58.0416666666667	0.24612	450.501390399373\\
58.0416666666667	0.24778	457.130404036295\\
58.0416666666667	0.24944	463.759417673216\\
58.0416666666667	0.2511	470.388431310138\\
58.0416666666667	0.25276	477.017444947059\\
58.0416666666667	0.25442	483.646458583981\\
58.0416666666667	0.25608	490.275472220902\\
58.0416666666667	0.25774	496.904485857824\\
58.0416666666667	0.2594	503.533499494745\\
58.0416666666667	0.26106	510.162513131666\\
58.0416666666667	0.26272	516.791526768588\\
58.0416666666667	0.26438	523.420540405509\\
58.0416666666667	0.26604	530.049554042431\\
58.0416666666667	0.2677	536.678567679352\\
58.0416666666667	0.26936	543.307581316274\\
58.0416666666667	0.27102	549.936594953195\\
58.0416666666667	0.27268	556.565608590117\\
58.0416666666667	0.27434	563.194622227038\\
58.0416666666667	0.276	569.82363586396\\
58.25	0.193	238.442243356522\\
58.25	0.19466	245.116804107063\\
58.25	0.19632	251.791364857604\\
58.25	0.19798	258.465925608145\\
58.25	0.19964	265.140486358687\\
58.25	0.2013	271.815047109228\\
58.25	0.20296	278.489607859769\\
58.25	0.20462	285.164168610311\\
58.25	0.20628	291.838729360852\\
58.25	0.20794	298.513290111394\\
58.25	0.2096	305.187850861935\\
58.25	0.21126	311.862411612476\\
58.25	0.21292	318.536972363017\\
58.25	0.21458	325.211533113559\\
58.25	0.21624	331.8860938641\\
58.25	0.2179	338.560654614641\\
58.25	0.21956	345.235215365183\\
58.25	0.22122	351.909776115724\\
58.25	0.22288	358.584336866265\\
58.25	0.22454	365.258897616807\\
58.25	0.2262	371.933458367348\\
58.25	0.22786	378.608019117889\\
58.25	0.22952	385.282579868431\\
58.25	0.23118	391.957140618972\\
58.25	0.23284	398.631701369513\\
58.25	0.2345	405.306262120054\\
58.25	0.23616	411.980822870596\\
58.25	0.23782	418.655383621137\\
58.25	0.23948	425.329944371678\\
58.25	0.24114	432.00450512222\\
58.25	0.2428	438.679065872761\\
58.25	0.24446	445.353626623302\\
58.25	0.24612	452.028187373844\\
58.25	0.24778	458.702748124385\\
58.25	0.24944	465.377308874926\\
58.25	0.2511	472.051869625467\\
58.25	0.25276	478.726430376009\\
58.25	0.25442	485.40099112655\\
58.25	0.25608	492.075551877091\\
58.25	0.25774	498.750112627633\\
58.25	0.2594	505.424673378174\\
58.25	0.26106	512.099234128715\\
58.25	0.26272	518.773794879257\\
58.25	0.26438	525.448355629798\\
58.25	0.26604	532.12291638034\\
58.25	0.2677	538.79747713088\\
58.25	0.26936	545.472037881422\\
58.25	0.27102	552.146598631963\\
58.25	0.27268	558.821159382504\\
58.25	0.27434	565.495720133046\\
58.25	0.276	572.170280883587\\
58.4583333333333	0.193	238.511532695157\\
58.4583333333333	0.19466	245.231640559319\\
58.4583333333333	0.19632	251.95174842348\\
58.4583333333333	0.19798	258.671856287641\\
58.4583333333333	0.19964	265.391964151802\\
58.4583333333333	0.2013	272.112072015963\\
58.4583333333333	0.20296	278.832179880124\\
58.4583333333333	0.20462	285.552287744285\\
58.4583333333333	0.20628	292.272395608446\\
58.4583333333333	0.20794	298.992503472607\\
58.4583333333333	0.2096	305.712611336769\\
58.4583333333333	0.21126	312.43271920093\\
58.4583333333333	0.21292	319.152827065091\\
58.4583333333333	0.21458	325.872934929252\\
58.4583333333333	0.21624	332.593042793413\\
58.4583333333333	0.2179	339.313150657574\\
58.4583333333333	0.21956	346.033258521736\\
58.4583333333333	0.22122	352.753366385897\\
58.4583333333333	0.22288	359.473474250058\\
58.4583333333333	0.22454	366.193582114219\\
58.4583333333333	0.2262	372.91368997838\\
58.4583333333333	0.22786	379.633797842542\\
58.4583333333333	0.22952	386.353905706703\\
58.4583333333333	0.23118	393.074013570864\\
58.4583333333333	0.23284	399.794121435025\\
58.4583333333333	0.2345	406.514229299186\\
58.4583333333333	0.23616	413.234337163347\\
58.4583333333333	0.23782	419.954445027508\\
58.4583333333333	0.23948	426.674552891669\\
58.4583333333333	0.24114	433.394660755831\\
58.4583333333333	0.2428	440.114768619992\\
58.4583333333333	0.24446	446.834876484153\\
58.4583333333333	0.24612	453.554984348314\\
58.4583333333333	0.24778	460.275092212475\\
58.4583333333333	0.24944	466.995200076636\\
58.4583333333333	0.2511	473.715307940797\\
58.4583333333333	0.25276	480.435415804958\\
58.4583333333333	0.25442	487.15552366912\\
58.4583333333333	0.25608	493.875631533281\\
58.4583333333333	0.25774	500.595739397442\\
58.4583333333333	0.2594	507.315847261603\\
58.4583333333333	0.26106	514.035955125764\\
58.4583333333333	0.26272	520.756062989925\\
58.4583333333333	0.26438	527.476170854086\\
58.4583333333333	0.26604	534.196278718248\\
58.4583333333333	0.2677	540.916386582409\\
58.4583333333333	0.26936	547.63649444657\\
58.4583333333333	0.27102	554.356602310731\\
58.4583333333333	0.27268	561.076710174892\\
58.4583333333333	0.27434	567.796818039054\\
58.4583333333333	0.276	574.516925903215\\
58.6666666666667	0.193	238.580822033793\\
58.6666666666667	0.19466	245.346477011574\\
58.6666666666667	0.19632	252.112131989355\\
58.6666666666667	0.19798	258.877786967136\\
58.6666666666667	0.19964	265.643441944917\\
58.6666666666667	0.2013	272.409096922698\\
58.6666666666667	0.20296	279.174751900479\\
58.6666666666667	0.20462	285.94040687826\\
58.6666666666667	0.20628	292.706061856041\\
58.6666666666667	0.20794	299.471716833822\\
58.6666666666667	0.2096	306.237371811603\\
58.6666666666667	0.21126	313.003026789384\\
58.6666666666667	0.21292	319.768681767164\\
58.6666666666667	0.21458	326.534336744945\\
58.6666666666667	0.21624	333.299991722727\\
58.6666666666667	0.2179	340.065646700508\\
58.6666666666667	0.21956	346.831301678288\\
58.6666666666667	0.22122	353.59695665607\\
58.6666666666667	0.22288	360.362611633851\\
58.6666666666667	0.22454	367.128266611632\\
58.6666666666667	0.2262	373.893921589412\\
58.6666666666667	0.22786	380.659576567193\\
58.6666666666667	0.22952	387.425231544974\\
58.6666666666667	0.23118	394.190886522756\\
58.6666666666667	0.23284	400.956541500536\\
58.6666666666667	0.2345	407.722196478317\\
58.6666666666667	0.23616	414.487851456098\\
58.6666666666667	0.23782	421.253506433879\\
58.6666666666667	0.23948	428.01916141166\\
58.6666666666667	0.24114	434.784816389441\\
58.6666666666667	0.2428	441.550471367222\\
58.6666666666667	0.24446	448.316126345003\\
58.6666666666667	0.24612	455.081781322784\\
58.6666666666667	0.24778	461.847436300565\\
58.6666666666667	0.24944	468.613091278346\\
58.6666666666667	0.2511	475.378746256127\\
58.6666666666667	0.25276	482.144401233908\\
58.6666666666667	0.25442	488.910056211689\\
58.6666666666667	0.25608	495.67571118947\\
58.6666666666667	0.25774	502.441366167251\\
58.6666666666667	0.2594	509.207021145032\\
58.6666666666667	0.26106	515.972676122813\\
58.6666666666667	0.26272	522.738331100594\\
58.6666666666667	0.26438	529.503986078375\\
58.6666666666667	0.26604	536.269641056156\\
58.6666666666667	0.2677	543.035296033937\\
58.6666666666667	0.26936	549.800951011718\\
58.6666666666667	0.27102	556.566605989499\\
58.6666666666667	0.27268	563.33226096728\\
58.6666666666667	0.27434	570.097915945061\\
58.6666666666667	0.276	576.863570922842\\
58.875	0.193	238.650111372429\\
58.875	0.19466	245.461313463829\\
58.875	0.19632	252.27251555523\\
58.875	0.19798	259.083717646631\\
58.875	0.19964	265.894919738032\\
58.875	0.2013	272.706121829433\\
58.875	0.20296	279.517323920833\\
58.875	0.20462	286.328526012234\\
58.875	0.20628	293.139728103635\\
58.875	0.20794	299.950930195036\\
58.875	0.2096	306.762132286437\\
58.875	0.21126	313.573334377837\\
58.875	0.21292	320.384536469238\\
58.875	0.21458	327.195738560639\\
58.875	0.21624	334.00694065204\\
58.875	0.2179	340.818142743441\\
58.875	0.21956	347.629344834842\\
58.875	0.22122	354.440546926242\\
58.875	0.22288	361.251749017643\\
58.875	0.22454	368.062951109044\\
58.875	0.2262	374.874153200445\\
58.875	0.22786	381.685355291845\\
58.875	0.22952	388.496557383246\\
58.875	0.23118	395.307759474647\\
58.875	0.23284	402.118961566048\\
58.875	0.2345	408.930163657449\\
58.875	0.23616	415.74136574885\\
58.875	0.23782	422.552567840251\\
58.875	0.23948	429.363769931651\\
58.875	0.24114	436.174972023052\\
58.875	0.2428	442.986174114453\\
58.875	0.24446	449.797376205854\\
58.875	0.24612	456.608578297255\\
58.875	0.24778	463.419780388655\\
58.875	0.24944	470.230982480056\\
58.875	0.2511	477.042184571457\\
58.875	0.25276	483.853386662858\\
58.875	0.25442	490.664588754259\\
58.875	0.25608	497.47579084566\\
58.875	0.25774	504.28699293706\\
58.875	0.2594	511.098195028461\\
58.875	0.26106	517.909397119862\\
58.875	0.26272	524.720599211263\\
58.875	0.26438	531.531801302663\\
58.875	0.26604	538.343003394064\\
58.875	0.2677	545.154205485465\\
58.875	0.26936	551.965407576866\\
58.875	0.27102	558.776609668267\\
58.875	0.27268	565.587811759668\\
58.875	0.27434	572.399013851069\\
58.875	0.276	579.210215942469\\
59.0833333333333	0.193	238.719400711064\\
59.0833333333333	0.19466	245.576149916085\\
59.0833333333333	0.19632	252.432899121105\\
59.0833333333333	0.19798	259.289648326126\\
59.0833333333333	0.19964	266.146397531147\\
59.0833333333333	0.2013	273.003146736168\\
59.0833333333333	0.20296	279.859895941188\\
59.0833333333333	0.20462	286.716645146209\\
59.0833333333333	0.20628	293.573394351229\\
59.0833333333333	0.20794	300.43014355625\\
59.0833333333333	0.2096	307.286892761271\\
59.0833333333333	0.21126	314.143641966291\\
59.0833333333333	0.21292	321.000391171312\\
59.0833333333333	0.21458	327.857140376333\\
59.0833333333333	0.21624	334.713889581353\\
59.0833333333333	0.2179	341.570638786374\\
59.0833333333333	0.21956	348.427387991394\\
59.0833333333333	0.22122	355.284137196415\\
59.0833333333333	0.22288	362.140886401436\\
59.0833333333333	0.22454	368.997635606457\\
59.0833333333333	0.2262	375.854384811477\\
59.0833333333333	0.22786	382.711134016498\\
59.0833333333333	0.22952	389.567883221519\\
59.0833333333333	0.23118	396.424632426539\\
59.0833333333333	0.23284	403.28138163156\\
59.0833333333333	0.2345	410.13813083658\\
59.0833333333333	0.23616	416.994880041601\\
59.0833333333333	0.23782	423.851629246622\\
59.0833333333333	0.23948	430.708378451642\\
59.0833333333333	0.24114	437.565127656663\\
59.0833333333333	0.2428	444.421876861684\\
59.0833333333333	0.24446	451.278626066704\\
59.0833333333333	0.24612	458.135375271725\\
59.0833333333333	0.24778	464.992124476745\\
59.0833333333333	0.24944	471.848873681766\\
59.0833333333333	0.2511	478.705622886787\\
59.0833333333333	0.25276	485.562372091807\\
59.0833333333333	0.25442	492.419121296828\\
59.0833333333333	0.25608	499.275870501849\\
59.0833333333333	0.25774	506.13261970687\\
59.0833333333333	0.2594	512.98936891189\\
59.0833333333333	0.26106	519.846118116911\\
59.0833333333333	0.26272	526.702867321931\\
59.0833333333333	0.26438	533.559616526952\\
59.0833333333333	0.26604	540.416365731973\\
59.0833333333333	0.2677	547.273114936994\\
59.0833333333333	0.26936	554.129864142014\\
59.0833333333333	0.27102	560.986613347035\\
59.0833333333333	0.27268	567.843362552055\\
59.0833333333333	0.27434	574.700111757076\\
59.0833333333333	0.276	581.556860962097\\
59.2916666666667	0.193	238.7886900497\\
59.2916666666667	0.19466	245.69098636834\\
59.2916666666667	0.19632	252.593282686981\\
59.2916666666667	0.19798	259.495579005621\\
59.2916666666667	0.19964	266.397875324262\\
59.2916666666667	0.2013	273.300171642902\\
59.2916666666667	0.20296	280.202467961542\\
59.2916666666667	0.20462	287.104764280183\\
59.2916666666667	0.20628	294.007060598824\\
59.2916666666667	0.20794	300.909356917464\\
59.2916666666667	0.2096	307.811653236104\\
59.2916666666667	0.21126	314.713949554745\\
59.2916666666667	0.21292	321.616245873385\\
59.2916666666667	0.21458	328.518542192026\\
59.2916666666667	0.21624	335.420838510666\\
59.2916666666667	0.2179	342.323134829307\\
59.2916666666667	0.21956	349.225431147947\\
59.2916666666667	0.22122	356.127727466588\\
59.2916666666667	0.22288	363.030023785228\\
59.2916666666667	0.22454	369.932320103869\\
59.2916666666667	0.2262	376.834616422509\\
59.2916666666667	0.22786	383.736912741149\\
59.2916666666667	0.22952	390.63920905979\\
59.2916666666667	0.23118	397.54150537843\\
59.2916666666667	0.23284	404.443801697071\\
59.2916666666667	0.2345	411.346098015712\\
59.2916666666667	0.23616	418.248394334352\\
59.2916666666667	0.23782	425.150690652993\\
59.2916666666667	0.23948	432.052986971633\\
59.2916666666667	0.24114	438.955283290274\\
59.2916666666667	0.2428	445.857579608914\\
59.2916666666667	0.24446	452.759875927555\\
59.2916666666667	0.24612	459.662172246195\\
59.2916666666667	0.24778	466.564468564836\\
59.2916666666667	0.24944	473.466764883476\\
59.2916666666667	0.2511	480.369061202116\\
59.2916666666667	0.25276	487.271357520757\\
59.2916666666667	0.25442	494.173653839398\\
59.2916666666667	0.25608	501.075950158038\\
59.2916666666667	0.25774	507.978246476678\\
59.2916666666667	0.2594	514.880542795319\\
59.2916666666667	0.26106	521.782839113959\\
59.2916666666667	0.26272	528.6851354326\\
59.2916666666667	0.26438	535.58743175124\\
59.2916666666667	0.26604	542.489728069881\\
59.2916666666667	0.2677	549.392024388521\\
59.2916666666667	0.26936	556.294320707162\\
59.2916666666667	0.27102	563.196617025802\\
59.2916666666667	0.27268	570.098913344443\\
59.2916666666667	0.27434	577.001209663083\\
59.2916666666667	0.276	583.903505981723\\
59.5	0.193	238.857979388335\\
59.5	0.19466	245.805822820595\\
59.5	0.19632	252.753666252856\\
59.5	0.19798	259.701509685116\\
59.5	0.19964	266.649353117376\\
59.5	0.2013	273.597196549637\\
59.5	0.20296	280.545039981897\\
59.5	0.20462	287.492883414157\\
59.5	0.20628	294.440726846418\\
59.5	0.20794	301.388570278678\\
59.5	0.2096	308.336413710939\\
59.5	0.21126	315.284257143199\\
59.5	0.21292	322.232100575459\\
59.5	0.21458	329.179944007719\\
59.5	0.21624	336.12778743998\\
59.5	0.2179	343.07563087224\\
59.5	0.21956	350.0234743045\\
59.5	0.22122	356.971317736761\\
59.5	0.22288	363.919161169021\\
59.5	0.22454	370.867004601281\\
59.5	0.2262	377.814848033541\\
59.5	0.22786	384.762691465802\\
59.5	0.22952	391.710534898062\\
59.5	0.23118	398.658378330323\\
59.5	0.23284	405.606221762583\\
59.5	0.2345	412.554065194843\\
59.5	0.23616	419.501908627103\\
59.5	0.23782	426.449752059364\\
59.5	0.23948	433.397595491624\\
59.5	0.24114	440.345438923884\\
59.5	0.2428	447.293282356145\\
59.5	0.24446	454.241125788405\\
59.5	0.24612	461.188969220665\\
59.5	0.24778	468.136812652926\\
59.5	0.24944	475.084656085186\\
59.5	0.2511	482.032499517446\\
59.5	0.25276	488.980342949706\\
59.5	0.25442	495.928186381967\\
59.5	0.25608	502.876029814227\\
59.5	0.25774	509.823873246488\\
59.5	0.2594	516.771716678748\\
59.5	0.26106	523.719560111008\\
59.5	0.26272	530.667403543269\\
59.5	0.26438	537.615246975529\\
59.5	0.26604	544.563090407789\\
59.5	0.2677	551.51093384005\\
59.5	0.26936	558.45877727231\\
59.5	0.27102	565.40662070457\\
59.5	0.27268	572.354464136831\\
59.5	0.27434	579.302307569091\\
59.5	0.276	586.250151001351\\
59.7083333333333	0.193	238.927268726971\\
59.7083333333333	0.19466	245.920659272851\\
59.7083333333333	0.19632	252.914049818731\\
59.7083333333333	0.19798	259.907440364611\\
59.7083333333333	0.19964	266.900830910492\\
59.7083333333333	0.2013	273.894221456372\\
59.7083333333333	0.20296	280.887612002252\\
59.7083333333333	0.20462	287.881002548132\\
59.7083333333333	0.20628	294.874393094012\\
59.7083333333333	0.20794	301.867783639892\\
59.7083333333333	0.2096	308.861174185772\\
59.7083333333333	0.21126	315.854564731653\\
59.7083333333333	0.21292	322.847955277533\\
59.7083333333333	0.21458	329.841345823413\\
59.7083333333333	0.21624	336.834736369293\\
59.7083333333333	0.2179	343.828126915173\\
59.7083333333333	0.21956	350.821517461053\\
59.7083333333333	0.22122	357.814908006933\\
59.7083333333333	0.22288	364.808298552814\\
59.7083333333333	0.22454	371.801689098694\\
59.7083333333333	0.2262	378.795079644574\\
59.7083333333333	0.22786	385.788470190454\\
59.7083333333333	0.22952	392.781860736334\\
59.7083333333333	0.23118	399.775251282214\\
59.7083333333333	0.23284	406.768641828095\\
59.7083333333333	0.2345	413.762032373975\\
59.7083333333333	0.23616	420.755422919855\\
59.7083333333333	0.23782	427.748813465735\\
59.7083333333333	0.23948	434.742204011615\\
59.7083333333333	0.24114	441.735594557495\\
59.7083333333333	0.2428	448.728985103375\\
59.7083333333333	0.24446	455.722375649256\\
59.7083333333333	0.24612	462.715766195136\\
59.7083333333333	0.24778	469.709156741016\\
59.7083333333333	0.24944	476.702547286896\\
59.7083333333333	0.2511	483.695937832776\\
59.7083333333333	0.25276	490.689328378656\\
59.7083333333333	0.25442	497.682718924537\\
59.7083333333333	0.25608	504.676109470417\\
59.7083333333333	0.25774	511.669500016297\\
59.7083333333333	0.2594	518.662890562177\\
59.7083333333333	0.26106	525.656281108057\\
59.7083333333333	0.26272	532.649671653937\\
59.7083333333333	0.26438	539.643062199817\\
59.7083333333333	0.26604	546.636452745698\\
59.7083333333333	0.2677	553.629843291578\\
59.7083333333333	0.26936	560.623233837458\\
59.7083333333333	0.27102	567.616624383338\\
59.7083333333333	0.27268	574.610014929218\\
59.7083333333333	0.27434	581.603405475098\\
59.7083333333333	0.276	588.596796020978\\
59.9166666666667	0.193	238.996558065606\\
59.9166666666667	0.19466	246.035495725107\\
59.9166666666667	0.19632	253.074433384606\\
59.9166666666667	0.19798	260.113371044106\\
59.9166666666667	0.19964	267.152308703606\\
59.9166666666667	0.2013	274.191246363106\\
59.9166666666667	0.20296	281.230184022606\\
59.9166666666667	0.20462	288.269121682107\\
59.9166666666667	0.20628	295.308059341606\\
59.9166666666667	0.20794	302.346997001107\\
59.9166666666667	0.2096	309.385934660606\\
59.9166666666667	0.21126	316.424872320106\\
59.9166666666667	0.21292	323.463809979606\\
59.9166666666667	0.21458	330.502747639106\\
59.9166666666667	0.21624	337.541685298606\\
59.9166666666667	0.2179	344.580622958106\\
59.9166666666667	0.21956	351.619560617606\\
59.9166666666667	0.22122	358.658498277106\\
59.9166666666667	0.22288	365.697435936606\\
59.9166666666667	0.22454	372.736373596106\\
59.9166666666667	0.2262	379.775311255606\\
59.9166666666667	0.22786	386.814248915106\\
59.9166666666667	0.22952	393.853186574606\\
59.9166666666667	0.23118	400.892124234106\\
59.9166666666667	0.23284	407.931061893606\\
59.9166666666667	0.2345	414.969999553106\\
59.9166666666667	0.23616	422.008937212606\\
59.9166666666667	0.23782	429.047874872106\\
59.9166666666667	0.23948	436.086812531606\\
59.9166666666667	0.24114	443.125750191106\\
59.9166666666667	0.2428	450.164687850606\\
59.9166666666667	0.24446	457.203625510106\\
59.9166666666667	0.24612	464.242563169606\\
59.9166666666667	0.24778	471.281500829106\\
59.9166666666667	0.24944	478.320438488606\\
59.9166666666667	0.2511	485.359376148106\\
59.9166666666667	0.25276	492.398313807606\\
59.9166666666667	0.25442	499.437251467106\\
59.9166666666667	0.25608	506.476189126606\\
59.9166666666667	0.25774	513.515126786106\\
59.9166666666667	0.2594	520.554064445606\\
59.9166666666667	0.26106	527.593002105105\\
59.9166666666667	0.26272	534.631939764606\\
59.9166666666667	0.26438	541.670877424106\\
59.9166666666667	0.26604	548.709815083606\\
59.9166666666667	0.2677	555.748752743106\\
59.9166666666667	0.26936	562.787690402606\\
59.9166666666667	0.27102	569.826628062106\\
59.9166666666667	0.27268	576.865565721605\\
59.9166666666667	0.27434	583.904503381106\\
59.9166666666667	0.276	590.943441040606\\
60.125	0.193	239.065847404242\\
60.125	0.19466	246.150332177362\\
60.125	0.19632	253.234816950482\\
60.125	0.19798	260.319301723602\\
60.125	0.19964	267.403786496722\\
60.125	0.2013	274.488271269841\\
60.125	0.20296	281.572756042961\\
60.125	0.20462	288.657240816081\\
60.125	0.20628	295.741725589201\\
60.125	0.20794	302.82621036232\\
60.125	0.2096	309.91069513544\\
60.125	0.21126	316.99517990856\\
60.125	0.21292	324.07966468168\\
60.125	0.21458	331.1641494548\\
60.125	0.21624	338.24863422792\\
60.125	0.2179	345.333119001039\\
60.125	0.21956	352.417603774159\\
60.125	0.22122	359.502088547279\\
60.125	0.22288	366.586573320399\\
60.125	0.22454	373.671058093519\\
60.125	0.2262	380.755542866638\\
60.125	0.22786	387.840027639758\\
60.125	0.22952	394.924512412878\\
60.125	0.23118	402.008997185998\\
60.125	0.23284	409.093481959118\\
60.125	0.2345	416.177966732238\\
60.125	0.23616	423.262451505357\\
60.125	0.23782	430.346936278477\\
60.125	0.23948	437.431421051597\\
60.125	0.24114	444.515905824717\\
60.125	0.2428	451.600390597837\\
60.125	0.24446	458.684875370956\\
60.125	0.24612	465.769360144076\\
60.125	0.24778	472.853844917196\\
60.125	0.24944	479.938329690316\\
60.125	0.2511	487.022814463436\\
60.125	0.25276	494.107299236555\\
60.125	0.25442	501.191784009676\\
60.125	0.25608	508.276268782795\\
60.125	0.25774	515.360753555915\\
60.125	0.2594	522.445238329035\\
60.125	0.26106	529.529723102155\\
60.125	0.26272	536.614207875274\\
60.125	0.26438	543.698692648394\\
60.125	0.26604	550.783177421514\\
60.125	0.2677	557.867662194634\\
60.125	0.26936	564.952146967754\\
60.125	0.27102	572.036631740874\\
60.125	0.27268	579.121116513993\\
60.125	0.27434	586.205601287113\\
60.125	0.276	593.290086060233\\
60.3333333333333	0.193	239.135136742878\\
60.3333333333333	0.19466	246.265168629617\\
60.3333333333333	0.19632	253.395200516357\\
60.3333333333333	0.19798	260.525232403097\\
60.3333333333333	0.19964	267.655264289836\\
60.3333333333333	0.2013	274.785296176576\\
60.3333333333333	0.20296	281.915328063316\\
60.3333333333333	0.20462	289.045359950055\\
60.3333333333333	0.20628	296.175391836795\\
60.3333333333333	0.20794	303.305423723535\\
60.3333333333333	0.2096	310.435455610274\\
60.3333333333333	0.21126	317.565487497014\\
60.3333333333333	0.21292	324.695519383754\\
60.3333333333333	0.21458	331.825551270493\\
60.3333333333333	0.21624	338.955583157233\\
60.3333333333333	0.2179	346.085615043973\\
60.3333333333333	0.21956	353.215646930712\\
60.3333333333333	0.22122	360.345678817452\\
60.3333333333333	0.22288	367.475710704192\\
60.3333333333333	0.22454	374.605742590931\\
60.3333333333333	0.2262	381.735774477671\\
60.3333333333333	0.22786	388.86580636441\\
60.3333333333333	0.22952	395.99583825115\\
60.3333333333333	0.23118	403.12587013789\\
60.3333333333333	0.23284	410.255902024629\\
60.3333333333333	0.2345	417.385933911369\\
60.3333333333333	0.23616	424.515965798109\\
60.3333333333333	0.23782	431.645997684848\\
60.3333333333333	0.23948	438.776029571588\\
60.3333333333333	0.24114	445.906061458328\\
60.3333333333333	0.2428	453.036093345067\\
60.3333333333333	0.24446	460.166125231807\\
60.3333333333333	0.24612	467.296157118547\\
60.3333333333333	0.24778	474.426189005286\\
60.3333333333333	0.24944	481.556220892026\\
60.3333333333333	0.2511	488.686252778765\\
60.3333333333333	0.25276	495.816284665505\\
60.3333333333333	0.25442	502.946316552245\\
60.3333333333333	0.25608	510.076348438985\\
60.3333333333333	0.25774	517.206380325724\\
60.3333333333333	0.2594	524.336412212464\\
60.3333333333333	0.26106	531.466444099203\\
60.3333333333333	0.26272	538.596475985943\\
60.3333333333333	0.26438	545.726507872683\\
60.3333333333333	0.26604	552.856539759423\\
60.3333333333333	0.2677	559.986571646162\\
60.3333333333333	0.26936	567.116603532902\\
60.3333333333333	0.27102	574.246635419642\\
60.3333333333333	0.27268	581.376667306381\\
60.3333333333333	0.27434	588.506699193121\\
60.3333333333333	0.276	595.636731079861\\
60.5416666666667	0.193	239.204426081513\\
60.5416666666667	0.19466	246.380005081873\\
60.5416666666667	0.19632	253.555584082232\\
60.5416666666667	0.19798	260.731163082592\\
60.5416666666667	0.19964	267.906742082951\\
60.5416666666667	0.2013	275.082321083311\\
60.5416666666667	0.20296	282.25790008367\\
60.5416666666667	0.20462	289.43347908403\\
60.5416666666667	0.20628	296.609058084389\\
60.5416666666667	0.20794	303.784637084749\\
60.5416666666667	0.2096	310.960216085108\\
60.5416666666667	0.21126	318.135795085468\\
60.5416666666667	0.21292	325.311374085827\\
60.5416666666667	0.21458	332.486953086187\\
60.5416666666667	0.21624	339.662532086546\\
60.5416666666667	0.2179	346.838111086906\\
60.5416666666667	0.21956	354.013690087265\\
60.5416666666667	0.22122	361.189269087625\\
60.5416666666667	0.22288	368.364848087984\\
60.5416666666667	0.22454	375.540427088343\\
60.5416666666667	0.2262	382.716006088703\\
60.5416666666667	0.22786	389.891585089062\\
60.5416666666667	0.22952	397.067164089422\\
60.5416666666667	0.23118	404.242743089781\\
60.5416666666667	0.23284	411.418322090141\\
60.5416666666667	0.2345	418.593901090501\\
60.5416666666667	0.23616	425.76948009086\\
60.5416666666667	0.23782	432.945059091219\\
60.5416666666667	0.23948	440.120638091579\\
60.5416666666667	0.24114	447.296217091939\\
60.5416666666667	0.2428	454.471796092298\\
60.5416666666667	0.24446	461.647375092658\\
60.5416666666667	0.24612	468.822954093017\\
60.5416666666667	0.24778	475.998533093376\\
60.5416666666667	0.24944	483.174112093736\\
60.5416666666667	0.2511	490.349691094095\\
60.5416666666667	0.25276	497.525270094455\\
60.5416666666667	0.25442	504.700849094814\\
60.5416666666667	0.25608	511.876428095174\\
60.5416666666667	0.25774	519.052007095533\\
60.5416666666667	0.2594	526.227586095893\\
60.5416666666667	0.26106	533.403165096252\\
60.5416666666667	0.26272	540.578744096612\\
60.5416666666667	0.26438	547.754323096971\\
60.5416666666667	0.26604	554.929902097331\\
60.5416666666667	0.2677	562.10548109769\\
60.5416666666667	0.26936	569.28106009805\\
60.5416666666667	0.27102	576.456639098409\\
60.5416666666667	0.27268	583.632218098769\\
60.5416666666667	0.27434	590.807797099128\\
60.5416666666667	0.276	597.983376099487\\
60.75	0.193	239.273715420149\\
60.75	0.19466	246.494841534128\\
60.75	0.19632	253.715967648108\\
60.75	0.19798	260.937093762087\\
60.75	0.19964	268.158219876066\\
60.75	0.2013	275.379345990046\\
60.75	0.20296	282.600472104025\\
60.75	0.20462	289.821598218004\\
60.75	0.20628	297.042724331984\\
60.75	0.20794	304.263850445963\\
60.75	0.2096	311.484976559942\\
60.75	0.21126	318.706102673921\\
60.75	0.21292	325.927228787901\\
60.75	0.21458	333.14835490188\\
60.75	0.21624	340.369481015859\\
60.75	0.2179	347.590607129839\\
60.75	0.21956	354.811733243818\\
60.75	0.22122	362.032859357798\\
60.75	0.22288	369.253985471777\\
60.75	0.22454	376.475111585756\\
60.75	0.2262	383.696237699735\\
60.75	0.22786	390.917363813715\\
60.75	0.22952	398.138489927694\\
60.75	0.23118	405.359616041673\\
60.75	0.23284	412.580742155653\\
60.75	0.2345	419.801868269632\\
60.75	0.23616	427.022994383611\\
60.75	0.23782	434.244120497591\\
60.75	0.23948	441.46524661157\\
60.75	0.24114	448.686372725549\\
60.75	0.2428	455.907498839529\\
60.75	0.24446	463.128624953508\\
60.75	0.24612	470.349751067487\\
60.75	0.24778	477.570877181467\\
60.75	0.24944	484.792003295446\\
60.75	0.2511	492.013129409425\\
60.75	0.25276	499.234255523404\\
60.75	0.25442	506.455381637384\\
60.75	0.25608	513.676507751363\\
60.75	0.25774	520.897633865343\\
60.75	0.2594	528.118759979322\\
60.75	0.26106	535.339886093301\\
60.75	0.26272	542.561012207281\\
60.75	0.26438	549.78213832126\\
60.75	0.26604	557.003264435239\\
60.75	0.2677	564.224390549219\\
60.75	0.26936	571.445516663198\\
60.75	0.27102	578.666642777177\\
60.75	0.27268	585.887768891156\\
60.75	0.27434	593.108895005136\\
60.75	0.276	600.330021119115\\
60.9583333333333	0.193	239.343004758785\\
60.9583333333333	0.19466	246.609677986384\\
60.9583333333333	0.19632	253.876351213983\\
60.9583333333333	0.19798	261.143024441582\\
60.9583333333333	0.19964	268.409697669181\\
60.9583333333333	0.2013	275.676370896781\\
60.9583333333333	0.20296	282.94304412438\\
60.9583333333333	0.20462	290.209717351979\\
60.9583333333333	0.20628	297.476390579578\\
60.9583333333333	0.20794	304.743063807177\\
60.9583333333333	0.2096	312.009737034776\\
60.9583333333333	0.21126	319.276410262376\\
60.9583333333333	0.21292	326.543083489975\\
60.9583333333333	0.21458	333.809756717574\\
60.9583333333333	0.21624	341.076429945173\\
60.9583333333333	0.2179	348.343103172772\\
60.9583333333333	0.21956	355.609776400371\\
60.9583333333333	0.22122	362.87644962797\\
60.9583333333333	0.22288	370.14312285557\\
60.9583333333333	0.22454	377.409796083169\\
60.9583333333333	0.2262	384.676469310768\\
60.9583333333333	0.22786	391.943142538367\\
60.9583333333333	0.22952	399.209815765966\\
60.9583333333333	0.23118	406.476488993565\\
60.9583333333333	0.23284	413.743162221165\\
60.9583333333333	0.2345	421.009835448764\\
60.9583333333333	0.23616	428.276508676363\\
60.9583333333333	0.23782	435.543181903962\\
60.9583333333333	0.23948	442.809855131561\\
60.9583333333333	0.24114	450.07652835916\\
60.9583333333333	0.2428	457.343201586759\\
60.9583333333333	0.24446	464.609874814359\\
60.9583333333333	0.24612	471.876548041958\\
60.9583333333333	0.24778	479.143221269557\\
60.9583333333333	0.24944	486.409894497156\\
60.9583333333333	0.2511	493.676567724755\\
60.9583333333333	0.25276	500.943240952354\\
60.9583333333333	0.25442	508.209914179954\\
60.9583333333333	0.25608	515.476587407553\\
60.9583333333333	0.25774	522.743260635152\\
60.9583333333333	0.2594	530.009933862751\\
60.9583333333333	0.26106	537.27660709035\\
60.9583333333333	0.26272	544.543280317949\\
60.9583333333333	0.26438	551.809953545548\\
60.9583333333333	0.26604	559.076626773148\\
60.9583333333333	0.2677	566.343300000747\\
60.9583333333333	0.26936	573.609973228346\\
60.9583333333333	0.27102	580.876646455945\\
60.9583333333333	0.27268	588.143319683544\\
60.9583333333333	0.27434	595.409992911143\\
60.9583333333333	0.276	602.676666138742\\
61.1666666666667	0.193	239.41229409742\\
61.1666666666667	0.19466	246.724514438639\\
61.1666666666667	0.19632	254.036734779858\\
61.1666666666667	0.19798	261.348955121077\\
61.1666666666667	0.19964	268.661175462296\\
61.1666666666667	0.2013	275.973395803515\\
61.1666666666667	0.20296	283.285616144734\\
61.1666666666667	0.20462	290.597836485953\\
61.1666666666667	0.20628	297.910056827172\\
61.1666666666667	0.20794	305.222277168391\\
61.1666666666667	0.2096	312.53449750961\\
61.1666666666667	0.21126	319.846717850829\\
61.1666666666667	0.21292	327.158938192048\\
61.1666666666667	0.21458	334.471158533267\\
61.1666666666667	0.21624	341.783378874486\\
61.1666666666667	0.2179	349.095599215705\\
61.1666666666667	0.21956	356.407819556924\\
61.1666666666667	0.22122	363.720039898143\\
61.1666666666667	0.22288	371.032260239362\\
61.1666666666667	0.22454	378.344480580581\\
61.1666666666667	0.2262	385.6567009218\\
61.1666666666667	0.22786	392.968921263019\\
61.1666666666667	0.22952	400.281141604238\\
61.1666666666667	0.23118	407.593361945457\\
61.1666666666667	0.23284	414.905582286676\\
61.1666666666667	0.2345	422.217802627895\\
61.1666666666667	0.23616	429.530022969114\\
61.1666666666667	0.23782	436.842243310333\\
61.1666666666667	0.23948	444.154463651552\\
61.1666666666667	0.24114	451.466683992771\\
61.1666666666667	0.2428	458.77890433399\\
61.1666666666667	0.24446	466.091124675209\\
61.1666666666667	0.24612	473.403345016428\\
61.1666666666667	0.24778	480.715565357647\\
61.1666666666667	0.24944	488.027785698866\\
61.1666666666667	0.2511	495.340006040085\\
61.1666666666667	0.25276	502.652226381304\\
61.1666666666667	0.25442	509.964446722523\\
61.1666666666667	0.25608	517.276667063742\\
61.1666666666667	0.25774	524.588887404961\\
61.1666666666667	0.2594	531.90110774618\\
61.1666666666667	0.26106	539.213328087399\\
61.1666666666667	0.26272	546.525548428618\\
61.1666666666667	0.26438	553.837768769837\\
61.1666666666667	0.26604	561.149989111056\\
61.1666666666667	0.2677	568.462209452275\\
61.1666666666667	0.26936	575.774429793494\\
61.1666666666667	0.27102	583.086650134713\\
61.1666666666667	0.27268	590.398870475932\\
61.1666666666667	0.27434	597.711090817151\\
61.1666666666667	0.276	605.02331115837\\
61.375	0.193	239.481583436056\\
61.375	0.19466	246.839350890895\\
61.375	0.19632	254.197118345734\\
61.375	0.19798	261.554885800572\\
61.375	0.19964	268.912653255411\\
61.375	0.2013	276.27042071025\\
61.375	0.20296	283.628188165089\\
61.375	0.20462	290.985955619928\\
61.375	0.20628	298.343723074767\\
61.375	0.20794	305.701490529605\\
61.375	0.2096	313.059257984444\\
61.375	0.21126	320.417025439283\\
61.375	0.21292	327.774792894122\\
61.375	0.21458	335.132560348961\\
61.375	0.21624	342.4903278038\\
61.375	0.2179	349.848095258638\\
61.375	0.21956	357.205862713477\\
61.375	0.22122	364.563630168316\\
61.375	0.22288	371.921397623155\\
61.375	0.22454	379.279165077994\\
61.375	0.2262	386.636932532832\\
61.375	0.22786	393.994699987671\\
61.375	0.22952	401.35246744251\\
61.375	0.23118	408.710234897349\\
61.375	0.23284	416.068002352188\\
61.375	0.2345	423.425769807026\\
61.375	0.23616	430.783537261865\\
61.375	0.23782	438.141304716704\\
61.375	0.23948	445.499072171543\\
61.375	0.24114	452.856839626382\\
61.375	0.2428	460.214607081221\\
61.375	0.24446	467.572374536059\\
61.375	0.24612	474.930141990898\\
61.375	0.24778	482.287909445737\\
61.375	0.24944	489.645676900576\\
61.375	0.2511	497.003444355415\\
61.375	0.25276	504.361211810253\\
61.375	0.25442	511.718979265093\\
61.375	0.25608	519.076746719931\\
61.375	0.25774	526.43451417477\\
61.375	0.2594	533.792281629609\\
61.375	0.26106	541.150049084448\\
61.375	0.26272	548.507816539286\\
61.375	0.26438	555.865583994125\\
61.375	0.26604	563.223351448964\\
61.375	0.2677	570.581118903803\\
61.375	0.26936	577.938886358642\\
61.375	0.27102	585.296653813481\\
61.375	0.27268	592.65442126832\\
61.375	0.27434	600.012188723158\\
61.375	0.276	607.369956177997\\
61.5833333333333	0.193	239.550872774692\\
61.5833333333333	0.19466	246.95418734315\\
61.5833333333333	0.19632	254.357501911609\\
61.5833333333333	0.19798	261.760816480067\\
61.5833333333333	0.19964	269.164131048526\\
61.5833333333333	0.2013	276.567445616985\\
61.5833333333333	0.20296	283.970760185443\\
61.5833333333333	0.20462	291.374074753902\\
61.5833333333333	0.20628	298.777389322361\\
61.5833333333333	0.20794	306.18070389082\\
61.5833333333333	0.2096	313.584018459278\\
61.5833333333333	0.21126	320.987333027737\\
61.5833333333333	0.21292	328.390647596196\\
61.5833333333333	0.21458	335.793962164654\\
61.5833333333333	0.21624	343.197276733113\\
61.5833333333333	0.2179	350.600591301572\\
61.5833333333333	0.21956	358.00390587003\\
61.5833333333333	0.22122	365.407220438489\\
61.5833333333333	0.22288	372.810535006947\\
61.5833333333333	0.22454	380.213849575406\\
61.5833333333333	0.2262	387.617164143865\\
61.5833333333333	0.22786	395.020478712323\\
61.5833333333333	0.22952	402.423793280782\\
61.5833333333333	0.23118	409.827107849241\\
61.5833333333333	0.23284	417.230422417699\\
61.5833333333333	0.2345	424.633736986158\\
61.5833333333333	0.23616	432.037051554617\\
61.5833333333333	0.23782	439.440366123075\\
61.5833333333333	0.23948	446.843680691534\\
61.5833333333333	0.24114	454.246995259993\\
61.5833333333333	0.2428	461.650309828451\\
61.5833333333333	0.24446	469.05362439691\\
61.5833333333333	0.24612	476.456938965369\\
61.5833333333333	0.24778	483.860253533828\\
61.5833333333333	0.24944	491.263568102286\\
61.5833333333333	0.2511	498.666882670744\\
61.5833333333333	0.25276	506.070197239203\\
61.5833333333333	0.25442	513.473511807662\\
61.5833333333333	0.25608	520.876826376121\\
61.5833333333333	0.25774	528.280140944579\\
61.5833333333333	0.2594	535.683455513038\\
61.5833333333333	0.26106	543.086770081497\\
61.5833333333333	0.26272	550.490084649955\\
61.5833333333333	0.26438	557.893399218414\\
61.5833333333333	0.26604	565.296713786873\\
61.5833333333333	0.2677	572.700028355331\\
61.5833333333333	0.26936	580.10334292379\\
61.5833333333333	0.27102	587.506657492249\\
61.5833333333333	0.27268	594.909972060707\\
61.5833333333333	0.27434	602.313286629166\\
61.5833333333333	0.276	609.716601197625\\
61.7916666666667	0.193	239.620162113327\\
61.7916666666667	0.19466	247.069023795406\\
61.7916666666667	0.19632	254.517885477484\\
61.7916666666667	0.19798	261.966747159563\\
61.7916666666667	0.19964	269.415608841641\\
61.7916666666667	0.2013	276.86447052372\\
61.7916666666667	0.20296	284.313332205798\\
61.7916666666667	0.20462	291.762193887877\\
61.7916666666667	0.20628	299.211055569955\\
61.7916666666667	0.20794	306.659917252034\\
61.7916666666667	0.2096	314.108778934112\\
61.7916666666667	0.21126	321.557640616191\\
61.7916666666667	0.21292	329.006502298269\\
61.7916666666667	0.21458	336.455363980347\\
61.7916666666667	0.21624	343.904225662426\\
61.7916666666667	0.2179	351.353087344505\\
61.7916666666667	0.21956	358.801949026583\\
61.7916666666667	0.22122	366.250810708661\\
61.7916666666667	0.22288	373.69967239074\\
61.7916666666667	0.22454	381.148534072818\\
61.7916666666667	0.2262	388.597395754897\\
61.7916666666667	0.22786	396.046257436975\\
61.7916666666667	0.22952	403.495119119054\\
61.7916666666667	0.23118	410.943980801132\\
61.7916666666667	0.23284	418.392842483211\\
61.7916666666667	0.2345	425.841704165289\\
61.7916666666667	0.23616	433.290565847368\\
61.7916666666667	0.23782	440.739427529446\\
61.7916666666667	0.23948	448.188289211525\\
61.7916666666667	0.24114	455.637150893604\\
61.7916666666667	0.2428	463.086012575682\\
61.7916666666667	0.24446	470.534874257761\\
61.7916666666667	0.24612	477.983735939839\\
61.7916666666667	0.24778	485.432597621917\\
61.7916666666667	0.24944	492.881459303996\\
61.7916666666667	0.2511	500.330320986074\\
61.7916666666667	0.25276	507.779182668153\\
61.7916666666667	0.25442	515.228044350231\\
61.7916666666667	0.25608	522.67690603231\\
61.7916666666667	0.25774	530.125767714388\\
61.7916666666667	0.2594	537.574629396467\\
61.7916666666667	0.26106	545.023491078545\\
61.7916666666667	0.26272	552.472352760624\\
61.7916666666667	0.26438	559.921214442702\\
61.7916666666667	0.26604	567.370076124781\\
61.7916666666667	0.2677	574.818937806859\\
61.7916666666667	0.26936	582.267799488938\\
61.7916666666667	0.27102	589.716661171017\\
61.7916666666667	0.27268	597.165522853095\\
61.7916666666667	0.27434	604.614384535173\\
61.7916666666667	0.276	612.063246217252\\
62	0.193	239.689451451963\\
62	0.19466	247.183860247661\\
62	0.19632	254.678269043359\\
62	0.19798	262.172677839058\\
62	0.19964	269.667086634756\\
62	0.2013	277.161495430454\\
62	0.20296	284.655904226153\\
62	0.20462	292.150313021851\\
62	0.20628	299.644721817549\\
62	0.20794	307.139130613248\\
62	0.2096	314.633539408946\\
62	0.21126	322.127948204644\\
62	0.21292	329.622357000343\\
62	0.21458	337.116765796041\\
62	0.21624	344.611174591739\\
62	0.2179	352.105583387438\\
62	0.21956	359.599992183136\\
62	0.22122	367.094400978835\\
62	0.22288	374.588809774533\\
62	0.22454	382.083218570231\\
62	0.2262	389.577627365929\\
62	0.22786	397.072036161628\\
62	0.22952	404.566444957326\\
62	0.23118	412.060853753025\\
62	0.23284	419.555262548723\\
62	0.2345	427.049671344421\\
62	0.23616	434.544080140119\\
62	0.23782	442.038488935818\\
62	0.23948	449.532897731516\\
62	0.24114	457.027306527214\\
62	0.2428	464.521715322913\\
62	0.24446	472.016124118611\\
62	0.24612	479.510532914309\\
62	0.24778	487.004941710007\\
62	0.24944	494.499350505706\\
62	0.2511	501.993759301404\\
62	0.25276	509.488168097102\\
62	0.25442	516.982576892801\\
62	0.25608	524.476985688499\\
62	0.25774	531.971394484198\\
62	0.2594	539.465803279896\\
62	0.26106	546.960212075594\\
62	0.26272	554.454620871292\\
62	0.26438	561.949029666991\\
62	0.26604	569.443438462689\\
62	0.2677	576.937847258388\\
62	0.26936	584.432256054086\\
62	0.27102	591.926664849784\\
62	0.27268	599.421073645483\\
62	0.27434	606.915482441181\\
62	0.276	614.409891236879\\
62.2083333333333	0.193	239.758740790598\\
62.2083333333333	0.19466	247.298696699917\\
62.2083333333333	0.19632	254.838652609235\\
62.2083333333333	0.19798	262.378608518553\\
62.2083333333333	0.19964	269.918564427871\\
62.2083333333333	0.2013	277.458520337189\\
62.2083333333333	0.20296	284.998476246507\\
62.2083333333333	0.20462	292.538432155826\\
62.2083333333333	0.20628	300.078388065144\\
62.2083333333333	0.20794	307.618343974462\\
62.2083333333333	0.2096	315.15829988378\\
62.2083333333333	0.21126	322.698255793098\\
62.2083333333333	0.21292	330.238211702416\\
62.2083333333333	0.21458	337.778167611734\\
62.2083333333333	0.21624	345.318123521053\\
62.2083333333333	0.2179	352.858079430371\\
62.2083333333333	0.21956	360.398035339689\\
62.2083333333333	0.22122	367.937991249007\\
62.2083333333333	0.22288	375.477947158325\\
62.2083333333333	0.22454	383.017903067644\\
62.2083333333333	0.2262	390.557858976962\\
62.2083333333333	0.22786	398.09781488628\\
62.2083333333333	0.22952	405.637770795598\\
62.2083333333333	0.23118	413.177726704916\\
62.2083333333333	0.23284	420.717682614234\\
62.2083333333333	0.2345	428.257638523552\\
62.2083333333333	0.23616	435.797594432871\\
62.2083333333333	0.23782	443.337550342189\\
62.2083333333333	0.23948	450.877506251507\\
62.2083333333333	0.24114	458.417462160825\\
62.2083333333333	0.2428	465.957418070143\\
62.2083333333333	0.24446	473.497373979462\\
62.2083333333333	0.24612	481.03732988878\\
62.2083333333333	0.24778	488.577285798098\\
62.2083333333333	0.24944	496.117241707416\\
62.2083333333333	0.2511	503.657197616734\\
62.2083333333333	0.25276	511.197153526052\\
62.2083333333333	0.25442	518.737109435371\\
62.2083333333333	0.25608	526.277065344688\\
62.2083333333333	0.25774	533.817021254007\\
62.2083333333333	0.2594	541.356977163325\\
62.2083333333333	0.26106	548.896933072643\\
62.2083333333333	0.26272	556.436888981961\\
62.2083333333333	0.26438	563.976844891279\\
62.2083333333333	0.26604	571.516800800598\\
62.2083333333333	0.2677	579.056756709916\\
62.2083333333333	0.26936	586.596712619234\\
62.2083333333333	0.27102	594.136668528552\\
62.2083333333333	0.27268	601.67662443787\\
62.2083333333333	0.27434	609.216580347188\\
62.2083333333333	0.276	616.756536256506\\
62.4166666666667	0.193	239.828030129234\\
62.4166666666667	0.19466	247.413533152172\\
62.4166666666667	0.19632	254.99903617511\\
62.4166666666667	0.19798	262.584539198048\\
62.4166666666667	0.19964	270.170042220986\\
62.4166666666667	0.2013	277.755545243924\\
62.4166666666667	0.20296	285.341048266862\\
62.4166666666667	0.20462	292.9265512898\\
62.4166666666667	0.20628	300.512054312738\\
62.4166666666667	0.20794	308.097557335676\\
62.4166666666667	0.2096	315.683060358614\\
62.4166666666667	0.21126	323.268563381552\\
62.4166666666667	0.21292	330.85406640449\\
62.4166666666667	0.21458	338.439569427428\\
62.4166666666667	0.21624	346.025072450366\\
62.4166666666667	0.2179	353.610575473304\\
62.4166666666667	0.21956	361.196078496242\\
62.4166666666667	0.22122	368.78158151918\\
62.4166666666667	0.22288	376.367084542118\\
62.4166666666667	0.22454	383.952587565056\\
62.4166666666667	0.2262	391.538090587994\\
62.4166666666667	0.22786	399.123593610932\\
62.4166666666667	0.22952	406.70909663387\\
62.4166666666667	0.23118	414.294599656808\\
62.4166666666667	0.23284	421.880102679746\\
62.4166666666667	0.2345	429.465605702684\\
62.4166666666667	0.23616	437.051108725622\\
62.4166666666667	0.23782	444.63661174856\\
62.4166666666667	0.23948	452.222114771498\\
62.4166666666667	0.24114	459.807617794436\\
62.4166666666667	0.2428	467.393120817374\\
62.4166666666667	0.24446	474.978623840312\\
62.4166666666667	0.24612	482.56412686325\\
62.4166666666667	0.24778	490.149629886188\\
62.4166666666667	0.24944	497.735132909126\\
62.4166666666667	0.2511	505.320635932064\\
62.4166666666667	0.25276	512.906138955002\\
62.4166666666667	0.25442	520.49164197794\\
62.4166666666667	0.25608	528.077145000878\\
62.4166666666667	0.25774	535.662648023816\\
62.4166666666667	0.2594	543.248151046754\\
62.4166666666667	0.26106	550.833654069692\\
62.4166666666667	0.26272	558.41915709263\\
62.4166666666667	0.26438	566.004660115568\\
62.4166666666667	0.26604	573.590163138506\\
62.4166666666667	0.2677	581.175666161444\\
62.4166666666667	0.26936	588.761169184382\\
62.4166666666667	0.27102	596.34667220732\\
62.4166666666667	0.27268	603.932175230258\\
62.4166666666667	0.27434	611.517678253196\\
62.4166666666667	0.276	619.103181276134\\
62.625	0.193	239.89731946787\\
62.625	0.19466	247.528369604427\\
62.625	0.19632	255.159419740985\\
62.625	0.19798	262.790469877543\\
62.625	0.19964	270.421520014101\\
62.625	0.2013	278.052570150659\\
62.625	0.20296	285.683620287216\\
62.625	0.20462	293.314670423774\\
62.625	0.20628	300.945720560332\\
62.625	0.20794	308.57677069689\\
62.625	0.2096	316.207820833448\\
62.625	0.21126	323.838870970006\\
62.625	0.21292	331.469921106563\\
62.625	0.21458	339.100971243121\\
62.625	0.21624	346.732021379679\\
62.625	0.2179	354.363071516237\\
62.625	0.21956	361.994121652795\\
62.625	0.22122	369.625171789352\\
62.625	0.22288	377.25622192591\\
62.625	0.22454	384.887272062468\\
62.625	0.2262	392.518322199026\\
62.625	0.22786	400.149372335584\\
62.625	0.22952	407.780422472142\\
62.625	0.23118	415.4114726087\\
62.625	0.23284	423.042522745257\\
62.625	0.2345	430.673572881815\\
62.625	0.23616	438.304623018373\\
62.625	0.23782	445.935673154931\\
62.625	0.23948	453.566723291489\\
62.625	0.24114	461.197773428047\\
62.625	0.2428	468.828823564604\\
62.625	0.24446	476.459873701162\\
62.625	0.24612	484.09092383772\\
62.625	0.24778	491.721973974278\\
62.625	0.24944	499.353024110836\\
62.625	0.2511	506.984074247393\\
62.625	0.25276	514.615124383951\\
62.625	0.25442	522.246174520509\\
62.625	0.25608	529.877224657067\\
62.625	0.25774	537.508274793625\\
62.625	0.2594	545.139324930183\\
62.625	0.26106	552.77037506674\\
62.625	0.26272	560.401425203298\\
62.625	0.26438	568.032475339856\\
62.625	0.26604	575.663525476414\\
62.625	0.2677	583.294575612972\\
62.625	0.26936	590.92562574953\\
62.625	0.27102	598.556675886088\\
62.625	0.27268	606.187726022645\\
62.625	0.27434	613.818776159203\\
62.625	0.276	621.449826295761\\
62.8333333333333	0.193	239.966608806505\\
62.8333333333333	0.19466	247.643206056683\\
62.8333333333333	0.19632	255.31980330686\\
62.8333333333333	0.19798	262.996400557038\\
62.8333333333333	0.19964	270.672997807216\\
62.8333333333333	0.2013	278.349595057393\\
62.8333333333333	0.20296	286.026192307571\\
62.8333333333333	0.20462	293.702789557748\\
62.8333333333333	0.20628	301.379386807926\\
62.8333333333333	0.20794	309.055984058104\\
62.8333333333333	0.2096	316.732581308282\\
62.8333333333333	0.21126	324.409178558459\\
62.8333333333333	0.21292	332.085775808637\\
62.8333333333333	0.21458	339.762373058815\\
62.8333333333333	0.21624	347.438970308992\\
62.8333333333333	0.2179	355.11556755917\\
62.8333333333333	0.21956	362.792164809347\\
62.8333333333333	0.22122	370.468762059525\\
62.8333333333333	0.22288	378.145359309703\\
62.8333333333333	0.22454	385.821956559881\\
62.8333333333333	0.2262	393.498553810058\\
62.8333333333333	0.22786	401.175151060236\\
62.8333333333333	0.22952	408.851748310414\\
62.8333333333333	0.23118	416.528345560591\\
62.8333333333333	0.23284	424.204942810769\\
62.8333333333333	0.2345	431.881540060946\\
62.8333333333333	0.23616	439.558137311124\\
62.8333333333333	0.23782	447.234734561302\\
62.8333333333333	0.23948	454.91133181148\\
62.8333333333333	0.24114	462.587929061657\\
62.8333333333333	0.2428	470.264526311835\\
62.8333333333333	0.24446	477.941123562012\\
62.8333333333333	0.24612	485.61772081219\\
62.8333333333333	0.24778	493.294318062368\\
62.8333333333333	0.24944	500.970915312545\\
62.8333333333333	0.2511	508.647512562723\\
62.8333333333333	0.25276	516.324109812901\\
62.8333333333333	0.25442	524.000707063079\\
62.8333333333333	0.25608	531.677304313256\\
62.8333333333333	0.25774	539.353901563434\\
62.8333333333333	0.2594	547.030498813611\\
62.8333333333333	0.26106	554.707096063789\\
62.8333333333333	0.26272	562.383693313967\\
62.8333333333333	0.26438	570.060290564144\\
62.8333333333333	0.26604	577.736887814322\\
62.8333333333333	0.2677	585.4134850645\\
62.8333333333333	0.26936	593.090082314678\\
62.8333333333333	0.27102	600.766679564855\\
62.8333333333333	0.27268	608.443276815033\\
62.8333333333333	0.27434	616.11987406521\\
62.8333333333333	0.276	623.796471315388\\
63.0416666666667	0.193	240.035898145141\\
63.0416666666667	0.19466	247.758042508938\\
63.0416666666667	0.19632	255.480186872736\\
63.0416666666667	0.19798	263.202331236533\\
63.0416666666667	0.19964	270.924475600331\\
63.0416666666667	0.2013	278.646619964128\\
63.0416666666667	0.20296	286.368764327926\\
63.0416666666667	0.20462	294.090908691723\\
63.0416666666667	0.20628	301.813053055521\\
63.0416666666667	0.20794	309.535197419318\\
63.0416666666667	0.2096	317.257341783116\\
63.0416666666667	0.21126	324.979486146913\\
63.0416666666667	0.21292	332.701630510711\\
63.0416666666667	0.21458	340.423774874508\\
63.0416666666667	0.21624	348.145919238306\\
63.0416666666667	0.2179	355.868063602103\\
63.0416666666667	0.21956	363.590207965901\\
63.0416666666667	0.22122	371.312352329698\\
63.0416666666667	0.22288	379.034496693496\\
63.0416666666667	0.22454	386.756641057293\\
63.0416666666667	0.2262	394.478785421091\\
63.0416666666667	0.22786	402.200929784888\\
63.0416666666667	0.22952	409.923074148686\\
63.0416666666667	0.23118	417.645218512483\\
63.0416666666667	0.23284	425.367362876281\\
63.0416666666667	0.2345	433.089507240078\\
63.0416666666667	0.23616	440.811651603876\\
63.0416666666667	0.23782	448.533795967673\\
63.0416666666667	0.23948	456.255940331471\\
63.0416666666667	0.24114	463.978084695268\\
63.0416666666667	0.2428	471.700229059066\\
63.0416666666667	0.24446	479.422373422863\\
63.0416666666667	0.24612	487.144517786661\\
63.0416666666667	0.24778	494.866662150458\\
63.0416666666667	0.24944	502.588806514256\\
63.0416666666667	0.2511	510.310950878053\\
63.0416666666667	0.25276	518.03309524185\\
63.0416666666667	0.25442	525.755239605648\\
63.0416666666667	0.25608	533.477383969446\\
63.0416666666667	0.25774	541.199528333243\\
63.0416666666667	0.2594	548.92167269704\\
63.0416666666667	0.26106	556.643817060838\\
63.0416666666667	0.26272	564.365961424636\\
63.0416666666667	0.26438	572.088105788433\\
63.0416666666667	0.26604	579.810250152231\\
63.0416666666667	0.2677	587.532394516028\\
63.0416666666667	0.26936	595.254538879825\\
63.0416666666667	0.27102	602.976683243623\\
63.0416666666667	0.27268	610.698827607421\\
63.0416666666667	0.27434	618.420971971218\\
63.0416666666667	0.276	626.143116335016\\
63.25	0.193	240.105187483776\\
63.25	0.19466	247.872878961194\\
63.25	0.19632	255.640570438611\\
63.25	0.19798	263.408261916028\\
63.25	0.19964	271.175953393446\\
63.25	0.2013	278.943644870863\\
63.25	0.20296	286.71133634828\\
63.25	0.20462	294.479027825698\\
63.25	0.20628	302.246719303115\\
63.25	0.20794	310.014410780532\\
63.25	0.2096	317.782102257949\\
63.25	0.21126	325.549793735367\\
63.25	0.21292	333.317485212784\\
63.25	0.21458	341.085176690201\\
63.25	0.21624	348.852868167619\\
63.25	0.2179	356.620559645036\\
63.25	0.21956	364.388251122454\\
63.25	0.22122	372.155942599871\\
63.25	0.22288	379.923634077288\\
63.25	0.22454	387.691325554705\\
63.25	0.2262	395.459017032123\\
63.25	0.22786	403.22670850954\\
63.25	0.22952	410.994399986957\\
63.25	0.23118	418.762091464375\\
63.25	0.23284	426.529782941792\\
63.25	0.2345	434.297474419209\\
63.25	0.23616	442.065165896627\\
63.25	0.23782	449.832857374044\\
63.25	0.23948	457.600548851462\\
63.25	0.24114	465.368240328879\\
63.25	0.2428	473.135931806296\\
63.25	0.24446	480.903623283713\\
63.25	0.24612	488.671314761131\\
63.25	0.24778	496.439006238548\\
63.25	0.24944	504.206697715965\\
63.25	0.2511	511.974389193383\\
63.25	0.25276	519.7420806708\\
63.25	0.25442	527.509772148218\\
63.25	0.25608	535.277463625635\\
63.25	0.25774	543.045155103052\\
63.25	0.2594	550.812846580469\\
63.25	0.26106	558.580538057887\\
63.25	0.26272	566.348229535304\\
63.25	0.26438	574.115921012721\\
63.25	0.26604	581.883612490139\\
63.25	0.2677	589.651303967556\\
63.25	0.26936	597.418995444973\\
63.25	0.27102	605.186686922391\\
63.25	0.27268	612.954378399808\\
63.25	0.27434	620.722069877226\\
63.25	0.276	628.489761354643\\
63.4583333333333	0.193	240.174476822412\\
63.4583333333333	0.19466	247.987715413449\\
63.4583333333333	0.19632	255.800954004486\\
63.4583333333333	0.19798	263.614192595524\\
63.4583333333333	0.19964	271.427431186561\\
63.4583333333333	0.2013	279.240669777598\\
63.4583333333333	0.20296	287.053908368635\\
63.4583333333333	0.20462	294.867146959672\\
63.4583333333333	0.20628	302.680385550709\\
63.4583333333333	0.20794	310.493624141747\\
63.4583333333333	0.2096	318.306862732784\\
63.4583333333333	0.21126	326.120101323821\\
63.4583333333333	0.21292	333.933339914858\\
63.4583333333333	0.21458	341.746578505895\\
63.4583333333333	0.21624	349.559817096932\\
63.4583333333333	0.2179	357.37305568797\\
63.4583333333333	0.21956	365.186294279007\\
63.4583333333333	0.22122	372.999532870044\\
63.4583333333333	0.22288	380.812771461081\\
63.4583333333333	0.22454	388.626010052118\\
63.4583333333333	0.2262	396.439248643155\\
63.4583333333333	0.22786	404.252487234193\\
63.4583333333333	0.22952	412.06572582523\\
63.4583333333333	0.23118	419.878964416267\\
63.4583333333333	0.23284	427.692203007304\\
63.4583333333333	0.2345	435.505441598341\\
63.4583333333333	0.23616	443.318680189378\\
63.4583333333333	0.23782	451.131918780416\\
63.4583333333333	0.23948	458.945157371453\\
63.4583333333333	0.24114	466.75839596249\\
63.4583333333333	0.2428	474.571634553527\\
63.4583333333333	0.24446	482.384873144564\\
63.4583333333333	0.24612	490.198111735601\\
63.4583333333333	0.24778	498.011350326638\\
63.4583333333333	0.24944	505.824588917676\\
63.4583333333333	0.2511	513.637827508713\\
63.4583333333333	0.25276	521.45106609975\\
63.4583333333333	0.25442	529.264304690787\\
63.4583333333333	0.25608	537.077543281825\\
63.4583333333333	0.25774	544.890781872862\\
63.4583333333333	0.2594	552.704020463899\\
63.4583333333333	0.26106	560.517259054936\\
63.4583333333333	0.26272	568.330497645973\\
63.4583333333333	0.26438	576.14373623701\\
63.4583333333333	0.26604	583.956974828047\\
63.4583333333333	0.2677	591.770213419085\\
63.4583333333333	0.26936	599.583452010122\\
63.4583333333333	0.27102	607.396690601159\\
63.4583333333333	0.27268	615.209929192196\\
63.4583333333333	0.27434	623.023167783233\\
63.4583333333333	0.276	630.836406374271\\
63.6666666666667	0.193	240.243766161048\\
63.6666666666667	0.19466	248.102551865705\\
63.6666666666667	0.19632	255.961337570362\\
63.6666666666667	0.19798	263.820123275019\\
63.6666666666667	0.19964	271.678908979676\\
63.6666666666667	0.2013	279.537694684333\\
63.6666666666667	0.20296	287.39648038899\\
63.6666666666667	0.20462	295.255266093647\\
63.6666666666667	0.20628	303.114051798304\\
63.6666666666667	0.20794	310.972837502961\\
63.6666666666667	0.2096	318.831623207618\\
63.6666666666667	0.21126	326.690408912275\\
63.6666666666667	0.21292	334.549194616932\\
63.6666666666667	0.21458	342.407980321589\\
63.6666666666667	0.21624	350.266766026246\\
63.6666666666667	0.2179	358.125551730903\\
63.6666666666667	0.21956	365.98433743556\\
63.6666666666667	0.22122	373.843123140217\\
63.6666666666667	0.22288	381.701908844874\\
63.6666666666667	0.22454	389.560694549531\\
63.6666666666667	0.2262	397.419480254188\\
63.6666666666667	0.22786	405.278265958845\\
63.6666666666667	0.22952	413.137051663502\\
63.6666666666667	0.23118	420.995837368159\\
63.6666666666667	0.23284	428.854623072816\\
63.6666666666667	0.2345	436.713408777473\\
63.6666666666667	0.23616	444.57219448213\\
63.6666666666667	0.23782	452.430980186787\\
63.6666666666667	0.23948	460.289765891444\\
63.6666666666667	0.24114	468.148551596101\\
63.6666666666667	0.2428	476.007337300758\\
63.6666666666667	0.24446	483.866123005414\\
63.6666666666667	0.24612	491.724908710072\\
63.6666666666667	0.24778	499.583694414729\\
63.6666666666667	0.24944	507.442480119386\\
63.6666666666667	0.2511	515.301265824043\\
63.6666666666667	0.25276	523.1600515287\\
63.6666666666667	0.25442	531.018837233357\\
63.6666666666667	0.25608	538.877622938014\\
63.6666666666667	0.25774	546.736408642671\\
63.6666666666667	0.2594	554.595194347328\\
63.6666666666667	0.26106	562.453980051985\\
63.6666666666667	0.26272	570.312765756642\\
63.6666666666667	0.26438	578.171551461299\\
63.6666666666667	0.26604	586.030337165956\\
63.6666666666667	0.2677	593.889122870613\\
63.6666666666667	0.26936	601.74790857527\\
63.6666666666667	0.27102	609.606694279927\\
63.6666666666667	0.27268	617.465479984584\\
63.6666666666667	0.27434	625.324265689241\\
63.6666666666667	0.276	633.183051393898\\
63.875	0.193	240.313055499683\\
63.875	0.19466	248.21738831796\\
63.875	0.19632	256.121721136237\\
63.875	0.19798	264.026053954514\\
63.875	0.19964	271.930386772791\\
63.875	0.2013	279.834719591067\\
63.875	0.20296	287.739052409344\\
63.875	0.20462	295.643385227621\\
63.875	0.20628	303.547718045898\\
63.875	0.20794	311.452050864175\\
63.875	0.2096	319.356383682451\\
63.875	0.21126	327.260716500728\\
63.875	0.21292	335.165049319005\\
63.875	0.21458	343.069382137282\\
63.875	0.21624	350.973714955559\\
63.875	0.2179	358.878047773836\\
63.875	0.21956	366.782380592113\\
63.875	0.22122	374.686713410389\\
63.875	0.22288	382.591046228666\\
63.875	0.22454	390.495379046943\\
63.875	0.2262	398.39971186522\\
63.875	0.22786	406.304044683497\\
63.875	0.22952	414.208377501774\\
63.875	0.23118	422.11271032005\\
63.875	0.23284	430.017043138327\\
63.875	0.2345	437.921375956604\\
63.875	0.23616	445.825708774881\\
63.875	0.23782	453.730041593158\\
63.875	0.23948	461.634374411435\\
63.875	0.24114	469.538707229712\\
63.875	0.2428	477.443040047988\\
63.875	0.24446	485.347372866265\\
63.875	0.24612	493.251705684542\\
63.875	0.24778	501.156038502819\\
63.875	0.24944	509.060371321096\\
63.875	0.2511	516.964704139372\\
63.875	0.25276	524.869036957649\\
63.875	0.25442	532.773369775926\\
63.875	0.25608	540.677702594203\\
63.875	0.25774	548.58203541248\\
63.875	0.2594	556.486368230757\\
63.875	0.26106	564.390701049033\\
63.875	0.26272	572.29503386731\\
63.875	0.26438	580.199366685587\\
63.875	0.26604	588.103699503864\\
63.875	0.2677	596.008032322141\\
63.875	0.26936	603.912365140418\\
63.875	0.27102	611.816697958695\\
63.875	0.27268	619.721030776971\\
63.875	0.27434	627.625363595248\\
63.875	0.276	635.529696413525\\
64.0833333333333	0.193	240.382344838319\\
64.0833333333333	0.19466	248.332224770216\\
64.0833333333333	0.19632	256.282104702112\\
64.0833333333333	0.19798	264.231984634009\\
64.0833333333333	0.19964	272.181864565906\\
64.0833333333333	0.2013	280.131744497802\\
64.0833333333333	0.20296	288.081624429699\\
64.0833333333333	0.20462	296.031504361596\\
64.0833333333333	0.20628	303.981384293492\\
64.0833333333333	0.20794	311.931264225389\\
64.0833333333333	0.2096	319.881144157286\\
64.0833333333333	0.21126	327.831024089182\\
64.0833333333333	0.21292	335.780904021079\\
64.0833333333333	0.21458	343.730783952976\\
64.0833333333333	0.21624	351.680663884873\\
64.0833333333333	0.2179	359.630543816769\\
64.0833333333333	0.21956	367.580423748666\\
64.0833333333333	0.22122	375.530303680562\\
64.0833333333333	0.22288	383.480183612459\\
64.0833333333333	0.22454	391.430063544356\\
64.0833333333333	0.2262	399.379943476252\\
64.0833333333333	0.22786	407.329823408149\\
64.0833333333333	0.22952	415.279703340046\\
64.0833333333333	0.23118	423.229583271942\\
64.0833333333333	0.23284	431.179463203839\\
64.0833333333333	0.2345	439.129343135736\\
64.0833333333333	0.23616	447.079223067632\\
64.0833333333333	0.23782	455.029102999529\\
64.0833333333333	0.23948	462.978982931426\\
64.0833333333333	0.24114	470.928862863323\\
64.0833333333333	0.2428	478.878742795219\\
64.0833333333333	0.24446	486.828622727116\\
64.0833333333333	0.24612	494.778502659012\\
64.0833333333333	0.24778	502.728382590909\\
64.0833333333333	0.24944	510.678262522806\\
64.0833333333333	0.2511	518.628142454702\\
64.0833333333333	0.25276	526.578022386599\\
64.0833333333333	0.25442	534.527902318496\\
64.0833333333333	0.25608	542.477782250392\\
64.0833333333333	0.25774	550.427662182289\\
64.0833333333333	0.2594	558.377542114186\\
64.0833333333333	0.26106	566.327422046082\\
64.0833333333333	0.26272	574.277301977979\\
64.0833333333333	0.26438	582.227181909876\\
64.0833333333333	0.26604	590.177061841773\\
64.0833333333333	0.2677	598.126941773669\\
64.0833333333333	0.26936	606.076821705566\\
64.0833333333333	0.27102	614.026701637463\\
64.0833333333333	0.27268	621.976581569359\\
64.0833333333333	0.27434	629.926461501256\\
64.0833333333333	0.276	637.876341433152\\
64.2916666666667	0.193	240.451634176955\\
64.2916666666667	0.19466	248.447061222471\\
64.2916666666667	0.19632	256.442488267988\\
64.2916666666667	0.19798	264.437915313504\\
64.2916666666667	0.19964	272.433342359021\\
64.2916666666667	0.2013	280.428769404537\\
64.2916666666667	0.20296	288.424196450054\\
64.2916666666667	0.20462	296.41962349557\\
64.2916666666667	0.20628	304.415050541087\\
64.2916666666667	0.20794	312.410477586603\\
64.2916666666667	0.2096	320.405904632119\\
64.2916666666667	0.21126	328.401331677636\\
64.2916666666667	0.21292	336.396758723153\\
64.2916666666667	0.21458	344.392185768669\\
64.2916666666667	0.21624	352.387612814186\\
64.2916666666667	0.2179	360.383039859702\\
64.2916666666667	0.21956	368.378466905219\\
64.2916666666667	0.22122	376.373893950735\\
64.2916666666667	0.22288	384.369320996252\\
64.2916666666667	0.22454	392.364748041768\\
64.2916666666667	0.2262	400.360175087285\\
64.2916666666667	0.22786	408.355602132801\\
64.2916666666667	0.22952	416.351029178318\\
64.2916666666667	0.23118	424.346456223834\\
64.2916666666667	0.23284	432.341883269351\\
64.2916666666667	0.2345	440.337310314867\\
64.2916666666667	0.23616	448.332737360384\\
64.2916666666667	0.23782	456.3281644059\\
64.2916666666667	0.23948	464.323591451417\\
64.2916666666667	0.24114	472.319018496933\\
64.2916666666667	0.2428	480.31444554245\\
64.2916666666667	0.24446	488.309872587966\\
64.2916666666667	0.24612	496.305299633483\\
64.2916666666667	0.24778	504.300726679\\
64.2916666666667	0.24944	512.296153724516\\
64.2916666666667	0.2511	520.291580770032\\
64.2916666666667	0.25276	528.287007815549\\
64.2916666666667	0.25442	536.282434861065\\
64.2916666666667	0.25608	544.277861906582\\
64.2916666666667	0.25774	552.273288952098\\
64.2916666666667	0.2594	560.268715997615\\
64.2916666666667	0.26106	568.264143043131\\
64.2916666666667	0.26272	576.259570088648\\
64.2916666666667	0.26438	584.254997134164\\
64.2916666666667	0.26604	592.250424179681\\
64.2916666666667	0.2677	600.245851225197\\
64.2916666666667	0.26936	608.241278270714\\
64.2916666666667	0.27102	616.23670531623\\
64.2916666666667	0.27268	624.232132361747\\
64.2916666666667	0.27434	632.227559407263\\
64.2916666666667	0.276	640.22298645278\\
64.5	0.193	240.52092351559\\
64.5	0.19466	248.561897674726\\
64.5	0.19632	256.602871833863\\
64.5	0.19798	264.643845992999\\
64.5	0.19964	272.684820152135\\
64.5	0.2013	280.725794311272\\
64.5	0.20296	288.766768470408\\
64.5	0.20462	296.807742629544\\
64.5	0.20628	304.848716788681\\
64.5	0.20794	312.889690947817\\
64.5	0.2096	320.930665106953\\
64.5	0.21126	328.97163926609\\
64.5	0.21292	337.012613425226\\
64.5	0.21458	345.053587584362\\
64.5	0.21624	353.094561743499\\
64.5	0.2179	361.135535902635\\
64.5	0.21956	369.176510061771\\
64.5	0.22122	377.217484220908\\
64.5	0.22288	385.258458380044\\
64.5	0.22454	393.29943253918\\
64.5	0.2262	401.340406698317\\
64.5	0.22786	409.381380857453\\
64.5	0.22952	417.422355016589\\
64.5	0.23118	425.463329175726\\
64.5	0.23284	433.504303334862\\
64.5	0.2345	441.545277493998\\
64.5	0.23616	449.586251653135\\
64.5	0.23782	457.627225812271\\
64.5	0.23948	465.668199971407\\
64.5	0.24114	473.709174130544\\
64.5	0.2428	481.75014828968\\
64.5	0.24446	489.791122448817\\
64.5	0.24612	497.832096607953\\
64.5	0.24778	505.873070767089\\
64.5	0.24944	513.914044926226\\
64.5	0.2511	521.955019085362\\
64.5	0.25276	529.995993244498\\
64.5	0.25442	538.036967403635\\
64.5	0.25608	546.077941562771\\
64.5	0.25774	554.118915721907\\
64.5	0.2594	562.159889881043\\
64.5	0.26106	570.20086404018\\
64.5	0.26272	578.241838199316\\
64.5	0.26438	586.282812358452\\
64.5	0.26604	594.323786517589\\
64.5	0.2677	602.364760676725\\
64.5	0.26936	610.405734835861\\
64.5	0.27102	618.446708994998\\
64.5	0.27268	626.487683154134\\
64.5	0.27434	634.528657313271\\
64.5	0.276	642.569631472407\\
64.7083333333333	0.193	240.590212854226\\
64.7083333333333	0.19466	248.676734126982\\
64.7083333333333	0.19632	256.763255399738\\
64.7083333333333	0.19798	264.849776672494\\
64.7083333333333	0.19964	272.936297945251\\
64.7083333333333	0.2013	281.022819218007\\
64.7083333333333	0.20296	289.109340490763\\
64.7083333333333	0.20462	297.195861763519\\
64.7083333333333	0.20628	305.282383036275\\
64.7083333333333	0.20794	313.368904309031\\
64.7083333333333	0.2096	321.455425581787\\
64.7083333333333	0.21126	329.541946854544\\
64.7083333333333	0.21292	337.6284681273\\
64.7083333333333	0.21458	345.714989400056\\
64.7083333333333	0.21624	353.801510672812\\
64.7083333333333	0.2179	361.888031945568\\
64.7083333333333	0.21956	369.974553218324\\
64.7083333333333	0.22122	378.06107449108\\
64.7083333333333	0.22288	386.147595763837\\
64.7083333333333	0.22454	394.234117036593\\
64.7083333333333	0.2262	402.320638309349\\
64.7083333333333	0.22786	410.407159582105\\
64.7083333333333	0.22952	418.493680854861\\
64.7083333333333	0.23118	426.580202127617\\
64.7083333333333	0.23284	434.666723400374\\
64.7083333333333	0.2345	442.75324467313\\
64.7083333333333	0.23616	450.839765945886\\
64.7083333333333	0.23782	458.926287218642\\
64.7083333333333	0.23948	467.012808491399\\
64.7083333333333	0.24114	475.099329764155\\
64.7083333333333	0.2428	483.185851036911\\
64.7083333333333	0.24446	491.272372309667\\
64.7083333333333	0.24612	499.358893582423\\
64.7083333333333	0.24778	507.445414855179\\
64.7083333333333	0.24944	515.531936127936\\
64.7083333333333	0.2511	523.618457400692\\
64.7083333333333	0.25276	531.704978673448\\
64.7083333333333	0.25442	539.791499946204\\
64.7083333333333	0.25608	547.87802121896\\
64.7083333333333	0.25774	555.964542491716\\
64.7083333333333	0.2594	564.051063764472\\
64.7083333333333	0.26106	572.137585037228\\
64.7083333333333	0.26272	580.224106309985\\
64.7083333333333	0.26438	588.310627582741\\
64.7083333333333	0.26604	596.397148855498\\
64.7083333333333	0.2677	604.483670128253\\
64.7083333333333	0.26936	612.57019140101\\
64.7083333333333	0.27102	620.656712673766\\
64.7083333333333	0.27268	628.743233946522\\
64.7083333333333	0.27434	636.829755219278\\
64.7083333333333	0.276	644.916276492034\\
64.9166666666667	0.193	240.659502192862\\
64.9166666666667	0.19466	248.791570579237\\
64.9166666666667	0.19632	256.923638965613\\
64.9166666666667	0.19798	265.055707351989\\
64.9166666666667	0.19964	273.187775738365\\
64.9166666666667	0.2013	281.319844124741\\
64.9166666666667	0.20296	289.451912511117\\
64.9166666666667	0.20462	297.583980897494\\
64.9166666666667	0.20628	305.716049283869\\
64.9166666666667	0.20794	313.848117670246\\
64.9166666666667	0.2096	321.980186056621\\
64.9166666666667	0.21126	330.112254442997\\
64.9166666666667	0.21292	338.244322829373\\
64.9166666666667	0.21458	346.376391215749\\
64.9166666666667	0.21624	354.508459602125\\
64.9166666666667	0.2179	362.640527988501\\
64.9166666666667	0.21956	370.772596374877\\
64.9166666666667	0.22122	378.904664761253\\
64.9166666666667	0.22288	387.03673314763\\
64.9166666666667	0.22454	395.168801534006\\
64.9166666666667	0.2262	403.300869920382\\
64.9166666666667	0.22786	411.432938306757\\
64.9166666666667	0.22952	419.565006693133\\
64.9166666666667	0.23118	427.697075079509\\
64.9166666666667	0.23284	435.829143465886\\
64.9166666666667	0.2345	443.961211852262\\
64.9166666666667	0.23616	452.093280238637\\
64.9166666666667	0.23782	460.225348625014\\
64.9166666666667	0.23948	468.35741701139\\
64.9166666666667	0.24114	476.489485397766\\
64.9166666666667	0.2428	484.621553784141\\
64.9166666666667	0.24446	492.753622170518\\
64.9166666666667	0.24612	500.885690556893\\
64.9166666666667	0.24778	509.01775894327\\
64.9166666666667	0.24944	517.149827329646\\
64.9166666666667	0.2511	525.281895716021\\
64.9166666666667	0.25276	533.413964102398\\
64.9166666666667	0.25442	541.546032488774\\
64.9166666666667	0.25608	549.67810087515\\
64.9166666666667	0.25774	557.810169261526\\
64.9166666666667	0.2594	565.942237647902\\
64.9166666666667	0.26106	574.074306034277\\
64.9166666666667	0.26272	582.206374420653\\
64.9166666666667	0.26438	590.33844280703\\
64.9166666666667	0.26604	598.470511193406\\
64.9166666666667	0.2677	606.602579579782\\
64.9166666666667	0.26936	614.734647966158\\
64.9166666666667	0.27102	622.866716352534\\
64.9166666666667	0.27268	630.99878473891\\
64.9166666666667	0.27434	639.130853125286\\
64.9166666666667	0.276	647.262921511662\\
65.125	0.193	240.728791531497\\
65.125	0.19466	248.906407031493\\
65.125	0.19632	257.084022531489\\
65.125	0.19798	265.261638031485\\
65.125	0.19964	273.439253531481\\
65.125	0.2013	281.616869031476\\
65.125	0.20296	289.794484531472\\
65.125	0.20462	297.972100031468\\
65.125	0.20628	306.149715531464\\
65.125	0.20794	314.32733103146\\
65.125	0.2096	322.504946531455\\
65.125	0.21126	330.682562031451\\
65.125	0.21292	338.860177531447\\
65.125	0.21458	347.037793031443\\
65.125	0.21624	355.215408531439\\
65.125	0.2179	363.393024031435\\
65.125	0.21956	371.570639531431\\
65.125	0.22122	379.748255031426\\
65.125	0.22288	387.925870531422\\
65.125	0.22454	396.103486031418\\
65.125	0.2262	404.281101531414\\
65.125	0.22786	412.45871703141\\
65.125	0.22952	420.636332531406\\
65.125	0.23118	428.813948031401\\
65.125	0.23284	436.991563531397\\
65.125	0.2345	445.169179031393\\
65.125	0.23616	453.346794531389\\
65.125	0.23782	461.524410031385\\
65.125	0.23948	469.702025531381\\
65.125	0.24114	477.879641031377\\
65.125	0.2428	486.057256531372\\
65.125	0.24446	494.234872031368\\
65.125	0.24612	502.412487531364\\
65.125	0.24778	510.59010303136\\
65.125	0.24944	518.767718531356\\
65.125	0.2511	526.945334031351\\
65.125	0.25276	535.122949531347\\
65.125	0.25442	543.300565031343\\
65.125	0.25608	551.478180531339\\
65.125	0.25774	559.655796031335\\
65.125	0.2594	567.833411531331\\
65.125	0.26106	576.011027031326\\
65.125	0.26272	584.188642531322\\
65.125	0.26438	592.366258031318\\
65.125	0.26604	600.543873531314\\
65.125	0.2677	608.72148903131\\
65.125	0.26936	616.899104531306\\
65.125	0.27102	625.076720031302\\
65.125	0.27268	633.254335531297\\
65.125	0.27434	641.431951031293\\
65.125	0.276	649.609566531289\\
65.3333333333333	0.193	240.798080870132\\
65.3333333333333	0.19466	249.021243483748\\
65.3333333333333	0.19632	257.244406097364\\
65.3333333333333	0.19798	265.46756871098\\
65.3333333333333	0.19964	273.690731324595\\
65.3333333333333	0.2013	281.913893938211\\
65.3333333333333	0.20296	290.137056551827\\
65.3333333333333	0.20462	298.360219165442\\
65.3333333333333	0.20628	306.583381779058\\
65.3333333333333	0.20794	314.806544392674\\
65.3333333333333	0.2096	323.02970700629\\
65.3333333333333	0.21126	331.252869619905\\
65.3333333333333	0.21292	339.476032233521\\
65.3333333333333	0.21458	347.699194847136\\
65.3333333333333	0.21624	355.922357460752\\
65.3333333333333	0.2179	364.145520074368\\
65.3333333333333	0.21956	372.368682687983\\
65.3333333333333	0.22122	380.591845301599\\
65.3333333333333	0.22288	388.815007915215\\
65.3333333333333	0.22454	397.038170528831\\
65.3333333333333	0.2262	405.261333142446\\
65.3333333333333	0.22786	413.484495756062\\
65.3333333333333	0.22952	421.707658369677\\
65.3333333333333	0.23118	429.930820983293\\
65.3333333333333	0.23284	438.153983596909\\
65.3333333333333	0.2345	446.377146210524\\
65.3333333333333	0.23616	454.60030882414\\
65.3333333333333	0.23782	462.823471437756\\
65.3333333333333	0.23948	471.046634051371\\
65.3333333333333	0.24114	479.269796664987\\
65.3333333333333	0.2428	487.492959278603\\
65.3333333333333	0.24446	495.716121892218\\
65.3333333333333	0.24612	503.939284505834\\
65.3333333333333	0.24778	512.16244711945\\
65.3333333333333	0.24944	520.385609733065\\
65.3333333333333	0.2511	528.608772346681\\
65.3333333333333	0.25276	536.831934960297\\
65.3333333333333	0.25442	545.055097573913\\
65.3333333333333	0.25608	553.278260187528\\
65.3333333333333	0.25774	561.501422801144\\
65.3333333333333	0.2594	569.72458541476\\
65.3333333333333	0.26106	577.947748028375\\
65.3333333333333	0.26272	586.170910641991\\
65.3333333333333	0.26438	594.394073255607\\
65.3333333333333	0.26604	602.617235869222\\
65.3333333333333	0.2677	610.840398482838\\
65.3333333333333	0.26936	619.063561096454\\
65.3333333333333	0.27102	627.28672371007\\
65.3333333333333	0.27268	635.509886323685\\
65.3333333333333	0.27434	643.733048937301\\
65.3333333333333	0.276	651.956211550917\\
65.5416666666667	0.193	240.867370208769\\
65.5416666666667	0.19466	249.136079936004\\
65.5416666666667	0.19632	257.404789663239\\
65.5416666666667	0.19798	265.673499390475\\
65.5416666666667	0.19964	273.94220911771\\
65.5416666666667	0.2013	282.210918844946\\
65.5416666666667	0.20296	290.479628572181\\
65.5416666666667	0.20462	298.748338299417\\
65.5416666666667	0.20628	307.017048026652\\
65.5416666666667	0.20794	315.285757753888\\
65.5416666666667	0.2096	323.554467481124\\
65.5416666666667	0.21126	331.823177208359\\
65.5416666666667	0.21292	340.091886935595\\
65.5416666666667	0.21458	348.36059666283\\
65.5416666666667	0.21624	356.629306390066\\
65.5416666666667	0.2179	364.898016117301\\
65.5416666666667	0.21956	373.166725844537\\
65.5416666666667	0.22122	381.435435571772\\
65.5416666666667	0.22288	389.704145299008\\
65.5416666666667	0.22454	397.972855026243\\
65.5416666666667	0.2262	406.241564753479\\
65.5416666666667	0.22786	414.510274480714\\
65.5416666666667	0.22952	422.77898420795\\
65.5416666666667	0.23118	431.047693935185\\
65.5416666666667	0.23284	439.316403662421\\
65.5416666666667	0.2345	447.585113389656\\
65.5416666666667	0.23616	455.853823116892\\
65.5416666666667	0.23782	464.122532844127\\
65.5416666666667	0.23948	472.391242571363\\
65.5416666666667	0.24114	480.659952298598\\
65.5416666666667	0.2428	488.928662025834\\
65.5416666666667	0.24446	497.197371753069\\
65.5416666666667	0.24612	505.466081480305\\
65.5416666666667	0.24778	513.734791207541\\
65.5416666666667	0.24944	522.003500934776\\
65.5416666666667	0.2511	530.272210662011\\
65.5416666666667	0.25276	538.540920389247\\
65.5416666666667	0.25442	546.809630116482\\
65.5416666666667	0.25608	555.078339843718\\
65.5416666666667	0.25774	563.347049570953\\
65.5416666666667	0.2594	571.615759298189\\
65.5416666666667	0.26106	579.884469025424\\
65.5416666666667	0.26272	588.15317875266\\
65.5416666666667	0.26438	596.421888479895\\
65.5416666666667	0.26604	604.690598207131\\
65.5416666666667	0.2677	612.959307934366\\
65.5416666666667	0.26936	621.228017661602\\
65.5416666666667	0.27102	629.496727388838\\
65.5416666666667	0.27268	637.765437116073\\
65.5416666666667	0.27434	646.034146843308\\
65.5416666666667	0.276	654.302856570544\\
65.75	0.193	240.936659547403\\
65.75	0.19466	249.250916388259\\
65.75	0.19632	257.565173229114\\
65.75	0.19798	265.87943006997\\
65.75	0.19964	274.193686910825\\
65.75	0.2013	282.50794375168\\
65.75	0.20296	290.822200592536\\
65.75	0.20462	299.136457433391\\
65.75	0.20628	307.450714274246\\
65.75	0.20794	315.764971115102\\
65.75	0.2096	324.079227955957\\
65.75	0.21126	332.393484796812\\
65.75	0.21292	340.707741637668\\
65.75	0.21458	349.021998478523\\
65.75	0.21624	357.336255319378\\
65.75	0.2179	365.650512160234\\
65.75	0.21956	373.964769001089\\
65.75	0.22122	382.279025841945\\
65.75	0.22288	390.5932826828\\
65.75	0.22454	398.907539523655\\
65.75	0.2262	407.22179636451\\
65.75	0.22786	415.536053205366\\
65.75	0.22952	423.850310046221\\
65.75	0.23118	432.164566887076\\
65.75	0.23284	440.478823727932\\
65.75	0.2345	448.793080568787\\
65.75	0.23616	457.107337409642\\
65.75	0.23782	465.421594250498\\
65.75	0.23948	473.735851091353\\
65.75	0.24114	482.050107932208\\
65.75	0.2428	490.364364773064\\
65.75	0.24446	498.678621613919\\
65.75	0.24612	506.992878454775\\
65.75	0.24778	515.30713529563\\
65.75	0.24944	523.621392136485\\
65.75	0.2511	531.93564897734\\
65.75	0.25276	540.249905818196\\
65.75	0.25442	548.564162659051\\
65.75	0.25608	556.878419499907\\
65.75	0.25774	565.192676340762\\
65.75	0.2594	573.506933181617\\
65.75	0.26106	581.821190022473\\
65.75	0.26272	590.135446863328\\
65.75	0.26438	598.449703704183\\
65.75	0.26604	606.763960545039\\
65.75	0.2677	615.078217385894\\
65.75	0.26936	623.392474226749\\
65.75	0.27102	631.706731067605\\
65.75	0.27268	640.02098790846\\
65.75	0.27434	648.335244749315\\
65.75	0.276	656.649501590171\\
65.9583333333333	0.193	241.005948886039\\
65.9583333333333	0.19466	249.365752840514\\
65.9583333333333	0.19632	257.72555679499\\
65.9583333333333	0.19798	266.085360749465\\
65.9583333333333	0.19964	274.44516470394\\
65.9583333333333	0.2013	282.804968658415\\
65.9583333333333	0.20296	291.16477261289\\
65.9583333333333	0.20462	299.524576567365\\
65.9583333333333	0.20628	307.884380521841\\
65.9583333333333	0.20794	316.244184476316\\
65.9583333333333	0.2096	324.603988430791\\
65.9583333333333	0.21126	332.963792385266\\
65.9583333333333	0.21292	341.323596339741\\
65.9583333333333	0.21458	349.683400294216\\
65.9583333333333	0.21624	358.043204248692\\
65.9583333333333	0.2179	366.403008203167\\
65.9583333333333	0.21956	374.762812157642\\
65.9583333333333	0.22122	383.122616112117\\
65.9583333333333	0.22288	391.482420066593\\
65.9583333333333	0.22454	399.842224021068\\
65.9583333333333	0.2262	408.202027975543\\
65.9583333333333	0.22786	416.561831930018\\
65.9583333333333	0.22952	424.921635884493\\
65.9583333333333	0.23118	433.281439838968\\
65.9583333333333	0.23284	441.641243793444\\
65.9583333333333	0.2345	450.001047747919\\
65.9583333333333	0.23616	458.360851702394\\
65.9583333333333	0.23782	466.720655656869\\
65.9583333333333	0.23948	475.080459611344\\
65.9583333333333	0.24114	483.440263565819\\
65.9583333333333	0.2428	491.800067520295\\
65.9583333333333	0.24446	500.15987147477\\
65.9583333333333	0.24612	508.519675429245\\
65.9583333333333	0.24778	516.87947938372\\
65.9583333333333	0.24944	525.239283338195\\
65.9583333333333	0.2511	533.59908729267\\
65.9583333333333	0.25276	541.958891247145\\
65.9583333333333	0.25442	550.318695201621\\
65.9583333333333	0.25608	558.678499156096\\
65.9583333333333	0.25774	567.038303110571\\
65.9583333333333	0.2594	575.398107065046\\
65.9583333333333	0.26106	583.757911019522\\
65.9583333333333	0.26272	592.117714973997\\
65.9583333333333	0.26438	600.477518928472\\
65.9583333333333	0.26604	608.837322882947\\
65.9583333333333	0.2677	617.197126837422\\
65.9583333333333	0.26936	625.556930791897\\
65.9583333333333	0.27102	633.916734746373\\
65.9583333333333	0.27268	642.276538700848\\
65.9583333333333	0.27434	650.636342655323\\
65.9583333333333	0.276	658.996146609798\\
66.1666666666667	0.193	241.075238224675\\
66.1666666666667	0.19466	249.48058929277\\
66.1666666666667	0.19632	257.885940360865\\
66.1666666666667	0.19798	266.29129142896\\
66.1666666666667	0.19964	274.696642497055\\
66.1666666666667	0.2013	283.10199356515\\
66.1666666666667	0.20296	291.507344633245\\
66.1666666666667	0.20462	299.91269570134\\
66.1666666666667	0.20628	308.318046769435\\
66.1666666666667	0.20794	316.72339783753\\
66.1666666666667	0.2096	325.128748905625\\
66.1666666666667	0.21126	333.53409997372\\
66.1666666666667	0.21292	341.939451041815\\
66.1666666666667	0.21458	350.34480210991\\
66.1666666666667	0.21624	358.750153178005\\
66.1666666666667	0.2179	367.1555042461\\
66.1666666666667	0.21956	375.560855314195\\
66.1666666666667	0.22122	383.96620638229\\
66.1666666666667	0.22288	392.371557450385\\
66.1666666666667	0.22454	400.77690851848\\
66.1666666666667	0.2262	409.182259586575\\
66.1666666666667	0.22786	417.58761065467\\
66.1666666666667	0.22952	425.992961722765\\
66.1666666666667	0.23118	434.39831279086\\
66.1666666666667	0.23284	442.803663858955\\
66.1666666666667	0.2345	451.20901492705\\
66.1666666666667	0.23616	459.614365995145\\
66.1666666666667	0.23782	468.01971706324\\
66.1666666666667	0.23948	476.425068131335\\
66.1666666666667	0.24114	484.83041919943\\
66.1666666666667	0.2428	493.235770267525\\
66.1666666666667	0.24446	501.64112133562\\
66.1666666666667	0.24612	510.046472403715\\
66.1666666666667	0.24778	518.45182347181\\
66.1666666666667	0.24944	526.857174539905\\
66.1666666666667	0.2511	535.262525608\\
66.1666666666667	0.25276	543.667876676095\\
66.1666666666667	0.25442	552.07322774419\\
66.1666666666667	0.25608	560.478578812285\\
66.1666666666667	0.25774	568.88392988038\\
66.1666666666667	0.2594	577.289280948475\\
66.1666666666667	0.26106	585.69463201657\\
66.1666666666667	0.26272	594.099983084666\\
66.1666666666667	0.26438	602.50533415276\\
66.1666666666667	0.26604	610.910685220856\\
66.1666666666667	0.2677	619.316036288951\\
66.1666666666667	0.26936	627.721387357045\\
66.1666666666667	0.27102	636.126738425141\\
66.1666666666667	0.27268	644.532089493235\\
66.1666666666667	0.27434	652.93744056133\\
66.1666666666667	0.276	661.342791629426\\
66.375	0.193	241.14452756331\\
66.375	0.19466	249.595425745026\\
66.375	0.19632	258.04632392674\\
66.375	0.19798	266.497222108455\\
66.375	0.19964	274.94812029017\\
66.375	0.2013	283.399018471885\\
66.375	0.20296	291.8499166536\\
66.375	0.20462	300.300814835314\\
66.375	0.20628	308.751713017029\\
66.375	0.20794	317.202611198744\\
66.375	0.2096	325.653509380459\\
66.375	0.21126	334.104407562174\\
66.375	0.21292	342.555305743889\\
66.375	0.21458	351.006203925604\\
66.375	0.21624	359.457102107319\\
66.375	0.2179	367.908000289033\\
66.375	0.21956	376.358898470748\\
66.375	0.22122	384.809796652463\\
66.375	0.22288	393.260694834178\\
66.375	0.22454	401.711593015893\\
66.375	0.2262	410.162491197608\\
66.375	0.22786	418.613389379322\\
66.375	0.22952	427.064287561037\\
66.375	0.23118	435.515185742752\\
66.375	0.23284	443.966083924467\\
66.375	0.2345	452.416982106182\\
66.375	0.23616	460.867880287896\\
66.375	0.23782	469.318778469612\\
66.375	0.23948	477.769676651327\\
66.375	0.24114	486.220574833041\\
66.375	0.2428	494.671473014756\\
66.375	0.24446	503.122371196471\\
66.375	0.24612	511.573269378186\\
66.375	0.24778	520.0241675599\\
66.375	0.24944	528.475065741615\\
66.375	0.2511	536.92596392333\\
66.375	0.25276	545.376862105045\\
66.375	0.25442	553.82776028676\\
66.375	0.25608	562.278658468475\\
66.375	0.25774	570.72955665019\\
66.375	0.2594	579.180454831904\\
66.375	0.26106	587.631353013619\\
66.375	0.26272	596.082251195334\\
66.375	0.26438	604.533149377049\\
66.375	0.26604	612.984047558764\\
66.375	0.2677	621.434945740479\\
66.375	0.26936	629.885843922194\\
66.375	0.27102	638.336742103908\\
66.375	0.27268	646.787640285623\\
66.375	0.27434	655.238538467338\\
66.375	0.276	663.689436649053\\
66.5833333333333	0.193	241.213816901946\\
66.5833333333333	0.19466	249.710262197281\\
66.5833333333333	0.19632	258.206707492616\\
66.5833333333333	0.19798	266.703152787951\\
66.5833333333333	0.19964	275.199598083285\\
66.5833333333333	0.2013	283.69604337862\\
66.5833333333333	0.20296	292.192488673954\\
66.5833333333333	0.20462	300.688933969289\\
66.5833333333333	0.20628	309.185379264624\\
66.5833333333333	0.20794	317.681824559958\\
66.5833333333333	0.2096	326.178269855293\\
66.5833333333333	0.21126	334.674715150628\\
66.5833333333333	0.21292	343.171160445962\\
66.5833333333333	0.21458	351.667605741297\\
66.5833333333333	0.21624	360.164051036632\\
66.5833333333333	0.2179	368.660496331967\\
66.5833333333333	0.21956	377.156941627301\\
66.5833333333333	0.22122	385.653386922636\\
66.5833333333333	0.22288	394.149832217971\\
66.5833333333333	0.22454	402.646277513305\\
66.5833333333333	0.2262	411.14272280864\\
66.5833333333333	0.22786	419.639168103975\\
66.5833333333333	0.22952	428.135613399309\\
66.5833333333333	0.23118	436.632058694644\\
66.5833333333333	0.23284	445.128503989979\\
66.5833333333333	0.2345	453.624949285314\\
66.5833333333333	0.23616	462.121394580648\\
66.5833333333333	0.23782	470.617839875983\\
66.5833333333333	0.23948	479.114285171317\\
66.5833333333333	0.24114	487.610730466652\\
66.5833333333333	0.2428	496.107175761987\\
66.5833333333333	0.24446	504.603621057321\\
66.5833333333333	0.24612	513.100066352656\\
66.5833333333333	0.24778	521.596511647991\\
66.5833333333333	0.24944	530.092956943326\\
66.5833333333333	0.2511	538.58940223866\\
66.5833333333333	0.25276	547.085847533995\\
66.5833333333333	0.25442	555.58229282933\\
66.5833333333333	0.25608	564.078738124664\\
66.5833333333333	0.25774	572.575183419999\\
66.5833333333333	0.2594	581.071628715334\\
66.5833333333333	0.26106	589.568074010668\\
66.5833333333333	0.26272	598.064519306003\\
66.5833333333333	0.26438	606.560964601338\\
66.5833333333333	0.26604	615.057409896672\\
66.5833333333333	0.2677	623.553855192007\\
66.5833333333333	0.26936	632.050300487342\\
66.5833333333333	0.27102	640.546745782676\\
66.5833333333333	0.27268	649.043191078011\\
66.5833333333333	0.27434	657.539636373346\\
66.5833333333333	0.276	666.03608166868\\
66.7916666666667	0.193	241.283106240582\\
66.7916666666667	0.19466	249.825098649537\\
66.7916666666667	0.19632	258.367091058491\\
66.7916666666667	0.19798	266.909083467446\\
66.7916666666667	0.19964	275.4510758764\\
66.7916666666667	0.2013	283.993068285355\\
66.7916666666667	0.20296	292.535060694309\\
66.7916666666667	0.20462	301.077053103264\\
66.7916666666667	0.20628	309.619045512218\\
66.7916666666667	0.20794	318.161037921173\\
66.7916666666667	0.2096	326.703030330127\\
66.7916666666667	0.21126	335.245022739082\\
66.7916666666667	0.21292	343.787015148036\\
66.7916666666667	0.21458	352.329007556991\\
66.7916666666667	0.21624	360.870999965945\\
66.7916666666667	0.2179	369.4129923749\\
66.7916666666667	0.21956	377.954984783854\\
66.7916666666667	0.22122	386.496977192809\\
66.7916666666667	0.22288	395.038969601763\\
66.7916666666667	0.22454	403.580962010718\\
66.7916666666667	0.2262	412.122954419672\\
66.7916666666667	0.22786	420.664946828627\\
66.7916666666667	0.22952	429.206939237581\\
66.7916666666667	0.23118	437.748931646536\\
66.7916666666667	0.23284	446.290924055491\\
66.7916666666667	0.2345	454.832916464445\\
66.7916666666667	0.23616	463.374908873399\\
66.7916666666667	0.23782	471.916901282354\\
66.7916666666667	0.23948	480.458893691309\\
66.7916666666667	0.24114	489.000886100263\\
66.7916666666667	0.2428	497.542878509218\\
66.7916666666667	0.24446	506.084870918172\\
66.7916666666667	0.24612	514.626863327127\\
66.7916666666667	0.24778	523.168855736081\\
66.7916666666667	0.24944	531.710848145035\\
66.7916666666667	0.2511	540.25284055399\\
66.7916666666667	0.25276	548.794832962944\\
66.7916666666667	0.25442	557.336825371899\\
66.7916666666667	0.25608	565.878817780854\\
66.7916666666667	0.25774	574.420810189808\\
66.7916666666667	0.2594	582.962802598763\\
66.7916666666667	0.26106	591.504795007717\\
66.7916666666667	0.26272	600.046787416672\\
66.7916666666667	0.26438	608.588779825626\\
66.7916666666667	0.26604	617.130772234581\\
66.7916666666667	0.2677	625.672764643536\\
66.7916666666667	0.26936	634.21475705249\\
66.7916666666667	0.27102	642.756749461445\\
66.7916666666667	0.27268	651.298741870399\\
66.7916666666667	0.27434	659.840734279353\\
66.7916666666667	0.276	668.382726688308\\
67	0.193	241.352395579217\\
67	0.19466	249.939935101792\\
67	0.19632	258.527474624366\\
67	0.19798	267.115014146941\\
67	0.19964	275.702553669515\\
67	0.2013	284.290093192089\\
67	0.20296	292.877632714663\\
67	0.20462	301.465172237238\\
67	0.20628	310.052711759812\\
67	0.20794	318.640251282387\\
67	0.2096	327.227790804961\\
67	0.21126	335.815330327535\\
67	0.21292	344.402869850109\\
67	0.21458	352.990409372684\\
67	0.21624	361.577948895258\\
67	0.2179	370.165488417833\\
67	0.21956	378.753027940407\\
67	0.22122	387.340567462981\\
67	0.22288	395.928106985556\\
67	0.22454	404.51564650813\\
67	0.2262	413.103186030704\\
67	0.22786	421.690725553279\\
67	0.22952	430.278265075853\\
67	0.23118	438.865804598428\\
67	0.23284	447.453344121002\\
67	0.2345	456.040883643576\\
67	0.23616	464.62842316615\\
67	0.23782	473.215962688725\\
67	0.23948	481.803502211299\\
67	0.24114	490.391041733873\\
67	0.2428	498.978581256448\\
67	0.24446	507.566120779022\\
67	0.24612	516.153660301597\\
67	0.24778	524.741199824171\\
67	0.24944	533.328739346745\\
67	0.2511	541.916278869319\\
67	0.25276	550.503818391894\\
67	0.25442	559.091357914468\\
67	0.25608	567.678897437042\\
67	0.25774	576.266436959617\\
67	0.2594	584.853976482191\\
67	0.26106	593.441516004766\\
67	0.26272	602.02905552734\\
67	0.26438	610.616595049914\\
67	0.26604	619.204134572489\\
67	0.2677	627.791674095063\\
67	0.26936	636.379213617637\\
67	0.27102	644.966753140212\\
67	0.27268	653.554292662786\\
67	0.27434	662.14183218536\\
67	0.276	670.729371707935\\
};
\end{axis}

\begin{axis}[%
width=4.927496cm,
height=3.050847cm,
at={(0cm,12.711864cm)},
scale only axis,
xmin=57,
xmax=67,
tick align=outside,
xlabel={$L_{cut}$},
xmajorgrids,
ymin=0.193,
ymax=0.276,
ylabel={$D_{rlx}$},
ymajorgrids,
zmin=-12.9755258490928,
zmax=120.850514492363,
zlabel={$x_1,x_4$},
zmajorgrids,
view={-140}{50},
legend style={at={(1.03,1)},anchor=north west,legend cell align=left,align=left,draw=white!15!black}
]
\addplot3[only marks,mark=*,mark options={},mark size=1.5000pt,color=mycolor1] plot table[row sep=crcr,]{%
67	0.276	36.259285565961\\
66	0.255	30.6908518556827\\
62	0.209	13.7570269936015\\
57	0.193	9.54134196861955\\
};
\addplot3[only marks,mark=*,mark options={},mark size=1.5000pt,color=black] plot table[row sep=crcr,]{%
64	0.23	24.0290939017834\\
};

\addplot3[%
surf,
opacity=0.7,
shader=interp,
colormap={mymap}{[1pt] rgb(0pt)=(0.0901961,0.239216,0.0745098); rgb(1pt)=(0.0945149,0.242058,0.0739522); rgb(2pt)=(0.0988592,0.244894,0.0733566); rgb(3pt)=(0.103229,0.247724,0.0727241); rgb(4pt)=(0.107623,0.250549,0.0720557); rgb(5pt)=(0.112043,0.253367,0.0713525); rgb(6pt)=(0.116487,0.25618,0.0706154); rgb(7pt)=(0.120956,0.258986,0.0698456); rgb(8pt)=(0.125449,0.261787,0.0690441); rgb(9pt)=(0.129967,0.264581,0.0682118); rgb(10pt)=(0.134508,0.26737,0.06735); rgb(11pt)=(0.139074,0.270152,0.0664596); rgb(12pt)=(0.143663,0.272929,0.0655416); rgb(13pt)=(0.148275,0.275699,0.0645971); rgb(14pt)=(0.152911,0.278463,0.0636271); rgb(15pt)=(0.15757,0.281221,0.0626328); rgb(16pt)=(0.162252,0.283973,0.0616151); rgb(17pt)=(0.166957,0.286719,0.060575); rgb(18pt)=(0.171685,0.289458,0.0595136); rgb(19pt)=(0.176434,0.292191,0.0584321); rgb(20pt)=(0.181207,0.294918,0.0573313); rgb(21pt)=(0.186001,0.297639,0.0562123); rgb(22pt)=(0.190817,0.300353,0.0550763); rgb(23pt)=(0.195655,0.303061,0.0539242); rgb(24pt)=(0.200514,0.305763,0.052757); rgb(25pt)=(0.205395,0.308459,0.0515759); rgb(26pt)=(0.210296,0.311149,0.0503624); rgb(27pt)=(0.215212,0.313846,0.0490067); rgb(28pt)=(0.220142,0.316548,0.0475043); rgb(29pt)=(0.22509,0.319254,0.0458704); rgb(30pt)=(0.230056,0.321962,0.0441205); rgb(31pt)=(0.235042,0.324671,0.04227); rgb(32pt)=(0.240048,0.327379,0.0403343); rgb(33pt)=(0.245078,0.330085,0.0383287); rgb(34pt)=(0.250131,0.332786,0.0362688); rgb(35pt)=(0.25521,0.335482,0.0341698); rgb(36pt)=(0.260317,0.33817,0.0320472); rgb(37pt)=(0.265451,0.340849,0.0299163); rgb(38pt)=(0.270616,0.343517,0.0277927); rgb(39pt)=(0.275813,0.346172,0.0256916); rgb(40pt)=(0.281043,0.348814,0.0236284); rgb(41pt)=(0.286307,0.35144,0.0216186); rgb(42pt)=(0.291607,0.354048,0.0196776); rgb(43pt)=(0.296945,0.356637,0.0178207); rgb(44pt)=(0.302322,0.359206,0.0160634); rgb(45pt)=(0.307739,0.361753,0.0144211); rgb(46pt)=(0.313198,0.364275,0.0129091); rgb(47pt)=(0.318701,0.366772,0.0115428); rgb(48pt)=(0.324249,0.369242,0.0103377); rgb(49pt)=(0.329843,0.371682,0.00930909); rgb(50pt)=(0.335485,0.374093,0.00847245); rgb(51pt)=(0.341176,0.376471,0.00784314); rgb(52pt)=(0.346925,0.378826,0.00732741); rgb(53pt)=(0.352735,0.381168,0.00682184); rgb(54pt)=(0.358605,0.383497,0.00632729); rgb(55pt)=(0.364532,0.385812,0.00584464); rgb(56pt)=(0.370516,0.388113,0.00537476); rgb(57pt)=(0.376552,0.390399,0.00491852); rgb(58pt)=(0.38264,0.39267,0.00447681); rgb(59pt)=(0.388777,0.394925,0.00405048); rgb(60pt)=(0.394962,0.397164,0.00364042); rgb(61pt)=(0.401191,0.399386,0.00324749); rgb(62pt)=(0.407464,0.401592,0.00287258); rgb(63pt)=(0.413777,0.40378,0.00251655); rgb(64pt)=(0.420129,0.40595,0.00218028); rgb(65pt)=(0.426518,0.408102,0.00186463); rgb(66pt)=(0.432942,0.410234,0.00157049); rgb(67pt)=(0.439399,0.412348,0.00129873); rgb(68pt)=(0.445885,0.414441,0.00105022); rgb(69pt)=(0.452401,0.416515,0.000825833); rgb(70pt)=(0.458942,0.418567,0.000626441); rgb(71pt)=(0.465508,0.420599,0.00045292); rgb(72pt)=(0.472096,0.422609,0.000306141); rgb(73pt)=(0.478704,0.424596,0.000186979); rgb(74pt)=(0.485331,0.426562,9.63073e-05); rgb(75pt)=(0.491973,0.428504,3.49981e-05); rgb(76pt)=(0.498628,0.430422,3.92506e-06); rgb(77pt)=(0.505323,0.432315,0); rgb(78pt)=(0.512206,0.434168,0); rgb(79pt)=(0.519282,0.435983,0); rgb(80pt)=(0.526529,0.437764,0); rgb(81pt)=(0.533922,0.439512,0); rgb(82pt)=(0.54144,0.441232,0); rgb(83pt)=(0.549059,0.442927,0); rgb(84pt)=(0.556756,0.444599,0); rgb(85pt)=(0.564508,0.446252,0); rgb(86pt)=(0.572292,0.447889,0); rgb(87pt)=(0.580084,0.449514,0); rgb(88pt)=(0.587863,0.451129,0); rgb(89pt)=(0.595604,0.452737,0); rgb(90pt)=(0.603284,0.454343,0); rgb(91pt)=(0.610882,0.455948,0); rgb(92pt)=(0.618373,0.457556,0); rgb(93pt)=(0.625734,0.459171,0); rgb(94pt)=(0.632943,0.460795,0); rgb(95pt)=(0.639976,0.462432,0); rgb(96pt)=(0.64681,0.464084,0); rgb(97pt)=(0.653423,0.465756,0); rgb(98pt)=(0.659791,0.46745,0); rgb(99pt)=(0.665891,0.469169,0); rgb(100pt)=(0.6717,0.470916,0); rgb(101pt)=(0.677195,0.472696,0); rgb(102pt)=(0.682353,0.47451,0); rgb(103pt)=(0.687242,0.476355,0); rgb(104pt)=(0.691952,0.478225,0); rgb(105pt)=(0.696497,0.480118,0); rgb(106pt)=(0.700887,0.482033,0); rgb(107pt)=(0.705134,0.483968,0); rgb(108pt)=(0.709251,0.485921,0); rgb(109pt)=(0.713249,0.487891,0); rgb(110pt)=(0.71714,0.489876,0); rgb(111pt)=(0.720936,0.491875,0); rgb(112pt)=(0.724649,0.493887,0); rgb(113pt)=(0.72829,0.495909,0); rgb(114pt)=(0.731872,0.49794,0); rgb(115pt)=(0.735406,0.499979,0); rgb(116pt)=(0.738904,0.502025,0); rgb(117pt)=(0.742378,0.504075,0); rgb(118pt)=(0.74584,0.506128,0); rgb(119pt)=(0.749302,0.508182,0); rgb(120pt)=(0.752775,0.510237,0); rgb(121pt)=(0.756272,0.51229,0); rgb(122pt)=(0.759804,0.514339,0); rgb(123pt)=(0.763384,0.516385,0); rgb(124pt)=(0.767022,0.518424,0); rgb(125pt)=(0.770731,0.520455,0); rgb(126pt)=(0.774523,0.522478,0); rgb(127pt)=(0.77841,0.524489,0); rgb(128pt)=(0.782391,0.526491,0); rgb(129pt)=(0.786402,0.528496,0); rgb(130pt)=(0.790431,0.530506,0); rgb(131pt)=(0.794478,0.532521,0); rgb(132pt)=(0.798541,0.534539,0); rgb(133pt)=(0.802619,0.53656,0); rgb(134pt)=(0.806712,0.538584,0); rgb(135pt)=(0.81082,0.540609,0); rgb(136pt)=(0.81494,0.542635,0); rgb(137pt)=(0.819074,0.54466,0); rgb(138pt)=(0.823219,0.546686,0); rgb(139pt)=(0.827374,0.548709,0); rgb(140pt)=(0.831541,0.55073,0); rgb(141pt)=(0.835716,0.552749,0); rgb(142pt)=(0.8399,0.554763,0); rgb(143pt)=(0.844092,0.556774,0); rgb(144pt)=(0.848292,0.558779,0); rgb(145pt)=(0.852497,0.560778,0); rgb(146pt)=(0.856708,0.562771,0); rgb(147pt)=(0.860924,0.564756,0); rgb(148pt)=(0.865143,0.566733,0); rgb(149pt)=(0.869366,0.568701,0); rgb(150pt)=(0.873592,0.57066,0); rgb(151pt)=(0.877819,0.572608,0); rgb(152pt)=(0.882047,0.574545,0); rgb(153pt)=(0.886275,0.576471,0); rgb(154pt)=(0.890659,0.578362,0); rgb(155pt)=(0.895333,0.580203,0); rgb(156pt)=(0.900258,0.581999,0); rgb(157pt)=(0.905397,0.583755,0); rgb(158pt)=(0.910711,0.585479,0); rgb(159pt)=(0.916164,0.587176,0); rgb(160pt)=(0.921717,0.588852,0); rgb(161pt)=(0.927333,0.590513,0); rgb(162pt)=(0.932974,0.592166,0); rgb(163pt)=(0.938602,0.593815,0); rgb(164pt)=(0.94418,0.595468,0); rgb(165pt)=(0.949669,0.59713,0); rgb(166pt)=(0.955033,0.598808,0); rgb(167pt)=(0.960233,0.600507,0); rgb(168pt)=(0.965232,0.602233,0); rgb(169pt)=(0.969992,0.603992,0); rgb(170pt)=(0.974475,0.605791,0); rgb(171pt)=(0.978643,0.607636,0); rgb(172pt)=(0.98246,0.609532,0); rgb(173pt)=(0.985886,0.611486,0); rgb(174pt)=(0.988885,0.613503,0); rgb(175pt)=(0.991419,0.61559,0); rgb(176pt)=(0.99345,0.617753,0); rgb(177pt)=(0.99494,0.619997,0); rgb(178pt)=(0.995851,0.622329,0); rgb(179pt)=(0.996226,0.624763,0); rgb(180pt)=(0.996512,0.627352,0); rgb(181pt)=(0.996788,0.630095,0); rgb(182pt)=(0.997053,0.632982,0); rgb(183pt)=(0.997308,0.636004,0); rgb(184pt)=(0.997552,0.639152,0); rgb(185pt)=(0.997785,0.642416,0); rgb(186pt)=(0.998006,0.645786,0); rgb(187pt)=(0.998217,0.649253,0); rgb(188pt)=(0.998416,0.652807,0); rgb(189pt)=(0.998605,0.656439,0); rgb(190pt)=(0.998781,0.660138,0); rgb(191pt)=(0.998946,0.663897,0); rgb(192pt)=(0.9991,0.667704,0); rgb(193pt)=(0.999242,0.67155,0); rgb(194pt)=(0.999372,0.675427,0); rgb(195pt)=(0.99949,0.679323,0); rgb(196pt)=(0.999596,0.68323,0); rgb(197pt)=(0.99969,0.687139,0); rgb(198pt)=(0.999771,0.691039,0); rgb(199pt)=(0.999841,0.694921,0); rgb(200pt)=(0.999898,0.698775,0); rgb(201pt)=(0.999942,0.702592,0); rgb(202pt)=(0.999974,0.706363,0); rgb(203pt)=(0.999994,0.710077,0); rgb(204pt)=(1,0.713725,0); rgb(205pt)=(1,0.717341,0); rgb(206pt)=(1,0.720963,0); rgb(207pt)=(1,0.724591,0); rgb(208pt)=(1,0.728226,0); rgb(209pt)=(1,0.731867,0); rgb(210pt)=(1,0.735514,0); rgb(211pt)=(1,0.739167,0); rgb(212pt)=(1,0.742827,0); rgb(213pt)=(1,0.746493,0); rgb(214pt)=(1,0.750165,0); rgb(215pt)=(1,0.753843,0); rgb(216pt)=(1,0.757527,0); rgb(217pt)=(1,0.761217,0); rgb(218pt)=(1,0.764913,0); rgb(219pt)=(1,0.768615,0); rgb(220pt)=(1,0.772324,0); rgb(221pt)=(1,0.776038,0); rgb(222pt)=(1,0.779758,0); rgb(223pt)=(1,0.783484,0); rgb(224pt)=(1,0.787215,0); rgb(225pt)=(1,0.790953,0); rgb(226pt)=(1,0.794696,0); rgb(227pt)=(1,0.798445,0); rgb(228pt)=(1,0.8022,0); rgb(229pt)=(1,0.805961,0); rgb(230pt)=(1,0.809727,0); rgb(231pt)=(1,0.8135,0); rgb(232pt)=(1,0.817278,0); rgb(233pt)=(1,0.821063,0); rgb(234pt)=(1,0.824854,0); rgb(235pt)=(1,0.828652,0); rgb(236pt)=(1,0.832455,0); rgb(237pt)=(1,0.836265,0); rgb(238pt)=(1,0.840081,0); rgb(239pt)=(1,0.843903,0); rgb(240pt)=(1,0.847732,0); rgb(241pt)=(1,0.851566,0); rgb(242pt)=(1,0.855406,0); rgb(243pt)=(1,0.859253,0); rgb(244pt)=(1,0.863106,0); rgb(245pt)=(1,0.866964,0); rgb(246pt)=(1,0.870829,0); rgb(247pt)=(1,0.8747,0); rgb(248pt)=(1,0.878577,0); rgb(249pt)=(1,0.88246,0); rgb(250pt)=(1,0.886349,0); rgb(251pt)=(1,0.890243,0); rgb(252pt)=(1,0.894144,0); rgb(253pt)=(1,0.898051,0); rgb(254pt)=(1,0.901964,0); rgb(255pt)=(1,0.905882,0)},
mesh/rows=49]
table[row sep=crcr,header=false] {%
%
57	0.193	9.54134196858149\\
57	0.19466	11.7675254190573\\
57	0.19632	13.9937088695328\\
57	0.19798	16.2198923200085\\
57	0.19964	18.4460757704841\\
57	0.2013	20.6722592209596\\
57	0.20296	22.8984426714354\\
57	0.20462	25.1246261219111\\
57	0.20628	27.3508095723868\\
57	0.20794	29.5769930228623\\
57	0.2096	31.8031764733379\\
57	0.21126	34.0293599238134\\
57	0.21292	36.2555433742892\\
57	0.21458	38.4817268247649\\
57	0.21624	40.7079102752406\\
57	0.2179	42.9340937257161\\
57	0.21956	45.1602771761917\\
57	0.22122	47.3864606266673\\
57	0.22288	49.6126440771429\\
57	0.22454	51.8388275276188\\
57	0.2262	54.0650109780944\\
57	0.22786	56.2911944285701\\
57	0.22952	58.5173778790456\\
57	0.23118	60.7435613295211\\
57	0.23284	62.9697447799966\\
57	0.2345	65.1959282304728\\
57	0.23616	67.4221116809483\\
57	0.23782	69.6482951314238\\
57	0.23948	71.8744785818993\\
57	0.24114	74.1006620323749\\
57	0.2428	76.3268454828506\\
57	0.24446	78.5530289333265\\
57	0.24612	80.7792123838021\\
57	0.24778	83.0053958342776\\
57	0.24944	85.2315792847533\\
57	0.2511	87.4577627352289\\
57	0.25276	89.6839461857044\\
57	0.25442	91.9101296361803\\
57	0.25608	94.1363130866557\\
57	0.25774	96.3624965371314\\
57	0.2594	98.5886799876068\\
57	0.26106	100.814863438083\\
57	0.26272	103.041046888559\\
57	0.26438	105.267230339034\\
57	0.26604	107.49341378951\\
57	0.2677	109.719597239985\\
57	0.26936	111.945780690461\\
57	0.27102	114.171964140937\\
57	0.27268	116.398147591412\\
57	0.27434	118.624331041888\\
57	0.276	120.850514492363\\
57.2083333333333	0.193	9.07224055571317\\
57.2083333333333	0.19466	11.2725596890604\\
57.2083333333333	0.19632	13.4728788224073\\
57.2083333333333	0.19798	15.6731979557543\\
57.2083333333333	0.19964	17.8735170891013\\
57.2083333333333	0.2013	20.0738362224481\\
57.2083333333333	0.20296	22.2741553557953\\
57.2083333333333	0.20462	24.4744744891423\\
57.2083333333333	0.20628	26.6747936224892\\
57.2083333333333	0.20794	28.8751127558362\\
57.2083333333333	0.2096	31.0754318891831\\
57.2083333333333	0.21126	33.27575102253\\
57.2083333333333	0.21292	35.4760701558772\\
57.2083333333333	0.21458	37.6763892892241\\
57.2083333333333	0.21624	39.8767084225711\\
57.2083333333333	0.2179	42.077027555918\\
57.2083333333333	0.21956	44.277346689265\\
57.2083333333333	0.22122	46.4776658226119\\
57.2083333333333	0.22288	48.6779849559589\\
57.2083333333333	0.22454	50.8783040893061\\
57.2083333333333	0.2262	53.0786232226531\\
57.2083333333333	0.22786	55.2789423559999\\
57.2083333333333	0.22952	57.4792614893469\\
57.2083333333333	0.23118	59.6795806226938\\
57.2083333333333	0.23284	61.8798997560408\\
57.2083333333333	0.2345	64.080218889388\\
57.2083333333333	0.23616	66.2805380227348\\
57.2083333333333	0.23782	68.4808571560819\\
57.2083333333333	0.23948	70.6811762894288\\
57.2083333333333	0.24114	72.8814954227757\\
57.2083333333333	0.2428	75.0818145561227\\
57.2083333333333	0.24446	77.2821336894699\\
57.2083333333333	0.24612	79.4824528228166\\
57.2083333333333	0.24778	81.6827719561638\\
57.2083333333333	0.24944	83.8830910895108\\
57.2083333333333	0.2511	86.0834102228578\\
57.2083333333333	0.25276	88.2837293562047\\
57.2083333333333	0.25442	90.4840484895515\\
57.2083333333333	0.25608	92.6843676228987\\
57.2083333333333	0.25774	94.8846867562456\\
57.2083333333333	0.2594	97.0850058895926\\
57.2083333333333	0.26106	99.2853250229396\\
57.2083333333333	0.26272	101.485644156287\\
57.2083333333333	0.26438	103.685963289633\\
57.2083333333333	0.26604	105.88628242298\\
57.2083333333333	0.2677	108.086601556328\\
57.2083333333333	0.26936	110.286920689675\\
57.2083333333333	0.27102	112.487239823021\\
57.2083333333333	0.27268	114.687558956368\\
57.2083333333333	0.27434	116.887878089716\\
57.2083333333333	0.276	119.088197223062\\
57.4166666666667	0.193	8.60313914284495\\
57.4166666666667	0.19466	10.7775939590634\\
57.4166666666667	0.19632	12.9520487752817\\
57.4166666666667	0.19798	15.1265035915001\\
57.4166666666667	0.19964	17.3009584077183\\
57.4166666666667	0.2013	19.4754132239366\\
57.4166666666667	0.20296	21.6498680401552\\
57.4166666666667	0.20462	23.8243228563734\\
57.4166666666667	0.20628	25.9987776725918\\
57.4166666666667	0.20794	28.1732324888101\\
57.4166666666667	0.2096	30.3476873050283\\
57.4166666666667	0.21126	32.5221421212466\\
57.4166666666667	0.21292	34.6965969374651\\
57.4166666666667	0.21458	36.8710517536833\\
57.4166666666667	0.21624	39.0455065699017\\
57.4166666666667	0.2179	41.21996138612\\
57.4166666666667	0.21956	43.3944162023382\\
57.4166666666667	0.22122	45.5688710185565\\
57.4166666666667	0.22288	47.7433258347749\\
57.4166666666667	0.22454	49.9177806509933\\
57.4166666666667	0.2262	52.0922354672116\\
57.4166666666667	0.22786	54.2666902834297\\
57.4166666666667	0.22952	56.441145099648\\
57.4166666666667	0.23118	58.6155999158666\\
57.4166666666667	0.23284	60.7900547320849\\
57.4166666666667	0.2345	62.9645095483032\\
57.4166666666667	0.23616	65.1389643645215\\
57.4166666666667	0.23782	67.3134191807401\\
57.4166666666667	0.23948	69.4878739969583\\
57.4166666666667	0.24114	71.6623288131766\\
57.4166666666667	0.2428	73.8367836293949\\
57.4166666666667	0.24446	76.0112384456131\\
57.4166666666667	0.24612	78.1856932618314\\
57.4166666666667	0.24778	80.3601480780499\\
57.4166666666667	0.24944	82.5346028942683\\
57.4166666666667	0.2511	84.7090577104866\\
57.4166666666667	0.25276	86.8835125267049\\
57.4166666666667	0.25442	89.0579673429229\\
57.4166666666667	0.25608	91.2324221591416\\
57.4166666666667	0.25774	93.4068769753599\\
57.4166666666667	0.2594	95.5813317915781\\
57.4166666666667	0.26106	97.7557866077964\\
57.4166666666667	0.26272	99.9302414240146\\
57.4166666666667	0.26438	102.104696240233\\
57.4166666666667	0.26604	104.279151056451\\
57.4166666666667	0.2677	106.45360587267\\
57.4166666666667	0.26936	108.628060688888\\
57.4166666666667	0.27102	110.802515505106\\
57.4166666666667	0.27268	112.976970321324\\
57.4166666666667	0.27434	115.151425137543\\
57.4166666666667	0.276	117.325879953761\\
57.625	0.193	8.13403772997674\\
57.625	0.19466	10.2826282290665\\
57.625	0.19632	12.4312187281562\\
57.625	0.19798	14.5798092272458\\
57.625	0.19964	16.7283997263355\\
57.625	0.2013	18.8769902254251\\
57.625	0.20296	21.025580724515\\
57.625	0.20462	23.1741712236046\\
57.625	0.20628	25.3227617226943\\
57.625	0.20794	27.4713522217838\\
57.625	0.2096	29.6199427208735\\
57.625	0.21126	31.7685332199632\\
57.625	0.21292	33.9171237190529\\
57.625	0.21458	36.0657142181426\\
57.625	0.21624	38.2143047172323\\
57.625	0.2179	40.3628952163219\\
57.625	0.21956	42.5114857154115\\
57.625	0.22122	44.6600762145011\\
57.625	0.22288	46.8086667135908\\
57.625	0.22454	48.9572572126807\\
57.625	0.2262	51.1058477117703\\
57.625	0.22786	53.2544382108599\\
57.625	0.22952	55.4030287099495\\
57.625	0.23118	57.5516192090392\\
57.625	0.23284	59.7002097081288\\
57.625	0.2345	61.8488002072187\\
57.625	0.23616	63.9973907063082\\
57.625	0.23782	66.1459812053979\\
57.625	0.23948	68.2945717044877\\
57.625	0.24114	70.4431622035772\\
57.625	0.2428	72.5917527026668\\
57.625	0.24446	74.7403432017566\\
57.625	0.24612	76.8889337008463\\
57.625	0.24778	79.0375241999359\\
57.625	0.24944	81.1861146990257\\
57.625	0.2511	83.3347051981152\\
57.625	0.25276	85.483295697205\\
57.625	0.25442	87.6318861962945\\
57.625	0.25608	89.7804766953843\\
57.625	0.25774	91.9290671944739\\
57.625	0.2594	94.0776576935637\\
57.625	0.26106	96.2262481926532\\
57.625	0.26272	98.374838691743\\
57.625	0.26438	100.523429190833\\
57.625	0.26604	102.672019689922\\
57.625	0.2677	104.820610189012\\
57.625	0.26936	106.969200688102\\
57.625	0.27102	109.117791187191\\
57.625	0.27268	111.266381686281\\
57.625	0.27434	113.414972185371\\
57.625	0.276	115.56356268446\\
57.8333333333333	0.193	7.66493631710853\\
57.8333333333333	0.19466	9.78766249906971\\
57.8333333333333	0.19632	11.9103886810307\\
57.8333333333333	0.19798	14.0331148629917\\
57.8333333333333	0.19964	16.1558410449527\\
57.8333333333333	0.2013	18.2785672269137\\
57.8333333333333	0.20296	20.4012934088748\\
57.8333333333333	0.20462	22.5240195908357\\
57.8333333333333	0.20628	24.6467457727967\\
57.8333333333333	0.20794	26.7694719547577\\
57.8333333333333	0.2096	28.8921981367187\\
57.8333333333333	0.21126	31.0149243186796\\
57.8333333333333	0.21292	33.1376505006408\\
57.8333333333333	0.21458	35.2603766826018\\
57.8333333333333	0.21624	37.3831028645628\\
57.8333333333333	0.2179	39.5058290465238\\
57.8333333333333	0.21956	41.6285552284847\\
57.8333333333333	0.22122	43.7512814104457\\
57.8333333333333	0.22288	45.8740075924068\\
57.8333333333333	0.22454	47.996733774368\\
57.8333333333333	0.2262	50.119459956329\\
57.8333333333333	0.22786	52.24218613829\\
57.8333333333333	0.22952	54.3649123202509\\
57.8333333333333	0.23118	56.4876385022118\\
57.8333333333333	0.23284	58.6103646841727\\
57.8333333333333	0.2345	60.7330908661341\\
57.8333333333333	0.23616	62.855817048095\\
57.8333333333333	0.23782	64.9785432300559\\
57.8333333333333	0.23948	67.1012694120168\\
57.8333333333333	0.24114	69.2239955939779\\
57.8333333333333	0.2428	71.3467217759387\\
57.8333333333333	0.24446	73.4694479579002\\
57.8333333333333	0.24612	75.5921741398611\\
57.8333333333333	0.24778	77.7149003218219\\
57.8333333333333	0.24944	79.837626503783\\
57.8333333333333	0.2511	81.9603526857441\\
57.8333333333333	0.25276	84.0830788677049\\
57.8333333333333	0.25442	86.205805049666\\
57.8333333333333	0.25608	88.328531231627\\
57.8333333333333	0.25774	90.4512574135879\\
57.8333333333333	0.2594	92.573983595549\\
57.8333333333333	0.26106	94.6967097775098\\
57.8333333333333	0.26272	96.8194359594713\\
57.8333333333333	0.26438	98.9421621414322\\
57.8333333333333	0.26604	101.064888323393\\
57.8333333333333	0.2677	103.187614505354\\
57.8333333333333	0.26936	105.310340687315\\
57.8333333333333	0.27102	107.433066869276\\
57.8333333333333	0.27268	109.555793051237\\
57.8333333333333	0.27434	111.678519233198\\
57.8333333333333	0.276	113.801245415159\\
58.0416666666667	0.193	7.19583490424031\\
58.0416666666667	0.19466	9.2926967690729\\
58.0416666666667	0.19632	11.3895586339052\\
58.0416666666667	0.19798	13.4864204987375\\
58.0416666666667	0.19964	15.5832823635699\\
58.0416666666667	0.2013	17.6801442284021\\
58.0416666666667	0.20296	19.7770060932347\\
58.0416666666667	0.20462	21.873867958067\\
58.0416666666667	0.20628	23.9707298228993\\
58.0416666666667	0.20794	26.0675916877317\\
58.0416666666667	0.2096	28.1644535525639\\
58.0416666666667	0.21126	30.2613154173963\\
58.0416666666667	0.21292	32.3581772822288\\
58.0416666666667	0.21458	34.4550391470611\\
58.0416666666667	0.21624	36.5519010118935\\
58.0416666666667	0.2179	38.6487628767258\\
58.0416666666667	0.21956	40.7456247415581\\
58.0416666666667	0.22122	42.8424866063904\\
58.0416666666667	0.22288	44.9393484712228\\
58.0416666666667	0.22454	47.0362103360553\\
58.0416666666667	0.2262	49.1330722008877\\
58.0416666666667	0.22786	51.22993406572\\
58.0416666666667	0.22952	53.3267959305523\\
58.0416666666667	0.23118	55.4236577953845\\
58.0416666666667	0.23284	57.5205196602168\\
58.0416666666667	0.2345	59.6173815250495\\
58.0416666666667	0.23616	61.7142433898819\\
58.0416666666667	0.23782	63.811105254714\\
58.0416666666667	0.23948	65.9079671195464\\
58.0416666666667	0.24114	68.0048289843787\\
58.0416666666667	0.2428	70.1016908492109\\
58.0416666666667	0.24446	72.1985527140437\\
58.0416666666667	0.24612	74.295414578876\\
58.0416666666667	0.24778	76.3922764437082\\
58.0416666666667	0.24944	78.4891383085405\\
58.0416666666667	0.2511	80.5860001733729\\
58.0416666666667	0.25276	82.682862038205\\
58.0416666666667	0.25442	84.7797239030376\\
58.0416666666667	0.25608	86.87658576787\\
58.0416666666667	0.25774	88.9734476327023\\
58.0416666666667	0.2594	91.0703094975345\\
58.0416666666667	0.26106	93.1671713623668\\
58.0416666666667	0.26272	95.2640332271997\\
58.0416666666667	0.26438	97.3608950920318\\
58.0416666666667	0.26604	99.4577569568639\\
58.0416666666667	0.2677	101.554618821697\\
58.0416666666667	0.26936	103.651480686529\\
58.0416666666667	0.27102	105.748342551361\\
58.0416666666667	0.27268	107.845204416194\\
58.0416666666667	0.27434	109.942066281026\\
58.0416666666667	0.276	112.038928145858\\
58.25	0.193	6.72673349137222\\
58.25	0.19466	8.7977310390761\\
58.25	0.19632	10.8687285867798\\
58.25	0.19798	12.9397261344835\\
58.25	0.19964	15.0107236821871\\
58.25	0.2013	17.0817212298907\\
58.25	0.20296	19.1527187775946\\
58.25	0.20462	21.2237163252983\\
58.25	0.20628	23.294713873002\\
58.25	0.20794	25.3657114207057\\
58.25	0.2096	27.4367089684093\\
58.25	0.21126	29.5077065161129\\
58.25	0.21292	31.5787040638168\\
58.25	0.21458	33.6497016115204\\
58.25	0.21624	35.7206991592242\\
58.25	0.2179	37.7916967069278\\
58.25	0.21956	39.8626942546315\\
58.25	0.22122	41.9336918023351\\
58.25	0.22288	44.0046893500388\\
58.25	0.22454	46.0756868977426\\
58.25	0.2262	48.1466844454462\\
58.25	0.22786	50.2176819931499\\
58.25	0.22952	52.2886795408535\\
58.25	0.23118	54.3596770885575\\
58.25	0.23284	56.4306746362612\\
58.25	0.2345	58.5016721839647\\
58.25	0.23616	60.5726697316684\\
58.25	0.23782	62.6436672793725\\
58.25	0.23948	64.7146648270759\\
58.25	0.24114	66.7856623747796\\
58.25	0.2428	68.8566599224832\\
58.25	0.24446	70.9276574701869\\
58.25	0.24612	72.9986550178905\\
58.25	0.24778	75.0696525655947\\
58.25	0.24944	77.1406501132983\\
58.25	0.2511	79.211647661002\\
58.25	0.25276	81.2826452087056\\
58.25	0.25442	83.353642756409\\
58.25	0.25608	85.4246403041132\\
58.25	0.25774	87.4956378518168\\
58.25	0.2594	89.5666353995205\\
58.25	0.26106	91.6376329472241\\
58.25	0.26272	93.7086304949275\\
58.25	0.26438	95.7796280426312\\
58.25	0.26604	97.8506255903351\\
58.25	0.2677	99.9216231380387\\
58.25	0.26936	101.992620685742\\
58.25	0.27102	104.063618233446\\
58.25	0.27268	106.13461578115\\
58.25	0.27434	108.205613328854\\
58.25	0.276	110.276610876557\\
58.4583333333333	0.193	6.257632078504\\
58.4583333333333	0.19466	8.30276530907918\\
58.4583333333333	0.19632	10.3478985396541\\
58.4583333333333	0.19798	12.3930317702293\\
58.4583333333333	0.19964	14.4381650008042\\
58.4583333333333	0.2013	16.4832982313792\\
58.4583333333333	0.20296	18.5284314619545\\
58.4583333333333	0.20462	20.5735646925294\\
58.4583333333333	0.20628	22.6186979231045\\
58.4583333333333	0.20794	24.6638311536796\\
58.4583333333333	0.2096	26.7089643842545\\
58.4583333333333	0.21126	28.7540976148294\\
58.4583333333333	0.21292	30.7992308454046\\
58.4583333333333	0.21458	32.8443640759797\\
58.4583333333333	0.21624	34.8894973065547\\
58.4583333333333	0.2179	36.9346305371297\\
58.4583333333333	0.21956	38.9797637677048\\
58.4583333333333	0.22122	41.0248969982797\\
58.4583333333333	0.22288	43.0700302288548\\
58.4583333333333	0.22454	45.1151634594301\\
58.4583333333333	0.2262	47.160296690005\\
58.4583333333333	0.22786	49.2054299205799\\
58.4583333333333	0.22952	51.250563151155\\
58.4583333333333	0.23118	53.2956963817301\\
58.4583333333333	0.23284	55.340829612305\\
58.4583333333333	0.2345	57.3859628428802\\
58.4583333333333	0.23616	59.4310960734551\\
58.4583333333333	0.23782	61.4762293040303\\
58.4583333333333	0.23948	63.5213625346053\\
58.4583333333333	0.24114	65.5664957651802\\
58.4583333333333	0.2428	67.6116289957552\\
58.4583333333333	0.24446	69.6567622263306\\
58.4583333333333	0.24612	71.7018954569055\\
58.4583333333333	0.24778	73.7470286874805\\
58.4583333333333	0.24944	75.7921619180556\\
58.4583333333333	0.2511	77.8372951486306\\
58.4583333333333	0.25276	79.8824283792055\\
58.4583333333333	0.25442	81.9275616097805\\
58.4583333333333	0.25608	83.9726948403559\\
58.4583333333333	0.25774	86.0178280709308\\
58.4583333333333	0.2594	88.0629613015058\\
58.4583333333333	0.26106	90.1080945320807\\
58.4583333333333	0.26272	92.1532277626559\\
58.4583333333333	0.26438	94.1983609932308\\
58.4583333333333	0.26604	96.243494223806\\
58.4583333333333	0.2677	98.2886274543812\\
58.4583333333333	0.26936	100.333760684956\\
58.4583333333333	0.27102	102.378893915531\\
58.4583333333333	0.27268	104.424027146106\\
58.4583333333333	0.27434	106.469160376681\\
58.4583333333333	0.276	108.514293607256\\
58.6666666666667	0.193	5.78853066563579\\
58.6666666666667	0.19466	7.80779957908237\\
58.6666666666667	0.19632	9.82706849252872\\
58.6666666666667	0.19798	11.8463374059751\\
58.6666666666667	0.19964	13.8656063194214\\
58.6666666666667	0.2013	15.8848752328678\\
58.6666666666667	0.20296	17.9041441463144\\
58.6666666666667	0.20462	19.9234130597607\\
58.6666666666667	0.20628	21.9426819732071\\
58.6666666666667	0.20794	23.9619508866534\\
58.6666666666667	0.2096	25.9812198000998\\
58.6666666666667	0.21126	28.0004887135461\\
58.6666666666667	0.21292	30.0197576269926\\
58.6666666666667	0.21458	32.039026540439\\
58.6666666666667	0.21624	34.0582954538854\\
58.6666666666667	0.2179	36.0775643673318\\
58.6666666666667	0.21956	38.096833280778\\
58.6666666666667	0.22122	40.1161021942244\\
58.6666666666667	0.22288	42.1353711076708\\
58.6666666666667	0.22454	44.1546400211174\\
58.6666666666667	0.2262	46.1739089345637\\
58.6666666666667	0.22786	48.19317784801\\
58.6666666666667	0.22952	50.2124467614564\\
58.6666666666667	0.23118	52.2317156749028\\
58.6666666666667	0.23284	54.2509845883491\\
58.6666666666667	0.2345	56.2702535017957\\
58.6666666666667	0.23616	58.2895224152421\\
58.6666666666667	0.23782	60.3087913286884\\
58.6666666666667	0.23948	62.3280602421348\\
58.6666666666667	0.24114	64.3473291555811\\
58.6666666666667	0.2428	66.3665980690275\\
58.6666666666667	0.24446	68.385866982474\\
58.6666666666667	0.24612	70.4051358959202\\
58.6666666666667	0.24778	72.4244048093667\\
58.6666666666667	0.24944	74.4436737228132\\
58.6666666666667	0.2511	76.4629426362594\\
58.6666666666667	0.25276	78.4822115497059\\
58.6666666666667	0.25442	80.5014804631521\\
58.6666666666667	0.25608	82.5207493765988\\
58.6666666666667	0.25774	84.5400182900451\\
58.6666666666667	0.2594	86.5592872034913\\
58.6666666666667	0.26106	88.5785561169378\\
58.6666666666667	0.26272	90.5978250303842\\
58.6666666666667	0.26438	92.6170939438307\\
58.6666666666667	0.26604	94.6363628572769\\
58.6666666666667	0.2677	96.6556317707234\\
58.6666666666667	0.26936	98.6749006841699\\
58.6666666666667	0.27102	100.694169597616\\
58.6666666666667	0.27268	102.713438511062\\
58.6666666666667	0.27434	104.732707424509\\
58.6666666666667	0.276	106.751976337955\\
58.875	0.193	5.31942925276758\\
58.875	0.19466	7.31283384908545\\
58.875	0.19632	9.3062384454031\\
58.875	0.19798	11.299643041721\\
58.875	0.19964	13.2930476380386\\
58.875	0.2013	15.2864522343563\\
58.875	0.20296	17.2798568306741\\
58.875	0.20462	19.2732614269919\\
58.875	0.20628	21.2666660233097\\
58.875	0.20794	23.2600706196273\\
58.875	0.2096	25.2534752159449\\
58.875	0.21126	27.2468798122626\\
58.875	0.21292	29.2402844085806\\
58.875	0.21458	31.2336890048982\\
58.875	0.21624	33.227093601216\\
58.875	0.2179	35.2204981975336\\
58.875	0.21956	37.2139027938513\\
58.875	0.22122	39.207307390169\\
58.875	0.22288	41.2007119864868\\
58.875	0.22454	43.1941165828048\\
58.875	0.2262	45.1875211791224\\
58.875	0.22786	47.1809257754401\\
58.875	0.22952	49.1743303717578\\
58.875	0.23118	51.1677349680754\\
58.875	0.23284	53.1611395643931\\
58.875	0.2345	55.1545441607111\\
58.875	0.23616	57.1479487570289\\
58.875	0.23782	59.1413533533464\\
58.875	0.23948	61.1347579496639\\
58.875	0.24114	63.1281625459817\\
58.875	0.2428	65.1215671422995\\
58.875	0.24446	67.1149717386174\\
58.875	0.24612	69.1083763349352\\
58.875	0.24778	71.1017809312527\\
58.875	0.24944	73.0951855275705\\
58.875	0.2511	75.0885901238883\\
58.875	0.25276	77.0819947202058\\
58.875	0.25442	79.0753993165238\\
58.875	0.25608	81.0688039128413\\
58.875	0.25774	83.0622085091591\\
58.875	0.2594	85.0556131054768\\
58.875	0.26106	87.0490177017944\\
58.875	0.26272	89.0424222981126\\
58.875	0.26438	91.0358268944303\\
58.875	0.26604	93.0292314907476\\
58.875	0.2677	95.0226360870658\\
58.875	0.26936	97.0160406833834\\
58.875	0.27102	99.0094452797011\\
58.875	0.27268	101.002849876019\\
58.875	0.27434	102.996254472336\\
58.875	0.276	104.989659068654\\
59.0833333333333	0.193	4.85032783989936\\
59.0833333333333	0.19466	6.81786811908864\\
59.0833333333333	0.19632	8.78540839827758\\
59.0833333333333	0.19798	10.7529486774667\\
59.0833333333333	0.19964	12.7204889566557\\
59.0833333333333	0.2013	14.6880292358447\\
59.0833333333333	0.20296	16.655569515034\\
59.0833333333333	0.20462	18.623109794223\\
59.0833333333333	0.20628	20.5906500734121\\
59.0833333333333	0.20794	22.5581903526012\\
59.0833333333333	0.2096	24.5257306317901\\
59.0833333333333	0.21126	26.4932709109792\\
59.0833333333333	0.21292	28.4608111901684\\
59.0833333333333	0.21458	30.4283514693574\\
59.0833333333333	0.21624	32.3958917485465\\
59.0833333333333	0.2179	34.3634320277356\\
59.0833333333333	0.21956	36.3309723069245\\
59.0833333333333	0.22122	38.2985125861136\\
59.0833333333333	0.22288	40.2660528653028\\
59.0833333333333	0.22454	42.2335931444919\\
59.0833333333333	0.2262	44.201133423681\\
59.0833333333333	0.22786	46.16867370287\\
59.0833333333333	0.22952	48.136213982059\\
59.0833333333333	0.23118	50.1037542612481\\
59.0833333333333	0.23284	52.0712945404371\\
59.0833333333333	0.2345	54.0388348196263\\
59.0833333333333	0.23616	56.0063750988154\\
59.0833333333333	0.23782	57.9739153780044\\
59.0833333333333	0.23948	59.9414556571935\\
59.0833333333333	0.24114	61.9089959363826\\
59.0833333333333	0.2428	63.8765362155716\\
59.0833333333333	0.24446	65.8440764947609\\
59.0833333333333	0.24612	67.8116167739497\\
59.0833333333333	0.24778	69.779157053139\\
59.0833333333333	0.24944	71.746697332328\\
59.0833333333333	0.2511	73.7142376115171\\
59.0833333333333	0.25276	75.6817778907061\\
59.0833333333333	0.25442	77.6493181698952\\
59.0833333333333	0.25608	79.6168584490842\\
59.0833333333333	0.25774	81.5843987282733\\
59.0833333333333	0.2594	83.5519390074624\\
59.0833333333333	0.26106	85.5194792866514\\
59.0833333333333	0.26272	87.4870195658407\\
59.0833333333333	0.26438	89.4545598450297\\
59.0833333333333	0.26604	91.4221001242186\\
59.0833333333333	0.2677	93.3896404034078\\
59.0833333333333	0.26936	95.3571806825969\\
59.0833333333333	0.27102	97.3247209617859\\
59.0833333333333	0.27268	99.292261240975\\
59.0833333333333	0.27434	101.259801520164\\
59.0833333333333	0.276	103.227341799353\\
59.2916666666667	0.193	4.38122642703115\\
59.2916666666667	0.19466	6.32290238909172\\
59.2916666666667	0.19632	8.26457835115207\\
59.2916666666667	0.19798	10.2062543132125\\
59.2916666666667	0.19964	12.1479302752729\\
59.2916666666667	0.2013	14.0896062373332\\
59.2916666666667	0.20296	16.0312821993938\\
59.2916666666667	0.20462	17.9729581614541\\
59.2916666666667	0.20628	19.9146341235146\\
59.2916666666667	0.20794	21.856310085575\\
59.2916666666667	0.2096	23.7979860476354\\
59.2916666666667	0.21126	25.7396620096957\\
59.2916666666667	0.21292	27.6813379717563\\
59.2916666666667	0.21458	29.6230139338167\\
59.2916666666667	0.21624	31.5646898958771\\
59.2916666666667	0.2179	33.5063658579375\\
59.2916666666667	0.21956	35.4480418199978\\
59.2916666666667	0.22122	37.3897177820581\\
59.2916666666667	0.22288	39.3313937441187\\
59.2916666666667	0.22454	41.2730697061792\\
59.2916666666667	0.2262	43.2147456682395\\
59.2916666666667	0.22786	45.1564216302999\\
59.2916666666667	0.22952	47.0980975923602\\
59.2916666666667	0.23118	49.0397735544209\\
59.2916666666667	0.23284	50.9814495164812\\
59.2916666666667	0.2345	52.9231254785416\\
59.2916666666667	0.23616	54.8648014406019\\
59.2916666666667	0.23782	56.8064774026627\\
59.2916666666667	0.23948	58.7481533647231\\
59.2916666666667	0.24114	60.6898293267834\\
59.2916666666667	0.2428	62.6315052888438\\
59.2916666666667	0.24446	64.5731812509041\\
59.2916666666667	0.24612	66.5148572129644\\
59.2916666666667	0.24778	68.4565331750252\\
59.2916666666667	0.24944	70.3982091370856\\
59.2916666666667	0.2511	72.3398850991459\\
59.2916666666667	0.25276	74.2815610612063\\
59.2916666666667	0.25442	76.2232370232664\\
59.2916666666667	0.25608	78.1649129853272\\
59.2916666666667	0.25774	80.1065889473875\\
59.2916666666667	0.2594	82.0482649094479\\
59.2916666666667	0.26106	83.9899408715082\\
59.2916666666667	0.26272	85.9316168335688\\
59.2916666666667	0.26438	87.8732927956291\\
59.2916666666667	0.26604	89.8149687576897\\
59.2916666666667	0.2677	91.7566447197501\\
59.2916666666667	0.26936	93.6983206818104\\
59.2916666666667	0.27102	95.6399966438707\\
59.2916666666667	0.27268	97.5816726059311\\
59.2916666666667	0.27434	99.5233485679919\\
59.2916666666667	0.276	101.465024530052\\
59.5	0.193	3.91212501416283\\
59.5	0.19466	5.8279366590948\\
59.5	0.19632	7.74374830402655\\
59.5	0.19798	9.6595599489583\\
59.5	0.19964	11.5753715938901\\
59.5	0.2013	13.4911832388217\\
59.5	0.20296	15.4069948837537\\
59.5	0.20462	17.3228065286853\\
59.5	0.20628	19.2386181736172\\
59.5	0.20794	21.1544298185488\\
59.5	0.2096	23.0702414634806\\
59.5	0.21126	24.9860531084123\\
59.5	0.21292	26.9018647533442\\
59.5	0.21458	28.8176763982759\\
59.5	0.21624	30.7334880432077\\
59.5	0.2179	32.6492996881394\\
59.5	0.21956	34.5651113330711\\
59.5	0.22122	36.4809229780028\\
59.5	0.22288	38.3967346229346\\
59.5	0.22454	40.3125462678665\\
59.5	0.2262	42.2283579127982\\
59.5	0.22786	44.1441695577299\\
59.5	0.22952	46.0599812026617\\
59.5	0.23118	47.9757928475935\\
59.5	0.23284	49.8916044925252\\
59.5	0.2345	51.807416137457\\
59.5	0.23616	53.7232277823887\\
59.5	0.23782	55.6390394273205\\
59.5	0.23948	57.5548510722522\\
59.5	0.24114	59.470662717184\\
59.5	0.2428	61.3864743621157\\
59.5	0.24446	63.3022860070475\\
59.5	0.24612	65.2180976519794\\
59.5	0.24778	67.133909296911\\
59.5	0.24944	69.0497209418429\\
59.5	0.2511	70.9655325867745\\
59.5	0.25276	72.8813442317064\\
59.5	0.25442	74.797155876638\\
59.5	0.25608	76.7129675215699\\
59.5	0.25774	78.6287791665015\\
59.5	0.2594	80.5445908114334\\
59.5	0.26106	82.460402456365\\
59.5	0.26272	84.3762141012969\\
59.5	0.26438	86.2920257462288\\
59.5	0.26604	88.2078373911604\\
59.5	0.2677	90.1236490360923\\
59.5	0.26936	92.0394606810239\\
59.5	0.27102	93.9552723259558\\
59.5	0.27268	95.8710839708874\\
59.5	0.27434	97.7868956158193\\
59.5	0.276	99.7027072607511\\
59.7083333333333	0.193	3.44302360129461\\
59.7083333333333	0.19466	5.33297092909788\\
59.7083333333333	0.19632	7.22291825690093\\
59.7083333333333	0.19798	9.11286558470408\\
59.7083333333333	0.19964	11.0028129125071\\
59.7083333333333	0.2013	12.8927602403102\\
59.7083333333333	0.20296	14.7827075681134\\
59.7083333333333	0.20462	16.6726548959165\\
59.7083333333333	0.20628	18.5626022237196\\
59.7083333333333	0.20794	20.4525495515227\\
59.7083333333333	0.2096	22.3424968793257\\
59.7083333333333	0.21126	24.2324442071288\\
59.7083333333333	0.21292	26.122391534932\\
59.7083333333333	0.21458	28.0123388627351\\
59.7083333333333	0.21624	29.9022861905382\\
59.7083333333333	0.2179	31.7922335183413\\
59.7083333333333	0.21956	33.6821808461443\\
59.7083333333333	0.22122	35.5721281739474\\
59.7083333333333	0.22288	37.4620755017504\\
59.7083333333333	0.22454	39.3520228295539\\
59.7083333333333	0.2262	41.2419701573569\\
59.7083333333333	0.22786	43.13191748516\\
59.7083333333333	0.22952	45.021864812963\\
59.7083333333333	0.23118	46.911812140766\\
59.7083333333333	0.23284	48.8017594685691\\
59.7083333333333	0.2345	50.6917067963725\\
59.7083333333333	0.23616	52.5816541241757\\
59.7083333333333	0.23782	54.4716014519786\\
59.7083333333333	0.23948	56.3615487797815\\
59.7083333333333	0.24114	58.2514961075847\\
59.7083333333333	0.2428	60.1414434353876\\
59.7083333333333	0.24446	62.0313907631912\\
59.7083333333333	0.24612	63.9213380909941\\
59.7083333333333	0.24778	65.8112854187971\\
59.7083333333333	0.24944	67.7012327466002\\
59.7083333333333	0.2511	69.5911800744034\\
59.7083333333333	0.25276	71.4811274022063\\
59.7083333333333	0.25442	73.3710747300097\\
59.7083333333333	0.25608	75.2610220578126\\
59.7083333333333	0.25774	77.1509693856158\\
59.7083333333333	0.2594	79.0409167134187\\
59.7083333333333	0.26106	80.9308640412219\\
59.7083333333333	0.26272	82.8208113690253\\
59.7083333333333	0.26438	84.7107586968284\\
59.7083333333333	0.26604	86.6007060246311\\
59.7083333333333	0.2677	88.4906533524347\\
59.7083333333333	0.26936	90.3806006802376\\
59.7083333333333	0.27102	92.2705480080408\\
59.7083333333333	0.27268	94.1604953358437\\
59.7083333333333	0.27434	96.0504426636469\\
59.7083333333333	0.276	97.9403899914498\\
59.9166666666667	0.193	2.97392218842651\\
59.9166666666667	0.19466	4.83800519910108\\
59.9166666666667	0.19632	6.70208820977552\\
59.9166666666667	0.19798	8.56617122044997\\
59.9166666666667	0.19964	10.4302542311243\\
59.9166666666667	0.2013	12.2943372417988\\
59.9166666666667	0.20296	14.1584202524733\\
59.9166666666667	0.20462	16.0225032631478\\
59.9166666666667	0.20628	17.8865862738222\\
59.9166666666667	0.20794	19.7506692844967\\
59.9166666666667	0.2096	21.614752295171\\
59.9166666666667	0.21126	23.4788353058455\\
59.9166666666667	0.21292	25.34291831652\\
59.9166666666667	0.21458	27.2070013271945\\
59.9166666666667	0.21624	29.0710843378689\\
59.9166666666667	0.2179	30.9351673485434\\
59.9166666666667	0.21956	32.7992503592177\\
59.9166666666667	0.22122	34.6633333698921\\
59.9166666666667	0.22288	36.5274163805665\\
59.9166666666667	0.22454	38.3914993912413\\
59.9166666666667	0.2262	40.2555824019157\\
59.9166666666667	0.22786	42.1196654125901\\
59.9166666666667	0.22952	43.9837484232644\\
59.9166666666667	0.23118	45.8478314339388\\
59.9166666666667	0.23284	47.7119144446131\\
59.9166666666667	0.2345	49.575997455288\\
59.9166666666667	0.23616	51.4400804659624\\
59.9166666666667	0.23782	53.3041634766366\\
59.9166666666667	0.23948	55.1682464873111\\
59.9166666666667	0.24114	57.0323294979853\\
59.9166666666667	0.2428	58.8964125086598\\
59.9166666666667	0.24446	60.7604955193347\\
59.9166666666667	0.24612	62.6245785300091\\
59.9166666666667	0.24778	64.4886615406833\\
59.9166666666667	0.24944	66.3527445513578\\
59.9166666666667	0.2511	68.2168275620322\\
59.9166666666667	0.25276	70.0809105727067\\
59.9166666666667	0.25442	71.9449935833811\\
59.9166666666667	0.25608	73.8090765940556\\
59.9166666666667	0.25774	75.67315960473\\
59.9166666666667	0.2594	77.5372426154042\\
59.9166666666667	0.26106	79.4013256260787\\
59.9166666666667	0.26272	81.2654086367536\\
59.9166666666667	0.26438	83.129491647428\\
59.9166666666667	0.26604	84.993574658102\\
59.9166666666667	0.2677	86.8576576687769\\
59.9166666666667	0.26936	88.7217406794514\\
59.9166666666667	0.27102	90.5858236901258\\
59.9166666666667	0.27268	92.4499067008001\\
59.9166666666667	0.27434	94.3139897114745\\
59.9166666666667	0.276	96.178072722149\\
60.125	0.193	2.50482077555819\\
60.125	0.19466	4.34303946910416\\
60.125	0.19632	6.1812581626499\\
60.125	0.19798	8.01947685619575\\
60.125	0.19964	9.85769554974149\\
60.125	0.2013	11.6959142432872\\
60.125	0.20296	13.5341329368332\\
60.125	0.20462	15.3723516303789\\
60.125	0.20628	17.2105703239248\\
60.125	0.20794	19.0487890174705\\
60.125	0.2096	20.8870077110163\\
60.125	0.21126	22.7252264045619\\
60.125	0.21292	24.5634450981079\\
60.125	0.21458	26.4016637916536\\
60.125	0.21624	28.2398824851995\\
60.125	0.2179	30.0781011787452\\
60.125	0.21956	31.916319872291\\
60.125	0.22122	33.7545385658367\\
60.125	0.22288	35.5927572593826\\
60.125	0.22454	37.4309759529285\\
60.125	0.2262	39.2691946464743\\
60.125	0.22786	41.10741334002\\
60.125	0.22952	42.9456320335657\\
60.125	0.23118	44.7838507271115\\
60.125	0.23284	46.6220694206572\\
60.125	0.2345	48.4602881142032\\
60.125	0.23616	50.2985068077489\\
60.125	0.23782	52.1367255012947\\
60.125	0.23948	53.9749441948404\\
60.125	0.24114	55.8131628883862\\
60.125	0.2428	57.6513815819319\\
60.125	0.24446	59.4896002754779\\
60.125	0.24612	61.3278189690236\\
60.125	0.24778	63.1660376625696\\
60.125	0.24944	65.0042563561153\\
60.125	0.2511	66.8424750496611\\
60.125	0.25276	68.6806937432068\\
60.125	0.25442	70.5189124367525\\
60.125	0.25608	72.3571311302985\\
60.125	0.25774	74.1953498238443\\
60.125	0.2594	76.03356851739\\
60.125	0.26106	77.8717872109357\\
60.125	0.26272	79.7100059044817\\
60.125	0.26438	81.5482245980274\\
60.125	0.26604	83.3864432915732\\
60.125	0.2677	85.2246619851192\\
60.125	0.26936	87.0628806786649\\
60.125	0.27102	88.9010993722104\\
60.125	0.27268	90.7393180657562\\
60.125	0.27434	92.5775367593021\\
60.125	0.276	94.4157554528479\\
60.3333333333333	0.193	2.03571936269009\\
60.3333333333333	0.19466	3.84807373910746\\
60.3333333333333	0.19632	5.6604281155245\\
60.3333333333333	0.19798	7.47278249194164\\
60.3333333333333	0.19964	9.28513686835879\\
60.3333333333333	0.2013	11.0974912447758\\
60.3333333333333	0.20296	12.9098456211931\\
60.3333333333333	0.20462	14.7221999976102\\
60.3333333333333	0.20628	16.5345543740274\\
60.3333333333333	0.20794	18.3469087504444\\
60.3333333333333	0.2096	20.1592631268616\\
60.3333333333333	0.21126	21.9716175032786\\
60.3333333333333	0.21292	23.783971879696\\
60.3333333333333	0.21458	25.596326256113\\
60.3333333333333	0.21624	27.4086806325301\\
60.3333333333333	0.2179	29.2210350089473\\
60.3333333333333	0.21956	31.0333893853643\\
60.3333333333333	0.22122	32.8457437617814\\
60.3333333333333	0.22288	34.6580981381986\\
60.3333333333333	0.22454	36.4704525146159\\
60.3333333333333	0.2262	38.2828068910329\\
60.3333333333333	0.22786	40.0951612674501\\
60.3333333333333	0.22952	41.9075156438671\\
60.3333333333333	0.23118	43.7198700202844\\
60.3333333333333	0.23284	45.5322243967014\\
60.3333333333333	0.2345	47.3445787731187\\
60.3333333333333	0.23616	49.1569331495357\\
60.3333333333333	0.23782	50.969287525953\\
60.3333333333333	0.23948	52.78164190237\\
60.3333333333333	0.24114	54.593996278787\\
60.3333333333333	0.2428	56.4063506552041\\
60.3333333333333	0.24446	58.2187050316215\\
60.3333333333333	0.24612	60.0310594080386\\
60.3333333333333	0.24778	61.8434137844556\\
60.3333333333333	0.24944	63.6557681608729\\
60.3333333333333	0.2511	65.4681225372899\\
60.3333333333333	0.25276	67.2804769137072\\
60.3333333333333	0.25442	69.0928312901242\\
60.3333333333333	0.25608	70.9051856665415\\
60.3333333333333	0.25774	72.7175400429585\\
60.3333333333333	0.2594	74.5298944193755\\
60.3333333333333	0.26106	76.3422487957926\\
60.3333333333333	0.26272	78.15460317221\\
60.3333333333333	0.26438	79.9669575486271\\
60.3333333333333	0.26604	81.7793119250441\\
60.3333333333333	0.2677	83.5916663014614\\
60.3333333333333	0.26936	85.4040206778784\\
60.3333333333333	0.27102	87.2163750542954\\
60.3333333333333	0.27268	89.0287294307127\\
60.3333333333333	0.27434	90.84108380713\\
60.3333333333333	0.276	92.653438183547\\
60.5416666666667	0.193	1.56661794982176\\
60.5416666666667	0.19466	3.35310800911043\\
60.5416666666667	0.19632	5.13959806839887\\
60.5416666666667	0.19798	6.92608812768731\\
60.5416666666667	0.19964	8.71257818697575\\
60.5416666666667	0.2013	10.4990682462642\\
60.5416666666667	0.20296	12.2855583055529\\
60.5416666666667	0.20462	14.0720483648413\\
60.5416666666667	0.20628	15.8585384241297\\
60.5416666666667	0.20794	17.6450284834182\\
60.5416666666667	0.2096	19.4315185427066\\
60.5416666666667	0.21126	21.2180086019951\\
60.5416666666667	0.21292	23.0044986612837\\
60.5416666666667	0.21458	24.7909887205722\\
60.5416666666667	0.21624	26.5774787798606\\
60.5416666666667	0.2179	28.363968839149\\
60.5416666666667	0.21956	30.1504588984375\\
60.5416666666667	0.22122	31.9369489577259\\
60.5416666666667	0.22288	33.7234390170144\\
60.5416666666667	0.22454	35.509929076303\\
60.5416666666667	0.2262	37.2964191355916\\
60.5416666666667	0.22786	39.0829091948799\\
60.5416666666667	0.22952	40.8693992541682\\
60.5416666666667	0.23118	42.6558893134568\\
60.5416666666667	0.23284	44.4423793727453\\
60.5416666666667	0.2345	46.2288694320339\\
60.5416666666667	0.23616	48.0153594913222\\
60.5416666666667	0.23782	49.8018495506108\\
60.5416666666667	0.23948	51.5883396098993\\
60.5416666666667	0.24114	53.3748296691876\\
60.5416666666667	0.2428	55.1613197284762\\
60.5416666666667	0.24446	56.9478097877648\\
60.5416666666667	0.24612	58.7342998470531\\
60.5416666666667	0.24778	60.5207899063416\\
60.5416666666667	0.24944	62.3072799656302\\
60.5416666666667	0.2511	64.0937700249187\\
60.5416666666667	0.25276	65.8802600842071\\
60.5416666666667	0.25442	67.6667501434954\\
60.5416666666667	0.25608	69.4532402027842\\
60.5416666666667	0.25774	71.2397302620725\\
60.5416666666667	0.2594	73.0262203213611\\
60.5416666666667	0.26106	74.8127103806494\\
60.5416666666667	0.26272	76.5992004399379\\
60.5416666666667	0.26438	78.3856904992265\\
60.5416666666667	0.26604	80.1721805585148\\
60.5416666666667	0.2677	81.9586706178034\\
60.5416666666667	0.26936	83.7451606770919\\
60.5416666666667	0.27102	85.5316507363802\\
60.5416666666667	0.27268	87.3181407956688\\
60.5416666666667	0.27434	89.1046308549573\\
60.5416666666667	0.276	90.8911209142459\\
60.75	0.193	1.09751653695355\\
60.75	0.19466	2.85814227911351\\
60.75	0.19632	4.61876802127324\\
60.75	0.19798	6.3793937634332\\
60.75	0.19964	8.14001950559293\\
60.75	0.2013	9.90064524775266\\
60.75	0.20296	11.6612709899126\\
60.75	0.20462	13.4218967320725\\
60.75	0.20628	15.1825224742323\\
60.75	0.20794	16.943148216392\\
60.75	0.2096	18.7037739585518\\
60.75	0.21126	20.4643997007116\\
60.75	0.21292	22.2250254428716\\
60.75	0.21458	23.9856511850313\\
60.75	0.21624	25.7462769271912\\
60.75	0.2179	27.506902669351\\
60.75	0.21956	29.2675284115109\\
60.75	0.22122	31.0281541536706\\
60.75	0.22288	32.7887798958302\\
60.75	0.22454	34.5494056379905\\
60.75	0.2262	36.3100313801501\\
60.75	0.22786	38.07065712231\\
60.75	0.22952	39.8312828644698\\
60.75	0.23118	41.5919086066294\\
60.75	0.23284	43.352534348789\\
60.75	0.2345	45.1131600909493\\
60.75	0.23616	46.8737858331092\\
60.75	0.23782	48.6344115752688\\
60.75	0.23948	50.3950373174284\\
60.75	0.24114	52.1556630595883\\
60.75	0.2428	53.9162888017481\\
60.75	0.24446	55.6769145439082\\
60.75	0.24612	57.437540286068\\
60.75	0.24778	59.1981660282277\\
60.75	0.24944	60.9587917703875\\
60.75	0.2511	62.7194175125474\\
60.75	0.25276	64.480043254707\\
60.75	0.25442	66.240668996867\\
60.75	0.25608	68.0012947390269\\
60.75	0.25774	69.7619204811865\\
60.75	0.2594	71.5225462233464\\
60.75	0.26106	73.2831719655062\\
60.75	0.26272	75.0437977076663\\
60.75	0.26438	76.8044234498261\\
60.75	0.26604	78.5650491919855\\
60.75	0.2677	80.3256749341458\\
60.75	0.26936	82.0863006763057\\
60.75	0.27102	83.8469264184653\\
60.75	0.27268	85.6075521606251\\
60.75	0.27434	87.3681779027847\\
60.75	0.276	89.1288036449446\\
60.9583333333333	0.193	0.628415124085336\\
60.9583333333333	0.19466	2.3631765491167\\
60.9583333333333	0.19632	4.09793797414773\\
60.9583333333333	0.19798	5.83269939917898\\
60.9583333333333	0.19964	7.56746082421012\\
60.9583333333333	0.2013	9.30222224924114\\
60.9583333333333	0.20296	11.0369836742725\\
60.9583333333333	0.20462	12.7717450993036\\
60.9583333333333	0.20628	14.5065065243348\\
60.9583333333333	0.20794	16.2412679493659\\
60.9583333333333	0.2096	17.9760293743971\\
60.9583333333333	0.21126	19.7107907994281\\
60.9583333333333	0.21292	21.4455522244594\\
60.9583333333333	0.21458	23.1803136494906\\
60.9583333333333	0.21624	24.9150750745217\\
60.9583333333333	0.2179	26.6498364995529\\
60.9583333333333	0.21956	28.384597924584\\
60.9583333333333	0.22122	30.1193593496151\\
60.9583333333333	0.22288	31.8541207746463\\
60.9583333333333	0.22454	33.5888821996776\\
60.9583333333333	0.2262	35.3236436247087\\
60.9583333333333	0.22786	37.0584050497398\\
60.9583333333333	0.22952	38.7931664747709\\
60.9583333333333	0.23118	40.5279278998021\\
60.9583333333333	0.23284	42.2626893248332\\
60.9583333333333	0.2345	43.9974507498646\\
60.9583333333333	0.23616	45.7322121748957\\
60.9583333333333	0.23782	47.4669735999269\\
60.9583333333333	0.23948	49.201735024958\\
60.9583333333333	0.24114	50.9364964499891\\
60.9583333333333	0.2428	52.6712578750203\\
60.9583333333333	0.24446	54.4060193000516\\
60.9583333333333	0.24612	56.1407807250825\\
60.9583333333333	0.24778	57.8755421501139\\
60.9583333333333	0.24944	59.6103035751451\\
60.9583333333333	0.2511	61.3450650001762\\
60.9583333333333	0.25276	63.0798264252073\\
60.9583333333333	0.25442	64.8145878502385\\
60.9583333333333	0.25608	66.5493492752698\\
60.9583333333333	0.25774	68.2841107003007\\
60.9583333333333	0.2594	70.0188721253319\\
60.9583333333333	0.26106	71.753633550363\\
60.9583333333333	0.26272	73.4883949753944\\
60.9583333333333	0.26438	75.2231564004255\\
60.9583333333333	0.26604	76.9579178254567\\
60.9583333333333	0.2677	78.6926792504878\\
60.9583333333333	0.26936	80.4274406755189\\
60.9583333333333	0.27102	82.1622021005501\\
60.9583333333333	0.27268	83.8969635255812\\
60.9583333333333	0.27434	85.6317249506126\\
60.9583333333333	0.276	87.3664863756437\\
61.1666666666667	0.193	0.159313711217237\\
61.1666666666667	0.19466	1.86821081911989\\
61.1666666666667	0.19632	3.57710792702244\\
61.1666666666667	0.19798	5.28600503492498\\
61.1666666666667	0.19964	6.99490214282741\\
61.1666666666667	0.2013	8.70379925072984\\
61.1666666666667	0.20296	10.4126963586325\\
61.1666666666667	0.20462	12.1215934665349\\
61.1666666666667	0.20628	13.8304905744375\\
61.1666666666667	0.20794	15.5393876823399\\
61.1666666666667	0.2096	17.2482847902425\\
61.1666666666667	0.21126	18.9571818981449\\
61.1666666666667	0.21292	20.6660790060475\\
61.1666666666667	0.21458	22.37497611395\\
61.1666666666667	0.21624	24.0838732218525\\
61.1666666666667	0.2179	25.7927703297549\\
61.1666666666667	0.21956	27.5016674376574\\
61.1666666666667	0.22122	29.2105645455598\\
61.1666666666667	0.22288	30.9194616534626\\
61.1666666666667	0.22454	32.628358761365\\
61.1666666666667	0.2262	34.3372558692674\\
61.1666666666667	0.22786	36.0461529771699\\
61.1666666666667	0.22952	37.7550500850723\\
61.1666666666667	0.23118	39.4639471929752\\
61.1666666666667	0.23284	41.1728443008776\\
61.1666666666667	0.2345	42.88174140878\\
61.1666666666667	0.23616	44.5906385166825\\
61.1666666666667	0.23782	46.2995356245854\\
61.1666666666667	0.23948	48.0084327324878\\
61.1666666666667	0.24114	49.7173298403902\\
61.1666666666667	0.2428	51.4262269482927\\
61.1666666666667	0.24446	53.1351240561951\\
61.1666666666667	0.24612	54.8440211640975\\
61.1666666666667	0.24778	56.5529182720004\\
61.1666666666667	0.24944	58.2618153799028\\
61.1666666666667	0.2511	59.9707124878053\\
61.1666666666667	0.25276	61.6796095957077\\
61.1666666666667	0.25442	63.3885067036099\\
61.1666666666667	0.25608	65.0974038115128\\
61.1666666666667	0.25774	66.8063009194154\\
61.1666666666667	0.2594	68.5151980273179\\
61.1666666666667	0.26106	70.2240951352203\\
61.1666666666667	0.26272	71.9329922431227\\
61.1666666666667	0.26438	73.6418893510252\\
61.1666666666667	0.26604	75.3507864589278\\
61.1666666666667	0.2677	77.0596835668302\\
61.1666666666667	0.26936	78.7685806747327\\
61.1666666666667	0.27102	80.4774777826351\\
61.1666666666667	0.27268	82.1863748905375\\
61.1666666666667	0.27434	83.8952719984404\\
61.1666666666667	0.276	85.6041691063429\\
61.375	0.193	-0.309787701650976\\
61.375	0.19466	1.37324508912309\\
61.375	0.19632	3.05627787989681\\
61.375	0.19798	4.73931067067076\\
61.375	0.19964	6.42234346144448\\
61.375	0.2013	8.10537625221832\\
61.375	0.20296	9.78840904299238\\
61.375	0.20462	11.4714418337661\\
61.375	0.20628	13.1544746245401\\
61.375	0.20794	14.8375074153138\\
61.375	0.2096	16.5205402060876\\
61.375	0.21126	18.2035729968615\\
61.375	0.21292	19.8866057876354\\
61.375	0.21458	21.5696385784092\\
61.375	0.21624	23.2526713691831\\
61.375	0.2179	24.9357041599569\\
61.375	0.21956	26.6187369507307\\
61.375	0.22122	28.3017697415045\\
61.375	0.22288	29.9848025322784\\
61.375	0.22454	31.6678353230525\\
61.375	0.2262	33.3508681138262\\
61.375	0.22786	35.0339009045999\\
61.375	0.22952	36.7169336953739\\
61.375	0.23118	38.3999664861476\\
61.375	0.23284	40.0829992769216\\
61.375	0.2345	41.7660320676955\\
61.375	0.23616	43.4490648584692\\
61.375	0.23782	45.1320976492432\\
61.375	0.23948	46.8151304400169\\
61.375	0.24114	48.4981632307909\\
61.375	0.2428	50.1811960215646\\
61.375	0.24446	51.8642288123385\\
61.375	0.24612	53.5472616031125\\
61.375	0.24778	55.2302943938862\\
61.375	0.24944	56.9133271846601\\
61.375	0.2511	58.5963599754341\\
61.375	0.25276	60.2793927662078\\
61.375	0.25442	61.9624255569815\\
61.375	0.25608	63.6454583477555\\
61.375	0.25774	65.3284911385294\\
61.375	0.2594	67.0115239293032\\
61.375	0.26106	68.6945567200769\\
61.375	0.26272	70.3775895108511\\
61.375	0.26438	72.0606223016248\\
61.375	0.26604	73.7436550923985\\
61.375	0.2677	75.4266878831725\\
61.375	0.26936	77.1097206739464\\
61.375	0.27102	78.7927534647201\\
61.375	0.27268	80.4757862554939\\
61.375	0.27434	82.158819046268\\
61.375	0.276	83.8418518370418\\
61.5833333333333	0.193	-0.778889114519188\\
61.5833333333333	0.19466	0.878279359126168\\
61.5833333333333	0.19632	2.5354478327713\\
61.5833333333333	0.19798	4.19261630641654\\
61.5833333333333	0.19964	5.84978478006167\\
61.5833333333333	0.2013	7.5069532537068\\
61.5833333333333	0.20296	9.16412172735215\\
61.5833333333333	0.20462	10.8212902009973\\
61.5833333333333	0.20628	12.4784586746425\\
61.5833333333333	0.20794	14.1356271482877\\
61.5833333333333	0.2096	15.7927956219328\\
61.5833333333333	0.21126	17.4499640955779\\
61.5833333333333	0.21292	19.1071325692233\\
61.5833333333333	0.21458	20.7643010428684\\
61.5833333333333	0.21624	22.4214695165136\\
61.5833333333333	0.2179	24.0786379901589\\
61.5833333333333	0.21956	25.7358064638041\\
61.5833333333333	0.22122	27.3929749374493\\
61.5833333333333	0.22288	29.0501434110943\\
61.5833333333333	0.22454	30.7073118847397\\
61.5833333333333	0.2262	32.364480358385\\
61.5833333333333	0.22786	34.02164883203\\
61.5833333333333	0.22952	35.6788173056752\\
61.5833333333333	0.23118	37.3359857793203\\
61.5833333333333	0.23284	38.9931542529653\\
61.5833333333333	0.2345	40.650322726611\\
61.5833333333333	0.23616	42.3074912002562\\
61.5833333333333	0.23782	43.964659673901\\
61.5833333333333	0.23948	45.6218281475462\\
61.5833333333333	0.24114	47.2789966211915\\
61.5833333333333	0.2428	48.9361650948365\\
61.5833333333333	0.24446	50.5933335684822\\
61.5833333333333	0.24612	52.2505020421272\\
61.5833333333333	0.24778	53.9076705157722\\
61.5833333333333	0.24944	55.5648389894175\\
61.5833333333333	0.2511	57.2220074630627\\
61.5833333333333	0.25276	58.8791759367077\\
61.5833333333333	0.25442	60.5363444103532\\
61.5833333333333	0.25608	62.1935128839982\\
61.5833333333333	0.25774	63.8506813576435\\
61.5833333333333	0.2594	65.5078498312885\\
61.5833333333333	0.26106	67.1650183049337\\
61.5833333333333	0.26272	68.8221867785792\\
61.5833333333333	0.26438	70.4793552522244\\
61.5833333333333	0.26604	72.1365237258692\\
61.5833333333333	0.2677	73.7936921995149\\
61.5833333333333	0.26936	75.4508606731599\\
61.5833333333333	0.27102	77.1080291468052\\
61.5833333333333	0.27268	78.7651976204502\\
61.5833333333333	0.27434	80.4223660940954\\
61.5833333333333	0.276	82.0795345677407\\
61.7916666666667	0.193	-1.2479905273874\\
61.7916666666667	0.19466	0.383313629129361\\
61.7916666666667	0.19632	2.01461778564578\\
61.7916666666667	0.19798	3.64592194216243\\
61.7916666666667	0.19964	5.27722609867885\\
61.7916666666667	0.2013	6.90853025519539\\
61.7916666666667	0.20296	8.53983441171204\\
61.7916666666667	0.20462	10.1711385682286\\
61.7916666666667	0.20628	11.8024427247451\\
61.7916666666667	0.20794	13.4337468812616\\
61.7916666666667	0.2096	15.0650510377781\\
61.7916666666667	0.21126	16.6963551942946\\
61.7916666666667	0.21292	18.3276593508112\\
61.7916666666667	0.21458	19.9589635073278\\
61.7916666666667	0.21624	21.5902676638443\\
61.7916666666667	0.2179	23.221571820361\\
61.7916666666667	0.21956	24.8528759768774\\
61.7916666666667	0.22122	26.4841801333939\\
61.7916666666667	0.22288	28.1154842899102\\
61.7916666666667	0.22454	29.7467884464272\\
61.7916666666667	0.2262	31.3780926029438\\
61.7916666666667	0.22786	33.0093967594601\\
61.7916666666667	0.22952	34.6407009159766\\
61.7916666666667	0.23118	36.2720050724929\\
61.7916666666667	0.23284	37.9033092290094\\
61.7916666666667	0.2345	39.5346133855264\\
61.7916666666667	0.23616	41.165917542043\\
61.7916666666667	0.23782	42.7972216985593\\
61.7916666666667	0.23948	44.4285258550758\\
61.7916666666667	0.24114	46.0598300115921\\
61.7916666666667	0.2428	47.6911341681086\\
61.7916666666667	0.24446	49.3224383246256\\
61.7916666666667	0.24612	50.9537424811422\\
61.7916666666667	0.24778	52.5850466376585\\
61.7916666666667	0.24944	54.216350794175\\
61.7916666666667	0.2511	55.8476549506915\\
61.7916666666667	0.25276	57.4789591072081\\
61.7916666666667	0.25442	59.1102632637246\\
61.7916666666667	0.25608	60.7415674202412\\
61.7916666666667	0.25774	62.3728715767577\\
61.7916666666667	0.2594	64.0041757332742\\
61.7916666666667	0.26106	65.6354798897905\\
61.7916666666667	0.26272	67.2667840463075\\
61.7916666666667	0.26438	68.8980882028241\\
61.7916666666667	0.26604	70.5293923593404\\
61.7916666666667	0.2677	72.1606965158571\\
61.7916666666667	0.26936	73.7920006723737\\
61.7916666666667	0.27102	75.4233048288902\\
61.7916666666667	0.27268	77.0546089854067\\
61.7916666666667	0.27434	78.685913141923\\
61.7916666666667	0.276	80.3172172984396\\
62	0.193	-1.71709194025561\\
62	0.19466	-0.111652100867559\\
62	0.19632	1.49378773852027\\
62	0.19798	3.09922757790821\\
62	0.19964	4.70466741729604\\
62	0.2013	6.31010725668386\\
62	0.20296	7.91554709607192\\
62	0.20462	9.52098693545975\\
62	0.20628	11.1264267748477\\
62	0.20794	12.7318666142355\\
62	0.2096	14.3373064536233\\
62	0.21126	15.9427462930111\\
62	0.21292	17.5481861323991\\
62	0.21458	19.1536259717869\\
62	0.21624	20.7590658111749\\
62	0.2179	22.3645056505627\\
62	0.21956	23.9699454899505\\
62	0.22122	25.5753853293384\\
62	0.22288	27.1808251687264\\
62	0.22454	28.7862650081145\\
62	0.2262	30.3917048475023\\
62	0.22786	31.9971446868901\\
62	0.22952	33.6025845262777\\
62	0.23118	35.2080243656658\\
62	0.23284	36.8134642050536\\
62	0.2345	38.4189040444417\\
62	0.23616	40.0243438838295\\
62	0.23782	41.6297837232173\\
62	0.23948	43.2352235626051\\
62	0.24114	44.840663401993\\
62	0.2428	46.4461032413808\\
62	0.24446	48.0515430807689\\
62	0.24612	49.6569829201567\\
62	0.24778	51.2624227595447\\
62	0.24944	52.8678625989326\\
62	0.2511	54.4733024383204\\
62	0.25276	56.0787422777082\\
62	0.25442	57.684182117096\\
62	0.25608	59.2896219564841\\
62	0.25774	60.8950617958719\\
62	0.2594	62.5005016352598\\
62	0.26106	64.1059414746476\\
62	0.26272	65.7113813140356\\
62	0.26438	67.3168211534235\\
62	0.26604	68.9222609928113\\
62	0.2677	70.5277008321993\\
62	0.26936	72.1331406715872\\
62	0.27102	73.738580510975\\
62	0.27268	75.3440203503628\\
62	0.27434	76.9494601897509\\
62	0.276	78.5549000291387\\
62.2083333333333	0.193	-2.18619335312371\\
62.2083333333333	0.19466	-0.606617830864366\\
62.2083333333333	0.19632	0.972957691394868\\
62.2083333333333	0.19798	2.5525332136541\\
62.2083333333333	0.19964	4.13210873591322\\
62.2083333333333	0.2013	5.71168425817245\\
62.2083333333333	0.20296	7.2912597804318\\
62.2083333333333	0.20462	8.87083530269103\\
62.2083333333333	0.20628	10.4504108249503\\
62.2083333333333	0.20794	12.0299863472094\\
62.2083333333333	0.2096	13.6095618694686\\
62.2083333333333	0.21126	15.1891373917277\\
62.2083333333333	0.21292	16.7687129139872\\
62.2083333333333	0.21458	18.3482884362463\\
62.2083333333333	0.21624	19.9278639585056\\
62.2083333333333	0.2179	21.5074394807648\\
62.2083333333333	0.21956	23.0870150030239\\
62.2083333333333	0.22122	24.666590525283\\
62.2083333333333	0.22288	26.2461660475424\\
62.2083333333333	0.22454	27.8257415698017\\
62.2083333333333	0.2262	29.4053170920608\\
62.2083333333333	0.22786	30.9848926143202\\
62.2083333333333	0.22952	32.5644681365793\\
62.2083333333333	0.23118	34.1440436588384\\
62.2083333333333	0.23284	35.7236191810978\\
62.2083333333333	0.2345	37.3031947033571\\
62.2083333333333	0.23616	38.8827702256162\\
62.2083333333333	0.23782	40.4623457478756\\
62.2083333333333	0.23948	42.0419212701347\\
62.2083333333333	0.24114	43.6214967923938\\
62.2083333333333	0.2428	45.2010723146529\\
62.2083333333333	0.24446	46.7806478369125\\
62.2083333333333	0.24612	48.3602233591716\\
62.2083333333333	0.24778	49.9397988814308\\
62.2083333333333	0.24944	51.5193744036901\\
62.2083333333333	0.2511	53.0989499259492\\
62.2083333333333	0.25276	54.6785254482086\\
62.2083333333333	0.25442	56.2581009704677\\
62.2083333333333	0.25608	57.837676492727\\
62.2083333333333	0.25774	59.4172520149862\\
62.2083333333333	0.2594	60.9968275372453\\
62.2083333333333	0.26106	62.5764030595046\\
62.2083333333333	0.26272	64.155978581764\\
62.2083333333333	0.26438	65.7355541040231\\
62.2083333333333	0.26604	67.3151296262822\\
62.2083333333333	0.2677	68.8947051485416\\
62.2083333333333	0.26936	70.4742806708009\\
62.2083333333333	0.27102	72.05385619306\\
62.2083333333333	0.27268	73.6334317153191\\
62.2083333333333	0.27434	75.2130072375785\\
62.2083333333333	0.276	76.7925827598376\\
62.4166666666667	0.193	-2.65529476599204\\
62.4166666666667	0.19466	-1.10158356086129\\
62.4166666666667	0.19632	0.452127644269126\\
62.4166666666667	0.19798	2.00583884939977\\
62.4166666666667	0.19964	3.55955005453029\\
62.4166666666667	0.2013	5.11326125966082\\
62.4166666666667	0.20296	6.66697246479157\\
62.4166666666667	0.20462	8.2206836699221\\
62.4166666666667	0.20628	9.77439487505262\\
62.4166666666667	0.20794	11.3281060801831\\
62.4166666666667	0.2096	12.8818172853137\\
62.4166666666667	0.21126	14.4355284904442\\
62.4166666666667	0.21292	15.989239695575\\
62.4166666666667	0.21458	17.5429509007055\\
62.4166666666667	0.21624	19.096662105836\\
62.4166666666667	0.2179	20.6503733109665\\
62.4166666666667	0.21956	22.2040845160971\\
62.4166666666667	0.22122	23.7577957212277\\
62.4166666666667	0.22288	25.3115069263581\\
62.4166666666667	0.22454	26.865218131489\\
62.4166666666667	0.2262	28.4189293366194\\
62.4166666666667	0.22786	29.97264054175\\
62.4166666666667	0.22952	31.5263517468804\\
62.4166666666667	0.23118	33.0800629520111\\
62.4166666666667	0.23284	34.6337741571415\\
62.4166666666667	0.2345	36.1874853622724\\
62.4166666666667	0.23616	37.7411965674028\\
62.4166666666667	0.23782	39.2949077725334\\
62.4166666666667	0.23948	40.848618977664\\
62.4166666666667	0.24114	42.4023301827945\\
62.4166666666667	0.2428	43.9560413879251\\
62.4166666666667	0.24446	45.5097525930557\\
62.4166666666667	0.24612	47.0634637981861\\
62.4166666666667	0.24778	48.6171750033168\\
62.4166666666667	0.24944	50.1708862084474\\
62.4166666666667	0.2511	51.7245974135781\\
62.4166666666667	0.25276	53.2783086187085\\
62.4166666666667	0.25442	54.8320198238389\\
62.4166666666667	0.25608	56.3857310289698\\
62.4166666666667	0.25774	57.9394422341002\\
62.4166666666667	0.2594	59.4931534392308\\
62.4166666666667	0.26106	61.0468646443612\\
62.4166666666667	0.26272	62.6005758494921\\
62.4166666666667	0.26438	64.1542870546225\\
62.4166666666667	0.26604	65.7079982597529\\
62.4166666666667	0.2677	67.2617094648838\\
62.4166666666667	0.26936	68.8154206700142\\
62.4166666666667	0.27102	70.3691318751448\\
62.4166666666667	0.27268	71.9228430802752\\
62.4166666666667	0.27434	73.4765542854061\\
62.4166666666667	0.276	75.0302654905365\\
62.625	0.193	-3.12439617886037\\
62.625	0.19466	-1.59654929085821\\
62.625	0.19632	-0.0687024028563883\\
62.625	0.19798	1.45914448514554\\
62.625	0.19964	2.98699137314748\\
62.625	0.2013	4.51483826114929\\
62.625	0.20296	6.04268514915134\\
62.625	0.20462	7.57053203715327\\
62.625	0.20628	9.0983789251552\\
62.625	0.20794	10.626225813157\\
62.625	0.2096	12.1540727011588\\
62.625	0.21126	13.6819195891608\\
62.625	0.21292	15.2097664771628\\
62.625	0.21458	16.7376133651646\\
62.625	0.21624	18.2654602531668\\
62.625	0.2179	19.7933071411686\\
62.625	0.21956	21.3211540291704\\
62.625	0.22122	22.8490009171724\\
62.625	0.22288	24.3768478051741\\
62.625	0.22454	25.9046946931762\\
62.625	0.2262	27.4325415811782\\
62.625	0.22786	28.9603884691801\\
62.625	0.22952	30.4882353571818\\
62.625	0.23118	32.0160822451837\\
62.625	0.23284	33.5439291331854\\
62.625	0.2345	35.0717760211878\\
62.625	0.23616	36.5996229091897\\
62.625	0.23782	38.1274697971915\\
62.625	0.23948	39.6553166851932\\
62.625	0.24114	41.1831635731951\\
62.625	0.2428	42.711010461197\\
62.625	0.24446	44.2388573491992\\
62.625	0.24612	45.7667042372011\\
62.625	0.24778	47.2945511252028\\
62.625	0.24944	48.8223980132047\\
62.625	0.2511	50.3502449012067\\
62.625	0.25276	51.8780917892086\\
62.625	0.25442	53.4059386772105\\
62.625	0.25608	54.9337855652125\\
62.625	0.25774	56.4616324532142\\
62.625	0.2594	57.9894793412161\\
62.625	0.26106	59.517326229218\\
62.625	0.26272	61.0451731172202\\
62.625	0.26438	62.5730200052221\\
62.625	0.26604	64.1008668932238\\
62.625	0.2677	65.628713781226\\
62.625	0.26936	67.1565606692279\\
62.625	0.27102	68.6844075572299\\
62.625	0.27268	70.2122544452316\\
62.625	0.27434	71.7401013332335\\
62.625	0.276	73.2679482212354\\
62.8333333333333	0.193	-3.59349759172858\\
62.8333333333333	0.19466	-2.09151502085513\\
62.8333333333333	0.19632	-0.589532449981903\\
62.8333333333333	0.19798	0.912450120891322\\
62.8333333333333	0.19964	2.41443269176455\\
62.8333333333333	0.2013	3.91641526263777\\
62.8333333333333	0.20296	5.41839783351122\\
62.8333333333333	0.20462	6.92038040438445\\
62.8333333333333	0.20628	8.42236297525767\\
62.8333333333333	0.20794	9.92434554613089\\
62.8333333333333	0.2096	11.4263281170041\\
62.8333333333333	0.21126	12.9283106878772\\
62.8333333333333	0.21292	14.4302932587507\\
62.8333333333333	0.21458	15.9322758296239\\
62.8333333333333	0.21624	17.4342584004972\\
62.8333333333333	0.2179	18.9362409713704\\
62.8333333333333	0.21956	20.4382235422436\\
62.8333333333333	0.22122	21.9402061131168\\
62.8333333333333	0.22288	23.44218868399\\
62.8333333333333	0.22454	24.9441712548635\\
62.8333333333333	0.2262	26.4461538257367\\
62.8333333333333	0.22786	27.9481363966099\\
62.8333333333333	0.22952	29.4501189674831\\
62.8333333333333	0.23118	30.9521015383564\\
62.8333333333333	0.23284	32.4540841092296\\
62.8333333333333	0.2345	33.956066680103\\
62.8333333333333	0.23616	35.4580492509763\\
62.8333333333333	0.23782	36.9600318218495\\
62.8333333333333	0.23948	38.4620143927227\\
62.8333333333333	0.24114	39.9639969635959\\
62.8333333333333	0.2428	41.4659795344692\\
62.8333333333333	0.24446	42.9679621053426\\
62.8333333333333	0.24612	44.4699446762158\\
62.8333333333333	0.24778	45.9719272470891\\
62.8333333333333	0.24944	47.4739098179623\\
62.8333333333333	0.2511	48.9758923888355\\
62.8333333333333	0.25276	50.4778749597087\\
62.8333333333333	0.25442	51.979857530582\\
62.8333333333333	0.25608	53.4818401014554\\
62.8333333333333	0.25774	54.9838226723286\\
62.8333333333333	0.2594	56.4858052432016\\
62.8333333333333	0.26106	57.9877878140749\\
62.8333333333333	0.26272	59.4897703849483\\
62.8333333333333	0.26438	60.9917529558215\\
62.8333333333333	0.26604	62.4937355266948\\
62.8333333333333	0.2677	63.9957180975682\\
62.8333333333333	0.26936	65.4977006684414\\
62.8333333333333	0.27102	66.9996832393144\\
62.8333333333333	0.27268	68.5016658101877\\
62.8333333333333	0.27434	70.0036483810611\\
62.8333333333333	0.276	71.5056309519343\\
63.0416666666667	0.193	-4.06259900459656\\
63.0416666666667	0.19466	-2.58648075085182\\
63.0416666666667	0.19632	-1.1103624971073\\
63.0416666666667	0.19798	0.365755756637327\\
63.0416666666667	0.19964	1.84187401038184\\
63.0416666666667	0.2013	3.31799226412647\\
63.0416666666667	0.20296	4.79411051787122\\
63.0416666666667	0.20462	6.27022877161573\\
63.0416666666667	0.20628	7.74634702536036\\
63.0416666666667	0.20794	9.22246527910488\\
63.0416666666667	0.2096	10.6985835328495\\
63.0416666666667	0.21126	12.174701786594\\
63.0416666666667	0.21292	13.6508200403388\\
63.0416666666667	0.21458	15.1269382940834\\
63.0416666666667	0.21624	16.6030565478278\\
63.0416666666667	0.2179	18.0791748015724\\
63.0416666666667	0.21956	19.555293055317\\
63.0416666666667	0.22122	21.0314113090615\\
63.0416666666667	0.22288	22.5075295628064\\
63.0416666666667	0.22454	23.983647816551\\
63.0416666666667	0.2262	25.4597660702955\\
63.0416666666667	0.22786	26.93588432404\\
63.0416666666667	0.22952	28.4120025777845\\
63.0416666666667	0.23118	29.8881208315295\\
63.0416666666667	0.23284	31.364239085274\\
63.0416666666667	0.2345	32.8403573390185\\
63.0416666666667	0.23616	34.316475592763\\
63.0416666666667	0.23782	35.7925938465078\\
63.0416666666667	0.23948	37.2687121002525\\
63.0416666666667	0.24114	38.744830353997\\
63.0416666666667	0.2428	40.2209486077415\\
63.0416666666667	0.24446	41.6970668614861\\
63.0416666666667	0.24612	43.1731851152306\\
63.0416666666667	0.24778	44.6493033689756\\
63.0416666666667	0.24944	46.1254216227201\\
63.0416666666667	0.2511	47.6015398764646\\
63.0416666666667	0.25276	49.0776581302093\\
63.0416666666667	0.25442	50.5537763839534\\
63.0416666666667	0.25608	52.0298946376986\\
63.0416666666667	0.25774	53.5060128914431\\
63.0416666666667	0.2594	54.9821311451876\\
63.0416666666667	0.26106	56.4582493989321\\
63.0416666666667	0.26272	57.9343676526767\\
63.0416666666667	0.26438	59.4104859064212\\
63.0416666666667	0.26604	60.8866041601659\\
63.0416666666667	0.2677	62.3627224139104\\
63.0416666666667	0.26936	63.8388406676549\\
63.0416666666667	0.27102	65.3149589213995\\
63.0416666666667	0.27268	66.7910771751442\\
63.0416666666667	0.27434	68.2671954288892\\
63.0416666666667	0.276	69.7433136826337\\
63.25	0.193	-4.53170041746489\\
63.25	0.19466	-3.08144648084874\\
63.25	0.19632	-1.63119254423282\\
63.25	0.19798	-0.180938607616895\\
63.25	0.19964	1.26931532899903\\
63.25	0.2013	2.71956926561495\\
63.25	0.20296	4.16982320223099\\
63.25	0.20462	5.62007713884691\\
63.25	0.20628	7.07033107546295\\
63.25	0.20794	8.52058501207875\\
63.25	0.2096	9.97083894869468\\
63.25	0.21126	11.4210928853106\\
63.25	0.21292	12.8713468219266\\
63.25	0.21458	14.3216007585426\\
63.25	0.21624	15.7718546951585\\
63.25	0.2179	17.2221086317745\\
63.25	0.21956	18.6723625683903\\
63.25	0.22122	20.1226165050061\\
63.25	0.22288	21.5728704416222\\
63.25	0.22454	23.0231243782382\\
63.25	0.2262	24.4733783148542\\
63.25	0.22786	25.9236322514701\\
63.25	0.22952	27.3738861880859\\
63.25	0.23118	28.8241401247019\\
63.25	0.23284	30.2743940613177\\
63.25	0.2345	31.724647997934\\
63.25	0.23616	33.1749019345498\\
63.25	0.23782	34.6251558711658\\
63.25	0.23948	36.0754098077816\\
63.25	0.24114	37.5256637443977\\
63.25	0.2428	38.9759176810135\\
63.25	0.24446	40.4261716176295\\
63.25	0.24612	41.8764255542455\\
63.25	0.24778	43.3266794908614\\
63.25	0.24944	44.7769334274774\\
63.25	0.2511	46.2271873640934\\
63.25	0.25276	47.6774413007092\\
63.25	0.25442	49.127695237325\\
63.25	0.25608	50.5779491739411\\
63.25	0.25774	52.0282031105571\\
63.25	0.2594	53.4784570471729\\
63.25	0.26106	54.928710983789\\
63.25	0.26272	56.378964920405\\
63.25	0.26438	57.8292188570208\\
63.25	0.26604	59.2794727936366\\
63.25	0.2677	60.7297267302529\\
63.25	0.26936	62.1799806668687\\
63.25	0.27102	63.6302346034845\\
63.25	0.27268	65.0804885401005\\
63.25	0.27434	66.5307424767166\\
63.25	0.276	67.9809964133324\\
63.4583333333333	0.193	-5.0008018303331\\
63.4583333333333	0.19466	-3.57641221084566\\
63.4583333333333	0.19632	-2.15202259135833\\
63.4583333333333	0.19798	-0.727632971871003\\
63.4583333333333	0.19964	0.696756647616212\\
63.4583333333333	0.2013	2.12114626710343\\
63.4583333333333	0.20296	3.54553588659087\\
63.4583333333333	0.20462	4.96992550607808\\
63.4583333333333	0.20628	6.39431512556541\\
63.4583333333333	0.20794	7.81870474505263\\
63.4583333333333	0.2096	9.24309436453984\\
63.4583333333333	0.21126	10.6674839840271\\
63.4583333333333	0.21292	12.0918736035145\\
63.4583333333333	0.21458	13.5162632230017\\
63.4583333333333	0.21624	14.9406528424893\\
63.4583333333333	0.2179	16.3650424619764\\
63.4583333333333	0.21956	17.7894320814637\\
63.4583333333333	0.22122	19.2138217009508\\
63.4583333333333	0.22288	20.6382113204379\\
63.4583333333333	0.22454	22.0626009399257\\
63.4583333333333	0.2262	23.486990559413\\
63.4583333333333	0.22786	24.9113801789001\\
63.4583333333333	0.22952	26.3357697983874\\
63.4583333333333	0.23118	27.7601594178745\\
63.4583333333333	0.23284	29.1845490373616\\
63.4583333333333	0.2345	30.6089386568494\\
63.4583333333333	0.23616	32.0333282763365\\
63.4583333333333	0.23782	33.4577178958236\\
63.4583333333333	0.23948	34.882107515311\\
63.4583333333333	0.24114	36.3064971347981\\
63.4583333333333	0.2428	37.7308867542854\\
63.4583333333333	0.24446	39.1552763737732\\
63.4583333333333	0.24612	40.5796659932603\\
63.4583333333333	0.24778	42.0040556127474\\
63.4583333333333	0.24944	43.4284452322347\\
63.4583333333333	0.2511	44.852834851722\\
63.4583333333333	0.25276	46.2772244712091\\
63.4583333333333	0.25442	47.7016140906967\\
63.4583333333333	0.25608	49.1260037101838\\
63.4583333333333	0.25774	50.5503933296711\\
63.4583333333333	0.2594	51.9747829491585\\
63.4583333333333	0.26106	53.3991725686456\\
63.4583333333333	0.26272	54.8235621881333\\
63.4583333333333	0.26438	56.2479518076204\\
63.4583333333333	0.26604	57.6723414271075\\
63.4583333333333	0.2677	59.0967310465951\\
63.4583333333333	0.26936	60.5211206660824\\
63.4583333333333	0.27102	61.9455102855695\\
63.4583333333333	0.27268	63.3698999050569\\
63.4583333333333	0.27434	64.794289524544\\
63.4583333333333	0.276	66.2186791440313\\
63.6666666666667	0.193	-5.46990324320143\\
63.6666666666667	0.19466	-4.07137794084258\\
63.6666666666667	0.19632	-2.67285263848407\\
63.6666666666667	0.19798	-1.27432733612534\\
63.6666666666667	0.19964	0.124197966233169\\
63.6666666666667	0.2013	1.52272326859179\\
63.6666666666667	0.20296	2.92124857095052\\
63.6666666666667	0.20462	4.31977387330915\\
63.6666666666667	0.20628	5.71829917566777\\
63.6666666666667	0.20794	7.11682447802639\\
63.6666666666667	0.2096	8.51534978038501\\
63.6666666666667	0.21126	9.91387508274352\\
63.6666666666667	0.21292	11.3124003851023\\
63.6666666666667	0.21458	12.7109256874608\\
63.6666666666667	0.21624	14.1094509898196\\
63.6666666666667	0.2179	15.5079762921782\\
63.6666666666667	0.21956	16.9065015945368\\
63.6666666666667	0.22122	18.3050268968955\\
63.6666666666667	0.22288	19.7035521992539\\
63.6666666666667	0.22454	21.1020775016129\\
63.6666666666667	0.2262	22.5006028039713\\
63.6666666666667	0.22786	23.89912810633\\
63.6666666666667	0.22952	25.2976534086886\\
63.6666666666667	0.23118	26.696178711047\\
63.6666666666667	0.23284	28.0947040134056\\
63.6666666666667	0.2345	29.4932293157647\\
63.6666666666667	0.23616	30.8917546181233\\
63.6666666666667	0.23782	32.2902799204817\\
63.6666666666667	0.23948	33.6888052228403\\
63.6666666666667	0.24114	35.0873305251987\\
63.6666666666667	0.2428	36.4858558275573\\
63.6666666666667	0.24446	37.8843811299164\\
63.6666666666667	0.24612	39.282906432275\\
63.6666666666667	0.24778	40.6814317346334\\
63.6666666666667	0.24944	42.079957036992\\
63.6666666666667	0.2511	43.4784823393506\\
63.6666666666667	0.25276	44.8770076417093\\
63.6666666666667	0.25442	46.2755329440681\\
63.6666666666667	0.25608	47.6740582464265\\
63.6666666666667	0.25774	49.0725835487851\\
63.6666666666667	0.2594	50.4711088511438\\
63.6666666666667	0.26106	51.8696341535024\\
63.6666666666667	0.26272	53.2681594558612\\
63.6666666666667	0.26438	54.6666847582198\\
63.6666666666667	0.26604	56.0652100605782\\
63.6666666666667	0.2677	57.4637353629373\\
63.6666666666667	0.26936	58.8622606652957\\
63.6666666666667	0.27102	60.2607859676543\\
63.6666666666667	0.27268	61.6593112700129\\
63.6666666666667	0.27434	63.0578365723716\\
63.6666666666667	0.276	64.4563618747302\\
63.875	0.193	-5.93900465606964\\
63.875	0.19466	-4.5663436708395\\
63.875	0.19632	-3.19368268560959\\
63.875	0.19798	-1.82102170037956\\
63.875	0.19964	-0.448360715149647\\
63.875	0.2013	0.924300270080266\\
63.875	0.20296	2.29696125531041\\
63.875	0.20462	3.66962224054032\\
63.875	0.20628	5.04228322577035\\
63.875	0.20794	6.41494421100026\\
63.875	0.2096	7.78760519623017\\
63.875	0.21126	9.16026618146009\\
63.875	0.21292	10.5329271666902\\
63.875	0.21458	11.90558815192\\
63.875	0.21624	13.2782491371502\\
63.875	0.2179	14.6509101223801\\
63.875	0.21956	16.02357110761\\
63.875	0.22122	17.3962320928399\\
63.875	0.22288	18.7688930780698\\
63.875	0.22454	20.1415540633\\
63.875	0.2262	21.5142150485299\\
63.875	0.22786	22.8868760337598\\
63.875	0.22952	24.2595370189897\\
63.875	0.23118	25.6321980042198\\
63.875	0.23284	27.0048589894498\\
63.875	0.2345	28.3775199746799\\
63.875	0.23616	29.7501809599098\\
63.875	0.23782	31.1228419451397\\
63.875	0.23948	32.4955029303696\\
63.875	0.24114	33.8681639155996\\
63.875	0.2428	35.2408249008295\\
63.875	0.24446	36.6134858860596\\
63.875	0.24612	37.9861468712895\\
63.875	0.24778	39.3588078565197\\
63.875	0.24944	40.7314688417496\\
63.875	0.2511	42.1041298269795\\
63.875	0.25276	43.4767908122094\\
63.875	0.25442	44.8494517974393\\
63.875	0.25608	46.2221127826695\\
63.875	0.25774	47.5947737678994\\
63.875	0.2594	48.9674347531293\\
63.875	0.26106	50.3400957383592\\
63.875	0.26272	51.7127567235893\\
63.875	0.26438	53.0854177088193\\
63.875	0.26604	54.4580786940492\\
63.875	0.2677	55.8307396792793\\
63.875	0.26936	57.2034006645092\\
63.875	0.27102	58.5760616497391\\
63.875	0.27268	59.948722634969\\
63.875	0.27434	61.3213836201992\\
63.875	0.276	62.6940446054291\\
64.0833333333333	0.193	-6.40810606893774\\
64.0833333333333	0.19466	-5.06130940083619\\
64.0833333333333	0.19632	-3.71451273273499\\
64.0833333333333	0.19798	-2.36771606463356\\
64.0833333333333	0.19964	-1.02091939653235\\
64.0833333333333	0.2013	0.32587727156897\\
64.0833333333333	0.20296	1.6726739396704\\
64.0833333333333	0.20462	3.01947060777172\\
64.0833333333333	0.20628	4.36626727587304\\
64.0833333333333	0.20794	5.71306394397425\\
64.0833333333333	0.2096	7.05986061207557\\
64.0833333333333	0.21126	8.40665728017677\\
64.0833333333333	0.21292	9.75345394827843\\
64.0833333333333	0.21458	11.1002506163798\\
64.0833333333333	0.21624	12.4470472844807\\
64.0833333333333	0.2179	13.7938439525822\\
64.0833333333333	0.21956	15.1406406206834\\
64.0833333333333	0.22122	16.4874372887846\\
64.0833333333333	0.22288	17.8342339568862\\
64.0833333333333	0.22454	19.1810306249874\\
64.0833333333333	0.2262	20.5278272930886\\
64.0833333333333	0.22786	21.8746239611899\\
64.0833333333333	0.22952	23.2214206292913\\
64.0833333333333	0.23118	24.5682172973927\\
64.0833333333333	0.23284	25.9150139654942\\
64.0833333333333	0.2345	27.2618106335954\\
64.0833333333333	0.23616	28.6086073016966\\
64.0833333333333	0.23782	29.9554039697982\\
64.0833333333333	0.23948	31.3022006378994\\
64.0833333333333	0.24114	32.6489973060006\\
64.0833333333333	0.2428	33.9957939741018\\
64.0833333333333	0.24446	35.3425906422033\\
64.0833333333333	0.24612	36.6893873103045\\
64.0833333333333	0.24778	38.0361839784059\\
64.0833333333333	0.24944	39.3829806465073\\
64.0833333333333	0.2511	40.7297773146088\\
64.0833333333333	0.25276	42.07657398271\\
64.0833333333333	0.25442	43.423370650811\\
64.0833333333333	0.25608	44.7701673189126\\
64.0833333333333	0.25774	46.1169639870138\\
64.0833333333333	0.2594	47.4637606551153\\
64.0833333333333	0.26106	48.8105573232165\\
64.0833333333333	0.26272	50.1573539913177\\
64.0833333333333	0.26438	51.5041506594189\\
64.0833333333333	0.26604	52.8509473275203\\
64.0833333333333	0.2677	54.1977439956217\\
64.0833333333333	0.26936	55.544540663723\\
64.0833333333333	0.27102	56.8913373318242\\
64.0833333333333	0.27268	58.2381339999254\\
64.0833333333333	0.27434	59.5849306680273\\
64.0833333333333	0.276	60.9317273361285\\
64.2916666666667	0.193	-6.87720748180595\\
64.2916666666667	0.19466	-5.55627513083311\\
64.2916666666667	0.19632	-4.23534277986062\\
64.2916666666667	0.19798	-2.91441042888789\\
64.2916666666667	0.19964	-1.59347807791528\\
64.2916666666667	0.2013	-0.272545726942667\\
64.2916666666667	0.20296	1.04838662403017\\
64.2916666666667	0.20462	2.36931897500278\\
64.2916666666667	0.20628	3.69025132597551\\
64.2916666666667	0.20794	5.01118367694801\\
64.2916666666667	0.2096	6.33211602792062\\
64.2916666666667	0.21126	7.65304837889323\\
64.2916666666667	0.21292	8.97398072986596\\
64.2916666666667	0.21458	10.2949130808386\\
64.2916666666667	0.21624	11.6158454318115\\
64.2916666666667	0.2179	12.936777782784\\
64.2916666666667	0.21956	14.2577101337567\\
64.2916666666667	0.22122	15.5786424847292\\
64.2916666666667	0.22288	16.8995748357017\\
64.2916666666667	0.22454	18.2205071866749\\
64.2916666666667	0.2262	19.5414395376474\\
64.2916666666667	0.22786	20.8623718886201\\
64.2916666666667	0.22952	22.1833042395926\\
64.2916666666667	0.23118	23.5042365905651\\
64.2916666666667	0.23284	24.8251689415376\\
64.2916666666667	0.2345	26.1461012925108\\
64.2916666666667	0.23616	27.4670336434833\\
64.2916666666667	0.23782	28.7879659944558\\
64.2916666666667	0.23948	30.1088983454285\\
64.2916666666667	0.24114	31.429830696401\\
64.2916666666667	0.2428	32.7507630473738\\
64.2916666666667	0.24446	34.0716953983467\\
64.2916666666667	0.24612	35.3926277493192\\
64.2916666666667	0.24778	36.7135601002917\\
64.2916666666667	0.24944	38.0344924512644\\
64.2916666666667	0.2511	39.3554248022372\\
64.2916666666667	0.25276	40.6763571532097\\
64.2916666666667	0.25442	41.9972895041826\\
64.2916666666667	0.25608	43.3182218551551\\
64.2916666666667	0.25774	44.6391542061276\\
64.2916666666667	0.2594	45.9600865571003\\
64.2916666666667	0.26106	47.2810189080728\\
64.2916666666667	0.26272	48.601951259046\\
64.2916666666667	0.26438	49.9228836100185\\
64.2916666666667	0.26604	51.243815960991\\
64.2916666666667	0.2677	52.564748311964\\
64.2916666666667	0.26936	53.8856806629365\\
64.2916666666667	0.27102	55.2066130139092\\
64.2916666666667	0.27268	56.5275453648817\\
64.2916666666667	0.27434	57.8484777158544\\
64.2916666666667	0.276	59.1694100668269\\
64.5	0.193	-7.34630889467417\\
64.5	0.19466	-6.05124086083003\\
64.5	0.19632	-4.75617282698602\\
64.5	0.19798	-3.461104793142\\
64.5	0.19964	-2.16603675929809\\
64.5	0.2013	-0.870968725454077\\
64.5	0.20296	0.424099308390055\\
64.5	0.20462	1.71916734223396\\
64.5	0.20628	3.01423537607809\\
64.5	0.20794	4.30930340992199\\
64.5	0.2096	5.6043714437659\\
64.5	0.21126	6.89943947761003\\
64.5	0.21292	8.19450751145405\\
64.5	0.21458	9.48957554529784\\
64.5	0.21624	10.7846435791421\\
64.5	0.2179	12.0797116129861\\
64.5	0.21956	13.3747796468301\\
64.5	0.22122	14.6698476806739\\
64.5	0.22288	15.9649157145179\\
64.5	0.22454	17.2599837483622\\
64.5	0.2262	18.5550517822062\\
64.5	0.22786	19.8501198160502\\
64.5	0.22952	21.145187849894\\
64.5	0.23118	22.4402558837378\\
64.5	0.23284	23.7353239175818\\
64.5	0.2345	25.0303919514263\\
64.5	0.23616	26.3254599852701\\
64.5	0.23782	27.6205280191141\\
64.5	0.23948	28.9155960529579\\
64.5	0.24114	30.2106640868019\\
64.5	0.2428	31.5057321206459\\
64.5	0.24446	32.8008001544902\\
64.5	0.24612	34.0958681883342\\
64.5	0.24778	35.390936222178\\
64.5	0.24944	36.686004256022\\
64.5	0.2511	37.981072289866\\
64.5	0.25276	39.27614032371\\
64.5	0.25442	40.571208357554\\
64.5	0.25608	41.8662763913981\\
64.5	0.25774	43.1613444252421\\
64.5	0.2594	44.4564124590859\\
64.5	0.26106	45.7514804929299\\
64.5	0.26272	47.0465485267744\\
64.5	0.26438	48.3416165606181\\
64.5	0.26604	49.6366845944619\\
64.5	0.2677	50.9317526283064\\
64.5	0.26936	52.2268206621502\\
64.5	0.27102	53.5218886959942\\
64.5	0.27268	54.8169567298382\\
64.5	0.27434	56.112024763682\\
64.5	0.276	57.407092797526\\
64.7083333333333	0.193	-7.81541030754249\\
64.7083333333333	0.19466	-6.54620659082696\\
64.7083333333333	0.19632	-5.27700287411176\\
64.7083333333333	0.19798	-4.00779915739633\\
64.7083333333333	0.19964	-2.73859544068102\\
64.7083333333333	0.2013	-1.46939172396571\\
64.7083333333333	0.20296	-0.200188007250176\\
64.7083333333333	0.20462	1.06901570946502\\
64.7083333333333	0.20628	2.33821942618044\\
64.7083333333333	0.20794	3.60742314289575\\
64.7083333333333	0.2096	4.87662685961106\\
64.7083333333333	0.21126	6.14583057632638\\
64.7083333333333	0.21292	7.4150342930418\\
64.7083333333333	0.21458	8.68423800975711\\
64.7083333333333	0.21624	9.95344172647265\\
64.7083333333333	0.2179	11.222645443188\\
64.7083333333333	0.21956	12.4918491599033\\
64.7083333333333	0.22122	13.7610528766186\\
64.7083333333333	0.22288	15.0302565933337\\
64.7083333333333	0.22454	16.2994603100494\\
64.7083333333333	0.2262	17.5686640267647\\
64.7083333333333	0.22786	18.83786774348\\
64.7083333333333	0.22952	20.1070714601954\\
64.7083333333333	0.23118	21.3762751769104\\
64.7083333333333	0.23284	22.6454788936258\\
64.7083333333333	0.2345	23.9146826103415\\
64.7083333333333	0.23616	25.1838863270568\\
64.7083333333333	0.23782	26.4530900437719\\
64.7083333333333	0.23948	27.7222937604872\\
64.7083333333333	0.24114	28.9914974772025\\
64.7083333333333	0.2428	30.2607011939178\\
64.7083333333333	0.24446	31.5299049106336\\
64.7083333333333	0.24612	32.7991086273489\\
64.7083333333333	0.24778	34.068312344064\\
64.7083333333333	0.24944	35.3375160607793\\
64.7083333333333	0.2511	36.6067197774946\\
64.7083333333333	0.25276	37.8759234942099\\
64.7083333333333	0.25442	39.1451272109255\\
64.7083333333333	0.25608	40.4143309276408\\
64.7083333333333	0.25774	41.6835346443561\\
64.7083333333333	0.2594	42.9527383610714\\
64.7083333333333	0.26106	44.2219420777867\\
64.7083333333333	0.26272	45.4911457945022\\
64.7083333333333	0.26438	46.7603495112176\\
64.7083333333333	0.26604	48.0295532279326\\
64.7083333333333	0.2677	49.2987569446484\\
64.7083333333333	0.26936	50.5679606613637\\
64.7083333333333	0.27102	51.837164378079\\
64.7083333333333	0.27268	53.1063680947943\\
64.7083333333333	0.27434	54.3755718115096\\
64.7083333333333	0.276	55.644775528225\\
64.9166666666667	0.193	-8.28451172041059\\
64.9166666666667	0.19466	-7.04117232082376\\
64.9166666666667	0.19632	-5.79783292123716\\
64.9166666666667	0.19798	-4.55449352165044\\
64.9166666666667	0.19964	-3.31115412206373\\
64.9166666666667	0.2013	-2.06781472247712\\
64.9166666666667	0.20296	-0.824475322890294\\
64.9166666666667	0.20462	0.418864076696309\\
64.9166666666667	0.20628	1.66220347628303\\
64.9166666666667	0.20794	2.90554287586974\\
64.9166666666667	0.2096	4.14888227545634\\
64.9166666666667	0.21126	5.39222167504295\\
64.9166666666667	0.21292	6.63556107463\\
64.9166666666667	0.21458	7.87890047421661\\
64.9166666666667	0.21624	9.12223987380298\\
64.9166666666667	0.2179	10.3655792733898\\
64.9166666666667	0.21956	11.6089186729764\\
64.9166666666667	0.22122	12.852258072563\\
64.9166666666667	0.22288	14.0955974721498\\
64.9166666666667	0.22454	15.3389368717367\\
64.9166666666667	0.2262	16.5822762713233\\
64.9166666666667	0.22786	17.8256156709099\\
64.9166666666667	0.22952	19.0689550704965\\
64.9166666666667	0.23118	20.3122944700833\\
64.9166666666667	0.23284	21.5556338696701\\
64.9166666666667	0.2345	22.7989732692567\\
64.9166666666667	0.23616	24.0423126688434\\
64.9166666666667	0.23782	25.2856520684302\\
64.9166666666667	0.23948	26.528991468017\\
64.9166666666667	0.24114	27.7723308676036\\
64.9166666666667	0.2428	29.0156702671902\\
64.9166666666667	0.24446	30.2590096667768\\
64.9166666666667	0.24612	31.5023490663634\\
64.9166666666667	0.24778	32.7456884659505\\
64.9166666666667	0.24944	33.9890278655371\\
64.9166666666667	0.2511	35.2323672651237\\
64.9166666666667	0.25276	36.4757066647105\\
64.9166666666667	0.25442	37.7190460642969\\
64.9166666666667	0.25608	38.9623854638839\\
64.9166666666667	0.25774	40.2057248634706\\
64.9166666666667	0.2594	41.4490642630572\\
64.9166666666667	0.26106	42.6924036626438\\
64.9166666666667	0.26272	43.9357430622304\\
64.9166666666667	0.26438	45.179082461817\\
64.9166666666667	0.26604	46.4224218614038\\
64.9166666666667	0.2677	47.6657612609904\\
64.9166666666667	0.26936	48.9091006605772\\
64.9166666666667	0.27102	50.1524400601638\\
64.9166666666667	0.27268	51.3957794597504\\
64.9166666666667	0.27434	52.6391188593375\\
64.9166666666667	0.276	53.8824582589241\\
65.125	0.193	-8.75361313327892\\
65.125	0.19466	-7.53613805082068\\
65.125	0.19632	-6.31866296836267\\
65.125	0.19798	-5.10118788590466\\
65.125	0.19964	-3.88371280344666\\
65.125	0.2013	-2.66623772098865\\
65.125	0.20296	-1.44876263853041\\
65.125	0.20462	-0.231287556072516\\
65.125	0.20628	0.986187526385606\\
65.125	0.20794	2.20366260884362\\
65.125	0.2096	3.42113769130151\\
65.125	0.21126	4.63861277375941\\
65.125	0.21292	5.85608785621775\\
65.125	0.21458	7.07356293867565\\
65.125	0.21624	8.29103802113377\\
65.125	0.2179	9.50851310359167\\
65.125	0.21956	10.7259881860498\\
65.125	0.22122	11.9434632685077\\
65.125	0.22288	13.1609383509658\\
65.125	0.22454	14.3784134334239\\
65.125	0.2262	15.5958885158821\\
65.125	0.22786	16.8133635983399\\
65.125	0.22952	18.0308386807978\\
65.125	0.23118	19.248313763256\\
65.125	0.23284	20.4657888457139\\
65.125	0.2345	21.6832639281722\\
65.125	0.23616	22.9007390106301\\
65.125	0.23782	24.1182140930882\\
65.125	0.23948	25.3356891755461\\
65.125	0.24114	26.5531642580042\\
65.125	0.2428	27.7706393404621\\
65.125	0.24446	28.9881144229203\\
65.125	0.24612	30.2055895053784\\
65.125	0.24778	31.4230645878363\\
65.125	0.24944	32.6405396702944\\
65.125	0.2511	33.8580147527525\\
65.125	0.25276	35.0754898352104\\
65.125	0.25442	36.2929649176683\\
65.125	0.25608	37.5104400001267\\
65.125	0.25774	38.7279150825846\\
65.125	0.2594	39.9453901650425\\
65.125	0.26106	41.1628652475006\\
65.125	0.26272	42.3803403299587\\
65.125	0.26438	43.5978154124166\\
65.125	0.26604	44.8152904948747\\
65.125	0.2677	46.0327655773328\\
65.125	0.26936	47.2502406597907\\
65.125	0.27102	48.4677157422489\\
65.125	0.27268	49.6851908247068\\
65.125	0.27434	50.9026659071649\\
65.125	0.276	52.120140989623\\
65.3333333333333	0.193	-9.22271454614702\\
65.3333333333333	0.19466	-8.03110378081749\\
65.3333333333333	0.19632	-6.83949301548819\\
65.3333333333333	0.19798	-5.64788225015877\\
65.3333333333333	0.19964	-4.45627148482947\\
65.3333333333333	0.2013	-3.26466071950006\\
65.3333333333333	0.20296	-2.07304995417053\\
65.3333333333333	0.20462	-0.881439188841227\\
65.3333333333333	0.20628	0.310171576488187\\
65.3333333333333	0.20794	1.50178234181749\\
65.3333333333333	0.2096	2.6933931071469\\
65.3333333333333	0.21126	3.88500387247609\\
65.3333333333333	0.21292	5.07661463780573\\
65.3333333333333	0.21458	6.26822540313492\\
65.3333333333333	0.21624	7.45983616846434\\
65.3333333333333	0.2179	8.65144693379375\\
65.3333333333333	0.21956	9.84305769912316\\
65.3333333333333	0.22122	11.0346684644524\\
65.3333333333333	0.22288	12.2262792297818\\
65.3333333333333	0.22454	13.4178899951114\\
65.3333333333333	0.2262	14.6095007604406\\
65.3333333333333	0.22786	15.80111152577\\
65.3333333333333	0.22952	16.9927222910992\\
65.3333333333333	0.23118	18.1843330564286\\
65.3333333333333	0.23284	19.375943821758\\
65.3333333333333	0.2345	20.5675545870877\\
65.3333333333333	0.23616	21.7591653524169\\
65.3333333333333	0.23782	22.9507761177463\\
65.3333333333333	0.23948	24.1423868830757\\
65.3333333333333	0.24114	25.3339976484049\\
65.3333333333333	0.2428	26.5256084137343\\
65.3333333333333	0.24446	27.7172191790639\\
65.3333333333333	0.24612	28.9088299443931\\
65.3333333333333	0.24778	30.1004407097225\\
65.3333333333333	0.24944	31.2920514750519\\
65.3333333333333	0.2511	32.4836622403814\\
65.3333333333333	0.25276	33.6752730057108\\
65.3333333333333	0.25442	34.86688377104\\
65.3333333333333	0.25608	36.0584945363696\\
65.3333333333333	0.25774	37.2501053016988\\
65.3333333333333	0.2594	38.4417160670282\\
65.3333333333333	0.26106	39.6333268323574\\
65.3333333333333	0.26272	40.824937597687\\
65.3333333333333	0.26438	42.0165483630162\\
65.3333333333333	0.26604	43.2081591283456\\
65.3333333333333	0.2677	44.3997698936751\\
65.3333333333333	0.26936	45.5913806590045\\
65.3333333333333	0.27102	46.7829914243339\\
65.3333333333333	0.27268	47.9746021896631\\
65.3333333333333	0.27434	49.1662129549927\\
65.3333333333333	0.276	50.3578237203219\\
65.5416666666667	0.193	-9.69181595901534\\
65.5416666666667	0.19466	-8.52606951081441\\
65.5416666666667	0.19632	-7.36032306261382\\
65.5416666666667	0.19798	-6.19457661441299\\
65.5416666666667	0.19964	-5.0288301662124\\
65.5416666666667	0.2013	-3.86308371801169\\
65.5416666666667	0.20296	-2.69733726981076\\
65.5416666666667	0.20462	-1.53159082161017\\
65.5416666666667	0.20628	-0.365844373409345\\
65.5416666666667	0.20794	0.799902074791248\\
65.5416666666667	0.2096	1.96564852299184\\
65.5416666666667	0.21126	3.13139497119255\\
65.5416666666667	0.21292	4.29714141939348\\
65.5416666666667	0.21458	5.46288786759419\\
65.5416666666667	0.21624	6.6286343157949\\
65.5416666666667	0.2179	7.7943807639956\\
65.5416666666667	0.21956	8.96012721219631\\
65.5416666666667	0.22122	10.1258736603968\\
65.5416666666667	0.22288	11.2916201085977\\
65.5416666666667	0.22454	12.4573665567984\\
65.5416666666667	0.2262	13.6231130049991\\
65.5416666666667	0.22786	14.7888594531998\\
65.5416666666667	0.22952	15.9546059014006\\
65.5416666666667	0.23118	17.1203523496013\\
65.5416666666667	0.23284	18.286098797802\\
65.5416666666667	0.2345	19.4518452460029\\
65.5416666666667	0.23616	20.6175916942036\\
65.5416666666667	0.23782	21.7833381424043\\
65.5416666666667	0.23948	22.949084590605\\
65.5416666666667	0.24114	24.1148310388055\\
65.5416666666667	0.2428	25.2805774870062\\
65.5416666666667	0.24446	26.4463239352071\\
65.5416666666667	0.24612	27.6120703834079\\
65.5416666666667	0.24778	28.7778168316086\\
65.5416666666667	0.24944	29.9435632798093\\
65.5416666666667	0.2511	31.10930972801\\
65.5416666666667	0.25276	32.2750561762107\\
65.5416666666667	0.25442	33.4408026244114\\
65.5416666666667	0.25608	34.6065490726121\\
65.5416666666667	0.25774	35.7722955208128\\
65.5416666666667	0.2594	36.9380419690135\\
65.5416666666667	0.26106	38.1037884172142\\
65.5416666666667	0.26272	39.2695348654152\\
65.5416666666667	0.26438	40.4352813136156\\
65.5416666666667	0.26604	41.6010277618163\\
65.5416666666667	0.2677	42.7667742100173\\
65.5416666666667	0.26936	43.932520658218\\
65.5416666666667	0.27102	45.0982671064185\\
65.5416666666667	0.27268	46.2640135546192\\
65.5416666666667	0.27434	47.4297600028201\\
65.5416666666667	0.276	48.5955064510208\\
65.75	0.193	-10.1609173718836\\
65.75	0.19466	-9.02103524081122\\
65.75	0.19632	-7.88115310973922\\
65.75	0.19798	-6.74127097866722\\
65.75	0.19964	-5.6013888475951\\
65.75	0.2013	-4.4615067165231\\
65.75	0.20296	-3.32162458545088\\
65.75	0.20462	-2.18174245437888\\
65.75	0.20628	-1.04186032330676\\
65.75	0.20794	0.0980218077652353\\
65.75	0.2096	1.23790393883723\\
65.75	0.21126	2.37778606990923\\
65.75	0.21292	3.51766820098146\\
65.75	0.21458	4.65755033205346\\
65.75	0.21624	5.79743246312546\\
65.75	0.2179	6.93731459419746\\
65.75	0.21956	8.07719672526969\\
65.75	0.22122	9.21707885634169\\
65.75	0.22288	10.3569609874137\\
65.75	0.22454	11.4968431184859\\
65.75	0.2262	12.6367252495579\\
65.75	0.22786	13.7766073806299\\
65.75	0.22952	14.9164895117019\\
65.75	0.23118	16.0563716427741\\
65.75	0.23284	17.1962537738461\\
65.75	0.2345	18.3361359049184\\
65.75	0.23616	19.4760180359904\\
65.75	0.23782	20.6159001670624\\
65.75	0.23948	21.7557822981344\\
65.75	0.24114	22.8956644292064\\
65.75	0.2428	24.0355465602784\\
65.75	0.24446	25.1754286913506\\
65.75	0.24612	26.3153108224226\\
65.75	0.24778	27.4551929534948\\
65.75	0.24944	28.5950750845668\\
65.75	0.2511	29.7349572156388\\
65.75	0.25276	30.8748393467108\\
65.75	0.25442	32.0147214777828\\
65.75	0.25608	33.154603608855\\
65.75	0.25774	34.294485739927\\
65.75	0.2594	35.434367870999\\
65.75	0.26106	36.574250002071\\
65.75	0.26272	37.7141321331433\\
65.75	0.26438	38.8540142642155\\
65.75	0.26604	39.9938963952873\\
65.75	0.2677	41.1337785263595\\
65.75	0.26936	42.2736606574315\\
65.75	0.27102	43.4135427885035\\
65.75	0.27268	44.5534249195757\\
65.75	0.27434	45.6933070506479\\
65.75	0.276	46.8331891817199\\
65.9583333333333	0.193	-10.6300187847515\\
65.9583333333333	0.19466	-9.51600097080802\\
65.9583333333333	0.19632	-8.40198315686462\\
65.9583333333333	0.19798	-7.28796534292121\\
65.9583333333333	0.19964	-6.17394752897781\\
65.9583333333333	0.2013	-5.05992971503451\\
65.9583333333333	0.20296	-3.94591190109088\\
65.9583333333333	0.20462	-2.83189408714759\\
65.9583333333333	0.20628	-1.71787627320407\\
65.9583333333333	0.20794	-0.603858459260891\\
65.9583333333333	0.2096	0.510159354682401\\
65.9583333333333	0.21126	1.62417716862592\\
65.9583333333333	0.21292	2.73819498256967\\
65.9583333333333	0.21458	3.85221279651296\\
65.9583333333333	0.21624	4.96623061045625\\
65.9583333333333	0.2179	6.08024842439954\\
65.9583333333333	0.21956	7.19426623834283\\
65.9583333333333	0.22122	8.30828405228635\\
65.9583333333333	0.22288	9.42230186622987\\
65.9583333333333	0.22454	10.5363196801734\\
65.9583333333333	0.2262	11.6503374941167\\
65.9583333333333	0.22786	12.76435530806\\
65.9583333333333	0.22952	13.8783731220033\\
65.9583333333333	0.23118	14.992390935947\\
65.9583333333333	0.23284	16.1064087498903\\
65.9583333333333	0.2345	17.2204265638338\\
65.9583333333333	0.23616	18.3344443777771\\
65.9583333333333	0.23782	19.4484621917209\\
65.9583333333333	0.23948	20.5624800056642\\
65.9583333333333	0.24114	21.6764978196074\\
65.9583333333333	0.2428	22.7905156335507\\
65.9583333333333	0.24446	23.9045334474943\\
65.9583333333333	0.24612	25.0185512614376\\
65.9583333333333	0.24778	26.1325690753813\\
65.9583333333333	0.24944	27.2465868893246\\
65.9583333333333	0.2511	28.3606047032681\\
65.9583333333333	0.25276	29.4746225172114\\
65.9583333333333	0.25442	30.5886403311545\\
65.9583333333333	0.25608	31.7026581450982\\
65.9583333333333	0.25774	32.8166759590417\\
65.9583333333333	0.2594	33.930693772985\\
65.9583333333333	0.26106	35.0447115869283\\
65.9583333333333	0.26272	36.1587294008716\\
65.9583333333333	0.26438	37.2727472148151\\
65.9583333333333	0.26604	38.3867650287586\\
65.9583333333333	0.2677	39.5007828427019\\
65.9583333333333	0.26936	40.6148006566452\\
65.9583333333333	0.27102	41.7288184705885\\
65.9583333333333	0.27268	42.842836284532\\
65.9583333333333	0.27434	43.9568540984758\\
65.9583333333333	0.276	45.0708719124191\\
66.1666666666667	0.193	-11.09912019762\\
66.1666666666667	0.19466	-10.0109667008051\\
66.1666666666667	0.19632	-8.92281320399036\\
66.1666666666667	0.19798	-7.83465970717555\\
66.1666666666667	0.19964	-6.74650621036085\\
66.1666666666667	0.2013	-5.65835271354615\\
66.1666666666667	0.20296	-4.57019921673123\\
66.1666666666667	0.20462	-3.48204571991653\\
66.1666666666667	0.20628	-2.39389222310183\\
66.1666666666667	0.20794	-1.30573872628702\\
66.1666666666667	0.2096	-0.217585229472434\\
66.1666666666667	0.21126	0.870568267342378\\
66.1666666666667	0.21292	1.95872176415719\\
66.1666666666667	0.21458	3.046875260972\\
66.1666666666667	0.21624	4.13502875778681\\
66.1666666666667	0.2179	5.2231822546014\\
66.1666666666667	0.21956	6.31133575141598\\
66.1666666666667	0.22122	7.39948924823079\\
66.1666666666667	0.22288	8.4876427450456\\
66.1666666666667	0.22454	9.57579624186042\\
66.1666666666667	0.2262	10.6639497386752\\
66.1666666666667	0.22786	11.7521032354898\\
66.1666666666667	0.22952	12.8402567323046\\
66.1666666666667	0.23118	13.9284102291194\\
66.1666666666667	0.23284	15.016563725934\\
66.1666666666667	0.2345	16.1047172227491\\
66.1666666666667	0.23616	17.1928707195636\\
66.1666666666667	0.23782	18.2810242163785\\
66.1666666666667	0.23948	19.3691777131933\\
66.1666666666667	0.24114	20.4573312100079\\
66.1666666666667	0.2428	21.5454847068224\\
66.1666666666667	0.24446	22.6336382036375\\
66.1666666666667	0.24612	23.7217917004521\\
66.1666666666667	0.24778	24.8099451972669\\
66.1666666666667	0.24944	25.8980986940817\\
66.1666666666667	0.2511	26.9862521908965\\
66.1666666666667	0.25276	28.0744056877111\\
66.1666666666667	0.25442	29.1625591845259\\
66.1666666666667	0.25608	30.2507126813407\\
66.1666666666667	0.25774	31.3388661781555\\
66.1666666666667	0.2594	32.4270196749701\\
66.1666666666667	0.26106	33.5151731717849\\
66.1666666666667	0.26272	34.6033266685997\\
66.1666666666667	0.26438	35.6914801654145\\
66.1666666666667	0.26604	36.7796336622291\\
66.1666666666667	0.2677	37.8677871590439\\
66.1666666666667	0.26936	38.9559406558587\\
66.1666666666667	0.27102	40.0440941526733\\
66.1666666666667	0.27268	41.1322476494881\\
66.1666666666667	0.27434	42.2204011463029\\
66.1666666666667	0.276	43.3085546431178\\
66.375	0.193	-11.5682216104881\\
66.375	0.19466	-10.5059324308017\\
66.375	0.19632	-9.44364325111576\\
66.375	0.19798	-8.38135407142954\\
66.375	0.19964	-7.31906489174355\\
66.375	0.2013	-6.25677571205756\\
66.375	0.20296	-5.19448653237123\\
66.375	0.20462	-4.13219735268524\\
66.375	0.20628	-3.06990817299913\\
66.375	0.20794	-2.00761899331292\\
66.375	0.2096	-0.945329813626813\\
66.375	0.21126	0.116959366059064\\
66.375	0.21292	1.17924854574517\\
66.375	0.21458	2.24153772543127\\
66.375	0.21624	3.3038269051176\\
66.375	0.2179	4.36611608480371\\
66.375	0.21956	5.42840526448958\\
66.375	0.22122	6.49069444417569\\
66.375	0.22288	7.55298362386156\\
66.375	0.22454	8.61527280354812\\
66.375	0.2262	9.67756198323423\\
66.375	0.22786	10.7398511629201\\
66.375	0.22952	11.8021403426062\\
66.375	0.23118	12.8644295222921\\
66.375	0.23284	13.9267187019782\\
66.375	0.2345	14.9890078816647\\
66.375	0.23616	16.0512970613506\\
66.375	0.23782	17.1135862410365\\
66.375	0.23948	18.1758754207226\\
66.375	0.24114	19.2381646004087\\
66.375	0.2428	20.3004537800948\\
66.375	0.24446	21.3627429597811\\
66.375	0.24612	22.4250321394672\\
66.375	0.24778	23.4873213191531\\
66.375	0.24944	24.5496104988392\\
66.375	0.2511	25.6118996785253\\
66.375	0.25276	26.6741888582114\\
66.375	0.25442	27.7364780378975\\
66.375	0.25608	28.7987672175836\\
66.375	0.25774	29.8610563972697\\
66.375	0.2594	30.9233455769559\\
66.375	0.26106	31.9856347566417\\
66.375	0.26272	33.0479239363283\\
66.375	0.26438	34.1102131160144\\
66.375	0.26604	35.1725022957\\
66.375	0.2677	36.2347914753866\\
66.375	0.26936	37.2970806550727\\
66.375	0.27102	38.3593698347586\\
66.375	0.27268	39.4216590144447\\
66.375	0.27434	40.4839481941308\\
66.375	0.276	41.5462373738167\\
66.5833333333333	0.193	-12.0373230233564\\
66.5833333333333	0.19466	-11.0008981607988\\
66.5833333333333	0.19632	-9.96447329824139\\
66.5833333333333	0.19798	-8.92804843568388\\
66.5833333333333	0.19964	-7.89162357312648\\
66.5833333333333	0.2013	-6.8551987105692\\
66.5833333333333	0.20296	-5.81877384801157\\
66.5833333333333	0.20462	-4.78234898545418\\
66.5833333333333	0.20628	-3.74592412289678\\
66.5833333333333	0.20794	-2.70949926033927\\
66.5833333333333	0.2096	-1.67307439778187\\
66.5833333333333	0.21126	-0.636649535224478\\
66.5833333333333	0.21292	0.399775327332918\\
66.5833333333333	0.21458	1.43620018989031\\
66.5833333333333	0.21624	2.47262505244794\\
66.5833333333333	0.2179	3.50904991500533\\
66.5833333333333	0.21956	4.54547477756273\\
66.5833333333333	0.22122	5.58189964012013\\
66.5833333333333	0.22288	6.61832450267752\\
66.5833333333333	0.22454	7.65474936523538\\
66.5833333333333	0.2262	8.69117422779254\\
66.5833333333333	0.22786	9.72759909034994\\
66.5833333333333	0.22952	10.7640239529073\\
66.5833333333333	0.23118	11.8004488154647\\
66.5833333333333	0.23284	12.8368736780221\\
66.5833333333333	0.2345	13.87329854058\\
66.5833333333333	0.23616	14.9097234031374\\
66.5833333333333	0.23782	15.9461482656945\\
66.5833333333333	0.23948	16.9825731282519\\
66.5833333333333	0.24114	18.0189979908093\\
66.5833333333333	0.2428	19.0554228533667\\
66.5833333333333	0.24446	20.0918477159246\\
66.5833333333333	0.24612	21.128272578482\\
66.5833333333333	0.24778	22.1646974410392\\
66.5833333333333	0.24944	23.2011223035965\\
66.5833333333333	0.2511	24.2375471661539\\
66.5833333333333	0.25276	25.2739720287113\\
66.5833333333333	0.25442	26.310396891269\\
66.5833333333333	0.25608	27.3468217538264\\
66.5833333333333	0.25774	28.3832466163838\\
66.5833333333333	0.2594	29.4196714789412\\
66.5833333333333	0.26106	30.4560963414986\\
66.5833333333333	0.26272	31.4925212040564\\
66.5833333333333	0.26438	32.5289460666138\\
66.5833333333333	0.26604	33.5653709291707\\
66.5833333333333	0.2677	34.6017957917286\\
66.5833333333333	0.26936	35.638220654286\\
66.5833333333333	0.27102	36.6746455168434\\
66.5833333333333	0.27268	37.7110703794008\\
66.5833333333333	0.27434	38.7474952419582\\
66.5833333333333	0.276	39.7839201045156\\
66.7916666666667	0.193	-12.5064244362245\\
66.7916666666667	0.19466	-11.4958638907956\\
66.7916666666667	0.19632	-10.4853033453668\\
66.7916666666667	0.19798	-9.47474279993799\\
66.7916666666667	0.19964	-8.4641822545093\\
66.7916666666667	0.2013	-7.45362170908049\\
66.7916666666667	0.20296	-6.44306116365158\\
66.7916666666667	0.20462	-5.43250061822289\\
66.7916666666667	0.20628	-4.42194007279409\\
66.7916666666667	0.20794	-3.4113795273654\\
66.7916666666667	0.2096	-2.40081898193671\\
66.7916666666667	0.21126	-1.39025843650802\\
66.7916666666667	0.21292	-0.379697891078877\\
66.7916666666667	0.21458	0.630862654349812\\
66.7916666666667	0.21624	1.6414231997785\\
66.7916666666667	0.2179	2.65198374520719\\
66.7916666666667	0.21956	3.66254429063588\\
66.7916666666667	0.22122	4.67310483606457\\
66.7916666666667	0.22288	5.68366538149371\\
66.7916666666667	0.22454	6.6942259269224\\
66.7916666666667	0.2262	7.70478647235109\\
66.7916666666667	0.22786	8.71534701778\\
66.7916666666667	0.22952	9.72590756320869\\
66.7916666666667	0.23118	10.7364681086376\\
66.7916666666667	0.23284	11.7470286540663\\
66.7916666666667	0.2345	12.7575891994952\\
66.7916666666667	0.23616	13.7681497449239\\
66.7916666666667	0.23782	14.7787102903528\\
66.7916666666667	0.23948	15.7892708357815\\
66.7916666666667	0.24114	16.7998313812104\\
66.7916666666667	0.2428	17.8103919266391\\
66.7916666666667	0.24446	18.8209524720678\\
66.7916666666667	0.24612	19.8315130174965\\
66.7916666666667	0.24778	20.8420735629256\\
66.7916666666667	0.24944	21.8526341083543\\
66.7916666666667	0.2511	22.8631946537832\\
66.7916666666667	0.25276	23.8737551992119\\
66.7916666666667	0.25442	24.8843157446404\\
66.7916666666667	0.25608	25.8948762900695\\
66.7916666666667	0.25774	26.9054368354982\\
66.7916666666667	0.2594	27.9159973809269\\
66.7916666666667	0.26106	28.9265579263556\\
66.7916666666667	0.26272	29.9371184717845\\
66.7916666666667	0.26438	30.9476790172132\\
66.7916666666667	0.26604	31.9582395626421\\
66.7916666666667	0.2677	32.9688001080708\\
66.7916666666667	0.26936	33.9793606534995\\
66.7916666666667	0.27102	34.9899211989282\\
66.7916666666667	0.27268	36.0004817443569\\
66.7916666666667	0.27434	37.011042289786\\
66.7916666666667	0.276	38.0216028352149\\
67	0.193	-12.9755258490928\\
67	0.19466	-11.9908296207925\\
67	0.19632	-11.0061333924925\\
67	0.19798	-10.0214371641923\\
67	0.19964	-9.03674093589223\\
67	0.2013	-8.05204470759213\\
67	0.20296	-7.06734847929192\\
67	0.20462	-6.08265225099183\\
67	0.20628	-5.09795602269173\\
67	0.20794	-4.11325979439175\\
67	0.2096	-3.12856356609154\\
67	0.21126	-2.14386733779156\\
67	0.21292	-1.15917110949113\\
67	0.21458	-0.174474881190918\\
67	0.21624	0.810221347109064\\
67	0.2179	1.79491757540904\\
67	0.21956	2.77961380370903\\
67	0.22122	3.76431003200923\\
67	0.22288	4.74900626030944\\
67	0.22454	5.73370248860965\\
67	0.2262	6.71839871690963\\
67	0.22786	7.70309494520984\\
67	0.22952	8.68779117350982\\
67	0.23118	9.67248740181026\\
67	0.23284	10.6571836301102\\
67	0.2345	11.6418798584104\\
67	0.23616	12.6265760867104\\
67	0.23782	13.6112723150109\\
67	0.23948	14.5959685433108\\
67	0.24114	15.5806647716111\\
67	0.2428	16.565360999911\\
67	0.24446	17.550057228211\\
67	0.24612	18.5347534565112\\
67	0.24778	19.5194496848114\\
67	0.24944	20.5041459131116\\
67	0.2511	21.4888421414119\\
67	0.25276	22.4735383697118\\
67	0.25442	23.4582345980116\\
67	0.25608	24.4429308263122\\
67	0.25774	25.4276270546122\\
67	0.2594	26.4123232829124\\
67	0.26106	27.3970195112124\\
67	0.26272	28.3817157395124\\
67	0.26438	29.3664119678126\\
67	0.26604	30.3511081961128\\
67	0.2677	31.3358044244128\\
67	0.26936	32.320500652713\\
67	0.27102	33.305196881013\\
67	0.27268	34.289893109313\\
67	0.27434	35.2745893376136\\
67	0.276	36.2592855659136\\
};
\end{axis}

\begin{axis}[%
width=4.927496cm,
height=3.050847cm,
at={(0cm,16.949153cm)},
scale only axis,
xmin=57,
xmax=67,
tick align=outside,
xlabel={$L_{cut}$},
xmajorgrids,
ymin=0.193,
ymax=0.276,
ylabel={$D_{rlx}$},
ymajorgrids,
zmin=-203.591576630033,
zmax=-56.9553064079365,
zlabel={$x_4,x_4$},
zmajorgrids,
view={-140}{50},
legend style={at={(1.03,1)},anchor=north west,legend cell align=left,align=left,draw=white!15!black}
]
\addplot3[only marks,mark=*,mark options={},mark size=1.5000pt,color=mycolor1] plot table[row sep=crcr,]{%
67	0.276	-199.229625869816\\
66	0.255	-163.931267020869\\
62	0.209	-90.7511623786065\\
57	0.193	-71.6947490376631\\
};
\addplot3[only marks,mark=*,mark options={},mark size=1.5000pt,color=black] plot table[row sep=crcr,]{%
64	0.23	-123.412797755139\\
};

\addplot3[%
surf,
opacity=0.7,
shader=interp,
colormap={mymap}{[1pt] rgb(0pt)=(0.0901961,0.239216,0.0745098); rgb(1pt)=(0.0945149,0.242058,0.0739522); rgb(2pt)=(0.0988592,0.244894,0.0733566); rgb(3pt)=(0.103229,0.247724,0.0727241); rgb(4pt)=(0.107623,0.250549,0.0720557); rgb(5pt)=(0.112043,0.253367,0.0713525); rgb(6pt)=(0.116487,0.25618,0.0706154); rgb(7pt)=(0.120956,0.258986,0.0698456); rgb(8pt)=(0.125449,0.261787,0.0690441); rgb(9pt)=(0.129967,0.264581,0.0682118); rgb(10pt)=(0.134508,0.26737,0.06735); rgb(11pt)=(0.139074,0.270152,0.0664596); rgb(12pt)=(0.143663,0.272929,0.0655416); rgb(13pt)=(0.148275,0.275699,0.0645971); rgb(14pt)=(0.152911,0.278463,0.0636271); rgb(15pt)=(0.15757,0.281221,0.0626328); rgb(16pt)=(0.162252,0.283973,0.0616151); rgb(17pt)=(0.166957,0.286719,0.060575); rgb(18pt)=(0.171685,0.289458,0.0595136); rgb(19pt)=(0.176434,0.292191,0.0584321); rgb(20pt)=(0.181207,0.294918,0.0573313); rgb(21pt)=(0.186001,0.297639,0.0562123); rgb(22pt)=(0.190817,0.300353,0.0550763); rgb(23pt)=(0.195655,0.303061,0.0539242); rgb(24pt)=(0.200514,0.305763,0.052757); rgb(25pt)=(0.205395,0.308459,0.0515759); rgb(26pt)=(0.210296,0.311149,0.0503624); rgb(27pt)=(0.215212,0.313846,0.0490067); rgb(28pt)=(0.220142,0.316548,0.0475043); rgb(29pt)=(0.22509,0.319254,0.0458704); rgb(30pt)=(0.230056,0.321962,0.0441205); rgb(31pt)=(0.235042,0.324671,0.04227); rgb(32pt)=(0.240048,0.327379,0.0403343); rgb(33pt)=(0.245078,0.330085,0.0383287); rgb(34pt)=(0.250131,0.332786,0.0362688); rgb(35pt)=(0.25521,0.335482,0.0341698); rgb(36pt)=(0.260317,0.33817,0.0320472); rgb(37pt)=(0.265451,0.340849,0.0299163); rgb(38pt)=(0.270616,0.343517,0.0277927); rgb(39pt)=(0.275813,0.346172,0.0256916); rgb(40pt)=(0.281043,0.348814,0.0236284); rgb(41pt)=(0.286307,0.35144,0.0216186); rgb(42pt)=(0.291607,0.354048,0.0196776); rgb(43pt)=(0.296945,0.356637,0.0178207); rgb(44pt)=(0.302322,0.359206,0.0160634); rgb(45pt)=(0.307739,0.361753,0.0144211); rgb(46pt)=(0.313198,0.364275,0.0129091); rgb(47pt)=(0.318701,0.366772,0.0115428); rgb(48pt)=(0.324249,0.369242,0.0103377); rgb(49pt)=(0.329843,0.371682,0.00930909); rgb(50pt)=(0.335485,0.374093,0.00847245); rgb(51pt)=(0.341176,0.376471,0.00784314); rgb(52pt)=(0.346925,0.378826,0.00732741); rgb(53pt)=(0.352735,0.381168,0.00682184); rgb(54pt)=(0.358605,0.383497,0.00632729); rgb(55pt)=(0.364532,0.385812,0.00584464); rgb(56pt)=(0.370516,0.388113,0.00537476); rgb(57pt)=(0.376552,0.390399,0.00491852); rgb(58pt)=(0.38264,0.39267,0.00447681); rgb(59pt)=(0.388777,0.394925,0.00405048); rgb(60pt)=(0.394962,0.397164,0.00364042); rgb(61pt)=(0.401191,0.399386,0.00324749); rgb(62pt)=(0.407464,0.401592,0.00287258); rgb(63pt)=(0.413777,0.40378,0.00251655); rgb(64pt)=(0.420129,0.40595,0.00218028); rgb(65pt)=(0.426518,0.408102,0.00186463); rgb(66pt)=(0.432942,0.410234,0.00157049); rgb(67pt)=(0.439399,0.412348,0.00129873); rgb(68pt)=(0.445885,0.414441,0.00105022); rgb(69pt)=(0.452401,0.416515,0.000825833); rgb(70pt)=(0.458942,0.418567,0.000626441); rgb(71pt)=(0.465508,0.420599,0.00045292); rgb(72pt)=(0.472096,0.422609,0.000306141); rgb(73pt)=(0.478704,0.424596,0.000186979); rgb(74pt)=(0.485331,0.426562,9.63073e-05); rgb(75pt)=(0.491973,0.428504,3.49981e-05); rgb(76pt)=(0.498628,0.430422,3.92506e-06); rgb(77pt)=(0.505323,0.432315,0); rgb(78pt)=(0.512206,0.434168,0); rgb(79pt)=(0.519282,0.435983,0); rgb(80pt)=(0.526529,0.437764,0); rgb(81pt)=(0.533922,0.439512,0); rgb(82pt)=(0.54144,0.441232,0); rgb(83pt)=(0.549059,0.442927,0); rgb(84pt)=(0.556756,0.444599,0); rgb(85pt)=(0.564508,0.446252,0); rgb(86pt)=(0.572292,0.447889,0); rgb(87pt)=(0.580084,0.449514,0); rgb(88pt)=(0.587863,0.451129,0); rgb(89pt)=(0.595604,0.452737,0); rgb(90pt)=(0.603284,0.454343,0); rgb(91pt)=(0.610882,0.455948,0); rgb(92pt)=(0.618373,0.457556,0); rgb(93pt)=(0.625734,0.459171,0); rgb(94pt)=(0.632943,0.460795,0); rgb(95pt)=(0.639976,0.462432,0); rgb(96pt)=(0.64681,0.464084,0); rgb(97pt)=(0.653423,0.465756,0); rgb(98pt)=(0.659791,0.46745,0); rgb(99pt)=(0.665891,0.469169,0); rgb(100pt)=(0.6717,0.470916,0); rgb(101pt)=(0.677195,0.472696,0); rgb(102pt)=(0.682353,0.47451,0); rgb(103pt)=(0.687242,0.476355,0); rgb(104pt)=(0.691952,0.478225,0); rgb(105pt)=(0.696497,0.480118,0); rgb(106pt)=(0.700887,0.482033,0); rgb(107pt)=(0.705134,0.483968,0); rgb(108pt)=(0.709251,0.485921,0); rgb(109pt)=(0.713249,0.487891,0); rgb(110pt)=(0.71714,0.489876,0); rgb(111pt)=(0.720936,0.491875,0); rgb(112pt)=(0.724649,0.493887,0); rgb(113pt)=(0.72829,0.495909,0); rgb(114pt)=(0.731872,0.49794,0); rgb(115pt)=(0.735406,0.499979,0); rgb(116pt)=(0.738904,0.502025,0); rgb(117pt)=(0.742378,0.504075,0); rgb(118pt)=(0.74584,0.506128,0); rgb(119pt)=(0.749302,0.508182,0); rgb(120pt)=(0.752775,0.510237,0); rgb(121pt)=(0.756272,0.51229,0); rgb(122pt)=(0.759804,0.514339,0); rgb(123pt)=(0.763384,0.516385,0); rgb(124pt)=(0.767022,0.518424,0); rgb(125pt)=(0.770731,0.520455,0); rgb(126pt)=(0.774523,0.522478,0); rgb(127pt)=(0.77841,0.524489,0); rgb(128pt)=(0.782391,0.526491,0); rgb(129pt)=(0.786402,0.528496,0); rgb(130pt)=(0.790431,0.530506,0); rgb(131pt)=(0.794478,0.532521,0); rgb(132pt)=(0.798541,0.534539,0); rgb(133pt)=(0.802619,0.53656,0); rgb(134pt)=(0.806712,0.538584,0); rgb(135pt)=(0.81082,0.540609,0); rgb(136pt)=(0.81494,0.542635,0); rgb(137pt)=(0.819074,0.54466,0); rgb(138pt)=(0.823219,0.546686,0); rgb(139pt)=(0.827374,0.548709,0); rgb(140pt)=(0.831541,0.55073,0); rgb(141pt)=(0.835716,0.552749,0); rgb(142pt)=(0.8399,0.554763,0); rgb(143pt)=(0.844092,0.556774,0); rgb(144pt)=(0.848292,0.558779,0); rgb(145pt)=(0.852497,0.560778,0); rgb(146pt)=(0.856708,0.562771,0); rgb(147pt)=(0.860924,0.564756,0); rgb(148pt)=(0.865143,0.566733,0); rgb(149pt)=(0.869366,0.568701,0); rgb(150pt)=(0.873592,0.57066,0); rgb(151pt)=(0.877819,0.572608,0); rgb(152pt)=(0.882047,0.574545,0); rgb(153pt)=(0.886275,0.576471,0); rgb(154pt)=(0.890659,0.578362,0); rgb(155pt)=(0.895333,0.580203,0); rgb(156pt)=(0.900258,0.581999,0); rgb(157pt)=(0.905397,0.583755,0); rgb(158pt)=(0.910711,0.585479,0); rgb(159pt)=(0.916164,0.587176,0); rgb(160pt)=(0.921717,0.588852,0); rgb(161pt)=(0.927333,0.590513,0); rgb(162pt)=(0.932974,0.592166,0); rgb(163pt)=(0.938602,0.593815,0); rgb(164pt)=(0.94418,0.595468,0); rgb(165pt)=(0.949669,0.59713,0); rgb(166pt)=(0.955033,0.598808,0); rgb(167pt)=(0.960233,0.600507,0); rgb(168pt)=(0.965232,0.602233,0); rgb(169pt)=(0.969992,0.603992,0); rgb(170pt)=(0.974475,0.605791,0); rgb(171pt)=(0.978643,0.607636,0); rgb(172pt)=(0.98246,0.609532,0); rgb(173pt)=(0.985886,0.611486,0); rgb(174pt)=(0.988885,0.613503,0); rgb(175pt)=(0.991419,0.61559,0); rgb(176pt)=(0.99345,0.617753,0); rgb(177pt)=(0.99494,0.619997,0); rgb(178pt)=(0.995851,0.622329,0); rgb(179pt)=(0.996226,0.624763,0); rgb(180pt)=(0.996512,0.627352,0); rgb(181pt)=(0.996788,0.630095,0); rgb(182pt)=(0.997053,0.632982,0); rgb(183pt)=(0.997308,0.636004,0); rgb(184pt)=(0.997552,0.639152,0); rgb(185pt)=(0.997785,0.642416,0); rgb(186pt)=(0.998006,0.645786,0); rgb(187pt)=(0.998217,0.649253,0); rgb(188pt)=(0.998416,0.652807,0); rgb(189pt)=(0.998605,0.656439,0); rgb(190pt)=(0.998781,0.660138,0); rgb(191pt)=(0.998946,0.663897,0); rgb(192pt)=(0.9991,0.667704,0); rgb(193pt)=(0.999242,0.67155,0); rgb(194pt)=(0.999372,0.675427,0); rgb(195pt)=(0.99949,0.679323,0); rgb(196pt)=(0.999596,0.68323,0); rgb(197pt)=(0.99969,0.687139,0); rgb(198pt)=(0.999771,0.691039,0); rgb(199pt)=(0.999841,0.694921,0); rgb(200pt)=(0.999898,0.698775,0); rgb(201pt)=(0.999942,0.702592,0); rgb(202pt)=(0.999974,0.706363,0); rgb(203pt)=(0.999994,0.710077,0); rgb(204pt)=(1,0.713725,0); rgb(205pt)=(1,0.717341,0); rgb(206pt)=(1,0.720963,0); rgb(207pt)=(1,0.724591,0); rgb(208pt)=(1,0.728226,0); rgb(209pt)=(1,0.731867,0); rgb(210pt)=(1,0.735514,0); rgb(211pt)=(1,0.739167,0); rgb(212pt)=(1,0.742827,0); rgb(213pt)=(1,0.746493,0); rgb(214pt)=(1,0.750165,0); rgb(215pt)=(1,0.753843,0); rgb(216pt)=(1,0.757527,0); rgb(217pt)=(1,0.761217,0); rgb(218pt)=(1,0.764913,0); rgb(219pt)=(1,0.768615,0); rgb(220pt)=(1,0.772324,0); rgb(221pt)=(1,0.776038,0); rgb(222pt)=(1,0.779758,0); rgb(223pt)=(1,0.783484,0); rgb(224pt)=(1,0.787215,0); rgb(225pt)=(1,0.790953,0); rgb(226pt)=(1,0.794696,0); rgb(227pt)=(1,0.798445,0); rgb(228pt)=(1,0.8022,0); rgb(229pt)=(1,0.805961,0); rgb(230pt)=(1,0.809727,0); rgb(231pt)=(1,0.8135,0); rgb(232pt)=(1,0.817278,0); rgb(233pt)=(1,0.821063,0); rgb(234pt)=(1,0.824854,0); rgb(235pt)=(1,0.828652,0); rgb(236pt)=(1,0.832455,0); rgb(237pt)=(1,0.836265,0); rgb(238pt)=(1,0.840081,0); rgb(239pt)=(1,0.843903,0); rgb(240pt)=(1,0.847732,0); rgb(241pt)=(1,0.851566,0); rgb(242pt)=(1,0.855406,0); rgb(243pt)=(1,0.859253,0); rgb(244pt)=(1,0.863106,0); rgb(245pt)=(1,0.866964,0); rgb(246pt)=(1,0.870829,0); rgb(247pt)=(1,0.8747,0); rgb(248pt)=(1,0.878577,0); rgb(249pt)=(1,0.88246,0); rgb(250pt)=(1,0.886349,0); rgb(251pt)=(1,0.890243,0); rgb(252pt)=(1,0.894144,0); rgb(253pt)=(1,0.898051,0); rgb(254pt)=(1,0.901964,0); rgb(255pt)=(1,0.905882,0)},
mesh/rows=49]
table[row sep=crcr,header=false] {%
%
57	0.193	-71.694749037699\\
57	0.19466	-74.3326855895457\\
57	0.19632	-76.9706221413924\\
57	0.19798	-79.6085586932391\\
57	0.19964	-82.2464952450858\\
57	0.2013	-84.8844317969324\\
57	0.20296	-87.5223683487791\\
57	0.20462	-90.1603049006258\\
57	0.20628	-92.7982414524725\\
57	0.20794	-95.4361780043192\\
57	0.2096	-98.0741145561659\\
57	0.21126	-100.712051108013\\
57	0.21292	-103.349987659859\\
57	0.21458	-105.987924211706\\
57	0.21624	-108.625860763553\\
57	0.2179	-111.263797315399\\
57	0.21956	-113.901733867246\\
57	0.22122	-116.539670419093\\
57	0.22288	-119.177606970939\\
57	0.22454	-121.815543522786\\
57	0.2262	-124.453480074633\\
57	0.22786	-127.091416626479\\
57	0.22952	-129.729353178326\\
57	0.23118	-132.367289730173\\
57	0.23284	-135.005226282019\\
57	0.2345	-137.643162833866\\
57	0.23616	-140.281099385713\\
57	0.23782	-142.919035937559\\
57	0.23948	-145.556972489406\\
57	0.24114	-148.194909041253\\
57	0.2428	-150.8328455931\\
57	0.24446	-153.470782144946\\
57	0.24612	-156.108718696793\\
57	0.24778	-158.74665524864\\
57	0.24944	-161.384591800486\\
57	0.2511	-164.022528352333\\
57	0.25276	-166.66046490418\\
57	0.25442	-169.298401456026\\
57	0.25608	-171.936338007873\\
57	0.25774	-174.57427455972\\
57	0.2594	-177.212211111566\\
57	0.26106	-179.850147663413\\
57	0.26272	-182.48808421526\\
57	0.26438	-185.126020767106\\
57	0.26604	-187.763957318953\\
57	0.2677	-190.4018938708\\
57	0.26936	-193.039830422646\\
57	0.27102	-195.677766974493\\
57	0.27268	-198.31570352634\\
57	0.27434	-200.953640078186\\
57	0.276	-203.591576630033\\
57.2083333333333	0.193	-71.3876773162456\\
57.2083333333333	0.19466	-74.029937823038\\
57.2083333333333	0.19632	-76.6721983298303\\
57.2083333333333	0.19798	-79.3144588366226\\
57.2083333333333	0.19964	-81.9567193434149\\
57.2083333333333	0.2013	-84.5989798502072\\
57.2083333333333	0.20296	-87.2412403569995\\
57.2083333333333	0.20462	-89.8835008637918\\
57.2083333333333	0.20628	-92.5257613705841\\
57.2083333333333	0.20794	-95.1680218773764\\
57.2083333333333	0.2096	-97.8102823841687\\
57.2083333333333	0.21126	-100.452542890961\\
57.2083333333333	0.21292	-103.094803397753\\
57.2083333333333	0.21458	-105.737063904546\\
57.2083333333333	0.21624	-108.379324411338\\
57.2083333333333	0.2179	-111.02158491813\\
57.2083333333333	0.21956	-113.663845424923\\
57.2083333333333	0.22122	-116.306105931715\\
57.2083333333333	0.22288	-118.948366438507\\
57.2083333333333	0.22454	-121.5906269453\\
57.2083333333333	0.2262	-124.232887452092\\
57.2083333333333	0.22786	-126.875147958884\\
57.2083333333333	0.22952	-129.517408465676\\
57.2083333333333	0.23118	-132.159668972469\\
57.2083333333333	0.23284	-134.801929479261\\
57.2083333333333	0.2345	-137.444189986053\\
57.2083333333333	0.23616	-140.086450492846\\
57.2083333333333	0.23782	-142.728710999638\\
57.2083333333333	0.23948	-145.37097150643\\
57.2083333333333	0.24114	-148.013232013223\\
57.2083333333333	0.2428	-150.655492520015\\
57.2083333333333	0.24446	-153.297753026807\\
57.2083333333333	0.24612	-155.9400135336\\
57.2083333333333	0.24778	-158.582274040392\\
57.2083333333333	0.24944	-161.224534547184\\
57.2083333333333	0.2511	-163.866795053976\\
57.2083333333333	0.25276	-166.509055560769\\
57.2083333333333	0.25442	-169.151316067561\\
57.2083333333333	0.25608	-171.793576574353\\
57.2083333333333	0.25774	-174.435837081146\\
57.2083333333333	0.2594	-177.078097587938\\
57.2083333333333	0.26106	-179.72035809473\\
57.2083333333333	0.26272	-182.362618601523\\
57.2083333333333	0.26438	-185.004879108315\\
57.2083333333333	0.26604	-187.647139615107\\
57.2083333333333	0.2677	-190.2894001219\\
57.2083333333333	0.26936	-192.931660628692\\
57.2083333333333	0.27102	-195.573921135484\\
57.2083333333333	0.27268	-198.216181642277\\
57.2083333333333	0.27434	-200.858442149069\\
57.2083333333333	0.276	-203.500702655861\\
57.4166666666667	0.193	-71.0806055947923\\
57.4166666666667	0.19466	-73.7271900565302\\
57.4166666666667	0.19632	-76.3737745182681\\
57.4166666666667	0.19798	-79.0203589800061\\
57.4166666666667	0.19964	-81.6669434417441\\
57.4166666666667	0.2013	-84.313527903482\\
57.4166666666667	0.20296	-86.9601123652199\\
57.4166666666667	0.20462	-89.6066968269579\\
57.4166666666667	0.20628	-92.2532812886958\\
57.4166666666667	0.20794	-94.8998657504337\\
57.4166666666667	0.2096	-97.5464502121717\\
57.4166666666667	0.21126	-100.19303467391\\
57.4166666666667	0.21292	-102.839619135648\\
57.4166666666667	0.21458	-105.486203597385\\
57.4166666666667	0.21624	-108.132788059123\\
57.4166666666667	0.2179	-110.779372520861\\
57.4166666666667	0.21956	-113.425956982599\\
57.4166666666667	0.22122	-116.072541444337\\
57.4166666666667	0.22288	-118.719125906075\\
57.4166666666667	0.22454	-121.365710367813\\
57.4166666666667	0.2262	-124.012294829551\\
57.4166666666667	0.22786	-126.658879291289\\
57.4166666666667	0.22952	-129.305463753027\\
57.4166666666667	0.23118	-131.952048214765\\
57.4166666666667	0.23284	-134.598632676503\\
57.4166666666667	0.2345	-137.245217138241\\
57.4166666666667	0.23616	-139.891801599979\\
57.4166666666667	0.23782	-142.538386061717\\
57.4166666666667	0.23948	-145.184970523455\\
57.4166666666667	0.24114	-147.831554985193\\
57.4166666666667	0.2428	-150.47813944693\\
57.4166666666667	0.24446	-153.124723908668\\
57.4166666666667	0.24612	-155.771308370406\\
57.4166666666667	0.24778	-158.417892832144\\
57.4166666666667	0.24944	-161.064477293882\\
57.4166666666667	0.2511	-163.71106175562\\
57.4166666666667	0.25276	-166.357646217358\\
57.4166666666667	0.25442	-169.004230679096\\
57.4166666666667	0.25608	-171.650815140834\\
57.4166666666667	0.25774	-174.297399602572\\
57.4166666666667	0.2594	-176.94398406431\\
57.4166666666667	0.26106	-179.590568526048\\
57.4166666666667	0.26272	-182.237152987786\\
57.4166666666667	0.26438	-184.883737449524\\
57.4166666666667	0.26604	-187.530321911262\\
57.4166666666667	0.2677	-190.176906373\\
57.4166666666667	0.26936	-192.823490834738\\
57.4166666666667	0.27102	-195.470075296476\\
57.4166666666667	0.27268	-198.116659758213\\
57.4166666666667	0.27434	-200.763244219951\\
57.4166666666667	0.276	-203.409828681689\\
57.625	0.193	-70.7735338733389\\
57.625	0.19466	-73.4244422900225\\
57.625	0.19632	-76.075350706706\\
57.625	0.19798	-78.7262591233896\\
57.625	0.19964	-81.3771675400732\\
57.625	0.2013	-84.0280759567567\\
57.625	0.20296	-86.6789843734403\\
57.625	0.20462	-89.3298927901239\\
57.625	0.20628	-91.9808012068075\\
57.625	0.20794	-94.631709623491\\
57.625	0.2096	-97.2826180401746\\
57.625	0.21126	-99.9335264568581\\
57.625	0.21292	-102.584434873542\\
57.625	0.21458	-105.235343290225\\
57.625	0.21624	-107.886251706909\\
57.625	0.2179	-110.537160123592\\
57.625	0.21956	-113.188068540276\\
57.625	0.22122	-115.83897695696\\
57.625	0.22288	-118.489885373643\\
57.625	0.22454	-121.140793790327\\
57.625	0.2262	-123.79170220701\\
57.625	0.22786	-126.442610623694\\
57.625	0.22952	-129.093519040377\\
57.625	0.23118	-131.744427457061\\
57.625	0.23284	-134.395335873745\\
57.625	0.2345	-137.046244290428\\
57.625	0.23616	-139.697152707112\\
57.625	0.23782	-142.348061123795\\
57.625	0.23948	-144.998969540479\\
57.625	0.24114	-147.649877957162\\
57.625	0.2428	-150.300786373846\\
57.625	0.24446	-152.95169479053\\
57.625	0.24612	-155.602603207213\\
57.625	0.24778	-158.253511623897\\
57.625	0.24944	-160.90442004058\\
57.625	0.2511	-163.555328457264\\
57.625	0.25276	-166.206236873947\\
57.625	0.25442	-168.857145290631\\
57.625	0.25608	-171.508053707315\\
57.625	0.25774	-174.158962123998\\
57.625	0.2594	-176.809870540682\\
57.625	0.26106	-179.460778957365\\
57.625	0.26272	-182.111687374049\\
57.625	0.26438	-184.762595790732\\
57.625	0.26604	-187.413504207416\\
57.625	0.2677	-190.0644126241\\
57.625	0.26936	-192.715321040783\\
57.625	0.27102	-195.366229457467\\
57.625	0.27268	-198.01713787415\\
57.625	0.27434	-200.668046290834\\
57.625	0.276	-203.318954707517\\
57.8333333333333	0.193	-70.4664621518855\\
57.8333333333333	0.19466	-73.1216945235147\\
57.8333333333333	0.19632	-75.7769268951439\\
57.8333333333333	0.19798	-78.4321592667731\\
57.8333333333333	0.19964	-81.0873916384023\\
57.8333333333333	0.2013	-83.7426240100315\\
57.8333333333333	0.20296	-86.3978563816607\\
57.8333333333333	0.20462	-89.0530887532899\\
57.8333333333333	0.20628	-91.7083211249191\\
57.8333333333333	0.20794	-94.3635534965483\\
57.8333333333333	0.2096	-97.0187858681775\\
57.8333333333333	0.21126	-99.6740182398067\\
57.8333333333333	0.21292	-102.329250611436\\
57.8333333333333	0.21458	-104.984482983065\\
57.8333333333333	0.21624	-107.639715354694\\
57.8333333333333	0.2179	-110.294947726323\\
57.8333333333333	0.21956	-112.950180097953\\
57.8333333333333	0.22122	-115.605412469582\\
57.8333333333333	0.22288	-118.260644841211\\
57.8333333333333	0.22454	-120.91587721284\\
57.8333333333333	0.2262	-123.571109584469\\
57.8333333333333	0.22786	-126.226341956099\\
57.8333333333333	0.22952	-128.881574327728\\
57.8333333333333	0.23118	-131.536806699357\\
57.8333333333333	0.23284	-134.192039070986\\
57.8333333333333	0.2345	-136.847271442615\\
57.8333333333333	0.23616	-139.502503814245\\
57.8333333333333	0.23782	-142.157736185874\\
57.8333333333333	0.23948	-144.812968557503\\
57.8333333333333	0.24114	-147.468200929132\\
57.8333333333333	0.2428	-150.123433300761\\
57.8333333333333	0.24446	-152.778665672391\\
57.8333333333333	0.24612	-155.43389804402\\
57.8333333333333	0.24778	-158.089130415649\\
57.8333333333333	0.24944	-160.744362787278\\
57.8333333333333	0.2511	-163.399595158907\\
57.8333333333333	0.25276	-166.054827530537\\
57.8333333333333	0.25442	-168.710059902166\\
57.8333333333333	0.25608	-171.365292273795\\
57.8333333333333	0.25774	-174.020524645424\\
57.8333333333333	0.2594	-176.675757017054\\
57.8333333333333	0.26106	-179.330989388683\\
57.8333333333333	0.26272	-181.986221760312\\
57.8333333333333	0.26438	-184.641454131941\\
57.8333333333333	0.26604	-187.29668650357\\
57.8333333333333	0.2677	-189.9519188752\\
57.8333333333333	0.26936	-192.607151246829\\
57.8333333333333	0.27102	-195.262383618458\\
57.8333333333333	0.27268	-197.917615990087\\
57.8333333333333	0.27434	-200.572848361716\\
57.8333333333333	0.276	-203.228080733346\\
58.0416666666667	0.193	-70.1593904304321\\
58.0416666666667	0.19466	-72.8189467570069\\
58.0416666666667	0.19632	-75.4785030835818\\
58.0416666666667	0.19798	-78.1380594101566\\
58.0416666666667	0.19964	-80.7976157367314\\
58.0416666666667	0.2013	-83.4571720633062\\
58.0416666666667	0.20296	-86.1167283898811\\
58.0416666666667	0.20462	-88.7762847164559\\
58.0416666666667	0.20628	-91.4358410430308\\
58.0416666666667	0.20794	-94.0953973696056\\
58.0416666666667	0.2096	-96.7549536961804\\
58.0416666666667	0.21126	-99.4145100227552\\
58.0416666666667	0.21292	-102.07406634933\\
58.0416666666667	0.21458	-104.733622675905\\
58.0416666666667	0.21624	-107.39317900248\\
58.0416666666667	0.2179	-110.052735329055\\
58.0416666666667	0.21956	-112.712291655629\\
58.0416666666667	0.22122	-115.371847982204\\
58.0416666666667	0.22288	-118.031404308779\\
58.0416666666667	0.22454	-120.690960635354\\
58.0416666666667	0.2262	-123.350516961929\\
58.0416666666667	0.22786	-126.010073288503\\
58.0416666666667	0.22952	-128.669629615078\\
58.0416666666667	0.23118	-131.329185941653\\
58.0416666666667	0.23284	-133.988742268228\\
58.0416666666667	0.2345	-136.648298594803\\
58.0416666666667	0.23616	-139.307854921378\\
58.0416666666667	0.23782	-141.967411247953\\
58.0416666666667	0.23948	-144.626967574527\\
58.0416666666667	0.24114	-147.286523901102\\
58.0416666666667	0.2428	-149.946080227677\\
58.0416666666667	0.24446	-152.605636554252\\
58.0416666666667	0.24612	-155.265192880827\\
58.0416666666667	0.24778	-157.924749207401\\
58.0416666666667	0.24944	-160.584305533976\\
58.0416666666667	0.2511	-163.243861860551\\
58.0416666666667	0.25276	-165.903418187126\\
58.0416666666667	0.25442	-168.562974513701\\
58.0416666666667	0.25608	-171.222530840276\\
58.0416666666667	0.25774	-173.88208716685\\
58.0416666666667	0.2594	-176.541643493425\\
58.0416666666667	0.26106	-179.20119982\\
58.0416666666667	0.26272	-181.860756146575\\
58.0416666666667	0.26438	-184.52031247315\\
58.0416666666667	0.26604	-187.179868799725\\
58.0416666666667	0.2677	-189.839425126299\\
58.0416666666667	0.26936	-192.498981452874\\
58.0416666666667	0.27102	-195.158537779449\\
58.0416666666667	0.27268	-197.818094106024\\
58.0416666666667	0.27434	-200.477650432599\\
58.0416666666667	0.276	-203.137206759174\\
58.25	0.193	-69.8523187089787\\
58.25	0.19466	-72.5161989904992\\
58.25	0.19632	-75.1800792720196\\
58.25	0.19798	-77.8439595535401\\
58.25	0.19964	-80.5078398350605\\
58.25	0.2013	-83.171720116581\\
58.25	0.20296	-85.8356003981015\\
58.25	0.20462	-88.4994806796219\\
58.25	0.20628	-91.1633609611424\\
58.25	0.20794	-93.8272412426628\\
58.25	0.2096	-96.4911215241833\\
58.25	0.21126	-99.1550018057037\\
58.25	0.21292	-101.818882087224\\
58.25	0.21458	-104.482762368745\\
58.25	0.21624	-107.146642650265\\
58.25	0.2179	-109.810522931786\\
58.25	0.21956	-112.474403213306\\
58.25	0.22122	-115.138283494827\\
58.25	0.22288	-117.802163776347\\
58.25	0.22454	-120.466044057867\\
58.25	0.2262	-123.129924339388\\
58.25	0.22786	-125.793804620908\\
58.25	0.22952	-128.457684902429\\
58.25	0.23118	-131.121565183949\\
58.25	0.23284	-133.78544546547\\
58.25	0.2345	-136.44932574699\\
58.25	0.23616	-139.113206028511\\
58.25	0.23782	-141.777086310031\\
58.25	0.23948	-144.440966591552\\
58.25	0.24114	-147.104846873072\\
58.25	0.2428	-149.768727154592\\
58.25	0.24446	-152.432607436113\\
58.25	0.24612	-155.096487717633\\
58.25	0.24778	-157.760367999154\\
58.25	0.24944	-160.424248280674\\
58.25	0.2511	-163.088128562195\\
58.25	0.25276	-165.752008843715\\
58.25	0.25442	-168.415889125236\\
58.25	0.25608	-171.079769406756\\
58.25	0.25774	-173.743649688277\\
58.25	0.2594	-176.407529969797\\
58.25	0.26106	-179.071410251318\\
58.25	0.26272	-181.735290532838\\
58.25	0.26438	-184.399170814358\\
58.25	0.26604	-187.063051095879\\
58.25	0.2677	-189.726931377399\\
58.25	0.26936	-192.39081165892\\
58.25	0.27102	-195.05469194044\\
58.25	0.27268	-197.718572221961\\
58.25	0.27434	-200.382452503481\\
58.25	0.276	-203.046332785002\\
58.4583333333333	0.193	-69.5452469875253\\
58.4583333333333	0.19466	-72.2134512239914\\
58.4583333333333	0.19632	-74.8816554604575\\
58.4583333333333	0.19798	-77.5498596969236\\
58.4583333333333	0.19964	-80.2180639333897\\
58.4583333333333	0.2013	-82.8862681698558\\
58.4583333333333	0.20296	-85.5544724063219\\
58.4583333333333	0.20462	-88.2226766427879\\
58.4583333333333	0.20628	-90.8908808792541\\
58.4583333333333	0.20794	-93.5590851157201\\
58.4583333333333	0.2096	-96.2272893521862\\
58.4583333333333	0.21126	-98.8954935886523\\
58.4583333333333	0.21292	-101.563697825118\\
58.4583333333333	0.21458	-104.231902061584\\
58.4583333333333	0.21624	-106.900106298051\\
58.4583333333333	0.2179	-109.568310534517\\
58.4583333333333	0.21956	-112.236514770983\\
58.4583333333333	0.22122	-114.904719007449\\
58.4583333333333	0.22288	-117.572923243915\\
58.4583333333333	0.22454	-120.241127480381\\
58.4583333333333	0.2262	-122.909331716847\\
58.4583333333333	0.22786	-125.577535953313\\
58.4583333333333	0.22952	-128.245740189779\\
58.4583333333333	0.23118	-130.913944426245\\
58.4583333333333	0.23284	-133.582148662711\\
58.4583333333333	0.2345	-136.250352899178\\
58.4583333333333	0.23616	-138.918557135644\\
58.4583333333333	0.23782	-141.58676137211\\
58.4583333333333	0.23948	-144.254965608576\\
58.4583333333333	0.24114	-146.923169845042\\
58.4583333333333	0.2428	-149.591374081508\\
58.4583333333333	0.24446	-152.259578317974\\
58.4583333333333	0.24612	-154.92778255444\\
58.4583333333333	0.24778	-157.595986790906\\
58.4583333333333	0.24944	-160.264191027372\\
58.4583333333333	0.2511	-162.932395263838\\
58.4583333333333	0.25276	-165.600599500304\\
58.4583333333333	0.25442	-168.268803736771\\
58.4583333333333	0.25608	-170.937007973237\\
58.4583333333333	0.25774	-173.605212209703\\
58.4583333333333	0.2594	-176.273416446169\\
58.4583333333333	0.26106	-178.941620682635\\
58.4583333333333	0.26272	-181.609824919101\\
58.4583333333333	0.26438	-184.278029155567\\
58.4583333333333	0.26604	-186.946233392033\\
58.4583333333333	0.2677	-189.614437628499\\
58.4583333333333	0.26936	-192.282641864965\\
58.4583333333333	0.27102	-194.950846101432\\
58.4583333333333	0.27268	-197.619050337898\\
58.4583333333333	0.27434	-200.287254574364\\
58.4583333333333	0.276	-202.95545881083\\
58.6666666666667	0.193	-69.2381752660719\\
58.6666666666667	0.19466	-71.9107034574837\\
58.6666666666667	0.19632	-74.5832316488954\\
58.6666666666667	0.19798	-77.2557598403071\\
58.6666666666667	0.19964	-79.9282880317188\\
58.6666666666667	0.2013	-82.6008162231305\\
58.6666666666667	0.20296	-85.2733444145423\\
58.6666666666667	0.20462	-87.945872605954\\
58.6666666666667	0.20628	-90.6184007973657\\
58.6666666666667	0.20794	-93.2909289887774\\
58.6666666666667	0.2096	-95.9634571801891\\
58.6666666666667	0.21126	-98.6359853716008\\
58.6666666666667	0.21292	-101.308513563013\\
58.6666666666667	0.21458	-103.981041754424\\
58.6666666666667	0.21624	-106.653569945836\\
58.6666666666667	0.2179	-109.326098137248\\
58.6666666666667	0.21956	-111.998626328659\\
58.6666666666667	0.22122	-114.671154520071\\
58.6666666666667	0.22288	-117.343682711483\\
58.6666666666667	0.22454	-120.016210902895\\
58.6666666666667	0.2262	-122.688739094306\\
58.6666666666667	0.22786	-125.361267285718\\
58.6666666666667	0.22952	-128.03379547713\\
58.6666666666667	0.23118	-130.706323668542\\
58.6666666666667	0.23284	-133.378851859953\\
58.6666666666667	0.2345	-136.051380051365\\
58.6666666666667	0.23616	-138.723908242777\\
58.6666666666667	0.23782	-141.396436434188\\
58.6666666666667	0.23948	-144.0689646256\\
58.6666666666667	0.24114	-146.741492817012\\
58.6666666666667	0.2428	-149.414021008424\\
58.6666666666667	0.24446	-152.086549199835\\
58.6666666666667	0.24612	-154.759077391247\\
58.6666666666667	0.24778	-157.431605582659\\
58.6666666666667	0.24944	-160.10413377407\\
58.6666666666667	0.2511	-162.776661965482\\
58.6666666666667	0.25276	-165.449190156894\\
58.6666666666667	0.25442	-168.121718348306\\
58.6666666666667	0.25608	-170.794246539717\\
58.6666666666667	0.25774	-173.466774731129\\
58.6666666666667	0.2594	-176.139302922541\\
58.6666666666667	0.26106	-178.811831113952\\
58.6666666666667	0.26272	-181.484359305364\\
58.6666666666667	0.26438	-184.156887496776\\
58.6666666666667	0.26604	-186.829415688188\\
58.6666666666667	0.2677	-189.501943879599\\
58.6666666666667	0.26936	-192.174472071011\\
58.6666666666667	0.27102	-194.847000262423\\
58.6666666666667	0.27268	-197.519528453834\\
58.6666666666667	0.27434	-200.192056645246\\
58.6666666666667	0.276	-202.864584836658\\
58.875	0.193	-68.9311035446185\\
58.875	0.19466	-71.6079556909759\\
58.875	0.19632	-74.2848078373332\\
58.875	0.19798	-76.9616599836906\\
58.875	0.19964	-79.638512130048\\
58.875	0.2013	-82.3153642764053\\
58.875	0.20296	-84.9922164227627\\
58.875	0.20462	-87.66906856912\\
58.875	0.20628	-90.3459207154773\\
58.875	0.20794	-93.0227728618347\\
58.875	0.2096	-95.699625008192\\
58.875	0.21126	-98.3764771545493\\
58.875	0.21292	-101.053329300907\\
58.875	0.21458	-103.730181447264\\
58.875	0.21624	-106.407033593621\\
58.875	0.2179	-109.083885739979\\
58.875	0.21956	-111.760737886336\\
58.875	0.22122	-114.437590032693\\
58.875	0.22288	-117.114442179051\\
58.875	0.22454	-119.791294325408\\
58.875	0.2262	-122.468146471765\\
58.875	0.22786	-125.144998618123\\
58.875	0.22952	-127.82185076448\\
58.875	0.23118	-130.498702910838\\
58.875	0.23284	-133.175555057195\\
58.875	0.2345	-135.852407203552\\
58.875	0.23616	-138.52925934991\\
58.875	0.23782	-141.206111496267\\
58.875	0.23948	-143.882963642624\\
58.875	0.24114	-146.559815788982\\
58.875	0.2428	-149.236667935339\\
58.875	0.24446	-151.913520081696\\
58.875	0.24612	-154.590372228054\\
58.875	0.24778	-157.267224374411\\
58.875	0.24944	-159.944076520768\\
58.875	0.2511	-162.620928667126\\
58.875	0.25276	-165.297780813483\\
58.875	0.25442	-167.97463295984\\
58.875	0.25608	-170.651485106198\\
58.875	0.25774	-173.328337252555\\
58.875	0.2594	-176.005189398912\\
58.875	0.26106	-178.68204154527\\
58.875	0.26272	-181.358893691627\\
58.875	0.26438	-184.035745837984\\
58.875	0.26604	-186.712597984342\\
58.875	0.2677	-189.389450130699\\
58.875	0.26936	-192.066302277057\\
58.875	0.27102	-194.743154423414\\
58.875	0.27268	-197.420006569771\\
58.875	0.27434	-200.096858716129\\
58.875	0.276	-202.773710862486\\
59.0833333333333	0.193	-68.6240318231652\\
59.0833333333333	0.19466	-71.3052079244682\\
59.0833333333333	0.19632	-73.9863840257711\\
59.0833333333333	0.19798	-76.6675601270741\\
59.0833333333333	0.19964	-79.3487362283771\\
59.0833333333333	0.2013	-82.0299123296801\\
59.0833333333333	0.20296	-84.711088430983\\
59.0833333333333	0.20462	-87.392264532286\\
59.0833333333333	0.20628	-90.073440633589\\
59.0833333333333	0.20794	-92.754616734892\\
59.0833333333333	0.2096	-95.435792836195\\
59.0833333333333	0.21126	-98.1169689374979\\
59.0833333333333	0.21292	-100.798145038801\\
59.0833333333333	0.21458	-103.479321140104\\
59.0833333333333	0.21624	-106.160497241407\\
59.0833333333333	0.2179	-108.84167334271\\
59.0833333333333	0.21956	-111.522849444013\\
59.0833333333333	0.22122	-114.204025545316\\
59.0833333333333	0.22288	-116.885201646619\\
59.0833333333333	0.22454	-119.566377747922\\
59.0833333333333	0.2262	-122.247553849225\\
59.0833333333333	0.22786	-124.928729950528\\
59.0833333333333	0.22952	-127.609906051831\\
59.0833333333333	0.23118	-130.291082153134\\
59.0833333333333	0.23284	-132.972258254437\\
59.0833333333333	0.2345	-135.65343435574\\
59.0833333333333	0.23616	-138.334610457043\\
59.0833333333333	0.23782	-141.015786558346\\
59.0833333333333	0.23948	-143.696962659649\\
59.0833333333333	0.24114	-146.378138760952\\
59.0833333333333	0.2428	-149.059314862255\\
59.0833333333333	0.24446	-151.740490963557\\
59.0833333333333	0.24612	-154.42166706486\\
59.0833333333333	0.24778	-157.102843166163\\
59.0833333333333	0.24944	-159.784019267466\\
59.0833333333333	0.2511	-162.465195368769\\
59.0833333333333	0.25276	-165.146371470072\\
59.0833333333333	0.25442	-167.827547571375\\
59.0833333333333	0.25608	-170.508723672678\\
59.0833333333333	0.25774	-173.189899773981\\
59.0833333333333	0.2594	-175.871075875284\\
59.0833333333333	0.26106	-178.552251976587\\
59.0833333333333	0.26272	-181.23342807789\\
59.0833333333333	0.26438	-183.914604179193\\
59.0833333333333	0.26604	-186.595780280496\\
59.0833333333333	0.2677	-189.276956381799\\
59.0833333333333	0.26936	-191.958132483102\\
59.0833333333333	0.27102	-194.639308584405\\
59.0833333333333	0.27268	-197.320484685708\\
59.0833333333333	0.27434	-200.001660787011\\
59.0833333333333	0.276	-202.682836888314\\
59.2916666666667	0.193	-68.3169601017118\\
59.2916666666667	0.19466	-71.0024601579604\\
59.2916666666667	0.19632	-73.687960214209\\
59.2916666666667	0.19798	-76.3734602704576\\
59.2916666666667	0.19964	-79.0589603267062\\
59.2916666666667	0.2013	-81.7444603829548\\
59.2916666666667	0.20296	-84.4299604392034\\
59.2916666666667	0.20462	-87.115460495452\\
59.2916666666667	0.20628	-89.8009605517007\\
59.2916666666667	0.20794	-92.4864606079493\\
59.2916666666667	0.2096	-95.1719606641979\\
59.2916666666667	0.21126	-97.8574607204465\\
59.2916666666667	0.21292	-100.542960776695\\
59.2916666666667	0.21458	-103.228460832944\\
59.2916666666667	0.21624	-105.913960889192\\
59.2916666666667	0.2179	-108.599460945441\\
59.2916666666667	0.21956	-111.284961001689\\
59.2916666666667	0.22122	-113.970461057938\\
59.2916666666667	0.22288	-116.655961114187\\
59.2916666666667	0.22454	-119.341461170435\\
59.2916666666667	0.2262	-122.026961226684\\
59.2916666666667	0.22786	-124.712461282933\\
59.2916666666667	0.22952	-127.397961339181\\
59.2916666666667	0.23118	-130.08346139543\\
59.2916666666667	0.23284	-132.768961451678\\
59.2916666666667	0.2345	-135.454461507927\\
59.2916666666667	0.23616	-138.139961564176\\
59.2916666666667	0.23782	-140.825461620424\\
59.2916666666667	0.23948	-143.510961676673\\
59.2916666666667	0.24114	-146.196461732921\\
59.2916666666667	0.2428	-148.88196178917\\
59.2916666666667	0.24446	-151.567461845419\\
59.2916666666667	0.24612	-154.252961901667\\
59.2916666666667	0.24778	-156.938461957916\\
59.2916666666667	0.24944	-159.623962014164\\
59.2916666666667	0.2511	-162.309462070413\\
59.2916666666667	0.25276	-164.994962126662\\
59.2916666666667	0.25442	-167.68046218291\\
59.2916666666667	0.25608	-170.365962239159\\
59.2916666666667	0.25774	-173.051462295407\\
59.2916666666667	0.2594	-175.736962351656\\
59.2916666666667	0.26106	-178.422462407905\\
59.2916666666667	0.26272	-181.107962464153\\
59.2916666666667	0.26438	-183.793462520402\\
59.2916666666667	0.26604	-186.478962576651\\
59.2916666666667	0.2677	-189.164462632899\\
59.2916666666667	0.26936	-191.849962689148\\
59.2916666666667	0.27102	-194.535462745396\\
59.2916666666667	0.27268	-197.220962801645\\
59.2916666666667	0.27434	-199.906462857894\\
59.2916666666667	0.276	-202.591962914142\\
59.5	0.193	-68.0098883802584\\
59.5	0.19466	-70.6997123914526\\
59.5	0.19632	-73.3895364026468\\
59.5	0.19798	-76.0793604138411\\
59.5	0.19964	-78.7691844250353\\
59.5	0.2013	-81.4590084362295\\
59.5	0.20296	-84.1488324474238\\
59.5	0.20462	-86.838656458618\\
59.5	0.20628	-89.5284804698123\\
59.5	0.20794	-92.2183044810065\\
59.5	0.2096	-94.9081284922007\\
59.5	0.21126	-97.597952503395\\
59.5	0.21292	-100.287776514589\\
59.5	0.21458	-102.977600525783\\
59.5	0.21624	-105.667424536978\\
59.5	0.2179	-108.357248548172\\
59.5	0.21956	-111.047072559366\\
59.5	0.22122	-113.73689657056\\
59.5	0.22288	-116.426720581755\\
59.5	0.22454	-119.116544592949\\
59.5	0.2262	-121.806368604143\\
59.5	0.22786	-124.496192615337\\
59.5	0.22952	-127.186016626532\\
59.5	0.23118	-129.875840637726\\
59.5	0.23284	-132.56566464892\\
59.5	0.2345	-135.255488660114\\
59.5	0.23616	-137.945312671308\\
59.5	0.23782	-140.635136682503\\
59.5	0.23948	-143.324960693697\\
59.5	0.24114	-146.014784704891\\
59.5	0.2428	-148.704608716085\\
59.5	0.24446	-151.39443272728\\
59.5	0.24612	-154.084256738474\\
59.5	0.24778	-156.774080749668\\
59.5	0.24944	-159.463904760862\\
59.5	0.2511	-162.153728772057\\
59.5	0.25276	-164.843552783251\\
59.5	0.25442	-167.533376794445\\
59.5	0.25608	-170.223200805639\\
59.5	0.25774	-172.913024816834\\
59.5	0.2594	-175.602848828028\\
59.5	0.26106	-178.292672839222\\
59.5	0.26272	-180.982496850416\\
59.5	0.26438	-183.672320861611\\
59.5	0.26604	-186.362144872805\\
59.5	0.2677	-189.051968883999\\
59.5	0.26936	-191.741792895193\\
59.5	0.27102	-194.431616906388\\
59.5	0.27268	-197.121440917582\\
59.5	0.27434	-199.811264928776\\
59.5	0.276	-202.50108893997\\
59.7083333333333	0.193	-67.702816658805\\
59.7083333333333	0.19466	-70.3969646249449\\
59.7083333333333	0.19632	-73.0911125910847\\
59.7083333333333	0.19798	-75.7852605572246\\
59.7083333333333	0.19964	-78.4794085233645\\
59.7083333333333	0.2013	-81.1735564895043\\
59.7083333333333	0.20296	-83.8677044556442\\
59.7083333333333	0.20462	-86.5618524217841\\
59.7083333333333	0.20628	-89.256000387924\\
59.7083333333333	0.20794	-91.9501483540638\\
59.7083333333333	0.2096	-94.6442963202037\\
59.7083333333333	0.21126	-97.3384442863435\\
59.7083333333333	0.21292	-100.032592252483\\
59.7083333333333	0.21458	-102.726740218623\\
59.7083333333333	0.21624	-105.420888184763\\
59.7083333333333	0.2179	-108.115036150903\\
59.7083333333333	0.21956	-110.809184117043\\
59.7083333333333	0.22122	-113.503332083183\\
59.7083333333333	0.22288	-116.197480049323\\
59.7083333333333	0.22454	-118.891628015462\\
59.7083333333333	0.2262	-121.585775981602\\
59.7083333333333	0.22786	-124.279923947742\\
59.7083333333333	0.22952	-126.974071913882\\
59.7083333333333	0.23118	-129.668219880022\\
59.7083333333333	0.23284	-132.362367846162\\
59.7083333333333	0.2345	-135.056515812302\\
59.7083333333333	0.23616	-137.750663778442\\
59.7083333333333	0.23782	-140.444811744581\\
59.7083333333333	0.23948	-143.138959710721\\
59.7083333333333	0.24114	-145.833107676861\\
59.7083333333333	0.2428	-148.527255643001\\
59.7083333333333	0.24446	-151.221403609141\\
59.7083333333333	0.24612	-153.915551575281\\
59.7083333333333	0.24778	-156.609699541421\\
59.7083333333333	0.24944	-159.30384750756\\
59.7083333333333	0.2511	-161.9979954737\\
59.7083333333333	0.25276	-164.69214343984\\
59.7083333333333	0.25442	-167.38629140598\\
59.7083333333333	0.25608	-170.08043937212\\
59.7083333333333	0.25774	-172.77458733826\\
59.7083333333333	0.2594	-175.4687353044\\
59.7083333333333	0.26106	-178.16288327054\\
59.7083333333333	0.26272	-180.857031236679\\
59.7083333333333	0.26438	-183.551179202819\\
59.7083333333333	0.26604	-186.245327168959\\
59.7083333333333	0.2677	-188.939475135099\\
59.7083333333333	0.26936	-191.633623101239\\
59.7083333333333	0.27102	-194.327771067379\\
59.7083333333333	0.27268	-197.021919033519\\
59.7083333333333	0.27434	-199.716066999658\\
59.7083333333333	0.276	-202.410214965798\\
59.9166666666667	0.193	-67.3957449373516\\
59.9166666666667	0.19466	-70.0942168584371\\
59.9166666666667	0.19632	-72.7926887795226\\
59.9166666666667	0.19798	-75.4911607006081\\
59.9166666666667	0.19964	-78.1896326216936\\
59.9166666666667	0.2013	-80.8881045427791\\
59.9166666666667	0.20296	-83.5865764638646\\
59.9166666666667	0.20462	-86.2850483849501\\
59.9166666666667	0.20628	-88.9835203060356\\
59.9166666666667	0.20794	-91.6819922271211\\
59.9166666666667	0.2096	-94.3804641482066\\
59.9166666666667	0.21126	-97.078936069292\\
59.9166666666667	0.21292	-99.7774079903776\\
59.9166666666667	0.21458	-102.475879911463\\
59.9166666666667	0.21624	-105.174351832549\\
59.9166666666667	0.2179	-107.872823753634\\
59.9166666666667	0.21956	-110.57129567472\\
59.9166666666667	0.22122	-113.269767595805\\
59.9166666666667	0.22288	-115.968239516891\\
59.9166666666667	0.22454	-118.666711437976\\
59.9166666666667	0.2262	-121.365183359062\\
59.9166666666667	0.22786	-124.063655280147\\
59.9166666666667	0.22952	-126.762127201232\\
59.9166666666667	0.23118	-129.460599122318\\
59.9166666666667	0.23284	-132.159071043404\\
59.9166666666667	0.2345	-134.857542964489\\
59.9166666666667	0.23616	-137.556014885574\\
59.9166666666667	0.23782	-140.25448680666\\
59.9166666666667	0.23948	-142.952958727745\\
59.9166666666667	0.24114	-145.651430648831\\
59.9166666666667	0.2428	-148.349902569916\\
59.9166666666667	0.24446	-151.048374491002\\
59.9166666666667	0.24612	-153.746846412087\\
59.9166666666667	0.24778	-156.445318333173\\
59.9166666666667	0.24944	-159.143790254258\\
59.9166666666667	0.2511	-161.842262175344\\
59.9166666666667	0.25276	-164.540734096429\\
59.9166666666667	0.25442	-167.239206017515\\
59.9166666666667	0.25608	-169.9376779386\\
59.9166666666667	0.25774	-172.636149859686\\
59.9166666666667	0.2594	-175.334621780771\\
59.9166666666667	0.26106	-178.033093701857\\
59.9166666666667	0.26272	-180.731565622942\\
59.9166666666667	0.26438	-183.430037544028\\
59.9166666666667	0.26604	-186.128509465113\\
59.9166666666667	0.2677	-188.826981386199\\
59.9166666666667	0.26936	-191.525453307284\\
59.9166666666667	0.27102	-194.22392522837\\
59.9166666666667	0.27268	-196.922397149455\\
59.9166666666667	0.27434	-199.620869070541\\
59.9166666666667	0.276	-202.319340991626\\
60.125	0.193	-67.0886732158982\\
60.125	0.19466	-69.7914690919294\\
60.125	0.19632	-72.4942649679605\\
60.125	0.19798	-75.1970608439916\\
60.125	0.19964	-77.8998567200228\\
60.125	0.2013	-80.6026525960539\\
60.125	0.20296	-83.305448472085\\
60.125	0.20462	-86.0082443481161\\
60.125	0.20628	-88.7110402241473\\
60.125	0.20794	-91.4138361001784\\
60.125	0.2096	-94.1166319762095\\
60.125	0.21126	-96.8194278522406\\
60.125	0.21292	-99.5222237282717\\
60.125	0.21458	-102.225019604303\\
60.125	0.21624	-104.927815480334\\
60.125	0.2179	-107.630611356365\\
60.125	0.21956	-110.333407232396\\
60.125	0.22122	-113.036203108427\\
60.125	0.22288	-115.738998984459\\
60.125	0.22454	-118.44179486049\\
60.125	0.2262	-121.144590736521\\
60.125	0.22786	-123.847386612552\\
60.125	0.22952	-126.550182488583\\
60.125	0.23118	-129.252978364614\\
60.125	0.23284	-131.955774240645\\
60.125	0.2345	-134.658570116676\\
60.125	0.23616	-137.361365992707\\
60.125	0.23782	-140.064161868739\\
60.125	0.23948	-142.76695774477\\
60.125	0.24114	-145.469753620801\\
60.125	0.2428	-148.172549496832\\
60.125	0.24446	-150.875345372863\\
60.125	0.24612	-153.578141248894\\
60.125	0.24778	-156.280937124925\\
60.125	0.24944	-158.983733000957\\
60.125	0.2511	-161.686528876988\\
60.125	0.25276	-164.389324753019\\
60.125	0.25442	-167.09212062905\\
60.125	0.25608	-169.794916505081\\
60.125	0.25774	-172.497712381112\\
60.125	0.2594	-175.200508257143\\
60.125	0.26106	-177.903304133174\\
60.125	0.26272	-180.606100009205\\
60.125	0.26438	-183.308895885237\\
60.125	0.26604	-186.011691761268\\
60.125	0.2677	-188.714487637299\\
60.125	0.26936	-191.41728351333\\
60.125	0.27102	-194.120079389361\\
60.125	0.27268	-196.822875265392\\
60.125	0.27434	-199.525671141423\\
60.125	0.276	-202.228467017455\\
60.3333333333333	0.193	-66.7816014944448\\
60.3333333333333	0.19466	-69.4887213254216\\
60.3333333333333	0.19632	-72.1958411563984\\
60.3333333333333	0.19798	-74.9029609873751\\
60.3333333333333	0.19964	-77.6100808183519\\
60.3333333333333	0.2013	-80.3172006493286\\
60.3333333333333	0.20296	-83.0243204803054\\
60.3333333333333	0.20462	-85.7314403112821\\
60.3333333333333	0.20628	-88.4385601422589\\
60.3333333333333	0.20794	-91.1456799732356\\
60.3333333333333	0.2096	-93.8527998042124\\
60.3333333333333	0.21126	-96.5599196351891\\
60.3333333333333	0.21292	-99.2670394661659\\
60.3333333333333	0.21458	-101.974159297143\\
60.3333333333333	0.21624	-104.681279128119\\
60.3333333333333	0.2179	-107.388398959096\\
60.3333333333333	0.21956	-110.095518790073\\
60.3333333333333	0.22122	-112.80263862105\\
60.3333333333333	0.22288	-115.509758452026\\
60.3333333333333	0.22454	-118.216878283003\\
60.3333333333333	0.2262	-120.92399811398\\
60.3333333333333	0.22786	-123.631117944957\\
60.3333333333333	0.22952	-126.338237775933\\
60.3333333333333	0.23118	-129.04535760691\\
60.3333333333333	0.23284	-131.752477437887\\
60.3333333333333	0.2345	-134.459597268864\\
60.3333333333333	0.23616	-137.16671709984\\
60.3333333333333	0.23782	-139.873836930817\\
60.3333333333333	0.23948	-142.580956761794\\
60.3333333333333	0.24114	-145.288076592771\\
60.3333333333333	0.2428	-147.995196423748\\
60.3333333333333	0.24446	-150.702316254724\\
60.3333333333333	0.24612	-153.409436085701\\
60.3333333333333	0.24778	-156.116555916678\\
60.3333333333333	0.24944	-158.823675747655\\
60.3333333333333	0.2511	-161.530795578631\\
60.3333333333333	0.25276	-164.237915409608\\
60.3333333333333	0.25442	-166.945035240585\\
60.3333333333333	0.25608	-169.652155071562\\
60.3333333333333	0.25774	-172.359274902538\\
60.3333333333333	0.2594	-175.066394733515\\
60.3333333333333	0.26106	-177.773514564492\\
60.3333333333333	0.26272	-180.480634395469\\
60.3333333333333	0.26438	-183.187754226445\\
60.3333333333333	0.26604	-185.894874057422\\
60.3333333333333	0.2677	-188.601993888399\\
60.3333333333333	0.26936	-191.309113719376\\
60.3333333333333	0.27102	-194.016233550352\\
60.3333333333333	0.27268	-196.723353381329\\
60.3333333333333	0.27434	-199.430473212306\\
60.3333333333333	0.276	-202.137593043283\\
60.5416666666667	0.193	-66.4745297729914\\
60.5416666666667	0.19466	-69.1859735589138\\
60.5416666666667	0.19632	-71.8974173448362\\
60.5416666666667	0.19798	-74.6088611307586\\
60.5416666666667	0.19964	-77.320304916681\\
60.5416666666667	0.2013	-80.0317487026034\\
60.5416666666667	0.20296	-82.7431924885258\\
60.5416666666667	0.20462	-85.4546362744481\\
60.5416666666667	0.20628	-88.1660800603705\\
60.5416666666667	0.20794	-90.8775238462929\\
60.5416666666667	0.2096	-93.5889676322153\\
60.5416666666667	0.21126	-96.3004114181377\\
60.5416666666667	0.21292	-99.0118552040601\\
60.5416666666667	0.21458	-101.723298989982\\
60.5416666666667	0.21624	-104.434742775905\\
60.5416666666667	0.2179	-107.146186561827\\
60.5416666666667	0.21956	-109.85763034775\\
60.5416666666667	0.22122	-112.569074133672\\
60.5416666666667	0.22288	-115.280517919594\\
60.5416666666667	0.22454	-117.991961705517\\
60.5416666666667	0.2262	-120.703405491439\\
60.5416666666667	0.22786	-123.414849277362\\
60.5416666666667	0.22952	-126.126293063284\\
60.5416666666667	0.23118	-128.837736849206\\
60.5416666666667	0.23284	-131.549180635129\\
60.5416666666667	0.2345	-134.260624421051\\
60.5416666666667	0.23616	-136.972068206973\\
60.5416666666667	0.23782	-139.683511992896\\
60.5416666666667	0.23948	-142.394955778818\\
60.5416666666667	0.24114	-145.106399564741\\
60.5416666666667	0.2428	-147.817843350663\\
60.5416666666667	0.24446	-150.529287136585\\
60.5416666666667	0.24612	-153.240730922508\\
60.5416666666667	0.24778	-155.95217470843\\
60.5416666666667	0.24944	-158.663618494353\\
60.5416666666667	0.2511	-161.375062280275\\
60.5416666666667	0.25276	-164.086506066197\\
60.5416666666667	0.25442	-166.79794985212\\
60.5416666666667	0.25608	-169.509393638042\\
60.5416666666667	0.25774	-172.220837423964\\
60.5416666666667	0.2594	-174.932281209887\\
60.5416666666667	0.26106	-177.643724995809\\
60.5416666666667	0.26272	-180.355168781732\\
60.5416666666667	0.26438	-183.066612567654\\
60.5416666666667	0.26604	-185.778056353576\\
60.5416666666667	0.2677	-188.489500139499\\
60.5416666666667	0.26936	-191.200943925421\\
60.5416666666667	0.27102	-193.912387711344\\
60.5416666666667	0.27268	-196.623831497266\\
60.5416666666667	0.27434	-199.335275283188\\
60.5416666666667	0.276	-202.046719069111\\
60.75	0.193	-66.1674580515381\\
60.75	0.19466	-68.8832257924061\\
60.75	0.19632	-71.5989935332741\\
60.75	0.19798	-74.3147612741422\\
60.75	0.19964	-77.0305290150102\\
60.75	0.2013	-79.7462967558781\\
60.75	0.20296	-82.4620644967462\\
60.75	0.20462	-85.1778322376142\\
60.75	0.20628	-87.8935999784822\\
60.75	0.20794	-90.6093677193502\\
60.75	0.2096	-93.3251354602182\\
60.75	0.21126	-96.0409032010862\\
60.75	0.21292	-98.7566709419542\\
60.75	0.21458	-101.472438682822\\
60.75	0.21624	-104.18820642369\\
60.75	0.2179	-106.903974164558\\
60.75	0.21956	-109.619741905426\\
60.75	0.22122	-112.335509646294\\
60.75	0.22288	-115.051277387162\\
60.75	0.22454	-117.76704512803\\
60.75	0.2262	-120.482812868898\\
60.75	0.22786	-123.198580609766\\
60.75	0.22952	-125.914348350634\\
60.75	0.23118	-128.630116091502\\
60.75	0.23284	-131.34588383237\\
60.75	0.2345	-134.061651573238\\
60.75	0.23616	-136.777419314106\\
60.75	0.23782	-139.493187054974\\
60.75	0.23948	-142.208954795842\\
60.75	0.24114	-144.924722536711\\
60.75	0.2428	-147.640490277579\\
60.75	0.24446	-150.356258018447\\
60.75	0.24612	-153.072025759314\\
60.75	0.24778	-155.787793500183\\
60.75	0.24944	-158.503561241051\\
60.75	0.2511	-161.219328981919\\
60.75	0.25276	-163.935096722787\\
60.75	0.25442	-166.650864463655\\
60.75	0.25608	-169.366632204523\\
60.75	0.25774	-172.082399945391\\
60.75	0.2594	-174.798167686259\\
60.75	0.26106	-177.513935427127\\
60.75	0.26272	-180.229703167995\\
60.75	0.26438	-182.945470908863\\
60.75	0.26604	-185.661238649731\\
60.75	0.2677	-188.377006390599\\
60.75	0.26936	-191.092774131467\\
60.75	0.27102	-193.808541872335\\
60.75	0.27268	-196.524309613203\\
60.75	0.27434	-199.240077354071\\
60.75	0.276	-201.955845094939\\
60.9583333333333	0.193	-65.8603863300847\\
60.9583333333333	0.19466	-68.5804780258983\\
60.9583333333333	0.19632	-71.300569721712\\
60.9583333333333	0.19798	-74.0206614175256\\
60.9583333333333	0.19964	-76.7407531133393\\
60.9583333333333	0.2013	-79.4608448091529\\
60.9583333333333	0.20296	-82.1809365049666\\
60.9583333333333	0.20462	-84.9010282007802\\
60.9583333333333	0.20628	-87.6211198965939\\
60.9583333333333	0.20794	-90.3412115924075\\
60.9583333333333	0.2096	-93.0613032882211\\
60.9583333333333	0.21126	-95.7813949840347\\
60.9583333333333	0.21292	-98.5014866798484\\
60.9583333333333	0.21458	-101.221578375662\\
60.9583333333333	0.21624	-103.941670071476\\
60.9583333333333	0.2179	-106.661761767289\\
60.9583333333333	0.21956	-109.381853463103\\
60.9583333333333	0.22122	-112.101945158917\\
60.9583333333333	0.22288	-114.82203685473\\
60.9583333333333	0.22454	-117.542128550544\\
60.9583333333333	0.2262	-120.262220246358\\
60.9583333333333	0.22786	-122.982311942171\\
60.9583333333333	0.22952	-125.702403637985\\
60.9583333333333	0.23118	-128.422495333799\\
60.9583333333333	0.23284	-131.142587029612\\
60.9583333333333	0.2345	-133.862678725426\\
60.9583333333333	0.23616	-136.582770421239\\
60.9583333333333	0.23782	-139.302862117053\\
60.9583333333333	0.23948	-142.022953812867\\
60.9583333333333	0.24114	-144.74304550868\\
60.9583333333333	0.2428	-147.463137204494\\
60.9583333333333	0.24446	-150.183228900308\\
60.9583333333333	0.24612	-152.903320596121\\
60.9583333333333	0.24778	-155.623412291935\\
60.9583333333333	0.24944	-158.343503987749\\
60.9583333333333	0.2511	-161.063595683562\\
60.9583333333333	0.25276	-163.783687379376\\
60.9583333333333	0.25442	-166.503779075189\\
60.9583333333333	0.25608	-169.223870771003\\
60.9583333333333	0.25774	-171.943962466817\\
60.9583333333333	0.2594	-174.66405416263\\
60.9583333333333	0.26106	-177.384145858444\\
60.9583333333333	0.26272	-180.104237554258\\
60.9583333333333	0.26438	-182.824329250071\\
60.9583333333333	0.26604	-185.544420945885\\
60.9583333333333	0.2677	-188.264512641699\\
60.9583333333333	0.26936	-190.984604337512\\
60.9583333333333	0.27102	-193.704696033326\\
60.9583333333333	0.27268	-196.42478772914\\
60.9583333333333	0.27434	-199.144879424953\\
60.9583333333333	0.276	-201.864971120767\\
61.1666666666667	0.193	-65.5533146086313\\
61.1666666666667	0.19466	-68.2777302593906\\
61.1666666666667	0.19632	-71.0021459101498\\
61.1666666666667	0.19798	-73.7265615609091\\
61.1666666666667	0.19964	-76.4509772116684\\
61.1666666666667	0.2013	-79.1753928624276\\
61.1666666666667	0.20296	-81.8998085131869\\
61.1666666666667	0.20462	-84.6242241639462\\
61.1666666666667	0.20628	-87.3486398147055\\
61.1666666666667	0.20794	-90.0730554654648\\
61.1666666666667	0.2096	-92.797471116224\\
61.1666666666667	0.21126	-95.5218867669833\\
61.1666666666667	0.21292	-98.2463024177426\\
61.1666666666667	0.21458	-100.970718068502\\
61.1666666666667	0.21624	-103.695133719261\\
61.1666666666667	0.2179	-106.41954937002\\
61.1666666666667	0.21956	-109.14396502078\\
61.1666666666667	0.22122	-111.868380671539\\
61.1666666666667	0.22288	-114.592796322298\\
61.1666666666667	0.22454	-117.317211973057\\
61.1666666666667	0.2262	-120.041627623817\\
61.1666666666667	0.22786	-122.766043274576\\
61.1666666666667	0.22952	-125.490458925335\\
61.1666666666667	0.23118	-128.214874576095\\
61.1666666666667	0.23284	-130.939290226854\\
61.1666666666667	0.2345	-133.663705877613\\
61.1666666666667	0.23616	-136.388121528372\\
61.1666666666667	0.23782	-139.112537179132\\
61.1666666666667	0.23948	-141.836952829891\\
61.1666666666667	0.24114	-144.56136848065\\
61.1666666666667	0.2428	-147.285784131409\\
61.1666666666667	0.24446	-150.010199782169\\
61.1666666666667	0.24612	-152.734615432928\\
61.1666666666667	0.24778	-155.459031083687\\
61.1666666666667	0.24944	-158.183446734447\\
61.1666666666667	0.2511	-160.907862385206\\
61.1666666666667	0.25276	-163.632278035965\\
61.1666666666667	0.25442	-166.356693686724\\
61.1666666666667	0.25608	-169.081109337484\\
61.1666666666667	0.25774	-171.805524988243\\
61.1666666666667	0.2594	-174.529940639002\\
61.1666666666667	0.26106	-177.254356289761\\
61.1666666666667	0.26272	-179.978771940521\\
61.1666666666667	0.26438	-182.70318759128\\
61.1666666666667	0.26604	-185.427603242039\\
61.1666666666667	0.2677	-188.152018892799\\
61.1666666666667	0.26936	-190.876434543558\\
61.1666666666667	0.27102	-193.600850194317\\
61.1666666666667	0.27268	-196.325265845076\\
61.1666666666667	0.27434	-199.049681495836\\
61.1666666666667	0.276	-201.774097146595\\
61.375	0.193	-65.2462428871779\\
61.375	0.19466	-67.9749824928828\\
61.375	0.19632	-70.7037220985877\\
61.375	0.19798	-73.4324617042926\\
61.375	0.19964	-76.1612013099975\\
61.375	0.2013	-78.8899409157024\\
61.375	0.20296	-81.6186805214073\\
61.375	0.20462	-84.3474201271122\\
61.375	0.20628	-87.0761597328172\\
61.375	0.20794	-89.804899338522\\
61.375	0.2096	-92.5336389442269\\
61.375	0.21126	-95.2623785499318\\
61.375	0.21292	-97.9911181556367\\
61.375	0.21458	-100.719857761342\\
61.375	0.21624	-103.448597367047\\
61.375	0.2179	-106.177336972751\\
61.375	0.21956	-108.906076578456\\
61.375	0.22122	-111.634816184161\\
61.375	0.22288	-114.363555789866\\
61.375	0.22454	-117.092295395571\\
61.375	0.2262	-119.821035001276\\
61.375	0.22786	-122.549774606981\\
61.375	0.22952	-125.278514212686\\
61.375	0.23118	-128.007253818391\\
61.375	0.23284	-130.735993424096\\
61.375	0.2345	-133.4647330298\\
61.375	0.23616	-136.193472635505\\
61.375	0.23782	-138.92221224121\\
61.375	0.23948	-141.650951846915\\
61.375	0.24114	-144.37969145262\\
61.375	0.2428	-147.108431058325\\
61.375	0.24446	-149.83717066403\\
61.375	0.24612	-152.565910269735\\
61.375	0.24778	-155.29464987544\\
61.375	0.24944	-158.023389481145\\
61.375	0.2511	-160.752129086849\\
61.375	0.25276	-163.480868692554\\
61.375	0.25442	-166.209608298259\\
61.375	0.25608	-168.938347903964\\
61.375	0.25774	-171.667087509669\\
61.375	0.2594	-174.395827115374\\
61.375	0.26106	-177.124566721079\\
61.375	0.26272	-179.853306326784\\
61.375	0.26438	-182.582045932489\\
61.375	0.26604	-185.310785538194\\
61.375	0.2677	-188.039525143899\\
61.375	0.26936	-190.768264749603\\
61.375	0.27102	-193.497004355308\\
61.375	0.27268	-196.225743961013\\
61.375	0.27434	-198.954483566718\\
61.375	0.276	-201.683223172423\\
61.5833333333333	0.193	-64.9391711657245\\
61.5833333333333	0.19466	-67.6722347263751\\
61.5833333333333	0.19632	-70.4052982870256\\
61.5833333333333	0.19798	-73.1383618476762\\
61.5833333333333	0.19964	-75.8714254083266\\
61.5833333333333	0.2013	-78.6044889689772\\
61.5833333333333	0.20296	-81.3375525296277\\
61.5833333333333	0.20462	-84.0706160902782\\
61.5833333333333	0.20628	-86.8036796509288\\
61.5833333333333	0.20794	-89.5367432115793\\
61.5833333333333	0.2096	-92.2698067722298\\
61.5833333333333	0.21126	-95.0028703328803\\
61.5833333333333	0.21292	-97.7359338935309\\
61.5833333333333	0.21458	-100.468997454181\\
61.5833333333333	0.21624	-103.202061014832\\
61.5833333333333	0.2179	-105.935124575482\\
61.5833333333333	0.21956	-108.668188136133\\
61.5833333333333	0.22122	-111.401251696784\\
61.5833333333333	0.22288	-114.134315257434\\
61.5833333333333	0.22454	-116.867378818085\\
61.5833333333333	0.2262	-119.600442378735\\
61.5833333333333	0.22786	-122.333505939386\\
61.5833333333333	0.22952	-125.066569500036\\
61.5833333333333	0.23118	-127.799633060687\\
61.5833333333333	0.23284	-130.532696621337\\
61.5833333333333	0.2345	-133.265760181988\\
61.5833333333333	0.23616	-135.998823742638\\
61.5833333333333	0.23782	-138.731887303289\\
61.5833333333333	0.23948	-141.464950863939\\
61.5833333333333	0.24114	-144.19801442459\\
61.5833333333333	0.2428	-146.93107798524\\
61.5833333333333	0.24446	-149.664141545891\\
61.5833333333333	0.24612	-152.397205106542\\
61.5833333333333	0.24778	-155.130268667192\\
61.5833333333333	0.24944	-157.863332227843\\
61.5833333333333	0.2511	-160.596395788493\\
61.5833333333333	0.25276	-163.329459349144\\
61.5833333333333	0.25442	-166.062522909794\\
61.5833333333333	0.25608	-168.795586470445\\
61.5833333333333	0.25774	-171.528650031095\\
61.5833333333333	0.2594	-174.261713591746\\
61.5833333333333	0.26106	-176.994777152396\\
61.5833333333333	0.26272	-179.727840713047\\
61.5833333333333	0.26438	-182.460904273697\\
61.5833333333333	0.26604	-185.193967834348\\
61.5833333333333	0.2677	-187.927031394998\\
61.5833333333333	0.26936	-190.660094955649\\
61.5833333333333	0.27102	-193.3931585163\\
61.5833333333333	0.27268	-196.12622207695\\
61.5833333333333	0.27434	-198.859285637601\\
61.5833333333333	0.276	-201.592349198251\\
61.7916666666667	0.193	-64.6320994442711\\
61.7916666666667	0.19466	-67.3694869598673\\
61.7916666666667	0.19632	-70.1068744754635\\
61.7916666666667	0.19798	-72.8442619910596\\
61.7916666666667	0.19964	-75.5816495066558\\
61.7916666666667	0.2013	-78.3190370222519\\
61.7916666666667	0.20296	-81.0564245378481\\
61.7916666666667	0.20462	-83.7938120534442\\
61.7916666666667	0.20628	-86.5311995690404\\
61.7916666666667	0.20794	-89.2685870846366\\
61.7916666666667	0.2096	-92.0059746002327\\
61.7916666666667	0.21126	-94.7433621158289\\
61.7916666666667	0.21292	-97.4807496314251\\
61.7916666666667	0.21458	-100.218137147021\\
61.7916666666667	0.21624	-102.955524662617\\
61.7916666666667	0.2179	-105.692912178214\\
61.7916666666667	0.21956	-108.43029969381\\
61.7916666666667	0.22122	-111.167687209406\\
61.7916666666667	0.22288	-113.905074725002\\
61.7916666666667	0.22454	-116.642462240598\\
61.7916666666667	0.2262	-119.379849756194\\
61.7916666666667	0.22786	-122.117237271791\\
61.7916666666667	0.22952	-124.854624787387\\
61.7916666666667	0.23118	-127.592012302983\\
61.7916666666667	0.23284	-130.329399818579\\
61.7916666666667	0.2345	-133.066787334175\\
61.7916666666667	0.23616	-135.804174849771\\
61.7916666666667	0.23782	-138.541562365367\\
61.7916666666667	0.23948	-141.278949880964\\
61.7916666666667	0.24114	-144.01633739656\\
61.7916666666667	0.2428	-146.753724912156\\
61.7916666666667	0.24446	-149.491112427752\\
61.7916666666667	0.24612	-152.228499943348\\
61.7916666666667	0.24778	-154.965887458944\\
61.7916666666667	0.24944	-157.703274974541\\
61.7916666666667	0.2511	-160.440662490137\\
61.7916666666667	0.25276	-163.178050005733\\
61.7916666666667	0.25442	-165.915437521329\\
61.7916666666667	0.25608	-168.652825036925\\
61.7916666666667	0.25774	-171.390212552521\\
61.7916666666667	0.2594	-174.127600068118\\
61.7916666666667	0.26106	-176.864987583714\\
61.7916666666667	0.26272	-179.60237509931\\
61.7916666666667	0.26438	-182.339762614906\\
61.7916666666667	0.26604	-185.077150130502\\
61.7916666666667	0.2677	-187.814537646098\\
61.7916666666667	0.26936	-190.551925161695\\
61.7916666666667	0.27102	-193.289312677291\\
61.7916666666667	0.27268	-196.026700192887\\
61.7916666666667	0.27434	-198.764087708483\\
61.7916666666667	0.276	-201.501475224079\\
62	0.193	-64.3250277228178\\
62	0.19466	-67.0667391933596\\
62	0.19632	-69.8084506639013\\
62	0.19798	-72.5501621344432\\
62	0.19964	-75.2918736049849\\
62	0.2013	-78.0335850755267\\
62	0.20296	-80.7752965460685\\
62	0.20462	-83.5170080166103\\
62	0.20628	-86.2587194871521\\
62	0.20794	-89.0004309576939\\
62	0.2096	-91.7421424282357\\
62	0.21126	-94.4838538987775\\
62	0.21292	-97.2255653693192\\
62	0.21458	-99.967276839861\\
62	0.21624	-102.708988310403\\
62	0.2179	-105.450699780945\\
62	0.21956	-108.192411251486\\
62	0.22122	-110.934122722028\\
62	0.22288	-113.67583419257\\
62	0.22454	-116.417545663112\\
62	0.2262	-119.159257133654\\
62	0.22786	-121.900968604195\\
62	0.22952	-124.642680074737\\
62	0.23118	-127.384391545279\\
62	0.23284	-130.126103015821\\
62	0.2345	-132.867814486363\\
62	0.23616	-135.609525956904\\
62	0.23782	-138.351237427446\\
62	0.23948	-141.092948897988\\
62	0.24114	-143.83466036853\\
62	0.2428	-146.576371839071\\
62	0.24446	-149.318083309613\\
62	0.24612	-152.059794780155\\
62	0.24778	-154.801506250697\\
62	0.24944	-157.543217721239\\
62	0.2511	-160.28492919178\\
62	0.25276	-163.026640662322\\
62	0.25442	-165.768352132864\\
62	0.25608	-168.510063603406\\
62	0.25774	-171.251775073948\\
62	0.2594	-173.993486544489\\
62	0.26106	-176.735198015031\\
62	0.26272	-179.476909485573\\
62	0.26438	-182.218620956115\\
62	0.26604	-184.960332426657\\
62	0.2677	-187.702043897198\\
62	0.26936	-190.44375536774\\
62	0.27102	-193.185466838282\\
62	0.27268	-195.927178308824\\
62	0.27434	-198.668889779366\\
62	0.276	-201.410601249907\\
62.2083333333333	0.193	-64.0179560013644\\
62.2083333333333	0.19466	-66.7639914268518\\
62.2083333333333	0.19632	-69.5100268523392\\
62.2083333333333	0.19798	-72.2560622778267\\
62.2083333333333	0.19964	-75.0020977033141\\
62.2083333333333	0.2013	-77.7481331288015\\
62.2083333333333	0.20296	-80.4941685542889\\
62.2083333333333	0.20462	-83.2402039797763\\
62.2083333333333	0.20628	-85.9862394052637\\
62.2083333333333	0.20794	-88.7322748307512\\
62.2083333333333	0.2096	-91.4783102562386\\
62.2083333333333	0.21126	-94.224345681726\\
62.2083333333333	0.21292	-96.9703811072134\\
62.2083333333333	0.21458	-99.7164165327008\\
62.2083333333333	0.21624	-102.462451958188\\
62.2083333333333	0.2179	-105.208487383676\\
62.2083333333333	0.21956	-107.954522809163\\
62.2083333333333	0.22122	-110.700558234651\\
62.2083333333333	0.22288	-113.446593660138\\
62.2083333333333	0.22454	-116.192629085625\\
62.2083333333333	0.2262	-118.938664511113\\
62.2083333333333	0.22786	-121.6846999366\\
62.2083333333333	0.22952	-124.430735362088\\
62.2083333333333	0.23118	-127.176770787575\\
62.2083333333333	0.23284	-129.922806213062\\
62.2083333333333	0.2345	-132.66884163855\\
62.2083333333333	0.23616	-135.414877064037\\
62.2083333333333	0.23782	-138.160912489525\\
62.2083333333333	0.23948	-140.906947915012\\
62.2083333333333	0.24114	-143.6529833405\\
62.2083333333333	0.2428	-146.399018765987\\
62.2083333333333	0.24446	-149.145054191474\\
62.2083333333333	0.24612	-151.891089616962\\
62.2083333333333	0.24778	-154.637125042449\\
62.2083333333333	0.24944	-157.383160467937\\
62.2083333333333	0.2511	-160.129195893424\\
62.2083333333333	0.25276	-162.875231318911\\
62.2083333333333	0.25442	-165.621266744399\\
62.2083333333333	0.25608	-168.367302169886\\
62.2083333333333	0.25774	-171.113337595374\\
62.2083333333333	0.2594	-173.859373020861\\
62.2083333333333	0.26106	-176.605408446349\\
62.2083333333333	0.26272	-179.351443871836\\
62.2083333333333	0.26438	-182.097479297323\\
62.2083333333333	0.26604	-184.843514722811\\
62.2083333333333	0.2677	-187.589550148298\\
62.2083333333333	0.26936	-190.335585573786\\
62.2083333333333	0.27102	-193.081620999273\\
62.2083333333333	0.27268	-195.827656424761\\
62.2083333333333	0.27434	-198.573691850248\\
62.2083333333333	0.276	-201.319727275735\\
62.4166666666667	0.193	-63.710884279911\\
62.4166666666667	0.19466	-66.461243660344\\
62.4166666666667	0.19632	-69.2116030407771\\
62.4166666666667	0.19798	-71.9619624212102\\
62.4166666666667	0.19964	-74.7123218016432\\
62.4166666666667	0.2013	-77.4626811820762\\
62.4166666666667	0.20296	-80.2130405625093\\
62.4166666666667	0.20462	-82.9633999429423\\
62.4166666666667	0.20628	-85.7137593233754\\
62.4166666666667	0.20794	-88.4641187038084\\
62.4166666666667	0.2096	-91.2144780842415\\
62.4166666666667	0.21126	-93.9648374646745\\
62.4166666666667	0.21292	-96.7151968451076\\
62.4166666666667	0.21458	-99.4655562255406\\
62.4166666666667	0.21624	-102.215915605974\\
62.4166666666667	0.2179	-104.966274986407\\
62.4166666666667	0.21956	-107.71663436684\\
62.4166666666667	0.22122	-110.466993747273\\
62.4166666666667	0.22288	-113.217353127706\\
62.4166666666667	0.22454	-115.967712508139\\
62.4166666666667	0.2262	-118.718071888572\\
62.4166666666667	0.22786	-121.468431269005\\
62.4166666666667	0.22952	-124.218790649438\\
62.4166666666667	0.23118	-126.969150029871\\
62.4166666666667	0.23284	-129.719509410304\\
62.4166666666667	0.2345	-132.469868790737\\
62.4166666666667	0.23616	-135.22022817117\\
62.4166666666667	0.23782	-137.970587551603\\
62.4166666666667	0.23948	-140.720946932036\\
62.4166666666667	0.24114	-143.471306312469\\
62.4166666666667	0.2428	-146.221665692902\\
62.4166666666667	0.24446	-148.972025073336\\
62.4166666666667	0.24612	-151.722384453769\\
62.4166666666667	0.24778	-154.472743834202\\
62.4166666666667	0.24944	-157.223103214635\\
62.4166666666667	0.2511	-159.973462595068\\
62.4166666666667	0.25276	-162.723821975501\\
62.4166666666667	0.25442	-165.474181355934\\
62.4166666666667	0.25608	-168.224540736367\\
62.4166666666667	0.25774	-170.9749001168\\
62.4166666666667	0.2594	-173.725259497233\\
62.4166666666667	0.26106	-176.475618877666\\
62.4166666666667	0.26272	-179.225978258099\\
62.4166666666667	0.26438	-181.976337638532\\
62.4166666666667	0.26604	-184.726697018965\\
62.4166666666667	0.2677	-187.477056399398\\
62.4166666666667	0.26936	-190.227415779831\\
62.4166666666667	0.27102	-192.977775160264\\
62.4166666666667	0.27268	-195.728134540697\\
62.4166666666667	0.27434	-198.47849392113\\
62.4166666666667	0.276	-201.228853301563\\
62.625	0.193	-63.4038125584576\\
62.625	0.19466	-66.1584958938363\\
62.625	0.19632	-68.9131792292149\\
62.625	0.19798	-71.6678625645937\\
62.625	0.19964	-74.4225458999723\\
62.625	0.2013	-77.177229235351\\
62.625	0.20296	-79.9319125707297\\
62.625	0.20462	-82.6865959061083\\
62.625	0.20628	-85.4412792414871\\
62.625	0.20794	-88.1959625768657\\
62.625	0.2096	-90.9506459122444\\
62.625	0.21126	-93.705329247623\\
62.625	0.21292	-96.4600125830017\\
62.625	0.21458	-99.2146959183804\\
62.625	0.21624	-101.969379253759\\
62.625	0.2179	-104.724062589138\\
62.625	0.21956	-107.478745924516\\
62.625	0.22122	-110.233429259895\\
62.625	0.22288	-112.988112595274\\
62.625	0.22454	-115.742795930653\\
62.625	0.2262	-118.497479266031\\
62.625	0.22786	-121.25216260141\\
62.625	0.22952	-124.006845936788\\
62.625	0.23118	-126.761529272167\\
62.625	0.23284	-129.516212607546\\
62.625	0.2345	-132.270895942925\\
62.625	0.23616	-135.025579278303\\
62.625	0.23782	-137.780262613682\\
62.625	0.23948	-140.534945949061\\
62.625	0.24114	-143.289629284439\\
62.625	0.2428	-146.044312619818\\
62.625	0.24446	-148.798995955197\\
62.625	0.24612	-151.553679290575\\
62.625	0.24778	-154.308362625954\\
62.625	0.24944	-157.063045961333\\
62.625	0.2511	-159.817729296711\\
62.625	0.25276	-162.57241263209\\
62.625	0.25442	-165.327095967469\\
62.625	0.25608	-168.081779302847\\
62.625	0.25774	-170.836462638226\\
62.625	0.2594	-173.591145973605\\
62.625	0.26106	-176.345829308983\\
62.625	0.26272	-179.100512644362\\
62.625	0.26438	-181.855195979741\\
62.625	0.26604	-184.60987931512\\
62.625	0.2677	-187.364562650498\\
62.625	0.26936	-190.119245985877\\
62.625	0.27102	-192.873929321256\\
62.625	0.27268	-195.628612656634\\
62.625	0.27434	-198.383295992013\\
62.625	0.276	-201.137979327392\\
62.8333333333333	0.193	-63.0967408370042\\
62.8333333333333	0.19466	-65.8557481273285\\
62.8333333333333	0.19632	-68.6147554176528\\
62.8333333333333	0.19798	-71.3737627079772\\
62.8333333333333	0.19964	-74.1327699983015\\
62.8333333333333	0.2013	-76.8917772886257\\
62.8333333333333	0.20296	-79.6507845789501\\
62.8333333333333	0.20462	-82.4097918692744\\
62.8333333333333	0.20628	-85.1687991595987\\
62.8333333333333	0.20794	-87.927806449923\\
62.8333333333333	0.2096	-90.6868137402473\\
62.8333333333333	0.21126	-93.4458210305716\\
62.8333333333333	0.21292	-96.2048283208959\\
62.8333333333333	0.21458	-98.9638356112202\\
62.8333333333333	0.21624	-101.722842901545\\
62.8333333333333	0.2179	-104.481850191869\\
62.8333333333333	0.21956	-107.240857482193\\
62.8333333333333	0.22122	-109.999864772517\\
62.8333333333333	0.22288	-112.758872062842\\
62.8333333333333	0.22454	-115.517879353166\\
62.8333333333333	0.2262	-118.27688664349\\
62.8333333333333	0.22786	-121.035893933815\\
62.8333333333333	0.22952	-123.794901224139\\
62.8333333333333	0.23118	-126.553908514463\\
62.8333333333333	0.23284	-129.312915804788\\
62.8333333333333	0.2345	-132.071923095112\\
62.8333333333333	0.23616	-134.830930385436\\
62.8333333333333	0.23782	-137.589937675761\\
62.8333333333333	0.23948	-140.348944966085\\
62.8333333333333	0.24114	-143.107952256409\\
62.8333333333333	0.2428	-145.866959546733\\
62.8333333333333	0.24446	-148.625966837058\\
62.8333333333333	0.24612	-151.384974127382\\
62.8333333333333	0.24778	-154.143981417706\\
62.8333333333333	0.24944	-156.902988708031\\
62.8333333333333	0.2511	-159.661995998355\\
62.8333333333333	0.25276	-162.421003288679\\
62.8333333333333	0.25442	-165.180010579004\\
62.8333333333333	0.25608	-167.939017869328\\
62.8333333333333	0.25774	-170.698025159652\\
62.8333333333333	0.2594	-173.457032449977\\
62.8333333333333	0.26106	-176.216039740301\\
62.8333333333333	0.26272	-178.975047030625\\
62.8333333333333	0.26438	-181.734054320949\\
62.8333333333333	0.26604	-184.493061611274\\
62.8333333333333	0.2677	-187.252068901598\\
62.8333333333333	0.26936	-190.011076191922\\
62.8333333333333	0.27102	-192.770083482247\\
62.8333333333333	0.27268	-195.529090772571\\
62.8333333333333	0.27434	-198.288098062895\\
62.8333333333333	0.276	-201.04710535322\\
63.0416666666667	0.193	-62.7896691155508\\
63.0416666666667	0.19466	-65.5530003608208\\
63.0416666666667	0.19632	-68.3163316060907\\
63.0416666666667	0.19798	-71.0796628513607\\
63.0416666666667	0.19964	-73.8429940966306\\
63.0416666666667	0.2013	-76.6063253419005\\
63.0416666666667	0.20296	-79.3696565871705\\
63.0416666666667	0.20462	-82.1329878324404\\
63.0416666666667	0.20628	-84.8963190777104\\
63.0416666666667	0.20794	-87.6596503229803\\
63.0416666666667	0.2096	-90.4229815682502\\
63.0416666666667	0.21126	-93.1863128135201\\
63.0416666666667	0.21292	-95.9496440587901\\
63.0416666666667	0.21458	-98.71297530406\\
63.0416666666667	0.21624	-101.47630654933\\
63.0416666666667	0.2179	-104.2396377946\\
63.0416666666667	0.21956	-107.00296903987\\
63.0416666666667	0.22122	-109.76630028514\\
63.0416666666667	0.22288	-112.52963153041\\
63.0416666666667	0.22454	-115.29296277568\\
63.0416666666667	0.2262	-118.05629402095\\
63.0416666666667	0.22786	-120.81962526622\\
63.0416666666667	0.22952	-123.582956511489\\
63.0416666666667	0.23118	-126.346287756759\\
63.0416666666667	0.23284	-129.109619002029\\
63.0416666666667	0.2345	-131.872950247299\\
63.0416666666667	0.23616	-134.636281492569\\
63.0416666666667	0.23782	-137.399612737839\\
63.0416666666667	0.23948	-140.162943983109\\
63.0416666666667	0.24114	-142.926275228379\\
63.0416666666667	0.2428	-145.689606473649\\
63.0416666666667	0.24446	-148.452937718919\\
63.0416666666667	0.24612	-151.216268964189\\
63.0416666666667	0.24778	-153.979600209459\\
63.0416666666667	0.24944	-156.742931454729\\
63.0416666666667	0.2511	-159.506262699999\\
63.0416666666667	0.25276	-162.269593945269\\
63.0416666666667	0.25442	-165.032925190539\\
63.0416666666667	0.25608	-167.796256435808\\
63.0416666666667	0.25774	-170.559587681078\\
63.0416666666667	0.2594	-173.322918926348\\
63.0416666666667	0.26106	-176.086250171618\\
63.0416666666667	0.26272	-178.849581416888\\
63.0416666666667	0.26438	-181.612912662158\\
63.0416666666667	0.26604	-184.376243907428\\
63.0416666666667	0.2677	-187.139575152698\\
63.0416666666667	0.26936	-189.902906397968\\
63.0416666666667	0.27102	-192.666237643238\\
63.0416666666667	0.27268	-195.429568888508\\
63.0416666666667	0.27434	-198.192900133778\\
63.0416666666667	0.276	-200.956231379048\\
63.25	0.193	-62.4825973940975\\
63.25	0.19466	-65.250252594313\\
63.25	0.19632	-68.0179077945286\\
63.25	0.19798	-70.7855629947442\\
63.25	0.19964	-73.5532181949598\\
63.25	0.2013	-76.3208733951753\\
63.25	0.20296	-79.0885285953909\\
63.25	0.20462	-81.8561837956064\\
63.25	0.20628	-84.623838995822\\
63.25	0.20794	-87.3914941960376\\
63.25	0.2096	-90.1591493962531\\
63.25	0.21126	-92.9268045964687\\
63.25	0.21292	-95.6944597966843\\
63.25	0.21458	-98.4621149968998\\
63.25	0.21624	-101.229770197115\\
63.25	0.2179	-103.997425397331\\
63.25	0.21956	-106.765080597547\\
63.25	0.22122	-109.532735797762\\
63.25	0.22288	-112.300390997978\\
63.25	0.22454	-115.068046198193\\
63.25	0.2262	-117.835701398409\\
63.25	0.22786	-120.603356598624\\
63.25	0.22952	-123.37101179884\\
63.25	0.23118	-126.138666999056\\
63.25	0.23284	-128.906322199271\\
63.25	0.2345	-131.673977399487\\
63.25	0.23616	-134.441632599702\\
63.25	0.23782	-137.209287799918\\
63.25	0.23948	-139.976943000133\\
63.25	0.24114	-142.744598200349\\
63.25	0.2428	-145.512253400565\\
63.25	0.24446	-148.27990860078\\
63.25	0.24612	-151.047563800996\\
63.25	0.24778	-153.815219001211\\
63.25	0.24944	-156.582874201427\\
63.25	0.2511	-159.350529401642\\
63.25	0.25276	-162.118184601858\\
63.25	0.25442	-164.885839802074\\
63.25	0.25608	-167.653495002289\\
63.25	0.25774	-170.421150202505\\
63.25	0.2594	-173.18880540272\\
63.25	0.26106	-175.956460602936\\
63.25	0.26272	-178.724115803151\\
63.25	0.26438	-181.491771003367\\
63.25	0.26604	-184.259426203583\\
63.25	0.2677	-187.027081403798\\
63.25	0.26936	-189.794736604014\\
63.25	0.27102	-192.562391804229\\
63.25	0.27268	-195.330047004445\\
63.25	0.27434	-198.09770220466\\
63.25	0.276	-200.865357404876\\
63.4583333333333	0.193	-62.175525672644\\
63.4583333333333	0.19466	-64.9475048278052\\
63.4583333333333	0.19632	-67.7194839829664\\
63.4583333333333	0.19798	-70.4914631381276\\
63.4583333333333	0.19964	-73.2634422932888\\
63.4583333333333	0.2013	-76.03542144845\\
63.4583333333333	0.20296	-78.8074006036112\\
63.4583333333333	0.20462	-81.5793797587724\\
63.4583333333333	0.20628	-84.3513589139336\\
63.4583333333333	0.20794	-87.1233380690948\\
63.4583333333333	0.2096	-89.895317224256\\
63.4583333333333	0.21126	-92.6672963794172\\
63.4583333333333	0.21292	-95.4392755345784\\
63.4583333333333	0.21458	-98.2112546897396\\
63.4583333333333	0.21624	-100.983233844901\\
63.4583333333333	0.2179	-103.755213000062\\
63.4583333333333	0.21956	-106.527192155223\\
63.4583333333333	0.22122	-109.299171310384\\
63.4583333333333	0.22288	-112.071150465546\\
63.4583333333333	0.22454	-114.843129620707\\
63.4583333333333	0.2262	-117.615108775868\\
63.4583333333333	0.22786	-120.387087931029\\
63.4583333333333	0.22952	-123.15906708619\\
63.4583333333333	0.23118	-125.931046241352\\
63.4583333333333	0.23284	-128.703025396513\\
63.4583333333333	0.2345	-131.475004551674\\
63.4583333333333	0.23616	-134.246983706835\\
63.4583333333333	0.23782	-137.018962861996\\
63.4583333333333	0.23948	-139.790942017158\\
63.4583333333333	0.24114	-142.562921172319\\
63.4583333333333	0.2428	-145.33490032748\\
63.4583333333333	0.24446	-148.106879482641\\
63.4583333333333	0.24612	-150.878858637802\\
63.4583333333333	0.24778	-153.650837792964\\
63.4583333333333	0.24944	-156.422816948125\\
63.4583333333333	0.2511	-159.194796103286\\
63.4583333333333	0.25276	-161.966775258447\\
63.4583333333333	0.25442	-164.738754413608\\
63.4583333333333	0.25608	-167.51073356877\\
63.4583333333333	0.25774	-170.282712723931\\
63.4583333333333	0.2594	-173.054691879092\\
63.4583333333333	0.26106	-175.826671034253\\
63.4583333333333	0.26272	-178.598650189414\\
63.4583333333333	0.26438	-181.370629344575\\
63.4583333333333	0.26604	-184.142608499737\\
63.4583333333333	0.2677	-186.914587654898\\
63.4583333333333	0.26936	-189.686566810059\\
63.4583333333333	0.27102	-192.45854596522\\
63.4583333333333	0.27268	-195.230525120382\\
63.4583333333333	0.27434	-198.002504275543\\
63.4583333333333	0.276	-200.774483430704\\
63.6666666666667	0.193	-61.8684539511907\\
63.6666666666667	0.19466	-64.6447570612975\\
63.6666666666667	0.19632	-67.4210601714043\\
63.6666666666667	0.19798	-70.1973632815112\\
63.6666666666667	0.19964	-72.973666391618\\
63.6666666666667	0.2013	-75.7499695017248\\
63.6666666666667	0.20296	-78.5262726118317\\
63.6666666666667	0.20462	-81.3025757219385\\
63.6666666666667	0.20628	-84.0788788320453\\
63.6666666666667	0.20794	-86.8551819421522\\
63.6666666666667	0.2096	-89.631485052259\\
63.6666666666667	0.21126	-92.4077881623658\\
63.6666666666667	0.21292	-95.1840912724726\\
63.6666666666667	0.21458	-97.9603943825794\\
63.6666666666667	0.21624	-100.736697492686\\
63.6666666666667	0.2179	-103.513000602793\\
63.6666666666667	0.21956	-106.2893037129\\
63.6666666666667	0.22122	-109.065606823007\\
63.6666666666667	0.22288	-111.841909933114\\
63.6666666666667	0.22454	-114.61821304322\\
63.6666666666667	0.2262	-117.394516153327\\
63.6666666666667	0.22786	-120.170819263434\\
63.6666666666667	0.22952	-122.947122373541\\
63.6666666666667	0.23118	-125.723425483648\\
63.6666666666667	0.23284	-128.499728593755\\
63.6666666666667	0.2345	-131.276031703861\\
63.6666666666667	0.23616	-134.052334813968\\
63.6666666666667	0.23782	-136.828637924075\\
63.6666666666667	0.23948	-139.604941034182\\
63.6666666666667	0.24114	-142.381244144289\\
63.6666666666667	0.2428	-145.157547254396\\
63.6666666666667	0.24446	-147.933850364502\\
63.6666666666667	0.24612	-150.710153474609\\
63.6666666666667	0.24778	-153.486456584716\\
63.6666666666667	0.24944	-156.262759694823\\
63.6666666666667	0.2511	-159.03906280493\\
63.6666666666667	0.25276	-161.815365915036\\
63.6666666666667	0.25442	-164.591669025143\\
63.6666666666667	0.25608	-167.36797213525\\
63.6666666666667	0.25774	-170.144275245357\\
63.6666666666667	0.2594	-172.920578355464\\
63.6666666666667	0.26106	-175.696881465571\\
63.6666666666667	0.26272	-178.473184575677\\
63.6666666666667	0.26438	-181.249487685784\\
63.6666666666667	0.26604	-184.025790795891\\
63.6666666666667	0.2677	-186.802093905998\\
63.6666666666667	0.26936	-189.578397016105\\
63.6666666666667	0.27102	-192.354700126212\\
63.6666666666667	0.27268	-195.131003236318\\
63.6666666666667	0.27434	-197.907306346425\\
63.6666666666667	0.276	-200.683609456532\\
63.875	0.193	-61.5613822297373\\
63.875	0.19466	-64.3420092947897\\
63.875	0.19632	-67.1226363598422\\
63.875	0.19798	-69.9032634248947\\
63.875	0.19964	-72.6838904899471\\
63.875	0.2013	-75.4645175549996\\
63.875	0.20296	-78.245144620052\\
63.875	0.20462	-81.0257716851045\\
63.875	0.20628	-83.806398750157\\
63.875	0.20794	-86.5870258152094\\
63.875	0.2096	-89.3676528802618\\
63.875	0.21126	-92.1482799453143\\
63.875	0.21292	-94.9289070103667\\
63.875	0.21458	-97.7095340754192\\
63.875	0.21624	-100.490161140472\\
63.875	0.2179	-103.270788205524\\
63.875	0.21956	-106.051415270577\\
63.875	0.22122	-108.832042335629\\
63.875	0.22288	-111.612669400682\\
63.875	0.22454	-114.393296465734\\
63.875	0.2262	-117.173923530786\\
63.875	0.22786	-119.954550595839\\
63.875	0.22952	-122.735177660891\\
63.875	0.23118	-125.515804725944\\
63.875	0.23284	-128.296431790996\\
63.875	0.2345	-131.077058856049\\
63.875	0.23616	-133.857685921101\\
63.875	0.23782	-136.638312986154\\
63.875	0.23948	-139.418940051206\\
63.875	0.24114	-142.199567116259\\
63.875	0.2428	-144.980194181311\\
63.875	0.24446	-147.760821246363\\
63.875	0.24612	-150.541448311416\\
63.875	0.24778	-153.322075376468\\
63.875	0.24944	-156.102702441521\\
63.875	0.2511	-158.883329506573\\
63.875	0.25276	-161.663956571626\\
63.875	0.25442	-164.444583636678\\
63.875	0.25608	-167.225210701731\\
63.875	0.25774	-170.005837766783\\
63.875	0.2594	-172.786464831836\\
63.875	0.26106	-175.567091896888\\
63.875	0.26272	-178.34771896194\\
63.875	0.26438	-181.128346026993\\
63.875	0.26604	-183.908973092045\\
63.875	0.2677	-186.689600157098\\
63.875	0.26936	-189.47022722215\\
63.875	0.27102	-192.250854287203\\
63.875	0.27268	-195.031481352255\\
63.875	0.27434	-197.812108417308\\
63.875	0.276	-200.59273548236\\
64.0833333333333	0.193	-61.2543105082839\\
64.0833333333333	0.19466	-64.039261528282\\
64.0833333333333	0.19632	-66.8242125482801\\
64.0833333333333	0.19798	-69.6091635682782\\
64.0833333333333	0.19964	-72.3941145882762\\
64.0833333333333	0.2013	-75.1790656082743\\
64.0833333333333	0.20296	-77.9640166282724\\
64.0833333333333	0.20462	-80.7489676482705\\
64.0833333333333	0.20628	-83.5339186682686\\
64.0833333333333	0.20794	-86.3188696882667\\
64.0833333333333	0.2096	-89.1038207082647\\
64.0833333333333	0.21126	-91.8887717282628\\
64.0833333333333	0.21292	-94.6737227482609\\
64.0833333333333	0.21458	-97.458673768259\\
64.0833333333333	0.21624	-100.243624788257\\
64.0833333333333	0.2179	-103.028575808255\\
64.0833333333333	0.21956	-105.813526828253\\
64.0833333333333	0.22122	-108.598477848251\\
64.0833333333333	0.22288	-111.383428868249\\
64.0833333333333	0.22454	-114.168379888248\\
64.0833333333333	0.2262	-116.953330908246\\
64.0833333333333	0.22786	-119.738281928244\\
64.0833333333333	0.22952	-122.523232948242\\
64.0833333333333	0.23118	-125.30818396824\\
64.0833333333333	0.23284	-128.093134988238\\
64.0833333333333	0.2345	-130.878086008236\\
64.0833333333333	0.23616	-133.663037028234\\
64.0833333333333	0.23782	-136.447988048232\\
64.0833333333333	0.23948	-139.23293906823\\
64.0833333333333	0.24114	-142.017890088228\\
64.0833333333333	0.2428	-144.802841108226\\
64.0833333333333	0.24446	-147.587792128225\\
64.0833333333333	0.24612	-150.372743148223\\
64.0833333333333	0.24778	-153.157694168221\\
64.0833333333333	0.24944	-155.942645188219\\
64.0833333333333	0.2511	-158.727596208217\\
64.0833333333333	0.25276	-161.512547228215\\
64.0833333333333	0.25442	-164.297498248213\\
64.0833333333333	0.25608	-167.082449268211\\
64.0833333333333	0.25774	-169.867400288209\\
64.0833333333333	0.2594	-172.652351308207\\
64.0833333333333	0.26106	-175.437302328205\\
64.0833333333333	0.26272	-178.222253348203\\
64.0833333333333	0.26438	-181.007204368202\\
64.0833333333333	0.26604	-183.7921553882\\
64.0833333333333	0.2677	-186.577106408198\\
64.0833333333333	0.26936	-189.362057428196\\
64.0833333333333	0.27102	-192.147008448194\\
64.0833333333333	0.27268	-194.931959468192\\
64.0833333333333	0.27434	-197.71691048819\\
64.0833333333333	0.276	-200.501861508188\\
64.2916666666667	0.193	-60.9472387868305\\
64.2916666666667	0.19466	-63.7365137617742\\
64.2916666666667	0.19632	-66.5257887367179\\
64.2916666666667	0.19798	-69.3150637116617\\
64.2916666666667	0.19964	-72.1043386866054\\
64.2916666666667	0.2013	-74.8936136615491\\
64.2916666666667	0.20296	-77.6828886364928\\
64.2916666666667	0.20462	-80.4721636114365\\
64.2916666666667	0.20628	-83.2614385863803\\
64.2916666666667	0.20794	-86.050713561324\\
64.2916666666667	0.2096	-88.8399885362677\\
64.2916666666667	0.21126	-91.6292635112114\\
64.2916666666667	0.21292	-94.4185384861551\\
64.2916666666667	0.21458	-97.2078134610988\\
64.2916666666667	0.21624	-99.9970884360426\\
64.2916666666667	0.2179	-102.786363410986\\
64.2916666666667	0.21956	-105.57563838593\\
64.2916666666667	0.22122	-108.364913360874\\
64.2916666666667	0.22288	-111.154188335817\\
64.2916666666667	0.22454	-113.943463310761\\
64.2916666666667	0.2262	-116.732738285705\\
64.2916666666667	0.22786	-119.522013260649\\
64.2916666666667	0.22952	-122.311288235592\\
64.2916666666667	0.23118	-125.100563210536\\
64.2916666666667	0.23284	-127.88983818548\\
64.2916666666667	0.2345	-130.679113160423\\
64.2916666666667	0.23616	-133.468388135367\\
64.2916666666667	0.23782	-136.257663110311\\
64.2916666666667	0.23948	-139.046938085255\\
64.2916666666667	0.24114	-141.836213060198\\
64.2916666666667	0.2428	-144.625488035142\\
64.2916666666667	0.24446	-147.414763010086\\
64.2916666666667	0.24612	-150.204037985029\\
64.2916666666667	0.24778	-152.993312959973\\
64.2916666666667	0.24944	-155.782587934917\\
64.2916666666667	0.2511	-158.571862909861\\
64.2916666666667	0.25276	-161.361137884804\\
64.2916666666667	0.25442	-164.150412859748\\
64.2916666666667	0.25608	-166.939687834692\\
64.2916666666667	0.25774	-169.728962809635\\
64.2916666666667	0.2594	-172.518237784579\\
64.2916666666667	0.26106	-175.307512759523\\
64.2916666666667	0.26272	-178.096787734467\\
64.2916666666667	0.26438	-180.88606270941\\
64.2916666666667	0.26604	-183.675337684354\\
64.2916666666667	0.2677	-186.464612659298\\
64.2916666666667	0.26936	-189.253887634241\\
64.2916666666667	0.27102	-192.043162609185\\
64.2916666666667	0.27268	-194.832437584129\\
64.2916666666667	0.27434	-197.621712559073\\
64.2916666666667	0.276	-200.410987534016\\
64.5	0.193	-60.6401670653771\\
64.5	0.19466	-63.4337659952665\\
64.5	0.19632	-66.2273649251558\\
64.5	0.19798	-69.0209638550452\\
64.5	0.19964	-71.8145627849345\\
64.5	0.2013	-74.6081617148239\\
64.5	0.20296	-77.4017606447132\\
64.5	0.20462	-80.1953595746025\\
64.5	0.20628	-82.9889585044919\\
64.5	0.20794	-85.7825574343813\\
64.5	0.2096	-88.5761563642706\\
64.5	0.21126	-91.3697552941599\\
64.5	0.21292	-94.1633542240493\\
64.5	0.21458	-96.9569531539386\\
64.5	0.21624	-99.750552083828\\
64.5	0.2179	-102.544151013717\\
64.5	0.21956	-105.337749943607\\
64.5	0.22122	-108.131348873496\\
64.5	0.22288	-110.924947803385\\
64.5	0.22454	-113.718546733275\\
64.5	0.2262	-116.512145663164\\
64.5	0.22786	-119.305744593053\\
64.5	0.22952	-122.099343522943\\
64.5	0.23118	-124.892942452832\\
64.5	0.23284	-127.686541382721\\
64.5	0.2345	-130.480140312611\\
64.5	0.23616	-133.2737392425\\
64.5	0.23782	-136.067338172389\\
64.5	0.23948	-138.860937102279\\
64.5	0.24114	-141.654536032168\\
64.5	0.2428	-144.448134962057\\
64.5	0.24446	-147.241733891947\\
64.5	0.24612	-150.035332821836\\
64.5	0.24778	-152.828931751726\\
64.5	0.24944	-155.622530681615\\
64.5	0.2511	-158.416129611504\\
64.5	0.25276	-161.209728541394\\
64.5	0.25442	-164.003327471283\\
64.5	0.25608	-166.796926401172\\
64.5	0.25774	-169.590525331062\\
64.5	0.2594	-172.384124260951\\
64.5	0.26106	-175.17772319084\\
64.5	0.26272	-177.97132212073\\
64.5	0.26438	-180.764921050619\\
64.5	0.26604	-183.558519980508\\
64.5	0.2677	-186.352118910398\\
64.5	0.26936	-189.145717840287\\
64.5	0.27102	-191.939316770176\\
64.5	0.27268	-194.732915700066\\
64.5	0.27434	-197.526514629955\\
64.5	0.276	-200.320113559844\\
64.7083333333333	0.193	-60.3330953439237\\
64.7083333333333	0.19466	-63.1310182287587\\
64.7083333333333	0.19632	-65.9289411135937\\
64.7083333333333	0.19798	-68.7268639984287\\
64.7083333333333	0.19964	-71.5247868832637\\
64.7083333333333	0.2013	-74.3227097680986\\
64.7083333333333	0.20296	-77.1206326529336\\
64.7083333333333	0.20462	-79.9185555377686\\
64.7083333333333	0.20628	-82.7164784226036\\
64.7083333333333	0.20794	-85.5144013074385\\
64.7083333333333	0.2096	-88.3123241922735\\
64.7083333333333	0.21126	-91.1102470771085\\
64.7083333333333	0.21292	-93.9081699619434\\
64.7083333333333	0.21458	-96.7060928467784\\
64.7083333333333	0.21624	-99.5040157316134\\
64.7083333333333	0.2179	-102.301938616448\\
64.7083333333333	0.21956	-105.099861501283\\
64.7083333333333	0.22122	-107.897784386118\\
64.7083333333333	0.22288	-110.695707270953\\
64.7083333333333	0.22454	-113.493630155788\\
64.7083333333333	0.2262	-116.291553040623\\
64.7083333333333	0.22786	-119.089475925458\\
64.7083333333333	0.22952	-121.887398810293\\
64.7083333333333	0.23118	-124.685321695128\\
64.7083333333333	0.23284	-127.483244579963\\
64.7083333333333	0.2345	-130.281167464798\\
64.7083333333333	0.23616	-133.079090349633\\
64.7083333333333	0.23782	-135.877013234468\\
64.7083333333333	0.23948	-138.674936119303\\
64.7083333333333	0.24114	-141.472859004138\\
64.7083333333333	0.2428	-144.270781888973\\
64.7083333333333	0.24446	-147.068704773808\\
64.7083333333333	0.24612	-149.866627658643\\
64.7083333333333	0.24778	-152.664550543478\\
64.7083333333333	0.24944	-155.462473428313\\
64.7083333333333	0.2511	-158.260396313148\\
64.7083333333333	0.25276	-161.058319197983\\
64.7083333333333	0.25442	-163.856242082818\\
64.7083333333333	0.25608	-166.654164967653\\
64.7083333333333	0.25774	-169.452087852488\\
64.7083333333333	0.2594	-172.250010737323\\
64.7083333333333	0.26106	-175.047933622158\\
64.7083333333333	0.26272	-177.845856506993\\
64.7083333333333	0.26438	-180.643779391828\\
64.7083333333333	0.26604	-183.441702276663\\
64.7083333333333	0.2677	-186.239625161498\\
64.7083333333333	0.26936	-189.037548046333\\
64.7083333333333	0.27102	-191.835470931168\\
64.7083333333333	0.27268	-194.633393816003\\
64.7083333333333	0.27434	-197.431316700837\\
64.7083333333333	0.276	-200.229239585673\\
64.9166666666667	0.193	-60.0260236224703\\
64.9166666666667	0.19466	-62.8282704622509\\
64.9166666666667	0.19632	-65.6305173020315\\
64.9166666666667	0.19798	-68.4327641418122\\
64.9166666666667	0.19964	-71.2350109815928\\
64.9166666666667	0.2013	-74.0372578213733\\
64.9166666666667	0.20296	-76.839504661154\\
64.9166666666667	0.20462	-79.6417515009346\\
64.9166666666667	0.20628	-82.4439983407152\\
64.9166666666667	0.20794	-85.2462451804958\\
64.9166666666667	0.2096	-88.0484920202764\\
64.9166666666667	0.21126	-90.850738860057\\
64.9166666666667	0.21292	-93.6529856998376\\
64.9166666666667	0.21458	-96.4552325396182\\
64.9166666666667	0.21624	-99.2574793793988\\
64.9166666666667	0.2179	-102.059726219179\\
64.9166666666667	0.21956	-104.86197305896\\
64.9166666666667	0.22122	-107.664219898741\\
64.9166666666667	0.22288	-110.466466738521\\
64.9166666666667	0.22454	-113.268713578302\\
64.9166666666667	0.2262	-116.070960418082\\
64.9166666666667	0.22786	-118.873207257863\\
64.9166666666667	0.22952	-121.675454097644\\
64.9166666666667	0.23118	-124.477700937424\\
64.9166666666667	0.23284	-127.279947777205\\
64.9166666666667	0.2345	-130.082194616985\\
64.9166666666667	0.23616	-132.884441456766\\
64.9166666666667	0.23782	-135.686688296547\\
64.9166666666667	0.23948	-138.488935136327\\
64.9166666666667	0.24114	-141.291181976108\\
64.9166666666667	0.2428	-144.093428815888\\
64.9166666666667	0.24446	-146.895675655669\\
64.9166666666667	0.24612	-149.69792249545\\
64.9166666666667	0.24778	-152.50016933523\\
64.9166666666667	0.24944	-155.302416175011\\
64.9166666666667	0.2511	-158.104663014791\\
64.9166666666667	0.25276	-160.906909854572\\
64.9166666666667	0.25442	-163.709156694353\\
64.9166666666667	0.25608	-166.511403534133\\
64.9166666666667	0.25774	-169.313650373914\\
64.9166666666667	0.2594	-172.115897213695\\
64.9166666666667	0.26106	-174.918144053475\\
64.9166666666667	0.26272	-177.720390893256\\
64.9166666666667	0.26438	-180.522637733036\\
64.9166666666667	0.26604	-183.324884572817\\
64.9166666666667	0.2677	-186.127131412598\\
64.9166666666667	0.26936	-188.929378252378\\
64.9166666666667	0.27102	-191.731625092159\\
64.9166666666667	0.27268	-194.533871931939\\
64.9166666666667	0.27434	-197.33611877172\\
64.9166666666667	0.276	-200.138365611501\\
65.125	0.193	-59.7189519010169\\
65.125	0.19466	-62.5255226957432\\
65.125	0.19632	-65.3320934904694\\
65.125	0.19798	-68.1386642851957\\
65.125	0.19964	-70.9452350799219\\
65.125	0.2013	-73.7518058746481\\
65.125	0.20296	-76.5583766693744\\
65.125	0.20462	-79.3649474641006\\
65.125	0.20628	-82.1715182588268\\
65.125	0.20794	-84.9780890535531\\
65.125	0.2096	-87.7846598482793\\
65.125	0.21126	-90.5912306430055\\
65.125	0.21292	-93.3978014377317\\
65.125	0.21458	-96.204372232458\\
65.125	0.21624	-99.0109430271842\\
65.125	0.2179	-101.81751382191\\
65.125	0.21956	-104.624084616637\\
65.125	0.22122	-107.430655411363\\
65.125	0.22288	-110.237226206089\\
65.125	0.22454	-113.043797000815\\
65.125	0.2262	-115.850367795542\\
65.125	0.22786	-118.656938590268\\
65.125	0.22952	-121.463509384994\\
65.125	0.23118	-124.27008017972\\
65.125	0.23284	-127.076650974447\\
65.125	0.2345	-129.883221769173\\
65.125	0.23616	-132.689792563899\\
65.125	0.23782	-135.496363358625\\
65.125	0.23948	-138.302934153352\\
65.125	0.24114	-141.109504948078\\
65.125	0.2428	-143.916075742804\\
65.125	0.24446	-146.72264653753\\
65.125	0.24612	-149.529217332256\\
65.125	0.24778	-152.335788126983\\
65.125	0.24944	-155.142358921709\\
65.125	0.2511	-157.948929716435\\
65.125	0.25276	-160.755500511161\\
65.125	0.25442	-163.562071305888\\
65.125	0.25608	-166.368642100614\\
65.125	0.25774	-169.17521289534\\
65.125	0.2594	-171.981783690066\\
65.125	0.26106	-174.788354484793\\
65.125	0.26272	-177.594925279519\\
65.125	0.26438	-180.401496074245\\
65.125	0.26604	-183.208066868971\\
65.125	0.2677	-186.014637663697\\
65.125	0.26936	-188.821208458424\\
65.125	0.27102	-191.62777925315\\
65.125	0.27268	-194.434350047876\\
65.125	0.27434	-197.240920842602\\
65.125	0.276	-200.047491637329\\
65.3333333333333	0.193	-59.4118801795636\\
65.3333333333333	0.19466	-62.2227749292354\\
65.3333333333333	0.19632	-65.0336696789073\\
65.3333333333333	0.19798	-67.8445644285792\\
65.3333333333333	0.19964	-70.655459178251\\
65.3333333333333	0.2013	-73.4663539279229\\
65.3333333333333	0.20296	-76.2772486775947\\
65.3333333333333	0.20462	-79.0881434272666\\
65.3333333333333	0.20628	-81.8990381769385\\
65.3333333333333	0.20794	-84.7099329266103\\
65.3333333333333	0.2096	-87.5208276762822\\
65.3333333333333	0.21126	-90.331722425954\\
65.3333333333333	0.21292	-93.1426171756259\\
65.3333333333333	0.21458	-95.9535119252978\\
65.3333333333333	0.21624	-98.7644066749696\\
65.3333333333333	0.2179	-101.575301424642\\
65.3333333333333	0.21956	-104.386196174313\\
65.3333333333333	0.22122	-107.197090923985\\
65.3333333333333	0.22288	-110.007985673657\\
65.3333333333333	0.22454	-112.818880423329\\
65.3333333333333	0.2262	-115.629775173001\\
65.3333333333333	0.22786	-118.440669922673\\
65.3333333333333	0.22952	-121.251564672345\\
65.3333333333333	0.23118	-124.062459422016\\
65.3333333333333	0.23284	-126.873354171688\\
65.3333333333333	0.2345	-129.68424892136\\
65.3333333333333	0.23616	-132.495143671032\\
65.3333333333333	0.23782	-135.306038420704\\
65.3333333333333	0.23948	-138.116933170376\\
65.3333333333333	0.24114	-140.927827920048\\
65.3333333333333	0.2428	-143.738722669719\\
65.3333333333333	0.24446	-146.549617419391\\
65.3333333333333	0.24612	-149.360512169063\\
65.3333333333333	0.24778	-152.171406918735\\
65.3333333333333	0.24944	-154.982301668407\\
65.3333333333333	0.2511	-157.793196418079\\
65.3333333333333	0.25276	-160.604091167751\\
65.3333333333333	0.25442	-163.414985917423\\
65.3333333333333	0.25608	-166.225880667094\\
65.3333333333333	0.25774	-169.036775416766\\
65.3333333333333	0.2594	-171.847670166438\\
65.3333333333333	0.26106	-174.65856491611\\
65.3333333333333	0.26272	-177.469459665782\\
65.3333333333333	0.26438	-180.280354415454\\
65.3333333333333	0.26604	-183.091249165126\\
65.3333333333333	0.2677	-185.902143914797\\
65.3333333333333	0.26936	-188.713038664469\\
65.3333333333333	0.27102	-191.523933414141\\
65.3333333333333	0.27268	-194.334828163813\\
65.3333333333333	0.27434	-197.145722913485\\
65.3333333333333	0.276	-199.956617663157\\
65.5416666666667	0.193	-59.1048084581102\\
65.5416666666667	0.19466	-61.9200271627277\\
65.5416666666667	0.19632	-64.7352458673452\\
65.5416666666667	0.19798	-67.5504645719627\\
65.5416666666667	0.19964	-70.3656832765802\\
65.5416666666667	0.2013	-73.1809019811976\\
65.5416666666667	0.20296	-75.9961206858152\\
65.5416666666667	0.20462	-78.8113393904326\\
65.5416666666667	0.20628	-81.6265580950502\\
65.5416666666667	0.20794	-84.4417767996677\\
65.5416666666667	0.2096	-87.2569955042851\\
65.5416666666667	0.21126	-90.0722142089026\\
65.5416666666667	0.21292	-92.8874329135201\\
65.5416666666667	0.21458	-95.7026516181376\\
65.5416666666667	0.21624	-98.5178703227551\\
65.5416666666667	0.2179	-101.333089027373\\
65.5416666666667	0.21956	-104.14830773199\\
65.5416666666667	0.22122	-106.963526436608\\
65.5416666666667	0.22288	-109.778745141225\\
65.5416666666667	0.22454	-112.593963845843\\
65.5416666666667	0.2262	-115.40918255046\\
65.5416666666667	0.22786	-118.224401255078\\
65.5416666666667	0.22952	-121.039619959695\\
65.5416666666667	0.23118	-123.854838664313\\
65.5416666666667	0.23284	-126.67005736893\\
65.5416666666667	0.2345	-129.485276073548\\
65.5416666666667	0.23616	-132.300494778165\\
65.5416666666667	0.23782	-135.115713482783\\
65.5416666666667	0.23948	-137.9309321874\\
65.5416666666667	0.24114	-140.746150892018\\
65.5416666666667	0.2428	-143.561369596635\\
65.5416666666667	0.24446	-146.376588301252\\
65.5416666666667	0.24612	-149.19180700587\\
65.5416666666667	0.24778	-152.007025710487\\
65.5416666666667	0.24944	-154.822244415105\\
65.5416666666667	0.2511	-157.637463119722\\
65.5416666666667	0.25276	-160.45268182434\\
65.5416666666667	0.25442	-163.267900528957\\
65.5416666666667	0.25608	-166.083119233575\\
65.5416666666667	0.25774	-168.898337938192\\
65.5416666666667	0.2594	-171.71355664281\\
65.5416666666667	0.26106	-174.528775347427\\
65.5416666666667	0.26272	-177.343994052045\\
65.5416666666667	0.26438	-180.159212756662\\
65.5416666666667	0.26604	-182.97443146128\\
65.5416666666667	0.2677	-185.789650165897\\
65.5416666666667	0.26936	-188.604868870515\\
65.5416666666667	0.27102	-191.420087575132\\
65.5416666666667	0.27268	-194.23530627975\\
65.5416666666667	0.27434	-197.050524984367\\
65.5416666666667	0.276	-199.865743688985\\
65.75	0.193	-58.7977367366568\\
65.75	0.19466	-61.6172793962199\\
65.75	0.19632	-64.4368220557831\\
65.75	0.19798	-67.2563647153462\\
65.75	0.19964	-70.0759073749093\\
65.75	0.2013	-72.8954500344724\\
65.75	0.20296	-75.7149926940356\\
65.75	0.20462	-78.5345353535987\\
65.75	0.20628	-81.3540780131618\\
65.75	0.20794	-84.173620672725\\
65.75	0.2096	-86.993163332288\\
65.75	0.21126	-89.8127059918511\\
65.75	0.21292	-92.6322486514143\\
65.75	0.21458	-95.4517913109774\\
65.75	0.21624	-98.2713339705406\\
65.75	0.2179	-101.090876630104\\
65.75	0.21956	-103.910419289667\\
65.75	0.22122	-106.72996194923\\
65.75	0.22288	-109.549504608793\\
65.75	0.22454	-112.369047268356\\
65.75	0.2262	-115.188589927919\\
65.75	0.22786	-118.008132587482\\
65.75	0.22952	-120.827675247045\\
65.75	0.23118	-123.647217906609\\
65.75	0.23284	-126.466760566172\\
65.75	0.2345	-129.286303225735\\
65.75	0.23616	-132.105845885298\\
65.75	0.23782	-134.925388544861\\
65.75	0.23948	-137.744931204424\\
65.75	0.24114	-140.564473863987\\
65.75	0.2428	-143.384016523551\\
65.75	0.24446	-146.203559183114\\
65.75	0.24612	-149.023101842677\\
65.75	0.24778	-151.84264450224\\
65.75	0.24944	-154.662187161803\\
65.75	0.2511	-157.481729821366\\
65.75	0.25276	-160.301272480929\\
65.75	0.25442	-163.120815140492\\
65.75	0.25608	-165.940357800055\\
65.75	0.25774	-168.759900459619\\
65.75	0.2594	-171.579443119182\\
65.75	0.26106	-174.398985778745\\
65.75	0.26272	-177.218528438308\\
65.75	0.26438	-180.038071097871\\
65.75	0.26604	-182.857613757434\\
65.75	0.2677	-185.677156416997\\
65.75	0.26936	-188.496699076561\\
65.75	0.27102	-191.316241736124\\
65.75	0.27268	-194.135784395687\\
65.75	0.27434	-196.95532705525\\
65.75	0.276	-199.774869714813\\
65.9583333333333	0.193	-58.4906650152034\\
65.9583333333333	0.19466	-61.3145316297122\\
65.9583333333333	0.19632	-64.1383982442209\\
65.9583333333333	0.19798	-66.9622648587297\\
65.9583333333333	0.19964	-69.7861314732384\\
65.9583333333333	0.2013	-72.6099980877472\\
65.9583333333333	0.20296	-75.433864702256\\
65.9583333333333	0.20462	-78.2577313167647\\
65.9583333333333	0.20628	-81.0815979312735\\
65.9583333333333	0.20794	-83.9054645457822\\
65.9583333333333	0.2096	-86.7293311602909\\
65.9583333333333	0.21126	-89.5531977747997\\
65.9583333333333	0.21292	-92.3770643893085\\
65.9583333333333	0.21458	-95.2009310038172\\
65.9583333333333	0.21624	-98.024797618326\\
65.9583333333333	0.2179	-100.848664232835\\
65.9583333333333	0.21956	-103.672530847343\\
65.9583333333333	0.22122	-106.496397461852\\
65.9583333333333	0.22288	-109.320264076361\\
65.9583333333333	0.22454	-112.14413069087\\
65.9583333333333	0.2262	-114.967997305378\\
65.9583333333333	0.22786	-117.791863919887\\
65.9583333333333	0.22952	-120.615730534396\\
65.9583333333333	0.23118	-123.439597148905\\
65.9583333333333	0.23284	-126.263463763414\\
65.9583333333333	0.2345	-129.087330377922\\
65.9583333333333	0.23616	-131.911196992431\\
65.9583333333333	0.23782	-134.73506360694\\
65.9583333333333	0.23948	-137.558930221448\\
65.9583333333333	0.24114	-140.382796835957\\
65.9583333333333	0.2428	-143.206663450466\\
65.9583333333333	0.24446	-146.030530064975\\
65.9583333333333	0.24612	-148.854396679483\\
65.9583333333333	0.24778	-151.678263293992\\
65.9583333333333	0.24944	-154.502129908501\\
65.9583333333333	0.2511	-157.32599652301\\
65.9583333333333	0.25276	-160.149863137518\\
65.9583333333333	0.25442	-162.973729752027\\
65.9583333333333	0.25608	-165.797596366536\\
65.9583333333333	0.25774	-168.621462981045\\
65.9583333333333	0.2594	-171.445329595554\\
65.9583333333333	0.26106	-174.269196210062\\
65.9583333333333	0.26272	-177.093062824571\\
65.9583333333333	0.26438	-179.91692943908\\
65.9583333333333	0.26604	-182.740796053589\\
65.9583333333333	0.2677	-185.564662668097\\
65.9583333333333	0.26936	-188.388529282606\\
65.9583333333333	0.27102	-191.212395897115\\
65.9583333333333	0.27268	-194.036262511624\\
65.9583333333333	0.27434	-196.860129126132\\
65.9583333333333	0.276	-199.683995740641\\
66.1666666666667	0.193	-58.18359329375\\
66.1666666666667	0.19466	-61.0117838632044\\
66.1666666666667	0.19632	-63.8399744326587\\
66.1666666666667	0.19798	-66.6681650021132\\
66.1666666666667	0.19964	-69.4963555715675\\
66.1666666666667	0.2013	-72.3245461410219\\
66.1666666666667	0.20296	-75.1527367104763\\
66.1666666666667	0.20462	-77.9809272799307\\
66.1666666666667	0.20628	-80.8091178493851\\
66.1666666666667	0.20794	-83.6373084188395\\
66.1666666666667	0.2096	-86.4654989882938\\
66.1666666666667	0.21126	-89.2936895577482\\
66.1666666666667	0.21292	-92.1218801272026\\
66.1666666666667	0.21458	-94.950070696657\\
66.1666666666667	0.21624	-97.7782612661114\\
66.1666666666667	0.2179	-100.606451835566\\
66.1666666666667	0.21956	-103.43464240502\\
66.1666666666667	0.22122	-106.262832974474\\
66.1666666666667	0.22288	-109.091023543929\\
66.1666666666667	0.22454	-111.919214113383\\
66.1666666666667	0.2262	-114.747404682838\\
66.1666666666667	0.22786	-117.575595252292\\
66.1666666666667	0.22952	-120.403785821746\\
66.1666666666667	0.23118	-123.231976391201\\
66.1666666666667	0.23284	-126.060166960655\\
66.1666666666667	0.2345	-128.88835753011\\
66.1666666666667	0.23616	-131.716548099564\\
66.1666666666667	0.23782	-134.544738669018\\
66.1666666666667	0.23948	-137.372929238473\\
66.1666666666667	0.24114	-140.201119807927\\
66.1666666666667	0.2428	-143.029310377381\\
66.1666666666667	0.24446	-145.857500946836\\
66.1666666666667	0.24612	-148.68569151629\\
66.1666666666667	0.24778	-151.513882085745\\
66.1666666666667	0.24944	-154.342072655199\\
66.1666666666667	0.2511	-157.170263224653\\
66.1666666666667	0.25276	-159.998453794108\\
66.1666666666667	0.25442	-162.826644363562\\
66.1666666666667	0.25608	-165.654834933017\\
66.1666666666667	0.25774	-168.483025502471\\
66.1666666666667	0.2594	-171.311216071925\\
66.1666666666667	0.26106	-174.13940664138\\
66.1666666666667	0.26272	-176.967597210834\\
66.1666666666667	0.26438	-179.795787780288\\
66.1666666666667	0.26604	-182.623978349743\\
66.1666666666667	0.2677	-185.452168919197\\
66.1666666666667	0.26936	-188.280359488652\\
66.1666666666667	0.27102	-191.108550058106\\
66.1666666666667	0.27268	-193.93674062756\\
66.1666666666667	0.27434	-196.764931197015\\
66.1666666666667	0.276	-199.593121766469\\
66.375	0.193	-57.8765215722967\\
66.375	0.19466	-60.7090360966967\\
66.375	0.19632	-63.5415506210967\\
66.375	0.19798	-66.3740651454967\\
66.375	0.19964	-69.2065796698967\\
66.375	0.2013	-72.0390941942967\\
66.375	0.20296	-74.8716087186968\\
66.375	0.20462	-77.7041232430967\\
66.375	0.20628	-80.5366377674968\\
66.375	0.20794	-83.3691522918968\\
66.375	0.2096	-86.2016668162968\\
66.375	0.21126	-89.0341813406968\\
66.375	0.21292	-91.8666958650968\\
66.375	0.21458	-94.6992103894968\\
66.375	0.21624	-97.5317249138969\\
66.375	0.2179	-100.364239438297\\
66.375	0.21956	-103.196753962697\\
66.375	0.22122	-106.029268487097\\
66.375	0.22288	-108.861783011497\\
66.375	0.22454	-111.694297535897\\
66.375	0.2262	-114.526812060297\\
66.375	0.22786	-117.359326584697\\
66.375	0.22952	-120.191841109097\\
66.375	0.23118	-123.024355633497\\
66.375	0.23284	-125.856870157897\\
66.375	0.2345	-128.689384682297\\
66.375	0.23616	-131.521899206697\\
66.375	0.23782	-134.354413731097\\
66.375	0.23948	-137.186928255497\\
66.375	0.24114	-140.019442779897\\
66.375	0.2428	-142.851957304297\\
66.375	0.24446	-145.684471828697\\
66.375	0.24612	-148.516986353097\\
66.375	0.24778	-151.349500877497\\
66.375	0.24944	-154.182015401897\\
66.375	0.2511	-157.014529926297\\
66.375	0.25276	-159.847044450697\\
66.375	0.25442	-162.679558975097\\
66.375	0.25608	-165.512073499497\\
66.375	0.25774	-168.344588023897\\
66.375	0.2594	-171.177102548297\\
66.375	0.26106	-174.009617072697\\
66.375	0.26272	-176.842131597097\\
66.375	0.26438	-179.674646121497\\
66.375	0.26604	-182.507160645897\\
66.375	0.2677	-185.339675170297\\
66.375	0.26936	-188.172189694697\\
66.375	0.27102	-191.004704219097\\
66.375	0.27268	-193.837218743497\\
66.375	0.27434	-196.669733267897\\
66.375	0.276	-199.502247792297\\
66.5833333333333	0.193	-57.5694498508433\\
66.5833333333333	0.19466	-60.4062883301889\\
66.5833333333333	0.19632	-63.2431268095345\\
66.5833333333333	0.19798	-66.0799652888802\\
66.5833333333333	0.19964	-68.9168037682259\\
66.5833333333333	0.2013	-71.7536422475715\\
66.5833333333333	0.20296	-74.5904807269171\\
66.5833333333333	0.20462	-77.4273192062628\\
66.5833333333333	0.20628	-80.2641576856084\\
66.5833333333333	0.20794	-83.1009961649541\\
66.5833333333333	0.2096	-85.9378346442997\\
66.5833333333333	0.21126	-88.7746731236453\\
66.5833333333333	0.21292	-91.611511602991\\
66.5833333333333	0.21458	-94.4483500823366\\
66.5833333333333	0.21624	-97.2851885616823\\
66.5833333333333	0.2179	-100.122027041028\\
66.5833333333333	0.21956	-102.958865520374\\
66.5833333333333	0.22122	-105.795703999719\\
66.5833333333333	0.22288	-108.632542479065\\
66.5833333333333	0.22454	-111.46938095841\\
66.5833333333333	0.2262	-114.306219437756\\
66.5833333333333	0.22786	-117.143057917102\\
66.5833333333333	0.22952	-119.979896396447\\
66.5833333333333	0.23118	-122.816734875793\\
66.5833333333333	0.23284	-125.653573355139\\
66.5833333333333	0.2345	-128.490411834484\\
66.5833333333333	0.23616	-131.32725031383\\
66.5833333333333	0.23782	-134.164088793176\\
66.5833333333333	0.23948	-137.000927272521\\
66.5833333333333	0.24114	-139.837765751867\\
66.5833333333333	0.2428	-142.674604231212\\
66.5833333333333	0.24446	-145.511442710558\\
66.5833333333333	0.24612	-148.348281189904\\
66.5833333333333	0.24778	-151.185119669249\\
66.5833333333333	0.24944	-154.021958148595\\
66.5833333333333	0.2511	-156.858796627941\\
66.5833333333333	0.25276	-159.695635107286\\
66.5833333333333	0.25442	-162.532473586632\\
66.5833333333333	0.25608	-165.369312065978\\
66.5833333333333	0.25774	-168.206150545323\\
66.5833333333333	0.2594	-171.042989024669\\
66.5833333333333	0.26106	-173.879827504015\\
66.5833333333333	0.26272	-176.71666598336\\
66.5833333333333	0.26438	-179.553504462706\\
66.5833333333333	0.26604	-182.390342942052\\
66.5833333333333	0.2677	-185.227181421397\\
66.5833333333333	0.26936	-188.064019900743\\
66.5833333333333	0.27102	-190.900858380088\\
66.5833333333333	0.27268	-193.737696859434\\
66.5833333333333	0.27434	-196.57453533878\\
66.5833333333333	0.276	-199.411373818125\\
66.7916666666667	0.193	-57.2623781293899\\
66.7916666666667	0.19466	-60.1035405636812\\
66.7916666666667	0.19632	-62.9447029979724\\
66.7916666666667	0.19798	-65.7858654322637\\
66.7916666666667	0.19964	-68.627027866555\\
66.7916666666667	0.2013	-71.4681903008462\\
66.7916666666667	0.20296	-74.3093527351375\\
66.7916666666667	0.20462	-77.1505151694288\\
66.7916666666667	0.20628	-79.9916776037201\\
66.7916666666667	0.20794	-82.8328400380113\\
66.7916666666667	0.2096	-85.6740024723026\\
66.7916666666667	0.21126	-88.5151649065938\\
66.7916666666667	0.21292	-91.3563273408851\\
66.7916666666667	0.21458	-94.1974897751764\\
66.7916666666667	0.21624	-97.0386522094677\\
66.7916666666667	0.2179	-99.8798146437589\\
66.7916666666667	0.21956	-102.72097707805\\
66.7916666666667	0.22122	-105.562139512341\\
66.7916666666667	0.22288	-108.403301946633\\
66.7916666666667	0.22454	-111.244464380924\\
66.7916666666667	0.2262	-114.085626815215\\
66.7916666666667	0.22786	-116.926789249507\\
66.7916666666667	0.22952	-119.767951683798\\
66.7916666666667	0.23118	-122.609114118089\\
66.7916666666667	0.23284	-125.45027655238\\
66.7916666666667	0.2345	-128.291438986672\\
66.7916666666667	0.23616	-131.132601420963\\
66.7916666666667	0.23782	-133.973763855254\\
66.7916666666667	0.23948	-136.814926289545\\
66.7916666666667	0.24114	-139.656088723837\\
66.7916666666667	0.2428	-142.497251158128\\
66.7916666666667	0.24446	-145.338413592419\\
66.7916666666667	0.24612	-148.179576026711\\
66.7916666666667	0.24778	-151.020738461002\\
66.7916666666667	0.24944	-153.861900895293\\
66.7916666666667	0.2511	-156.703063329584\\
66.7916666666667	0.25276	-159.544225763876\\
66.7916666666667	0.25442	-162.385388198167\\
66.7916666666667	0.25608	-165.226550632458\\
66.7916666666667	0.25774	-168.067713066749\\
66.7916666666667	0.2594	-170.908875501041\\
66.7916666666667	0.26106	-173.750037935332\\
66.7916666666667	0.26272	-176.591200369623\\
66.7916666666667	0.26438	-179.432362803914\\
66.7916666666667	0.26604	-182.273525238206\\
66.7916666666667	0.2677	-185.114687672497\\
66.7916666666667	0.26936	-187.955850106788\\
66.7916666666667	0.27102	-190.79701254108\\
66.7916666666667	0.27268	-193.638174975371\\
66.7916666666667	0.27434	-196.479337409662\\
66.7916666666667	0.276	-199.320499843953\\
67	0.193	-56.9553064079365\\
67	0.19466	-59.8007927971734\\
67	0.19632	-62.6462791864103\\
67	0.19798	-65.4917655756472\\
67	0.19964	-68.3372519648841\\
67	0.2013	-71.182738354121\\
67	0.20296	-74.0282247433579\\
67	0.20462	-76.8737111325948\\
67	0.20628	-79.7191975218317\\
67	0.20794	-82.5646839110686\\
67	0.2096	-85.4101703003055\\
67	0.21126	-88.2556566895424\\
67	0.21292	-91.1011430787793\\
67	0.21458	-93.9466294680162\\
67	0.21624	-96.7921158572531\\
67	0.2179	-99.63760224649\\
67	0.21956	-102.483088635727\\
67	0.22122	-105.328575024964\\
67	0.22288	-108.174061414201\\
67	0.22454	-111.019547803438\\
67	0.2262	-113.865034192674\\
67	0.22786	-116.710520581911\\
67	0.22952	-119.556006971148\\
67	0.23118	-122.401493360385\\
67	0.23284	-125.246979749622\\
67	0.2345	-128.092466138859\\
67	0.23616	-130.937952528096\\
67	0.23782	-133.783438917333\\
67	0.23948	-136.62892530657\\
67	0.24114	-139.474411695807\\
67	0.2428	-142.319898085043\\
67	0.24446	-145.16538447428\\
67	0.24612	-148.010870863517\\
67	0.24778	-150.856357252754\\
67	0.24944	-153.701843641991\\
67	0.2511	-156.547330031228\\
67	0.25276	-159.392816420465\\
67	0.25442	-162.238302809702\\
67	0.25608	-165.083789198939\\
67	0.25774	-167.929275588176\\
67	0.2594	-170.774761977412\\
67	0.26106	-173.620248366649\\
67	0.26272	-176.465734755886\\
67	0.26438	-179.311221145123\\
67	0.26604	-182.15670753436\\
67	0.2677	-185.002193923597\\
67	0.26936	-187.847680312834\\
67	0.27102	-190.693166702071\\
67	0.27268	-193.538653091308\\
67	0.27434	-196.384139480545\\
67	0.276	-199.229625869782\\
};
\end{axis}

\begin{axis}[%
width=4.927496cm,
height=3.050847cm,
at={(6.483547cm,12.711864cm)},
scale only axis,
xmin=57,
xmax=67,
tick align=outside,
xlabel={$L_{cut}$},
xmajorgrids,
ymin=0.193,
ymax=0.276,
ylabel={$D_{rlx}$},
ymajorgrids,
zmin=-500,
zmax=0,
zlabel={$x_2$},
zmajorgrids,
view={-140}{50},
legend style={at={(1.03,1)},anchor=north west,legend cell align=left,align=left,draw=white!15!black}
]
\addplot3[only marks,mark=*,mark options={},mark size=1.5000pt,color=mycolor1] plot table[row sep=crcr,]{%
67	0.276	-237.955472525381\\
66	0.255	-196.972248838965\\
62	0.209	-97.9896643933193\\
57	0.193	-69.3873457002779\\
};
\addplot3[only marks,mark=*,mark options={},mark size=1.5000pt,color=black] plot table[row sep=crcr,]{%
64	0.23	-148.601446012025\\
};

\addplot3[%
surf,
opacity=0.7,
shader=interp,
colormap={mymap}{[1pt] rgb(0pt)=(0.0901961,0.239216,0.0745098); rgb(1pt)=(0.0945149,0.242058,0.0739522); rgb(2pt)=(0.0988592,0.244894,0.0733566); rgb(3pt)=(0.103229,0.247724,0.0727241); rgb(4pt)=(0.107623,0.250549,0.0720557); rgb(5pt)=(0.112043,0.253367,0.0713525); rgb(6pt)=(0.116487,0.25618,0.0706154); rgb(7pt)=(0.120956,0.258986,0.0698456); rgb(8pt)=(0.125449,0.261787,0.0690441); rgb(9pt)=(0.129967,0.264581,0.0682118); rgb(10pt)=(0.134508,0.26737,0.06735); rgb(11pt)=(0.139074,0.270152,0.0664596); rgb(12pt)=(0.143663,0.272929,0.0655416); rgb(13pt)=(0.148275,0.275699,0.0645971); rgb(14pt)=(0.152911,0.278463,0.0636271); rgb(15pt)=(0.15757,0.281221,0.0626328); rgb(16pt)=(0.162252,0.283973,0.0616151); rgb(17pt)=(0.166957,0.286719,0.060575); rgb(18pt)=(0.171685,0.289458,0.0595136); rgb(19pt)=(0.176434,0.292191,0.0584321); rgb(20pt)=(0.181207,0.294918,0.0573313); rgb(21pt)=(0.186001,0.297639,0.0562123); rgb(22pt)=(0.190817,0.300353,0.0550763); rgb(23pt)=(0.195655,0.303061,0.0539242); rgb(24pt)=(0.200514,0.305763,0.052757); rgb(25pt)=(0.205395,0.308459,0.0515759); rgb(26pt)=(0.210296,0.311149,0.0503624); rgb(27pt)=(0.215212,0.313846,0.0490067); rgb(28pt)=(0.220142,0.316548,0.0475043); rgb(29pt)=(0.22509,0.319254,0.0458704); rgb(30pt)=(0.230056,0.321962,0.0441205); rgb(31pt)=(0.235042,0.324671,0.04227); rgb(32pt)=(0.240048,0.327379,0.0403343); rgb(33pt)=(0.245078,0.330085,0.0383287); rgb(34pt)=(0.250131,0.332786,0.0362688); rgb(35pt)=(0.25521,0.335482,0.0341698); rgb(36pt)=(0.260317,0.33817,0.0320472); rgb(37pt)=(0.265451,0.340849,0.0299163); rgb(38pt)=(0.270616,0.343517,0.0277927); rgb(39pt)=(0.275813,0.346172,0.0256916); rgb(40pt)=(0.281043,0.348814,0.0236284); rgb(41pt)=(0.286307,0.35144,0.0216186); rgb(42pt)=(0.291607,0.354048,0.0196776); rgb(43pt)=(0.296945,0.356637,0.0178207); rgb(44pt)=(0.302322,0.359206,0.0160634); rgb(45pt)=(0.307739,0.361753,0.0144211); rgb(46pt)=(0.313198,0.364275,0.0129091); rgb(47pt)=(0.318701,0.366772,0.0115428); rgb(48pt)=(0.324249,0.369242,0.0103377); rgb(49pt)=(0.329843,0.371682,0.00930909); rgb(50pt)=(0.335485,0.374093,0.00847245); rgb(51pt)=(0.341176,0.376471,0.00784314); rgb(52pt)=(0.346925,0.378826,0.00732741); rgb(53pt)=(0.352735,0.381168,0.00682184); rgb(54pt)=(0.358605,0.383497,0.00632729); rgb(55pt)=(0.364532,0.385812,0.00584464); rgb(56pt)=(0.370516,0.388113,0.00537476); rgb(57pt)=(0.376552,0.390399,0.00491852); rgb(58pt)=(0.38264,0.39267,0.00447681); rgb(59pt)=(0.388777,0.394925,0.00405048); rgb(60pt)=(0.394962,0.397164,0.00364042); rgb(61pt)=(0.401191,0.399386,0.00324749); rgb(62pt)=(0.407464,0.401592,0.00287258); rgb(63pt)=(0.413777,0.40378,0.00251655); rgb(64pt)=(0.420129,0.40595,0.00218028); rgb(65pt)=(0.426518,0.408102,0.00186463); rgb(66pt)=(0.432942,0.410234,0.00157049); rgb(67pt)=(0.439399,0.412348,0.00129873); rgb(68pt)=(0.445885,0.414441,0.00105022); rgb(69pt)=(0.452401,0.416515,0.000825833); rgb(70pt)=(0.458942,0.418567,0.000626441); rgb(71pt)=(0.465508,0.420599,0.00045292); rgb(72pt)=(0.472096,0.422609,0.000306141); rgb(73pt)=(0.478704,0.424596,0.000186979); rgb(74pt)=(0.485331,0.426562,9.63073e-05); rgb(75pt)=(0.491973,0.428504,3.49981e-05); rgb(76pt)=(0.498628,0.430422,3.92506e-06); rgb(77pt)=(0.505323,0.432315,0); rgb(78pt)=(0.512206,0.434168,0); rgb(79pt)=(0.519282,0.435983,0); rgb(80pt)=(0.526529,0.437764,0); rgb(81pt)=(0.533922,0.439512,0); rgb(82pt)=(0.54144,0.441232,0); rgb(83pt)=(0.549059,0.442927,0); rgb(84pt)=(0.556756,0.444599,0); rgb(85pt)=(0.564508,0.446252,0); rgb(86pt)=(0.572292,0.447889,0); rgb(87pt)=(0.580084,0.449514,0); rgb(88pt)=(0.587863,0.451129,0); rgb(89pt)=(0.595604,0.452737,0); rgb(90pt)=(0.603284,0.454343,0); rgb(91pt)=(0.610882,0.455948,0); rgb(92pt)=(0.618373,0.457556,0); rgb(93pt)=(0.625734,0.459171,0); rgb(94pt)=(0.632943,0.460795,0); rgb(95pt)=(0.639976,0.462432,0); rgb(96pt)=(0.64681,0.464084,0); rgb(97pt)=(0.653423,0.465756,0); rgb(98pt)=(0.659791,0.46745,0); rgb(99pt)=(0.665891,0.469169,0); rgb(100pt)=(0.6717,0.470916,0); rgb(101pt)=(0.677195,0.472696,0); rgb(102pt)=(0.682353,0.47451,0); rgb(103pt)=(0.687242,0.476355,0); rgb(104pt)=(0.691952,0.478225,0); rgb(105pt)=(0.696497,0.480118,0); rgb(106pt)=(0.700887,0.482033,0); rgb(107pt)=(0.705134,0.483968,0); rgb(108pt)=(0.709251,0.485921,0); rgb(109pt)=(0.713249,0.487891,0); rgb(110pt)=(0.71714,0.489876,0); rgb(111pt)=(0.720936,0.491875,0); rgb(112pt)=(0.724649,0.493887,0); rgb(113pt)=(0.72829,0.495909,0); rgb(114pt)=(0.731872,0.49794,0); rgb(115pt)=(0.735406,0.499979,0); rgb(116pt)=(0.738904,0.502025,0); rgb(117pt)=(0.742378,0.504075,0); rgb(118pt)=(0.74584,0.506128,0); rgb(119pt)=(0.749302,0.508182,0); rgb(120pt)=(0.752775,0.510237,0); rgb(121pt)=(0.756272,0.51229,0); rgb(122pt)=(0.759804,0.514339,0); rgb(123pt)=(0.763384,0.516385,0); rgb(124pt)=(0.767022,0.518424,0); rgb(125pt)=(0.770731,0.520455,0); rgb(126pt)=(0.774523,0.522478,0); rgb(127pt)=(0.77841,0.524489,0); rgb(128pt)=(0.782391,0.526491,0); rgb(129pt)=(0.786402,0.528496,0); rgb(130pt)=(0.790431,0.530506,0); rgb(131pt)=(0.794478,0.532521,0); rgb(132pt)=(0.798541,0.534539,0); rgb(133pt)=(0.802619,0.53656,0); rgb(134pt)=(0.806712,0.538584,0); rgb(135pt)=(0.81082,0.540609,0); rgb(136pt)=(0.81494,0.542635,0); rgb(137pt)=(0.819074,0.54466,0); rgb(138pt)=(0.823219,0.546686,0); rgb(139pt)=(0.827374,0.548709,0); rgb(140pt)=(0.831541,0.55073,0); rgb(141pt)=(0.835716,0.552749,0); rgb(142pt)=(0.8399,0.554763,0); rgb(143pt)=(0.844092,0.556774,0); rgb(144pt)=(0.848292,0.558779,0); rgb(145pt)=(0.852497,0.560778,0); rgb(146pt)=(0.856708,0.562771,0); rgb(147pt)=(0.860924,0.564756,0); rgb(148pt)=(0.865143,0.566733,0); rgb(149pt)=(0.869366,0.568701,0); rgb(150pt)=(0.873592,0.57066,0); rgb(151pt)=(0.877819,0.572608,0); rgb(152pt)=(0.882047,0.574545,0); rgb(153pt)=(0.886275,0.576471,0); rgb(154pt)=(0.890659,0.578362,0); rgb(155pt)=(0.895333,0.580203,0); rgb(156pt)=(0.900258,0.581999,0); rgb(157pt)=(0.905397,0.583755,0); rgb(158pt)=(0.910711,0.585479,0); rgb(159pt)=(0.916164,0.587176,0); rgb(160pt)=(0.921717,0.588852,0); rgb(161pt)=(0.927333,0.590513,0); rgb(162pt)=(0.932974,0.592166,0); rgb(163pt)=(0.938602,0.593815,0); rgb(164pt)=(0.94418,0.595468,0); rgb(165pt)=(0.949669,0.59713,0); rgb(166pt)=(0.955033,0.598808,0); rgb(167pt)=(0.960233,0.600507,0); rgb(168pt)=(0.965232,0.602233,0); rgb(169pt)=(0.969992,0.603992,0); rgb(170pt)=(0.974475,0.605791,0); rgb(171pt)=(0.978643,0.607636,0); rgb(172pt)=(0.98246,0.609532,0); rgb(173pt)=(0.985886,0.611486,0); rgb(174pt)=(0.988885,0.613503,0); rgb(175pt)=(0.991419,0.61559,0); rgb(176pt)=(0.99345,0.617753,0); rgb(177pt)=(0.99494,0.619997,0); rgb(178pt)=(0.995851,0.622329,0); rgb(179pt)=(0.996226,0.624763,0); rgb(180pt)=(0.996512,0.627352,0); rgb(181pt)=(0.996788,0.630095,0); rgb(182pt)=(0.997053,0.632982,0); rgb(183pt)=(0.997308,0.636004,0); rgb(184pt)=(0.997552,0.639152,0); rgb(185pt)=(0.997785,0.642416,0); rgb(186pt)=(0.998006,0.645786,0); rgb(187pt)=(0.998217,0.649253,0); rgb(188pt)=(0.998416,0.652807,0); rgb(189pt)=(0.998605,0.656439,0); rgb(190pt)=(0.998781,0.660138,0); rgb(191pt)=(0.998946,0.663897,0); rgb(192pt)=(0.9991,0.667704,0); rgb(193pt)=(0.999242,0.67155,0); rgb(194pt)=(0.999372,0.675427,0); rgb(195pt)=(0.99949,0.679323,0); rgb(196pt)=(0.999596,0.68323,0); rgb(197pt)=(0.99969,0.687139,0); rgb(198pt)=(0.999771,0.691039,0); rgb(199pt)=(0.999841,0.694921,0); rgb(200pt)=(0.999898,0.698775,0); rgb(201pt)=(0.999942,0.702592,0); rgb(202pt)=(0.999974,0.706363,0); rgb(203pt)=(0.999994,0.710077,0); rgb(204pt)=(1,0.713725,0); rgb(205pt)=(1,0.717341,0); rgb(206pt)=(1,0.720963,0); rgb(207pt)=(1,0.724591,0); rgb(208pt)=(1,0.728226,0); rgb(209pt)=(1,0.731867,0); rgb(210pt)=(1,0.735514,0); rgb(211pt)=(1,0.739167,0); rgb(212pt)=(1,0.742827,0); rgb(213pt)=(1,0.746493,0); rgb(214pt)=(1,0.750165,0); rgb(215pt)=(1,0.753843,0); rgb(216pt)=(1,0.757527,0); rgb(217pt)=(1,0.761217,0); rgb(218pt)=(1,0.764913,0); rgb(219pt)=(1,0.768615,0); rgb(220pt)=(1,0.772324,0); rgb(221pt)=(1,0.776038,0); rgb(222pt)=(1,0.779758,0); rgb(223pt)=(1,0.783484,0); rgb(224pt)=(1,0.787215,0); rgb(225pt)=(1,0.790953,0); rgb(226pt)=(1,0.794696,0); rgb(227pt)=(1,0.798445,0); rgb(228pt)=(1,0.8022,0); rgb(229pt)=(1,0.805961,0); rgb(230pt)=(1,0.809727,0); rgb(231pt)=(1,0.8135,0); rgb(232pt)=(1,0.817278,0); rgb(233pt)=(1,0.821063,0); rgb(234pt)=(1,0.824854,0); rgb(235pt)=(1,0.828652,0); rgb(236pt)=(1,0.832455,0); rgb(237pt)=(1,0.836265,0); rgb(238pt)=(1,0.840081,0); rgb(239pt)=(1,0.843903,0); rgb(240pt)=(1,0.847732,0); rgb(241pt)=(1,0.851566,0); rgb(242pt)=(1,0.855406,0); rgb(243pt)=(1,0.859253,0); rgb(244pt)=(1,0.863106,0); rgb(245pt)=(1,0.866964,0); rgb(246pt)=(1,0.870829,0); rgb(247pt)=(1,0.8747,0); rgb(248pt)=(1,0.878577,0); rgb(249pt)=(1,0.88246,0); rgb(250pt)=(1,0.886349,0); rgb(251pt)=(1,0.890243,0); rgb(252pt)=(1,0.894144,0); rgb(253pt)=(1,0.898051,0); rgb(254pt)=(1,0.901964,0); rgb(255pt)=(1,0.905882,0)},
mesh/rows=49]
table[row sep=crcr,header=false] {%
%
57	0.193	-69.387345700108\\
57	0.19466	-76.5606084058529\\
57	0.19632	-83.7338711115976\\
57	0.19798	-90.9071338173426\\
57	0.19964	-98.0803965230875\\
57	0.2013	-105.253659228833\\
57	0.20296	-112.426921934577\\
57	0.20462	-119.600184640322\\
57	0.20628	-126.773447346067\\
57	0.20794	-133.946710051812\\
57	0.2096	-141.119972757557\\
57	0.21126	-148.293235463302\\
57	0.21292	-155.466498169047\\
57	0.21458	-162.639760874792\\
57	0.21624	-169.813023580537\\
57	0.2179	-176.986286286281\\
57	0.21956	-184.159548992026\\
57	0.22122	-191.332811697771\\
57	0.22288	-198.506074403516\\
57	0.22454	-205.679337109261\\
57	0.2262	-212.852599815006\\
57	0.22786	-220.025862520751\\
57	0.22952	-227.199125226496\\
57	0.23118	-234.372387932241\\
57	0.23284	-241.545650637986\\
57	0.2345	-248.71891334373\\
57	0.23616	-255.892176049475\\
57	0.23782	-263.06543875522\\
57	0.23948	-270.238701460965\\
57	0.24114	-277.41196416671\\
57	0.2428	-284.585226872455\\
57	0.24446	-291.7584895782\\
57	0.24612	-298.931752283945\\
57	0.24778	-306.10501498969\\
57	0.24944	-313.278277695434\\
57	0.2511	-320.451540401179\\
57	0.25276	-327.624803106924\\
57	0.25442	-334.798065812669\\
57	0.25608	-341.971328518414\\
57	0.25774	-349.144591224159\\
57	0.2594	-356.317853929904\\
57	0.26106	-363.491116635649\\
57	0.26272	-370.664379341394\\
57	0.26438	-377.837642047139\\
57	0.26604	-385.010904752884\\
57	0.2677	-392.184167458629\\
57	0.26936	-399.357430164373\\
57	0.27102	-406.530692870118\\
57	0.27268	-413.703955575863\\
57	0.27434	-420.877218281608\\
57	0.276	-428.050480987353\\
57.2083333333333	0.193	-68.2406680599865\\
57.2083333333333	0.19466	-75.3576580650079\\
57.2083333333333	0.19632	-82.4746480700292\\
57.2083333333333	0.19798	-89.5916380750507\\
57.2083333333333	0.19964	-96.7086280800722\\
57.2083333333333	0.2013	-103.825618085094\\
57.2083333333333	0.20296	-110.942608090115\\
57.2083333333333	0.20462	-118.059598095137\\
57.2083333333333	0.20628	-125.176588100158\\
57.2083333333333	0.20794	-132.29357810518\\
57.2083333333333	0.2096	-139.410568110201\\
57.2083333333333	0.21126	-146.527558115223\\
57.2083333333333	0.21292	-153.644548120244\\
57.2083333333333	0.21458	-160.761538125265\\
57.2083333333333	0.21624	-167.878528130288\\
57.2083333333333	0.2179	-174.995518135309\\
57.2083333333333	0.21956	-182.11250814033\\
57.2083333333333	0.22122	-189.229498145351\\
57.2083333333333	0.22288	-196.346488150373\\
57.2083333333333	0.22454	-203.463478155394\\
57.2083333333333	0.2262	-210.580468160416\\
57.2083333333333	0.22786	-217.697458165438\\
57.2083333333333	0.22952	-224.814448170459\\
57.2083333333333	0.23118	-231.93143817548\\
57.2083333333333	0.23284	-239.048428180502\\
57.2083333333333	0.2345	-246.165418185523\\
57.2083333333333	0.23616	-253.282408190545\\
57.2083333333333	0.23782	-260.399398195566\\
57.2083333333333	0.23948	-267.516388200588\\
57.2083333333333	0.24114	-274.633378205609\\
57.2083333333333	0.2428	-281.750368210631\\
57.2083333333333	0.24446	-288.867358215652\\
57.2083333333333	0.24612	-295.984348220673\\
57.2083333333333	0.24778	-303.101338225695\\
57.2083333333333	0.24944	-310.218328230717\\
57.2083333333333	0.2511	-317.335318235738\\
57.2083333333333	0.25276	-324.45230824076\\
57.2083333333333	0.25442	-331.569298245782\\
57.2083333333333	0.25608	-338.686288250803\\
57.2083333333333	0.25774	-345.803278255824\\
57.2083333333333	0.2594	-352.920268260846\\
57.2083333333333	0.26106	-360.037258265867\\
57.2083333333333	0.26272	-367.154248270888\\
57.2083333333333	0.26438	-374.27123827591\\
57.2083333333333	0.26604	-381.388228280931\\
57.2083333333333	0.2677	-388.505218285953\\
57.2083333333333	0.26936	-395.622208290974\\
57.2083333333333	0.27102	-402.739198295996\\
57.2083333333333	0.27268	-409.856188301017\\
57.2083333333333	0.27434	-416.973178306038\\
57.2083333333333	0.276	-424.090168311061\\
57.4166666666667	0.193	-67.0939904198647\\
57.4166666666667	0.19466	-74.1547077241626\\
57.4166666666667	0.19632	-81.2154250284605\\
57.4166666666667	0.19798	-88.2761423327588\\
57.4166666666667	0.19964	-95.3368596370567\\
57.4166666666667	0.2013	-102.397576941355\\
57.4166666666667	0.20296	-109.458294245653\\
57.4166666666667	0.20462	-116.519011549951\\
57.4166666666667	0.20628	-123.579728854249\\
57.4166666666667	0.20794	-130.640446158547\\
57.4166666666667	0.2096	-137.701163462845\\
57.4166666666667	0.21126	-144.761880767143\\
57.4166666666667	0.21292	-151.822598071441\\
57.4166666666667	0.21458	-158.883315375739\\
57.4166666666667	0.21624	-165.944032680038\\
57.4166666666667	0.2179	-173.004749984336\\
57.4166666666667	0.21956	-180.065467288634\\
57.4166666666667	0.22122	-187.126184592932\\
57.4166666666667	0.22288	-194.186901897229\\
57.4166666666667	0.22454	-201.247619201527\\
57.4166666666667	0.2262	-208.308336505826\\
57.4166666666667	0.22786	-215.369053810124\\
57.4166666666667	0.22952	-222.429771114422\\
57.4166666666667	0.23118	-229.49048841872\\
57.4166666666667	0.23284	-236.551205723018\\
57.4166666666667	0.2345	-243.611923027316\\
57.4166666666667	0.23616	-250.672640331614\\
57.4166666666667	0.23782	-257.733357635912\\
57.4166666666667	0.23948	-264.79407494021\\
57.4166666666667	0.24114	-271.854792244508\\
57.4166666666667	0.2428	-278.915509548806\\
57.4166666666667	0.24446	-285.976226853104\\
57.4166666666667	0.24612	-293.036944157402\\
57.4166666666667	0.24778	-300.097661461701\\
57.4166666666667	0.24944	-307.158378765999\\
57.4166666666667	0.2511	-314.219096070297\\
57.4166666666667	0.25276	-321.279813374595\\
57.4166666666667	0.25442	-328.340530678893\\
57.4166666666667	0.25608	-335.401247983191\\
57.4166666666667	0.25774	-342.461965287489\\
57.4166666666667	0.2594	-349.522682591787\\
57.4166666666667	0.26106	-356.583399896085\\
57.4166666666667	0.26272	-363.644117200383\\
57.4166666666667	0.26438	-370.704834504681\\
57.4166666666667	0.26604	-377.765551808979\\
57.4166666666667	0.2677	-384.826269113277\\
57.4166666666667	0.26936	-391.886986417575\\
57.4166666666667	0.27102	-398.947703721873\\
57.4166666666667	0.27268	-406.008421026171\\
57.4166666666667	0.27434	-413.069138330469\\
57.4166666666667	0.276	-420.129855634768\\
57.625	0.193	-65.947312779743\\
57.625	0.19466	-72.9517573833175\\
57.625	0.19632	-79.9562019868922\\
57.625	0.19798	-86.9606465904669\\
57.625	0.19964	-93.9650911940414\\
57.625	0.2013	-100.969535797616\\
57.625	0.20296	-107.973980401191\\
57.625	0.20462	-114.978425004766\\
57.625	0.20628	-121.98286960834\\
57.625	0.20794	-128.987314211915\\
57.625	0.2096	-135.991758815489\\
57.625	0.21126	-142.996203419064\\
57.625	0.21292	-150.000648022639\\
57.625	0.21458	-157.005092626213\\
57.625	0.21624	-164.009537229788\\
57.625	0.2179	-171.013981833363\\
57.625	0.21956	-178.018426436937\\
57.625	0.22122	-185.022871040512\\
57.625	0.22288	-192.027315644087\\
57.625	0.22454	-199.031760247661\\
57.625	0.2262	-206.036204851236\\
57.625	0.22786	-213.040649454811\\
57.625	0.22952	-220.045094058385\\
57.625	0.23118	-227.04953866196\\
57.625	0.23284	-234.053983265534\\
57.625	0.2345	-241.058427869109\\
57.625	0.23616	-248.062872472683\\
57.625	0.23782	-255.067317076258\\
57.625	0.23948	-262.071761679833\\
57.625	0.24114	-269.076206283408\\
57.625	0.2428	-276.080650886982\\
57.625	0.24446	-283.085095490557\\
57.625	0.24612	-290.089540094131\\
57.625	0.24778	-297.093984697706\\
57.625	0.24944	-304.098429301281\\
57.625	0.2511	-311.102873904856\\
57.625	0.25276	-318.10731850843\\
57.625	0.25442	-325.111763112005\\
57.625	0.25608	-332.11620771558\\
57.625	0.25774	-339.120652319154\\
57.625	0.2594	-346.125096922729\\
57.625	0.26106	-353.129541526303\\
57.625	0.26272	-360.133986129878\\
57.625	0.26438	-367.138430733452\\
57.625	0.26604	-374.142875337027\\
57.625	0.2677	-381.147319940602\\
57.625	0.26936	-388.151764544177\\
57.625	0.27102	-395.156209147751\\
57.625	0.27268	-402.160653751326\\
57.625	0.27434	-409.1650983549\\
57.625	0.276	-416.169542958475\\
57.8333333333333	0.193	-64.8006351396214\\
57.8333333333333	0.19466	-71.7488070424727\\
57.8333333333333	0.19632	-78.6969789453237\\
57.8333333333333	0.19798	-85.645150848175\\
57.8333333333333	0.19964	-92.5933227510261\\
57.8333333333333	0.2013	-99.5414946538776\\
57.8333333333333	0.20296	-106.489666556729\\
57.8333333333333	0.20462	-113.43783845958\\
57.8333333333333	0.20628	-120.386010362431\\
57.8333333333333	0.20794	-127.334182265282\\
57.8333333333333	0.2096	-134.282354168133\\
57.8333333333333	0.21126	-141.230526070985\\
57.8333333333333	0.21292	-148.178697973837\\
57.8333333333333	0.21458	-155.126869876688\\
57.8333333333333	0.21624	-162.075041779538\\
57.8333333333333	0.2179	-169.02321368239\\
57.8333333333333	0.21956	-175.971385585241\\
57.8333333333333	0.22122	-182.919557488092\\
57.8333333333333	0.22288	-189.867729390944\\
57.8333333333333	0.22454	-196.815901293795\\
57.8333333333333	0.2262	-203.764073196646\\
57.8333333333333	0.22786	-210.712245099497\\
57.8333333333333	0.22952	-217.660417002348\\
57.8333333333333	0.23118	-224.6085889052\\
57.8333333333333	0.23284	-231.556760808051\\
57.8333333333333	0.2345	-238.504932710902\\
57.8333333333333	0.23616	-245.453104613753\\
57.8333333333333	0.23782	-252.401276516604\\
57.8333333333333	0.23948	-259.349448419456\\
57.8333333333333	0.24114	-266.297620322307\\
57.8333333333333	0.2428	-273.245792225158\\
57.8333333333333	0.24446	-280.19396412801\\
57.8333333333333	0.24612	-287.14213603086\\
57.8333333333333	0.24778	-294.090307933711\\
57.8333333333333	0.24944	-301.038479836563\\
57.8333333333333	0.2511	-307.986651739414\\
57.8333333333333	0.25276	-314.934823642265\\
57.8333333333333	0.25442	-321.882995545117\\
57.8333333333333	0.25608	-328.831167447968\\
57.8333333333333	0.25774	-335.779339350819\\
57.8333333333333	0.2594	-342.72751125367\\
57.8333333333333	0.26106	-349.675683156521\\
57.8333333333333	0.26272	-356.623855059373\\
57.8333333333333	0.26438	-363.572026962224\\
57.8333333333333	0.26604	-370.520198865076\\
57.8333333333333	0.2677	-377.468370767927\\
57.8333333333333	0.26936	-384.416542670778\\
57.8333333333333	0.27102	-391.364714573629\\
57.8333333333333	0.27268	-398.31288647648\\
57.8333333333333	0.27434	-405.261058379332\\
57.8333333333333	0.276	-412.209230282182\\
58.0416666666667	0.193	-63.6539574994997\\
58.0416666666667	0.19466	-70.5458567016274\\
58.0416666666667	0.19632	-77.437755903755\\
58.0416666666667	0.19798	-84.3296551058829\\
58.0416666666667	0.19964	-91.2215543080106\\
58.0416666666667	0.2013	-98.1134535101387\\
58.0416666666667	0.20296	-105.005352712267\\
58.0416666666667	0.20462	-111.897251914394\\
58.0416666666667	0.20628	-118.789151116522\\
58.0416666666667	0.20794	-125.68105031865\\
58.0416666666667	0.2096	-132.572949520777\\
58.0416666666667	0.21126	-139.464848722905\\
58.0416666666667	0.21292	-146.356747925034\\
58.0416666666667	0.21458	-153.248647127161\\
58.0416666666667	0.21624	-160.140546329289\\
58.0416666666667	0.2179	-167.032445531417\\
58.0416666666667	0.21956	-173.924344733544\\
58.0416666666667	0.22122	-180.816243935672\\
58.0416666666667	0.22288	-187.7081431378\\
58.0416666666667	0.22454	-194.600042339928\\
58.0416666666667	0.2262	-201.491941542056\\
58.0416666666667	0.22786	-208.383840744184\\
58.0416666666667	0.22952	-215.275739946311\\
58.0416666666667	0.23118	-222.167639148439\\
58.0416666666667	0.23284	-229.059538350567\\
58.0416666666667	0.2345	-235.951437552695\\
58.0416666666667	0.23616	-242.843336754823\\
58.0416666666667	0.23782	-249.73523595695\\
58.0416666666667	0.23948	-256.627135159079\\
58.0416666666667	0.24114	-263.519034361206\\
58.0416666666667	0.2428	-270.410933563334\\
58.0416666666667	0.24446	-277.302832765462\\
58.0416666666667	0.24612	-284.19473196759\\
58.0416666666667	0.24778	-291.086631169717\\
58.0416666666667	0.24944	-297.978530371844\\
58.0416666666667	0.2511	-304.870429573973\\
58.0416666666667	0.25276	-311.7623287761\\
58.0416666666667	0.25442	-318.654227978228\\
58.0416666666667	0.25608	-325.546127180356\\
58.0416666666667	0.25774	-332.438026382484\\
58.0416666666667	0.2594	-339.329925584611\\
58.0416666666667	0.26106	-346.221824786739\\
58.0416666666667	0.26272	-353.113723988868\\
58.0416666666667	0.26438	-360.005623190995\\
58.0416666666667	0.26604	-366.897522393123\\
58.0416666666667	0.2677	-373.789421595251\\
58.0416666666667	0.26936	-380.681320797379\\
58.0416666666667	0.27102	-387.573219999507\\
58.0416666666667	0.27268	-394.465119201634\\
58.0416666666667	0.27434	-401.357018403762\\
58.0416666666667	0.276	-408.24891760589\\
58.25	0.193	-62.5072798593781\\
58.25	0.19466	-69.3429063607823\\
58.25	0.19632	-76.1785328621866\\
58.25	0.19798	-83.014159363591\\
58.25	0.19964	-89.8497858649953\\
58.25	0.2013	-96.6854123664002\\
58.25	0.20296	-103.521038867804\\
58.25	0.20462	-110.356665369209\\
58.25	0.20628	-117.192291870613\\
58.25	0.20794	-124.027918372017\\
58.25	0.2096	-130.863544873422\\
58.25	0.21126	-137.699171374826\\
58.25	0.21292	-144.534797876231\\
58.25	0.21458	-151.370424377635\\
58.25	0.21624	-158.20605087904\\
58.25	0.2179	-165.041677380444\\
58.25	0.21956	-171.877303881848\\
58.25	0.22122	-178.712930383253\\
58.25	0.22288	-185.548556884657\\
58.25	0.22454	-192.384183386061\\
58.25	0.2262	-199.219809887466\\
58.25	0.22786	-206.05543638887\\
58.25	0.22952	-212.891062890274\\
58.25	0.23118	-219.726689391679\\
58.25	0.23284	-226.562315893083\\
58.25	0.2345	-233.397942394487\\
58.25	0.23616	-240.233568895892\\
58.25	0.23782	-247.069195397296\\
58.25	0.23948	-253.904821898701\\
58.25	0.24114	-260.740448400105\\
58.25	0.2428	-267.57607490151\\
58.25	0.24446	-274.411701402914\\
58.25	0.24612	-281.247327904318\\
58.25	0.24778	-288.082954405723\\
58.25	0.24944	-294.918580907127\\
58.25	0.2511	-301.754207408531\\
58.25	0.25276	-308.589833909936\\
58.25	0.25442	-315.42546041134\\
58.25	0.25608	-322.261086912745\\
58.25	0.25774	-329.096713414149\\
58.25	0.2594	-335.932339915553\\
58.25	0.26106	-342.767966416958\\
58.25	0.26272	-349.603592918362\\
58.25	0.26438	-356.439219419767\\
58.25	0.26604	-363.274845921171\\
58.25	0.2677	-370.110472422576\\
58.25	0.26936	-376.94609892398\\
58.25	0.27102	-383.781725425384\\
58.25	0.27268	-390.617351926789\\
58.25	0.27434	-397.452978428193\\
58.25	0.276	-404.288604929598\\
58.4583333333333	0.193	-61.3606022192564\\
58.4583333333333	0.19466	-68.1399560199372\\
58.4583333333333	0.19632	-74.9193098206181\\
58.4583333333333	0.19798	-81.6986636212991\\
58.4583333333333	0.19964	-88.4780174219802\\
58.4583333333333	0.2013	-95.2573712226615\\
58.4583333333333	0.20296	-102.036725023342\\
58.4583333333333	0.20462	-108.816078824023\\
58.4583333333333	0.20628	-115.595432624704\\
58.4583333333333	0.20794	-122.374786425385\\
58.4583333333333	0.2096	-129.154140226066\\
58.4583333333333	0.21126	-135.933494026747\\
58.4583333333333	0.21292	-142.712847827428\\
58.4583333333333	0.21458	-149.492201628109\\
58.4583333333333	0.21624	-156.27155542879\\
58.4583333333333	0.2179	-163.050909229471\\
58.4583333333333	0.21956	-169.830263030152\\
58.4583333333333	0.22122	-176.609616830833\\
58.4583333333333	0.22288	-183.388970631514\\
58.4583333333333	0.22454	-190.168324432194\\
58.4583333333333	0.2262	-196.947678232876\\
58.4583333333333	0.22786	-203.727032033557\\
58.4583333333333	0.22952	-210.506385834238\\
58.4583333333333	0.23118	-217.285739634919\\
58.4583333333333	0.23284	-224.065093435599\\
58.4583333333333	0.2345	-230.84444723628\\
58.4583333333333	0.23616	-237.623801036961\\
58.4583333333333	0.23782	-244.403154837643\\
58.4583333333333	0.23948	-251.182508638324\\
58.4583333333333	0.24114	-257.961862439005\\
58.4583333333333	0.2428	-264.741216239685\\
58.4583333333333	0.24446	-271.520570040366\\
58.4583333333333	0.24612	-278.299923841047\\
58.4583333333333	0.24778	-285.079277641728\\
58.4583333333333	0.24944	-291.858631442409\\
58.4583333333333	0.2511	-298.63798524309\\
58.4583333333333	0.25276	-305.417339043771\\
58.4583333333333	0.25442	-312.196692844453\\
58.4583333333333	0.25608	-318.976046645133\\
58.4583333333333	0.25774	-325.755400445814\\
58.4583333333333	0.2594	-332.534754246495\\
58.4583333333333	0.26106	-339.314108047176\\
58.4583333333333	0.26272	-346.093461847857\\
58.4583333333333	0.26438	-352.872815648538\\
58.4583333333333	0.26604	-359.652169449219\\
58.4583333333333	0.2677	-366.4315232499\\
58.4583333333333	0.26936	-373.210877050581\\
58.4583333333333	0.27102	-379.990230851262\\
58.4583333333333	0.27268	-386.769584651943\\
58.4583333333333	0.27434	-393.548938452624\\
58.4583333333333	0.276	-400.328292253305\\
58.6666666666667	0.193	-60.2139245791345\\
58.6666666666667	0.19466	-66.937005679092\\
58.6666666666667	0.19632	-73.6600867790496\\
58.6666666666667	0.19798	-80.3831678790073\\
58.6666666666667	0.19964	-87.1062489789647\\
58.6666666666667	0.2013	-93.8293300789226\\
58.6666666666667	0.20296	-100.55241117888\\
58.6666666666667	0.20462	-107.275492278837\\
58.6666666666667	0.20628	-113.998573378795\\
58.6666666666667	0.20794	-120.721654478753\\
58.6666666666667	0.2096	-127.44473557871\\
58.6666666666667	0.21126	-134.167816678668\\
58.6666666666667	0.21292	-140.890897778625\\
58.6666666666667	0.21458	-147.613978878583\\
58.6666666666667	0.21624	-154.337059978541\\
58.6666666666667	0.2179	-161.060141078498\\
58.6666666666667	0.21956	-167.783222178456\\
58.6666666666667	0.22122	-174.506303278413\\
58.6666666666667	0.22288	-181.22938437837\\
58.6666666666667	0.22454	-187.952465478328\\
58.6666666666667	0.2262	-194.675546578286\\
58.6666666666667	0.22786	-201.398627678243\\
58.6666666666667	0.22952	-208.121708778201\\
58.6666666666667	0.23118	-214.844789878158\\
58.6666666666667	0.23284	-221.567870978115\\
58.6666666666667	0.2345	-228.290952078073\\
58.6666666666667	0.23616	-235.01403317803\\
58.6666666666667	0.23782	-241.737114277988\\
58.6666666666667	0.23948	-248.460195377946\\
58.6666666666667	0.24114	-255.183276477903\\
58.6666666666667	0.2428	-261.906357577861\\
58.6666666666667	0.24446	-268.629438677818\\
58.6666666666667	0.24612	-275.352519777776\\
58.6666666666667	0.24778	-282.075600877734\\
58.6666666666667	0.24944	-288.798681977692\\
58.6666666666667	0.2511	-295.521763077649\\
58.6666666666667	0.25276	-302.244844177606\\
58.6666666666667	0.25442	-308.967925277565\\
58.6666666666667	0.25608	-315.691006377522\\
58.6666666666667	0.25774	-322.414087477479\\
58.6666666666667	0.2594	-329.137168577437\\
58.6666666666667	0.26106	-335.860249677394\\
58.6666666666667	0.26272	-342.583330777351\\
58.6666666666667	0.26438	-349.306411877309\\
58.6666666666667	0.26604	-356.029492977267\\
58.6666666666667	0.2677	-362.752574077224\\
58.6666666666667	0.26936	-369.475655177182\\
58.6666666666667	0.27102	-376.198736277139\\
58.6666666666667	0.27268	-382.921817377097\\
58.6666666666667	0.27434	-389.644898477054\\
58.6666666666667	0.276	-396.367979577013\\
58.875	0.193	-59.0672469390131\\
58.875	0.19466	-65.7340553382471\\
58.875	0.19632	-72.4008637374811\\
58.875	0.19798	-79.0676721367154\\
58.875	0.19964	-85.7344805359494\\
58.875	0.2013	-92.4012889351839\\
58.875	0.20296	-99.0680973344179\\
58.875	0.20462	-105.734905733652\\
58.875	0.20628	-112.401714132886\\
58.875	0.20794	-119.06852253212\\
58.875	0.2096	-125.735330931354\\
58.875	0.21126	-132.402139330588\\
58.875	0.21292	-139.068947729823\\
58.875	0.21458	-145.735756129057\\
58.875	0.21624	-152.402564528291\\
58.875	0.2179	-159.069372927525\\
58.875	0.21956	-165.736181326759\\
58.875	0.22122	-172.402989725993\\
58.875	0.22288	-179.069798125227\\
58.875	0.22454	-185.736606524461\\
58.875	0.2262	-192.403414923696\\
58.875	0.22786	-199.07022332293\\
58.875	0.22952	-205.737031722164\\
58.875	0.23118	-212.403840121398\\
58.875	0.23284	-219.070648520632\\
58.875	0.2345	-225.737456919866\\
58.875	0.23616	-232.4042653191\\
58.875	0.23782	-239.071073718334\\
58.875	0.23948	-245.737882117569\\
58.875	0.24114	-252.404690516803\\
58.875	0.2428	-259.071498916037\\
58.875	0.24446	-265.738307315271\\
58.875	0.24612	-272.405115714505\\
58.875	0.24778	-279.071924113739\\
58.875	0.24944	-285.738732512973\\
58.875	0.2511	-292.405540912207\\
58.875	0.25276	-299.072349311441\\
58.875	0.25442	-305.739157710676\\
58.875	0.25608	-312.40596610991\\
58.875	0.25774	-319.072774509144\\
58.875	0.2594	-325.739582908378\\
58.875	0.26106	-332.406391307612\\
58.875	0.26272	-339.073199706846\\
58.875	0.26438	-345.740008106081\\
58.875	0.26604	-352.406816505315\\
58.875	0.2677	-359.073624904549\\
58.875	0.26936	-365.740433303783\\
58.875	0.27102	-372.407241703017\\
58.875	0.27268	-379.074050102251\\
58.875	0.27434	-385.740858501485\\
58.875	0.276	-392.40766690072\\
59.0833333333333	0.193	-57.9205692988915\\
59.0833333333333	0.19466	-64.5311049974021\\
59.0833333333333	0.19632	-71.1416406959127\\
59.0833333333333	0.19798	-77.7521763944235\\
59.0833333333333	0.19964	-84.3627120929341\\
59.0833333333333	0.2013	-90.9732477914451\\
59.0833333333333	0.20296	-97.583783489956\\
59.0833333333333	0.20462	-104.194319188467\\
59.0833333333333	0.20628	-110.804854886977\\
59.0833333333333	0.20794	-117.415390585488\\
59.0833333333333	0.2096	-124.025926283998\\
59.0833333333333	0.21126	-130.636461982509\\
59.0833333333333	0.21292	-137.24699768102\\
59.0833333333333	0.21458	-143.857533379531\\
59.0833333333333	0.21624	-150.468069078041\\
59.0833333333333	0.2179	-157.078604776552\\
59.0833333333333	0.21956	-163.689140475063\\
59.0833333333333	0.22122	-170.299676173573\\
59.0833333333333	0.22288	-176.910211872084\\
59.0833333333333	0.22454	-183.520747570595\\
59.0833333333333	0.2262	-190.131283269106\\
59.0833333333333	0.22786	-196.741818967616\\
59.0833333333333	0.22952	-203.352354666127\\
59.0833333333333	0.23118	-209.962890364638\\
59.0833333333333	0.23284	-216.573426063149\\
59.0833333333333	0.2345	-223.183961761659\\
59.0833333333333	0.23616	-229.79449746017\\
59.0833333333333	0.23782	-236.40503315868\\
59.0833333333333	0.23948	-243.015568857192\\
59.0833333333333	0.24114	-249.626104555703\\
59.0833333333333	0.2428	-256.236640254213\\
59.0833333333333	0.24446	-262.847175952724\\
59.0833333333333	0.24612	-269.457711651234\\
59.0833333333333	0.24778	-276.068247349745\\
59.0833333333333	0.24944	-282.678783048255\\
59.0833333333333	0.2511	-289.289318746766\\
59.0833333333333	0.25276	-295.899854445277\\
59.0833333333333	0.25442	-302.510390143788\\
59.0833333333333	0.25608	-309.120925842298\\
59.0833333333333	0.25774	-315.731461540809\\
59.0833333333333	0.2594	-322.341997239319\\
59.0833333333333	0.26106	-328.95253293783\\
59.0833333333333	0.26272	-335.563068636342\\
59.0833333333333	0.26438	-342.173604334852\\
59.0833333333333	0.26604	-348.784140033363\\
59.0833333333333	0.2677	-355.394675731874\\
59.0833333333333	0.26936	-362.005211430384\\
59.0833333333333	0.27102	-368.615747128895\\
59.0833333333333	0.27268	-375.226282827406\\
59.0833333333333	0.27434	-381.836818525916\\
59.0833333333333	0.276	-388.447354224427\\
59.2916666666667	0.193	-56.7738916587696\\
59.2916666666667	0.19466	-63.3281546565568\\
59.2916666666667	0.19632	-69.882417654344\\
59.2916666666667	0.19798	-76.4366806521314\\
59.2916666666667	0.19964	-82.9909436499186\\
59.2916666666667	0.2013	-89.5452066477062\\
59.2916666666667	0.20296	-96.0994696454936\\
59.2916666666667	0.20462	-102.653732643281\\
59.2916666666667	0.20628	-109.207995641068\\
59.2916666666667	0.20794	-115.762258638855\\
59.2916666666667	0.2096	-122.316521636642\\
59.2916666666667	0.21126	-128.870784634429\\
59.2916666666667	0.21292	-135.425047632218\\
59.2916666666667	0.21458	-141.979310630005\\
59.2916666666667	0.21624	-148.533573627792\\
59.2916666666667	0.2179	-155.087836625579\\
59.2916666666667	0.21956	-161.642099623366\\
59.2916666666667	0.22122	-168.196362621153\\
59.2916666666667	0.22288	-174.750625618941\\
59.2916666666667	0.22454	-181.304888616728\\
59.2916666666667	0.2262	-187.859151614516\\
59.2916666666667	0.22786	-194.413414612303\\
59.2916666666667	0.22952	-200.96767761009\\
59.2916666666667	0.23118	-207.521940607878\\
59.2916666666667	0.23284	-214.076203605665\\
59.2916666666667	0.2345	-220.630466603452\\
59.2916666666667	0.23616	-227.184729601239\\
59.2916666666667	0.23782	-233.738992599026\\
59.2916666666667	0.23948	-240.293255596814\\
59.2916666666667	0.24114	-246.847518594601\\
59.2916666666667	0.2428	-253.401781592388\\
59.2916666666667	0.24446	-259.956044590176\\
59.2916666666667	0.24612	-266.510307587963\\
59.2916666666667	0.24778	-273.06457058575\\
59.2916666666667	0.24944	-279.618833583537\\
59.2916666666667	0.2511	-286.173096581325\\
59.2916666666667	0.25276	-292.727359579112\\
59.2916666666667	0.25442	-299.281622576899\\
59.2916666666667	0.25608	-305.835885574686\\
59.2916666666667	0.25774	-312.390148572474\\
59.2916666666667	0.2594	-318.944411570261\\
59.2916666666667	0.26106	-325.498674568049\\
59.2916666666667	0.26272	-332.052937565837\\
59.2916666666667	0.26438	-338.607200563624\\
59.2916666666667	0.26604	-345.161463561411\\
59.2916666666667	0.2677	-351.715726559198\\
59.2916666666667	0.26936	-358.269989556985\\
59.2916666666667	0.27102	-364.824252554773\\
59.2916666666667	0.27268	-371.37851555256\\
59.2916666666667	0.27434	-377.932778550347\\
59.2916666666667	0.276	-384.487041548134\\
59.5	0.193	-55.6272140186479\\
59.5	0.19466	-62.1252043157117\\
59.5	0.19632	-68.6231946127755\\
59.5	0.19798	-75.1211849098395\\
59.5	0.19964	-81.6191752069033\\
59.5	0.2013	-88.1171655039675\\
59.5	0.20296	-94.6151558010313\\
59.5	0.20462	-101.113146098095\\
59.5	0.20628	-107.611136395159\\
59.5	0.20794	-114.109126692223\\
59.5	0.2096	-120.607116989287\\
59.5	0.21126	-127.10510728635\\
59.5	0.21292	-133.603097583415\\
59.5	0.21458	-140.101087880478\\
59.5	0.21624	-146.599078177542\\
59.5	0.2179	-153.097068474606\\
59.5	0.21956	-159.59505877167\\
59.5	0.22122	-166.093049068734\\
59.5	0.22288	-172.591039365798\\
59.5	0.22454	-179.089029662861\\
59.5	0.2262	-185.587019959926\\
59.5	0.22786	-192.085010256989\\
59.5	0.22952	-198.583000554053\\
59.5	0.23118	-205.080990851117\\
59.5	0.23284	-211.578981148181\\
59.5	0.2345	-218.076971445245\\
59.5	0.23616	-224.574961742308\\
59.5	0.23782	-231.072952039372\\
59.5	0.23948	-237.570942336437\\
59.5	0.24114	-244.0689326335\\
59.5	0.2428	-250.566922930564\\
59.5	0.24446	-257.064913227628\\
59.5	0.24612	-263.562903524692\\
59.5	0.24778	-270.060893821756\\
59.5	0.24944	-276.55888411882\\
59.5	0.2511	-283.056874415884\\
59.5	0.25276	-289.554864712948\\
59.5	0.25442	-296.052855010012\\
59.5	0.25608	-302.550845307076\\
59.5	0.25774	-309.048835604139\\
59.5	0.2594	-315.546825901203\\
59.5	0.26106	-322.044816198267\\
59.5	0.26272	-328.542806495331\\
59.5	0.26438	-335.040796792395\\
59.5	0.26604	-341.538787089459\\
59.5	0.2677	-348.036777386523\\
59.5	0.26936	-354.534767683586\\
59.5	0.27102	-361.03275798065\\
59.5	0.27268	-367.530748277714\\
59.5	0.27434	-374.028738574778\\
59.5	0.276	-380.526728871842\\
59.7083333333333	0.193	-54.4805363785263\\
59.7083333333333	0.19466	-60.9222539748666\\
59.7083333333333	0.19632	-67.363971571207\\
59.7083333333333	0.19798	-73.8056891675476\\
59.7083333333333	0.19964	-80.247406763888\\
59.7083333333333	0.2013	-86.6891243602288\\
59.7083333333333	0.20296	-93.1308419565689\\
59.7083333333333	0.20462	-99.5725595529093\\
59.7083333333333	0.20628	-106.01427714925\\
59.7083333333333	0.20794	-112.455994745591\\
59.7083333333333	0.2096	-118.897712341931\\
59.7083333333333	0.21126	-125.339429938271\\
59.7083333333333	0.21292	-131.781147534612\\
59.7083333333333	0.21458	-138.222865130952\\
59.7083333333333	0.21624	-144.664582727293\\
59.7083333333333	0.2179	-151.106300323634\\
59.7083333333333	0.21956	-157.548017919974\\
59.7083333333333	0.22122	-163.989735516314\\
59.7083333333333	0.22288	-170.431453112654\\
59.7083333333333	0.22454	-176.873170708995\\
59.7083333333333	0.2262	-183.314888305336\\
59.7083333333333	0.22786	-189.756605901676\\
59.7083333333333	0.22952	-196.198323498017\\
59.7083333333333	0.23118	-202.640041094357\\
59.7083333333333	0.23284	-209.081758690697\\
59.7083333333333	0.2345	-215.523476287038\\
59.7083333333333	0.23616	-221.965193883378\\
59.7083333333333	0.23782	-228.406911479719\\
59.7083333333333	0.23948	-234.848629076059\\
59.7083333333333	0.24114	-241.2903466724\\
59.7083333333333	0.2428	-247.73206426874\\
59.7083333333333	0.24446	-254.17378186508\\
59.7083333333333	0.24612	-260.615499461421\\
59.7083333333333	0.24778	-267.057217057762\\
59.7083333333333	0.24944	-273.498934654102\\
59.7083333333333	0.2511	-279.940652250443\\
59.7083333333333	0.25276	-286.382369846783\\
59.7083333333333	0.25442	-292.824087443124\\
59.7083333333333	0.25608	-299.265805039464\\
59.7083333333333	0.25774	-305.707522635805\\
59.7083333333333	0.2594	-312.149240232145\\
59.7083333333333	0.26106	-318.590957828485\\
59.7083333333333	0.26272	-325.032675424826\\
59.7083333333333	0.26438	-331.474393021166\\
59.7083333333333	0.26604	-337.916110617507\\
59.7083333333333	0.2677	-344.357828213847\\
59.7083333333333	0.26936	-350.799545810187\\
59.7083333333333	0.27102	-357.241263406528\\
59.7083333333333	0.27268	-363.682981002868\\
59.7083333333333	0.27434	-370.124698599208\\
59.7083333333333	0.276	-376.56641619555\\
59.9166666666667	0.193	-53.3338587384046\\
59.9166666666667	0.19466	-59.7193036340216\\
59.9166666666667	0.19632	-66.1047485296385\\
59.9166666666667	0.19798	-72.4901934252555\\
59.9166666666667	0.19964	-78.8756383208724\\
59.9166666666667	0.2013	-85.2610832164899\\
59.9166666666667	0.20296	-91.6465281121066\\
59.9166666666667	0.20462	-98.0319730077235\\
59.9166666666667	0.20628	-104.417417903341\\
59.9166666666667	0.20794	-110.802862798958\\
59.9166666666667	0.2096	-117.188307694575\\
59.9166666666667	0.21126	-123.573752590192\\
59.9166666666667	0.21292	-129.959197485809\\
59.9166666666667	0.21458	-136.344642381426\\
59.9166666666667	0.21624	-142.730087277043\\
59.9166666666667	0.2179	-149.115532172661\\
59.9166666666667	0.21956	-155.500977068277\\
59.9166666666667	0.22122	-161.886421963894\\
59.9166666666667	0.22288	-168.271866859511\\
59.9166666666667	0.22454	-174.657311755128\\
59.9166666666667	0.2262	-181.042756650746\\
59.9166666666667	0.22786	-187.428201546363\\
59.9166666666667	0.22952	-193.813646441979\\
59.9166666666667	0.23118	-200.199091337596\\
59.9166666666667	0.23284	-206.584536233213\\
59.9166666666667	0.2345	-212.96998112883\\
59.9166666666667	0.23616	-219.355426024447\\
59.9166666666667	0.23782	-225.740870920065\\
59.9166666666667	0.23948	-232.126315815682\\
59.9166666666667	0.24114	-238.511760711299\\
59.9166666666667	0.2428	-244.897205606916\\
59.9166666666667	0.24446	-251.282650502533\\
59.9166666666667	0.24612	-257.668095398149\\
59.9166666666667	0.24778	-264.053540293767\\
59.9166666666667	0.24944	-270.438985189384\\
59.9166666666667	0.2511	-276.824430085001\\
59.9166666666667	0.25276	-283.209874980618\\
59.9166666666667	0.25442	-289.595319876235\\
59.9166666666667	0.25608	-295.980764771853\\
59.9166666666667	0.25774	-302.366209667469\\
59.9166666666667	0.2594	-308.751654563087\\
59.9166666666667	0.26106	-315.137099458703\\
59.9166666666667	0.26272	-321.52254435432\\
59.9166666666667	0.26438	-327.907989249937\\
59.9166666666667	0.26604	-334.293434145554\\
59.9166666666667	0.2677	-340.678879041172\\
59.9166666666667	0.26936	-347.064323936788\\
59.9166666666667	0.27102	-353.449768832405\\
59.9166666666667	0.27268	-359.835213728022\\
59.9166666666667	0.27434	-366.220658623639\\
59.9166666666667	0.276	-372.606103519257\\
60.125	0.193	-52.187181098283\\
60.125	0.19466	-58.5163532931765\\
60.125	0.19632	-64.8455254880701\\
60.125	0.19798	-71.1746976829638\\
60.125	0.19964	-77.5038698778571\\
60.125	0.2013	-83.8330420727511\\
60.125	0.20296	-90.1622142676447\\
60.125	0.20462	-96.4913864625382\\
60.125	0.20628	-102.820558657432\\
60.125	0.20794	-109.149730852326\\
60.125	0.2096	-115.478903047219\\
60.125	0.21126	-121.808075242112\\
60.125	0.21292	-128.137247437006\\
60.125	0.21458	-134.4664196319\\
60.125	0.21624	-140.795591826794\\
60.125	0.2179	-147.124764021687\\
60.125	0.21956	-153.453936216581\\
60.125	0.22122	-159.783108411475\\
60.125	0.22288	-166.112280606368\\
60.125	0.22454	-172.441452801262\\
60.125	0.2262	-178.770624996156\\
60.125	0.22786	-185.099797191049\\
60.125	0.22952	-191.428969385943\\
60.125	0.23118	-197.758141580836\\
60.125	0.23284	-204.08731377573\\
60.125	0.2345	-210.416485970623\\
60.125	0.23616	-216.745658165517\\
60.125	0.23782	-223.07483036041\\
60.125	0.23948	-229.404002555304\\
60.125	0.24114	-235.733174750198\\
60.125	0.2428	-242.062346945092\\
60.125	0.24446	-248.391519139985\\
60.125	0.24612	-254.720691334879\\
60.125	0.24778	-261.049863529773\\
60.125	0.24944	-267.379035724666\\
60.125	0.2511	-273.70820791956\\
60.125	0.25276	-280.037380114453\\
60.125	0.25442	-286.366552309347\\
60.125	0.25608	-292.695724504241\\
60.125	0.25774	-299.024896699134\\
60.125	0.2594	-305.354068894028\\
60.125	0.26106	-311.683241088921\\
60.125	0.26272	-318.012413283815\\
60.125	0.26438	-324.341585478709\\
60.125	0.26604	-330.670757673603\\
60.125	0.2677	-336.999929868496\\
60.125	0.26936	-343.32910206339\\
60.125	0.27102	-349.658274258283\\
60.125	0.27268	-355.987446453177\\
60.125	0.27434	-362.31661864807\\
60.125	0.276	-368.645790842964\\
60.3333333333333	0.193	-51.0405034581613\\
60.3333333333333	0.19466	-57.3134029523314\\
60.3333333333333	0.19632	-63.5863024465016\\
60.3333333333333	0.19798	-69.8592019406719\\
60.3333333333333	0.19964	-76.1321014348421\\
60.3333333333333	0.2013	-82.4050009290127\\
60.3333333333333	0.20296	-88.6779004231828\\
60.3333333333333	0.20462	-94.9507999173529\\
60.3333333333333	0.20628	-101.223699411523\\
60.3333333333333	0.20794	-107.496598905693\\
60.3333333333333	0.2096	-113.769498399863\\
60.3333333333333	0.21126	-120.042397894033\\
60.3333333333333	0.21292	-126.315297388204\\
60.3333333333333	0.21458	-132.588196882374\\
60.3333333333333	0.21624	-138.861096376544\\
60.3333333333333	0.2179	-145.133995870714\\
60.3333333333333	0.21956	-151.406895364884\\
60.3333333333333	0.22122	-157.679794859054\\
60.3333333333333	0.22288	-163.952694353225\\
60.3333333333333	0.22454	-170.225593847395\\
60.3333333333333	0.2262	-176.498493341565\\
60.3333333333333	0.22786	-182.771392835736\\
60.3333333333333	0.22952	-189.044292329906\\
60.3333333333333	0.23118	-195.317191824076\\
60.3333333333333	0.23284	-201.590091318246\\
60.3333333333333	0.2345	-207.862990812417\\
60.3333333333333	0.23616	-214.135890306587\\
60.3333333333333	0.23782	-220.408789800756\\
60.3333333333333	0.23948	-226.681689294928\\
60.3333333333333	0.24114	-232.954588789098\\
60.3333333333333	0.2428	-239.227488283268\\
60.3333333333333	0.24446	-245.500387777438\\
60.3333333333333	0.24612	-251.773287271608\\
60.3333333333333	0.24778	-258.046186765778\\
60.3333333333333	0.24944	-264.319086259948\\
60.3333333333333	0.2511	-270.591985754118\\
60.3333333333333	0.25276	-276.864885248288\\
60.3333333333333	0.25442	-283.137784742459\\
60.3333333333333	0.25608	-289.410684236629\\
60.3333333333333	0.25774	-295.683583730799\\
60.3333333333333	0.2594	-301.956483224969\\
60.3333333333333	0.26106	-308.229382719139\\
60.3333333333333	0.26272	-314.502282213311\\
60.3333333333333	0.26438	-320.77518170748\\
60.3333333333333	0.26604	-327.048081201651\\
60.3333333333333	0.2677	-333.320980695821\\
60.3333333333333	0.26936	-339.593880189991\\
60.3333333333333	0.27102	-345.866779684161\\
60.3333333333333	0.27268	-352.139679178331\\
60.3333333333333	0.27434	-358.412578672502\\
60.3333333333333	0.276	-364.685478166672\\
60.5416666666667	0.193	-49.8938258180394\\
60.5416666666667	0.19466	-56.1104526114862\\
60.5416666666667	0.19632	-62.3270794049329\\
60.5416666666667	0.19798	-68.5437061983798\\
60.5416666666667	0.19964	-74.7603329918265\\
60.5416666666667	0.2013	-80.9769597852739\\
60.5416666666667	0.20296	-87.1935865787204\\
60.5416666666667	0.20462	-93.4102133721672\\
60.5416666666667	0.20628	-99.6268401656137\\
60.5416666666667	0.20794	-105.84346695906\\
60.5416666666667	0.2096	-112.060093752507\\
60.5416666666667	0.21126	-118.276720545954\\
60.5416666666667	0.21292	-124.493347339401\\
60.5416666666667	0.21458	-130.709974132848\\
60.5416666666667	0.21624	-136.926600926294\\
60.5416666666667	0.2179	-143.143227719741\\
60.5416666666667	0.21956	-149.359854513188\\
60.5416666666667	0.22122	-155.576481306634\\
60.5416666666667	0.22288	-161.793108100082\\
60.5416666666667	0.22454	-168.009734893529\\
60.5416666666667	0.2262	-174.226361686975\\
60.5416666666667	0.22786	-180.442988480422\\
60.5416666666667	0.22952	-186.659615273868\\
60.5416666666667	0.23118	-192.876242067316\\
60.5416666666667	0.23284	-199.092868860763\\
60.5416666666667	0.2345	-205.30949565421\\
60.5416666666667	0.23616	-211.526122447656\\
60.5416666666667	0.23782	-217.742749241103\\
60.5416666666667	0.23948	-223.95937603455\\
60.5416666666667	0.24114	-230.176002827997\\
60.5416666666667	0.2428	-236.392629621444\\
60.5416666666667	0.24446	-242.60925641489\\
60.5416666666667	0.24612	-248.825883208337\\
60.5416666666667	0.24778	-255.042510001783\\
60.5416666666667	0.24944	-261.25913679523\\
60.5416666666667	0.2511	-267.475763588677\\
60.5416666666667	0.25276	-273.692390382123\\
60.5416666666667	0.25442	-279.909017175571\\
60.5416666666667	0.25608	-286.125643969017\\
60.5416666666667	0.25774	-292.342270762464\\
60.5416666666667	0.2594	-298.558897555911\\
60.5416666666667	0.26106	-304.775524349358\\
60.5416666666667	0.26272	-310.992151142805\\
60.5416666666667	0.26438	-317.208777936252\\
60.5416666666667	0.26604	-323.425404729699\\
60.5416666666667	0.2677	-329.642031523145\\
60.5416666666667	0.26936	-335.858658316592\\
60.5416666666667	0.27102	-342.075285110039\\
60.5416666666667	0.27268	-348.291911903485\\
60.5416666666667	0.27434	-354.508538696932\\
60.5416666666667	0.276	-360.725165490379\\
60.75	0.193	-48.747148177918\\
60.75	0.19466	-54.9075022706411\\
60.75	0.19632	-61.0678563633644\\
60.75	0.19798	-67.2282104560879\\
60.75	0.19964	-73.3885645488112\\
60.75	0.2013	-79.548918641535\\
60.75	0.20296	-85.7092727342581\\
60.75	0.20462	-91.8696268269814\\
60.75	0.20628	-98.0299809197049\\
60.75	0.20794	-104.190335012428\\
60.75	0.2096	-110.350689105152\\
60.75	0.21126	-116.511043197875\\
60.75	0.21292	-122.671397290599\\
60.75	0.21458	-128.831751383322\\
60.75	0.21624	-134.992105476045\\
60.75	0.2179	-141.152459568768\\
60.75	0.21956	-147.312813661492\\
60.75	0.22122	-153.473167754215\\
60.75	0.22288	-159.633521846938\\
60.75	0.22454	-165.793875939662\\
60.75	0.2262	-171.954230032386\\
60.75	0.22786	-178.114584125109\\
60.75	0.22952	-184.274938217832\\
60.75	0.23118	-190.435292310555\\
60.75	0.23284	-196.595646403279\\
60.75	0.2345	-202.756000496002\\
60.75	0.23616	-208.916354588725\\
60.75	0.23782	-215.076708681449\\
60.75	0.23948	-221.237062774173\\
60.75	0.24114	-227.397416866896\\
60.75	0.2428	-233.557770959619\\
60.75	0.24446	-239.718125052342\\
60.75	0.24612	-245.878479145066\\
60.75	0.24778	-252.038833237789\\
60.75	0.24944	-258.199187330512\\
60.75	0.2511	-264.359541423236\\
60.75	0.25276	-270.519895515959\\
60.75	0.25442	-276.680249608683\\
60.75	0.25608	-282.840603701406\\
60.75	0.25774	-289.000957794129\\
60.75	0.2594	-295.161311886853\\
60.75	0.26106	-301.321665979576\\
60.75	0.26272	-307.4820200723\\
60.75	0.26438	-313.642374165023\\
60.75	0.26604	-319.802728257746\\
60.75	0.2677	-325.96308235047\\
60.75	0.26936	-332.123436443193\\
60.75	0.27102	-338.283790535917\\
60.75	0.27268	-344.44414462864\\
60.75	0.27434	-350.604498721363\\
60.75	0.276	-356.764852814087\\
60.9583333333333	0.193	-47.6004705377964\\
60.9583333333333	0.19466	-53.7045519297963\\
60.9583333333333	0.19632	-59.8086333217959\\
60.9583333333333	0.19798	-65.912714713796\\
60.9583333333333	0.19964	-72.0167961057962\\
60.9583333333333	0.2013	-78.1208774977961\\
60.9583333333333	0.20296	-84.224958889796\\
60.9583333333333	0.20462	-90.3290402817956\\
60.9583333333333	0.20628	-96.4331216737962\\
60.9583333333333	0.20794	-102.537203065796\\
60.9583333333333	0.2096	-108.641284457796\\
60.9583333333333	0.21126	-114.745365849796\\
60.9583333333333	0.21292	-120.849447241796\\
60.9583333333333	0.21458	-126.953528633795\\
60.9583333333333	0.21624	-133.057610025796\\
60.9583333333333	0.2179	-139.161691417796\\
60.9583333333333	0.21956	-145.265772809796\\
60.9583333333333	0.22122	-151.369854201796\\
60.9583333333333	0.22288	-157.473935593795\\
60.9583333333333	0.22454	-163.578016985795\\
60.9583333333333	0.2262	-169.682098377796\\
60.9583333333333	0.22786	-175.786179769796\\
60.9583333333333	0.22952	-181.890261161795\\
60.9583333333333	0.23118	-187.994342553795\\
60.9583333333333	0.23284	-194.098423945795\\
60.9583333333333	0.2345	-200.202505337795\\
60.9583333333333	0.23616	-206.306586729795\\
60.9583333333333	0.23782	-212.410668121795\\
60.9583333333333	0.23948	-218.514749513795\\
60.9583333333333	0.24114	-224.618830905795\\
60.9583333333333	0.2428	-230.722912297795\\
60.9583333333333	0.24446	-236.826993689795\\
60.9583333333333	0.24612	-242.931075081794\\
60.9583333333333	0.24778	-249.035156473795\\
60.9583333333333	0.24944	-255.139237865795\\
60.9583333333333	0.2511	-261.243319257795\\
60.9583333333333	0.25276	-267.347400649795\\
60.9583333333333	0.25442	-273.451482041795\\
60.9583333333333	0.25608	-279.555563433795\\
60.9583333333333	0.25774	-285.659644825795\\
60.9583333333333	0.2594	-291.763726217795\\
60.9583333333333	0.26106	-297.867807609794\\
60.9583333333333	0.26272	-303.971889001794\\
60.9583333333333	0.26438	-310.075970393794\\
60.9583333333333	0.26604	-316.180051785794\\
60.9583333333333	0.2677	-322.284133177794\\
60.9583333333333	0.26936	-328.388214569794\\
60.9583333333333	0.27102	-334.492295961794\\
60.9583333333333	0.27268	-340.596377353794\\
60.9583333333333	0.27434	-346.700458745794\\
60.9583333333333	0.276	-352.804540137794\\
61.1666666666667	0.193	-46.4537928976745\\
61.1666666666667	0.19466	-52.501601588951\\
61.1666666666667	0.19632	-58.5494102802274\\
61.1666666666667	0.19798	-64.5972189715039\\
61.1666666666667	0.19964	-70.6450276627806\\
61.1666666666667	0.2013	-76.6928363540571\\
61.1666666666667	0.20296	-82.7406450453336\\
61.1666666666667	0.20462	-88.7884537366099\\
61.1666666666667	0.20628	-94.836262427887\\
61.1666666666667	0.20794	-100.884071119163\\
61.1666666666667	0.2096	-106.93187981044\\
61.1666666666667	0.21126	-112.979688501716\\
61.1666666666667	0.21292	-119.027497192993\\
61.1666666666667	0.21458	-125.075305884269\\
61.1666666666667	0.21624	-131.123114575546\\
61.1666666666667	0.2179	-137.170923266823\\
61.1666666666667	0.21956	-143.218731958099\\
61.1666666666667	0.22122	-149.266540649376\\
61.1666666666667	0.22288	-155.314349340652\\
61.1666666666667	0.22454	-161.362158031928\\
61.1666666666667	0.2262	-167.409966723206\\
61.1666666666667	0.22786	-173.457775414482\\
61.1666666666667	0.22952	-179.505584105758\\
61.1666666666667	0.23118	-185.553392797035\\
61.1666666666667	0.23284	-191.601201488311\\
61.1666666666667	0.2345	-197.649010179588\\
61.1666666666667	0.23616	-203.696818870864\\
61.1666666666667	0.23782	-209.744627562141\\
61.1666666666667	0.23948	-215.792436253417\\
61.1666666666667	0.24114	-221.840244944694\\
61.1666666666667	0.2428	-227.88805363597\\
61.1666666666667	0.24446	-233.935862327247\\
61.1666666666667	0.24612	-239.983671018523\\
61.1666666666667	0.24778	-246.0314797098\\
61.1666666666667	0.24944	-252.079288401077\\
61.1666666666667	0.2511	-258.127097092353\\
61.1666666666667	0.25276	-264.17490578363\\
61.1666666666667	0.25442	-270.222714474907\\
61.1666666666667	0.25608	-276.270523166183\\
61.1666666666667	0.25774	-282.31833185746\\
61.1666666666667	0.2594	-288.366140548736\\
61.1666666666667	0.26106	-294.413949240012\\
61.1666666666667	0.26272	-300.461757931289\\
61.1666666666667	0.26438	-306.509566622565\\
61.1666666666667	0.26604	-312.557375313842\\
61.1666666666667	0.2677	-318.605184005119\\
61.1666666666667	0.26936	-324.652992696395\\
61.1666666666667	0.27102	-330.700801387672\\
61.1666666666667	0.27268	-336.748610078948\\
61.1666666666667	0.27434	-342.796418770224\\
61.1666666666667	0.276	-348.844227461502\\
61.375	0.193	-45.3071152575528\\
61.375	0.19466	-51.2986512481059\\
61.375	0.19632	-57.290187238659\\
61.375	0.19798	-63.281723229212\\
61.375	0.19964	-69.2732592197651\\
61.375	0.2013	-75.2647952103187\\
61.375	0.20296	-81.2563312008717\\
61.375	0.20462	-87.2478671914246\\
61.375	0.20628	-93.2394031819777\\
61.375	0.20794	-99.2309391725307\\
61.375	0.2096	-105.222475163084\\
61.375	0.21126	-111.214011153637\\
61.375	0.21292	-117.20554714419\\
61.375	0.21458	-123.197083134743\\
61.375	0.21624	-129.188619125297\\
61.375	0.2179	-135.18015511585\\
61.375	0.21956	-141.171691106403\\
61.375	0.22122	-147.163227096956\\
61.375	0.22288	-153.154763087509\\
61.375	0.22454	-159.146299078062\\
61.375	0.2262	-165.137835068615\\
61.375	0.22786	-171.129371059169\\
61.375	0.22952	-177.120907049722\\
61.375	0.23118	-183.112443040275\\
61.375	0.23284	-189.103979030828\\
61.375	0.2345	-195.095515021381\\
61.375	0.23616	-201.087051011934\\
61.375	0.23782	-207.078587002487\\
61.375	0.23948	-213.070122993041\\
61.375	0.24114	-219.061658983594\\
61.375	0.2428	-225.053194974146\\
61.375	0.24446	-231.044730964699\\
61.375	0.24612	-237.036266955252\\
61.375	0.24778	-243.027802945805\\
61.375	0.24944	-249.019338936359\\
61.375	0.2511	-255.010874926912\\
61.375	0.25276	-261.002410917465\\
61.375	0.25442	-266.993946908018\\
61.375	0.25608	-272.985482898571\\
61.375	0.25774	-278.977018889124\\
61.375	0.2594	-284.968554879677\\
61.375	0.26106	-290.96009087023\\
61.375	0.26272	-296.951626860784\\
61.375	0.26438	-302.943162851337\\
61.375	0.26604	-308.93469884189\\
61.375	0.2677	-314.926234832443\\
61.375	0.26936	-320.917770822996\\
61.375	0.27102	-326.909306813549\\
61.375	0.27268	-332.900842804102\\
61.375	0.27434	-338.892378794656\\
61.375	0.276	-344.883914785209\\
61.5833333333333	0.193	-44.1604376174314\\
61.5833333333333	0.19466	-50.0957009072608\\
61.5833333333333	0.19632	-56.0309641970905\\
61.5833333333333	0.19798	-61.9662274869204\\
61.5833333333333	0.19964	-67.9014907767496\\
61.5833333333333	0.2013	-73.8367540665802\\
61.5833333333333	0.20296	-79.7720173564098\\
61.5833333333333	0.20462	-85.7072806462393\\
61.5833333333333	0.20628	-91.6425439360687\\
61.5833333333333	0.20794	-97.5778072258981\\
61.5833333333333	0.2096	-103.513070515728\\
61.5833333333333	0.21126	-109.448333805557\\
61.5833333333333	0.21292	-115.383597095388\\
61.5833333333333	0.21458	-121.318860385218\\
61.5833333333333	0.21624	-127.254123675047\\
61.5833333333333	0.2179	-133.189386964877\\
61.5833333333333	0.21956	-139.124650254706\\
61.5833333333333	0.22122	-145.059913544536\\
61.5833333333333	0.22288	-150.995176834366\\
61.5833333333333	0.22454	-156.930440124196\\
61.5833333333333	0.2262	-162.865703414025\\
61.5833333333333	0.22786	-168.800966703855\\
61.5833333333333	0.22952	-174.736229993684\\
61.5833333333333	0.23118	-180.671493283515\\
61.5833333333333	0.23284	-186.606756573344\\
61.5833333333333	0.2345	-192.542019863174\\
61.5833333333333	0.23616	-198.477283153004\\
61.5833333333333	0.23782	-204.412546442833\\
61.5833333333333	0.23948	-210.347809732663\\
61.5833333333333	0.24114	-216.283073022493\\
61.5833333333333	0.2428	-222.218336312323\\
61.5833333333333	0.24446	-228.153599602152\\
61.5833333333333	0.24612	-234.088862891982\\
61.5833333333333	0.24778	-240.024126181811\\
61.5833333333333	0.24944	-245.95938947164\\
61.5833333333333	0.2511	-251.89465276147\\
61.5833333333333	0.25276	-257.8299160513\\
61.5833333333333	0.25442	-263.76517934113\\
61.5833333333333	0.25608	-269.70044263096\\
61.5833333333333	0.25774	-275.635705920789\\
61.5833333333333	0.2594	-281.570969210619\\
61.5833333333333	0.26106	-287.506232500448\\
61.5833333333333	0.26272	-293.441495790279\\
61.5833333333333	0.26438	-299.376759080109\\
61.5833333333333	0.26604	-305.312022369938\\
61.5833333333333	0.2677	-311.247285659768\\
61.5833333333333	0.26936	-317.182548949598\\
61.5833333333333	0.27102	-323.117812239427\\
61.5833333333333	0.27268	-329.053075529257\\
61.5833333333333	0.27434	-334.988338819086\\
61.5833333333333	0.276	-340.923602108916\\
61.7916666666667	0.193	-43.0137599773095\\
61.7916666666667	0.19466	-48.8927505664155\\
61.7916666666667	0.19632	-54.7717411555218\\
61.7916666666667	0.19798	-60.650731744628\\
61.7916666666667	0.19964	-66.5297223337341\\
61.7916666666667	0.2013	-72.4087129228412\\
61.7916666666667	0.20296	-78.2877035119475\\
61.7916666666667	0.20462	-84.1666941010535\\
61.7916666666667	0.20628	-90.0456846901593\\
61.7916666666667	0.20794	-95.9246752792656\\
61.7916666666667	0.2096	-101.803665868372\\
61.7916666666667	0.21126	-107.682656457478\\
61.7916666666667	0.21292	-113.561647046585\\
61.7916666666667	0.21458	-119.440637635691\\
61.7916666666667	0.21624	-125.319628224797\\
61.7916666666667	0.2179	-131.198618813904\\
61.7916666666667	0.21956	-137.07760940301\\
61.7916666666667	0.22122	-142.956599992116\\
61.7916666666667	0.22288	-148.835590581223\\
61.7916666666667	0.22454	-154.714581170329\\
61.7916666666667	0.2262	-160.593571759435\\
61.7916666666667	0.22786	-166.472562348541\\
61.7916666666667	0.22952	-172.351552937647\\
61.7916666666667	0.23118	-178.230543526754\\
61.7916666666667	0.23284	-184.10953411586\\
61.7916666666667	0.2345	-189.988524704966\\
61.7916666666667	0.23616	-195.867515294073\\
61.7916666666667	0.23782	-201.746505883179\\
61.7916666666667	0.23948	-207.625496472286\\
61.7916666666667	0.24114	-213.504487061392\\
61.7916666666667	0.2428	-219.383477650498\\
61.7916666666667	0.24446	-225.262468239604\\
61.7916666666667	0.24612	-231.141458828711\\
61.7916666666667	0.24778	-237.020449417817\\
61.7916666666667	0.24944	-242.899440006923\\
61.7916666666667	0.2511	-248.778430596029\\
61.7916666666667	0.25276	-254.657421185135\\
61.7916666666667	0.25442	-260.536411774242\\
61.7916666666667	0.25608	-266.415402363348\\
61.7916666666667	0.25774	-272.294392952454\\
61.7916666666667	0.2594	-278.17338354156\\
61.7916666666667	0.26106	-284.052374130667\\
61.7916666666667	0.26272	-289.931364719774\\
61.7916666666667	0.26438	-295.81035530888\\
61.7916666666667	0.26604	-301.689345897986\\
61.7916666666667	0.2677	-307.568336487092\\
61.7916666666667	0.26936	-313.447327076199\\
61.7916666666667	0.27102	-319.326317665305\\
61.7916666666667	0.27268	-325.205308254411\\
61.7916666666667	0.27434	-331.084298843517\\
61.7916666666667	0.276	-336.963289432623\\
62	0.193	-41.8670823371879\\
62	0.19466	-47.6898002255707\\
62	0.19632	-53.5125181139533\\
62	0.19798	-59.3352360023364\\
62	0.19964	-65.1579538907192\\
62	0.2013	-70.9806717791023\\
62	0.20296	-76.8033896674851\\
62	0.20462	-82.6261075558677\\
62	0.20628	-88.4488254442508\\
62	0.20794	-94.2715433326334\\
62	0.2096	-100.094261221016\\
62	0.21126	-105.916979109399\\
62	0.21292	-111.739696997782\\
62	0.21458	-117.562414886165\\
62	0.21624	-123.385132774548\\
62	0.2179	-129.207850662931\\
62	0.21956	-135.030568551314\\
62	0.22122	-140.853286439696\\
62	0.22288	-146.676004328079\\
62	0.22454	-152.498722216462\\
62	0.2262	-158.321440104845\\
62	0.22786	-164.144157993228\\
62	0.22952	-169.966875881611\\
62	0.23118	-175.789593769994\\
62	0.23284	-181.612311658377\\
62	0.2345	-187.435029546759\\
62	0.23616	-193.257747435142\\
62	0.23782	-199.080465323525\\
62	0.23948	-204.903183211908\\
62	0.24114	-210.725901100291\\
62	0.2428	-216.548618988674\\
62	0.24446	-222.371336877056\\
62	0.24612	-228.194054765439\\
62	0.24778	-234.016772653822\\
62	0.24944	-239.839490542205\\
62	0.2511	-245.662208430588\\
62	0.25276	-251.484926318971\\
62	0.25442	-257.307644207354\\
62	0.25608	-263.130362095736\\
62	0.25774	-268.95307998412\\
62	0.2594	-274.775797872502\\
62	0.26106	-280.598515760885\\
62	0.26272	-286.421233649268\\
62	0.26438	-292.243951537651\\
62	0.26604	-298.066669426034\\
62	0.2677	-303.889387314417\\
62	0.26936	-309.7121052028\\
62	0.27102	-315.534823091182\\
62	0.27268	-321.357540979565\\
62	0.27434	-327.180258867948\\
62	0.276	-333.002976756331\\
62.2083333333333	0.193	-40.7204046970662\\
62.2083333333333	0.19466	-46.4868498847256\\
62.2083333333333	0.19632	-52.2532950723851\\
62.2083333333333	0.19798	-58.0197402600447\\
62.2083333333333	0.19964	-63.7861854477042\\
62.2083333333333	0.2013	-69.5526306353634\\
62.2083333333333	0.20296	-75.3190758230226\\
62.2083333333333	0.20462	-81.085521010682\\
62.2083333333333	0.20628	-86.8519661983419\\
62.2083333333333	0.20794	-92.6184113860013\\
62.2083333333333	0.2096	-98.3848565736607\\
62.2083333333333	0.21126	-104.15130176132\\
62.2083333333333	0.21292	-109.91774694898\\
62.2083333333333	0.21458	-115.684192136639\\
62.2083333333333	0.21624	-121.450637324299\\
62.2083333333333	0.2179	-127.217082511958\\
62.2083333333333	0.21956	-132.983527699617\\
62.2083333333333	0.22122	-138.749972887277\\
62.2083333333333	0.22288	-144.516418074936\\
62.2083333333333	0.22454	-150.282863262595\\
62.2083333333333	0.2262	-156.049308450255\\
62.2083333333333	0.22786	-161.815753637915\\
62.2083333333333	0.22952	-167.582198825574\\
62.2083333333333	0.23118	-173.348644013233\\
62.2083333333333	0.23284	-179.115089200893\\
62.2083333333333	0.2345	-184.881534388552\\
62.2083333333333	0.23616	-190.647979576212\\
62.2083333333333	0.23782	-196.414424763871\\
62.2083333333333	0.23948	-202.180869951531\\
62.2083333333333	0.24114	-207.94731513919\\
62.2083333333333	0.2428	-213.713760326849\\
62.2083333333333	0.24446	-219.480205514509\\
62.2083333333333	0.24612	-225.246650702168\\
62.2083333333333	0.24778	-231.013095889828\\
62.2083333333333	0.24944	-236.779541077487\\
62.2083333333333	0.2511	-242.545986265147\\
62.2083333333333	0.25276	-248.312431452806\\
62.2083333333333	0.25442	-254.078876640466\\
62.2083333333333	0.25608	-259.845321828126\\
62.2083333333333	0.25774	-265.611767015785\\
62.2083333333333	0.2594	-271.378212203444\\
62.2083333333333	0.26106	-277.144657391103\\
62.2083333333333	0.26272	-282.911102578763\\
62.2083333333333	0.26438	-288.677547766422\\
62.2083333333333	0.26604	-294.443992954082\\
62.2083333333333	0.2677	-300.210438141741\\
62.2083333333333	0.26936	-305.976883329401\\
62.2083333333333	0.27102	-311.74332851706\\
62.2083333333333	0.27268	-317.509773704719\\
62.2083333333333	0.27434	-323.276218892378\\
62.2083333333333	0.276	-329.042664080039\\
62.4166666666667	0.193	-39.5737270569446\\
62.4166666666667	0.19466	-45.2838995438806\\
62.4166666666667	0.19632	-50.9940720308166\\
62.4166666666667	0.19798	-56.7042445177526\\
62.4166666666667	0.19964	-62.4144170046886\\
62.4166666666667	0.2013	-68.1245894916249\\
62.4166666666667	0.20296	-73.8347619785609\\
62.4166666666667	0.20462	-79.5449344654967\\
62.4166666666667	0.20628	-85.2551069524329\\
62.4166666666667	0.20794	-90.9652794393687\\
62.4166666666667	0.2096	-96.675451926305\\
62.4166666666667	0.21126	-102.385624413241\\
62.4166666666667	0.21292	-108.095796900177\\
62.4166666666667	0.21458	-113.805969387113\\
62.4166666666667	0.21624	-119.516141874049\\
62.4166666666667	0.2179	-125.226314360985\\
62.4166666666667	0.21956	-130.936486847921\\
62.4166666666667	0.22122	-136.646659334857\\
62.4166666666667	0.22288	-142.356831821793\\
62.4166666666667	0.22454	-148.067004308729\\
62.4166666666667	0.2262	-153.777176795666\\
62.4166666666667	0.22786	-159.487349282601\\
62.4166666666667	0.22952	-165.197521769537\\
62.4166666666667	0.23118	-170.907694256473\\
62.4166666666667	0.23284	-176.617866743409\\
62.4166666666667	0.2345	-182.328039230345\\
62.4166666666667	0.23616	-188.038211717281\\
62.4166666666667	0.23782	-193.748384204217\\
62.4166666666667	0.23948	-199.458556691154\\
62.4166666666667	0.24114	-205.16872917809\\
62.4166666666667	0.2428	-210.878901665025\\
62.4166666666667	0.24446	-216.589074151962\\
62.4166666666667	0.24612	-222.299246638897\\
62.4166666666667	0.24778	-228.009419125834\\
62.4166666666667	0.24944	-233.719591612769\\
62.4166666666667	0.2511	-239.429764099706\\
62.4166666666667	0.25276	-245.139936586641\\
62.4166666666667	0.25442	-250.850109073578\\
62.4166666666667	0.25608	-256.560281560514\\
62.4166666666667	0.25774	-262.27045404745\\
62.4166666666667	0.2594	-267.980626534385\\
62.4166666666667	0.26106	-273.690799021322\\
62.4166666666667	0.26272	-279.400971508258\\
62.4166666666667	0.26438	-285.111143995194\\
62.4166666666667	0.26604	-290.82131648213\\
62.4166666666667	0.2677	-296.531488969066\\
62.4166666666667	0.26936	-302.241661456002\\
62.4166666666667	0.27102	-307.951833942938\\
62.4166666666667	0.27268	-313.662006429874\\
62.4166666666667	0.27434	-319.37217891681\\
62.4166666666667	0.276	-325.082351403746\\
62.625	0.193	-38.4270494168231\\
62.625	0.19466	-44.0809492030355\\
62.625	0.19632	-49.7348489892483\\
62.625	0.19798	-55.3887487754605\\
62.625	0.19964	-61.0426485616731\\
62.625	0.2013	-66.6965483478866\\
62.625	0.20296	-72.3504481340992\\
62.625	0.20462	-78.0043479203114\\
62.625	0.20628	-83.658247706524\\
62.625	0.20794	-89.3121474927361\\
62.625	0.2096	-94.9660472789487\\
62.625	0.21126	-100.619947065161\\
62.625	0.21292	-106.273846851375\\
62.625	0.21458	-111.927746637587\\
62.625	0.21624	-117.5816464238\\
62.625	0.2179	-123.235546210012\\
62.625	0.21956	-128.889445996224\\
62.625	0.22122	-134.543345782437\\
62.625	0.22288	-140.19724556865\\
62.625	0.22454	-145.851145354863\\
62.625	0.2262	-151.505045141075\\
62.625	0.22786	-157.158944927288\\
62.625	0.22952	-162.8128447135\\
62.625	0.23118	-168.466744499714\\
62.625	0.23284	-174.120644285926\\
62.625	0.2345	-179.774544072138\\
62.625	0.23616	-185.428443858351\\
62.625	0.23782	-191.082343644563\\
62.625	0.23948	-196.736243430777\\
62.625	0.24114	-202.390143216989\\
62.625	0.2428	-208.044043003202\\
62.625	0.24446	-213.697942789414\\
62.625	0.24612	-219.351842575627\\
62.625	0.24778	-225.005742361839\\
62.625	0.24944	-230.659642148051\\
62.625	0.2511	-236.313541934264\\
62.625	0.25276	-241.967441720476\\
62.625	0.25442	-247.62134150669\\
62.625	0.25608	-253.275241292902\\
62.625	0.25774	-258.929141079115\\
62.625	0.2594	-264.583040865327\\
62.625	0.26106	-270.236940651539\\
62.625	0.26272	-275.890840437753\\
62.625	0.26438	-281.544740223966\\
62.625	0.26604	-287.198640010178\\
62.625	0.2677	-292.852539796391\\
62.625	0.26936	-298.506439582603\\
62.625	0.27102	-304.160339368816\\
62.625	0.27268	-309.814239155028\\
62.625	0.27434	-315.468138941241\\
62.625	0.276	-321.122038727453\\
62.8333333333333	0.193	-37.2803717767015\\
62.8333333333333	0.19466	-42.8779988621907\\
62.8333333333333	0.19632	-48.4756259476796\\
62.8333333333333	0.19798	-54.0732530331688\\
62.8333333333333	0.19964	-59.670880118658\\
62.8333333333333	0.2013	-65.2685072041477\\
62.8333333333333	0.20296	-70.8661342896366\\
62.8333333333333	0.20462	-76.4637613751256\\
62.8333333333333	0.20628	-82.061388460615\\
62.8333333333333	0.20794	-87.659015546104\\
62.8333333333333	0.2096	-93.2566426315934\\
62.8333333333333	0.21126	-98.8542697170824\\
62.8333333333333	0.21292	-104.451896802572\\
62.8333333333333	0.21458	-110.049523888061\\
62.8333333333333	0.21624	-115.64715097355\\
62.8333333333333	0.2179	-121.244778059039\\
62.8333333333333	0.21956	-126.842405144529\\
62.8333333333333	0.22122	-132.440032230018\\
62.8333333333333	0.22288	-138.037659315507\\
62.8333333333333	0.22454	-143.635286400996\\
62.8333333333333	0.2262	-149.232913486485\\
62.8333333333333	0.22786	-154.830540571974\\
62.8333333333333	0.22952	-160.428167657464\\
62.8333333333333	0.23118	-166.025794742953\\
62.8333333333333	0.23284	-171.623421828442\\
62.8333333333333	0.2345	-177.221048913931\\
62.8333333333333	0.23616	-182.81867599942\\
62.8333333333333	0.23782	-188.416303084909\\
62.8333333333333	0.23948	-194.013930170399\\
62.8333333333333	0.24114	-199.611557255888\\
62.8333333333333	0.2428	-205.209184341377\\
62.8333333333333	0.24446	-210.806811426866\\
62.8333333333333	0.24612	-216.404438512355\\
62.8333333333333	0.24778	-222.002065597845\\
62.8333333333333	0.24944	-227.599692683334\\
62.8333333333333	0.2511	-233.197319768823\\
62.8333333333333	0.25276	-238.794946854312\\
62.8333333333333	0.25442	-244.392573939802\\
62.8333333333333	0.25608	-249.990201025291\\
62.8333333333333	0.25774	-255.58782811078\\
62.8333333333333	0.2594	-261.185455196269\\
62.8333333333333	0.26106	-266.783082281758\\
62.8333333333333	0.26272	-272.380709367248\\
62.8333333333333	0.26438	-277.978336452737\\
62.8333333333333	0.26604	-283.575963538226\\
62.8333333333333	0.2677	-289.173590623715\\
62.8333333333333	0.26936	-294.771217709204\\
62.8333333333333	0.27102	-300.368844794693\\
62.8333333333333	0.27268	-305.966471880182\\
62.8333333333333	0.27434	-311.564098965672\\
62.8333333333333	0.276	-317.161726051161\\
63.0416666666667	0.193	-36.1336941365796\\
63.0416666666667	0.19466	-41.6750485213454\\
63.0416666666667	0.19632	-47.2164029061109\\
63.0416666666667	0.19798	-52.757757290877\\
63.0416666666667	0.19964	-58.2991116756425\\
63.0416666666667	0.2013	-63.8404660604087\\
63.0416666666667	0.20296	-69.3818204451745\\
63.0416666666667	0.20462	-74.9231748299399\\
63.0416666666667	0.20628	-80.4645292147061\\
63.0416666666667	0.20794	-86.0058835994714\\
63.0416666666667	0.2096	-91.5472379842372\\
63.0416666666667	0.21126	-97.088592369003\\
63.0416666666667	0.21292	-102.629946753769\\
63.0416666666667	0.21458	-108.171301138535\\
63.0416666666667	0.21624	-113.712655523301\\
63.0416666666667	0.2179	-119.254009908066\\
63.0416666666667	0.21956	-124.795364292832\\
63.0416666666667	0.22122	-130.336718677598\\
63.0416666666667	0.22288	-135.878073062363\\
63.0416666666667	0.22454	-141.419427447129\\
63.0416666666667	0.2262	-146.960781831895\\
63.0416666666667	0.22786	-152.502136216661\\
63.0416666666667	0.22952	-158.043490601427\\
63.0416666666667	0.23118	-163.584844986192\\
63.0416666666667	0.23284	-169.126199370958\\
63.0416666666667	0.2345	-174.667553755724\\
63.0416666666667	0.23616	-180.20890814049\\
63.0416666666667	0.23782	-185.750262525255\\
63.0416666666667	0.23948	-191.291616910021\\
63.0416666666667	0.24114	-196.832971294787\\
63.0416666666667	0.2428	-202.374325679553\\
63.0416666666667	0.24446	-207.915680064319\\
63.0416666666667	0.24612	-213.457034449084\\
63.0416666666667	0.24778	-218.99838883385\\
63.0416666666667	0.24944	-224.539743218616\\
63.0416666666667	0.2511	-230.081097603382\\
63.0416666666667	0.25276	-235.622451988148\\
63.0416666666667	0.25442	-241.163806372913\\
63.0416666666667	0.25608	-246.705160757679\\
63.0416666666667	0.25774	-252.246515142445\\
63.0416666666667	0.2594	-257.78786952721\\
63.0416666666667	0.26106	-263.329223911976\\
63.0416666666667	0.26272	-268.870578296742\\
63.0416666666667	0.26438	-274.411932681508\\
63.0416666666667	0.26604	-279.953287066274\\
63.0416666666667	0.2677	-285.49464145104\\
63.0416666666667	0.26936	-291.035995835805\\
63.0416666666667	0.27102	-296.577350220571\\
63.0416666666667	0.27268	-302.118704605336\\
63.0416666666667	0.27434	-307.660058990102\\
63.0416666666667	0.276	-313.201413374868\\
63.25	0.193	-34.987016496458\\
63.25	0.19466	-40.4720981805003\\
63.25	0.19632	-45.9571798645422\\
63.25	0.19798	-51.4422615485853\\
63.25	0.19964	-56.9273432326277\\
63.25	0.2013	-62.4124249166698\\
63.25	0.20296	-67.8975066007119\\
63.25	0.20462	-73.3825882847541\\
63.25	0.20628	-78.8676699687971\\
63.25	0.20794	-84.3527516528393\\
63.25	0.2096	-89.8378333368819\\
63.25	0.21126	-95.322915020924\\
63.25	0.21292	-100.807996704966\\
63.25	0.21458	-106.293078389008\\
63.25	0.21624	-111.778160073051\\
63.25	0.2179	-117.263241757094\\
63.25	0.21956	-122.748323441136\\
63.25	0.22122	-128.233405125178\\
63.25	0.22288	-133.71848680922\\
63.25	0.22454	-139.203568493263\\
63.25	0.2262	-144.688650177306\\
63.25	0.22786	-150.173731861348\\
63.25	0.22952	-155.65881354539\\
63.25	0.23118	-161.143895229432\\
63.25	0.23284	-166.628976913474\\
63.25	0.2345	-172.114058597516\\
63.25	0.23616	-177.599140281559\\
63.25	0.23782	-183.084221965602\\
63.25	0.23948	-188.569303649644\\
63.25	0.24114	-194.054385333686\\
63.25	0.2428	-199.539467017728\\
63.25	0.24446	-205.024548701771\\
63.25	0.24612	-210.509630385813\\
63.25	0.24778	-215.994712069856\\
63.25	0.24944	-221.479793753898\\
63.25	0.2511	-226.96487543794\\
63.25	0.25276	-232.449957121983\\
63.25	0.25442	-237.935038806026\\
63.25	0.25608	-243.420120490068\\
63.25	0.25774	-248.90520217411\\
63.25	0.2594	-254.390283858152\\
63.25	0.26106	-259.875365542195\\
63.25	0.26272	-265.360447226237\\
63.25	0.26438	-270.845528910279\\
63.25	0.26604	-276.330610594322\\
63.25	0.2677	-281.815692278364\\
63.25	0.26936	-287.300773962406\\
63.25	0.27102	-292.785855646448\\
63.25	0.27268	-298.270937330491\\
63.25	0.27434	-303.756019014533\\
63.25	0.276	-309.241100698576\\
63.4583333333333	0.193	-33.8403388563365\\
63.4583333333333	0.19466	-39.2691478396553\\
63.4583333333333	0.19632	-44.6979568229742\\
63.4583333333333	0.19798	-50.1267658062932\\
63.4583333333333	0.19964	-55.5555747896119\\
63.4583333333333	0.2013	-60.9843837729313\\
63.4583333333333	0.20296	-66.4131927562503\\
63.4583333333333	0.20462	-71.8420017395688\\
63.4583333333333	0.20628	-77.2708107228882\\
63.4583333333333	0.20794	-82.6996197062067\\
63.4583333333333	0.2096	-88.1284286895257\\
63.4583333333333	0.21126	-93.5572376728446\\
63.4583333333333	0.21292	-98.9860466561636\\
63.4583333333333	0.21458	-104.414855639483\\
63.4583333333333	0.21624	-109.843664622801\\
63.4583333333333	0.2179	-115.27247360612\\
63.4583333333333	0.21956	-120.701282589439\\
63.4583333333333	0.22122	-126.130091572758\\
63.4583333333333	0.22288	-131.558900556077\\
63.4583333333333	0.22454	-136.987709539396\\
63.4583333333333	0.2262	-142.416518522715\\
63.4583333333333	0.22786	-147.845327506034\\
63.4583333333333	0.22952	-153.274136489353\\
63.4583333333333	0.23118	-158.702945472672\\
63.4583333333333	0.23284	-164.131754455991\\
63.4583333333333	0.2345	-169.56056343931\\
63.4583333333333	0.23616	-174.989372422629\\
63.4583333333333	0.23782	-180.418181405948\\
63.4583333333333	0.23948	-185.846990389267\\
63.4583333333333	0.24114	-191.275799372586\\
63.4583333333333	0.2428	-196.704608355904\\
63.4583333333333	0.24446	-202.133417339223\\
63.4583333333333	0.24612	-207.562226322542\\
63.4583333333333	0.24778	-212.991035305861\\
63.4583333333333	0.24944	-218.41984428918\\
63.4583333333333	0.2511	-223.848653272499\\
63.4583333333333	0.25276	-229.277462255818\\
63.4583333333333	0.25442	-234.706271239137\\
63.4583333333333	0.25608	-240.135080222456\\
63.4583333333333	0.25774	-245.563889205775\\
63.4583333333333	0.2594	-250.992698189094\\
63.4583333333333	0.26106	-256.421507172412\\
63.4583333333333	0.26272	-261.850316155732\\
63.4583333333333	0.26438	-267.279125139051\\
63.4583333333333	0.26604	-272.70793412237\\
63.4583333333333	0.2677	-278.136743105689\\
63.4583333333333	0.26936	-283.565552089008\\
63.4583333333333	0.27102	-288.994361072326\\
63.4583333333333	0.27268	-294.423170055645\\
63.4583333333333	0.27434	-299.851979038964\\
63.4583333333333	0.276	-305.280788022284\\
63.6666666666667	0.193	-32.6936612162147\\
63.6666666666667	0.19466	-38.06619749881\\
63.6666666666667	0.19632	-43.4387337814055\\
63.6666666666667	0.19798	-48.8112700640008\\
63.6666666666667	0.19964	-54.1838063465966\\
63.6666666666667	0.2013	-59.5563426291924\\
63.6666666666667	0.20296	-64.9288789117877\\
63.6666666666667	0.20462	-70.301415194383\\
63.6666666666667	0.20628	-75.6739514769788\\
63.6666666666667	0.20794	-81.0464877595741\\
63.6666666666667	0.2096	-86.4190240421694\\
63.6666666666667	0.21126	-91.7915603247652\\
63.6666666666667	0.21292	-97.164096607361\\
63.6666666666667	0.21458	-102.536632889956\\
63.6666666666667	0.21624	-107.909169172552\\
63.6666666666667	0.2179	-113.281705455147\\
63.6666666666667	0.21956	-118.654241737743\\
63.6666666666667	0.22122	-124.026778020338\\
63.6666666666667	0.22288	-129.399314302934\\
63.6666666666667	0.22454	-134.771850585529\\
63.6666666666667	0.2262	-140.144386868125\\
63.6666666666667	0.22786	-145.516923150721\\
63.6666666666667	0.22952	-150.889459433316\\
63.6666666666667	0.23118	-156.261995715912\\
63.6666666666667	0.23284	-161.634531998507\\
63.6666666666667	0.2345	-167.007068281102\\
63.6666666666667	0.23616	-172.379604563698\\
63.6666666666667	0.23782	-177.752140846294\\
63.6666666666667	0.23948	-183.124677128889\\
63.6666666666667	0.24114	-188.497213411485\\
63.6666666666667	0.2428	-193.86974969408\\
63.6666666666667	0.24446	-199.242285976676\\
63.6666666666667	0.24612	-204.614822259271\\
63.6666666666667	0.24778	-209.987358541867\\
63.6666666666667	0.24944	-215.359894824462\\
63.6666666666667	0.2511	-220.732431107058\\
63.6666666666667	0.25276	-226.104967389653\\
63.6666666666667	0.25442	-231.477503672249\\
63.6666666666667	0.25608	-236.850039954844\\
63.6666666666667	0.25774	-242.22257623744\\
63.6666666666667	0.2594	-247.595112520035\\
63.6666666666667	0.26106	-252.967648802631\\
63.6666666666667	0.26272	-258.340185085226\\
63.6666666666667	0.26438	-263.712721367822\\
63.6666666666667	0.26604	-269.085257650418\\
63.6666666666667	0.2677	-274.457793933013\\
63.6666666666667	0.26936	-279.830330215609\\
63.6666666666667	0.27102	-285.202866498204\\
63.6666666666667	0.27268	-290.575402780799\\
63.6666666666667	0.27434	-295.947939063395\\
63.6666666666667	0.276	-301.320475345991\\
63.875	0.193	-31.546983576093\\
63.875	0.19466	-36.8632471579654\\
63.875	0.19632	-42.1795107398373\\
63.875	0.19798	-47.4957743217092\\
63.875	0.19964	-52.8120379035809\\
63.875	0.2013	-58.1283014854539\\
63.875	0.20296	-63.444565067326\\
63.875	0.20462	-68.7608286491977\\
63.875	0.20628	-74.0770922310699\\
63.875	0.20794	-79.3933558129415\\
63.875	0.2096	-84.7096193948137\\
63.875	0.21126	-90.0258829766858\\
63.875	0.21292	-95.3421465585584\\
63.875	0.21458	-100.658410140431\\
63.875	0.21624	-105.974673722302\\
63.875	0.2179	-111.290937304174\\
63.875	0.21956	-116.607200886046\\
63.875	0.22122	-121.923464467918\\
63.875	0.22288	-127.239728049791\\
63.875	0.22454	-132.555991631663\\
63.875	0.2262	-137.872255213535\\
63.875	0.22786	-143.188518795407\\
63.875	0.22952	-148.504782377279\\
63.875	0.23118	-153.821045959151\\
63.875	0.23284	-159.137309541024\\
63.875	0.2345	-164.453573122896\\
63.875	0.23616	-169.769836704768\\
63.875	0.23782	-175.08610028664\\
63.875	0.23948	-180.402363868512\\
63.875	0.24114	-185.718627450384\\
63.875	0.2428	-191.034891032256\\
63.875	0.24446	-196.351154614128\\
63.875	0.24612	-201.667418196\\
63.875	0.24778	-206.983681777872\\
63.875	0.24944	-212.299945359744\\
63.875	0.2511	-217.616208941616\\
63.875	0.25276	-222.932472523488\\
63.875	0.25442	-228.24873610536\\
63.875	0.25608	-233.564999687233\\
63.875	0.25774	-238.881263269105\\
63.875	0.2594	-244.197526850977\\
63.875	0.26106	-249.513790432849\\
63.875	0.26272	-254.830054014722\\
63.875	0.26438	-260.146317596594\\
63.875	0.26604	-265.462581178466\\
63.875	0.2677	-270.778844760338\\
63.875	0.26936	-276.09510834221\\
63.875	0.27102	-281.411371924082\\
63.875	0.27268	-286.727635505954\\
63.875	0.27434	-292.043899087826\\
63.875	0.276	-297.360162669698\\
64.0833333333333	0.193	-30.4003059359716\\
64.0833333333333	0.19466	-35.6602968171203\\
64.0833333333333	0.19632	-40.920287698269\\
64.0833333333333	0.19798	-46.1802785794171\\
64.0833333333333	0.19964	-51.440269460566\\
64.0833333333333	0.2013	-56.7002603417154\\
64.0833333333333	0.20296	-61.9602512228639\\
64.0833333333333	0.20462	-67.2202421040124\\
64.0833333333333	0.20628	-72.4802329851609\\
64.0833333333333	0.20794	-77.7402238663094\\
64.0833333333333	0.2096	-83.0002147474579\\
64.0833333333333	0.21126	-88.2602056286064\\
64.0833333333333	0.21292	-93.5201965097563\\
64.0833333333333	0.21458	-98.7801873909048\\
64.0833333333333	0.21624	-104.040178272053\\
64.0833333333333	0.2179	-109.300169153202\\
64.0833333333333	0.21956	-114.56016003435\\
64.0833333333333	0.22122	-119.820150915499\\
64.0833333333333	0.22288	-125.080141796648\\
64.0833333333333	0.22454	-130.340132677797\\
64.0833333333333	0.2262	-135.600123558945\\
64.0833333333333	0.22786	-140.860114440094\\
64.0833333333333	0.22952	-146.120105321242\\
64.0833333333333	0.23118	-151.380096202392\\
64.0833333333333	0.23284	-156.64008708354\\
64.0833333333333	0.2345	-161.900077964689\\
64.0833333333333	0.23616	-167.160068845837\\
64.0833333333333	0.23782	-172.420059726986\\
64.0833333333333	0.23948	-177.680050608135\\
64.0833333333333	0.24114	-182.940041489284\\
64.0833333333333	0.2428	-188.200032370432\\
64.0833333333333	0.24446	-193.460023251581\\
64.0833333333333	0.24612	-198.720014132729\\
64.0833333333333	0.24778	-203.980005013878\\
64.0833333333333	0.24944	-209.239995895026\\
64.0833333333333	0.2511	-214.499986776175\\
64.0833333333333	0.25276	-219.759977657324\\
64.0833333333333	0.25442	-225.019968538473\\
64.0833333333333	0.25608	-230.279959419621\\
64.0833333333333	0.25774	-235.53995030077\\
64.0833333333333	0.2594	-240.799941181919\\
64.0833333333333	0.26106	-246.059932063067\\
64.0833333333333	0.26272	-251.319922944217\\
64.0833333333333	0.26438	-256.579913825365\\
64.0833333333333	0.26604	-261.839904706514\\
64.0833333333333	0.2677	-267.099895587663\\
64.0833333333333	0.26936	-272.359886468812\\
64.0833333333333	0.27102	-277.61987734996\\
64.0833333333333	0.27268	-282.879868231109\\
64.0833333333333	0.27434	-288.139859112257\\
64.0833333333333	0.276	-293.399849993406\\
64.2916666666667	0.193	-29.2536282958497\\
64.2916666666667	0.19466	-34.4573464762748\\
64.2916666666667	0.19632	-39.6610646567001\\
64.2916666666667	0.19798	-44.8647828371259\\
64.2916666666667	0.19964	-50.0685010175507\\
64.2916666666667	0.2013	-55.272219197976\\
64.2916666666667	0.20296	-60.4759373784013\\
64.2916666666667	0.20462	-65.6796555588262\\
64.2916666666667	0.20628	-70.8833737392524\\
64.2916666666667	0.20794	-76.0870919196773\\
64.2916666666667	0.2096	-81.2908101001026\\
64.2916666666667	0.21126	-86.4945282805279\\
64.2916666666667	0.21292	-91.6982464609528\\
64.2916666666667	0.21458	-96.9019646413781\\
64.2916666666667	0.21624	-102.105682821804\\
64.2916666666667	0.2179	-107.309401002229\\
64.2916666666667	0.21956	-112.513119182654\\
64.2916666666667	0.22122	-117.716837363079\\
64.2916666666667	0.22288	-122.920555543504\\
64.2916666666667	0.22454	-128.124273723929\\
64.2916666666667	0.2262	-133.327991904356\\
64.2916666666667	0.22786	-138.531710084781\\
64.2916666666667	0.22952	-143.735428265206\\
64.2916666666667	0.23118	-148.939146445631\\
64.2916666666667	0.23284	-154.142864626056\\
64.2916666666667	0.2345	-159.346582806481\\
64.2916666666667	0.23616	-164.550300986906\\
64.2916666666667	0.23782	-169.754019167332\\
64.2916666666667	0.23948	-174.957737347757\\
64.2916666666667	0.24114	-180.161455528182\\
64.2916666666667	0.2428	-185.365173708607\\
64.2916666666667	0.24446	-190.568891889033\\
64.2916666666667	0.24612	-195.772610069458\\
64.2916666666667	0.24778	-200.976328249884\\
64.2916666666667	0.24944	-206.180046430309\\
64.2916666666667	0.2511	-211.383764610734\\
64.2916666666667	0.25276	-216.587482791159\\
64.2916666666667	0.25442	-221.791200971585\\
64.2916666666667	0.25608	-226.99491915201\\
64.2916666666667	0.25774	-232.198637332435\\
64.2916666666667	0.2594	-237.402355512861\\
64.2916666666667	0.26106	-242.606073693285\\
64.2916666666667	0.26272	-247.809791873711\\
64.2916666666667	0.26438	-253.013510054136\\
64.2916666666667	0.26604	-258.217228234561\\
64.2916666666667	0.2677	-263.420946414987\\
64.2916666666667	0.26936	-268.624664595412\\
64.2916666666667	0.27102	-273.828382775837\\
64.2916666666667	0.27268	-279.032100956262\\
64.2916666666667	0.27434	-284.235819136687\\
64.2916666666667	0.276	-289.439537317113\\
64.5	0.193	-28.1069506557278\\
64.5	0.19466	-33.2543961354295\\
64.5	0.19632	-38.4018416151312\\
64.5	0.19798	-43.5492870948337\\
64.5	0.19964	-48.6967325745354\\
64.5	0.2013	-53.8441780542371\\
64.5	0.20296	-58.9916235339388\\
64.5	0.20462	-64.1390690136404\\
64.5	0.20628	-69.286514493343\\
64.5	0.20794	-74.4339599730447\\
64.5	0.2096	-79.5814054527464\\
64.5	0.21126	-84.7288509324485\\
64.5	0.21292	-89.8762964121502\\
64.5	0.21458	-95.0237418918518\\
64.5	0.21624	-100.171187371554\\
64.5	0.2179	-105.318632851256\\
64.5	0.21956	-110.466078330958\\
64.5	0.22122	-115.613523810659\\
64.5	0.22288	-120.760969290361\\
64.5	0.22454	-125.908414770063\\
64.5	0.2262	-131.055860249765\\
64.5	0.22786	-136.203305729467\\
64.5	0.22952	-141.350751209169\\
64.5	0.23118	-146.49819668887\\
64.5	0.23284	-151.645642168572\\
64.5	0.2345	-156.793087648274\\
64.5	0.23616	-161.940533127975\\
64.5	0.23782	-167.087978607678\\
64.5	0.23948	-172.23542408738\\
64.5	0.24114	-177.382869567081\\
64.5	0.2428	-182.530315046783\\
64.5	0.24446	-187.677760526485\\
64.5	0.24612	-192.825206006187\\
64.5	0.24778	-197.972651485889\\
64.5	0.24944	-203.120096965591\\
64.5	0.2511	-208.267542445293\\
64.5	0.25276	-213.414987924994\\
64.5	0.25442	-218.562433404697\\
64.5	0.25608	-223.709878884398\\
64.5	0.25774	-228.8573243641\\
64.5	0.2594	-234.004769843802\\
64.5	0.26106	-239.152215323504\\
64.5	0.26272	-244.299660803205\\
64.5	0.26438	-249.447106282907\\
64.5	0.26604	-254.594551762609\\
64.5	0.2677	-259.741997242311\\
64.5	0.26936	-264.889442722013\\
64.5	0.27102	-270.036888201715\\
64.5	0.27268	-275.184333681416\\
64.5	0.27434	-280.331779161118\\
64.5	0.276	-285.479224640821\\
64.7083333333333	0.193	-26.9602730156064\\
64.7083333333333	0.19466	-32.0514457945847\\
64.7083333333333	0.19632	-37.1426185735631\\
64.7083333333333	0.19798	-42.2337913525421\\
64.7083333333333	0.19964	-47.3249641315201\\
64.7083333333333	0.2013	-52.4161369104986\\
64.7083333333333	0.20296	-57.5073096894771\\
64.7083333333333	0.20462	-62.5984824684551\\
64.7083333333333	0.20628	-67.6896552474341\\
64.7083333333333	0.20794	-72.7808280264126\\
64.7083333333333	0.2096	-77.8720008053911\\
64.7083333333333	0.21126	-82.9631735843691\\
64.7083333333333	0.21292	-88.0543463633476\\
64.7083333333333	0.21458	-93.1455191423261\\
64.7083333333333	0.21624	-98.236691921305\\
64.7083333333333	0.2179	-103.327864700283\\
64.7083333333333	0.21956	-108.419037479262\\
64.7083333333333	0.22122	-113.51021025824\\
64.7083333333333	0.22288	-118.601383037218\\
64.7083333333333	0.22454	-123.692555816196\\
64.7083333333333	0.2262	-128.783728595175\\
64.7083333333333	0.22786	-133.874901374154\\
64.7083333333333	0.22952	-138.966074153132\\
64.7083333333333	0.23118	-144.05724693211\\
64.7083333333333	0.23284	-149.148419711089\\
64.7083333333333	0.2345	-154.239592490067\\
64.7083333333333	0.23616	-159.330765269045\\
64.7083333333333	0.23782	-164.421938048024\\
64.7083333333333	0.23948	-169.513110827002\\
64.7083333333333	0.24114	-174.604283605981\\
64.7083333333333	0.2428	-179.695456384959\\
64.7083333333333	0.24446	-184.786629163938\\
64.7083333333333	0.24612	-189.877801942916\\
64.7083333333333	0.24778	-194.968974721895\\
64.7083333333333	0.24944	-200.060147500873\\
64.7083333333333	0.2511	-205.151320279852\\
64.7083333333333	0.25276	-210.24249305883\\
64.7083333333333	0.25442	-215.333665837809\\
64.7083333333333	0.25608	-220.424838616787\\
64.7083333333333	0.25774	-225.516011395765\\
64.7083333333333	0.2594	-230.607184174744\\
64.7083333333333	0.26106	-235.698356953722\\
64.7083333333333	0.26272	-240.7895297327\\
64.7083333333333	0.26438	-245.880702511679\\
64.7083333333333	0.26604	-250.971875290657\\
64.7083333333333	0.2677	-256.063048069636\\
64.7083333333333	0.26936	-261.154220848614\\
64.7083333333333	0.27102	-266.245393627592\\
64.7083333333333	0.27268	-271.336566406571\\
64.7083333333333	0.27434	-276.427739185549\\
64.7083333333333	0.276	-281.518911964528\\
64.9166666666667	0.193	-25.813595375485\\
64.9166666666667	0.19466	-30.8484954537398\\
64.9166666666667	0.19632	-35.8833955319947\\
64.9166666666667	0.19798	-40.9182956102495\\
64.9166666666667	0.19964	-45.9531956885044\\
64.9166666666667	0.2013	-50.9880957667601\\
64.9166666666667	0.20296	-56.022995845015\\
64.9166666666667	0.20462	-61.0578959232703\\
64.9166666666667	0.20628	-66.0927960015247\\
64.9166666666667	0.20794	-71.1276960797795\\
64.9166666666667	0.2096	-76.1625961580344\\
64.9166666666667	0.21126	-81.1974962362892\\
64.9166666666667	0.21292	-86.2323963145454\\
64.9166666666667	0.21458	-91.2672963928003\\
64.9166666666667	0.21624	-96.3021964710547\\
64.9166666666667	0.2179	-101.33709654931\\
64.9166666666667	0.21956	-106.371996627565\\
64.9166666666667	0.22122	-111.40689670582\\
64.9166666666667	0.22288	-116.441796784075\\
64.9166666666667	0.22454	-121.47669686233\\
64.9166666666667	0.2262	-126.511596940585\\
64.9166666666667	0.22786	-131.54649701884\\
64.9166666666667	0.22952	-136.581397097095\\
64.9166666666667	0.23118	-141.61629717535\\
64.9166666666667	0.23284	-146.651197253605\\
64.9166666666667	0.2345	-151.68609733186\\
64.9166666666667	0.23616	-156.720997410115\\
64.9166666666667	0.23782	-161.75589748837\\
64.9166666666667	0.23948	-166.790797566625\\
64.9166666666667	0.24114	-171.825697644881\\
64.9166666666667	0.2428	-176.860597723135\\
64.9166666666667	0.24446	-181.89549780139\\
64.9166666666667	0.24612	-186.930397879645\\
64.9166666666667	0.24778	-191.9652979579\\
64.9166666666667	0.24944	-197.000198036155\\
64.9166666666667	0.2511	-202.03509811441\\
64.9166666666667	0.25276	-207.069998192665\\
64.9166666666667	0.25442	-212.10489827092\\
64.9166666666667	0.25608	-217.139798349175\\
64.9166666666667	0.25774	-222.17469842743\\
64.9166666666667	0.2594	-227.209598505685\\
64.9166666666667	0.26106	-232.24449858394\\
64.9166666666667	0.26272	-237.279398662195\\
64.9166666666667	0.26438	-242.31429874045\\
64.9166666666667	0.26604	-247.349198818706\\
64.9166666666667	0.2677	-252.38409889696\\
64.9166666666667	0.26936	-257.418998975215\\
64.9166666666667	0.27102	-262.453899053471\\
64.9166666666667	0.27268	-267.488799131726\\
64.9166666666667	0.27434	-272.52369920998\\
64.9166666666667	0.276	-277.558599288235\\
65.125	0.193	-24.6669177353631\\
65.125	0.19466	-29.6455451128945\\
65.125	0.19632	-34.6241724904262\\
65.125	0.19798	-39.6027998679574\\
65.125	0.19964	-44.5814272454891\\
65.125	0.2013	-49.5600546230212\\
65.125	0.20296	-54.5386820005529\\
65.125	0.20462	-59.5173093780845\\
65.125	0.20628	-64.4959367556157\\
65.125	0.20794	-69.4745641331469\\
65.125	0.2096	-74.4531915106786\\
65.125	0.21126	-79.4318188882098\\
65.125	0.21292	-84.4104462657424\\
65.125	0.21458	-89.3890736432741\\
65.125	0.21624	-94.3677010208053\\
65.125	0.2179	-99.3463283983365\\
65.125	0.21956	-104.324955775868\\
65.125	0.22122	-109.3035831534\\
65.125	0.22288	-114.282210530932\\
65.125	0.22454	-119.260837908463\\
65.125	0.2262	-124.239465285995\\
65.125	0.22786	-129.218092663526\\
65.125	0.22952	-134.196720041058\\
65.125	0.23118	-139.17534741859\\
65.125	0.23284	-144.153974796121\\
65.125	0.2345	-149.132602173653\\
65.125	0.23616	-154.111229551184\\
65.125	0.23782	-159.089856928716\\
65.125	0.23948	-164.068484306248\\
65.125	0.24114	-169.047111683779\\
65.125	0.2428	-174.025739061311\\
65.125	0.24446	-179.004366438843\\
65.125	0.24612	-183.982993816374\\
65.125	0.24778	-188.961621193905\\
65.125	0.24944	-193.940248571437\\
65.125	0.2511	-198.918875948968\\
65.125	0.25276	-203.8975033265\\
65.125	0.25442	-208.876130704032\\
65.125	0.25608	-213.854758081563\\
65.125	0.25774	-218.833385459095\\
65.125	0.2594	-223.812012836626\\
65.125	0.26106	-228.790640214158\\
65.125	0.26272	-233.76926759169\\
65.125	0.26438	-238.747894969222\\
65.125	0.26604	-243.726522346753\\
65.125	0.2677	-248.705149724285\\
65.125	0.26936	-253.683777101816\\
65.125	0.27102	-258.662404479348\\
65.125	0.27268	-263.64103185688\\
65.125	0.27434	-268.619659234411\\
65.125	0.276	-273.598286611943\\
65.3333333333333	0.193	-23.5202400952417\\
65.3333333333333	0.19466	-28.4425947720497\\
65.3333333333333	0.19632	-33.3649494488577\\
65.3333333333333	0.19798	-38.2873041256657\\
65.3333333333333	0.19964	-43.2096588024738\\
65.3333333333333	0.2013	-48.1320134792827\\
65.3333333333333	0.20296	-53.0543681560907\\
65.3333333333333	0.20462	-57.9767228328988\\
65.3333333333333	0.20628	-62.8990775097068\\
65.3333333333333	0.20794	-67.8214321865148\\
65.3333333333333	0.2096	-72.7437868633228\\
65.3333333333333	0.21126	-77.6661415401309\\
65.3333333333333	0.21292	-82.5884962169398\\
65.3333333333333	0.21458	-87.5108508937478\\
65.3333333333333	0.21624	-92.4332055705559\\
65.3333333333333	0.2179	-97.3555602473639\\
65.3333333333333	0.21956	-102.277914924172\\
65.3333333333333	0.22122	-107.20026960098\\
65.3333333333333	0.22288	-112.122624277789\\
65.3333333333333	0.22454	-117.044978954597\\
65.3333333333333	0.2262	-121.967333631405\\
65.3333333333333	0.22786	-126.889688308213\\
65.3333333333333	0.22952	-131.812042985021\\
65.3333333333333	0.23118	-136.73439766183\\
65.3333333333333	0.23284	-141.656752338638\\
65.3333333333333	0.2345	-146.579107015446\\
65.3333333333333	0.23616	-151.501461692254\\
65.3333333333333	0.23782	-156.423816369062\\
65.3333333333333	0.23948	-161.346171045871\\
65.3333333333333	0.24114	-166.268525722679\\
65.3333333333333	0.2428	-171.190880399487\\
65.3333333333333	0.24446	-176.113235076295\\
65.3333333333333	0.24612	-181.035589753103\\
65.3333333333333	0.24778	-185.957944429911\\
65.3333333333333	0.24944	-190.880299106719\\
65.3333333333333	0.2511	-195.802653783528\\
65.3333333333333	0.25276	-200.725008460336\\
65.3333333333333	0.25442	-205.647363137144\\
65.3333333333333	0.25608	-210.569717813952\\
65.3333333333333	0.25774	-215.49207249076\\
65.3333333333333	0.2594	-220.414427167568\\
65.3333333333333	0.26106	-225.336781844376\\
65.3333333333333	0.26272	-230.259136521185\\
65.3333333333333	0.26438	-235.181491197993\\
65.3333333333333	0.26604	-240.103845874802\\
65.3333333333333	0.2677	-245.02620055161\\
65.3333333333333	0.26936	-249.948555228418\\
65.3333333333333	0.27102	-254.870909905226\\
65.3333333333333	0.27268	-259.793264582034\\
65.3333333333333	0.27434	-264.715619258842\\
65.3333333333333	0.276	-269.63797393565\\
65.5416666666667	0.193	-22.3735624551196\\
65.5416666666667	0.19466	-27.2396444312044\\
65.5416666666667	0.19632	-32.1057264072888\\
65.5416666666667	0.19798	-36.9718083833741\\
65.5416666666667	0.19964	-41.8378903594589\\
65.5416666666667	0.2013	-46.7039723355433\\
65.5416666666667	0.20296	-51.5700543116282\\
65.5416666666667	0.20462	-56.436136287713\\
65.5416666666667	0.20628	-61.3022182637983\\
65.5416666666667	0.20794	-66.1683002398827\\
65.5416666666667	0.2096	-71.0343822159675\\
65.5416666666667	0.21126	-75.9004641920519\\
65.5416666666667	0.21292	-80.7665461681368\\
65.5416666666667	0.21458	-85.6326281442216\\
65.5416666666667	0.21624	-90.4987101203069\\
65.5416666666667	0.2179	-95.3647920963913\\
65.5416666666667	0.21956	-100.230874072476\\
65.5416666666667	0.22122	-105.096956048561\\
65.5416666666667	0.22288	-109.963038024645\\
65.5416666666667	0.22454	-114.82912000073\\
65.5416666666667	0.2262	-119.695201976815\\
65.5416666666667	0.22786	-124.5612839529\\
65.5416666666667	0.22952	-129.427365928985\\
65.5416666666667	0.23118	-134.293447905069\\
65.5416666666667	0.23284	-139.159529881154\\
65.5416666666667	0.2345	-144.025611857238\\
65.5416666666667	0.23616	-148.891693833323\\
65.5416666666667	0.23782	-153.757775809408\\
65.5416666666667	0.23948	-158.623857785493\\
65.5416666666667	0.24114	-163.489939761578\\
65.5416666666667	0.2428	-168.356021737662\\
65.5416666666667	0.24446	-173.222103713747\\
65.5416666666667	0.24612	-178.088185689831\\
65.5416666666667	0.24778	-182.954267665917\\
65.5416666666667	0.24944	-187.820349642001\\
65.5416666666667	0.2511	-192.686431618086\\
65.5416666666667	0.25276	-197.552513594171\\
65.5416666666667	0.25442	-202.418595570256\\
65.5416666666667	0.25608	-207.284677546341\\
65.5416666666667	0.25774	-212.150759522425\\
65.5416666666667	0.2594	-217.01684149851\\
65.5416666666667	0.26106	-221.882923474595\\
65.5416666666667	0.26272	-226.749005450679\\
65.5416666666667	0.26438	-231.615087426764\\
65.5416666666667	0.26604	-236.481169402849\\
65.5416666666667	0.2677	-241.347251378934\\
65.5416666666667	0.26936	-246.213333355018\\
65.5416666666667	0.27102	-251.079415331103\\
65.5416666666667	0.27268	-255.945497307188\\
65.5416666666667	0.27434	-260.811579283272\\
65.5416666666667	0.276	-265.677661259358\\
65.75	0.193	-21.2268848149979\\
65.75	0.19466	-26.0366940903591\\
65.75	0.19632	-30.8465033657203\\
65.75	0.19798	-35.656312641082\\
65.75	0.19964	-40.4661219164432\\
65.75	0.2013	-45.2759311918044\\
65.75	0.20296	-50.0857404671656\\
65.75	0.20462	-54.8955497425272\\
65.75	0.20628	-59.7053590178889\\
65.75	0.20794	-64.5151682932501\\
65.75	0.2096	-69.3249775686113\\
65.75	0.21126	-74.1347868439725\\
65.75	0.21292	-78.9445961193337\\
65.75	0.21458	-83.7544053946949\\
65.75	0.21624	-88.564214670057\\
65.75	0.2179	-93.3740239454182\\
65.75	0.21956	-98.1838332207794\\
65.75	0.22122	-102.993642496141\\
65.75	0.22288	-107.803451771502\\
65.75	0.22454	-112.613261046863\\
65.75	0.2262	-117.423070322225\\
65.75	0.22786	-122.232879597586\\
65.75	0.22952	-127.042688872948\\
65.75	0.23118	-131.852498148308\\
65.75	0.23284	-136.66230742367\\
65.75	0.2345	-141.472116699031\\
65.75	0.23616	-146.281925974392\\
65.75	0.23782	-151.091735249754\\
65.75	0.23948	-155.901544525115\\
65.75	0.24114	-160.711353800476\\
65.75	0.2428	-165.521163075838\\
65.75	0.24446	-170.330972351199\\
65.75	0.24612	-175.140781626561\\
65.75	0.24778	-179.950590901922\\
65.75	0.24944	-184.760400177283\\
65.75	0.2511	-189.570209452645\\
65.75	0.25276	-194.380018728006\\
65.75	0.25442	-199.189828003368\\
65.75	0.25608	-203.999637278729\\
65.75	0.25774	-208.80944655409\\
65.75	0.2594	-213.619255829452\\
65.75	0.26106	-218.429065104813\\
65.75	0.26272	-223.238874380174\\
65.75	0.26438	-228.048683655535\\
65.75	0.26604	-232.858492930897\\
65.75	0.2677	-237.668302206258\\
65.75	0.26936	-242.478111481619\\
65.75	0.27102	-247.28792075698\\
65.75	0.27268	-252.097730032342\\
65.75	0.27434	-256.907539307703\\
65.75	0.276	-261.717348583065\\
65.9583333333333	0.193	-20.0802071748763\\
65.9583333333333	0.19466	-24.8337437495143\\
65.9583333333333	0.19632	-29.5872803241518\\
65.9583333333333	0.19798	-34.3408168987903\\
65.9583333333333	0.19964	-39.0943534734283\\
65.9583333333333	0.2013	-43.8478900480659\\
65.9583333333333	0.20296	-48.6014266227039\\
65.9583333333333	0.20462	-53.3549631973415\\
65.9583333333333	0.20628	-58.10849977198\\
65.9583333333333	0.20794	-62.862036346618\\
65.9583333333333	0.2096	-67.615572921256\\
65.9583333333333	0.21126	-72.3691094958936\\
65.9583333333333	0.21292	-77.1226460705316\\
65.9583333333333	0.21458	-81.8761826451691\\
65.9583333333333	0.21624	-86.6297192198076\\
65.9583333333333	0.2179	-91.3832557944456\\
65.9583333333333	0.21956	-96.1367923690832\\
65.9583333333333	0.22122	-100.890328943721\\
65.9583333333333	0.22288	-105.643865518359\\
65.9583333333333	0.22454	-110.397402092996\\
65.9583333333333	0.2262	-115.150938667635\\
65.9583333333333	0.22786	-119.904475242273\\
65.9583333333333	0.22952	-124.658011816911\\
65.9583333333333	0.23118	-129.411548391548\\
65.9583333333333	0.23284	-134.165084966186\\
65.9583333333333	0.2345	-138.918621540824\\
65.9583333333333	0.23616	-143.672158115462\\
65.9583333333333	0.23782	-148.4256946901\\
65.9583333333333	0.23948	-153.179231264738\\
65.9583333333333	0.24114	-157.932767839376\\
65.9583333333333	0.2428	-162.686304414014\\
65.9583333333333	0.24446	-167.439840988652\\
65.9583333333333	0.24612	-172.19337756329\\
65.9583333333333	0.24778	-176.946914137928\\
65.9583333333333	0.24944	-181.700450712566\\
65.9583333333333	0.2511	-186.453987287204\\
65.9583333333333	0.25276	-191.207523861842\\
65.9583333333333	0.25442	-195.96106043648\\
65.9583333333333	0.25608	-200.714597011118\\
65.9583333333333	0.25774	-205.468133585756\\
65.9583333333333	0.2594	-210.221670160393\\
65.9583333333333	0.26106	-214.975206735031\\
65.9583333333333	0.26272	-219.728743309669\\
65.9583333333333	0.26438	-224.482279884307\\
65.9583333333333	0.26604	-229.235816458945\\
65.9583333333333	0.2677	-233.989353033583\\
65.9583333333333	0.26936	-238.742889608221\\
65.9583333333333	0.27102	-243.496426182859\\
65.9583333333333	0.27268	-248.249962757497\\
65.9583333333333	0.27434	-253.003499332134\\
65.9583333333333	0.276	-257.757035906773\\
66.1666666666667	0.193	-18.9335295347551\\
66.1666666666667	0.19466	-23.6307934086694\\
66.1666666666667	0.19632	-28.3280572825838\\
66.1666666666667	0.19798	-33.0253211564977\\
66.1666666666667	0.19964	-37.7225850304121\\
66.1666666666667	0.2013	-42.4198489043274\\
66.1666666666667	0.20296	-47.1171127782422\\
66.1666666666667	0.20462	-51.8143766521566\\
66.1666666666667	0.20628	-56.5116405260706\\
66.1666666666667	0.20794	-61.2089043999849\\
66.1666666666667	0.2096	-65.9061682738993\\
66.1666666666667	0.21126	-70.6034321478137\\
66.1666666666667	0.21292	-75.300696021729\\
66.1666666666667	0.21458	-79.9979598956434\\
66.1666666666667	0.21624	-84.6952237695577\\
66.1666666666667	0.2179	-89.3924876434721\\
66.1666666666667	0.21956	-94.0897515173865\\
66.1666666666667	0.22122	-98.7870153913009\\
66.1666666666667	0.22288	-103.484279265216\\
66.1666666666667	0.22454	-108.18154313913\\
66.1666666666667	0.2262	-112.878807013045\\
66.1666666666667	0.22786	-117.576070886959\\
66.1666666666667	0.22952	-122.273334760874\\
66.1666666666667	0.23118	-126.970598634789\\
66.1666666666667	0.23284	-131.667862508703\\
66.1666666666667	0.2345	-136.365126382617\\
66.1666666666667	0.23616	-141.062390256532\\
66.1666666666667	0.23782	-145.759654130446\\
66.1666666666667	0.23948	-150.456918004361\\
66.1666666666667	0.24114	-155.154181878276\\
66.1666666666667	0.2428	-159.85144575219\\
66.1666666666667	0.24446	-164.548709626104\\
66.1666666666667	0.24612	-169.245973500019\\
66.1666666666667	0.24778	-173.943237373933\\
66.1666666666667	0.24944	-178.640501247847\\
66.1666666666667	0.2511	-183.337765121762\\
66.1666666666667	0.25276	-188.035028995676\\
66.1666666666667	0.25442	-192.732292869591\\
66.1666666666667	0.25608	-197.429556743506\\
66.1666666666667	0.25774	-202.12682061742\\
66.1666666666667	0.2594	-206.824084491334\\
66.1666666666667	0.26106	-211.521348365249\\
66.1666666666667	0.26272	-216.218612239164\\
66.1666666666667	0.26438	-220.915876113078\\
66.1666666666667	0.26604	-225.613139986993\\
66.1666666666667	0.2677	-230.310403860908\\
66.1666666666667	0.26936	-235.007667734822\\
66.1666666666667	0.27102	-239.704931608736\\
66.1666666666667	0.27268	-244.402195482651\\
66.1666666666667	0.27434	-249.099459356565\\
66.1666666666667	0.276	-253.79672323048\\
66.375	0.193	-17.7868518946329\\
66.375	0.19466	-22.4278430678241\\
66.375	0.19632	-27.0688342410153\\
66.375	0.19798	-31.7098254142056\\
66.375	0.19964	-36.3508165873968\\
66.375	0.2013	-40.9918077605889\\
66.375	0.20296	-45.6327989337797\\
66.375	0.20462	-50.2737901069709\\
66.375	0.20628	-54.9147812801616\\
66.375	0.20794	-59.5557724533523\\
66.375	0.2096	-64.1967636265435\\
66.375	0.21126	-68.8377547997343\\
66.375	0.21292	-73.4787459729264\\
66.375	0.21458	-78.1197371461171\\
66.375	0.21624	-82.7607283193079\\
66.375	0.2179	-87.4017194924991\\
66.375	0.21956	-92.0427106656898\\
66.375	0.22122	-96.683701838881\\
66.375	0.22288	-101.324693012073\\
66.375	0.22454	-105.965684185263\\
66.375	0.2262	-110.606675358455\\
66.375	0.22786	-115.247666531645\\
66.375	0.22952	-119.888657704837\\
66.375	0.23118	-124.529648878028\\
66.375	0.23284	-129.170640051219\\
66.375	0.2345	-133.81163122441\\
66.375	0.23616	-138.452622397601\\
66.375	0.23782	-143.093613570792\\
66.375	0.23948	-147.734604743984\\
66.375	0.24114	-152.375595917174\\
66.375	0.2428	-157.016587090366\\
66.375	0.24446	-161.657578263557\\
66.375	0.24612	-166.298569436748\\
66.375	0.24778	-170.939560609938\\
66.375	0.24944	-175.58055178313\\
66.375	0.2511	-180.221542956321\\
66.375	0.25276	-184.862534129511\\
66.375	0.25442	-189.503525302703\\
66.375	0.25608	-194.144516475894\\
66.375	0.25774	-198.785507649085\\
66.375	0.2594	-203.426498822276\\
66.375	0.26106	-208.067489995467\\
66.375	0.26272	-212.708481168659\\
66.375	0.26438	-217.34947234185\\
66.375	0.26604	-221.990463515041\\
66.375	0.2677	-226.631454688232\\
66.375	0.26936	-231.272445861423\\
66.375	0.27102	-235.913437034614\\
66.375	0.27268	-240.554428207805\\
66.375	0.27434	-245.195419380996\\
66.375	0.276	-249.836410554187\\
66.5833333333333	0.193	-16.6401742545117\\
66.5833333333333	0.19466	-21.2248927269793\\
66.5833333333333	0.19632	-25.8096111994469\\
66.5833333333333	0.19798	-30.394329671914\\
66.5833333333333	0.19964	-34.9790481443815\\
66.5833333333333	0.2013	-39.56376661685\\
66.5833333333333	0.20296	-44.1484850893175\\
66.5833333333333	0.20462	-48.7332035617851\\
66.5833333333333	0.20628	-53.3179220342527\\
66.5833333333333	0.20794	-57.9026405067202\\
66.5833333333333	0.2096	-62.4873589791878\\
66.5833333333333	0.21126	-67.0720774516553\\
66.5833333333333	0.21292	-71.6567959241238\\
66.5833333333333	0.21458	-76.2415143965914\\
66.5833333333333	0.21624	-80.8262328690585\\
66.5833333333333	0.2179	-85.4109513415265\\
66.5833333333333	0.21956	-89.995669813994\\
66.5833333333333	0.22122	-94.5803882864616\\
66.5833333333333	0.22288	-99.1651067589296\\
66.5833333333333	0.22454	-103.749825231397\\
66.5833333333333	0.2262	-108.334543703865\\
66.5833333333333	0.22786	-112.919262176332\\
66.5833333333333	0.22952	-117.5039806488\\
66.5833333333333	0.23118	-122.088699121268\\
66.5833333333333	0.23284	-126.673417593736\\
66.5833333333333	0.2345	-131.258136066203\\
66.5833333333333	0.23616	-135.842854538671\\
66.5833333333333	0.23782	-140.427573011138\\
66.5833333333333	0.23948	-145.012291483607\\
66.5833333333333	0.24114	-149.597009956074\\
66.5833333333333	0.2428	-154.181728428542\\
66.5833333333333	0.24446	-158.766446901009\\
66.5833333333333	0.24612	-163.351165373477\\
66.5833333333333	0.24778	-167.935883845944\\
66.5833333333333	0.24944	-172.520602318412\\
66.5833333333333	0.2511	-177.105320790879\\
66.5833333333333	0.25276	-181.690039263347\\
66.5833333333333	0.25442	-186.274757735815\\
66.5833333333333	0.25608	-190.859476208283\\
66.5833333333333	0.25774	-195.44419468075\\
66.5833333333333	0.2594	-200.028913153218\\
66.5833333333333	0.26106	-204.613631625685\\
66.5833333333333	0.26272	-209.198350098154\\
66.5833333333333	0.26438	-213.783068570621\\
66.5833333333333	0.26604	-218.367787043089\\
66.5833333333333	0.2677	-222.952505515557\\
66.5833333333333	0.26936	-227.537223988024\\
66.5833333333333	0.27102	-232.121942460492\\
66.5833333333333	0.27268	-236.706660932959\\
66.5833333333333	0.27434	-241.291379405427\\
66.5833333333333	0.276	-245.876097877895\\
66.7916666666667	0.193	-15.4934966143896\\
66.7916666666667	0.19466	-20.0219423861336\\
66.7916666666667	0.19632	-24.5503881578779\\
66.7916666666667	0.19798	-29.0788339296228\\
66.7916666666667	0.19964	-33.6072797013667\\
66.7916666666667	0.2013	-38.1357254731111\\
66.7916666666667	0.20296	-42.664171244855\\
66.7916666666667	0.20462	-47.1926170165993\\
66.7916666666667	0.20628	-51.7210627883442\\
66.7916666666667	0.20794	-56.2495085600881\\
66.7916666666667	0.2096	-60.7779543318325\\
66.7916666666667	0.21126	-65.3064001035764\\
66.7916666666667	0.21292	-69.8348458753208\\
66.7916666666667	0.21458	-74.3632916470647\\
66.7916666666667	0.21624	-78.8917374188095\\
66.7916666666667	0.2179	-83.4201831905539\\
66.7916666666667	0.21956	-87.9486289622978\\
66.7916666666667	0.22122	-92.4770747340422\\
66.7916666666667	0.22288	-97.0055205057856\\
66.7916666666667	0.22454	-101.53396627753\\
66.7916666666667	0.2262	-106.062412049275\\
66.7916666666667	0.22786	-110.590857821019\\
66.7916666666667	0.22952	-115.119303592764\\
66.7916666666667	0.23118	-119.647749364507\\
66.7916666666667	0.23284	-124.176195136251\\
66.7916666666667	0.2345	-128.704640907995\\
66.7916666666667	0.23616	-133.23308667974\\
66.7916666666667	0.23782	-137.761532451485\\
66.7916666666667	0.23948	-142.289978223228\\
66.7916666666667	0.24114	-146.818423994973\\
66.7916666666667	0.2428	-151.346869766717\\
66.7916666666667	0.24446	-155.875315538461\\
66.7916666666667	0.24612	-160.403761310205\\
66.7916666666667	0.24778	-164.93220708195\\
66.7916666666667	0.24944	-169.460652853694\\
66.7916666666667	0.2511	-173.989098625439\\
66.7916666666667	0.25276	-178.517544397183\\
66.7916666666667	0.25442	-183.045990168927\\
66.7916666666667	0.25608	-187.574435940671\\
66.7916666666667	0.25774	-192.102881712416\\
66.7916666666667	0.2594	-196.63132748416\\
66.7916666666667	0.26106	-201.159773255904\\
66.7916666666667	0.26272	-205.688219027648\\
66.7916666666667	0.26438	-210.216664799392\\
66.7916666666667	0.26604	-214.745110571137\\
66.7916666666667	0.2677	-219.273556342881\\
66.7916666666667	0.26936	-223.802002114625\\
66.7916666666667	0.27102	-228.330447886369\\
66.7916666666667	0.27268	-232.858893658113\\
66.7916666666667	0.27434	-237.387339429858\\
66.7916666666667	0.276	-241.915785201602\\
67	0.193	-14.3468189742675\\
67	0.19466	-18.8189920452883\\
67	0.19632	-23.291165116309\\
67	0.19798	-27.7633381873306\\
67	0.19964	-32.2355112583514\\
67	0.2013	-36.7076843293721\\
67	0.20296	-41.1798574003928\\
67	0.20462	-45.6520304714136\\
67	0.20628	-50.1242035424348\\
67	0.20794	-54.5963766134555\\
67	0.2096	-59.0685496844762\\
67	0.21126	-63.540722755497\\
67	0.21292	-68.0128958265177\\
67	0.21458	-72.4850688975384\\
67	0.21624	-76.9572419685596\\
67	0.2179	-81.4294150395804\\
67	0.21956	-85.9015881106011\\
67	0.22122	-90.3737611816218\\
67	0.22288	-94.8459342526426\\
67	0.22454	-99.3181073236633\\
67	0.2262	-103.790280394685\\
67	0.22786	-108.262453465706\\
67	0.22952	-112.734626536726\\
67	0.23118	-117.206799607747\\
67	0.23284	-121.678972678767\\
67	0.2345	-126.151145749788\\
67	0.23616	-130.623318820809\\
67	0.23782	-135.095491891831\\
67	0.23948	-139.567664962851\\
67	0.24114	-144.039838033872\\
67	0.2428	-148.512011104893\\
67	0.24446	-152.984184175913\\
67	0.24612	-157.456357246934\\
67	0.24778	-161.928530317955\\
67	0.24944	-166.400703388976\\
67	0.2511	-170.872876459997\\
67	0.25276	-175.345049531018\\
67	0.25442	-179.817222602039\\
67	0.25608	-184.28939567306\\
67	0.25774	-188.76156874408\\
67	0.2594	-193.233741815101\\
67	0.26106	-197.705914886122\\
67	0.26272	-202.178087957142\\
67	0.26438	-206.650261028163\\
67	0.26604	-211.122434099184\\
67	0.2677	-215.594607170205\\
67	0.26936	-220.066780241226\\
67	0.27102	-224.538953312247\\
67	0.27268	-229.011126383267\\
67	0.27434	-233.483299454288\\
67	0.276	-237.95547252531\\
};
\end{axis}
\end{tikzpicture}%
	\caption{The  surfaces (\ref{tikz:theta_surf}) fitted to the internal parameter values estimated for the sets C1-C5 (\ref{tikz:thetas2}) are  used to compute the internal model parameters corresponding to the settings of choice (\ref{tikz:thetaidentified}).}\label{fig:surfaces_C610}
\end{figure}

\begin{figure}[!h]
	\centering
	\subfloat[C3]{% This file was created by matlab2tikz.
% Minimal pgfplots version: 1.3
%
\definecolor{mycolor1}{rgb}{0.00000,0.44700,0.74100}%
\definecolor{mycolor2}{rgb}{0.85000,0.32500,0.09800}%
%
\begin{tikzpicture}

\begin{axis}[%
width=11.411043cm,
height=3.8cm,
at={(0cm,0cm)},
scale only axis,
xmin=1000,
xmax=1500,
ymin=-40.283,
ymax=0,
legend style={legend cell align=left,align=left,draw=white!15!black,font=\small}
]
\addplot [color=mycolor1,solid]
  table[row sep=crcr]{%
1001	-17.09\\
1002	-19.531\\
1003	-14.648\\
1004	-14.648\\
1005	-19.531\\
1006	-18.311\\
1007	-23.193\\
1008	-18.311\\
1009	-9.766\\
1010	-15.869\\
1011	-12.207\\
1012	-14.648\\
1013	-10.986\\
1014	-3.662\\
1015	-2.441\\
1016	-4.883\\
1017	-6.104\\
1018	-14.648\\
1019	-17.09\\
1020	-17.09\\
1021	-13.428\\
1022	-9.766\\
1023	-18.311\\
1024	-14.648\\
1025	-10.986\\
1026	-13.428\\
1027	-12.207\\
1028	-10.986\\
1029	-20.752\\
1030	-19.531\\
1031	-12.207\\
1032	-20.752\\
1033	-20.752\\
1034	-15.869\\
1035	-13.428\\
1036	-9.766\\
1037	-14.648\\
1038	-14.648\\
1039	-14.648\\
1040	-14.648\\
1041	-14.648\\
1042	-15.869\\
1043	-21.973\\
1044	-20.752\\
1045	-13.428\\
1046	-13.428\\
1047	-8.545\\
1048	-12.207\\
1049	-12.207\\
1050	-13.428\\
1051	-10.986\\
1052	-13.428\\
1053	-14.648\\
1054	-13.428\\
1055	-20.752\\
1056	-13.428\\
1057	-9.766\\
1058	-7.324\\
1059	-8.545\\
1060	-10.986\\
1061	-6.104\\
1062	-9.766\\
1063	-10.986\\
1064	-6.104\\
1065	-6.104\\
1066	-9.766\\
1067	-9.766\\
1068	-15.869\\
1069	-14.648\\
1070	-17.09\\
1071	-15.869\\
1072	-18.311\\
1073	-13.428\\
1074	-14.648\\
1075	-12.207\\
1076	-10.986\\
1077	-12.207\\
1078	-23.193\\
1079	-28.076\\
1080	-30.518\\
1081	-29.297\\
1082	-18.311\\
1083	-28.076\\
1084	-32.959\\
1085	-31.738\\
1086	-20.752\\
1087	-34.18\\
1088	-40.283\\
1089	-29.297\\
1090	-24.414\\
1091	-19.531\\
1092	-15.869\\
1093	-12.207\\
1094	-14.648\\
1095	-9.766\\
1096	-7.324\\
1097	-7.324\\
1098	-10.986\\
1099	-13.428\\
1100	-12.207\\
1101	-12.207\\
1102	-15.869\\
1103	-18.311\\
1104	-19.531\\
1105	-15.869\\
1106	-21.973\\
1107	-15.869\\
1108	-15.869\\
1109	-17.09\\
1110	-12.207\\
1111	-12.207\\
1112	-15.869\\
1113	-14.648\\
1114	-8.545\\
1115	-12.207\\
1116	-8.545\\
1117	-9.766\\
1118	-9.766\\
1119	-7.324\\
1120	-9.766\\
1121	-12.207\\
1122	-17.09\\
1123	-17.09\\
1124	-17.09\\
1125	-10.986\\
1126	-12.207\\
1127	-23.193\\
1128	-15.869\\
1129	-20.752\\
1130	-21.973\\
1131	-12.207\\
1132	-9.766\\
1133	-12.207\\
1134	-13.428\\
1135	-21.973\\
1136	-25.635\\
1137	-26.855\\
1138	-19.531\\
1139	-20.752\\
1140	-18.311\\
1141	-17.09\\
1142	-18.311\\
1143	-13.428\\
1144	-12.207\\
1145	-9.766\\
1146	-9.766\\
1147	-10.986\\
1148	-15.869\\
1149	-23.193\\
1150	-17.09\\
1151	-13.428\\
1152	-10.986\\
1153	-10.986\\
1154	-7.324\\
1155	-6.104\\
1156	-6.104\\
1157	-12.207\\
1158	-7.324\\
1159	-8.545\\
1160	-8.545\\
1161	-8.545\\
1162	-7.324\\
1163	-4.883\\
1164	-3.662\\
1165	-8.545\\
1166	-15.869\\
1167	-18.311\\
1168	-20.752\\
1169	-15.869\\
1170	-10.986\\
1171	-8.545\\
1172	-4.883\\
1173	-9.766\\
1174	-8.545\\
1175	-15.869\\
1176	-17.09\\
1177	-29.297\\
1178	-32.959\\
1179	-25.635\\
1180	-25.635\\
1181	-18.311\\
1182	-23.193\\
1183	-21.973\\
1184	-24.414\\
1185	-15.869\\
1186	-17.09\\
1187	-17.09\\
1188	-15.869\\
1189	-15.869\\
1190	-13.428\\
1191	-23.193\\
1192	-25.635\\
1193	-19.531\\
1194	-13.428\\
1195	-14.648\\
1196	-23.193\\
1197	-24.414\\
1198	-28.076\\
1199	-30.518\\
1200	-29.297\\
1201	-23.193\\
1202	-21.973\\
1203	-25.635\\
1204	-28.076\\
1205	-18.311\\
1206	-12.207\\
1207	-19.531\\
1208	-14.648\\
1209	-9.766\\
1210	-13.428\\
1211	-14.648\\
1212	-10.986\\
1213	-10.986\\
1214	-10.986\\
1215	-8.545\\
1216	-10.986\\
1217	-18.311\\
1218	-17.09\\
1219	-12.207\\
1220	-12.207\\
1221	-18.311\\
1222	-26.855\\
1223	-17.09\\
1224	-14.648\\
1225	-10.986\\
1226	-13.428\\
1227	-10.986\\
1228	-8.545\\
1229	-8.545\\
1230	-10.986\\
1231	-13.428\\
1232	-14.648\\
1233	-12.207\\
1234	-15.869\\
1235	-20.752\\
1236	-17.09\\
1237	-12.207\\
1238	-15.869\\
1239	-20.752\\
1240	-13.428\\
1241	-7.324\\
1242	-13.428\\
1243	-10.986\\
1244	-12.207\\
1245	-9.766\\
1246	-8.545\\
1247	-13.428\\
1248	-13.428\\
1249	-9.766\\
1250	-12.207\\
1251	-9.766\\
1252	-7.324\\
1253	-12.207\\
1254	-8.545\\
1255	-9.766\\
1256	-7.324\\
1257	-4.883\\
1258	-12.207\\
1259	-15.869\\
1260	-17.09\\
1261	-23.193\\
1262	-15.869\\
1263	-9.766\\
1264	-8.545\\
1265	-8.545\\
1266	-10.986\\
1267	-8.545\\
1268	-4.883\\
1269	-10.986\\
1270	-15.869\\
1271	-13.428\\
1272	-20.752\\
1273	-15.869\\
1274	-14.648\\
1275	-8.545\\
1276	-10.986\\
1277	-4.883\\
1278	-3.662\\
1279	-4.883\\
1280	-9.766\\
1281	-10.986\\
1282	-12.207\\
1283	-13.428\\
1284	-20.752\\
1285	-17.09\\
1286	-14.648\\
1287	-17.09\\
1288	-19.531\\
1289	-20.752\\
1290	-17.09\\
1291	-13.428\\
1292	-7.324\\
1293	-6.104\\
1294	-4.883\\
1295	-7.324\\
1296	-7.324\\
1297	-4.883\\
1298	-8.545\\
1299	-12.207\\
1300	-10.986\\
1301	-12.207\\
1302	-12.207\\
1303	-8.545\\
1304	-4.883\\
1305	-9.766\\
1306	-15.869\\
1307	-13.428\\
1308	-15.869\\
1309	-13.428\\
1310	-9.766\\
1311	-14.648\\
1312	-18.311\\
1313	-18.311\\
1314	-13.428\\
1315	-20.752\\
1316	-15.869\\
1317	-17.09\\
1318	-18.311\\
1319	-19.531\\
1320	-13.428\\
1321	-13.428\\
1322	-18.311\\
1323	-26.855\\
1324	-21.973\\
1325	-13.428\\
1326	-13.428\\
1327	-13.428\\
1328	-15.869\\
1329	-17.09\\
1330	-12.207\\
1331	-14.648\\
1332	-18.311\\
1333	-20.752\\
1334	-13.428\\
1335	-12.207\\
1336	-15.869\\
1337	-23.193\\
1338	-20.752\\
1339	-21.973\\
1340	-14.648\\
1341	-14.648\\
1342	-12.207\\
1343	-9.766\\
1344	-4.883\\
1345	-3.662\\
1346	-3.662\\
1347	-7.324\\
1348	-13.428\\
1349	-14.648\\
1350	-9.766\\
1351	-12.207\\
1352	-13.428\\
1353	-9.766\\
1354	-8.545\\
1355	-8.545\\
1356	-6.104\\
1357	-9.766\\
1358	-14.648\\
1359	-15.869\\
1360	-10.986\\
1361	-8.545\\
1362	-8.545\\
1363	-8.545\\
1364	-7.324\\
1365	-10.986\\
1366	-20.752\\
1367	-18.311\\
1368	-18.311\\
1369	-24.414\\
1370	-23.193\\
1371	-18.311\\
1372	-14.648\\
1373	-14.648\\
1374	-14.648\\
1375	-17.09\\
1376	-19.531\\
1377	-19.531\\
1378	-25.635\\
1379	-26.855\\
1380	-31.738\\
1381	-24.414\\
1382	-25.635\\
1383	-26.855\\
1384	-21.973\\
1385	-24.414\\
1386	-20.752\\
1387	-13.428\\
1388	-9.766\\
1389	-9.766\\
1390	-12.207\\
1391	-9.766\\
1392	-7.324\\
1393	-10.986\\
1394	-8.545\\
1395	-6.104\\
1396	-7.324\\
1397	-9.766\\
1398	-6.104\\
1399	-12.207\\
1400	-14.648\\
1401	-10.986\\
1402	-17.09\\
1403	-24.414\\
1404	-24.414\\
1405	-28.076\\
1406	-23.193\\
1407	-21.973\\
1408	-17.09\\
1409	-9.766\\
1410	-10.986\\
1411	-10.986\\
1412	-7.324\\
1413	-10.986\\
1414	-9.766\\
1415	-2.441\\
1416	-6.104\\
1417	-9.766\\
1418	-12.207\\
1419	-14.648\\
1420	-17.09\\
1421	-14.648\\
1422	-17.09\\
1423	-14.648\\
1424	-13.428\\
1425	-12.207\\
1426	-10.986\\
1427	-14.648\\
1428	-13.428\\
1429	-10.986\\
1430	-13.428\\
1431	-18.311\\
1432	-13.428\\
1433	-10.986\\
1434	-10.986\\
1435	-10.986\\
1436	-8.545\\
1437	-7.324\\
1438	-10.986\\
1439	-13.428\\
1440	-13.428\\
1441	-10.986\\
1442	-13.428\\
1443	-13.428\\
1444	-8.545\\
1445	-7.324\\
1446	-15.869\\
1447	-19.531\\
1448	-18.311\\
1449	-18.311\\
1450	-15.869\\
1451	-13.428\\
1452	-10.986\\
1453	-9.766\\
1454	-13.428\\
1455	-13.428\\
1456	-8.545\\
1457	-12.207\\
1458	-14.648\\
1459	-14.648\\
1460	-17.09\\
1461	-17.09\\
1462	-20.752\\
1463	-28.076\\
1464	-30.518\\
1465	-20.752\\
1466	-13.428\\
1467	-8.545\\
1468	-6.104\\
1469	-8.545\\
1470	-7.324\\
1471	-10.986\\
1472	-12.207\\
1473	-12.207\\
1474	-13.428\\
1475	-15.869\\
1476	-19.531\\
1477	-19.531\\
1478	-28.076\\
1479	-23.193\\
1480	-18.311\\
1481	-14.648\\
1482	-14.648\\
1483	-14.648\\
1484	-17.09\\
1485	-14.648\\
1486	-12.207\\
1487	-13.428\\
1488	-8.545\\
1489	-7.324\\
1490	-17.09\\
1491	-14.648\\
1492	-12.207\\
1493	-23.193\\
1494	-18.311\\
1495	-15.869\\
1496	-21.973\\
1497	-23.193\\
1498	-14.648\\
1499	-8.545\\
1500	-9.766\\
};
\addlegendentry{True output};

\addplot [color=mycolor2,dashed]
  table[row sep=crcr]{%
1001	-15.6213866308512\\
1002	-18.9226907743792\\
1003	-17.7306564641883\\
1004	-16.1248413489558\\
1005	-19.8892543622883\\
1006	-20.0988173229198\\
1007	-21.2824284526109\\
1008	-18.1818313873644\\
1009	-11.3736034017043\\
1010	-12.2448466380569\\
1011	-13.3190169208678\\
1012	-15.0616757186513\\
1013	-13.7760468447006\\
1014	-7.23031075294326\\
1015	-4.90980104807397\\
1016	-4.75272944104692\\
1017	-9.99858538731541\\
1018	-18.0281888482618\\
1019	-17.5243612126982\\
1020	-16.7459595678504\\
1021	-12.8872704037604\\
1022	-8.86267067819354\\
1023	-15.6768588155112\\
1024	-13.2391904298932\\
1025	-10.1775642086538\\
1026	-13.7277813501494\\
1027	-14.5792762874875\\
1028	-11.9343432911495\\
1029	-16.856672519785\\
1030	-17.0433181084214\\
1031	-13.4241337398592\\
1032	-17.3901554716475\\
1033	-18.053542246585\\
1034	-14.8642842644229\\
1035	-13.0216362755021\\
1036	-11.7283007094428\\
1037	-12.6170129658387\\
1038	-15.1281843651973\\
1039	-15.0205309680204\\
1040	-14.1902674824356\\
1041	-14.6802668383256\\
1042	-15.494839227207\\
1043	-19.9286934462234\\
1044	-19.3865757367961\\
1045	-13.9859706287427\\
1046	-12.7715039010784\\
1047	-9.10252058227124\\
1048	-8.59837191742753\\
1049	-11.4025201816866\\
1050	-10.6670427116973\\
1051	-10.3786364861359\\
1052	-12.7168281799714\\
1053	-14.2377429862503\\
1054	-13.3982331999723\\
1055	-18.3508529817852\\
1056	-16.0621151269731\\
1057	-9.82510804460818\\
1058	-8.43417949915416\\
1059	-10.7483065873254\\
1060	-12.5193854910545\\
1061	-9.04023546088802\\
1062	-7.53584826651679\\
1063	-10.0536050744307\\
1064	-9.73278526820674\\
1065	-8.24726841722031\\
1066	-9.33796341666224\\
1067	-10.618512631069\\
1068	-14.4707332971501\\
1069	-14.7047046205133\\
1070	-15.4770093991787\\
1071	-14.5533444191285\\
1072	-16.2243661565006\\
1073	-14.620398814154\\
1074	-13.3623547680762\\
1075	-12.8893567549469\\
1076	-12.8811837849832\\
1077	-12.7212501694853\\
1078	-18.7052088497742\\
1079	-27.0695351501065\\
1080	-27.0059657797637\\
1081	-25.3134208097211\\
1082	-17.0500538592601\\
1083	-22.9860486699165\\
1084	-28.330093079408\\
1085	-28.3113933966475\\
1086	-23.5471328297674\\
1087	-28.0021510335615\\
1088	-35.4427593824786\\
1089	-29.8562098955251\\
1090	-24.9185747709467\\
1091	-18.6263386933145\\
1092	-15.69692628829\\
1093	-14.444930647167\\
1094	-16.1885344898367\\
1095	-14.6509252815562\\
1096	-9.84780392901648\\
1097	-8.61075144058132\\
1098	-10.6133134906741\\
1099	-14.5936638374606\\
1100	-14.4188449156918\\
1101	-13.7547467603006\\
1102	-15.8335134343058\\
1103	-16.5483540046346\\
1104	-21.9435002870484\\
1105	-19.6755874222658\\
1106	-20.3384209355202\\
1107	-17.4803134896188\\
1108	-14.9515955052577\\
1109	-17.6401611904945\\
1110	-14.8875872355714\\
1111	-12.8271315006999\\
1112	-15.2227780609688\\
1113	-15.5962853212595\\
1114	-13.1410967553011\\
1115	-12.4069929422671\\
1116	-10.9531286482946\\
1117	-10.5912876097008\\
1118	-12.2802605935941\\
1119	-9.36046560680082\\
1120	-9.95410173855537\\
1121	-12.4407438109364\\
1122	-16.7659878485675\\
1123	-16.9382199293043\\
1124	-16.7756802601101\\
1125	-12.4139415049622\\
1126	-15.1487194751417\\
1127	-22.4842234156016\\
1128	-18.2019910072049\\
1129	-19.2070664202649\\
1130	-19.255331189183\\
1131	-13.8435861026251\\
1132	-10.4682837871594\\
1133	-12.616231408749\\
1134	-14.8911952323473\\
1135	-21.338064971096\\
1136	-25.5977400182952\\
1137	-23.9102479712986\\
1138	-19.6644710385396\\
1139	-19.1675431045858\\
1140	-19.1571085864167\\
1141	-18.3561990409063\\
1142	-18.2520911504805\\
1143	-14.3391732737\\
1144	-12.3999744915984\\
1145	-12.3841861979205\\
1146	-11.9240012018464\\
1147	-12.5146748545714\\
1148	-16.3509369445295\\
1149	-22.2962320091277\\
1150	-19.5865202886395\\
1151	-13.729237424162\\
1152	-11.8396491363613\\
1153	-12.2988123344155\\
1154	-8.92863938444588\\
1155	-6.89343148293733\\
1156	-7.49107396743029\\
1157	-11.633406572876\\
1158	-10.6537920531546\\
1159	-9.97238292701811\\
1160	-10.5791780608721\\
1161	-10.1429851435348\\
1162	-8.63010462626863\\
1163	-6.56403720747054\\
1164	-5.71341948362295\\
1165	-7.20863706808424\\
1166	-13.6451559385508\\
1167	-19.2980159985127\\
1168	-20.0666048547843\\
1169	-13.7232132458244\\
1170	-10.0362785716691\\
1171	-7.96119213100671\\
1172	-6.19989321250652\\
1173	-10.7565760177081\\
1174	-11.869692756574\\
1175	-14.3758780649479\\
1176	-17.675472103296\\
1177	-26.8537178789162\\
1178	-31.3089377803998\\
1179	-25.6761224225149\\
1180	-24.5175037011525\\
1181	-19.4426273235149\\
1182	-19.359926151872\\
1183	-21.5243273191282\\
1184	-22.663217576552\\
1185	-19.1392150228086\\
1186	-17.7059564612251\\
1187	-18.0656202137026\\
1188	-16.8518286730534\\
1189	-19.1598720198262\\
1190	-16.0478396266023\\
1191	-20.7977007876908\\
1192	-25.2274452977742\\
1193	-19.4047104302792\\
1194	-13.1535992448782\\
1195	-13.8477875888819\\
1196	-20.8899878387252\\
1197	-24.7660068570266\\
1198	-27.8498726706433\\
1199	-28.94736503778\\
1200	-28.6025227253854\\
1201	-23.4798940932225\\
1202	-22.0556808606695\\
1203	-24.5314533624123\\
1204	-27.1278435645515\\
1205	-20.1873796919526\\
1206	-13.4916882749729\\
1207	-17.4230004957317\\
1208	-16.7857259093499\\
1209	-12.0109574352025\\
1210	-14.0445523225441\\
1211	-16.3206400541892\\
1212	-13.9787822305807\\
1213	-11.3324244486637\\
1214	-13.1208012907519\\
1215	-12.8129676056499\\
1216	-14.7338933919404\\
1217	-19.2360995190519\\
1218	-17.4483480491695\\
1219	-13.5456457859631\\
1220	-13.0020421186739\\
1221	-18.1606397685547\\
1222	-27.2274244687358\\
1223	-22.6665136574251\\
1224	-14.6337966459155\\
1225	-12.2077719886288\\
1226	-13.1251636906158\\
1227	-13.9735733205055\\
1228	-11.2652476644407\\
1229	-11.7060858411892\\
1230	-12.9126909660492\\
1231	-15.5208517089666\\
1232	-16.5509823940162\\
1233	-12.1093770669256\\
1234	-15.591563867371\\
1235	-19.0887474203805\\
1236	-18.216961513668\\
1237	-15.4994450929366\\
1238	-15.9507278800614\\
1239	-20.5048251423267\\
1240	-15.410786431233\\
1241	-7.98423316118386\\
1242	-9.09841958175882\\
1243	-11.3829903921888\\
1244	-14.8198415346373\\
1245	-13.76076832206\\
1246	-10.9723412455419\\
1247	-13.9757127672483\\
1248	-14.3691581942328\\
1249	-11.4773349572636\\
1250	-12.7696457582252\\
1251	-12.3523525005286\\
1252	-11.6814657384975\\
1253	-12.322242304768\\
1254	-9.42588415962198\\
1255	-9.46041907206408\\
1256	-8.42353160572685\\
1257	-8.67119359243628\\
1258	-13.1544063962765\\
1259	-14.9621206404998\\
1260	-17.865483669245\\
1261	-22.6209434753499\\
1262	-16.5928402942029\\
1263	-10.5274610964636\\
1264	-8.59522823965741\\
1265	-8.86312773447735\\
1266	-11.9464201273617\\
1267	-9.31130142719068\\
1268	-9.06036066596336\\
1269	-11.0529363489856\\
1270	-16.1711494245468\\
1271	-16.2116803575953\\
1272	-18.5328573484796\\
1273	-14.5069166680833\\
1274	-12.6192522404578\\
1275	-8.49749070304824\\
1276	-7.75769772576135\\
1277	-6.2158107642333\\
1278	-6.26165494077431\\
1279	-7.52189070411335\\
1280	-12.1418170559522\\
1281	-12.3272013967668\\
1282	-13.1704930487617\\
1283	-13.4021688141536\\
1284	-19.5149685230983\\
1285	-19.228157429386\\
1286	-13.896007532526\\
1287	-15.6317794147275\\
1288	-16.5779472035904\\
1289	-20.7427443119152\\
1290	-18.0299613692076\\
1291	-13.0217858258921\\
1292	-8.13034235656072\\
1293	-6.09772949480035\\
1294	-6.67019012349659\\
1295	-7.84854281478225\\
1296	-8.17520229953675\\
1297	-7.96223504139015\\
1298	-9.01326092227061\\
1299	-12.3873656812783\\
1300	-12.3710059763143\\
1301	-11.548188534054\\
1302	-12.7656491950152\\
1303	-9.46676055001622\\
1304	-6.10043476720269\\
1305	-9.39913205662793\\
1306	-13.8204145832819\\
1307	-14.1054157214721\\
1308	-13.8183647448611\\
1309	-12.7467181545046\\
1310	-10.4282590469344\\
1311	-12.5936635372292\\
1312	-16.9352378750382\\
1313	-17.432632653495\\
1314	-13.0551789879034\\
1315	-19.2117427939298\\
1316	-17.0765210475517\\
1317	-15.3858175881533\\
1318	-17.7690962442865\\
1319	-17.2454624712406\\
1320	-14.6304466550607\\
1321	-12.9724255210198\\
1322	-17.8595026907498\\
1323	-25.7435098526211\\
1324	-20.5815607116016\\
1325	-12.9791441109451\\
1326	-12.2227540927244\\
1327	-14.1457187026535\\
1328	-16.4077143836101\\
1329	-18.4533103514842\\
1330	-14.9271519358609\\
1331	-15.4028291657669\\
1332	-18.6700864494616\\
1333	-19.964231397069\\
1334	-15.278141642421\\
1335	-12.7891340396186\\
1336	-15.401961332287\\
1337	-23.2087144248931\\
1338	-21.1177507134956\\
1339	-20.2795284519899\\
1340	-14.0757942354711\\
1341	-12.4977572069838\\
1342	-13.4579579684601\\
1343	-9.35150715311656\\
1344	-7.23965945125137\\
1345	-5.92499466822323\\
1346	-5.51424729655416\\
1347	-9.04793631523814\\
1348	-13.098394992874\\
1349	-14.7153926354083\\
1350	-11.4736581983076\\
1351	-11.2976850544595\\
1352	-12.4695553293072\\
1353	-9.71767388956623\\
1354	-9.22065115033983\\
1355	-9.70057860034292\\
1356	-8.46529188702432\\
1357	-9.10801969592774\\
1358	-12.6984856928942\\
1359	-15.972238858232\\
1360	-11.4429215440562\\
1361	-8.82877099548872\\
1362	-8.10384272130869\\
1363	-9.47366561427586\\
1364	-7.76549149747697\\
1365	-13.1234822464872\\
1366	-20.7813268235037\\
1367	-16.8731067501891\\
1368	-19.6398606650433\\
1369	-23.4118265361861\\
1370	-22.5480418319051\\
1371	-18.6485521660634\\
1372	-14.4972352820393\\
1373	-14.6392001672205\\
1374	-15.5592916268862\\
1375	-17.3849427463311\\
1376	-19.1122837064827\\
1377	-19.4194687045023\\
1378	-22.4133385261958\\
1379	-25.5526975044879\\
1380	-29.2951412693545\\
1381	-22.8148219987133\\
1382	-22.3694343643773\\
1383	-26.3087900490259\\
1384	-22.6425522570721\\
1385	-23.3742206032378\\
1386	-22.1617410424119\\
1387	-13.5967565638778\\
1388	-9.95022496725979\\
1389	-10.7940338007407\\
1390	-15.1366792333744\\
1391	-13.7068910175749\\
1392	-10.1586574287926\\
1393	-11.815546592817\\
1394	-10.1888864302823\\
1395	-8.17850230426176\\
1396	-9.22229683586261\\
1397	-10.1810612937753\\
1398	-8.64080738185653\\
1399	-11.8539169963605\\
1400	-15.7013776355268\\
1401	-12.8071586507225\\
1402	-16.635740755502\\
1403	-24.1692391852916\\
1404	-22.6085934307069\\
1405	-25.6669554607131\\
1406	-20.4590148719265\\
1407	-19.9485144541588\\
1408	-16.4423535969898\\
1409	-11.084322609779\\
1410	-13.319079816423\\
1411	-11.877309366679\\
1412	-9.65158987516324\\
1413	-11.7681627484458\\
1414	-8.53218035299907\\
1415	-6.07419435366859\\
1416	-6.6941519050977\\
1417	-10.6879092954257\\
1418	-14.944146704929\\
1419	-16.5851498904621\\
1420	-16.1321495718634\\
1421	-15.1087792893587\\
1422	-16.3009668341976\\
1423	-14.8122173851085\\
1424	-12.727824810183\\
1425	-13.4097972872901\\
1426	-11.9463141258906\\
1427	-15.6793573734047\\
1428	-12.6559202779009\\
1429	-10.3066000188834\\
1430	-13.5752198992549\\
1431	-18.4834843827229\\
1432	-15.6731502294313\\
1433	-11.8407165151894\\
1434	-10.7152981029094\\
1435	-12.0354818116499\\
1436	-9.714158283891\\
1437	-8.75378016171661\\
1438	-11.1089146838316\\
1439	-12.8512168922195\\
1440	-14.289740567273\\
1441	-12.1517674925831\\
1442	-13.4422176084617\\
1443	-13.5353771881279\\
1444	-9.56305465574163\\
1445	-9.06953137421049\\
1446	-14.4488659239069\\
1447	-20.2496164701768\\
1448	-19.1818695208778\\
1449	-16.2568358611493\\
1450	-15.8214850115897\\
1451	-13.6677963460482\\
1452	-13.1022128218605\\
1453	-11.6015212698359\\
1454	-14.1529503660841\\
1455	-13.8869669948353\\
1456	-9.88485908852515\\
1457	-12.3756154443445\\
1458	-14.02262944968\\
1459	-16.3142076090992\\
1460	-16.6848977924679\\
1461	-16.3930608584241\\
1462	-20.9420598332632\\
1463	-25.3778106179864\\
1464	-28.2777900051375\\
1465	-20.5397150696481\\
1466	-12.8684335692276\\
1467	-9.48910278517388\\
1468	-7.12808457980855\\
1469	-9.32382120331183\\
1470	-10.3453168585071\\
1471	-14.3303617097567\\
1472	-13.9715372526342\\
1473	-11.6607992673428\\
1474	-13.9473134311294\\
1475	-15.7456474612117\\
1476	-17.3594672353493\\
1477	-18.3674348439599\\
1478	-25.8676085184465\\
1479	-23.9030502962153\\
1480	-19.1132445593847\\
1481	-14.4826027314458\\
1482	-13.8875914293882\\
1483	-15.6389674196055\\
1484	-17.935322873781\\
1485	-14.8141094655102\\
1486	-13.9504929835041\\
1487	-14.0614905919954\\
1488	-9.57900150247952\\
1489	-12.4828713593455\\
1490	-17.6287088827132\\
1491	-14.1912452311247\\
1492	-13.756460381748\\
1493	-21.7888656095558\\
1494	-18.6346534458374\\
1495	-16.5971207155494\\
1496	-21.4398789009632\\
1497	-21.2181984423236\\
1498	-15.5778507746306\\
1499	-10.0178514793398\\
1500	-10.9987223226482\\
};
\addlegendentry{Generated output};

\end{axis}
\end{tikzpicture}%\label{fig:c3c15}}\\
	\subfloat[C8]{% This file was created by matlab2tikz.
% Minimal pgfplots version: 1.3
%
\definecolor{mycolor1}{rgb}{0.00000,0.44700,0.74100}%
\definecolor{mycolor2}{rgb}{0.85000,0.32500,0.09800}%
%
\begin{tikzpicture}

\begin{axis}[%
width=11.411043cm,
height=3.8cm,
at={(0cm,0cm)},
scale only axis,
xmin=1000,
xmax=1500,
ymin=-300,
ymax=0,
legend style={legend cell align=left,align=left,draw=white!15!black,font=\small}
]
\addplot [color=mycolor1,solid]
  table[row sep=crcr]{%
1001	-96.436\\
1002	-122.07\\
1003	-100.098\\
1004	-93.994\\
1005	-130.615\\
1006	-125.732\\
1007	-153.809\\
1008	-115.967\\
1009	-61.035\\
1010	-74.463\\
1011	-73.242\\
1012	-86.67\\
1013	-73.242\\
1014	-34.18\\
1015	-20.752\\
1016	-23.193\\
1017	-58.594\\
1018	-100.098\\
1019	-109.863\\
1020	-114.746\\
1021	-83.008\\
1022	-52.49\\
1023	-104.98\\
1024	-81.787\\
1025	-59.814\\
1026	-97.656\\
1027	-83.008\\
1028	-72.021\\
1029	-118.408\\
1030	-107.422\\
1031	-83.008\\
1032	-119.629\\
1033	-123.291\\
1034	-95.215\\
1035	-80.566\\
1036	-62.256\\
1037	-75.684\\
1038	-95.215\\
1039	-87.891\\
1040	-91.553\\
1041	-96.436\\
1042	-102.539\\
1043	-141.602\\
1044	-128.174\\
1045	-85.449\\
1046	-79.346\\
1047	-47.607\\
1048	-48.828\\
1049	-68.359\\
1050	-52.49\\
1051	-57.373\\
1052	-75.684\\
1053	-89.111\\
1054	-81.787\\
1055	-128.174\\
1056	-98.877\\
1057	-57.373\\
1058	-45.166\\
1059	-62.256\\
1060	-72.021\\
1061	-40.283\\
1062	-42.725\\
1063	-56.152\\
1064	-46.387\\
1065	-40.283\\
1066	-52.49\\
1067	-62.256\\
1068	-92.773\\
1069	-90.332\\
1070	-107.422\\
1071	-98.877\\
1072	-115.967\\
1073	-91.553\\
1074	-91.553\\
1075	-79.346\\
1076	-75.684\\
1077	-76.904\\
1078	-133.057\\
1079	-189.209\\
1080	-194.092\\
1081	-189.209\\
1082	-119.629\\
1083	-170.898\\
1084	-213.623\\
1085	-219.727\\
1086	-169.678\\
1087	-219.727\\
1088	-279.541\\
1089	-197.754\\
1090	-161.133\\
1091	-109.863\\
1092	-85.449\\
1093	-73.242\\
1094	-91.553\\
1095	-62.256\\
1096	-46.387\\
1097	-42.725\\
1098	-54.932\\
1099	-79.346\\
1100	-74.463\\
1101	-75.684\\
1102	-100.098\\
1103	-103.76\\
1104	-141.602\\
1105	-114.746\\
1106	-142.822\\
1107	-104.98\\
1108	-90.332\\
1109	-103.76\\
1110	-76.904\\
1111	-69.58\\
1112	-84.229\\
1113	-86.67\\
1114	-59.814\\
1115	-64.697\\
1116	-48.828\\
1117	-53.711\\
1118	-57.373\\
1119	-41.504\\
1120	-52.49\\
1121	-65.918\\
1122	-101.318\\
1123	-104.98\\
1124	-114.746\\
1125	-68.359\\
1126	-93.994\\
1127	-150.146\\
1128	-114.746\\
1129	-125.732\\
1130	-128.174\\
1131	-80.566\\
1132	-53.711\\
1133	-75.684\\
1134	-85.449\\
1135	-137.939\\
1136	-172.119\\
1137	-170.898\\
1138	-123.291\\
1139	-130.615\\
1140	-115.967\\
1141	-113.525\\
1142	-111.084\\
1143	-76.904\\
1144	-68.359\\
1145	-61.035\\
1146	-57.373\\
1147	-62.256\\
1148	-91.553\\
1149	-140.381\\
1150	-111.084\\
1151	-79.346\\
1152	-64.697\\
1153	-62.256\\
1154	-40.283\\
1155	-29.297\\
1156	-36.621\\
1157	-63.477\\
1158	-51.27\\
1159	-52.49\\
1160	-58.594\\
1161	-54.932\\
1162	-37.842\\
1163	-28.076\\
1164	-23.193\\
1165	-39.063\\
1166	-83.008\\
1167	-117.188\\
1168	-130.615\\
1169	-86.67\\
1170	-58.594\\
1171	-45.166\\
1172	-32.959\\
1173	-67.139\\
1174	-65.918\\
1175	-91.553\\
1176	-117.188\\
1177	-186.768\\
1178	-216.064\\
1179	-177.002\\
1180	-168.457\\
1181	-109.863\\
1182	-122.07\\
1183	-131.836\\
1184	-146.484\\
1185	-108.643\\
1186	-100.098\\
1187	-98.877\\
1188	-89.111\\
1189	-106.201\\
1190	-78.125\\
1191	-130.615\\
1192	-159.912\\
1193	-113.525\\
1194	-70.801\\
1195	-73.242\\
1196	-123.291\\
1197	-153.809\\
1198	-189.209\\
1199	-202.637\\
1200	-202.637\\
1201	-153.809\\
1202	-137.939\\
1203	-158.691\\
1204	-187.988\\
1205	-109.863\\
1206	-72.021\\
1207	-101.318\\
1208	-86.67\\
1209	-53.711\\
1210	-74.463\\
1211	-84.229\\
1212	-67.139\\
1213	-51.27\\
1214	-70.801\\
1215	-58.594\\
1216	-79.346\\
1217	-107.422\\
1218	-100.098\\
1219	-69.58\\
1220	-70.801\\
1221	-107.422\\
1222	-177.002\\
1223	-128.174\\
1224	-79.346\\
1225	-61.035\\
1226	-70.801\\
1227	-64.697\\
1228	-48.828\\
1229	-53.711\\
1230	-65.918\\
1231	-83.008\\
1232	-91.553\\
1233	-63.477\\
1234	-104.98\\
1235	-124.512\\
1236	-118.408\\
1237	-91.553\\
1238	-100.098\\
1239	-135.498\\
1240	-87.891\\
1241	-42.725\\
1242	-57.373\\
1243	-62.256\\
1244	-81.787\\
1245	-72.021\\
1246	-52.49\\
1247	-79.346\\
1248	-80.566\\
1249	-57.373\\
1250	-70.801\\
1251	-63.477\\
1252	-56.152\\
1253	-67.139\\
1254	-41.504\\
1255	-48.828\\
1256	-36.621\\
1257	-40.283\\
1258	-75.684\\
1259	-84.229\\
1260	-113.525\\
1261	-146.484\\
1262	-98.877\\
1263	-58.594\\
1264	-46.387\\
1265	-43.945\\
1266	-63.477\\
1267	-42.725\\
1268	-47.607\\
1269	-61.035\\
1270	-102.539\\
1271	-93.994\\
1272	-134.277\\
1273	-89.111\\
1274	-81.787\\
1275	-46.387\\
1276	-45.166\\
1277	-28.076\\
1278	-35.4\\
1279	-37.842\\
1280	-67.139\\
1281	-70.801\\
1282	-79.346\\
1283	-85.449\\
1284	-125.732\\
1285	-112.305\\
1286	-84.229\\
1287	-102.539\\
1288	-109.863\\
1289	-144.043\\
1290	-115.967\\
1291	-75.684\\
1292	-40.283\\
1293	-29.297\\
1294	-29.297\\
1295	-36.621\\
1296	-39.063\\
1297	-37.842\\
1298	-50.049\\
1299	-74.463\\
1300	-68.359\\
1301	-70.801\\
1302	-83.008\\
1303	-51.27\\
1304	-30.518\\
1305	-58.594\\
1306	-90.332\\
1307	-87.891\\
1308	-97.656\\
1309	-83.008\\
1310	-61.035\\
1311	-85.449\\
1312	-118.408\\
1313	-118.408\\
1314	-84.229\\
1315	-136.719\\
1316	-100.098\\
1317	-101.318\\
1318	-117.188\\
1319	-115.967\\
1320	-86.67\\
1321	-76.904\\
1322	-128.174\\
1323	-178.223\\
1324	-139.16\\
1325	-79.346\\
1326	-76.904\\
1327	-79.346\\
1328	-98.877\\
1329	-114.746\\
1330	-85.449\\
1331	-90.332\\
1332	-120.85\\
1333	-131.836\\
1334	-87.891\\
1335	-68.359\\
1336	-91.553\\
1337	-156.25\\
1338	-137.939\\
1339	-140.381\\
1340	-84.229\\
1341	-76.904\\
1342	-72.021\\
1343	-47.607\\
1344	-30.518\\
1345	-26.855\\
1346	-23.193\\
1347	-54.932\\
1348	-70.801\\
1349	-87.891\\
1350	-65.918\\
1351	-73.242\\
1352	-83.008\\
1353	-53.711\\
1354	-56.152\\
1355	-53.711\\
1356	-40.283\\
1357	-51.27\\
1358	-84.229\\
1359	-106.201\\
1360	-63.477\\
1361	-48.828\\
1362	-40.283\\
1363	-50.049\\
1364	-35.4\\
1365	-76.904\\
1366	-130.615\\
1367	-104.98\\
1368	-139.16\\
1369	-162.354\\
1370	-164.795\\
1371	-124.512\\
1372	-90.332\\
1373	-89.111\\
1374	-89.111\\
1375	-106.201\\
1376	-125.732\\
1377	-125.732\\
1378	-168.457\\
1379	-183.105\\
1380	-219.727\\
1381	-147.705\\
1382	-166.016\\
1383	-184.326\\
1384	-140.381\\
1385	-162.354\\
1386	-139.16\\
1387	-80.566\\
1388	-52.49\\
1389	-56.152\\
1390	-81.787\\
1391	-65.918\\
1392	-43.945\\
1393	-62.256\\
1394	-47.607\\
1395	-36.621\\
1396	-46.387\\
1397	-52.49\\
1398	-36.621\\
1399	-70.801\\
1400	-92.773\\
1401	-68.359\\
1402	-114.746\\
1403	-164.795\\
1404	-150.146\\
1405	-196.533\\
1406	-131.836\\
1407	-144.043\\
1408	-96.436\\
1409	-63.477\\
1410	-76.904\\
1411	-59.814\\
1412	-45.166\\
1413	-64.697\\
1414	-36.621\\
1415	-28.076\\
1416	-34.18\\
1417	-70.801\\
1418	-86.67\\
1419	-107.422\\
1420	-103.76\\
1421	-100.098\\
1422	-114.746\\
1423	-93.994\\
1424	-76.904\\
1425	-76.904\\
1426	-62.256\\
1427	-97.656\\
1428	-68.359\\
1429	-56.152\\
1430	-81.787\\
1431	-113.525\\
1432	-86.67\\
1433	-65.918\\
1434	-56.152\\
1435	-65.918\\
1436	-45.166\\
1437	-45.166\\
1438	-59.814\\
1439	-75.684\\
1440	-84.229\\
1441	-65.918\\
1442	-86.67\\
1443	-79.346\\
1444	-47.607\\
1445	-50.049\\
1446	-95.215\\
1447	-131.836\\
1448	-131.836\\
1449	-107.422\\
1450	-103.76\\
1451	-79.346\\
1452	-76.904\\
1453	-61.035\\
1454	-89.111\\
1455	-75.684\\
1456	-50.049\\
1457	-74.463\\
1458	-85.449\\
1459	-101.318\\
1460	-109.863\\
1461	-109.863\\
1462	-148.926\\
1463	-181.885\\
1464	-205.078\\
1465	-129.395\\
1466	-73.242\\
1467	-47.607\\
1468	-34.18\\
1469	-54.932\\
1470	-46.387\\
1471	-80.566\\
1472	-72.021\\
1473	-64.697\\
1474	-85.449\\
1475	-98.877\\
1476	-117.188\\
1477	-123.291\\
1478	-175.781\\
1479	-150.146\\
1480	-117.188\\
1481	-80.566\\
1482	-80.566\\
1483	-83.008\\
1484	-106.201\\
1485	-75.684\\
1486	-79.346\\
1487	-74.463\\
1488	-42.725\\
1489	-74.463\\
1490	-107.422\\
1491	-73.242\\
1492	-80.566\\
1493	-147.705\\
1494	-109.863\\
1495	-106.201\\
1496	-147.705\\
1497	-140.381\\
1498	-87.891\\
1499	-56.152\\
1500	-58.594\\
};
\addlegendentry{True output};

\addplot [color=mycolor2,dashed]
  table[row sep=crcr]{%
1001	-82.3778659442859\\
1002	-104.276566421284\\
1003	-90.0705431579143\\
1004	-86.4281306586284\\
1005	-116.033091809973\\
1006	-100.986521487038\\
1007	-130.191078680898\\
1008	-100.794137471448\\
1009	-47.6278060443451\\
1010	-61.4131512685879\\
1011	-59.8289243315649\\
1012	-80.9957412149499\\
1013	-65.985047353\\
1014	-24.3062490549714\\
1015	-15.864130155365\\
1016	-17.392109232912\\
1017	-50.8945204015309\\
1018	-93.4709899342724\\
1019	-98.7125349482615\\
1020	-94.8759707613303\\
1021	-66.3500259728016\\
1022	-35.3292618861059\\
1023	-88.729195500119\\
1024	-58.2972631734349\\
1025	-47.8440722280281\\
1026	-83.7325597078435\\
1027	-68.127554756146\\
1028	-60.1499360330854\\
1029	-97.3663678112424\\
1030	-84.1203128822858\\
1031	-67.7895040777209\\
1032	-94.8540932872013\\
1033	-90.8926558139785\\
1034	-77.5415510948552\\
1035	-65.2990425263578\\
1036	-51.3663047212859\\
1037	-63.0620235122234\\
1038	-74.1374868952414\\
1039	-74.761845587564\\
1040	-74.5471943047829\\
1041	-76.398496637042\\
1042	-77.5333109618045\\
1043	-111.024353554249\\
1044	-101.481662059359\\
1045	-72.0235787619804\\
1046	-64.4685053768329\\
1047	-34.1778360301877\\
1048	-36.786556980638\\
1049	-56.8034053765719\\
1050	-44.6904481865286\\
1051	-47.7568665394353\\
1052	-65.3608619186186\\
1053	-71.4997326147141\\
1054	-68.1108597590753\\
1055	-105.999691879093\\
1056	-76.1753301423187\\
1057	-44.7980968233754\\
1058	-36.4626822166486\\
1059	-47.0455547758902\\
1060	-60.5446123916081\\
1061	-32.5879459611773\\
1062	-34.32401927458\\
1063	-46.9932980463236\\
1064	-39.0342192084918\\
1065	-34.6513866889097\\
1066	-44.5564644919361\\
1067	-46.5117483613439\\
1068	-82.5679116018798\\
1069	-69.6464707987306\\
1070	-87.6610722364101\\
1071	-77.2245138222914\\
1072	-83.8069226210118\\
1073	-71.4188129751244\\
1074	-68.3086513097735\\
1075	-63.0729821850763\\
1076	-63.2013581262469\\
1077	-61.7948461228732\\
1078	-102.443493676212\\
1079	-151.969788743745\\
1080	-158.949966813685\\
1081	-151.006068713971\\
1082	-94.1779042067157\\
1083	-135.419348996082\\
1084	-160.939711331894\\
1085	-171.916218230394\\
1086	-143.072254126012\\
1087	-170.826291255187\\
1088	-224.496603121509\\
1089	-171.838238151212\\
1090	-150.531397505333\\
1091	-95.4270805087629\\
1092	-73.6354293843106\\
1093	-66.6996213934639\\
1094	-84.946584468772\\
1095	-61.1846498685716\\
1096	-38.6613039184279\\
1097	-35.6202205260474\\
1098	-49.5992259728102\\
1099	-74.9506892029147\\
1100	-69.4358411923796\\
1101	-70.4843931927291\\
1102	-86.7106542880815\\
1103	-90.5087927653352\\
1104	-123.588168234572\\
1105	-107.870713433547\\
1106	-117.825759513443\\
1107	-91.1432484441053\\
1108	-76.2937008009595\\
1109	-91.4700319691591\\
1110	-68.6147766178257\\
1111	-64.6666373703596\\
1112	-79.645035503963\\
1113	-78.2410709824807\\
1114	-61.0813468337076\\
1115	-62.8485403849498\\
1116	-45.7193322437585\\
1117	-47.7349681007851\\
1118	-55.5956157833467\\
1119	-41.9673791039147\\
1120	-49.6952558415998\\
1121	-60.3845141613265\\
1122	-94.7289830331278\\
1123	-87.8716254898876\\
1124	-101.720981106486\\
1125	-54.6734630785754\\
1126	-80.8734999227488\\
1127	-123.23440835112\\
1128	-99.1144520318398\\
1129	-112.065001231471\\
1130	-104.200651801854\\
1131	-65.0959404336305\\
1132	-41.6767803216175\\
1133	-64.5685239110135\\
1134	-72.1236975670632\\
1135	-124.910643059581\\
1136	-149.703317602366\\
1137	-145.502233087732\\
1138	-103.842288957267\\
1139	-113.349550747134\\
1140	-97.8432729857386\\
1141	-98.3813456367001\\
1142	-101.307981325602\\
1143	-66.6021733408805\\
1144	-63.3271959305907\\
1145	-55.8348231515981\\
1146	-52.2303956577959\\
1147	-62.0492611390378\\
1148	-87.4889858247062\\
1149	-124.226071987782\\
1150	-108.495774368757\\
1151	-70.0695751154775\\
1152	-60.639566977426\\
1153	-55.5975674978929\\
1154	-33.7938590994951\\
1155	-24.3274615854947\\
1156	-31.4366626762458\\
1157	-54.0715804074412\\
1158	-44.8810264179441\\
1159	-49.3431442039601\\
1160	-50.109057716954\\
1161	-48.2153589034539\\
1162	-34.0247940817057\\
1163	-26.4019265562188\\
1164	-20.7093019876944\\
1165	-30.1360725557611\\
1166	-71.7689646861325\\
1167	-105.731441366341\\
1168	-112.779540609304\\
1169	-75.1663198752019\\
1170	-45.4666921824341\\
1171	-32.9261994387721\\
1172	-23.3503132649666\\
1173	-54.5485494791158\\
1174	-53.4282414459466\\
1175	-76.5795759740743\\
1176	-97.5269629432756\\
1177	-164.408463737357\\
1178	-190.979975771626\\
1179	-159.243138815801\\
1180	-140.962855720958\\
1181	-90.3475954430859\\
1182	-102.181663048581\\
1183	-115.408036865231\\
1184	-124.068110831904\\
1185	-104.71821626804\\
1186	-87.4794540866097\\
1187	-95.6049177595303\\
1188	-86.5320027755512\\
1189	-98.3907550412042\\
1190	-76.0410247191198\\
1191	-120.084705355599\\
1192	-145.716721570945\\
1193	-104.181185128948\\
1194	-69.7969531978693\\
1195	-64.3328906307408\\
1196	-108.679094120461\\
1197	-137.928090602033\\
1198	-161.191578182321\\
1199	-170.043118567344\\
1200	-170.972131330784\\
1201	-129.33663612864\\
1202	-122.810054915668\\
1203	-135.488512057264\\
1204	-158.149627575828\\
1205	-96.809845293976\\
1206	-62.0132899114095\\
1207	-96.4488827793314\\
1208	-79.1053541970713\\
1209	-52.8573608534151\\
1210	-72.1538552417825\\
1211	-79.4322615913263\\
1212	-66.4152010291039\\
1213	-58.4594178671426\\
1214	-68.2029861010598\\
1215	-58.9896363150717\\
1216	-71.47428387376\\
1217	-106.451171626366\\
1218	-92.1392219851958\\
1219	-69.8955544852761\\
1220	-65.8347699098736\\
1221	-95.2389480067523\\
1222	-158.403220368997\\
1223	-127.151021262835\\
1224	-73.8507485147838\\
1225	-57.7836849764125\\
1226	-62.2053687637256\\
1227	-64.1894959066802\\
1228	-46.3766052281606\\
1229	-54.6603009257994\\
1230	-61.876136818966\\
1231	-76.8979179318976\\
1232	-88.1420296434433\\
1233	-60.7523988799379\\
1234	-94.2538059164859\\
1235	-111.132767175273\\
1236	-101.391060948687\\
1237	-82.7819294615246\\
1238	-86.4198206303831\\
1239	-111.914602223546\\
1240	-84.9556723982653\\
1241	-30.3464678316232\\
1242	-43.1038711929693\\
1243	-52.6546163895272\\
1244	-76.7491849088598\\
1245	-69.8670559693766\\
1246	-53.0820801701805\\
1247	-76.6615398897418\\
1248	-67.9898016358561\\
1249	-51.2135364044089\\
1250	-67.4135073616537\\
1251	-56.6884625222931\\
1252	-51.4981927314668\\
1253	-63.5737058448799\\
1254	-37.9110468659675\\
1255	-45.1332986598866\\
1256	-33.2616571033559\\
1257	-36.4157306797742\\
1258	-69.6367447554807\\
1259	-71.5070135096157\\
1260	-97.0030750095896\\
1261	-127.442917110104\\
1262	-84.3913769953004\\
1263	-48.5079266989673\\
1264	-34.4661479729335\\
1265	-34.7697679219084\\
1266	-59.2339833917329\\
1267	-38.7854723992251\\
1268	-42.3122585790866\\
1269	-52.7738964789733\\
1270	-90.4809609065712\\
1271	-82.526989410378\\
1272	-116.671435729693\\
1273	-77.7832184088595\\
1274	-63.725291611166\\
1275	-32.5592548487384\\
1276	-32.0917462874524\\
1277	-22.8758092884285\\
1278	-27.7830675251476\\
1279	-33.2662570173647\\
1280	-56.018871731646\\
1281	-60.9670800455901\\
1282	-64.9396440683909\\
1283	-67.254555716675\\
1284	-102.349946814638\\
1285	-100.518371632798\\
1286	-70.6071238278902\\
1287	-81.6975742921129\\
1288	-87.991480105349\\
1289	-113.925867433157\\
1290	-93.3190382353838\\
1291	-62.2893600310182\\
1292	-25.0419223971251\\
1293	-18.1280588100038\\
1294	-23.0023929055259\\
1295	-31.0194879353349\\
1296	-34.2531244334585\\
1297	-33.2541210478086\\
1298	-38.7323443647249\\
1299	-65.0185023856796\\
1300	-58.9980697172189\\
1301	-56.0255489709667\\
1302	-63.9274203700664\\
1303	-39.5149443065196\\
1304	-23.1004356246376\\
1305	-44.1136949515988\\
1306	-70.7590463142498\\
1307	-69.7929891501577\\
1308	-80.994162751942\\
1309	-63.0867654011431\\
1310	-46.5625269998861\\
1311	-64.3402866992171\\
1312	-89.3663710918302\\
1313	-92.9941286230541\\
1314	-67.8898095594328\\
1315	-108.074718898314\\
1316	-81.5742823446585\\
1317	-77.1634516934616\\
1318	-97.4218576661334\\
1319	-90.0803331363406\\
1320	-75.0834633237246\\
1321	-67.5231710612994\\
1322	-104.619672911048\\
1323	-150.726416889631\\
1324	-118.412735924767\\
1325	-68.6742537005776\\
1326	-60.0238805010907\\
1327	-69.0666464866704\\
1328	-82.489451543136\\
1329	-96.9702280085322\\
1330	-75.8353549267878\\
1331	-83.2310552822605\\
1332	-101.858461355379\\
1333	-105.050866081145\\
1334	-75.344492425085\\
1335	-65.1525197574303\\
1336	-80.0568484048588\\
1337	-138.125447684258\\
1338	-120.930877459044\\
1339	-119.965136786944\\
1340	-72.1588102874628\\
1341	-60.6610711478365\\
1342	-62.8874830458857\\
1343	-39.1232659885085\\
1344	-24.5141726287676\\
1345	-20.5240109289548\\
1346	-20.2039941926775\\
1347	-42.9802511160649\\
1348	-63.2353001457975\\
1349	-77.5980389749593\\
1350	-57.7490530314026\\
1351	-56.3514966834513\\
1352	-64.2487813256792\\
1353	-39.2061232668753\\
1354	-43.1371789662058\\
1355	-42.4662368143916\\
1356	-32.543205616383\\
1357	-43.1010799326734\\
1358	-66.7535367538531\\
1359	-86.7956682102376\\
1360	-51.3908211864472\\
1361	-38.4444940371545\\
1362	-33.7011852787576\\
1363	-42.2392475814029\\
1364	-29.40194352961\\
1365	-67.270393904756\\
1366	-113.651147532231\\
1367	-92.9818942072699\\
1368	-115.037980824411\\
1369	-127.634843726185\\
1370	-131.473911982151\\
1371	-97.9021688709255\\
1372	-71.6775301969054\\
1373	-75.023725486911\\
1374	-73.8455040392391\\
1375	-91.0895193827509\\
1376	-106.062919677493\\
1377	-102.956199707826\\
1378	-137.889577285478\\
1379	-151.121530442797\\
1380	-176.542371698278\\
1381	-126.421344541504\\
1382	-140.95056146035\\
1383	-145.676825588611\\
1384	-123.076059695345\\
1385	-140.328459196124\\
1386	-122.315586405816\\
1387	-70.2786219905578\\
1388	-40.7923179962775\\
1389	-45.6107999989472\\
1390	-71.6799575825904\\
1391	-61.9926004000134\\
1392	-44.7897125196563\\
1393	-61.1018939393279\\
1394	-44.1069520810133\\
1395	-35.5124994841433\\
1396	-43.6451542142113\\
1397	-43.9542506602862\\
1398	-37.7078787551948\\
1399	-60.789612263328\\
1400	-81.4666108168268\\
1401	-61.3278828123711\\
1402	-104.866553581902\\
1403	-150.191840476322\\
1404	-136.298888056764\\
1405	-169.396652100171\\
1406	-111.439307205749\\
1407	-123.553387816224\\
1408	-81.4884924920724\\
1409	-47.3148247501774\\
1410	-65.1215747605182\\
1411	-54.7128234347311\\
1412	-41.5889895205543\\
1413	-57.8166995793567\\
1414	-38.310623576541\\
1415	-21.8040443096835\\
1416	-28.1929219185247\\
1417	-58.8119417880875\\
1418	-80.5582404869346\\
1419	-94.5446650628759\\
1420	-91.5790831010046\\
1421	-79.897208150619\\
1422	-90.1273528243885\\
1423	-75.2135291610096\\
1424	-63.9892014194011\\
1425	-66.1281134770207\\
1426	-54.1669266047147\\
1427	-85.3293065097382\\
1428	-56.4050815301926\\
1429	-48.8853850527563\\
1430	-73.8703286602123\\
1431	-93.6783584357159\\
1432	-74.9163356473811\\
1433	-57.9224580623498\\
1434	-50.7287883371443\\
1435	-57.8302659688285\\
1436	-40.0601697108641\\
1437	-38.7781656962154\\
1438	-53.6479749281159\\
1439	-62.068967239284\\
1440	-71.5163047833607\\
1441	-59.629685880435\\
1442	-70.8624073603298\\
1443	-63.5885004654938\\
1444	-39.0901123733423\\
1445	-42.561944867379\\
1446	-79.2342228856937\\
1447	-118.131764883754\\
1448	-111.418161000284\\
1449	-90.2811896375625\\
1450	-82.5567441717623\\
1451	-68.0474207166709\\
1452	-62.7755946537658\\
1453	-55.9469685280625\\
1454	-71.6401509095514\\
1455	-70.3762376993066\\
1456	-43.3575650573866\\
1457	-67.0948782617924\\
1458	-68.8824226608455\\
1459	-83.3404569790037\\
1460	-94.5644851013501\\
1461	-85.5304857288044\\
1462	-127.219630277279\\
1463	-149.537515586237\\
1464	-169.451098924072\\
1465	-116.691023517776\\
1466	-56.3836576667316\\
1467	-31.3904472424651\\
1468	-22.4153579566101\\
1469	-42.0159155190308\\
1470	-40.155926115912\\
1471	-73.5255164815445\\
1472	-64.4317872097617\\
1473	-57.47564633052\\
1474	-72.9345453573633\\
1475	-79.3066411751537\\
1476	-97.1039181554501\\
1477	-101.589236670919\\
1478	-147.863394386314\\
1479	-133.527912345022\\
1480	-100.482370897092\\
1481	-72.6707946015177\\
1482	-65.525278889421\\
1483	-77.3428878855186\\
1484	-93.1559694657467\\
1485	-71.0694750035224\\
1486	-70.9693950237044\\
1487	-74.4734030967362\\
1488	-34.2176098686109\\
1489	-65.2550946759291\\
1490	-92.7215305295799\\
1491	-68.4384947444119\\
1492	-68.5772433741785\\
1493	-135.451873846715\\
1494	-91.1450253593166\\
1495	-95.8336016670712\\
1496	-123.004456648776\\
1497	-113.976967658871\\
1498	-80.028856757846\\
1499	-48.3390891944178\\
1500	-48.4388618091224\\
};
\addlegendentry{Generated output};

\end{axis}
\end{tikzpicture}%\label{fig:c8c610}}
	\caption{Samples of the output obtained experimentally and the output generated by the models identified from datasets corresponding to Foams (a) F1 and (b) F2.}\label{fig:Csubsets}
\end{figure}
\printbibliography
\end{document}