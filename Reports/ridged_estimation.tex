\documentclass[a4paper,11pt,twoside]{article}
%%%%%%%%%%%%%%%%%%%%%%%%%%%%%%%%%%%%%%%%%%%%%%%%%%%%%%%%%%%%%%%%%%%%%%%%%%%
% A generic report compiler for publishing rables and plots for a partcular choise of lags in the NARX system
% A path to media corresponding to the setting is defined below, as well as chosen parameter settings
\makeatletter
\def\input@path{{../Results_set_C_ny_0_nu_4/}}
\def\dataset{C}
\def\ny{4}
\def\nu{4}
\def\order{2}
\makeatother
% Packages
\usepackage[final]{pdfpages}
\usepackage{verbatim}
\usepackage{inputenc}
\usepackage{graphicx} 
\usepackage{amsmath,amssymb,mathrsfs,amsfonts}
%\usepackage{unicode-math}
\usepackage{mathtools}
\usepackage{amsthm}
\usepackage{mathtools}
\usepackage{calrsfs}
\usepackage{graphicx}
\usepackage{subfig}
\usepackage{eucal}    
\usepackage{amssymb}  
\usepackage{pifont}
\usepackage{color} 
\usepackage{cancel}
\usepackage[toc,page]{appendix}
\usepackage{pgfplots}
\pgfplotsset{every axis/.append style={line width=0.5pt},label style={font=\scriptsize},tick label style={font=\scriptsize},x tick label style={/pgf/number format/.cd,fixed,precision=3, set thousands separator={}},z tick label style={/pgf/number format/.cd,fixed,precision=3, set thousands separator={}}}
\usetikzlibrary{shapes,shadows,arrows,backgrounds,patterns,positioning,automata,calc,decorations.markings,decorations.pathreplacing,bayesnet,arrows.meta}
%\usepackage{tikzexternal}
%\usepackage{tikz}
%\usepgfplotslibrary{external} 
%\tikzexternalize
\usepackage{varwidth}
\usepackage{lscape}
\usepackage{array} 
\usepackage[colorlinks=false,pdfborder={0 0 0}]{hyperref}
\usepackage{tabularx}
\usepackage{textcomp}
\usepackage{multicol} 
\usepackage{booktabs}
\usepackage{multirow}
\usepackage[font=small,labelfont=bf]{caption}                                                           
\usepackage{textcase}
\usepackage{bbm} 
\usepackage{fancyhdr}
\usepackage{enumitem}
\usepackage{soul}
%\usepackage[british]{babel}
\usepackage{wrapfig}
%\usepackage{glossaries}
%%%%%%%%%%%%%%%%%%%%%%%%%%%%%%%%%%%%%%%%%%%%%%%%%%%%%%%%%%%%%%%%%%%%%%%%%%%
\setlength{\parindent}{2em}
\setlength{\parskip}{0.5em}
\renewcommand{\baselinestretch}{1.2}
\usepackage[left=2cm, right=2cm, top=2.5cm, bottom=3cm, headheight=13.6pt]{geometry}
\allowdisplaybreaks 
% Bibliography
\usepackage[backend=bibtex,style=ieee,sorting=none]{biblatex} 
\bibliography{bibliography_ridge}
\renewcommand*{\bibfont}{\scriptsize}
\makeatletter
\def\input@path{{../Results_set_C_ny_0_nu_4/}}
\newcommand*{\rom}[1]{\expandafter\@slowromancap\romannumeral #1@}
\newcommand{\ie}{\textit{i.e.} }
\newcommand{\eg}{\textit{e.g.} }
\newcommand\id{\ensuremath{\mathbbm{1}}} 
\newcommand{\norm}[1]{\left\lVert#1\right\rVert}
\DeclareMathOperator{\E}{\mathbb{E}}
\DeclareMathOperator{\eye}{\mathbb{I}}
\DeclareMathOperator{\R}{\rm I\!R}
\DeclareMathOperator{\zeros}{\mathbb{O}}
\DeclareMathOperator{\tr}{\textrm{tr}}
\DeclareMathOperator{\vvec}{\textrm{vec}}
\DeclareMathOperator{\ik}{\mathrm{k}}
\DeclareMathOperator{\ip}{\mathrm{p}}
\DeclareMathOperator{\inn}{\mathrm{n}}
\DeclareMathOperator{\im}{\mathrm{m}}
\DeclareMathOperator{\td}{\mathrm{t}}
\DeclareMathOperator{\kd}{\mathrm{k}}
\DeclareMathOperator{\T}{\mathrm{T}}
\DeclareMathOperator{\K}{\mathrm{K}}
\DeclareMathOperator{\rk}{\mathrm{rk}}
\DeclareMathOperator{\vc}{\mathrm{vec}}
\DeclareSymbolFontAlphabet{\mathcal} {symbols}
\DeclareSymbolFont{symbols}{OMS}{cm}{m}{n}
\DeclareMathAlphabet{\mathbfit}{OML}{cmm}{b}{it}
\makeatother
% Number equations
%\numberwithin{equation}{section}

%%%%%%%%%%%%%%%%%%%%%%%%%%%%%%%%%%%%%%%%%%%%%%%%%%%%%%%%%%%%%%%%%%%%%%%%%%%
%Theorems
\newtheoremstyle{mytheoremstyle} % name
{.5em}                    % Space above
{.8em}                    % Space below
{\itshape}                % Body font
{1em}                           % Indent amount
{\bfseries}                   % Theorem head font
{:}                          % Punctuation after theorem head
{.5em}                       % Space after theorem head
{}  % Theorem head spec (can be left empty, meaning ‘normal’)

\theoremstyle{mytheoremstyle}
\newtheorem{theorem}{Theorem}[section]
\newtheorem{remark}{Remark}[section]
\newtheorem{assumption}{Assumption}[section]
\newtheorem{lemma}{Lemma}[section]
\newtheorem{condition}{Condition}[section]
\newtheorem{definition}{Definition}[section]
\newtheorem{property}{Property}[section]
\newtheorem{corollary}{Corollary}[section]
\renewcommand\qedsymbol{$\blacksquare$}
%%%%%%%%%%%%%%%%%%%%%%%%%%%%%%%%%%%%%%%%%%%%%%%%%%%%%%%%%%%%%%%%%%%%%%%%%%%
% Nomenclature
\usepackage[intoc]{nomencl}
\makenomenclature

%\usepackage[ruled,chapter]{algorithm}
%\usepackage{float}

%\usepackage{algorithmic}
%\algsetup{linenosize=\scriptsize}
%\usepackage{etoolbox}
%\AtBeginEnvironment{algorithmic}{\scriptsize}
%\renewcommand{\thealgorithm}{\thechapter.\arabic{algorithm}} 
%\usepackage{chngcntr}
%\counterwithin{algorithm}{section}

% correct bad hyphenation here
\hyphenation{op-tical net-works semi-conduc-tor}
%%%%%%%%%%%%%%%%%%%%%%%%%%%%%%%%%%%%%%%%%%%%%%%%%%%%%%%%%%%%%%%%%%%%%%%%%%%%%
% Captions
%\newcommand{\xLanguage}{british}          % <-- Added this command
%
%\usepackage[\xLanguage]{babel}             % <-- Implemented here
%
%\expandafter\addto\csname captions\xLanguage\endcsname{% <-- and here
%	\renewcommand{\tableshortname}{Table}%
%	\renewcommand{\figureshortname}{Figure}%
%}
%
%\captionsetup[table]{format=plain,indention=1.15cm,justification=justified}
%\captionsetup[figure]{format=plain,indention=1.25cm,justification=justified}

%%%%%%%%%%%%%%%%%%%%%%%%%%%%%%%%%%%%%%%%%%%%%%%%%%%%%%%%%%%%%%%%%%%%%%%%%%%%%
\usepackage[explicit]{titlesec}
\usepackage{titletoc}
%\titleformat{\section}[block]{\normalfont\Large\rm\filright\bfseries}{\thesection}{1em}{#1}
%\titlecontents{chapter}[1.5em]{}{\scshape\contentslabel{2.3em}}{}{\titlerule*[1pc]{}\contentspage}
%\definecolor{gray75}{gray}{0.5}
%\newcommand{\hsp}{\hspace{20pt}}
%\titleformat{\chapter}[hang]{\huge\scshape\filright\bfseries}{\color{gray75}\thechapter}{20pt}{\begin{tabular}[t]{@{\color{gray75}\vrule width 2pt\hsp}p{0.85\textwidth}}\raggedright#1\end{tabular}}
%\titleformat{name=\chapter,numberless}[display]{}{}{0pt}{\normalfont\huge\bfseries #1} % format for numberless chapters
%\titleformat{\subsection}[block]{\normalfont\large\rm\filright\bfseries}{\thesubsection}{1em}{#1}
%\renewcommand{\sectionmark}[1]{\markright{\thesection ~ \ #1}}
%%\renewcommand{\chaptermark}[1]{\markboth{\chaptername\ \thechapter ~ \ #1}{}} 
%\pagestyle{fancy}
%%\fancyhf{}
%\fancyhead[RO,LE]{\thepage}
%\fancyhead[RE]{\itshape \nouppercase \rightmark}      % chaptertitle left
%\fancyhead[LO]{\itshape \nouppercase \leftmark}       % sectiontitle right
%\renewcommand{\headrulewidth}{0.5pt} 				   % no rule
%\cfoot{}
\interfootnotelinepenalty=10000

\title{Regularised Least Squares estimation of external parameters}
\begin{document}
	\maketitle
\section*{Preliminaries}
\par Recall the notation from the previous reports:
\begin{itemize}
	\item[] $k = 1, \dots, K$ - index of the dataset used for identification.
	\item[] $t = 1, \dots, T$ - discrete time index.
	\item[] $N_s$ - number of significant terms in the model.
	\item[] $i = 1, \dots, N_s$ - term index.
	\item[] $\Xi = \R^2$ - space of external parameters.
	\item[] $\Theta = \R^{N_s}$ - space of internal parameters.
	\item[]	$\theta^{k}_{i}$ - an internal regression parameter corresponding to the $i$-th term in the $k$-th dataset.
	\item[] $\mathbf{\theta}^k = \{\theta^{k}_{i}\}_{i=1}^{N_s}$ - vector of internal parameters.
\end{itemize}
\par The analysis deals with the time-series data from auxetic foam vibration tests introduced in the previous report. A non-linear FIR filter is considered with the input lag of 3. The filter structure is approximated using polynomial regressors of order $\order$. Number the significant terms and corresponding scaling parameters $\theta$ are estimated via EFOR-CMSS algorithm based on 8 datasets out of 10 available. The remaining to datasets (\dataset3 and \dataset8) are used for model validation. 
\par This report deals only with identification of the relationship between the estimated internal parameters $\mathbf{\theta}^k$ and the pre-specified design settings $\xi^k = (L_{cut}, D_{rlx})$. The first section examines the effect on the polynomial structure and sampling from the small subspace of external parameters on the generalisation of the identified model. The remaining sections briefly describe approaches to improving the estimation procedure. The results section only demonstrates numerical results and does not draw any conclusions as this report is intermediate.
\section{Collinearity in the selected model}
\par Recall the assumed polynomial model that links internal parameters to the external ones:
\begin{equation}\label{eq:polyn}
\theta^{k}_{i}(L_{cut},D_{rlx}) = \beta_0 + \beta_1 L_{cut} + \beta_2 D_{rlx} + \beta_3 L_{cut}^{2} + \beta_4 D_{rlx}^{2} + \beta_5 L_{cut} D_{rlx}, \qquad i=1,\dots,N_s,
\end{equation}
where polynomial coefficients $\mathbf{\beta}^i = \left[ \beta_0 \dots \beta_5 \right]$ are unknown. In the original paper \cite{Wei2008} they are estimated via batch LS estimator as follows. First, the matrix of "independent" variables is formed from the external parameter values:
\begin{equation}
\mathbf{A} = \left[\begin{array}{cccccc}
1 & L& D& L\odot L& D \odot D& L\odot  D
\end{array}\right],
\end{equation}
where $L = \{L_{cut}^{k}\}^{K}_{k=1}$, $D = \{D_{rlx}^{k}\}^{K}_{k=1}$, the number of columns $M + 1$, where $M$ is the order of polynomial \eqref{eq:polyn}, and where $\odot$ denotes Hadamard entry-wise product:
\begin{equation*}
L\odot L = \Big[ L_{cut}^{(k)} L_{cut}^{(k)}\Big]^{K}_{k=1}.
\end{equation*}
Then, for each internal parameter a vector of estimated values is formed across $K$ training datasets:
\begin{equation*}
\bar{\theta}^i = \left\{ \theta^{1}_{i}, \theta^{1}_{i}, \dots, \theta^{K}_{i} \right\}^{\top}.
\end{equation*}
Note that while the internal parameters are considered to be fixed in the estimation, their estimates are obtained from the random data and therefore are random variables 
\begin{equation*}
\bar{\mathbf{\theta}}^i \sim \mathcal{N}(\mu_i, \sigma^{2}_{i}).
\end{equation*}
\begin{remark}
For the purposes of this analysis the factors contributing to this variance are not considered and the vector of LSEs is treated as a response signal in the LS formulation.
\end{remark}
Estimation of the polynomial coefficients for the $i$-th internal parameter $\mathbf{\beta}^i$ is an ordinary least squares (OLS) problem:
\begin{equation}
\hat{\mathbf{\beta}}^i = \underset{\beta \in \R}{\arg \min} \Biggl\{\norm{ \biggl(\bar{\theta}^i - \mathbf{A} \mathbf{\beta}^i \biggr) }^{2}_{2}\Biggr\}
\end{equation}
with closed form solution
\begin{equation}
\hat{\mathbf{\beta}}^{i}_{OLS} = \mathbf{R}^{-1}_{aa} \mathbf{A}^{\top}\bar{\theta}^i,  
\end{equation} 
where $\mathbf{R}^{\ast}_{aa} = (\mathbf{A}^\ast)^{\top}\mathbf{A}^\ast$ is the regression matrix. The OLS estimator is unbiased and provides the solution that best fits the training data, where the fit is characterised by the sum of squared residuals (RSS). The unbiasedness is usually at the cost of the estimate variance. Moreover, OLS regression always utilises all provided covariates and if the model structure is selected arbitrarily (as in this report) and is overly detailed, the estimation leads to the overfitting to the training data and bad generalisation of the resultant model.
\par It can be seen from Table 1 in previous report that the experimental data is collected from the narrow subspace of design parameters. Combined with the polynomial structure that means that the columns of the data matrix $\mathbf{A} $ are powers of one another, this may lead to severe collinearity problem. The quickest way to access collinearity is to check the condition number of the scaled matrix $\mathbf{A}^\ast$. As per recommendation in \cite{Seber2003}, the data is not centred to see the effect on the raw data perturbation on the LSE variance:
\begin{equation*}
a^{\ast}_{k,j} = \frac{a_{k,j}}{\Big\{ \sum^{K}_{k=1} a^{2}_{k,j}\Big\}^{1/2}}, 
\end{equation*}
where $j$ is the column index.
For the given data $\kappa(\mathbf{A}^\ast) = 3308.29$ which indicates that some eigenvalues of the regression matrix $\mathbf{R}^{\ast}_{aa} = (\mathbf{A}^\ast)^{\top}\mathbf{A}^\ast$ are close to zero:
\begin{equation*}
\text{eig}(\mathbf{R}^{\ast}_{aa}) = \left[\begin{array}{r}
0.00016 \\
0.00094\\
0.01672\\
1.11563\\
18.7669\\
1806.28
%4.78 \times 10^{-6} \\
%3.50 \times 10^{-5} \\
%0.0015	\\
%0.0199	\\
%0.3625	\\
%5.6161	\\
\end{array}\right]
\end{equation*}
Some collinearity metrics for centred and scaled matrix $\mathbf{A}_{s}^{\star}$, where $\mathbf{A} = \left[\right]1, \mathbf{A}_{s}$ are summarised in Table 1. The correlation of each columns and linear combinations of other columns is assessed by computing the coefficients of determination for individual columns
\begin{equation*}
R^{2}_{j} = 1 - \frac{(a^{\star}_{j})^{\top} H_j a^{\star}_{j}}{\sum_{i}( a^{\star}_{j} - \hat{a}^{\star}_{j})^2},
\end{equation*} 
where $ \mathbf{H}_j = \mathbf{A}_{j}^{\star}((\mathbf{A}_{j}^{\star})^{\top}\mathbf{A}_{j}^{\star})^{-1}(\mathbf{A}_{j}^{\star})^{\top}$ is the hat matrix and where $\mathbf{A}_{j}^{\star}$ is the data matrix that excludes the column $a^{\star}_{j}$. The ratio in this case quantifies the distance between the column of interest and its linear  regression on other columns. The smaller this distance is, the higher determination coefficient for the column $a^{\star}_{j}$ will be. However, coefficients of determination do not indicate which particular columns have near-linear dependence. Thus, the effect of collinearity on LSE variance is assessed via variance inflation coefficients
\begin{equation*}
\text{VIF}_j = \frac{\text{Var}(\beta_j)}{\sigma^{2}_{j}} = \frac{\sigma^{2}_{j}}{1 - R^{2}_{j}},
\end{equation*}
while the number of correlated column maybe better assessed from eigenvalues of the scaled and centred regression matrix are
\begin{equation*}
\text{eig}(\mathbf{R}^{\star}_{aa}) = \left[\begin{array}{r}
0.00066\\
0.00103\\
0.01662\\
8.07508\\
18.3753\\
57.8643
%0.0002\\
%0.0004\\
%0.0075\\
%1.9647\\
%3.0271\\
%8
\end{array}\right]
\end{equation*}
It can be seen that at least two columns of the normalised data matrix suffer from collinearity, and its effect on variance inflation confirms that the model may suffer bad generalisation outside of the considered subspace of design parameters $\xi$. This indicates the need for regularisation of the estimation procedure in order to reduce the dispersion of the parameter LSEs.
\begin{table}[!h]
	\centering
	\caption{Collinearity metrics of the data.}\label{tab:vifs}
	\begin{tabular}{rrrrrr}
		Column & $L$ & $R$ & $L \odot R$ & $L \odot L$ & $R \odot R$ \\
		\hline
		$R^{2}_{j}$ & 0.9992 &   0.9977  &  0.9825  &  0.9406  & 0.9600 \\
		VIF$_j$ & 1180.97 &	443.31 &	56.98 &	16.82 &4	25.025 \\
		
%		 1256.04 & 862.29 & 2867.46 & 1212.49 & 665.06 \\
		\hline
	\end{tabular}
\end{table}
\section{Regularised Least Squares}
\par A common treatment for models that suffer from bad generalisation and collinearity is to introduce a constrained LS problem, where the constraint is normally posed on the unknown parameter vector:
\begin{equation}\label{eq:rls_constr}
\hat{\mathbf{\beta}}^i = \underset{\beta \in \R}{\arg \min}\norm{ \biggl(\bar{\theta}^i - \mathbf{A} \mathbf{\beta}^i \biggr) }^{2}_{2}, \quad f_{R}(\mathbf{\beta}^i) < \gamma.
\end{equation}
where $\gamma$ is a pre-specified parameter that defines the size of the constraint in the parameter space $\Xi$.
Lagrangian formulation of the constrained problem is called Regularised Least Squares (RLS),
\begin{equation}\label{eq:rls_problem}
\hat{\mathbf{\beta}}^i = \underset{\beta \in \R}{\arg \min} \Biggl\{\norm{ \biggl(\bar{\theta}^i - \mathbf{A} \mathbf{\beta}^i \biggr) }^{2}_{2} + \lambda f_{R}(\mathbf{\beta}^i)\Biggr\}
\end{equation}
where the regularisation coefficient $\lambda$ is directly linked to  $\gamma$ in \eqref{eq:rls_constr} \cite{Seber2003}. Different types of the constraint function can be considered depending on the problem.
\subsection{Tikhonov regularisation}
Innovation regularisation  constrains the 2-norm of the parameter vector
\begin{equation}
f_{R}(\mathbf{\beta}^i) = \norm{\mathbf{\beta}^i}^{2}_{2}.
\end{equation}
This is the only RLS formulation that has a closed form solution that is usually obtained for the normalised data. 
\begin{equation}\label{eq:ridge}
\hat{\mathbf{\beta}}^{i}_{RLS} = (\mathbf{R}^{*}_{aa} + \lambda \eye_M)^{-1} (\mathbf{A}^{\star})^{\top}\bar{\theta}^i,  
\end{equation}
This solution is referred to as ridge regression, because increasing $\lambda$ shrinks the coefficients $\beta_j$. The shrinkage is shown is best interpreted via singular values decomposition (SVD) of the normalised data matrix. Denote the SVD of $\mathbf{A}^{\star}$ as
\begin{equation}
\underbrace{\mathbf{A}^{\star}}_{K \times M} = \underbrace{\mathbf{U}}_{K \times M}\underbrace{\mathbf{D}}_{M \times M}\underbrace{\mathbf{V}^{\top}}_{M \times M},
\end{equation}
where columns of $\mathbf{U}$ are principal components of $\mathbf{A}^{\star}$, diagonal elements of  $\mathbf{D}$ are the singular values, and where $\mathbf{V}$ is the rotation. All orthonormality assumption are the same as in the general case. The ridge regression \eqref{eq:ridge} then takes form
\begin{equation}
\hat{\mathbf{\beta}}^{i}_{RLS} = \left( \mathbf{V}\mathbf{D}\mathbf{U}^{\top}\mathbf{U}\mathbf{D}\mathbf{V}^{\top}  + \lambda \eye_M\right)^{-1} \left( \mathbf{U}\mathbf{D}\mathbf{V}^{\top}\right)^{\top}\bar{\theta}^i.
\end{equation}
Simple linear algebra yields the following
\begin{equation}
\hat{\mathbf{\beta}}^{i}_{RLS} =  \mathbf{V}\left(\mathbf{D}^2  + \lambda \eye_M\right)^{-1}\mathbf{V}^{\top} \mathbf{V}\mathbf{D}\mathbf{U}^{\top}\bar{\theta}^i.
\end{equation}
The finale expression is
\begin{equation}\label{eq:rls_svd}
\hat{\mathbf{\beta}}^{i}_{RLS} = \mathbf{V}\left(\mathbf{D}^2  + \lambda \eye_M\right)^{-1}\mathbf{D}\mathbf{U}^{\top}\bar{\theta}^i,
\end{equation}
which can be compared to the SVD of OLS regression:
\begin{equation}\label{eq:ols_svd}
\hat{\mathbf{\beta}}^{i}_{OLS} = \mathbf{V}\mathbf{D}^{-1}\mathbf{U}^{\top}\bar{\theta}^i \qquad \left( \approx (\mathbf{A}^{\star})^{-1}\bar{\theta}^i\right).
\end{equation}
It can be seen from \eqref{eq:rls_svd}-\eqref{eq:ols_svd} that in Tikhonov regularisation the inverse of the diagonal matrix is obtained as $\mathbf{D} / (\mathbf{D}^2  + \lambda \eye_M)$ where $\lambda$ is non-negative. Increasing regularisation coefficient thus leads to shrinkage of the singular values, and the estimates asymptotically approach zero. This shows that in Tikhonov regularisation $\lambda$ quantifies the trade-off between the bias and the variance in the estimates. Small regularisation coefficient leads to near-OLS solution that overfits the model to the training data, while large value leads to biased estimates and drives all coefficients to near-zero values.
\subsection{LASSO regularisation}
\par While in the ridge regression LSEs asymptotically approach zero, none of the parameters can be zeroed-out explicitly if the model structure is overly detailed. Another formulation of RLS, called Least absolute shrinkage and selection operator (LASSO) regression, performs variable selection and the regularisation simultaneously . The regularisation term is the 1-norm of the parameter vector:
\begin{equation}
f_{R}(\mathbf{\beta}^i) = \norm{\mathbf{\beta}^i}_{1}.
\end{equation}
The Lagrangian optimisation problem then takes for of basis pursuit denoising that can be solved numerically using quadratic programming or convex techniques. This report uses the shooting algorithm proposed in \cite{Fu1998} because of its relative simplicity.
\begin{remark}
	Results of ridge estimation can be used as the initial point in convex optimisation while solving the LASSO regression. 
\end{remark}
\subsection{Selection of the regularisation coefficient}
\par The important stage of solving RLS problem is selecting the regularisation parameter that will result into a interpretable but parsimonious model. The constraint $\gamma$ in \eqref{eq:rls_constr} is often selected arbitrary since the shape of the parameter space is unknown. As a result, finding $\lambda$ relies on iterative schemes most of which do not guarantee convergence [CITE MANY]. For the lack of universal approach for selecting the optimal value of $\lambda$, the choice of the method remains application-specific.
\par In this report, ridge estimation is applied to a fixed model structure, hence Aikaike’s information criterion (AIC) may be used
\begin{equation}
\text{AIC}_{\lambda} = p + \frac{1}{2}\log p(\bar{\theta}^i \mid \mathbf{\beta}^i, \lambda),
\end{equation}
where the first term quantifies model complexity and the second term is the log-likelihood of the selected model fitting the data. The model complexity is determined as simply the trace of the hat matrix of the ridge estimator
\begin{equation}
p = \tr(\mathbf{H}_{RLS}) = \tr((\mathbf{A}^{\top})(\mathbf{R}_{aa} + \lambda \eye_M)^{-1} \mathbf{A}^{\top})
\end{equation}.
\par Selcting regularisation coefficient for LASSO regression is more nuanced as different model structure may arise for different values of $\lambda$. The most popular approach described in the literature uses cross-validation. Both the data matrix and the response vector are partitioned into pairs. Then each pair is exuded from the 
\section{Results}
\par The results are obtained for the auxetic form data for 8 datasets using both types of regularisation. Ridge traces are shown for values of $\lambda$ from $10^{-6}$ to $10^0$. While the RLS is usually used for the normalised (scaled and centred) data, results for the raw data are presented to show the difference in performances.  
Figures are external 

\begin{figure}[!t]
	\centering
	\subfloat[Normalised data.]{% This file was created by matlab2tikz.
%
\definecolor{mycolor1}{rgb}{0.00000,0.44700,0.74100}%
\definecolor{mycolor2}{rgb}{0.92900,0.69400,0.12500}%
\definecolor{mycolor3}{rgb}{0.46600,0.67400,0.18800}%
\definecolor{mycolor4}{rgb}{0.63500,0.07800,0.18400}%
\definecolor{mycolor5}{rgb}{0.85000,0.32500,0.09800}%
\definecolor{mycolor6}{rgb}{0.49400,0.18400,0.55600}%
\definecolor{mycolor7}{rgb}{0.30100,0.74500,0.93300}%
%
\begin{tikzpicture}

\begin{axis}[%
width=4.927cm,
height=7cm,
at={(0cm,0cm)},
scale only axis,
xmode=log,
xmin=1e-06,
xmax=1,
xminorticks=true,
xlabel style={font=\color{white!15!black}},
xlabel={$\lambda$},
ymin=-40,
ymax=20,
ylabel style={font=\color{white!15!black}},
ylabel={AIC},
axis background/.style={fill=white}
]
\addplot [color=mycolor1, mark size=2.5pt, mark=*, mark options={solid, fill=mycolor1, mycolor1}, forget plot]
  table[row sep=crcr]{%
1e-06	-16.1100762768366\\
1.31825673855641e-06	-16.1098189248134\\
1.73780082874938e-06	-16.1094826141248\\
2.29086765276777e-06	-16.1090443813874\\
3.01995172040202e-06	-16.1084755434187\\
3.98107170553497e-06	-16.1077410367926\\
5.24807460249772e-06	-16.1067993876171\\
6.91830970918936e-06	-16.1056041128175\\
9.1201083935591e-06	-16.1041080282003\\
1.20226443461741e-05	-16.1022731067139\\
1.58489319246111e-05	-16.1000905211595\\
2.08929613085404e-05	-16.0976188396684\\
2.75422870333816e-05	-16.0950538178934\\
3.63078054770102e-05	-16.0928520246899\\
4.78630092322638e-05	-16.0919442962636\\
6.30957344480193e-05	-16.0940959357245\\
8.31763771102671e-05	-16.102501546789\\
0.000109647819614319	-16.1227476545108\\
0.000144543977074593	-16.164344358308\\
0.000190546071796325	-16.243141610637\\
0.000251188643150958	-16.3851789392213\\
0.000331131121482591	-16.6331010165889\\
0.000436515832240166	-17.0579804790174\\
0.000575439937337157	-17.7851837112095\\
0.000758577575029184	-19.067386665534\\
0.001	-21.5965565794556\\
0.00131825673855641	-35.0911523897701\\
0.00173780082874938	-25.1710641102243\\
0.00229086765276777	-20.2245273775358\\
0.00301995172040202	-16.3275866372931\\
0.00398107170553497	-13.9840691809218\\
0.00524807460249772	-12.379457093385\\
0.00691830970918936	-11.292923365024\\
0.00912010839355909	-10.6926644441557\\
0.0120226443461741	-10.7905250318404\\
0.0158489319246111	-14.0209921201432\\
0.0208929613085404	-9.52608069629022\\
0.0275422870333816	-7.34983204024062\\
0.0363078054770102	-6.10824856885074\\
0.0478630092322638	-5.52234561114557\\
0.0630957344480194	-5.84748309960264\\
0.0831763771102671	-10.1542732161172\\
0.109647819614319	-6.40361399619278\\
0.144543977074593	-7.28521889091203\\
0.190546071796325	-3.52432710792105\\
0.251188643150958	-0.602008594015828\\
0.331131121482591	3.87589269227156\\
0.436515832240166	6.61993476019849\\
0.575439937337157	8.5437970016753\\
0.758577575029184	9.78424022840082\\
1	10.1795019440876\\
};
\addplot [color=black, draw=none, mark size=5.0pt, mark=*, mark options={solid, fill=black, black}, forget plot]
  table[row sep=crcr]{%
0.00131825673855641	-35.0911523897701\\
};
\addplot [color=mycolor2, mark size=2.5pt, mark=*, mark options={solid, fill=mycolor2, mycolor2}, forget plot]
  table[row sep=crcr]{%
1e-06	-13.8980525976146\\
1.31825673855641e-06	-13.8978495364948\\
1.73780082874938e-06	-13.8975858278423\\
2.29086765276777e-06	-13.8972450957641\\
3.01995172040202e-06	-13.8968078991094\\
3.98107170553497e-06	-13.8962523242575\\
5.24807460249772e-06	-13.8955559043274\\
6.91830970918936e-06	-13.8947001021323\\
9.1201083935591e-06	-13.893679559871\\
1.20226443461741e-05	-13.892519978185\\
1.58489319246111e-05	-13.8913113078414\\
2.08929613085404e-05	-13.8902676762861\\
2.75422870333816e-05	-13.8898333437394\\
3.63078054770102e-05	-13.8908669341598\\
4.78630092322638e-05	-13.8949573965228\\
6.30957344480193e-05	-13.9049601941747\\
8.31763771102671e-05	-13.9259022349396\\
0.000109647819614319	-13.9665155600023\\
0.000144543977074593	-14.0418973952871\\
0.000190546071796325	-14.1783997451328\\
0.000251188643150958	-14.4237389579762\\
0.000331131121482591	-14.872791282964\\
0.000436515832240166	-15.7623868177245\\
0.000575439937337157	-18.2838461387037\\
0.000758577575029184	-17.1759918374316\\
0.001	-15.5339843798348\\
0.00131825673855641	-15.1918351791075\\
0.00173780082874938	-15.8313323911414\\
0.00229086765276777	-18.897723664605\\
0.00301995172040202	-17.6388015048376\\
0.00398107170553497	-18.9428380213931\\
0.00524807460249772	-15.6964552030909\\
0.00691830970918936	-13.5708051848777\\
0.00912010839355909	-12.5884687077415\\
0.0120226443461741	-12.2451278116172\\
0.0158489319246111	-12.5609895326806\\
0.0208929613085404	-14.3728353812488\\
0.0275422870333816	-14.0525131927774\\
0.0363078054770102	-12.518968562819\\
0.0478630092322638	-14.4479683923377\\
0.0630957344480194	-8.50872694508441\\
0.0831763771102671	-5.81242050651034\\
0.109647819614319	-3.73293667115627\\
0.144543977074593	-2.0574570756355\\
0.190546071796325	-0.909008980900263\\
0.251188643150958	-1.05984005325812\\
0.331131121482591	0.269366566919311\\
0.436515832240166	3.96931435242107\\
0.575439937337157	6.01479718745912\\
0.758577575029184	7.17714320161328\\
1	7.16318140988667\\
};
\addplot [color=black, draw=none, mark size=5.0pt, mark=*, mark options={solid, fill=black, black}, forget plot]
  table[row sep=crcr]{%
0.00398107170553497	-18.9428380213931\\
};
\addplot [color=mycolor3, mark size=2.5pt, mark=*, mark options={solid, fill=mycolor3, mycolor3}, forget plot]
  table[row sep=crcr]{%
1e-06	-11.3065753632887\\
1.31825673855641e-06	-11.3117403229758\\
1.73780082874938e-06	-11.3185673177957\\
2.29086765276777e-06	-11.3275989269301\\
3.01995172040202e-06	-11.3395606709797\\
3.98107170553497e-06	-11.355427083613\\
5.24807460249772e-06	-11.3765149780065\\
6.91830970918936e-06	-11.4046176340227\\
9.1201083935591e-06	-11.4422024929363\\
1.20226443461741e-05	-11.4927110897773\\
1.58489319246111e-05	-11.5610308977734\\
2.08929613085404e-05	-11.6542718258007\\
2.75422870333816e-05	-11.7831188199253\\
3.63078054770102e-05	-11.9643683654985\\
4.78630092322638e-05	-12.2261845782514\\
6.30957344480193e-05	-12.6206709664676\\
8.31763771102671e-05	-13.2615450086331\\
0.000109647819614319	-14.4961353091103\\
0.000144543977074593	-22.765486481837\\
0.000190546071796325	-14.5101784160123\\
0.000251188643150958	-13.381096027612\\
0.000331131121482591	-13.5342381515056\\
0.000436515832240166	-15.4045448515061\\
0.000575439937337157	-10.5461378993776\\
0.000758577575029184	-8.36350448765259\\
0.001	-6.77182378475862\\
0.00131825673855641	-5.5219920158762\\
0.00173780082874938	-4.54676518860262\\
0.00229086765276777	-3.83103579534946\\
0.00301995172040202	-3.38210289466305\\
0.00398107170553497	-3.22551218429557\\
0.00524807460249772	-3.42130174870004\\
0.00691830970918936	-4.13768922581167\\
0.00912010839355909	-6.18892923652345\\
0.0120226443461741	-6.21154397185188\\
0.0158489319246111	-4.16764799916263\\
0.0208929613085404	-3.44388627243779\\
0.0275422870333816	-3.21155244315808\\
0.0363078054770102	-3.30559139253348\\
0.0478630092322638	-3.70337000424458\\
0.0630957344480194	-4.50739413201378\\
0.0831763771102671	-6.16359065947705\\
0.109647819614319	-12.6088240101446\\
0.144543977074593	-14.4962004547636\\
0.190546071796325	-4.43359405988417\\
0.251188643150958	-1.94843788749126\\
0.331131121482591	-0.582495758234165\\
0.436515832240166	-0.172160234407154\\
0.575439937337157	-2.20332769585256\\
0.758577575029184	0.828765652557699\\
1	3.03049831525293\\
};
\addplot [color=black, draw=none, mark size=5.0pt, mark=*, mark options={solid, fill=black, black}, forget plot]
  table[row sep=crcr]{%
0.000144543977074593	-22.765486481837\\
};
\addplot [color=mycolor4, mark size=2.5pt, mark=*, mark options={solid, fill=mycolor4, mycolor4}, forget plot]
  table[row sep=crcr]{%
1e-06	-1.61141575181165\\
1.31825673855641e-06	-1.60614383777842\\
1.73780082874938e-06	-1.59920552152598\\
2.29086765276777e-06	-1.59007880514087\\
3.01995172040202e-06	-1.57808161728085\\
3.98107170553497e-06	-1.56232520836998\\
5.24807460249772e-06	-1.54165575500796\\
6.91830970918936e-06	-1.51458246878715\\
9.1201083935591e-06	-1.47919123968375\\
1.20226443461741e-05	-1.43304449178317\\
1.58489319246111e-05	-1.373071005804\\
2.08929613085404e-05	-1.2954545708079\\
2.75422870333816e-05	-1.19553796869737\\
3.63078054770102e-05	-1.06776897001638\\
4.78630092322638e-05	-0.905726587365882\\
6.30957344480193e-05	-0.702275826776539\\
8.31763771102671e-05	-0.449902639254489\\
0.000109647819614319	-0.141271913389291\\
0.000144543977074593	0.229973770739882\\
0.000190546071796325	0.668197784582196\\
0.000251188643150958	1.17461098998704\\
0.000331131121482591	1.74610672355573\\
0.000436515832240166	2.37429069095774\\
0.000575439937337157	3.04496877397831\\
0.000758577575029184	3.73837746682234\\
0.001	4.4303639867392\\
0.00131825673855641	5.0945013548392\\
0.00173780082874938	5.70478860839593\\
0.00229086765276777	6.23827315342289\\
0.00301995172040202	6.67682936955888\\
0.00398107170553497	7.00754985010526\\
0.00524807460249772	7.22168352415932\\
0.00691830970918936	7.31255386754387\\
0.00912010839355909	7.27317059181334\\
0.0120226443461741	7.09422066181121\\
0.0158489319246111	6.76286272694892\\
0.0208929613085404	6.26238849994169\\
0.0275422870333816	5.57249454947717\\
0.0363078054770102	4.66981344660073\\
0.0478630092322638	3.52853273489049\\
0.0630957344480194	2.1188925011544\\
0.0831763771102671	0.37602150378842\\
0.109647819614319	-2.0940583573224\\
0.144543977074593	-5.83170619304556\\
0.190546071796325	-5.25810509287021\\
0.251188643150958	-1.20490931313747\\
0.331131121482591	1.26777583831775\\
0.436515832240166	2.15826333028945\\
0.575439937337157	1.05676352710522\\
0.758577575029184	0.194218594506204\\
1	3.69451469694044\\
};
\addplot [color=black, draw=none, mark size=5.0pt, mark=*, mark options={solid, fill=black, black}, forget plot]
  table[row sep=crcr]{%
0.144543977074593	-5.83170619304556\\
};
\addplot [color=mycolor5, mark size=2.5pt, mark=*, mark options={solid, fill=mycolor5, mycolor5}, forget plot]
  table[row sep=crcr]{%
1e-06	-26.795462558887\\
1.31825673855641e-06	-26.7973755077172\\
1.73780082874938e-06	-26.7999101576736\\
2.29086765276777e-06	-26.8032739031148\\
3.01995172040202e-06	-26.807747206306\\
3.98107170553497e-06	-26.8137121027131\\
5.24807460249772e-06	-26.8216937078821\\
6.91830970918936e-06	-26.8324218427355\\
9.1201083935591e-06	-26.8469244778048\\
1.20226443461741e-05	-26.8666726423477\\
1.58489319246111e-05	-26.8938105540287\\
2.08929613085404e-05	-26.9315305741986\\
2.75422870333816e-05	-26.9847020139223\\
3.63078054770102e-05	-27.0609631240175\\
4.78630092322638e-05	-27.1727069458779\\
6.30957344480193e-05	-27.3409394074633\\
8.31763771102671e-05	-27.6035709268871\\
0.000109647819614319	-28.0363830761184\\
0.000144543977074593	-28.82343122489\\
0.000190546071796325	-30.6991921477098\\
0.000251188643150958	-31.4321547418727\\
0.000331131121482591	-29.45845627289\\
0.000436515832240166	-31.0830195812822\\
0.000575439937337157	-27.1856067708289\\
0.000758577575029184	-24.3947353209208\\
0.001	-22.5920415134746\\
0.00131825673855641	-21.3353839432455\\
0.00173780082874938	-20.5985596638568\\
0.00229086765276777	-20.6189072640702\\
0.00301995172040202	-23.6806777763727\\
0.00398107170553497	-19.6014943461672\\
0.00524807460249772	-17.6930366716083\\
0.00691830970918936	-16.9604666550719\\
0.00912010839355909	-17.6168559266654\\
0.0120226443461741	-17.7486326353274\\
0.0158489319246111	-14.6656180930344\\
0.0208929613085404	-13.9249384268056\\
0.0275422870333816	-15.3911570667374\\
0.0363078054770102	-10.0192401320121\\
0.0478630092322638	-7.64708558666386\\
0.0630957344480194	-6.14768270707033\\
0.0831763771102671	-5.47370613120231\\
0.109647819614319	-6.33407273547572\\
0.144543977074593	-9.5138649453242\\
0.190546071796325	-2.53630645458039\\
0.251188643150958	1.70508289616184\\
0.331131121482591	4.4615362586715\\
0.436515832240166	6.34142266878095\\
0.575439937337157	7.23017530156692\\
0.758577575029184	2.98538234137804\\
1	10.4332209255353\\
};
\addplot [color=black, draw=none, mark size=5.0pt, mark=*, mark options={solid, fill=black, black}, forget plot]
  table[row sep=crcr]{%
0.000251188643150958	-31.4321547418727\\
};
\addplot [color=mycolor6, mark size=2.5pt, mark=*, mark options={solid, fill=mycolor6, mycolor6}, forget plot]
  table[row sep=crcr]{%
1e-06	7.9639169346924\\
1.31825673855641e-06	7.96348487378778\\
1.73780082874938e-06	7.96291523964822\\
2.29086765276777e-06	7.96216419966892\\
3.01995172040202e-06	7.96117393557898\\
3.98107170553497e-06	7.95986816676222\\
5.24807460249772e-06	7.95814623075252\\
6.91830970918936e-06	7.95587524942909\\
9.1201083935591e-06	7.95287975167475\\
1.20226443461741e-05	7.94892791127148\\
1.58489319246111e-05	7.94371327447688\\
2.08929613085404e-05	7.93683047303083\\
2.75422870333816e-05	7.92774292228991\\
3.63078054770102e-05	7.91573987704793\\
4.78630092322638e-05	7.89987948165752\\
6.30957344480193e-05	7.87891372767061\\
8.31763771102671e-05	7.85119087881601\\
0.000109647819614319	7.81453180796429\\
0.000144543977074593	7.76608070237383\\
0.000190546071796325	7.70214149436431\\
0.000251188643150958	7.61803605486741\\
0.000331131121482591	7.50807016843168\\
0.000436515832240166	7.36578586123128\\
0.000575439937337157	7.18483542534302\\
0.000758577575029184	6.96104647305845\\
0.001	6.69649598585763\\
0.00131825673855641	6.40626451676154\\
0.00173780082874938	6.12651806386078\\
0.00229086765276777	5.91600103542245\\
0.00301995172040202	5.83544143336717\\
0.00398107170553497	5.90465436688017\\
0.00524807460249772	6.07860439109495\\
0.00691830970918936	6.27075307883063\\
0.00912010839355909	6.38543079868361\\
0.0120226443461741	6.31798970243658\\
0.0158489319246111	5.90538515954428\\
0.0208929613085404	4.69181732646577\\
0.0275422870333816	0.804419774097276\\
0.0363078054770102	5.00874061853094\\
0.0478630092322638	5.13716432172071\\
0.0630957344480194	4.25666050053883\\
0.0831763771102671	8.345157207368\\
0.109647819614319	10.2104286827625\\
0.144543977074593	11.4618652433295\\
0.190546071796325	12.3612012755901\\
0.251188643150958	12.9970331408919\\
0.331131121482591	13.3935183751785\\
0.436515832240166	13.5357357446928\\
0.575439937337157	13.3741623282616\\
0.758577575029184	12.8043496220355\\
1	11.5418643802399\\
};
\addplot [color=black, draw=none, mark size=5.0pt, mark=*, mark options={solid, fill=black, black}, forget plot]
  table[row sep=crcr]{%
0.0275422870333816	0.804419774097276\\
};
\addplot [color=mycolor7, mark size=2.5pt, mark=*, mark options={solid, fill=mycolor7, mycolor7}, forget plot]
  table[row sep=crcr]{%
1e-06	6.67394497790529\\
1.31825673855641e-06	6.67356170286963\\
1.73780082874938e-06	6.67305647603675\\
2.29086765276777e-06	6.67239050637585\\
3.01995172040202e-06	6.67151267311915\\
3.98107170553497e-06	6.67035561377951\\
5.24807460249772e-06	6.668830576549\\
6.91830970918936e-06	6.66682065142453\\
9.1201083935591e-06	6.66417187810605\\
1.20226443461741e-05	6.66068158117539\\
1.58489319246111e-05	6.65608309990191\\
2.08929613085404e-05	6.65002586109213\\
2.75422870333816e-05	6.64204949990029\\
3.63078054770102e-05	6.6315505037745\\
4.78630092322638e-05	6.61773973668572\\
6.30957344480193e-05	6.59958941802501\\
8.31763771102671e-05	6.57576916034139\\
0.000109647819614319	6.54457347740217\\
0.000144543977074593	6.50384960390334\\
0.000190546071796325	6.45094780488936\\
0.000251188643150958	6.38274185667109\\
0.000331131121482591	6.29581199401979\\
0.000436515832240166	6.18695147595935\\
0.000575439937337157	6.05424173822662\\
0.000758577575029184	5.89898153555949\\
0.001	5.7285696035523\\
0.00131825673855641	5.55962850762012\\
0.00173780082874938	5.41879856073485\\
0.00229086765276777	5.33665439423117\\
0.00301995172040202	5.3324846665465\\
0.00398107170553497	5.39740136577741\\
0.00524807460249772	5.48880890592715\\
0.00691830970918936	5.53583024834762\\
0.00912010839355909	5.43637886886274\\
0.0120226443461741	5.00828253762011\\
0.0158489319246111	3.67784569148105\\
0.0208929613085404	1.97865064437015\\
0.0275422870333816	5.06433373246439\\
0.0363078054770102	5.73250065636033\\
0.0478630092322638	3.96613111847911\\
0.0630957344480194	6.9412324676327\\
0.0831763771102671	9.47469856226898\\
0.109647819614319	11.0166390450484\\
0.144543977074593	12.1234325318138\\
0.190546071796325	12.9532711734112\\
0.251188643150958	13.5698987887193\\
0.331131121482591	13.9925981809269\\
0.436515832240166	14.2124858810844\\
0.575439937337157	14.2023031790918\\
0.758577575029184	13.9219399263806\\
1	13.3157120841669\\
};
\addplot [color=black, draw=none, mark size=5.0pt, mark=*, mark options={solid, fill=black, black}, forget plot]
  table[row sep=crcr]{%
0.0208929613085404	1.97865064437015\\
};
\end{axis}

\begin{axis}[%
width=4.927cm,
height=7cm,
at={(6.484cm,0cm)},
scale only axis,
xmode=log,
xmin=1e-06,
xmax=1,
xminorticks=true,
xlabel style={font=\color{white!15!black}},
xlabel={$\lambda$},
ymin=-20,
ymax=40,
ylabel style={font=\color{white!15!black}},
ylabel={ICOMP},
axis background/.style={fill=white}
]
\addplot [color=mycolor1, mark size=2.5pt, mark=*, mark options={solid, fill=mycolor1, mycolor1}, forget plot]
  table[row sep=crcr]{%
1e-06	10.0084563717418\\
1.31825673855641e-06	10.0024148898314\\
1.73780082874938e-06	9.99446745386215\\
2.29086765276777e-06	9.98401968727037\\
3.01995172040202e-06	9.97029682592348\\
3.98107170553497e-06	9.95229249730779\\
5.24807460249772e-06	9.92870540778212\\
6.91830970918936e-06	9.89786276780424\\
9.1201083935591e-06	9.85763038840068\\
1.20226443461741e-05	9.80531138397484\\
1.58489319246111e-05	9.7375384884137\\
2.08929613085404e-05	9.65016884872573\\
2.75422870333816e-05	9.53819348121193\\
3.63078054770102e-05	9.39567297402082\\
4.78630092322638e-05	9.21570011694581\\
6.30957344480193e-05	8.9903594364347\\
8.31763771102671e-05	8.71059289170815\\
0.000109647819614319	8.36578496125191\\
0.000144543977074593	7.94275396545013\\
0.000190546071796325	7.42368701518956\\
0.000251188643150958	6.78234621122786\\
0.000331131121482591	5.97738449850047\\
0.000436515832240166	4.94003191310995\\
0.000575439937337157	3.54771514616476\\
0.000758577575029184	1.55069914067486\\
0.001	-1.73997814978454\\
0.00131825673855641	-16.0395946555737\\
0.00173780082874938	-6.96454245658777\\
0.00229086765276777	-2.89923462822476\\
0.00301995172040202	0.084232740117848\\
0.00398107170553497	1.48574273282072\\
0.00524807460249772	2.12288046391217\\
0.00691830970918936	2.21859853958899\\
0.00912010839355909	1.80580989766045\\
0.0120226443461741	0.672954419730166\\
0.0158489319246111	-3.61459154225008\\
0.0208929613085404	-0.198918419854648\\
0.0275422870333816	0.876323401902532\\
0.0363078054770102	0.99620604353821\\
0.0478630092322638	0.441476217217405\\
0.0630957344480194	-1.04093686534709\\
0.0831763771102671	-6.51907254294927\\
0.109647819614319	-3.95122789998872\\
0.144543977074593	-6.0247157567878\\
0.190546071796325	-3.46276662303268\\
0.251188643150958	-1.74499366302481\\
0.331131121482591	1.52353340211601\\
0.436515832240166	3.05337693642035\\
0.575439937337157	3.75743971078027\\
0.758577575029184	3.77102940361651\\
1	2.93049265072023\\
};
\addplot [color=black, draw=none, mark size=5.0pt, mark=*, mark options={solid, fill=black, black}, forget plot]
  table[row sep=crcr]{%
0.00131825673855641	-16.0395946555737\\
};
\addplot [color=mycolor2, mark size=2.5pt, mark=*, mark options={solid, fill=mycolor2, mycolor2}, forget plot]
  table[row sep=crcr]{%
1e-06	12.2204800509638\\
1.31825673855641e-06	12.2143842781499\\
1.73780082874938e-06	12.2063642401447\\
2.29086765276777e-06	12.1958189728936\\
3.01995172040202e-06	12.1819644702328\\
3.98107170553497e-06	12.1637812098429\\
5.24807460249772e-06	12.1399488910718\\
6.91830970918936e-06	12.1087667784894\\
9.1201083935591e-06	12.0680588567301\\
1.20226443461741e-05	12.0150645125037\\
1.58489319246111e-05	11.9463177017318\\
2.08929613085404e-05	11.857520012108\\
2.75422870333816e-05	11.7434139553659\\
3.63078054770102e-05	11.5976580645509\\
4.78630092322638e-05	11.4126870166866\\
6.30957344480193e-05	11.1794951779845\\
8.31763771102671e-05	10.8871922035575\\
0.000109647819614319	10.5220170557605\\
0.000144543977074593	10.065200928471\\
0.000190546071796325	9.48842888069381\\
0.000251188643150958	8.74378619247295\\
0.000331131121482591	7.73769423212541\\
0.000436515832240166	6.2356255744029\\
0.000575439937337157	3.04905271867056\\
0.000758577575029184	3.44209396877719\\
0.001	4.32259404983631\\
0.00131825673855641	3.8597225550889\\
0.00173780082874938	2.37518926249506\\
0.00229086765276777	-1.57243091529397\\
0.00301995172040202	-1.22698212742664\\
0.00398107170553497	-3.47302610765059\\
0.00524807460249772	-1.1941176457937\\
0.00691830970918936	-0.0592832802647081\\
0.00912010839355909	-0.0899943659253708\\
0.0120226443461741	-0.781648360046683\\
0.0158489319246111	-2.15458895478751\\
0.0208929613085404	-5.04567310481327\\
0.0275422870333816	-5.82635775063424\\
0.0363078054770102	-5.41451395043007\\
0.0478630092322638	-8.48414656397468\\
0.0630957344480194	-3.70218071082887\\
0.0831763771102671	-2.17721983334242\\
0.109647819614319	-1.28055057495222\\
0.144543977074593	-0.796953941511266\\
0.190546071796325	-0.847448496011897\\
0.251188643150958	-2.20282512226711\\
0.331131121482591	-2.08299272323624\\
0.436515832240166	0.402756528642929\\
0.575439937337157	1.22843989656409\\
0.758577575029184	1.16393237682898\\
1	-0.085827883480726\\
};
\addplot [color=black, draw=none, mark size=5.0pt, mark=*, mark options={solid, fill=black, black}, forget plot]
  table[row sep=crcr]{%
0.0478630092322638	-8.48414656397468\\
};
\addplot [color=mycolor3, mark size=2.5pt, mark=*, mark options={solid, fill=mycolor3, mycolor3}, forget plot]
  table[row sep=crcr]{%
1e-06	14.8119572852897\\
1.31825673855641e-06	14.800493491669\\
1.73780082874938e-06	14.7853827501912\\
2.29086765276777e-06	14.7654651417276\\
3.01995172040202e-06	14.7392116983625\\
3.98107170553497e-06	14.7046064504874\\
5.24807460249772e-06	14.6589898173927\\
6.91830970918936e-06	14.598849246599\\
9.1201083935591e-06	14.5195359236647\\
1.20226443461741e-05	14.4148734009114\\
1.58489319246111e-05	14.2765981117997\\
2.08929613085404e-05	14.0935158625934\\
2.75422870333816e-05	13.85012847918\\
3.63078054770102e-05	13.5241566332122\\
4.78630092322638e-05	13.081459834958\\
6.30957344480193e-05	12.4637844056917\\
8.31763771102671e-05	11.5515494298641\\
0.000109647819614319	9.99239730665244\\
0.000144543977074593	1.34161184192113\\
0.000190546071796325	9.15665020981429\\
0.000251188643150958	9.78642912283719\\
0.000331131121482591	9.07624736358379\\
0.000436515832240166	6.59346754062122\\
0.000575439937337157	10.7867609579966\\
0.000758577575029184	12.2545813185562\\
0.001	13.0847546449125\\
0.00131825673855641	13.5295657183202\\
0.00173780082874938	13.6597564650339\\
0.00229086765276777	13.4942569539616\\
0.00301995172040202	13.0297164827479\\
0.00398107170553497	12.244299729447\\
0.00524807460249772	11.0810358085971\\
0.00691830970918936	9.37383267880133\\
0.00912010839355909	6.30954510529269\\
0.0120226443461741	5.25193547971867\\
0.0158489319246111	6.23875257873048\\
0.0208929613085404	5.88327600399778\\
0.0275422870333816	5.01460299898508\\
0.0363078054770102	3.79886321985547\\
0.0478630092322638	2.2604518241184\\
0.0630957344480194	0.299152102241763\\
0.0831763771102671	-2.52838998630913\\
0.109647819614319	-10.1564379139405\\
0.144543977074593	-13.2356973206394\\
0.190546071796325	-4.3720335749958\\
0.251188643150958	-3.09142295650024\\
0.331131121482591	-2.93485504838972\\
0.436515832240166	-3.7387180581853\\
0.575439937337157	-6.98968498674759\\
0.758577575029184	-5.18444517222661\\
1	-4.21851097811447\\
};
\addplot [color=black, draw=none, mark size=5.0pt, mark=*, mark options={solid, fill=black, black}, forget plot]
  table[row sep=crcr]{%
0.144543977074593	-13.2356973206394\\
};
\addplot [color=mycolor4, mark size=2.5pt, mark=*, mark options={solid, fill=mycolor4, mycolor4}, forget plot]
  table[row sep=crcr]{%
1e-06	24.5071168967668\\
1.31825673855641e-06	24.5060899768664\\
1.73780082874938e-06	24.504744546461\\
2.29086765276777e-06	24.5029852635169\\
3.01995172040202e-06	24.5006907520613\\
3.98107170553497e-06	24.4977083257304\\
5.24807460249772e-06	24.4938490403912\\
6.91830970918936e-06	24.4888844118346\\
9.1201083935591e-06	24.4825471769173\\
1.20226443461741e-05	24.4745399989056\\
1.58489319246111e-05	24.4645580037692\\
2.08929613085404e-05	24.4523331175862\\
2.75422870333816e-05	24.4377093304079\\
3.63078054770102e-05	24.4207560286943\\
4.78630092322638e-05	24.4019178258435\\
6.30957344480193e-05	24.3821795453827\\
8.31763771102671e-05	24.3631917992427\\
0.000109647819614319	24.3472607023734\\
0.000144543977074593	24.337072094498\\
0.000190546071796325	24.3350264104088\\
0.000251188643150958	24.3421361404362\\
0.000331131121482591	24.3565922386451\\
0.000436515832240166	24.3723030830851\\
0.000575439937337157	24.3778676313525\\
0.000758577575029184	24.3564632730312\\
0.001	24.2869424164103\\
0.00131825673855641	24.1460590890356\\
0.00173780082874938	23.9113102620324\\
0.00229086765276777	23.5635659027339\\
0.00301995172040202	23.0886487469698\\
0.00398107170553497	22.4773617638478\\
0.00524807460249772	21.7240210814565\\
0.00691830970918936	20.8240757721569\\
0.00912010839355909	19.7716449336295\\
0.0120226443461741	18.5577001133818\\
0.0158489319246111	17.169263304842\\
0.0208929613085404	15.5895507763773\\
0.0275422870333816	13.7986499916203\\
0.0363078054770102	11.7742680589897\\
0.0478630092322638	9.49235456325347\\
0.0630957344480194	6.92543873540994\\
0.0831763771102671	4.01122217695634\\
0.109647819614319	0.358327738881659\\
0.144543977074593	-4.57120305892132\\
0.190546071796325	-5.19654460798185\\
0.251188643150958	-2.34789438214645\\
0.331131121482591	-1.08458345183781\\
0.436515832240166	-1.40829449348869\\
0.575439937337157	-3.72959376378982\\
0.758577575029184	-5.8189922302781\\
1	-3.55449459642696\\
};
\addplot [color=black, draw=none, mark size=5.0pt, mark=*, mark options={solid, fill=black, black}, forget plot]
  table[row sep=crcr]{%
0.758577575029184	-5.8189922302781\\
};
\addplot [color=mycolor5, mark size=2.5pt, mark=*, mark options={solid, fill=mycolor5, mycolor5}, forget plot]
  table[row sep=crcr]{%
1e-06	-0.676929910308552\\
1.31825673855641e-06	-0.685141693072431\\
1.73780082874938e-06	-0.69596008968669\\
2.29086765276777e-06	-0.71020983445705\\
3.01995172040202e-06	-0.728974836963786\\
3.98107170553497e-06	-0.753678568612671\\
5.24807460249772e-06	-0.786188912482942\\
6.91830970918936e-06	-0.82895496211383\\
9.1201083935591e-06	-0.885186061203814\\
1.20226443461741e-05	-0.959088151659007\\
1.58489319246111e-05	-1.05618154445548\\
2.08929613085404e-05	-1.18374288580447\\
2.75422870333816e-05	-1.35145471481703\\
3.63078054770102e-05	-1.57243812530685\\
4.78630092322638e-05	-1.86506253266849\\
6.30957344480193e-05	-2.25648403530406\\
8.31763771102671e-05	-2.79047648838993\\
0.000109647819614319	-3.54785046035565\\
0.000144543977074593	-4.71633290113187\\
0.000190546071796325	-7.03236352188325\\
0.000251188643150958	-8.26462959142353\\
0.000331131121482591	-6.84797075780067\\
0.000436515832240166	-9.08500718915479\\
0.000575439937337157	-5.85270791345468\\
0.000758577575029184	-3.776649514712\\
0.001	-2.73546308380351\\
0.00131825673855641	-2.28382620904905\\
0.00173780082874938	-2.39203801022035\\
0.00229086765276777	-3.29361451475915\\
0.00301995172040202	-7.26885839896175\\
0.00398107170553497	-4.13168243242471\\
0.00524807460249772	-3.1906991143111\\
0.00691830970918936	-3.44894475045894\\
0.00912010839355909	-5.11838158484928\\
0.0120226443461741	-6.28515318375683\\
0.0158489319246111	-4.2592175151413\\
0.0208929613085404	-4.59777615037002\\
0.0275422870333816	-7.16500162459428\\
0.0363078054770102	-2.91478551962316\\
0.0478630092322638	-1.68326375830088\\
0.0630957344480194	-1.34113647281479\\
0.0831763771102671	-1.83850545803439\\
0.109647819614319	-3.88168663927166\\
0.144543977074593	-8.25336181119997\\
0.190546071796325	-2.47474596969202\\
0.251188643150958	0.562097827152859\\
0.331131121482591	2.10917696851594\\
0.436515832240166	2.77486484500281\\
0.575439937337157	2.44381801067189\\
0.758577575029184	-3.02782848340627\\
1	3.18421163216791\\
};
\addplot [color=black, draw=none, mark size=5.0pt, mark=*, mark options={solid, fill=black, black}, forget plot]
  table[row sep=crcr]{%
0.000436515832240166	-9.08500718915479\\
};
\addplot [color=mycolor6, mark size=2.5pt, mark=*, mark options={solid, fill=mycolor6, mycolor6}, forget plot]
  table[row sep=crcr]{%
1e-06	34.0824495832708\\
1.31825673855641e-06	34.0757186884326\\
1.73780082874938e-06	34.0668653076352\\
2.29086765276777e-06	34.0552282683266\\
3.01995172040202e-06	34.0399463049212\\
3.98107170553497e-06	34.0199017008626\\
5.24807460249772e-06	33.9936510261517\\
6.91830970918936e-06	33.9593421300508\\
9.1201083935591e-06	33.9146181682758\\
1.20226443461741e-05	33.8565124019602\\
1.58489319246111e-05	33.7813422840501\\
2.08929613085404e-05	33.6846181614249\\
2.75422870333816e-05	33.5609902213952\\
3.63078054770102e-05	33.4042648757586\\
4.78630092322638e-05	33.2075238948669\\
6.30957344480193e-05	32.9633690998298\\
8.31763771102671e-05	32.6642853173132\\
0.000109647819614319	32.303064423727\\
0.000144543977074593	31.873179026132\\
0.000190546071796325	31.3689701201909\\
0.000251188643150958	30.7855612053166\\
0.000331131121482591	30.1185556835211\\
0.000436515832240166	29.3637982533586\\
0.000575439937337157	28.5177342827172\\
0.000758577575029184	27.5791322792673\\
0.001	26.5530744155287\\
0.00131825673855641	25.457822250958\\
0.00173780082874938	24.3330397174973\\
0.00229086765276777	23.2412937847335\\
0.00301995172040202	22.2472608107781\\
0.00398107170553497	21.3744662806227\\
0.00524807460249772	20.5809419483921\\
0.00691830970918936	19.7822749834436\\
0.00912010839355909	18.8839051404998\\
0.0120226443461741	17.7814691540071\\
0.0158489319246111	16.3117857374374\\
0.0208929613085404	14.0189796029013\\
0.0275422870333816	9.03057521624043\\
0.0363078054770102	12.1131952309199\\
0.0478630092322638	11.1009861500837\\
0.0630957344480194	9.06320673479438\\
0.0831763771102671	11.9803578805359\\
0.109647819614319	12.6628147789666\\
0.144543977074593	12.7223683774537\\
0.190546071796325	12.4227617604785\\
0.251188643150958	11.8540480718829\\
0.331131121482591	11.0411590850229\\
0.436515832240166	9.96917792091468\\
0.575439937337157	8.58780503736656\\
0.758577575029184	6.79113879725118\\
1	4.29285508687246\\
};
\addplot [color=black, draw=none, mark size=5.0pt, mark=*, mark options={solid, fill=black, black}, forget plot]
  table[row sep=crcr]{%
1	4.29285508687246\\
};
\addplot [color=mycolor7, mark size=2.5pt, mark=*, mark options={solid, fill=mycolor7, mycolor7}, forget plot]
  table[row sep=crcr]{%
1e-06	32.7924776264837\\
1.31825673855641e-06	32.7857955175144\\
1.73780082874938e-06	32.7770065440237\\
2.29086765276777e-06	32.7654545750336\\
3.01995172040202e-06	32.7502850424613\\
3.98107170553497e-06	32.7303891478799\\
5.24807460249772e-06	32.7043353719482\\
6.91830970918936e-06	32.6702875320462\\
9.1201083935591e-06	32.6259102947071\\
1.20226443461741e-05	32.5682660718641\\
1.58489319246111e-05	32.4937121094751\\
2.08929613085404e-05	32.3978135494862\\
2.75422870333816e-05	32.2752967990056\\
3.63078054770102e-05	32.1200755024852\\
4.78630092322638e-05	31.9253841498951\\
6.30957344480193e-05	31.6840447901842\\
8.31763771102671e-05	31.3888635988386\\
0.000109647819614319	31.0331060931649\\
0.000144543977074593	30.6109479276615\\
0.000190546071796325	30.1177764307159\\
0.000251188643150958	29.5502670071203\\
0.000331131121482591	28.9062975091092\\
0.000436515832240166	28.1849638680867\\
0.000575439937337157	27.3871405956008\\
0.000758577575029184	26.5170673417683\\
0.001	25.5851480332234\\
0.00131825673855641	24.6111862418166\\
0.00173780082874938	23.6253202143713\\
0.00229086765276777	22.6619471435422\\
0.00301995172040202	21.7443040439574\\
0.00398107170553497	20.8672132795199\\
0.00524807460249772	19.9911464632243\\
0.00691830970918936	19.0473521529606\\
0.00912010839355909	17.9348532106789\\
0.0120226443461741	16.4717619891907\\
0.0158489319246111	14.0842462693742\\
0.0208929613085404	11.3058129208057\\
0.0275422870333816	13.2904891746076\\
0.0363078054770102	12.8369552687493\\
0.0478630092322638	9.92995294684208\\
0.0630957344480194	11.7477787018882\\
0.0831763771102671	13.1098992354369\\
0.109647819614319	13.4690251412524\\
0.144543977074593	13.3839356659381\\
0.190546071796325	13.0148316582995\\
0.251188643150958	12.4269137197103\\
0.331131121482591	11.6402388907714\\
0.436515832240166	10.6459280573062\\
0.575439937337157	9.4159458881968\\
0.758577575029184	7.90872910159633\\
1	6.06670279079951\\
};
\addplot [color=black, draw=none, mark size=5.0pt, mark=*, mark options={solid, fill=black, black}, forget plot]
  table[row sep=crcr]{%
1	6.06670279079951\\
};
\end{axis}
\end{tikzpicture}%\label{fig:AIC_norm}}\\
	\subfloat[Raw data.]{% This file was created by matlab2tikz.
%
\definecolor{mycolor1}{rgb}{0.00000,0.44700,0.74100}%
\definecolor{mycolor2}{rgb}{0.92900,0.69400,0.12500}%
\definecolor{mycolor3}{rgb}{0.46600,0.67400,0.18800}%
\definecolor{mycolor4}{rgb}{0.63500,0.07800,0.18400}%
\definecolor{mycolor5}{rgb}{0.85000,0.32500,0.09800}%
\definecolor{mycolor6}{rgb}{0.49400,0.18400,0.55600}%
\definecolor{mycolor7}{rgb}{0.30100,0.74500,0.93300}%
%
\begin{tikzpicture}

\begin{axis}[%
width=11.411cm,
height=7cm,
at={(0cm,0cm)},
scale only axis,
xmode=log,
xmin=1e-06,
xmax=1,
xminorticks=true,
xlabel style={font=\color{white!15!black}},
xlabel={$\gamma$},
ymin=0,
ymax=140,
ylabel style={font=\color{white!15!black}},
ylabel={AIC},
axis background/.style={fill=white}
]
\addplot [color=mycolor1, mark size=2.5pt, mark=*, mark options={solid, fill=mycolor1, mycolor1}, forget plot]
  table[row sep=crcr]{%
1e-06	94.9764908085287\\
1.31825673855641e-06	94.8909899232243\\
1.73780082874938e-06	94.7866747688406\\
2.29086765276777e-06	94.6617447085826\\
3.01995172040202e-06	94.5152450052637\\
3.98107170553497e-06	94.3473383624201\\
5.24807460249772e-06	94.159351815712\\
6.91830970918936e-06	93.9534752862609\\
9.1201083935591e-06	93.732090096189\\
1.20226443461741e-05	93.4968599104392\\
1.58489319246111e-05	93.2478415024529\\
2.08929613085404e-05	92.9828927839117\\
2.75422870333816e-05	92.6975633921735\\
3.63078054770102e-05	92.3855140301106\\
4.78630092322638e-05	92.0393981409025\\
6.30957344480193e-05	91.652078474563\\
8.31763771102671e-05	91.2180187346834\\
0.000109647819614319	90.7346584402688\\
0.000144543977074593	90.2035463783674\\
0.000190546071796325	89.630997949052\\
0.000251188643150958	89.0280776962381\\
0.000331131121482591	88.4098026796279\\
0.000436515832240166	87.7936156955351\\
0.000575439937337157	87.1973688414783\\
0.000758577575029184	86.6372230369007\\
0.001	86.1259181658186\\
0.00131825673855641	85.6717555321954\\
0.00173780082874938	85.2784066903627\\
0.00229086765276777	84.9454299956607\\
0.00301995172040202	84.6692316347683\\
0.00398107170553497	84.444179775086\\
0.00524807460249772	84.2636400720454\\
0.00691830970918936	84.1207979352232\\
0.00912010839355909	84.0092249811289\\
0.0120226443461741	83.9232110296169\\
0.0158489319246111	83.8579147018177\\
0.0208929613085404	83.8093923563806\\
0.0275422870333816	83.7745574607484\\
0.0363078054770102	83.7511095302559\\
0.0478630092322638	83.7374589971969\\
0.0630957344480194	83.7326642136822\\
0.0831763771102671	83.7363896457646\\
0.109647819614319	83.7488896394946\\
0.144543977074593	83.7710190028838\\
0.190546071796325	83.8042690819267\\
0.251188643150958	83.8508250690791\\
0.331131121482591	83.9136360974626\\
0.436515832240166	83.9964835579425\\
0.575439937337157	84.1040248418364\\
0.758577575029184	84.2417802430048\\
1	84.4160227478024\\
};
\addplot [color=black, draw=none, mark size=5.0pt, mark=*, mark options={solid, fill=black, black}, forget plot]
  table[row sep=crcr]{%
0.0630957344480194	83.7326642136822\\
};
\addplot [color=mycolor2, mark size=2.5pt, mark=*, mark options={solid, fill=mycolor2, mycolor2}, forget plot]
  table[row sep=crcr]{%
1e-06	110.955848742877\\
1.31825673855641e-06	110.873596680614\\
1.73780082874938e-06	110.773197004626\\
2.29086765276777e-06	110.652879747365\\
3.01995172040202e-06	110.511670411422\\
3.98107170553497e-06	110.349646933135\\
5.24807460249772e-06	110.167982590997\\
6.91830970918936e-06	109.968658691201\\
9.1201083935591e-06	109.753830018416\\
1.20226443461741e-05	109.524974821862\\
1.58489319246111e-05	109.282080745543\\
2.08929613085404e-05	109.023135829721\\
2.75422870333816e-05	108.744102485456\\
3.63078054770102e-05	108.43941421804\\
4.78630092322638e-05	108.102919231125\\
6.30957344480193e-05	107.729125986309\\
8.31763771102671e-05	107.314564084513\\
0.000109647819614319	106.859037365876\\
0.000144543977074593	106.366519022476\\
0.000190546071796325	105.845447344556\\
0.000251188643150958	105.308252535782\\
0.000331131121482591	104.770093987698\\
0.000436515832240166	104.246999206094\\
0.000575439937337157	103.753805338042\\
0.000758577575029184	103.302399364999\\
0.001	102.900651898176\\
0.00131825673855641	102.552183068751\\
0.00173780082874938	102.256831790902\\
0.00229086765276777	102.01154798344\\
0.00301995172040202	101.811415831654\\
0.00398107170553497	101.650592304612\\
0.00524807460249772	101.523044943537\\
0.00691830970918936	101.423057035688\\
0.00912010839355909	101.34552258385\\
0.0120226443461741	101.2860790581\\
0.0158489319246111	101.241130627111\\
0.0208929613085404	101.207807521667\\
0.0275422870333816	101.183895779338\\
0.0363078054770102	101.167760425118\\
0.0478630092322638	101.158276227655\\
0.0630957344480194	101.154773976989\\
0.0831763771102671	101.157006364809\\
0.109647819614319	101.165135372404\\
0.144543977074593	101.179741905613\\
0.190546071796325	101.201857658189\\
0.251188643150958	101.233018266596\\
0.331131121482591	101.275335155532\\
0.436515832240166	101.331580422284\\
0.575439937337157	101.405274007563\\
0.758577575029184	101.500754763961\\
1	101.623207013483\\
};
\addplot [color=black, draw=none, mark size=5.0pt, mark=*, mark options={solid, fill=black, black}, forget plot]
  table[row sep=crcr]{%
0.0630957344480194	101.154773976989\\
};
\addplot [color=mycolor3, mark size=2.5pt, mark=*, mark options={solid, fill=mycolor3, mycolor3}, forget plot]
  table[row sep=crcr]{%
1e-06	46.3357812817728\\
1.31825673855641e-06	46.2559911556683\\
1.73780082874938e-06	46.1592712014049\\
2.29086765276777e-06	46.0443553972012\\
3.01995172040202e-06	45.9108908521017\\
3.98107170553497e-06	45.7596640187562\\
5.24807460249772e-06	45.5925812688893\\
6.91830970918936e-06	45.4122924546694\\
9.1201083935591e-06	45.2214504666274\\
1.20226443461741e-05	45.021741315964\\
1.58489319246111e-05	44.8129197528977\\
2.08929613085404e-05	44.5920858602582\\
2.75422870333816e-05	44.3533501005388\\
3.63078054770102e-05	44.0879273393806\\
4.78630092322638e-05	43.7846323379352\\
6.30957344480193e-05	43.4307183762282\\
8.31763771102671e-05	43.0129606034074\\
0.000109647819614319	42.518801286726\\
0.000144543977074593	41.9372577843058\\
0.000190546071796325	41.2591800525253\\
0.000251188643150958	40.4763075486193\\
0.000331131121482591	39.5782202139262\\
0.000436515832240166	38.5450390621125\\
0.000575439937337157	37.3288672897452\\
0.000758577575029184	35.7920173732483\\
0.001	33.3303212930682\\
0.00131825673855641	31.604355620276\\
0.00173780082874938	33.7285381444413\\
0.00229086765276777	34.3275208842406\\
0.00301995172040202	34.5696596135909\\
0.00398107170553497	34.6675750800307\\
0.00524807460249772	34.6984845200704\\
0.00691830970918936	34.6976138430131\\
0.00912010839355909	34.6825827864295\\
0.0120226443461741	34.6623972322233\\
0.0158489319246111	34.6415135487525\\
0.0208929613085404	34.6219133257168\\
0.0275422870333816	34.6042308866151\\
0.0363078054770102	34.5883771098254\\
0.0478630092322638	34.5738771902434\\
0.0630957344480194	34.56003956704\\
0.0831763771102671	34.5460214488379\\
0.109647819614319	34.5308260853375\\
0.144543977074593	34.5132469046255\\
0.190546071796325	34.4917577381902\\
0.251188643150958	34.4643316014363\\
0.331131121482591	34.4281464170399\\
0.436515832240166	34.3790927172866\\
0.575439937337157	34.3109085094697\\
0.758577575029184	34.2135584493005\\
1	34.0699315256439\\
};
\addplot [color=black, draw=none, mark size=5.0pt, mark=*, mark options={solid, fill=black, black}, forget plot]
  table[row sep=crcr]{%
0.00131825673855641	31.604355620276\\
};
\addplot [color=mycolor4, mark size=2.5pt, mark=*, mark options={solid, fill=mycolor4, mycolor4}, forget plot]
  table[row sep=crcr]{%
1e-06	19.620602728182\\
1.31825673855641e-06	19.4883251038547\\
1.73780082874938e-06	19.3261047744484\\
2.29086765276777e-06	19.1308068896155\\
3.01995172040202e-06	18.9007925909081\\
3.98107170553497e-06	18.6367310546416\\
5.24807460249772e-06	18.3423242999593\\
6.91830970918936e-06	18.0247624587237\\
9.1201083935591e-06	17.694782629069\\
1.20226443461741e-05	17.366308841905\\
1.58489319246111e-05	17.0557044356913\\
2.08929613085404e-05	16.7805691411625\\
2.75422870333816e-05	16.5577992742058\\
3.63078054770102e-05	16.4005993303825\\
4.78630092322638e-05	16.3147155794731\\
6.30957344480193e-05	16.2952942316351\\
8.31763771102671e-05	16.3263325898207\\
0.000109647819614319	16.3835314035761\\
0.000144543977074593	16.4392503797153\\
0.000190546071796325	16.4673162193333\\
0.000251188643150958	16.4462082381872\\
0.000331131121482591	16.3603732222788\\
0.000436515832240166	16.2000530402427\\
0.000575439937337157	15.960068082678\\
0.000758577575029184	15.6378668010203\\
0.001	15.2309755401037\\
0.00131825673855641	14.7336545326062\\
0.00173780082874938	14.1317560808804\\
0.00229086765276777	13.3923889798909\\
0.00301995172040202	12.4356046067073\\
0.00398107170553497	11.0182144330543\\
0.00524807460249772	7.40119227847827\\
0.00691830970918936	9.64492052989101\\
0.00912010839355909	11.0757072722776\\
0.0120226443461741	11.7472795611716\\
0.0158489319246111	12.1467950907257\\
0.0208929613085404	12.4080609671109\\
0.0275422870333816	12.5875900095105\\
0.0363078054770102	12.7145613122297\\
0.0478630092322638	12.805909379075\\
0.0630957344480194	12.8722182774281\\
0.0831763771102671	12.9204284238643\\
0.109647819614319	12.955226881354\\
0.144543977074593	12.9798210862724\\
0.190546071796325	12.9963940882861\\
0.251188643150958	13.006381001501\\
0.331131121482591	13.0106368550163\\
0.436515832240166	13.0095326158871\\
0.575439937337157	13.0029985901451\\
0.758577575029184	12.9905240968815\\
1	12.9711151887367\\
};
\addplot [color=black, draw=none, mark size=5.0pt, mark=*, mark options={solid, fill=black, black}, forget plot]
  table[row sep=crcr]{%
0.00524807460249772	7.40119227847827\\
};
\addplot [color=mycolor5, mark size=2.5pt, mark=*, mark options={solid, fill=mycolor5, mycolor5}, forget plot]
  table[row sep=crcr]{%
1e-06	52.0082903050912\\
1.31825673855641e-06	52.390666213866\\
1.73780082874938e-06	52.7230974458758\\
2.29086765276777e-06	52.967757820456\\
3.01995172040202e-06	53.0623345541823\\
3.98107170553497e-06	52.8810234317905\\
5.24807460249772e-06	52.0411409488737\\
6.91830970918936e-06	46.9433179182162\\
9.1201083935591e-06	53.2981760478533\\
1.20226443461741e-05	55.0394952918322\\
1.58489319246111e-05	56.1590166153486\\
2.08929613085404e-05	56.9626947004026\\
2.75422870333816e-05	57.5433197098502\\
3.63078054770102e-05	57.9361233589992\\
4.78630092322638e-05	58.151435561748\\
6.30957344480193e-05	58.1823734715749\\
8.31763771102671e-05	58.0002251986036\\
0.000109647819614319	57.5287382165398\\
0.000144543977074593	56.5289418900206\\
0.000190546071796325	53.2703615097473\\
0.000251188643150958	55.8006955735973\\
0.000331131121482591	57.4730491142577\\
0.000436515832240166	58.3371773833032\\
0.000575439937337157	58.8871325935763\\
0.000758577575029184	59.263596739586\\
0.001	59.5298935467937\\
0.00131825673855641	59.7215023079635\\
0.00173780082874938	59.8608070597095\\
0.00229086765276777	59.9628523078755\\
0.00301995172040202	60.0380871067033\\
0.00398107170553497	60.0938983853377\\
0.00524807460249772	60.1355676878595\\
0.00691830970918936	60.1669073274043\\
0.00912010839355909	60.1906970268186\\
0.0120226443461741	60.2089884654038\\
0.0158489319246111	60.2233201460002\\
0.0208929613085404	60.2348710363081\\
0.0275422870333816	60.2445725713988\\
0.0363078054770102	60.2531926241591\\
0.0478630092322638	60.2614009466681\\
0.0630957344480194	60.2698227527412\\
0.0831763771102671	60.2790850873142\\
0.109647819614319	60.2898590565328\\
0.144543977074593	60.3028995247631\\
0.190546071796325	60.3190821739946\\
0.251188643150958	60.3394354663235\\
0.331131121482591	60.3651615919113\\
0.436515832240166	60.3976354548377\\
0.575439937337157	60.4383638402293\\
0.758577575029184	60.4888782993831\\
1	60.5505261122413\\
};
\addplot [color=black, draw=none, mark size=5.0pt, mark=*, mark options={solid, fill=black, black}, forget plot]
  table[row sep=crcr]{%
6.91830970918936e-06	46.9433179182162\\
};
\addplot [color=mycolor6, mark size=2.5pt, mark=*, mark options={solid, fill=mycolor6, mycolor6}, forget plot]
  table[row sep=crcr]{%
1e-06	124.904406714506\\
1.31825673855641e-06	124.817844092436\\
1.73780082874938e-06	124.712170050989\\
2.29086765276777e-06	124.585522819044\\
3.01995172040202e-06	124.43688851506\\
3.98107170553497e-06	124.266380637668\\
5.24807460249772e-06	124.075294367976\\
6.91830970918936e-06	123.865807400752\\
9.1201083935591e-06	123.64030034642\\
1.20226443461741e-05	123.400425925219\\
1.58489319246111e-05	123.146185309356\\
2.08929613085404e-05	122.875294004854\\
2.75422870333816e-05	122.583028897175\\
3.63078054770102e-05	122.262607715106\\
4.78630092322638e-05	121.906037845685\\
6.30957344480193e-05	121.505310528121\\
8.31763771102671e-05	121.053786546546\\
0.000109647819614319	120.54759135102\\
0.000144543977074593	119.986808943062\\
0.000190546071796325	119.376256958527\\
0.000251188643150958	118.725659848077\\
0.000331131121482591	118.049117783585\\
0.000436515832240166	117.363889670812\\
0.000575439937337157	116.688653661064\\
0.000758577575029184	116.04153903081\\
0.001	115.438285935255\\
0.00131825673855641	114.890853913598\\
0.00173780082874938	114.406681864721\\
0.00229086765276777	113.988646253677\\
0.00301995172040202	113.63561610642\\
0.00398107170553497	113.343398993558\\
0.00524807460249772	113.105834802325\\
0.00691830970918936	112.915825317238\\
0.00912010839355909	112.766163879092\\
0.0120226443461741	112.650114762354\\
0.0158489319246111	112.561757912535\\
0.0208929613085404	112.496150386726\\
0.0275422870333816	112.449365246425\\
0.0363078054770102	112.41846201595\\
0.0478630092322638	112.401429732347\\
0.0630957344480194	112.397130160751\\
0.0831763771102671	112.405257536189\\
0.109647819614319	112.42632259961\\
0.144543977074593	112.461661902637\\
0.190546071796325	112.513467184647\\
0.251188643150958	112.584822982805\\
0.331131121482591	112.679732943148\\
0.436515832240166	112.803107020664\\
0.575439937337157	112.960674991245\\
0.758577575029184	113.158790445623\\
1	113.40409867736\\
};
\addplot [color=black, draw=none, mark size=5.0pt, mark=*, mark options={solid, fill=black, black}, forget plot]
  table[row sep=crcr]{%
0.0630957344480194	112.397130160751\\
};
\addplot [color=mycolor7, mark size=2.5pt, mark=*, mark options={solid, fill=mycolor7, mycolor7}, forget plot]
  table[row sep=crcr]{%
1e-06	130.539709519667\\
1.31825673855641e-06	130.451438362081\\
1.73780082874938e-06	130.343736136211\\
2.29086765276777e-06	130.214750443081\\
3.01995172040202e-06	130.063516969731\\
3.98107170553497e-06	129.89025121742\\
5.24807460249772e-06	129.696409658729\\
6.91830970918936e-06	129.484387567827\\
9.1201083935591e-06	129.256823724261\\
1.20226443461741e-05	129.01564163351\\
1.58489319246111e-05	128.761089407954\\
2.08929613085404e-05	128.491066075936\\
2.75422870333816e-05	128.200930616075\\
3.63078054770102e-05	127.883848126585\\
4.78630092322638e-05	127.531612946944\\
6.30957344480193e-05	127.135828989137\\
8.31763771102671e-05	126.689298571363\\
0.000109647819614319	126.187440979683\\
0.000144543977074593	125.629528949127\\
0.000190546071796325	125.019521053784\\
0.000251188643150958	124.36630393521\\
0.000331131121482591	123.683243802237\\
0.000436515832240166	122.987068499673\\
0.000575439937337157	122.296237504977\\
0.000758577575029184	121.629070294775\\
0.001	121.001951618325\\
0.00131825673855641	120.427899232639\\
0.00173780082874938	119.915687346791\\
0.00229086765276777	119.469601903239\\
0.00301995172040202	119.089785027422\\
0.00398107170553497	118.773023049404\\
0.00524807460249772	118.513768820925\\
0.00691830970918936	118.3051842875\\
0.00912010839355909	118.140039958525\\
0.0120226443461741	118.011386503141\\
0.0158489319246111	117.912987278263\\
0.0208929613085404	117.83954894284\\
0.0275422870333816	117.786807454814\\
0.0363078054770102	117.751526391173\\
0.0478630092322638	117.731453765042\\
0.0630957344480194	117.725269958779\\
0.0831763771102671	117.732547008966\\
0.109647819614319	117.753729469183\\
0.144543977074593	117.79013897841\\
0.190546071796325	117.843997400213\\
0.251188643150958	117.918455896821\\
0.331131121482591	118.017609059421\\
0.436515832240166	118.146464897512\\
0.575439937337157	118.310835521149\\
0.758577575029184	118.517113985591\\
1	118.771914797659\\
};
\addplot [color=black, draw=none, mark size=5.0pt, mark=*, mark options={solid, fill=black, black}, forget plot]
  table[row sep=crcr]{%
0.0630957344480194	117.725269958779\\
};
\addplot [color=mycolor1, mark size=2.5pt, mark=*, mark options={solid, fill=mycolor1, mycolor1}, forget plot]
  table[row sep=crcr]{%
1e-06	86.468822681411\\
1.31825673855641e-06	86.3853172557616\\
1.73780082874938e-06	86.2834732664543\\
2.29086765276777e-06	86.1615526319272\\
3.01995172040202e-06	86.0186459398418\\
3.98107170553497e-06	85.8549338698744\\
5.24807460249772e-06	85.6717288291668\\
6.91830970918936e-06	85.4711796408625\\
9.1201083935591e-06	85.2556236463106\\
1.20226443461741e-05	85.026721272784\\
1.58489319246111e-05	84.7846278864499\\
2.08929613085404e-05	84.5274737006208\\
2.75422870333816e-05	84.2513289937794\\
3.63078054770102e-05	83.9506926548678\\
4.78630092322638e-05	83.6194272263091\\
6.30957344480193e-05	83.2519965277273\\
8.31763771102671e-05	82.8448217655687\\
0.000109647819614319	82.3975353567973\\
0.000144543977074593	81.9138828692839\\
0.000190546071796325	81.4020309184933\\
0.000251188643150958	80.8741112519454\\
0.000331131121482591	80.3449815941736\\
0.000436515832240166	79.8303943667743\\
0.000575439937337157	79.3449689636472\\
0.000758577575029184	78.9004536391472\\
0.001	78.5046637418683\\
0.00131825673855641	78.1612348491903\\
0.00173780082874938	77.8700698951855\\
0.00229086765276777	77.628209145427\\
0.00301995172040202	77.4308365052968\\
0.00398107170553497	77.2722072341735\\
0.00524807460249772	77.1463795941133\\
0.00691830970918936	77.0477167958616\\
0.00912010839355909	76.9711804509655\\
0.0120226443461741	76.9124628278928\\
0.0158489319246111	76.8680101855832\\
0.0208929613085404	76.8349825194265\\
0.0275422870333816	76.81118370391\\
0.0363078054770102	76.7949848695432\\
0.0478630092322638	76.7852549855671\\
0.0630957344480194	76.7813064720411\\
0.0831763771102671	76.7828598432054\\
0.109647819614319	76.7900292555321\\
0.144543977074593	76.8033297361755\\
0.190546071796325	76.8237062223254\\
0.251188643150958	76.8525838122167\\
0.331131121482591	76.8919372650389\\
0.436515832240166	76.9443751746282\\
0.575439937337157	77.0132297078092\\
0.758577575029184	77.1026357499349\\
1	77.2175736096436\\
};
\addplot [color=black, draw=none, mark size=5.0pt, mark=*, mark options={solid, fill=black, black}, forget plot]
  table[row sep=crcr]{%
0.0630957344480194	76.7813064720411\\
};
\end{axis}
\end{tikzpicture}%\label{fig:AIC_raw}}
	\caption{Akaike information criterion agains regularisation coefficient.}\label{fig:Callout}
\end{figure}
\section{Future work}
\begin{itemize}
	\item Regularised LS for the direct estimation of $B$ from joint datasets. Long time-series and polynomial structure of the regressors may also lead to collinearity problems. It's not clear yet whether the data needs to be normalised separately.
	\item Confidence and covariance regions to demonstrate bias-variance trade off.
	\item Multiple methods for LASSO regression can be investigated.
	\item Cross-validation and other methods to select the regularisation term.
\end{itemize}
\printbibliography
\end{document}