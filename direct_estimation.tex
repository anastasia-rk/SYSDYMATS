\documentclass[a4paper,11pt,twoside]{article}
%%%%%%%%%%%%%%%%%%%%%%%%%%%%%%%%%%%%%%%%%%%%%%%%%%%%%%%%%%%%%%%%%%%%%%%%%%%
% Packages
\usepackage[final]{pdfpages}
\usepackage{verbatim}
\usepackage{inputenc}
\usepackage{graphicx} 
\usepackage{amsmath,amssymb,mathrsfs,amsfonts}
\usepackage{mathtools}
\usepackage{amsthm}
\usepackage{mathtools}
\usepackage{calrsfs}
\usepackage{graphicx}
\usepackage{subfig}
\usepackage{eucal}    
\usepackage{amssymb}  
\usepackage{pifont}
\usepackage{color} 
\usepackage{cancel}
\usepackage[toc,page]{appendix}
\usepackage{pgfplots}
\pgfplotsset{every axis/.append style={line width=0.5pt},label style={font=\scriptsize},tick label style={font=\scriptsize},x tick label style={/pgf/number format/.cd,fixed,precision=3, set thousands separator={}},z tick label style={/pgf/number format/.cd,fixed,precision=3, set thousands separator={}}}
\usetikzlibrary{shapes,shadows,arrows,backgrounds,patterns,positioning,automata,calc,decorations.markings,decorations.pathreplacing,bayesnet,arrows.meta}
%\usepackage{tikzexternal}
%\usepackage{tikz}
%\usepgfplotslibrary{external} 
%\tikzexternalize
\usepackage{varwidth}
\usepackage{lscape}
\usepackage{array} 
\usepackage[colorlinks=false,pdfborder={0 0 0}]{hyperref}
\usepackage{tabularx}
\usepackage{textcomp}
\usepackage{multicol} 
\usepackage{booktabs}
\usepackage{multirow}
\usepackage[font=small,labelfont=bf]{caption}                                                           
\usepackage{textcase}
\usepackage{bbm} 
\usepackage{fancyhdr}
\usepackage{enumitem}
\usepackage{soul}
%\usepackage[british]{babel}
\usepackage{wrapfig}
%\usepackage{glossaries}
%%%%%%%%%%%%%%%%%%%%%%%%%%%%%%%%%%%%%%%%%%%%%%%%%%%%%%%%%%%%%%%%%%%%%%%%%%%
\setlength{\parindent}{2em}
\setlength{\parskip}{0.5em}
\renewcommand{\baselinestretch}{1.2}
\usepackage[left=2cm, right=2cm, top=2.5cm, bottom=3cm, headheight=13.6pt]{geometry}
\allowdisplaybreaks 
% Bibliography
\usepackage[backend=bibtex,style=ieee,sorting=none]{biblatex} 
\bibliography{bibliography}
\renewcommand*{\bibfont}{\scriptsize}
\makeatletter
\newcommand*{\rom}[1]{\expandafter\@slowromancap\romannumeral #1@}
\newcommand{\ie}{\textit{i.e.} }
\newcommand{\eg}{\textit{e.g.} }
\makeatother
\newcommand\id{\ensuremath{\mathbbm{1}}} 
\DeclareMathOperator{\E}{\mathbb{E}}
\DeclareMathOperator{\eye}{\mathbb{I}}
\DeclareMathOperator{\zeros}{\mathbb{O}}
\DeclareMathOperator{\tr}{\textrm{tr}}
\DeclareMathOperator{\vvec}{\textrm{vec}}
\DeclareMathOperator{\ik}{\mathrm{k}}
\DeclareMathOperator{\ip}{\mathrm{p}}
\DeclareMathOperator{\inn}{\mathrm{n}}
\DeclareMathOperator{\im}{\mathrm{m}}
\DeclareMathOperator{\td}{\mathrm{t}}
\DeclareMathOperator{\kd}{\mathrm{k}}
\DeclareMathOperator{\T}{\mathrm{T}}
\DeclareMathOperator{\K}{\mathrm{K}}
\DeclareMathOperator{\rk}{\mathrm{rk}}
\DeclareMathOperator{\vc}{\mathrm{vec}}
\DeclareSymbolFontAlphabet{\mathcal} {symbols}
\DeclareSymbolFont{symbols}{OMS}{cm}{m}{n}
\DeclareMathAlphabet{\mathbfit}{OML}{cmm}{b}{it}

% Number equations
%\numberwithin{equation}{section}

%%%%%%%%%%%%%%%%%%%%%%%%%%%%%%%%%%%%%%%%%%%%%%%%%%%%%%%%%%%%%%%%%%%%%%%%%%%
%Theorems
\newtheoremstyle{mytheoremstyle} % name
{.5em}                    % Space above
{.8em}                    % Space below
{\itshape}                % Body font
{1em}                           % Indent amount
{\bfseries}                   % Theorem head font
{:}                          % Punctuation after theorem head
{.5em}                       % Space after theorem head
{}  % Theorem head spec (can be left empty, meaning ‘normal’)

\theoremstyle{mytheoremstyle}
\newtheorem{theorem}{Theorem}[section]
\newtheorem{remark}{Remark}[section]
\newtheorem{assumption}{Assumption}[section]
\newtheorem{lemma}{Lemma}[section]
\newtheorem{condition}{Condition}[section]
\newtheorem{definition}{Definition}[section]
\newtheorem{property}{Property}[section]
\newtheorem{corollary}{Corollary}[section]
\renewcommand\qedsymbol{$\blacksquare$}
%%%%%%%%%%%%%%%%%%%%%%%%%%%%%%%%%%%%%%%%%%%%%%%%%%%%%%%%%%%%%%%%%%%%%%%%%%%
% Nomenclature
\usepackage[intoc]{nomencl}
\makenomenclature

%\usepackage[ruled,chapter]{algorithm}
%\usepackage{float}

%\usepackage{algorithmic}
%\algsetup{linenosize=\scriptsize}
%\usepackage{etoolbox}
%\AtBeginEnvironment{algorithmic}{\scriptsize}
%\renewcommand{\thealgorithm}{\thechapter.\arabic{algorithm}} 
%\usepackage{chngcntr}
%\counterwithin{algorithm}{section}

% correct bad hyphenation here
\hyphenation{op-tical net-works semi-conduc-tor}
%%%%%%%%%%%%%%%%%%%%%%%%%%%%%%%%%%%%%%%%%%%%%%%%%%%%%%%%%%%%%%%%%%%%%%%%%%%%%
% Captions
%\newcommand{\xLanguage}{british}          % <-- Added this command
%
%\usepackage[\xLanguage]{babel}             % <-- Implemented here
%
%\expandafter\addto\csname captions\xLanguage\endcsname{% <-- and here
%	\renewcommand{\tableshortname}{Table}%
%	\renewcommand{\figureshortname}{Figure}%
%}
%
%\captionsetup[table]{format=plain,indention=1.15cm,justification=justified}
%\captionsetup[figure]{format=plain,indention=1.25cm,justification=justified}

%%%%%%%%%%%%%%%%%%%%%%%%%%%%%%%%%%%%%%%%%%%%%%%%%%%%%%%%%%%%%%%%%%%%%%%%%%%%%
\usepackage[explicit]{titlesec}
\usepackage{titletoc}
%\titleformat{\section}[block]{\normalfont\Large\rm\filright\bfseries}{\thesection}{1em}{#1}
%\titlecontents{chapter}[1.5em]{}{\scshape\contentslabel{2.3em}}{}{\titlerule*[1pc]{}\contentspage}
%\definecolor{gray75}{gray}{0.5}
%\newcommand{\hsp}{\hspace{20pt}}
%\titleformat{\chapter}[hang]{\huge\scshape\filright\bfseries}{\color{gray75}\thechapter}{20pt}{\begin{tabular}[t]{@{\color{gray75}\vrule width 2pt\hsp}p{0.85\textwidth}}\raggedright#1\end{tabular}}
%\titleformat{name=\chapter,numberless}[display]{}{}{0pt}{\normalfont\huge\bfseries #1} % format for numberless chapters
%\titleformat{\subsection}[block]{\normalfont\large\rm\filright\bfseries}{\thesubsection}{1em}{#1}
%\renewcommand{\sectionmark}[1]{\markright{\thesection ~ \ #1}}
%%\renewcommand{\chaptermark}[1]{\markboth{\chaptername\ \thechapter ~ \ #1}{}} 
%\pagestyle{fancy}
%%\fancyhf{}
%\fancyhead[RO,LE]{\thepage}
%\fancyhead[RE]{\itshape \nouppercase \rightmark}      % chaptertitle left
%\fancyhead[LO]{\itshape \nouppercase \leftmark}       % sectiontitle right
%\renewcommand{\headrulewidth}{0.5pt} 				   % no rule
%\cfoot{}
\interfootnotelinepenalty=10000

\title{Direct estimation of the dynamical model from batch data from multiple experiments}

\begin{document}
	\maketitle
\section{Direct estimation of surface coefficients}
\par Orthogonalisation and forward search are applied to all datasets to identify significant model terms prior to parameter estimation as described in \cite{Wei2008}.\\
\par Model structure for an individual dataset:
\begin{equation}
\underbrace{\mathbfit{y}^k}_{\T\times 1} = \underbrace{\Phi^k}_{\T \times N} \underbrace{\theta^k}_{N \times 1},
\end{equation}
where $\T$ is the length of time-series, $k = 1,\dots, K$ is the dataset index, $N$ is identified number of significant terms.
Model structure for $K$ datasets:
\begin{equation}\label{eq:batchtimeser}
\underbrace{\bar{\mathbf{Y}}}_{\T K\times 1} = \underbrace{\bar{\Phi}}_{\T K \times NK} \underbrace{\bar{\Theta}}_{NK \times 1},
\end{equation}
where the block matrices have the following structure:
\begin{equation}
\underbrace{\bar{\mathbf{Y}}}_{\T K\times 1} = \left[\begin{array}{c} 
\underbrace{\mathbfit{y}^1}_{\T\times 1} \\
\underbrace{\mathbfit{y}^2}_{\T\times 1} \\
\vdots \\
\underbrace{\mathbfit{y}^K}_{\T\times 1}
\end{array}\right]; \quad 
\underbrace{\bar{\Phi}}_{\T K \times NK} = \left[\begin{array}{cccc} 
\underbrace{\Phi^1}_{\T \times N} & \dots & \dots & \zeros \\
\zeros & \underbrace{\Phi^2}_{\T \times N} & \dots & \zeros \\
\vdots & \vdots & \vdots & \vdots  \\
\zeros & \dots & \dots & \underbrace{\Phi^K}_{\T \times N}
\end{array}\right]; \quad 
\underbrace{\bar{\Theta}}_{NK \times 1} = \left[\begin{array}{c} 
\underbrace{\theta^1}_{N \times 1} \\
\underbrace{\theta^2}_{N \times 1} \\
\vdots \\
\underbrace{\theta^K}_{N \times 1}
\end{array}\right].
\end{equation}
\par The relationship of the design parameters known from the experiments and the internal parameters of NARMAX model  is defined by the following linear function:
\begin{equation}
\underbrace{\Theta}_{N \times K} = \underbrace{B}_{N \times L} \underbrace{A}_{L \times K},
\end{equation}
where $A$ is the matrix where each row is a function of the vector of design parameters. The example structure is
\begin{equation}
A = \left[\begin{array}{cccccc}
\eye_{K \times 1} & L_{K \times 1} & D_{K \times 1} & L D_{K \times 1} &  L^{2}_{K \times 1} & D^{2}_{K \times 1} 
\end{array}\right]^{\top},
\end{equation} 
and where $B$ denotes the matrix of unknown coefficients of a hypersurface of order $L$ that maps a point in external parameter space, $\xi^k = (L_k, D_k)$, onto the point in the space of internal parameters, $\theta^k$.
In can be seen that $\bar{\Theta} = \text{vec}(\Theta)$, then
\begin{equation}
\underbrace{\bar{\Theta}}_{NK \times 1} = \text{vec}\left(\underbrace{B}_{N \times L} \underbrace{A}_{L \times K}\right).
\end{equation}
This vectorisation can be obtained using Kronecker product:
\begin{equation}
\text{vec}\left(\underbrace{B}_{N \times L} \underbrace{A}_{L \times K}\right) = (\underbrace{A^{\top}}_{K \times L} \otimes \underbrace{\eye}_{N \times N}) \underbrace{\text{vec}(B)}_{NL \times 1}.
\end{equation}
Denoting the result of Kronecker product as $\underbrace{\mathbf{Kr}}_{NK \times NL} \triangleq (\underbrace{A^{\top}}_{K \times L} \otimes \underbrace{\eye}_{N \times N})$ and vectorised coefficient matrix as $\underbrace{\bar{\mathbf{B}}}_{NL \times 1} \triangleq \text{vec}(B)$  yields the following:
\begin{equation}\label{eq:BtoTheta}
\underbrace{\bar{\Theta}}_{NK \times 1} = \underbrace{\mathbf{Kr}}_{KN \times NL} \underbrace{\bar{\mathbf{B}}}_{NL \times 1}.
\end{equation}
Substituting the above expression into \eqref{eq:batchtimeser} renders an expression that directly links the design parameters and the timeseries data
\begin{equation}\label{eq:BtoY}
\underbrace{\bar{\mathbf{Y}}}_{\T K\times 1} = \underbrace{\bar{\Phi}}_{\T K \times NK} \underbrace{\mathbf{Kr}}_{KN \times NL} \underbrace{\bar{\mathbf{B}}}_{NL \times 1},
\end{equation}
where $\bar{\mathbf{B}}$ is the unknown vector and all other factors are known form the experiments or defined prior to structure identification. 
\par The representation \eqref{eq:BtoY} allows estimating the coefficients in  $\bar{\mathbf{B}}$ directly from the timeseries data bypassing the intermediate estimation of the internal coefficients in NARMAX model. It must be noted, however, that its not increasing the dimension of the matrix that guarantees identifiability of the system but its rank. The following condition must be satisfied:
\begin{equation}\label{eq:rankcond}
\rk(\bar{\Phi}\mathbf{Kr}) \geq NL.
\end{equation}
The rank of the linear system \eqref{eq:BtoY} satisfies the following:
\begin{equation}
\rk (\bar{\Phi}\mathbf{Kr}) \leq \min\left(\rk(\bar{\Phi}), \rk(\mathbf{Kr})\right),
\end{equation}
where the rank of Kronecker product can be found as
\begin{equation}
\rk(\mathbf{Kr}) = \rk(\underbrace{A^{\top}}_{K \times L})\rk(\underbrace{\eye}_{N \times N}),
\end{equation}
where the first factor is $\rk(A^{\top}) = \min\{K, L\}$ (note that all rows and columns in the matrix $A$ are known to be independent ``by design"), and where second factor equals $N$. Given the above two expressions, it is evident that regardless of the length of the available time series, the identifiability of the system depends on the number of experiments, $K$, and the selected order of the hypersurface fitted to the points in the design parameter space, $L-1$. Moreover, if insufficient number of datapoints are available ($K < L$), the condition \eqref{eq:rankcond} can not be satisfied rendering the system \eqref{eq:BtoY} unidentifiable.
\section{Estimation results for joint datasets}
\par To test this approach, 8 out of 10 datasets are used for model identification and two remaining datasets (C3 and C8) are used for validation. The surface coefficients estimated via ordinary least squares from system \eqref{eq:BtoY} are presented in Table \ref{tab:Betas_direct_all}. The estimation results are similar to those obtained in the two-stage process in \cite{Wei2008}. The output generated by the identified models for C3 and C8 are compared with the outputs in Figure \ref{fig:Betas_direct_all}. RMSEs for C3 and C8 are equal to 1.97 and 12.08, respectively.
\begin{table}[!h]
	\centering
	\caption{Estimated polynomial coefficients for the sample length 2000.}\label{tab:Betas_direct_all}
	\small
	\begin{tabular}{rrrrrrrrr}
Step & Terms & Input terms & $\beta_{0}$ & $\beta_{1}$ & $\beta_{2}$ & $\beta_{3}$ & $\beta_{4}$ & $\beta_{5}$ \\ 
\hline 
1 & $x_4,x_4$ & $u^2(t)$ & -170.63 & 4.28 & 830.86 & -1.99 & -0.03 & -4566.31 \\ 
2 & $x_3$ & $u(t-1)$& 143.21 & -3.39 & -1375.84 & -4.07 & 0.04 & 11461.92 \\ 
3 & $x_1,x_4$ & $u(t-3)u(t)$ & 6.37 & 0.13 & -201.68 & 2.8 & 0 & 168.2 \\ 
4 & $x_1,x_1$& $u^2(t-3)$ & 1.31 & -0.18 & 77.49 & -1.36 & 0 & 35.9 \\ 
5 & $x_2$ & $u(t-2)$ & 75.9 & -1.75 & -435.03 & 5.27 & 0.01 & 952.37 \\ 
6 & $x_4$ & $u(t)$ & -1213.8 & 31.15 & 5349.77 & -8.89 & -0.25 & -30663.41 \\ 
7 & $c$ & $const$ & -2421.38 & 62.88 & 8922.96 & -22.5 & -0.49 & -45455.29 \\ 
8 & $x_3,x_4$ & $u(t-1)u(t)$ & 61.25 & -1.53 & -369.5 & 0.28 & 0.01 & 2480.51 \\ 
\hline 
\end{tabular}
\end{table}

\begin{figure}[!h]
	\centering
	\subfloat[C3]{% This file was created by matlab2tikz.
%
\definecolor{mycolor1}{rgb}{0.00000,0.44700,0.74100}%
\definecolor{mycolor2}{rgb}{0.85000,0.32500,0.09800}%
%
\begin{tikzpicture}

\begin{axis}[%
width=11.411cm,
height=3.5cm,
at={(0cm,0cm)},
scale only axis,
xmin=1000,
xmax=1500,
ymin=-40.283,
ymax=0,
axis background/.style={fill=white},
legend style={legend cell align=left, align=left, draw=white!15!black}
]
\addplot [color=mycolor1]
  table[row sep=crcr]{%
1001	-17.09\\
1002	-19.531\\
1003	-14.648\\
1004	-14.648\\
1005	-19.531\\
1006	-18.311\\
1007	-23.193\\
1008	-18.311\\
1009	-9.766\\
1010	-15.869\\
1011	-12.207\\
1012	-14.648\\
1013	-10.986\\
1014	-3.662\\
1015	-2.441\\
1016	-4.883\\
1017	-6.104\\
1018	-14.648\\
1019	-17.09\\
1020	-17.09\\
1021	-13.428\\
1022	-9.766\\
1023	-18.311\\
1024	-14.648\\
1025	-10.986\\
1026	-13.428\\
1027	-12.207\\
1028	-10.986\\
1029	-20.752\\
1030	-19.531\\
1031	-12.207\\
1032	-20.752\\
1033	-20.752\\
1034	-15.869\\
1035	-13.428\\
1036	-9.766\\
1037	-14.648\\
1038	-14.648\\
1039	-14.648\\
1040	-14.648\\
1041	-14.648\\
1042	-15.869\\
1043	-21.973\\
1044	-20.752\\
1045	-13.428\\
1046	-13.428\\
1047	-8.545\\
1048	-12.207\\
1049	-12.207\\
1050	-13.428\\
1051	-10.986\\
1052	-13.428\\
1053	-14.648\\
1054	-13.428\\
1055	-20.752\\
1056	-13.428\\
1057	-9.766\\
1058	-7.324\\
1059	-8.545\\
1060	-10.986\\
1061	-6.104\\
1062	-9.766\\
1063	-10.986\\
1064	-6.104\\
1065	-6.104\\
1066	-9.766\\
1067	-9.766\\
1068	-15.869\\
1069	-14.648\\
1070	-17.09\\
1071	-15.869\\
1072	-18.311\\
1073	-13.428\\
1074	-14.648\\
1075	-12.207\\
1076	-10.986\\
1077	-12.207\\
1078	-23.193\\
1079	-28.076\\
1080	-30.518\\
1081	-29.297\\
1082	-18.311\\
1083	-28.076\\
1084	-32.959\\
1085	-31.738\\
1086	-20.752\\
1087	-34.18\\
1088	-40.283\\
1089	-29.297\\
1090	-24.414\\
1091	-19.531\\
1092	-15.869\\
1093	-12.207\\
1094	-14.648\\
1095	-9.766\\
1096	-7.324\\
1097	-7.324\\
1098	-10.986\\
1099	-13.428\\
1100	-12.207\\
1101	-12.207\\
1102	-15.869\\
1103	-18.311\\
1104	-19.531\\
1105	-15.869\\
1106	-21.973\\
1107	-15.869\\
1108	-15.869\\
1109	-17.09\\
1110	-12.207\\
1111	-12.207\\
1112	-15.869\\
1113	-14.648\\
1114	-8.545\\
1115	-12.207\\
1116	-8.545\\
1117	-9.766\\
1118	-9.766\\
1119	-7.324\\
1120	-9.766\\
1121	-12.207\\
1122	-17.09\\
1123	-17.09\\
1124	-17.09\\
1125	-10.986\\
1126	-12.207\\
1127	-23.193\\
1128	-15.869\\
1129	-20.752\\
1130	-21.973\\
1131	-12.207\\
1132	-9.766\\
1133	-12.207\\
1134	-13.428\\
1135	-21.973\\
1136	-25.635\\
1137	-26.855\\
1138	-19.531\\
1139	-20.752\\
1140	-18.311\\
1141	-17.09\\
1142	-18.311\\
1143	-13.428\\
1144	-12.207\\
1145	-9.766\\
1146	-9.766\\
1147	-10.986\\
1148	-15.869\\
1149	-23.193\\
1150	-17.09\\
1151	-13.428\\
1152	-10.986\\
1153	-10.986\\
1154	-7.324\\
1155	-6.104\\
1156	-6.104\\
1157	-12.207\\
1158	-7.324\\
1159	-8.545\\
1160	-8.545\\
1161	-8.545\\
1162	-7.324\\
1163	-4.883\\
1164	-3.662\\
1165	-8.545\\
1166	-15.869\\
1167	-18.311\\
1168	-20.752\\
1169	-15.869\\
1170	-10.986\\
1171	-8.545\\
1172	-4.883\\
1173	-9.766\\
1174	-8.545\\
1175	-15.869\\
1176	-17.09\\
1177	-29.297\\
1178	-32.959\\
1179	-25.635\\
1180	-25.635\\
1181	-18.311\\
1182	-23.193\\
1183	-21.973\\
1184	-24.414\\
1185	-15.869\\
1186	-17.09\\
1187	-17.09\\
1188	-15.869\\
1189	-15.869\\
1190	-13.428\\
1191	-23.193\\
1192	-25.635\\
1193	-19.531\\
1194	-13.428\\
1195	-14.648\\
1196	-23.193\\
1197	-24.414\\
1198	-28.076\\
1199	-30.518\\
1200	-29.297\\
1201	-23.193\\
1202	-21.973\\
1203	-25.635\\
1204	-28.076\\
1205	-18.311\\
1206	-12.207\\
1207	-19.531\\
1208	-14.648\\
1209	-9.766\\
1210	-13.428\\
1211	-14.648\\
1212	-10.986\\
1213	-10.986\\
1214	-10.986\\
1215	-8.545\\
1216	-10.986\\
1217	-18.311\\
1218	-17.09\\
1219	-12.207\\
1220	-12.207\\
1221	-18.311\\
1222	-26.855\\
1223	-17.09\\
1224	-14.648\\
1225	-10.986\\
1226	-13.428\\
1227	-10.986\\
1228	-8.545\\
1229	-8.545\\
1230	-10.986\\
1231	-13.428\\
1232	-14.648\\
1233	-12.207\\
1234	-15.869\\
1235	-20.752\\
1236	-17.09\\
1237	-12.207\\
1238	-15.869\\
1239	-20.752\\
1240	-13.428\\
1241	-7.324\\
1242	-13.428\\
1243	-10.986\\
1244	-12.207\\
1245	-9.766\\
1246	-8.545\\
1247	-13.428\\
1248	-13.428\\
1249	-9.766\\
1250	-12.207\\
1251	-9.766\\
1252	-7.324\\
1253	-12.207\\
1254	-8.545\\
1255	-9.766\\
1256	-7.324\\
1257	-4.883\\
1258	-12.207\\
1259	-15.869\\
1260	-17.09\\
1261	-23.193\\
1262	-15.869\\
1263	-9.766\\
1264	-8.545\\
1265	-8.545\\
1266	-10.986\\
1267	-8.545\\
1268	-4.883\\
1269	-10.986\\
1270	-15.869\\
1271	-13.428\\
1272	-20.752\\
1273	-15.869\\
1274	-14.648\\
1275	-8.545\\
1276	-10.986\\
1277	-4.883\\
1278	-3.662\\
1279	-4.883\\
1280	-9.766\\
1281	-10.986\\
1282	-12.207\\
1283	-13.428\\
1284	-20.752\\
1285	-17.09\\
1286	-14.648\\
1287	-17.09\\
1288	-19.531\\
1289	-20.752\\
1290	-17.09\\
1291	-13.428\\
1292	-7.324\\
1293	-6.104\\
1294	-4.883\\
1295	-7.324\\
1296	-7.324\\
1297	-4.883\\
1298	-8.545\\
1299	-12.207\\
1300	-10.986\\
1301	-12.207\\
1302	-12.207\\
1303	-8.545\\
1304	-4.883\\
1305	-9.766\\
1306	-15.869\\
1307	-13.428\\
1308	-15.869\\
1309	-13.428\\
1310	-9.766\\
1311	-14.648\\
1312	-18.311\\
1313	-18.311\\
1314	-13.428\\
1315	-20.752\\
1316	-15.869\\
1317	-17.09\\
1318	-18.311\\
1319	-19.531\\
1320	-13.428\\
1321	-13.428\\
1322	-18.311\\
1323	-26.855\\
1324	-21.973\\
1325	-13.428\\
1326	-13.428\\
1327	-13.428\\
1328	-15.869\\
1329	-17.09\\
1330	-12.207\\
1331	-14.648\\
1332	-18.311\\
1333	-20.752\\
1334	-13.428\\
1335	-12.207\\
1336	-15.869\\
1337	-23.193\\
1338	-20.752\\
1339	-21.973\\
1340	-14.648\\
1341	-14.648\\
1342	-12.207\\
1343	-9.766\\
1344	-4.883\\
1345	-3.662\\
1346	-3.662\\
1347	-7.324\\
1348	-13.428\\
1349	-14.648\\
1350	-9.766\\
1351	-12.207\\
1352	-13.428\\
1353	-9.766\\
1354	-8.545\\
1355	-8.545\\
1356	-6.104\\
1357	-9.766\\
1358	-14.648\\
1359	-15.869\\
1360	-10.986\\
1361	-8.545\\
1362	-8.545\\
1363	-8.545\\
1364	-7.324\\
1365	-10.986\\
1366	-20.752\\
1367	-18.311\\
1368	-18.311\\
1369	-24.414\\
1370	-23.193\\
1371	-18.311\\
1372	-14.648\\
1373	-14.648\\
1374	-14.648\\
1375	-17.09\\
1376	-19.531\\
1377	-19.531\\
1378	-25.635\\
1379	-26.855\\
1380	-31.738\\
1381	-24.414\\
1382	-25.635\\
1383	-26.855\\
1384	-21.973\\
1385	-24.414\\
1386	-20.752\\
1387	-13.428\\
1388	-9.766\\
1389	-9.766\\
1390	-12.207\\
1391	-9.766\\
1392	-7.324\\
1393	-10.986\\
1394	-8.545\\
1395	-6.104\\
1396	-7.324\\
1397	-9.766\\
1398	-6.104\\
1399	-12.207\\
1400	-14.648\\
1401	-10.986\\
1402	-17.09\\
1403	-24.414\\
1404	-24.414\\
1405	-28.076\\
1406	-23.193\\
1407	-21.973\\
1408	-17.09\\
1409	-9.766\\
1410	-10.986\\
1411	-10.986\\
1412	-7.324\\
1413	-10.986\\
1414	-9.766\\
1415	-2.441\\
1416	-6.104\\
1417	-9.766\\
1418	-12.207\\
1419	-14.648\\
1420	-17.09\\
1421	-14.648\\
1422	-17.09\\
1423	-14.648\\
1424	-13.428\\
1425	-12.207\\
1426	-10.986\\
1427	-14.648\\
1428	-13.428\\
1429	-10.986\\
1430	-13.428\\
1431	-18.311\\
1432	-13.428\\
1433	-10.986\\
1434	-10.986\\
1435	-10.986\\
1436	-8.545\\
1437	-7.324\\
1438	-10.986\\
1439	-13.428\\
1440	-13.428\\
1441	-10.986\\
1442	-13.428\\
1443	-13.428\\
1444	-8.545\\
1445	-7.324\\
1446	-15.869\\
1447	-19.531\\
1448	-18.311\\
1449	-18.311\\
1450	-15.869\\
1451	-13.428\\
1452	-10.986\\
1453	-9.766\\
1454	-13.428\\
1455	-13.428\\
1456	-8.545\\
1457	-12.207\\
1458	-14.648\\
1459	-14.648\\
1460	-17.09\\
1461	-17.09\\
1462	-20.752\\
1463	-28.076\\
1464	-30.518\\
1465	-20.752\\
1466	-13.428\\
1467	-8.545\\
1468	-6.104\\
1469	-8.545\\
1470	-7.324\\
1471	-10.986\\
1472	-12.207\\
1473	-12.207\\
1474	-13.428\\
1475	-15.869\\
1476	-19.531\\
1477	-19.531\\
1478	-28.076\\
1479	-23.193\\
1480	-18.311\\
1481	-14.648\\
1482	-14.648\\
1483	-14.648\\
1484	-17.09\\
1485	-14.648\\
1486	-12.207\\
1487	-13.428\\
1488	-8.545\\
1489	-7.324\\
1490	-17.09\\
1491	-14.648\\
1492	-12.207\\
1493	-23.193\\
1494	-18.311\\
1495	-15.869\\
1496	-21.973\\
1497	-23.193\\
1498	-14.648\\
1499	-8.545\\
1500	-9.766\\
};
\addlegendentry{True output}

\addplot [color=mycolor2, dashed]
  table[row sep=crcr]{%
1001	-16.2539230739907\\
1002	-19.6780520415553\\
1003	-18.469302237404\\
1004	-16.8080058865357\\
1005	-20.6957282599679\\
1006	-20.9226048725852\\
1007	-22.1580165130341\\
1008	-18.9518290289044\\
1009	-11.8532255560373\\
1010	-12.7094778579294\\
1011	-13.8179607362094\\
1012	-15.6676410156176\\
1013	-14.3385593179044\\
1014	-7.44456767678184\\
1015	-4.84212856487212\\
1016	-4.54619238282197\\
1017	-10.2785971043246\\
1018	-18.6668177063403\\
1019	-18.1577790223973\\
1020	-17.419624042712\\
1021	-13.4252139071657\\
1022	-9.17868514381313\\
1023	-16.280527196778\\
1024	-13.7580359232491\\
1025	-10.5801771762864\\
1026	-14.2590547823766\\
1027	-15.1542873846733\\
1028	-12.4173422729559\\
1029	-17.5313443661036\\
1030	-17.7338761110506\\
1031	-13.9934524151653\\
1032	-18.0970293295857\\
1033	-18.7878487338414\\
1034	-15.4978702674208\\
1035	-13.5599726341209\\
1036	-12.1798162283371\\
1037	-13.1048898918884\\
1038	-15.7290880554037\\
1039	-15.6323469059163\\
1040	-14.7762364826088\\
1041	-15.2795750922203\\
1042	-16.1264460788797\\
1043	-20.7289117767489\\
1044	-20.1843200896907\\
1045	-14.5950967919077\\
1046	-13.2923323949386\\
1047	-9.39908460219986\\
1048	-8.85432943021028\\
1049	-11.7979120299693\\
1050	-11.0473129766188\\
1051	-10.7567174205842\\
1052	-13.198426949789\\
1053	-14.7950097571651\\
1054	-13.9362850087787\\
1055	-19.0822368141187\\
1056	-16.7253966801447\\
1057	-10.2278538557543\\
1058	-8.69417777823647\\
1059	-11.1018614705337\\
1060	-12.9871438663928\\
1061	-9.34952037071325\\
1062	-7.73521105539386\\
1063	-10.3734786059553\\
1064	-10.0454800943999\\
1065	-8.48755962757926\\
1066	-9.63047759177737\\
1067	-10.9784840991665\\
1068	-15.0186046839667\\
1069	-15.2811088744891\\
1070	-16.1018944552936\\
1071	-15.1501122378512\\
1072	-16.8880163566434\\
1073	-15.223840534098\\
1074	-13.9139363777144\\
1075	-13.40364396027\\
1076	-13.3905775549969\\
1077	-13.2241621429197\\
1078	-19.447236643897\\
1079	-28.0915906678499\\
1080	-28.0506563512639\\
1081	-26.3396930259809\\
1082	-17.8009977070953\\
1083	-23.9047242810522\\
1084	-29.4185349614902\\
1085	-29.4298245178528\\
1086	-24.5323302243264\\
1087	-29.1028255275749\\
1088	-36.7554854289053\\
1089	-31.0414281213296\\
1090	-25.9555782684485\\
1091	-19.3995789200637\\
1092	-16.3212779257356\\
1093	-14.9983478293814\\
1094	-16.8286134098941\\
1095	-15.2468567575013\\
1096	-10.2248823365089\\
1097	-8.87405887169777\\
1098	-10.9633616525056\\
1099	-15.1535846501346\\
1100	-14.9941435061284\\
1101	-14.3162680178793\\
1102	-16.4744491614032\\
1103	-17.2237237142614\\
1104	-22.8147684132588\\
1105	-20.4912683973127\\
1106	-21.1845932822107\\
1107	-18.2175792726756\\
1108	-15.5834974916142\\
1109	-18.354878665448\\
1110	-15.5018900417249\\
1111	-13.3582592799504\\
1112	-15.8302154536316\\
1113	-16.2281786823007\\
1114	-13.6858974246602\\
1115	-12.9040400269245\\
1116	-11.3626753315738\\
1117	-10.9779846862752\\
1118	-12.7438185967629\\
1119	-9.68154591687903\\
1120	-10.3048325743132\\
1121	-12.8997286711409\\
1122	-17.4202210661855\\
1123	-17.6197012659321\\
1124	-17.469207430894\\
1125	-12.932268872015\\
1126	-15.7631410114535\\
1127	-23.3617271482332\\
1128	-18.9660507014177\\
1129	-20.0119174135404\\
1130	-20.0475990720454\\
1131	-14.4374385670613\\
1132	-10.8744460429319\\
1133	-13.0819452830625\\
1134	-15.4733736962833\\
1135	-22.177808497815\\
1136	-26.5877088513597\\
1137	-24.8767237472474\\
1138	-20.5076402025991\\
1139	-19.9673025534405\\
1140	-19.9421944087727\\
1141	-19.1189797190637\\
1142	-19.0096726818086\\
1143	-14.9406840289458\\
1144	-12.8990245804064\\
1145	-12.8556563276776\\
1146	-12.378291714666\\
1147	-13.0020992375691\\
1148	-17.0015352235398\\
1149	-23.1676061246407\\
1150	-20.3956184850823\\
1151	-14.3349727562941\\
1152	-12.3099804768701\\
1153	-12.758255263183\\
1154	-9.22023022707761\\
1155	-7.03572633789989\\
1156	-7.63680912740635\\
1157	-12.031621516445\\
1158	-11.0241800905432\\
1159	-10.3251313346702\\
1160	-10.9519400445727\\
1161	-10.494936371324\\
1162	-8.89742073208249\\
1163	-6.66266352080711\\
1164	-5.69737985947421\\
1165	-7.31800424616763\\
1166	-14.1180474343725\\
1167	-19.9886057959006\\
1168	-20.8198735551309\\
1169	-14.311159761974\\
1170	-10.4338291866424\\
1171	-8.17049903359285\\
1172	-6.24703120947181\\
1173	-11.1059591396286\\
1174	-12.289052230204\\
1175	-14.9266748968001\\
1176	-18.3692094609253\\
1177	-27.8516325556974\\
1178	-32.4586608941558\\
1179	-26.7151033478238\\
1180	-25.5288339164571\\
1181	-20.2590126338423\\
1182	-20.1575209363349\\
1183	-22.3863841765549\\
1184	-23.5824694591454\\
1185	-19.9500313612636\\
1186	-18.4511982503646\\
1187	-18.8044157267348\\
1188	-17.5499112265555\\
1189	-19.9445121480006\\
1190	-16.7210525145171\\
1191	-21.6415953900423\\
1192	-26.2193616723197\\
1193	-20.2354977087544\\
1194	-13.7330574829656\\
1195	-14.3873306316012\\
1196	-21.7042130720529\\
1197	-25.739994218845\\
1198	-28.9321133304504\\
1199	-30.0802835354787\\
1200	-29.7365990835702\\
1201	-24.4591682286259\\
1202	-22.965576230733\\
1203	-25.5036612952033\\
1204	-28.1957971014082\\
1205	-21.049850138535\\
1206	-14.0732148539597\\
1207	-18.0994846278193\\
1208	-17.454294894628\\
1209	-12.5130253074846\\
1210	-14.6034340716407\\
1211	-16.9679005832604\\
1212	-14.5590251993914\\
1213	-11.7887546220519\\
1214	-13.631192736275\\
1215	-13.3119983488425\\
1216	-15.3274977167194\\
1217	-20.0025583306443\\
1218	-18.1720360052906\\
1219	-14.1273434659414\\
1220	-13.5266338341512\\
1221	-18.8814105774277\\
1222	-28.261227535936\\
1223	-23.5932166347219\\
1224	-15.2993517119231\\
1225	-12.6924985403799\\
1226	-13.6123623384401\\
1227	-14.5223193891605\\
1228	-11.7060237170626\\
1229	-12.1575720520966\\
1230	-13.4088443461588\\
1231	-16.1405281351675\\
1232	-17.2230153540734\\
1233	-12.6126620640347\\
1234	-16.223694365303\\
1235	-19.8498191718542\\
1236	-18.9720101941369\\
1237	-16.1610905868729\\
1238	-16.610904977064\\
1239	-21.3268956646753\\
1240	-16.0674705976619\\
1241	-8.28003162313434\\
1242	-9.36523662845529\\
1243	-11.7603886376218\\
1244	-15.3974598318374\\
1245	-14.3127090400631\\
1246	-11.4075253480717\\
1247	-14.5262446786404\\
1248	-14.939621076983\\
1249	-11.9384180013251\\
1250	-13.2735177532064\\
1251	-12.8286784046346\\
1252	-12.1369183539869\\
1253	-12.7997035597817\\
1254	-9.75459153510669\\
1255	-9.78163625239314\\
1256	-8.65801574155545\\
1257	-8.92664927170209\\
1258	-13.6379109578393\\
1259	-15.5377035576408\\
1260	-18.5674324301422\\
1261	-23.5023918885621\\
1262	-17.3028975464599\\
1263	-10.9768649442763\\
1264	-8.85213173661993\\
1265	-9.10490717211669\\
1266	-12.3752580182473\\
1267	-9.62077540552941\\
1268	-9.36111178184319\\
1269	-11.4383582833858\\
1270	-16.7922510608452\\
1271	-16.8559738231921\\
1272	-19.2790126301891\\
1273	-15.1184668074552\\
1274	-13.1440106741284\\
1275	-8.76836309900033\\
1276	-7.95330887462065\\
1277	-6.24892884689786\\
1278	-6.30405656908711\\
1279	-7.65606753166752\\
1280	-12.5545566760042\\
1281	-12.7652479679523\\
1282	-13.6706334294884\\
1283	-13.9273472743644\\
1284	-20.2782881443599\\
1285	-20.0046353266574\\
1286	-14.4990612448987\\
1287	-16.2763971731633\\
1288	-17.2474068984667\\
1289	-21.578753659502\\
1290	-18.7868797139578\\
1291	-13.5855117303878\\
1292	-8.38855966563831\\
1293	-6.13907769094542\\
1294	-6.72706197487921\\
1295	-8.00888832950053\\
1296	-8.37623344141533\\
1297	-8.1547185136819\\
1298	-9.27364056981452\\
1299	-12.8265877025395\\
1300	-12.827159761967\\
1301	-11.9868369304656\\
1302	-13.2596175289529\\
1303	-9.79893970180592\\
1304	-6.1848106171375\\
1305	-9.67413298334181\\
1306	-14.3217428073644\\
1307	-14.6446073121796\\
1308	-14.3701941478023\\
1309	-13.2571071569034\\
1310	-10.8232205536851\\
1311	-13.0780208843682\\
1312	-17.6012985885539\\
1313	-18.1387056303394\\
1314	-13.6085979378672\\
1315	-19.9839325490534\\
1316	-17.779287563572\\
1317	-16.042192250216\\
1318	-18.4946856282427\\
1319	-17.9578457684789\\
1320	-15.2500255730766\\
1321	-13.5017539088877\\
1322	-18.572348076345\\
1323	-26.7344708618669\\
1324	-21.4411892553042\\
1325	-13.5662976880119\\
1326	-12.6951517337681\\
1327	-14.6816668305842\\
1328	-17.067954030467\\
1329	-19.2054903471361\\
1330	-15.5597698952217\\
1331	-16.0445225823114\\
1332	-19.4228056176553\\
1333	-20.7793833799127\\
1334	-15.9369854479256\\
1335	-13.3219470871975\\
1336	-16.011989974564\\
1337	-24.1139735456751\\
1338	-21.983336236925\\
1339	-21.1310311116869\\
1340	-14.6803716468835\\
1341	-13.0007032324487\\
1342	-13.9712730720293\\
1343	-9.675397465137\\
1344	-7.41610084887329\\
1345	-5.93051396586524\\
1346	-5.45530180236698\\
1347	-9.28545387472525\\
1348	-13.5517203909519\\
1349	-15.2535167562427\\
1350	-11.9156222131554\\
1351	-11.7326379265984\\
1352	-12.9443086331619\\
1353	-10.0643671786901\\
1354	-9.53169789452859\\
1355	-10.0149982150875\\
1356	-8.71107638701354\\
1357	-9.39196025296291\\
1358	-13.1620861098204\\
1359	-16.5862599114221\\
1360	-11.8957876530638\\
1361	-9.14133527721\\
1362	-8.31722431432862\\
1363	-9.76645095955545\\
1364	-7.95327385853207\\
1365	-13.6066838129525\\
1366	-21.5601563859836\\
1367	-17.5479704658023\\
1368	-20.4373015079638\\
1369	-24.3406031641331\\
1370	-23.4696289346703\\
1371	-19.4493475913192\\
1372	-15.1140271031427\\
1373	-15.2262981128224\\
1374	-16.1807470052193\\
1375	-18.0944956465427\\
1376	-19.8944236524624\\
1377	-20.2223849660781\\
1378	-23.3203412962895\\
1379	-26.5660262071652\\
1380	-30.43035680189\\
1381	-23.774721769576\\
1382	-23.2999449671533\\
1383	-27.340377550019\\
1384	-23.5776928979972\\
1385	-24.3325025357088\\
1386	-23.0665025767692\\
1387	-14.1848998367042\\
1388	-10.3097664339943\\
1389	-11.1364364820812\\
1390	-15.714055990445\\
1391	-14.2592991734158\\
1392	-10.5563700933549\\
1393	-12.2612279622968\\
1394	-10.5438050984089\\
1395	-8.42678553971259\\
1396	-9.50739921851141\\
1397	-10.5177500686832\\
1398	-8.90267907601711\\
1399	-12.2856217428823\\
1400	-16.3023155324535\\
1401	-13.3163071147417\\
1402	-17.3041532241307\\
1403	-25.0985590346364\\
1404	-23.5158048645782\\
1405	-26.6851615193597\\
1406	-21.3243853095907\\
1407	-20.7879163377641\\
1408	-17.1262290771534\\
1409	-11.5261614862112\\
1410	-13.8251741495308\\
1411	-12.3162159614708\\
1412	-10.0008444518946\\
1413	-12.202866065259\\
1414	-8.78888452476564\\
1415	-6.14811414228488\\
1416	-6.77025807031817\\
1417	-11.025014069855\\
1418	-15.4913084097193\\
1419	-17.2169296997092\\
1420	-16.7827595075409\\
1421	-15.7352167946492\\
1422	-16.9705653951228\\
1423	-15.4252304239812\\
1424	-13.2515645648776\\
1425	-13.9444910489864\\
1426	-12.4106886385845\\
1427	-16.306982845209\\
1428	-13.1636901466916\\
1429	-10.7077479492376\\
1430	-14.0977013199461\\
1431	-19.2111820654393\\
1432	-16.324149150138\\
1433	-12.3389049776214\\
1434	-11.1154674241061\\
1435	-12.4834202850968\\
1436	-10.0529425378744\\
1437	-9.0348993986101\\
1438	-11.4983950807885\\
1439	-13.3354624919173\\
1440	-14.8518907046436\\
1441	-12.6346660122174\\
1442	-13.979435900401\\
1443	-14.0721778760564\\
1444	-9.9174258940787\\
1445	-9.37050366384558\\
1446	-14.9930623264498\\
1447	-21.0318543940542\\
1448	-19.9538620348605\\
1449	-16.9504484428642\\
1450	-16.4822623577218\\
1451	-14.2274847004841\\
1452	-13.6308267945847\\
1453	-12.0478591963532\\
1454	-14.7124875992284\\
1455	-14.4415783431988\\
1456	-10.2630190378167\\
1457	-12.8502250107637\\
1458	-14.5652665717799\\
1459	-16.970469489618\\
1460	-17.3673163984328\\
1461	-17.0739940894659\\
1462	-21.7836709428202\\
1463	-26.3719331222928\\
1464	-29.3742834171161\\
1465	-21.4244186699024\\
1466	-13.4368479317095\\
1467	-9.79392109581274\\
1468	-7.23658293344496\\
1469	-9.58491802336514\\
1470	-10.6836344274412\\
1471	-14.8807549706112\\
1472	-14.5222809898189\\
1473	-12.1268327261536\\
1474	-14.5010641450403\\
1475	-16.3745739795042\\
1476	-18.065229042577\\
1477	-19.1203217664647\\
1478	-26.868946475005\\
1479	-24.8689519708786\\
1480	-19.9406023180345\\
1481	-15.0994066498417\\
1482	-14.4371606807785\\
1483	-16.2571380328895\\
1484	-18.6628681557596\\
1485	-15.4379176798584\\
1486	-14.5310082290384\\
1487	-14.6252133426737\\
1488	-9.92936209875618\\
1489	-12.9604194125152\\
1490	-18.3164034390473\\
1491	-14.7787102197957\\
1492	-14.3294940533448\\
1493	-22.6419765817315\\
1494	-19.4067227763452\\
1495	-17.310459670167\\
1496	-22.3010333406962\\
1497	-22.0853712715304\\
1498	-16.2599916577794\\
1499	-10.411695300332\\
1500	-11.3792218269772\\
};
\addlegendentry{Generated output}

\end{axis}
\end{tikzpicture}%\label{fig:c3all}}\\
	\subfloat[C8]{% This file was created by matlab2tikz.
%
\definecolor{mycolor1}{rgb}{0.00000,0.44700,0.74100}%
\definecolor{mycolor2}{rgb}{0.85000,0.32500,0.09800}%
%
\begin{tikzpicture}

\begin{axis}[%
width=11.411cm,
height=3.5cm,
at={(0cm,0cm)},
scale only axis,
xmin=1000,
xmax=1500,
ymin=-300,
ymax=0,
axis background/.style={fill=white},
legend style={legend cell align=left, align=left, draw=white!15!black}
]
\addplot [color=mycolor1]
  table[row sep=crcr]{%
1001	-96.436\\
1002	-122.07\\
1003	-100.098\\
1004	-93.994\\
1005	-130.615\\
1006	-125.732\\
1007	-153.809\\
1008	-115.967\\
1009	-61.035\\
1010	-74.463\\
1011	-73.242\\
1012	-86.67\\
1013	-73.242\\
1014	-34.18\\
1015	-20.752\\
1016	-23.193\\
1017	-58.594\\
1018	-100.098\\
1019	-109.863\\
1020	-114.746\\
1021	-83.008\\
1022	-52.49\\
1023	-104.98\\
1024	-81.787\\
1025	-59.814\\
1026	-97.656\\
1027	-83.008\\
1028	-72.021\\
1029	-118.408\\
1030	-107.422\\
1031	-83.008\\
1032	-119.629\\
1033	-123.291\\
1034	-95.215\\
1035	-80.566\\
1036	-62.256\\
1037	-75.684\\
1038	-95.215\\
1039	-87.891\\
1040	-91.553\\
1041	-96.436\\
1042	-102.539\\
1043	-141.602\\
1044	-128.174\\
1045	-85.449\\
1046	-79.346\\
1047	-47.607\\
1048	-48.828\\
1049	-68.359\\
1050	-52.49\\
1051	-57.373\\
1052	-75.684\\
1053	-89.111\\
1054	-81.787\\
1055	-128.174\\
1056	-98.877\\
1057	-57.373\\
1058	-45.166\\
1059	-62.256\\
1060	-72.021\\
1061	-40.283\\
1062	-42.725\\
1063	-56.152\\
1064	-46.387\\
1065	-40.283\\
1066	-52.49\\
1067	-62.256\\
1068	-92.773\\
1069	-90.332\\
1070	-107.422\\
1071	-98.877\\
1072	-115.967\\
1073	-91.553\\
1074	-91.553\\
1075	-79.346\\
1076	-75.684\\
1077	-76.904\\
1078	-133.057\\
1079	-189.209\\
1080	-194.092\\
1081	-189.209\\
1082	-119.629\\
1083	-170.898\\
1084	-213.623\\
1085	-219.727\\
1086	-169.678\\
1087	-219.727\\
1088	-279.541\\
1089	-197.754\\
1090	-161.133\\
1091	-109.863\\
1092	-85.449\\
1093	-73.242\\
1094	-91.553\\
1095	-62.256\\
1096	-46.387\\
1097	-42.725\\
1098	-54.932\\
1099	-79.346\\
1100	-74.463\\
1101	-75.684\\
1102	-100.098\\
1103	-103.76\\
1104	-141.602\\
1105	-114.746\\
1106	-142.822\\
1107	-104.98\\
1108	-90.332\\
1109	-103.76\\
1110	-76.904\\
1111	-69.58\\
1112	-84.229\\
1113	-86.67\\
1114	-59.814\\
1115	-64.697\\
1116	-48.828\\
1117	-53.711\\
1118	-57.373\\
1119	-41.504\\
1120	-52.49\\
1121	-65.918\\
1122	-101.318\\
1123	-104.98\\
1124	-114.746\\
1125	-68.359\\
1126	-93.994\\
1127	-150.146\\
1128	-114.746\\
1129	-125.732\\
1130	-128.174\\
1131	-80.566\\
1132	-53.711\\
1133	-75.684\\
1134	-85.449\\
1135	-137.939\\
1136	-172.119\\
1137	-170.898\\
1138	-123.291\\
1139	-130.615\\
1140	-115.967\\
1141	-113.525\\
1142	-111.084\\
1143	-76.904\\
1144	-68.359\\
1145	-61.035\\
1146	-57.373\\
1147	-62.256\\
1148	-91.553\\
1149	-140.381\\
1150	-111.084\\
1151	-79.346\\
1152	-64.697\\
1153	-62.256\\
1154	-40.283\\
1155	-29.297\\
1156	-36.621\\
1157	-63.477\\
1158	-51.27\\
1159	-52.49\\
1160	-58.594\\
1161	-54.932\\
1162	-37.842\\
1163	-28.076\\
1164	-23.193\\
1165	-39.063\\
1166	-83.008\\
1167	-117.188\\
1168	-130.615\\
1169	-86.67\\
1170	-58.594\\
1171	-45.166\\
1172	-32.959\\
1173	-67.139\\
1174	-65.918\\
1175	-91.553\\
1176	-117.188\\
1177	-186.768\\
1178	-216.064\\
1179	-177.002\\
1180	-168.457\\
1181	-109.863\\
1182	-122.07\\
1183	-131.836\\
1184	-146.484\\
1185	-108.643\\
1186	-100.098\\
1187	-98.877\\
1188	-89.111\\
1189	-106.201\\
1190	-78.125\\
1191	-130.615\\
1192	-159.912\\
1193	-113.525\\
1194	-70.801\\
1195	-73.242\\
1196	-123.291\\
1197	-153.809\\
1198	-189.209\\
1199	-202.637\\
1200	-202.637\\
1201	-153.809\\
1202	-137.939\\
1203	-158.691\\
1204	-187.988\\
1205	-109.863\\
1206	-72.021\\
1207	-101.318\\
1208	-86.67\\
1209	-53.711\\
1210	-74.463\\
1211	-84.229\\
1212	-67.139\\
1213	-51.27\\
1214	-70.801\\
1215	-58.594\\
1216	-79.346\\
1217	-107.422\\
1218	-100.098\\
1219	-69.58\\
1220	-70.801\\
1221	-107.422\\
1222	-177.002\\
1223	-128.174\\
1224	-79.346\\
1225	-61.035\\
1226	-70.801\\
1227	-64.697\\
1228	-48.828\\
1229	-53.711\\
1230	-65.918\\
1231	-83.008\\
1232	-91.553\\
1233	-63.477\\
1234	-104.98\\
1235	-124.512\\
1236	-118.408\\
1237	-91.553\\
1238	-100.098\\
1239	-135.498\\
1240	-87.891\\
1241	-42.725\\
1242	-57.373\\
1243	-62.256\\
1244	-81.787\\
1245	-72.021\\
1246	-52.49\\
1247	-79.346\\
1248	-80.566\\
1249	-57.373\\
1250	-70.801\\
1251	-63.477\\
1252	-56.152\\
1253	-67.139\\
1254	-41.504\\
1255	-48.828\\
1256	-36.621\\
1257	-40.283\\
1258	-75.684\\
1259	-84.229\\
1260	-113.525\\
1261	-146.484\\
1262	-98.877\\
1263	-58.594\\
1264	-46.387\\
1265	-43.945\\
1266	-63.477\\
1267	-42.725\\
1268	-47.607\\
1269	-61.035\\
1270	-102.539\\
1271	-93.994\\
1272	-134.277\\
1273	-89.111\\
1274	-81.787\\
1275	-46.387\\
1276	-45.166\\
1277	-28.076\\
1278	-35.4\\
1279	-37.842\\
1280	-67.139\\
1281	-70.801\\
1282	-79.346\\
1283	-85.449\\
1284	-125.732\\
1285	-112.305\\
1286	-84.229\\
1287	-102.539\\
1288	-109.863\\
1289	-144.043\\
1290	-115.967\\
1291	-75.684\\
1292	-40.283\\
1293	-29.297\\
1294	-29.297\\
1295	-36.621\\
1296	-39.063\\
1297	-37.842\\
1298	-50.049\\
1299	-74.463\\
1300	-68.359\\
1301	-70.801\\
1302	-83.008\\
1303	-51.27\\
1304	-30.518\\
1305	-58.594\\
1306	-90.332\\
1307	-87.891\\
1308	-97.656\\
1309	-83.008\\
1310	-61.035\\
1311	-85.449\\
1312	-118.408\\
1313	-118.408\\
1314	-84.229\\
1315	-136.719\\
1316	-100.098\\
1317	-101.318\\
1318	-117.188\\
1319	-115.967\\
1320	-86.67\\
1321	-76.904\\
1322	-128.174\\
1323	-178.223\\
1324	-139.16\\
1325	-79.346\\
1326	-76.904\\
1327	-79.346\\
1328	-98.877\\
1329	-114.746\\
1330	-85.449\\
1331	-90.332\\
1332	-120.85\\
1333	-131.836\\
1334	-87.891\\
1335	-68.359\\
1336	-91.553\\
1337	-156.25\\
1338	-137.939\\
1339	-140.381\\
1340	-84.229\\
1341	-76.904\\
1342	-72.021\\
1343	-47.607\\
1344	-30.518\\
1345	-26.855\\
1346	-23.193\\
1347	-54.932\\
1348	-70.801\\
1349	-87.891\\
1350	-65.918\\
1351	-73.242\\
1352	-83.008\\
1353	-53.711\\
1354	-56.152\\
1355	-53.711\\
1356	-40.283\\
1357	-51.27\\
1358	-84.229\\
1359	-106.201\\
1360	-63.477\\
1361	-48.828\\
1362	-40.283\\
1363	-50.049\\
1364	-35.4\\
1365	-76.904\\
1366	-130.615\\
1367	-104.98\\
1368	-139.16\\
1369	-162.354\\
1370	-164.795\\
1371	-124.512\\
1372	-90.332\\
1373	-89.111\\
1374	-89.111\\
1375	-106.201\\
1376	-125.732\\
1377	-125.732\\
1378	-168.457\\
1379	-183.105\\
1380	-219.727\\
1381	-147.705\\
1382	-166.016\\
1383	-184.326\\
1384	-140.381\\
1385	-162.354\\
1386	-139.16\\
1387	-80.566\\
1388	-52.49\\
1389	-56.152\\
1390	-81.787\\
1391	-65.918\\
1392	-43.945\\
1393	-62.256\\
1394	-47.607\\
1395	-36.621\\
1396	-46.387\\
1397	-52.49\\
1398	-36.621\\
1399	-70.801\\
1400	-92.773\\
1401	-68.359\\
1402	-114.746\\
1403	-164.795\\
1404	-150.146\\
1405	-196.533\\
1406	-131.836\\
1407	-144.043\\
1408	-96.436\\
1409	-63.477\\
1410	-76.904\\
1411	-59.814\\
1412	-45.166\\
1413	-64.697\\
1414	-36.621\\
1415	-28.076\\
1416	-34.18\\
1417	-70.801\\
1418	-86.67\\
1419	-107.422\\
1420	-103.76\\
1421	-100.098\\
1422	-114.746\\
1423	-93.994\\
1424	-76.904\\
1425	-76.904\\
1426	-62.256\\
1427	-97.656\\
1428	-68.359\\
1429	-56.152\\
1430	-81.787\\
1431	-113.525\\
1432	-86.67\\
1433	-65.918\\
1434	-56.152\\
1435	-65.918\\
1436	-45.166\\
1437	-45.166\\
1438	-59.814\\
1439	-75.684\\
1440	-84.229\\
1441	-65.918\\
1442	-86.67\\
1443	-79.346\\
1444	-47.607\\
1445	-50.049\\
1446	-95.215\\
1447	-131.836\\
1448	-131.836\\
1449	-107.422\\
1450	-103.76\\
1451	-79.346\\
1452	-76.904\\
1453	-61.035\\
1454	-89.111\\
1455	-75.684\\
1456	-50.049\\
1457	-74.463\\
1458	-85.449\\
1459	-101.318\\
1460	-109.863\\
1461	-109.863\\
1462	-148.926\\
1463	-181.885\\
1464	-205.078\\
1465	-129.395\\
1466	-73.242\\
1467	-47.607\\
1468	-34.18\\
1469	-54.932\\
1470	-46.387\\
1471	-80.566\\
1472	-72.021\\
1473	-64.697\\
1474	-85.449\\
1475	-98.877\\
1476	-117.188\\
1477	-123.291\\
1478	-175.781\\
1479	-150.146\\
1480	-117.188\\
1481	-80.566\\
1482	-80.566\\
1483	-83.008\\
1484	-106.201\\
1485	-75.684\\
1486	-79.346\\
1487	-74.463\\
1488	-42.725\\
1489	-74.463\\
1490	-107.422\\
1491	-73.242\\
1492	-80.566\\
1493	-147.705\\
1494	-109.863\\
1495	-106.201\\
1496	-147.705\\
1497	-140.381\\
1498	-87.891\\
1499	-56.152\\
1500	-58.594\\
};
\addlegendentry{True output}

\addplot [color=mycolor2, dashed]
  table[row sep=crcr]{%
1001	-85.8762831091721\\
1002	-108.784124280739\\
1003	-93.9992083274412\\
1004	-89.4789624272588\\
1005	-119.594908486225\\
1006	-105.372710246536\\
1007	-132.751137435116\\
1008	-104.934380142473\\
1009	-50.7603990602984\\
1010	-64.7964254677585\\
1011	-63.0900594685661\\
1012	-84.4139013907196\\
1013	-69.2129213510655\\
1014	-26.7701659676819\\
1015	-18.3267116830798\\
1016	-15.8752908495389\\
1017	-53.0749613737942\\
1018	-97.3173530145666\\
1019	-102.852597882015\\
1020	-98.0089509928733\\
1021	-69.5598964387952\\
1022	-38.1337309924441\\
1023	-92.1835488384898\\
1024	-61.2330399498336\\
1025	-51.6248712414172\\
1026	-87.1623819948547\\
1027	-71.4212047042794\\
1028	-63.5644613469746\\
1029	-100.657978145167\\
1030	-87.9968163200163\\
1031	-70.9663826251699\\
1032	-98.0248268221356\\
1033	-94.9412663045923\\
1034	-80.7530684364841\\
1035	-68.4956403859135\\
1036	-54.3481196875678\\
1037	-66.2226993095825\\
1038	-77.4636196494415\\
1039	-78.0917839912172\\
1040	-77.828391945307\\
1041	-79.7058035541552\\
1042	-80.8803123338785\\
1043	-114.386210241492\\
1044	-105.336259097501\\
1045	-75.0901030046798\\
1046	-67.6863329672772\\
1047	-36.4167006177363\\
1048	-39.506864257619\\
1049	-59.4440585831399\\
1050	-46.7759632394666\\
1051	-51.5225140776071\\
1052	-68.6417201309655\\
1053	-74.7719187548922\\
1054	-71.5512758302251\\
1055	-109.227114520831\\
1056	-79.7410015091079\\
1057	-48.1126539841226\\
1058	-39.4806105364441\\
1059	-49.3065077010599\\
1060	-63.3154475886696\\
1061	-34.8507266176985\\
1062	-38.5350135666809\\
1063	-49.2208093931926\\
1064	-40.8326772661673\\
1065	-38.5003692829551\\
1066	-47.8762429578702\\
1067	-48.911344420182\\
1068	-86.508459894091\\
1069	-73.0369177410674\\
1070	-91.1264827755545\\
1071	-80.5817977775393\\
1072	-87.0690255578996\\
1073	-74.7071421228594\\
1074	-71.5432286051009\\
1075	-66.2457510656305\\
1076	-66.363738058722\\
1077	-64.9174221145264\\
1078	-105.872896976835\\
1079	-156.124769090965\\
1080	-161.94947730798\\
1081	-152.350895082567\\
1082	-97.4821982350004\\
1083	-138.243356128133\\
1084	-167.53423672494\\
1085	-175.711469769163\\
1086	-145.932006448752\\
1087	-173.473656946891\\
1088	-230.108805763586\\
1089	-177.495083045445\\
1090	-152.002737918958\\
1091	-100.85651253904\\
1092	-78.4133486339169\\
1093	-71.208091339476\\
1094	-89.2222089208183\\
1095	-64.773242430858\\
1096	-41.5752509613795\\
1097	-38.1948246236963\\
1098	-51.9320591543516\\
1099	-78.182584796106\\
1100	-72.7743217952733\\
1101	-73.9528340307556\\
1102	-90.0847661694956\\
1103	-94.0705241623871\\
1104	-126.594744003361\\
1105	-111.535185796976\\
1106	-120.378976807539\\
1107	-94.8696078393408\\
1108	-79.6216132821677\\
1109	-95.1129775828472\\
1110	-72.3365820869673\\
1111	-67.8755614458916\\
1112	-83.0992826355222\\
1113	-81.8308204407072\\
1114	-64.242506340193\\
1115	-66.0856216491263\\
1116	-48.3029961632262\\
1117	-50.7108201636143\\
1118	-58.449865059395\\
1119	-44.3357532594332\\
1120	-53.1787602508411\\
1121	-63.3663685430911\\
1122	-98.3241621883036\\
1123	-91.4223111876355\\
1124	-104.602685099613\\
1125	-57.7562156826426\\
1126	-84.0883195956316\\
1127	-127.748678299861\\
1128	-103.126129951981\\
1129	-114.12956090748\\
1130	-108.02638762792\\
1131	-68.5565722357187\\
1132	-44.7634549544086\\
1133	-67.8557405941594\\
1134	-75.5630689213167\\
1135	-128.194148132515\\
1136	-153.480877514927\\
1137	-147.864905872582\\
1138	-106.520702724974\\
1139	-116.338667736026\\
1140	-102.684118532217\\
1141	-102.064748729559\\
1142	-105.158200865756\\
1143	-70.2755744837186\\
1144	-66.5957499650008\\
1145	-58.9891005648119\\
1146	-55.1197417979864\\
1147	-65.1699525457851\\
1148	-90.9778324047824\\
1149	-127.91462084839\\
1150	-111.906152922604\\
1151	-72.9958598788125\\
1152	-63.877401916698\\
1153	-58.6770659326691\\
1154	-35.8627631627116\\
1155	-26.6142974751909\\
1156	-33.2279480009013\\
1157	-56.3502360909423\\
1158	-47.3307510279854\\
1159	-53.8169157336715\\
1160	-53.188489096499\\
1161	-51.2101466057516\\
1162	-36.8145451852406\\
1163	-29.1924887907089\\
1164	-21.5791230282536\\
1165	-32.4748858307246\\
1166	-75.1263823339248\\
1167	-110.06808308038\\
1168	-116.110427865471\\
1169	-78.314188495095\\
1170	-48.7267670020994\\
1171	-35.3437368194996\\
1172	-24.1705404104341\\
1173	-57.6064304872029\\
1174	-55.4798792764699\\
1175	-80.5888902230449\\
1176	-101.085646926058\\
1177	-167.27816215922\\
1178	-194.108810672455\\
1179	-160.481731475629\\
1180	-142.40136728142\\
1181	-94.7798127953719\\
1182	-106.062849095658\\
1183	-120.718739060434\\
1184	-128.488708650217\\
1185	-108.394942999058\\
1186	-90.9321733012043\\
1187	-99.475925367418\\
1188	-90.7335619354821\\
1189	-101.892610885516\\
1190	-79.7301442329217\\
1191	-123.160752125015\\
1192	-150.715384928792\\
1193	-107.700577138832\\
1194	-72.6281651455957\\
1195	-67.8914877763055\\
1196	-113.05252760212\\
1197	-142.893891216864\\
1198	-163.725217571046\\
1199	-172.267853902642\\
1200	-173.846390905472\\
1201	-133.213668714255\\
1202	-126.163012731312\\
1203	-140.602080038357\\
1204	-163.357301777031\\
1205	-101.406857612175\\
1206	-65.2388230765356\\
1207	-100.683829455895\\
1208	-84.1826348417494\\
1209	-55.9677390191809\\
1210	-75.4747212663817\\
1211	-83.2568465786494\\
1212	-69.6619374838555\\
1213	-61.6796589562525\\
1214	-71.4465698934263\\
1215	-61.9732825642788\\
1216	-74.7911866348237\\
1217	-110.026023767852\\
1218	-95.866410701024\\
1219	-72.9695868397996\\
1220	-69.0703715502117\\
1221	-98.9467144220961\\
1222	-162.631067557506\\
1223	-131.064287491573\\
1224	-76.2743171590869\\
1225	-61.1094374311413\\
1226	-65.7472712202646\\
1227	-67.4707990203507\\
1228	-49.1846393558442\\
1229	-57.8858708304647\\
1230	-64.8110195486875\\
1231	-80.2596458343279\\
1232	-91.5658369643472\\
1233	-63.9242575266956\\
1234	-97.3887562487172\\
1235	-115.410891020333\\
1236	-104.741832608371\\
1237	-85.7331256523242\\
1238	-89.7713173573033\\
1239	-115.927747744489\\
1240	-88.8155623726259\\
1241	-33.2357993247614\\
1242	-46.5717359336632\\
1243	-54.8017938460728\\
1244	-80.0854941073337\\
1245	-73.1984594191224\\
1246	-56.401789916101\\
1247	-80.0472630603189\\
1248	-71.2060426368483\\
1249	-54.4118132145828\\
1250	-70.8100752755705\\
1251	-59.4731918255339\\
1252	-54.6794427875095\\
1253	-66.8551147065798\\
1254	-39.9650367043296\\
1255	-48.7852060814489\\
1256	-34.7756462284547\\
1257	-39.2544096950069\\
1258	-73.0580630020344\\
1259	-74.7932135017664\\
1260	-100.497535288483\\
1261	-130.693515383372\\
1262	-87.8100380905156\\
1263	-51.7345068039808\\
1264	-37.2405171413641\\
1265	-36.4251435658494\\
1266	-61.9997674367019\\
1267	-40.6456362225111\\
1268	-46.4622047075623\\
1269	-55.6033302800259\\
1270	-94.089563314563\\
1271	-86.1126597256405\\
1272	-119.315482311144\\
1273	-81.2779090201304\\
1274	-66.8850140031823\\
1275	-34.9807099453937\\
1276	-34.4895028514446\\
1277	-23.0428447769049\\
1278	-29.7141004667976\\
1279	-34.8118489521966\\
1280	-59.4229005871432\\
1281	-64.413870833479\\
1282	-69.0580530576396\\
1283	-70.8604767867404\\
1284	-105.816827374977\\
1285	-104.227579606376\\
1286	-73.7347695262345\\
1287	-84.826254621154\\
1288	-91.8196671379012\\
1289	-117.470859510051\\
1290	-96.8522234237031\\
1291	-65.3893510057121\\
1292	-27.4918786965907\\
1293	-19.5794628294214\\
1294	-22.8966581103761\\
1295	-31.5127686733018\\
1296	-36.3986226930307\\
1297	-36.5745036461059\\
1298	-42.3802331229758\\
1299	-68.782095295184\\
1300	-62.3194595420974\\
1301	-60.3264798737706\\
1302	-67.4752580598264\\
1303	-41.8633650841853\\
1304	-26.1551292055853\\
1305	-47.4366824455269\\
1306	-73.4418266131094\\
1307	-73.4542661694572\\
1308	-84.7289678300993\\
1309	-66.3038217382843\\
1310	-49.7084065419002\\
1311	-67.6611681982403\\
1312	-92.970517199196\\
1313	-96.5338823235853\\
1314	-71.0880633226789\\
1315	-111.013661756372\\
1316	-85.6649431720042\\
1317	-80.2090747616906\\
1318	-100.802862636313\\
1319	-94.0522796568385\\
1320	-78.2918109016627\\
1321	-70.7638531389289\\
1322	-108.256727383596\\
1323	-155.324258118517\\
1324	-121.966864452667\\
1325	-71.3447144505018\\
1326	-63.3968303075712\\
1327	-72.7771024406351\\
1328	-86.1926532240595\\
1329	-100.447950064312\\
1330	-79.1834082764696\\
1331	-86.386891478547\\
1332	-105.629891576995\\
1333	-108.874691952224\\
1334	-78.5698047907948\\
1335	-68.3457773042127\\
1336	-83.6259813588194\\
1337	-142.328902770289\\
1338	-125.264123581255\\
1339	-121.788278061586\\
1340	-75.6890134497439\\
1341	-63.9981213539003\\
1342	-66.325257934072\\
1343	-41.4781565668884\\
1344	-26.8343682633372\\
1345	-21.6041507431188\\
1346	-19.6442229137285\\
1347	-45.6772134085831\\
1348	-66.0578715924101\\
1349	-81.9564124519156\\
1350	-61.6571821762035\\
1351	-60.1553654227696\\
1352	-67.4134384666234\\
1353	-41.5377714412497\\
1354	-46.8785713320144\\
1355	-44.630826002835\\
1356	-34.5452908185203\\
1357	-46.6425060444052\\
1358	-69.8960393842048\\
1359	-90.4857408702799\\
1360	-54.7839193079671\\
1361	-42.557196229673\\
1362	-35.7667041644564\\
1363	-44.3542246289337\\
1364	-30.8619476549391\\
1365	-71.5954633413298\\
1366	-117.820251991978\\
1367	-96.6424283968202\\
1368	-117.084286949369\\
1369	-131.281897157965\\
1370	-134.937189659449\\
1371	-101.030258813131\\
1372	-74.8861560776439\\
1373	-78.6809474202002\\
1374	-77.6430257411513\\
1375	-94.6283721686031\\
1376	-109.770041193298\\
1377	-106.471412017524\\
1378	-140.68185127602\\
1379	-155.276279263581\\
1380	-179.085316174912\\
1381	-130.122714158473\\
1382	-143.114573586038\\
1383	-151.235740440644\\
1384	-127.778222462798\\
1385	-143.478540799289\\
1386	-127.519001795906\\
1387	-74.2197484417393\\
1388	-44.1342724878998\\
1389	-48.6772531852347\\
1390	-75.098805671269\\
1391	-65.0788116609935\\
1392	-48.0040672032164\\
1393	-64.4307884478713\\
1394	-46.1488574481558\\
1395	-38.6035997772589\\
1396	-46.6240040036867\\
1397	-45.9897926198788\\
1398	-40.5282852069521\\
1399	-64.5954809321952\\
1400	-84.926419238619\\
1401	-64.7296777396992\\
1402	-107.982629554391\\
1403	-154.573034193076\\
1404	-139.630568183439\\
1405	-169.691793930278\\
1406	-115.746168111641\\
1407	-125.927597185961\\
1408	-86.2492677128511\\
1409	-50.6542744569439\\
1410	-68.6021774836962\\
1411	-57.77852259256\\
1412	-44.3089241507875\\
1413	-61.054812622758\\
1414	-40.0092253523359\\
1415	-24.2635374336246\\
1416	-31.0046411072238\\
1417	-61.1038023480658\\
1418	-84.1580619448106\\
1419	-98.3111495770722\\
1420	-94.851193946791\\
1421	-83.1207129931809\\
1422	-93.37614498032\\
1423	-78.6591172340538\\
1424	-67.1809826604063\\
1425	-69.3664455593897\\
1426	-57.0803617790559\\
1427	-88.6949358217139\\
1428	-59.3592212170825\\
1429	-52.3280580472745\\
1430	-77.1933802985452\\
1431	-97.508930298979\\
1432	-78.2429766923298\\
1433	-61.2232013067576\\
1434	-53.7292449271272\\
1435	-60.7584652090472\\
1436	-42.3137888438046\\
1437	-41.9702016847961\\
1438	-56.4876040046579\\
1439	-64.9892518423339\\
1440	-75.0551523367701\\
1441	-62.9356688708432\\
1442	-74.299218530446\\
1443	-66.7051112334447\\
1444	-41.9116062625669\\
1445	-45.893008668356\\
1446	-82.5229697163427\\
1447	-122.288983438513\\
1448	-114.608465279946\\
1449	-92.9298635011883\\
1450	-85.7570074342962\\
1451	-71.501670767595\\
1452	-66.0040661589164\\
1453	-58.9969253655803\\
1454	-74.9233372990496\\
1455	-73.6603956078856\\
1456	-46.2228543449492\\
1457	-70.5874902568005\\
1458	-72.0300302271355\\
1459	-86.7306166273478\\
1460	-97.9152009659649\\
1461	-88.8830544135171\\
1462	-130.050384644329\\
1463	-153.805041602\\
1464	-171.912477594926\\
1465	-119.784829711513\\
1466	-59.5548643132568\\
1467	-34.5384187682205\\
1468	-23.6486063109028\\
1469	-43.9671176249208\\
1470	-41.4966668830162\\
1471	-77.2744904439316\\
1472	-67.549829683471\\
1473	-61.1458205866426\\
1474	-76.3728687540093\\
1475	-82.7232702780406\\
1476	-100.454179229279\\
1477	-104.927569388569\\
1478	-150.566793277589\\
1479	-137.57221339616\\
1480	-103.064818564611\\
1481	-75.9499256963497\\
1482	-69.176146007848\\
1483	-81.0944512296612\\
1484	-96.9794651788868\\
1485	-74.4555623786637\\
1486	-74.1623903209633\\
1487	-77.8991157568218\\
1488	-36.4623475325687\\
1489	-68.8618983977556\\
1490	-96.6528720857114\\
1491	-71.7691564000188\\
1492	-71.9612816482287\\
1493	-138.918179977806\\
1494	-95.8542301660225\\
1495	-98.001525643261\\
1496	-126.562581833742\\
1497	-118.632011738493\\
1498	-83.1300001156759\\
1499	-51.5159970659107\\
1500	-51.4897540104418\\
};
\addlegendentry{Generated output}

\end{axis}
\end{tikzpicture}%\label{fig:c8all}}
	\caption{Samples of the output obtained experimentally and the output generated by the model identified from all datasets.}\label{fig:Betas_direct_all}
\end{figure}
\printbibliography

\end{document}