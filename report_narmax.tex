\documentclass[a4paper,11pt,twoside]{article}
%%%%%%%%%%%%%%%%%%%%%%%%%%%%%%%%%%%%%%%%%%%%%%%%%%%%%%%%%%%%%%%%%%%%%%%%%%%
% Packages
\usepackage[final]{pdfpages}
\usepackage{verbatim}
\usepackage{inputenc}
\usepackage{graphicx} 
\usepackage{amsmath,amssymb,mathrsfs,amsfonts}
\usepackage{mathtools}
\usepackage{amsthm}
\usepackage{mathtools}
\usepackage{calrsfs}
\usepackage{graphicx}
\usepackage{subfig}
\usepackage{eucal}    
\usepackage{amssymb}  
\usepackage{pifont}
\usepackage{color} 
\usepackage{cancel}
\usepackage[toc,page]{appendix}
\usepackage{pgfplots}
\pgfplotsset{every axis/.append style={line width=0.5pt},label style={font=\scriptsize},tick label style={font=\scriptsize},x tick label style={/pgf/number format/.cd,fixed,precision=3, set thousands separator={}},z tick label style={/pgf/number format/.cd,fixed,precision=3, set thousands separator={}}}
\usetikzlibrary{shapes,shadows,arrows,backgrounds,patterns,positioning,automata,calc,decorations.markings,decorations.pathreplacing,bayesnet,arrows.meta}
%\usepackage{tikzexternal}
%\usepackage{tikz}
%\usepgfplotslibrary{external} 
%\tikzexternalize
\usepackage{varwidth}
\usepackage{lscape}
\usepackage{array} 
\usepackage[colorlinks=false,pdfborder={0 0 0}]{hyperref}
\usepackage{tabularx}
\usepackage{textcomp}
\usepackage{multicol} 
\usepackage{booktabs}
\usepackage{multirow}
\usepackage[font=small,labelfont=bf]{caption}                                                           
\usepackage{textcase}
\usepackage{bbm} 
\usepackage{fancyhdr}
\usepackage{enumitem}
\usepackage{soul}
%\usepackage[british]{babel}
\usepackage{wrapfig}
%\usepackage{glossaries}
%%%%%%%%%%%%%%%%%%%%%%%%%%%%%%%%%%%%%%%%%%%%%%%%%%%%%%%%%%%%%%%%%%%%%%%%%%%
\setlength{\parindent}{2em}
\setlength{\parskip}{0.5em}
\renewcommand{\baselinestretch}{1.2}
\usepackage[left=2cm, right=2cm, top=2.5cm, bottom=3cm, headheight=13.6pt]{geometry}
\allowdisplaybreaks 
% Bibliography
%\usepackage[backend=bibtex,style=ieee,sorting=none]{biblatex} 
%\bibliography{Bibliography-thesis}
%\renewcommand*{\bibfont}{\scriptsize}
\makeatletter
\newcommand*{\rom}[1]{\expandafter\@slowromancap\romannumeral #1@}
\newcommand{\ie}{\textit{i.e.} }
\newcommand{\eg}{\textit{e.g.} }
\makeatother
\newcommand\id{\ensuremath{\mathbbm{1}}} 
\DeclareMathOperator{\E}{\mathbb{E}}
\DeclareMathOperator{\eye}{\mathbb{I}}
\DeclareMathOperator{\zeros}{\mathbb{O}}
\DeclareMathOperator{\tr}{\textrm{tr}}
\DeclareMathOperator{\vvec}{\textrm{vec}}
\DeclareMathOperator{\ik}{\mathrm{k}}
\DeclareMathOperator{\ip}{\mathrm{p}}
\DeclareMathOperator{\inn}{\mathrm{n}}
\DeclareMathOperator{\im}{\mathrm{m}}
\DeclareMathOperator{\td}{\mathrm{t}}
\DeclareMathOperator{\kd}{\mathrm{k}}
\DeclareMathOperator{\T}{\mathrm{T}}
\DeclareMathOperator{\K}{\mathrm{K}}
\DeclareSymbolFontAlphabet{\mathcal} {symbols}
\DeclareSymbolFont{symbols}{OMS}{cm}{m}{n}
\DeclareMathAlphabet{\mathbfit}{OML}{cmm}{b}{it}

% Number equations
%\numberwithin{equation}{section}

%%%%%%%%%%%%%%%%%%%%%%%%%%%%%%%%%%%%%%%%%%%%%%%%%%%%%%%%%%%%%%%%%%%%%%%%%%%
%Theorems
\newtheoremstyle{mytheoremstyle} % name
{.5em}                    % Space above
{.8em}                    % Space below
{\itshape}                % Body font
{1em}                           % Indent amount
{\bfseries}                   % Theorem head font
{:}                          % Punctuation after theorem head
{.5em}                       % Space after theorem head
{}  % Theorem head spec (can be left empty, meaning ‘normal’)

\theoremstyle{mytheoremstyle}
\newtheorem{theorem}{Theorem}[section]
\newtheorem{remark}{Remark}[section]
\newtheorem{assumption}{Assumption}[section]
\newtheorem{lemma}{Lemma}[section]
\newtheorem{condition}{Condition}[section]
\newtheorem{definition}{Definition}[section]
\newtheorem{property}{Property}[section]
\newtheorem{corollary}{Corollary}[section]
\renewcommand\qedsymbol{$\blacksquare$}
%%%%%%%%%%%%%%%%%%%%%%%%%%%%%%%%%%%%%%%%%%%%%%%%%%%%%%%%%%%%%%%%%%%%%%%%%%%
% Nomenclature
\usepackage[intoc]{nomencl}
\makenomenclature

%\usepackage[ruled,chapter]{algorithm}
%\usepackage{float}

%\usepackage{algorithmic}
%\algsetup{linenosize=\scriptsize}
%\usepackage{etoolbox}
%\AtBeginEnvironment{algorithmic}{\scriptsize}
%\renewcommand{\thealgorithm}{\thechapter.\arabic{algorithm}} 
%\usepackage{chngcntr}
%\counterwithin{algorithm}{section}

% correct bad hyphenation here
\hyphenation{op-tical net-works semi-conduc-tor}
%%%%%%%%%%%%%%%%%%%%%%%%%%%%%%%%%%%%%%%%%%%%%%%%%%%%%%%%%%%%%%%%%%%%%%%%%%%%%
% Captions
%\newcommand{\xLanguage}{british}          % <-- Added this command
%
%\usepackage[\xLanguage]{babel}             % <-- Implemented here
%
%\expandafter\addto\csname captions\xLanguage\endcsname{% <-- and here
%	\renewcommand{\tableshortname}{Table}%
%	\renewcommand{\figureshortname}{Figure}%
%}
%
%\captionsetup[table]{format=plain,indention=1.15cm,justification=justified}
%\captionsetup[figure]{format=plain,indention=1.25cm,justification=justified}

%%%%%%%%%%%%%%%%%%%%%%%%%%%%%%%%%%%%%%%%%%%%%%%%%%%%%%%%%%%%%%%%%%%%%%%%%%%%%
\usepackage[explicit]{titlesec}
\usepackage{titletoc}
%\titleformat{\section}[block]{\normalfont\Large\rm\filright\bfseries}{\thesection}{1em}{#1}
%\titlecontents{chapter}[1.5em]{}{\scshape\contentslabel{2.3em}}{}{\titlerule*[1pc]{}\contentspage}
%\definecolor{gray75}{gray}{0.5}
%\newcommand{\hsp}{\hspace{20pt}}
%\titleformat{\chapter}[hang]{\huge\scshape\filright\bfseries}{\color{gray75}\thechapter}{20pt}{\begin{tabular}[t]{@{\color{gray75}\vrule width 2pt\hsp}p{0.85\textwidth}}\raggedright#1\end{tabular}}
%\titleformat{name=\chapter,numberless}[display]{}{}{0pt}{\normalfont\huge\bfseries #1} % format for numberless chapters
%\titleformat{\subsection}[block]{\normalfont\large\rm\filright\bfseries}{\thesubsection}{1em}{#1}
%\renewcommand{\sectionmark}[1]{\markright{\thesection ~ \ #1}}
%%\renewcommand{\chaptermark}[1]{\markboth{\chaptername\ \thechapter ~ \ #1}{}} 
%\pagestyle{fancy}
%%\fancyhf{}
%\fancyhead[RO,LE]{\thepage}
%\fancyhead[RE]{\itshape \nouppercase \rightmark}      % chaptertitle left
%\fancyhead[LO]{\itshape \nouppercase \leftmark}       % sectiontitle right
%\renewcommand{\headrulewidth}{0.5pt} 				   % no rule
%\cfoot{}
\interfootnotelinepenalty=10000

\title{Structure and parameter identification of the dynamical model of auxetic foam}


\begin{document}
	\maketitle
%	\section{Experimental setup}
%	
\section{Experimental data}
\par The experiments is conducted on ten samples of authentic elastomeric foams prepared in different manufacturing settings. In this report, the influence of two manufacturing parameters on the dynamical behaviour of the foam is evaluated: the length of the foam sample after cutting, $L_{cut}$ (mm), and the relaxed density of the sample, $D_{rlx}$ (g/cm$^3$). The parameter values for each specimen are presented in Table \ref{tab:externparams}.
\begin{table}[!h]
	\centering
	\caption{manufacturing parameters of auxetic form used in 10 experiments.}\label{tab:externparams}
	\small
	\begin{tabular}{rrrrrrrrrrr}
		Parameter  & C1 & C2 & C3 & C4 & C5 & C6 & C7 & C8 & C9 & C10 \\ 
		\hline 
		$L_{cut}$ & 74 & 72 & 69 & 61 & 56 & 67 & 66 & 64 & 62 & 57 \\ 
		$D_{rlx}$ & 0.123 & 0.113 & 0.104 & 0.095 & 0.093 & 0.276 & 0.255 & 0.230 & 0.209 & .193 \\ 
		\hline 
	\end{tabular}
\end{table}
The random input signal is adopted for all foam specimen is illustrated in Figure \ref{fig:input}. 
\begin{figure}[!h]
	\centering
	% This file was created by matlab2tikz.
%
\definecolor{mycolor1}{rgb}{0.00000,0.44700,0.74100}%
%
\begin{tikzpicture}

\begin{axis}[%
width=11.411cm,
height=5cm,
at={(0cm,0cm)},
scale only axis,
xmin=1,
xmax=1500,
xlabel style={font=\color{white!15!black}},
xlabel={Sample index},
ymin=-7,
ymax=-4.999,
ylabel style={font=\color{white!15!black}},
ylabel={$\Delta x$, mm},
axis background/.style={fill=white},
legend style={legend cell align=left, align=left, draw=white!15!black}
]
\addplot [color=mycolor1]
  table[row sep=crcr]{%
1	-5.951\\
3	-5.933\\
5	-5.933\\
7	-5.914\\
9	-5.933\\
11	-5.933\\
13	-5.914\\
15	-5.914\\
17	-5.933\\
19	-5.933\\
21	-5.933\\
23	-5.933\\
25	-5.914\\
27	-5.914\\
29	-5.933\\
31	-5.933\\
33	-5.933\\
35	-5.914\\
37	-5.933\\
39	-5.933\\
41	-5.933\\
43	-5.933\\
45	-5.933\\
47	-5.933\\
49	-5.933\\
51	-5.933\\
53	-5.914\\
55	-5.914\\
57	-5.914\\
59	-5.933\\
61	-5.933\\
63	-5.933\\
65	-5.933\\
67	-5.914\\
69	-5.914\\
71	-5.914\\
73	-5.933\\
75	-5.914\\
77	-5.914\\
79	-5.933\\
81	-5.933\\
83	-5.951\\
85	-5.933\\
87	-5.933\\
89	-5.914\\
91	-5.933\\
93	-5.933\\
95	-5.933\\
97	-5.933\\
99	-5.914\\
101	-5.914\\
103	-5.933\\
105	-5.933\\
107	-5.933\\
109	-5.933\\
111	-5.914\\
113	-5.896\\
115	-5.878\\
117	-5.841\\
119	-5.859\\
121	-6.061\\
123	-6.226\\
125	-6.116\\
127	-6.171\\
129	-6.281\\
131	-6.061\\
133	-5.878\\
135	-6.226\\
137	-6.061\\
139	-6.226\\
141	-5.859\\
143	-5.933\\
145	-6.354\\
147	-6.079\\
149	-5.896\\
151	-5.64\\
153	-5.64\\
155	-5.365\\
157	-5.621\\
159	-5.878\\
161	-5.786\\
163	-5.548\\
165	-5.988\\
167	-6.207\\
169	-5.933\\
171	-6.097\\
173	-5.713\\
175	-5.511\\
177	-5.548\\
179	-5.42\\
181	-5.511\\
183	-5.2\\
185	-4.999\\
187	-5.219\\
189	-5.731\\
191	-5.566\\
193	-5.585\\
195	-5.621\\
197	-5.823\\
199	-6.079\\
201	-6.006\\
203	-5.475\\
205	-5.64\\
207	-5.786\\
209	-5.951\\
211	-5.841\\
213	-6.171\\
215	-6.207\\
217	-6.006\\
219	-5.731\\
221	-5.64\\
223	-5.969\\
225	-5.768\\
227	-5.841\\
229	-5.933\\
231	-6.299\\
233	-5.859\\
235	-5.878\\
237	-5.75\\
239	-5.493\\
241	-5.768\\
243	-5.951\\
245	-5.896\\
247	-5.933\\
249	-6.207\\
251	-5.914\\
253	-6.042\\
255	-5.64\\
257	-5.896\\
259	-5.823\\
261	-5.658\\
263	-5.585\\
265	-6.097\\
267	-5.768\\
269	-5.658\\
271	-5.695\\
273	-5.676\\
275	-5.914\\
277	-6.299\\
279	-6.299\\
281	-5.859\\
283	-5.804\\
285	-5.878\\
287	-6.024\\
289	-6.207\\
291	-5.969\\
293	-6.024\\
295	-6.061\\
297	-6.244\\
299	-6.097\\
301	-5.878\\
303	-5.804\\
305	-5.475\\
307	-5.328\\
309	-5.621\\
311	-5.621\\
313	-5.64\\
315	-5.804\\
317	-5.988\\
319	-6.335\\
321	-6.647\\
323	-6.354\\
325	-5.933\\
327	-6.042\\
329	-5.951\\
331	-5.896\\
333	-6.152\\
335	-6.793\\
337	-6.573\\
339	-6.519\\
341	-6.061\\
343	-5.75\\
345	-5.951\\
347	-5.64\\
349	-5.2\\
351	-5.585\\
353	-6.024\\
355	-6.464\\
357	-6.445\\
359	-6.299\\
361	-6.207\\
363	-6.354\\
365	-6.097\\
367	-6.171\\
369	-5.914\\
371	-5.859\\
373	-5.969\\
375	-6.537\\
377	-6.409\\
379	-6.061\\
381	-5.42\\
383	-5.383\\
385	-5.676\\
387	-6.024\\
389	-6.244\\
391	-6.72\\
393	-6.555\\
395	-6.152\\
397	-6.628\\
399	-6.39\\
401	-6.042\\
403	-5.804\\
405	-5.804\\
407	-5.493\\
409	-5.53\\
411	-5.878\\
413	-5.75\\
415	-6.152\\
417	-6.281\\
419	-6.354\\
421	-6.079\\
423	-5.969\\
425	-5.896\\
427	-5.969\\
429	-5.841\\
431	-5.951\\
433	-5.548\\
435	-6.006\\
437	-6.573\\
439	-6.592\\
441	-6.647\\
443	-6.281\\
445	-6.171\\
447	-5.878\\
449	-5.53\\
451	-5.53\\
453	-5.878\\
455	-6.226\\
457	-6.024\\
459	-6.317\\
461	-6.354\\
463	-6.024\\
465	-5.878\\
467	-5.768\\
469	-6.006\\
471	-5.933\\
473	-5.969\\
475	-6.024\\
477	-6.189\\
479	-6.189\\
481	-6.317\\
483	-6.006\\
485	-5.804\\
487	-5.896\\
489	-6.042\\
491	-6.042\\
493	-5.988\\
495	-5.75\\
497	-6.061\\
499	-5.914\\
501	-5.695\\
503	-5.585\\
505	-5.621\\
507	-5.933\\
509	-6.226\\
511	-6.354\\
513	-6.464\\
515	-6.573\\
517	-6.134\\
519	-5.841\\
521	-6.079\\
523	-5.713\\
525	-5.53\\
527	-5.548\\
529	-5.731\\
531	-5.933\\
533	-6.226\\
535	-6.061\\
537	-6.024\\
539	-6.006\\
541	-5.933\\
543	-6.116\\
545	-6.427\\
547	-6.244\\
549	-6.281\\
551	-5.969\\
553	-6.39\\
555	-5.914\\
557	-6.116\\
559	-6.207\\
561	-5.75\\
563	-5.713\\
565	-5.585\\
567	-5.658\\
569	-5.402\\
571	-5.219\\
573	-5.273\\
575	-5.566\\
577	-5.878\\
579	-6.097\\
581	-5.713\\
583	-5.493\\
585	-5.566\\
587	-5.237\\
589	-5.164\\
591	-5.493\\
593	-5.603\\
595	-5.475\\
597	-5.823\\
599	-6.079\\
601	-6.152\\
603	-6.097\\
605	-5.969\\
607	-5.786\\
609	-5.603\\
611	-5.786\\
613	-5.548\\
615	-5.951\\
617	-6.079\\
619	-6.079\\
621	-6.189\\
623	-6.042\\
625	-5.841\\
627	-5.804\\
629	-6.189\\
631	-6.207\\
633	-6.207\\
635	-5.878\\
637	-6.024\\
639	-6.244\\
641	-6.006\\
643	-5.878\\
645	-5.621\\
647	-5.914\\
649	-5.896\\
651	-5.493\\
653	-5.585\\
655	-5.878\\
657	-5.804\\
659	-5.951\\
661	-5.603\\
663	-5.53\\
665	-5.713\\
667	-5.823\\
669	-6.171\\
671	-6.262\\
673	-6.024\\
675	-6.116\\
677	-6.006\\
679	-6.006\\
681	-5.988\\
683	-6.189\\
685	-6.079\\
687	-6.189\\
689	-6.335\\
691	-6.372\\
693	-6.189\\
695	-6.39\\
697	-6.702\\
699	-6.262\\
701	-6.171\\
703	-6.134\\
705	-6.5\\
707	-6.226\\
709	-6.171\\
711	-6.024\\
713	-6.207\\
715	-6.317\\
717	-6.006\\
719	-5.878\\
721	-5.75\\
723	-5.75\\
725	-6.079\\
727	-6.281\\
729	-6.702\\
731	-6.39\\
733	-6.116\\
735	-5.786\\
737	-5.988\\
739	-6.189\\
741	-5.896\\
743	-6.042\\
745	-6.134\\
747	-6.189\\
749	-5.841\\
751	-5.969\\
753	-5.603\\
755	-5.951\\
757	-5.695\\
759	-5.768\\
761	-5.859\\
763	-6.354\\
765	-6.152\\
767	-6.042\\
769	-6.171\\
771	-6.079\\
773	-5.603\\
775	-5.713\\
777	-5.878\\
779	-5.988\\
781	-6.152\\
783	-5.933\\
785	-6.262\\
787	-6.207\\
789	-6.207\\
791	-6.207\\
793	-6.116\\
795	-5.75\\
797	-6.061\\
799	-6.006\\
801	-5.823\\
803	-5.896\\
805	-5.878\\
807	-5.786\\
809	-5.713\\
811	-5.64\\
813	-5.585\\
815	-5.841\\
817	-5.75\\
819	-6.134\\
821	-5.988\\
823	-6.061\\
825	-5.53\\
827	-5.676\\
829	-6.134\\
831	-6.061\\
833	-6.024\\
835	-6.207\\
837	-5.988\\
839	-5.786\\
841	-5.75\\
843	-5.768\\
845	-5.75\\
847	-6.244\\
849	-5.878\\
851	-5.457\\
853	-5.42\\
855	-5.493\\
857	-5.566\\
859	-6.024\\
861	-6.372\\
863	-6.244\\
865	-6.409\\
867	-6.042\\
869	-5.731\\
871	-5.713\\
873	-5.621\\
875	-5.768\\
877	-5.896\\
879	-5.988\\
881	-6.189\\
883	-6.061\\
885	-6.061\\
887	-6.226\\
889	-6.281\\
891	-6.39\\
893	-6.061\\
895	-5.951\\
897	-5.658\\
899	-5.621\\
901	-5.328\\
903	-5.731\\
905	-5.621\\
907	-5.511\\
909	-5.53\\
911	-5.786\\
913	-5.566\\
915	-6.134\\
917	-6.097\\
919	-6.372\\
921	-6.39\\
923	-6.226\\
925	-6.262\\
927	-5.768\\
929	-5.658\\
931	-5.676\\
933	-5.969\\
935	-5.676\\
937	-5.731\\
939	-5.383\\
941	-5.859\\
943	-5.676\\
945	-5.823\\
947	-5.804\\
949	-6.189\\
951	-6.244\\
953	-6.244\\
955	-6.061\\
957	-6.281\\
959	-6.262\\
961	-6.006\\
963	-5.988\\
965	-5.786\\
967	-5.64\\
969	-5.676\\
971	-5.823\\
973	-5.695\\
975	-5.896\\
977	-5.676\\
979	-5.695\\
981	-5.841\\
983	-5.804\\
985	-6.317\\
987	-5.933\\
989	-6.152\\
991	-6.189\\
993	-6.299\\
995	-6.537\\
997	-6.39\\
999	-6.152\\
1001	-6.244\\
1003	-6.134\\
1005	-6.281\\
1007	-6.244\\
1009	-5.914\\
1011	-5.969\\
1013	-5.585\\
1015	-5.145\\
1017	-5.896\\
1019	-6.061\\
1021	-5.731\\
1023	-5.914\\
1025	-5.933\\
1027	-5.878\\
1029	-6.079\\
1031	-6.079\\
1033	-6.061\\
1035	-5.878\\
1037	-5.969\\
1039	-5.969\\
1041	-6.006\\
1043	-6.207\\
1045	-5.969\\
1047	-5.658\\
1049	-5.676\\
1051	-5.804\\
1053	-5.859\\
1055	-6.042\\
1057	-5.658\\
1059	-5.786\\
1061	-5.493\\
1063	-5.566\\
1065	-5.548\\
1067	-5.841\\
1069	-5.969\\
1071	-6.024\\
1073	-5.951\\
1075	-5.878\\
1077	-6.116\\
1079	-6.519\\
1081	-6.335\\
1083	-6.61\\
1085	-6.555\\
1087	-6.866\\
1089	-6.665\\
1091	-6.244\\
1093	-6.152\\
1095	-5.786\\
1097	-5.695\\
1099	-5.896\\
1101	-6.006\\
1103	-6.226\\
1105	-6.281\\
1107	-6.097\\
1109	-6.042\\
1111	-6.006\\
1113	-5.914\\
1115	-5.804\\
1117	-5.768\\
1119	-5.676\\
1121	-5.969\\
1123	-6.116\\
1125	-5.988\\
1127	-6.226\\
1129	-6.281\\
1131	-5.859\\
1133	-5.969\\
1135	-6.409\\
1137	-6.354\\
1139	-6.281\\
1141	-6.262\\
1143	-5.969\\
1145	-5.841\\
1147	-6.006\\
1149	-6.207\\
1151	-5.951\\
1153	-5.695\\
1155	-5.457\\
1157	-5.64\\
1159	-5.658\\
1161	-5.53\\
1163	-5.219\\
1165	-5.676\\
1167	-6.134\\
1169	-5.804\\
1171	-5.438\\
1173	-5.713\\
1175	-6.024\\
1177	-6.61\\
1179	-6.555\\
1181	-6.335\\
1183	-6.427\\
1185	-6.244\\
1187	-6.171\\
1189	-6.116\\
1191	-6.427\\
1193	-6.079\\
1195	-6.226\\
1197	-6.537\\
1199	-6.683\\
1201	-6.5\\
1203	-6.61\\
1205	-6.152\\
1207	-6.171\\
1209	-5.969\\
1211	-5.969\\
1213	-5.896\\
1215	-5.914\\
1217	-6.134\\
1219	-5.969\\
1221	-6.427\\
1223	-6.152\\
1225	-5.951\\
1227	-5.804\\
1229	-5.859\\
1231	-6.006\\
1233	-6.042\\
1235	-6.207\\
1237	-6.134\\
1239	-6.152\\
1241	-5.731\\
1243	-5.896\\
1245	-5.786\\
1247	-5.933\\
1249	-5.878\\
1251	-5.804\\
1253	-5.676\\
1255	-5.548\\
1257	-5.768\\
1259	-6.024\\
1261	-6.152\\
1263	-5.695\\
1265	-5.75\\
1267	-5.585\\
1269	-5.896\\
1271	-6.134\\
1273	-5.951\\
1275	-5.566\\
1277	-5.328\\
1279	-5.603\\
1281	-5.768\\
1283	-6.097\\
1285	-6.024\\
1287	-6.116\\
1289	-6.226\\
1291	-5.695\\
1293	-5.383\\
1295	-5.42\\
1297	-5.457\\
1299	-5.731\\
1301	-5.786\\
1303	-5.402\\
1305	-5.804\\
1307	-5.914\\
1309	-5.75\\
1311	-6.024\\
1313	-5.988\\
1315	-6.134\\
1317	-6.134\\
1319	-6.061\\
1321	-6.171\\
1323	-6.39\\
1325	-6.006\\
1327	-6.079\\
1329	-6.079\\
1331	-6.171\\
1333	-6.116\\
1335	-6.024\\
1337	-6.299\\
1339	-6.097\\
1341	-5.969\\
1343	-5.511\\
1345	-5.2\\
1347	-5.676\\
1349	-5.768\\
1351	-5.841\\
1353	-5.621\\
1355	-5.511\\
1357	-5.768\\
1359	-5.768\\
1361	-5.548\\
1363	-5.457\\
1365	-6.079\\
1367	-6.189\\
1369	-6.39\\
1371	-6.134\\
1373	-6.061\\
1375	-6.207\\
1377	-6.39\\
1379	-6.628\\
1381	-6.519\\
1383	-6.482\\
1385	-6.482\\
1387	-5.933\\
1389	-5.969\\
1391	-5.75\\
1393	-5.713\\
1395	-5.566\\
1397	-5.53\\
1399	-5.896\\
1401	-6.024\\
1403	-6.354\\
1405	-6.39\\
1407	-6.226\\
1409	-5.969\\
1411	-5.731\\
1413	-5.676\\
1415	-5.365\\
1417	-5.859\\
1419	-6.042\\
1421	-6.097\\
1423	-5.951\\
1425	-5.841\\
1427	-5.914\\
1429	-5.878\\
1431	-6.006\\
1433	-5.804\\
1435	-5.713\\
1437	-5.676\\
1439	-5.878\\
1441	-5.878\\
1443	-5.695\\
1445	-5.896\\
1447	-6.207\\
1449	-6.116\\
1451	-5.951\\
1453	-5.951\\
1455	-5.768\\
1457	-5.914\\
1459	-6.079\\
1461	-6.262\\
1463	-6.592\\
1465	-6.097\\
1467	-5.603\\
1469	-5.621\\
1471	-5.841\\
1473	-5.896\\
1475	-6.079\\
1477	-6.372\\
1479	-6.317\\
1481	-6.061\\
1483	-6.152\\
1485	-6.006\\
1487	-5.75\\
1489	-6.061\\
1491	-5.969\\
1493	-6.189\\
1495	-6.299\\
1497	-6.134\\
1499	-5.823\\
1501	-6.281\\
};
\addlegendentry{data1}

\end{axis}
\end{tikzpicture}%
	\caption{The random input signal for the set of authentic foams is the displacement of the hydraulic actuator of the machine in mm.}\label{fig:input}
\end{figure}
Samples of recorded foam responses are presented in  Figure \ref{fig:output}. The ability to predict the response of a foam specimen to a known input based on manufacturing parameters is crucial in for optimal design. The aim of this work is to identify the dynamical model that could describe the relationship between measured input and output time series and the external parameters.	
\begin{figure}[!h]
	\centering
	% This file was created by matlab2tikz.
% Minimal pgfplots version: 1.3
%
\definecolor{mycolor1}{rgb}{0.00000,0.44700,0.74100}%
%
\begin{tikzpicture}

\begin{axis}[%
width=5.090835cm,
height=2.240107cm,
at={(0cm,0cm)},
scale only axis,
xmin=1000,
xmax=1500,
xlabel={Sample index},
ymin=-200,
ymax=0,
ylabel={C9 load, kN},
legend style={legend cell align=left,align=left,draw=white!15!black}
]
\addplot [color=mycolor1,solid,forget plot]
  table[row sep=crcr]{%
1000	-68.359\\
1001	-85.449\\
1002	-69.58\\
1003	-65.918\\
1004	-90.332\\
1005	-86.67\\
1006	-103.76\\
1007	-79.346\\
1008	-45.166\\
1009	-57.373\\
1010	-51.27\\
1011	-63.477\\
1012	-48.828\\
1013	-24.414\\
1014	-17.09\\
1015	-17.09\\
1016	-43.945\\
1017	-73.242\\
1018	-76.904\\
1019	-79.346\\
1020	-58.594\\
1021	-36.621\\
1022	-76.904\\
1023	-53.711\\
1024	-43.945\\
1025	-69.58\\
1026	-61.035\\
1027	-48.828\\
1028	-84.229\\
1029	-79.346\\
1030	-59.814\\
1031	-87.891\\
1032	-85.449\\
1033	-67.139\\
1034	-54.932\\
1035	-46.387\\
1036	-56.152\\
1037	-64.697\\
1038	-64.697\\
1039	-67.139\\
1040	-68.359\\
1041	-73.242\\
1042	-100.098\\
1043	-87.891\\
1044	-61.035\\
1045	-56.152\\
1046	-34.18\\
1047	-34.18\\
1048	-48.828\\
1049	-37.842\\
1050	-39.063\\
1051	-56.152\\
1052	-62.256\\
1053	-58.594\\
1054	-90.332\\
1055	-64.697\\
1056	-41.504\\
1057	-34.18\\
1058	-45.166\\
1059	-51.27\\
1060	-28.076\\
1061	-25.635\\
1062	-41.504\\
1063	-34.18\\
1064	-29.297\\
1065	-40.283\\
1066	-43.945\\
1067	-68.359\\
1068	-63.477\\
1069	-79.346\\
1070	-69.58\\
1071	-79.346\\
1072	-63.477\\
1073	-62.256\\
1074	-54.932\\
1075	-53.711\\
1076	-53.711\\
1077	-91.553\\
1078	-126.953\\
1079	-131.836\\
1080	-129.395\\
1081	-83.008\\
1082	-114.746\\
1083	-141.602\\
1084	-147.705\\
1085	-111.084\\
1086	-146.484\\
1087	-185.547\\
1088	-131.836\\
1089	-112.305\\
1090	-75.684\\
1091	-61.035\\
1092	-52.49\\
1093	-64.697\\
1094	-46.387\\
1095	-31.738\\
1096	-31.738\\
1097	-40.283\\
1098	-57.373\\
1099	-52.49\\
1100	-53.711\\
1101	-72.021\\
1102	-72.021\\
1103	-97.656\\
1104	-79.346\\
1105	-101.318\\
1106	-72.021\\
1107	-64.697\\
1108	-73.242\\
1109	-54.932\\
1110	-51.27\\
1111	-59.814\\
1112	-62.256\\
1113	-43.945\\
1114	-47.607\\
1115	-34.18\\
1116	-40.283\\
1117	-42.725\\
1118	-30.518\\
1119	-39.063\\
1120	-47.607\\
1121	-72.021\\
1122	-74.463\\
1123	-81.787\\
1124	-48.828\\
1125	-70.801\\
1126	-101.318\\
1127	-84.229\\
1128	-93.994\\
1129	-91.553\\
1130	-54.932\\
1131	-40.283\\
1132	-56.152\\
1133	-61.035\\
1134	-97.656\\
1135	-119.629\\
1136	-117.188\\
1137	-89.111\\
1138	-92.773\\
1139	-81.787\\
1140	-80.566\\
1141	-79.346\\
1142	-54.932\\
1143	-48.828\\
1144	-45.166\\
1145	-40.283\\
1146	-45.166\\
1147	-64.697\\
1148	-96.436\\
1149	-74.463\\
1150	-52.49\\
1151	-45.166\\
1152	-46.387\\
1153	-29.297\\
1154	-23.193\\
1155	-26.855\\
1156	-46.387\\
1157	-35.4\\
1158	-41.504\\
1159	-40.283\\
1160	-39.063\\
1161	-28.076\\
1162	-21.973\\
1163	-17.09\\
1164	-29.297\\
1165	-58.594\\
1166	-80.566\\
1167	-91.553\\
1168	-59.814\\
1169	-43.945\\
1170	-32.959\\
1171	-23.193\\
1172	-47.607\\
1173	-46.387\\
1174	-67.139\\
1175	-81.787\\
1176	-131.836\\
1177	-147.705\\
1178	-120.85\\
1179	-117.188\\
1180	-78.125\\
1181	-81.787\\
1182	-91.553\\
1183	-98.877\\
1184	-73.242\\
1185	-68.359\\
1186	-69.58\\
1187	-62.256\\
1188	-75.684\\
1189	-53.711\\
1190	-93.994\\
1191	-108.643\\
1192	-80.566\\
1193	-51.27\\
1194	-54.932\\
1195	-92.773\\
1196	-109.863\\
1197	-134.277\\
1198	-140.381\\
1199	-140.381\\
1200	-106.201\\
1201	-96.436\\
1202	-111.084\\
1203	-129.395\\
1204	-78.125\\
1205	-50.049\\
1206	-73.242\\
1207	-57.373\\
1208	-39.063\\
1209	-56.152\\
1210	-61.035\\
1211	-46.387\\
1212	-37.842\\
1213	-50.049\\
1214	-39.063\\
1215	-53.711\\
1216	-76.904\\
1217	-72.021\\
1218	-51.27\\
1219	-51.27\\
1220	-78.125\\
1221	-123.291\\
1222	-86.67\\
1223	-56.152\\
1224	-43.945\\
1225	-48.828\\
1226	-43.945\\
1227	-34.18\\
1228	-37.842\\
1229	-47.607\\
1230	-59.814\\
1231	-64.697\\
1232	-46.387\\
1233	-74.463\\
1234	-85.449\\
1235	-83.008\\
1236	-65.918\\
1237	-73.242\\
1238	-96.436\\
1239	-61.035\\
1240	-29.297\\
1241	-42.725\\
1242	-43.945\\
1243	-58.594\\
1244	-50.049\\
1245	-40.283\\
1246	-58.594\\
1247	-54.932\\
1248	-39.063\\
1249	-48.828\\
1250	-43.945\\
1251	-39.063\\
1252	-50.049\\
1253	-29.297\\
1254	-36.621\\
1255	-26.855\\
1256	-31.738\\
1257	-53.711\\
1258	-61.035\\
1259	-76.904\\
1260	-101.318\\
1261	-67.139\\
1262	-42.725\\
1263	-32.959\\
1264	-31.738\\
1265	-47.607\\
1266	-30.518\\
1267	-36.621\\
1268	-43.945\\
1269	-65.918\\
1270	-62.256\\
1271	-86.67\\
1272	-59.814\\
1273	-56.152\\
1274	-31.738\\
1275	-32.959\\
1276	-20.752\\
1277	-24.414\\
1278	-26.855\\
1279	-47.607\\
1280	-47.607\\
1281	-56.152\\
1282	-59.814\\
1283	-93.994\\
1284	-79.346\\
1285	-62.256\\
1286	-73.242\\
1287	-79.346\\
1288	-101.318\\
1289	-80.566\\
1290	-52.49\\
1291	-30.518\\
1292	-21.973\\
1293	-23.193\\
1294	-29.297\\
1295	-29.297\\
1296	-26.855\\
1297	-36.621\\
1298	-53.711\\
1299	-47.607\\
1300	-51.27\\
1301	-57.373\\
1302	-37.842\\
1303	-20.752\\
1304	-42.725\\
1305	-63.477\\
1306	-61.035\\
1307	-69.58\\
1308	-58.594\\
1309	-45.166\\
1310	-59.814\\
1311	-84.229\\
1312	-85.449\\
1313	-59.814\\
1314	-102.539\\
1315	-76.904\\
1316	-74.463\\
1317	-84.229\\
1318	-83.008\\
1319	-63.477\\
1320	-54.932\\
1321	-90.332\\
1322	-124.512\\
1323	-95.215\\
1324	-57.373\\
1325	-53.711\\
1326	-54.932\\
1327	-72.021\\
1328	-83.008\\
1329	-59.814\\
1330	-64.697\\
1331	-83.008\\
1332	-90.332\\
1333	-61.035\\
1334	-50.049\\
1335	-63.477\\
1336	-104.98\\
1337	-92.773\\
1338	-95.215\\
1339	-58.594\\
1340	-53.711\\
1341	-51.27\\
1342	-34.18\\
1343	-23.193\\
1344	-20.752\\
1345	-17.09\\
1346	-40.283\\
1347	-54.932\\
1348	-62.256\\
1349	-47.607\\
1350	-52.49\\
1351	-57.373\\
1352	-37.842\\
1353	-39.063\\
1354	-39.063\\
1355	-28.076\\
1356	-39.063\\
1357	-57.373\\
1358	-73.242\\
1359	-48.828\\
1360	-35.4\\
1361	-31.738\\
1362	-36.621\\
1363	-26.855\\
1364	-54.932\\
1365	-93.994\\
1366	-72.021\\
1367	-97.656\\
1368	-113.525\\
1369	-114.746\\
1370	-85.449\\
1371	-63.477\\
1372	-63.477\\
1373	-63.477\\
1374	-75.684\\
1375	-89.111\\
1376	-86.67\\
1377	-115.967\\
1378	-125.732\\
1379	-150.146\\
1380	-96.436\\
1381	-109.863\\
1382	-122.07\\
1383	-92.773\\
1384	-107.422\\
1385	-92.773\\
1386	-54.932\\
1387	-39.063\\
1388	-40.283\\
1389	-57.373\\
1390	-42.725\\
1391	-32.959\\
1392	-45.166\\
1393	-34.18\\
1394	-26.855\\
1395	-34.18\\
1396	-37.842\\
1397	-25.635\\
1398	-48.828\\
1399	-64.697\\
1400	-46.387\\
1401	-81.787\\
1402	-115.967\\
1403	-106.201\\
1404	-133.057\\
1405	-92.773\\
1406	-97.656\\
1407	-69.58\\
1408	-43.945\\
1409	-53.711\\
1410	-42.725\\
1411	-34.18\\
1412	-45.166\\
1413	-26.855\\
1414	-20.752\\
1415	-25.635\\
1416	-50.049\\
1417	-61.035\\
1418	-78.125\\
1419	-74.463\\
1420	-69.58\\
1421	-80.566\\
1422	-64.697\\
1423	-53.711\\
1424	-54.932\\
1425	-43.945\\
1426	-69.58\\
1427	-50.049\\
1428	-40.283\\
1429	-58.594\\
1430	-80.566\\
1431	-59.814\\
1432	-46.387\\
1433	-41.504\\
1434	-47.607\\
1435	-32.959\\
1436	-31.738\\
1437	-43.945\\
1438	-53.711\\
1439	-59.814\\
1440	-48.828\\
1441	-61.035\\
1442	-57.373\\
1443	-34.18\\
1444	-37.842\\
1445	-65.918\\
1446	-91.553\\
1447	-90.332\\
1448	-75.684\\
1449	-73.242\\
1450	-57.373\\
1451	-53.711\\
1452	-43.945\\
1453	-61.035\\
1454	-54.932\\
1455	-36.621\\
1456	-52.49\\
1457	-61.035\\
1458	-70.801\\
1459	-76.904\\
1460	-76.904\\
1461	-102.539\\
1462	-124.512\\
1463	-140.381\\
1464	-91.553\\
1465	-52.49\\
1466	-34.18\\
1467	-25.635\\
1468	-36.621\\
1469	-32.959\\
1470	-58.594\\
1471	-53.711\\
1472	-46.387\\
1473	-62.256\\
1474	-72.021\\
1475	-81.787\\
1476	-86.67\\
1477	-122.07\\
1478	-101.318\\
1479	-81.787\\
1480	-58.594\\
1481	-58.594\\
1482	-59.814\\
1483	-75.684\\
1484	-54.932\\
1485	-56.152\\
1486	-53.711\\
1487	-30.518\\
1488	-54.932\\
1489	-70.801\\
1490	-50.049\\
1491	-53.711\\
1492	-100.098\\
1493	-75.684\\
1494	-72.021\\
1495	-102.539\\
1496	-98.877\\
1497	-62.256\\
1498	-40.283\\
1499	-42.725\\
1500	-70.801\\
};
\end{axis}

\begin{axis}[%
width=5.090835cm,
height=2.240107cm,
at={(6.698467cm,15.759893cm)},
scale only axis,
xmin=1000,
xmax=1500,
xlabel={Sample index},
ymin=-50,
ymax=0,
ylabel={C2 load, kN},
legend style={legend cell align=left,align=left,draw=white!15!black}
]
\addplot [color=mycolor1,solid,forget plot]
  table[row sep=crcr]{%
1000	-19.531\\
1001	-24.414\\
1002	-19.531\\
1003	-20.752\\
1004	-25.635\\
1005	-24.414\\
1006	-28.076\\
1007	-23.193\\
1008	-13.428\\
1009	-17.09\\
1010	-17.09\\
1011	-18.311\\
1012	-15.869\\
1013	-8.545\\
1014	-6.104\\
1015	-7.324\\
1016	-9.766\\
1017	-21.973\\
1018	-23.193\\
1019	-23.193\\
1020	-19.531\\
1021	-10.986\\
1022	-17.09\\
1023	-19.531\\
1024	-13.428\\
1025	-18.311\\
1026	-19.531\\
1027	-14.648\\
1028	-24.414\\
1029	-21.973\\
1030	-15.869\\
1031	-23.193\\
1032	-25.635\\
1033	-19.531\\
1034	-17.09\\
1035	-14.648\\
1036	-17.09\\
1037	-20.752\\
1038	-18.311\\
1039	-18.311\\
1040	-19.531\\
1041	-20.752\\
1042	-28.076\\
1043	-25.635\\
1044	-17.09\\
1045	-15.869\\
1046	-12.207\\
1047	-10.986\\
1048	-15.869\\
1049	-12.207\\
1050	-12.207\\
1051	-15.869\\
1052	-18.311\\
1053	-15.869\\
1054	-25.635\\
1055	-21.973\\
1056	-12.207\\
1057	-9.766\\
1058	-13.428\\
1059	-15.869\\
1060	-10.986\\
1061	-10.986\\
1062	-12.207\\
1063	-9.766\\
1064	-8.545\\
1065	-12.207\\
1066	-14.648\\
1067	-17.09\\
1068	-18.311\\
1069	-23.193\\
1070	-21.973\\
1071	-21.973\\
1072	-17.09\\
1073	-17.09\\
1074	-17.09\\
1075	-15.869\\
1076	-17.09\\
1077	-24.414\\
1078	-35.4\\
1079	-36.621\\
1080	-34.18\\
1081	-24.414\\
1082	-29.297\\
1083	-36.621\\
1084	-40.283\\
1085	-31.738\\
1086	-37.842\\
1087	-48.828\\
1088	-39.063\\
1089	-30.518\\
1090	-25.635\\
1091	-19.531\\
1092	-14.648\\
1093	-19.531\\
1094	-15.869\\
1095	-9.766\\
1096	-9.766\\
1097	-12.207\\
1098	-17.09\\
1099	-15.869\\
1100	-15.869\\
1101	-19.531\\
1102	-19.531\\
1103	-28.076\\
1104	-23.193\\
1105	-25.635\\
1106	-23.193\\
1107	-18.311\\
1108	-20.752\\
1109	-17.09\\
1110	-14.648\\
1111	-18.311\\
1112	-17.09\\
1113	-15.869\\
1114	-13.428\\
1115	-10.986\\
1116	-12.207\\
1117	-13.428\\
1118	-10.986\\
1119	-9.766\\
1120	-14.648\\
1121	-19.531\\
1122	-23.193\\
1123	-23.193\\
1124	-17.09\\
1125	-18.311\\
1126	-30.518\\
1127	-23.193\\
1128	-24.414\\
1129	-26.855\\
1130	-17.09\\
1131	-10.986\\
1132	-17.09\\
1133	-18.311\\
1134	-26.855\\
1135	-34.18\\
1136	-31.738\\
1137	-25.635\\
1138	-24.414\\
1139	-23.193\\
1140	-23.193\\
1141	-21.973\\
1142	-18.311\\
1143	-14.648\\
1144	-13.428\\
1145	-12.207\\
1146	-13.428\\
1147	-19.531\\
1148	-25.635\\
1149	-21.973\\
1150	-14.648\\
1151	-13.428\\
1152	-14.648\\
1153	-12.207\\
1154	-8.545\\
1155	-8.545\\
1156	-13.428\\
1157	-12.207\\
1158	-12.207\\
1159	-12.207\\
1160	-12.207\\
1161	-8.545\\
1162	-7.324\\
1163	-4.883\\
1164	-7.324\\
1165	-17.09\\
1166	-24.414\\
1167	-24.414\\
1168	-19.531\\
1169	-12.207\\
1170	-12.207\\
1171	-8.545\\
1172	-10.986\\
1173	-14.648\\
1174	-14.648\\
1175	-24.414\\
1176	-34.18\\
1177	-40.283\\
1178	-32.959\\
1179	-31.738\\
1180	-23.193\\
1181	-25.635\\
1182	-26.855\\
1183	-28.076\\
1184	-21.973\\
1185	-20.752\\
1186	-19.531\\
1187	-19.531\\
1188	-21.973\\
1189	-19.531\\
1190	-23.193\\
1191	-29.297\\
1192	-24.414\\
1193	-14.648\\
1194	-15.869\\
1195	-25.635\\
1196	-32.959\\
1197	-35.4\\
1198	-39.063\\
1199	-37.842\\
1200	-30.518\\
1201	-26.855\\
1202	-30.518\\
1203	-35.4\\
1204	-25.635\\
1205	-14.648\\
1206	-19.531\\
1207	-20.752\\
1208	-12.207\\
1209	-13.428\\
1210	-18.311\\
1211	-14.648\\
1212	-12.207\\
1213	-18.311\\
1214	-10.986\\
1215	-14.648\\
1216	-24.414\\
1217	-21.973\\
1218	-14.648\\
1219	-14.648\\
1220	-20.752\\
1221	-34.18\\
1222	-28.076\\
1223	-15.869\\
1224	-14.648\\
1225	-14.648\\
1226	-12.207\\
1227	-10.986\\
1228	-10.986\\
1229	-13.428\\
1230	-17.09\\
1231	-19.531\\
1232	-14.648\\
1233	-19.531\\
1234	-26.855\\
1235	-23.193\\
1236	-17.09\\
1237	-21.973\\
1238	-25.635\\
1239	-21.973\\
1240	-8.545\\
1241	-12.207\\
1242	-13.428\\
1243	-15.869\\
1244	-15.869\\
1245	-10.986\\
1246	-17.09\\
1247	-15.869\\
1248	-10.986\\
1249	-14.648\\
1250	-14.648\\
1251	-12.207\\
1252	-14.648\\
1253	-10.986\\
1254	-9.766\\
1255	-12.207\\
1256	-8.545\\
1257	-17.09\\
1258	-18.311\\
1259	-20.752\\
1260	-29.297\\
1261	-20.752\\
1262	-10.986\\
1263	-10.986\\
1264	-8.545\\
1265	-13.428\\
1266	-12.207\\
1267	-9.766\\
1268	-15.869\\
1269	-19.531\\
1270	-20.752\\
1271	-23.193\\
1272	-23.193\\
1273	-17.09\\
1274	-15.869\\
1275	-10.986\\
1276	-10.986\\
1277	-7.324\\
1278	-8.545\\
1279	-10.986\\
1280	-17.09\\
1281	-18.311\\
1282	-17.09\\
1283	-24.414\\
1284	-25.635\\
1285	-15.869\\
1286	-19.531\\
1287	-23.193\\
1288	-26.855\\
1289	-25.635\\
1290	-17.09\\
1291	-10.986\\
1292	-7.324\\
1293	-6.104\\
1294	-8.545\\
1295	-9.766\\
1296	-8.545\\
1297	-10.986\\
1298	-17.09\\
1299	-15.869\\
1300	-13.428\\
1301	-15.869\\
1302	-12.207\\
1303	-6.104\\
1304	-9.766\\
1305	-20.752\\
1306	-17.09\\
1307	-18.311\\
1308	-15.869\\
1309	-10.986\\
1310	-17.09\\
1311	-23.193\\
1312	-23.193\\
1313	-17.09\\
1314	-26.855\\
1315	-25.635\\
1316	-18.311\\
1317	-23.193\\
1318	-25.635\\
1319	-19.531\\
1320	-15.869\\
1321	-24.414\\
1322	-35.4\\
1323	-29.297\\
1324	-17.09\\
1325	-17.09\\
1326	-18.311\\
1327	-20.752\\
1328	-23.193\\
1329	-19.531\\
1330	-17.09\\
1331	-24.414\\
1332	-25.635\\
1333	-18.311\\
1334	-15.869\\
1335	-18.311\\
1336	-28.076\\
1337	-26.855\\
1338	-24.414\\
1339	-19.531\\
1340	-17.09\\
1341	-18.311\\
1342	-13.428\\
1343	-6.104\\
1344	-6.104\\
1345	-6.104\\
1346	-9.766\\
1347	-19.531\\
1348	-15.869\\
1349	-14.648\\
1350	-13.428\\
1351	-17.09\\
1352	-13.428\\
1353	-9.766\\
1354	-12.207\\
1355	-9.766\\
1356	-8.545\\
1357	-17.09\\
1358	-20.752\\
1359	-17.09\\
1360	-10.986\\
1361	-10.986\\
1362	-13.428\\
1363	-9.766\\
1364	-10.986\\
1365	-28.076\\
1366	-20.752\\
1367	-25.635\\
1368	-35.4\\
1369	-30.518\\
1370	-24.414\\
1371	-18.311\\
1372	-18.311\\
1373	-19.531\\
1374	-21.973\\
1375	-25.635\\
1376	-25.635\\
1377	-31.738\\
1378	-34.18\\
1379	-41.504\\
1380	-32.959\\
1381	-28.076\\
1382	-35.4\\
1383	-26.855\\
1384	-29.297\\
1385	-28.076\\
1386	-17.09\\
1387	-10.986\\
1388	-12.207\\
1389	-17.09\\
1390	-14.648\\
1391	-9.766\\
1392	-13.428\\
1393	-13.428\\
1394	-7.324\\
1395	-8.545\\
1396	-12.207\\
1397	-7.324\\
1398	-14.648\\
1399	-20.752\\
1400	-13.428\\
1401	-21.973\\
1402	-30.518\\
1403	-29.297\\
1404	-35.4\\
1405	-29.297\\
1406	-28.076\\
1407	-23.193\\
1408	-12.207\\
1409	-14.648\\
1410	-15.869\\
1411	-10.986\\
1412	-12.207\\
1413	-13.428\\
1414	-3.662\\
1415	-8.545\\
1416	-17.09\\
1417	-19.531\\
1418	-20.752\\
1419	-23.193\\
1420	-20.752\\
1421	-23.193\\
1422	-18.311\\
1423	-15.869\\
1424	-15.869\\
1425	-12.207\\
1426	-18.311\\
1427	-17.09\\
1428	-12.207\\
1429	-17.09\\
1430	-23.193\\
1431	-17.09\\
1432	-13.428\\
1433	-12.207\\
1434	-14.648\\
1435	-7.324\\
1436	-8.545\\
1437	-12.207\\
1438	-17.09\\
1439	-17.09\\
1440	-14.648\\
1441	-15.869\\
1442	-18.311\\
1443	-12.207\\
1444	-9.766\\
1445	-18.311\\
1446	-26.855\\
1447	-26.855\\
1448	-21.973\\
1449	-19.531\\
1450	-17.09\\
1451	-14.648\\
1452	-13.428\\
1453	-17.09\\
1454	-17.09\\
1455	-10.986\\
1456	-14.648\\
1457	-18.311\\
1458	-19.531\\
1459	-21.973\\
1460	-21.973\\
1461	-29.297\\
1462	-35.4\\
1463	-36.621\\
1464	-26.855\\
1465	-14.648\\
1466	-10.986\\
1467	-7.324\\
1468	-10.986\\
1469	-10.986\\
1470	-14.648\\
1471	-13.428\\
1472	-12.207\\
1473	-15.869\\
1474	-20.752\\
1475	-20.752\\
1476	-24.414\\
1477	-31.738\\
1478	-29.297\\
1479	-26.855\\
1480	-18.311\\
1481	-17.09\\
1482	-20.752\\
1483	-20.752\\
1484	-17.09\\
1485	-14.648\\
1486	-18.311\\
1487	-12.207\\
1488	-13.428\\
1489	-20.752\\
1490	-17.09\\
1491	-18.311\\
1492	-32.959\\
1493	-25.635\\
1494	-20.752\\
1495	-28.076\\
1496	-28.076\\
1497	-19.531\\
1498	-13.428\\
1499	-13.428\\
1500	-19.531\\
};
\end{axis}

\begin{axis}[%
width=5.090835cm,
height=2.240107cm,
at={(0cm,15.759893cm)},
scale only axis,
xmin=1000,
xmax=1500,
xlabel={Sample index},
ymin=-58.594,
ymax=0,
ylabel={C1 load, kN},
legend style={legend cell align=left,align=left,draw=white!15!black}
]
\addplot [color=mycolor1,solid,forget plot]
  table[row sep=crcr]{%
1000	-23.193\\
1001	-28.076\\
1002	-26.855\\
1003	-20.752\\
1004	-29.297\\
1005	-28.076\\
1006	-32.959\\
1007	-25.635\\
1008	-14.648\\
1009	-20.752\\
1010	-17.09\\
1011	-18.311\\
1012	-20.752\\
1013	-7.324\\
1014	-4.883\\
1015	-4.883\\
1016	-13.428\\
1017	-23.193\\
1018	-24.414\\
1019	-25.635\\
1020	-19.531\\
1021	-12.207\\
1022	-24.414\\
1023	-19.531\\
1024	-14.648\\
1025	-20.752\\
1026	-18.311\\
1027	-17.09\\
1028	-29.297\\
1029	-25.635\\
1030	-18.311\\
1031	-28.076\\
1032	-28.076\\
1033	-21.973\\
1034	-18.311\\
1035	-15.869\\
1036	-15.869\\
1037	-21.973\\
1038	-21.973\\
1039	-21.973\\
1040	-21.973\\
1041	-23.193\\
1042	-30.518\\
1043	-29.297\\
1044	-19.531\\
1045	-18.311\\
1046	-10.986\\
1047	-14.648\\
1048	-15.869\\
1049	-12.207\\
1050	-13.428\\
1051	-18.311\\
1052	-19.531\\
1053	-18.311\\
1054	-29.297\\
1055	-21.973\\
1056	-12.207\\
1057	-12.207\\
1058	-13.428\\
1059	-17.09\\
1060	-9.766\\
1061	-10.986\\
1062	-14.648\\
1063	-9.766\\
1064	-7.324\\
1065	-13.428\\
1066	-13.428\\
1067	-21.973\\
1068	-20.752\\
1069	-25.635\\
1070	-21.973\\
1071	-25.635\\
1072	-20.752\\
1073	-20.752\\
1074	-18.311\\
1075	-18.311\\
1076	-17.09\\
1077	-30.518\\
1078	-41.504\\
1079	-41.504\\
1080	-42.725\\
1081	-28.076\\
1082	-37.842\\
1083	-46.387\\
1084	-46.387\\
1085	-35.4\\
1086	-47.607\\
1087	-58.594\\
1088	-43.945\\
1089	-34.18\\
1090	-24.414\\
1091	-20.752\\
1092	-17.09\\
1093	-21.973\\
1094	-17.09\\
1095	-12.207\\
1096	-9.766\\
1097	-12.207\\
1098	-18.311\\
1099	-18.311\\
1100	-18.311\\
1101	-21.973\\
1102	-23.193\\
1103	-30.518\\
1104	-28.076\\
1105	-30.518\\
1106	-25.635\\
1107	-20.752\\
1108	-24.414\\
1109	-18.311\\
1110	-13.428\\
1111	-19.531\\
1112	-20.752\\
1113	-12.207\\
1114	-15.869\\
1115	-12.207\\
1116	-13.428\\
1117	-13.428\\
1118	-9.766\\
1119	-10.986\\
1120	-17.09\\
1121	-23.193\\
1122	-25.635\\
1123	-26.855\\
1124	-15.869\\
1125	-20.752\\
1126	-32.959\\
1127	-24.414\\
1128	-28.076\\
1129	-29.297\\
1130	-18.311\\
1131	-12.207\\
1132	-17.09\\
1133	-18.311\\
1134	-30.518\\
1135	-36.621\\
1136	-36.621\\
1137	-28.076\\
1138	-28.076\\
1139	-25.635\\
1140	-25.635\\
1141	-25.635\\
1142	-20.752\\
1143	-15.869\\
1144	-14.648\\
1145	-13.428\\
1146	-13.428\\
1147	-20.752\\
1148	-31.738\\
1149	-26.855\\
1150	-17.09\\
1151	-15.869\\
1152	-15.869\\
1153	-9.766\\
1154	-7.324\\
1155	-7.324\\
1156	-15.869\\
1157	-10.986\\
1158	-14.648\\
1159	-14.648\\
1160	-13.428\\
1161	-9.766\\
1162	-6.104\\
1163	-4.883\\
1164	-7.324\\
1165	-18.311\\
1166	-26.855\\
1167	-28.076\\
1168	-20.752\\
1169	-13.428\\
1170	-10.986\\
1171	-6.104\\
1172	-12.207\\
1173	-14.648\\
1174	-18.311\\
1175	-26.855\\
1176	-39.063\\
1177	-47.607\\
1178	-42.725\\
1179	-37.842\\
1180	-28.076\\
1181	-30.518\\
1182	-31.738\\
1183	-34.18\\
1184	-24.414\\
1185	-23.193\\
1186	-23.193\\
1187	-21.973\\
1188	-23.193\\
1189	-18.311\\
1190	-30.518\\
1191	-37.842\\
1192	-28.076\\
1193	-18.311\\
1194	-18.311\\
1195	-30.518\\
1196	-35.4\\
1197	-40.283\\
1198	-45.166\\
1199	-45.166\\
1200	-34.18\\
1201	-32.959\\
1202	-36.621\\
1203	-41.504\\
1204	-29.297\\
1205	-17.09\\
1206	-23.193\\
1207	-20.752\\
1208	-13.428\\
1209	-18.311\\
1210	-19.531\\
1211	-14.648\\
1212	-13.428\\
1213	-15.869\\
1214	-13.428\\
1215	-17.09\\
1216	-25.635\\
1217	-25.635\\
1218	-15.869\\
1219	-14.648\\
1220	-24.414\\
1221	-40.283\\
1222	-29.297\\
1223	-20.752\\
1224	-14.648\\
1225	-18.311\\
1226	-15.869\\
1227	-12.207\\
1228	-13.428\\
1229	-15.869\\
1230	-18.311\\
1231	-20.752\\
1232	-15.869\\
1233	-23.193\\
1234	-29.297\\
1235	-25.635\\
1236	-20.752\\
1237	-23.193\\
1238	-30.518\\
1239	-21.973\\
1240	-8.545\\
1241	-13.428\\
1242	-13.428\\
1243	-17.09\\
1244	-17.09\\
1245	-13.428\\
1246	-17.09\\
1247	-19.531\\
1248	-13.428\\
1249	-15.869\\
1250	-15.869\\
1251	-12.207\\
1252	-15.869\\
1253	-9.766\\
1254	-10.986\\
1255	-8.545\\
1256	-8.545\\
1257	-14.648\\
1258	-20.752\\
1259	-24.414\\
1260	-35.4\\
1261	-24.414\\
1262	-13.428\\
1263	-12.207\\
1264	-10.986\\
1265	-13.428\\
1266	-10.986\\
1267	-12.207\\
1268	-14.648\\
1269	-24.414\\
1270	-21.973\\
1271	-26.855\\
1272	-23.193\\
1273	-17.09\\
1274	-13.428\\
1275	-10.986\\
1276	-8.545\\
1277	-6.104\\
1278	-9.766\\
1279	-14.648\\
1280	-18.311\\
1281	-18.311\\
1282	-19.531\\
1283	-29.297\\
1284	-25.635\\
1285	-18.311\\
1286	-24.414\\
1287	-24.414\\
1288	-31.738\\
1289	-26.855\\
1290	-17.09\\
1291	-9.766\\
1292	-6.104\\
1293	-7.324\\
1294	-9.766\\
1295	-8.545\\
1296	-8.545\\
1297	-12.207\\
1298	-17.09\\
1299	-15.869\\
1300	-15.869\\
1301	-19.531\\
1302	-13.428\\
1303	-4.883\\
1304	-9.766\\
1305	-21.973\\
1306	-19.531\\
1307	-21.973\\
1308	-18.311\\
1309	-14.648\\
1310	-19.531\\
1311	-25.635\\
1312	-25.635\\
1313	-19.531\\
1314	-26.855\\
1315	-26.855\\
1316	-23.193\\
1317	-28.076\\
1318	-26.855\\
1319	-20.752\\
1320	-19.531\\
1321	-29.297\\
1322	-40.283\\
1323	-30.518\\
1324	-18.311\\
1325	-17.09\\
1326	-18.311\\
1327	-21.973\\
1328	-26.855\\
1329	-19.531\\
1330	-19.531\\
1331	-26.855\\
1332	-28.076\\
1333	-20.752\\
1334	-14.648\\
1335	-20.752\\
1336	-34.18\\
1337	-30.518\\
1338	-31.738\\
1339	-21.973\\
1340	-17.09\\
1341	-17.09\\
1342	-10.986\\
1343	-6.104\\
1344	-6.104\\
1345	-4.883\\
1346	-10.986\\
1347	-19.531\\
1348	-18.311\\
1349	-15.869\\
1350	-14.648\\
1351	-20.752\\
1352	-14.648\\
1353	-9.766\\
1354	-13.428\\
1355	-8.545\\
1356	-10.986\\
1357	-18.311\\
1358	-23.193\\
1359	-17.09\\
1360	-9.766\\
1361	-10.986\\
1362	-12.207\\
1363	-9.766\\
1364	-13.428\\
1365	-31.738\\
1366	-25.635\\
1367	-31.738\\
1368	-37.842\\
1369	-36.621\\
1370	-26.855\\
1371	-20.752\\
1372	-18.311\\
1373	-21.973\\
1374	-23.193\\
1375	-29.297\\
1376	-29.297\\
1377	-36.621\\
1378	-42.725\\
1379	-46.387\\
1380	-37.842\\
1381	-35.4\\
1382	-40.283\\
1383	-31.738\\
1384	-34.18\\
1385	-31.738\\
1386	-18.311\\
1387	-12.207\\
1388	-13.428\\
1389	-18.311\\
1390	-17.09\\
1391	-8.545\\
1392	-14.648\\
1393	-13.428\\
1394	-6.104\\
1395	-10.986\\
1396	-12.207\\
1397	-7.324\\
1398	-18.311\\
1399	-20.752\\
1400	-14.648\\
1401	-23.193\\
1402	-37.842\\
1403	-32.959\\
1404	-37.842\\
1405	-32.959\\
1406	-29.297\\
1407	-20.752\\
1408	-14.648\\
1409	-17.09\\
1410	-13.428\\
1411	-10.986\\
1412	-14.648\\
1413	-13.428\\
1414	-3.662\\
1415	-7.324\\
1416	-15.869\\
1417	-20.752\\
1418	-21.973\\
1419	-23.193\\
1420	-21.973\\
1421	-26.855\\
1422	-21.973\\
1423	-17.09\\
1424	-17.09\\
1425	-14.648\\
1426	-20.752\\
1427	-18.311\\
1428	-13.428\\
1429	-18.311\\
1430	-26.855\\
1431	-20.752\\
1432	-15.869\\
1433	-13.428\\
1434	-14.648\\
1435	-12.207\\
1436	-8.545\\
1437	-14.648\\
1438	-19.531\\
1439	-18.311\\
1440	-15.869\\
1441	-19.531\\
1442	-18.311\\
1443	-10.986\\
1444	-10.986\\
1445	-21.973\\
1446	-30.518\\
1447	-29.297\\
1448	-23.193\\
1449	-21.973\\
1450	-19.531\\
1451	-18.311\\
1452	-15.869\\
1453	-19.531\\
1454	-17.09\\
1455	-13.428\\
1456	-17.09\\
1457	-19.531\\
1458	-21.973\\
1459	-25.635\\
1460	-25.635\\
1461	-32.959\\
1462	-41.504\\
1463	-45.166\\
1464	-31.738\\
1465	-17.09\\
1466	-13.428\\
1467	-8.545\\
1468	-10.986\\
1469	-13.428\\
1470	-17.09\\
1471	-18.311\\
1472	-13.428\\
1473	-19.531\\
1474	-24.414\\
1475	-25.635\\
1476	-26.855\\
1477	-39.063\\
1478	-35.4\\
1479	-25.635\\
1480	-20.752\\
1481	-20.752\\
1482	-19.531\\
1483	-23.193\\
1484	-19.531\\
1485	-15.869\\
1486	-17.09\\
1487	-10.986\\
1488	-14.648\\
1489	-28.076\\
1490	-19.531\\
1491	-15.869\\
1492	-34.18\\
1493	-28.076\\
1494	-21.973\\
1495	-34.18\\
1496	-31.738\\
1497	-20.752\\
1498	-15.869\\
1499	-13.428\\
1500	-23.193\\
};
\end{axis}

\begin{axis}[%
width=5.090835cm,
height=2.240107cm,
at={(0cm,3.939973cm)},
scale only axis,
xmin=1000,
xmax=1500,
xlabel={Sample index},
ymin=-400,
ymax=0,
ylabel={C7 load, kN},
legend style={legend cell align=left,align=left,draw=white!15!black}
]
\addplot [color=mycolor1,solid,forget plot]
  table[row sep=crcr]{%
1000	-112.305\\
1001	-144.043\\
1002	-117.188\\
1003	-111.084\\
1004	-156.25\\
1005	-147.705\\
1006	-175.781\\
1007	-134.277\\
1008	-68.359\\
1009	-80.566\\
1010	-80.566\\
1011	-100.098\\
1012	-81.787\\
1013	-34.18\\
1014	-24.414\\
1015	-23.193\\
1016	-73.242\\
1017	-128.174\\
1018	-133.057\\
1019	-137.939\\
1020	-92.773\\
1021	-56.152\\
1022	-125.732\\
1023	-86.67\\
1024	-68.359\\
1025	-114.746\\
1026	-100.098\\
1027	-85.449\\
1028	-142.822\\
1029	-133.057\\
1030	-102.539\\
1031	-150.146\\
1032	-145.264\\
1033	-113.525\\
1034	-92.773\\
1035	-72.021\\
1036	-86.67\\
1037	-112.305\\
1038	-103.76\\
1039	-109.863\\
1040	-112.305\\
1041	-122.07\\
1042	-175.781\\
1043	-155.029\\
1044	-102.539\\
1045	-92.773\\
1046	-53.711\\
1047	-58.594\\
1048	-79.346\\
1049	-59.814\\
1050	-62.256\\
1051	-89.111\\
1052	-102.539\\
1053	-95.215\\
1054	-151.367\\
1055	-111.084\\
1056	-64.697\\
1057	-51.27\\
1058	-69.58\\
1059	-81.787\\
1060	-43.945\\
1061	-45.166\\
1062	-67.139\\
1063	-51.27\\
1064	-42.725\\
1065	-61.035\\
1066	-70.801\\
1067	-109.863\\
1068	-104.98\\
1069	-129.395\\
1070	-120.85\\
1071	-137.939\\
1072	-109.863\\
1073	-107.422\\
1074	-92.773\\
1075	-91.553\\
1076	-87.891\\
1077	-157.471\\
1078	-224.609\\
1079	-231.934\\
1080	-225.83\\
1081	-140.381\\
1082	-205.078\\
1083	-252.686\\
1084	-261.23\\
1085	-200.195\\
1086	-270.996\\
1087	-335.693\\
1088	-235.596\\
1089	-191.65\\
1090	-128.174\\
1091	-98.877\\
1092	-84.229\\
1093	-104.98\\
1094	-74.463\\
1095	-50.049\\
1096	-48.828\\
1097	-63.477\\
1098	-95.215\\
1099	-86.67\\
1100	-89.111\\
1101	-119.629\\
1102	-123.291\\
1103	-167.236\\
1104	-134.277\\
1105	-169.678\\
1106	-122.07\\
1107	-108.643\\
1108	-125.732\\
1109	-89.111\\
1110	-80.566\\
1111	-97.656\\
1112	-98.877\\
1113	-68.359\\
1114	-73.242\\
1115	-54.932\\
1116	-61.035\\
1117	-64.697\\
1118	-45.166\\
1119	-58.594\\
1120	-72.021\\
1121	-117.188\\
1122	-122.07\\
1123	-136.719\\
1124	-76.904\\
1125	-109.863\\
1126	-173.34\\
1127	-131.836\\
1128	-157.471\\
1129	-157.471\\
1130	-85.449\\
1131	-59.814\\
1132	-85.449\\
1133	-98.877\\
1134	-161.133\\
1135	-202.637\\
1136	-198.975\\
1137	-145.264\\
1138	-150.146\\
1139	-135.498\\
1140	-131.836\\
1141	-126.953\\
1142	-85.449\\
1143	-76.904\\
1144	-69.58\\
1145	-62.256\\
1146	-70.801\\
1147	-107.422\\
1148	-164.795\\
1149	-125.732\\
1150	-86.67\\
1151	-73.242\\
1152	-70.801\\
1153	-42.725\\
1154	-31.738\\
1155	-39.063\\
1156	-72.021\\
1157	-57.373\\
1158	-59.814\\
1159	-67.139\\
1160	-63.477\\
1161	-43.945\\
1162	-29.297\\
1163	-25.635\\
1164	-43.945\\
1165	-96.436\\
1166	-140.381\\
1167	-155.029\\
1168	-101.318\\
1169	-69.58\\
1170	-50.049\\
1171	-34.18\\
1172	-83.008\\
1173	-81.787\\
1174	-119.629\\
1175	-145.264\\
1176	-230.713\\
1177	-262.451\\
1178	-211.182\\
1179	-202.637\\
1180	-136.719\\
1181	-151.367\\
1182	-159.912\\
1183	-172.119\\
1184	-129.395\\
1185	-118.408\\
1186	-115.967\\
1187	-103.76\\
1188	-123.291\\
1189	-87.891\\
1190	-153.809\\
1191	-189.209\\
1192	-133.057\\
1193	-78.125\\
1194	-85.449\\
1195	-146.484\\
1196	-181.885\\
1197	-229.492\\
1198	-241.699\\
1199	-240.479\\
1200	-180.664\\
1201	-163.574\\
1202	-187.988\\
1203	-222.168\\
1204	-139.16\\
1205	-81.787\\
1206	-122.07\\
1207	-102.539\\
1208	-65.918\\
1209	-87.891\\
1210	-98.877\\
1211	-76.904\\
1212	-58.594\\
1213	-80.566\\
1214	-65.918\\
1215	-91.553\\
1216	-133.057\\
1217	-122.07\\
1218	-79.346\\
1219	-83.008\\
1220	-126.953\\
1221	-209.961\\
1222	-150.146\\
1223	-92.773\\
1224	-69.58\\
1225	-83.008\\
1226	-74.463\\
1227	-56.152\\
1228	-62.256\\
1229	-74.463\\
1230	-96.436\\
1231	-107.422\\
1232	-72.021\\
1233	-122.07\\
1234	-144.043\\
1235	-137.939\\
1236	-106.201\\
1237	-118.408\\
1238	-158.691\\
1239	-101.318\\
1240	-41.504\\
1241	-58.594\\
1242	-65.918\\
1243	-91.553\\
1244	-81.787\\
1245	-59.814\\
1246	-96.436\\
1247	-91.553\\
1248	-65.918\\
1249	-83.008\\
1250	-76.904\\
1251	-67.139\\
1252	-79.346\\
1253	-46.387\\
1254	-53.711\\
1255	-40.283\\
1256	-43.945\\
1257	-86.67\\
1258	-100.098\\
1259	-137.939\\
1260	-177.002\\
1261	-115.967\\
1262	-68.359\\
1263	-48.828\\
1264	-48.828\\
1265	-73.242\\
1266	-46.387\\
1267	-52.49\\
1268	-70.801\\
1269	-119.629\\
1270	-107.422\\
1271	-150.146\\
1272	-102.539\\
1273	-91.553\\
1274	-47.607\\
1275	-47.607\\
1276	-31.738\\
1277	-34.18\\
1278	-41.504\\
1279	-75.684\\
1280	-83.008\\
1281	-92.773\\
1282	-97.656\\
1283	-152.588\\
1284	-130.615\\
1285	-97.656\\
1286	-119.629\\
1287	-129.395\\
1288	-167.236\\
1289	-133.057\\
1290	-84.229\\
1291	-43.945\\
1292	-31.738\\
1293	-34.18\\
1294	-41.504\\
1295	-43.945\\
1296	-40.283\\
1297	-56.152\\
1298	-87.891\\
1299	-79.346\\
1300	-81.787\\
1301	-96.436\\
1302	-58.594\\
1303	-29.297\\
1304	-64.697\\
1305	-97.656\\
1306	-96.436\\
1307	-109.863\\
1308	-93.994\\
1309	-69.58\\
1310	-97.656\\
1311	-137.939\\
1312	-139.16\\
1313	-97.656\\
1314	-162.354\\
1315	-128.174\\
1316	-118.408\\
1317	-137.939\\
1318	-137.939\\
1319	-103.76\\
1320	-90.332\\
1321	-153.809\\
1322	-214.844\\
1323	-164.795\\
1324	-92.773\\
1325	-92.773\\
1326	-91.553\\
1327	-113.525\\
1328	-136.719\\
1329	-100.098\\
1330	-104.98\\
1331	-140.381\\
1332	-153.809\\
1333	-103.76\\
1334	-79.346\\
1335	-104.98\\
1336	-175.781\\
1337	-159.912\\
1338	-162.354\\
1339	-95.215\\
1340	-84.229\\
1341	-83.008\\
1342	-52.49\\
1343	-34.18\\
1344	-28.076\\
1345	-23.193\\
1346	-59.814\\
1347	-86.67\\
1348	-104.98\\
1349	-76.904\\
1350	-87.891\\
1351	-97.656\\
1352	-59.814\\
1353	-63.477\\
1354	-61.035\\
1355	-45.166\\
1356	-57.373\\
1357	-97.656\\
1358	-124.512\\
1359	-78.125\\
1360	-56.152\\
1361	-46.387\\
1362	-59.814\\
1363	-41.504\\
1364	-91.553\\
1365	-158.691\\
1366	-122.07\\
1367	-167.236\\
1368	-194.092\\
1369	-197.754\\
1370	-140.381\\
1371	-106.201\\
1372	-106.201\\
1373	-104.98\\
1374	-120.85\\
1375	-150.146\\
1376	-147.705\\
1377	-200.195\\
1378	-220.947\\
1379	-263.672\\
1380	-178.223\\
1381	-190.43\\
1382	-212.402\\
1383	-163.574\\
1384	-189.209\\
1385	-163.574\\
1386	-91.553\\
1387	-58.594\\
1388	-62.256\\
1389	-95.215\\
1390	-75.684\\
1391	-51.27\\
1392	-72.021\\
1393	-52.49\\
1394	-40.283\\
1395	-48.828\\
1396	-56.152\\
1397	-40.283\\
1398	-76.904\\
1399	-109.863\\
1400	-78.125\\
1401	-134.277\\
1402	-195.313\\
1403	-178.223\\
1404	-222.168\\
1405	-150.146\\
1406	-161.133\\
1407	-111.084\\
1408	-67.139\\
1409	-85.449\\
1410	-67.139\\
1411	-48.828\\
1412	-72.021\\
1413	-41.504\\
1414	-25.635\\
1415	-36.621\\
1416	-79.346\\
1417	-104.98\\
1418	-128.174\\
1419	-124.512\\
1420	-117.188\\
1421	-136.719\\
1422	-111.084\\
1423	-90.332\\
1424	-90.332\\
1425	-73.242\\
1426	-115.967\\
1427	-78.125\\
1428	-67.139\\
1429	-100.098\\
1430	-137.939\\
1431	-103.76\\
1432	-76.904\\
1433	-64.697\\
1434	-74.463\\
1435	-51.27\\
1436	-50.049\\
1437	-72.021\\
1438	-86.67\\
1439	-98.877\\
1440	-79.346\\
1441	-101.318\\
1442	-92.773\\
1443	-52.49\\
1444	-54.932\\
1445	-112.305\\
1446	-158.691\\
1447	-153.809\\
1448	-129.395\\
1449	-122.07\\
1450	-92.773\\
1451	-89.111\\
1452	-70.801\\
1453	-102.539\\
1454	-90.332\\
1455	-54.932\\
1456	-84.229\\
1457	-101.318\\
1458	-118.408\\
1459	-129.395\\
1460	-129.395\\
1461	-175.781\\
1462	-216.064\\
1463	-244.141\\
1464	-153.809\\
1465	-85.449\\
1466	-54.932\\
1467	-37.842\\
1468	-57.373\\
1469	-53.711\\
1470	-95.215\\
1471	-85.449\\
1472	-75.684\\
1473	-102.539\\
1474	-118.408\\
1475	-141.602\\
1476	-148.926\\
1477	-208.74\\
1478	-181.885\\
1479	-136.719\\
1480	-90.332\\
1481	-92.773\\
1482	-95.215\\
1483	-123.291\\
1484	-87.891\\
1485	-89.111\\
1486	-87.891\\
1487	-47.607\\
1488	-81.787\\
1489	-128.174\\
1490	-90.332\\
1491	-96.436\\
1492	-170.898\\
1493	-129.395\\
1494	-122.07\\
1495	-177.002\\
1496	-166.016\\
1497	-102.539\\
1498	-64.697\\
1499	-65.918\\
1500	-114.746\\
};
\end{axis}

\begin{axis}[%
width=5.090835cm,
height=2.240107cm,
at={(6.698467cm,7.879947cm)},
scale only axis,
xmin=1000,
xmax=1500,
xlabel={Sample index},
ymin=-404.053,
ymax=0,
ylabel={C6 load, kN},
legend style={legend cell align=left,align=left,draw=white!15!black}
]
\addplot [color=mycolor1,solid,forget plot]
  table[row sep=crcr]{%
1000	-140.381\\
1001	-178.223\\
1002	-147.705\\
1003	-139.16\\
1004	-192.871\\
1005	-185.547\\
1006	-219.727\\
1007	-169.678\\
1008	-86.67\\
1009	-107.422\\
1010	-102.539\\
1011	-126.953\\
1012	-103.76\\
1013	-41.504\\
1014	-31.738\\
1015	-28.076\\
1016	-89.111\\
1017	-157.471\\
1018	-167.236\\
1019	-174.561\\
1020	-124.512\\
1021	-74.463\\
1022	-158.691\\
1023	-111.084\\
1024	-83.008\\
1025	-142.822\\
1026	-123.291\\
1027	-103.76\\
1028	-173.34\\
1029	-164.795\\
1030	-124.512\\
1031	-180.664\\
1032	-179.443\\
1033	-142.822\\
1034	-117.188\\
1035	-89.111\\
1036	-108.643\\
1037	-137.939\\
1038	-130.615\\
1039	-136.719\\
1040	-140.381\\
1041	-151.367\\
1042	-213.623\\
1043	-191.65\\
1044	-126.953\\
1045	-115.967\\
1046	-64.697\\
1047	-69.58\\
1048	-96.436\\
1049	-74.463\\
1050	-78.125\\
1051	-111.084\\
1052	-128.174\\
1053	-119.629\\
1054	-190.43\\
1055	-139.16\\
1056	-81.787\\
1057	-63.477\\
1058	-85.449\\
1059	-101.318\\
1060	-51.27\\
1061	-58.594\\
1062	-81.787\\
1063	-65.918\\
1064	-56.152\\
1065	-74.463\\
1066	-86.67\\
1067	-139.16\\
1068	-130.615\\
1069	-163.574\\
1070	-146.484\\
1071	-170.898\\
1072	-139.16\\
1073	-133.057\\
1074	-115.967\\
1075	-111.084\\
1076	-111.084\\
1077	-196.533\\
1078	-280.762\\
1079	-291.748\\
1080	-280.762\\
1081	-175.781\\
1082	-240.479\\
1083	-303.955\\
1084	-317.383\\
1085	-239.258\\
1086	-313.721\\
1087	-404.053\\
1088	-289.307\\
1089	-238.037\\
1090	-158.691\\
1091	-122.07\\
1092	-102.539\\
1093	-129.395\\
1094	-92.773\\
1095	-61.035\\
1096	-59.814\\
1097	-75.684\\
1098	-118.408\\
1099	-107.422\\
1100	-111.084\\
1101	-146.484\\
1102	-152.588\\
1103	-207.52\\
1104	-167.236\\
1105	-208.74\\
1106	-148.926\\
1107	-130.615\\
1108	-151.367\\
1109	-108.643\\
1110	-98.877\\
1111	-122.07\\
1112	-123.291\\
1113	-85.449\\
1114	-95.215\\
1115	-67.139\\
1116	-76.904\\
1117	-81.787\\
1118	-57.373\\
1119	-75.684\\
1120	-93.994\\
1121	-151.367\\
1122	-152.588\\
1123	-170.898\\
1124	-97.656\\
1125	-140.381\\
1126	-211.182\\
1127	-161.133\\
1128	-186.768\\
1129	-191.65\\
1130	-112.305\\
1131	-75.684\\
1132	-107.422\\
1133	-123.291\\
1134	-205.078\\
1135	-252.686\\
1136	-252.686\\
1137	-184.326\\
1138	-189.209\\
1139	-167.236\\
1140	-163.574\\
1141	-162.354\\
1142	-108.643\\
1143	-96.436\\
1144	-86.67\\
1145	-78.125\\
1146	-87.891\\
1147	-133.057\\
1148	-203.857\\
1149	-161.133\\
1150	-109.863\\
1151	-92.773\\
1152	-89.111\\
1153	-52.49\\
1154	-41.504\\
1155	-47.607\\
1156	-90.332\\
1157	-67.139\\
1158	-78.125\\
1159	-85.449\\
1160	-80.566\\
1161	-54.932\\
1162	-37.842\\
1163	-30.518\\
1164	-54.932\\
1165	-119.629\\
1166	-172.119\\
1167	-194.092\\
1168	-136.719\\
1169	-91.553\\
1170	-64.697\\
1171	-43.945\\
1172	-98.877\\
1173	-91.553\\
1174	-137.939\\
1175	-168.457\\
1176	-275.879\\
1177	-316.162\\
1178	-260.01\\
1179	-249.023\\
1180	-159.912\\
1181	-185.547\\
1182	-195.313\\
1183	-212.402\\
1184	-158.691\\
1185	-144.043\\
1186	-142.822\\
1187	-129.395\\
1188	-156.25\\
1189	-109.863\\
1190	-195.313\\
1191	-234.375\\
1192	-175.781\\
1193	-104.98\\
1194	-111.084\\
1195	-192.871\\
1196	-234.375\\
1197	-289.307\\
1198	-303.955\\
1199	-302.734\\
1200	-228.271\\
1201	-205.078\\
1202	-235.596\\
1203	-275.879\\
1204	-170.898\\
1205	-104.98\\
1206	-152.588\\
1207	-122.07\\
1208	-79.346\\
1209	-109.863\\
1210	-122.07\\
1211	-95.215\\
1212	-75.684\\
1213	-101.318\\
1214	-80.566\\
1215	-115.967\\
1216	-162.354\\
1217	-147.705\\
1218	-100.098\\
1219	-101.318\\
1220	-157.471\\
1221	-261.23\\
1222	-185.547\\
1223	-118.408\\
1224	-85.449\\
1225	-101.318\\
1226	-90.332\\
1227	-67.139\\
1228	-76.904\\
1229	-92.773\\
1230	-120.85\\
1231	-134.277\\
1232	-89.111\\
1233	-156.25\\
1234	-179.443\\
1235	-170.898\\
1236	-140.381\\
1237	-150.146\\
1238	-202.637\\
1239	-125.732\\
1240	-53.711\\
1241	-78.125\\
1242	-85.449\\
1243	-119.629\\
1244	-101.318\\
1245	-74.463\\
1246	-118.408\\
1247	-112.305\\
1248	-79.346\\
1249	-102.539\\
1250	-90.332\\
1251	-80.566\\
1252	-95.215\\
1253	-54.932\\
1254	-68.359\\
1255	-47.607\\
1256	-51.27\\
1257	-106.201\\
1258	-122.07\\
1259	-167.236\\
1260	-219.727\\
1261	-142.822\\
1262	-83.008\\
1263	-63.477\\
1264	-62.256\\
1265	-91.553\\
1266	-54.932\\
1267	-70.801\\
1268	-85.449\\
1269	-150.146\\
1270	-133.057\\
1271	-190.43\\
1272	-124.512\\
1273	-117.188\\
1274	-56.152\\
1275	-62.256\\
1276	-34.18\\
1277	-46.387\\
1278	-51.27\\
1279	-95.215\\
1280	-100.098\\
1281	-117.188\\
1282	-123.291\\
1283	-197.754\\
1284	-170.898\\
1285	-128.174\\
1286	-153.809\\
1287	-163.574\\
1288	-213.623\\
1289	-161.133\\
1290	-104.98\\
1291	-53.711\\
1292	-40.283\\
1293	-41.504\\
1294	-52.49\\
1295	-54.932\\
1296	-51.27\\
1297	-70.801\\
1298	-108.643\\
1299	-98.877\\
1300	-103.76\\
1301	-118.408\\
1302	-69.58\\
1303	-36.621\\
1304	-81.787\\
1305	-125.732\\
1306	-125.732\\
1307	-142.822\\
1308	-118.408\\
1309	-87.891\\
1310	-123.291\\
1311	-172.119\\
1312	-172.119\\
1313	-120.85\\
1314	-207.52\\
1315	-155.029\\
1316	-151.367\\
1317	-173.34\\
1318	-172.119\\
1319	-130.615\\
1320	-112.305\\
1321	-192.871\\
1322	-266.113\\
1323	-202.637\\
1324	-124.512\\
1325	-118.408\\
1326	-115.967\\
1327	-150.146\\
1328	-173.34\\
1329	-125.732\\
1330	-134.277\\
1331	-177.002\\
1332	-192.871\\
1333	-120.85\\
1334	-97.656\\
1335	-130.615\\
1336	-218.506\\
1337	-195.313\\
1338	-203.857\\
1339	-115.967\\
1340	-108.643\\
1341	-101.318\\
1342	-63.477\\
1343	-41.504\\
1344	-34.18\\
1345	-29.297\\
1346	-78.125\\
1347	-108.643\\
1348	-131.836\\
1349	-93.994\\
1350	-107.422\\
1351	-119.629\\
1352	-72.021\\
1353	-79.346\\
1354	-73.242\\
1355	-54.932\\
1356	-75.684\\
1357	-120.85\\
1358	-156.25\\
1359	-95.215\\
1360	-73.242\\
1361	-58.594\\
1362	-74.463\\
1363	-45.166\\
1364	-117.188\\
1365	-195.313\\
1366	-163.574\\
1367	-216.064\\
1368	-241.699\\
1369	-249.023\\
1370	-181.885\\
1371	-131.836\\
1372	-130.615\\
1373	-129.395\\
1374	-155.029\\
1375	-184.326\\
1376	-181.885\\
1377	-247.803\\
1378	-270.996\\
1379	-328.369\\
1380	-217.285\\
1381	-238.037\\
1382	-264.893\\
1383	-205.078\\
1384	-239.258\\
1385	-200.195\\
1386	-113.525\\
1387	-74.463\\
1388	-79.346\\
1389	-117.188\\
1390	-97.656\\
1391	-64.697\\
1392	-90.332\\
1393	-63.477\\
1394	-48.828\\
1395	-64.697\\
1396	-72.021\\
1397	-48.828\\
1398	-100.098\\
1399	-135.498\\
1400	-97.656\\
1401	-173.34\\
1402	-241.699\\
1403	-220.947\\
1404	-279.541\\
1405	-187.988\\
1406	-206.299\\
1407	-134.277\\
1408	-85.449\\
1409	-108.643\\
1410	-87.891\\
1411	-63.477\\
1412	-92.773\\
1413	-50.049\\
1414	-32.959\\
1415	-43.945\\
1416	-98.877\\
1417	-125.732\\
1418	-157.471\\
1419	-152.588\\
1420	-146.484\\
1421	-168.457\\
1422	-135.498\\
1423	-108.643\\
1424	-112.305\\
1425	-89.111\\
1426	-146.484\\
1427	-96.436\\
1428	-80.566\\
1429	-120.85\\
1430	-164.795\\
1431	-123.291\\
1432	-93.994\\
1433	-80.566\\
1434	-91.553\\
1435	-61.035\\
1436	-61.035\\
1437	-87.891\\
1438	-109.863\\
1439	-124.512\\
1440	-96.436\\
1441	-128.174\\
1442	-114.746\\
1443	-61.035\\
1444	-69.58\\
1445	-135.498\\
1446	-190.43\\
1447	-186.768\\
1448	-157.471\\
1449	-152.588\\
1450	-114.746\\
1451	-109.863\\
1452	-87.891\\
1453	-125.732\\
1454	-117.188\\
1455	-70.801\\
1456	-107.422\\
1457	-126.953\\
1458	-150.146\\
1459	-164.795\\
1460	-163.574\\
1461	-222.168\\
1462	-267.334\\
1463	-302.734\\
1464	-190.43\\
1465	-106.201\\
1466	-68.359\\
1467	-45.166\\
1468	-70.801\\
1469	-64.697\\
1470	-119.629\\
1471	-102.539\\
1472	-93.994\\
1473	-124.512\\
1474	-145.264\\
1475	-172.119\\
1476	-183.105\\
1477	-258.789\\
1478	-231.934\\
1479	-177.002\\
1480	-120.85\\
1481	-120.85\\
1482	-123.291\\
1483	-157.471\\
1484	-108.643\\
1485	-113.525\\
1486	-107.422\\
1487	-58.594\\
1488	-102.539\\
1489	-152.588\\
1490	-107.422\\
1491	-115.967\\
1492	-216.064\\
1493	-159.912\\
1494	-156.25\\
1495	-219.727\\
1496	-206.299\\
1497	-129.395\\
1498	-85.449\\
1499	-85.449\\
1500	-146.484\\
};
\end{axis}

\begin{axis}[%
width=5.090835cm,
height=2.240107cm,
at={(0cm,7.879947cm)},
scale only axis,
xmin=1000,
xmax=1500,
xlabel={Sample index},
ymin=-28.076,
ymax=0,
ylabel={C5 load, kN},
legend style={legend cell align=left,align=left,draw=white!15!black}
]
\addplot [color=mycolor1,solid,forget plot]
  table[row sep=crcr]{%
1000	-10.986\\
1001	-15.869\\
1002	-10.986\\
1003	-12.207\\
1004	-13.428\\
1005	-15.869\\
1006	-14.648\\
1007	-14.648\\
1008	-7.324\\
1009	-8.545\\
1010	-10.986\\
1011	-9.766\\
1012	-9.766\\
1013	-6.104\\
1014	-2.441\\
1015	-3.662\\
1016	-2.441\\
1017	-12.207\\
1018	-13.428\\
1019	-12.207\\
1020	-10.986\\
1021	-6.104\\
1022	-6.104\\
1023	-1.221\\
1024	-6.104\\
1025	-10.986\\
1026	-12.207\\
1027	-8.545\\
1028	-14.648\\
1029	-14.648\\
1030	-9.766\\
1031	-13.428\\
1032	-12.207\\
1033	-9.766\\
1034	-9.766\\
1035	-8.545\\
1036	-8.545\\
1037	-12.207\\
1038	-10.986\\
1039	-10.986\\
1040	-10.986\\
1041	-10.986\\
1042	-14.648\\
1043	-15.869\\
1044	-9.766\\
1045	-8.545\\
1046	-8.545\\
1047	-6.104\\
1048	-7.324\\
1049	-7.324\\
1050	-4.883\\
1051	-9.766\\
1052	-10.986\\
1053	-8.545\\
1054	-13.428\\
1055	-15.869\\
1056	-8.545\\
1057	-6.104\\
1058	-6.104\\
1059	-9.766\\
1060	-6.104\\
1061	-4.883\\
1062	-7.324\\
1063	-7.324\\
1064	-3.662\\
1065	-6.104\\
1066	-8.545\\
1067	-10.986\\
1068	-10.986\\
1069	-9.766\\
1070	-13.428\\
1071	-10.986\\
1072	-12.207\\
1073	-10.986\\
1074	-10.986\\
1075	-7.324\\
1076	-8.545\\
1077	-14.648\\
1078	-20.752\\
1079	-20.752\\
1080	-19.531\\
1081	-13.428\\
1082	-15.869\\
1083	-24.414\\
1084	-23.193\\
1085	-15.869\\
1086	-21.973\\
1087	-28.076\\
1088	-23.193\\
1089	-17.09\\
1090	-14.648\\
1091	-13.428\\
1092	-9.766\\
1093	-10.986\\
1094	-12.207\\
1095	-4.883\\
1096	-4.883\\
1097	-7.324\\
1098	-10.986\\
1099	-9.766\\
1100	-9.766\\
1101	-10.986\\
1102	-13.428\\
1103	-14.648\\
1104	-15.869\\
1105	-14.648\\
1106	-17.09\\
1107	-12.207\\
1108	-12.207\\
1109	-10.986\\
1110	-8.545\\
1111	-9.766\\
1112	-12.207\\
1113	-8.545\\
1114	-9.766\\
1115	-7.324\\
1116	-6.104\\
1117	-8.545\\
1118	-6.104\\
1119	-4.883\\
1120	-8.545\\
1121	-13.428\\
1122	-12.207\\
1123	-10.986\\
1124	-7.324\\
1125	-8.545\\
1126	-19.531\\
1127	-14.648\\
1128	-10.986\\
1129	-17.09\\
1130	-12.207\\
1131	-6.104\\
1132	-9.766\\
1133	-8.545\\
1134	-14.648\\
1135	-15.869\\
1136	-19.531\\
1137	-14.648\\
1138	-14.648\\
1139	-13.428\\
1140	-13.428\\
1141	-14.648\\
1142	-10.986\\
1143	-7.324\\
1144	-7.324\\
1145	-6.104\\
1146	-8.545\\
1147	-10.986\\
1148	-15.869\\
1149	-14.648\\
1150	-8.545\\
1151	-8.545\\
1152	-9.766\\
1153	-6.104\\
1154	-2.441\\
1155	-4.883\\
1156	-7.324\\
1157	-8.545\\
1158	-6.104\\
1159	-7.324\\
1160	-6.104\\
1161	-4.883\\
1162	-4.883\\
1163	-1.221\\
1164	-3.662\\
1165	-12.207\\
1166	-14.648\\
1167	-15.869\\
1168	-12.207\\
1169	-7.324\\
1170	-6.104\\
1171	-3.662\\
1172	-7.324\\
1173	-8.545\\
1174	-9.766\\
1175	-12.207\\
1176	-18.311\\
1177	-18.311\\
1178	-19.531\\
1179	-18.311\\
1180	-15.869\\
1181	-15.869\\
1182	-17.09\\
1183	-18.311\\
1184	-12.207\\
1185	-10.986\\
1186	-12.207\\
1187	-10.986\\
1188	-12.207\\
1189	-9.766\\
1190	-15.869\\
1191	-18.311\\
1192	-12.207\\
1193	-7.324\\
1194	-9.766\\
1195	-17.09\\
1196	-18.311\\
1197	-18.311\\
1198	-21.973\\
1199	-20.752\\
1200	-17.09\\
1201	-15.869\\
1202	-18.311\\
1203	-20.752\\
1204	-17.09\\
1205	-8.545\\
1206	-14.648\\
1207	-12.207\\
1208	-7.324\\
1209	-10.986\\
1210	-9.766\\
1211	-8.545\\
1212	-7.324\\
1213	-8.545\\
1214	-8.545\\
1215	-9.766\\
1216	-13.428\\
1217	-14.648\\
1218	-7.324\\
1219	-8.545\\
1220	-12.207\\
1221	-17.09\\
1222	-15.869\\
1223	-8.545\\
1224	-8.545\\
1225	-9.766\\
1226	-8.545\\
1227	-7.324\\
1228	-6.104\\
1229	-8.545\\
1230	-9.766\\
1231	-9.766\\
1232	-9.766\\
1233	-12.207\\
1234	-14.648\\
1235	-15.869\\
1236	-10.986\\
1237	-13.428\\
1238	-17.09\\
1239	-12.207\\
1240	-3.662\\
1241	-9.766\\
1242	-8.545\\
1243	-9.766\\
1244	-8.545\\
1245	-8.545\\
1246	-8.545\\
1247	-10.986\\
1248	-7.324\\
1249	-7.324\\
1250	-9.766\\
1251	-6.104\\
1252	-8.545\\
1253	-6.104\\
1254	-3.662\\
1255	-6.104\\
1256	-4.883\\
1257	-10.986\\
1258	-12.207\\
1259	-12.207\\
1260	-18.311\\
1261	-10.986\\
1262	-4.883\\
1263	-6.104\\
1264	-6.104\\
1265	-8.545\\
1266	-6.104\\
1267	-8.545\\
1268	-7.324\\
1269	-13.428\\
1270	-9.766\\
1271	-14.648\\
1272	-10.986\\
1273	-9.766\\
1274	-6.104\\
1275	-3.662\\
1276	-3.662\\
1277	-1.221\\
1278	-2.441\\
1279	-8.545\\
1280	-8.545\\
1281	-7.324\\
1282	-9.766\\
1283	-15.869\\
1284	-10.986\\
1285	-9.766\\
1286	-9.766\\
1287	-13.428\\
1288	-14.648\\
1289	-12.207\\
1290	-12.207\\
1291	-6.104\\
1292	-2.441\\
1293	-4.883\\
1294	-4.883\\
1295	-6.104\\
1296	-3.662\\
1297	-4.883\\
1298	-9.766\\
1299	-7.324\\
1300	-7.324\\
1301	-10.986\\
1302	-7.324\\
1303	-4.883\\
1304	-9.766\\
1305	-12.207\\
1306	-9.766\\
1307	-10.986\\
1308	-7.324\\
1309	-7.324\\
1310	-10.986\\
1311	-14.648\\
1312	-13.428\\
1313	-8.545\\
1314	-14.648\\
1315	-13.428\\
1316	-13.428\\
1317	-13.428\\
1318	-13.428\\
1319	-10.986\\
1320	-10.986\\
1321	-12.207\\
1322	-18.311\\
1323	-17.09\\
1324	-10.986\\
1325	-10.986\\
1326	-10.986\\
1327	-13.428\\
1328	-13.428\\
1329	-9.766\\
1330	-12.207\\
1331	-14.648\\
1332	-14.648\\
1333	-9.766\\
1334	-8.545\\
1335	-10.986\\
1336	-18.311\\
1337	-14.648\\
1338	-13.428\\
1339	-9.766\\
1340	-10.986\\
1341	-9.766\\
1342	-6.104\\
1343	-4.883\\
1344	-3.662\\
1345	-3.662\\
1346	-7.324\\
1347	-10.986\\
1348	-9.766\\
1349	-6.104\\
1350	-7.324\\
1351	-9.766\\
1352	-6.104\\
1353	-6.104\\
1354	-9.766\\
1355	-6.104\\
1356	-7.324\\
1357	-12.207\\
1358	-12.207\\
1359	-8.545\\
1360	-4.883\\
1361	-6.104\\
1362	-7.324\\
1363	-6.104\\
1364	-7.324\\
1365	-14.648\\
1366	-9.766\\
1367	-17.09\\
1368	-20.752\\
1369	-18.311\\
1370	-13.428\\
1371	-10.986\\
1372	-10.986\\
1373	-12.207\\
1374	-13.428\\
1375	-14.648\\
1376	-15.869\\
1377	-20.752\\
1378	-20.752\\
1379	-24.414\\
1380	-15.869\\
1381	-20.752\\
1382	-20.752\\
1383	-14.648\\
1384	-17.09\\
1385	-18.311\\
1386	-9.766\\
1387	-7.324\\
1388	-8.545\\
1389	-9.766\\
1390	-7.324\\
1391	-6.104\\
1392	-7.324\\
1393	-6.104\\
1394	-3.662\\
1395	-4.883\\
1396	-7.324\\
1397	-2.441\\
1398	-10.986\\
1399	-8.545\\
1400	-7.324\\
1401	-15.869\\
1402	-19.531\\
1403	-19.531\\
1404	-19.531\\
1405	-14.648\\
1406	-17.09\\
1407	-12.207\\
1408	-7.324\\
1409	-10.986\\
1410	-7.324\\
1411	-3.662\\
1412	-6.104\\
1413	-4.883\\
1414	-2.441\\
1415	-4.883\\
1416	-10.986\\
1417	-9.766\\
1418	-10.986\\
1419	-10.986\\
1420	-12.207\\
1421	-14.648\\
1422	-10.986\\
1423	-10.986\\
1424	-8.545\\
1425	-7.324\\
1426	-13.428\\
1427	-9.766\\
1428	-6.104\\
1429	-10.986\\
1430	-13.428\\
1431	-10.986\\
1432	-7.324\\
1433	-7.324\\
1434	-7.324\\
1435	-4.883\\
1436	-6.104\\
1437	-7.324\\
1438	-10.986\\
1439	-8.545\\
1440	-7.324\\
1441	-10.986\\
1442	-8.545\\
1443	-6.104\\
1444	-7.324\\
1445	-13.428\\
1446	-14.648\\
1447	-13.428\\
1448	-12.207\\
1449	-12.207\\
1450	-8.545\\
1451	-9.766\\
1452	-8.545\\
1453	-12.207\\
1454	-7.324\\
1455	-4.883\\
1456	-9.766\\
1457	-9.766\\
1458	-10.986\\
1459	-12.207\\
1460	-10.986\\
1461	-17.09\\
1462	-20.752\\
1463	-23.193\\
1464	-15.869\\
1465	-9.766\\
1466	-6.104\\
1467	-6.104\\
1468	-9.766\\
1469	-6.104\\
1470	-9.766\\
1471	-6.104\\
1472	-7.324\\
1473	-8.545\\
1474	-10.986\\
1475	-13.428\\
1476	-14.648\\
1477	-19.531\\
1478	-14.648\\
1479	-13.428\\
1480	-10.986\\
1481	-10.986\\
1482	-10.986\\
1483	-12.207\\
1484	-9.766\\
1485	-10.986\\
1486	-9.766\\
1487	-7.324\\
1488	-12.207\\
1489	-13.428\\
1490	-8.545\\
1491	-9.766\\
1492	-18.311\\
1493	-12.207\\
1494	-12.207\\
1495	-17.09\\
1496	-14.648\\
1497	-10.986\\
1498	-7.324\\
1499	-7.324\\
1500	-12.207\\
};
\end{axis}

\begin{axis}[%
width=5.090835cm,
height=2.240107cm,
at={(0cm,11.81992cm)},
scale only axis,
xmin=1000,
xmax=1500,
xlabel={Sample index},
ymin=-40.283,
ymax=0,
ylabel={C3 load, kN},
legend style={legend cell align=left,align=left,draw=white!15!black}
]
\addplot [color=mycolor1,solid,forget plot]
  table[row sep=crcr]{%
1000	-17.09\\
1001	-19.531\\
1002	-14.648\\
1003	-14.648\\
1004	-19.531\\
1005	-18.311\\
1006	-23.193\\
1007	-18.311\\
1008	-9.766\\
1009	-15.869\\
1010	-12.207\\
1011	-14.648\\
1012	-10.986\\
1013	-3.662\\
1014	-2.441\\
1015	-4.883\\
1016	-6.104\\
1017	-14.648\\
1018	-17.09\\
1019	-17.09\\
1020	-13.428\\
1021	-9.766\\
1022	-18.311\\
1023	-14.648\\
1024	-10.986\\
1025	-13.428\\
1026	-12.207\\
1027	-10.986\\
1028	-20.752\\
1029	-19.531\\
1030	-12.207\\
1031	-20.752\\
1032	-20.752\\
1033	-15.869\\
1034	-13.428\\
1035	-9.766\\
1036	-14.648\\
1037	-14.648\\
1038	-14.648\\
1039	-14.648\\
1040	-14.648\\
1041	-15.869\\
1042	-21.973\\
1043	-20.752\\
1044	-13.428\\
1045	-13.428\\
1046	-8.545\\
1047	-12.207\\
1048	-12.207\\
1049	-13.428\\
1050	-10.986\\
1051	-13.428\\
1052	-14.648\\
1053	-13.428\\
1054	-20.752\\
1055	-13.428\\
1056	-9.766\\
1057	-7.324\\
1058	-8.545\\
1059	-10.986\\
1060	-6.104\\
1061	-9.766\\
1062	-10.986\\
1063	-6.104\\
1064	-6.104\\
1065	-9.766\\
1066	-9.766\\
1067	-15.869\\
1068	-14.648\\
1069	-17.09\\
1070	-15.869\\
1071	-18.311\\
1072	-13.428\\
1073	-14.648\\
1074	-12.207\\
1075	-10.986\\
1076	-12.207\\
1077	-23.193\\
1078	-28.076\\
1079	-30.518\\
1080	-29.297\\
1081	-18.311\\
1082	-28.076\\
1083	-32.959\\
1084	-31.738\\
1085	-20.752\\
1086	-34.18\\
1087	-40.283\\
1088	-29.297\\
1089	-24.414\\
1090	-19.531\\
1091	-15.869\\
1092	-12.207\\
1093	-14.648\\
1094	-9.766\\
1095	-7.324\\
1096	-7.324\\
1097	-10.986\\
1098	-13.428\\
1099	-12.207\\
1100	-12.207\\
1101	-15.869\\
1102	-18.311\\
1103	-19.531\\
1104	-15.869\\
1105	-21.973\\
1106	-15.869\\
1107	-15.869\\
1108	-17.09\\
1109	-12.207\\
1110	-12.207\\
1111	-15.869\\
1112	-14.648\\
1113	-8.545\\
1114	-12.207\\
1115	-8.545\\
1116	-9.766\\
1117	-9.766\\
1118	-7.324\\
1119	-9.766\\
1120	-12.207\\
1121	-17.09\\
1122	-17.09\\
1123	-17.09\\
1124	-10.986\\
1125	-12.207\\
1126	-23.193\\
1127	-15.869\\
1128	-20.752\\
1129	-21.973\\
1130	-12.207\\
1131	-9.766\\
1132	-12.207\\
1133	-13.428\\
1134	-21.973\\
1135	-25.635\\
1136	-26.855\\
1137	-19.531\\
1138	-20.752\\
1139	-18.311\\
1140	-17.09\\
1141	-18.311\\
1142	-13.428\\
1143	-12.207\\
1144	-9.766\\
1145	-9.766\\
1146	-10.986\\
1147	-15.869\\
1148	-23.193\\
1149	-17.09\\
1150	-13.428\\
1151	-10.986\\
1152	-10.986\\
1153	-7.324\\
1154	-6.104\\
1155	-6.104\\
1156	-12.207\\
1157	-7.324\\
1158	-8.545\\
1159	-8.545\\
1160	-8.545\\
1161	-7.324\\
1162	-4.883\\
1163	-3.662\\
1164	-8.545\\
1165	-15.869\\
1166	-18.311\\
1167	-20.752\\
1168	-15.869\\
1169	-10.986\\
1170	-8.545\\
1171	-4.883\\
1172	-9.766\\
1173	-8.545\\
1174	-15.869\\
1175	-17.09\\
1176	-29.297\\
1177	-32.959\\
1178	-25.635\\
1179	-25.635\\
1180	-18.311\\
1181	-23.193\\
1182	-21.973\\
1183	-24.414\\
1184	-15.869\\
1185	-17.09\\
1186	-17.09\\
1187	-15.869\\
1188	-15.869\\
1189	-13.428\\
1190	-23.193\\
1191	-25.635\\
1192	-19.531\\
1193	-13.428\\
1194	-14.648\\
1195	-23.193\\
1196	-24.414\\
1197	-28.076\\
1198	-30.518\\
1199	-29.297\\
1200	-23.193\\
1201	-21.973\\
1202	-25.635\\
1203	-28.076\\
1204	-18.311\\
1205	-12.207\\
1206	-19.531\\
1207	-14.648\\
1208	-9.766\\
1209	-13.428\\
1210	-14.648\\
1211	-10.986\\
1212	-10.986\\
1213	-10.986\\
1214	-8.545\\
1215	-10.986\\
1216	-18.311\\
1217	-17.09\\
1218	-12.207\\
1219	-12.207\\
1220	-18.311\\
1221	-26.855\\
1222	-17.09\\
1223	-14.648\\
1224	-10.986\\
1225	-13.428\\
1226	-10.986\\
1227	-8.545\\
1228	-8.545\\
1229	-10.986\\
1230	-13.428\\
1231	-14.648\\
1232	-12.207\\
1233	-15.869\\
1234	-20.752\\
1235	-17.09\\
1236	-12.207\\
1237	-15.869\\
1238	-20.752\\
1239	-13.428\\
1240	-7.324\\
1241	-13.428\\
1242	-10.986\\
1243	-12.207\\
1244	-9.766\\
1245	-8.545\\
1246	-13.428\\
1247	-13.428\\
1248	-9.766\\
1249	-12.207\\
1250	-9.766\\
1251	-7.324\\
1252	-12.207\\
1253	-8.545\\
1254	-9.766\\
1255	-7.324\\
1256	-4.883\\
1257	-12.207\\
1258	-15.869\\
1259	-17.09\\
1260	-23.193\\
1261	-15.869\\
1262	-9.766\\
1263	-8.545\\
1264	-8.545\\
1265	-10.986\\
1266	-8.545\\
1267	-4.883\\
1268	-10.986\\
1269	-15.869\\
1270	-13.428\\
1271	-20.752\\
1272	-15.869\\
1273	-14.648\\
1274	-8.545\\
1275	-10.986\\
1276	-4.883\\
1277	-3.662\\
1278	-4.883\\
1279	-9.766\\
1280	-10.986\\
1281	-12.207\\
1282	-13.428\\
1283	-20.752\\
1284	-17.09\\
1285	-14.648\\
1286	-17.09\\
1287	-19.531\\
1288	-20.752\\
1289	-17.09\\
1290	-13.428\\
1291	-7.324\\
1292	-6.104\\
1293	-4.883\\
1294	-7.324\\
1295	-7.324\\
1296	-4.883\\
1297	-8.545\\
1298	-12.207\\
1299	-10.986\\
1300	-12.207\\
1301	-12.207\\
1302	-8.545\\
1303	-4.883\\
1304	-9.766\\
1305	-15.869\\
1306	-13.428\\
1307	-15.869\\
1308	-13.428\\
1309	-9.766\\
1310	-14.648\\
1311	-18.311\\
1312	-18.311\\
1313	-13.428\\
1314	-20.752\\
1315	-15.869\\
1316	-17.09\\
1317	-18.311\\
1318	-19.531\\
1319	-13.428\\
1320	-13.428\\
1321	-18.311\\
1322	-26.855\\
1323	-21.973\\
1324	-13.428\\
1325	-13.428\\
1326	-13.428\\
1327	-15.869\\
1328	-17.09\\
1329	-12.207\\
1330	-14.648\\
1331	-18.311\\
1332	-20.752\\
1333	-13.428\\
1334	-12.207\\
1335	-15.869\\
1336	-23.193\\
1337	-20.752\\
1338	-21.973\\
1339	-14.648\\
1340	-14.648\\
1341	-12.207\\
1342	-9.766\\
1343	-4.883\\
1344	-3.662\\
1345	-3.662\\
1346	-7.324\\
1347	-13.428\\
1348	-14.648\\
1349	-9.766\\
1350	-12.207\\
1351	-13.428\\
1352	-9.766\\
1353	-8.545\\
1354	-8.545\\
1355	-6.104\\
1356	-9.766\\
1357	-14.648\\
1358	-15.869\\
1359	-10.986\\
1360	-8.545\\
1361	-8.545\\
1362	-8.545\\
1363	-7.324\\
1364	-10.986\\
1365	-20.752\\
1366	-18.311\\
1367	-18.311\\
1368	-24.414\\
1369	-23.193\\
1370	-18.311\\
1371	-14.648\\
1372	-14.648\\
1373	-14.648\\
1374	-17.09\\
1375	-19.531\\
1376	-19.531\\
1377	-25.635\\
1378	-26.855\\
1379	-31.738\\
1380	-24.414\\
1381	-25.635\\
1382	-26.855\\
1383	-21.973\\
1384	-24.414\\
1385	-20.752\\
1386	-13.428\\
1387	-9.766\\
1388	-9.766\\
1389	-12.207\\
1390	-9.766\\
1391	-7.324\\
1392	-10.986\\
1393	-8.545\\
1394	-6.104\\
1395	-7.324\\
1396	-9.766\\
1397	-6.104\\
1398	-12.207\\
1399	-14.648\\
1400	-10.986\\
1401	-17.09\\
1402	-24.414\\
1403	-24.414\\
1404	-28.076\\
1405	-23.193\\
1406	-21.973\\
1407	-17.09\\
1408	-9.766\\
1409	-10.986\\
1410	-10.986\\
1411	-7.324\\
1412	-10.986\\
1413	-9.766\\
1414	-2.441\\
1415	-6.104\\
1416	-9.766\\
1417	-12.207\\
1418	-14.648\\
1419	-17.09\\
1420	-14.648\\
1421	-17.09\\
1422	-14.648\\
1423	-13.428\\
1424	-12.207\\
1425	-10.986\\
1426	-14.648\\
1427	-13.428\\
1428	-10.986\\
1429	-13.428\\
1430	-18.311\\
1431	-13.428\\
1432	-10.986\\
1433	-10.986\\
1434	-10.986\\
1435	-8.545\\
1436	-7.324\\
1437	-10.986\\
1438	-13.428\\
1439	-13.428\\
1440	-10.986\\
1441	-13.428\\
1442	-13.428\\
1443	-8.545\\
1444	-7.324\\
1445	-15.869\\
1446	-19.531\\
1447	-18.311\\
1448	-18.311\\
1449	-15.869\\
1450	-13.428\\
1451	-10.986\\
1452	-9.766\\
1453	-13.428\\
1454	-13.428\\
1455	-8.545\\
1456	-12.207\\
1457	-14.648\\
1458	-14.648\\
1459	-17.09\\
1460	-17.09\\
1461	-20.752\\
1462	-28.076\\
1463	-30.518\\
1464	-20.752\\
1465	-13.428\\
1466	-8.545\\
1467	-6.104\\
1468	-8.545\\
1469	-7.324\\
1470	-10.986\\
1471	-12.207\\
1472	-12.207\\
1473	-13.428\\
1474	-15.869\\
1475	-19.531\\
1476	-19.531\\
1477	-28.076\\
1478	-23.193\\
1479	-18.311\\
1480	-14.648\\
1481	-14.648\\
1482	-14.648\\
1483	-17.09\\
1484	-14.648\\
1485	-12.207\\
1486	-13.428\\
1487	-8.545\\
1488	-7.324\\
1489	-17.09\\
1490	-14.648\\
1491	-12.207\\
1492	-23.193\\
1493	-18.311\\
1494	-15.869\\
1495	-21.973\\
1496	-23.193\\
1497	-14.648\\
1498	-8.545\\
1499	-9.766\\
1500	-15.869\\
};
\end{axis}

\begin{axis}[%
width=5.090835cm,
height=2.240107cm,
at={(6.698467cm,11.81992cm)},
scale only axis,
xmin=1000,
xmax=1500,
xlabel={Sample index},
ymin=-26.855,
ymax=0,
ylabel={C4 load, kN},
legend style={legend cell align=left,align=left,draw=white!15!black}
]
\addplot [color=mycolor1,solid,forget plot]
  table[row sep=crcr]{%
1000	-10.986\\
1001	-15.869\\
1002	-10.986\\
1003	-10.986\\
1004	-14.648\\
1005	-14.648\\
1006	-17.09\\
1007	-10.986\\
1008	-7.324\\
1009	-14.648\\
1010	-8.545\\
1011	-9.766\\
1012	-6.104\\
1013	-3.662\\
1014	-3.662\\
1015	-3.662\\
1016	-2.441\\
1017	-13.428\\
1018	-13.428\\
1019	-13.428\\
1020	-9.766\\
1021	-7.324\\
1022	-10.986\\
1023	-12.207\\
1024	-6.104\\
1025	-9.766\\
1026	-13.428\\
1027	-7.324\\
1028	-15.869\\
1029	-13.428\\
1030	-10.986\\
1031	-17.09\\
1032	-13.428\\
1033	-10.986\\
1034	-8.545\\
1035	-8.545\\
1036	-10.986\\
1037	-12.207\\
1038	-10.986\\
1039	-9.766\\
1040	-10.986\\
1041	-10.986\\
1042	-15.869\\
1043	-14.648\\
1044	-10.986\\
1045	-10.986\\
1046	-6.104\\
1047	-4.883\\
1048	-8.545\\
1049	-7.324\\
1050	-7.324\\
1051	-10.986\\
1052	-9.766\\
1053	-9.766\\
1054	-15.869\\
1055	-8.545\\
1056	-6.104\\
1057	-7.324\\
1058	-8.545\\
1059	-9.766\\
1060	-4.883\\
1061	-7.324\\
1062	-7.324\\
1063	-7.324\\
1064	-6.104\\
1065	-7.324\\
1066	-8.545\\
1067	-9.766\\
1068	-10.986\\
1069	-10.986\\
1070	-12.207\\
1071	-13.428\\
1072	-10.986\\
1073	-12.207\\
1074	-9.766\\
1075	-9.766\\
1076	-9.766\\
1077	-17.09\\
1078	-20.752\\
1079	-19.531\\
1080	-19.531\\
1081	-15.869\\
1082	-20.752\\
1083	-23.193\\
1084	-23.193\\
1085	-14.648\\
1086	-23.193\\
1087	-26.855\\
1088	-21.973\\
1089	-15.869\\
1090	-14.648\\
1091	-10.986\\
1092	-8.545\\
1093	-10.986\\
1094	-8.545\\
1095	-6.104\\
1096	-6.104\\
1097	-8.545\\
1098	-9.766\\
1099	-10.986\\
1100	-9.766\\
1101	-12.207\\
1102	-10.986\\
1103	-15.869\\
1104	-13.428\\
1105	-18.311\\
1106	-12.207\\
1107	-10.986\\
1108	-12.207\\
1109	-9.766\\
1110	-8.545\\
1111	-10.986\\
1112	-10.986\\
1113	-8.545\\
1114	-8.545\\
1115	-7.324\\
1116	-7.324\\
1117	-7.324\\
1118	-4.883\\
1119	-6.104\\
1120	-8.545\\
1121	-13.428\\
1122	-12.207\\
1123	-13.428\\
1124	-8.545\\
1125	-15.869\\
1126	-17.09\\
1127	-13.428\\
1128	-14.648\\
1129	-14.648\\
1130	-10.986\\
1131	-7.324\\
1132	-8.545\\
1133	-8.545\\
1134	-17.09\\
1135	-20.752\\
1136	-19.531\\
1137	-14.648\\
1138	-15.869\\
1139	-13.428\\
1140	-12.207\\
1141	-12.207\\
1142	-9.766\\
1143	-8.545\\
1144	-8.545\\
1145	-9.766\\
1146	-8.545\\
1147	-13.428\\
1148	-15.869\\
1149	-10.986\\
1150	-8.545\\
1151	-9.766\\
1152	-8.545\\
1153	-6.104\\
1154	-3.662\\
1155	-4.883\\
1156	-8.545\\
1157	-9.766\\
1158	-4.883\\
1159	-7.324\\
1160	-7.324\\
1161	-3.662\\
1162	-3.662\\
1163	-2.441\\
1164	-4.883\\
1165	-10.986\\
1166	-12.207\\
1167	-14.648\\
1168	-13.428\\
1169	-8.545\\
1170	-7.324\\
1171	-4.883\\
1172	-7.324\\
1173	-6.104\\
1174	-10.986\\
1175	-12.207\\
1176	-21.973\\
1177	-23.193\\
1178	-21.973\\
1179	-18.311\\
1180	-13.428\\
1181	-17.09\\
1182	-15.869\\
1183	-19.531\\
1184	-12.207\\
1185	-13.428\\
1186	-12.207\\
1187	-13.428\\
1188	-13.428\\
1189	-10.986\\
1190	-18.311\\
1191	-17.09\\
1192	-13.428\\
1193	-7.324\\
1194	-12.207\\
1195	-15.869\\
1196	-17.09\\
1197	-19.531\\
1198	-21.973\\
1199	-20.752\\
1200	-15.869\\
1201	-14.648\\
1202	-18.311\\
1203	-19.531\\
1204	-13.428\\
1205	-9.766\\
1206	-13.428\\
1207	-9.766\\
1208	-7.324\\
1209	-9.766\\
1210	-10.986\\
1211	-7.324\\
1212	-7.324\\
1213	-9.766\\
1214	-6.104\\
1215	-10.986\\
1216	-13.428\\
1217	-13.428\\
1218	-8.545\\
1219	-9.766\\
1220	-12.207\\
1221	-18.311\\
1222	-15.869\\
1223	-9.766\\
1224	-8.545\\
1225	-9.766\\
1226	-7.324\\
1227	-8.545\\
1228	-7.324\\
1229	-8.545\\
1230	-10.986\\
1231	-10.986\\
1232	-6.104\\
1233	-10.986\\
1234	-13.428\\
1235	-13.428\\
1236	-9.766\\
1237	-12.207\\
1238	-13.428\\
1239	-10.986\\
1240	-4.883\\
1241	-10.986\\
1242	-8.545\\
1243	-8.545\\
1244	-6.104\\
1245	-6.104\\
1246	-10.986\\
1247	-10.986\\
1248	-6.104\\
1249	-8.545\\
1250	-8.545\\
1251	-4.883\\
1252	-8.545\\
1253	-4.883\\
1254	-7.324\\
1255	-4.883\\
1256	-4.883\\
1257	-10.986\\
1258	-12.207\\
1259	-12.207\\
1260	-15.869\\
1261	-10.986\\
1262	-6.104\\
1263	-7.324\\
1264	-4.883\\
1265	-7.324\\
1266	-6.104\\
1267	-3.662\\
1268	-8.545\\
1269	-9.766\\
1270	-9.766\\
1271	-17.09\\
1272	-10.986\\
1273	-13.428\\
1274	-7.324\\
1275	-8.545\\
1276	-3.662\\
1277	-1.221\\
1278	-6.104\\
1279	-9.766\\
1280	-8.545\\
1281	-9.766\\
1282	-9.766\\
1283	-17.09\\
1284	-10.986\\
1285	-10.986\\
1286	-10.986\\
1287	-13.428\\
1288	-15.869\\
1289	-15.869\\
1290	-10.986\\
1291	-6.104\\
1292	-3.662\\
1293	-4.883\\
1294	-6.104\\
1295	-4.883\\
1296	-6.104\\
1297	-6.104\\
1298	-9.766\\
1299	-9.766\\
1300	-8.545\\
1301	-9.766\\
1302	-6.104\\
1303	-3.662\\
1304	-7.324\\
1305	-10.986\\
1306	-8.545\\
1307	-12.207\\
1308	-8.545\\
1309	-7.324\\
1310	-8.545\\
1311	-12.207\\
1312	-14.648\\
1313	-10.986\\
1314	-17.09\\
1315	-9.766\\
1316	-12.207\\
1317	-13.428\\
1318	-13.428\\
1319	-9.766\\
1320	-9.766\\
1321	-12.207\\
1322	-18.311\\
1323	-15.869\\
1324	-8.545\\
1325	-12.207\\
1326	-9.766\\
1327	-12.207\\
1328	-13.428\\
1329	-9.766\\
1330	-9.766\\
1331	-14.648\\
1332	-13.428\\
1333	-12.207\\
1334	-8.545\\
1335	-9.766\\
1336	-15.869\\
1337	-14.648\\
1338	-15.869\\
1339	-12.207\\
1340	-10.986\\
1341	-9.766\\
1342	-8.545\\
1343	-4.883\\
1344	-3.662\\
1345	-3.662\\
1346	-7.324\\
1347	-10.986\\
1348	-9.766\\
1349	-8.545\\
1350	-10.986\\
1351	-9.766\\
1352	-8.545\\
1353	-3.662\\
1354	-6.104\\
1355	-6.104\\
1356	-6.104\\
1357	-8.545\\
1358	-10.986\\
1359	-7.324\\
1360	-6.104\\
1361	-7.324\\
1362	-7.324\\
1363	-6.104\\
1364	-8.545\\
1365	-15.869\\
1366	-14.648\\
1367	-17.09\\
1368	-17.09\\
1369	-17.09\\
1370	-13.428\\
1371	-10.986\\
1372	-9.766\\
1373	-10.986\\
1374	-12.207\\
1375	-14.648\\
1376	-14.648\\
1377	-17.09\\
1378	-19.531\\
1379	-21.973\\
1380	-18.311\\
1381	-21.973\\
1382	-19.531\\
1383	-14.648\\
1384	-17.09\\
1385	-14.648\\
1386	-10.986\\
1387	-8.545\\
1388	-8.545\\
1389	-10.986\\
1390	-7.324\\
1391	-7.324\\
1392	-7.324\\
1393	-7.324\\
1394	-3.662\\
1395	-7.324\\
1396	-7.324\\
1397	-6.104\\
1398	-8.545\\
1399	-12.207\\
1400	-8.545\\
1401	-12.207\\
1402	-18.311\\
1403	-17.09\\
1404	-19.531\\
1405	-17.09\\
1406	-14.648\\
1407	-14.648\\
1408	-7.324\\
1409	-9.766\\
1410	-8.545\\
1411	-6.104\\
1412	-7.324\\
1413	-8.545\\
1414	-2.441\\
1415	-4.883\\
1416	-8.545\\
1417	-9.766\\
1418	-12.207\\
1419	-12.207\\
1420	-10.986\\
1421	-12.207\\
1422	-14.648\\
1423	-10.986\\
1424	-9.766\\
1425	-7.324\\
1426	-10.986\\
1427	-10.986\\
1428	-7.324\\
1429	-10.986\\
1430	-13.428\\
1431	-10.986\\
1432	-8.545\\
1433	-8.545\\
1434	-8.545\\
1435	-6.104\\
1436	-7.324\\
1437	-6.104\\
1438	-9.766\\
1439	-8.545\\
1440	-10.986\\
1441	-10.986\\
1442	-8.545\\
1443	-6.104\\
1444	-6.104\\
1445	-10.986\\
1446	-14.648\\
1447	-14.648\\
1448	-12.207\\
1449	-10.986\\
1450	-9.766\\
1451	-8.545\\
1452	-8.545\\
1453	-9.766\\
1454	-12.207\\
1455	-7.324\\
1456	-7.324\\
1457	-10.986\\
1458	-10.986\\
1459	-13.428\\
1460	-13.428\\
1461	-14.648\\
1462	-20.752\\
1463	-21.973\\
1464	-14.648\\
1465	-9.766\\
1466	-7.324\\
1467	-6.104\\
1468	-7.324\\
1469	-4.883\\
1470	-7.324\\
1471	-7.324\\
1472	-7.324\\
1473	-10.986\\
1474	-12.207\\
1475	-13.428\\
1476	-13.428\\
1477	-18.311\\
1478	-17.09\\
1479	-12.207\\
1480	-12.207\\
1481	-9.766\\
1482	-9.766\\
1483	-13.428\\
1484	-9.766\\
1485	-8.545\\
1486	-10.986\\
1487	-6.104\\
1488	-10.986\\
1489	-14.648\\
1490	-8.545\\
1491	-8.545\\
1492	-15.869\\
1493	-13.428\\
1494	-10.986\\
1495	-17.09\\
1496	-17.09\\
1497	-10.986\\
1498	-8.545\\
1499	-9.766\\
1500	-12.207\\
};
\end{axis}

\begin{axis}[%
width=5.090835cm,
height=2.240107cm,
at={(6.698467cm,3.939973cm)},
scale only axis,
xmin=1000,
xmax=1500,
xlabel={Sample index},
ymin=-279.541,
ymax=0,
ylabel={C8 load, kN},
legend style={legend cell align=left,align=left,draw=white!15!black}
]
\addplot [color=mycolor1,solid,forget plot]
  table[row sep=crcr]{%
1000	-96.436\\
1001	-122.07\\
1002	-100.098\\
1003	-93.994\\
1004	-130.615\\
1005	-125.732\\
1006	-153.809\\
1007	-115.967\\
1008	-61.035\\
1009	-74.463\\
1010	-73.242\\
1011	-86.67\\
1012	-73.242\\
1013	-34.18\\
1014	-20.752\\
1015	-23.193\\
1016	-58.594\\
1017	-100.098\\
1018	-109.863\\
1019	-114.746\\
1020	-83.008\\
1021	-52.49\\
1022	-104.98\\
1023	-81.787\\
1024	-59.814\\
1025	-97.656\\
1026	-83.008\\
1027	-72.021\\
1028	-118.408\\
1029	-107.422\\
1030	-83.008\\
1031	-119.629\\
1032	-123.291\\
1033	-95.215\\
1034	-80.566\\
1035	-62.256\\
1036	-75.684\\
1037	-95.215\\
1038	-87.891\\
1039	-91.553\\
1040	-96.436\\
1041	-102.539\\
1042	-141.602\\
1043	-128.174\\
1044	-85.449\\
1045	-79.346\\
1046	-47.607\\
1047	-48.828\\
1048	-68.359\\
1049	-52.49\\
1050	-57.373\\
1051	-75.684\\
1052	-89.111\\
1053	-81.787\\
1054	-128.174\\
1055	-98.877\\
1056	-57.373\\
1057	-45.166\\
1058	-62.256\\
1059	-72.021\\
1060	-40.283\\
1061	-42.725\\
1062	-56.152\\
1063	-46.387\\
1064	-40.283\\
1065	-52.49\\
1066	-62.256\\
1067	-92.773\\
1068	-90.332\\
1069	-107.422\\
1070	-98.877\\
1071	-115.967\\
1072	-91.553\\
1073	-91.553\\
1074	-79.346\\
1075	-75.684\\
1076	-76.904\\
1077	-133.057\\
1078	-189.209\\
1079	-194.092\\
1080	-189.209\\
1081	-119.629\\
1082	-170.898\\
1083	-213.623\\
1084	-219.727\\
1085	-169.678\\
1086	-219.727\\
1087	-279.541\\
1088	-197.754\\
1089	-161.133\\
1090	-109.863\\
1091	-85.449\\
1092	-73.242\\
1093	-91.553\\
1094	-62.256\\
1095	-46.387\\
1096	-42.725\\
1097	-54.932\\
1098	-79.346\\
1099	-74.463\\
1100	-75.684\\
1101	-100.098\\
1102	-103.76\\
1103	-141.602\\
1104	-114.746\\
1105	-142.822\\
1106	-104.98\\
1107	-90.332\\
1108	-103.76\\
1109	-76.904\\
1110	-69.58\\
1111	-84.229\\
1112	-86.67\\
1113	-59.814\\
1114	-64.697\\
1115	-48.828\\
1116	-53.711\\
1117	-57.373\\
1118	-41.504\\
1119	-52.49\\
1120	-65.918\\
1121	-101.318\\
1122	-104.98\\
1123	-114.746\\
1124	-68.359\\
1125	-93.994\\
1126	-150.146\\
1127	-114.746\\
1128	-125.732\\
1129	-128.174\\
1130	-80.566\\
1131	-53.711\\
1132	-75.684\\
1133	-85.449\\
1134	-137.939\\
1135	-172.119\\
1136	-170.898\\
1137	-123.291\\
1138	-130.615\\
1139	-115.967\\
1140	-113.525\\
1141	-111.084\\
1142	-76.904\\
1143	-68.359\\
1144	-61.035\\
1145	-57.373\\
1146	-62.256\\
1147	-91.553\\
1148	-140.381\\
1149	-111.084\\
1150	-79.346\\
1151	-64.697\\
1152	-62.256\\
1153	-40.283\\
1154	-29.297\\
1155	-36.621\\
1156	-63.477\\
1157	-51.27\\
1158	-52.49\\
1159	-58.594\\
1160	-54.932\\
1161	-37.842\\
1162	-28.076\\
1163	-23.193\\
1164	-39.063\\
1165	-83.008\\
1166	-117.188\\
1167	-130.615\\
1168	-86.67\\
1169	-58.594\\
1170	-45.166\\
1171	-32.959\\
1172	-67.139\\
1173	-65.918\\
1174	-91.553\\
1175	-117.188\\
1176	-186.768\\
1177	-216.064\\
1178	-177.002\\
1179	-168.457\\
1180	-109.863\\
1181	-122.07\\
1182	-131.836\\
1183	-146.484\\
1184	-108.643\\
1185	-100.098\\
1186	-98.877\\
1187	-89.111\\
1188	-106.201\\
1189	-78.125\\
1190	-130.615\\
1191	-159.912\\
1192	-113.525\\
1193	-70.801\\
1194	-73.242\\
1195	-123.291\\
1196	-153.809\\
1197	-189.209\\
1198	-202.637\\
1199	-202.637\\
1200	-153.809\\
1201	-137.939\\
1202	-158.691\\
1203	-187.988\\
1204	-109.863\\
1205	-72.021\\
1206	-101.318\\
1207	-86.67\\
1208	-53.711\\
1209	-74.463\\
1210	-84.229\\
1211	-67.139\\
1212	-51.27\\
1213	-70.801\\
1214	-58.594\\
1215	-79.346\\
1216	-107.422\\
1217	-100.098\\
1218	-69.58\\
1219	-70.801\\
1220	-107.422\\
1221	-177.002\\
1222	-128.174\\
1223	-79.346\\
1224	-61.035\\
1225	-70.801\\
1226	-64.697\\
1227	-48.828\\
1228	-53.711\\
1229	-65.918\\
1230	-83.008\\
1231	-91.553\\
1232	-63.477\\
1233	-104.98\\
1234	-124.512\\
1235	-118.408\\
1236	-91.553\\
1237	-100.098\\
1238	-135.498\\
1239	-87.891\\
1240	-42.725\\
1241	-57.373\\
1242	-62.256\\
1243	-81.787\\
1244	-72.021\\
1245	-52.49\\
1246	-79.346\\
1247	-80.566\\
1248	-57.373\\
1249	-70.801\\
1250	-63.477\\
1251	-56.152\\
1252	-67.139\\
1253	-41.504\\
1254	-48.828\\
1255	-36.621\\
1256	-40.283\\
1257	-75.684\\
1258	-84.229\\
1259	-113.525\\
1260	-146.484\\
1261	-98.877\\
1262	-58.594\\
1263	-46.387\\
1264	-43.945\\
1265	-63.477\\
1266	-42.725\\
1267	-47.607\\
1268	-61.035\\
1269	-102.539\\
1270	-93.994\\
1271	-134.277\\
1272	-89.111\\
1273	-81.787\\
1274	-46.387\\
1275	-45.166\\
1276	-28.076\\
1277	-35.4\\
1278	-37.842\\
1279	-67.139\\
1280	-70.801\\
1281	-79.346\\
1282	-85.449\\
1283	-125.732\\
1284	-112.305\\
1285	-84.229\\
1286	-102.539\\
1287	-109.863\\
1288	-144.043\\
1289	-115.967\\
1290	-75.684\\
1291	-40.283\\
1292	-29.297\\
1293	-29.297\\
1294	-36.621\\
1295	-39.063\\
1296	-37.842\\
1297	-50.049\\
1298	-74.463\\
1299	-68.359\\
1300	-70.801\\
1301	-83.008\\
1302	-51.27\\
1303	-30.518\\
1304	-58.594\\
1305	-90.332\\
1306	-87.891\\
1307	-97.656\\
1308	-83.008\\
1309	-61.035\\
1310	-85.449\\
1311	-118.408\\
1312	-118.408\\
1313	-84.229\\
1314	-136.719\\
1315	-100.098\\
1316	-101.318\\
1317	-117.188\\
1318	-115.967\\
1319	-86.67\\
1320	-76.904\\
1321	-128.174\\
1322	-178.223\\
1323	-139.16\\
1324	-79.346\\
1325	-76.904\\
1326	-79.346\\
1327	-98.877\\
1328	-114.746\\
1329	-85.449\\
1330	-90.332\\
1331	-120.85\\
1332	-131.836\\
1333	-87.891\\
1334	-68.359\\
1335	-91.553\\
1336	-156.25\\
1337	-137.939\\
1338	-140.381\\
1339	-84.229\\
1340	-76.904\\
1341	-72.021\\
1342	-47.607\\
1343	-30.518\\
1344	-26.855\\
1345	-23.193\\
1346	-54.932\\
1347	-70.801\\
1348	-87.891\\
1349	-65.918\\
1350	-73.242\\
1351	-83.008\\
1352	-53.711\\
1353	-56.152\\
1354	-53.711\\
1355	-40.283\\
1356	-51.27\\
1357	-84.229\\
1358	-106.201\\
1359	-63.477\\
1360	-48.828\\
1361	-40.283\\
1362	-50.049\\
1363	-35.4\\
1364	-76.904\\
1365	-130.615\\
1366	-104.98\\
1367	-139.16\\
1368	-162.354\\
1369	-164.795\\
1370	-124.512\\
1371	-90.332\\
1372	-89.111\\
1373	-89.111\\
1374	-106.201\\
1375	-125.732\\
1376	-125.732\\
1377	-168.457\\
1378	-183.105\\
1379	-219.727\\
1380	-147.705\\
1381	-166.016\\
1382	-184.326\\
1383	-140.381\\
1384	-162.354\\
1385	-139.16\\
1386	-80.566\\
1387	-52.49\\
1388	-56.152\\
1389	-81.787\\
1390	-65.918\\
1391	-43.945\\
1392	-62.256\\
1393	-47.607\\
1394	-36.621\\
1395	-46.387\\
1396	-52.49\\
1397	-36.621\\
1398	-70.801\\
1399	-92.773\\
1400	-68.359\\
1401	-114.746\\
1402	-164.795\\
1403	-150.146\\
1404	-196.533\\
1405	-131.836\\
1406	-144.043\\
1407	-96.436\\
1408	-63.477\\
1409	-76.904\\
1410	-59.814\\
1411	-45.166\\
1412	-64.697\\
1413	-36.621\\
1414	-28.076\\
1415	-34.18\\
1416	-70.801\\
1417	-86.67\\
1418	-107.422\\
1419	-103.76\\
1420	-100.098\\
1421	-114.746\\
1422	-93.994\\
1423	-76.904\\
1424	-76.904\\
1425	-62.256\\
1426	-97.656\\
1427	-68.359\\
1428	-56.152\\
1429	-81.787\\
1430	-113.525\\
1431	-86.67\\
1432	-65.918\\
1433	-56.152\\
1434	-65.918\\
1435	-45.166\\
1436	-45.166\\
1437	-59.814\\
1438	-75.684\\
1439	-84.229\\
1440	-65.918\\
1441	-86.67\\
1442	-79.346\\
1443	-47.607\\
1444	-50.049\\
1445	-95.215\\
1446	-131.836\\
1447	-131.836\\
1448	-107.422\\
1449	-103.76\\
1450	-79.346\\
1451	-76.904\\
1452	-61.035\\
1453	-89.111\\
1454	-75.684\\
1455	-50.049\\
1456	-74.463\\
1457	-85.449\\
1458	-101.318\\
1459	-109.863\\
1460	-109.863\\
1461	-148.926\\
1462	-181.885\\
1463	-205.078\\
1464	-129.395\\
1465	-73.242\\
1466	-47.607\\
1467	-34.18\\
1468	-54.932\\
1469	-46.387\\
1470	-80.566\\
1471	-72.021\\
1472	-64.697\\
1473	-85.449\\
1474	-98.877\\
1475	-117.188\\
1476	-123.291\\
1477	-175.781\\
1478	-150.146\\
1479	-117.188\\
1480	-80.566\\
1481	-80.566\\
1482	-83.008\\
1483	-106.201\\
1484	-75.684\\
1485	-79.346\\
1486	-74.463\\
1487	-42.725\\
1488	-74.463\\
1489	-107.422\\
1490	-73.242\\
1491	-80.566\\
1492	-147.705\\
1493	-109.863\\
1494	-106.201\\
1495	-147.705\\
1496	-140.381\\
1497	-87.891\\
1498	-56.152\\
1499	-58.594\\
1500	-97.656\\
};
\end{axis}

\begin{axis}[%
width=5.090835cm,
height=2.240107cm,
at={(6.698467cm,0cm)},
scale only axis,
xmin=1000,
xmax=1500,
xlabel={Sample index},
ymin=-158.691,
ymax=-17.09,
ylabel={C10 load, kN},
legend style={legend cell align=left,align=left,draw=white!15!black}
]
\addplot [color=mycolor1,solid,forget plot]
  table[row sep=crcr]{%
1000	-63.477\\
1001	-75.684\\
1002	-62.256\\
1003	-59.814\\
1004	-80.566\\
1005	-76.904\\
1006	-90.332\\
1007	-68.359\\
1008	-40.283\\
1009	-48.828\\
1010	-46.387\\
1011	-54.932\\
1012	-45.166\\
1013	-21.973\\
1014	-17.09\\
1015	-18.311\\
1016	-40.283\\
1017	-62.256\\
1018	-67.139\\
1019	-69.58\\
1020	-52.49\\
1021	-34.18\\
1022	-64.697\\
1023	-52.49\\
1024	-39.063\\
1025	-61.035\\
1026	-53.711\\
1027	-45.166\\
1028	-73.242\\
1029	-67.139\\
1030	-52.49\\
1031	-75.684\\
1032	-74.463\\
1033	-58.594\\
1034	-50.049\\
1035	-41.504\\
1036	-47.607\\
1037	-58.594\\
1038	-54.932\\
1039	-59.814\\
1040	-59.814\\
1041	-65.918\\
1042	-85.449\\
1043	-76.904\\
1044	-54.932\\
1045	-50.049\\
1046	-35.4\\
1047	-34.18\\
1048	-45.166\\
1049	-34.18\\
1050	-36.621\\
1051	-51.27\\
1052	-54.932\\
1053	-51.27\\
1054	-78.125\\
1055	-62.256\\
1056	-36.621\\
1057	-30.518\\
1058	-40.283\\
1059	-46.387\\
1060	-29.297\\
1061	-29.297\\
1062	-36.621\\
1063	-31.738\\
1064	-26.855\\
1065	-35.4\\
1066	-40.283\\
1067	-58.594\\
1068	-56.152\\
1069	-65.918\\
1070	-61.035\\
1071	-70.801\\
1072	-57.373\\
1073	-58.594\\
1074	-51.27\\
1075	-50.049\\
1076	-48.828\\
1077	-80.566\\
1078	-111.084\\
1079	-114.746\\
1080	-112.305\\
1081	-74.463\\
1082	-100.098\\
1083	-125.732\\
1084	-128.174\\
1085	-96.436\\
1086	-124.512\\
1087	-158.691\\
1088	-115.967\\
1089	-97.656\\
1090	-68.359\\
1091	-53.711\\
1092	-47.607\\
1093	-56.152\\
1094	-45.166\\
1095	-30.518\\
1096	-30.518\\
1097	-36.621\\
1098	-50.049\\
1099	-48.828\\
1100	-47.607\\
1101	-62.256\\
1102	-64.697\\
1103	-85.449\\
1104	-72.021\\
1105	-85.449\\
1106	-63.477\\
1107	-54.932\\
1108	-64.697\\
1109	-48.828\\
1110	-43.945\\
1111	-53.711\\
1112	-54.932\\
1113	-40.283\\
1114	-43.945\\
1115	-32.959\\
1116	-36.621\\
1117	-40.283\\
1118	-30.518\\
1119	-34.18\\
1120	-43.945\\
1121	-63.477\\
1122	-65.918\\
1123	-72.021\\
1124	-45.166\\
1125	-56.152\\
1126	-89.111\\
1127	-70.801\\
1128	-81.787\\
1129	-80.566\\
1130	-50.049\\
1131	-36.621\\
1132	-47.607\\
1133	-52.49\\
1134	-81.787\\
1135	-103.76\\
1136	-102.539\\
1137	-76.904\\
1138	-80.566\\
1139	-72.021\\
1140	-69.58\\
1141	-68.359\\
1142	-51.27\\
1143	-45.166\\
1144	-39.063\\
1145	-37.842\\
1146	-41.504\\
1147	-56.152\\
1148	-84.229\\
1149	-70.801\\
1150	-50.049\\
1151	-42.725\\
1152	-40.283\\
1153	-28.076\\
1154	-23.193\\
1155	-24.414\\
1156	-41.504\\
1157	-32.959\\
1158	-34.18\\
1159	-37.842\\
1160	-35.4\\
1161	-26.855\\
1162	-21.973\\
1163	-18.311\\
1164	-25.635\\
1165	-51.27\\
1166	-73.242\\
1167	-78.125\\
1168	-54.932\\
1169	-39.063\\
1170	-31.738\\
1171	-24.414\\
1172	-41.504\\
1173	-42.725\\
1174	-54.932\\
1175	-73.242\\
1176	-108.643\\
1177	-125.732\\
1178	-103.76\\
1179	-101.318\\
1180	-67.139\\
1181	-75.684\\
1182	-79.346\\
1183	-86.67\\
1184	-69.58\\
1185	-62.256\\
1186	-61.035\\
1187	-56.152\\
1188	-67.139\\
1189	-52.49\\
1190	-76.904\\
1191	-97.656\\
1192	-72.021\\
1193	-45.166\\
1194	-47.607\\
1195	-78.125\\
1196	-93.994\\
1197	-112.305\\
1198	-119.629\\
1199	-119.629\\
1200	-91.553\\
1201	-84.229\\
1202	-96.436\\
1203	-111.084\\
1204	-74.463\\
1205	-46.387\\
1206	-62.256\\
1207	-54.932\\
1208	-34.18\\
1209	-47.607\\
1210	-53.711\\
1211	-42.725\\
1212	-34.18\\
1213	-43.945\\
1214	-40.283\\
1215	-52.49\\
1216	-65.918\\
1217	-62.256\\
1218	-45.166\\
1219	-45.166\\
1220	-67.139\\
1221	-104.98\\
1222	-79.346\\
1223	-51.27\\
1224	-40.283\\
1225	-47.607\\
1226	-40.283\\
1227	-31.738\\
1228	-35.4\\
1229	-42.725\\
1230	-51.27\\
1231	-57.373\\
1232	-42.725\\
1233	-63.477\\
1234	-75.684\\
1235	-70.801\\
1236	-57.373\\
1237	-62.256\\
1238	-81.787\\
1239	-52.49\\
1240	-28.076\\
1241	-36.621\\
1242	-42.725\\
1243	-51.27\\
1244	-46.387\\
1245	-36.621\\
1246	-52.49\\
1247	-52.49\\
1248	-37.842\\
1249	-47.607\\
1250	-41.504\\
1251	-37.842\\
1252	-45.166\\
1253	-29.297\\
1254	-31.738\\
1255	-29.297\\
1256	-26.855\\
1257	-47.607\\
1258	-53.711\\
1259	-69.58\\
1260	-89.111\\
1261	-62.256\\
1262	-36.621\\
1263	-31.738\\
1264	-29.297\\
1265	-40.283\\
1266	-31.738\\
1267	-30.518\\
1268	-39.063\\
1269	-61.035\\
1270	-53.711\\
1271	-79.346\\
1272	-54.932\\
1273	-48.828\\
1274	-32.959\\
1275	-29.297\\
1276	-23.193\\
1277	-23.193\\
1278	-25.635\\
1279	-40.283\\
1280	-46.387\\
1281	-48.828\\
1282	-51.27\\
1283	-79.346\\
1284	-73.242\\
1285	-53.711\\
1286	-63.477\\
1287	-69.58\\
1288	-85.449\\
1289	-72.021\\
1290	-47.607\\
1291	-28.076\\
1292	-21.973\\
1293	-21.973\\
1294	-26.855\\
1295	-26.855\\
1296	-25.635\\
1297	-32.959\\
1298	-47.607\\
1299	-43.945\\
1300	-45.166\\
1301	-52.49\\
1302	-35.4\\
1303	-20.752\\
1304	-35.4\\
1305	-56.152\\
1306	-54.932\\
1307	-59.814\\
1308	-53.711\\
1309	-40.283\\
1310	-51.27\\
1311	-72.021\\
1312	-70.801\\
1313	-51.27\\
1314	-81.787\\
1315	-62.256\\
1316	-63.477\\
1317	-73.242\\
1318	-70.801\\
1319	-54.932\\
1320	-47.607\\
1321	-78.125\\
1322	-108.643\\
1323	-85.449\\
1324	-50.049\\
1325	-51.27\\
1326	-51.27\\
1327	-61.035\\
1328	-72.021\\
1329	-54.932\\
1330	-56.152\\
1331	-73.242\\
1332	-79.346\\
1333	-56.152\\
1334	-45.166\\
1335	-57.373\\
1336	-87.891\\
1337	-78.125\\
1338	-80.566\\
1339	-54.932\\
1340	-45.166\\
1341	-47.607\\
1342	-31.738\\
1343	-20.752\\
1344	-20.752\\
1345	-17.09\\
1346	-31.738\\
1347	-48.828\\
1348	-54.932\\
1349	-40.283\\
1350	-45.166\\
1351	-52.49\\
1352	-35.4\\
1353	-35.4\\
1354	-35.4\\
1355	-29.297\\
1356	-31.738\\
1357	-51.27\\
1358	-64.697\\
1359	-41.504\\
1360	-32.959\\
1361	-28.076\\
1362	-32.959\\
1363	-26.855\\
1364	-45.166\\
1365	-80.566\\
1366	-63.477\\
1367	-81.787\\
1368	-98.877\\
1369	-97.656\\
1370	-76.904\\
1371	-58.594\\
1372	-54.932\\
1373	-57.373\\
1374	-65.918\\
1375	-76.904\\
1376	-76.904\\
1377	-100.098\\
1378	-108.643\\
1379	-130.615\\
1380	-91.553\\
1381	-98.877\\
1382	-108.643\\
1383	-85.449\\
1384	-96.436\\
1385	-85.449\\
1386	-51.27\\
1387	-34.18\\
1388	-36.621\\
1389	-51.27\\
1390	-43.945\\
1391	-31.738\\
1392	-41.504\\
1393	-32.959\\
1394	-23.193\\
1395	-31.738\\
1396	-34.18\\
1397	-25.635\\
1398	-42.725\\
1399	-58.594\\
1400	-45.166\\
1401	-68.359\\
1402	-100.098\\
1403	-87.891\\
1404	-109.863\\
1405	-81.787\\
1406	-81.787\\
1407	-62.256\\
1408	-37.842\\
1409	-46.387\\
1410	-42.725\\
1411	-29.297\\
1412	-40.283\\
1413	-21.973\\
1414	-19.531\\
1415	-24.414\\
1416	-45.166\\
1417	-57.373\\
1418	-65.918\\
1419	-65.918\\
1420	-62.256\\
1421	-69.58\\
1422	-57.373\\
1423	-48.828\\
1424	-50.049\\
1425	-43.945\\
1426	-59.814\\
1427	-46.387\\
1428	-36.621\\
1429	-51.27\\
1430	-69.58\\
1431	-54.932\\
1432	-41.504\\
1433	-37.842\\
1434	-41.504\\
1435	-32.959\\
1436	-30.518\\
1437	-41.504\\
1438	-47.607\\
1439	-52.49\\
1440	-42.725\\
1441	-53.711\\
1442	-51.27\\
1443	-32.959\\
1444	-32.959\\
1445	-58.594\\
1446	-80.566\\
1447	-78.125\\
1448	-67.139\\
1449	-62.256\\
1450	-50.049\\
1451	-48.828\\
1452	-41.504\\
1453	-54.932\\
1454	-50.049\\
1455	-32.959\\
1456	-45.166\\
1457	-52.49\\
1458	-61.035\\
1459	-68.359\\
1460	-68.359\\
1461	-87.891\\
1462	-107.422\\
1463	-119.629\\
1464	-80.566\\
1465	-46.387\\
1466	-32.959\\
1467	-24.414\\
1468	-34.18\\
1469	-29.297\\
1470	-51.27\\
1471	-48.828\\
1472	-40.283\\
1473	-52.49\\
1474	-63.477\\
1475	-70.801\\
1476	-75.684\\
1477	-103.76\\
1478	-92.773\\
1479	-72.021\\
1480	-51.27\\
1481	-52.49\\
1482	-54.932\\
1483	-65.918\\
1484	-51.27\\
1485	-50.049\\
1486	-48.828\\
1487	-30.518\\
1488	-45.166\\
1489	-68.359\\
1490	-48.828\\
1491	-52.49\\
1492	-86.67\\
1493	-63.477\\
1494	-63.477\\
1495	-85.449\\
1496	-81.787\\
1497	-53.711\\
1498	-35.4\\
1499	-36.621\\
1500	-61.035\\
};
\end{axis}
\end{tikzpicture}%
	\caption{The measured output for each foam sample is the force of the specimen response in kN.}
\end{figure}\label{fig:output}
\section{Structure identification}
\par The following model structure is assumed. The output of the NARX model $\mathbfit{y}(t)$ is the measured load. The input vector is composed as
\begin{equation}
	\mathbfit{x}(t) = \{x_i(t)\}^{d}_{i=1} = \left[\{y(t - k + 1)\}^{n_y}_{k=1} \quad \{u(t - k + n_y + 1)\}^{n_y + n_u}_{k= n_y + 1} \right]^{\top},
\end{equation}
where $n_u$ is the length of the input lag and $n_y$ is the length of the output lag in discrete time, and where $d = n_u + n_y$. In this case, the identification is performed under the following assumptions:
\begin{itemize}[noitemsep,topsep=0.3pt,parsep=0.3pt,partopsep=0.2pt,labelindent=1cm] 
	\item only the input signal affects the output ($n_y = 0$).
	\item the input signal has a lag of length $n_u = 4$.
\end{itemize}
The resultant input vector of the NARX model then takes the following form:
\begin{equation}
\mathbfit{x}(t) = \left[\begin{array}{cccc}
u(t-3) & u(t-2) & u(t-1) & u(t)
\end{array}\right]^{\top}.
\end{equation}
The unknown model is approximated with a sum of polynomial basis functions up to second degree ($\lambda = 2$), rendering the following structure
\begin{equation}
	\mathbfit{y}(t) = \theta^0 + \sum_{i=1}^{d} \theta_i x_i(t) + \sum_{i=1}^{d} \sum_{j=1}^{d} \theta_{i,j} x_i(t) x_j(t) + e(t).
\end{equation}
The number and order of significant terms are identified within the EFOR-CMSS algorithm based on the data from 8 out of 10 datasets. Figure \ref{fig:aamdl} illustrates the relationship between the number of model terms and the selected criterion of significance, AAMDL.
\begin{figure}[!h]
	\centering
	\definecolor{mycolor1}{rgb}{0.00000,0.44700,0.74100}%
	\definecolor{mycolor2}{rgb}{0.85000,0.32500,0.09800}%
	\subfloat[Sample size 2000.]{% This file was created by matlab2tikz.
% Minimal pgfplots version: 1.3
%
\definecolor{mycolor1}{rgb}{0.00000,0.44700,0.74100}%
\definecolor{mycolor2}{rgb}{0.85000,0.32500,0.09800}%
%
\begin{tikzpicture}

\begin{axis}[%
width=6cm,
height=6cm,
at={(0cm,0cm)},
scale only axis,
xmin=1,
xmax=15,
xlabel={Number of terms},
ymin=-2.21881347579609,
ymax=-1.56472398176224,
ylabel={AAMDL},
legend style={legend cell align=left,align=left,draw=white!15!black}
]
\addplot [color=mycolor1,only marks,mark=o,mark options={solid},forget plot]
  table[row sep=crcr]{%
1	-1.56472398176224\\
2	-1.97701270064064\\
3	-2.12029297016719\\
4	-2.18293755649416\\
5	-2.19697120568313\\
6	-2.19619682060115\\
7	-2.20287253171251\\
8	-2.21881347579609\\
9	-2.21849462617766\\
10	-2.21775809851537\\
11	-2.21624604641005\\
12	-2.21472746636226\\
13	-2.21467778403799\\
14	-2.2138532692636\\
15	-2.21310607093081\\
};
\addplot [color=mycolor2,line width=5.0pt,only marks,mark=asterisk,mark options={solid},forget plot]
  table[row sep=crcr]
	\subfloat[Sample size 4000.]{% This file was created by matlab2tikz.
% Minimal pgfplots version: 1.3
%
\definecolor{mycolor1}{rgb}{0.00000,0.44700,0.74100}%
\definecolor{mycolor2}{rgb}{0.85000,0.32500,0.09800}%
%
\begin{tikzpicture}

\begin{axis}[%
width=5.979382cm,
height=6cm,
at={(0cm,0cm)},
scale only axis,
xmin=1,
xmax=15,
xlabel={Number of terms},
ymin=-2.8,
ymax=-1.8,
ylabel={AAMDL},
legend style={legend cell align=left,align=left,draw=white!15!black}
]
\addplot [color=mycolor1,only marks,mark=o,mark options={solid},forget plot]
  table[row sep=crcr]{%
1	-1.85238639353984\\
2	-2.51661709688035\\
3	-2.54325120703526\\
4	-2.55128539901149\\
5	-2.55878900600603\\
6	-2.55830335858106\\
7	-2.65674313484709\\
8	-2.69442575885501\\
9	-2.69503767764109\\
10	-2.69462917143197\\
11	-2.69378515117799\\
12	-2.69252058304669\\
13	-2.69122725766994\\
14	-2.68930444709195\\
15	-2.68735035007295\\
};
\addplot [color=mycolor2,line width=5.0pt,only marks,mark=asterisk,mark options={solid},forget plot]
  table[row sep=crcr]
	\caption{AAMDL evolution with the growing number of terms for samples of different size. The optimal number of terms (\ref{tikz:nterms}) increases with the growing sample size.}\label{fig:aamdl}
\end{figure}	
\section{Parameter estimation}
\par The results of internal parameter estimation via the EFOR-CMSS method for sample sizes of 2000 and 4000 points are presented in Tables \ref{tab:thetas2000} and \ref{tab:thetas4000}, respectively. It can be seen that the first 7 significant terms are the same for both samples; moreover, the estimated values are of the same order in both cases. The further analysis deals only with the parameter estimates obtained from the smaller data sample. 
\begin{table}[!h]
	\centering
	\caption{Estimated parameters for the sample length 2000.}\label{tab:thetas2000}
	\small
	\begin{tabular}{rrrrrrrrrrr}
Step & Terms & C1 & C2 & C4 & C5 & C6 & C7 & C9 & C10 & AEER($\%$) \\ 
\hline 
1 & $x_4,x_4$ & -26.04 & -20.99 & -10.69 & -10.96 & -191.78 & -157.64 & -87.42 & -69.8 & 89.511 \\ 
2 & $x_3$ & 75.42 & 59.58 & 33 & 26.06 & 508.94 & 419.54 & 242.35 & 195.15 & 8.849 \\ 
3 & $x_1,x_4$ & 0.62 & 0.76 & 0.32 & 0.48 & 8.55 & 7.83 & 2.66 & 1.15 & 0.139 \\ 
4 & $x_1,x_1$ & 0.01 & -0.19 & -0.15 & -0.22 & 0.05 & -0.48 & 0.44 & 0.76 & 0.045 \\ 
5 & $x_2$ & 0.71 & -0.73 & -2.24 & -0.66 & 45.94 & 36.57 & 18.4 & 12.68 & 0.032 \\ 
6 & $x_4$ & -171.24 & -139.22 & -69.61 & -73.69 & -1273.02 & -1046.38 & -579.72 & -465.59 & 0.006 \\ 
7 & $c$ & -233.16 & -200.83 & -93.74 & -119.7 & -1805.9 & -1488.55 & -803.7 & -648.8 & 0.308 \\ 
8 & $x_3,x_4$ & 15.47 & 12.1 & 6.36 & 5.68 & 110.13 & 90.43 & 51.77 & 41.43 & 0.093 \\ 
\hline 
\end{tabular}
\end{table}
\begin{table}[!h]
	\centering
	\caption{Estimated parameters for the sample length 4000.}\label{tab:thetas4000}
	\small
	\begin{tabular}{llllllllllllll}
Step & Terms & C1 & C2 & C3 & C4 & C5 & C6 & C7 & C8 & C9 & C10 & AEER & AAMDL \\ 
\hline 
1 & $c$ & -165.57 & -140.83 & -106.8 & -57.88 & -93.13 & -1037.61 & -964.13 & -843.17 & -559.51 & -370.04 & 0 & -1.665 \\ 
2 & $x_4$ & -156.14 & -123.14 & -93.42 & -60.44 & -67.23 & -1089.72 & -931.89 & -801 & -530.57 & -398.02 & 0 & -2.209 \\ 
3 & $x_3$ & 77.52 & 56.15 & 39.44 & 33.75 & 22.39 & 466.78 & 421.82 & 365.29 & 244.52 & 171.74 & 0 & -2.389 \\ 
4 & $x_4,x_4$ & -19.62 & -12.42 & -9.15 & -4.16 & -7.3 & -103.98 & -95.74 & -83.16 & -55.36 & -35.81 & 0 & -2.477 \\ 
5 & $x_3,x_4$ & -0.61 & -10.91 & -10.1 & -11.57 & -7.03 & -126.87 & -80.74 & -60.96 & -39.74 & -50.79 & 0 & -2.499 \\ 
6 & $x_3,x_3$ & 16.49 & 22.71 & 16.85 & 13.74 & 12.56 & 259.29 & 184.73 & 152.18 & 99.4 & 95.7 & 0 & -2.499 \\ 
7 & $x_2$ & -19.39 & -14.83 & -7.99 & -13.43 & -5.4 & -54.34 & -71.89 & -59.62 & -35.85 & -9.18 & 0 & -2.507 \\ 
8 & $x_2,x_4$ & 1.31 & 5.3 & 5.13 & 4.35 & 5.21 & 55.56 & 39.19 & 29.86 & 20.47 & 21.91 & 0 & -2.506 \\ 
9 & $x_2,x_3$ & -16.39 & -22.92 & -15.24 & -9.28 & -13.26 & -288.73 & -197.82 & -165.46 & -106.81 & -103.02 & 0 & -2.504 \\ 
10 & $x_2,x_2$ & 5.77 & 7.59 & 4.45 & 1.49 & 3.72 & 108.29 & 70.29 & 60.34 & 38.62 & 38.72 & 0 & -2.506 \\ 
11 & $x_1$ & 25.77 & 21.68 & 15.22 & 13.42 & 10.92 & 220.93 & 169.96 & 139.84 & 85.56 & 72.68 & 0 & -2.526 \\ 
12 & $x_1,x_4$ & 5.03 & 4.03 & 2.82 & 2.28 & 1.83 & 45.48 & 35.21 & 28.83 & 17.78 & 14.85 & 0 & -2.527 \\ 
\hline 
\end{tabular}
\end{table}
\begin{figure}[!h]
	\centering
	% This file was created by matlab2tikz.
%
\definecolor{mycolor1}{rgb}{0.00000,0.44700,0.74100}%
%
\begin{tikzpicture}

\begin{axis}[%
width=2.306cm,
height=2.512cm,
at={(0cm,3.488cm)},
scale only axis,
xmin=0,
xmax=10,
xlabel style={font=\color{white!15!black}},
xlabel={Dataset index},
ymin=-1570.60720264055,
ymax=0,
ylabel style={font=\color{white!15!black}},
ylabel={$c$},
axis background/.style={fill=white},
legend style={legend cell align=left, align=left, draw=white!15!black}
]
\addplot [color=mycolor1, line width=2.0pt, draw=none, mark=o, mark options={solid, mycolor1}]
  table[row sep=crcr]{%
1	-234.539140897964\\
2	-204.620222005464\\
4	-69.3183215856622\\
5	-119.469410422321\\
6	-1570.60720264055\\
7	-1369.94256508144\\
9	-810.157687342949\\
10	-578.871880489588\\
};
\addlegendentry{data1}

\end{axis}

\begin{axis}[%
width=2.306cm,
height=2.512cm,
at={(3.035cm,3.488cm)},
scale only axis,
xmin=0,
xmax=10,
xlabel style={font=\color{white!15!black}},
xlabel={Dataset index},
ymin=-1500,
ymax=0,
ylabel style={font=\color{white!15!black}},
ylabel={$x_4$},
axis background/.style={fill=white},
legend style={legend cell align=left, align=left, draw=white!15!black}
]
\addplot [color=mycolor1, line width=2.0pt, draw=none, mark=o, mark options={solid, mycolor1}]
  table[row sep=crcr]{%
1	-177.027842821286\\
2	-144.091783081883\\
4	-68.5172968049575\\
5	-76.681359016238\\
6	-1355.31197379512\\
7	-1111.81339717774\\
9	-630.593824253468\\
10	-490.279895335194\\
};
\addlegendentry{data1}

\end{axis}

\begin{axis}[%
width=2.306cm,
height=2.512cm,
at={(6.07cm,3.488cm)},
scale only axis,
xmin=0,
xmax=10,
xlabel style={font=\color{white!15!black}},
xlabel={Dataset index},
ymin=0,
ymax=725.984005688338,
ylabel style={font=\color{white!15!black}},
ylabel={$x_3$},
axis background/.style={fill=white},
legend style={legend cell align=left, align=left, draw=white!15!black}
]
\addplot [color=mycolor1, line width=2.0pt, draw=none, mark=o, mark options={solid, mycolor1}]
  table[row sep=crcr]{%
1	84.0784415628057\\
2	64.2993799213961\\
4	40.1997153300434\\
5	28.8321342630344\\
6	725.984005688338\\
7	566.327570839676\\
9	311.636312916187\\
10	259.452196820681\\
};
\addlegendentry{data1}

\end{axis}

\begin{axis}[%
width=2.306cm,
height=2.512cm,
at={(9.105cm,3.488cm)},
scale only axis,
xmin=0,
xmax=10,
xlabel style={font=\color{white!15!black}},
xlabel={Dataset index},
ymin=-150.420523327553,
ymax=0,
ylabel style={font=\color{white!15!black}},
ylabel={$x_4,x_4$},
axis background/.style={fill=white},
legend style={legend cell align=left, align=left, draw=white!15!black}
]
\addplot [color=mycolor1, line width=2.0pt, draw=none, mark=o, mark options={solid, mycolor1}]
  table[row sep=crcr]{%
1	-23.3162164993923\\
2	-17.551126535841\\
4	-8.57889852771006\\
5	-11.1249029147396\\
6	-150.420523327553\\
7	-144.424085070836\\
9	-83.7837850877327\\
10	-56.0717545660385\\
};
\addlegendentry{data1}

\end{axis}

\begin{axis}[%
width=2.306cm,
height=2.512cm,
at={(0cm,0cm)},
scale only axis,
xmin=0,
xmax=10,
xlabel style={font=\color{white!15!black}},
xlabel={Dataset index},
ymin=0,
ymax=60,
ylabel style={font=\color{white!15!black}},
ylabel={$x_3,x_4$},
axis background/.style={fill=white},
legend style={legend cell align=left, align=left, draw=white!15!black}
]
\addplot [color=mycolor1, line width=2.0pt, draw=none, mark=o, mark options={solid, mycolor1}]
  table[row sep=crcr]{%
1	9.41071532819576\\
2	4.85648939202435\\
4	2.52525319022002\\
5	5.82144512126454\\
6	19.5828294495475\\
7	58.5359665260704\\
9	37.4478718322197\\
10	9.84958308987871\\
};
\addlegendentry{data1}

\end{axis}

\begin{axis}[%
width=2.306cm,
height=2.512cm,
at={(3.035cm,0cm)},
scale only axis,
xmin=0,
xmax=10,
xlabel style={font=\color{white!15!black}},
xlabel={Dataset index},
ymin=0,
ymax=63.8645034703178,
ylabel style={font=\color{white!15!black}},
ylabel={$x_3,x_3$},
axis background/.style={fill=white},
legend style={legend cell align=left, align=left, draw=white!15!black}
]
\addplot [color=mycolor1, line width=2.0pt, draw=none, mark=o, mark options={solid, mycolor1}]
  table[row sep=crcr]{%
1	3.84735837114033\\
2	4.18589479318199\\
4	2.5253492774925\\
5	0.202004646323669\\
6	63.8645034703178\\
7	28.8636620018487\\
9	13.1339134575995\\
10	21.4862675861994\\
};
\addlegendentry{data1}

\end{axis}

\begin{axis}[%
width=2.306cm,
height=2.512cm,
at={(6.07cm,0cm)},
scale only axis,
xmin=0,
xmax=10,
xlabel style={font=\color{white!15!black}},
xlabel={Dataset index},
ymin=-2.6962358835339,
ymax=43.652313591371,
ylabel style={font=\color{white!15!black}},
ylabel={$x_2$},
axis background/.style={fill=white},
legend style={legend cell align=left, align=left, draw=white!15!black}
]
\addplot [color=mycolor1, line width=2.0pt, draw=none, mark=o, mark options={solid, mycolor1}]
  table[row sep=crcr]{%
1	1.42162596296746\\
2	0.746168274194868\\
4	-2.6962358835339\\
5	-0.0502271623069443\\
6	43.652313591371\\
7	38.6573138130893\\
9	17.6606791673818\\
10	13.6793355027508\\
};
\addlegendentry{data1}

\end{axis}

\begin{axis}[%
width=2.306cm,
height=2.512cm,
at={(9.105cm,0cm)},
scale only axis,
xmin=0,
xmax=10,
xlabel style={font=\color{white!15!black}},
xlabel={Dataset index},
ymin=-54.5712613439143,
ymax=0.18652536446538,
ylabel style={font=\color{white!15!black}},
ylabel={$x_1$},
axis background/.style={fill=white},
legend style={legend cell align=left, align=left, draw=white!15!black}
]
\addplot [color=mycolor1, line width=2.0pt, draw=none, mark=o, mark options={solid, mycolor1}]
  table[row sep=crcr]{%
1	-4.315102766666\\
2	-2.89970791020909\\
4	0.18652536446538\\
5	-0.786268351112874\\
6	-54.5712613439143\\
7	-45.0319465291698\\
9	-21.7631864713255\\
10	-17.6985827852262\\
};
\addlegendentry{data1}

\end{axis}
\end{tikzpicture}%
	\caption{Estimated values of internal parameters.}
\end{figure}
\par In order to link the external and internal parameters, an arbitrary polynomial function of two arguments is formed
\begin{equation}
\theta_i(L_{cut},D_{rlx}) = \beta_0 + \beta_1 L_{cut} + \beta_2 D_{rlx} + \beta_3 L_{cut}^{2} + \beta_4 D_{rlx}^{2} + \beta_5 L_{cut} D_{rlx}, \qquad i=1,\dots,N_s,
\end{equation}
$L_{cut}$ and $D_{rlx}$ are the external parameters of manufacturing process, and where $N_s$ is the estimated number of the significant terms. The coefficients $B = \left[ \beta_0 \dots \beta_5 \right]$ are unknown and must be estimated from the internal parameter values available. The linear relationship between the batch of internal parameter values corresponding to each significant term $\Theta_i$ and the surface coefficients can be established in the following form:
\begin{equation}\label{eq:linrel}
\left[\begin{array}{c}
\theta_{i}^{1} \\
\vdots \\
\theta_{i}^{j} \\
\vdots \\
\theta_{i}^{N}
\end{array}\right] =
\left[\begin{array}{cccccc}
1 & L& D& L\times L& D\times D& L\times D
\end{array}\right] B,
\end{equation}
where $j = 1, \dots, N$ is the index of the dataset, and where $\times$ denotes the by-element product such that
\begin{equation*}
L\times L = \Big[ L_{cut}^{(j)} L_{cut}^{(j)}\Big]^{N}_{j=1}.
\end{equation*}
Given the equation \eqref{eq:linrel}, the vector of unknown coefficients $B$ can be identified via the classical least squares (LS) method.
Curve fitting results are presented in Table
\begin{table}[!h]
	\centering
	\caption{Estimated polynomial coefficients for the sample length 2000.}
	\small
	\begin{tabular}{lllllll}
Terms & $b_0$ & $b_1$ & $b_2$ & $b_3$ & $b_4$ & $b_5$ \\ 
\hline 
$c$ & -1322.084 & 20.004 & 14767.273 & 0.118 & -7870.234 & -303.531 \\ 
$x_4$ & -1075.382 & 24.835 & 6667.522 & -0.167 & -28913.028 & -45.153 \\ 
$x_3$ & 450.825 & -13.657 & -939.78 & 0.18 & 24429.196 & -66.177 \\ 
$x_4,x_4$ & 54.376 & -5.123 & 2240.158 & 0.093 & 1927.023 & -56.82 \\ 
$x_3,x_4$ & -362.541 & 16.453 & -3169.244 & -0.232 & -10119.363 & 106.874 \\ 
$x_3,x_3$ & 239.666 & -9.93 & 1441.955 & 0.135 & 7401.423 & -58.63 \\ 
$x_2$ & 116.545 & -3.045 & -440.422 & 0.017 & 702.364 & 6.7 \\ 
$x_1$ & -101.419 & 2.818 & 300.219 & -0.022 & -1527.744 & -0.581 \\ 
\hline 
\end{tabular}
\end{table}

\begin{figure}[!h]
	\centering
	% This file was created by matlab2tikz.
% Minimal pgfplots version: 1.3
%
\definecolor{mycolor1}{rgb}{0.00000,0.44700,0.74100}%
\definecolor{mycolor2}{rgb}{0.85000,0.32500,0.09800}%
%
\begin{tikzpicture}

\begin{axis}[%
width=4.527496cm,
height=3.870968cm,
at={(0cm,0cm)},
scale only axis,
xmin=56,
xmax=74,
tick align=outside,
xlabel={$L_{cut}$},
xmajorgrids,
ymin=0.093,
ymax=0.276,
ylabel={$D_{rlx}$},
ymajorgrids,
zmin=-2000,
zmax=0,
zlabel={$c$},
zmajorgrids,
view={-140}{50},
legend style={at={(1.03,1)},anchor=north west,legend cell align=left,align=left,draw=white!15!black}
]
\addplot3[only marks,mark=*,mark options={},mark size=1.5000pt,color=mycolor1] plot table[row sep=crcr,]{%
74	0.123	-233.157966898601\\
72	0.113	-200.830593420783\\
61	0.095	-93.7391747783605\\
56	0.093	-119.696332564413\\
};\label{tikz:thetas1}
\addplot3[only marks,mark=*,mark options={},mark size=1.5000pt,color=mycolor2] plot table[row sep=crcr,]{%
67	0.276	-1805.89673785913\\
66	0.255	-1488.55090215304\\
62	0.209	-803.703476355143\\
57	0.193	-648.796609601896\\
};\label{tikz:thetas2}
\addplot3[only marks,mark=*,mark options={},mark size=1.5000pt,color=black] plot table[row sep=crcr,]{%
69	0.104	-138.628962092727\\
};\label{tikz:thetaidentified}
\addplot3[only marks,mark=*,mark options={},mark size=1.5000pt,color=black] plot table[row sep=crcr,]{%
64	0.23	-1085.45181546038\\
};

\addplot3[%
surf,
opacity=0.7,
shader=interp,
colormap={mymap}{[1pt] rgb(0pt)=(0.0901961,0.239216,0.0745098); rgb(1pt)=(0.0945149,0.242058,0.0739522); rgb(2pt)=(0.0988592,0.244894,0.0733566); rgb(3pt)=(0.103229,0.247724,0.0727241); rgb(4pt)=(0.107623,0.250549,0.0720557); rgb(5pt)=(0.112043,0.253367,0.0713525); rgb(6pt)=(0.116487,0.25618,0.0706154); rgb(7pt)=(0.120956,0.258986,0.0698456); rgb(8pt)=(0.125449,0.261787,0.0690441); rgb(9pt)=(0.129967,0.264581,0.0682118); rgb(10pt)=(0.134508,0.26737,0.06735); rgb(11pt)=(0.139074,0.270152,0.0664596); rgb(12pt)=(0.143663,0.272929,0.0655416); rgb(13pt)=(0.148275,0.275699,0.0645971); rgb(14pt)=(0.152911,0.278463,0.0636271); rgb(15pt)=(0.15757,0.281221,0.0626328); rgb(16pt)=(0.162252,0.283973,0.0616151); rgb(17pt)=(0.166957,0.286719,0.060575); rgb(18pt)=(0.171685,0.289458,0.0595136); rgb(19pt)=(0.176434,0.292191,0.0584321); rgb(20pt)=(0.181207,0.294918,0.0573313); rgb(21pt)=(0.186001,0.297639,0.0562123); rgb(22pt)=(0.190817,0.300353,0.0550763); rgb(23pt)=(0.195655,0.303061,0.0539242); rgb(24pt)=(0.200514,0.305763,0.052757); rgb(25pt)=(0.205395,0.308459,0.0515759); rgb(26pt)=(0.210296,0.311149,0.0503624); rgb(27pt)=(0.215212,0.313846,0.0490067); rgb(28pt)=(0.220142,0.316548,0.0475043); rgb(29pt)=(0.22509,0.319254,0.0458704); rgb(30pt)=(0.230056,0.321962,0.0441205); rgb(31pt)=(0.235042,0.324671,0.04227); rgb(32pt)=(0.240048,0.327379,0.0403343); rgb(33pt)=(0.245078,0.330085,0.0383287); rgb(34pt)=(0.250131,0.332786,0.0362688); rgb(35pt)=(0.25521,0.335482,0.0341698); rgb(36pt)=(0.260317,0.33817,0.0320472); rgb(37pt)=(0.265451,0.340849,0.0299163); rgb(38pt)=(0.270616,0.343517,0.0277927); rgb(39pt)=(0.275813,0.346172,0.0256916); rgb(40pt)=(0.281043,0.348814,0.0236284); rgb(41pt)=(0.286307,0.35144,0.0216186); rgb(42pt)=(0.291607,0.354048,0.0196776); rgb(43pt)=(0.296945,0.356637,0.0178207); rgb(44pt)=(0.302322,0.359206,0.0160634); rgb(45pt)=(0.307739,0.361753,0.0144211); rgb(46pt)=(0.313198,0.364275,0.0129091); rgb(47pt)=(0.318701,0.366772,0.0115428); rgb(48pt)=(0.324249,0.369242,0.0103377); rgb(49pt)=(0.329843,0.371682,0.00930909); rgb(50pt)=(0.335485,0.374093,0.00847245); rgb(51pt)=(0.341176,0.376471,0.00784314); rgb(52pt)=(0.346925,0.378826,0.00732741); rgb(53pt)=(0.352735,0.381168,0.00682184); rgb(54pt)=(0.358605,0.383497,0.00632729); rgb(55pt)=(0.364532,0.385812,0.00584464); rgb(56pt)=(0.370516,0.388113,0.00537476); rgb(57pt)=(0.376552,0.390399,0.00491852); rgb(58pt)=(0.38264,0.39267,0.00447681); rgb(59pt)=(0.388777,0.394925,0.00405048); rgb(60pt)=(0.394962,0.397164,0.00364042); rgb(61pt)=(0.401191,0.399386,0.00324749); rgb(62pt)=(0.407464,0.401592,0.00287258); rgb(63pt)=(0.413777,0.40378,0.00251655); rgb(64pt)=(0.420129,0.40595,0.00218028); rgb(65pt)=(0.426518,0.408102,0.00186463); rgb(66pt)=(0.432942,0.410234,0.00157049); rgb(67pt)=(0.439399,0.412348,0.00129873); rgb(68pt)=(0.445885,0.414441,0.00105022); rgb(69pt)=(0.452401,0.416515,0.000825833); rgb(70pt)=(0.458942,0.418567,0.000626441); rgb(71pt)=(0.465508,0.420599,0.00045292); rgb(72pt)=(0.472096,0.422609,0.000306141); rgb(73pt)=(0.478704,0.424596,0.000186979); rgb(74pt)=(0.485331,0.426562,9.63073e-05); rgb(75pt)=(0.491973,0.428504,3.49981e-05); rgb(76pt)=(0.498628,0.430422,3.92506e-06); rgb(77pt)=(0.505323,0.432315,0); rgb(78pt)=(0.512206,0.434168,0); rgb(79pt)=(0.519282,0.435983,0); rgb(80pt)=(0.526529,0.437764,0); rgb(81pt)=(0.533922,0.439512,0); rgb(82pt)=(0.54144,0.441232,0); rgb(83pt)=(0.549059,0.442927,0); rgb(84pt)=(0.556756,0.444599,0); rgb(85pt)=(0.564508,0.446252,0); rgb(86pt)=(0.572292,0.447889,0); rgb(87pt)=(0.580084,0.449514,0); rgb(88pt)=(0.587863,0.451129,0); rgb(89pt)=(0.595604,0.452737,0); rgb(90pt)=(0.603284,0.454343,0); rgb(91pt)=(0.610882,0.455948,0); rgb(92pt)=(0.618373,0.457556,0); rgb(93pt)=(0.625734,0.459171,0); rgb(94pt)=(0.632943,0.460795,0); rgb(95pt)=(0.639976,0.462432,0); rgb(96pt)=(0.64681,0.464084,0); rgb(97pt)=(0.653423,0.465756,0); rgb(98pt)=(0.659791,0.46745,0); rgb(99pt)=(0.665891,0.469169,0); rgb(100pt)=(0.6717,0.470916,0); rgb(101pt)=(0.677195,0.472696,0); rgb(102pt)=(0.682353,0.47451,0); rgb(103pt)=(0.687242,0.476355,0); rgb(104pt)=(0.691952,0.478225,0); rgb(105pt)=(0.696497,0.480118,0); rgb(106pt)=(0.700887,0.482033,0); rgb(107pt)=(0.705134,0.483968,0); rgb(108pt)=(0.709251,0.485921,0); rgb(109pt)=(0.713249,0.487891,0); rgb(110pt)=(0.71714,0.489876,0); rgb(111pt)=(0.720936,0.491875,0); rgb(112pt)=(0.724649,0.493887,0); rgb(113pt)=(0.72829,0.495909,0); rgb(114pt)=(0.731872,0.49794,0); rgb(115pt)=(0.735406,0.499979,0); rgb(116pt)=(0.738904,0.502025,0); rgb(117pt)=(0.742378,0.504075,0); rgb(118pt)=(0.74584,0.506128,0); rgb(119pt)=(0.749302,0.508182,0); rgb(120pt)=(0.752775,0.510237,0); rgb(121pt)=(0.756272,0.51229,0); rgb(122pt)=(0.759804,0.514339,0); rgb(123pt)=(0.763384,0.516385,0); rgb(124pt)=(0.767022,0.518424,0); rgb(125pt)=(0.770731,0.520455,0); rgb(126pt)=(0.774523,0.522478,0); rgb(127pt)=(0.77841,0.524489,0); rgb(128pt)=(0.782391,0.526491,0); rgb(129pt)=(0.786402,0.528496,0); rgb(130pt)=(0.790431,0.530506,0); rgb(131pt)=(0.794478,0.532521,0); rgb(132pt)=(0.798541,0.534539,0); rgb(133pt)=(0.802619,0.53656,0); rgb(134pt)=(0.806712,0.538584,0); rgb(135pt)=(0.81082,0.540609,0); rgb(136pt)=(0.81494,0.542635,0); rgb(137pt)=(0.819074,0.54466,0); rgb(138pt)=(0.823219,0.546686,0); rgb(139pt)=(0.827374,0.548709,0); rgb(140pt)=(0.831541,0.55073,0); rgb(141pt)=(0.835716,0.552749,0); rgb(142pt)=(0.8399,0.554763,0); rgb(143pt)=(0.844092,0.556774,0); rgb(144pt)=(0.848292,0.558779,0); rgb(145pt)=(0.852497,0.560778,0); rgb(146pt)=(0.856708,0.562771,0); rgb(147pt)=(0.860924,0.564756,0); rgb(148pt)=(0.865143,0.566733,0); rgb(149pt)=(0.869366,0.568701,0); rgb(150pt)=(0.873592,0.57066,0); rgb(151pt)=(0.877819,0.572608,0); rgb(152pt)=(0.882047,0.574545,0); rgb(153pt)=(0.886275,0.576471,0); rgb(154pt)=(0.890659,0.578362,0); rgb(155pt)=(0.895333,0.580203,0); rgb(156pt)=(0.900258,0.581999,0); rgb(157pt)=(0.905397,0.583755,0); rgb(158pt)=(0.910711,0.585479,0); rgb(159pt)=(0.916164,0.587176,0); rgb(160pt)=(0.921717,0.588852,0); rgb(161pt)=(0.927333,0.590513,0); rgb(162pt)=(0.932974,0.592166,0); rgb(163pt)=(0.938602,0.593815,0); rgb(164pt)=(0.94418,0.595468,0); rgb(165pt)=(0.949669,0.59713,0); rgb(166pt)=(0.955033,0.598808,0); rgb(167pt)=(0.960233,0.600507,0); rgb(168pt)=(0.965232,0.602233,0); rgb(169pt)=(0.969992,0.603992,0); rgb(170pt)=(0.974475,0.605791,0); rgb(171pt)=(0.978643,0.607636,0); rgb(172pt)=(0.98246,0.609532,0); rgb(173pt)=(0.985886,0.611486,0); rgb(174pt)=(0.988885,0.613503,0); rgb(175pt)=(0.991419,0.61559,0); rgb(176pt)=(0.99345,0.617753,0); rgb(177pt)=(0.99494,0.619997,0); rgb(178pt)=(0.995851,0.622329,0); rgb(179pt)=(0.996226,0.624763,0); rgb(180pt)=(0.996512,0.627352,0); rgb(181pt)=(0.996788,0.630095,0); rgb(182pt)=(0.997053,0.632982,0); rgb(183pt)=(0.997308,0.636004,0); rgb(184pt)=(0.997552,0.639152,0); rgb(185pt)=(0.997785,0.642416,0); rgb(186pt)=(0.998006,0.645786,0); rgb(187pt)=(0.998217,0.649253,0); rgb(188pt)=(0.998416,0.652807,0); rgb(189pt)=(0.998605,0.656439,0); rgb(190pt)=(0.998781,0.660138,0); rgb(191pt)=(0.998946,0.663897,0); rgb(192pt)=(0.9991,0.667704,0); rgb(193pt)=(0.999242,0.67155,0); rgb(194pt)=(0.999372,0.675427,0); rgb(195pt)=(0.99949,0.679323,0); rgb(196pt)=(0.999596,0.68323,0); rgb(197pt)=(0.99969,0.687139,0); rgb(198pt)=(0.999771,0.691039,0); rgb(199pt)=(0.999841,0.694921,0); rgb(200pt)=(0.999898,0.698775,0); rgb(201pt)=(0.999942,0.702592,0); rgb(202pt)=(0.999974,0.706363,0); rgb(203pt)=(0.999994,0.710077,0); rgb(204pt)=(1,0.713725,0); rgb(205pt)=(1,0.717341,0); rgb(206pt)=(1,0.720963,0); rgb(207pt)=(1,0.724591,0); rgb(208pt)=(1,0.728226,0); rgb(209pt)=(1,0.731867,0); rgb(210pt)=(1,0.735514,0); rgb(211pt)=(1,0.739167,0); rgb(212pt)=(1,0.742827,0); rgb(213pt)=(1,0.746493,0); rgb(214pt)=(1,0.750165,0); rgb(215pt)=(1,0.753843,0); rgb(216pt)=(1,0.757527,0); rgb(217pt)=(1,0.761217,0); rgb(218pt)=(1,0.764913,0); rgb(219pt)=(1,0.768615,0); rgb(220pt)=(1,0.772324,0); rgb(221pt)=(1,0.776038,0); rgb(222pt)=(1,0.779758,0); rgb(223pt)=(1,0.783484,0); rgb(224pt)=(1,0.787215,0); rgb(225pt)=(1,0.790953,0); rgb(226pt)=(1,0.794696,0); rgb(227pt)=(1,0.798445,0); rgb(228pt)=(1,0.8022,0); rgb(229pt)=(1,0.805961,0); rgb(230pt)=(1,0.809727,0); rgb(231pt)=(1,0.8135,0); rgb(232pt)=(1,0.817278,0); rgb(233pt)=(1,0.821063,0); rgb(234pt)=(1,0.824854,0); rgb(235pt)=(1,0.828652,0); rgb(236pt)=(1,0.832455,0); rgb(237pt)=(1,0.836265,0); rgb(238pt)=(1,0.840081,0); rgb(239pt)=(1,0.843903,0); rgb(240pt)=(1,0.847732,0); rgb(241pt)=(1,0.851566,0); rgb(242pt)=(1,0.855406,0); rgb(243pt)=(1,0.859253,0); rgb(244pt)=(1,0.863106,0); rgb(245pt)=(1,0.866964,0); rgb(246pt)=(1,0.870829,0); rgb(247pt)=(1,0.8747,0); rgb(248pt)=(1,0.878577,0); rgb(249pt)=(1,0.88246,0); rgb(250pt)=(1,0.886349,0); rgb(251pt)=(1,0.890243,0); rgb(252pt)=(1,0.894144,0); rgb(253pt)=(1,0.898051,0); rgb(254pt)=(1,0.901964,0); rgb(255pt)=(1,0.905882,0)},
mesh/rows=49]
table[row sep=crcr,header=false] {%
%
56	0.093	-115.68341116986\\
56	0.09666	-119.178384949931\\
56	0.10032	-123.892976330784\\
56	0.10398	-129.827185312418\\
56	0.10764	-136.981011894832\\
56	0.1113	-145.354456078027\\
56	0.11496	-154.947517862003\\
56	0.11862	-165.76019724676\\
56	0.12228	-177.792494232298\\
56	0.12594	-191.044408818617\\
56	0.1296	-205.515941005717\\
56	0.13326	-221.207090793597\\
56	0.13692	-238.117858182259\\
56	0.14058	-256.248243171702\\
56	0.14424	-275.598245761925\\
56	0.1479	-296.16786595293\\
56	0.15156	-317.957103744714\\
56	0.15522	-340.965959137281\\
56	0.15888	-365.194432130628\\
56	0.16254	-390.642522724757\\
56	0.1662	-417.310230919665\\
56	0.16986	-445.197556715355\\
56	0.17352	-474.304500111826\\
56	0.17718	-504.631061109078\\
56	0.18084	-536.177239707109\\
56	0.1845	-568.943035905924\\
56	0.18816	-602.928449705518\\
56	0.19182	-638.133481105893\\
56	0.19548	-674.558130107049\\
56	0.19914	-712.202396708986\\
56	0.2028	-751.066280911705\\
56	0.20646	-791.149782715203\\
56	0.21012	-832.452902119484\\
56	0.21378	-874.975639124543\\
56	0.21744	-918.717993730386\\
56	0.2211	-963.679965937008\\
56	0.22476	-1009.86155574441\\
56	0.22842	-1057.2627631526\\
56	0.23208	-1105.88358816156\\
56	0.23574	-1155.72403077131\\
56	0.2394	-1206.78409098184\\
56	0.24306	-1259.06376879314\\
56	0.24672	-1312.56306420523\\
56	0.25038	-1367.2819772181\\
56	0.25404	-1423.22050783175\\
56	0.2577	-1480.37865604619\\
56	0.26136	-1538.7564218614\\
56	0.26502	-1598.35380527739\\
56	0.26868	-1659.17080629417\\
56	0.27234	-1721.20742491172\\
56	0.276	-1784.46366113006\\
56.375	0.093	-113.514447153983\\
56.375	0.09666	-117.038923411893\\
56.375	0.10032	-121.783017270585\\
56.375	0.10398	-127.746728730057\\
56.375	0.10764	-134.93005779031\\
56.375	0.1113	-143.333004451344\\
56.375	0.11496	-152.955568713159\\
56.375	0.11862	-163.797750575755\\
56.375	0.12228	-175.859550039132\\
56.375	0.12594	-189.140967103289\\
56.375	0.1296	-203.642001768228\\
56.375	0.13326	-219.362654033947\\
56.375	0.13692	-236.302923900447\\
56.375	0.14058	-254.462811367729\\
56.375	0.14424	-273.842316435791\\
56.375	0.1479	-294.441439104634\\
56.375	0.15156	-316.260179374258\\
56.375	0.15522	-339.298537244663\\
56.375	0.15888	-363.556512715849\\
56.375	0.16254	-389.034105787816\\
56.375	0.1662	-415.731316460563\\
56.375	0.16986	-443.648144734093\\
56.375	0.17352	-472.784590608401\\
56.375	0.17718	-503.140654083492\\
56.375	0.18084	-534.716335159363\\
56.375	0.1845	-567.511633836016\\
56.375	0.18816	-601.526550113449\\
56.375	0.19182	-636.761083991663\\
56.375	0.19548	-673.215235470658\\
56.375	0.19914	-710.889004550434\\
56.375	0.2028	-749.782391230991\\
56.375	0.20646	-789.895395512328\\
56.375	0.21012	-831.228017394447\\
56.375	0.21378	-873.780256877346\\
56.375	0.21744	-917.552113961027\\
56.375	0.2211	-962.543588645488\\
56.375	0.22476	-1008.75468093073\\
56.375	0.22842	-1056.18539081675\\
56.375	0.23208	-1104.83571830356\\
56.375	0.23574	-1154.70566339114\\
56.375	0.2394	-1205.79522607951\\
56.375	0.24306	-1258.10440636866\\
56.375	0.24672	-1311.63320425858\\
56.375	0.25038	-1366.38161974929\\
56.375	0.25404	-1422.34965284078\\
56.375	0.2577	-1479.53730353305\\
56.375	0.26136	-1537.9445718261\\
56.375	0.26502	-1597.57145771994\\
56.375	0.26868	-1658.41796121455\\
56.375	0.27234	-1720.48408230994\\
56.375	0.276	-1783.76982100612\\
56.75	0.093	-111.484086745226\\
56.75	0.09666	-115.038065480975\\
56.75	0.10032	-119.811661817505\\
56.75	0.10398	-125.804875754816\\
56.75	0.10764	-133.017707292908\\
56.75	0.1113	-141.45015643178\\
56.75	0.11496	-151.102223171434\\
56.75	0.11862	-161.973907511869\\
56.75	0.12228	-174.065209453084\\
56.75	0.12594	-187.376128995081\\
56.75	0.1296	-201.906666137858\\
56.75	0.13326	-217.656820881416\\
56.75	0.13692	-234.626593225755\\
56.75	0.14058	-252.815983170875\\
56.75	0.14424	-272.224990716777\\
56.75	0.1479	-292.853615863458\\
56.75	0.15156	-314.701858610921\\
56.75	0.15522	-337.769718959164\\
56.75	0.15888	-362.05719690819\\
56.75	0.16254	-387.564292457995\\
56.75	0.1662	-414.291005608582\\
56.75	0.16986	-442.237336359949\\
56.75	0.17352	-471.403284712098\\
56.75	0.17718	-501.788850665026\\
56.75	0.18084	-533.394034218737\\
56.75	0.1845	-566.218835373227\\
56.75	0.18816	-600.2632541285\\
56.75	0.19182	-635.527290484552\\
56.75	0.19548	-672.010944441386\\
56.75	0.19914	-709.714215999001\\
56.75	0.2028	-748.637105157396\\
56.75	0.20646	-788.779611916573\\
56.75	0.21012	-830.14173627653\\
56.75	0.21378	-872.723478237269\\
56.75	0.21744	-916.524837798787\\
56.75	0.2211	-961.545814961088\\
56.75	0.22476	-1007.78640972417\\
56.75	0.22842	-1055.24662208803\\
56.75	0.23208	-1103.92645205267\\
56.75	0.23574	-1153.8258996181\\
56.75	0.2394	-1204.9449647843\\
56.75	0.24306	-1257.28364755129\\
56.75	0.24672	-1310.84194791905\\
56.75	0.25038	-1365.6198658876\\
56.75	0.25404	-1421.61740145693\\
56.75	0.2577	-1478.83455462704\\
56.75	0.26136	-1537.27132539793\\
56.75	0.26502	-1596.9277137696\\
56.75	0.26868	-1657.80371974205\\
56.75	0.27234	-1719.89934331529\\
56.75	0.276	-1783.2145844893\\
57.125	0.093	-109.592329943587\\
57.125	0.09666	-113.175811157175\\
57.125	0.10032	-117.978909971543\\
57.125	0.10398	-124.001626386693\\
57.125	0.10764	-131.243960402624\\
57.125	0.1113	-139.705912019335\\
57.125	0.11496	-149.387481236828\\
57.125	0.11862	-160.288668055101\\
57.125	0.12228	-172.409472474155\\
57.125	0.12594	-185.749894493991\\
57.125	0.1296	-200.309934114607\\
57.125	0.13326	-216.089591336004\\
57.125	0.13692	-233.088866158182\\
57.125	0.14058	-251.30775858114\\
57.125	0.14424	-270.746268604881\\
57.125	0.1479	-291.404396229401\\
57.125	0.15156	-313.282141454703\\
57.125	0.15522	-336.379504280785\\
57.125	0.15888	-360.696484707648\\
57.125	0.16254	-386.233082735293\\
57.125	0.1662	-412.989298363718\\
57.125	0.16986	-440.965131592924\\
57.125	0.17352	-470.160582422912\\
57.125	0.17718	-500.575650853679\\
57.125	0.18084	-532.210336885228\\
57.125	0.1845	-565.064640517558\\
57.125	0.18816	-599.138561750668\\
57.125	0.19182	-634.43210058456\\
57.125	0.19548	-670.945257019233\\
57.125	0.19914	-708.678031054686\\
57.125	0.2028	-747.630422690921\\
57.125	0.20646	-787.802431927936\\
57.125	0.21012	-829.194058765732\\
57.125	0.21378	-871.805303204309\\
57.125	0.21744	-915.636165243666\\
57.125	0.2211	-960.686644883806\\
57.125	0.22476	-1006.95674212473\\
57.125	0.22842	-1054.44645696643\\
57.125	0.23208	-1103.15578940891\\
57.125	0.23574	-1153.08473945217\\
57.125	0.2394	-1204.23330709621\\
57.125	0.24306	-1256.60149234104\\
57.125	0.24672	-1310.18929518664\\
57.125	0.25038	-1364.99671563303\\
57.125	0.25404	-1421.0237536802\\
57.125	0.2577	-1478.27040932815\\
57.125	0.26136	-1536.73668257687\\
57.125	0.26502	-1596.42257342638\\
57.125	0.26868	-1657.32808187668\\
57.125	0.27234	-1719.45320792775\\
57.125	0.276	-1782.7979515796\\
57.5	0.093	-107.839176749067\\
57.5	0.09666	-111.452160440494\\
57.5	0.10032	-116.284761732701\\
57.5	0.10398	-122.33698062569\\
57.5	0.10764	-129.608817119459\\
57.5	0.1113	-138.10027121401\\
57.5	0.11496	-147.811342909341\\
57.5	0.11862	-158.742032205453\\
57.5	0.12228	-170.892339102346\\
57.5	0.12594	-184.26226360002\\
57.5	0.1296	-198.851805698475\\
57.5	0.13326	-214.66096539771\\
57.5	0.13692	-231.689742697727\\
57.5	0.14058	-249.938137598525\\
57.5	0.14424	-269.406150100104\\
57.5	0.1479	-290.093780202463\\
57.5	0.15156	-312.001027905603\\
57.5	0.15522	-335.127893209524\\
57.5	0.15888	-359.474376114227\\
57.5	0.16254	-385.04047661971\\
57.5	0.1662	-411.826194725974\\
57.5	0.16986	-439.831530433019\\
57.5	0.17352	-469.056483740845\\
57.5	0.17718	-499.501054649451\\
57.5	0.18084	-531.165243158839\\
57.5	0.1845	-564.049049269008\\
57.5	0.18816	-598.152472979957\\
57.5	0.19182	-633.475514291687\\
57.5	0.19548	-670.018173204199\\
57.5	0.19914	-707.780449717491\\
57.5	0.2028	-746.762343831564\\
57.5	0.20646	-786.963855546418\\
57.5	0.21012	-828.384984862053\\
57.5	0.21378	-871.025731778469\\
57.5	0.21744	-914.886096295665\\
57.5	0.2211	-959.966078413643\\
57.5	0.22476	-1006.2656781324\\
57.5	0.22842	-1053.78489545194\\
57.5	0.23208	-1102.52373037226\\
57.5	0.23574	-1152.48218289336\\
57.5	0.2394	-1203.66025301525\\
57.5	0.24306	-1256.05794073791\\
57.5	0.24672	-1309.67524606135\\
57.5	0.25038	-1364.51216898558\\
57.5	0.25404	-1420.56870951058\\
57.5	0.2577	-1477.84486763637\\
57.5	0.26136	-1536.34064336294\\
57.5	0.26502	-1596.05603669029\\
57.5	0.26868	-1656.99104761842\\
57.5	0.27234	-1719.14567614733\\
57.5	0.276	-1782.51992227702\\
57.875	0.093	-106.224627161667\\
57.875	0.09666	-109.867113330933\\
57.875	0.10032	-114.729217100979\\
57.875	0.10398	-120.810938471806\\
57.875	0.10764	-128.112277443414\\
57.875	0.1113	-136.633234015804\\
57.875	0.11496	-146.373808188973\\
57.875	0.11862	-157.333999962924\\
57.875	0.12228	-169.513809337656\\
57.875	0.12594	-182.913236313169\\
57.875	0.1296	-197.532280889462\\
57.875	0.13326	-213.370943066537\\
57.875	0.13692	-230.429222844393\\
57.875	0.14058	-248.707120223029\\
57.875	0.14424	-268.204635202446\\
57.875	0.1479	-288.921767782645\\
57.875	0.15156	-310.858517963623\\
57.875	0.15522	-334.014885745384\\
57.875	0.15888	-358.390871127924\\
57.875	0.16254	-383.986474111247\\
57.875	0.1662	-410.801694695349\\
57.875	0.16986	-438.836532880234\\
57.875	0.17352	-468.090988665897\\
57.875	0.17718	-498.565062052343\\
57.875	0.18084	-530.258753039569\\
57.875	0.1845	-563.172061627577\\
57.875	0.18816	-597.304987816365\\
57.875	0.19182	-632.657531605935\\
57.875	0.19548	-669.229692996284\\
57.875	0.19914	-707.021471987415\\
57.875	0.2028	-746.032868579328\\
57.875	0.20646	-786.26388277202\\
57.875	0.21012	-827.714514565494\\
57.875	0.21378	-870.384763959748\\
57.875	0.21744	-914.274630954784\\
57.875	0.2211	-959.3841155506\\
57.875	0.22476	-1005.7132177472\\
57.875	0.22842	-1053.26193754458\\
57.875	0.23208	-1102.03027494274\\
57.875	0.23574	-1152.01822994168\\
57.875	0.2394	-1203.2258025414\\
57.875	0.24306	-1255.6529927419\\
57.875	0.24672	-1309.29980054318\\
57.875	0.25038	-1364.16622594525\\
57.875	0.25404	-1420.25226894809\\
57.875	0.2577	-1477.55792955172\\
57.875	0.26136	-1536.08320775612\\
57.875	0.26502	-1595.82810356131\\
57.875	0.26868	-1656.79261696728\\
57.875	0.27234	-1718.97674797403\\
57.875	0.276	-1782.38049658156\\
58.25	0.093	-104.748681181386\\
58.25	0.09666	-108.42066982849\\
58.25	0.10032	-113.312276076375\\
58.25	0.10398	-119.423499925041\\
58.25	0.10764	-126.754341374488\\
58.25	0.1113	-135.304800424716\\
58.25	0.11496	-145.074877075725\\
58.25	0.11862	-156.064571327514\\
58.25	0.12228	-168.273883180085\\
58.25	0.12594	-181.702812633437\\
58.25	0.1296	-196.351359687569\\
58.25	0.13326	-212.219524342482\\
58.25	0.13692	-229.307306598176\\
58.25	0.14058	-247.614706454652\\
58.25	0.14424	-267.141723911908\\
58.25	0.1479	-287.888358969945\\
58.25	0.15156	-309.854611628762\\
58.25	0.15522	-333.040481888362\\
58.25	0.15888	-357.445969748741\\
58.25	0.16254	-383.071075209902\\
58.25	0.1662	-409.915798271843\\
58.25	0.16986	-437.980138934566\\
58.25	0.17352	-467.264097198069\\
58.25	0.17718	-497.767673062354\\
58.25	0.18084	-529.490866527418\\
58.25	0.1845	-562.433677593265\\
58.25	0.18816	-596.596106259892\\
58.25	0.19182	-631.9781525273\\
58.25	0.19548	-668.579816395489\\
58.25	0.19914	-706.401097864459\\
58.25	0.2028	-745.44199693421\\
58.25	0.20646	-785.702513604741\\
58.25	0.21012	-827.182647876054\\
58.25	0.21378	-869.882399748146\\
58.25	0.21744	-913.801769221021\\
58.25	0.2211	-958.940756294676\\
58.25	0.22476	-1005.29936096911\\
58.25	0.22842	-1052.87758324433\\
58.25	0.23208	-1101.67542312033\\
58.25	0.23574	-1151.69288059711\\
58.25	0.2394	-1202.92995567467\\
58.25	0.24306	-1255.38664835301\\
58.25	0.24672	-1309.06295863213\\
58.25	0.25038	-1363.95888651203\\
58.25	0.25404	-1420.07443199271\\
58.25	0.2577	-1477.40959507418\\
58.25	0.26136	-1535.96437575642\\
58.25	0.26502	-1595.73877403945\\
58.25	0.26868	-1656.73278992326\\
58.25	0.27234	-1718.94642340785\\
58.25	0.276	-1782.37967449321\\
58.625	0.093	-103.411338808224\\
58.625	0.09666	-107.112829933167\\
58.625	0.10032	-112.033938658891\\
58.625	0.10398	-118.174664985396\\
58.625	0.10764	-125.535008912681\\
58.625	0.1113	-134.114970440748\\
58.625	0.11496	-143.914549569595\\
58.625	0.11862	-154.933746299224\\
58.625	0.12228	-167.172560629633\\
58.625	0.12594	-180.630992560824\\
58.625	0.1296	-195.309042092795\\
58.625	0.13326	-211.206709225547\\
58.625	0.13692	-228.32399395908\\
58.625	0.14058	-246.660896293393\\
58.625	0.14424	-266.217416228489\\
58.625	0.1479	-286.993553764364\\
58.625	0.15156	-308.989308901021\\
58.625	0.15522	-332.204681638459\\
58.625	0.15888	-356.639671976677\\
58.625	0.16254	-382.294279915676\\
58.625	0.1662	-409.168505455457\\
58.625	0.16986	-437.262348596018\\
58.625	0.17352	-466.575809337361\\
58.625	0.17718	-497.108887679483\\
58.625	0.18084	-528.861583622388\\
58.625	0.1845	-561.833897166072\\
58.625	0.18816	-596.025828310538\\
58.625	0.19182	-631.437377055785\\
58.625	0.19548	-668.068543401812\\
58.625	0.19914	-705.919327348621\\
58.625	0.2028	-744.98972889621\\
58.625	0.20646	-785.279748044581\\
58.625	0.21012	-826.789384793732\\
58.625	0.21378	-869.518639143664\\
58.625	0.21744	-913.467511094377\\
58.625	0.2211	-958.636000645871\\
58.625	0.22476	-1005.02410779815\\
58.625	0.22842	-1052.6318325512\\
58.625	0.23208	-1101.45917490504\\
58.625	0.23574	-1151.50613485966\\
58.625	0.2394	-1202.77271241506\\
58.625	0.24306	-1255.25890757123\\
58.625	0.24672	-1308.96472032819\\
58.625	0.25038	-1363.89015068594\\
58.625	0.25404	-1420.03519864446\\
58.625	0.2577	-1477.39986420376\\
58.625	0.26136	-1535.98414736385\\
58.625	0.26502	-1595.78804812471\\
58.625	0.26868	-1656.81156648636\\
58.625	0.27234	-1719.05470244878\\
58.625	0.276	-1782.51745601199\\
59	0.093	-102.212600042181\\
59	0.09666	-105.943593644963\\
59	0.10032	-110.894204848525\\
59	0.10398	-117.064433652869\\
59	0.10764	-124.454280057993\\
59	0.1113	-133.063744063899\\
59	0.11496	-142.892825670585\\
59	0.11862	-153.941524878052\\
59	0.12228	-166.2098416863\\
59	0.12594	-179.697776095329\\
59	0.1296	-194.405328105139\\
59	0.13326	-210.33249771573\\
59	0.13692	-227.479284927102\\
59	0.14058	-245.845689739254\\
59	0.14424	-265.431712152189\\
59	0.1479	-286.237352165903\\
59	0.15156	-308.262609780399\\
59	0.15522	-331.507484995674\\
59	0.15888	-355.971977811732\\
59	0.16254	-381.65608822857\\
59	0.1662	-408.559816246189\\
59	0.16986	-436.683161864589\\
59	0.17352	-466.02612508377\\
59	0.17718	-496.588705903732\\
59	0.18084	-528.370904324475\\
59	0.1845	-561.372720345998\\
59	0.18816	-595.594153968303\\
59	0.19182	-631.035205191389\\
59	0.19548	-667.695874015255\\
59	0.19914	-705.576160439902\\
59	0.2028	-744.67606446533\\
59	0.20646	-784.99558609154\\
59	0.21012	-826.534725318529\\
59	0.21378	-869.293482146301\\
59	0.21744	-913.271856574852\\
59	0.2211	-958.469848604185\\
59	0.22476	-1004.8874582343\\
59	0.22842	-1052.52468546519\\
59	0.23208	-1101.38153029687\\
59	0.23574	-1151.45799272933\\
59	0.2394	-1202.75407276256\\
59	0.24306	-1255.26977039658\\
59	0.24672	-1309.00508563138\\
59	0.25038	-1363.96001846696\\
59	0.25404	-1420.13456890332\\
59	0.2577	-1477.52873694046\\
59	0.26136	-1536.14252257839\\
59	0.26502	-1595.97592581709\\
59	0.26868	-1657.02894665657\\
59	0.27234	-1719.30158509684\\
59	0.276	-1782.79384113789\\
59.375	0.093	-101.152464883257\\
59.375	0.09666	-104.912960963878\\
59.375	0.10032	-109.893074645279\\
59.375	0.10398	-116.092805927461\\
59.375	0.10764	-123.512154810424\\
59.375	0.1113	-132.151121294169\\
59.375	0.11496	-142.009705378694\\
59.375	0.11862	-153.087907064\\
59.375	0.12228	-165.385726350087\\
59.375	0.12594	-178.903163236955\\
59.375	0.1296	-193.640217724603\\
59.375	0.13326	-209.596889813033\\
59.375	0.13692	-226.773179502243\\
59.375	0.14058	-245.169086792235\\
59.375	0.14424	-264.784611683007\\
59.375	0.1479	-285.619754174561\\
59.375	0.15156	-307.674514266894\\
59.375	0.15522	-330.94889196001\\
59.375	0.15888	-355.442887253906\\
59.375	0.16254	-381.156500148584\\
59.375	0.1662	-408.089730644041\\
59.375	0.16986	-436.24257874028\\
59.375	0.17352	-465.615044437299\\
59.375	0.17718	-496.207127735101\\
59.375	0.18084	-528.018828633681\\
59.375	0.1845	-561.050147133044\\
59.375	0.18816	-595.301083233187\\
59.375	0.19182	-630.771636934112\\
59.375	0.19548	-667.461808235817\\
59.375	0.19914	-705.371597138303\\
59.375	0.2028	-744.50100364157\\
59.375	0.20646	-784.850027745618\\
59.375	0.21012	-826.418669450447\\
59.375	0.21378	-869.206928756056\\
59.375	0.21744	-913.214805662447\\
59.375	0.2211	-958.442300169618\\
59.375	0.22476	-1004.88941227757\\
59.375	0.22842	-1052.5561419863\\
59.375	0.23208	-1101.44248929582\\
59.375	0.23574	-1151.54845420611\\
59.375	0.2394	-1202.87403671719\\
59.375	0.24306	-1255.41923682905\\
59.375	0.24672	-1309.18405454168\\
59.375	0.25038	-1364.1684898551\\
59.375	0.25404	-1420.3725427693\\
59.375	0.2577	-1477.79621328428\\
59.375	0.26136	-1536.43950140005\\
59.375	0.26502	-1596.30240711659\\
59.375	0.26868	-1657.38493043391\\
59.375	0.27234	-1719.68707135202\\
59.375	0.276	-1783.2088298709\\
59.75	0.093	-100.230933331452\\
59.75	0.09666	-104.020931889911\\
59.75	0.10032	-109.030548049152\\
59.75	0.10398	-115.259781809173\\
59.75	0.10764	-122.708633169975\\
59.75	0.1113	-131.377102131558\\
59.75	0.11496	-141.265188693922\\
59.75	0.11862	-152.372892857067\\
59.75	0.12228	-164.700214620992\\
59.75	0.12594	-178.247153985699\\
59.75	0.1296	-193.013710951186\\
59.75	0.13326	-208.999885517455\\
59.75	0.13692	-226.205677684504\\
59.75	0.14058	-244.631087452335\\
59.75	0.14424	-264.276114820945\\
59.75	0.1479	-285.140759790338\\
59.75	0.15156	-307.22502236051\\
59.75	0.15522	-330.528902531465\\
59.75	0.15888	-355.052400303199\\
59.75	0.16254	-380.795515675716\\
59.75	0.1662	-407.758248649012\\
59.75	0.16986	-435.94059922309\\
59.75	0.17352	-465.342567397948\\
59.75	0.17718	-495.964153173587\\
59.75	0.18084	-527.805356550007\\
59.75	0.1845	-560.866177527209\\
59.75	0.18816	-595.146616105191\\
59.75	0.19182	-630.646672283954\\
59.75	0.19548	-667.366346063498\\
59.75	0.19914	-705.305637443822\\
59.75	0.2028	-744.464546424929\\
59.75	0.20646	-784.843073006815\\
59.75	0.21012	-826.441217189483\\
59.75	0.21378	-869.258978972931\\
59.75	0.21744	-913.296358357161\\
59.75	0.2211	-958.55335534217\\
59.75	0.22476	-1005.02996992796\\
59.75	0.22842	-1052.72620211453\\
59.75	0.23208	-1101.64205190189\\
59.75	0.23574	-1151.77751929002\\
59.75	0.2394	-1203.13260427894\\
59.75	0.24306	-1255.70730686863\\
59.75	0.24672	-1309.50162705911\\
59.75	0.25038	-1364.51556485037\\
59.75	0.25404	-1420.74912024241\\
59.75	0.2577	-1478.20229323522\\
59.75	0.26136	-1536.87508382882\\
59.75	0.26502	-1596.76749202321\\
59.75	0.26868	-1657.87951781837\\
59.75	0.27234	-1720.21116121431\\
59.75	0.276	-1783.76242221104\\
60.125	0.093	-99.448005386767\\
60.125	0.09666	-103.267506423065\\
60.125	0.10032	-108.306625060144\\
60.125	0.10398	-114.565361298004\\
60.125	0.10764	-122.043715136645\\
60.125	0.1113	-130.741686576066\\
60.125	0.11496	-140.659275616269\\
60.125	0.11862	-151.796482257253\\
60.125	0.12228	-164.153306499017\\
60.125	0.12594	-177.729748341563\\
60.125	0.1296	-192.525807784889\\
60.125	0.13326	-208.541484828996\\
60.125	0.13692	-225.776779473884\\
60.125	0.14058	-244.231691719553\\
60.125	0.14424	-263.906221566003\\
60.125	0.1479	-284.800369013234\\
60.125	0.15156	-306.914134061246\\
60.125	0.15522	-330.247516710038\\
60.125	0.15888	-354.800516959612\\
60.125	0.16254	-380.573134809966\\
60.125	0.1662	-407.565370261102\\
60.125	0.16986	-435.777223313018\\
60.125	0.17352	-465.208693965716\\
60.125	0.17718	-495.859782219193\\
60.125	0.18084	-527.730488073453\\
60.125	0.1845	-560.820811528492\\
60.125	0.18816	-595.130752584313\\
60.125	0.19182	-630.660311240915\\
60.125	0.19548	-667.409487498298\\
60.125	0.19914	-705.378281356461\\
60.125	0.2028	-744.566692815406\\
60.125	0.20646	-784.974721875132\\
60.125	0.21012	-826.602368535638\\
60.125	0.21378	-869.449632796925\\
60.125	0.21744	-913.516514658993\\
60.125	0.2211	-958.803014121842\\
60.125	0.22476	-1005.30913118547\\
60.125	0.22842	-1053.03486584988\\
60.125	0.23208	-1101.98021811507\\
60.125	0.23574	-1152.14518798105\\
60.125	0.2394	-1203.5297754478\\
60.125	0.24306	-1256.13398051534\\
60.125	0.24672	-1309.95780318365\\
60.125	0.25038	-1365.00124345275\\
60.125	0.25404	-1421.26430132263\\
60.125	0.2577	-1478.74697679328\\
60.125	0.26136	-1537.44926986472\\
60.125	0.26502	-1597.37118053694\\
60.125	0.26868	-1658.51270880994\\
60.125	0.27234	-1720.87385468373\\
60.125	0.276	-1784.45461815829\\
60.5	0.093	-98.8036810492002\\
60.5	0.09666	-102.652684563337\\
60.5	0.10032	-107.721305678255\\
60.5	0.10398	-114.009544393953\\
60.5	0.10764	-121.517400710433\\
60.5	0.1113	-130.244874627694\\
60.5	0.11496	-140.191966145735\\
60.5	0.11862	-151.358675264557\\
60.5	0.12228	-163.74500198416\\
60.5	0.12594	-177.350946304545\\
60.5	0.1296	-192.17650822571\\
60.5	0.13326	-208.221687747656\\
60.5	0.13692	-225.486484870382\\
60.5	0.14058	-243.97089959389\\
60.5	0.14424	-263.674931918179\\
60.5	0.1479	-284.598581843248\\
60.5	0.15156	-306.741849369099\\
60.5	0.15522	-330.10473449573\\
60.5	0.15888	-354.687237223143\\
60.5	0.16254	-380.489357551336\\
60.5	0.1662	-407.511095480311\\
60.5	0.16986	-435.752451010066\\
60.5	0.17352	-465.213424140602\\
60.5	0.17718	-495.894014871918\\
60.5	0.18084	-527.794223204016\\
60.5	0.1845	-560.914049136895\\
60.5	0.18816	-595.253492670555\\
60.5	0.19182	-630.812553804995\\
60.5	0.19548	-667.591232540217\\
60.5	0.19914	-705.589528876219\\
60.5	0.2028	-744.807442813003\\
60.5	0.20646	-785.244974350567\\
60.5	0.21012	-826.902123488912\\
60.5	0.21378	-869.778890228038\\
60.5	0.21744	-913.875274567944\\
60.5	0.2211	-959.191276508633\\
60.5	0.22476	-1005.7268960501\\
60.5	0.22842	-1053.48213319235\\
60.5	0.23208	-1102.45698793538\\
60.5	0.23574	-1152.65146027919\\
60.5	0.2394	-1204.06555022379\\
60.5	0.24306	-1256.69925776916\\
60.5	0.24672	-1310.55258291531\\
60.5	0.25038	-1365.62552566225\\
60.5	0.25404	-1421.91808600996\\
60.5	0.2577	-1479.43026395846\\
60.5	0.26136	-1538.16205950774\\
60.5	0.26502	-1598.1134726578\\
60.5	0.26868	-1659.28450340864\\
60.5	0.27234	-1721.67515176026\\
60.5	0.276	-1785.28541771266\\
60.875	0.093	-98.2979603187528\\
60.875	0.09666	-102.176466310728\\
60.875	0.10032	-107.274589903485\\
60.875	0.10398	-113.592331097022\\
60.875	0.10764	-121.129689891341\\
60.875	0.1113	-129.88666628644\\
60.875	0.11496	-139.86326028232\\
60.875	0.11862	-151.059471878981\\
60.875	0.12228	-163.475301076423\\
60.875	0.12594	-177.110747874646\\
60.875	0.1296	-191.96581227365\\
60.875	0.13326	-208.040494273434\\
60.875	0.13692	-225.334793874\\
60.875	0.14058	-243.848711075346\\
60.875	0.14424	-263.582245877475\\
60.875	0.1479	-284.535398280383\\
60.875	0.15156	-306.708168284072\\
60.875	0.15522	-330.100555888542\\
60.875	0.15888	-354.712561093794\\
60.875	0.16254	-380.544183899826\\
60.875	0.1662	-407.595424306639\\
60.875	0.16986	-435.866282314232\\
60.875	0.17352	-465.356757922607\\
60.875	0.17718	-496.066851131763\\
60.875	0.18084	-527.996561941699\\
60.875	0.1845	-561.145890352417\\
60.875	0.18816	-595.514836363915\\
60.875	0.19182	-631.103399976195\\
60.875	0.19548	-667.911581189255\\
60.875	0.19914	-705.939380003096\\
60.875	0.2028	-745.186796417718\\
60.875	0.20646	-785.653830433122\\
60.875	0.21012	-827.340482049305\\
60.875	0.21378	-870.24675126627\\
60.875	0.21744	-914.372638084015\\
60.875	0.2211	-959.718142502542\\
60.875	0.22476	-1006.28326452185\\
60.875	0.22842	-1054.06800414194\\
60.875	0.23208	-1103.07236136281\\
60.875	0.23574	-1153.29633618446\\
60.875	0.2394	-1204.73992860689\\
60.875	0.24306	-1257.4031386301\\
60.875	0.24672	-1311.28596625409\\
60.875	0.25038	-1366.38841147887\\
60.875	0.25404	-1422.71047430442\\
60.875	0.2577	-1480.25215473076\\
60.875	0.26136	-1539.01345275788\\
60.875	0.26502	-1598.99436838577\\
60.875	0.26868	-1660.19490161445\\
60.875	0.27234	-1722.61505244391\\
60.875	0.276	-1786.25482087415\\
61.25	0.093	-97.9308431954249\\
61.25	0.09666	-101.838851665239\\
61.25	0.10032	-106.966477735835\\
61.25	0.10398	-113.31372140721\\
61.25	0.10764	-120.880582679368\\
61.25	0.1113	-129.667061552306\\
61.25	0.11496	-139.673158026025\\
61.25	0.11862	-150.898872100525\\
61.25	0.12228	-163.344203775805\\
61.25	0.12594	-177.009153051867\\
61.25	0.1296	-191.89371992871\\
61.25	0.13326	-207.997904406333\\
61.25	0.13692	-225.321706484738\\
61.25	0.14058	-243.865126163923\\
61.25	0.14424	-263.628163443889\\
61.25	0.1479	-284.610818324637\\
61.25	0.15156	-306.813090806164\\
61.25	0.15522	-330.234980888474\\
61.25	0.15888	-354.876488571563\\
61.25	0.16254	-380.737613855435\\
61.25	0.1662	-407.818356740086\\
61.25	0.16986	-436.118717225519\\
61.25	0.17352	-465.638695311732\\
61.25	0.17718	-496.378290998727\\
61.25	0.18084	-528.337504286502\\
61.25	0.1845	-561.516335175059\\
61.25	0.18816	-595.914783664396\\
61.25	0.19182	-631.532849754514\\
61.25	0.19548	-668.370533445413\\
61.25	0.19914	-706.427834737093\\
61.25	0.2028	-745.704753629554\\
61.25	0.20646	-786.201290122795\\
61.25	0.21012	-827.917444216818\\
61.25	0.21378	-870.853215911621\\
61.25	0.21744	-915.008605207206\\
61.25	0.2211	-960.383612103571\\
61.25	0.22476	-1006.97823660072\\
61.25	0.22842	-1054.79247869865\\
61.25	0.23208	-1103.82633839735\\
61.25	0.23574	-1154.07981569684\\
61.25	0.2394	-1205.55291059711\\
61.25	0.24306	-1258.24562309816\\
61.25	0.24672	-1312.15795319999\\
61.25	0.25038	-1367.28990090261\\
61.25	0.25404	-1423.641466206\\
61.25	0.2577	-1481.21264911018\\
61.25	0.26136	-1540.00344961513\\
61.25	0.26502	-1600.01386772087\\
61.25	0.26868	-1661.24390342738\\
61.25	0.27234	-1723.69355673468\\
61.25	0.276	-1787.36282764276\\
61.625	0.093	-97.7023296792157\\
61.625	0.09666	-101.639840626868\\
61.625	0.10032	-106.796969175303\\
61.625	0.10398	-113.173715324518\\
61.625	0.10764	-120.770079074514\\
61.625	0.1113	-129.586060425291\\
61.625	0.11496	-139.621659376848\\
61.625	0.11862	-150.876875929187\\
61.625	0.12228	-163.351710082306\\
61.625	0.12594	-177.046161836207\\
61.625	0.1296	-191.960231190888\\
61.625	0.13326	-208.09391814635\\
61.625	0.13692	-225.447222702594\\
61.625	0.14058	-244.020144859618\\
61.625	0.14424	-263.812684617423\\
61.625	0.1479	-284.824841976009\\
61.625	0.15156	-307.056616935375\\
61.625	0.15522	-330.508009495524\\
61.625	0.15888	-355.179019656452\\
61.625	0.16254	-381.069647418162\\
61.625	0.1662	-408.179892780652\\
61.625	0.16986	-436.509755743924\\
61.625	0.17352	-466.059236307976\\
61.625	0.17718	-496.82833447281\\
61.625	0.18084	-528.817050238423\\
61.625	0.1845	-562.025383604818\\
61.625	0.18816	-596.453334571994\\
61.625	0.19182	-632.100903139951\\
61.625	0.19548	-668.968089308689\\
61.625	0.19914	-707.054893078208\\
61.625	0.2028	-746.361314448508\\
61.625	0.20646	-786.887353419588\\
61.625	0.21012	-828.63300999145\\
61.625	0.21378	-871.598284164091\\
61.625	0.21744	-915.783175937515\\
61.625	0.2211	-961.187685311718\\
61.625	0.22476	-1007.8118122867\\
61.625	0.22842	-1055.65555686247\\
61.625	0.23208	-1104.71891903902\\
61.625	0.23574	-1155.00189881635\\
61.625	0.2394	-1206.50449619445\\
61.625	0.24306	-1259.22671117334\\
61.625	0.24672	-1313.16854375301\\
61.625	0.25038	-1368.32999393347\\
61.625	0.25404	-1424.7110617147\\
61.625	0.2577	-1482.31174709671\\
61.625	0.26136	-1541.13205007951\\
61.625	0.26502	-1601.17197066308\\
61.625	0.26868	-1662.43150884744\\
61.625	0.27234	-1724.91066463257\\
61.625	0.276	-1788.60943801849\\
62	0.093	-97.6124197701259\\
62	0.09666	-101.579433195618\\
62	0.10032	-106.766064221891\\
62	0.10398	-113.172312848944\\
62	0.10764	-120.798179076779\\
62	0.1113	-129.643662905395\\
62	0.11496	-139.708764334791\\
62	0.11862	-150.993483364968\\
62	0.12228	-163.497819995927\\
62	0.12594	-177.221774227666\\
62	0.1296	-192.165346060186\\
62	0.13326	-208.328535493487\\
62	0.13692	-225.711342527569\\
62	0.14058	-244.313767162432\\
62	0.14424	-264.135809398076\\
62	0.1479	-285.1774692345\\
62	0.15156	-307.438746671706\\
62	0.15522	-330.919641709693\\
62	0.15888	-355.62015434846\\
62	0.16254	-381.540284588008\\
62	0.1662	-408.680032428338\\
62	0.16986	-437.039397869448\\
62	0.17352	-466.618380911339\\
62	0.17718	-497.416981554011\\
62	0.18084	-529.435199797464\\
62	0.1845	-562.673035641698\\
62	0.18816	-597.130489086713\\
62	0.19182	-632.807560132508\\
62	0.19548	-669.704248779085\\
62	0.19914	-707.820555026442\\
62	0.2028	-747.15647887458\\
62	0.20646	-787.7120203235\\
62	0.21012	-829.4871793732\\
62	0.21378	-872.481956023681\\
62	0.21744	-916.696350274943\\
62	0.2211	-962.130362126986\\
62	0.22476	-1008.78399157981\\
62	0.22842	-1056.65723863342\\
62	0.23208	-1105.7501032878\\
62	0.23574	-1156.06258554297\\
62	0.2394	-1207.59468539892\\
62	0.24306	-1260.34640285564\\
62	0.24672	-1314.31773791315\\
62	0.25038	-1369.50869057144\\
62	0.25404	-1425.91926083051\\
62	0.2577	-1483.54944869037\\
62	0.26136	-1542.399254151\\
62	0.26502	-1602.46867721241\\
62	0.26868	-1663.75771787461\\
62	0.27234	-1726.26637613758\\
62	0.276	-1789.99465200134\\
62.375	0.093	-97.6611134681548\\
62.375	0.09666	-101.657629371485\\
62.375	0.10032	-106.873762875597\\
62.375	0.10398	-113.309513980489\\
62.375	0.10764	-120.964882686163\\
62.375	0.1113	-129.839868992617\\
62.375	0.11496	-139.934472899853\\
62.375	0.11862	-151.248694407869\\
62.375	0.12228	-163.782533516666\\
62.375	0.12594	-177.535990226244\\
62.375	0.1296	-192.509064536603\\
62.375	0.13326	-208.701756447742\\
62.375	0.13692	-226.114065959663\\
62.375	0.14058	-244.745993072365\\
62.375	0.14424	-264.597537785848\\
62.375	0.1479	-285.668700100111\\
62.375	0.15156	-307.959480015156\\
62.375	0.15522	-331.46987753098\\
62.375	0.15888	-356.199892647587\\
62.375	0.16254	-382.149525364974\\
62.375	0.1662	-409.318775683143\\
62.375	0.16986	-437.707643602091\\
62.375	0.17352	-467.316129121821\\
62.375	0.17718	-498.144232242332\\
62.375	0.18084	-530.191952963623\\
62.375	0.1845	-563.459291285696\\
62.375	0.18816	-597.946247208549\\
62.375	0.19182	-633.652820732184\\
62.375	0.19548	-670.579011856599\\
62.375	0.19914	-708.724820581796\\
62.375	0.2028	-748.090246907773\\
62.375	0.20646	-788.675290834531\\
62.375	0.21012	-830.47995236207\\
62.375	0.21378	-873.50423149039\\
62.375	0.21744	-917.74812821949\\
62.375	0.2211	-963.211642549372\\
62.375	0.22476	-1009.89477448003\\
62.375	0.22842	-1057.79752401148\\
62.375	0.23208	-1106.9198911437\\
62.375	0.23574	-1157.26187587671\\
62.375	0.2394	-1208.8234782105\\
62.375	0.24306	-1261.60469814506\\
62.375	0.24672	-1315.60553568041\\
62.375	0.25038	-1370.82599081654\\
62.375	0.25404	-1427.26606355345\\
62.375	0.2577	-1484.92575389114\\
62.375	0.26136	-1543.80506182961\\
62.375	0.26502	-1603.90398736886\\
62.375	0.26868	-1665.2225305089\\
62.375	0.27234	-1727.76069124971\\
62.375	0.276	-1791.51846959131\\
62.75	0.093	-97.848410773303\\
62.75	0.09666	-101.874429154473\\
62.75	0.10032	-107.120065136423\\
62.75	0.10398	-113.585318719155\\
62.75	0.10764	-121.270189902667\\
62.75	0.1113	-130.17467868696\\
62.75	0.11496	-140.298785072034\\
62.75	0.11862	-151.642509057889\\
62.75	0.12228	-164.205850644525\\
62.75	0.12594	-177.988809831941\\
62.75	0.1296	-192.991386620139\\
62.75	0.13326	-209.213581009118\\
62.75	0.13692	-226.655392998877\\
62.75	0.14058	-245.316822589418\\
62.75	0.14424	-265.197869780739\\
62.75	0.1479	-286.298534572841\\
62.75	0.15156	-308.618816965724\\
62.75	0.15522	-332.158716959389\\
62.75	0.15888	-356.918234553834\\
62.75	0.16254	-382.89736974906\\
62.75	0.1662	-410.096122545066\\
62.75	0.16986	-438.514492941855\\
62.75	0.17352	-468.152480939422\\
62.75	0.17718	-499.010086537772\\
62.75	0.18084	-531.087309736902\\
62.75	0.1845	-564.384150536814\\
62.75	0.18816	-598.900608937506\\
62.75	0.19182	-634.636684938979\\
62.75	0.19548	-671.592378541233\\
62.75	0.19914	-709.767689744269\\
62.75	0.2028	-749.162618548085\\
62.75	0.20646	-789.777164952681\\
62.75	0.21012	-831.611328958059\\
62.75	0.21378	-874.665110564217\\
62.75	0.21744	-918.938509771157\\
62.75	0.2211	-964.431526578877\\
62.75	0.22476	-1011.14416098738\\
62.75	0.22842	-1059.07641299666\\
62.75	0.23208	-1108.22828260672\\
62.75	0.23574	-1158.59976981757\\
62.75	0.2394	-1210.19087462919\\
62.75	0.24306	-1263.0015970416\\
62.75	0.24672	-1317.03193705479\\
62.75	0.25038	-1372.28189466876\\
62.75	0.25404	-1428.7514698835\\
62.75	0.2577	-1486.44066269903\\
62.75	0.26136	-1545.34947311534\\
62.75	0.26502	-1605.47790113244\\
62.75	0.26868	-1666.82594675031\\
62.75	0.27234	-1729.39360996896\\
62.75	0.276	-1793.18089078839\\
63.125	0.093	-98.1743116855703\\
63.125	0.09666	-102.229832544579\\
63.125	0.10032	-107.504971004368\\
63.125	0.10398	-113.999727064938\\
63.125	0.10764	-121.714100726289\\
63.125	0.1113	-130.648091988421\\
63.125	0.11496	-140.801700851334\\
63.125	0.11862	-152.174927315027\\
63.125	0.12228	-164.767771379502\\
63.125	0.12594	-178.580233044758\\
63.125	0.1296	-193.612312310794\\
63.125	0.13326	-209.864009177611\\
63.125	0.13692	-227.335323645209\\
63.125	0.14058	-246.026255713589\\
63.125	0.14424	-265.936805382749\\
63.125	0.1479	-287.06697265269\\
63.125	0.15156	-309.416757523412\\
63.125	0.15522	-332.986159994915\\
63.125	0.15888	-357.775180067199\\
63.125	0.16254	-383.783817740264\\
63.125	0.1662	-411.012073014109\\
63.125	0.16986	-439.459945888736\\
63.125	0.17352	-469.127436364143\\
63.125	0.17718	-500.014544440331\\
63.125	0.18084	-532.1212701173\\
63.125	0.1845	-565.447613395051\\
63.125	0.18816	-599.993574273582\\
63.125	0.19182	-635.759152752893\\
63.125	0.19548	-672.744348832987\\
63.125	0.19914	-710.94916251386\\
63.125	0.2028	-750.373593795515\\
63.125	0.20646	-791.01764267795\\
63.125	0.21012	-832.881309161168\\
63.125	0.21378	-875.964593245164\\
63.125	0.21744	-920.267494929943\\
63.125	0.2211	-965.790014215502\\
63.125	0.22476	-1012.53215110184\\
63.125	0.22842	-1060.49390558896\\
63.125	0.23208	-1109.67527767687\\
63.125	0.23574	-1160.07626736555\\
63.125	0.2394	-1211.69687465501\\
63.125	0.24306	-1264.53709954526\\
63.125	0.24672	-1318.59694203628\\
63.125	0.25038	-1373.87640212809\\
63.125	0.25404	-1430.37547982068\\
63.125	0.2577	-1488.09417511405\\
63.125	0.26136	-1547.03248800819\\
63.125	0.26502	-1607.19041850313\\
63.125	0.26868	-1668.56796659884\\
63.125	0.27234	-1731.16513229533\\
63.125	0.276	-1794.9819155926\\
63.5	0.093	-98.6388162049566\\
63.5	0.09666	-102.723839541804\\
63.5	0.10032	-108.028480479432\\
63.5	0.10398	-114.552739017841\\
63.5	0.10764	-122.29661515703\\
63.5	0.1113	-131.260108897001\\
63.5	0.11496	-141.443220237753\\
63.5	0.11862	-152.845949179285\\
63.5	0.12228	-165.468295721598\\
63.5	0.12594	-179.310259864693\\
63.5	0.1296	-194.371841608568\\
63.5	0.13326	-210.653040953224\\
63.5	0.13692	-228.153857898661\\
63.5	0.14058	-246.874292444879\\
63.5	0.14424	-266.814344591878\\
63.5	0.1479	-287.974014339658\\
63.5	0.15156	-310.353301688218\\
63.5	0.15522	-333.952206637561\\
63.5	0.15888	-358.770729187683\\
63.5	0.16254	-384.808869338587\\
63.5	0.1662	-412.066627090271\\
63.5	0.16986	-440.544002442736\\
63.5	0.17352	-470.240995395982\\
63.5	0.17718	-501.157605950009\\
63.5	0.18084	-533.293834104817\\
63.5	0.1845	-566.649679860406\\
63.5	0.18816	-601.225143216776\\
63.5	0.19182	-637.020224173927\\
63.5	0.19548	-674.034922731858\\
63.5	0.19914	-712.269238890571\\
63.5	0.2028	-751.723172650065\\
63.5	0.20646	-792.396724010339\\
63.5	0.21012	-834.289892971394\\
63.5	0.21378	-877.40267953323\\
63.5	0.21744	-921.735083695848\\
63.5	0.2211	-967.287105459245\\
63.5	0.22476	-1014.05874482342\\
63.5	0.22842	-1062.05000178838\\
63.5	0.23208	-1111.26087635412\\
63.5	0.23574	-1161.69136852065\\
63.5	0.2394	-1213.34147828795\\
63.5	0.24306	-1266.21120565603\\
63.5	0.24672	-1320.3005506249\\
63.5	0.25038	-1375.60951319454\\
63.5	0.25404	-1432.13809336497\\
63.5	0.2577	-1489.88629113618\\
63.5	0.26136	-1548.85410650816\\
63.5	0.26502	-1609.04153948093\\
63.5	0.26868	-1670.44859005448\\
63.5	0.27234	-1733.07525822881\\
63.5	0.276	-1796.92154400393\\
63.875	0.093	-99.2419243314626\\
63.875	0.09666	-103.356450146148\\
63.875	0.10032	-108.690593561615\\
63.875	0.10398	-115.244354577863\\
63.875	0.10764	-123.017733194891\\
63.875	0.1113	-132.010729412701\\
63.875	0.11496	-142.223343231291\\
63.875	0.11862	-153.655574650662\\
63.875	0.12228	-166.307423670815\\
63.875	0.12594	-180.178890291748\\
63.875	0.1296	-195.269974513462\\
63.875	0.13326	-211.580676335956\\
63.875	0.13692	-229.110995759232\\
63.875	0.14058	-247.860932783289\\
63.875	0.14424	-267.830487408127\\
63.875	0.1479	-289.019659633745\\
63.875	0.15156	-311.428449460145\\
63.875	0.15522	-335.056856887325\\
63.875	0.15888	-359.904881915287\\
63.875	0.16254	-385.972524544029\\
63.875	0.1662	-413.259784773552\\
63.875	0.16986	-441.766662603856\\
63.875	0.17352	-471.493158034941\\
63.875	0.17718	-502.439271066807\\
63.875	0.18084	-534.605001699454\\
63.875	0.1845	-567.990349932881\\
63.875	0.18816	-602.59531576709\\
63.875	0.19182	-638.419899202079\\
63.875	0.19548	-675.46410023785\\
63.875	0.19914	-713.727918874401\\
63.875	0.2028	-753.211355111733\\
63.875	0.20646	-793.914408949847\\
63.875	0.21012	-835.837080388741\\
63.875	0.21378	-878.979369428416\\
63.875	0.21744	-923.341276068871\\
63.875	0.2211	-968.922800310108\\
63.875	0.22476	-1015.72394215213\\
63.875	0.22842	-1063.74470159493\\
63.875	0.23208	-1112.9850786385\\
63.875	0.23574	-1163.44507328286\\
63.875	0.2394	-1215.12468552801\\
63.875	0.24306	-1268.02391537393\\
63.875	0.24672	-1322.14276282063\\
63.875	0.25038	-1377.48122786812\\
63.875	0.25404	-1434.03931051638\\
63.875	0.2577	-1491.81701076543\\
63.875	0.26136	-1550.81432861525\\
63.875	0.26502	-1611.03126406586\\
63.875	0.26868	-1672.46781711725\\
63.875	0.27234	-1735.12398776942\\
63.875	0.276	-1798.99977602237\\
64.25	0.093	-99.9836360650873\\
64.25	0.09666	-104.127664357612\\
64.25	0.10032	-109.491310250918\\
64.25	0.10398	-116.074573745004\\
64.25	0.10764	-123.877454839871\\
64.25	0.1113	-132.89995353552\\
64.25	0.11496	-143.142069831948\\
64.25	0.11862	-154.603803729158\\
64.25	0.12228	-167.28515522715\\
64.25	0.12594	-181.186124325921\\
64.25	0.1296	-196.306711025474\\
64.25	0.13326	-212.646915325808\\
64.25	0.13692	-230.206737226923\\
64.25	0.14058	-248.986176728817\\
64.25	0.14424	-268.985233831494\\
64.25	0.1479	-290.203908534951\\
64.25	0.15156	-312.64220083919\\
64.25	0.15522	-336.300110744209\\
64.25	0.15888	-361.177638250009\\
64.25	0.16254	-387.27478335659\\
64.25	0.1662	-414.591546063952\\
64.25	0.16986	-443.127926372095\\
64.25	0.17352	-472.883924281019\\
64.25	0.17718	-503.859539790723\\
64.25	0.18084	-536.054772901209\\
64.25	0.1845	-569.469623612475\\
64.25	0.18816	-604.104091924523\\
64.25	0.19182	-639.958177837351\\
64.25	0.19548	-677.03188135096\\
64.25	0.19914	-715.32520246535\\
64.25	0.2028	-754.838141180521\\
64.25	0.20646	-795.570697496474\\
64.25	0.21012	-837.522871413206\\
64.25	0.21378	-880.69466293072\\
64.25	0.21744	-925.086072049014\\
64.25	0.2211	-970.69709876809\\
64.25	0.22476	-1017.52774308795\\
64.25	0.22842	-1065.57800500858\\
64.25	0.23208	-1114.84788453\\
64.25	0.23574	-1165.3373816522\\
64.25	0.2394	-1217.04649637518\\
64.25	0.24306	-1269.97522869894\\
64.25	0.24672	-1324.12357862348\\
64.25	0.25038	-1379.49154614881\\
64.25	0.25404	-1436.07913127491\\
64.25	0.2577	-1493.8863340018\\
64.25	0.26136	-1552.91315432946\\
64.25	0.26502	-1613.15959225791\\
64.25	0.26868	-1674.62564778714\\
64.25	0.27234	-1737.31132091714\\
64.25	0.276	-1801.21661164793\\
64.625	0.093	-100.863951405831\\
64.625	0.09666	-105.037482176195\\
64.625	0.10032	-110.430630547339\\
64.625	0.10398	-117.043396519264\\
64.625	0.10764	-124.87578009197\\
64.625	0.1113	-133.927781265457\\
64.625	0.11496	-144.199400039726\\
64.625	0.11862	-155.690636414774\\
64.625	0.12228	-168.401490390604\\
64.625	0.12594	-182.331961967215\\
64.625	0.1296	-197.482051144606\\
64.625	0.13326	-213.851757922778\\
64.625	0.13692	-231.441082301732\\
64.625	0.14058	-250.250024281466\\
64.625	0.14424	-270.278583861981\\
64.625	0.1479	-291.526761043277\\
64.625	0.15156	-313.994555825354\\
64.625	0.15522	-337.681968208213\\
64.625	0.15888	-362.588998191851\\
64.625	0.16254	-388.715645776271\\
64.625	0.1662	-416.061910961472\\
64.625	0.16986	-444.627793747454\\
64.625	0.17352	-474.413294134215\\
64.625	0.17718	-505.41841212176\\
64.625	0.18084	-537.643147710083\\
64.625	0.1845	-571.087500899189\\
64.625	0.18816	-605.751471689075\\
64.625	0.19182	-641.635060079742\\
64.625	0.19548	-678.73826607119\\
64.625	0.19914	-717.061089663419\\
64.625	0.2028	-756.603530856429\\
64.625	0.20646	-797.365589650219\\
64.625	0.21012	-839.347266044791\\
64.625	0.21378	-882.548560040143\\
64.625	0.21744	-926.969471636277\\
64.625	0.2211	-972.610000833191\\
64.625	0.22476	-1019.47014763089\\
64.625	0.22842	-1067.54991202936\\
64.625	0.23208	-1116.84929402862\\
64.625	0.23574	-1167.36829362866\\
64.625	0.2394	-1219.10691082948\\
64.625	0.24306	-1272.06514563108\\
64.625	0.24672	-1326.24299803346\\
64.625	0.25038	-1381.64046803662\\
64.625	0.25404	-1438.25755564056\\
64.625	0.2577	-1496.09426084529\\
64.625	0.26136	-1555.15058365079\\
64.625	0.26502	-1615.42652405707\\
64.625	0.26868	-1676.92208206414\\
64.625	0.27234	-1739.63725767199\\
64.625	0.276	-1803.57205088062\\
65	0.093	-101.882870353695\\
65	0.09666	-106.085903601897\\
65	0.10032	-111.50855445088\\
65	0.10398	-118.150822900644\\
65	0.10764	-126.012708951189\\
65	0.1113	-135.094212602515\\
65	0.11496	-145.395333854621\\
65	0.11862	-156.916072707509\\
65	0.12228	-169.656429161177\\
65	0.12594	-183.616403215627\\
65	0.1296	-198.795994870857\\
65	0.13326	-215.195204126868\\
65	0.13692	-232.81403098366\\
65	0.14058	-251.652475441233\\
65	0.14424	-271.710537499588\\
65	0.1479	-292.988217158722\\
65	0.15156	-315.485514418638\\
65	0.15522	-339.202429279334\\
65	0.15888	-364.138961740812\\
65	0.16254	-390.295111803071\\
65	0.1662	-417.670879466111\\
65	0.16986	-446.266264729931\\
65	0.17352	-476.081267594532\\
65	0.17718	-507.115888059914\\
65	0.18084	-539.370126126077\\
65	0.1845	-572.843981793021\\
65	0.18816	-607.537455060746\\
65	0.19182	-643.450545929252\\
65	0.19548	-680.583254398539\\
65	0.19914	-718.935580468606\\
65	0.2028	-758.507524139455\\
65	0.20646	-799.299085411085\\
65	0.21012	-841.310264283495\\
65	0.21378	-884.541060756686\\
65	0.21744	-928.991474830658\\
65	0.2211	-974.661506505412\\
65	0.22476	-1021.55115578095\\
65	0.22842	-1069.66042265726\\
65	0.23208	-1118.98930713436\\
65	0.23574	-1169.53780921223\\
65	0.2394	-1221.30592889089\\
65	0.24306	-1274.29366617033\\
65	0.24672	-1328.50102105055\\
65	0.25038	-1383.92799353155\\
65	0.25404	-1440.57458361333\\
65	0.2577	-1498.44079129589\\
65	0.26136	-1557.52661657924\\
65	0.26502	-1617.83205946336\\
65	0.26868	-1679.35711994827\\
65	0.27234	-1742.10179803395\\
65	0.276	-1806.06609372042\\
65.375	0.093	-103.040392908677\\
65.375	0.09666	-107.272928634718\\
65.375	0.10032	-112.725081961539\\
65.375	0.10398	-119.396852889142\\
65.375	0.10764	-127.288241417526\\
65.375	0.1113	-136.39924754669\\
65.375	0.11496	-146.729871276636\\
65.375	0.11862	-158.280112607362\\
65.375	0.12228	-171.049971538869\\
65.375	0.12594	-185.039448071157\\
65.375	0.1296	-200.248542204227\\
65.375	0.13326	-216.677253938076\\
65.375	0.13692	-234.325583272707\\
65.375	0.14058	-253.193530208119\\
65.375	0.14424	-273.281094744312\\
65.375	0.1479	-294.588276881285\\
65.375	0.15156	-317.11507661904\\
65.375	0.15522	-340.861493957575\\
65.375	0.15888	-365.827528896892\\
65.375	0.16254	-392.013181436989\\
65.375	0.1662	-419.418451577868\\
65.375	0.16986	-448.043339319527\\
65.375	0.17352	-477.887844661967\\
65.375	0.17718	-508.951967605188\\
65.375	0.18084	-541.23570814919\\
65.375	0.1845	-574.739066293972\\
65.375	0.18816	-609.462042039536\\
65.375	0.19182	-645.404635385881\\
65.375	0.19548	-682.566846333006\\
65.375	0.19914	-720.948674880912\\
65.375	0.2028	-760.5501210296\\
65.375	0.20646	-801.371184779068\\
65.375	0.21012	-843.411866129317\\
65.375	0.21378	-886.672165080347\\
65.375	0.21744	-931.152081632158\\
65.375	0.2211	-976.85161578475\\
65.375	0.22476	-1023.77076753812\\
65.375	0.22842	-1071.90953689228\\
65.375	0.23208	-1121.26792384721\\
65.375	0.23574	-1171.84592840293\\
65.375	0.2394	-1223.64355055942\\
65.375	0.24306	-1276.6607903167\\
65.375	0.24672	-1330.89764767476\\
65.375	0.25038	-1386.3541226336\\
65.375	0.25404	-1443.03021519322\\
65.375	0.2577	-1500.92592535362\\
65.375	0.26136	-1560.0412531148\\
65.375	0.26502	-1620.37619847676\\
65.375	0.26868	-1681.93076143951\\
65.375	0.27234	-1744.70494200303\\
65.375	0.276	-1808.69874016734\\
65.75	0.093	-104.336519070778\\
65.75	0.09666	-108.598557274657\\
65.75	0.10032	-114.080213079318\\
65.75	0.10398	-120.78148648476\\
65.75	0.10764	-128.702377490982\\
65.75	0.1113	-137.842886097985\\
65.75	0.11496	-148.20301230577\\
65.75	0.11862	-159.782756114334\\
65.75	0.12228	-172.582117523681\\
65.75	0.12594	-186.601096533808\\
65.75	0.1296	-201.839693144715\\
65.75	0.13326	-218.297907356404\\
65.75	0.13692	-235.975739168874\\
65.75	0.14058	-254.873188582124\\
65.75	0.14424	-274.990255596156\\
65.75	0.1479	-296.326940210968\\
65.75	0.15156	-318.883242426562\\
65.75	0.15522	-342.659162242936\\
65.75	0.15888	-367.654699660091\\
65.75	0.16254	-393.869854678027\\
65.75	0.1662	-421.304627296745\\
65.75	0.16986	-449.959017516242\\
65.75	0.17352	-479.833025336521\\
65.75	0.17718	-510.926650757581\\
65.75	0.18084	-543.239893779421\\
65.75	0.1845	-576.772754402043\\
65.75	0.18816	-611.525232625446\\
65.75	0.19182	-647.497328449629\\
65.75	0.19548	-684.689041874593\\
65.75	0.19914	-723.100372900338\\
65.75	0.2028	-762.731321526864\\
65.75	0.20646	-803.581887754172\\
65.75	0.21012	-845.652071582259\\
65.75	0.21378	-888.941873011128\\
65.75	0.21744	-933.451292040777\\
65.75	0.2211	-979.180328671209\\
65.75	0.22476	-1026.12898290242\\
65.75	0.22842	-1074.29725473441\\
65.75	0.23208	-1123.68514416719\\
65.75	0.23574	-1174.29265120074\\
65.75	0.2394	-1226.11977583508\\
65.75	0.24306	-1279.16651807019\\
65.75	0.24672	-1333.43287790609\\
65.75	0.25038	-1388.91885534277\\
65.75	0.25404	-1445.62445038023\\
65.75	0.2577	-1503.54966301847\\
65.75	0.26136	-1562.69449325749\\
65.75	0.26502	-1623.05894109729\\
65.75	0.26868	-1684.64300653787\\
65.75	0.27234	-1747.44668957923\\
65.75	0.276	-1811.46999022138\\
66.125	0.093	-105.771248839998\\
66.125	0.09666	-110.062789521716\\
66.125	0.10032	-115.573947804216\\
66.125	0.10398	-122.304723687496\\
66.125	0.10764	-130.255117171557\\
66.125	0.1113	-139.425128256399\\
66.125	0.11496	-149.814756942022\\
66.125	0.11862	-161.424003228426\\
66.125	0.12228	-174.252867115611\\
66.125	0.12594	-188.301348603577\\
66.125	0.1296	-203.569447692323\\
66.125	0.13326	-220.057164381851\\
66.125	0.13692	-237.764498672159\\
66.125	0.14058	-256.691450563249\\
66.125	0.14424	-276.838020055119\\
66.125	0.1479	-298.20420714777\\
66.125	0.15156	-320.790011841202\\
66.125	0.15522	-344.595434135415\\
66.125	0.15888	-369.620474030409\\
66.125	0.16254	-395.865131526184\\
66.125	0.1662	-423.32940662274\\
66.125	0.16986	-452.013299320077\\
66.125	0.17352	-481.916809618194\\
66.125	0.17718	-513.039937517092\\
66.125	0.18084	-545.382683016772\\
66.125	0.1845	-578.945046117232\\
66.125	0.18816	-613.727026818474\\
66.125	0.19182	-649.728625120496\\
66.125	0.19548	-686.949841023299\\
66.125	0.19914	-725.390674526883\\
66.125	0.2028	-765.051125631248\\
66.125	0.20646	-805.931194336394\\
66.125	0.21012	-848.03088064232\\
66.125	0.21378	-891.350184549028\\
66.125	0.21744	-935.889106056516\\
66.125	0.2211	-981.647645164786\\
66.125	0.22476	-1028.62580187384\\
66.125	0.22842	-1076.82357618367\\
66.125	0.23208	-1126.24096809428\\
66.125	0.23574	-1176.87797760567\\
66.125	0.2394	-1228.73460471785\\
66.125	0.24306	-1281.8108494308\\
66.125	0.24672	-1336.10671174454\\
66.125	0.25038	-1391.62219165905\\
66.125	0.25404	-1448.35728917435\\
66.125	0.2577	-1506.31200429043\\
66.125	0.26136	-1565.48633700729\\
66.125	0.26502	-1625.88028732493\\
66.125	0.26868	-1687.49385524335\\
66.125	0.27234	-1750.32704076255\\
66.125	0.276	-1814.37984388254\\
66.5	0.093	-107.344582216338\\
66.5	0.09666	-111.665625375895\\
66.5	0.10032	-117.206286136233\\
66.5	0.10398	-123.966564497352\\
66.5	0.10764	-131.946460459252\\
66.5	0.1113	-141.145974021933\\
66.5	0.11496	-151.565105185394\\
66.5	0.11862	-163.203853949637\\
66.5	0.12228	-176.062220314661\\
66.5	0.12594	-190.140204280465\\
66.5	0.1296	-205.437805847051\\
66.5	0.13326	-221.955025014417\\
66.5	0.13692	-239.691861782564\\
66.5	0.14058	-258.648316151492\\
66.5	0.14424	-278.824388121202\\
66.5	0.1479	-300.220077691691\\
66.5	0.15156	-322.835384862962\\
66.5	0.15522	-346.670309635014\\
66.5	0.15888	-371.724852007847\\
66.5	0.16254	-397.99901198146\\
66.5	0.1662	-425.492789555855\\
66.5	0.16986	-454.20618473103\\
66.5	0.17352	-484.139197506987\\
66.5	0.17718	-515.291827883724\\
66.5	0.18084	-547.664075861242\\
66.5	0.1845	-581.255941439541\\
66.5	0.18816	-616.067424618622\\
66.5	0.19182	-652.098525398482\\
66.5	0.19548	-689.349243779124\\
66.5	0.19914	-727.819579760547\\
66.5	0.2028	-767.509533342751\\
66.5	0.20646	-808.419104525736\\
66.5	0.21012	-850.5482933095\\
66.5	0.21378	-893.897099694047\\
66.5	0.21744	-938.465523679374\\
66.5	0.2211	-984.253565265482\\
66.5	0.22476	-1031.26122445237\\
66.5	0.22842	-1079.48850124004\\
66.5	0.23208	-1128.93539562849\\
66.5	0.23574	-1179.60190761772\\
66.5	0.2394	-1231.48803720774\\
66.5	0.24306	-1284.59378439853\\
66.5	0.24672	-1338.91914919011\\
66.5	0.25038	-1394.46413158246\\
66.5	0.25404	-1451.2287315756\\
66.5	0.2577	-1509.21294916952\\
66.5	0.26136	-1568.41678436421\\
66.5	0.26502	-1628.84023715969\\
66.5	0.26868	-1690.48330755595\\
66.5	0.27234	-1753.34599555299\\
66.5	0.276	-1817.42830115082\\
66.875	0.093	-109.056519199797\\
66.875	0.09666	-113.407064837193\\
66.875	0.10032	-118.97722807537\\
66.875	0.10398	-125.767008914328\\
66.875	0.10764	-133.776407354067\\
66.875	0.1113	-143.005423394586\\
66.875	0.11496	-153.454057035887\\
66.875	0.11862	-165.122308277968\\
66.875	0.12228	-178.01017712083\\
66.875	0.12594	-192.117663564474\\
66.875	0.1296	-207.444767608898\\
66.875	0.13326	-223.991489254103\\
66.875	0.13692	-241.757828500089\\
66.875	0.14058	-260.743785346856\\
66.875	0.14424	-280.949359794404\\
66.875	0.1479	-302.374551842732\\
66.875	0.15156	-325.019361491842\\
66.875	0.15522	-348.883788741733\\
66.875	0.15888	-373.967833592404\\
66.875	0.16254	-400.271496043857\\
66.875	0.1662	-427.79477609609\\
66.875	0.16986	-456.537673749104\\
66.875	0.17352	-486.500189002899\\
66.875	0.17718	-517.682321857475\\
66.875	0.18084	-550.084072312832\\
66.875	0.1845	-583.70544036897\\
66.875	0.18816	-618.546426025889\\
66.875	0.19182	-654.607029283589\\
66.875	0.19548	-691.887250142069\\
66.875	0.19914	-730.387088601331\\
66.875	0.2028	-770.106544661373\\
66.875	0.20646	-811.045618322197\\
66.875	0.21012	-853.204309583801\\
66.875	0.21378	-896.582618446186\\
66.875	0.21744	-941.180544909352\\
66.875	0.2211	-986.998088973299\\
66.875	0.22476	-1034.03525063803\\
66.875	0.22842	-1082.29202990354\\
66.875	0.23208	-1131.76842676983\\
66.875	0.23574	-1182.4644412369\\
66.875	0.2394	-1234.38007330475\\
66.875	0.24306	-1287.51532297338\\
66.875	0.24672	-1341.87019024279\\
66.875	0.25038	-1397.44467511299\\
66.875	0.25404	-1454.23877758396\\
66.875	0.2577	-1512.25249765572\\
66.875	0.26136	-1571.48583532826\\
66.875	0.26502	-1631.93879060157\\
66.875	0.26868	-1693.61136347567\\
66.875	0.27234	-1756.50355395055\\
66.875	0.276	-1820.61536202621\\
67.25	0.093	-110.907059790375\\
67.25	0.09666	-115.28710790561\\
67.25	0.10032	-120.886773621625\\
67.25	0.10398	-127.706056938422\\
67.25	0.10764	-135.744957855999\\
67.25	0.1113	-145.003476374358\\
67.25	0.11496	-155.481612493497\\
67.25	0.11862	-167.179366213417\\
67.25	0.12228	-180.096737534118\\
67.25	0.12594	-194.2337264556\\
67.25	0.1296	-209.590332977863\\
67.25	0.13326	-226.166557100907\\
67.25	0.13692	-243.962398824732\\
67.25	0.14058	-262.977858149337\\
67.25	0.14424	-283.212935074724\\
67.25	0.1479	-304.667629600891\\
67.25	0.15156	-327.34194172784\\
67.25	0.15522	-351.235871455569\\
67.25	0.15888	-376.349418784079\\
67.25	0.16254	-402.682583713371\\
67.25	0.1662	-430.235366243443\\
67.25	0.16986	-459.007766374296\\
67.25	0.17352	-488.99978410593\\
67.25	0.17718	-520.211419438344\\
67.25	0.18084	-552.64267237154\\
67.25	0.1845	-586.293542905517\\
67.25	0.18816	-621.164031040274\\
67.25	0.19182	-657.254136775813\\
67.25	0.19548	-694.563860112132\\
67.25	0.19914	-733.093201049233\\
67.25	0.2028	-772.842159587114\\
67.25	0.20646	-813.810735725777\\
67.25	0.21012	-855.998929465219\\
67.25	0.21378	-899.406740805443\\
67.25	0.21744	-944.034169746447\\
67.25	0.2211	-989.881216288234\\
67.25	0.22476	-1036.9478804308\\
67.25	0.22842	-1085.23416217415\\
67.25	0.23208	-1134.74006151828\\
67.25	0.23574	-1185.46557846319\\
67.25	0.2394	-1237.41071300888\\
67.25	0.24306	-1290.57546515535\\
67.25	0.24672	-1344.9598349026\\
67.25	0.25038	-1400.56382225063\\
67.25	0.25404	-1457.38742719945\\
67.25	0.2577	-1515.43064974904\\
67.25	0.26136	-1574.69348989942\\
67.25	0.26502	-1635.17594765057\\
67.25	0.26868	-1696.87802300251\\
67.25	0.27234	-1759.79971595523\\
67.25	0.276	-1823.94102650873\\
67.625	0.093	-112.896203988071\\
67.625	0.09666	-117.305754581144\\
67.625	0.10032	-122.934922774999\\
67.625	0.10398	-129.783708569635\\
67.625	0.10764	-137.852111965051\\
67.625	0.1113	-147.140132961248\\
67.625	0.11496	-157.647771558226\\
67.625	0.11862	-169.375027755985\\
67.625	0.12228	-182.321901554525\\
67.625	0.12594	-196.488392953846\\
67.625	0.1296	-211.874501953947\\
67.625	0.13326	-228.48022855483\\
67.625	0.13692	-246.305572756494\\
67.625	0.14058	-265.350534558938\\
67.625	0.14424	-285.615113962164\\
67.625	0.1479	-307.099310966169\\
67.625	0.15156	-329.803125570957\\
67.625	0.15522	-353.726557776525\\
67.625	0.15888	-378.869607582874\\
67.625	0.16254	-405.232274990004\\
67.625	0.1662	-432.814559997915\\
67.625	0.16986	-461.616462606607\\
67.625	0.17352	-491.637982816079\\
67.625	0.17718	-522.879120626333\\
67.625	0.18084	-555.339876037368\\
67.625	0.1845	-589.020249049182\\
67.625	0.18816	-623.920239661779\\
67.625	0.19182	-660.039847875157\\
67.625	0.19548	-697.379073689314\\
67.625	0.19914	-735.937917104254\\
67.625	0.2028	-775.716378119974\\
67.625	0.20646	-816.714456736475\\
67.625	0.21012	-858.932152953756\\
67.625	0.21378	-902.369466771819\\
67.625	0.21744	-947.026398190662\\
67.625	0.2211	-992.902947210287\\
67.625	0.22476	-1039.99911383069\\
67.625	0.22842	-1088.31489805188\\
67.625	0.23208	-1137.85029987385\\
67.625	0.23574	-1188.60531929659\\
67.625	0.2394	-1240.57995632012\\
67.625	0.24306	-1293.77421094443\\
67.625	0.24672	-1348.18808316952\\
67.625	0.25038	-1403.8215729954\\
67.625	0.25404	-1460.67468042205\\
67.625	0.2577	-1518.74740544948\\
67.625	0.26136	-1578.0397480777\\
67.625	0.26502	-1638.55170830669\\
67.625	0.26868	-1700.28328613647\\
67.625	0.27234	-1763.23448156703\\
67.625	0.276	-1827.40529459836\\
68	0.093	-115.023951792887\\
68	0.09666	-119.463004863799\\
68	0.10032	-125.121675535493\\
68	0.10398	-131.999963807967\\
68	0.10764	-140.097869681222\\
68	0.1113	-149.415393155258\\
68	0.11496	-159.952534230075\\
68	0.11862	-171.709292905672\\
68	0.12228	-184.685669182051\\
68	0.12594	-198.88166305921\\
68	0.1296	-214.297274537151\\
68	0.13326	-230.932503615873\\
68	0.13692	-248.787350295375\\
68	0.14058	-267.861814575658\\
68	0.14424	-288.155896456722\\
68	0.1479	-309.669595938567\\
68	0.15156	-332.402913021193\\
68	0.15522	-356.3558477046\\
68	0.15888	-381.528399988788\\
68	0.16254	-407.920569873757\\
68	0.1662	-435.532357359506\\
68	0.16986	-464.363762446037\\
68	0.17352	-494.414785133348\\
68	0.17718	-525.68542542144\\
68	0.18084	-558.175683310314\\
68	0.1845	-591.885558799968\\
68	0.18816	-626.815051890403\\
68	0.19182	-662.964162581619\\
68	0.19548	-700.332890873616\\
68	0.19914	-738.921236766394\\
68	0.2028	-778.729200259953\\
68	0.20646	-819.756781354293\\
68	0.21012	-862.003980049413\\
68	0.21378	-905.470796345314\\
68	0.21744	-950.157230241996\\
68	0.2211	-996.06328173946\\
68	0.22476	-1043.1889508377\\
68	0.22842	-1091.53423753673\\
68	0.23208	-1141.09914183654\\
68	0.23574	-1191.88366373712\\
68	0.2394	-1243.88780323849\\
68	0.24306	-1297.11156034064\\
68	0.24672	-1351.55493504357\\
68	0.25038	-1407.21792734728\\
68	0.25404	-1464.10053725177\\
68	0.2577	-1522.20276475704\\
68	0.26136	-1581.5246098631\\
68	0.26502	-1642.06607256993\\
68	0.26868	-1703.82715287755\\
68	0.27234	-1766.80785078594\\
68	0.276	-1831.00816629512\\
68.375	0.093	-117.290303204823\\
68.375	0.09666	-121.758858753574\\
68.375	0.10032	-127.447031903106\\
68.375	0.10398	-134.354822653419\\
68.375	0.10764	-142.482231004513\\
68.375	0.1113	-151.829256956388\\
68.375	0.11496	-162.395900509043\\
68.375	0.11862	-174.182161662479\\
68.375	0.12228	-187.188040416697\\
68.375	0.12594	-201.413536771695\\
68.375	0.1296	-216.858650727475\\
68.375	0.13326	-233.523382284035\\
68.375	0.13692	-251.407731441376\\
68.375	0.14058	-270.511698199498\\
68.375	0.14424	-290.835282558401\\
68.375	0.1479	-312.378484518084\\
68.375	0.15156	-335.141304078549\\
68.375	0.15522	-359.123741239795\\
68.375	0.15888	-384.325796001822\\
68.375	0.16254	-410.747468364629\\
68.375	0.1662	-438.388758328218\\
68.375	0.16986	-467.249665892587\\
68.375	0.17352	-497.330191057737\\
68.375	0.17718	-528.630333823668\\
68.375	0.18084	-561.15009419038\\
68.375	0.1845	-594.889472157873\\
68.375	0.18816	-629.848467726147\\
68.375	0.19182	-666.027080895202\\
68.375	0.19548	-703.425311665037\\
68.375	0.19914	-742.043160035654\\
68.375	0.2028	-781.880626007052\\
68.375	0.20646	-822.93770957923\\
68.375	0.21012	-865.214410752189\\
68.375	0.21378	-908.71072952593\\
68.375	0.21744	-953.42666590045\\
68.375	0.2211	-999.362219875753\\
68.375	0.22476	-1046.51739145184\\
68.375	0.22842	-1094.8921806287\\
68.375	0.23208	-1144.48658740634\\
68.375	0.23574	-1195.30061178477\\
68.375	0.2394	-1247.33425376398\\
68.375	0.24306	-1300.58751334396\\
68.375	0.24672	-1355.06039052473\\
68.375	0.25038	-1410.75288530628\\
68.375	0.25404	-1467.66499768861\\
68.375	0.2577	-1525.79672767172\\
68.375	0.26136	-1585.14807525562\\
68.375	0.26502	-1645.71904044029\\
68.375	0.26868	-1707.50962322574\\
68.375	0.27234	-1770.51982361198\\
68.375	0.276	-1834.74964159899\\
68.75	0.093	-119.695258223877\\
68.75	0.09666	-124.193316250467\\
68.75	0.10032	-129.910991877838\\
68.75	0.10398	-136.848285105989\\
68.75	0.10764	-145.005195934922\\
68.75	0.1113	-154.381724364636\\
68.75	0.11496	-164.97787039513\\
68.75	0.11862	-176.793634026405\\
68.75	0.12228	-189.829015258462\\
68.75	0.12594	-204.084014091299\\
68.75	0.1296	-219.558630524916\\
68.75	0.13326	-236.252864559315\\
68.75	0.13692	-254.166716194495\\
68.75	0.14058	-273.300185430456\\
68.75	0.14424	-293.653272267198\\
68.75	0.1479	-315.22597670472\\
68.75	0.15156	-338.018298743024\\
68.75	0.15522	-362.030238382108\\
68.75	0.15888	-387.261795621974\\
68.75	0.16254	-413.71297046262\\
68.75	0.1662	-441.383762904048\\
68.75	0.16986	-470.274172946255\\
68.75	0.17352	-500.384200589244\\
68.75	0.17718	-531.713845833014\\
68.75	0.18084	-564.263108677565\\
68.75	0.1845	-598.031989122896\\
68.75	0.18816	-633.020487169009\\
68.75	0.19182	-669.228602815903\\
68.75	0.19548	-706.656336063577\\
68.75	0.19914	-745.303686912033\\
68.75	0.2028	-785.170655361269\\
68.75	0.20646	-826.257241411287\\
68.75	0.21012	-868.563445062084\\
68.75	0.21378	-912.089266313663\\
68.75	0.21744	-956.834705166023\\
68.75	0.2211	-1002.79976161916\\
68.75	0.22476	-1049.98443567309\\
68.75	0.22842	-1098.38872732779\\
68.75	0.23208	-1148.01263658327\\
68.75	0.23574	-1198.85616343954\\
68.75	0.2394	-1250.91930789658\\
68.75	0.24306	-1304.20206995441\\
68.75	0.24672	-1358.70444961302\\
68.75	0.25038	-1414.4264468724\\
68.75	0.25404	-1471.36806173257\\
68.75	0.2577	-1529.52929419352\\
68.75	0.26136	-1588.91014425525\\
68.75	0.26502	-1649.51061191777\\
68.75	0.26868	-1711.33069718106\\
68.75	0.27234	-1774.37040004513\\
68.75	0.276	-1838.62972050999\\
69.125	0.093	-122.238816850051\\
69.125	0.09666	-126.766377354479\\
69.125	0.10032	-132.513555459689\\
69.125	0.10398	-139.480351165679\\
69.125	0.10764	-147.66676447245\\
69.125	0.1113	-157.072795380003\\
69.125	0.11496	-167.698443888336\\
69.125	0.11862	-179.54370999745\\
69.125	0.12228	-192.608593707345\\
69.125	0.12594	-206.893095018021\\
69.125	0.1296	-222.397213929478\\
69.125	0.13326	-239.120950441715\\
69.125	0.13692	-257.064304554734\\
69.125	0.14058	-276.227276268533\\
69.125	0.14424	-296.609865583114\\
69.125	0.1479	-318.212072498475\\
69.125	0.15156	-341.033897014618\\
69.125	0.15522	-365.075339131541\\
69.125	0.15888	-390.336398849245\\
69.125	0.16254	-416.81707616773\\
69.125	0.1662	-444.517371086996\\
69.125	0.16986	-473.437283607043\\
69.125	0.17352	-503.576813727871\\
69.125	0.17718	-534.935961449479\\
69.125	0.18084	-567.514726771869\\
69.125	0.1845	-601.313109695039\\
69.125	0.18816	-636.331110218991\\
69.125	0.19182	-672.568728343723\\
69.125	0.19548	-710.025964069236\\
69.125	0.19914	-748.70281739553\\
69.125	0.2028	-788.599288322606\\
69.125	0.20646	-829.715376850462\\
69.125	0.21012	-872.051082979098\\
69.125	0.21378	-915.606406708516\\
69.125	0.21744	-960.381348038714\\
69.125	0.2211	-1006.37590696969\\
69.125	0.22476	-1053.59008350145\\
69.125	0.22842	-1102.023877634\\
69.125	0.23208	-1151.67728936732\\
69.125	0.23574	-1202.55031870142\\
69.125	0.2394	-1254.64296563631\\
69.125	0.24306	-1307.95523017197\\
69.125	0.24672	-1362.48711230842\\
69.125	0.25038	-1418.23861204565\\
69.125	0.25404	-1475.20972938365\\
69.125	0.2577	-1533.40046432244\\
69.125	0.26136	-1592.81081686201\\
69.125	0.26502	-1653.44078700236\\
69.125	0.26868	-1715.29037474349\\
69.125	0.27234	-1778.3595800854\\
69.125	0.276	-1842.6484030281\\
69.5	0.093	-124.920979083343\\
69.5	0.09666	-129.47804206561\\
69.5	0.10032	-135.254722648658\\
69.5	0.10398	-142.251020832488\\
69.5	0.10764	-150.466936617098\\
69.5	0.1113	-159.902470002489\\
69.5	0.11496	-170.557620988661\\
69.5	0.11862	-182.432389575614\\
69.5	0.12228	-195.526775763347\\
69.5	0.12594	-209.840779551862\\
69.5	0.1296	-225.374400941157\\
69.5	0.13326	-242.127639931234\\
69.5	0.13692	-260.100496522091\\
69.5	0.14058	-279.292970713729\\
69.5	0.14424	-299.705062506149\\
69.5	0.1479	-321.336771899349\\
69.5	0.15156	-344.18809889333\\
69.5	0.15522	-368.259043488092\\
69.5	0.15888	-393.549605683635\\
69.5	0.16254	-420.059785479959\\
69.5	0.1662	-447.789582877064\\
69.5	0.16986	-476.738997874949\\
69.5	0.17352	-506.908030473616\\
69.5	0.17718	-538.296680673063\\
69.5	0.18084	-570.904948473291\\
69.5	0.1845	-604.7328338743\\
69.5	0.18816	-639.780336876091\\
69.5	0.19182	-676.047457478662\\
69.5	0.19548	-713.534195682014\\
69.5	0.19914	-752.240551486147\\
69.5	0.2028	-792.166524891061\\
69.5	0.20646	-833.312115896756\\
69.5	0.21012	-875.677324503231\\
69.5	0.21378	-919.262150710488\\
69.5	0.21744	-964.066594518525\\
69.5	0.2211	-1010.09065592734\\
69.5	0.22476	-1057.33433493694\\
69.5	0.22842	-1105.79763154732\\
69.5	0.23208	-1155.48054575848\\
69.5	0.23574	-1206.38307757043\\
69.5	0.2394	-1258.50522698315\\
69.5	0.24306	-1311.84699399665\\
69.5	0.24672	-1366.40837861094\\
69.5	0.25038	-1422.189380826\\
69.5	0.25404	-1479.19000064185\\
69.5	0.2577	-1537.41023805848\\
69.5	0.26136	-1596.85009307589\\
69.5	0.26502	-1657.50956569408\\
69.5	0.26868	-1719.38865591305\\
69.5	0.27234	-1782.4873637328\\
69.5	0.276	-1846.80568915333\\
69.875	0.093	-127.741744923754\\
69.875	0.09666	-132.32831038386\\
69.875	0.10032	-138.134493444747\\
69.875	0.10398	-145.160294106415\\
69.875	0.10764	-153.405712368864\\
69.875	0.1113	-162.870748232094\\
69.875	0.11496	-173.555401696105\\
69.875	0.11862	-185.459672760896\\
69.875	0.12228	-198.583561426469\\
69.875	0.12594	-212.927067692822\\
69.875	0.1296	-228.490191559956\\
69.875	0.13326	-245.272933027872\\
69.875	0.13692	-263.275292096568\\
69.875	0.14058	-282.497268766045\\
69.875	0.14424	-302.938863036303\\
69.875	0.1479	-324.600074907341\\
69.875	0.15156	-347.480904379162\\
69.875	0.15522	-371.581351451762\\
69.875	0.15888	-396.901416125144\\
69.875	0.16254	-423.441098399307\\
69.875	0.1662	-451.200398274251\\
69.875	0.16986	-480.179315749975\\
69.875	0.17352	-510.37785082648\\
69.875	0.17718	-541.796003503766\\
69.875	0.18084	-574.433773781833\\
69.875	0.1845	-608.291161660681\\
69.875	0.18816	-643.36816714031\\
69.875	0.19182	-679.66479022072\\
69.875	0.19548	-717.181030901911\\
69.875	0.19914	-755.916889183883\\
69.875	0.2028	-795.872365066635\\
69.875	0.20646	-837.047458550169\\
69.875	0.21012	-879.442169634483\\
69.875	0.21378	-923.056498319579\\
69.875	0.21744	-967.890444605454\\
69.875	0.2211	-1013.94400849211\\
69.875	0.22476	-1061.21718997955\\
69.875	0.22842	-1109.70998906777\\
69.875	0.23208	-1159.42240575677\\
69.875	0.23574	-1210.35444004655\\
69.875	0.2394	-1262.50609193711\\
69.875	0.24306	-1315.87736142845\\
69.875	0.24672	-1370.46824852058\\
69.875	0.25038	-1426.27875321348\\
69.875	0.25404	-1483.30887550717\\
69.875	0.2577	-1541.55861540163\\
69.875	0.26136	-1601.02797289688\\
69.875	0.26502	-1661.71694799291\\
69.875	0.26868	-1723.62554068972\\
69.875	0.27234	-1786.75375098731\\
69.875	0.276	-1851.10157888568\\
70.25	0.093	-130.701114371285\\
70.25	0.09666	-135.31718230923\\
70.25	0.10032	-141.152867847956\\
70.25	0.10398	-148.208170987463\\
70.25	0.10764	-156.483091727751\\
70.25	0.1113	-165.977630068819\\
70.25	0.11496	-176.691786010669\\
70.25	0.11862	-188.625559553299\\
70.25	0.12228	-201.77895069671\\
70.25	0.12594	-216.151959440903\\
70.25	0.1296	-231.744585785876\\
70.25	0.13326	-248.556829731629\\
70.25	0.13692	-266.588691278165\\
70.25	0.14058	-285.84017042548\\
70.25	0.14424	-306.311267173578\\
70.25	0.1479	-328.001981522455\\
70.25	0.15156	-350.912313472114\\
70.25	0.15522	-375.042263022553\\
70.25	0.15888	-400.391830173774\\
70.25	0.16254	-426.961014925775\\
70.25	0.1662	-454.749817278558\\
70.25	0.16986	-483.758237232121\\
70.25	0.17352	-513.986274786464\\
70.25	0.17718	-545.433929941589\\
70.25	0.18084	-578.101202697495\\
70.25	0.1845	-611.988093054182\\
70.25	0.18816	-647.09460101165\\
70.25	0.19182	-683.420726569899\\
70.25	0.19548	-720.966469728928\\
70.25	0.19914	-759.731830488739\\
70.25	0.2028	-799.71680884933\\
70.25	0.20646	-840.921404810703\\
70.25	0.21012	-883.345618372855\\
70.25	0.21378	-926.98944953579\\
70.25	0.21744	-971.852898299504\\
70.25	0.2211	-1017.935964664\\
70.25	0.22476	-1065.23864862928\\
70.25	0.22842	-1113.76095019534\\
70.25	0.23208	-1163.50286936217\\
70.25	0.23574	-1214.46440612979\\
70.25	0.2394	-1266.64556049819\\
70.25	0.24306	-1320.04633246738\\
70.25	0.24672	-1374.66672203734\\
70.25	0.25038	-1430.50672920808\\
70.25	0.25404	-1487.56635397961\\
70.25	0.2577	-1545.84559635191\\
70.25	0.26136	-1605.344456325\\
70.25	0.26502	-1666.06293389886\\
70.25	0.26868	-1728.00102907351\\
70.25	0.27234	-1791.15874184894\\
70.25	0.276	-1855.53607222515\\
70.625	0.093	-133.799087425935\\
70.625	0.09666	-138.444657841719\\
70.625	0.10032	-144.309845858284\\
70.625	0.10398	-151.394651475629\\
70.625	0.10764	-159.699074693756\\
70.625	0.1113	-169.223115512663\\
70.625	0.11496	-179.966773932352\\
70.625	0.11862	-191.930049952821\\
70.625	0.12228	-205.112943574071\\
70.625	0.12594	-219.515454796102\\
70.625	0.1296	-235.137583618913\\
70.625	0.13326	-251.979330042506\\
70.625	0.13692	-270.04069406688\\
70.625	0.14058	-289.321675692034\\
70.625	0.14424	-309.822274917971\\
70.625	0.1479	-331.542491744686\\
70.625	0.15156	-354.482326172184\\
70.625	0.15522	-378.641778200462\\
70.625	0.15888	-404.020847829521\\
70.625	0.16254	-430.619535059362\\
70.625	0.1662	-458.437839889983\\
70.625	0.16986	-487.475762321385\\
70.625	0.17352	-517.733302353568\\
70.625	0.17718	-549.210459986531\\
70.625	0.18084	-581.907235220276\\
70.625	0.1845	-615.823628054801\\
70.625	0.18816	-650.959638490108\\
70.625	0.19182	-687.315266526195\\
70.625	0.19548	-724.890512163064\\
70.625	0.19914	-763.685375400713\\
70.625	0.2028	-803.699856239143\\
70.625	0.20646	-844.933954678355\\
70.625	0.21012	-887.387670718346\\
70.625	0.21378	-931.061004359119\\
70.625	0.21744	-975.953955600673\\
70.625	0.2211	-1022.06652444301\\
70.625	0.22476	-1069.39871088612\\
70.625	0.22842	-1117.95051493002\\
70.625	0.23208	-1167.7219365747\\
70.625	0.23574	-1218.71297582016\\
70.625	0.2394	-1270.92363266639\\
70.625	0.24306	-1324.35390711342\\
70.625	0.24672	-1379.00379916122\\
70.625	0.25038	-1434.8733088098\\
70.625	0.25404	-1491.96243605916\\
70.625	0.2577	-1550.27118090931\\
70.625	0.26136	-1609.79954336023\\
70.625	0.26502	-1670.54752341193\\
70.625	0.26868	-1732.51512106442\\
70.625	0.27234	-1795.70233631769\\
70.625	0.276	-1860.10916917174\\
71	0.093	-137.035664087704\\
71	0.09666	-141.710736981327\\
71	0.10032	-147.60542747573\\
71	0.10398	-154.719735570914\\
71	0.10764	-163.053661266879\\
71	0.1113	-172.607204563626\\
71	0.11496	-183.380365461153\\
71	0.11862	-195.373143959461\\
71	0.12228	-208.585540058549\\
71	0.12594	-223.017553758419\\
71	0.1296	-238.66918505907\\
71	0.13326	-255.540433960501\\
71	0.13692	-273.631300462714\\
71	0.14058	-292.941784565707\\
71	0.14424	-313.471886269482\\
71	0.1479	-335.221605574036\\
71	0.15156	-358.190942479373\\
71	0.15522	-382.37989698549\\
71	0.15888	-407.788469092388\\
71	0.16254	-434.416658800067\\
71	0.1662	-462.264466108527\\
71	0.16986	-491.331891017767\\
71	0.17352	-521.618933527789\\
71	0.17718	-553.125593638591\\
71	0.18084	-585.851871350175\\
71	0.1845	-619.797766662539\\
71	0.18816	-654.963279575685\\
71	0.19182	-691.348410089611\\
71	0.19548	-728.953158204318\\
71	0.19914	-767.777523919806\\
71	0.2028	-807.821507236075\\
71	0.20646	-849.085108153125\\
71	0.21012	-891.568326670955\\
71	0.21378	-935.271162789567\\
71	0.21744	-980.193616508959\\
71	0.2211	-1026.33568782913\\
71	0.22476	-1073.69737675009\\
71	0.22842	-1122.27868327182\\
71	0.23208	-1172.07960739434\\
71	0.23574	-1223.10014911764\\
71	0.2394	-1275.34030844171\\
71	0.24306	-1328.80008536657\\
71	0.24672	-1383.47947989221\\
71	0.25038	-1439.37849201863\\
71	0.25404	-1496.49712174584\\
71	0.2577	-1554.83536907382\\
71	0.26136	-1614.39323400258\\
71	0.26502	-1675.17071653213\\
71	0.26868	-1737.16781666245\\
71	0.27234	-1800.38453439356\\
71	0.276	-1864.82086972544\\
71.375	0.093	-140.410844356592\\
71.375	0.09666	-145.115419728054\\
71.375	0.10032	-151.039612700296\\
71.375	0.10398	-158.183423273318\\
71.375	0.10764	-166.546851447123\\
71.375	0.1113	-176.129897221708\\
71.375	0.11496	-186.932560597073\\
71.375	0.11862	-198.95484157322\\
71.375	0.12228	-212.196740150148\\
71.375	0.12594	-226.658256327856\\
71.375	0.1296	-242.339390106345\\
71.375	0.13326	-259.240141485616\\
71.375	0.13692	-277.360510465667\\
71.375	0.14058	-296.700497046499\\
71.375	0.14424	-317.260101228113\\
71.375	0.1479	-339.039323010506\\
71.375	0.15156	-362.038162393681\\
71.375	0.15522	-386.256619377637\\
71.375	0.15888	-411.694693962374\\
71.375	0.16254	-438.352386147892\\
71.375	0.1662	-466.229695934191\\
71.375	0.16986	-495.32662332127\\
71.375	0.17352	-525.64316830913\\
71.375	0.17718	-557.179330897771\\
71.375	0.18084	-589.935111087193\\
71.375	0.1845	-623.910508877396\\
71.375	0.18816	-659.105524268381\\
71.375	0.19182	-695.520157260146\\
71.375	0.19548	-733.154407852692\\
71.375	0.19914	-772.008276046018\\
71.375	0.2028	-812.081761840126\\
71.375	0.20646	-853.374865235015\\
71.375	0.21012	-895.887586230684\\
71.375	0.21378	-939.619924827135\\
71.375	0.21744	-984.571881024365\\
71.375	0.2211	-1030.74345482238\\
71.375	0.22476	-1078.13464622117\\
71.375	0.22842	-1126.74545522075\\
71.375	0.23208	-1176.5758818211\\
71.375	0.23574	-1227.62592602224\\
71.375	0.2394	-1279.89558782415\\
71.375	0.24306	-1333.38486722685\\
71.375	0.24672	-1388.09376423033\\
71.375	0.25038	-1444.02227883459\\
71.375	0.25404	-1501.17041103963\\
71.375	0.2577	-1559.53816084545\\
71.375	0.26136	-1619.12552825205\\
71.375	0.26502	-1679.93251325944\\
71.375	0.26868	-1741.9591158676\\
71.375	0.27234	-1805.20533607655\\
71.375	0.276	-1869.67117388627\\
71.75	0.093	-143.924628232599\\
71.75	0.09666	-148.658706081899\\
71.75	0.10032	-154.612401531981\\
71.75	0.10398	-161.785714582842\\
71.75	0.10764	-170.178645234485\\
71.75	0.1113	-179.791193486909\\
71.75	0.11496	-190.623359340113\\
71.75	0.11862	-202.675142794099\\
71.75	0.12228	-215.946543848865\\
71.75	0.12594	-230.437562504413\\
71.75	0.1296	-246.148198760741\\
71.75	0.13326	-263.07845261785\\
71.75	0.13692	-281.228324075739\\
71.75	0.14058	-300.597813134411\\
71.75	0.14424	-321.186919793863\\
71.75	0.1479	-342.995644054095\\
71.75	0.15156	-366.023985915109\\
71.75	0.15522	-390.271945376903\\
71.75	0.15888	-415.739522439479\\
71.75	0.16254	-442.426717102836\\
71.75	0.1662	-470.333529366973\\
71.75	0.16986	-499.459959231891\\
71.75	0.17352	-529.80600669759\\
71.75	0.17718	-561.37167176407\\
71.75	0.18084	-594.156954431332\\
71.75	0.1845	-628.161854699373\\
71.75	0.18816	-663.386372568196\\
71.75	0.19182	-699.8305080378\\
71.75	0.19548	-737.494261108185\\
71.75	0.19914	-776.37763177935\\
71.75	0.2028	-816.480620051296\\
71.75	0.20646	-857.803225924024\\
71.75	0.21012	-900.345449397532\\
71.75	0.21378	-944.107290471822\\
71.75	0.21744	-989.088749146891\\
71.75	0.2211	-1035.28982542274\\
71.75	0.22476	-1082.71051929937\\
71.75	0.22842	-1131.35083077679\\
71.75	0.23208	-1181.21075985498\\
71.75	0.23574	-1232.29030653396\\
71.75	0.2394	-1284.58947081371\\
71.75	0.24306	-1338.10825269425\\
71.75	0.24672	-1392.84665217556\\
71.75	0.25038	-1448.80466925766\\
71.75	0.25404	-1505.98230394054\\
71.75	0.2577	-1564.3795562242\\
71.75	0.26136	-1623.99642610864\\
71.75	0.26502	-1684.83291359387\\
71.75	0.26868	-1746.88901867987\\
71.75	0.27234	-1810.16474136665\\
71.75	0.276	-1874.66008165422\\
72.125	0.093	-147.577015715726\\
72.125	0.09666	-152.340596042865\\
72.125	0.10032	-158.323793970785\\
72.125	0.10398	-165.526609499485\\
72.125	0.10764	-173.949042628967\\
72.125	0.1113	-183.59109335923\\
72.125	0.11496	-194.452761690273\\
72.125	0.11862	-206.534047622097\\
72.125	0.12228	-219.834951154702\\
72.125	0.12594	-234.355472288088\\
72.125	0.1296	-250.095611022255\\
72.125	0.13326	-267.055367357203\\
72.125	0.13692	-285.234741292932\\
72.125	0.14058	-304.633732829441\\
72.125	0.14424	-325.252341966733\\
72.125	0.1479	-347.090568704803\\
72.125	0.15156	-370.148413043656\\
72.125	0.15522	-394.425874983289\\
72.125	0.15888	-419.922954523704\\
72.125	0.16254	-446.639651664899\\
72.125	0.1662	-474.575966406876\\
72.125	0.16986	-503.731898749633\\
72.125	0.17352	-534.10744869317\\
72.125	0.17718	-565.702616237489\\
72.125	0.18084	-598.517401382589\\
72.125	0.1845	-632.551804128469\\
72.125	0.18816	-667.805824475131\\
72.125	0.19182	-704.279462422574\\
72.125	0.19548	-741.972717970797\\
72.125	0.19914	-780.885591119801\\
72.125	0.2028	-821.018081869586\\
72.125	0.20646	-862.370190220153\\
72.125	0.21012	-904.9419161715\\
72.125	0.21378	-948.733259723628\\
72.125	0.21744	-993.744220876536\\
72.125	0.2211	-1039.97479963023\\
72.125	0.22476	-1087.4249959847\\
72.125	0.22842	-1136.09480993995\\
72.125	0.23208	-1185.98424149598\\
72.125	0.23574	-1237.0932906528\\
72.125	0.2394	-1289.42195741039\\
72.125	0.24306	-1342.97024176876\\
72.125	0.24672	-1397.73814372792\\
72.125	0.25038	-1453.72566328786\\
72.125	0.25404	-1510.93280044858\\
72.125	0.2577	-1569.35955521007\\
72.125	0.26136	-1629.00592757235\\
72.125	0.26502	-1689.87191753542\\
72.125	0.26868	-1751.95752509926\\
72.125	0.27234	-1815.26275026388\\
72.125	0.276	-1879.78759302928\\
72.5	0.093	-151.368006805972\\
72.5	0.09666	-156.161089610949\\
72.5	0.10032	-162.173790016708\\
72.5	0.10398	-169.406108023247\\
72.5	0.10764	-177.858043630568\\
72.5	0.1113	-187.529596838669\\
72.5	0.11496	-198.420767647551\\
72.5	0.11862	-210.531556057214\\
72.5	0.12228	-223.861962067658\\
72.5	0.12594	-238.411985678882\\
72.5	0.1296	-254.181626890888\\
72.5	0.13326	-271.170885703675\\
72.5	0.13692	-289.379762117242\\
72.5	0.14058	-308.808256131591\\
72.5	0.14424	-329.456367746721\\
72.5	0.1479	-351.32409696263\\
72.5	0.15156	-374.411443779322\\
72.5	0.15522	-398.718408196794\\
72.5	0.15888	-424.244990215047\\
72.5	0.16254	-450.991189834081\\
72.5	0.1662	-478.957007053897\\
72.5	0.16986	-508.142441874492\\
72.5	0.17352	-538.547494295869\\
72.5	0.17718	-570.172164318026\\
72.5	0.18084	-603.016451940965\\
72.5	0.1845	-637.080357164684\\
72.5	0.18816	-672.363879989184\\
72.5	0.19182	-708.867020414466\\
72.5	0.19548	-746.589778440528\\
72.5	0.19914	-785.532154067371\\
72.5	0.2028	-825.694147294995\\
72.5	0.20646	-867.0757581234\\
72.5	0.21012	-909.676986552586\\
72.5	0.21378	-953.497832582553\\
72.5	0.21744	-998.5382962133\\
72.5	0.2211	-1044.79837744483\\
72.5	0.22476	-1092.27807627714\\
72.5	0.22842	-1140.97739271023\\
72.5	0.23208	-1190.8963267441\\
72.5	0.23574	-1242.03487837875\\
72.5	0.2394	-1294.39304761419\\
72.5	0.24306	-1347.9708344504\\
72.5	0.24672	-1402.76823888739\\
72.5	0.25038	-1458.78526092517\\
72.5	0.25404	-1516.02190056373\\
72.5	0.2577	-1574.47815780306\\
72.5	0.26136	-1634.15403264318\\
72.5	0.26502	-1695.04952508408\\
72.5	0.26868	-1757.16463512576\\
72.5	0.27234	-1820.49936276822\\
72.5	0.276	-1885.05370801147\\
72.875	0.093	-155.297601503336\\
72.875	0.09666	-160.120186786153\\
72.875	0.10032	-166.16238966975\\
72.875	0.10398	-173.424210154128\\
72.875	0.10764	-181.905648239287\\
72.875	0.1113	-191.606703925227\\
72.875	0.11496	-202.527377211948\\
72.875	0.11862	-214.66766809945\\
72.875	0.12228	-228.027576587732\\
72.875	0.12594	-242.607102676796\\
72.875	0.1296	-258.406246366641\\
72.875	0.13326	-275.425007657266\\
72.875	0.13692	-293.663386548672\\
72.875	0.14058	-313.121383040859\\
72.875	0.14424	-333.798997133828\\
72.875	0.1479	-355.696228827577\\
72.875	0.15156	-378.813078122107\\
72.875	0.15522	-403.149545017417\\
72.875	0.15888	-428.70562951351\\
72.875	0.16254	-455.481331610383\\
72.875	0.1662	-483.476651308037\\
72.875	0.16986	-512.691588606471\\
72.875	0.17352	-543.126143505686\\
72.875	0.17718	-574.780316005682\\
72.875	0.18084	-607.65410610646\\
72.875	0.1845	-641.747513808018\\
72.875	0.18816	-677.060539110357\\
72.875	0.19182	-713.593182013477\\
72.875	0.19548	-751.345442517378\\
72.875	0.19914	-790.31732062206\\
72.875	0.2028	-830.508816327523\\
72.875	0.20646	-871.919929633767\\
72.875	0.21012	-914.550660540791\\
72.875	0.21378	-958.401009048597\\
72.875	0.21744	-1003.47097515718\\
72.875	0.2211	-1049.76055886655\\
72.875	0.22476	-1097.2697601767\\
72.875	0.22842	-1145.99857908763\\
72.875	0.23208	-1195.94701559934\\
72.875	0.23574	-1247.11506971183\\
72.875	0.2394	-1299.5027414251\\
72.875	0.24306	-1353.11003073915\\
72.875	0.24672	-1407.93693765399\\
72.875	0.25038	-1463.9834621696\\
72.875	0.25404	-1521.249604286\\
72.875	0.2577	-1579.73536400317\\
72.875	0.26136	-1639.44074132113\\
72.875	0.26502	-1700.36573623987\\
72.875	0.26868	-1762.51034875939\\
72.875	0.27234	-1825.87457887969\\
72.875	0.276	-1890.45842660077\\
73.25	0.093	-159.36579980782\\
73.25	0.09666	-164.217887568475\\
73.25	0.10032	-170.289592929911\\
73.25	0.10398	-177.580915892128\\
73.25	0.10764	-186.091856455126\\
73.25	0.1113	-195.822414618904\\
73.25	0.11496	-206.772590383464\\
73.25	0.11862	-218.942383748805\\
73.25	0.12228	-232.331794714926\\
73.25	0.12594	-246.940823281829\\
73.25	0.1296	-262.769469449512\\
73.25	0.13326	-279.817733217976\\
73.25	0.13692	-298.085614587221\\
73.25	0.14058	-317.573113557247\\
73.25	0.14424	-338.280230128054\\
73.25	0.1479	-360.206964299642\\
73.25	0.15156	-383.353316072011\\
73.25	0.15522	-407.71928544516\\
73.25	0.15888	-433.304872419091\\
73.25	0.16254	-460.110076993803\\
73.25	0.1662	-488.134899169295\\
73.25	0.16986	-517.379338945568\\
73.25	0.17352	-547.843396322623\\
73.25	0.17718	-579.527071300457\\
73.25	0.18084	-612.430363879074\\
73.25	0.1845	-646.55327405847\\
73.25	0.18816	-681.895801838649\\
73.25	0.19182	-718.457947219608\\
73.25	0.19548	-756.239710201347\\
73.25	0.19914	-795.241090783868\\
73.25	0.2028	-835.462088967169\\
73.25	0.20646	-876.902704751252\\
73.25	0.21012	-919.562938136115\\
73.25	0.21378	-963.44278912176\\
73.25	0.21744	-1008.54225770818\\
73.25	0.2211	-1054.86134389539\\
73.25	0.22476	-1102.40004768338\\
73.25	0.22842	-1151.15836907215\\
73.25	0.23208	-1201.13630806169\\
73.25	0.23574	-1252.33386465203\\
73.25	0.2394	-1304.75103884314\\
73.25	0.24306	-1358.38783063503\\
73.25	0.24672	-1413.2442400277\\
73.25	0.25038	-1469.32026702115\\
73.25	0.25404	-1526.61591161539\\
73.25	0.2577	-1585.1311738104\\
73.25	0.26136	-1644.8660536062\\
73.25	0.26502	-1705.82055100278\\
73.25	0.26868	-1767.99466600013\\
73.25	0.27234	-1831.38839859827\\
73.25	0.276	-1896.00174879719\\
73.625	0.093	-163.572601719423\\
73.625	0.09666	-168.454191957917\\
73.625	0.10032	-174.555399797192\\
73.625	0.10398	-181.876225237248\\
73.625	0.10764	-190.416668278084\\
73.625	0.1113	-200.176728919702\\
73.625	0.11496	-211.1564071621\\
73.625	0.11862	-223.35570300528\\
73.625	0.12228	-236.77461644924\\
73.625	0.12594	-251.413147493981\\
73.625	0.1296	-267.271296139503\\
73.625	0.13326	-284.349062385806\\
73.625	0.13692	-302.64644623289\\
73.625	0.14058	-322.163447680755\\
73.625	0.14424	-342.900066729401\\
73.625	0.1479	-364.856303378827\\
73.625	0.15156	-388.032157629035\\
73.625	0.15522	-412.427629480023\\
73.625	0.15888	-438.042718931793\\
73.625	0.16254	-464.877425984343\\
73.625	0.1662	-492.931750637675\\
73.625	0.16986	-522.205692891786\\
73.625	0.17352	-552.699252746679\\
73.625	0.17718	-584.412430202353\\
73.625	0.18084	-617.345225258808\\
73.625	0.1845	-651.497637916043\\
73.625	0.18816	-686.869668174061\\
73.625	0.19182	-723.461316032858\\
73.625	0.19548	-761.272581492437\\
73.625	0.19914	-800.303464552796\\
73.625	0.2028	-840.553965213936\\
73.625	0.20646	-882.024083475858\\
73.625	0.21012	-924.713819338559\\
73.625	0.21378	-968.623172802043\\
73.625	0.21744	-1013.75214386631\\
73.625	0.2211	-1060.10073253135\\
73.625	0.22476	-1107.66893879718\\
73.625	0.22842	-1156.45676266378\\
73.625	0.23208	-1206.46420413117\\
73.625	0.23574	-1257.69126319934\\
73.625	0.2394	-1310.13793986829\\
73.625	0.24306	-1363.80423413802\\
73.625	0.24672	-1418.69014600853\\
73.625	0.25038	-1474.79567547982\\
73.625	0.25404	-1532.1208225519\\
73.625	0.2577	-1590.66558722475\\
73.625	0.26136	-1650.42996949839\\
73.625	0.26502	-1711.4139693728\\
73.625	0.26868	-1773.617586848\\
73.625	0.27234	-1837.04082192398\\
73.625	0.276	-1901.68367460073\\
74	0.093	-167.918007238146\\
74	0.09666	-172.829099954478\\
74	0.10032	-178.959810271592\\
74	0.10398	-186.310138189486\\
74	0.10764	-194.880083708161\\
74	0.1113	-204.669646827618\\
74	0.11496	-215.678827547855\\
74	0.11862	-227.907625868873\\
74	0.12228	-241.356041790672\\
74	0.12594	-256.024075313252\\
74	0.1296	-271.911726436613\\
74	0.13326	-289.018995160754\\
74	0.13692	-307.345881485677\\
74	0.14058	-326.892385411381\\
74	0.14424	-347.658506937865\\
74	0.1479	-369.644246065131\\
74	0.15156	-392.849602793177\\
74	0.15522	-417.274577122004\\
74	0.15888	-442.919169051612\\
74	0.16254	-469.783378582002\\
74	0.1662	-497.867205713172\\
74	0.16986	-527.170650445122\\
74	0.17352	-557.693712777855\\
74	0.17718	-589.436392711367\\
74	0.18084	-622.39869024566\\
74	0.1845	-656.580605380735\\
74	0.18816	-691.98213811659\\
74	0.19182	-728.603288453227\\
74	0.19548	-766.444056390644\\
74	0.19914	-805.504441928842\\
74	0.2028	-845.784445067821\\
74	0.20646	-887.284065807582\\
74	0.21012	-930.003304148122\\
74	0.21378	-973.942160089444\\
74	0.21744	-1019.10063363155\\
74	0.2211	-1065.47872477443\\
74	0.22476	-1113.0764335181\\
74	0.22842	-1161.89375986254\\
74	0.23208	-1211.93070380777\\
74	0.23574	-1263.18726535377\\
74	0.2394	-1315.66344450056\\
74	0.24306	-1369.35924124813\\
74	0.24672	-1424.27465559648\\
74	0.25038	-1480.40968754561\\
74	0.25404	-1537.76433709553\\
74	0.2577	-1596.33860424622\\
74	0.26136	-1656.13248899769\\
74	0.26502	-1717.14599134995\\
74	0.26868	-1779.37911130298\\
74	0.27234	-1842.8318488568\\
74	0.276	-1907.50420401139\\
};
\end{axis}

\begin{axis}[%
width=4.527496cm,
height=3.870968cm,
at={(6.483547cm,5.376344cm)},
scale only axis,
xmin=56,
xmax=74,
tick align=outside,
xlabel={$L_{cut}$},
xmajorgrids,
ymin=0.093,
ymax=0.276,
ylabel={$D_{rlx}$},
ymajorgrids,
zmin=-1328.01931269086,
zmax=0,
zlabel={$x_4$},
zmajorgrids,
view={-140}{50},
legend style={at={(1.03,1)},anchor=north west,legend cell align=left,align=left,draw=white!15!black}
]
\addplot3[only marks,mark=*,mark options={},mark size=1.5000pt,color=mycolor1] plot table[row sep=crcr,]{%
74	0.123	-171.244880190644\\
72	0.113	-139.217460396701\\
61	0.095	-69.6051966848396\\
56	0.093	-73.6870961160989\\
};
\addplot3[only marks,mark=*,mark options={},mark size=1.5000pt,color=mycolor2] plot table[row sep=crcr,]{%
67	0.276	-1273.01999820883\\
66	0.255	-1046.38133574319\\
62	0.209	-579.722288788632\\
57	0.193	-465.590859275979\\
};
\addplot3[only marks,mark=*,mark options={},mark size=1.5000pt,color=black] plot table[row sep=crcr,]{%
69	0.104	-102.299172261903\\
};
\addplot3[only marks,mark=*,mark options={},mark size=1.5000pt,color=black] plot table[row sep=crcr,]{%
64	0.23	-773.899909377579\\
};

\addplot3[%
surf,
opacity=0.7,
shader=interp,
colormap={mymap}{[1pt] rgb(0pt)=(0.0901961,0.239216,0.0745098); rgb(1pt)=(0.0945149,0.242058,0.0739522); rgb(2pt)=(0.0988592,0.244894,0.0733566); rgb(3pt)=(0.103229,0.247724,0.0727241); rgb(4pt)=(0.107623,0.250549,0.0720557); rgb(5pt)=(0.112043,0.253367,0.0713525); rgb(6pt)=(0.116487,0.25618,0.0706154); rgb(7pt)=(0.120956,0.258986,0.0698456); rgb(8pt)=(0.125449,0.261787,0.0690441); rgb(9pt)=(0.129967,0.264581,0.0682118); rgb(10pt)=(0.134508,0.26737,0.06735); rgb(11pt)=(0.139074,0.270152,0.0664596); rgb(12pt)=(0.143663,0.272929,0.0655416); rgb(13pt)=(0.148275,0.275699,0.0645971); rgb(14pt)=(0.152911,0.278463,0.0636271); rgb(15pt)=(0.15757,0.281221,0.0626328); rgb(16pt)=(0.162252,0.283973,0.0616151); rgb(17pt)=(0.166957,0.286719,0.060575); rgb(18pt)=(0.171685,0.289458,0.0595136); rgb(19pt)=(0.176434,0.292191,0.0584321); rgb(20pt)=(0.181207,0.294918,0.0573313); rgb(21pt)=(0.186001,0.297639,0.0562123); rgb(22pt)=(0.190817,0.300353,0.0550763); rgb(23pt)=(0.195655,0.303061,0.0539242); rgb(24pt)=(0.200514,0.305763,0.052757); rgb(25pt)=(0.205395,0.308459,0.0515759); rgb(26pt)=(0.210296,0.311149,0.0503624); rgb(27pt)=(0.215212,0.313846,0.0490067); rgb(28pt)=(0.220142,0.316548,0.0475043); rgb(29pt)=(0.22509,0.319254,0.0458704); rgb(30pt)=(0.230056,0.321962,0.0441205); rgb(31pt)=(0.235042,0.324671,0.04227); rgb(32pt)=(0.240048,0.327379,0.0403343); rgb(33pt)=(0.245078,0.330085,0.0383287); rgb(34pt)=(0.250131,0.332786,0.0362688); rgb(35pt)=(0.25521,0.335482,0.0341698); rgb(36pt)=(0.260317,0.33817,0.0320472); rgb(37pt)=(0.265451,0.340849,0.0299163); rgb(38pt)=(0.270616,0.343517,0.0277927); rgb(39pt)=(0.275813,0.346172,0.0256916); rgb(40pt)=(0.281043,0.348814,0.0236284); rgb(41pt)=(0.286307,0.35144,0.0216186); rgb(42pt)=(0.291607,0.354048,0.0196776); rgb(43pt)=(0.296945,0.356637,0.0178207); rgb(44pt)=(0.302322,0.359206,0.0160634); rgb(45pt)=(0.307739,0.361753,0.0144211); rgb(46pt)=(0.313198,0.364275,0.0129091); rgb(47pt)=(0.318701,0.366772,0.0115428); rgb(48pt)=(0.324249,0.369242,0.0103377); rgb(49pt)=(0.329843,0.371682,0.00930909); rgb(50pt)=(0.335485,0.374093,0.00847245); rgb(51pt)=(0.341176,0.376471,0.00784314); rgb(52pt)=(0.346925,0.378826,0.00732741); rgb(53pt)=(0.352735,0.381168,0.00682184); rgb(54pt)=(0.358605,0.383497,0.00632729); rgb(55pt)=(0.364532,0.385812,0.00584464); rgb(56pt)=(0.370516,0.388113,0.00537476); rgb(57pt)=(0.376552,0.390399,0.00491852); rgb(58pt)=(0.38264,0.39267,0.00447681); rgb(59pt)=(0.388777,0.394925,0.00405048); rgb(60pt)=(0.394962,0.397164,0.00364042); rgb(61pt)=(0.401191,0.399386,0.00324749); rgb(62pt)=(0.407464,0.401592,0.00287258); rgb(63pt)=(0.413777,0.40378,0.00251655); rgb(64pt)=(0.420129,0.40595,0.00218028); rgb(65pt)=(0.426518,0.408102,0.00186463); rgb(66pt)=(0.432942,0.410234,0.00157049); rgb(67pt)=(0.439399,0.412348,0.00129873); rgb(68pt)=(0.445885,0.414441,0.00105022); rgb(69pt)=(0.452401,0.416515,0.000825833); rgb(70pt)=(0.458942,0.418567,0.000626441); rgb(71pt)=(0.465508,0.420599,0.00045292); rgb(72pt)=(0.472096,0.422609,0.000306141); rgb(73pt)=(0.478704,0.424596,0.000186979); rgb(74pt)=(0.485331,0.426562,9.63073e-05); rgb(75pt)=(0.491973,0.428504,3.49981e-05); rgb(76pt)=(0.498628,0.430422,3.92506e-06); rgb(77pt)=(0.505323,0.432315,0); rgb(78pt)=(0.512206,0.434168,0); rgb(79pt)=(0.519282,0.435983,0); rgb(80pt)=(0.526529,0.437764,0); rgb(81pt)=(0.533922,0.439512,0); rgb(82pt)=(0.54144,0.441232,0); rgb(83pt)=(0.549059,0.442927,0); rgb(84pt)=(0.556756,0.444599,0); rgb(85pt)=(0.564508,0.446252,0); rgb(86pt)=(0.572292,0.447889,0); rgb(87pt)=(0.580084,0.449514,0); rgb(88pt)=(0.587863,0.451129,0); rgb(89pt)=(0.595604,0.452737,0); rgb(90pt)=(0.603284,0.454343,0); rgb(91pt)=(0.610882,0.455948,0); rgb(92pt)=(0.618373,0.457556,0); rgb(93pt)=(0.625734,0.459171,0); rgb(94pt)=(0.632943,0.460795,0); rgb(95pt)=(0.639976,0.462432,0); rgb(96pt)=(0.64681,0.464084,0); rgb(97pt)=(0.653423,0.465756,0); rgb(98pt)=(0.659791,0.46745,0); rgb(99pt)=(0.665891,0.469169,0); rgb(100pt)=(0.6717,0.470916,0); rgb(101pt)=(0.677195,0.472696,0); rgb(102pt)=(0.682353,0.47451,0); rgb(103pt)=(0.687242,0.476355,0); rgb(104pt)=(0.691952,0.478225,0); rgb(105pt)=(0.696497,0.480118,0); rgb(106pt)=(0.700887,0.482033,0); rgb(107pt)=(0.705134,0.483968,0); rgb(108pt)=(0.709251,0.485921,0); rgb(109pt)=(0.713249,0.487891,0); rgb(110pt)=(0.71714,0.489876,0); rgb(111pt)=(0.720936,0.491875,0); rgb(112pt)=(0.724649,0.493887,0); rgb(113pt)=(0.72829,0.495909,0); rgb(114pt)=(0.731872,0.49794,0); rgb(115pt)=(0.735406,0.499979,0); rgb(116pt)=(0.738904,0.502025,0); rgb(117pt)=(0.742378,0.504075,0); rgb(118pt)=(0.74584,0.506128,0); rgb(119pt)=(0.749302,0.508182,0); rgb(120pt)=(0.752775,0.510237,0); rgb(121pt)=(0.756272,0.51229,0); rgb(122pt)=(0.759804,0.514339,0); rgb(123pt)=(0.763384,0.516385,0); rgb(124pt)=(0.767022,0.518424,0); rgb(125pt)=(0.770731,0.520455,0); rgb(126pt)=(0.774523,0.522478,0); rgb(127pt)=(0.77841,0.524489,0); rgb(128pt)=(0.782391,0.526491,0); rgb(129pt)=(0.786402,0.528496,0); rgb(130pt)=(0.790431,0.530506,0); rgb(131pt)=(0.794478,0.532521,0); rgb(132pt)=(0.798541,0.534539,0); rgb(133pt)=(0.802619,0.53656,0); rgb(134pt)=(0.806712,0.538584,0); rgb(135pt)=(0.81082,0.540609,0); rgb(136pt)=(0.81494,0.542635,0); rgb(137pt)=(0.819074,0.54466,0); rgb(138pt)=(0.823219,0.546686,0); rgb(139pt)=(0.827374,0.548709,0); rgb(140pt)=(0.831541,0.55073,0); rgb(141pt)=(0.835716,0.552749,0); rgb(142pt)=(0.8399,0.554763,0); rgb(143pt)=(0.844092,0.556774,0); rgb(144pt)=(0.848292,0.558779,0); rgb(145pt)=(0.852497,0.560778,0); rgb(146pt)=(0.856708,0.562771,0); rgb(147pt)=(0.860924,0.564756,0); rgb(148pt)=(0.865143,0.566733,0); rgb(149pt)=(0.869366,0.568701,0); rgb(150pt)=(0.873592,0.57066,0); rgb(151pt)=(0.877819,0.572608,0); rgb(152pt)=(0.882047,0.574545,0); rgb(153pt)=(0.886275,0.576471,0); rgb(154pt)=(0.890659,0.578362,0); rgb(155pt)=(0.895333,0.580203,0); rgb(156pt)=(0.900258,0.581999,0); rgb(157pt)=(0.905397,0.583755,0); rgb(158pt)=(0.910711,0.585479,0); rgb(159pt)=(0.916164,0.587176,0); rgb(160pt)=(0.921717,0.588852,0); rgb(161pt)=(0.927333,0.590513,0); rgb(162pt)=(0.932974,0.592166,0); rgb(163pt)=(0.938602,0.593815,0); rgb(164pt)=(0.94418,0.595468,0); rgb(165pt)=(0.949669,0.59713,0); rgb(166pt)=(0.955033,0.598808,0); rgb(167pt)=(0.960233,0.600507,0); rgb(168pt)=(0.965232,0.602233,0); rgb(169pt)=(0.969992,0.603992,0); rgb(170pt)=(0.974475,0.605791,0); rgb(171pt)=(0.978643,0.607636,0); rgb(172pt)=(0.98246,0.609532,0); rgb(173pt)=(0.985886,0.611486,0); rgb(174pt)=(0.988885,0.613503,0); rgb(175pt)=(0.991419,0.61559,0); rgb(176pt)=(0.99345,0.617753,0); rgb(177pt)=(0.99494,0.619997,0); rgb(178pt)=(0.995851,0.622329,0); rgb(179pt)=(0.996226,0.624763,0); rgb(180pt)=(0.996512,0.627352,0); rgb(181pt)=(0.996788,0.630095,0); rgb(182pt)=(0.997053,0.632982,0); rgb(183pt)=(0.997308,0.636004,0); rgb(184pt)=(0.997552,0.639152,0); rgb(185pt)=(0.997785,0.642416,0); rgb(186pt)=(0.998006,0.645786,0); rgb(187pt)=(0.998217,0.649253,0); rgb(188pt)=(0.998416,0.652807,0); rgb(189pt)=(0.998605,0.656439,0); rgb(190pt)=(0.998781,0.660138,0); rgb(191pt)=(0.998946,0.663897,0); rgb(192pt)=(0.9991,0.667704,0); rgb(193pt)=(0.999242,0.67155,0); rgb(194pt)=(0.999372,0.675427,0); rgb(195pt)=(0.99949,0.679323,0); rgb(196pt)=(0.999596,0.68323,0); rgb(197pt)=(0.99969,0.687139,0); rgb(198pt)=(0.999771,0.691039,0); rgb(199pt)=(0.999841,0.694921,0); rgb(200pt)=(0.999898,0.698775,0); rgb(201pt)=(0.999942,0.702592,0); rgb(202pt)=(0.999974,0.706363,0); rgb(203pt)=(0.999994,0.710077,0); rgb(204pt)=(1,0.713725,0); rgb(205pt)=(1,0.717341,0); rgb(206pt)=(1,0.720963,0); rgb(207pt)=(1,0.724591,0); rgb(208pt)=(1,0.728226,0); rgb(209pt)=(1,0.731867,0); rgb(210pt)=(1,0.735514,0); rgb(211pt)=(1,0.739167,0); rgb(212pt)=(1,0.742827,0); rgb(213pt)=(1,0.746493,0); rgb(214pt)=(1,0.750165,0); rgb(215pt)=(1,0.753843,0); rgb(216pt)=(1,0.757527,0); rgb(217pt)=(1,0.761217,0); rgb(218pt)=(1,0.764913,0); rgb(219pt)=(1,0.768615,0); rgb(220pt)=(1,0.772324,0); rgb(221pt)=(1,0.776038,0); rgb(222pt)=(1,0.779758,0); rgb(223pt)=(1,0.783484,0); rgb(224pt)=(1,0.787215,0); rgb(225pt)=(1,0.790953,0); rgb(226pt)=(1,0.794696,0); rgb(227pt)=(1,0.798445,0); rgb(228pt)=(1,0.8022,0); rgb(229pt)=(1,0.805961,0); rgb(230pt)=(1,0.809727,0); rgb(231pt)=(1,0.8135,0); rgb(232pt)=(1,0.817278,0); rgb(233pt)=(1,0.821063,0); rgb(234pt)=(1,0.824854,0); rgb(235pt)=(1,0.828652,0); rgb(236pt)=(1,0.832455,0); rgb(237pt)=(1,0.836265,0); rgb(238pt)=(1,0.840081,0); rgb(239pt)=(1,0.843903,0); rgb(240pt)=(1,0.847732,0); rgb(241pt)=(1,0.851566,0); rgb(242pt)=(1,0.855406,0); rgb(243pt)=(1,0.859253,0); rgb(244pt)=(1,0.863106,0); rgb(245pt)=(1,0.866964,0); rgb(246pt)=(1,0.870829,0); rgb(247pt)=(1,0.8747,0); rgb(248pt)=(1,0.878577,0); rgb(249pt)=(1,0.88246,0); rgb(250pt)=(1,0.886349,0); rgb(251pt)=(1,0.890243,0); rgb(252pt)=(1,0.894144,0); rgb(253pt)=(1,0.898051,0); rgb(254pt)=(1,0.901964,0); rgb(255pt)=(1,0.905882,0)},
mesh/rows=49]
table[row sep=crcr,header=false] {%
%
56	0.093	-73.0247201609133\\
56	0.09666	-76.5488775261567\\
56	0.10032	-80.8947757668137\\
56	0.10398	-86.0624148828836\\
56	0.10764	-92.0517948743666\\
56	0.1113	-98.8629157412631\\
56	0.11496	-106.495777483573\\
56	0.11862	-114.950380101296\\
56	0.12228	-124.226723594431\\
56	0.12594	-134.324807962981\\
56	0.1296	-145.244633206943\\
56	0.13326	-156.986199326319\\
56	0.13692	-169.549506321107\\
56	0.14058	-182.934554191309\\
56	0.14424	-197.141342936925\\
56	0.1479	-212.169872557953\\
56	0.15156	-228.020143054394\\
56	0.15522	-244.692154426249\\
56	0.15888	-262.185906673517\\
56	0.16254	-280.501399796198\\
56	0.1662	-299.638633794293\\
56	0.16986	-319.5976086678\\
56	0.17352	-340.378324416721\\
56	0.17718	-361.980781041055\\
56	0.18084	-384.404978540802\\
56	0.1845	-407.650916915963\\
56	0.18816	-431.718596166536\\
56	0.19182	-456.608016292523\\
56	0.19548	-482.319177293923\\
56	0.19914	-508.852079170736\\
56	0.2028	-536.206721922963\\
56	0.20646	-564.383105550602\\
56	0.21012	-593.381230053655\\
56	0.21378	-623.201095432121\\
56	0.21744	-653.842701686\\
56	0.2211	-685.306048815292\\
56	0.22476	-717.591136819998\\
56	0.22842	-750.697965700116\\
56	0.23208	-784.626535455648\\
56	0.23574	-819.376846086594\\
56	0.2394	-854.948897592952\\
56	0.24306	-891.342689974723\\
56	0.24672	-928.558223231908\\
56	0.25038	-966.595497364506\\
56	0.25404	-1005.45451237252\\
56	0.2577	-1045.13526825594\\
56	0.26136	-1085.63776501478\\
56	0.26502	-1126.96200264903\\
56	0.26868	-1169.10798115869\\
56	0.27234	-1212.07570054377\\
56	0.276	-1255.86516080426\\
56.375	0.093	-72.2605225046986\\
56.375	0.09666	-75.7966422256901\\
56.375	0.10032	-80.1545028220945\\
56.375	0.10398	-85.3341042939122\\
56.375	0.10764	-91.3354466411432\\
56.375	0.1113	-98.1585298637872\\
56.375	0.11496	-105.803353961845\\
56.375	0.11862	-114.269918935315\\
56.375	0.12228	-123.558224784199\\
56.375	0.12594	-133.668271508496\\
56.375	0.1296	-144.600059108206\\
56.375	0.13326	-156.353587583329\\
56.375	0.13692	-168.928856933866\\
56.375	0.14058	-182.325867159816\\
56.375	0.14424	-196.544618261178\\
56.375	0.1479	-211.585110237955\\
56.375	0.15156	-227.447343090144\\
56.375	0.15522	-244.131316817747\\
56.375	0.15888	-261.637031420762\\
56.375	0.16254	-279.964486899192\\
56.375	0.1662	-299.113683253033\\
56.375	0.16986	-319.084620482289\\
56.375	0.17352	-339.877298586957\\
56.375	0.17718	-361.491717567039\\
56.375	0.18084	-383.927877422534\\
56.375	0.1845	-407.185778153442\\
56.375	0.18816	-431.265419759763\\
56.375	0.19182	-456.166802241498\\
56.375	0.19548	-481.889925598646\\
56.375	0.19914	-508.434789831206\\
56.375	0.2028	-535.801394939181\\
56.375	0.20646	-563.989740922568\\
56.375	0.21012	-592.999827781368\\
56.375	0.21378	-622.831655515582\\
56.375	0.21744	-653.485224125209\\
56.375	0.2211	-684.960533610249\\
56.375	0.22476	-717.257583970702\\
56.375	0.22842	-750.376375206569\\
56.375	0.23208	-784.316907317849\\
56.375	0.23574	-819.079180304542\\
56.375	0.2394	-854.663194166648\\
56.375	0.24306	-891.068948904167\\
56.375	0.24672	-928.296444517099\\
56.375	0.25038	-966.345681005445\\
56.375	0.25404	-1005.2166583692\\
56.375	0.2577	-1044.90937660838\\
56.375	0.26136	-1085.42383572296\\
56.375	0.26502	-1126.76003571296\\
56.375	0.26868	-1168.91797657837\\
56.375	0.27234	-1211.8976583192\\
56.375	0.276	-1255.69908093543\\
56.75	0.093	-71.5673585236526\\
56.75	0.09666	-75.1154406003918\\
56.75	0.10032	-79.485263552544\\
56.75	0.10398	-84.6768273801095\\
56.75	0.10764	-90.6901320830883\\
56.75	0.1113	-97.52517766148\\
56.75	0.11496	-105.181964115285\\
56.75	0.11862	-113.660491444503\\
56.75	0.12228	-122.960759649135\\
56.75	0.12594	-133.08276872918\\
56.75	0.1296	-144.026518684638\\
56.75	0.13326	-155.792009515509\\
56.75	0.13692	-168.379241221793\\
56.75	0.14058	-181.788213803491\\
56.75	0.14424	-196.018927260601\\
56.75	0.1479	-211.071381593125\\
56.75	0.15156	-226.945576801062\\
56.75	0.15522	-243.641512884412\\
56.75	0.15888	-261.159189843176\\
56.75	0.16254	-279.498607677353\\
56.75	0.1662	-298.659766386942\\
56.75	0.16986	-318.642665971946\\
56.75	0.17352	-339.447306432362\\
56.75	0.17718	-361.073687768191\\
56.75	0.18084	-383.521809979434\\
56.75	0.1845	-406.79167306609\\
56.75	0.18816	-430.883277028159\\
56.75	0.19182	-455.796621865641\\
56.75	0.19548	-481.531707578537\\
56.75	0.19914	-508.088534166845\\
56.75	0.2028	-535.467101630567\\
56.75	0.20646	-563.667409969703\\
56.75	0.21012	-592.689459184251\\
56.75	0.21378	-622.533249274212\\
56.75	0.21744	-653.198780239587\\
56.75	0.2211	-684.686052080375\\
56.75	0.22476	-716.995064796576\\
56.75	0.22842	-750.12581838819\\
56.75	0.23208	-784.078312855218\\
56.75	0.23574	-818.852548197658\\
56.75	0.2394	-854.448524415512\\
56.75	0.24306	-890.866241508779\\
56.75	0.24672	-928.105699477459\\
56.75	0.25038	-966.166898321553\\
56.75	0.25404	-1005.04983804106\\
56.75	0.2577	-1044.75451863598\\
56.75	0.26136	-1085.28094010631\\
56.75	0.26502	-1126.62910245206\\
56.75	0.26868	-1168.79900567322\\
56.75	0.27234	-1211.79064976979\\
56.75	0.276	-1255.60403474178\\
57.125	0.093	-70.9452282177755\\
57.125	0.09666	-74.5052726502622\\
57.125	0.10032	-78.8870579581624\\
57.125	0.10398	-84.0905841414756\\
57.125	0.10764	-90.1158512002021\\
57.125	0.1113	-96.9628591343419\\
57.125	0.11496	-104.631607943894\\
57.125	0.11862	-113.122097628861\\
57.125	0.12228	-122.43432818924\\
57.125	0.12594	-132.568299625032\\
57.125	0.1296	-143.524011936238\\
57.125	0.13326	-155.301465122857\\
57.125	0.13692	-167.900659184889\\
57.125	0.14058	-181.321594122334\\
57.125	0.14424	-195.564269935193\\
57.125	0.1479	-210.628686623464\\
57.125	0.15156	-226.514844187149\\
57.125	0.15522	-243.222742626247\\
57.125	0.15888	-260.752381940759\\
57.125	0.16254	-279.103762130683\\
57.125	0.1662	-298.276883196021\\
57.125	0.16986	-318.271745136771\\
57.125	0.17352	-339.088347952935\\
57.125	0.17718	-360.726691644512\\
57.125	0.18084	-383.186776211503\\
57.125	0.1845	-406.468601653907\\
57.125	0.18816	-430.572167971724\\
57.125	0.19182	-455.497475164954\\
57.125	0.19548	-481.244523233597\\
57.125	0.19914	-507.813312177653\\
57.125	0.2028	-535.203841997123\\
57.125	0.20646	-563.416112692006\\
57.125	0.21012	-592.450124262302\\
57.125	0.21378	-622.305876708011\\
57.125	0.21744	-652.983370029133\\
57.125	0.2211	-684.482604225669\\
57.125	0.22476	-716.803579297618\\
57.125	0.22842	-749.94629524498\\
57.125	0.23208	-783.910752067755\\
57.125	0.23574	-818.696949765944\\
57.125	0.2394	-854.304888339545\\
57.125	0.24306	-890.73456778856\\
57.125	0.24672	-927.985988112988\\
57.125	0.25038	-966.059149312829\\
57.125	0.25404	-1004.95405138808\\
57.125	0.2577	-1044.67069433875\\
57.125	0.26136	-1085.20907816483\\
57.125	0.26502	-1126.56920286633\\
57.125	0.26868	-1168.75106844323\\
57.125	0.27234	-1211.75467489555\\
57.125	0.276	-1255.58002222329\\
57.5	0.093	-70.394131587067\\
57.5	0.09666	-73.9661383753014\\
57.5	0.10032	-78.3598860389494\\
57.5	0.10398	-83.5753745780103\\
57.5	0.10764	-89.6126039924844\\
57.5	0.1113	-96.4715742823719\\
57.5	0.11496	-104.152285447672\\
57.5	0.11862	-112.654737488386\\
57.5	0.12228	-121.978930404513\\
57.5	0.12594	-132.124864196054\\
57.5	0.1296	-143.092538863007\\
57.5	0.13326	-154.881954405374\\
57.5	0.13692	-167.493110823153\\
57.5	0.14058	-180.926008116346\\
57.5	0.14424	-195.180646284953\\
57.5	0.1479	-210.257025328972\\
57.5	0.15156	-226.155145248405\\
57.5	0.15522	-242.87500604325\\
57.5	0.15888	-260.41660771351\\
57.5	0.16254	-278.779950259182\\
57.5	0.1662	-297.965033680267\\
57.5	0.16986	-317.971857976766\\
57.5	0.17352	-338.800423148678\\
57.5	0.17718	-360.450729196002\\
57.5	0.18084	-382.92277611874\\
57.5	0.1845	-406.216563916892\\
57.5	0.18816	-430.332092590457\\
57.5	0.19182	-455.269362139434\\
57.5	0.19548	-481.028372563825\\
57.5	0.19914	-507.60912386363\\
57.5	0.2028	-535.011616038847\\
57.5	0.20646	-563.235849089478\\
57.5	0.21012	-592.281823015521\\
57.5	0.21378	-622.149537816978\\
57.5	0.21744	-652.838993493849\\
57.5	0.2211	-684.350190046132\\
57.5	0.22476	-716.683127473828\\
57.5	0.22842	-749.837805776938\\
57.5	0.23208	-783.814224955461\\
57.5	0.23574	-818.612385009398\\
57.5	0.2394	-854.232285938747\\
57.5	0.24306	-890.673927743509\\
57.5	0.24672	-927.937310423685\\
57.5	0.25038	-966.022433979274\\
57.5	0.25404	-1004.92929841028\\
57.5	0.2577	-1044.65790371669\\
57.5	0.26136	-1085.20824989852\\
57.5	0.26502	-1126.58033695576\\
57.5	0.26868	-1168.77416488842\\
57.5	0.27234	-1211.78973369648\\
57.5	0.276	-1255.62704337997\\
57.875	0.093	-69.914068631527\\
57.875	0.09666	-73.4980377755096\\
57.875	0.10032	-77.9037477949049\\
57.875	0.10398	-83.1311986897136\\
57.875	0.10764	-89.1803904599356\\
57.875	0.1113	-96.0513231055706\\
57.875	0.11496	-103.743996626619\\
57.875	0.11862	-112.258411023081\\
57.875	0.12228	-121.594566294956\\
57.875	0.12594	-131.752462442243\\
57.875	0.1296	-142.732099464945\\
57.875	0.13326	-154.533477363059\\
57.875	0.13692	-167.156596136587\\
57.875	0.14058	-180.601455785527\\
57.875	0.14424	-194.868056309881\\
57.875	0.1479	-209.956397709649\\
57.875	0.15156	-225.866479984829\\
57.875	0.15522	-242.598303135422\\
57.875	0.15888	-260.151867161429\\
57.875	0.16254	-278.527172062849\\
57.875	0.1662	-297.724217839682\\
57.875	0.16986	-317.743004491929\\
57.875	0.17352	-338.583532019588\\
57.875	0.17718	-360.245800422661\\
57.875	0.18084	-382.729809701147\\
57.875	0.1845	-406.035559855046\\
57.875	0.18816	-430.163050884359\\
57.875	0.19182	-455.112282789084\\
57.875	0.19548	-480.883255569223\\
57.875	0.19914	-507.475969224775\\
57.875	0.2028	-534.89042375574\\
57.875	0.20646	-563.126619162118\\
57.875	0.21012	-592.18455544391\\
57.875	0.21378	-622.064232601115\\
57.875	0.21744	-652.765650633733\\
57.875	0.2211	-684.288809541764\\
57.875	0.22476	-716.633709325208\\
57.875	0.22842	-749.800349984065\\
57.875	0.23208	-783.788731518336\\
57.875	0.23574	-818.59885392802\\
57.875	0.2394	-854.230717213117\\
57.875	0.24306	-890.684321373628\\
57.875	0.24672	-927.959666409551\\
57.875	0.25038	-966.056752320888\\
57.875	0.25404	-1004.97557910764\\
57.875	0.2577	-1044.7161467698\\
57.875	0.26136	-1085.27845530738\\
57.875	0.26502	-1126.66250472037\\
57.875	0.26868	-1168.86829500877\\
57.875	0.27234	-1211.89582617259\\
57.875	0.276	-1255.74509821181\\
58.25	0.093	-69.5050393511555\\
58.25	0.09666	-73.1009708508859\\
58.25	0.10032	-77.5186432260292\\
58.25	0.10398	-82.7580564765856\\
58.25	0.10764	-88.8192106025554\\
58.25	0.1113	-95.7021056039382\\
58.25	0.11496	-103.406741480735\\
58.25	0.11862	-111.933118232944\\
58.25	0.12228	-121.281235860567\\
58.25	0.12594	-131.451094363602\\
58.25	0.1296	-142.442693742051\\
58.25	0.13326	-154.256033995913\\
58.25	0.13692	-166.891115125188\\
58.25	0.14058	-180.347937129877\\
58.25	0.14424	-194.626500009979\\
58.25	0.1479	-209.726803765494\\
58.25	0.15156	-225.648848396422\\
58.25	0.15522	-242.392633902763\\
58.25	0.15888	-259.958160284517\\
58.25	0.16254	-278.345427541685\\
58.25	0.1662	-297.554435674266\\
58.25	0.16986	-317.585184682261\\
58.25	0.17352	-338.437674565668\\
58.25	0.17718	-360.111905324488\\
58.25	0.18084	-382.607876958722\\
58.25	0.1845	-405.925589468369\\
58.25	0.18816	-430.065042853429\\
58.25	0.19182	-455.026237113902\\
58.25	0.19548	-480.809172249789\\
58.25	0.19914	-507.413848261088\\
58.25	0.2028	-534.840265147801\\
58.25	0.20646	-563.088422909928\\
58.25	0.21012	-592.158321547467\\
58.25	0.21378	-622.049961060419\\
58.25	0.21744	-652.763341448785\\
58.25	0.2211	-684.298462712564\\
58.25	0.22476	-716.655324851756\\
58.25	0.22842	-749.833927866361\\
58.25	0.23208	-783.83427175638\\
58.25	0.23574	-818.656356521812\\
58.25	0.2394	-854.300182162656\\
58.25	0.24306	-890.765748678914\\
58.25	0.24672	-928.053056070586\\
58.25	0.25038	-966.16210433767\\
58.25	0.25404	-1005.09289348017\\
58.25	0.2577	-1044.84542349808\\
58.25	0.26136	-1085.4196943914\\
58.25	0.26502	-1126.81570616014\\
58.25	0.26868	-1169.03345880429\\
58.25	0.27234	-1212.07295232385\\
58.25	0.276	-1255.93418671883\\
58.625	0.093	-69.1670437459534\\
58.625	0.09666	-72.7749376014311\\
58.625	0.10032	-77.2045723323224\\
58.625	0.10398	-82.4559479386266\\
58.625	0.10764	-88.5290644203441\\
58.625	0.1113	-95.4239217774749\\
58.625	0.11496	-103.140520010019\\
58.625	0.11862	-111.678859117976\\
58.625	0.12228	-121.038939101346\\
58.625	0.12594	-131.22075996013\\
58.625	0.1296	-142.224321694326\\
58.625	0.13326	-154.049624303936\\
58.625	0.13692	-166.696667788959\\
58.625	0.14058	-180.165452149395\\
58.625	0.14424	-194.455977385245\\
58.625	0.1479	-209.568243496508\\
58.625	0.15156	-225.502250483184\\
58.625	0.15522	-242.257998345272\\
58.625	0.15888	-259.835487082775\\
58.625	0.16254	-278.23471669569\\
58.625	0.1662	-297.455687184019\\
58.625	0.16986	-317.498398547761\\
58.625	0.17352	-338.362850786916\\
58.625	0.17718	-360.049043901484\\
58.625	0.18084	-382.556977891466\\
58.625	0.1845	-405.88665275686\\
58.625	0.18816	-430.038068497668\\
58.625	0.19182	-455.01122511389\\
58.625	0.19548	-480.806122605523\\
58.625	0.19914	-507.422760972571\\
58.625	0.2028	-534.861140215032\\
58.625	0.20646	-563.121260332906\\
58.625	0.21012	-592.203121326193\\
58.625	0.21378	-622.106723194893\\
58.625	0.21744	-652.832065939006\\
58.625	0.2211	-684.379149558533\\
58.625	0.22476	-716.747974053473\\
58.625	0.22842	-749.938539423826\\
58.625	0.23208	-783.950845669592\\
58.625	0.23574	-818.784892790772\\
58.625	0.2394	-854.440680787364\\
58.625	0.24306	-890.91820965937\\
58.625	0.24672	-928.217479406789\\
58.625	0.25038	-966.338490029622\\
58.625	0.25404	-1005.28124152787\\
58.625	0.2577	-1045.04573390153\\
58.625	0.26136	-1085.6319671506\\
58.625	0.26502	-1127.03994127508\\
58.625	0.26868	-1169.26965627498\\
58.625	0.27234	-1212.32111215029\\
58.625	0.276	-1256.19430890102\\
59	0.093	-68.9000818159197\\
59	0.09666	-72.5199380271451\\
59	0.10032	-76.9615351137841\\
59	0.10398	-82.2248730758361\\
59	0.10764	-88.3099519133012\\
59	0.1113	-95.2167716261797\\
59	0.11496	-102.945332214471\\
59	0.11862	-111.495633678176\\
59	0.12228	-120.867676017294\\
59	0.12594	-131.061459231826\\
59	0.1296	-142.07698332177\\
59	0.13326	-153.914248287128\\
59	0.13692	-166.573254127898\\
59	0.14058	-180.054000844083\\
59	0.14424	-194.35648843568\\
59	0.1479	-209.48071690269\\
59	0.15156	-225.426686245114\\
59	0.15522	-242.194396462951\\
59	0.15888	-259.783847556201\\
59	0.16254	-278.195039524864\\
59	0.1662	-297.42797236894\\
59	0.16986	-317.48264608843\\
59	0.17352	-338.359060683333\\
59	0.17718	-360.057216153649\\
59	0.18084	-382.577112499378\\
59	0.1845	-405.918749720521\\
59	0.18816	-430.082127817076\\
59	0.19182	-455.067246789045\\
59	0.19548	-480.874106636427\\
59	0.19914	-507.502707359222\\
59	0.2028	-534.953048957431\\
59	0.20646	-563.225131431052\\
59	0.21012	-592.318954780087\\
59	0.21378	-622.234519004535\\
59	0.21744	-652.971824104396\\
59	0.2211	-684.530870079671\\
59	0.22476	-716.911656930358\\
59	0.22842	-750.114184656459\\
59	0.23208	-784.138453257973\\
59	0.23574	-818.984462734901\\
59	0.2394	-854.652213087241\\
59	0.24306	-891.141704314994\\
59	0.24672	-928.452936418161\\
59	0.25038	-966.585909396741\\
59	0.25404	-1005.54062325073\\
59	0.2577	-1045.31707798014\\
59	0.26136	-1085.91527358496\\
59	0.26502	-1127.33521006519\\
59	0.26868	-1169.57688742084\\
59	0.27234	-1212.6403056519\\
59	0.276	-1256.52546475837\\
59.375	0.093	-68.7041535610545\\
59.375	0.09666	-72.3359721280279\\
59.375	0.10032	-76.7895315704145\\
59.375	0.10398	-82.0648318882142\\
59.375	0.10764	-88.1618730814272\\
59.375	0.1113	-95.0806551500533\\
59.375	0.11496	-102.821178094093\\
59.375	0.11862	-111.383441913545\\
59.375	0.12228	-120.767446608411\\
59.375	0.12594	-130.97319217869\\
59.375	0.1296	-142.000678624383\\
59.375	0.13326	-153.849905945488\\
59.375	0.13692	-166.520874142007\\
59.375	0.14058	-180.013583213938\\
59.375	0.14424	-194.328033161283\\
59.375	0.1479	-209.464223984042\\
59.375	0.15156	-225.422155682213\\
59.375	0.15522	-242.201828255797\\
59.375	0.15888	-259.803241704795\\
59.375	0.16254	-278.226396029207\\
59.375	0.1662	-297.47129122903\\
59.375	0.16986	-317.537927304268\\
59.375	0.17352	-338.426304254918\\
59.375	0.17718	-360.136422080982\\
59.375	0.18084	-382.668280782459\\
59.375	0.1845	-406.021880359349\\
59.375	0.18816	-430.197220811653\\
59.375	0.19182	-455.19430213937\\
59.375	0.19548	-481.013124342499\\
59.375	0.19914	-507.653687421042\\
59.375	0.2028	-535.115991374998\\
59.375	0.20646	-563.400036204368\\
59.375	0.21012	-592.50582190915\\
59.375	0.21378	-622.433348489346\\
59.375	0.21744	-653.182615944955\\
59.375	0.2211	-684.753624275977\\
59.375	0.22476	-717.146373482413\\
59.375	0.22842	-750.360863564261\\
59.375	0.23208	-784.397094521523\\
59.375	0.23574	-819.255066354198\\
59.375	0.2394	-854.934779062286\\
59.375	0.24306	-891.436232645788\\
59.375	0.24672	-928.759427104702\\
59.375	0.25038	-966.90436243903\\
59.375	0.25404	-1005.87103864877\\
59.375	0.2577	-1045.65945573393\\
59.375	0.26136	-1086.26961369449\\
59.375	0.26502	-1127.70151253047\\
59.375	0.26868	-1169.95515224187\\
59.375	0.27234	-1213.03053282867\\
59.375	0.276	-1256.92765429089\\
59.75	0.093	-68.5792589813578\\
59.75	0.09666	-72.2230399040792\\
59.75	0.10032	-76.6885617022135\\
59.75	0.10398	-81.975824375761\\
59.75	0.10764	-88.0848279247218\\
59.75	0.1113	-95.0155723490956\\
59.75	0.11496	-102.768057648883\\
59.75	0.11862	-111.342283824083\\
59.75	0.12228	-120.738250874697\\
59.75	0.12594	-130.955958800724\\
59.75	0.1296	-141.995407602164\\
59.75	0.13326	-153.856597279017\\
59.75	0.13692	-166.539527831283\\
59.75	0.14058	-180.044199258963\\
59.75	0.14424	-194.370611562056\\
59.75	0.1479	-209.518764740561\\
59.75	0.15156	-225.48865879448\\
59.75	0.15522	-242.280293723813\\
59.75	0.15888	-259.893669528558\\
59.75	0.16254	-278.328786208717\\
59.75	0.1662	-297.585643764289\\
59.75	0.16986	-317.664242195274\\
59.75	0.17352	-338.564581501672\\
59.75	0.17718	-360.286661683484\\
59.75	0.18084	-382.830482740709\\
59.75	0.1845	-406.196044673347\\
59.75	0.18816	-430.383347481398\\
59.75	0.19182	-455.392391164862\\
59.75	0.19548	-481.22317572374\\
59.75	0.19914	-507.875701158031\\
59.75	0.2028	-535.349967467735\\
59.75	0.20646	-563.645974652852\\
59.75	0.21012	-592.763722713382\\
59.75	0.21378	-622.703211649325\\
59.75	0.21744	-653.464441460682\\
59.75	0.2211	-685.047412147452\\
59.75	0.22476	-717.452123709635\\
59.75	0.22842	-750.678576147232\\
59.75	0.23208	-784.726769460241\\
59.75	0.23574	-819.596703648664\\
59.75	0.2394	-855.2883787125\\
59.75	0.24306	-891.801794651749\\
59.75	0.24672	-929.136951466411\\
59.75	0.25038	-967.293849156487\\
59.75	0.25404	-1006.27248772198\\
59.75	0.2577	-1046.07286716288\\
59.75	0.26136	-1086.69498747919\\
59.75	0.26502	-1128.13884867092\\
59.75	0.26868	-1170.40445073806\\
59.75	0.27234	-1213.49179368062\\
59.75	0.276	-1257.40087749859\\
60.125	0.093	-68.5253980768301\\
60.125	0.09666	-72.1811413552992\\
60.125	0.10032	-76.6586255091813\\
60.125	0.10398	-81.9578505384765\\
60.125	0.10764	-88.0788164431851\\
60.125	0.1113	-95.0215232233066\\
60.125	0.11496	-102.785970878842\\
60.125	0.11862	-111.37215940979\\
60.125	0.12228	-120.780088816151\\
60.125	0.12594	-131.009759097926\\
60.125	0.1296	-142.061170255113\\
60.125	0.13326	-153.934322287714\\
60.125	0.13692	-166.629215195728\\
60.125	0.14058	-180.145848979156\\
60.125	0.14424	-194.484223637996\\
60.125	0.1479	-209.64433917225\\
60.125	0.15156	-225.626195581917\\
60.125	0.15522	-242.429792866997\\
60.125	0.15888	-260.05513102749\\
60.125	0.16254	-278.502210063397\\
60.125	0.1662	-297.771029974716\\
60.125	0.16986	-317.86159076145\\
60.125	0.17352	-338.773892423595\\
60.125	0.17718	-360.507934961155\\
60.125	0.18084	-383.063718374127\\
60.125	0.1845	-406.441242662513\\
60.125	0.18816	-430.640507826312\\
60.125	0.19182	-455.661513865524\\
60.125	0.19548	-481.504260780149\\
60.125	0.19914	-508.168748570188\\
60.125	0.2028	-535.65497723564\\
60.125	0.20646	-563.962946776505\\
60.125	0.21012	-593.092657192783\\
60.125	0.21378	-623.044108484474\\
60.125	0.21744	-653.817300651578\\
60.125	0.2211	-685.412233694096\\
60.125	0.22476	-717.828907612027\\
60.125	0.22842	-751.067322405371\\
60.125	0.23208	-785.127478074128\\
60.125	0.23574	-820.009374618299\\
60.125	0.2394	-855.713012037883\\
60.125	0.24306	-892.238390332879\\
60.125	0.24672	-929.585509503289\\
60.125	0.25038	-967.754369549113\\
60.125	0.25404	-1006.74497047035\\
60.125	0.2577	-1046.557312267\\
60.125	0.26136	-1087.19139493906\\
60.125	0.26502	-1128.64721848654\\
60.125	0.26868	-1170.92478290943\\
60.125	0.27234	-1214.02408820773\\
60.125	0.276	-1257.94513438145\\
60.5	0.093	-68.5425708474715\\
60.5	0.09666	-72.210276481688\\
60.5	0.10032	-76.699722991318\\
60.5	0.10398	-82.010910376361\\
60.5	0.10764	-88.1438386368173\\
60.5	0.1113	-95.0985077726868\\
60.5	0.11496	-102.874917783969\\
60.5	0.11862	-111.473068670665\\
60.5	0.12228	-120.892960432774\\
60.5	0.12594	-131.134593070297\\
60.5	0.1296	-142.197966583232\\
60.5	0.13326	-154.083080971581\\
60.5	0.13692	-166.789936235342\\
60.5	0.14058	-180.318532374518\\
60.5	0.14424	-194.668869389106\\
60.5	0.1479	-209.840947279107\\
60.5	0.15156	-225.834766044522\\
60.5	0.15522	-242.65032568535\\
60.5	0.15888	-260.287626201591\\
60.5	0.16254	-278.746667593245\\
60.5	0.1662	-298.027449860313\\
60.5	0.16986	-318.129973002793\\
60.5	0.17352	-339.054237020687\\
60.5	0.17718	-360.800241913994\\
60.5	0.18084	-383.367987682714\\
60.5	0.1845	-406.757474326848\\
60.5	0.18816	-430.968701846395\\
60.5	0.19182	-456.001670241355\\
60.5	0.19548	-481.856379511728\\
60.5	0.19914	-508.532829657514\\
60.5	0.2028	-536.031020678714\\
60.5	0.20646	-564.350952575326\\
60.5	0.21012	-593.492625347352\\
60.5	0.21378	-623.456038994791\\
60.5	0.21744	-654.241193517643\\
60.5	0.2211	-685.848088915909\\
60.5	0.22476	-718.276725189587\\
60.5	0.22842	-751.527102338679\\
60.5	0.23208	-785.599220363184\\
60.5	0.23574	-820.493079263103\\
60.5	0.2394	-856.208679038434\\
60.5	0.24306	-892.746019689178\\
60.5	0.24672	-930.105101215336\\
60.5	0.25038	-968.285923616907\\
60.5	0.25404	-1007.28848689389\\
60.5	0.2577	-1047.11279104629\\
60.5	0.26136	-1087.7588360741\\
60.5	0.26502	-1129.22662197732\\
60.5	0.26868	-1171.51614875596\\
60.5	0.27234	-1214.62741641001\\
60.5	0.276	-1258.56042493947\\
60.875	0.093	-68.6307772932809\\
60.875	0.09666	-72.3104452832451\\
60.875	0.10032	-76.8118541486229\\
60.875	0.10398	-82.1350038894137\\
60.875	0.10764	-88.2798945056177\\
60.875	0.1113	-95.246525997235\\
60.875	0.11496	-103.034898364265\\
60.875	0.11862	-111.645011606709\\
60.875	0.12228	-121.076865724566\\
60.875	0.12594	-131.330460717836\\
60.875	0.1296	-142.405796586519\\
60.875	0.13326	-154.302873330616\\
60.875	0.13692	-167.021690950125\\
60.875	0.14058	-180.562249445048\\
60.875	0.14424	-194.924548815384\\
60.875	0.1479	-210.108589061133\\
60.875	0.15156	-226.114370182296\\
60.875	0.15522	-242.941892178871\\
60.875	0.15888	-260.59115505086\\
60.875	0.16254	-279.062158798262\\
60.875	0.1662	-298.354903421078\\
60.875	0.16986	-318.469388919306\\
60.875	0.17352	-339.405615292948\\
60.875	0.17718	-361.163582542002\\
60.875	0.18084	-383.74329066647\\
60.875	0.1845	-407.144739666352\\
60.875	0.18816	-431.367929541646\\
60.875	0.19182	-456.412860292354\\
60.875	0.19548	-482.279531918475\\
60.875	0.19914	-508.967944420009\\
60.875	0.2028	-536.478097796956\\
60.875	0.20646	-564.809992049316\\
60.875	0.21012	-593.96362717709\\
60.875	0.21378	-623.939003180277\\
60.875	0.21744	-654.736120058877\\
60.875	0.2211	-686.35497781289\\
60.875	0.22476	-718.795576442316\\
60.875	0.22842	-752.057915947156\\
60.875	0.23208	-786.141996327409\\
60.875	0.23574	-821.047817583075\\
60.875	0.2394	-856.775379714154\\
60.875	0.24306	-893.324682720646\\
60.875	0.24672	-930.695726602552\\
60.875	0.25038	-968.888511359871\\
60.875	0.25404	-1007.9030369926\\
60.875	0.2577	-1047.73930350075\\
60.875	0.26136	-1088.39731088431\\
60.875	0.26502	-1129.87705914328\\
60.875	0.26868	-1172.17854827766\\
60.875	0.27234	-1215.30177828746\\
60.875	0.276	-1259.24674917267\\
61.25	0.093	-68.7900174142592\\
61.25	0.09666	-72.4816477599717\\
61.25	0.10032	-76.995018981097\\
61.25	0.10398	-82.3301310776355\\
61.25	0.10764	-88.4869840495873\\
61.25	0.1113	-95.4655778969521\\
61.25	0.11496	-103.265912619731\\
61.25	0.11862	-111.887988217922\\
61.25	0.12228	-121.331804691527\\
61.25	0.12594	-131.597362040544\\
61.25	0.1296	-142.684660264975\\
61.25	0.13326	-154.593699364819\\
61.25	0.13692	-167.324479340077\\
61.25	0.14058	-180.877000190747\\
61.25	0.14424	-195.251261916831\\
61.25	0.1479	-210.447264518328\\
61.25	0.15156	-226.465007995238\\
61.25	0.15522	-243.304492347562\\
61.25	0.15888	-260.965717575298\\
61.25	0.16254	-279.448683678448\\
61.25	0.1662	-298.753390657011\\
61.25	0.16986	-318.879838510988\\
61.25	0.17352	-339.828027240377\\
61.25	0.17718	-361.597956845179\\
61.25	0.18084	-384.189627325395\\
61.25	0.1845	-407.603038681024\\
61.25	0.18816	-431.838190912066\\
61.25	0.19182	-456.895084018522\\
61.25	0.19548	-482.77371800039\\
61.25	0.19914	-509.474092857672\\
61.25	0.2028	-536.996208590367\\
61.25	0.20646	-565.340065198475\\
61.25	0.21012	-594.505662681997\\
61.25	0.21378	-624.493001040931\\
61.25	0.21744	-655.302080275279\\
61.25	0.2211	-686.93290038504\\
61.25	0.22476	-719.385461370214\\
61.25	0.22842	-752.659763230801\\
61.25	0.23208	-786.755805966802\\
61.25	0.23574	-821.673589578216\\
61.25	0.2394	-857.413114065043\\
61.25	0.24306	-893.974379427283\\
61.25	0.24672	-931.357385664936\\
61.25	0.25038	-969.562132778003\\
61.25	0.25404	-1008.58862076648\\
61.25	0.2577	-1048.43684963038\\
61.25	0.26136	-1089.10681936968\\
61.25	0.26502	-1130.5985299844\\
61.25	0.26868	-1172.91198147453\\
61.25	0.27234	-1216.04717384008\\
61.25	0.276	-1260.00410708104\\
61.625	0.093	-69.0202912104062\\
61.625	0.09666	-72.7238839118664\\
61.625	0.10032	-77.2492174887395\\
61.625	0.10398	-82.5962919410258\\
61.625	0.10764	-88.7651072687253\\
61.625	0.1113	-95.7556634718379\\
61.625	0.11496	-103.567960550364\\
61.625	0.11862	-112.201998504303\\
61.625	0.12228	-121.657777333656\\
61.625	0.12594	-131.935297038421\\
61.625	0.1296	-143.0345576186\\
61.625	0.13326	-154.955559074192\\
61.625	0.13692	-167.698301405197\\
61.625	0.14058	-181.262784611615\\
61.625	0.14424	-195.649008693447\\
61.625	0.1479	-210.856973650692\\
61.625	0.15156	-226.886679483349\\
61.625	0.15522	-243.738126191421\\
61.625	0.15888	-261.411313774905\\
61.625	0.16254	-279.906242233803\\
61.625	0.1662	-299.222911568113\\
61.625	0.16986	-319.361321777837\\
61.625	0.17352	-340.321472862974\\
61.625	0.17718	-362.103364823525\\
61.625	0.18084	-384.706997659488\\
61.625	0.1845	-408.132371370865\\
61.625	0.18816	-432.379485957655\\
61.625	0.19182	-457.448341419858\\
61.625	0.19548	-483.338937757474\\
61.625	0.19914	-510.051274970504\\
61.625	0.2028	-537.585353058947\\
61.625	0.20646	-565.941172022803\\
61.625	0.21012	-595.118731862072\\
61.625	0.21378	-625.118032576754\\
61.625	0.21744	-655.93907416685\\
61.625	0.2211	-687.581856632358\\
61.625	0.22476	-720.04637997328\\
61.625	0.22842	-753.332644189615\\
61.625	0.23208	-787.440649281364\\
61.625	0.23574	-822.370395248525\\
61.625	0.2394	-858.1218820911\\
61.625	0.24306	-894.695109809088\\
61.625	0.24672	-932.090078402489\\
61.625	0.25038	-970.306787871303\\
61.625	0.25404	-1009.34523821553\\
61.625	0.2577	-1049.20542943517\\
61.625	0.26136	-1089.88736153023\\
61.625	0.26502	-1131.39103450069\\
61.625	0.26868	-1173.71644834657\\
61.625	0.27234	-1216.86360306787\\
61.625	0.276	-1260.83249866457\\
62	0.093	-69.3215986817225\\
62	0.09666	-73.03715373893\\
62	0.10032	-77.574449671551\\
62	0.10398	-82.933486479585\\
62	0.10764	-89.1142641630324\\
62	0.1113	-96.1167827218929\\
62	0.11496	-103.941042156166\\
62	0.11862	-112.587042465854\\
62	0.12228	-122.054783650953\\
62	0.12594	-132.344265711467\\
62	0.1296	-143.455488647393\\
62	0.13326	-155.388452458733\\
62	0.13692	-168.143157145486\\
62	0.14058	-181.719602707652\\
62	0.14424	-196.117789145231\\
62	0.1479	-211.337716458224\\
62	0.15156	-227.37938464663\\
62	0.15522	-244.242793710448\\
62	0.15888	-261.927943649681\\
62	0.16254	-280.434834464326\\
62	0.1662	-299.763466154385\\
62	0.16986	-319.913838719856\\
62	0.17352	-340.885952160741\\
62	0.17718	-362.679806477039\\
62	0.18084	-385.29540166875\\
62	0.1845	-408.732737735875\\
62	0.18816	-432.991814678412\\
62	0.19182	-458.072632496364\\
62	0.19548	-483.975191189728\\
62	0.19914	-510.699490758505\\
62	0.2028	-538.245531202695\\
62	0.20646	-566.613312522299\\
62	0.21012	-595.802834717316\\
62	0.21378	-625.814097787746\\
62	0.21744	-656.647101733589\\
62	0.2211	-688.301846554846\\
62	0.22476	-720.778332251515\\
62	0.22842	-754.076558823598\\
62	0.23208	-788.196526271094\\
62	0.23574	-823.138234594004\\
62	0.2394	-858.901683792326\\
62	0.24306	-895.486873866062\\
62	0.24672	-932.893804815211\\
62	0.25038	-971.122476639773\\
62	0.25404	-1010.17288933975\\
62	0.2577	-1050.04504291514\\
62	0.26136	-1090.73893736594\\
62	0.26502	-1132.25457269215\\
62	0.26868	-1174.59194889378\\
62	0.27234	-1217.75106597082\\
62	0.276	-1261.73192392328\\
62.375	0.093	-69.6939398282066\\
62.375	0.09666	-73.4214572411619\\
62.375	0.10032	-77.9707155295307\\
62.375	0.10398	-83.3417146933124\\
62.375	0.10764	-89.5344547325075\\
62.375	0.1113	-96.5489356471159\\
62.375	0.11496	-104.385157437137\\
62.375	0.11862	-113.043120102572\\
62.375	0.12228	-122.52282364342\\
62.375	0.12594	-132.824268059681\\
62.375	0.1296	-143.947453351355\\
62.375	0.13326	-155.892379518443\\
62.375	0.13692	-168.659046560943\\
62.375	0.14058	-182.247454478857\\
62.375	0.14424	-196.657603272184\\
62.375	0.1479	-211.889492940925\\
62.375	0.15156	-227.943123485078\\
62.375	0.15522	-244.818494904645\\
62.375	0.15888	-262.515607199625\\
62.375	0.16254	-281.034460370018\\
62.375	0.1662	-300.375054415824\\
62.375	0.16986	-320.537389337043\\
62.375	0.17352	-341.521465133676\\
62.375	0.17718	-363.327281805722\\
62.375	0.18084	-385.954839353181\\
62.375	0.1845	-409.404137776053\\
62.375	0.18816	-433.675177074338\\
62.375	0.19182	-458.767957248037\\
62.375	0.19548	-484.682478297149\\
62.375	0.19914	-511.418740221674\\
62.375	0.2028	-538.976743021612\\
62.375	0.20646	-567.356486696964\\
62.375	0.21012	-596.557971247728\\
62.375	0.21378	-626.581196673906\\
62.375	0.21744	-657.426162975497\\
62.375	0.2211	-689.092870152501\\
62.375	0.22476	-721.581318204919\\
62.375	0.22842	-754.89150713275\\
62.375	0.23208	-789.023436935993\\
62.375	0.23574	-823.977107614651\\
62.375	0.2394	-859.752519168721\\
62.375	0.24306	-896.349671598204\\
62.375	0.24672	-933.768564903101\\
62.375	0.25038	-972.009199083411\\
62.375	0.25404	-1011.07157413913\\
62.375	0.2577	-1050.95569007027\\
62.375	0.26136	-1091.66154687682\\
62.375	0.26502	-1133.18914455878\\
62.375	0.26868	-1175.53848311616\\
62.375	0.27234	-1218.70956254895\\
62.375	0.276	-1262.70238285715\\
62.75	0.093	-70.1373146498598\\
62.75	0.09666	-73.8767944185632\\
62.75	0.10032	-78.4380150626795\\
62.75	0.10398	-83.8209765822091\\
62.75	0.10764	-90.0256789771519\\
62.75	0.1113	-97.0521222475078\\
62.75	0.11496	-104.900306393277\\
62.75	0.11862	-113.570231414459\\
62.75	0.12228	-123.061897311055\\
62.75	0.12594	-133.375304083064\\
62.75	0.1296	-144.510451730486\\
62.75	0.13326	-156.467340253321\\
62.75	0.13692	-169.24596965157\\
62.75	0.14058	-182.846339925232\\
62.75	0.14424	-197.268451074306\\
62.75	0.1479	-212.512303098794\\
62.75	0.15156	-228.577895998695\\
62.75	0.15522	-245.46522977401\\
62.75	0.15888	-263.174304424738\\
62.75	0.16254	-281.705119950879\\
62.75	0.1662	-301.057676352432\\
62.75	0.16986	-321.2319736294\\
62.75	0.17352	-342.22801178178\\
62.75	0.17718	-364.045790809573\\
62.75	0.18084	-386.68531071278\\
62.75	0.1845	-410.1465714914\\
62.75	0.18816	-434.429573145434\\
62.75	0.19182	-459.53431567488\\
62.75	0.19548	-485.46079907974\\
62.75	0.19914	-512.209023360012\\
62.75	0.2028	-539.778988515698\\
62.75	0.20646	-568.170694546798\\
62.75	0.21012	-597.38414145331\\
62.75	0.21378	-627.419329235236\\
62.75	0.21744	-658.276257892574\\
62.75	0.2211	-689.954927425326\\
62.75	0.22476	-722.455337833492\\
62.75	0.22842	-755.77748911707\\
62.75	0.23208	-789.921381276062\\
62.75	0.23574	-824.887014310467\\
62.75	0.2394	-860.674388220285\\
62.75	0.24306	-897.283503005516\\
62.75	0.24672	-934.71435866616\\
62.75	0.25038	-972.966955202218\\
62.75	0.25404	-1012.04129261369\\
62.75	0.2577	-1051.93737090057\\
62.75	0.26136	-1092.65519006287\\
62.75	0.26502	-1134.19475010058\\
62.75	0.26868	-1176.5560510137\\
62.75	0.27234	-1219.73909280224\\
62.75	0.276	-1263.74387546619\\
63.125	0.093	-70.6517231466816\\
63.125	0.09666	-74.4031652711328\\
63.125	0.10032	-78.9763482709969\\
63.125	0.10398	-84.3712721462742\\
63.125	0.10764	-90.5879368969648\\
63.125	0.1113	-97.6263425230684\\
63.125	0.11496	-105.486489024586\\
63.125	0.11862	-114.168376401516\\
63.125	0.12228	-123.672004653859\\
63.125	0.12594	-133.997373781616\\
63.125	0.1296	-145.144483784786\\
63.125	0.13326	-157.113334663368\\
63.125	0.13692	-169.903926417365\\
63.125	0.14058	-183.516259046774\\
63.125	0.14424	-197.950332551596\\
63.125	0.1479	-213.206146931832\\
63.125	0.15156	-229.283702187481\\
63.125	0.15522	-246.182998318544\\
63.125	0.15888	-263.904035325019\\
63.125	0.16254	-282.446813206908\\
63.125	0.1662	-301.811331964209\\
63.125	0.16986	-321.997591596925\\
63.125	0.17352	-343.005592105052\\
63.125	0.17718	-364.835333488594\\
63.125	0.18084	-387.486815747548\\
63.125	0.1845	-410.960038881916\\
63.125	0.18816	-435.255002891697\\
63.125	0.19182	-460.371707776891\\
63.125	0.19548	-486.310153537499\\
63.125	0.19914	-513.070340173519\\
63.125	0.2028	-540.652267684953\\
63.125	0.20646	-569.0559360718\\
63.125	0.21012	-598.28134533406\\
63.125	0.21378	-628.328495471734\\
63.125	0.21744	-659.19738648482\\
63.125	0.2211	-690.88801837332\\
63.125	0.22476	-723.400391137233\\
63.125	0.22842	-756.734504776559\\
63.125	0.23208	-790.890359291298\\
63.125	0.23574	-825.867954681451\\
63.125	0.2394	-861.667290947017\\
63.125	0.24306	-898.288368087996\\
63.125	0.24672	-935.731186104387\\
63.125	0.25038	-973.995744996193\\
63.125	0.25404	-1013.08204476341\\
63.125	0.2577	-1052.99008540604\\
63.125	0.26136	-1093.71986692409\\
63.125	0.26502	-1135.27138931755\\
63.125	0.26868	-1177.64465258642\\
63.125	0.27234	-1220.8396567307\\
63.125	0.276	-1264.8564017504\\
63.5	0.093	-71.2371653186723\\
63.5	0.09666	-75.0005697988711\\
63.5	0.10032	-79.5857151544831\\
63.5	0.10398	-84.9926013855082\\
63.5	0.10764	-91.2212284919465\\
63.5	0.1113	-98.2715964737981\\
63.5	0.11496	-106.143705331063\\
63.5	0.11862	-114.837555063741\\
63.5	0.12228	-124.353145671832\\
63.5	0.12594	-134.690477155336\\
63.5	0.1296	-145.849549514254\\
63.5	0.13326	-157.830362748584\\
63.5	0.13692	-170.632916858328\\
63.5	0.14058	-184.257211843485\\
63.5	0.14424	-198.703247704056\\
63.5	0.1479	-213.971024440039\\
63.5	0.15156	-230.060542051436\\
63.5	0.15522	-246.971800538246\\
63.5	0.15888	-264.70479990047\\
63.5	0.16254	-283.259540138106\\
63.5	0.1662	-302.636021251155\\
63.5	0.16986	-322.834243239618\\
63.5	0.17352	-343.854206103494\\
63.5	0.17718	-365.695909842783\\
63.5	0.18084	-388.359354457485\\
63.5	0.1845	-411.844539947601\\
63.5	0.18816	-436.151466313129\\
63.5	0.19182	-461.280133554071\\
63.5	0.19548	-487.230541670426\\
63.5	0.19914	-514.002690662195\\
63.5	0.2028	-541.596580529376\\
63.5	0.20646	-570.012211271971\\
63.5	0.21012	-599.249582889979\\
63.5	0.21378	-629.3086953834\\
63.5	0.21744	-660.189548752234\\
63.5	0.2211	-691.892142996482\\
63.5	0.22476	-724.416478116143\\
63.5	0.22842	-757.762554111217\\
63.5	0.23208	-791.930370981704\\
63.5	0.23574	-826.919928727604\\
63.5	0.2394	-862.731227348918\\
63.5	0.24306	-899.364266845644\\
63.5	0.24672	-936.819047217784\\
63.5	0.25038	-975.095568465337\\
63.5	0.25404	-1014.1938305883\\
63.5	0.2577	-1054.11383358668\\
63.5	0.26136	-1094.85557746048\\
63.5	0.26502	-1136.41906220968\\
63.5	0.26868	-1178.8042878343\\
63.5	0.27234	-1222.01125433433\\
63.5	0.276	-1266.03996170978\\
63.875	0.093	-71.8936411658314\\
63.875	0.09666	-75.6690080017777\\
63.875	0.10032	-80.2661157131375\\
63.875	0.10398	-85.6849642999103\\
63.875	0.10764	-91.9255537620966\\
63.875	0.1113	-98.987884099696\\
63.875	0.11496	-106.871955312708\\
63.875	0.11862	-115.577767401134\\
63.875	0.12228	-125.105320364973\\
63.875	0.12594	-135.454614204225\\
63.875	0.1296	-146.62564891889\\
63.875	0.13326	-158.618424508969\\
63.875	0.13692	-171.43294097446\\
63.875	0.14058	-185.069198315365\\
63.875	0.14424	-199.527196531683\\
63.875	0.1479	-214.806935623415\\
63.875	0.15156	-230.908415590559\\
63.875	0.15522	-247.831636433117\\
63.875	0.15888	-265.576598151088\\
63.875	0.16254	-284.143300744472\\
63.875	0.1662	-303.531744213269\\
63.875	0.16986	-323.74192855748\\
63.875	0.17352	-344.773853777104\\
63.875	0.17718	-366.62751987214\\
63.875	0.18084	-389.30292684259\\
63.875	0.1845	-412.800074688454\\
63.875	0.18816	-437.11896340973\\
63.875	0.19182	-462.25959300642\\
63.875	0.19548	-488.221963478523\\
63.875	0.19914	-515.006074826039\\
63.875	0.2028	-542.611927048968\\
63.875	0.20646	-571.039520147311\\
63.875	0.21012	-600.288854121066\\
63.875	0.21378	-630.359928970235\\
63.875	0.21744	-661.252744694817\\
63.875	0.2211	-692.967301294812\\
63.875	0.22476	-725.503598770221\\
63.875	0.22842	-758.861637121043\\
63.875	0.23208	-793.041416347277\\
63.875	0.23574	-828.042936448926\\
63.875	0.2394	-863.866197425987\\
63.875	0.24306	-900.511199278461\\
63.875	0.24672	-937.977942006348\\
63.875	0.25038	-976.26642560965\\
63.875	0.25404	-1015.37665008836\\
63.875	0.2577	-1055.30861544249\\
63.875	0.26136	-1096.06232167203\\
63.875	0.26502	-1137.63776877698\\
63.875	0.26868	-1180.03495675735\\
63.875	0.27234	-1223.25388561313\\
63.875	0.276	-1267.29455534433\\
64.25	0.093	-72.6211506881593\\
64.25	0.09666	-76.4084798798538\\
64.25	0.10032	-81.0175499469611\\
64.25	0.10398	-86.4483608894816\\
64.25	0.10764	-92.7009127074155\\
64.25	0.1113	-99.7752054007624\\
64.25	0.11496	-107.671238969523\\
64.25	0.11862	-116.389013413696\\
64.25	0.12228	-125.928528733283\\
64.25	0.12594	-136.289784928283\\
64.25	0.1296	-147.472781998696\\
64.25	0.13326	-159.477519944522\\
64.25	0.13692	-172.303998765762\\
64.25	0.14058	-185.952218462414\\
64.25	0.14424	-200.42217903448\\
64.25	0.1479	-215.713880481959\\
64.25	0.15156	-231.827322804851\\
64.25	0.15522	-248.762506003157\\
64.25	0.15888	-266.519430076876\\
64.25	0.16254	-285.098095026008\\
64.25	0.1662	-304.498500850552\\
64.25	0.16986	-324.720647550511\\
64.25	0.17352	-345.764535125882\\
64.25	0.17718	-367.630163576667\\
64.25	0.18084	-390.317532902864\\
64.25	0.1845	-413.826643104476\\
64.25	0.18816	-438.1574941815\\
64.25	0.19182	-463.310086133937\\
64.25	0.19548	-489.284418961788\\
64.25	0.19914	-516.080492665052\\
64.25	0.2028	-543.698307243729\\
64.25	0.20646	-572.137862697819\\
64.25	0.21012	-601.399159027323\\
64.25	0.21378	-631.482196232239\\
64.25	0.21744	-662.386974312569\\
64.25	0.2211	-694.113493268312\\
64.25	0.22476	-726.661753099468\\
64.25	0.22842	-760.031753806038\\
64.25	0.23208	-794.22349538802\\
64.25	0.23574	-829.236977845417\\
64.25	0.2394	-865.072201178225\\
64.25	0.24306	-901.729165386447\\
64.25	0.24672	-939.207870470083\\
64.25	0.25038	-977.508316429132\\
64.25	0.25404	-1016.63050326359\\
64.25	0.2577	-1056.57443097347\\
64.25	0.26136	-1097.34009955876\\
64.25	0.26502	-1138.92750901946\\
64.25	0.26868	-1181.33665935557\\
64.25	0.27234	-1224.5675505671\\
64.25	0.276	-1268.62018265404\\
64.625	0.093	-73.4196938856559\\
64.625	0.09666	-77.2189854330982\\
64.625	0.10032	-81.8400178559532\\
64.625	0.10398	-87.2827911542216\\
64.625	0.10764	-93.5473053279032\\
64.625	0.1113	-100.633560376998\\
64.625	0.11496	-108.541556301506\\
64.625	0.11862	-117.271293101427\\
64.625	0.12228	-126.822770776762\\
64.625	0.12594	-137.195989327509\\
64.625	0.1296	-148.39094875367\\
64.625	0.13326	-160.407649055244\\
64.625	0.13692	-173.246090232231\\
64.625	0.14058	-186.906272284632\\
64.625	0.14424	-201.388195212445\\
64.625	0.1479	-216.691859015672\\
64.625	0.15156	-232.817263694312\\
64.625	0.15522	-249.764409248365\\
64.625	0.15888	-267.533295677832\\
64.625	0.16254	-286.123922982712\\
64.625	0.1662	-305.536291163004\\
64.625	0.16986	-325.770400218711\\
64.625	0.17352	-346.826250149829\\
64.625	0.17718	-368.703840956362\\
64.625	0.18084	-391.403172638307\\
64.625	0.1845	-414.924245195666\\
64.625	0.18816	-439.267058628438\\
64.625	0.19182	-464.431612936623\\
64.625	0.19548	-490.417908120222\\
64.625	0.19914	-517.225944179233\\
64.625	0.2028	-544.855721113658\\
64.625	0.20646	-573.307238923496\\
64.625	0.21012	-602.580497608747\\
64.625	0.21378	-632.675497169412\\
64.625	0.21744	-663.592237605489\\
64.625	0.2211	-695.33071891698\\
64.625	0.22476	-727.890941103884\\
64.625	0.22842	-761.272904166201\\
64.625	0.23208	-795.476608103932\\
64.625	0.23574	-830.502052917076\\
64.625	0.2394	-866.349238605632\\
64.625	0.24306	-903.018165169602\\
64.625	0.24672	-940.508832608985\\
64.625	0.25038	-978.821240923782\\
64.625	0.25404	-1017.95539011399\\
64.625	0.2577	-1057.91128017961\\
64.625	0.26136	-1098.68891112065\\
64.625	0.26502	-1140.2882829371\\
64.625	0.26868	-1182.70939562896\\
64.625	0.27234	-1225.95224919624\\
64.625	0.276	-1270.01684363893\\
65	0.093	-74.2892707583216\\
65	0.09666	-78.1005246615111\\
65	0.10032	-82.7335194401144\\
65	0.10398	-88.1882550941305\\
65	0.10764	-94.4647316235599\\
65	0.1113	-101.562949028402\\
65	0.11496	-109.482907308658\\
65	0.11862	-118.224606464327\\
65	0.12228	-127.788046495409\\
65	0.12594	-138.173227401905\\
65	0.1296	-149.380149183813\\
65	0.13326	-161.408811841135\\
65	0.13692	-174.25921537387\\
65	0.14058	-187.931359782018\\
65	0.14424	-202.42524506558\\
65	0.1479	-217.740871224554\\
65	0.15156	-233.878238258942\\
65	0.15522	-250.837346168743\\
65	0.15888	-268.618194953957\\
65	0.16254	-287.220784614585\\
65	0.1662	-306.645115150625\\
65	0.16986	-326.891186562079\\
65	0.17352	-347.958998848946\\
65	0.17718	-369.848552011226\\
65	0.18084	-392.559846048919\\
65	0.1845	-416.092880962026\\
65	0.18816	-440.447656750545\\
65	0.19182	-465.624173414478\\
65	0.19548	-491.622430953825\\
65	0.19914	-518.442429368584\\
65	0.2028	-546.084168658757\\
65	0.20646	-574.547648824343\\
65	0.21012	-603.832869865341\\
65	0.21378	-633.939831781753\\
65	0.21744	-664.868534573579\\
65	0.2211	-696.618978240817\\
65	0.22476	-729.191162783469\\
65	0.22842	-762.585088201534\\
65	0.23208	-796.800754495012\\
65	0.23574	-831.838161663904\\
65	0.2394	-867.697309708208\\
65	0.24306	-904.378198627926\\
65	0.24672	-941.880828423057\\
65	0.25038	-980.205199093601\\
65	0.25404	-1019.35131063956\\
65	0.2577	-1059.31916306093\\
65	0.26136	-1100.10875635771\\
65	0.26502	-1141.72009052991\\
65	0.26868	-1184.15316557752\\
65	0.27234	-1227.40798150054\\
65	0.276	-1271.48453829898\\
65.375	0.093	-75.2298813061554\\
65.375	0.09666	-79.0530975650927\\
65.375	0.10032	-83.6980546994435\\
65.375	0.10398	-89.1647527092073\\
65.375	0.10764	-95.4531915943847\\
65.375	0.1113	-102.563371354975\\
65.375	0.11496	-110.495291990978\\
65.375	0.11862	-119.248953502395\\
65.375	0.12228	-128.824355889225\\
65.375	0.12594	-139.221499151468\\
65.375	0.1296	-150.440383289124\\
65.375	0.13326	-162.481008302194\\
65.375	0.13692	-175.343374190677\\
65.375	0.14058	-189.027480954573\\
65.375	0.14424	-203.533328593882\\
65.375	0.1479	-218.860917108604\\
65.375	0.15156	-235.01024649874\\
65.375	0.15522	-251.981316764288\\
65.375	0.15888	-269.774127905251\\
65.375	0.16254	-288.388679921625\\
65.375	0.1662	-307.824972813414\\
65.375	0.16986	-328.083006580615\\
65.375	0.17352	-349.16278122323\\
65.375	0.17718	-371.064296741258\\
65.375	0.18084	-393.787553134699\\
65.375	0.1845	-417.332550403553\\
65.375	0.18816	-441.699288547821\\
65.375	0.19182	-466.887767567502\\
65.375	0.19548	-492.897987462596\\
65.375	0.19914	-519.729948233103\\
65.375	0.2028	-547.383649879023\\
65.375	0.20646	-575.859092400357\\
65.375	0.21012	-605.156275797103\\
65.375	0.21378	-635.275200069263\\
65.375	0.21744	-666.215865216836\\
65.375	0.2211	-697.978271239822\\
65.375	0.22476	-730.562418138222\\
65.375	0.22842	-763.968305912035\\
65.375	0.23208	-798.195934561261\\
65.375	0.23574	-833.2453040859\\
65.375	0.2394	-869.116414485952\\
65.375	0.24306	-905.809265761417\\
65.375	0.24672	-943.323857912296\\
65.375	0.25038	-981.660190938589\\
65.375	0.25404	-1020.81826484029\\
65.375	0.2577	-1060.79807961741\\
65.375	0.26136	-1101.59963526994\\
65.375	0.26502	-1143.22293179789\\
65.375	0.26868	-1185.66796920125\\
65.375	0.27234	-1228.93474748002\\
65.375	0.276	-1273.0232666342\\
65.75	0.093	-76.2415255291582\\
65.75	0.09666	-80.0767041438437\\
65.75	0.10032	-84.733623633942\\
65.75	0.10398	-90.2122839994536\\
65.75	0.10764	-96.5126852403785\\
65.75	0.1113	-103.634827356717\\
65.75	0.11496	-111.578710348468\\
65.75	0.11862	-120.344334215632\\
65.75	0.12228	-129.93169895821\\
65.75	0.12594	-140.340804576201\\
65.75	0.1296	-151.571651069605\\
65.75	0.13326	-163.624238438422\\
65.75	0.13692	-176.498566682653\\
65.75	0.14058	-190.194635802297\\
65.75	0.14424	-204.712445797354\\
65.75	0.1479	-220.051996667824\\
65.75	0.15156	-236.213288413707\\
65.75	0.15522	-253.196321035003\\
65.75	0.15888	-271.001094531713\\
65.75	0.16254	-289.627608903836\\
65.75	0.1662	-309.075864151372\\
65.75	0.16986	-329.345860274322\\
65.75	0.17352	-350.437597272684\\
65.75	0.17718	-372.351075146459\\
65.75	0.18084	-395.086293895648\\
65.75	0.1845	-418.64325352025\\
65.75	0.18816	-443.021954020266\\
65.75	0.19182	-468.222395395694\\
65.75	0.19548	-494.244577646536\\
65.75	0.19914	-521.088500772791\\
65.75	0.2028	-548.754164774459\\
65.75	0.20646	-577.24156965154\\
65.75	0.21012	-606.550715404034\\
65.75	0.21378	-636.681602031942\\
65.75	0.21744	-667.634229535263\\
65.75	0.2211	-699.408597913997\\
65.75	0.22476	-732.004707168144\\
65.75	0.22842	-765.422557297705\\
65.75	0.23208	-799.662148302678\\
65.75	0.23574	-834.723480183065\\
65.75	0.2394	-870.606552938865\\
65.75	0.24306	-907.311366570079\\
65.75	0.24672	-944.837921076705\\
65.75	0.25038	-983.186216458745\\
65.75	0.25404	-1022.3562527162\\
65.75	0.2577	-1062.34802984906\\
65.75	0.26136	-1103.16154785734\\
65.75	0.26502	-1144.79680674104\\
65.75	0.26868	-1187.25380650014\\
65.75	0.27234	-1230.53254713466\\
65.75	0.276	-1274.63302864459\\
66.125	0.093	-77.3242034273296\\
66.125	0.09666	-81.1713443977628\\
66.125	0.10032	-85.8402262436089\\
66.125	0.10398	-91.3308489648683\\
66.125	0.10764	-97.6432125615409\\
66.125	0.1113	-104.777317033627\\
66.125	0.11496	-112.733162381126\\
66.125	0.11862	-121.510748604038\\
66.125	0.12228	-131.110075702364\\
66.125	0.12594	-141.531143676102\\
66.125	0.1296	-152.773952525254\\
66.125	0.13326	-164.838502249819\\
66.125	0.13692	-177.724792849797\\
66.125	0.14058	-191.432824325189\\
66.125	0.14424	-205.962596675993\\
66.125	0.1479	-221.314109902211\\
66.125	0.15156	-237.487364003842\\
66.125	0.15522	-254.482358980886\\
66.125	0.15888	-272.299094833344\\
66.125	0.16254	-290.937571561215\\
66.125	0.1662	-310.397789164498\\
66.125	0.16986	-330.679747643196\\
66.125	0.17352	-351.783446997306\\
66.125	0.17718	-373.708887226829\\
66.125	0.18084	-396.456068331766\\
66.125	0.1845	-420.024990312116\\
66.125	0.18816	-444.415653167879\\
66.125	0.19182	-469.628056899055\\
66.125	0.19548	-495.662201505644\\
66.125	0.19914	-522.518086987647\\
66.125	0.2028	-550.195713345063\\
66.125	0.20646	-578.695080577892\\
66.125	0.21012	-608.016188686134\\
66.125	0.21378	-638.159037669789\\
66.125	0.21744	-669.123627528858\\
66.125	0.2211	-700.90995826334\\
66.125	0.22476	-733.518029873235\\
66.125	0.22842	-766.947842358543\\
66.125	0.23208	-801.199395719264\\
66.125	0.23574	-836.272689955399\\
66.125	0.2394	-872.167725066947\\
66.125	0.24306	-908.884501053908\\
66.125	0.24672	-946.423017916282\\
66.125	0.25038	-984.78327565407\\
66.125	0.25404	-1023.96527426727\\
66.125	0.2577	-1063.96901375588\\
66.125	0.26136	-1104.79449411991\\
66.125	0.26502	-1146.44171535935\\
66.125	0.26868	-1188.9106774742\\
66.125	0.27234	-1232.20138046447\\
66.125	0.276	-1276.31382433015\\
66.5	0.093	-78.4779150006693\\
66.5	0.09666	-82.3370183268503\\
66.5	0.10032	-87.0178625284444\\
66.5	0.10398	-92.5204476054515\\
66.5	0.10764	-98.8447735578719\\
66.5	0.1113	-105.990840385705\\
66.5	0.11496	-113.958648088952\\
66.5	0.11862	-122.748196667612\\
66.5	0.12228	-132.359486121686\\
66.5	0.12594	-142.792516451172\\
66.5	0.1296	-154.047287656072\\
66.5	0.13326	-166.123799736384\\
66.5	0.13692	-179.02205269211\\
66.5	0.14058	-192.742046523249\\
66.5	0.14424	-207.283781229802\\
66.5	0.1479	-222.647256811768\\
66.5	0.15156	-238.832473269146\\
66.5	0.15522	-255.839430601938\\
66.5	0.15888	-273.668128810143\\
66.5	0.16254	-292.318567893762\\
66.5	0.1662	-311.790747852793\\
66.5	0.16986	-332.084668687239\\
66.5	0.17352	-353.200330397096\\
66.5	0.17718	-375.137732982367\\
66.5	0.18084	-397.896876443052\\
66.5	0.1845	-421.477760779149\\
66.5	0.18816	-445.88038599066\\
66.5	0.19182	-471.104752077584\\
66.5	0.19548	-497.150859039921\\
66.5	0.19914	-524.018706877672\\
66.5	0.2028	-551.708295590835\\
66.5	0.20646	-580.219625179412\\
66.5	0.21012	-609.552695643402\\
66.5	0.21378	-639.707506982805\\
66.5	0.21744	-670.684059197622\\
66.5	0.2211	-702.482352287851\\
66.5	0.22476	-735.102386253494\\
66.5	0.22842	-768.54416109455\\
66.5	0.23208	-802.807676811019\\
66.5	0.23574	-837.892933402901\\
66.5	0.2394	-873.799930870197\\
66.5	0.24306	-910.528669212906\\
66.5	0.24672	-948.079148431028\\
66.5	0.25038	-986.451368524563\\
66.5	0.25404	-1025.64532949351\\
66.5	0.2577	-1065.66103133787\\
66.5	0.26136	-1106.49847405765\\
66.5	0.26502	-1148.15765765284\\
66.5	0.26868	-1190.63858212344\\
66.5	0.27234	-1233.94124746945\\
66.5	0.276	-1278.06565369088\\
66.875	0.093	-79.7026602491782\\
66.875	0.09666	-83.573725931107\\
66.875	0.10032	-88.2665324884486\\
66.875	0.10398	-93.7810799212035\\
66.875	0.10764	-100.117368229372\\
66.875	0.1113	-107.275397412953\\
66.875	0.11496	-115.255167471948\\
66.875	0.11862	-124.056678406355\\
66.875	0.12228	-133.679930216176\\
66.875	0.12594	-144.12492290141\\
66.875	0.1296	-155.391656462058\\
66.875	0.13326	-167.480130898118\\
66.875	0.13692	-180.390346209592\\
66.875	0.14058	-194.122302396479\\
66.875	0.14424	-208.675999458779\\
66.875	0.1479	-224.051437396493\\
66.875	0.15156	-240.248616209619\\
66.875	0.15522	-257.267535898159\\
66.875	0.15888	-275.108196462112\\
66.875	0.16254	-293.770597901478\\
66.875	0.1662	-313.254740216257\\
66.875	0.16986	-333.56062340645\\
66.875	0.17352	-354.688247472056\\
66.875	0.17718	-376.637612413075\\
66.875	0.18084	-399.408718229507\\
66.875	0.1845	-423.001564921352\\
66.875	0.18816	-447.416152488611\\
66.875	0.19182	-472.652480931283\\
66.875	0.19548	-498.710550249368\\
66.875	0.19914	-525.590360442866\\
66.875	0.2028	-553.291911511777\\
66.875	0.20646	-581.815203456102\\
66.875	0.21012	-611.160236275839\\
66.875	0.21378	-641.32700997099\\
66.875	0.21744	-672.315524541554\\
66.875	0.2211	-704.125779987532\\
66.875	0.22476	-736.757776308922\\
66.875	0.22842	-770.211513505726\\
66.875	0.23208	-804.486991577942\\
66.875	0.23574	-839.584210525573\\
66.875	0.2394	-875.503170348617\\
66.875	0.24306	-912.243871047073\\
66.875	0.24672	-949.806312620942\\
66.875	0.25038	-988.190495070226\\
66.875	0.25404	-1027.39641839492\\
66.875	0.2577	-1067.42408259503\\
66.875	0.26136	-1108.27348767055\\
66.875	0.26502	-1149.94463362149\\
66.875	0.26868	-1192.43752044784\\
66.875	0.27234	-1235.7521481496\\
66.875	0.276	-1279.88851672678\\
67.25	0.093	-80.998439172856\\
67.25	0.09666	-84.8814672105325\\
67.25	0.10032	-89.5862361236219\\
67.25	0.10398	-95.1127459121245\\
67.25	0.10764	-101.46099657604\\
67.25	0.1113	-108.63098811537\\
67.25	0.11496	-116.622720530112\\
67.25	0.11862	-125.436193820267\\
67.25	0.12228	-135.071407985836\\
67.25	0.12594	-145.528363026818\\
67.25	0.1296	-156.807058943213\\
67.25	0.13326	-168.907495735021\\
67.25	0.13692	-181.829673402243\\
67.25	0.14058	-195.573591944878\\
67.25	0.14424	-210.139251362926\\
67.25	0.1479	-225.526651656387\\
67.25	0.15156	-241.735792825261\\
67.25	0.15522	-258.766674869549\\
67.25	0.15888	-276.619297789249\\
67.25	0.16254	-295.293661584363\\
67.25	0.1662	-314.78976625489\\
67.25	0.16986	-335.107611800831\\
67.25	0.17352	-356.247198222184\\
67.25	0.17718	-378.208525518951\\
67.25	0.18084	-400.991593691131\\
67.25	0.1845	-424.596402738724\\
67.25	0.18816	-449.02295266173\\
67.25	0.19182	-474.27124346015\\
67.25	0.19548	-500.341275133982\\
67.25	0.19914	-527.233047683228\\
67.25	0.2028	-554.946561107887\\
67.25	0.20646	-583.48181540796\\
67.25	0.21012	-612.838810583445\\
67.25	0.21378	-643.017546634344\\
67.25	0.21744	-674.018023560656\\
67.25	0.2211	-705.840241362381\\
67.25	0.22476	-738.484200039519\\
67.25	0.22842	-771.949899592071\\
67.25	0.23208	-806.237340020036\\
67.25	0.23574	-841.346521323414\\
67.25	0.2394	-877.277443502205\\
67.25	0.24306	-914.030106556408\\
67.25	0.24672	-951.604510486026\\
67.25	0.25038	-990.000655291057\\
67.25	0.25404	-1029.2185409715\\
67.25	0.2577	-1069.25816752736\\
67.25	0.26136	-1110.11953495863\\
67.25	0.26502	-1151.80264326531\\
67.25	0.26868	-1194.30749244741\\
67.25	0.27234	-1237.63408250492\\
67.25	0.276	-1281.78241343784\\
67.625	0.093	-82.365251771702\\
67.625	0.09666	-86.2602421651266\\
67.625	0.10032	-90.9769734339637\\
67.625	0.10398	-96.515445578214\\
67.625	0.10764	-102.875658597878\\
67.625	0.1113	-110.057612492954\\
67.625	0.11496	-118.061307263445\\
67.625	0.11862	-126.886742909348\\
67.625	0.12228	-136.533919430665\\
67.625	0.12594	-147.002836827394\\
67.625	0.1296	-158.293495099537\\
67.625	0.13326	-170.405894247093\\
67.625	0.13692	-183.340034270062\\
67.625	0.14058	-197.095915168445\\
67.625	0.14424	-211.673536942241\\
67.625	0.1479	-227.072899591449\\
67.625	0.15156	-243.294003116072\\
67.625	0.15522	-260.336847516107\\
67.625	0.15888	-278.201432791555\\
67.625	0.16254	-296.887758942417\\
67.625	0.1662	-316.395825968692\\
67.625	0.16986	-336.72563387038\\
67.625	0.17352	-357.877182647481\\
67.625	0.17718	-379.850472299995\\
67.625	0.18084	-402.645502827923\\
67.625	0.1845	-426.262274231264\\
67.625	0.18816	-450.700786510018\\
67.625	0.19182	-475.961039664186\\
67.625	0.19548	-502.043033693766\\
67.625	0.19914	-528.946768598759\\
67.625	0.2028	-556.672244379166\\
67.625	0.20646	-585.219461034987\\
67.625	0.21012	-614.58841856622\\
67.625	0.21378	-644.779116972866\\
67.625	0.21744	-675.791556254926\\
67.625	0.2211	-707.625736412399\\
67.625	0.22476	-740.281657445284\\
67.625	0.22842	-773.759319353584\\
67.625	0.23208	-808.058722137297\\
67.625	0.23574	-843.179865796423\\
67.625	0.2394	-879.122750330961\\
67.625	0.24306	-915.887375740913\\
67.625	0.24672	-953.473742026278\\
67.625	0.25038	-991.881849187057\\
67.625	0.25404	-1031.11169722325\\
67.625	0.2577	-1071.16328613485\\
67.625	0.26136	-1112.03661592187\\
67.625	0.26502	-1153.7316865843\\
67.625	0.26868	-1196.24849812215\\
67.625	0.27234	-1239.5870505354\\
67.625	0.276	-1283.74734382407\\
68	0.093	-83.8030980457169\\
68	0.09666	-87.710050794889\\
68	0.10032	-92.4387444194741\\
68	0.10398	-97.9891789194722\\
68	0.10764	-104.361354294884\\
68	0.1113	-111.555270545708\\
68	0.11496	-119.570927671946\\
68	0.11862	-128.408325673597\\
68	0.12228	-138.067464550661\\
68	0.12594	-148.548344303139\\
68	0.1296	-159.850964931029\\
68	0.13326	-171.975326434333\\
68	0.13692	-184.92142881305\\
68	0.14058	-198.689272067181\\
68	0.14424	-213.278856196724\\
68	0.1479	-228.690181201681\\
68	0.15156	-244.92324708205\\
68	0.15522	-261.978053837833\\
68	0.15888	-279.85460146903\\
68	0.16254	-298.552889975639\\
68	0.1662	-318.072919357662\\
68	0.16986	-338.414689615098\\
68	0.17352	-359.578200747947\\
68	0.17718	-381.563452756209\\
68	0.18084	-404.370445639884\\
68	0.1845	-427.999179398973\\
68	0.18816	-452.449654033475\\
68	0.19182	-477.72186954339\\
68	0.19548	-503.815825928718\\
68	0.19914	-530.731523189459\\
68	0.2028	-558.468961325614\\
68	0.20646	-587.028140337182\\
68	0.21012	-616.409060224163\\
68	0.21378	-646.611720986557\\
68	0.21744	-677.636122624365\\
68	0.2211	-709.482265137585\\
68	0.22476	-742.150148526219\\
68	0.22842	-775.639772790266\\
68	0.23208	-809.951137929726\\
68	0.23574	-845.0842439446\\
68	0.2394	-881.039090834886\\
68	0.24306	-917.815678600586\\
68	0.24672	-955.414007241699\\
68	0.25038	-993.834076758225\\
68	0.25404	-1033.07588715016\\
68	0.2577	-1073.13943841752\\
68	0.26136	-1114.02473056028\\
68	0.26502	-1155.73176357846\\
68	0.26868	-1198.26053747205\\
68	0.27234	-1241.61105224106\\
68	0.276	-1285.78330788548\\
68.375	0.093	-85.3119779949004\\
68.375	0.09666	-89.2308930998202\\
68.375	0.10032	-93.9715490801528\\
68.375	0.10398	-99.5339459358989\\
68.375	0.10764	-105.918083667058\\
68.375	0.1113	-113.12396227363\\
68.375	0.11496	-121.151581755616\\
68.375	0.11862	-130.000942113015\\
68.375	0.12228	-139.672043345827\\
68.375	0.12594	-150.164885454052\\
68.375	0.1296	-161.47946843769\\
68.375	0.13326	-173.615792296742\\
68.375	0.13692	-186.573857031207\\
68.375	0.14058	-200.353662641085\\
68.375	0.14424	-214.955209126376\\
68.375	0.1479	-230.37849648708\\
68.375	0.15156	-246.623524723198\\
68.375	0.15522	-263.690293834729\\
68.375	0.15888	-281.578803821673\\
68.375	0.16254	-300.28905468403\\
68.375	0.1662	-319.8210464218\\
68.375	0.16986	-340.174779034984\\
68.375	0.17352	-361.350252523581\\
68.375	0.17718	-383.34746688759\\
68.375	0.18084	-406.166422127014\\
68.375	0.1845	-429.80711824185\\
68.375	0.18816	-454.2695552321\\
68.375	0.19182	-479.553733097763\\
68.375	0.19548	-505.659651838838\\
68.375	0.19914	-532.587311455328\\
68.375	0.2028	-560.33671194723\\
68.375	0.20646	-588.907853314546\\
68.375	0.21012	-618.300735557274\\
68.375	0.21378	-648.515358675416\\
68.375	0.21744	-679.551722668972\\
68.375	0.2211	-711.40982753794\\
68.375	0.22476	-744.089673282321\\
68.375	0.22842	-777.591259902116\\
68.375	0.23208	-811.914587397324\\
68.375	0.23574	-847.059655767946\\
68.375	0.2394	-883.02646501398\\
68.375	0.24306	-919.815015135427\\
68.375	0.24672	-957.425306132288\\
68.375	0.25038	-995.857338004562\\
68.375	0.25404	-1035.11111075225\\
68.375	0.2577	-1075.18662437535\\
68.375	0.26136	-1116.08387887386\\
68.375	0.26502	-1157.80287424779\\
68.375	0.26868	-1200.34361049713\\
68.375	0.27234	-1243.70608762188\\
68.375	0.276	-1287.89030562205\\
68.75	0.093	-86.891891619253\\
68.75	0.09666	-90.8227690799206\\
68.75	0.10032	-95.575387416001\\
68.75	0.10398	-101.149746627495\\
68.75	0.10764	-107.545846714402\\
68.75	0.1113	-114.763687676722\\
68.75	0.11496	-122.803269514455\\
68.75	0.11862	-131.664592227602\\
68.75	0.12228	-141.347655816161\\
68.75	0.12594	-151.852460280134\\
68.75	0.1296	-163.179005619521\\
68.75	0.13326	-175.32729183432\\
68.75	0.13692	-188.297318924532\\
68.75	0.14058	-202.089086890158\\
68.75	0.14424	-216.702595731197\\
68.75	0.1479	-232.137845447649\\
68.75	0.15156	-248.394836039515\\
68.75	0.15522	-265.473567506793\\
68.75	0.15888	-283.374039849485\\
68.75	0.16254	-302.09625306759\\
68.75	0.1662	-321.640207161108\\
68.75	0.16986	-342.00590213004\\
68.75	0.17352	-363.193337974384\\
68.75	0.17718	-385.202514694142\\
68.75	0.18084	-408.033432289312\\
68.75	0.1845	-431.686090759897\\
68.75	0.18816	-456.160490105894\\
68.75	0.19182	-481.456630327305\\
68.75	0.19548	-507.574511424128\\
68.75	0.19914	-534.514133396365\\
68.75	0.2028	-562.275496244016\\
68.75	0.20646	-590.858599967079\\
68.75	0.21012	-620.263444565556\\
68.75	0.21378	-650.490030039445\\
68.75	0.21744	-681.538356388748\\
68.75	0.2211	-713.408423613464\\
68.75	0.22476	-746.100231713594\\
68.75	0.22842	-779.613780689136\\
68.75	0.23208	-813.949070540091\\
68.75	0.23574	-849.106101266461\\
68.75	0.2394	-885.084872868243\\
68.75	0.24306	-921.885385345438\\
68.75	0.24672	-959.507638698046\\
68.75	0.25038	-997.951632926068\\
68.75	0.25404	-1037.2173680295\\
68.75	0.2577	-1077.30484400835\\
68.75	0.26136	-1118.21406086261\\
68.75	0.26502	-1159.94501859229\\
68.75	0.26868	-1202.49771719737\\
68.75	0.27234	-1245.87215667788\\
68.75	0.276	-1290.06833703379\\
69.125	0.093	-88.5428389187738\\
69.125	0.09666	-92.4856787351894\\
69.125	0.10032	-97.2502594270175\\
69.125	0.10398	-102.836580994259\\
69.125	0.10764	-109.244643436914\\
69.125	0.1113	-116.474446754981\\
69.125	0.11496	-124.525990948463\\
69.125	0.11862	-133.399276017357\\
69.125	0.12228	-143.094301961665\\
69.125	0.12594	-153.611068781385\\
69.125	0.1296	-164.949576476519\\
69.125	0.13326	-177.109825047066\\
69.125	0.13692	-190.091814493027\\
69.125	0.14058	-203.8955448144\\
69.125	0.14424	-218.521016011187\\
69.125	0.1479	-233.968228083387\\
69.125	0.15156	-250.237181031\\
69.125	0.15522	-267.327874854026\\
69.125	0.15888	-285.240309552466\\
69.125	0.16254	-303.974485126319\\
69.125	0.1662	-323.530401575584\\
69.125	0.16986	-343.908058900263\\
69.125	0.17352	-365.107457100356\\
69.125	0.17718	-387.128596175861\\
69.125	0.18084	-409.97147612678\\
69.125	0.1845	-433.636096953112\\
69.125	0.18816	-458.122458654857\\
69.125	0.19182	-483.430561232015\\
69.125	0.19548	-509.560404684587\\
69.125	0.19914	-536.511989012571\\
69.125	0.2028	-564.285314215969\\
69.125	0.20646	-592.88038029478\\
69.125	0.21012	-622.297187249005\\
69.125	0.21378	-652.535735078642\\
69.125	0.21744	-683.596023783693\\
69.125	0.2211	-715.478053364156\\
69.125	0.22476	-748.181823820034\\
69.125	0.22842	-781.707335151324\\
69.125	0.23208	-816.054587358027\\
69.125	0.23574	-851.223580440144\\
69.125	0.2394	-887.214314397674\\
69.125	0.24306	-924.026789230617\\
69.125	0.24672	-961.661004938973\\
69.125	0.25038	-1000.11696152274\\
69.125	0.25404	-1039.39465898193\\
69.125	0.2577	-1079.49409731652\\
69.125	0.26136	-1120.41527652653\\
69.125	0.26502	-1162.15819661195\\
69.125	0.26868	-1204.72285757279\\
69.125	0.27234	-1248.10925940904\\
69.125	0.276	-1292.3174021207\\
69.5	0.093	-90.2648198934634\\
69.5	0.09666	-94.2196220656265\\
69.5	0.10032	-98.9961651132026\\
69.5	0.10398	-104.594449036192\\
69.5	0.10764	-111.014473834594\\
69.5	0.1113	-118.25623950841\\
69.5	0.11496	-126.319746057639\\
69.5	0.11862	-135.204993482281\\
69.5	0.12228	-144.911981782336\\
69.5	0.12594	-155.440710957804\\
69.5	0.1296	-166.791181008686\\
69.5	0.13326	-178.963391934981\\
69.5	0.13692	-191.957343736689\\
69.5	0.14058	-205.77303641381\\
69.5	0.14424	-220.410469966345\\
69.5	0.1479	-235.869644394293\\
69.5	0.15156	-252.150559697653\\
69.5	0.15522	-269.253215876427\\
69.5	0.15888	-287.177612930615\\
69.5	0.16254	-305.923750860215\\
69.5	0.1662	-325.491629665229\\
69.5	0.16986	-345.881249345656\\
69.5	0.17352	-367.092609901496\\
69.5	0.17718	-389.125711332749\\
69.5	0.18084	-411.980553639415\\
69.5	0.1845	-435.657136821495\\
69.5	0.18816	-460.155460878988\\
69.5	0.19182	-485.475525811894\\
69.5	0.19548	-511.617331620213\\
69.5	0.19914	-538.580878303946\\
69.5	0.2028	-566.366165863092\\
69.5	0.20646	-594.97319429765\\
69.5	0.21012	-624.401963607623\\
69.5	0.21378	-654.652473793008\\
69.5	0.21744	-685.724724853806\\
69.5	0.2211	-717.618716790018\\
69.5	0.22476	-750.334449601643\\
69.5	0.22842	-783.871923288681\\
69.5	0.23208	-818.231137851132\\
69.5	0.23574	-853.412093288996\\
69.5	0.2394	-889.414789602274\\
69.5	0.24306	-926.239226790965\\
69.5	0.24672	-963.885404855068\\
69.5	0.25038	-1002.35332379459\\
69.5	0.25404	-1041.64298360952\\
69.5	0.2577	-1081.75438429986\\
69.5	0.26136	-1122.68752586562\\
69.5	0.26502	-1164.44240830679\\
69.5	0.26868	-1207.01903162337\\
69.5	0.27234	-1250.41739581537\\
69.5	0.276	-1294.63750088278\\
69.875	0.093	-92.0578345433218\\
69.875	0.09666	-96.0245990712326\\
69.875	0.10032	-100.813104474556\\
69.875	0.10398	-106.423350753293\\
69.875	0.10764	-112.855337907444\\
69.875	0.1113	-120.109065937007\\
69.875	0.11496	-128.184534841984\\
69.875	0.11862	-137.081744622373\\
69.875	0.12228	-146.800695278177\\
69.875	0.12594	-157.341386809393\\
69.875	0.1296	-168.703819216022\\
69.875	0.13326	-180.887992498065\\
69.875	0.13692	-193.893906655521\\
69.875	0.14058	-207.72156168839\\
69.875	0.14424	-222.370957596672\\
69.875	0.1479	-237.842094380367\\
69.875	0.15156	-254.134972039476\\
69.875	0.15522	-271.249590573998\\
69.875	0.15888	-289.185949983933\\
69.875	0.16254	-307.944050269281\\
69.875	0.1662	-327.523891430042\\
69.875	0.16986	-347.925473466217\\
69.875	0.17352	-369.148796377805\\
69.875	0.17718	-391.193860164806\\
69.875	0.18084	-414.06066482722\\
69.875	0.1845	-437.749210365047\\
69.875	0.18816	-462.259496778288\\
69.875	0.19182	-487.591524066942\\
69.875	0.19548	-513.745292231009\\
69.875	0.19914	-540.720801270489\\
69.875	0.2028	-568.518051185382\\
69.875	0.20646	-597.137041975689\\
69.875	0.21012	-626.577773641409\\
69.875	0.21378	-656.840246182542\\
69.875	0.21744	-687.924459599088\\
69.875	0.2211	-719.830413891047\\
69.875	0.22476	-752.55810905842\\
69.875	0.22842	-786.107545101206\\
69.875	0.23208	-820.478722019405\\
69.875	0.23574	-855.671639813017\\
69.875	0.2394	-891.686298482043\\
69.875	0.24306	-928.522698026481\\
69.875	0.24672	-966.180838446333\\
69.875	0.25038	-1004.6607197416\\
69.875	0.25404	-1043.96234191228\\
69.875	0.2577	-1084.08570495837\\
69.875	0.26136	-1125.03080887987\\
69.875	0.26502	-1166.79765367679\\
69.875	0.26868	-1209.38623934912\\
69.875	0.27234	-1252.79656589686\\
69.875	0.276	-1297.02863332002\\
70.25	0.093	-93.921882868349\\
70.25	0.09666	-97.9006097520078\\
70.25	0.10032	-102.701077511079\\
70.25	0.10398	-108.323286145564\\
70.25	0.10764	-114.767235655462\\
70.25	0.1113	-122.032926040773\\
70.25	0.11496	-130.120357301498\\
70.25	0.11862	-139.029529437635\\
70.25	0.12228	-148.760442449186\\
70.25	0.12594	-159.31309633615\\
70.25	0.1296	-170.687491098527\\
70.25	0.13326	-182.883626736317\\
70.25	0.13692	-195.901503249521\\
70.25	0.14058	-209.741120638138\\
70.25	0.14424	-224.402478902168\\
70.25	0.1479	-239.885578041611\\
70.25	0.15156	-256.190418056467\\
70.25	0.15522	-273.316998946737\\
70.25	0.15888	-291.26532071242\\
70.25	0.16254	-310.035383353516\\
70.25	0.1662	-329.627186870025\\
70.25	0.16986	-350.040731261947\\
70.25	0.17352	-371.276016529283\\
70.25	0.17718	-393.333042672031\\
70.25	0.18084	-416.211809690193\\
70.25	0.1845	-439.912317583769\\
70.25	0.18816	-464.434566352757\\
70.25	0.19182	-489.778555997159\\
70.25	0.19548	-515.944286516973\\
70.25	0.19914	-542.931757912201\\
70.25	0.2028	-570.740970182843\\
70.25	0.20646	-599.371923328897\\
70.25	0.21012	-628.824617350364\\
70.25	0.21378	-659.099052247245\\
70.25	0.21744	-690.195228019539\\
70.25	0.2211	-722.113144667246\\
70.25	0.22476	-754.852802190366\\
70.25	0.22842	-788.4142005889\\
70.25	0.23208	-822.797339862847\\
70.25	0.23574	-858.002220012207\\
70.25	0.2394	-894.02884103698\\
70.25	0.24306	-930.877202937166\\
70.25	0.24672	-968.547305712766\\
70.25	0.25038	-1007.03914936378\\
70.25	0.25404	-1046.3527338902\\
70.25	0.2577	-1086.48805929204\\
70.25	0.26136	-1127.4451255693\\
70.25	0.26502	-1169.22393272196\\
70.25	0.26868	-1211.82448075004\\
70.25	0.27234	-1255.24676965353\\
70.25	0.276	-1299.49079943244\\
70.625	0.093	-95.8569648685447\\
70.625	0.09666	-99.847654107951\\
70.625	0.10032	-104.66008422277\\
70.625	0.10398	-110.294255213003\\
70.625	0.10764	-116.750167078649\\
70.625	0.1113	-124.027819819707\\
70.625	0.11496	-132.12721343618\\
70.625	0.11862	-141.048347928065\\
70.625	0.12228	-150.791223295364\\
70.625	0.12594	-161.355839538075\\
70.625	0.1296	-172.7421966562\\
70.625	0.13326	-184.950294649738\\
70.625	0.13692	-197.98013351869\\
70.625	0.14058	-211.831713263054\\
70.625	0.14424	-226.505033882832\\
70.625	0.1479	-242.000095378023\\
70.625	0.15156	-258.316897748627\\
70.625	0.15522	-275.455440994644\\
70.625	0.15888	-293.415725116075\\
70.625	0.16254	-312.197750112919\\
70.625	0.1662	-331.801515985176\\
70.625	0.16986	-352.227022732846\\
70.625	0.17352	-373.474270355929\\
70.625	0.17718	-395.543258854426\\
70.625	0.18084	-418.433988228335\\
70.625	0.1845	-442.146458477658\\
70.625	0.18816	-466.680669602395\\
70.625	0.19182	-492.036621602544\\
70.625	0.19548	-518.214314478106\\
70.625	0.19914	-545.213748229082\\
70.625	0.2028	-573.034922855471\\
70.625	0.20646	-601.677838357273\\
70.625	0.21012	-631.142494734488\\
70.625	0.21378	-661.428891987117\\
70.625	0.21744	-692.537030115159\\
70.625	0.2211	-724.466909118613\\
70.625	0.22476	-757.218528997481\\
70.625	0.22842	-790.791889751763\\
70.625	0.23208	-825.186991381458\\
70.625	0.23574	-860.403833886566\\
70.625	0.2394	-896.442417267086\\
70.625	0.24306	-933.30274152302\\
70.625	0.24672	-970.984806654368\\
70.625	0.25038	-1009.48861266113\\
70.625	0.25404	-1048.8141595433\\
70.625	0.2577	-1088.96144730089\\
70.625	0.26136	-1129.93047593389\\
70.625	0.26502	-1171.7212454423\\
70.625	0.26868	-1214.33375582613\\
70.625	0.27234	-1257.76800708537\\
70.625	0.276	-1302.02399922002\\
71	0.093	-97.863080543909\\
71	0.09666	-101.865732139063\\
71	0.10032	-106.69012460963\\
71	0.10398	-112.33625795561\\
71	0.10764	-118.804132177004\\
71	0.1113	-126.09374727381\\
71	0.11496	-134.205103246031\\
71	0.11862	-143.138200093663\\
71	0.12228	-152.89303781671\\
71	0.12594	-163.469616415169\\
71	0.1296	-174.867935889042\\
71	0.13326	-187.087996238328\\
71	0.13692	-200.129797463027\\
71	0.14058	-213.993339563139\\
71	0.14424	-228.678622538665\\
71	0.1479	-244.185646389604\\
71	0.15156	-260.514411115955\\
71	0.15522	-277.664916717721\\
71	0.15888	-295.637163194899\\
71	0.16254	-314.431150547491\\
71	0.1662	-334.046878775495\\
71	0.16986	-354.484347878913\\
71	0.17352	-375.743557857744\\
71	0.17718	-397.824508711988\\
71	0.18084	-420.727200441646\\
71	0.1845	-444.451633046717\\
71	0.18816	-468.997806527201\\
71	0.19182	-494.365720883098\\
71	0.19548	-520.555376114408\\
71	0.19914	-547.566772221131\\
71	0.2028	-575.399909203268\\
71	0.20646	-604.054787060818\\
71	0.21012	-633.531405793781\\
71	0.21378	-663.829765402157\\
71	0.21744	-694.949865885947\\
71	0.2211	-726.891707245149\\
71	0.22476	-759.655289479765\\
71	0.22842	-793.240612589794\\
71	0.23208	-827.647676575237\\
71	0.23574	-862.876481436092\\
71	0.2394	-898.927027172361\\
71	0.24306	-935.799313784042\\
71	0.24672	-973.493341271138\\
71	0.25038	-1012.00910963365\\
71	0.25404	-1051.34661887157\\
71	0.2577	-1091.5058689849\\
71	0.26136	-1132.48685997365\\
71	0.26502	-1174.28959183781\\
71	0.26868	-1216.91406457739\\
71	0.27234	-1260.36027819237\\
71	0.276	-1304.62823268277\\
71.375	0.093	-99.9402298944421\\
71.375	0.09666	-103.954843845344\\
71.375	0.10032	-108.791198671659\\
71.375	0.10398	-114.449294373387\\
71.375	0.10764	-120.929130950528\\
71.375	0.1113	-128.230708403082\\
71.375	0.11496	-136.35402673105\\
71.375	0.11862	-145.299085934431\\
71.375	0.12228	-155.065886013225\\
71.375	0.12594	-165.654426967432\\
71.375	0.1296	-177.064708797053\\
71.375	0.13326	-189.296731502086\\
71.375	0.13692	-202.350495082533\\
71.375	0.14058	-216.225999538393\\
71.375	0.14424	-230.923244869667\\
71.375	0.1479	-246.442231076353\\
71.375	0.15156	-262.782958158453\\
71.375	0.15522	-279.945426115965\\
71.375	0.15888	-297.929634948891\\
71.375	0.16254	-316.735584657231\\
71.375	0.1662	-336.363275240983\\
71.375	0.16986	-356.812706700149\\
71.375	0.17352	-378.083879034728\\
71.375	0.17718	-400.17679224472\\
71.375	0.18084	-423.091446330125\\
71.375	0.1845	-446.827841290944\\
71.375	0.18816	-471.385977127175\\
71.375	0.19182	-496.76585383882\\
71.375	0.19548	-522.967471425878\\
71.375	0.19914	-549.990829888349\\
71.375	0.2028	-577.835929226234\\
71.375	0.20646	-606.502769439532\\
71.375	0.21012	-635.991350528242\\
71.375	0.21378	-666.301672492366\\
71.375	0.21744	-697.433735331904\\
71.375	0.2211	-729.387539046854\\
71.375	0.22476	-762.163083637217\\
71.375	0.22842	-795.760369102994\\
71.375	0.23208	-830.179395444184\\
71.375	0.23574	-865.420162660788\\
71.375	0.2394	-901.482670752804\\
71.375	0.24306	-938.366919720233\\
71.375	0.24672	-976.072909563076\\
71.375	0.25038	-1014.60064028133\\
71.375	0.25404	-1053.950111875\\
71.375	0.2577	-1094.12132434408\\
71.375	0.26136	-1135.11427768858\\
71.375	0.26502	-1176.92897190849\\
71.375	0.26868	-1219.56540700381\\
71.375	0.27234	-1263.02358297455\\
71.375	0.276	-1307.30349982069\\
71.75	0.093	-102.088412920144\\
71.75	0.09666	-106.114989226794\\
71.75	0.10032	-110.963306408856\\
71.75	0.10398	-116.633364466332\\
71.75	0.10764	-123.125163399221\\
71.75	0.1113	-130.438703207523\\
71.75	0.11496	-138.573983891239\\
71.75	0.11862	-147.531005450367\\
71.75	0.12228	-157.309767884909\\
71.75	0.12594	-167.910271194864\\
71.75	0.1296	-179.332515380233\\
71.75	0.13326	-191.576500441014\\
71.75	0.13692	-204.642226377209\\
71.75	0.14058	-218.529693188817\\
71.75	0.14424	-233.238900875837\\
71.75	0.1479	-248.769849438272\\
71.75	0.15156	-265.122538876119\\
71.75	0.15522	-282.29696918938\\
71.75	0.15888	-300.293140378053\\
71.75	0.16254	-319.111052442141\\
71.75	0.1662	-338.750705381641\\
71.75	0.16986	-359.212099196554\\
71.75	0.17352	-380.495233886881\\
71.75	0.17718	-402.60010945262\\
71.75	0.18084	-425.526725893773\\
71.75	0.1845	-449.27508321034\\
71.75	0.18816	-473.845181402319\\
71.75	0.19182	-499.237020469712\\
71.75	0.19548	-525.450600412518\\
71.75	0.19914	-552.485921230736\\
71.75	0.2028	-580.342982924368\\
71.75	0.20646	-609.021785493414\\
71.75	0.21012	-638.522328937873\\
71.75	0.21378	-668.844613257744\\
71.75	0.21744	-699.988638453029\\
71.75	0.2211	-731.954404523728\\
71.75	0.22476	-764.741911469839\\
71.75	0.22842	-798.351159291364\\
71.75	0.23208	-832.782147988301\\
71.75	0.23574	-868.034877560653\\
71.75	0.2394	-904.109348008417\\
71.75	0.24306	-941.005559331594\\
71.75	0.24672	-978.723511530184\\
71.75	0.25038	-1017.26320460419\\
71.75	0.25404	-1056.6246385536\\
71.75	0.2577	-1096.80781337844\\
71.75	0.26136	-1137.81272907868\\
71.75	0.26502	-1179.63938565434\\
71.75	0.26868	-1222.2877831054\\
71.75	0.27234	-1265.75792143189\\
71.75	0.276	-1310.04980063378\\
72.125	0.093	-104.307629621015\\
72.125	0.09666	-108.346168283412\\
72.125	0.10032	-113.206447821222\\
72.125	0.10398	-118.888468234446\\
72.125	0.10764	-125.392229523083\\
72.125	0.1113	-132.717731687133\\
72.125	0.11496	-140.864974726596\\
72.125	0.11862	-149.833958641472\\
72.125	0.12228	-159.624683431762\\
72.125	0.12594	-170.237149097464\\
72.125	0.1296	-181.671355638581\\
72.125	0.13326	-193.92730305511\\
72.125	0.13692	-207.004991347052\\
72.125	0.14058	-220.904420514408\\
72.125	0.14424	-235.625590557176\\
72.125	0.1479	-251.168501475358\\
72.125	0.15156	-267.533153268953\\
72.125	0.15522	-284.719545937962\\
72.125	0.15888	-302.727679482383\\
72.125	0.16254	-321.557553902219\\
72.125	0.1662	-341.209169197466\\
72.125	0.16986	-361.682525368128\\
72.125	0.17352	-382.977622414202\\
72.125	0.17718	-405.094460335689\\
72.125	0.18084	-428.03303913259\\
72.125	0.1845	-451.793358804904\\
72.125	0.18816	-476.375419352631\\
72.125	0.19182	-501.779220775772\\
72.125	0.19548	-528.004763074325\\
72.125	0.19914	-555.052046248292\\
72.125	0.2028	-582.921070297672\\
72.125	0.20646	-611.611835222465\\
72.125	0.21012	-641.124341022672\\
72.125	0.21378	-671.458587698291\\
72.125	0.21744	-702.614575249323\\
72.125	0.2211	-734.59230367577\\
72.125	0.22476	-767.391772977629\\
72.125	0.22842	-801.012983154901\\
72.125	0.23208	-835.455934207587\\
72.125	0.23574	-870.720626135685\\
72.125	0.2394	-906.807058939198\\
72.125	0.24306	-943.715232618123\\
72.125	0.24672	-981.445147172461\\
72.125	0.25038	-1019.99680260221\\
72.125	0.25404	-1059.37019890738\\
72.125	0.2577	-1099.56533608796\\
72.125	0.26136	-1140.58221414395\\
72.125	0.26502	-1182.42083307535\\
72.125	0.26868	-1225.08119288217\\
72.125	0.27234	-1268.5632935644\\
72.125	0.276	-1312.86713512204\\
72.5	0.093	-106.597879997054\\
72.5	0.09666	-110.648381015199\\
72.5	0.10032	-115.520622908757\\
72.5	0.10398	-121.214605677728\\
72.5	0.10764	-127.730329322113\\
72.5	0.1113	-135.067793841911\\
72.5	0.11496	-143.226999237122\\
72.5	0.11862	-152.207945507745\\
72.5	0.12228	-162.010632653783\\
72.5	0.12594	-172.635060675233\\
72.5	0.1296	-184.081229572097\\
72.5	0.13326	-196.349139344374\\
72.5	0.13692	-209.438789992064\\
72.5	0.14058	-223.350181515168\\
72.5	0.14424	-238.083313913684\\
72.5	0.1479	-253.638187187614\\
72.5	0.15156	-270.014801336957\\
72.5	0.15522	-287.213156361713\\
72.5	0.15888	-305.233252261882\\
72.5	0.16254	-324.075089037465\\
72.5	0.1662	-343.738666688461\\
72.5	0.16986	-364.22398521487\\
72.5	0.17352	-385.531044616692\\
72.5	0.17718	-407.659844893927\\
72.5	0.18084	-430.610386046575\\
72.5	0.1845	-454.382668074637\\
72.5	0.18816	-478.976690978112\\
72.5	0.19182	-504.392454757\\
72.5	0.19548	-530.629959411302\\
72.5	0.19914	-557.689204941016\\
72.5	0.2028	-585.570191346143\\
72.5	0.20646	-614.272918626685\\
72.5	0.21012	-643.797386782639\\
72.5	0.21378	-674.143595814006\\
72.5	0.21744	-705.311545720786\\
72.5	0.2211	-737.30123650298\\
72.5	0.22476	-770.112668160587\\
72.5	0.22842	-803.745840693607\\
72.5	0.23208	-838.20075410204\\
72.5	0.23574	-873.477408385887\\
72.5	0.2394	-909.575803545147\\
72.5	0.24306	-946.49593957982\\
72.5	0.24672	-984.237816489906\\
72.5	0.25038	-1022.80143427541\\
72.5	0.25404	-1062.18679293632\\
72.5	0.2577	-1102.39389247264\\
72.5	0.26136	-1143.42273288438\\
72.5	0.26502	-1185.27331417153\\
72.5	0.26868	-1227.9456363341\\
72.5	0.27234	-1271.43969937208\\
72.5	0.276	-1315.75550328547\\
72.875	0.093	-108.959164048261\\
72.875	0.09666	-113.021627422155\\
72.875	0.10032	-117.90583167146\\
72.875	0.10398	-123.611776796179\\
72.875	0.10764	-130.139462796312\\
72.875	0.1113	-137.488889671857\\
72.875	0.11496	-145.660057422816\\
72.875	0.11862	-154.652966049188\\
72.875	0.12228	-164.467615550973\\
72.875	0.12594	-175.104005928171\\
72.875	0.1296	-186.562137180783\\
72.875	0.13326	-198.842009308807\\
72.875	0.13692	-211.943622312245\\
72.875	0.14058	-225.866976191096\\
72.875	0.14424	-240.612070945361\\
72.875	0.1479	-256.178906575038\\
72.875	0.15156	-272.567483080129\\
72.875	0.15522	-289.777800460633\\
72.875	0.15888	-307.80985871655\\
72.875	0.16254	-326.66365784788\\
72.875	0.1662	-346.339197854623\\
72.875	0.16986	-366.83647873678\\
72.875	0.17352	-388.15550049435\\
72.875	0.17718	-410.296263127333\\
72.875	0.18084	-433.258766635729\\
72.875	0.1845	-457.043011019539\\
72.875	0.18816	-481.648996278762\\
72.875	0.19182	-507.076722413398\\
72.875	0.19548	-533.326189423447\\
72.875	0.19914	-560.397397308908\\
72.875	0.2028	-588.290346069784\\
72.875	0.20646	-617.005035706073\\
72.875	0.21012	-646.541466217775\\
72.875	0.21378	-676.89963760489\\
72.875	0.21744	-708.079549867418\\
72.875	0.2211	-740.08120300536\\
72.875	0.22476	-772.904597018714\\
72.875	0.22842	-806.549731907482\\
72.875	0.23208	-841.016607671663\\
72.875	0.23574	-876.305224311257\\
72.875	0.2394	-912.415581826265\\
72.875	0.24306	-949.347680216686\\
72.875	0.24672	-987.101519482519\\
72.875	0.25038	-1025.67709962377\\
72.875	0.25404	-1065.07442064043\\
72.875	0.2577	-1105.2934825325\\
72.875	0.26136	-1146.33428529999\\
72.875	0.26502	-1188.19682894289\\
72.875	0.26868	-1230.8811134612\\
72.875	0.27234	-1274.38713885493\\
72.875	0.276	-1318.71490512407\\
73.25	0.093	-111.391481774638\\
73.25	0.09666	-115.465907504279\\
73.25	0.10032	-120.362074109332\\
73.25	0.10398	-126.079981589799\\
73.25	0.10764	-132.619629945679\\
73.25	0.1113	-139.981019176972\\
73.25	0.11496	-148.164149283679\\
73.25	0.11862	-157.169020265798\\
73.25	0.12228	-166.995632123331\\
73.25	0.12594	-177.643984856277\\
73.25	0.1296	-189.114078464637\\
73.25	0.13326	-201.405912948409\\
73.25	0.13692	-214.519488307595\\
73.25	0.14058	-228.454804542194\\
73.25	0.14424	-243.211861652206\\
73.25	0.1479	-258.790659637631\\
73.25	0.15156	-275.191198498469\\
73.25	0.15522	-292.413478234721\\
73.25	0.15888	-310.457498846386\\
73.25	0.16254	-329.323260333464\\
73.25	0.1662	-349.010762695955\\
73.25	0.16986	-369.52000593386\\
73.25	0.17352	-390.850990047177\\
73.25	0.17718	-413.003715035908\\
73.25	0.18084	-435.978180900052\\
73.25	0.1845	-459.774387639609\\
73.25	0.18816	-484.39233525458\\
73.25	0.19182	-509.832023744964\\
73.25	0.19548	-536.09345311076\\
73.25	0.19914	-563.17662335197\\
73.25	0.2028	-591.081534468593\\
73.25	0.20646	-619.80818646063\\
73.25	0.21012	-649.35657932808\\
73.25	0.21378	-679.726713070942\\
73.25	0.21744	-710.918587689218\\
73.25	0.2211	-742.932203182907\\
73.25	0.22476	-775.76755955201\\
73.25	0.22842	-809.424656796526\\
73.25	0.23208	-843.903494916454\\
73.25	0.23574	-879.204073911797\\
73.25	0.2394	-915.326393782552\\
73.25	0.24306	-952.27045452872\\
73.25	0.24672	-990.036256150302\\
73.25	0.25038	-1028.6237986473\\
73.25	0.25404	-1068.0330820197\\
73.25	0.2577	-1108.26410626753\\
73.25	0.26136	-1149.31687139076\\
73.25	0.26502	-1191.19137738941\\
73.25	0.26868	-1233.88762426347\\
73.25	0.27234	-1277.40561201294\\
73.25	0.276	-1321.74534063783\\
73.625	0.093	-113.894833176184\\
73.625	0.09666	-117.981221261572\\
73.625	0.10032	-122.889350222374\\
73.625	0.10398	-128.619220058588\\
73.625	0.10764	-135.170830770216\\
73.625	0.1113	-142.544182357257\\
73.625	0.11496	-150.739274819711\\
73.625	0.11862	-159.756108157579\\
73.625	0.12228	-169.594682370859\\
73.625	0.12594	-180.254997459553\\
73.625	0.1296	-191.73705342366\\
73.625	0.13326	-204.04085026318\\
73.625	0.13692	-217.166387978114\\
73.625	0.14058	-231.11366656846\\
73.625	0.14424	-245.88268603422\\
73.625	0.1479	-261.473446375393\\
73.625	0.15156	-277.885947591979\\
73.625	0.15522	-295.120189683979\\
73.625	0.15888	-313.176172651391\\
73.625	0.16254	-332.053896494217\\
73.625	0.1662	-351.753361212456\\
73.625	0.16986	-372.274566806109\\
73.625	0.17352	-393.617513275174\\
73.625	0.17718	-415.782200619652\\
73.625	0.18084	-438.768628839544\\
73.625	0.1845	-462.576797934849\\
73.625	0.18816	-487.206707905567\\
73.625	0.19182	-512.658358751698\\
73.625	0.19548	-538.931750473243\\
73.625	0.19914	-566.026883070201\\
73.625	0.2028	-593.943756542572\\
73.625	0.20646	-622.682370890356\\
73.625	0.21012	-652.242726113554\\
73.625	0.21378	-682.624822212164\\
73.625	0.21744	-713.828659186188\\
73.625	0.2211	-745.854237035625\\
73.625	0.22476	-778.701555760475\\
73.625	0.22842	-812.370615360738\\
73.625	0.23208	-846.861415836415\\
73.625	0.23574	-882.173957187505\\
73.625	0.2394	-918.308239414008\\
73.625	0.24306	-955.264262515924\\
73.625	0.24672	-993.042026493253\\
73.625	0.25038	-1031.641531346\\
73.625	0.25404	-1071.06277707415\\
73.625	0.2577	-1111.30576367772\\
73.625	0.26136	-1152.3704911567\\
73.625	0.26502	-1194.2569595111\\
73.625	0.26868	-1236.96516874091\\
73.625	0.27234	-1280.49511884613\\
73.625	0.276	-1324.84680982676\\
74	0.093	-116.469218252898\\
74	0.09666	-120.567568694034\\
74	0.10032	-125.487660010583\\
74	0.10398	-131.229492202545\\
74	0.10764	-137.793065269921\\
74	0.1113	-145.17837921271\\
74	0.11496	-153.385434030912\\
74	0.11862	-162.414229724527\\
74	0.12228	-172.264766293555\\
74	0.12594	-182.937043737997\\
74	0.1296	-194.431062057852\\
74	0.13326	-206.746821253119\\
74	0.13692	-219.884321323801\\
74	0.14058	-233.843562269895\\
74	0.14424	-248.624544091403\\
74	0.1479	-264.227266788323\\
74	0.15156	-280.651730360657\\
74	0.15522	-297.897934808405\\
74	0.15888	-315.965880131565\\
74	0.16254	-334.855566330139\\
74	0.1662	-354.566993404125\\
74	0.16986	-375.100161353525\\
74	0.17352	-396.455070178338\\
74	0.17718	-418.631719878564\\
74	0.18084	-441.630110454204\\
74	0.1845	-465.450241905257\\
74	0.18816	-490.092114231723\\
74	0.19182	-515.555727433602\\
74	0.19548	-541.841081510894\\
74	0.19914	-568.9481764636\\
74	0.2028	-596.877012291719\\
74	0.20646	-625.627588995251\\
74	0.21012	-655.199906574196\\
74	0.21378	-685.593965028554\\
74	0.21744	-716.809764358326\\
74	0.2211	-748.84730456351\\
74	0.22476	-781.706585644108\\
74	0.22842	-815.38760760012\\
74	0.23208	-849.890370431544\\
74	0.23574	-885.214874138382\\
74	0.2394	-921.361118720632\\
74	0.24306	-958.329104178296\\
74	0.24672	-996.118830511373\\
74	0.25038	-1034.73029771986\\
74	0.25404	-1074.16350580377\\
74	0.2577	-1114.41845476308\\
74	0.26136	-1155.49514459781\\
74	0.26502	-1197.39357530796\\
74	0.26868	-1240.11374689351\\
74	0.27234	-1283.65565935448\\
74	0.276	-1328.01931269086\\
};
\end{axis}

\begin{axis}[%
width=4.527496cm,
height=3.870968cm,
at={(6.483547cm,10.752688cm)},
scale only axis,
xmin=56,
xmax=74,
tick align=outside,
xlabel={$L_{cut}$},
xmajorgrids,
ymin=0.093,
ymax=0.276,
ylabel={$D_{rlx}$},
ymajorgrids,
zmin=-1.52969162398006,
zmax=2.51190540588646,
zlabel={$x_1,x_1$},
zmajorgrids,
view={-140}{50},
legend style={at={(1.03,1)},anchor=north west,legend cell align=left,align=left,draw=white!15!black}
]
\addplot3[only marks,mark=*,mark options={},mark size=1.5000pt,color=mycolor1] plot table[row sep=crcr,]{%
74	0.123	0.00740370360905991\\
72	0.113	-0.186626587578763\\
61	0.095	-0.150607769785267\\
56	0.093	-0.216975707230133\\
};
\addplot3[only marks,mark=*,mark options={},mark size=1.5000pt,color=mycolor2] plot table[row sep=crcr,]{%
67	0.276	0.0527524251509314\\
66	0.255	-0.475927501616939\\
62	0.209	0.435130634728223\\
57	0.193	0.76117565090573\\
};
\addplot3[only marks,mark=*,mark options={},mark size=1.5000pt,color=black] plot table[row sep=crcr,]{%
69	0.104	-0.0638888311101987\\
};
\addplot3[only marks,mark=*,mark options={},mark size=1.5000pt,color=black] plot table[row sep=crcr,]{%
64	0.23	0.0838528346972987\\
};

\addplot3[%
surf,
opacity=0.7,
shader=interp,
colormap={mymap}{[1pt] rgb(0pt)=(0.0901961,0.239216,0.0745098); rgb(1pt)=(0.0945149,0.242058,0.0739522); rgb(2pt)=(0.0988592,0.244894,0.0733566); rgb(3pt)=(0.103229,0.247724,0.0727241); rgb(4pt)=(0.107623,0.250549,0.0720557); rgb(5pt)=(0.112043,0.253367,0.0713525); rgb(6pt)=(0.116487,0.25618,0.0706154); rgb(7pt)=(0.120956,0.258986,0.0698456); rgb(8pt)=(0.125449,0.261787,0.0690441); rgb(9pt)=(0.129967,0.264581,0.0682118); rgb(10pt)=(0.134508,0.26737,0.06735); rgb(11pt)=(0.139074,0.270152,0.0664596); rgb(12pt)=(0.143663,0.272929,0.0655416); rgb(13pt)=(0.148275,0.275699,0.0645971); rgb(14pt)=(0.152911,0.278463,0.0636271); rgb(15pt)=(0.15757,0.281221,0.0626328); rgb(16pt)=(0.162252,0.283973,0.0616151); rgb(17pt)=(0.166957,0.286719,0.060575); rgb(18pt)=(0.171685,0.289458,0.0595136); rgb(19pt)=(0.176434,0.292191,0.0584321); rgb(20pt)=(0.181207,0.294918,0.0573313); rgb(21pt)=(0.186001,0.297639,0.0562123); rgb(22pt)=(0.190817,0.300353,0.0550763); rgb(23pt)=(0.195655,0.303061,0.0539242); rgb(24pt)=(0.200514,0.305763,0.052757); rgb(25pt)=(0.205395,0.308459,0.0515759); rgb(26pt)=(0.210296,0.311149,0.0503624); rgb(27pt)=(0.215212,0.313846,0.0490067); rgb(28pt)=(0.220142,0.316548,0.0475043); rgb(29pt)=(0.22509,0.319254,0.0458704); rgb(30pt)=(0.230056,0.321962,0.0441205); rgb(31pt)=(0.235042,0.324671,0.04227); rgb(32pt)=(0.240048,0.327379,0.0403343); rgb(33pt)=(0.245078,0.330085,0.0383287); rgb(34pt)=(0.250131,0.332786,0.0362688); rgb(35pt)=(0.25521,0.335482,0.0341698); rgb(36pt)=(0.260317,0.33817,0.0320472); rgb(37pt)=(0.265451,0.340849,0.0299163); rgb(38pt)=(0.270616,0.343517,0.0277927); rgb(39pt)=(0.275813,0.346172,0.0256916); rgb(40pt)=(0.281043,0.348814,0.0236284); rgb(41pt)=(0.286307,0.35144,0.0216186); rgb(42pt)=(0.291607,0.354048,0.0196776); rgb(43pt)=(0.296945,0.356637,0.0178207); rgb(44pt)=(0.302322,0.359206,0.0160634); rgb(45pt)=(0.307739,0.361753,0.0144211); rgb(46pt)=(0.313198,0.364275,0.0129091); rgb(47pt)=(0.318701,0.366772,0.0115428); rgb(48pt)=(0.324249,0.369242,0.0103377); rgb(49pt)=(0.329843,0.371682,0.00930909); rgb(50pt)=(0.335485,0.374093,0.00847245); rgb(51pt)=(0.341176,0.376471,0.00784314); rgb(52pt)=(0.346925,0.378826,0.00732741); rgb(53pt)=(0.352735,0.381168,0.00682184); rgb(54pt)=(0.358605,0.383497,0.00632729); rgb(55pt)=(0.364532,0.385812,0.00584464); rgb(56pt)=(0.370516,0.388113,0.00537476); rgb(57pt)=(0.376552,0.390399,0.00491852); rgb(58pt)=(0.38264,0.39267,0.00447681); rgb(59pt)=(0.388777,0.394925,0.00405048); rgb(60pt)=(0.394962,0.397164,0.00364042); rgb(61pt)=(0.401191,0.399386,0.00324749); rgb(62pt)=(0.407464,0.401592,0.00287258); rgb(63pt)=(0.413777,0.40378,0.00251655); rgb(64pt)=(0.420129,0.40595,0.00218028); rgb(65pt)=(0.426518,0.408102,0.00186463); rgb(66pt)=(0.432942,0.410234,0.00157049); rgb(67pt)=(0.439399,0.412348,0.00129873); rgb(68pt)=(0.445885,0.414441,0.00105022); rgb(69pt)=(0.452401,0.416515,0.000825833); rgb(70pt)=(0.458942,0.418567,0.000626441); rgb(71pt)=(0.465508,0.420599,0.00045292); rgb(72pt)=(0.472096,0.422609,0.000306141); rgb(73pt)=(0.478704,0.424596,0.000186979); rgb(74pt)=(0.485331,0.426562,9.63073e-05); rgb(75pt)=(0.491973,0.428504,3.49981e-05); rgb(76pt)=(0.498628,0.430422,3.92506e-06); rgb(77pt)=(0.505323,0.432315,0); rgb(78pt)=(0.512206,0.434168,0); rgb(79pt)=(0.519282,0.435983,0); rgb(80pt)=(0.526529,0.437764,0); rgb(81pt)=(0.533922,0.439512,0); rgb(82pt)=(0.54144,0.441232,0); rgb(83pt)=(0.549059,0.442927,0); rgb(84pt)=(0.556756,0.444599,0); rgb(85pt)=(0.564508,0.446252,0); rgb(86pt)=(0.572292,0.447889,0); rgb(87pt)=(0.580084,0.449514,0); rgb(88pt)=(0.587863,0.451129,0); rgb(89pt)=(0.595604,0.452737,0); rgb(90pt)=(0.603284,0.454343,0); rgb(91pt)=(0.610882,0.455948,0); rgb(92pt)=(0.618373,0.457556,0); rgb(93pt)=(0.625734,0.459171,0); rgb(94pt)=(0.632943,0.460795,0); rgb(95pt)=(0.639976,0.462432,0); rgb(96pt)=(0.64681,0.464084,0); rgb(97pt)=(0.653423,0.465756,0); rgb(98pt)=(0.659791,0.46745,0); rgb(99pt)=(0.665891,0.469169,0); rgb(100pt)=(0.6717,0.470916,0); rgb(101pt)=(0.677195,0.472696,0); rgb(102pt)=(0.682353,0.47451,0); rgb(103pt)=(0.687242,0.476355,0); rgb(104pt)=(0.691952,0.478225,0); rgb(105pt)=(0.696497,0.480118,0); rgb(106pt)=(0.700887,0.482033,0); rgb(107pt)=(0.705134,0.483968,0); rgb(108pt)=(0.709251,0.485921,0); rgb(109pt)=(0.713249,0.487891,0); rgb(110pt)=(0.71714,0.489876,0); rgb(111pt)=(0.720936,0.491875,0); rgb(112pt)=(0.724649,0.493887,0); rgb(113pt)=(0.72829,0.495909,0); rgb(114pt)=(0.731872,0.49794,0); rgb(115pt)=(0.735406,0.499979,0); rgb(116pt)=(0.738904,0.502025,0); rgb(117pt)=(0.742378,0.504075,0); rgb(118pt)=(0.74584,0.506128,0); rgb(119pt)=(0.749302,0.508182,0); rgb(120pt)=(0.752775,0.510237,0); rgb(121pt)=(0.756272,0.51229,0); rgb(122pt)=(0.759804,0.514339,0); rgb(123pt)=(0.763384,0.516385,0); rgb(124pt)=(0.767022,0.518424,0); rgb(125pt)=(0.770731,0.520455,0); rgb(126pt)=(0.774523,0.522478,0); rgb(127pt)=(0.77841,0.524489,0); rgb(128pt)=(0.782391,0.526491,0); rgb(129pt)=(0.786402,0.528496,0); rgb(130pt)=(0.790431,0.530506,0); rgb(131pt)=(0.794478,0.532521,0); rgb(132pt)=(0.798541,0.534539,0); rgb(133pt)=(0.802619,0.53656,0); rgb(134pt)=(0.806712,0.538584,0); rgb(135pt)=(0.81082,0.540609,0); rgb(136pt)=(0.81494,0.542635,0); rgb(137pt)=(0.819074,0.54466,0); rgb(138pt)=(0.823219,0.546686,0); rgb(139pt)=(0.827374,0.548709,0); rgb(140pt)=(0.831541,0.55073,0); rgb(141pt)=(0.835716,0.552749,0); rgb(142pt)=(0.8399,0.554763,0); rgb(143pt)=(0.844092,0.556774,0); rgb(144pt)=(0.848292,0.558779,0); rgb(145pt)=(0.852497,0.560778,0); rgb(146pt)=(0.856708,0.562771,0); rgb(147pt)=(0.860924,0.564756,0); rgb(148pt)=(0.865143,0.566733,0); rgb(149pt)=(0.869366,0.568701,0); rgb(150pt)=(0.873592,0.57066,0); rgb(151pt)=(0.877819,0.572608,0); rgb(152pt)=(0.882047,0.574545,0); rgb(153pt)=(0.886275,0.576471,0); rgb(154pt)=(0.890659,0.578362,0); rgb(155pt)=(0.895333,0.580203,0); rgb(156pt)=(0.900258,0.581999,0); rgb(157pt)=(0.905397,0.583755,0); rgb(158pt)=(0.910711,0.585479,0); rgb(159pt)=(0.916164,0.587176,0); rgb(160pt)=(0.921717,0.588852,0); rgb(161pt)=(0.927333,0.590513,0); rgb(162pt)=(0.932974,0.592166,0); rgb(163pt)=(0.938602,0.593815,0); rgb(164pt)=(0.94418,0.595468,0); rgb(165pt)=(0.949669,0.59713,0); rgb(166pt)=(0.955033,0.598808,0); rgb(167pt)=(0.960233,0.600507,0); rgb(168pt)=(0.965232,0.602233,0); rgb(169pt)=(0.969992,0.603992,0); rgb(170pt)=(0.974475,0.605791,0); rgb(171pt)=(0.978643,0.607636,0); rgb(172pt)=(0.98246,0.609532,0); rgb(173pt)=(0.985886,0.611486,0); rgb(174pt)=(0.988885,0.613503,0); rgb(175pt)=(0.991419,0.61559,0); rgb(176pt)=(0.99345,0.617753,0); rgb(177pt)=(0.99494,0.619997,0); rgb(178pt)=(0.995851,0.622329,0); rgb(179pt)=(0.996226,0.624763,0); rgb(180pt)=(0.996512,0.627352,0); rgb(181pt)=(0.996788,0.630095,0); rgb(182pt)=(0.997053,0.632982,0); rgb(183pt)=(0.997308,0.636004,0); rgb(184pt)=(0.997552,0.639152,0); rgb(185pt)=(0.997785,0.642416,0); rgb(186pt)=(0.998006,0.645786,0); rgb(187pt)=(0.998217,0.649253,0); rgb(188pt)=(0.998416,0.652807,0); rgb(189pt)=(0.998605,0.656439,0); rgb(190pt)=(0.998781,0.660138,0); rgb(191pt)=(0.998946,0.663897,0); rgb(192pt)=(0.9991,0.667704,0); rgb(193pt)=(0.999242,0.67155,0); rgb(194pt)=(0.999372,0.675427,0); rgb(195pt)=(0.99949,0.679323,0); rgb(196pt)=(0.999596,0.68323,0); rgb(197pt)=(0.99969,0.687139,0); rgb(198pt)=(0.999771,0.691039,0); rgb(199pt)=(0.999841,0.694921,0); rgb(200pt)=(0.999898,0.698775,0); rgb(201pt)=(0.999942,0.702592,0); rgb(202pt)=(0.999974,0.706363,0); rgb(203pt)=(0.999994,0.710077,0); rgb(204pt)=(1,0.713725,0); rgb(205pt)=(1,0.717341,0); rgb(206pt)=(1,0.720963,0); rgb(207pt)=(1,0.724591,0); rgb(208pt)=(1,0.728226,0); rgb(209pt)=(1,0.731867,0); rgb(210pt)=(1,0.735514,0); rgb(211pt)=(1,0.739167,0); rgb(212pt)=(1,0.742827,0); rgb(213pt)=(1,0.746493,0); rgb(214pt)=(1,0.750165,0); rgb(215pt)=(1,0.753843,0); rgb(216pt)=(1,0.757527,0); rgb(217pt)=(1,0.761217,0); rgb(218pt)=(1,0.764913,0); rgb(219pt)=(1,0.768615,0); rgb(220pt)=(1,0.772324,0); rgb(221pt)=(1,0.776038,0); rgb(222pt)=(1,0.779758,0); rgb(223pt)=(1,0.783484,0); rgb(224pt)=(1,0.787215,0); rgb(225pt)=(1,0.790953,0); rgb(226pt)=(1,0.794696,0); rgb(227pt)=(1,0.798445,0); rgb(228pt)=(1,0.8022,0); rgb(229pt)=(1,0.805961,0); rgb(230pt)=(1,0.809727,0); rgb(231pt)=(1,0.8135,0); rgb(232pt)=(1,0.817278,0); rgb(233pt)=(1,0.821063,0); rgb(234pt)=(1,0.824854,0); rgb(235pt)=(1,0.828652,0); rgb(236pt)=(1,0.832455,0); rgb(237pt)=(1,0.836265,0); rgb(238pt)=(1,0.840081,0); rgb(239pt)=(1,0.843903,0); rgb(240pt)=(1,0.847732,0); rgb(241pt)=(1,0.851566,0); rgb(242pt)=(1,0.855406,0); rgb(243pt)=(1,0.859253,0); rgb(244pt)=(1,0.863106,0); rgb(245pt)=(1,0.866964,0); rgb(246pt)=(1,0.870829,0); rgb(247pt)=(1,0.8747,0); rgb(248pt)=(1,0.878577,0); rgb(249pt)=(1,0.88246,0); rgb(250pt)=(1,0.886349,0); rgb(251pt)=(1,0.890243,0); rgb(252pt)=(1,0.894144,0); rgb(253pt)=(1,0.898051,0); rgb(254pt)=(1,0.901964,0); rgb(255pt)=(1,0.905882,0)},
mesh/rows=49]
table[row sep=crcr,header=false] {%
%
56	0.093	-0.183093463550335\\
56	0.09666	-0.15506912383271\\
56	0.10032	-0.125988635638709\\
56	0.10398	-0.095851998968341\\
56	0.10764	-0.0646592138216007\\
56	0.1113	-0.032410280198485\\
56	0.11496	0.000894801900998998\\
56	0.11862	0.0352560324768565\\
56	0.12228	0.0706734115290841\\
56	0.12594	0.107146939057687\\
56	0.1296	0.14467661506266\\
56	0.13326	0.183262439544003\\
56	0.13692	0.22290441250172\\
56	0.14058	0.263602533935807\\
56	0.14424	0.305356803846267\\
56	0.1479	0.348167222233099\\
56	0.15156	0.392033789096303\\
56	0.15522	0.436956504435879\\
56	0.15888	0.482935368251827\\
56	0.16254	0.529970380544145\\
56	0.1662	0.578061541312836\\
56	0.16986	0.627208850557897\\
56	0.17352	0.677412308279332\\
56	0.17718	0.728671914477137\\
56	0.18084	0.780987669151318\\
56	0.1845	0.834359572301868\\
56	0.18816	0.888787623928789\\
56	0.19182	0.944271824032083\\
56	0.19548	1.00081217261175\\
56	0.19914	1.05840866966778\\
56	0.2028	1.11706131520019\\
56	0.20646	1.17677010920897\\
56	0.21012	1.23753505169413\\
56	0.21378	1.29935614265565\\
56	0.21744	1.36223338209355\\
56	0.2211	1.42616677000781\\
56	0.22476	1.49115630639846\\
56	0.22842	1.55720199126547\\
56	0.23208	1.62430382460885\\
56	0.23574	1.69246180642861\\
56	0.2394	1.76167593672473\\
56	0.24306	1.83194621549724\\
56	0.24672	1.90327264274611\\
56	0.25038	1.97565521847135\\
56	0.25404	2.04909394267296\\
56	0.2577	2.12358881535095\\
56	0.26136	2.19913983650531\\
56	0.26502	2.27574700613604\\
56	0.26868	2.35341032424314\\
56	0.27234	2.43212979082662\\
56	0.276	2.51190540588646\\
56.375	0.093	-0.18948708924544\\
56.375	0.09666	-0.163384241142382\\
56.375	0.10032	-0.136225244562956\\
56.375	0.10398	-0.108010099507157\\
56.375	0.10764	-0.0787388059749854\\
56.375	0.1113	-0.0484113639664421\\
56.375	0.11496	-0.0170277734815271\\
56.375	0.11862	0.0154119654797598\\
56.375	0.12228	0.0489078529174185\\
56.375	0.12594	0.0834598888314491\\
56.375	0.1296	0.119068073221851\\
56.375	0.13326	0.155732406088625\\
56.375	0.13692	0.193452887431771\\
56.375	0.14058	0.232229517251288\\
56.375	0.14424	0.272062295547178\\
56.375	0.1479	0.312951222319441\\
56.375	0.15156	0.354896297568072\\
56.375	0.15522	0.397897521293079\\
56.375	0.15888	0.441954893494456\\
56.375	0.16254	0.487068414172203\\
56.375	0.1662	0.533238083326324\\
56.375	0.16986	0.580463900956817\\
56.375	0.17352	0.628745867063681\\
56.375	0.17718	0.678083981646915\\
56.375	0.18084	0.728478244706525\\
56.375	0.1845	0.779928656242505\\
56.375	0.18816	0.832435216254855\\
56.375	0.19182	0.88599792474358\\
56.375	0.19548	0.940616781708675\\
56.375	0.19914	0.99629178715014\\
56.375	0.2028	1.05302294106798\\
56.375	0.20646	1.11081024346219\\
56.375	0.21012	1.16965369433278\\
56.375	0.21378	1.22955329367973\\
56.375	0.21744	1.29050904150305\\
56.375	0.2211	1.35252093780275\\
56.375	0.22476	1.41558898257882\\
56.375	0.22842	1.47971317583126\\
56.375	0.23208	1.54489351756008\\
56.375	0.23574	1.61113000776526\\
56.375	0.2394	1.67842264644682\\
56.375	0.24306	1.74677143360475\\
56.375	0.24672	1.81617636923905\\
56.375	0.25038	1.88663745334972\\
56.375	0.25404	1.95815468593677\\
56.375	0.2577	2.03072806700018\\
56.375	0.26136	2.10435759653997\\
56.375	0.26502	2.17904327455613\\
56.375	0.26868	2.25478510104866\\
56.375	0.27234	2.33158307601756\\
56.375	0.276	2.40943719946284\\
56.75	0.093	-0.195103341821336\\
56.75	0.09666	-0.170921985332851\\
56.75	0.10032	-0.145684480367992\\
56.75	0.10398	-0.119390826926765\\
56.75	0.10764	-0.0920410250091641\\
56.75	0.1113	-0.0636350746151897\\
56.75	0.11496	-0.0341729757448453\\
56.75	0.11862	-0.00365472839812919\\
56.75	0.12228	0.0279196674249589\\
56.75	0.12594	0.0605502117244188\\
56.75	0.1296	0.0942369045002521\\
56.75	0.13326	0.128979745752454\\
56.75	0.13692	0.164778735481031\\
56.75	0.14058	0.201633873685977\\
56.75	0.14424	0.239545160367297\\
56.75	0.1479	0.278512595524988\\
56.75	0.15156	0.318536179159051\\
56.75	0.15522	0.359615911269485\\
56.75	0.15888	0.401751791856294\\
56.75	0.16254	0.44494382091947\\
56.75	0.1662	0.489191998459022\\
56.75	0.16986	0.534496324474942\\
56.75	0.17352	0.580856798967237\\
56.75	0.17718	0.628273421935903\\
56.75	0.18084	0.67674619338094\\
56.75	0.1845	0.726275113302351\\
56.75	0.18816	0.77686018170013\\
56.75	0.19182	0.828501398574285\\
56.75	0.19548	0.881198763924809\\
56.75	0.19914	0.934952277751706\\
56.75	0.2028	0.989761940054972\\
56.75	0.20646	1.04562775083461\\
56.75	0.21012	1.10254971009063\\
56.75	0.21378	1.16052781782301\\
56.75	0.21744	1.21956207403177\\
56.75	0.2211	1.27965247871689\\
56.75	0.22476	1.34079903187839\\
56.75	0.22842	1.40300173351626\\
56.75	0.23208	1.46626058363051\\
56.75	0.23574	1.53057558222112\\
56.75	0.2394	1.59594672928811\\
56.75	0.24306	1.66237402483147\\
56.75	0.24672	1.7298574688512\\
56.75	0.25038	1.7983970613473\\
56.75	0.25404	1.86799280231977\\
56.75	0.2577	1.93864469176862\\
56.75	0.26136	2.01035272969384\\
56.75	0.26502	2.08311691609543\\
56.75	0.26868	2.15693725097339\\
56.75	0.27234	2.23181373432773\\
56.75	0.276	2.30774636615843\\
57.125	0.093	-0.199942221278023\\
57.125	0.09666	-0.17768235640411\\
57.125	0.10032	-0.154366343053822\\
57.125	0.10398	-0.129994181227162\\
57.125	0.10764	-0.104565870924133\\
57.125	0.1113	-0.0780814121447296\\
57.125	0.11496	-0.0505408048889558\\
57.125	0.11862	-0.0219440491568086\\
57.125	0.12228	0.00770885505170882\\
57.125	0.12594	0.038417907736598\\
57.125	0.1296	0.0701831088978607\\
57.125	0.13326	0.103004458535494\\
57.125	0.13692	0.1368819566495\\
57.125	0.14058	0.171815603239875\\
57.125	0.14424	0.207805398306625\\
57.125	0.1479	0.244851341849747\\
57.125	0.15156	0.282953433869239\\
57.125	0.15522	0.322111674365103\\
57.125	0.15888	0.362326063337341\\
57.125	0.16254	0.403596600785948\\
57.125	0.1662	0.445923286710927\\
57.125	0.16986	0.489306121112278\\
57.125	0.17352	0.533745103990003\\
57.125	0.17718	0.579240235344098\\
57.125	0.18084	0.625791515174564\\
57.125	0.1845	0.673398943481404\\
57.125	0.18816	0.722062520264615\\
57.125	0.19182	0.771782245524197\\
57.125	0.19548	0.822558119260153\\
57.125	0.19914	0.874390141472479\\
57.125	0.2028	0.927278312161178\\
57.125	0.20646	0.981222631326246\\
57.125	0.21012	1.03622309896769\\
57.125	0.21378	1.0922797150855\\
57.125	0.21744	1.14939247967969\\
57.125	0.2211	1.20756139275025\\
57.125	0.22476	1.26678645429718\\
57.125	0.22842	1.32706766432047\\
57.125	0.23208	1.38840502282015\\
57.125	0.23574	1.45079852979619\\
57.125	0.2394	1.51424818524861\\
57.125	0.24306	1.5787539891774\\
57.125	0.24672	1.64431594158256\\
57.125	0.25038	1.71093404246409\\
57.125	0.25404	1.77860829182199\\
57.125	0.2577	1.84733868965627\\
57.125	0.26136	1.91712523596692\\
57.125	0.26502	1.98796793075394\\
57.125	0.26868	2.05986677401732\\
57.125	0.27234	2.13282176575709\\
57.125	0.276	2.20683290597322\\
57.5	0.093	-0.204003727615505\\
57.5	0.09666	-0.183665354356161\\
57.5	0.10032	-0.162270832620443\\
57.5	0.10398	-0.139820162408354\\
57.5	0.10764	-0.116313343719895\\
57.5	0.1113	-0.0917503765550616\\
57.5	0.11496	-0.0661312609138586\\
57.5	0.11862	-0.039455996796282\\
57.5	0.12228	-0.0117245842023352\\
57.5	0.12594	0.0170629768679851\\
57.5	0.1296	0.0469066864146771\\
57.5	0.13326	0.0778065444377393\\
57.5	0.13692	0.109762550937175\\
57.5	0.14058	0.142774705912979\\
57.5	0.14424	0.176843009365159\\
57.5	0.1479	0.21196746129371\\
57.5	0.15156	0.248148061698631\\
57.5	0.15522	0.285384810579926\\
57.5	0.15888	0.323677707937593\\
57.5	0.16254	0.36302675377163\\
57.5	0.1662	0.403431948082039\\
57.5	0.16986	0.444893290868819\\
57.5	0.17352	0.487410782131973\\
57.5	0.17718	0.530984421871497\\
57.5	0.18084	0.575614210087395\\
57.5	0.1845	0.621300146779664\\
57.5	0.18816	0.668042231948304\\
57.5	0.19182	0.715840465593317\\
57.5	0.19548	0.764694847714702\\
57.5	0.19914	0.814605378312456\\
57.5	0.2028	0.865572057386585\\
57.5	0.20646	0.917594884937085\\
57.5	0.21012	0.970673860963958\\
57.5	0.21378	1.0248089854672\\
57.5	0.21744	1.08000025844681\\
57.5	0.2211	1.1362476799028\\
57.5	0.22476	1.19355124983516\\
57.5	0.22842	1.25191096824389\\
57.5	0.23208	1.31132683512899\\
57.5	0.23574	1.37179885049047\\
57.5	0.2394	1.43332701432831\\
57.5	0.24306	1.49591132664253\\
57.5	0.24672	1.55955178743312\\
57.5	0.25038	1.62424839670008\\
57.5	0.25404	1.69000115444341\\
57.5	0.2577	1.75681006066312\\
57.5	0.26136	1.8246751153592\\
57.5	0.26502	1.89359631853165\\
57.5	0.26868	1.96357367018047\\
57.5	0.27234	2.03460717030566\\
57.5	0.276	2.10669681890722\\
57.875	0.093	-0.207287860833774\\
57.875	0.09666	-0.188870979188999\\
57.875	0.10032	-0.169397949067852\\
57.875	0.10398	-0.148868770470335\\
57.875	0.10764	-0.127283443396446\\
57.875	0.1113	-0.104641967846182\\
57.875	0.11496	-0.08094434381955\\
57.875	0.11862	-0.056190571316544\\
57.875	0.12228	-0.030380650337168\\
57.875	0.12594	-0.00351458088141832\\
57.875	0.1296	0.024407637050703\\
57.875	0.13326	0.0533860034591945\\
57.875	0.13692	0.0834205183440596\\
57.875	0.14058	0.114511181705295\\
57.875	0.14424	0.146657993542904\\
57.875	0.1479	0.179860953856886\\
57.875	0.15156	0.214120062647237\\
57.875	0.15522	0.249435319913961\\
57.875	0.15888	0.285806725657057\\
57.875	0.16254	0.323234279876524\\
57.875	0.1662	0.361717982572363\\
57.875	0.16986	0.401257833744573\\
57.875	0.17352	0.441853833393156\\
57.875	0.17718	0.48350598151811\\
57.875	0.18084	0.526214278119437\\
57.875	0.1845	0.569978723197137\\
57.875	0.18816	0.614799316751205\\
57.875	0.19182	0.660676058781649\\
57.875	0.19548	0.707608949288463\\
57.875	0.19914	0.755597988271646\\
57.875	0.2028	0.804643175731204\\
57.875	0.20646	0.854744511667134\\
57.875	0.21012	0.905901996079436\\
57.875	0.21378	0.958115628968109\\
57.875	0.21744	1.01138541033315\\
57.875	0.2211	1.06571134017457\\
57.875	0.22476	1.12109341849236\\
57.875	0.22842	1.17753164528652\\
57.875	0.23208	1.23502602055705\\
57.875	0.23574	1.29357654430396\\
57.875	0.2394	1.35318321652723\\
57.875	0.24306	1.41384603722688\\
57.875	0.24672	1.4755650064029\\
57.875	0.25038	1.53834012405529\\
57.875	0.25404	1.60217139018405\\
57.875	0.2577	1.66705880478919\\
57.875	0.26136	1.73300236787069\\
57.875	0.26502	1.80000207942857\\
57.875	0.26868	1.86805793946282\\
57.875	0.27234	1.93716994797345\\
57.875	0.276	2.00733810496044\\
58.25	0.093	-0.209794620932834\\
58.25	0.09666	-0.193299230902629\\
58.25	0.10032	-0.175747692396053\\
58.25	0.10398	-0.157140005413105\\
58.25	0.10764	-0.137476169953787\\
58.25	0.1113	-0.116756186018094\\
58.25	0.11496	-0.0949800536060318\\
58.25	0.11862	-0.0721477727175948\\
58.25	0.12228	-0.0482593433527894\\
58.25	0.12594	-0.0233147655116104\\
58.25	0.1296	0.00268596080594208\\
58.25	0.13326	0.0297428355998629\\
58.25	0.13692	0.0578558588701573\\
58.25	0.14058	0.0870250306168221\\
58.25	0.14424	0.117250350839862\\
58.25	0.1479	0.148531819539272\\
58.25	0.15156	0.180869436715052\\
58.25	0.15522	0.214263202367207\\
58.25	0.15888	0.248713116495733\\
58.25	0.16254	0.284219179100628\\
58.25	0.1662	0.320781390181897\\
58.25	0.16986	0.358399749739538\\
58.25	0.17352	0.397074257773551\\
58.25	0.17718	0.436804914283934\\
58.25	0.18084	0.477591719270692\\
58.25	0.1845	0.51943467273382\\
58.25	0.18816	0.562333774673319\\
58.25	0.19182	0.60628902508919\\
58.25	0.19548	0.651300423981436\\
58.25	0.19914	0.697367971350051\\
58.25	0.2028	0.744491667195039\\
58.25	0.20646	0.792671511516395\\
58.25	0.21012	0.841907504314129\\
58.25	0.21378	0.892199645588232\\
58.25	0.21744	0.943547935338703\\
58.25	0.2211	0.995952373565549\\
58.25	0.22476	1.04941296026877\\
58.25	0.22842	1.10392969544836\\
58.25	0.23208	1.15950257910432\\
58.25	0.23574	1.21613161123665\\
58.25	0.2394	1.27381679184536\\
58.25	0.24306	1.33255812093044\\
58.25	0.24672	1.39235559849189\\
58.25	0.25038	1.45320922452971\\
58.25	0.25404	1.5151189990439\\
58.25	0.2577	1.57808492203447\\
58.25	0.26136	1.6421069935014\\
58.25	0.26502	1.70718521344471\\
58.25	0.26868	1.77331958186439\\
58.25	0.27234	1.84051009876044\\
58.25	0.276	1.90875676413287\\
58.625	0.093	-0.211524007912687\\
58.625	0.09666	-0.196950109497054\\
58.625	0.10032	-0.181320062605044\\
58.625	0.10398	-0.164633867236669\\
58.625	0.10764	-0.14689152339192\\
58.625	0.1113	-0.128093031070799\\
58.625	0.11496	-0.108238390273306\\
58.625	0.11862	-0.0873276009994395\\
58.625	0.12228	-0.0653606632492048\\
58.625	0.12594	-0.0423375770225947\\
58.625	0.1296	-0.0182583423196147\\
58.625	0.13326	0.00687704085973728\\
58.625	0.13692	0.033068572515461\\
58.625	0.14058	0.0603162526475551\\
58.625	0.14424	0.0886200812560244\\
58.625	0.1479	0.117980058340864\\
58.625	0.15156	0.148396183902075\\
58.625	0.15522	0.179868457939659\\
58.625	0.15888	0.212396880453614\\
58.625	0.16254	0.245981451443939\\
58.625	0.1662	0.280622170910639\\
58.625	0.16986	0.316319038853708\\
58.625	0.17352	0.353072055273151\\
58.625	0.17718	0.390881220168964\\
58.625	0.18084	0.429746533541151\\
58.625	0.1845	0.469667995389708\\
58.625	0.18816	0.510645605714638\\
58.625	0.19182	0.552679364515939\\
58.625	0.19548	0.595769271793614\\
58.625	0.19914	0.639915327547659\\
58.625	0.2028	0.685117531778076\\
58.625	0.20646	0.731375884484865\\
58.625	0.21012	0.778690385668025\\
58.625	0.21378	0.827061035327557\\
58.625	0.21744	0.876487833463461\\
58.625	0.2211	0.926970780075737\\
58.625	0.22476	0.978509875164384\\
58.625	0.22842	1.0311051187294\\
58.625	0.23208	1.08475651077079\\
58.625	0.23574	1.13946405128856\\
58.625	0.2394	1.19522774028269\\
58.625	0.24306	1.2520475777532\\
58.625	0.24672	1.30992356370008\\
58.625	0.25038	1.36885569812333\\
58.625	0.25404	1.42884398102295\\
58.625	0.2577	1.48988841239895\\
58.625	0.26136	1.55198899225132\\
58.625	0.26502	1.61514572058005\\
58.625	0.26868	1.67935859738516\\
58.625	0.27234	1.74462762266664\\
58.625	0.276	1.8109527964245\\
59	0.093	-0.212476021773331\\
59	0.09666	-0.199823614972266\\
59	0.10032	-0.186115059694831\\
59	0.10398	-0.171350355941025\\
59	0.10764	-0.155529503710846\\
59	0.1113	-0.138652503004294\\
59	0.11496	-0.120719353821372\\
59	0.11862	-0.101730056162078\\
59	0.12228	-0.0816846100264124\\
59	0.12594	-0.0605830154143729\\
59	0.1296	-0.0384252723259618\\
59	0.13326	-0.0152113807611823\\
59	0.13692	0.00905865927997251\\
59	0.14058	0.034384847797496\\
59	0.14424	0.0607671847913945\\
59	0.1479	0.0882056702616648\\
59	0.15156	0.116700304208305\\
59	0.15522	0.146251086631317\\
59	0.15888	0.176858017530704\\
59	0.16254	0.208521096906459\\
59	0.1662	0.241240324758587\\
59	0.16986	0.275015701087087\\
59	0.17352	0.30984722589196\\
59	0.17718	0.345734899173201\\
59	0.18084	0.382678720930818\\
59	0.1845	0.420678691164807\\
59	0.18816	0.459734809875164\\
59	0.19182	0.499847077061896\\
59	0.19548	0.541015492725\\
59	0.19914	0.583240056864474\\
59	0.2028	0.626520769480321\\
59	0.20646	0.670857630572539\\
59	0.21012	0.716250640141132\\
59	0.21378	0.762699798186093\\
59	0.21744	0.810205104707423\\
59	0.2211	0.858766559705128\\
59	0.22476	0.908384163179208\\
59	0.22842	0.959057915129657\\
59	0.23208	1.01078781555648\\
59	0.23574	1.06357386445967\\
59	0.2394	1.11741606183924\\
59	0.24306	1.17231440769517\\
59	0.24672	1.22826890202748\\
59	0.25038	1.28527954483616\\
59	0.25404	1.34334633612121\\
59	0.2577	1.40246927588264\\
59	0.26136	1.46264836412043\\
59	0.26502	1.5238836008346\\
59	0.26868	1.58617498602514\\
59	0.27234	1.64952251969205\\
59	0.276	1.71392620183533\\
59.375	0.093	-0.212650662514767\\
59.375	0.09666	-0.201919747328275\\
59.375	0.10032	-0.190132683665409\\
59.375	0.10398	-0.177289471526173\\
59.375	0.10764	-0.163390110910565\\
59.375	0.1113	-0.148434601818582\\
59.375	0.11496	-0.132422944250232\\
59.375	0.11862	-0.115355138205507\\
59.375	0.12228	-0.0972311836844122\\
59.375	0.12594	-0.0780510806869434\\
59.375	0.1296	-0.057814829213103\\
59.375	0.13326	-0.0365224292628924\\
59.375	0.13692	-0.0141738808363082\\
59.375	0.14058	0.00923081606664455\\
59.375	0.14424	0.0336916614459725\\
59.375	0.1479	0.0592086553016721\\
59.375	0.15156	0.0857817976337418\\
59.375	0.15522	0.113411088442185\\
59.375	0.15888	0.142096527727001\\
59.375	0.16254	0.171838115488184\\
59.375	0.1662	0.202635851725743\\
59.375	0.16986	0.234489736439672\\
59.375	0.17352	0.267399769629974\\
59.375	0.17718	0.301365951296647\\
59.375	0.18084	0.336388281439693\\
59.375	0.1845	0.372466760059111\\
59.375	0.18816	0.409601387154897\\
59.375	0.19182	0.447792162727059\\
59.375	0.19548	0.487039086775592\\
59.375	0.19914	0.527342159300496\\
59.375	0.2028	0.568701380301772\\
59.375	0.20646	0.611116749779419\\
59.375	0.21012	0.654588267733441\\
59.375	0.21378	0.699115934163832\\
59.375	0.21744	0.744699749070594\\
59.375	0.2211	0.791339712453729\\
59.375	0.22476	0.839035824313239\\
59.375	0.22842	0.887788084649117\\
59.375	0.23208	0.937596493461367\\
59.375	0.23574	0.988461050749989\\
59.375	0.2394	1.04038175651498\\
59.375	0.24306	1.09335861075635\\
59.375	0.24672	1.14739161347409\\
59.375	0.25038	1.2024807646682\\
59.375	0.25404	1.25862606433868\\
59.375	0.2577	1.31582751248554\\
59.375	0.26136	1.37408510910876\\
59.375	0.26502	1.43339885420836\\
59.375	0.26868	1.49376874778432\\
59.375	0.27234	1.55519478983667\\
59.375	0.276	1.61767698036538\\
59.75	0.093	-0.212047930136994\\
59.75	0.09666	-0.203238506565073\\
59.75	0.10032	-0.193372934516779\\
59.75	0.10398	-0.182451213992112\\
59.75	0.10764	-0.170473344991073\\
59.75	0.1113	-0.157439327513662\\
59.75	0.11496	-0.143349161559881\\
59.75	0.11862	-0.128202847129727\\
59.75	0.12228	-0.112000384223202\\
59.75	0.12594	-0.0947417728403044\\
59.75	0.1296	-0.0764270129810346\\
59.75	0.13326	-0.0570561046453929\\
59.75	0.13692	-0.0366290478333794\\
59.75	0.14058	-0.0151458425449973\\
59.75	0.14424	0.00739351121976173\\
59.75	0.1479	0.0309890134608907\\
59.75	0.15156	0.0556406641783898\\
59.75	0.15522	0.0813484633722624\\
59.75	0.15888	0.108112411042507\\
59.75	0.16254	0.135932507189122\\
59.75	0.1662	0.16480875181211\\
59.75	0.16986	0.194741144911468\\
59.75	0.17352	0.2257296864872\\
59.75	0.17718	0.257774376539302\\
59.75	0.18084	0.290875215067777\\
59.75	0.1845	0.325032202072624\\
59.75	0.18816	0.360245337553842\\
59.75	0.19182	0.396514621511433\\
59.75	0.19548	0.433840053945396\\
59.75	0.19914	0.472221634855729\\
59.75	0.2028	0.511659364242437\\
59.75	0.20646	0.552153242105514\\
59.75	0.21012	0.593703268444965\\
59.75	0.21378	0.636309443260785\\
59.75	0.21744	0.679971766552977\\
59.75	0.2211	0.724690238321541\\
59.75	0.22476	0.77046485856648\\
59.75	0.22842	0.817295627287788\\
59.75	0.23208	0.865182544485467\\
59.75	0.23574	0.914125610159519\\
59.75	0.2394	0.964124824309942\\
59.75	0.24306	1.01518018693674\\
59.75	0.24672	1.06729169803991\\
59.75	0.25038	1.12045935761945\\
59.75	0.25404	1.17468316567536\\
59.75	0.2577	1.22996312220764\\
59.75	0.26136	1.2862992272163\\
59.75	0.26502	1.34369148070132\\
59.75	0.26868	1.40213988266272\\
59.75	0.27234	1.46164443310049\\
59.75	0.276	1.52220513201464\\
60.125	0.093	-0.210667824640015\\
60.125	0.09666	-0.203779892682662\\
60.125	0.10032	-0.195835812248937\\
60.125	0.10398	-0.186835583338841\\
60.125	0.10764	-0.176779205952375\\
60.125	0.1113	-0.165666680089534\\
60.125	0.11496	-0.153498005750322\\
60.125	0.11862	-0.140273182934739\\
60.125	0.12228	-0.125992211642785\\
60.125	0.12594	-0.110655091874456\\
60.125	0.1296	-0.0942618236297567\\
60.125	0.13326	-0.0768124069086874\\
60.125	0.13692	-0.0583068417112428\\
60.125	0.14058	-0.0387451280374296\\
60.125	0.14424	-0.018127265887243\\
60.125	0.1479	0.00354674473931527\\
60.125	0.15156	0.0262769038422437\\
60.125	0.15522	0.0500632114215475\\
60.125	0.15888	0.0749056674772216\\
60.125	0.16254	0.100804272009265\\
60.125	0.1662	0.127759025017683\\
60.125	0.16986	0.155769926502472\\
60.125	0.17352	0.184836976463633\\
60.125	0.17718	0.214960174901164\\
60.125	0.18084	0.246139521815069\\
60.125	0.1845	0.278375017205348\\
60.125	0.18816	0.311666661071995\\
60.125	0.19182	0.346014453415015\\
60.125	0.19548	0.381418394234407\\
60.125	0.19914	0.417878483530169\\
60.125	0.2028	0.455394721302307\\
60.125	0.20646	0.493967107550813\\
60.125	0.21012	0.533595642275694\\
60.125	0.21378	0.574280325476943\\
60.125	0.21744	0.616021157154565\\
60.125	0.2211	0.658818137308561\\
60.125	0.22476	0.70267126593893\\
60.125	0.22842	0.747580543045666\\
60.125	0.23208	0.793545968628775\\
60.125	0.23574	0.840567542688256\\
60.125	0.2394	0.888645265224112\\
60.125	0.24306	0.937779136236339\\
60.125	0.24672	0.987969155724935\\
60.125	0.25038	1.0392153236899\\
60.125	0.25404	1.09151764013125\\
60.125	0.2577	1.14487610504896\\
60.125	0.26136	1.19929071844304\\
60.125	0.26502	1.2547614803135\\
60.125	0.26868	1.31128839066033\\
60.125	0.27234	1.36887144948353\\
60.125	0.276	1.4275106567831\\
60.5	0.093	-0.208510346023826\\
60.5	0.09666	-0.203543905681044\\
60.5	0.10032	-0.19752131686189\\
60.5	0.10398	-0.190442579566362\\
60.5	0.10764	-0.182307693794467\\
60.5	0.1113	-0.173116659546195\\
60.5	0.11496	-0.162869476821556\\
60.5	0.11862	-0.151566145620541\\
60.5	0.12228	-0.139206665943158\\
60.5	0.12594	-0.125791037789399\\
60.5	0.1296	-0.111319261159271\\
60.5	0.13326	-0.0957913360527706\\
60.5	0.13692	-0.0792072624698984\\
60.5	0.14058	-0.0615670404106541\\
60.5	0.14424	-0.0428706698750382\\
60.5	0.1479	-0.0231181508630488\\
60.5	0.15156	-0.00230948337469106\\
60.5	0.15522	0.0195553325900421\\
60.5	0.15888	0.0424762970311455\\
60.5	0.16254	0.0664534099486187\\
60.5	0.1662	0.0914866713424674\\
60.5	0.16986	0.117576081212684\\
60.5	0.17352	0.144721639559276\\
60.5	0.17718	0.172923346382237\\
60.5	0.18084	0.202181201681573\\
60.5	0.1845	0.232495205457279\\
60.5	0.18816	0.263865357709355\\
60.5	0.19182	0.296291658437806\\
60.5	0.19548	0.329774107642628\\
60.5	0.19914	0.364312705323819\\
60.5	0.2028	0.399907451481387\\
60.5	0.20646	0.436558346115322\\
60.5	0.21012	0.474265389225636\\
60.5	0.21378	0.513028580812314\\
60.5	0.21744	0.552847920875365\\
60.5	0.2211	0.593723409414787\\
60.5	0.22476	0.635655046430589\\
60.5	0.22842	0.678642831922755\\
60.5	0.23208	0.722686765891293\\
60.5	0.23574	0.767786848336207\\
60.5	0.2394	0.813943079257488\\
60.5	0.24306	0.861155458655145\\
60.5	0.24672	0.90942398652917\\
60.5	0.25038	0.958748662879571\\
60.5	0.25404	1.00912948770634\\
60.5	0.2577	1.06056646100948\\
60.5	0.26136	1.113059582789\\
60.5	0.26502	1.16660885304488\\
60.5	0.26868	1.22121427177714\\
60.5	0.27234	1.27687583898577\\
60.5	0.276	1.33359355467078\\
60.875	0.093	-0.205575494288422\\
60.875	0.09666	-0.202530545560211\\
60.875	0.10032	-0.198429448355626\\
60.875	0.10398	-0.193272202674671\\
60.875	0.10764	-0.187058808517345\\
60.875	0.1113	-0.179789265883645\\
60.875	0.11496	-0.171463574773574\\
60.875	0.11862	-0.16208173518713\\
60.875	0.12228	-0.151643747124316\\
60.875	0.12594	-0.14014961058513\\
60.875	0.1296	-0.12759932556957\\
60.875	0.13326	-0.113992892077641\\
60.875	0.13692	-0.0993303101093391\\
60.875	0.14058	-0.0836115796646655\\
60.875	0.14424	-0.0668367007436202\\
60.875	0.1479	-0.0490056733462015\\
60.875	0.15156	-0.0301184974724127\\
60.875	0.15522	-0.010175173122252\\
60.875	0.15888	0.0108242997042826\\
60.875	0.16254	0.0328799210071868\\
60.875	0.1662	0.055991690786463\\
60.875	0.16986	0.0801596090421111\\
60.875	0.17352	0.105383675774132\\
60.875	0.17718	0.131663890982522\\
60.875	0.18084	0.159000254667288\\
60.875	0.1845	0.187392766828425\\
60.875	0.18816	0.216841427465932\\
60.875	0.19182	0.247346236579811\\
60.875	0.19548	0.278907194170062\\
60.875	0.19914	0.311524300236686\\
60.875	0.2028	0.345197554779679\\
60.875	0.20646	0.379926957799047\\
60.875	0.21012	0.415712509294787\\
60.875	0.21378	0.452554209266895\\
60.875	0.21744	0.490452057715378\\
60.875	0.2211	0.52940605464023\\
60.875	0.22476	0.569416200041457\\
60.875	0.22842	0.610482493919056\\
60.875	0.23208	0.652604936273023\\
60.875	0.23574	0.695783527103367\\
60.875	0.2394	0.740018266410078\\
60.875	0.24306	0.785309154193164\\
60.875	0.24672	0.831656190452622\\
60.875	0.25038	0.879059375188449\\
60.875	0.25404	0.92751870840065\\
60.875	0.2577	0.977034190089224\\
60.875	0.26136	1.02760582025417\\
60.875	0.26502	1.07923359889548\\
60.875	0.26868	1.13191752601317\\
60.875	0.27234	1.18565760160723\\
60.875	0.276	1.24045382567766\\
61.25	0.093	-0.20186326943382\\
61.25	0.09666	-0.200739812320177\\
61.25	0.10032	-0.198560206730163\\
61.25	0.10398	-0.195324452663778\\
61.25	0.10764	-0.191032550121022\\
61.25	0.1113	-0.185684499101892\\
61.25	0.11496	-0.179280299606392\\
61.25	0.11862	-0.171819951634519\\
61.25	0.12228	-0.163303455186275\\
61.25	0.12594	-0.15373081026166\\
61.25	0.1296	-0.143102016860671\\
61.25	0.13326	-0.131417074983312\\
61.25	0.13692	-0.118675984629579\\
61.25	0.14058	-0.104878745799476\\
61.25	0.14424	-0.0900253584930016\\
61.25	0.1479	-0.0741158227101536\\
61.25	0.15156	-0.0571501384509354\\
61.25	0.15522	-0.0391283057153436\\
61.25	0.15888	-0.0200503245033797\\
61.25	0.16254	8.38051849521015e-05\\
61.25	0.1662	0.0212740833496594\\
61.25	0.16986	0.0435205099907368\\
61.25	0.17352	0.0668230851081875\\
61.25	0.17718	0.0911818087020084\\
61.25	0.18084	0.116596680772203\\
61.25	0.1845	0.143067701318769\\
61.25	0.18816	0.170594870341706\\
61.25	0.19182	0.199178187841014\\
61.25	0.19548	0.228817653816694\\
61.25	0.19914	0.259513268268748\\
61.25	0.2028	0.291265031197174\\
61.25	0.20646	0.324072942601968\\
61.25	0.21012	0.357937002483141\\
61.25	0.21378	0.392857210840678\\
61.25	0.21744	0.428833567674587\\
61.25	0.2211	0.465866072984872\\
61.25	0.22476	0.503954726771528\\
61.25	0.22842	0.543099529034556\\
61.25	0.23208	0.583300479773953\\
61.25	0.23574	0.624557578989726\\
61.25	0.2394	0.666870826681866\\
61.25	0.24306	0.710240222850385\\
61.25	0.24672	0.754665767495269\\
61.25	0.25038	0.800147460616528\\
61.25	0.25404	0.846685302214156\\
61.25	0.2577	0.894279292288163\\
61.25	0.26136	0.942929430838534\\
61.25	0.26502	0.99263571786528\\
61.25	0.26868	1.04339815336839\\
61.25	0.27234	1.09521673734789\\
61.25	0.276	1.14809146980375\\
61.625	0.093	-0.19737367146\\
61.625	0.09666	-0.19817170596093\\
61.625	0.10032	-0.197913591985488\\
61.625	0.10398	-0.196599329533673\\
61.625	0.10764	-0.194228918605487\\
61.625	0.1113	-0.190802359200926\\
61.625	0.11496	-0.186319651319997\\
61.625	0.11862	-0.180780794962694\\
61.625	0.12228	-0.174185790129021\\
61.625	0.12594	-0.166534636818975\\
61.625	0.1296	-0.157827335032556\\
61.625	0.13326	-0.148063884769768\\
61.625	0.13692	-0.137244286030606\\
61.625	0.14058	-0.125368538815074\\
61.625	0.14424	-0.112436643123168\\
61.625	0.1479	-0.0984485989548908\\
61.625	0.15156	-0.0834044063102433\\
61.625	0.15522	-0.0673040651892222\\
61.625	0.15888	-0.0501475755918289\\
61.625	0.16254	-0.031934937518066\\
61.625	0.1662	-0.0126661509679293\\
61.625	0.16986	0.00765878405857734\\
61.625	0.17352	0.0290398675614574\\
61.625	0.17718	0.0514770995407077\\
61.625	0.18084	0.0749704799963316\\
61.625	0.1845	0.0995200089283272\\
61.625	0.18816	0.125125686336692\\
61.625	0.19182	0.151787512221433\\
61.625	0.19548	0.179505486582542\\
61.625	0.19914	0.208279609420025\\
61.625	0.2028	0.23810988073388\\
61.625	0.20646	0.268996300524104\\
61.625	0.21012	0.300938868790706\\
61.625	0.21378	0.333937585533672\\
61.625	0.21744	0.367992450753014\\
61.625	0.2211	0.403103464448728\\
61.625	0.22476	0.439270626620814\\
61.625	0.22842	0.476493937269272\\
61.625	0.23208	0.514773396394097\\
61.625	0.23574	0.5541090039953\\
61.625	0.2394	0.594500760072873\\
61.625	0.24306	0.635948664626818\\
61.625	0.24672	0.678452717657134\\
61.625	0.25038	0.72201291916382\\
61.625	0.25404	0.76662926914688\\
61.625	0.2577	0.812301767606316\\
61.625	0.26136	0.859030414542116\\
61.625	0.26502	0.906815209954292\\
61.625	0.26868	0.955656153842836\\
61.625	0.27234	1.00555324620776\\
61.625	0.276	1.05650648704905\\
62	0.093	-0.192106700366977\\
62	0.09666	-0.194826226482477\\
62	0.10032	-0.196489604121602\\
62	0.10398	-0.197096833284359\\
62	0.10764	-0.196647913970744\\
62	0.1113	-0.195142846180754\\
62	0.11496	-0.192581629914396\\
62	0.11862	-0.188964265171662\\
62	0.12228	-0.184290751952559\\
62	0.12594	-0.178561090257084\\
62	0.1296	-0.171775280085236\\
62	0.13326	-0.163933321437017\\
62	0.13692	-0.155035214312425\\
62	0.14058	-0.145080958711464\\
62	0.14424	-0.134070554634129\\
62	0.1479	-0.122004002080422\\
62	0.15156	-0.108881301050343\\
62	0.15522	-0.0947024515438929\\
62	0.15888	-0.0794674535610703\\
62	0.16254	-0.063176307101878\\
62	0.1662	-0.0458290121663121\\
62	0.16986	-0.0274255687543743\\
62	0.17352	-0.00796597686606493\\
62	0.17718	0.0125497634986147\\
62	0.18084	0.0341216523396679\\
62	0.1845	0.0567496896570929\\
62	0.18816	0.0804338754508884\\
62	0.19182	0.105174209721059\\
62	0.19548	0.130970692467601\\
62	0.19914	0.15782332369051\\
62	0.2028	0.185732103389795\\
62	0.20646	0.214697031565451\\
62	0.21012	0.244718108217478\\
62	0.21378	0.275795333345878\\
62	0.21744	0.307928706950649\\
62	0.2211	0.341118229031789\\
62	0.22476	0.375363899589308\\
62	0.22842	0.410665718623195\\
62	0.23208	0.44702368613345\\
62	0.23574	0.484437802120081\\
62	0.2394	0.522908066583084\\
62	0.24306	0.562434479522458\\
62	0.24672	0.603017040938204\\
62	0.25038	0.644655750830322\\
62	0.25404	0.687350609198812\\
62	0.2577	0.731101616043674\\
62	0.26136	0.775908771364907\\
62	0.26502	0.821772075162508\\
62	0.26868	0.868691527436485\\
62	0.27234	0.916667128186838\\
62	0.276	0.965698877413555\\
62.375	0.093	-0.186062356154745\\
62.375	0.09666	-0.190703373884815\\
62.375	0.10032	-0.194288243138512\\
62.375	0.10398	-0.196816963915838\\
62.375	0.10764	-0.198289536216792\\
62.375	0.1113	-0.198705960041374\\
62.375	0.11496	-0.198066235389585\\
62.375	0.11862	-0.196370362261423\\
62.375	0.12228	-0.19361834065689\\
62.375	0.12594	-0.189810170575985\\
62.375	0.1296	-0.184945852018706\\
62.375	0.13326	-0.179025384985059\\
62.375	0.13692	-0.172048769475037\\
62.375	0.14058	-0.164016005488646\\
62.375	0.14424	-0.154927093025881\\
62.375	0.1479	-0.144782032086744\\
62.375	0.15156	-0.133580822671237\\
62.375	0.15522	-0.121323464779356\\
62.375	0.15888	-0.108009958411104\\
62.375	0.16254	-0.0936403035664823\\
62.375	0.1662	-0.0782145002454853\\
62.375	0.16986	-0.0617325484481199\\
62.375	0.17352	-0.0441944481743795\\
62.375	0.17718	-0.0256001994242705\\
62.375	0.18084	-0.00594980219778618\\
62.375	0.1845	0.0147567435050682\\
62.375	0.18816	0.036519437684293\\
62.375	0.19182	0.0593382803398927\\
62.375	0.19548	0.0832132714718643\\
62.375	0.19914	0.108144411080203\\
62.375	0.2028	0.134131699164916\\
62.375	0.20646	0.161175135726002\\
62.375	0.21012	0.189274720763462\\
62.375	0.21378	0.218430454277291\\
62.375	0.21744	0.248642336267489\\
62.375	0.2211	0.279910366734061\\
62.375	0.22476	0.312234545677009\\
62.375	0.22842	0.345614873096322\\
62.375	0.23208	0.38005134899201\\
62.375	0.23574	0.415543973364071\\
62.375	0.2394	0.452092746212502\\
62.375	0.24306	0.489697667537309\\
62.375	0.24672	0.528358737338481\\
62.375	0.25038	0.568075955616029\\
62.375	0.25404	0.608849322369948\\
62.375	0.2577	0.650678837600243\\
62.375	0.26136	0.693564501306905\\
62.375	0.26502	0.737506313489936\\
62.375	0.26868	0.782504274149342\\
62.375	0.27234	0.828558383285124\\
62.375	0.276	0.875668640897274\\
62.75	0.093	-0.179240638823306\\
62.75	0.09666	-0.185803148167946\\
62.75	0.10032	-0.191309509036215\\
62.75	0.10398	-0.195759721428111\\
62.75	0.10764	-0.199153785343636\\
62.75	0.1113	-0.201491700782787\\
62.75	0.11496	-0.202773467745568\\
62.75	0.11862	-0.202999086231977\\
62.75	0.12228	-0.202168556242014\\
62.75	0.12594	-0.200281877775678\\
62.75	0.1296	-0.197339050832972\\
62.75	0.13326	-0.193340075413894\\
62.75	0.13692	-0.188284951518444\\
62.75	0.14058	-0.182173679146622\\
62.75	0.14424	-0.175006258298428\\
62.75	0.1479	-0.166782688973861\\
62.75	0.15156	-0.157502971172924\\
62.75	0.15522	-0.147167104895615\\
62.75	0.15888	-0.135775090141932\\
62.75	0.16254	-0.123326926911881\\
62.75	0.1662	-0.109822615205454\\
62.75	0.16986	-0.0952621550226596\\
62.75	0.17352	-0.0796455463634897\\
62.75	0.17718	-0.0629727892279497\\
62.75	0.18084	-0.0452438836160378\\
62.75	0.1845	-0.0264588295277524\\
62.75	0.18816	-0.0066176269630982\\
62.75	0.19182	0.0142797240779309\\
62.75	0.19548	0.0362332235953318\\
62.75	0.19914	0.0592428715891029\\
62.75	0.2028	0.0833086680592461\\
62.75	0.20646	0.108430613005761\\
62.75	0.21012	0.134608706428647\\
62.75	0.21378	0.161842948327905\\
62.75	0.21744	0.190133338703536\\
62.75	0.2211	0.219479877555538\\
62.75	0.22476	0.249882564883915\\
62.75	0.22842	0.281341400688657\\
62.75	0.23208	0.313856384969774\\
62.75	0.23574	0.347427517727264\\
62.75	0.2394	0.382054798961125\\
62.75	0.24306	0.417738228671362\\
62.75	0.24672	0.454477806857966\\
62.75	0.25038	0.492273533520943\\
62.75	0.25404	0.531125408660292\\
62.75	0.2577	0.571033432276016\\
62.75	0.26136	0.611997604368104\\
62.75	0.26502	0.654017924936568\\
62.75	0.26868	0.697094393981403\\
62.75	0.27234	0.741227011502615\\
62.75	0.276	0.786415777500194\\
63.125	0.093	-0.171641548372656\\
63.125	0.09666	-0.180125549331865\\
63.125	0.10032	-0.187553401814702\\
63.125	0.10398	-0.193925105821169\\
63.125	0.10764	-0.199240661351266\\
63.125	0.1113	-0.203500068404986\\
63.125	0.11496	-0.20670332698234\\
63.125	0.11862	-0.208850437083318\\
63.125	0.12228	-0.209941398707926\\
63.125	0.12594	-0.20997621185616\\
63.125	0.1296	-0.208954876528023\\
63.125	0.13326	-0.206877392723515\\
63.125	0.13692	-0.203743760442636\\
63.125	0.14058	-0.199553979685385\\
63.125	0.14424	-0.19430805045176\\
63.125	0.1479	-0.188005972741765\\
63.125	0.15156	-0.180647746555399\\
63.125	0.15522	-0.172233371892658\\
63.125	0.15888	-0.162762848753546\\
63.125	0.16254	-0.152236177138066\\
63.125	0.1662	-0.14065335704621\\
63.125	0.16986	-0.128014388477984\\
63.125	0.17352	-0.114319271433385\\
63.125	0.17718	-0.0995680059124158\\
63.125	0.18084	-0.0837605919150728\\
63.125	0.1845	-0.066897029441358\\
63.125	0.18816	-0.0489773184912745\\
63.125	0.19182	-0.0300014590648161\\
63.125	0.19548	-0.00996945116198589\\
63.125	0.19914	0.0111187052172146\\
63.125	0.2028	0.0332630100727871\\
63.125	0.20646	0.0564634634047312\\
63.125	0.21012	0.0807200652130504\\
63.125	0.21378	0.106032815497738\\
63.125	0.21744	0.132401714258797\\
63.125	0.2211	0.159826761496229\\
63.125	0.22476	0.188307957210032\\
63.125	0.22842	0.217845301400207\\
63.125	0.23208	0.248438794066753\\
63.125	0.23574	0.280088435209673\\
63.125	0.2394	0.312794224828963\\
63.125	0.24306	0.346556162924629\\
63.125	0.24672	0.381374249496663\\
63.125	0.25038	0.417248484545069\\
63.125	0.25404	0.454178868069847\\
63.125	0.2577	0.492165400071\\
63.125	0.26136	0.531208080548522\\
63.125	0.26502	0.571306909502415\\
63.125	0.26868	0.612461886932679\\
63.125	0.27234	0.65467301283932\\
63.125	0.276	0.697940287222329\\
63.5	0.093	-0.163265084802796\\
63.5	0.09666	-0.173670577376576\\
63.5	0.10032	-0.183019921473983\\
63.5	0.10398	-0.191313117095019\\
63.5	0.10764	-0.198550164239685\\
63.5	0.1113	-0.204731062907978\\
63.5	0.11496	-0.2098558130999\\
63.5	0.11862	-0.213924414815449\\
63.5	0.12228	-0.216936868054628\\
63.5	0.12594	-0.218893172817433\\
63.5	0.1296	-0.219793329103866\\
63.5	0.13326	-0.219637336913929\\
63.5	0.13692	-0.218425196247619\\
63.5	0.14058	-0.216156907104938\\
63.5	0.14424	-0.212832469485884\\
63.5	0.1479	-0.208451883390458\\
63.5	0.15156	-0.203015148818662\\
63.5	0.15522	-0.196522265770493\\
63.5	0.15888	-0.188973234245951\\
63.5	0.16254	-0.18036805424504\\
63.5	0.1662	-0.170706725767755\\
63.5	0.16986	-0.159989248814099\\
63.5	0.17352	-0.148215623384071\\
63.5	0.17718	-0.135385849477672\\
63.5	0.18084	-0.121499927094902\\
63.5	0.1845	-0.106557856235758\\
63.5	0.18816	-0.0905596369002413\\
63.5	0.19182	-0.0735052690883535\\
63.5	0.19548	-0.055394752800094\\
63.5	0.19914	-0.0362280880354642\\
63.5	0.2028	-0.0160052747944623\\
63.5	0.20646	0.00527368692291108\\
63.5	0.21012	0.0276087971166596\\
63.5	0.21378	0.0510000557867765\\
63.5	0.21744	0.0754474629332653\\
63.5	0.2211	0.100951018556126\\
63.5	0.22476	0.127510722655362\\
63.5	0.22842	0.155126575230966\\
63.5	0.23208	0.183798576282942\\
63.5	0.23574	0.213526725811295\\
63.5	0.2394	0.244311023816014\\
63.5	0.24306	0.276151470297109\\
63.5	0.24672	0.309048065254573\\
63.5	0.25038	0.343000808688408\\
63.5	0.25404	0.378009700598615\\
63.5	0.2577	0.414074740985198\\
63.5	0.26136	0.451195929848148\\
63.5	0.26502	0.489373267187471\\
63.5	0.26868	0.528606753003165\\
63.5	0.27234	0.568896387295235\\
63.5	0.276	0.610242170063673\\
63.875	0.093	-0.154111248113726\\
63.875	0.09666	-0.166438232302078\\
63.875	0.10032	-0.177709068014059\\
63.875	0.10398	-0.187923755249664\\
63.875	0.10764	-0.1970822940089\\
63.875	0.1113	-0.205184684291763\\
63.875	0.11496	-0.212230926098255\\
63.875	0.11862	-0.218221019428374\\
63.875	0.12228	-0.223154964282123\\
63.875	0.12594	-0.227032760659499\\
63.875	0.1296	-0.229854408560501\\
63.875	0.13326	-0.231619907985135\\
63.875	0.13692	-0.232329258933396\\
63.875	0.14058	-0.231982461405285\\
63.875	0.14424	-0.230579515400802\\
63.875	0.1479	-0.228120420919947\\
63.875	0.15156	-0.22460517796272\\
63.875	0.15522	-0.220033786529121\\
63.875	0.15888	-0.21440624661915\\
63.875	0.16254	-0.207722558232809\\
63.875	0.1662	-0.199982721370095\\
63.875	0.16986	-0.19118673603101\\
63.875	0.17352	-0.181334602215551\\
63.875	0.17718	-0.170426319923723\\
63.875	0.18084	-0.158461889155521\\
63.875	0.1845	-0.145441309910948\\
63.875	0.18816	-0.131364582190002\\
63.875	0.19182	-0.116231705992685\\
63.875	0.19548	-0.100042681318996\\
63.875	0.19914	-0.0827975081689369\\
63.875	0.2028	-0.0644961865425058\\
63.875	0.20646	-0.045138716439703\\
63.875	0.21012	-0.0247250978605216\\
63.875	0.21378	-0.00325533080497542\\
63.875	0.21744	0.0192705847269428\\
63.875	0.2211	0.0428526487352328\\
63.875	0.22476	0.0674908612198981\\
63.875	0.22842	0.0931852221809315\\
63.875	0.23208	0.119935731618337\\
63.875	0.23574	0.147742389532119\\
63.875	0.2394	0.176605195922268\\
63.875	0.24306	0.206524150788792\\
63.875	0.24672	0.237499254131685\\
63.875	0.25038	0.26953050595095\\
63.875	0.25404	0.302617906246589\\
63.875	0.2577	0.336761455018602\\
63.875	0.26136	0.371961152266981\\
63.875	0.26502	0.408216997991733\\
63.875	0.26868	0.445528992192857\\
63.875	0.27234	0.483897134870356\\
63.875	0.276	0.523321426024223\\
64.25	0.093	-0.144180038305451\\
64.25	0.09666	-0.158428514108372\\
64.25	0.10032	-0.171620841434921\\
64.25	0.10398	-0.183757020285098\\
64.25	0.10764	-0.194837050658906\\
64.25	0.1113	-0.204860932556338\\
64.25	0.11496	-0.213828665977401\\
64.25	0.11862	-0.22174025092209\\
64.25	0.12228	-0.22859568739041\\
64.25	0.12594	-0.234394975382354\\
64.25	0.1296	-0.239138114897929\\
64.25	0.13326	-0.242825105937132\\
64.25	0.13692	-0.245455948499963\\
64.25	0.14058	-0.247030642586423\\
64.25	0.14424	-0.247549188196511\\
64.25	0.1479	-0.247011585330224\\
64.25	0.15156	-0.24541783398757\\
64.25	0.15522	-0.24276793416854\\
64.25	0.15888	-0.239061885873139\\
64.25	0.16254	-0.234299689101369\\
64.25	0.1662	-0.228481343853224\\
64.25	0.16986	-0.22160685012871\\
64.25	0.17352	-0.213676207927821\\
64.25	0.17718	-0.204689417250564\\
64.25	0.18084	-0.194646478096931\\
64.25	0.1845	-0.183547390466928\\
64.25	0.18816	-0.171392154360557\\
64.25	0.19182	-0.158180769777807\\
64.25	0.19548	-0.143913236718689\\
64.25	0.19914	-0.1285895551832\\
64.25	0.2028	-0.112209725171336\\
64.25	0.20646	-0.094773746683104\\
64.25	0.21012	-0.0762816197184968\\
64.25	0.21378	-0.0567333442775213\\
64.25	0.21744	-0.0361289203601738\\
64.25	0.2211	-0.0144683479664509\\
64.25	0.22476	0.00824837290364377\\
64.25	0.22842	0.0320212422501065\\
64.25	0.23208	0.0568502600729412\\
64.25	0.23574	0.0827354263721523\\
64.25	0.2394	0.109676741147731\\
64.25	0.24306	0.137674204399684\\
64.25	0.24672	0.166727816128006\\
64.25	0.25038	0.196837576332704\\
64.25	0.25404	0.22800348501377\\
64.25	0.2577	0.260225542171211\\
64.25	0.26136	0.293503747805024\\
64.25	0.26502	0.327838101915205\\
64.25	0.26868	0.363228604501758\\
64.25	0.27234	0.399675255564687\\
64.25	0.276	0.437178055103987\\
64.625	0.093	-0.133471455377967\\
64.625	0.09666	-0.149641422795459\\
64.625	0.10032	-0.164755241736579\\
64.625	0.10398	-0.178812912201327\\
64.625	0.10764	-0.191814434189703\\
64.625	0.1113	-0.203759807701706\\
64.625	0.11496	-0.21464903273734\\
64.625	0.11862	-0.224482109296599\\
64.625	0.12228	-0.23325903737949\\
64.625	0.12594	-0.240979816986005\\
64.625	0.1296	-0.247644448116149\\
64.625	0.13326	-0.253252930769924\\
64.625	0.13692	-0.257805264947324\\
64.625	0.14058	-0.261301450648355\\
64.625	0.14424	-0.263741487873011\\
64.625	0.1479	-0.265125376621297\\
64.625	0.15156	-0.265453116893212\\
64.625	0.15522	-0.264724708688754\\
64.625	0.15888	-0.262940152007923\\
64.625	0.16254	-0.260099446850723\\
64.625	0.1662	-0.256202593217149\\
64.625	0.16986	-0.251249591107205\\
64.625	0.17352	-0.245240440520887\\
64.625	0.17718	-0.238175141458201\\
64.625	0.18084	-0.230053693919137\\
64.625	0.1845	-0.220876097903705\\
64.625	0.18816	-0.210642353411904\\
64.625	0.19182	-0.199352460443724\\
64.625	0.19548	-0.187006418999177\\
64.625	0.19914	-0.173604229078259\\
64.625	0.2028	-0.159145890680966\\
64.625	0.20646	-0.143631403807304\\
64.625	0.21012	-0.127060768457264\\
64.625	0.21378	-0.109433984630859\\
64.625	0.21744	-0.0907510523280826\\
64.625	0.2211	-0.0710119715489339\\
64.625	0.22476	-0.0502167422934063\\
64.625	0.22842	-0.0283653645615143\\
64.625	0.23208	-0.0054578383532502\\
64.625	0.23574	0.0185058363313901\\
64.625	0.2394	0.0435256594923978\\
64.625	0.24306	0.0696016311297845\\
64.625	0.24672	0.0967337512435358\\
64.625	0.25038	0.124922019833659\\
64.625	0.25404	0.154166436900158\\
64.625	0.2577	0.184467002443029\\
64.625	0.26136	0.215823716462267\\
64.625	0.26502	0.248236578957881\\
64.625	0.26868	0.281705589929863\\
64.625	0.27234	0.316230749378221\\
64.625	0.276	0.351812057302951\\
65	0.093	-0.121985499331276\\
65	0.09666	-0.140076958363339\\
65	0.10032	-0.157112268919029\\
65	0.10398	-0.173091430998346\\
65	0.10764	-0.188014444601293\\
65	0.1113	-0.201881309727866\\
65	0.11496	-0.21469202637807\\
65	0.11862	-0.226446594551899\\
65	0.12228	-0.237145014249361\\
65	0.12594	-0.246787285470446\\
65	0.1296	-0.255373408215161\\
65	0.13326	-0.262903382483505\\
65	0.13692	-0.269377208275475\\
65	0.14058	-0.274794885591077\\
65	0.14424	-0.279156414430304\\
65	0.1479	-0.282461794793159\\
65	0.15156	-0.284711026679646\\
65	0.15522	-0.285904110089757\\
65	0.15888	-0.286041045023497\\
65	0.16254	-0.285121831480868\\
65	0.1662	-0.283146469461864\\
65	0.16986	-0.28011495896649\\
65	0.17352	-0.276027299994742\\
65	0.17718	-0.270883492546624\\
65	0.18084	-0.264683536622135\\
65	0.1845	-0.25742743222127\\
65	0.18816	-0.249115179344039\\
65	0.19182	-0.239746777990431\\
65	0.19548	-0.229322228160454\\
65	0.19914	-0.217841529854103\\
65	0.2028	-0.205304683071384\\
65	0.20646	-0.19171168781229\\
65	0.21012	-0.177062544076824\\
65	0.21378	-0.16135725186499\\
65	0.21744	-0.14459581117678\\
65	0.2211	-0.126778222012202\\
65	0.22476	-0.107904484371245\\
65	0.22842	-0.0879745982539237\\
65	0.23208	-0.0669885636602303\\
65	0.23574	-0.0449463805901607\\
65	0.2394	-0.0218480490437201\\
65	0.24306	0.00230643097909233\\
65	0.24672	0.027517059478273\\
65	0.25038	0.0537838364538294\\
65	0.25404	0.0811067619057537\\
65	0.2577	0.109485835834057\\
65	0.26136	0.138921058238725\\
65	0.26502	0.169412429119769\\
65	0.26868	0.20095994847718\\
65	0.27234	0.233563616310967\\
65	0.276	0.267223432621126\\
65.375	0.093	-0.109722170165376\\
65.375	0.09666	-0.129735120812007\\
65.375	0.10032	-0.148691922982266\\
65.375	0.10398	-0.166592576676156\\
65.375	0.10764	-0.183437081893673\\
65.375	0.1113	-0.199225438634815\\
65.375	0.11496	-0.213957646899589\\
65.375	0.11862	-0.22763370668799\\
65.375	0.12228	-0.24025361800002\\
65.375	0.12594	-0.251817380835677\\
65.375	0.1296	-0.262324995194961\\
65.375	0.13326	-0.271776461077876\\
65.375	0.13692	-0.280171778484417\\
65.375	0.14058	-0.28751094741459\\
65.375	0.14424	-0.293793967868388\\
65.375	0.1479	-0.299020839845813\\
65.375	0.15156	-0.303191563346869\\
65.375	0.15522	-0.306306138371551\\
65.375	0.15888	-0.308364564919861\\
65.375	0.16254	-0.309366842991801\\
65.375	0.1662	-0.309312972587368\\
65.375	0.16986	-0.308202953706564\\
65.375	0.17352	-0.306036786349387\\
65.375	0.17718	-0.302814470515842\\
65.375	0.18084	-0.298536006205919\\
65.375	0.1845	-0.293201393419628\\
65.375	0.18816	-0.286810632156965\\
65.375	0.19182	-0.279363722417928\\
65.375	0.19548	-0.270860664202521\\
65.375	0.19914	-0.261301457510741\\
65.375	0.2028	-0.250686102342593\\
65.375	0.20646	-0.239014598698069\\
65.375	0.21012	-0.226286946577174\\
65.375	0.21378	-0.212503145979907\\
65.375	0.21744	-0.197663196906271\\
65.375	0.2211	-0.181767099356261\\
65.375	0.22476	-0.164814853329878\\
65.375	0.22842	-0.146806458827124\\
65.375	0.23208	-0.127741915848001\\
65.375	0.23574	-0.107621224392502\\
65.375	0.2394	-0.086444384460632\\
65.375	0.24306	-0.0642113960523902\\
65.375	0.24672	-0.0409222591677767\\
65.375	0.25038	-0.0165769738067909\\
65.375	0.25404	0.0088244600305627\\
65.375	0.2577	0.0352820423442957\\
65.375	0.26136	0.0627957731343929\\
65.375	0.26502	0.0913656524008655\\
65.375	0.26868	0.120991680143706\\
65.375	0.27234	0.151673856362927\\
65.375	0.276	0.183412181058511\\
65.75	0.093	-0.0966814678802672\\
65.75	0.09666	-0.118615910141467\\
65.75	0.10032	-0.139494203926299\\
65.75	0.10398	-0.159316349234758\\
65.75	0.10764	-0.178082346066846\\
65.75	0.1113	-0.195792194422559\\
65.75	0.11496	-0.212445894301903\\
65.75	0.11862	-0.228043445704874\\
65.75	0.12228	-0.242584848631473\\
65.75	0.12594	-0.256070103081701\\
65.75	0.1296	-0.268499209055556\\
65.75	0.13326	-0.279872166553042\\
65.75	0.13692	-0.290188975574153\\
65.75	0.14058	-0.299449636118895\\
65.75	0.14424	-0.307654148187263\\
65.75	0.1479	-0.314802511779258\\
65.75	0.15156	-0.320894726894884\\
65.75	0.15522	-0.325930793534137\\
65.75	0.15888	-0.329910711697017\\
65.75	0.16254	-0.332834481383528\\
65.75	0.1662	-0.334702102593665\\
65.75	0.16986	-0.335513575327433\\
65.75	0.17352	-0.335268899584825\\
65.75	0.17718	-0.333968075365848\\
65.75	0.18084	-0.3316111026705\\
65.75	0.1845	-0.328197981498776\\
65.75	0.18816	-0.323728711850684\\
65.75	0.19182	-0.318203293726216\\
65.75	0.19548	-0.311621727125381\\
65.75	0.19914	-0.303984012048172\\
65.75	0.2028	-0.29529014849459\\
65.75	0.20646	-0.285540136464641\\
65.75	0.21012	-0.274733975958313\\
65.75	0.21378	-0.26287166697562\\
65.75	0.21744	-0.249953209516552\\
65.75	0.2211	-0.235978603581111\\
65.75	0.22476	-0.220947849169299\\
65.75	0.22842	-0.204860946281116\\
65.75	0.23208	-0.187717894916564\\
65.75	0.23574	-0.169518695075635\\
65.75	0.2394	-0.150263346758336\\
65.75	0.24306	-0.129951849964661\\
65.75	0.24672	-0.108584204694622\\
65.75	0.25038	-0.086160410948207\\
65.75	0.25404	-0.0626804687254205\\
65.75	0.2577	-0.0381443780262618\\
65.75	0.26136	-0.0125521388507317\\
65.75	0.26502	0.0140962488011667\\
65.75	0.26868	0.0418007849294404\\
65.75	0.27234	0.0705614695340899\\
65.75	0.276	0.100378302615104\\
66.125	0.093	-0.0828633924759457\\
66.125	0.09666	-0.106719326351716\\
66.125	0.10032	-0.129519111751119\\
66.125	0.10398	-0.151262748674146\\
66.125	0.10764	-0.171950237120805\\
66.125	0.1113	-0.191581577091089\\
66.125	0.11496	-0.210156768585004\\
66.125	0.11862	-0.227675811602545\\
66.125	0.12228	-0.244138706143715\\
66.125	0.12594	-0.259545452208513\\
66.125	0.1296	-0.273896049796939\\
66.125	0.13326	-0.287190498908994\\
66.125	0.13692	-0.299428799544676\\
66.125	0.14058	-0.310610951703989\\
66.125	0.14424	-0.320736955386927\\
66.125	0.1479	-0.329806810593493\\
66.125	0.15156	-0.33782051732369\\
66.125	0.15522	-0.344778075577511\\
66.125	0.15888	-0.350679485354963\\
66.125	0.16254	-0.355524746656044\\
66.125	0.1662	-0.359313859480752\\
66.125	0.16986	-0.362046823829088\\
66.125	0.17352	-0.363723639701052\\
66.125	0.17718	-0.364344307096645\\
66.125	0.18084	-0.363908826015867\\
66.125	0.1845	-0.362417196458714\\
66.125	0.18816	-0.359869418425192\\
66.125	0.19182	-0.356265491915296\\
66.125	0.19548	-0.351605416929031\\
66.125	0.19914	-0.345889193466392\\
66.125	0.2028	-0.339116821527382\\
66.125	0.20646	-0.331288301111999\\
66.125	0.21012	-0.322403632220245\\
66.125	0.21378	-0.31246281485212\\
66.125	0.21744	-0.301465849007622\\
66.125	0.2211	-0.289412734686756\\
66.125	0.22476	-0.276303471889511\\
66.125	0.22842	-0.262138060615898\\
66.125	0.23208	-0.246916500865913\\
66.125	0.23574	-0.230638792639559\\
66.125	0.2394	-0.213304935936831\\
66.125	0.24306	-0.194914930757727\\
66.125	0.24672	-0.175468777102255\\
66.125	0.25038	-0.15496647497041\\
66.125	0.25404	-0.133408024362198\\
66.125	0.2577	-0.110793425277606\\
66.125	0.26136	-0.0871226777166467\\
66.125	0.26502	-0.062395781679319\\
66.125	0.26868	-0.036612737165616\\
66.125	0.27234	-0.00977354417553711\\
66.125	0.276	0.018121797290906\\
66.5	0.093	-0.0682679439524163\\
66.5	0.09666	-0.0940453694427614\\
66.5	0.10032	-0.118766646456731\\
66.5	0.10398	-0.142431774994331\\
66.5	0.10764	-0.16504075505556\\
66.5	0.1113	-0.186593586640414\\
66.5	0.11496	-0.207090269748898\\
66.5	0.11862	-0.226530804381011\\
66.5	0.12228	-0.244915190536751\\
66.5	0.12594	-0.26224342821612\\
66.5	0.1296	-0.278515517419115\\
66.5	0.13326	-0.293731458145742\\
66.5	0.13692	-0.307891250395993\\
66.5	0.14058	-0.320994894169876\\
66.5	0.14424	-0.333042389467384\\
66.5	0.1479	-0.344033736288522\\
66.5	0.15156	-0.353968934633287\\
66.5	0.15522	-0.362847984501682\\
66.5	0.15888	-0.370670885893702\\
66.5	0.16254	-0.377437638809354\\
66.5	0.1662	-0.383148243248631\\
66.5	0.16986	-0.387802699211539\\
66.5	0.17352	-0.391401006698074\\
66.5	0.17718	-0.393943165708237\\
66.5	0.18084	-0.395429176242027\\
66.5	0.1845	-0.395859038299444\\
66.5	0.18816	-0.395232751880493\\
66.5	0.19182	-0.393550316985168\\
66.5	0.19548	-0.390811733613473\\
66.5	0.19914	-0.387017001765405\\
66.5	0.2028	-0.382166121440965\\
66.5	0.20646	-0.376259092640154\\
66.5	0.21012	-0.369295915362967\\
66.5	0.21378	-0.361276589609412\\
66.5	0.21744	-0.352201115379488\\
66.5	0.2211	-0.34206949267319\\
66.5	0.22476	-0.330881721490515\\
66.5	0.22842	-0.318637801831473\\
66.5	0.23208	-0.305337733696059\\
66.5	0.23574	-0.290981517084272\\
66.5	0.2394	-0.275569151996117\\
66.5	0.24306	-0.259100638431584\\
66.5	0.24672	-0.241575976390683\\
66.5	0.25038	-0.222995165873409\\
66.5	0.25404	-0.203358206879764\\
66.5	0.2577	-0.182665099409743\\
66.5	0.26136	-0.160915843463358\\
66.5	0.26502	-0.138110439040597\\
66.5	0.26868	-0.114248886141465\\
66.5	0.27234	-0.0893311847659564\\
66.5	0.276	-0.0633573349140804\\
66.875	0.093	-0.052895122309681\\
66.875	0.09666	-0.0805940394145949\\
66.875	0.10032	-0.107236808043137\\
66.875	0.10398	-0.132823428195306\\
66.875	0.10764	-0.157353899871106\\
66.875	0.1113	-0.180828223070529\\
66.875	0.11496	-0.203246397793585\\
66.875	0.11862	-0.224608424040267\\
66.875	0.12228	-0.244914301810578\\
66.875	0.12594	-0.264164031104515\\
66.875	0.1296	-0.282357611922083\\
66.875	0.13326	-0.299495044263279\\
66.875	0.13692	-0.315576328128101\\
66.875	0.14058	-0.330601463516554\\
66.875	0.14424	-0.344570450428633\\
66.875	0.1479	-0.357483288864339\\
66.875	0.15156	-0.369339978823676\\
66.875	0.15522	-0.38014052030664\\
66.875	0.15888	-0.389884913313231\\
66.875	0.16254	-0.398573157843454\\
66.875	0.1662	-0.406205253897302\\
66.875	0.16986	-0.412781201474779\\
66.875	0.17352	-0.418301000575883\\
66.875	0.17718	-0.422764651200617\\
66.875	0.18084	-0.426172153348977\\
66.875	0.1845	-0.428523507020965\\
66.875	0.18816	-0.429818712216585\\
66.875	0.19182	-0.43005776893583\\
66.875	0.19548	-0.429240677178706\\
66.875	0.19914	-0.427367436945209\\
66.875	0.2028	-0.424438048235339\\
66.875	0.20646	-0.420452511049098\\
66.875	0.21012	-0.415410825386482\\
66.875	0.21378	-0.409312991247498\\
66.875	0.21744	-0.402159008632142\\
66.875	0.2211	-0.393948877540413\\
66.875	0.22476	-0.38468259797231\\
66.875	0.22842	-0.374360169927838\\
66.875	0.23208	-0.362981593406995\\
66.875	0.23574	-0.350546868409778\\
66.875	0.2394	-0.337055994936191\\
66.875	0.24306	-0.322508972986232\\
66.875	0.24672	-0.306905802559901\\
66.875	0.25038	-0.290246483657195\\
66.875	0.25404	-0.272531016278124\\
66.875	0.2577	-0.253759400422673\\
66.875	0.26136	-0.233931636090855\\
66.875	0.26502	-0.213047723282665\\
66.875	0.26868	-0.191107661998104\\
66.875	0.27234	-0.168111452237166\\
66.875	0.276	-0.144059093999861\\
67.25	0.093	-0.0367449275477378\\
67.25	0.09666	-0.0663653362672207\\
67.25	0.10032	-0.0949295965103316\\
67.25	0.10398	-0.122437708277073\\
67.25	0.10764	-0.148889671567442\\
67.25	0.1113	-0.174285486381437\\
67.25	0.11496	-0.198625152719063\\
67.25	0.11862	-0.221908670580315\\
67.25	0.12228	-0.244136039965196\\
67.25	0.12594	-0.265307260873705\\
67.25	0.1296	-0.285422333305841\\
67.25	0.13326	-0.304481257261607\\
67.25	0.13692	-0.322484032741\\
67.25	0.14058	-0.339430659744024\\
67.25	0.14424	-0.355321138270673\\
67.25	0.1479	-0.370155468320951\\
67.25	0.15156	-0.383933649894858\\
67.25	0.15522	-0.396655682992392\\
67.25	0.15888	-0.408321567613553\\
67.25	0.16254	-0.418931303758345\\
67.25	0.1662	-0.428484891426763\\
67.25	0.16986	-0.436982330618811\\
67.25	0.17352	-0.444423621334487\\
67.25	0.17718	-0.450808763573792\\
67.25	0.18084	-0.456137757336723\\
67.25	0.1845	-0.460410602623282\\
67.25	0.18816	-0.463627299433472\\
67.25	0.19182	-0.465787847767287\\
67.25	0.19548	-0.466892247624731\\
67.25	0.19914	-0.466940499005804\\
67.25	0.2028	-0.465932601910502\\
67.25	0.20646	-0.463868556338832\\
67.25	0.21012	-0.460748362290786\\
67.25	0.21378	-0.456572019766373\\
67.25	0.21744	-0.451339528765587\\
67.25	0.2211	-0.44505088928843\\
67.25	0.22476	-0.437706101334897\\
67.25	0.22842	-0.429305164904996\\
67.25	0.23208	-0.419848079998723\\
67.25	0.23574	-0.409334846616077\\
67.25	0.2394	-0.397765464757061\\
67.25	0.24306	-0.385139934421669\\
67.25	0.24672	-0.371458255609908\\
67.25	0.25038	-0.356720428321776\\
67.25	0.25404	-0.340926452557272\\
67.25	0.2577	-0.324076328316393\\
67.25	0.26136	-0.306170055599145\\
67.25	0.26502	-0.287207634405526\\
67.25	0.26868	-0.267189064735535\\
67.25	0.27234	-0.246114346589168\\
67.25	0.276	-0.223983479966433\\
67.625	0.093	-0.0198173596665869\\
67.625	0.09666	-0.0513592600006404\\
67.625	0.10032	-0.081845011858322\\
67.625	0.10398	-0.111274615239632\\
67.625	0.10764	-0.139648070144572\\
67.625	0.1113	-0.166965376573138\\
67.625	0.11496	-0.193226534525332\\
67.625	0.11862	-0.218431544001155\\
67.625	0.12228	-0.242580405000607\\
67.625	0.12594	-0.265673117523686\\
67.625	0.1296	-0.287709681570393\\
67.625	0.13326	-0.308690097140729\\
67.625	0.13692	-0.328614364234692\\
67.625	0.14058	-0.347482482852287\\
67.625	0.14424	-0.365294452993505\\
67.625	0.1479	-0.382050274658353\\
67.625	0.15156	-0.397749947846831\\
67.625	0.15522	-0.412393472558935\\
67.625	0.15888	-0.425980848794667\\
67.625	0.16254	-0.43851207655403\\
67.625	0.1662	-0.449987155837018\\
67.625	0.16986	-0.460406086643635\\
67.625	0.17352	-0.469768868973881\\
67.625	0.17718	-0.478075502827756\\
67.625	0.18084	-0.485325988205257\\
67.625	0.1845	-0.491520325106387\\
67.625	0.18816	-0.496658513531148\\
67.625	0.19182	-0.500740553479534\\
67.625	0.19548	-0.503766444951548\\
67.625	0.19914	-0.505736187947189\\
67.625	0.2028	-0.506649782466461\\
67.625	0.20646	-0.506507228509361\\
67.625	0.21012	-0.505308526075886\\
67.625	0.21378	-0.503053675166043\\
67.625	0.21744	-0.499742675779828\\
67.625	0.2211	-0.495375527917241\\
67.625	0.22476	-0.489952231578279\\
67.625	0.22842	-0.483472786762945\\
67.625	0.23208	-0.475937193471243\\
67.625	0.23574	-0.467345451703168\\
67.625	0.2394	-0.457697561458722\\
67.625	0.24306	-0.446993522737901\\
67.625	0.24672	-0.435233335540711\\
67.625	0.25038	-0.422416999867146\\
67.625	0.25404	-0.408544515717213\\
67.625	0.2577	-0.393615883090904\\
67.625	0.26136	-0.377631101988227\\
67.625	0.26502	-0.360590172409179\\
67.625	0.26868	-0.342493094353758\\
67.625	0.27234	-0.323339867821962\\
67.625	0.276	-0.303130492813798\\
68	0.093	-0.00211241866622647\\
68	0.09666	-0.0355758106148507\\
68	0.10032	-0.0679830540871029\\
68	0.10398	-0.0993341490829835\\
68	0.10764	-0.129629095602494\\
68	0.1113	-0.158867893645629\\
68	0.11496	-0.187050543212396\\
68	0.11862	-0.214177044302787\\
68	0.12228	-0.24024739691681\\
68	0.12594	-0.26526160105446\\
68	0.1296	-0.289219656715736\\
68	0.13326	-0.312121563900644\\
68	0.13692	-0.333967322609178\\
68	0.14058	-0.354756932841342\\
68	0.14424	-0.37449039459713\\
68	0.1479	-0.393167707876549\\
68	0.15156	-0.410788872679597\\
68	0.15522	-0.427353889006271\\
68	0.15888	-0.442862756856573\\
68	0.16254	-0.457315476230507\\
68	0.1662	-0.470712047128066\\
68	0.16986	-0.483052469549255\\
68	0.17352	-0.494336743494071\\
68	0.17718	-0.504564868962514\\
68	0.18084	-0.513736845954586\\
68	0.1845	-0.521852674470286\\
68	0.18816	-0.528912354509618\\
68	0.19182	-0.534915886072571\\
68	0.19548	-0.539863269159156\\
68	0.19914	-0.543754503769371\\
68	0.2028	-0.546589589903213\\
68	0.20646	-0.548368527560684\\
68	0.21012	-0.549091316741777\\
68	0.21378	-0.548757957446504\\
68	0.21744	-0.54736844967486\\
68	0.2211	-0.544922793426844\\
68	0.22476	-0.541420988702452\\
68	0.22842	-0.536863035501689\\
68	0.23208	-0.531248933824557\\
68	0.23574	-0.524578683671053\\
68	0.2394	-0.516852285041178\\
68	0.24306	-0.508069737934927\\
68	0.24672	-0.498231042352305\\
68	0.25038	-0.487336198293314\\
68	0.25404	-0.475385205757951\\
68	0.2577	-0.462378064746213\\
68	0.26136	-0.448314775258103\\
68	0.26502	-0.433195337293625\\
68	0.26868	-0.417019750852776\\
68	0.27234	-0.39978801593555\\
68	0.276	-0.381500132541957\\
68.375	0.093	0.0163698954533436\\
68.375	0.09666	-0.0190149881098495\\
68.375	0.10032	-0.0533437231966706\\
68.375	0.10398	-0.0866163098071237\\
68.375	0.10764	-0.118832747941203\\
68.375	0.1113	-0.149993037598909\\
68.375	0.11496	-0.180097178780246\\
68.375	0.11862	-0.209145171485208\\
68.375	0.12228	-0.237137015713802\\
68.375	0.12594	-0.264072711466021\\
68.375	0.1296	-0.289952258741869\\
68.375	0.13326	-0.314775657541346\\
68.375	0.13692	-0.338542907864451\\
68.375	0.14058	-0.361254009711185\\
68.375	0.14424	-0.382908963081544\\
68.375	0.1479	-0.403507767975534\\
68.375	0.15156	-0.423050424393151\\
68.375	0.15522	-0.441536932334395\\
68.375	0.15888	-0.458967291799268\\
68.375	0.16254	-0.475341502787772\\
68.375	0.1662	-0.4906595652999\\
68.375	0.16986	-0.504921479335659\\
68.375	0.17352	-0.518127244895045\\
68.375	0.17718	-0.530276861978062\\
68.375	0.18084	-0.541370330584705\\
68.375	0.1845	-0.551407650714972\\
68.375	0.18816	-0.560388822368874\\
68.375	0.19182	-0.568313845546402\\
68.375	0.19548	-0.575182720247554\\
68.375	0.19914	-0.580995446472339\\
68.375	0.2028	-0.585752024220749\\
68.375	0.20646	-0.589452453492791\\
68.375	0.21012	-0.592096734288457\\
68.375	0.21378	-0.593684866607756\\
68.375	0.21744	-0.594216850450678\\
68.375	0.2211	-0.593692685817233\\
68.375	0.22476	-0.592112372707412\\
68.375	0.22842	-0.589475911121223\\
68.375	0.23208	-0.585783301058659\\
68.375	0.23574	-0.581034542519725\\
68.375	0.2394	-0.57522963550442\\
68.375	0.24306	-0.568368580012737\\
68.375	0.24672	-0.560451376044689\\
68.375	0.25038	-0.551478023600265\\
68.375	0.25404	-0.541448522679473\\
68.375	0.2577	-0.530362873282305\\
68.375	0.26136	-0.51822107540877\\
68.375	0.26502	-0.505023129058859\\
68.375	0.26868	-0.49076903423258\\
68.375	0.27234	-0.475458790929925\\
68.375	0.276	-0.459092399150903\\
68.75	0.093	0.0356295826921285\\
68.75	0.09666	-0.00167679248563707\\
68.75	0.10032	-0.0379270191870306\\
68.75	0.10398	-0.0731210974120526\\
68.75	0.10764	-0.107259027160703\\
68.75	0.1113	-0.140340808432979\\
68.75	0.11496	-0.172366441228885\\
68.75	0.11862	-0.203335925548418\\
68.75	0.12228	-0.233249261391583\\
68.75	0.12594	-0.262106448758372\\
68.75	0.1296	-0.289907487648789\\
68.75	0.13326	-0.316652378062838\\
68.75	0.13692	-0.342341120000512\\
68.75	0.14058	-0.366973713461817\\
68.75	0.14424	-0.390550158446747\\
68.75	0.1479	-0.413070454955305\\
68.75	0.15156	-0.434534602987493\\
68.75	0.15522	-0.45494260254331\\
68.75	0.15888	-0.474294453622752\\
68.75	0.16254	-0.492590156225826\\
68.75	0.1662	-0.509829710352525\\
68.75	0.16986	-0.526013116002854\\
68.75	0.17352	-0.541140373176812\\
68.75	0.17718	-0.555211481874395\\
68.75	0.18084	-0.568226442095609\\
68.75	0.1845	-0.58018525384045\\
68.75	0.18816	-0.59108791710892\\
68.75	0.19182	-0.600934431901018\\
68.75	0.19548	-0.609724798216741\\
68.75	0.19914	-0.617459016056097\\
68.75	0.2028	-0.624137085419077\\
68.75	0.20646	-0.629759006305689\\
68.75	0.21012	-0.634324778715927\\
68.75	0.21378	-0.637834402649792\\
68.75	0.21744	-0.640287878107289\\
68.75	0.2211	-0.641685205088411\\
68.75	0.22476	-0.642026383593161\\
68.75	0.22842	-0.641311413621542\\
68.75	0.23208	-0.639540295173548\\
68.75	0.23574	-0.636713028249186\\
68.75	0.2394	-0.632829612848448\\
68.75	0.24306	-0.627890048971339\\
68.75	0.24672	-0.621894336617861\\
68.75	0.25038	-0.614842475788008\\
68.75	0.25404	-0.606734466481787\\
68.75	0.2577	-0.597570308699186\\
68.75	0.26136	-0.587350002440222\\
68.75	0.26502	-0.576073547704882\\
68.75	0.26868	-0.563740944493173\\
68.75	0.27234	-0.550352192805089\\
68.75	0.276	-0.535907292640633\\
69.125	0.093	0.055666643050114\\
69.125	0.09666	0.0164387762577778\\
69.125	0.10032	-0.0217329420581864\\
69.125	0.10398	-0.0588485118977772\\
69.125	0.10764	-0.0949079332609979\\
69.125	0.1113	-0.129911206147843\\
69.125	0.11496	-0.163858330558322\\
69.125	0.11862	-0.196749306492426\\
69.125	0.12228	-0.228584133950159\\
69.125	0.12594	-0.259362812931519\\
69.125	0.1296	-0.289085343436507\\
69.125	0.13326	-0.317751725465125\\
69.125	0.13692	-0.345361959017369\\
69.125	0.14058	-0.371916044093243\\
69.125	0.14424	-0.397413980692745\\
69.125	0.1479	-0.421855768815874\\
69.125	0.15156	-0.445241408462633\\
69.125	0.15522	-0.467570899633018\\
69.125	0.15888	-0.488844242327031\\
69.125	0.16254	-0.509061436544675\\
69.125	0.1662	-0.528222482285944\\
69.125	0.16986	-0.546327379550846\\
69.125	0.17352	-0.56337612833937\\
69.125	0.17718	-0.579368728651528\\
69.125	0.18084	-0.594305180487309\\
69.125	0.1845	-0.608185483846721\\
69.125	0.18816	-0.621009638729761\\
69.125	0.19182	-0.63277764513643\\
69.125	0.19548	-0.643489503066723\\
69.125	0.19914	-0.65314521252065\\
69.125	0.2028	-0.661744773498201\\
69.125	0.20646	-0.669288185999384\\
69.125	0.21012	-0.675775450024188\\
69.125	0.21378	-0.681206565572628\\
69.125	0.21744	-0.685581532644692\\
69.125	0.2211	-0.688900351240388\\
69.125	0.22476	-0.691163021359705\\
69.125	0.22842	-0.692369543002657\\
69.125	0.23208	-0.692519916169234\\
69.125	0.23574	-0.691614140859442\\
69.125	0.2394	-0.689652217073275\\
69.125	0.24306	-0.686634144810736\\
69.125	0.24672	-0.682559924071826\\
69.125	0.25038	-0.677429554856543\\
69.125	0.25404	-0.671243037164893\\
69.125	0.2577	-0.664000370996863\\
69.125	0.26136	-0.655701556352469\\
69.125	0.26502	-0.646346593231703\\
69.125	0.26868	-0.635935481634562\\
69.125	0.27234	-0.624468221561048\\
69.125	0.276	-0.611944813011164\\
69.5	0.093	0.0764810765273092\\
69.5	0.09666	0.0353317181204023\\
69.5	0.10032	-0.00476149181012903\\
69.5	0.10398	-0.0437985532642923\\
69.5	0.10764	-0.0817794662420837\\
69.5	0.1113	-0.118704230743501\\
69.5	0.11496	-0.154572846768549\\
69.5	0.11862	-0.189385314317223\\
69.5	0.12228	-0.223141633389527\\
69.5	0.12594	-0.255841803985456\\
69.5	0.1296	-0.287485826105015\\
69.5	0.13326	-0.318073699748203\\
69.5	0.13692	-0.347605424915018\\
69.5	0.14058	-0.376081001605465\\
69.5	0.14424	-0.403500429819534\\
69.5	0.1479	-0.429863709557234\\
69.5	0.15156	-0.455170840818563\\
69.5	0.15522	-0.479421823603519\\
69.5	0.15888	-0.502616657912105\\
69.5	0.16254	-0.524755343744317\\
69.5	0.1662	-0.545837881100159\\
69.5	0.16986	-0.565864269979627\\
69.5	0.17352	-0.584834510382726\\
69.5	0.17718	-0.602748602309451\\
69.5	0.18084	-0.619606545759802\\
69.5	0.1845	-0.635408340733785\\
69.5	0.18816	-0.650153987231396\\
69.5	0.19182	-0.663843485252636\\
69.5	0.19548	-0.6764768347975\\
69.5	0.19914	-0.688054035865993\\
69.5	0.2028	-0.698575088458119\\
69.5	0.20646	-0.708039992573869\\
69.5	0.21012	-0.716448748213248\\
69.5	0.21378	-0.723801355376254\\
69.5	0.21744	-0.730097814062889\\
69.5	0.2211	-0.735338124273156\\
69.5	0.22476	-0.739522286007043\\
69.5	0.22842	-0.742650299264566\\
69.5	0.23208	-0.744722164045714\\
69.5	0.23574	-0.745737880350489\\
69.5	0.2394	-0.745697448178896\\
69.5	0.24306	-0.744600867530925\\
69.5	0.24672	-0.742448138406588\\
69.5	0.25038	-0.739239260805876\\
69.5	0.25404	-0.734974234728793\\
69.5	0.2577	-0.729653060175337\\
69.5	0.26136	-0.723275737145511\\
69.5	0.26502	-0.715842265639315\\
69.5	0.26868	-0.707352645656745\\
69.5	0.27234	-0.697806877197802\\
69.5	0.276	-0.687204960262488\\
69.875	0.093	0.0980728831237121\\
69.875	0.09666	0.0550020331022363\\
69.875	0.10032	0.0129873315571326\\
69.875	0.10398	-0.0279712215115996\\
69.875	0.10764	-0.0678736261039616\\
69.875	0.1113	-0.106719882219948\\
69.875	0.11496	-0.144509989859567\\
69.875	0.11862	-0.181243949022811\\
69.875	0.12228	-0.216921759709686\\
69.875	0.12594	-0.251543421920186\\
69.875	0.1296	-0.285108935654315\\
69.875	0.13326	-0.317618300912074\\
69.875	0.13692	-0.349071517693458\\
69.875	0.14058	-0.379468585998473\\
69.875	0.14424	-0.408809505827115\\
69.875	0.1479	-0.437094277179386\\
69.875	0.15156	-0.464322900055286\\
69.875	0.15522	-0.490495374454811\\
69.875	0.15888	-0.515611700377965\\
69.875	0.16254	-0.53967187782475\\
69.875	0.1662	-0.562675906795159\\
69.875	0.16986	-0.584623787289202\\
69.875	0.17352	-0.605515519306867\\
69.875	0.17718	-0.625351102848163\\
69.875	0.18084	-0.644130537913088\\
69.875	0.1845	-0.661853824501639\\
69.875	0.18816	-0.67852096261382\\
69.875	0.19182	-0.69413195224963\\
69.875	0.19548	-0.708686793409065\\
69.875	0.19914	-0.722185486092129\\
69.875	0.2028	-0.734628030298822\\
69.875	0.20646	-0.746014426029146\\
69.875	0.21012	-0.756344673283092\\
69.875	0.21378	-0.765618772060669\\
69.875	0.21744	-0.773836722361878\\
69.875	0.2211	-0.780998524186712\\
69.875	0.22476	-0.78710417753517\\
69.875	0.22842	-0.792153682407264\\
69.875	0.23208	-0.796147038802982\\
69.875	0.23574	-0.799084246722328\\
69.875	0.2394	-0.800965306165306\\
69.875	0.24306	-0.801790217131905\\
69.875	0.24672	-0.801558979622136\\
69.875	0.25038	-0.800271593635995\\
69.875	0.25404	-0.797928059173485\\
69.875	0.2577	-0.794528376234597\\
69.875	0.26136	-0.790072544819341\\
69.875	0.26502	-0.784560564927716\\
69.875	0.26868	-0.777992436559716\\
69.875	0.27234	-0.77036815971534\\
69.875	0.276	-0.7616877343946\\
70.25	0.093	0.120442062839325\\
70.25	0.09666	0.0754497212032799\\
70.25	0.10032	0.0315135280436037\\
70.25	0.10398	-0.0113665166396974\\
70.25	0.10764	-0.0531904128466301\\
70.25	0.1113	-0.0939581605771874\\
70.25	0.11496	-0.133669759831375\\
70.25	0.11862	-0.17232521060919\\
70.25	0.12228	-0.209924512910634\\
70.25	0.12594	-0.246467666735706\\
70.25	0.1296	-0.281954672084404\\
70.25	0.13326	-0.316385528956734\\
70.25	0.13692	-0.349760237352688\\
70.25	0.14058	-0.382078797272274\\
70.25	0.14424	-0.413341208715487\\
70.25	0.1479	-0.443547471682326\\
70.25	0.15156	-0.472697586172797\\
70.25	0.15522	-0.500791552186892\\
70.25	0.15888	-0.527829369724617\\
70.25	0.16254	-0.553811038785973\\
70.25	0.1662	-0.578736559370953\\
70.25	0.16986	-0.602605931479562\\
70.25	0.17352	-0.625419155111802\\
70.25	0.17718	-0.647176230267669\\
70.25	0.18084	-0.667877156947161\\
70.25	0.1845	-0.687521935150282\\
70.25	0.18816	-0.706110564877034\\
70.25	0.19182	-0.723643046127415\\
70.25	0.19548	-0.74011937890142\\
70.25	0.19914	-0.755539563199056\\
70.25	0.2028	-0.769903599020319\\
70.25	0.20646	-0.78321148636521\\
70.25	0.21012	-0.79546322523373\\
70.25	0.21378	-0.806658815625878\\
70.25	0.21744	-0.816798257541654\\
70.25	0.2211	-0.825881550981059\\
70.25	0.22476	-0.833908695944091\\
70.25	0.22842	-0.840879692430752\\
70.25	0.23208	-0.846794540441041\\
70.25	0.23574	-0.851653239974957\\
70.25	0.2394	-0.855455791032502\\
70.25	0.24306	-0.858202193613676\\
70.25	0.24672	-0.859892447718477\\
70.25	0.25038	-0.860526553346907\\
70.25	0.25404	-0.860104510498965\\
70.25	0.2577	-0.858626319174647\\
70.25	0.26136	-0.856091979373965\\
70.25	0.26502	-0.852501491096907\\
70.25	0.26868	-0.847854854343478\\
70.25	0.27234	-0.842152069113673\\
70.25	0.276	-0.835393135407504\\
70.625	0.093	0.143588615674148\\
70.625	0.09666	0.0966747824235295\\
70.625	0.10032	0.0508170976492862\\
70.625	0.10398	0.00601556135141262\\
70.625	0.10764	-0.0377298264700889\\
70.625	0.1113	-0.0804190658152169\\
70.625	0.11496	-0.122052156683977\\
70.625	0.11862	-0.162629099076361\\
70.625	0.12228	-0.202149892992375\\
70.625	0.12594	-0.240614538432018\\
70.625	0.1296	-0.278023035395287\\
70.625	0.13326	-0.314375383882186\\
70.625	0.13692	-0.349671583892713\\
70.625	0.14058	-0.383911635426867\\
70.625	0.14424	-0.417095538484651\\
70.625	0.1479	-0.449223293066061\\
70.625	0.15156	-0.4802948991711\\
70.625	0.15522	-0.510310356799768\\
70.625	0.15888	-0.539269665952062\\
70.625	0.16254	-0.567172826627988\\
70.625	0.1662	-0.594019838827539\\
70.625	0.16986	-0.619810702550719\\
70.625	0.17352	-0.644545417797526\\
70.625	0.17718	-0.668223984567963\\
70.625	0.18084	-0.69084640286203\\
70.625	0.1845	-0.712412672679722\\
70.625	0.18816	-0.732922794021044\\
70.625	0.19182	-0.752376766885992\\
70.625	0.19548	-0.770774591274568\\
70.625	0.19914	-0.788116267186774\\
70.625	0.2028	-0.804401794622608\\
70.625	0.20646	-0.81963117358207\\
70.625	0.21012	-0.833804404065157\\
70.625	0.21378	-0.846921486071876\\
70.625	0.21744	-0.858982419602226\\
70.625	0.2211	-0.869987204656201\\
70.625	0.22476	-0.879935841233801\\
70.625	0.22842	-0.888828329335032\\
70.625	0.23208	-0.896664668959892\\
70.625	0.23574	-0.903444860108379\\
70.625	0.2394	-0.909168902780495\\
70.625	0.24306	-0.913836796976235\\
70.625	0.24672	-0.917448542695607\\
70.625	0.25038	-0.920004139938607\\
70.625	0.25404	-0.92150358870524\\
70.625	0.2577	-0.921946888995492\\
70.625	0.26136	-0.921334040809378\\
70.625	0.26502	-0.919665044146891\\
70.625	0.26868	-0.916939899008032\\
70.625	0.27234	-0.913158605392798\\
70.625	0.276	-0.908321163301196\\
71	0.093	0.167512541628178\\
71	0.09666	0.11867721676299\\
71	0.10032	0.0708980403741746\\
71	0.10398	0.0241750124617322\\
71	0.10764	-0.02149186697434\\
71	0.1113	-0.0661025979340387\\
71	0.11496	-0.109657180417367\\
71	0.11862	-0.152155614424322\\
71	0.12228	-0.193597899954907\\
71	0.12594	-0.233984037009119\\
71	0.1296	-0.273314025586958\\
71	0.13326	-0.31158786568843\\
71	0.13692	-0.348805557313525\\
71	0.14058	-0.384967100462251\\
71	0.14424	-0.420072495134603\\
71	0.1479	-0.454121741330584\\
71	0.15156	-0.487114839050196\\
71	0.15522	-0.519051788293433\\
71	0.15888	-0.549932589060297\\
71	0.16254	-0.579757241350794\\
71	0.1662	-0.608525745164915\\
71	0.16986	-0.636238100502666\\
71	0.17352	-0.662894307364044\\
71	0.17718	-0.688494365749052\\
71	0.18084	-0.713038275657686\\
71	0.1845	-0.73652603708995\\
71	0.18816	-0.758957650045841\\
71	0.19182	-0.780333114525362\\
71	0.19548	-0.800652430528507\\
71	0.19914	-0.819915598055283\\
71	0.2028	-0.838122617105687\\
71	0.20646	-0.85527348767972\\
71	0.21012	-0.871368209777378\\
71	0.21378	-0.886406783398668\\
71	0.21744	-0.900389208543585\\
71	0.2211	-0.913315485212131\\
71	0.22476	-0.925185613404301\\
71	0.22842	-0.935999593120103\\
71	0.23208	-0.945757424359533\\
71	0.23574	-0.954459107122591\\
71	0.2394	-0.962104641409277\\
71	0.24306	-0.968694027219589\\
71	0.24672	-0.974227264553531\\
71	0.25038	-0.978704353411102\\
71	0.25404	-0.982125293792302\\
71	0.2577	-0.984490085697125\\
71	0.26136	-0.985798729125581\\
71	0.26502	-0.986051224077665\\
71	0.26868	-0.985247570553377\\
71	0.27234	-0.983387768552713\\
71	0.276	-0.980471818075682\\
71.375	0.093	0.192213840701419\\
71.375	0.09666	0.141457024221661\\
71.375	0.10032	0.0917563562182744\\
71.375	0.10398	0.0431118366912613\\
71.375	0.10764	-0.00447653435938156\\
71.375	0.1113	-0.0510087569336491\\
71.375	0.11496	-0.0964848310315484\\
71.375	0.11862	-0.140904756653074\\
71.375	0.12228	-0.18426853379823\\
71.375	0.12594	-0.226576162467012\\
71.375	0.1296	-0.267827642659422\\
71.375	0.13326	-0.308022974375462\\
71.375	0.13692	-0.347162157615129\\
71.375	0.14058	-0.385245192378425\\
71.375	0.14424	-0.422272078665348\\
71.375	0.1479	-0.458242816475899\\
71.375	0.15156	-0.49315740581008\\
71.375	0.15522	-0.527015846667887\\
71.375	0.15888	-0.559818139049325\\
71.375	0.16254	-0.591564282954388\\
71.375	0.1662	-0.62225427838308\\
71.375	0.16986	-0.651888125335402\\
71.375	0.17352	-0.680465823811351\\
71.375	0.17718	-0.707987373810927\\
71.375	0.18084	-0.734452775334133\\
71.375	0.1845	-0.759862028380968\\
71.375	0.18816	-0.78421513295143\\
71.375	0.19182	-0.807512089045521\\
71.375	0.19548	-0.829752896663237\\
71.375	0.19914	-0.850937555804584\\
71.375	0.2028	-0.871066066469556\\
71.375	0.20646	-0.890138428658159\\
71.375	0.21012	-0.908154642370388\\
71.375	0.21378	-0.925114707606248\\
71.375	0.21744	-0.941018624365736\\
71.375	0.2211	-0.955866392648852\\
71.375	0.22476	-0.969658012455593\\
71.375	0.22842	-0.982393483785966\\
71.375	0.23208	-0.994072806639967\\
71.375	0.23574	-1.0046959810176\\
71.375	0.2394	-1.01426300691885\\
71.375	0.24306	-1.02277388434373\\
71.375	0.24672	-1.03022861329225\\
71.375	0.25038	-1.03662719376439\\
71.375	0.25404	-1.04196962576016\\
71.375	0.2577	-1.04625590927955\\
71.375	0.26136	-1.04948604432258\\
71.375	0.26502	-1.05166003088923\\
71.375	0.26868	-1.05277786897951\\
71.375	0.27234	-1.05283955859342\\
71.375	0.276	-1.05184509973096\\
71.75	0.093	0.217692512893865\\
71.75	0.09666	0.165014204799537\\
71.75	0.10032	0.113392045181582\\
71.75	0.10398	0.0628260340399965\\
71.75	0.10764	0.0133161713747847\\
71.75	0.1113	-0.0351375428140553\\
71.75	0.11496	-0.0825351085265252\\
71.75	0.11862	-0.12887652576262\\
71.75	0.12228	-0.174161794522346\\
71.75	0.12594	-0.218390914805699\\
71.75	0.1296	-0.26156388661268\\
71.75	0.13326	-0.303680709943289\\
71.75	0.13692	-0.344741384797526\\
71.75	0.14058	-0.384745911175393\\
71.75	0.14424	-0.423694289076887\\
71.75	0.1479	-0.461586518502008\\
71.75	0.15156	-0.498422599450758\\
71.75	0.15522	-0.534202531923136\\
71.75	0.15888	-0.568926315919142\\
71.75	0.16254	-0.602593951438779\\
71.75	0.1662	-0.635205438482041\\
71.75	0.16986	-0.666760777048934\\
71.75	0.17352	-0.697259967139455\\
71.75	0.17718	-0.726703008753603\\
71.75	0.18084	-0.755089901891373\\
71.75	0.1845	-0.782420646552779\\
71.75	0.18816	-0.808695242737812\\
71.75	0.19182	-0.833913690446473\\
71.75	0.19548	-0.85807598967876\\
71.75	0.19914	-0.881182140434674\\
71.75	0.2028	-0.90323214271422\\
71.75	0.20646	-0.924225996517394\\
71.75	0.21012	-0.944163701844193\\
71.75	0.21378	-0.963045258694624\\
71.75	0.21744	-0.980870667068683\\
71.75	0.2211	-0.997639926966366\\
71.75	0.22476	-1.01335303838768\\
71.75	0.22842	-1.02801000133262\\
71.75	0.23208	-1.04161081580119\\
71.75	0.23574	-1.05415548179339\\
71.75	0.2394	-1.06564399930922\\
71.75	0.24306	-1.07607636834867\\
71.75	0.24672	-1.08545258891175\\
71.75	0.25038	-1.09377266099846\\
71.75	0.25404	-1.10103658460881\\
71.75	0.2577	-1.10724435974277\\
71.75	0.26136	-1.11239598640036\\
71.75	0.26502	-1.11649146458159\\
71.75	0.26868	-1.11953079428644\\
71.75	0.27234	-1.12151397551492\\
71.75	0.276	-1.12244100826703\\
72.125	0.093	0.24394855820552\\
72.125	0.09666	0.189348758496622\\
72.125	0.10032	0.135805107264096\\
72.125	0.10398	0.0833176045079412\\
72.125	0.10764	0.0318862502281587\\
72.125	0.1113	-0.0184889555752502\\
72.125	0.11496	-0.0678080129022908\\
72.125	0.11862	-0.116070921752958\\
72.125	0.12228	-0.163277682127253\\
72.125	0.12594	-0.209428294025176\\
72.125	0.1296	-0.254522757446726\\
72.125	0.13326	-0.298561072391908\\
72.125	0.13692	-0.341543238860714\\
72.125	0.14058	-0.383469256853151\\
72.125	0.14424	-0.424339126369216\\
72.125	0.1479	-0.464152847408906\\
72.125	0.15156	-0.502910419972229\\
72.125	0.15522	-0.540611844059176\\
72.125	0.15888	-0.577257119669752\\
72.125	0.16254	-0.612846246803961\\
72.125	0.1662	-0.647379225461792\\
72.125	0.16986	-0.680856055643252\\
72.125	0.17352	-0.71327673734834\\
72.125	0.17718	-0.744641270577063\\
72.125	0.18084	-0.774949655329408\\
72.125	0.1845	-0.80420189160538\\
72.125	0.18816	-0.832397979404984\\
72.125	0.19182	-0.859537918728216\\
72.125	0.19548	-0.885621709575073\\
72.125	0.19914	-0.910649351945558\\
72.125	0.2028	-0.934620845839674\\
72.125	0.20646	-0.957536191257419\\
72.125	0.21012	-0.979395388198785\\
72.125	0.21378	-1.00019843666379\\
72.125	0.21744	-1.01994533665242\\
72.125	0.2211	-1.03863608816467\\
72.125	0.22476	-1.05627069120055\\
72.125	0.22842	-1.07284914576007\\
72.125	0.23208	-1.08837145184321\\
72.125	0.23574	-1.10283760944998\\
72.125	0.2394	-1.11624761858037\\
72.125	0.24306	-1.1286014792344\\
72.125	0.24672	-1.13989919141205\\
72.125	0.25038	-1.15014075511333\\
72.125	0.25404	-1.15932617033824\\
72.125	0.2577	-1.16745543708678\\
72.125	0.26136	-1.17452855535895\\
72.125	0.26502	-1.18054552515474\\
72.125	0.26868	-1.18550634647416\\
72.125	0.27234	-1.18941101931721\\
72.125	0.276	-1.19225954368389\\
72.5	0.093	0.270981976636384\\
72.5	0.09666	0.214460685312917\\
72.5	0.10032	0.158995542465822\\
72.5	0.10398	0.104586548095095\\
72.5	0.10764	0.0512337022007423\\
72.5	0.1113	-0.00106299521723724\\
72.5	0.11496	-0.0523035441588467\\
72.5	0.11862	-0.102487944624085\\
72.5	0.12228	-0.15161619661295\\
72.5	0.12594	-0.199688300125443\\
72.5	0.1296	-0.246704255161565\\
72.5	0.13326	-0.292664061721315\\
72.5	0.13692	-0.337567719804692\\
72.5	0.14058	-0.3814152294117\\
72.5	0.14424	-0.424206590542333\\
72.5	0.1479	-0.465941803196597\\
72.5	0.15156	-0.506620867374488\\
72.5	0.15522	-0.546243783076006\\
72.5	0.15888	-0.584810550301151\\
72.5	0.16254	-0.622321169049929\\
72.5	0.1662	-0.658775639322331\\
72.5	0.16986	-0.694173961118365\\
72.5	0.17352	-0.728516134438023\\
72.5	0.17718	-0.761802159281314\\
72.5	0.18084	-0.794032035648228\\
72.5	0.1845	-0.825205763538772\\
72.5	0.18816	-0.855323342952946\\
72.5	0.19182	-0.884384773890749\\
72.5	0.19548	-0.912390056352177\\
72.5	0.19914	-0.939339190337232\\
72.5	0.2028	-0.965232175845919\\
72.5	0.20646	-0.990069012878231\\
72.5	0.21012	-1.01384970143417\\
72.5	0.21378	-1.03657424151374\\
72.5	0.21744	-1.05824263311694\\
72.5	0.2211	-1.07885487624377\\
72.5	0.22476	-1.09841097089422\\
72.5	0.22842	-1.1169109170683\\
72.5	0.23208	-1.13435471476602\\
72.5	0.23574	-1.15074236398735\\
72.5	0.2394	-1.16607386473232\\
72.5	0.24306	-1.18034921700092\\
72.5	0.24672	-1.19356842079314\\
72.5	0.25038	-1.20573147610899\\
72.5	0.25404	-1.21683838294847\\
72.5	0.2577	-1.22688914131158\\
72.5	0.26136	-1.23588375119831\\
72.5	0.26502	-1.24382221260868\\
72.5	0.26868	-1.25070452554267\\
72.5	0.27234	-1.25653069000029\\
72.5	0.276	-1.26130070598154\\
72.875	0.093	0.298792768186456\\
72.875	0.09666	0.240349985248419\\
72.875	0.10032	0.182963350786753\\
72.875	0.10398	0.126632864801457\\
72.875	0.10764	0.0713585272925337\\
72.875	0.1113	0.0171403382599835\\
72.875	0.11496	-0.0360217022961985\\
72.875	0.11862	-0.0881275943760051\\
72.875	0.12228	-0.139177337979442\\
72.875	0.12594	-0.189170933106505\\
72.875	0.1296	-0.238108379757196\\
72.875	0.13326	-0.285989677931517\\
72.875	0.13692	-0.332814827629464\\
72.875	0.14058	-0.378583828851043\\
72.875	0.14424	-0.423296681596247\\
72.875	0.1479	-0.466953385865079\\
72.875	0.15156	-0.509553941657543\\
72.875	0.15522	-0.551098348973631\\
72.875	0.15888	-0.591586607813351\\
72.875	0.16254	-0.631018718176696\\
72.875	0.1662	-0.669394680063668\\
72.875	0.16986	-0.706714493474273\\
72.875	0.17352	-0.742978158408499\\
72.875	0.17718	-0.778185674866363\\
72.875	0.18084	-0.812337042847845\\
72.875	0.1845	-0.845432262352962\\
72.875	0.18816	-0.877471333381704\\
72.875	0.19182	-0.908454255934074\\
72.875	0.19548	-0.938381030010072\\
72.875	0.19914	-0.967251655609702\\
72.875	0.2028	-0.995066132732956\\
72.875	0.20646	-1.02182446137984\\
72.875	0.21012	-1.04752664155035\\
72.875	0.21378	-1.07217267324449\\
72.875	0.21744	-1.09576255646226\\
72.875	0.2211	-1.11829629120366\\
72.875	0.22476	-1.13977387746868\\
72.875	0.22842	-1.16019531525733\\
72.875	0.23208	-1.17956060456962\\
72.875	0.23574	-1.19786974540553\\
72.875	0.2394	-1.21512273776507\\
72.875	0.24306	-1.23131958164823\\
72.875	0.24672	-1.24646027705502\\
72.875	0.25038	-1.26054482398544\\
72.875	0.25404	-1.2735732224395\\
72.875	0.2577	-1.28554547241717\\
72.875	0.26136	-1.29646157391848\\
72.875	0.26502	-1.30632152694342\\
72.875	0.26868	-1.31512533149198\\
72.875	0.27234	-1.32287298756417\\
72.875	0.276	-1.32956449515999\\
73.25	0.093	0.327380932855736\\
73.25	0.09666	0.26701665830313\\
73.25	0.10032	0.207708532226892\\
73.25	0.10398	0.149456554627027\\
73.25	0.10764	0.0922607255035328\\
73.25	0.1113	0.0361210448564119\\
73.25	0.11496	-0.0189624873143389\\
73.25	0.11862	-0.0729898710087162\\
73.25	0.12228	-0.125961106226723\\
73.25	0.12594	-0.177876192968357\\
73.25	0.1296	-0.228735131233619\\
73.25	0.13326	-0.278537921022511\\
73.25	0.13692	-0.327284562335029\\
73.25	0.14058	-0.374975055171177\\
73.25	0.14424	-0.421609399530949\\
73.25	0.1479	-0.467187595414356\\
73.25	0.15156	-0.511709642821386\\
73.25	0.15522	-0.555175541752045\\
73.25	0.15888	-0.597585292206336\\
73.25	0.16254	-0.638938894184248\\
73.25	0.1662	-0.679236347685795\\
73.25	0.16986	-0.718477652710966\\
73.25	0.17352	-0.756662809259766\\
73.25	0.17718	-0.793791817332198\\
73.25	0.18084	-0.829864676928254\\
73.25	0.1845	-0.864881388047938\\
73.25	0.18816	-0.898841950691251\\
73.25	0.19182	-0.931746364858191\\
73.25	0.19548	-0.96359463054876\\
73.25	0.19914	-0.994386747762961\\
73.25	0.2028	-1.02412271650079\\
73.25	0.20646	-1.05280253676224\\
73.25	0.21012	-1.08042620854732\\
73.25	0.21378	-1.10699373185603\\
73.25	0.21744	-1.13250510668837\\
73.25	0.2211	-1.15696033304434\\
73.25	0.22476	-1.18035941092393\\
73.25	0.22842	-1.20270234032716\\
73.25	0.23208	-1.22398912125401\\
73.25	0.23574	-1.24421975370449\\
73.25	0.2394	-1.2633942376786\\
73.25	0.24306	-1.28151257317633\\
73.25	0.24672	-1.2985747601977\\
73.25	0.25038	-1.31458079874269\\
73.25	0.25404	-1.32953068881131\\
73.25	0.2577	-1.34342443040356\\
73.25	0.26136	-1.35626202351943\\
73.25	0.26502	-1.36804346815894\\
73.25	0.26868	-1.37876876432207\\
73.25	0.27234	-1.38843791200883\\
73.25	0.276	-1.39705091121922\\
73.625	0.093	0.35674647064423\\
73.625	0.09666	0.294460704477049\\
73.625	0.10032	0.233231086786244\\
73.625	0.10398	0.173057617571807\\
73.625	0.10764	0.113940296833742\\
73.625	0.1113	0.0558791245720517\\
73.625	0.11496	-0.00112589921326978\\
73.625	0.11862	-0.0570747745222178\\
73.625	0.12228	-0.111967501354796\\
73.625	0.12594	-0.165804079710998\\
73.625	0.1296	-0.218584509590831\\
73.625	0.13326	-0.270308790994293\\
73.625	0.13692	-0.32097692392138\\
73.625	0.14058	-0.3705889083721\\
73.625	0.14424	-0.419144744346445\\
73.625	0.1479	-0.466644431844419\\
73.625	0.15156	-0.51308797086602\\
73.625	0.15522	-0.55847536141125\\
73.625	0.15888	-0.602806603480111\\
73.625	0.16254	-0.646081697072597\\
73.625	0.1662	-0.688300642188708\\
73.625	0.16986	-0.729463438828454\\
73.625	0.17352	-0.769570086991824\\
73.625	0.17718	-0.808620586678823\\
73.625	0.18084	-0.84661493788945\\
73.625	0.1845	-0.883553140623705\\
73.625	0.18816	-0.919435194881591\\
73.625	0.19182	-0.954261100663099\\
73.625	0.19548	-0.988030857968239\\
73.625	0.19914	-1.02074446679701\\
73.625	0.2028	-1.05240192714941\\
73.625	0.20646	-1.08300323902543\\
73.625	0.21012	-1.11254840242508\\
73.625	0.21378	-1.14103741734836\\
73.625	0.21744	-1.16847028379527\\
73.625	0.2211	-1.19484700176581\\
73.625	0.22476	-1.22016757125997\\
73.625	0.22842	-1.24443199227777\\
73.625	0.23208	-1.26764026481919\\
73.625	0.23574	-1.28979238888424\\
73.625	0.2394	-1.31088836447292\\
73.625	0.24306	-1.33092819158522\\
73.625	0.24672	-1.34991187022116\\
73.625	0.25038	-1.36783940038072\\
73.625	0.25404	-1.38471078206391\\
73.625	0.2577	-1.40052601527073\\
73.625	0.26136	-1.41528510000117\\
73.625	0.26502	-1.42898803625525\\
73.625	0.26868	-1.44163482403296\\
73.625	0.27234	-1.45322546333429\\
73.625	0.276	-1.46375995415925\\
74	0.093	0.386889381551931\\
74	0.09666	0.322682123770181\\
74	0.10032	0.259531014464803\\
74	0.10398	0.197436053635797\\
74	0.10764	0.136397241283162\\
74	0.1113	0.0764145774069011\\
74	0.11496	0.0174880620070089\\
74	0.11862	-0.040382304916508\\
74	0.12228	-0.0971965233636565\\
74	0.12594	-0.15295459333443\\
74	0.1296	-0.207656514828833\\
74	0.13326	-0.261302287846864\\
74	0.13692	-0.313891912388524\\
74	0.14058	-0.365425388453815\\
74	0.14424	-0.415902716042727\\
74	0.1479	-0.465323895155271\\
74	0.15156	-0.513688925791443\\
74	0.15522	-0.560997807951247\\
74	0.15888	-0.607250541634672\\
74	0.16254	-0.652447126841732\\
74	0.1662	-0.696587563572413\\
74	0.16986	-0.73967185182673\\
74	0.17352	-0.781699991604667\\
74	0.17718	-0.82267198290624\\
74	0.18084	-0.862587825731434\\
74	0.1845	-0.901447520080264\\
74	0.18816	-0.939251065952717\\
74	0.19182	-0.975998463348799\\
74	0.19548	-1.01168971226851\\
74	0.19914	-1.04632481271185\\
74	0.2028	-1.07990376467881\\
74	0.20646	-1.11242656816941\\
74	0.21012	-1.14389322318363\\
74	0.21378	-1.17430372972148\\
74	0.21744	-1.20365808778296\\
74	0.2211	-1.23195629736807\\
74	0.22476	-1.2591983584768\\
74	0.22842	-1.28538427110917\\
74	0.23208	-1.31051403526516\\
74	0.23574	-1.33458765094478\\
74	0.2394	-1.35760511814803\\
74	0.24306	-1.3795664368749\\
74	0.24672	-1.40047160712541\\
74	0.25038	-1.42032062889954\\
74	0.25404	-1.4391135021973\\
74	0.2577	-1.45685022701869\\
74	0.26136	-1.47353080336371\\
74	0.26502	-1.48915523123236\\
74	0.26868	-1.50372351062463\\
74	0.27234	-1.51723564154053\\
74	0.276	-1.52969162398006\\
};
\end{axis}

\begin{axis}[%
width=4.527496cm,
height=3.870968cm,
at={(6.483547cm,0cm)},
scale only axis,
xmin=56,
xmax=74,
tick align=outside,
xlabel={$L_{cut}$},
xmajorgrids,
ymin=0.093,
ymax=0.276,
ylabel={$D_{rlx}$},
ymajorgrids,
zmin=0,
zmax=113.453205752381,
zlabel={$x_3,x_4$},
zmajorgrids,
view={-140}{50},
legend style={at={(1.03,1)},anchor=north west,legend cell align=left,align=left,draw=white!15!black}
]
\addplot3[only marks,mark=*,mark options={},mark size=1.5000pt,color=mycolor1] plot table[row sep=crcr,]{%
74	0.123	15.4680684467597\\
72	0.113	12.0981268654473\\
61	0.095	6.35969255839677\\
56	0.093	5.67589105085895\\
};
\addplot3[only marks,mark=*,mark options={},mark size=1.5000pt,color=mycolor2] plot table[row sep=crcr,]{%
67	0.276	110.131370256376\\
66	0.255	90.4271696447197\\
62	0.209	51.7720891508705\\
57	0.193	41.4264697981385\\
};
\addplot3[only marks,mark=*,mark options={},mark size=1.5000pt,color=black] plot table[row sep=crcr,]{%
69	0.104	9.34303234279614\\
};
\addplot3[only marks,mark=*,mark options={},mark size=1.5000pt,color=black] plot table[row sep=crcr,]{%
64	0.23	68.0446111694493\\
};

\addplot3[%
surf,
opacity=0.7,
shader=interp,
colormap={mymap}{[1pt] rgb(0pt)=(0.0901961,0.239216,0.0745098); rgb(1pt)=(0.0945149,0.242058,0.0739522); rgb(2pt)=(0.0988592,0.244894,0.0733566); rgb(3pt)=(0.103229,0.247724,0.0727241); rgb(4pt)=(0.107623,0.250549,0.0720557); rgb(5pt)=(0.112043,0.253367,0.0713525); rgb(6pt)=(0.116487,0.25618,0.0706154); rgb(7pt)=(0.120956,0.258986,0.0698456); rgb(8pt)=(0.125449,0.261787,0.0690441); rgb(9pt)=(0.129967,0.264581,0.0682118); rgb(10pt)=(0.134508,0.26737,0.06735); rgb(11pt)=(0.139074,0.270152,0.0664596); rgb(12pt)=(0.143663,0.272929,0.0655416); rgb(13pt)=(0.148275,0.275699,0.0645971); rgb(14pt)=(0.152911,0.278463,0.0636271); rgb(15pt)=(0.15757,0.281221,0.0626328); rgb(16pt)=(0.162252,0.283973,0.0616151); rgb(17pt)=(0.166957,0.286719,0.060575); rgb(18pt)=(0.171685,0.289458,0.0595136); rgb(19pt)=(0.176434,0.292191,0.0584321); rgb(20pt)=(0.181207,0.294918,0.0573313); rgb(21pt)=(0.186001,0.297639,0.0562123); rgb(22pt)=(0.190817,0.300353,0.0550763); rgb(23pt)=(0.195655,0.303061,0.0539242); rgb(24pt)=(0.200514,0.305763,0.052757); rgb(25pt)=(0.205395,0.308459,0.0515759); rgb(26pt)=(0.210296,0.311149,0.0503624); rgb(27pt)=(0.215212,0.313846,0.0490067); rgb(28pt)=(0.220142,0.316548,0.0475043); rgb(29pt)=(0.22509,0.319254,0.0458704); rgb(30pt)=(0.230056,0.321962,0.0441205); rgb(31pt)=(0.235042,0.324671,0.04227); rgb(32pt)=(0.240048,0.327379,0.0403343); rgb(33pt)=(0.245078,0.330085,0.0383287); rgb(34pt)=(0.250131,0.332786,0.0362688); rgb(35pt)=(0.25521,0.335482,0.0341698); rgb(36pt)=(0.260317,0.33817,0.0320472); rgb(37pt)=(0.265451,0.340849,0.0299163); rgb(38pt)=(0.270616,0.343517,0.0277927); rgb(39pt)=(0.275813,0.346172,0.0256916); rgb(40pt)=(0.281043,0.348814,0.0236284); rgb(41pt)=(0.286307,0.35144,0.0216186); rgb(42pt)=(0.291607,0.354048,0.0196776); rgb(43pt)=(0.296945,0.356637,0.0178207); rgb(44pt)=(0.302322,0.359206,0.0160634); rgb(45pt)=(0.307739,0.361753,0.0144211); rgb(46pt)=(0.313198,0.364275,0.0129091); rgb(47pt)=(0.318701,0.366772,0.0115428); rgb(48pt)=(0.324249,0.369242,0.0103377); rgb(49pt)=(0.329843,0.371682,0.00930909); rgb(50pt)=(0.335485,0.374093,0.00847245); rgb(51pt)=(0.341176,0.376471,0.00784314); rgb(52pt)=(0.346925,0.378826,0.00732741); rgb(53pt)=(0.352735,0.381168,0.00682184); rgb(54pt)=(0.358605,0.383497,0.00632729); rgb(55pt)=(0.364532,0.385812,0.00584464); rgb(56pt)=(0.370516,0.388113,0.00537476); rgb(57pt)=(0.376552,0.390399,0.00491852); rgb(58pt)=(0.38264,0.39267,0.00447681); rgb(59pt)=(0.388777,0.394925,0.00405048); rgb(60pt)=(0.394962,0.397164,0.00364042); rgb(61pt)=(0.401191,0.399386,0.00324749); rgb(62pt)=(0.407464,0.401592,0.00287258); rgb(63pt)=(0.413777,0.40378,0.00251655); rgb(64pt)=(0.420129,0.40595,0.00218028); rgb(65pt)=(0.426518,0.408102,0.00186463); rgb(66pt)=(0.432942,0.410234,0.00157049); rgb(67pt)=(0.439399,0.412348,0.00129873); rgb(68pt)=(0.445885,0.414441,0.00105022); rgb(69pt)=(0.452401,0.416515,0.000825833); rgb(70pt)=(0.458942,0.418567,0.000626441); rgb(71pt)=(0.465508,0.420599,0.00045292); rgb(72pt)=(0.472096,0.422609,0.000306141); rgb(73pt)=(0.478704,0.424596,0.000186979); rgb(74pt)=(0.485331,0.426562,9.63073e-05); rgb(75pt)=(0.491973,0.428504,3.49981e-05); rgb(76pt)=(0.498628,0.430422,3.92506e-06); rgb(77pt)=(0.505323,0.432315,0); rgb(78pt)=(0.512206,0.434168,0); rgb(79pt)=(0.519282,0.435983,0); rgb(80pt)=(0.526529,0.437764,0); rgb(81pt)=(0.533922,0.439512,0); rgb(82pt)=(0.54144,0.441232,0); rgb(83pt)=(0.549059,0.442927,0); rgb(84pt)=(0.556756,0.444599,0); rgb(85pt)=(0.564508,0.446252,0); rgb(86pt)=(0.572292,0.447889,0); rgb(87pt)=(0.580084,0.449514,0); rgb(88pt)=(0.587863,0.451129,0); rgb(89pt)=(0.595604,0.452737,0); rgb(90pt)=(0.603284,0.454343,0); rgb(91pt)=(0.610882,0.455948,0); rgb(92pt)=(0.618373,0.457556,0); rgb(93pt)=(0.625734,0.459171,0); rgb(94pt)=(0.632943,0.460795,0); rgb(95pt)=(0.639976,0.462432,0); rgb(96pt)=(0.64681,0.464084,0); rgb(97pt)=(0.653423,0.465756,0); rgb(98pt)=(0.659791,0.46745,0); rgb(99pt)=(0.665891,0.469169,0); rgb(100pt)=(0.6717,0.470916,0); rgb(101pt)=(0.677195,0.472696,0); rgb(102pt)=(0.682353,0.47451,0); rgb(103pt)=(0.687242,0.476355,0); rgb(104pt)=(0.691952,0.478225,0); rgb(105pt)=(0.696497,0.480118,0); rgb(106pt)=(0.700887,0.482033,0); rgb(107pt)=(0.705134,0.483968,0); rgb(108pt)=(0.709251,0.485921,0); rgb(109pt)=(0.713249,0.487891,0); rgb(110pt)=(0.71714,0.489876,0); rgb(111pt)=(0.720936,0.491875,0); rgb(112pt)=(0.724649,0.493887,0); rgb(113pt)=(0.72829,0.495909,0); rgb(114pt)=(0.731872,0.49794,0); rgb(115pt)=(0.735406,0.499979,0); rgb(116pt)=(0.738904,0.502025,0); rgb(117pt)=(0.742378,0.504075,0); rgb(118pt)=(0.74584,0.506128,0); rgb(119pt)=(0.749302,0.508182,0); rgb(120pt)=(0.752775,0.510237,0); rgb(121pt)=(0.756272,0.51229,0); rgb(122pt)=(0.759804,0.514339,0); rgb(123pt)=(0.763384,0.516385,0); rgb(124pt)=(0.767022,0.518424,0); rgb(125pt)=(0.770731,0.520455,0); rgb(126pt)=(0.774523,0.522478,0); rgb(127pt)=(0.77841,0.524489,0); rgb(128pt)=(0.782391,0.526491,0); rgb(129pt)=(0.786402,0.528496,0); rgb(130pt)=(0.790431,0.530506,0); rgb(131pt)=(0.794478,0.532521,0); rgb(132pt)=(0.798541,0.534539,0); rgb(133pt)=(0.802619,0.53656,0); rgb(134pt)=(0.806712,0.538584,0); rgb(135pt)=(0.81082,0.540609,0); rgb(136pt)=(0.81494,0.542635,0); rgb(137pt)=(0.819074,0.54466,0); rgb(138pt)=(0.823219,0.546686,0); rgb(139pt)=(0.827374,0.548709,0); rgb(140pt)=(0.831541,0.55073,0); rgb(141pt)=(0.835716,0.552749,0); rgb(142pt)=(0.8399,0.554763,0); rgb(143pt)=(0.844092,0.556774,0); rgb(144pt)=(0.848292,0.558779,0); rgb(145pt)=(0.852497,0.560778,0); rgb(146pt)=(0.856708,0.562771,0); rgb(147pt)=(0.860924,0.564756,0); rgb(148pt)=(0.865143,0.566733,0); rgb(149pt)=(0.869366,0.568701,0); rgb(150pt)=(0.873592,0.57066,0); rgb(151pt)=(0.877819,0.572608,0); rgb(152pt)=(0.882047,0.574545,0); rgb(153pt)=(0.886275,0.576471,0); rgb(154pt)=(0.890659,0.578362,0); rgb(155pt)=(0.895333,0.580203,0); rgb(156pt)=(0.900258,0.581999,0); rgb(157pt)=(0.905397,0.583755,0); rgb(158pt)=(0.910711,0.585479,0); rgb(159pt)=(0.916164,0.587176,0); rgb(160pt)=(0.921717,0.588852,0); rgb(161pt)=(0.927333,0.590513,0); rgb(162pt)=(0.932974,0.592166,0); rgb(163pt)=(0.938602,0.593815,0); rgb(164pt)=(0.94418,0.595468,0); rgb(165pt)=(0.949669,0.59713,0); rgb(166pt)=(0.955033,0.598808,0); rgb(167pt)=(0.960233,0.600507,0); rgb(168pt)=(0.965232,0.602233,0); rgb(169pt)=(0.969992,0.603992,0); rgb(170pt)=(0.974475,0.605791,0); rgb(171pt)=(0.978643,0.607636,0); rgb(172pt)=(0.98246,0.609532,0); rgb(173pt)=(0.985886,0.611486,0); rgb(174pt)=(0.988885,0.613503,0); rgb(175pt)=(0.991419,0.61559,0); rgb(176pt)=(0.99345,0.617753,0); rgb(177pt)=(0.99494,0.619997,0); rgb(178pt)=(0.995851,0.622329,0); rgb(179pt)=(0.996226,0.624763,0); rgb(180pt)=(0.996512,0.627352,0); rgb(181pt)=(0.996788,0.630095,0); rgb(182pt)=(0.997053,0.632982,0); rgb(183pt)=(0.997308,0.636004,0); rgb(184pt)=(0.997552,0.639152,0); rgb(185pt)=(0.997785,0.642416,0); rgb(186pt)=(0.998006,0.645786,0); rgb(187pt)=(0.998217,0.649253,0); rgb(188pt)=(0.998416,0.652807,0); rgb(189pt)=(0.998605,0.656439,0); rgb(190pt)=(0.998781,0.660138,0); rgb(191pt)=(0.998946,0.663897,0); rgb(192pt)=(0.9991,0.667704,0); rgb(193pt)=(0.999242,0.67155,0); rgb(194pt)=(0.999372,0.675427,0); rgb(195pt)=(0.99949,0.679323,0); rgb(196pt)=(0.999596,0.68323,0); rgb(197pt)=(0.99969,0.687139,0); rgb(198pt)=(0.999771,0.691039,0); rgb(199pt)=(0.999841,0.694921,0); rgb(200pt)=(0.999898,0.698775,0); rgb(201pt)=(0.999942,0.702592,0); rgb(202pt)=(0.999974,0.706363,0); rgb(203pt)=(0.999994,0.710077,0); rgb(204pt)=(1,0.713725,0); rgb(205pt)=(1,0.717341,0); rgb(206pt)=(1,0.720963,0); rgb(207pt)=(1,0.724591,0); rgb(208pt)=(1,0.728226,0); rgb(209pt)=(1,0.731867,0); rgb(210pt)=(1,0.735514,0); rgb(211pt)=(1,0.739167,0); rgb(212pt)=(1,0.742827,0); rgb(213pt)=(1,0.746493,0); rgb(214pt)=(1,0.750165,0); rgb(215pt)=(1,0.753843,0); rgb(216pt)=(1,0.757527,0); rgb(217pt)=(1,0.761217,0); rgb(218pt)=(1,0.764913,0); rgb(219pt)=(1,0.768615,0); rgb(220pt)=(1,0.772324,0); rgb(221pt)=(1,0.776038,0); rgb(222pt)=(1,0.779758,0); rgb(223pt)=(1,0.783484,0); rgb(224pt)=(1,0.787215,0); rgb(225pt)=(1,0.790953,0); rgb(226pt)=(1,0.794696,0); rgb(227pt)=(1,0.798445,0); rgb(228pt)=(1,0.8022,0); rgb(229pt)=(1,0.805961,0); rgb(230pt)=(1,0.809727,0); rgb(231pt)=(1,0.8135,0); rgb(232pt)=(1,0.817278,0); rgb(233pt)=(1,0.821063,0); rgb(234pt)=(1,0.824854,0); rgb(235pt)=(1,0.828652,0); rgb(236pt)=(1,0.832455,0); rgb(237pt)=(1,0.836265,0); rgb(238pt)=(1,0.840081,0); rgb(239pt)=(1,0.843903,0); rgb(240pt)=(1,0.847732,0); rgb(241pt)=(1,0.851566,0); rgb(242pt)=(1,0.855406,0); rgb(243pt)=(1,0.859253,0); rgb(244pt)=(1,0.863106,0); rgb(245pt)=(1,0.866964,0); rgb(246pt)=(1,0.870829,0); rgb(247pt)=(1,0.8747,0); rgb(248pt)=(1,0.878577,0); rgb(249pt)=(1,0.88246,0); rgb(250pt)=(1,0.886349,0); rgb(251pt)=(1,0.890243,0); rgb(252pt)=(1,0.894144,0); rgb(253pt)=(1,0.898051,0); rgb(254pt)=(1,0.901964,0); rgb(255pt)=(1,0.905882,0)},
mesh/rows=49]
table[row sep=crcr,header=false] {%
%
56	0.093	5.74613142666568\\
56	0.09666	6.17235695419101\\
56	0.10032	6.66497241887921\\
56	0.10398	7.22397782073029\\
56	0.10764	7.84937315974424\\
56	0.1113	8.54115843592107\\
56	0.11496	9.29933364926077\\
56	0.11862	10.1238987997633\\
56	0.12228	11.0148538874288\\
56	0.12594	11.9721989122571\\
56	0.1296	12.9959338742483\\
56	0.13326	14.0860587734024\\
56	0.13692	15.2425736097193\\
56	0.14058	16.4654783831992\\
56	0.14424	17.7547730938419\\
56	0.1479	19.1104577416474\\
56	0.15156	20.5325323266159\\
56	0.15522	22.0209968487472\\
56	0.15888	23.5758513080414\\
56	0.16254	25.1970957044985\\
56	0.1662	26.8847300381184\\
56	0.16986	28.6387543089012\\
56	0.17352	30.4591685168469\\
56	0.17718	32.3459726619555\\
56	0.18084	34.2991667442269\\
56	0.1845	36.3187507636612\\
56	0.18816	38.4047247202584\\
56	0.19182	40.5570886140185\\
56	0.19548	42.7758424449414\\
56	0.19914	45.0609862130273\\
56	0.2028	47.412519918276\\
56	0.20646	49.8304435606875\\
56	0.21012	52.314757140262\\
56	0.21378	54.8654606569993\\
56	0.21744	57.4825541108994\\
56	0.2211	60.1660375019625\\
56	0.22476	62.9159108301884\\
56	0.22842	65.7321740955772\\
56	0.23208	68.6148272981289\\
56	0.23574	71.5638704378435\\
56	0.2394	74.5793035147209\\
56	0.24306	77.6611265287612\\
56	0.24672	80.8093394799644\\
56	0.25038	84.0239423683305\\
56	0.25404	87.3049351938594\\
56	0.2577	90.6523179565513\\
56	0.26136	94.0660906564059\\
56	0.26502	97.5462532934235\\
56	0.26868	101.092805867604\\
56	0.27234	104.705748378947\\
56	0.276	108.385080827453\\
56.375	0.093	5.74291435535881\\
56.375	0.09666	6.16955702951079\\
56.375	0.10032	6.66258964082565\\
56.375	0.10398	7.22201218930338\\
56.375	0.10764	7.847824674944\\
56.375	0.1113	8.54002709774747\\
56.375	0.11496	9.29861945771383\\
56.375	0.11862	10.1236017548431\\
56.375	0.12228	11.0149739891352\\
56.375	0.12594	11.9727361605902\\
56.375	0.1296	12.996888269208\\
56.375	0.13326	14.0874303149887\\
56.375	0.13692	15.2443622979323\\
56.375	0.14058	16.4676842180388\\
56.375	0.14424	17.7573960753082\\
56.375	0.1479	19.1134978697404\\
56.375	0.15156	20.5359896013355\\
56.375	0.15522	22.0248712700935\\
56.375	0.15888	23.5801428760143\\
56.375	0.16254	25.201804419098\\
56.375	0.1662	26.8898558993446\\
56.375	0.16986	28.6442973167541\\
56.375	0.17352	30.4651286713265\\
56.375	0.17718	32.3523499630617\\
56.375	0.18084	34.3059611919598\\
56.375	0.1845	36.3259623580207\\
56.375	0.18816	38.4123534612446\\
56.375	0.19182	40.5651345016313\\
56.375	0.19548	42.7843054791809\\
56.375	0.19914	45.0698663938934\\
56.375	0.2028	47.4218172457687\\
56.375	0.20646	49.840158034807\\
56.375	0.21012	52.324888761008\\
56.375	0.21378	54.876009424372\\
56.375	0.21744	57.4935200248988\\
56.375	0.2211	60.1774205625886\\
56.375	0.22476	62.9277110374411\\
56.375	0.22842	65.7443914494566\\
56.375	0.23208	68.6274617986349\\
56.375	0.23574	71.5769220849762\\
56.375	0.2394	74.5927723084802\\
56.375	0.24306	77.6750124691472\\
56.375	0.24672	80.823642566977\\
56.375	0.25038	84.0386626019698\\
56.375	0.25404	87.3200725741253\\
56.375	0.2577	90.6678724834438\\
56.375	0.26136	94.0820623299252\\
56.375	0.26502	97.5626421135694\\
56.375	0.26868	101.109611834376\\
56.375	0.27234	104.722971492346\\
56.375	0.276	108.402721087479\\
56.75	0.093	5.74343965323961\\
56.75	0.09666	6.17049947401825\\
56.75	0.10032	6.66394923195977\\
56.75	0.10398	7.22378892706414\\
56.75	0.10764	7.85001855933142\\
56.75	0.1113	8.54263812876155\\
56.75	0.11496	9.30164763535456\\
56.75	0.11862	10.1270470791104\\
56.75	0.12228	11.0188364600292\\
56.75	0.12594	11.9770157781109\\
56.75	0.1296	13.0015850333553\\
56.75	0.13326	14.0925442257627\\
56.75	0.13692	15.249893355333\\
56.75	0.14058	16.4736324220661\\
56.75	0.14424	17.7637614259621\\
56.75	0.1479	19.120280367021\\
56.75	0.15156	20.5431892452428\\
56.75	0.15522	22.0324880606274\\
56.75	0.15888	23.5881768131749\\
56.75	0.16254	25.2102555028853\\
56.75	0.1662	26.8987241297585\\
56.75	0.16986	28.6535826937947\\
56.75	0.17352	30.4748311949937\\
56.75	0.17718	32.3624696333555\\
56.75	0.18084	34.3164980088803\\
56.75	0.1845	36.3369163215679\\
56.75	0.18816	38.4237245714184\\
56.75	0.19182	40.5769227584318\\
56.75	0.19548	42.796510882608\\
56.75	0.19914	45.0824889439472\\
56.75	0.2028	47.4348569424492\\
56.75	0.20646	49.8536148781141\\
56.75	0.21012	52.3387627509418\\
56.75	0.21378	54.8903005609324\\
56.75	0.21744	57.5082283080859\\
56.75	0.2211	60.1925459924023\\
56.75	0.22476	62.9432536138815\\
56.75	0.22842	65.7603511725236\\
56.75	0.23208	68.6438386683286\\
56.75	0.23574	71.5937161012965\\
56.75	0.2394	74.6099834714272\\
56.75	0.24306	77.6926407787209\\
56.75	0.24672	80.8416880231773\\
56.75	0.25038	84.0571252047967\\
56.75	0.25404	87.338952323579\\
56.75	0.2577	90.6871693795241\\
56.75	0.26136	94.1017763726321\\
56.75	0.26502	97.5827733029029\\
56.75	0.26868	101.130160170337\\
56.75	0.27234	104.743936974933\\
56.75	0.276	108.424103716693\\
57.125	0.093	5.74770732030808\\
57.125	0.09666	6.17518428771337\\
57.125	0.10032	6.66905119228155\\
57.125	0.10398	7.22930803401258\\
57.125	0.10764	7.85595481290651\\
57.125	0.1113	8.54899152896329\\
57.125	0.11496	9.30841818218295\\
57.125	0.11862	10.1342347725655\\
57.125	0.12228	11.0264413001109\\
57.125	0.12594	11.9850377648192\\
57.125	0.1296	13.0100241666904\\
57.125	0.13326	14.1014005057244\\
57.125	0.13692	15.2591667819213\\
57.125	0.14058	16.4833229952811\\
57.125	0.14424	17.7738691458038\\
57.125	0.1479	19.1308052334893\\
57.125	0.15156	20.5541312583377\\
57.125	0.15522	22.043847220349\\
57.125	0.15888	23.5999531195232\\
57.125	0.16254	25.2224489558602\\
57.125	0.1662	26.9113347293601\\
57.125	0.16986	28.6666104400229\\
57.125	0.17352	30.4882760878485\\
57.125	0.17718	32.376331672837\\
57.125	0.18084	34.3307771949885\\
57.125	0.1845	36.3516126543027\\
57.125	0.18816	38.4388380507799\\
57.125	0.19182	40.5924533844199\\
57.125	0.19548	42.8124586552228\\
57.125	0.19914	45.0988538631886\\
57.125	0.2028	47.4516390083173\\
57.125	0.20646	49.8708140906088\\
57.125	0.21012	52.3563791100632\\
57.125	0.21378	54.9083340666805\\
57.125	0.21744	57.5266789604606\\
57.125	0.2211	60.2114137914037\\
57.125	0.22476	62.9625385595095\\
57.125	0.22842	65.7800532647783\\
57.125	0.23208	68.6639579072099\\
57.125	0.23574	71.6142524868045\\
57.125	0.2394	74.6309370035619\\
57.125	0.24306	77.7140114574822\\
57.125	0.24672	80.8634758485653\\
57.125	0.25038	84.0793301768114\\
57.125	0.25404	87.3615744422202\\
57.125	0.2577	90.710208644792\\
57.125	0.26136	94.1252327845267\\
57.125	0.26502	97.6066468614242\\
57.125	0.26868	101.154450875485\\
57.125	0.27234	104.768644826708\\
57.125	0.276	108.449228715094\\
57.5	0.093	5.75571735656422\\
57.5	0.09666	6.18361147059617\\
57.5	0.10032	6.677895521791\\
57.5	0.10398	7.23856951014869\\
57.5	0.10764	7.86563343566927\\
57.5	0.1113	8.55908729835271\\
57.5	0.11496	9.31893109819903\\
57.5	0.11862	10.1451648352082\\
57.5	0.12228	11.0377885093803\\
57.5	0.12594	11.9968021207152\\
57.5	0.1296	13.022205669213\\
57.5	0.13326	14.1139991548737\\
57.5	0.13692	15.2721825776973\\
57.5	0.14058	16.4967559376837\\
57.5	0.14424	17.7877192348331\\
57.5	0.1479	19.1450724691453\\
57.5	0.15156	20.5688156406203\\
57.5	0.15522	22.0589487492583\\
57.5	0.15888	23.6154717950591\\
57.5	0.16254	25.2383847780228\\
57.5	0.1662	26.9276876981493\\
57.5	0.16986	28.6833805554388\\
57.5	0.17352	30.5054633498911\\
57.5	0.17718	32.3939360815063\\
57.5	0.18084	34.3487987502843\\
57.5	0.1845	36.3700513562252\\
57.5	0.18816	38.4576938993291\\
57.5	0.19182	40.6117263795957\\
57.5	0.19548	42.8321487970253\\
57.5	0.19914	45.1189611516177\\
57.5	0.2028	47.4721634433731\\
57.5	0.20646	49.8917556722913\\
57.5	0.21012	52.3777378383723\\
57.5	0.21378	54.9301099416162\\
57.5	0.21744	57.548871982023\\
57.5	0.2211	60.2340239595927\\
57.5	0.22476	62.9855658743252\\
57.5	0.22842	65.8034977262207\\
57.5	0.23208	68.687819515279\\
57.5	0.23574	71.6385312415002\\
57.5	0.2394	74.6556329048842\\
57.5	0.24306	77.7391245054311\\
57.5	0.24672	80.8890060431409\\
57.5	0.25038	84.1052775180137\\
57.5	0.25404	87.3879389300492\\
57.5	0.2577	90.7369902792476\\
57.5	0.26136	94.1524315656089\\
57.5	0.26502	97.6342627891331\\
57.5	0.26868	101.18248394982\\
57.5	0.27234	104.79709504767\\
57.5	0.276	108.478096082683\\
57.875	0.093	5.76746976200801\\
57.875	0.09666	6.19578102266661\\
57.875	0.10032	6.69048222048808\\
57.875	0.10398	7.25157335547244\\
57.875	0.10764	7.87905442761965\\
57.875	0.1113	8.57292543692976\\
57.875	0.11496	9.33318638340273\\
57.875	0.11862	10.1598372670386\\
57.875	0.12228	11.0528780878373\\
57.875	0.12594	12.0123088457989\\
57.875	0.1296	13.0381295409234\\
57.875	0.13326	14.1303401732107\\
57.875	0.13692	15.2889407426609\\
57.875	0.14058	16.5139312492741\\
57.875	0.14424	17.80531169305\\
57.875	0.1479	19.1630820739889\\
57.875	0.15156	20.5872423920906\\
57.875	0.15522	22.0777926473552\\
57.875	0.15888	23.6347328397826\\
57.875	0.16254	25.258062969373\\
57.875	0.1662	26.9477830361262\\
57.875	0.16986	28.7038930400423\\
57.875	0.17352	30.5263929811213\\
57.875	0.17718	32.4152828593631\\
57.875	0.18084	34.3705626747678\\
57.875	0.1845	36.3922324273354\\
57.875	0.18816	38.4802921170659\\
57.875	0.19182	40.6347417439592\\
57.875	0.19548	42.8555813080154\\
57.875	0.19914	45.1428108092345\\
57.875	0.2028	47.4964302476165\\
57.875	0.20646	49.9164396231613\\
57.875	0.21012	52.402838935869\\
57.875	0.21378	54.9556281857396\\
57.875	0.21744	57.5748073727731\\
57.875	0.2211	60.2603764969694\\
57.875	0.22476	63.0123355583286\\
57.875	0.22842	65.8306845568507\\
57.875	0.23208	68.7154234925356\\
57.875	0.23574	71.6665523653835\\
57.875	0.2394	74.6840711753942\\
57.875	0.24306	77.7679799225678\\
57.875	0.24672	80.9182786069042\\
57.875	0.25038	84.1349672284036\\
57.875	0.25404	87.4180457870658\\
57.875	0.2577	90.7675142828909\\
57.875	0.26136	94.1833727158788\\
57.875	0.26502	97.6656210860296\\
57.875	0.26868	101.214259393343\\
57.875	0.27234	104.82928763782\\
57.875	0.276	108.510705819459\\
58.25	0.093	5.78296453663947\\
58.25	0.09666	6.21169294392473\\
58.25	0.10032	6.70681128837286\\
58.25	0.10398	7.26831956998387\\
58.25	0.10764	7.89621778875775\\
58.25	0.1113	8.59050594469451\\
58.25	0.11496	9.35118403779413\\
58.25	0.11862	10.1782520680566\\
58.25	0.12228	11.071710035482\\
58.25	0.12594	12.0315579400703\\
58.25	0.1296	13.0577957818214\\
58.25	0.13326	14.1504235607354\\
58.25	0.13692	15.3094412768123\\
58.25	0.14058	16.534848930052\\
58.25	0.14424	17.8266465204546\\
58.25	0.1479	19.1848340480201\\
58.25	0.15156	20.6094115127485\\
58.25	0.15522	22.1003789146398\\
58.25	0.15888	23.6577362536939\\
58.25	0.16254	25.2814835299109\\
58.25	0.1662	26.9716207432908\\
58.25	0.16986	28.7281478938335\\
58.25	0.17352	30.5510649815391\\
58.25	0.17718	32.4403720064076\\
58.25	0.18084	34.396068968439\\
58.25	0.1845	36.4181558676332\\
58.25	0.18816	38.5066327039904\\
58.25	0.19182	40.6614994775103\\
58.25	0.19548	42.8827561881932\\
58.25	0.19914	45.170402836039\\
58.25	0.2028	47.5244394210476\\
58.25	0.20646	49.9448659432191\\
58.25	0.21012	52.4316824025534\\
58.25	0.21378	54.9848887990507\\
58.25	0.21744	57.6044851327108\\
58.25	0.2211	60.2904714035338\\
58.25	0.22476	63.0428476115196\\
58.25	0.22842	65.8616137566684\\
58.25	0.23208	68.74676983898\\
58.25	0.23574	71.6983158584545\\
58.25	0.2394	74.7162518150918\\
58.25	0.24306	77.8005777088921\\
58.25	0.24672	80.9512935398552\\
58.25	0.25038	84.1683993079812\\
58.25	0.25404	87.45189501327\\
58.25	0.2577	90.8017806557218\\
58.25	0.26136	94.2180562353364\\
58.25	0.26502	97.7007217521139\\
58.25	0.26868	101.249777206054\\
58.25	0.27234	104.865222597157\\
58.25	0.276	108.547057925424\\
58.625	0.093	5.80220168045861\\
58.625	0.09666	6.23134723437052\\
58.625	0.10032	6.7268827254453\\
58.625	0.10398	7.28880815368297\\
58.625	0.10764	7.9171235190835\\
58.625	0.1113	8.61182882164692\\
58.625	0.11496	9.37292406137319\\
58.625	0.11862	10.2004092382623\\
58.625	0.12228	11.0942843523144\\
58.625	0.12594	12.0545494035293\\
58.625	0.1296	13.0812043919071\\
58.625	0.13326	14.1742493174477\\
58.625	0.13692	15.3336841801513\\
58.625	0.14058	16.5595089800177\\
58.625	0.14424	17.851723717047\\
58.625	0.1479	19.2103283912391\\
58.625	0.15156	20.6353230025941\\
58.625	0.15522	22.126707551112\\
58.625	0.15888	23.6844820367928\\
58.625	0.16254	25.3086464596365\\
58.625	0.1662	26.999200819643\\
58.625	0.16986	28.7561451168124\\
58.625	0.17352	30.5794793511447\\
58.625	0.17718	32.4692035226398\\
58.625	0.18084	34.4253176312978\\
58.625	0.1845	36.4478216771187\\
58.625	0.18816	38.5367156601025\\
58.625	0.19182	40.6919995802492\\
58.625	0.19548	42.9136734375587\\
58.625	0.19914	45.2017372320311\\
58.625	0.2028	47.5561909636664\\
58.625	0.20646	49.9770346324645\\
58.625	0.21012	52.4642682384255\\
58.625	0.21378	55.0178917815494\\
58.625	0.21744	57.6379052618362\\
58.625	0.2211	60.3243086792858\\
58.625	0.22476	63.0771020338983\\
58.625	0.22842	65.8962853256737\\
58.625	0.23208	68.781858554612\\
58.625	0.23574	71.7338217207131\\
58.625	0.2394	74.7521748239772\\
58.625	0.24306	77.8369178644041\\
58.625	0.24672	80.9880508419938\\
58.625	0.25038	84.2055737567465\\
58.625	0.25404	87.489486608662\\
58.625	0.2577	90.8397893977404\\
58.625	0.26136	94.2564821239816\\
58.625	0.26502	97.7395647873858\\
58.625	0.26868	101.289037387953\\
58.625	0.27234	104.904899925683\\
58.625	0.276	108.587152400575\\
59	0.093	5.8251811934654\\
59	0.09666	6.25474389400397\\
59	0.10032	6.7506965317054\\
59	0.10398	7.31303910656973\\
59	0.10764	7.94177161859691\\
59	0.1113	8.63689406778698\\
59	0.11496	9.39840645413992\\
59	0.11862	10.2263087776557\\
59	0.12228	11.1206010383344\\
59	0.12594	12.081283236176\\
59	0.1296	13.1083553711804\\
59	0.13326	14.2018174433477\\
59	0.13692	15.3616694526779\\
59	0.14058	16.587911399171\\
59	0.14424	17.8805432828269\\
59	0.1479	19.2395651036457\\
59	0.15156	20.6649768616274\\
59	0.15522	22.1567785567719\\
59	0.15888	23.7149701890794\\
59	0.16254	25.3395517585497\\
59	0.1662	27.0305232651829\\
59	0.16986	28.7878847089789\\
59	0.17352	30.6116360899379\\
59	0.17718	32.5017774080597\\
59	0.18084	34.4583086633443\\
59	0.1845	36.4812298557919\\
59	0.18816	38.5705409854023\\
59	0.19182	40.7262420521756\\
59	0.19548	42.9483330561118\\
59	0.19914	45.2368139972108\\
59	0.2028	47.5916848754728\\
59	0.20646	50.0129456908976\\
59	0.21012	52.5005964434853\\
59	0.21378	55.0546371332358\\
59	0.21744	57.6750677601492\\
59	0.2211	60.3618883242255\\
59	0.22476	63.1150988254647\\
59	0.22842	65.9346992638667\\
59	0.23208	68.8206896394316\\
59	0.23574	71.7730699521595\\
59	0.2394	74.7918402020501\\
59	0.24306	77.8770003891037\\
59	0.24672	81.0285505133201\\
59	0.25038	84.2464905746994\\
59	0.25404	87.5308205732416\\
59	0.2577	90.8815405089466\\
59	0.26136	94.2986503818145\\
59	0.26502	97.7821501918453\\
59	0.26868	101.332039939039\\
59	0.27234	104.948319623396\\
59	0.276	108.630989244915\\
59.375	0.093	5.85190307565984\\
59.375	0.09666	6.28188292282507\\
59.375	0.10032	6.77825270715317\\
59.375	0.10398	7.34101242864412\\
59.375	0.10764	7.97016208729798\\
59.375	0.1113	8.66570168311469\\
59.375	0.11496	9.42763121609428\\
59.375	0.11862	10.2559506862368\\
59.375	0.12228	11.1506600935421\\
59.375	0.12594	12.1117594380103\\
59.375	0.1296	13.1392487196414\\
59.375	0.13326	14.2331279384354\\
59.375	0.13692	15.3933970943922\\
59.375	0.14058	16.6200561875119\\
59.375	0.14424	17.9131052177945\\
59.375	0.1479	19.27254418524\\
59.375	0.15156	20.6983730898483\\
59.375	0.15522	22.1905919316195\\
59.375	0.15888	23.7492007105536\\
59.375	0.16254	25.3741994266506\\
59.375	0.1662	27.0655880799104\\
59.375	0.16986	28.8233666703331\\
59.375	0.17352	30.6475351979187\\
59.375	0.17718	32.5380936626672\\
59.375	0.18084	34.4950420645785\\
59.375	0.1845	36.5183804036527\\
59.375	0.18816	38.6081086798898\\
59.375	0.19182	40.7642268932897\\
59.375	0.19548	42.9867350438526\\
59.375	0.19914	45.2756331315783\\
59.375	0.2028	47.6309211564669\\
59.375	0.20646	50.0525991185183\\
59.375	0.21012	52.5406670177326\\
59.375	0.21378	55.0951248541098\\
59.375	0.21744	57.7159726276499\\
59.375	0.2211	60.4032103383529\\
59.375	0.22476	63.1568379862187\\
59.375	0.22842	65.9768555712474\\
59.375	0.23208	68.863263093439\\
59.375	0.23574	71.8160605527934\\
59.375	0.2394	74.8352479493108\\
59.375	0.24306	77.920825282991\\
59.375	0.24672	81.072792553834\\
59.375	0.25038	84.29114976184\\
59.375	0.25404	87.5758969070088\\
59.375	0.2577	90.9270339893405\\
59.375	0.26136	94.3445610088351\\
59.375	0.26502	97.8284779654925\\
59.375	0.26868	101.378784859313\\
59.375	0.27234	104.995481690296\\
59.375	0.276	108.678568458442\\
59.75	0.093	5.88236732704198\\
59.75	0.09666	6.31276432083386\\
59.75	0.10032	6.80955125178861\\
59.75	0.10398	7.37272811990622\\
59.75	0.10764	8.00229492518674\\
59.75	0.1113	8.6982516676301\\
59.75	0.11496	9.46059834723635\\
59.75	0.11862	10.2893349640055\\
59.75	0.12228	11.1844615179375\\
59.75	0.12594	12.1459780090323\\
59.75	0.1296	13.1738844372901\\
59.75	0.13326	14.2681808027107\\
59.75	0.13692	15.4288671052942\\
59.75	0.14058	16.6559433450406\\
59.75	0.14424	17.9494095219498\\
59.75	0.1479	19.3092656360219\\
59.75	0.15156	20.7355116872569\\
59.75	0.15522	22.2281476756548\\
59.75	0.15888	23.7871736012155\\
59.75	0.16254	25.4125894639391\\
59.75	0.1662	27.1043952638256\\
59.75	0.16986	28.862591000875\\
59.75	0.17352	30.6871766750873\\
59.75	0.17718	32.5781522864623\\
59.75	0.18084	34.5355178350003\\
59.75	0.1845	36.5592733207012\\
59.75	0.18816	38.6494187435649\\
59.75	0.19182	40.8059541035915\\
59.75	0.19548	43.028879400781\\
59.75	0.19914	45.3181946351334\\
59.75	0.2028	47.6738998066486\\
59.75	0.20646	50.0959949153268\\
59.75	0.21012	52.5844799611677\\
59.75	0.21378	55.1393549441716\\
59.75	0.21744	57.7606198643383\\
59.75	0.2211	60.4482747216679\\
59.75	0.22476	63.2023195161604\\
59.75	0.22842	66.0227542478158\\
59.75	0.23208	68.909578916634\\
59.75	0.23574	71.8627935226151\\
59.75	0.2394	74.8823980657591\\
59.75	0.24306	77.9683925460659\\
59.75	0.24672	81.1207769635357\\
59.75	0.25038	84.3395513181683\\
59.75	0.25404	87.6247156099637\\
59.75	0.2577	90.9762698389221\\
59.75	0.26136	94.3942140050434\\
59.75	0.26502	97.8785481083274\\
59.75	0.26868	101.429272148774\\
59.75	0.27234	105.046386126384\\
59.75	0.276	108.729890041157\\
60.125	0.093	5.91657394761176\\
60.125	0.09666	6.3473880880303\\
60.125	0.10032	6.8445921656117\\
60.125	0.10398	7.40818618035596\\
60.125	0.10764	8.03817013226314\\
60.125	0.1113	8.73454402133316\\
60.125	0.11496	9.49730784756606\\
60.125	0.11862	10.3264616109618\\
60.125	0.12228	11.2220053115205\\
60.125	0.12594	12.183938949242\\
60.125	0.1296	13.2122625241264\\
60.125	0.13326	14.3069760361737\\
60.125	0.13692	15.4680794853838\\
60.125	0.14058	16.6955728717569\\
60.125	0.14424	17.9894561952928\\
60.125	0.1479	19.3497294559915\\
60.125	0.15156	20.7763926538532\\
60.125	0.15522	22.2694457888777\\
60.125	0.15888	23.8288888610651\\
60.125	0.16254	25.4547218704154\\
60.125	0.1662	27.1469448169285\\
60.125	0.16986	28.9055577006045\\
60.125	0.17352	30.7305605214434\\
60.125	0.17718	32.6219532794452\\
60.125	0.18084	34.5797359746098\\
60.125	0.1845	36.6039086069373\\
60.125	0.18816	38.6944711764277\\
60.125	0.19182	40.851423683081\\
60.125	0.19548	43.0747661268971\\
60.125	0.19914	45.3644985078762\\
60.125	0.2028	47.7206208260181\\
60.125	0.20646	50.1431330813228\\
60.125	0.21012	52.6320352737905\\
60.125	0.21378	55.187327403421\\
60.125	0.21744	57.8090094702144\\
60.125	0.2211	60.4970814741706\\
60.125	0.22476	63.2515434152898\\
60.125	0.22842	66.0723952935718\\
60.125	0.23208	68.9596371090166\\
60.125	0.23574	71.9132688616244\\
60.125	0.2394	74.9332905513951\\
60.125	0.24306	78.0197021783286\\
60.125	0.24672	81.1725037424249\\
60.125	0.25038	84.3916952436842\\
60.125	0.25404	87.6772766821063\\
60.125	0.2577	91.0292480576914\\
60.125	0.26136	94.4476093704392\\
60.125	0.26502	97.93236062035\\
60.125	0.26868	101.483501807424\\
60.125	0.27234	105.10103293166\\
60.125	0.276	108.78495399306\\
60.5	0.093	5.95452293736921\\
60.5	0.09666	6.3857542244144\\
60.5	0.10032	6.88337544862247\\
60.5	0.10398	7.44738660999339\\
60.5	0.10764	8.07778770852721\\
60.5	0.1113	8.77457874422389\\
60.5	0.11496	9.53775971708345\\
60.5	0.11862	10.3673306271059\\
60.5	0.12228	11.2632914742912\\
60.5	0.12594	12.2256422586394\\
60.5	0.1296	13.2543829801504\\
60.5	0.13326	14.3495136388244\\
60.5	0.13692	15.5110342346611\\
60.5	0.14058	16.7389447676608\\
60.5	0.14424	18.0332452378234\\
60.5	0.1479	19.3939356451488\\
60.5	0.15156	20.8210159896371\\
60.5	0.15522	22.3144862712883\\
60.5	0.15888	23.8743464901023\\
60.5	0.16254	25.5005966460793\\
60.5	0.1662	27.1932367392191\\
60.5	0.16986	28.9522667695217\\
60.5	0.17352	30.7776867369873\\
60.5	0.17718	32.6694966416157\\
60.5	0.18084	34.627696483407\\
60.5	0.1845	36.6522862623612\\
60.5	0.18816	38.7432659784782\\
60.5	0.19182	40.9006356317581\\
60.5	0.19548	43.1243952222009\\
60.5	0.19914	45.4145447498066\\
60.5	0.2028	47.7710842145752\\
60.5	0.20646	50.1940136165066\\
60.5	0.21012	52.6833329556009\\
60.5	0.21378	55.239042231858\\
60.5	0.21744	57.8611414452781\\
60.5	0.2211	60.549630595861\\
60.5	0.22476	63.3045096836068\\
60.5	0.22842	66.1257787085154\\
60.5	0.23208	69.013437670587\\
60.5	0.23574	71.9674865698214\\
60.5	0.2394	74.9879254062187\\
60.5	0.24306	78.0747541797789\\
60.5	0.24672	81.2279728905019\\
60.5	0.25038	84.4475815383878\\
60.5	0.25404	87.7335801234366\\
60.5	0.2577	91.0859686456483\\
60.5	0.26136	94.5047471050228\\
60.5	0.26502	97.9899155015602\\
60.5	0.26868	101.54147383526\\
60.5	0.27234	105.159422106124\\
60.5	0.276	108.84376031415\\
60.875	0.093	5.99621429631434\\
60.875	0.09666	6.42786272998619\\
60.875	0.10032	6.92590110082091\\
60.875	0.10398	7.49032940881848\\
60.875	0.10764	8.12114765397896\\
60.875	0.1113	8.81835583630229\\
60.875	0.11496	9.58195395578849\\
60.875	0.11862	10.4119420124376\\
60.875	0.12228	11.3083200062495\\
60.875	0.12594	12.2710879372244\\
60.875	0.1296	13.3002458053621\\
60.875	0.13326	14.3957936106627\\
60.875	0.13692	15.5577313531261\\
60.875	0.14058	16.7860590327524\\
60.875	0.14424	18.0807766495417\\
60.875	0.1479	19.4418842034937\\
60.875	0.15156	20.8693816946087\\
60.875	0.15522	22.3632691228865\\
60.875	0.15888	23.9235464883272\\
60.875	0.16254	25.5502137909308\\
60.875	0.1662	27.2432710306973\\
60.875	0.16986	29.0027182076266\\
60.875	0.17352	30.8285553217188\\
60.875	0.17718	32.7207823729739\\
60.875	0.18084	34.6793993613918\\
60.875	0.1845	36.7044062869727\\
60.875	0.18816	38.7958031497164\\
60.875	0.19182	40.9535899496229\\
60.875	0.19548	43.1777666866924\\
60.875	0.19914	45.4683333609247\\
60.875	0.2028	47.8252899723199\\
60.875	0.20646	50.248636520878\\
60.875	0.21012	52.7383730065989\\
60.875	0.21378	55.2944994294828\\
60.875	0.21744	57.9170157895294\\
60.875	0.2211	60.605922086739\\
60.875	0.22476	63.3612183211115\\
60.875	0.22842	66.1829044926468\\
60.875	0.23208	69.070980601345\\
60.875	0.23574	72.0254466472061\\
60.875	0.2394	75.04630263023\\
60.875	0.24306	78.1335485504168\\
60.875	0.24672	81.2871844077665\\
60.875	0.25038	84.5072102022791\\
60.875	0.25404	87.7936259339545\\
60.875	0.2577	91.1464316027929\\
60.875	0.26136	94.565627208794\\
60.875	0.26502	98.0512127519581\\
60.875	0.26868	101.603188232285\\
60.875	0.27234	105.221553649775\\
60.875	0.276	108.906309004428\\
61.25	0.093	6.04164802444712\\
61.25	0.09666	6.47371360474562\\
61.25	0.10032	6.97216912220697\\
61.25	0.10398	7.53701457683123\\
61.25	0.10764	8.16824996861833\\
61.25	0.1113	8.86587529756834\\
61.25	0.11496	9.6298905636812\\
61.25	0.11862	10.4602957669569\\
61.25	0.12228	11.3570909073956\\
61.25	0.12594	12.320275984997\\
61.25	0.1296	13.3498509997614\\
61.25	0.13326	14.4458159516886\\
61.25	0.13692	15.6081708407787\\
61.25	0.14058	16.8369156670317\\
61.25	0.14424	18.1320504304476\\
61.25	0.1479	19.4935751310263\\
61.25	0.15156	20.9214897687679\\
61.25	0.15522	22.4157943436724\\
61.25	0.15888	23.9764888557398\\
61.25	0.16254	25.60357330497\\
61.25	0.1662	27.2970476913631\\
61.25	0.16986	29.0569120149191\\
61.25	0.17352	30.883166275638\\
61.25	0.17718	32.7758104735197\\
61.25	0.18084	34.7348446085643\\
61.25	0.1845	36.7602686807718\\
61.25	0.18816	38.8520826901422\\
61.25	0.19182	41.0102866366754\\
61.25	0.19548	43.2348805203715\\
61.25	0.19914	45.5258643412305\\
61.25	0.2028	47.8832380992523\\
61.25	0.20646	50.3070017944371\\
61.25	0.21012	52.7971554267847\\
61.25	0.21378	55.3536989962951\\
61.25	0.21744	57.9766325029685\\
61.25	0.2211	60.6659559468047\\
61.25	0.22476	63.4216693278038\\
61.25	0.22842	66.2437726459658\\
61.25	0.23208	69.1322659012906\\
61.25	0.23574	72.0871490937784\\
61.25	0.2394	75.108422223429\\
61.25	0.24306	78.1960852902424\\
61.25	0.24672	81.3501382942188\\
61.25	0.25038	84.570581235358\\
61.25	0.25404	87.8574141136601\\
61.25	0.2577	91.2106369291251\\
61.25	0.26136	94.6302496817529\\
61.25	0.26502	98.1162523715436\\
61.25	0.26868	101.668644998497\\
61.25	0.27234	105.287427562614\\
61.25	0.276	108.972600063893\\
61.625	0.093	6.09082412176756\\
61.625	0.09666	6.52330684869272\\
61.625	0.10032	7.02217951278073\\
61.625	0.10398	7.58744211403165\\
61.625	0.10764	8.21909465244541\\
61.625	0.1113	8.91713712802206\\
61.625	0.11496	9.68156954076157\\
61.625	0.11862	10.512391890664\\
61.625	0.12228	11.4096041777292\\
61.625	0.12594	12.3732064019574\\
61.625	0.1296	13.4031985633484\\
61.625	0.13326	14.4995806619023\\
61.625	0.13692	15.662352697619\\
61.625	0.14058	16.8915146704987\\
61.625	0.14424	18.1870665805412\\
61.625	0.1479	19.5490084277466\\
61.625	0.15156	20.9773402121149\\
61.625	0.15522	22.472061933646\\
61.625	0.15888	24.03317359234\\
61.625	0.16254	25.6606751881969\\
61.625	0.1662	27.3545667212167\\
61.625	0.16986	29.1148481913993\\
61.625	0.17352	30.9415195987448\\
61.625	0.17718	32.8345809432532\\
61.625	0.18084	34.7940322249245\\
61.625	0.1845	36.8198734437586\\
61.625	0.18816	38.9121045997556\\
61.625	0.19182	41.0707256929155\\
61.625	0.19548	43.2957367232383\\
61.625	0.19914	45.5871376907239\\
61.625	0.2028	47.9449285953724\\
61.625	0.20646	50.3691094371838\\
61.625	0.21012	52.8596802161581\\
61.625	0.21378	55.4166409322952\\
61.625	0.21744	58.0399915855952\\
61.625	0.2211	60.7297321760581\\
61.625	0.22476	63.4858627036838\\
61.625	0.22842	66.3083831684724\\
61.625	0.23208	69.1972935704239\\
61.625	0.23574	72.1525939095383\\
61.625	0.2394	75.1742841858156\\
61.625	0.24306	78.2623643992557\\
61.625	0.24672	81.4168345498587\\
61.625	0.25038	84.6376946376246\\
61.625	0.25404	87.9249446625533\\
61.625	0.2577	91.278584624645\\
61.625	0.26136	94.6986145238995\\
61.625	0.26502	98.1850343603169\\
61.625	0.26868	101.737844133897\\
61.625	0.27234	105.35704384464\\
61.625	0.276	109.042633492546\\
62	0.093	6.14374258827568\\
62	0.09666	6.57664246182749\\
62	0.10032	7.07593227254216\\
62	0.10398	7.64161202041971\\
62	0.10764	8.27368170546014\\
62	0.1113	8.97214132766344\\
62	0.11496	9.73699088702962\\
62	0.11862	10.5682303835586\\
62	0.12228	11.4658598172506\\
62	0.12594	12.4298791881054\\
62	0.1296	13.4602884961231\\
62	0.13326	14.5570877413036\\
62	0.13692	15.720276923647\\
62	0.14058	16.9498560431533\\
62	0.14424	18.2458250998225\\
62	0.1479	19.6081840936545\\
62	0.15156	21.0369330246495\\
62	0.15522	22.5320718928073\\
62	0.15888	24.0936006981279\\
62	0.16254	25.7215194406115\\
62	0.1662	27.4158281202579\\
62	0.16986	29.1765267370672\\
62	0.17352	31.0036152910394\\
62	0.17718	32.8970937821744\\
62	0.18084	34.8569622104723\\
62	0.1845	36.8832205759331\\
62	0.18816	38.9758688785568\\
62	0.19182	41.1349071183433\\
62	0.19548	43.3603352952927\\
62	0.19914	45.652153409405\\
62	0.2028	48.0103614606802\\
62	0.20646	50.4349594491182\\
62	0.21012	52.9259473747191\\
62	0.21378	55.4833252374829\\
62	0.21744	58.1070930374096\\
62	0.2211	60.7972507744991\\
62	0.22476	63.5537984487515\\
62	0.22842	66.3767360601668\\
62	0.23208	69.2660636087449\\
62	0.23574	72.221781094486\\
62	0.2394	75.2438885173899\\
62	0.24306	78.3323858774567\\
62	0.24672	81.4872731746863\\
62	0.25038	84.7085504090789\\
62	0.25404	87.9962175806343\\
62	0.2577	91.3502746893526\\
62	0.26136	94.7707217352337\\
62	0.26502	98.2575587182777\\
62	0.26868	101.810785638485\\
62	0.27234	105.430402495854\\
62	0.276	109.116409290387\\
62.375	0.093	6.20040342397147\\
62.375	0.09666	6.63372044414993\\
62.375	0.10032	7.13342740149125\\
62.375	0.10398	7.69952429599547\\
62.375	0.10764	8.33201112766254\\
62.375	0.1113	9.03088789649249\\
62.375	0.11496	9.79615460248532\\
62.375	0.11862	10.627811245641\\
62.375	0.12228	11.5258578259596\\
62.375	0.12594	12.4902943434411\\
62.375	0.1296	13.5211207980854\\
62.375	0.13326	14.6183371898926\\
62.375	0.13692	15.7819435188627\\
62.375	0.14058	17.0119397849956\\
62.375	0.14424	18.3083259882914\\
62.375	0.1479	19.6711021287501\\
62.375	0.15156	21.1002682063717\\
62.375	0.15522	22.5958242211562\\
62.375	0.15888	24.1577701731035\\
62.375	0.16254	25.7861060622137\\
62.375	0.1662	27.4808318884868\\
62.375	0.16986	29.2419476519227\\
62.375	0.17352	31.0694533525215\\
62.375	0.17718	32.9633489902832\\
62.375	0.18084	34.9236345652078\\
62.375	0.1845	36.9503100772952\\
62.375	0.18816	39.0433755265456\\
62.375	0.19182	41.2028309129588\\
62.375	0.19548	43.4286762365348\\
62.375	0.19914	45.7209114972738\\
62.375	0.2028	48.0795366951756\\
62.375	0.20646	50.5045518302403\\
62.375	0.21012	52.9959569024678\\
62.375	0.21378	55.5537519118583\\
62.375	0.21744	58.1779368584116\\
62.375	0.2211	60.8685117421278\\
62.375	0.22476	63.6254765630068\\
62.375	0.22842	66.4488313210488\\
62.375	0.23208	69.3385760162536\\
62.375	0.23574	72.2947106486213\\
62.375	0.2394	75.3172352181519\\
62.375	0.24306	78.4061497248453\\
62.375	0.24672	81.5614541687016\\
62.375	0.25038	84.7831485497208\\
62.375	0.25404	88.0712328679028\\
62.375	0.2577	91.4257071232478\\
62.375	0.26136	94.8465713157556\\
62.375	0.26502	98.3338254454263\\
62.375	0.26868	101.88746951226\\
62.375	0.27234	105.507503516256\\
62.375	0.276	109.193927457416\\
62.75	0.093	6.26080662885489\\
62.75	0.09666	6.69454079566001\\
62.75	0.10032	7.19466489962801\\
62.75	0.10398	7.76117894075885\\
62.75	0.10764	8.3940829190526\\
62.75	0.1113	9.0933768345092\\
62.75	0.11496	9.85906068712869\\
62.75	0.11862	10.6911344769111\\
62.75	0.12228	11.5895982038563\\
62.75	0.12594	12.5544518679644\\
62.75	0.1296	13.5856954692354\\
62.75	0.13326	14.6833290076692\\
62.75	0.13692	15.8473524832659\\
62.75	0.14058	17.0777658960256\\
62.75	0.14424	18.374569245948\\
62.75	0.1479	19.7377625330334\\
62.75	0.15156	21.1673457572816\\
62.75	0.15522	22.6633189186927\\
62.75	0.15888	24.2256820172667\\
62.75	0.16254	25.8544350530036\\
62.75	0.1662	27.5495780259033\\
62.75	0.16986	29.3111109359659\\
62.75	0.17352	31.1390337831914\\
62.75	0.17718	33.0333465675797\\
62.75	0.18084	34.9940492891309\\
62.75	0.1845	37.021141947845\\
62.75	0.18816	39.114624543722\\
62.75	0.19182	41.2744970767618\\
62.75	0.19548	43.5007595469646\\
62.75	0.19914	45.7934119543302\\
62.75	0.2028	48.1524542988587\\
62.75	0.20646	50.57788658055\\
62.75	0.21012	53.0697087994042\\
62.75	0.21378	55.6279209554213\\
62.75	0.21744	58.2525230486013\\
62.75	0.2211	60.9435150789441\\
62.75	0.22476	63.7008970464498\\
62.75	0.22842	66.5246689511184\\
62.75	0.23208	69.4148307929499\\
62.75	0.23574	72.3713825719443\\
62.75	0.2394	75.3943242881015\\
62.75	0.24306	78.4836559414216\\
62.75	0.24672	81.6393775319045\\
62.75	0.25038	84.8614890595504\\
62.75	0.25404	88.1499905243591\\
62.75	0.2577	91.5048819263307\\
62.75	0.26136	94.9261632654652\\
62.75	0.26502	98.4138345417625\\
62.75	0.26868	101.967895755223\\
62.75	0.27234	105.588346905846\\
62.75	0.276	109.275187993632\\
63.125	0.093	6.324952202926\\
63.125	0.09666	6.75910351635778\\
63.125	0.10032	7.25964476695242\\
63.125	0.10398	7.82657595470992\\
63.125	0.10764	8.45989707963033\\
63.125	0.1113	9.15960814171358\\
63.125	0.11496	9.92570914095972\\
63.125	0.11862	10.7582000773687\\
63.125	0.12228	11.6570809509406\\
63.125	0.12594	12.6223517616754\\
63.125	0.1296	13.654012509573\\
63.125	0.13326	14.7520631946335\\
63.125	0.13692	15.9165038168569\\
63.125	0.14058	17.1473343762432\\
63.125	0.14424	18.4445548727923\\
63.125	0.1479	19.8081653065043\\
63.125	0.15156	21.2381656773792\\
63.125	0.15522	22.734555985417\\
63.125	0.15888	24.2973362306176\\
63.125	0.16254	25.9265064129811\\
63.125	0.1662	27.6220665325075\\
63.125	0.16986	29.3840165891967\\
63.125	0.17352	31.2123565830489\\
63.125	0.17718	33.1070865140639\\
63.125	0.18084	35.0682063822417\\
63.125	0.1845	37.0957161875825\\
63.125	0.18816	39.1896159300861\\
63.125	0.19182	41.3499056097526\\
63.125	0.19548	43.576585226582\\
63.125	0.19914	45.8696547805743\\
63.125	0.2028	48.2291142717294\\
63.125	0.20646	50.6549637000474\\
63.125	0.21012	53.1472030655283\\
63.125	0.21378	55.705832368172\\
63.125	0.21744	58.3308516079787\\
63.125	0.2211	61.0222607849482\\
63.125	0.22476	63.7800598990805\\
63.125	0.22842	66.6042489503758\\
63.125	0.23208	69.4948279388339\\
63.125	0.23574	72.4517968644549\\
63.125	0.2394	75.4751557272388\\
63.125	0.24306	78.5649045271855\\
63.125	0.24672	81.7210432642951\\
63.125	0.25038	84.9435719385677\\
63.125	0.25404	88.232490550003\\
63.125	0.2577	91.5877990986013\\
63.125	0.26136	95.0094975843624\\
63.125	0.26502	98.4975860072864\\
63.125	0.26868	102.052064367373\\
63.125	0.27234	105.672932664623\\
63.125	0.276	109.360190899036\\
63.5	0.093	6.39284014618478\\
63.5	0.09666	6.82740860624321\\
63.5	0.10032	7.32836700346451\\
63.5	0.10398	7.89571533784866\\
63.5	0.10764	8.52945360939573\\
63.5	0.1113	9.22958181810564\\
63.5	0.11496	9.99609996397843\\
63.5	0.11862	10.8290080470141\\
63.5	0.12228	11.7283060672126\\
63.5	0.12594	12.6939940245741\\
63.5	0.1296	13.7260719190983\\
63.5	0.13326	14.8245397507855\\
63.5	0.13692	15.9893975196355\\
63.5	0.14058	17.2206452256485\\
63.5	0.14424	18.5182828688243\\
63.5	0.1479	19.8823104491629\\
63.5	0.15156	21.3127279666645\\
63.5	0.15522	22.8095354213289\\
63.5	0.15888	24.3727328131562\\
63.5	0.16254	26.0023201421463\\
63.5	0.1662	27.6982974082994\\
63.5	0.16986	29.4606646116153\\
63.5	0.17352	31.2894217520941\\
63.5	0.17718	33.1845688297357\\
63.5	0.18084	35.1461058445402\\
63.5	0.1845	37.1740327965077\\
63.5	0.18816	39.2683496856379\\
63.5	0.19182	41.4290565119311\\
63.5	0.19548	43.6561532753871\\
63.5	0.19914	45.949639976006\\
63.5	0.2028	48.3095166137878\\
63.5	0.20646	50.7357831887325\\
63.5	0.21012	53.22843970084\\
63.5	0.21378	55.7874861501104\\
63.5	0.21744	58.4129225365437\\
63.5	0.2211	61.1047488601398\\
63.5	0.22476	63.8629651208989\\
63.5	0.22842	66.6875713188207\\
63.5	0.23208	69.5785674539055\\
63.5	0.23574	72.5359535261532\\
63.5	0.2394	75.5597295355637\\
63.5	0.24306	78.6498954821371\\
63.5	0.24672	81.8064513658734\\
63.5	0.25038	85.0293971867726\\
63.5	0.25404	88.3187329448346\\
63.5	0.2577	91.6744586400595\\
63.5	0.26136	95.0965742724473\\
63.5	0.26502	98.5850798419979\\
63.5	0.26868	102.139975348711\\
63.5	0.27234	105.761260792588\\
63.5	0.276	109.448936173627\\
63.875	0.093	6.46447045863123\\
63.875	0.09666	6.8994560653163\\
63.875	0.10032	7.40083160916426\\
63.875	0.10398	7.96859709017506\\
63.875	0.10764	8.60275250834878\\
63.875	0.1113	9.30329786368535\\
63.875	0.11496	10.0702331561848\\
63.875	0.11862	10.9035583858471\\
63.875	0.12228	11.8032735526723\\
63.875	0.12594	12.7693786566604\\
63.875	0.1296	13.8018736978113\\
63.875	0.13326	14.9007586761252\\
63.875	0.13692	16.0660335916018\\
63.875	0.14058	17.2976984442414\\
63.875	0.14424	18.5957532340439\\
63.875	0.1479	19.9601979610092\\
63.875	0.15156	21.3910326251374\\
63.875	0.15522	22.8882572264285\\
63.875	0.15888	24.4518717648824\\
63.875	0.16254	26.0818762404992\\
63.875	0.1662	27.7782706532789\\
63.875	0.16986	29.5410550032215\\
63.875	0.17352	31.3702292903269\\
63.875	0.17718	33.2657935145952\\
63.875	0.18084	35.2277476760264\\
63.875	0.1845	37.2560917746205\\
63.875	0.18816	39.3508258103774\\
63.875	0.19182	41.5119497832972\\
63.875	0.19548	43.7394636933799\\
63.875	0.19914	46.0333675406255\\
63.875	0.2028	48.3936613250339\\
63.875	0.20646	50.8203450466052\\
63.875	0.21012	53.3134187053394\\
63.875	0.21378	55.8728823012365\\
63.875	0.21744	58.4987358342964\\
63.875	0.2211	61.1909793045192\\
63.875	0.22476	63.9496127119049\\
63.875	0.22842	66.7746360564534\\
63.875	0.23208	69.6660493381648\\
63.875	0.23574	72.6238525570392\\
63.875	0.2394	75.6480457130764\\
63.875	0.24306	78.7386288062764\\
63.875	0.24672	81.8956018366393\\
63.875	0.25038	85.1189648041652\\
63.875	0.25404	88.4087177088538\\
63.875	0.2577	91.7648605507054\\
63.875	0.26136	95.1873933297198\\
63.875	0.26502	98.6763160458971\\
63.875	0.26868	102.231628699237\\
63.875	0.27234	105.85333128974\\
63.875	0.276	109.541423817406\\
64.25	0.093	6.53984314026532\\
64.25	0.09666	6.97524589357706\\
64.25	0.10032	7.47703858405167\\
64.25	0.10398	8.04522121168913\\
64.25	0.10764	8.6797937764895\\
64.25	0.1113	9.38075627845273\\
64.25	0.11496	10.1481087175788\\
64.25	0.11862	10.9818510938678\\
64.25	0.12228	11.8819834073196\\
64.25	0.12594	12.8485056579344\\
64.25	0.1296	13.881417845712\\
64.25	0.13326	14.9807199706525\\
64.25	0.13692	16.1464120327558\\
64.25	0.14058	17.378494032022\\
64.25	0.14424	18.6769659684511\\
64.25	0.1479	20.0418278420431\\
64.25	0.15156	21.4730796527979\\
64.25	0.15522	22.9707214007157\\
64.25	0.15888	24.5347530857963\\
64.25	0.16254	26.1651747080397\\
64.25	0.1662	27.8619862674461\\
64.25	0.16986	29.6251877640153\\
64.25	0.17352	31.4547791977474\\
64.25	0.17718	33.3507605686424\\
64.25	0.18084	35.3131318767002\\
64.25	0.1845	37.3418931219209\\
64.25	0.18816	39.4370443043045\\
64.25	0.19182	41.598585423851\\
64.25	0.19548	43.8265164805603\\
64.25	0.19914	46.1208374744325\\
64.25	0.2028	48.4815484054677\\
64.25	0.20646	50.9086492736656\\
64.25	0.21012	53.4021400790264\\
64.25	0.21378	55.9620208215501\\
64.25	0.21744	58.5882915012367\\
64.25	0.2211	61.2809521180862\\
64.25	0.22476	64.0400026720985\\
64.25	0.22842	66.8654431632737\\
64.25	0.23208	69.7572735916118\\
64.25	0.23574	72.7154939571128\\
64.25	0.2394	75.7401042597766\\
64.25	0.24306	78.8311044996034\\
64.25	0.24672	81.9884946765929\\
64.25	0.25038	85.2122747907454\\
64.25	0.25404	88.5024448420607\\
64.25	0.2577	91.859004830539\\
64.25	0.26136	95.2819547561801\\
64.25	0.26502	98.771294618984\\
64.25	0.26868	102.327024418951\\
64.25	0.27234	105.949144156081\\
64.25	0.276	109.637653830373\\
64.625	0.093	6.61895819108708\\
64.625	0.09666	7.05477809102547\\
64.625	0.10032	7.55698792812673\\
64.625	0.10398	8.12558770239087\\
64.625	0.10764	8.76057741381787\\
64.625	0.1113	9.46195706240776\\
64.625	0.11496	10.2297266481605\\
64.625	0.11862	11.0638861710761\\
64.625	0.12228	11.9644356311547\\
64.625	0.12594	12.931375028396\\
64.625	0.1296	13.9647043628003\\
64.625	0.13326	15.0644236343674\\
64.625	0.13692	16.2305328430974\\
64.625	0.14058	17.4630319889903\\
64.625	0.14424	18.7619210720461\\
64.625	0.1479	20.1272000922647\\
64.625	0.15156	21.5588690496462\\
64.625	0.15522	23.0569279441906\\
64.625	0.15888	24.6213767758978\\
64.625	0.16254	26.2522155447679\\
64.625	0.1662	27.9494442508009\\
64.625	0.16986	29.7130628939968\\
64.625	0.17352	31.5430714743556\\
64.625	0.17718	33.4394699918772\\
64.625	0.18084	35.4022584465617\\
64.625	0.1845	37.4314368384091\\
64.625	0.18816	39.5270051674193\\
64.625	0.19182	41.6889634335924\\
64.625	0.19548	43.9173116369284\\
64.625	0.19914	46.2120497774273\\
64.625	0.2028	48.573177855089\\
64.625	0.20646	51.0006958699137\\
64.625	0.21012	53.4946038219012\\
64.625	0.21378	56.0549017110515\\
64.625	0.21744	58.6815895373648\\
64.625	0.2211	61.3746673008409\\
64.625	0.22476	64.1341350014799\\
64.625	0.22842	66.9599926392817\\
64.625	0.23208	69.8522402142465\\
64.625	0.23574	72.8108777263741\\
64.625	0.2394	75.8359051756646\\
64.625	0.24306	78.927322562118\\
64.625	0.24672	82.0851298857342\\
64.625	0.25038	85.3093271465133\\
64.625	0.25404	88.5999143444553\\
64.625	0.2577	91.9568914795602\\
64.625	0.26136	95.3802585518279\\
64.625	0.26502	98.8700155612585\\
64.625	0.26868	102.426162507852\\
64.625	0.27234	106.048699391608\\
64.625	0.276	109.737626212528\\
65	0.093	6.70181561109652\\
65	0.09666	7.13805265766156\\
65	0.10032	7.64067964138947\\
65	0.10398	8.20969656228026\\
65	0.10764	8.84510342033392\\
65	0.1113	9.54690021555046\\
65	0.11496	10.3150869479299\\
65	0.11862	11.1496636174721\\
65	0.12228	12.0506302241773\\
65	0.12594	13.0179867680454\\
65	0.1296	14.0517332490763\\
65	0.13326	15.15186966727\\
65	0.13692	16.3183960226267\\
65	0.14058	17.5513123151463\\
65	0.14424	18.8506185448286\\
65	0.1479	20.2163147116739\\
65	0.15156	21.6484008156821\\
65	0.15522	23.1468768568531\\
65	0.15888	24.711742835187\\
65	0.16254	26.3429987506838\\
65	0.1662	28.0406446033435\\
65	0.16986	29.804680393166\\
65	0.17352	31.6351061201514\\
65	0.17718	33.5319217842997\\
65	0.18084	35.4951273856108\\
65	0.1845	37.5247229240848\\
65	0.18816	39.6207083997218\\
65	0.19182	41.7830838125215\\
65	0.19548	44.0118491624842\\
65	0.19914	46.3070044496097\\
65	0.2028	48.6685496738981\\
65	0.20646	51.0964848353494\\
65	0.21012	53.5908099339635\\
65	0.21378	56.1515249697406\\
65	0.21744	58.7786299426804\\
65	0.2211	61.4721248527832\\
65	0.22476	64.2320097000488\\
65	0.22842	67.0582844844774\\
65	0.23208	69.9509492060687\\
65	0.23574	72.9100038648231\\
65	0.2394	75.9354484607402\\
65	0.24306	79.0272829938203\\
65	0.24672	82.1855074640631\\
65	0.25038	85.4101218714689\\
65	0.25404	88.7011262160375\\
65	0.2577	92.0585204977691\\
65	0.26136	95.4823047166635\\
65	0.26502	98.9724788727207\\
65	0.26868	102.529042965941\\
65	0.27234	106.151996996324\\
65	0.276	109.84134096387\\
65.375	0.093	6.78841540029362\\
65.375	0.09666	7.22506959348531\\
65.375	0.10032	7.72811372383989\\
65.375	0.10398	8.29754779135734\\
65.375	0.10764	8.93337179603764\\
65.375	0.1113	9.63558573788084\\
65.375	0.11496	10.4041896168869\\
65.375	0.11862	11.2391834330558\\
65.375	0.12228	12.1405671863877\\
65.375	0.12594	13.1083408768824\\
65.375	0.1296	14.1425045045399\\
65.375	0.13326	15.2430580693603\\
65.375	0.13692	16.4100015713437\\
65.375	0.14058	17.6433350104899\\
65.375	0.14424	18.9430583867989\\
65.375	0.1479	20.3091717002709\\
65.375	0.15156	21.7416749509057\\
65.375	0.15522	23.2405681387034\\
65.375	0.15888	24.8058512636639\\
65.375	0.16254	26.4375243257874\\
65.375	0.1662	28.1355873250737\\
65.375	0.16986	29.9000402615229\\
65.375	0.17352	31.7308831351349\\
65.375	0.17718	33.6281159459098\\
65.375	0.18084	35.5917386938476\\
65.375	0.1845	37.6217513789483\\
65.375	0.18816	39.7181540012119\\
65.375	0.19182	41.8809465606383\\
65.375	0.19548	44.1101290572276\\
65.375	0.19914	46.4057014909798\\
65.375	0.2028	48.7676638618949\\
65.375	0.20646	51.1960161699728\\
65.375	0.21012	53.6907584152136\\
65.375	0.21378	56.2518905976173\\
65.375	0.21744	58.8794127171838\\
65.375	0.2211	61.5733247739132\\
65.375	0.22476	64.3336267678055\\
65.375	0.22842	67.1603186988607\\
65.375	0.23208	70.0534005670787\\
65.375	0.23574	73.0128723724597\\
65.375	0.2394	76.0387341150035\\
65.375	0.24306	79.1309857947102\\
65.375	0.24672	82.2896274115797\\
65.375	0.25038	85.5146589656122\\
65.375	0.25404	88.8060804568075\\
65.375	0.2577	92.1638918851656\\
65.375	0.26136	95.5880932506867\\
65.375	0.26502	99.0786845533706\\
65.375	0.26868	102.635665793217\\
65.375	0.27234	106.259036970227\\
65.375	0.276	109.9487980844\\
65.75	0.093	6.87875755867838\\
65.75	0.09666	7.31582889849674\\
65.75	0.10032	7.81929017547796\\
65.75	0.10398	8.38914138962205\\
65.75	0.10764	9.02538254092902\\
65.75	0.1113	9.72801362939887\\
65.75	0.11496	10.4970346550316\\
65.75	0.11862	11.3324456178272\\
65.75	0.12228	12.2342465177857\\
65.75	0.12594	13.202437354907\\
65.75	0.1296	14.2370181291912\\
65.75	0.13326	15.3379888406383\\
65.75	0.13692	16.5053494892483\\
65.75	0.14058	17.7391000750211\\
65.75	0.14424	19.0392405979568\\
65.75	0.1479	20.4057710580554\\
65.75	0.15156	21.8386914553169\\
65.75	0.15522	23.3380017897412\\
65.75	0.15888	24.9037020613285\\
65.75	0.16254	26.5357922700786\\
65.75	0.1662	28.2342724159915\\
65.75	0.16986	29.9991424990674\\
65.75	0.17352	31.8304025193061\\
65.75	0.17718	33.7280524767076\\
65.75	0.18084	35.6920923712721\\
65.75	0.1845	37.7225222029994\\
65.75	0.18816	39.8193419718897\\
65.75	0.19182	41.9825516779427\\
65.75	0.19548	44.2121513211587\\
65.75	0.19914	46.5081409015375\\
65.75	0.2028	48.8705204190793\\
65.75	0.20646	51.2992898737838\\
65.75	0.21012	53.7944492656513\\
65.75	0.21378	56.3559985946816\\
65.75	0.21744	58.9839378608748\\
65.75	0.2211	61.6782670642309\\
65.75	0.22476	64.4389862047499\\
65.75	0.22842	67.2660952824317\\
65.75	0.23208	70.1595942972764\\
65.75	0.23574	73.119483249284\\
65.75	0.2394	76.1457621384544\\
65.75	0.24306	79.2384309647878\\
65.75	0.24672	82.397489728284\\
65.75	0.25038	85.6229384289431\\
65.75	0.25404	88.914777066765\\
65.75	0.2577	92.2730056417498\\
65.75	0.26136	95.6976241538976\\
65.75	0.26502	99.1886326032081\\
65.75	0.26868	102.746030989682\\
65.75	0.27234	106.369819313318\\
65.75	0.276	110.059997574117\\
66.125	0.093	6.9728420862508\\
66.125	0.09666	7.4103305726958\\
66.125	0.10032	7.9142089963037\\
66.125	0.10398	8.48447735707443\\
66.125	0.10764	9.12113565500808\\
66.125	0.1113	9.82418389010456\\
66.125	0.11496	10.5936220623639\\
66.125	0.11862	11.4294501717862\\
66.125	0.12228	12.3316682183713\\
66.125	0.12594	13.3002762021193\\
66.125	0.1296	14.3352741230302\\
66.125	0.13326	15.4366619811039\\
66.125	0.13692	16.6044397763405\\
66.125	0.14058	17.8386075087401\\
66.125	0.14424	19.1391651783024\\
66.125	0.1479	20.5061127850277\\
66.125	0.15156	21.9394503289158\\
66.125	0.15522	23.4391778099668\\
66.125	0.15888	25.0052952281807\\
66.125	0.16254	26.6378025835574\\
66.125	0.1662	28.336699876097\\
66.125	0.16986	30.1019871057995\\
66.125	0.17352	31.9336642726649\\
66.125	0.17718	33.8317313766931\\
66.125	0.18084	35.7961884178842\\
66.125	0.1845	37.8270353962382\\
66.125	0.18816	39.9242723117551\\
66.125	0.19182	42.0878991644348\\
66.125	0.19548	44.3179159542774\\
66.125	0.19914	46.6143226812829\\
66.125	0.2028	48.9771193454513\\
66.125	0.20646	51.4063059467826\\
66.125	0.21012	53.9018824852767\\
66.125	0.21378	56.4638489609336\\
66.125	0.21744	59.0922053737535\\
66.125	0.2211	61.7869517237362\\
66.125	0.22476	64.5480880108818\\
66.125	0.22842	67.3756142351903\\
66.125	0.23208	70.2695303966617\\
66.125	0.23574	73.2298364952959\\
66.125	0.2394	76.2565325310931\\
66.125	0.24306	79.349618504053\\
66.125	0.24672	82.5090944141758\\
66.125	0.25038	85.7349602614616\\
66.125	0.25404	89.0272160459102\\
66.125	0.2577	92.3858617675217\\
66.125	0.26136	95.8108974262961\\
66.125	0.26502	99.3023230222333\\
66.125	0.26868	102.860138555333\\
66.125	0.27234	106.484344025596\\
66.125	0.276	110.174939433022\\
66.5	0.093	7.07066898301089\\
66.5	0.09666	7.50857461608255\\
66.5	0.10032	8.01287018631711\\
66.5	0.10398	8.58355569371449\\
66.5	0.10764	9.22063113827479\\
66.5	0.1113	9.92409651999794\\
66.5	0.11496	10.693951838884\\
66.5	0.11862	11.5301970949329\\
66.5	0.12228	12.4328322881446\\
66.5	0.12594	13.4018574185193\\
66.5	0.1296	14.4372724860568\\
66.5	0.13326	15.5390774907572\\
66.5	0.13692	16.7072724326205\\
66.5	0.14058	17.9418573116467\\
66.5	0.14424	19.2428321278357\\
66.5	0.1479	20.6101968811876\\
66.5	0.15156	22.0439515717024\\
66.5	0.15522	23.54409619938\\
66.5	0.15888	25.1106307642206\\
66.5	0.16254	26.7435552662239\\
66.5	0.1662	28.4428697053902\\
66.5	0.16986	30.2085740817194\\
66.5	0.17352	32.0406683952114\\
66.5	0.17718	33.9391526458663\\
66.5	0.18084	35.904026833684\\
66.5	0.1845	37.9352909586647\\
66.5	0.18816	40.0329450208082\\
66.5	0.19182	42.1969890201146\\
66.5	0.19548	44.4274229565839\\
66.5	0.19914	46.724246830216\\
66.5	0.2028	49.0874606410111\\
66.5	0.20646	51.517064388969\\
66.5	0.21012	54.0130580740897\\
66.5	0.21378	56.5754416963733\\
66.5	0.21744	59.2042152558199\\
66.5	0.2211	61.8993787524293\\
66.5	0.22476	64.6609321862015\\
66.5	0.22842	67.4888755571366\\
66.5	0.23208	70.3832088652346\\
66.5	0.23574	73.3439321104956\\
66.5	0.2394	76.3710452929193\\
66.5	0.24306	79.464548412506\\
66.5	0.24672	82.6244414692555\\
66.5	0.25038	85.8507244631679\\
66.5	0.25404	89.1433973942431\\
66.5	0.2577	92.5024602624813\\
66.5	0.26136	95.9279130678823\\
66.5	0.26502	99.4197558104462\\
66.5	0.26868	102.977988490173\\
66.5	0.27234	106.602611107063\\
66.5	0.276	110.293623661115\\
66.875	0.093	7.17223824895865\\
66.875	0.09666	7.61056102865697\\
66.875	0.10032	8.11527374551817\\
66.875	0.10398	8.6863763995422\\
66.875	0.10764	9.32386899072916\\
66.875	0.1113	10.027751519079\\
66.875	0.11496	10.7980239845916\\
66.875	0.11862	11.6346863872672\\
66.875	0.12228	12.5377387271056\\
66.875	0.12594	13.507181004107\\
66.875	0.1296	14.5430132182711\\
66.875	0.13326	15.6452353695982\\
66.875	0.13692	16.8138474580881\\
66.875	0.14058	18.0488494837409\\
66.875	0.14424	19.3502414465566\\
66.875	0.1479	20.7180233465352\\
66.875	0.15156	22.1521951836766\\
66.875	0.15522	23.6527569579809\\
66.875	0.15888	25.2197086694481\\
66.875	0.16254	26.8530503180781\\
66.875	0.1662	28.5527819038711\\
66.875	0.16986	30.3189034268269\\
66.875	0.17352	32.1514148869456\\
66.875	0.17718	34.0503162842271\\
66.875	0.18084	36.0156076186715\\
66.875	0.1845	38.0472888902788\\
66.875	0.18816	40.145360099049\\
66.875	0.19182	42.309821244982\\
66.875	0.19548	44.540672328078\\
66.875	0.19914	46.8379133483368\\
66.875	0.2028	49.2015443057585\\
66.875	0.20646	51.631565200343\\
66.875	0.21012	54.1279760320904\\
66.875	0.21378	56.6907768010007\\
66.875	0.21744	59.3199675070739\\
66.875	0.2211	62.0155481503099\\
66.875	0.22476	64.7775187307088\\
66.875	0.22842	67.6058792482706\\
66.875	0.23208	70.5006297029953\\
66.875	0.23574	73.4617700948828\\
66.875	0.2394	76.4893004239333\\
66.875	0.24306	79.5832206901466\\
66.875	0.24672	82.7435308935227\\
66.875	0.25038	85.9702310340618\\
66.875	0.25404	89.2633211117637\\
66.875	0.2577	92.6228011266285\\
66.875	0.26136	96.0486710786562\\
66.875	0.26502	99.5409309678467\\
66.875	0.26868	103.0995807942\\
66.875	0.27234	106.724620557716\\
66.875	0.276	110.416050258396\\
67.25	0.093	7.27754988409409\\
67.25	0.09666	7.71628981041905\\
67.25	0.10032	8.22141967390691\\
67.25	0.10398	8.79293947455761\\
67.25	0.10764	9.43084921237121\\
67.25	0.1113	10.1351488873477\\
67.25	0.11496	10.905838499487\\
67.25	0.11862	11.7429180487892\\
67.25	0.12228	12.6463875352543\\
67.25	0.12594	13.6162469588823\\
67.25	0.1296	14.6524963196731\\
67.25	0.13326	15.7551356176268\\
67.25	0.13692	16.9241648527434\\
67.25	0.14058	18.1595840250229\\
67.25	0.14424	19.4613931344652\\
67.25	0.1479	20.8295921810704\\
67.25	0.15156	22.2641811648385\\
67.25	0.15522	23.7651600857695\\
67.25	0.15888	25.3325289438633\\
67.25	0.16254	26.96628773912\\
67.25	0.1662	28.6664364715396\\
67.25	0.16986	30.432975141122\\
67.25	0.17352	32.2659037478674\\
67.25	0.17718	34.1652222917756\\
67.25	0.18084	36.1309307728466\\
67.25	0.1845	38.1630291910806\\
67.25	0.18816	40.2615175464774\\
67.25	0.19182	42.4263958390371\\
67.25	0.19548	44.6576640687597\\
67.25	0.19914	46.9553222356452\\
67.25	0.2028	49.3193703396935\\
67.25	0.20646	51.7498083809047\\
67.25	0.21012	54.2466363592788\\
67.25	0.21378	56.8098542748157\\
67.25	0.21744	59.4394621275156\\
67.25	0.2211	62.1354599173783\\
67.25	0.22476	64.8978476444038\\
67.25	0.22842	67.7266253085923\\
67.25	0.23208	70.6217929099436\\
67.25	0.23574	73.5833504484578\\
67.25	0.2394	76.6112979241349\\
67.25	0.24306	79.7056353369748\\
67.25	0.24672	82.8663626869777\\
67.25	0.25038	86.0934799741434\\
67.25	0.25404	89.3869871984719\\
67.25	0.2577	92.7468843599634\\
67.25	0.26136	96.1731714586177\\
67.25	0.26502	99.6658484944349\\
67.25	0.26868	103.224915467415\\
67.25	0.27234	106.850372377558\\
67.25	0.276	110.542219224864\\
67.625	0.093	7.38660388841716\\
67.625	0.09666	7.82576096136878\\
67.625	0.10032	8.33130797148327\\
67.625	0.10398	8.90324491876065\\
67.625	0.10764	9.54157180320088\\
67.625	0.1113	10.246288624804\\
67.625	0.11496	11.01739538357\\
67.625	0.11862	11.8548920794989\\
67.625	0.12228	12.7587787125906\\
67.625	0.12594	13.7290552828452\\
67.625	0.1296	14.7657217902627\\
67.625	0.13326	15.8687782348431\\
67.625	0.13692	17.0382246165863\\
67.625	0.14058	18.2740609354925\\
67.625	0.14424	19.5762871915614\\
67.625	0.1479	20.9449033847933\\
67.625	0.15156	22.379909515188\\
67.625	0.15522	23.8813055827457\\
67.625	0.15888	25.4490915874662\\
67.625	0.16254	27.0832675293495\\
67.625	0.1662	28.7838334083958\\
67.625	0.16986	30.5507892246049\\
67.625	0.17352	32.3841349779769\\
67.625	0.17718	34.2838706685117\\
67.625	0.18084	36.2499962962094\\
67.625	0.1845	38.28251186107\\
67.625	0.18816	40.3814173630935\\
67.625	0.19182	42.5467128022799\\
67.625	0.19548	44.7783981786291\\
67.625	0.19914	47.0764734921412\\
67.625	0.2028	49.4409387428162\\
67.625	0.20646	51.8717939306541\\
67.625	0.21012	54.3690390556548\\
67.625	0.21378	56.9326741178184\\
67.625	0.21744	59.5626991171449\\
67.625	0.2211	62.2591140536342\\
67.625	0.22476	65.0219189272865\\
67.625	0.22842	67.8511137381016\\
67.625	0.23208	70.7466984860795\\
67.625	0.23574	73.7086731712204\\
67.625	0.2394	76.7370377935241\\
67.625	0.24306	79.8317923529908\\
67.625	0.24672	82.9929368496202\\
67.625	0.25038	86.2204712834126\\
67.625	0.25404	89.5143956543678\\
67.625	0.2577	92.8747099624859\\
67.625	0.26136	96.3014142077669\\
67.625	0.26502	99.7945083902107\\
67.625	0.26868	103.353992509817\\
67.625	0.27234	106.979866566587\\
67.625	0.276	110.672130560519\\
68	0.093	7.49940026192791\\
68	0.09666	7.9389744815062\\
68	0.10032	8.44493863824733\\
68	0.10398	9.01729273215138\\
68	0.10764	9.65603676321827\\
68	0.1113	10.361170731448\\
68	0.11496	11.1326946368407\\
68	0.11862	11.9706084793962\\
68	0.12228	12.8749122591146\\
68	0.12594	13.8456059759959\\
68	0.1296	14.88268963004\\
68	0.13326	15.986163221247\\
68	0.13692	17.1560267496169\\
68	0.14058	18.3922802151497\\
68	0.14424	19.6949236178454\\
68	0.1479	21.0639569577039\\
68	0.15156	22.4993802347253\\
68	0.15522	24.0011934489096\\
68	0.15888	25.5693966002567\\
68	0.16254	27.2039896887667\\
68	0.1662	28.9049727144396\\
68	0.16986	30.6723456772754\\
68	0.17352	32.506108577274\\
68	0.17718	34.4062614144355\\
68	0.18084	36.3728041887599\\
68	0.1845	38.4057369002472\\
68	0.18816	40.5050595488973\\
68	0.19182	42.6707721347103\\
68	0.19548	44.9028746576862\\
68	0.19914	47.201367117825\\
68	0.2028	49.5662495151266\\
68	0.20646	51.9975218495911\\
68	0.21012	54.4951841212185\\
68	0.21378	57.0592363300088\\
68	0.21744	59.6896784759619\\
68	0.2211	62.3865105590779\\
68	0.22476	65.1497325793568\\
68	0.22842	67.9793445367985\\
68	0.23208	70.8753464314032\\
68	0.23574	73.8377382631707\\
68	0.2394	76.8665200321011\\
68	0.24306	79.9616917381943\\
68	0.24672	83.1232533814505\\
68	0.25038	86.3512049618695\\
68	0.25404	89.6455464794514\\
68	0.2577	93.0062779341961\\
68	0.26136	96.4333993261037\\
68	0.26502	99.9269106551743\\
68	0.26868	103.486811921408\\
68	0.27234	107.113103124804\\
68	0.276	110.805784265363\\
68.375	0.093	7.61593900462633\\
68.375	0.09666	8.05593037083127\\
68.375	0.10032	8.56231167419906\\
68.375	0.10398	9.13508291472976\\
68.375	0.10764	9.77424409242331\\
68.375	0.1113	10.4797952072797\\
68.375	0.11496	11.251736259299\\
68.375	0.11862	12.0900672484812\\
68.375	0.12228	12.9947881748263\\
68.375	0.12594	13.9658990383342\\
68.375	0.1296	15.003399839005\\
68.375	0.13326	16.1072905768387\\
68.375	0.13692	17.2775712518352\\
68.375	0.14058	18.5142418639947\\
68.375	0.14424	19.8173024133169\\
68.375	0.1479	21.1867528998021\\
68.375	0.15156	22.6225933234502\\
68.375	0.15522	24.1248236842611\\
68.375	0.15888	25.6934439822349\\
68.375	0.16254	27.3284542173716\\
68.375	0.1662	29.0298543896711\\
68.375	0.16986	30.7976444991335\\
68.375	0.17352	32.6318245457588\\
68.375	0.17718	34.532394529547\\
68.375	0.18084	36.499354450498\\
68.375	0.1845	38.532704308612\\
68.375	0.18816	40.6324441038887\\
68.375	0.19182	42.7985738363284\\
68.375	0.19548	45.031093505931\\
68.375	0.19914	47.3300031126964\\
68.375	0.2028	49.6953026566247\\
68.375	0.20646	52.1269921377159\\
68.375	0.21012	54.6250715559699\\
68.375	0.21378	57.1895409113868\\
68.375	0.21744	59.8204002039666\\
68.375	0.2211	62.5176494337093\\
68.375	0.22476	65.2812886006148\\
68.375	0.22842	68.1113177046832\\
68.375	0.23208	71.0077367459144\\
68.375	0.23574	73.9705457243086\\
68.375	0.2394	76.9997446398657\\
68.375	0.24306	80.0953334925856\\
68.375	0.24672	83.2573122824684\\
68.375	0.25038	86.4856810095141\\
68.375	0.25404	89.7804396737226\\
68.375	0.2577	93.141588275094\\
68.375	0.26136	96.5691268136283\\
68.375	0.26502	100.063055289325\\
68.375	0.26868	103.623373702185\\
68.375	0.27234	107.250082052208\\
68.375	0.276	110.943180339394\\
68.75	0.093	7.73622011651242\\
68.75	0.09666	8.17662862934401\\
68.75	0.10032	8.68342707933845\\
68.75	0.10398	9.2566154664958\\
68.75	0.10764	9.896193790816\\
68.75	0.1113	10.6021620522991\\
68.75	0.11496	11.3745202509451\\
68.75	0.11862	12.2132683867539\\
68.75	0.12228	13.1184064597256\\
68.75	0.12594	14.0899344698602\\
68.75	0.1296	15.1278524171576\\
68.75	0.13326	16.2321603016179\\
68.75	0.13692	17.4028581232412\\
68.75	0.14058	18.6399458820272\\
68.75	0.14424	19.9434235779762\\
68.75	0.1479	21.313291211088\\
68.75	0.15156	22.7495487813627\\
68.75	0.15522	24.2521962888003\\
68.75	0.15888	25.8212337334008\\
68.75	0.16254	27.4566611151641\\
68.75	0.1662	29.1584784340903\\
68.75	0.16986	30.9266856901794\\
68.75	0.17352	32.7612828834313\\
68.75	0.17718	34.6622700138461\\
68.75	0.18084	36.6296470814238\\
68.75	0.1845	38.6634140861644\\
68.75	0.18816	40.7635710280679\\
68.75	0.19182	42.9301179071342\\
68.75	0.19548	45.1630547233634\\
68.75	0.19914	47.4623814767554\\
68.75	0.2028	49.8280981673104\\
68.75	0.20646	52.2602047950282\\
68.75	0.21012	54.7587013599089\\
68.75	0.21378	57.3235878619525\\
68.75	0.21744	59.9548643011589\\
68.75	0.2211	62.6525306775282\\
68.75	0.22476	65.4165869910604\\
68.75	0.22842	68.2470332417555\\
68.75	0.23208	71.1438694296134\\
68.75	0.23574	74.1070955546343\\
68.75	0.2394	77.1367116168179\\
68.75	0.24306	80.2327176161645\\
68.75	0.24672	83.3951135526739\\
68.75	0.25038	86.6238994263463\\
68.75	0.25404	89.9190752371815\\
68.75	0.2577	93.2806409851795\\
68.75	0.26136	96.7085966703405\\
68.75	0.26502	100.202942292664\\
68.75	0.26868	103.763677852151\\
68.75	0.27234	107.390803348801\\
68.75	0.276	111.084318782613\\
69.125	0.093	7.86024359758616\\
69.125	0.09666	8.30106925704441\\
69.125	0.10032	8.80828485366552\\
69.125	0.10398	9.38189038744952\\
69.125	0.10764	10.0218858583964\\
69.125	0.1113	10.7282712665061\\
69.125	0.11496	11.5010466117787\\
69.125	0.11862	12.3402118942142\\
69.125	0.12228	13.2457671138126\\
69.125	0.12594	14.2177122705738\\
69.125	0.1296	15.2560473644979\\
69.125	0.13326	16.3607723955849\\
69.125	0.13692	17.5318873638348\\
69.125	0.14058	18.7693922692475\\
69.125	0.14424	20.0732871118231\\
69.125	0.1479	21.4435718915616\\
69.125	0.15156	22.880246608463\\
69.125	0.15522	24.3833112625272\\
69.125	0.15888	25.9527658537543\\
69.125	0.16254	27.5886103821443\\
69.125	0.1662	29.2908448476971\\
69.125	0.16986	31.0594692504129\\
69.125	0.17352	32.8944835902915\\
69.125	0.17718	34.7958878673329\\
69.125	0.18084	36.7636820815373\\
69.125	0.1845	38.7978662329045\\
69.125	0.18816	40.8984403214346\\
69.125	0.19182	43.0654043471276\\
69.125	0.19548	45.2987583099834\\
69.125	0.19914	47.5985022100022\\
69.125	0.2028	49.9646360471838\\
69.125	0.20646	52.3971598215283\\
69.125	0.21012	54.8960735330356\\
69.125	0.21378	57.4613771817058\\
69.125	0.21744	60.0930707675389\\
69.125	0.2211	62.7911542905349\\
69.125	0.22476	65.5556277506937\\
69.125	0.22842	68.3864911480154\\
69.125	0.23208	71.2837444825\\
69.125	0.23574	74.2473877541475\\
69.125	0.2394	77.2774209629579\\
69.125	0.24306	80.3738441089311\\
69.125	0.24672	83.5366571920672\\
69.125	0.25038	86.7658602123662\\
69.125	0.25404	90.061453169828\\
69.125	0.2577	93.4234360644527\\
69.125	0.26136	96.8518088962404\\
69.125	0.26502	100.346571665191\\
69.125	0.26868	103.907724371304\\
69.125	0.27234	107.53526701458\\
69.125	0.276	111.229199595019\\
69.5	0.093	7.98800944784757\\
69.5	0.09666	8.42925225393246\\
69.5	0.10032	8.93688499718024\\
69.5	0.10398	9.51090767759086\\
69.5	0.10764	10.1513202951644\\
69.5	0.1113	10.8581228499008\\
69.5	0.11496	11.6313153418001\\
69.5	0.11862	12.4708977708622\\
69.5	0.12228	13.3768701370872\\
69.5	0.12594	14.3492324404751\\
69.5	0.1296	15.3879846810259\\
69.5	0.13326	16.4931268587395\\
69.5	0.13692	17.664658973616\\
69.5	0.14058	18.9025810256554\\
69.5	0.14424	20.2068930148577\\
69.5	0.1479	21.5775949412228\\
69.5	0.15156	23.0146868047508\\
69.5	0.15522	24.5181686054417\\
69.5	0.15888	26.0880403432955\\
69.5	0.16254	27.7243020183121\\
69.5	0.1662	29.4269536304916\\
69.5	0.16986	31.195995179834\\
69.5	0.17352	33.0314266663393\\
69.5	0.17718	34.9332480900074\\
69.5	0.18084	36.9014594508384\\
69.5	0.1845	38.9360607488323\\
69.5	0.18816	41.037051983989\\
69.5	0.19182	43.2044331563087\\
69.5	0.19548	45.4382042657912\\
69.5	0.19914	47.7383653124366\\
69.5	0.2028	50.1049162962448\\
69.5	0.20646	52.537857217216\\
69.5	0.21012	55.03718807535\\
69.5	0.21378	57.6029088706468\\
69.5	0.21744	60.2350196031066\\
69.5	0.2211	62.9335202727292\\
69.5	0.22476	65.6984108795147\\
69.5	0.22842	68.5296914234631\\
69.5	0.23208	71.4273619045743\\
69.5	0.23574	74.3914223228485\\
69.5	0.2394	77.4218726782855\\
69.5	0.24306	80.5187129708853\\
69.5	0.24672	83.6819432006481\\
69.5	0.25038	86.9115633675737\\
69.5	0.25404	90.2075734716622\\
69.5	0.2577	93.5699735129136\\
69.5	0.26136	96.9987634913279\\
69.5	0.26502	100.493943406905\\
69.5	0.26868	104.055513259645\\
69.5	0.27234	107.683473049548\\
69.5	0.276	111.377822776614\\
69.875	0.093	8.11951766729666\\
69.875	0.09666	8.5611776200082\\
69.875	0.10032	9.06922750988263\\
69.875	0.10398	9.64366733691992\\
69.875	0.10764	10.2844971011201\\
69.875	0.1113	10.9917168024832\\
69.875	0.11496	11.7653264410091\\
69.875	0.11862	12.6053260166979\\
69.875	0.12228	13.5117155295495\\
69.875	0.12594	14.4844949795641\\
69.875	0.1296	15.5236643667415\\
69.875	0.13326	16.6292236910818\\
69.875	0.13692	17.801172952585\\
69.875	0.14058	19.039512151251\\
69.875	0.14424	20.3442412870799\\
69.875	0.1479	21.7153603600717\\
69.875	0.15156	23.1528693702264\\
69.875	0.15522	24.6567683175439\\
69.875	0.15888	26.2270572020244\\
69.875	0.16254	27.8637360236676\\
69.875	0.1662	29.5668047824738\\
69.875	0.16986	31.3362634784428\\
69.875	0.17352	33.1721121115748\\
69.875	0.17718	35.0743506818695\\
69.875	0.18084	37.0429791893272\\
69.875	0.1845	39.0779976339477\\
69.875	0.18816	41.1794060157311\\
69.875	0.19182	43.3472043346774\\
69.875	0.19548	45.5813925907866\\
69.875	0.19914	47.8819707840586\\
69.875	0.2028	50.2489389144936\\
69.875	0.20646	52.6822969820913\\
69.875	0.21012	55.182044986852\\
69.875	0.21378	57.7481829287755\\
69.875	0.21744	60.3807108078619\\
69.875	0.2211	63.0796286241112\\
69.875	0.22476	65.8449363775234\\
69.875	0.22842	68.6766340680984\\
69.875	0.23208	71.5747216958363\\
69.875	0.23574	74.5391992607371\\
69.875	0.2394	77.5700667628007\\
69.875	0.24306	80.6673242020273\\
69.875	0.24672	83.8309715784167\\
69.875	0.25038	87.061008891969\\
69.875	0.25404	90.3574361426841\\
69.875	0.2577	93.7202533305622\\
69.875	0.26136	97.149460455603\\
69.875	0.26502	100.645057517807\\
69.875	0.26868	104.207044517173\\
69.875	0.27234	107.835421453703\\
69.875	0.276	111.530188327395\\
70.25	0.093	8.25476825593339\\
70.25	0.09666	8.69684535527161\\
70.25	0.10032	9.20531239177269\\
70.25	0.10398	9.78016936543663\\
70.25	0.10764	10.4214162762635\\
70.25	0.1113	11.1290531242532\\
70.25	0.11496	11.9030799094057\\
70.25	0.11862	12.7434966317212\\
70.25	0.12228	13.6503032911995\\
70.25	0.12594	14.6234998878407\\
70.25	0.1296	15.6630864216448\\
70.25	0.13326	16.7690628926117\\
70.25	0.13692	17.9414293007416\\
70.25	0.14058	19.1801856460343\\
70.25	0.14424	20.4853319284898\\
70.25	0.1479	21.8568681481083\\
70.25	0.15156	23.2947943048896\\
70.25	0.15522	24.7991103988338\\
70.25	0.15888	26.3698164299409\\
70.25	0.16254	28.0069123982108\\
70.25	0.1662	29.7103983036436\\
70.25	0.16986	31.4802741462393\\
70.25	0.17352	33.3165399259979\\
70.25	0.17718	35.2191956429193\\
70.25	0.18084	37.1882412970036\\
70.25	0.1845	39.2236768882508\\
70.25	0.18816	41.3255024166609\\
70.25	0.19182	43.4937178822339\\
70.25	0.19548	45.7283232849697\\
70.25	0.19914	48.0293186248684\\
70.25	0.2028	50.3967039019299\\
70.25	0.20646	52.8304791161544\\
70.25	0.21012	55.3306442675417\\
70.25	0.21378	57.8971993560919\\
70.25	0.21744	60.5301443818049\\
70.25	0.2211	63.2294793446809\\
70.25	0.22476	65.9952042447197\\
70.25	0.22842	68.8273190819213\\
70.25	0.23208	71.7258238562859\\
70.25	0.23574	74.6907185678133\\
70.25	0.2394	77.7220032165037\\
70.25	0.24306	80.8196778023569\\
70.25	0.24672	83.9837423253729\\
70.25	0.25038	87.2141967855518\\
70.25	0.25404	90.5110411828936\\
70.25	0.2577	93.8742755173984\\
70.25	0.26136	97.3038997890659\\
70.25	0.26502	100.799913997896\\
70.25	0.26868	104.36231814389\\
70.25	0.27234	107.991112227046\\
70.25	0.276	111.686296247365\\
70.625	0.093	8.39376121375781\\
70.625	0.09666	8.83625545972267\\
70.625	0.10032	9.34513964285041\\
70.625	0.10398	9.920413763141\\
70.625	0.10764	10.5620778205945\\
70.625	0.1113	11.2701318152108\\
70.625	0.11496	12.0445757469901\\
70.625	0.11862	12.8854096159322\\
70.625	0.12228	13.7926334220372\\
70.625	0.12594	14.766247165305\\
70.625	0.1296	15.8062508457357\\
70.625	0.13326	16.9126444633294\\
70.625	0.13692	18.0854280180858\\
70.625	0.14058	19.3246015100052\\
70.625	0.14424	20.6301649390874\\
70.625	0.1479	22.0021183053325\\
70.625	0.15156	23.4404616087405\\
70.625	0.15522	24.9451948493113\\
70.625	0.15888	26.5163180270451\\
70.625	0.16254	28.1538311419417\\
70.625	0.1662	29.8577341940011\\
70.625	0.16986	31.6280271832235\\
70.625	0.17352	33.4647101096087\\
70.625	0.17718	35.3677829731568\\
70.625	0.18084	37.3372457738678\\
70.625	0.1845	39.3730985117416\\
70.625	0.18816	41.4753411867784\\
70.625	0.19182	43.6439737989779\\
70.625	0.19548	45.8789963483404\\
70.625	0.19914	48.1804088348658\\
70.625	0.2028	50.548211258554\\
70.625	0.20646	52.9824036194051\\
70.625	0.21012	55.482985917419\\
70.625	0.21378	58.0499581525959\\
70.625	0.21744	60.6833203249356\\
70.625	0.2211	63.3830724344382\\
70.625	0.22476	66.1492144811036\\
70.625	0.22842	68.981746464932\\
70.625	0.23208	71.8806683859232\\
70.625	0.23574	74.8459802440773\\
70.625	0.2394	77.8776820393942\\
70.625	0.24306	80.9757737718741\\
70.625	0.24672	84.1402554415168\\
70.625	0.25038	87.3711270483224\\
70.625	0.25404	90.6683885922909\\
70.625	0.2577	94.0320400734222\\
70.625	0.26136	97.4620814917164\\
70.625	0.26502	100.958512847174\\
70.625	0.26868	104.521334139793\\
70.625	0.27234	108.150545369576\\
70.625	0.276	111.846146536522\\
71	0.093	8.53649654076987\\
71	0.09666	8.97940793336138\\
71	0.10032	9.48870926311576\\
71	0.10398	10.064400530033\\
71	0.10764	10.7064817341132\\
71	0.1113	11.4149528753562\\
71	0.11496	12.1898139537621\\
71	0.11862	13.0310649693308\\
71	0.12228	13.9387059220625\\
71	0.12594	14.912736811957\\
71	0.1296	15.9531576390144\\
71	0.13326	17.0599684032346\\
71	0.13692	18.2331691046177\\
71	0.14058	19.4727597431638\\
71	0.14424	20.7787403188726\\
71	0.1479	22.1511108317444\\
71	0.15156	23.589871281779\\
71	0.15522	25.0950216689765\\
71	0.15888	26.6665619933369\\
71	0.16254	28.3044922548602\\
71	0.1662	30.0088124535463\\
71	0.16986	31.7795225893953\\
71	0.17352	33.6166226624072\\
71	0.17718	35.5201126725819\\
71	0.18084	37.4899926199195\\
71	0.1845	39.52626250442\\
71	0.18816	41.6289223260834\\
71	0.19182	43.7979720849097\\
71	0.19548	46.0334117808988\\
71	0.19914	48.3352414140508\\
71	0.2028	50.7034609843657\\
71	0.20646	53.1380704918434\\
71	0.21012	55.639069936484\\
71	0.21378	58.2064593182875\\
71	0.21744	60.8402386372539\\
71	0.2211	63.5404078933832\\
71	0.22476	66.3069670866753\\
71	0.22842	69.1399162171302\\
71	0.23208	72.0392552847481\\
71	0.23574	75.0049842895289\\
71	0.2394	78.0371032314725\\
71	0.24306	81.135612110579\\
71	0.24672	84.3005109268484\\
71	0.25038	87.5317996802806\\
71	0.25404	90.8294783708757\\
71	0.2577	94.1935469986337\\
71	0.26136	97.6240055635546\\
71	0.26502	101.120854065638\\
71	0.26868	104.684092504885\\
71	0.27234	108.313720881294\\
71	0.276	112.009739194867\\
71.375	0.093	8.68297423696959\\
71.375	0.09666	9.12630277618777\\
71.375	0.10032	9.63602125256881\\
71.375	0.10398	10.2121296661127\\
71.375	0.10764	10.8546280168195\\
71.375	0.1113	11.5635163046892\\
71.375	0.11496	12.3387945297217\\
71.375	0.11862	13.1804626919171\\
71.375	0.12228	14.0885207912754\\
71.375	0.12594	15.0629688277966\\
71.375	0.1296	16.1038068014806\\
71.375	0.13326	17.2110347123275\\
71.375	0.13692	18.3846525603373\\
71.375	0.14058	19.62466034551\\
71.375	0.14424	20.9310580678455\\
71.375	0.1479	22.3038457273439\\
71.375	0.15156	23.7430233240052\\
71.375	0.15522	25.2485908578294\\
71.375	0.15888	26.8205483288164\\
71.375	0.16254	28.4588957369663\\
71.375	0.1662	30.1636330822791\\
71.375	0.16986	31.9347603647548\\
71.375	0.17352	33.7722775843933\\
71.375	0.17718	35.6761847411947\\
71.375	0.18084	37.646481835159\\
71.375	0.1845	39.6831688662862\\
71.375	0.18816	41.7862458345762\\
71.375	0.19182	43.9557127400291\\
71.375	0.19548	46.1915695826449\\
71.375	0.19914	48.4938163624235\\
71.375	0.2028	50.8624530793651\\
71.375	0.20646	53.2974797334695\\
71.375	0.21012	55.7988963247367\\
71.375	0.21378	58.3667028531669\\
71.375	0.21744	61.0008993187599\\
71.375	0.2211	63.7014857215158\\
71.375	0.22476	66.4684620614346\\
71.375	0.22842	69.3018283385162\\
71.375	0.23208	72.2015845527607\\
71.375	0.23574	75.1677307041682\\
71.375	0.2394	78.2002667927384\\
71.375	0.24306	81.2991928184716\\
71.375	0.24672	84.4645087813676\\
71.375	0.25038	87.6962146814265\\
71.375	0.25404	90.9943105186483\\
71.375	0.2577	94.3587962930329\\
71.375	0.26136	97.7896720045804\\
71.375	0.26502	101.286937653291\\
71.375	0.26868	104.850593239164\\
71.375	0.27234	108.4806387622\\
71.375	0.276	112.177074222399\\
71.75	0.093	8.83319430235701\\
71.75	0.09666	9.27693998820184\\
71.75	0.10032	9.78707561120953\\
71.75	0.10398	10.3636011713801\\
71.75	0.10764	11.0065166687135\\
71.75	0.1113	11.7158221032099\\
71.75	0.11496	12.4915174748691\\
71.75	0.11862	13.3336027836911\\
71.75	0.12228	14.2420780296761\\
71.75	0.12594	15.2169432128239\\
71.75	0.1296	16.2581983331346\\
71.75	0.13326	17.3658433906082\\
71.75	0.13692	18.5398783852446\\
71.75	0.14058	19.7803033170439\\
71.75	0.14424	21.0871181860061\\
71.75	0.1479	22.4603229921312\\
71.75	0.15156	23.8999177354191\\
71.75	0.15522	25.4059024158699\\
71.75	0.15888	26.9782770334836\\
71.75	0.16254	28.6170415882602\\
71.75	0.1662	30.3221960801996\\
71.75	0.16986	32.0937405093019\\
71.75	0.17352	33.9316748755671\\
71.75	0.17718	35.8359991789952\\
71.75	0.18084	37.8067134195861\\
71.75	0.1845	39.8438175973399\\
71.75	0.18816	41.9473117122566\\
71.75	0.19182	44.1171957643362\\
71.75	0.19548	46.3534697535786\\
71.75	0.19914	48.6561336799839\\
71.75	0.2028	51.0251875435521\\
71.75	0.20646	53.4606313442832\\
71.75	0.21012	55.9624650821771\\
71.75	0.21378	58.5306887572339\\
71.75	0.21744	61.1653023694536\\
71.75	0.2211	63.8663059188361\\
71.75	0.22476	66.6336994053816\\
71.75	0.22842	69.4674828290898\\
71.75	0.23208	72.367656189961\\
71.75	0.23574	75.3342194879951\\
71.75	0.2394	78.367172723192\\
71.75	0.24306	81.4665158955518\\
71.75	0.24672	84.6322490050745\\
71.75	0.25038	87.8643720517601\\
71.75	0.25404	91.1628850356085\\
71.75	0.2577	94.5277879566198\\
71.75	0.26136	97.959080814794\\
71.75	0.26502	101.456763610131\\
71.75	0.26868	105.020836342631\\
71.75	0.27234	108.651299012294\\
71.75	0.276	112.348151619119\\
72.125	0.093	8.98715673693209\\
72.125	0.09666	9.43131956940356\\
72.125	0.10032	9.94187233903791\\
72.125	0.10398	10.5188150458352\\
72.125	0.10764	11.1621476897952\\
72.125	0.1113	11.8718702709182\\
72.125	0.11496	12.6479827892041\\
72.125	0.11862	13.4904852446528\\
72.125	0.12228	14.3993776372644\\
72.125	0.12594	15.3746599670389\\
72.125	0.1296	16.4163322339762\\
72.125	0.13326	17.5243944380764\\
72.125	0.13692	18.6988465793395\\
72.125	0.14058	19.9396886577655\\
72.125	0.14424	21.2469206733543\\
72.125	0.1479	22.6205426261061\\
72.125	0.15156	24.0605545160207\\
72.125	0.15522	25.5669563430981\\
72.125	0.15888	27.1397481073385\\
72.125	0.16254	28.7789298087417\\
72.125	0.1662	30.4845014473078\\
72.125	0.16986	32.2564630230368\\
72.125	0.17352	34.0948145359286\\
72.125	0.17718	35.9995559859833\\
72.125	0.18084	37.9706873732009\\
72.125	0.1845	40.0082086975814\\
72.125	0.18816	42.1121199591247\\
72.125	0.19182	44.2824211578309\\
72.125	0.19548	46.5191122937\\
72.125	0.19914	48.822193366732\\
72.125	0.2028	51.1916643769268\\
72.125	0.20646	53.6275253242845\\
72.125	0.21012	56.1297762088051\\
72.125	0.21378	58.6984170304886\\
72.125	0.21744	61.3334477893349\\
72.125	0.2211	64.0348684853441\\
72.125	0.22476	66.8026791185162\\
72.125	0.22842	69.6368796888511\\
72.125	0.23208	72.537470196349\\
72.125	0.23574	75.5044506410097\\
72.125	0.2394	78.5378210228333\\
72.125	0.24306	81.6375813418197\\
72.125	0.24672	84.8037315979691\\
72.125	0.25038	88.0362717912813\\
72.125	0.25404	91.3352019217564\\
72.125	0.2577	94.7005219893943\\
72.125	0.26136	98.1322319941951\\
72.125	0.26502	101.630331936159\\
72.125	0.26868	105.194821815285\\
72.125	0.27234	108.825701631575\\
72.125	0.276	112.522971385027\\
72.5	0.093	9.14486154069483\\
72.5	0.09666	9.58944151979296\\
72.5	0.10032	10.100411436054\\
72.5	0.10398	10.6777712894778\\
72.5	0.10764	11.3215210800646\\
72.5	0.1113	12.0316608078142\\
72.5	0.11496	12.8081904727267\\
72.5	0.11862	13.6511100748021\\
72.5	0.12228	14.5604196140404\\
72.5	0.12594	15.5361190904415\\
72.5	0.1296	16.5782085040055\\
72.5	0.13326	17.6866878547324\\
72.5	0.13692	18.8615571426221\\
72.5	0.14058	20.1028163676748\\
72.5	0.14424	21.4104655298903\\
72.5	0.1479	22.7845046292686\\
72.5	0.15156	24.2249336658099\\
72.5	0.15522	25.731752639514\\
72.5	0.15888	27.304961550381\\
72.5	0.16254	28.9445603984109\\
72.5	0.1662	30.6505491836036\\
72.5	0.16986	32.4229279059592\\
72.5	0.17352	34.2616965654777\\
72.5	0.17718	36.1668551621591\\
72.5	0.18084	38.1384036960033\\
72.5	0.1845	40.1763421670105\\
72.5	0.18816	42.2806705751805\\
72.5	0.19182	44.4513889205133\\
72.5	0.19548	46.6884972030091\\
72.5	0.19914	48.9919954226677\\
72.5	0.2028	51.3618835794892\\
72.5	0.20646	53.7981616734736\\
72.5	0.21012	56.3008297046208\\
72.5	0.21378	58.8698876729309\\
72.5	0.21744	61.5053355784039\\
72.5	0.2211	64.2071734210398\\
72.5	0.22476	66.9754012008385\\
72.5	0.22842	69.8100189178001\\
72.5	0.23208	72.7110265719245\\
72.5	0.23574	75.678424163212\\
72.5	0.2394	78.7122116916622\\
72.5	0.24306	81.8123891572753\\
72.5	0.24672	84.9789565600513\\
72.5	0.25038	88.2119138999902\\
72.5	0.25404	91.5112611770919\\
72.5	0.2577	94.8769983913565\\
72.5	0.26136	98.309125542784\\
72.5	0.26502	101.807642631374\\
72.5	0.26868	105.372549657128\\
72.5	0.27234	109.003846620044\\
72.5	0.276	112.701533520123\\
72.875	0.093	9.3063087136452\\
72.875	0.09666	9.75130583936999\\
72.875	0.10032	10.2626929022577\\
72.875	0.10398	10.8404699023082\\
72.875	0.10764	11.4846368395216\\
72.875	0.1113	12.1951937138979\\
72.875	0.11496	12.972140525437\\
72.875	0.11862	13.8154772741391\\
72.875	0.12228	14.725203960004\\
72.875	0.12594	15.7013205830318\\
72.875	0.1296	16.7438271432224\\
72.875	0.13326	17.852723640576\\
72.875	0.13692	19.0280100750924\\
72.875	0.14058	20.2696864467716\\
72.875	0.14424	21.5777527556138\\
72.875	0.1479	22.9522090016188\\
72.875	0.15156	24.3930551847867\\
72.875	0.15522	25.9002913051175\\
72.875	0.15888	27.4739173626112\\
72.875	0.16254	29.1139333572677\\
72.875	0.1662	30.8203392890871\\
72.875	0.16986	32.5931351580694\\
72.875	0.17352	34.4323209642145\\
72.875	0.17718	36.3378967075225\\
72.875	0.18084	38.3098623879934\\
72.875	0.1845	40.3482180056272\\
72.875	0.18816	42.4529635604239\\
72.875	0.19182	44.6240990523834\\
72.875	0.19548	46.8616244815058\\
72.875	0.19914	49.1655398477911\\
72.875	0.2028	51.5358451512392\\
72.875	0.20646	53.9725403918502\\
72.875	0.21012	56.4756255696241\\
72.875	0.21378	59.0451006845609\\
72.875	0.21744	61.6809657366605\\
72.875	0.2211	64.3832207259231\\
72.875	0.22476	67.1518656523484\\
72.875	0.22842	69.9869005159367\\
72.875	0.23208	72.8883253166878\\
72.875	0.23574	75.8561400546019\\
72.875	0.2394	78.8903447296788\\
72.875	0.24306	81.9909393419185\\
72.875	0.24672	85.1579238913211\\
72.875	0.25038	88.3912983778867\\
72.875	0.25404	91.6910628016151\\
72.875	0.2577	95.0572171625064\\
72.875	0.26136	98.4897614605605\\
72.875	0.26502	101.988695695777\\
72.875	0.26868	105.554019868157\\
72.875	0.27234	109.1857339777\\
72.875	0.276	112.883838024406\\
73.25	0.093	9.47149825578327\\
73.25	0.09666	9.91691252813471\\
73.25	0.10032	10.428716737649\\
73.25	0.10398	11.0069108843262\\
73.25	0.10764	11.6514949681663\\
73.25	0.1113	12.3624689891692\\
73.25	0.11496	13.139832947335\\
73.25	0.11862	13.9835868426637\\
73.25	0.12228	14.8937306751553\\
73.25	0.12594	15.8702644448097\\
73.25	0.1296	16.913188151627\\
73.25	0.13326	18.0225017956072\\
73.25	0.13692	19.1982053767503\\
73.25	0.14058	20.4402988950562\\
73.25	0.14424	21.748782350525\\
73.25	0.1479	23.1236557431567\\
73.25	0.15156	24.5649190729513\\
73.25	0.15522	26.0725723399087\\
73.25	0.15888	27.646615544029\\
73.25	0.16254	29.2870486853122\\
73.25	0.1662	30.9938717637582\\
73.25	0.16986	32.7670847793672\\
73.25	0.17352	34.606687732139\\
73.25	0.17718	36.5126806220736\\
73.25	0.18084	38.4850634491712\\
73.25	0.1845	40.5238362134316\\
73.25	0.18816	42.6289989148549\\
73.25	0.19182	44.8005515534411\\
73.25	0.19548	47.0384941291902\\
73.25	0.19914	49.3428266421021\\
73.25	0.2028	51.7135490921769\\
73.25	0.20646	54.1506614794146\\
73.25	0.21012	56.6541638038152\\
73.25	0.21378	59.2240560653785\\
73.25	0.21744	61.8603382641048\\
73.25	0.2211	64.563010399994\\
73.25	0.22476	67.3320724730461\\
73.25	0.22842	70.167524483261\\
73.25	0.23208	73.0693664306388\\
73.25	0.23574	76.0375983151795\\
73.25	0.2394	79.072220136883\\
73.25	0.24306	82.1732318957495\\
73.25	0.24672	85.3406335917787\\
73.25	0.25038	88.5744252249709\\
73.25	0.25404	91.8746067953259\\
73.25	0.2577	95.2411783028439\\
73.25	0.26136	98.6741397475247\\
73.25	0.26502	102.173491129368\\
73.25	0.26868	105.739232448375\\
73.25	0.27234	109.371363704544\\
73.25	0.276	113.069884897877\\
73.625	0.093	9.640430167109\\
73.625	0.09666	10.0862615860871\\
73.625	0.10032	10.5984829422281\\
73.625	0.10398	11.1770942355319\\
73.625	0.10764	11.8220954659986\\
73.625	0.1113	12.5334866336282\\
73.625	0.11496	13.3112677384207\\
73.625	0.11862	14.155438780376\\
73.625	0.12228	15.0659997594942\\
73.625	0.12594	16.0429506757754\\
73.625	0.1296	17.0862915292193\\
73.625	0.13326	18.1960223198261\\
73.625	0.13692	19.3721430475958\\
73.625	0.14058	20.6146537125284\\
73.625	0.14424	21.9235543146239\\
73.625	0.1479	23.2988448538823\\
73.625	0.15156	24.7405253303035\\
73.625	0.15522	26.2485957438876\\
73.625	0.15888	27.8230560946345\\
73.625	0.16254	29.4639063825444\\
73.625	0.1662	31.1711466076171\\
73.625	0.16986	32.9447767698526\\
73.625	0.17352	34.7847968692511\\
73.625	0.17718	36.6912069058124\\
73.625	0.18084	38.6640068795366\\
73.625	0.1845	40.7031967904237\\
73.625	0.18816	42.8087766384737\\
73.625	0.19182	44.9807464236865\\
73.625	0.19548	47.2191061460622\\
73.625	0.19914	49.5238558056008\\
73.625	0.2028	51.8949954023023\\
73.625	0.20646	54.3325249361666\\
73.625	0.21012	56.8364444071938\\
73.625	0.21378	59.4067538153839\\
73.625	0.21744	62.0434531607368\\
73.625	0.2211	64.7465424432527\\
73.625	0.22476	67.5160216629314\\
73.625	0.22842	70.3518908197729\\
73.625	0.23208	73.2541499137774\\
73.625	0.23574	76.2227989449447\\
73.625	0.2394	79.2578379132749\\
73.625	0.24306	82.359266818768\\
73.625	0.24672	85.5270856614239\\
73.625	0.25038	88.7612944412428\\
73.625	0.25404	92.0618931582245\\
73.625	0.2577	95.4288818123691\\
73.625	0.26136	98.8622604036765\\
73.625	0.26502	102.362028932147\\
73.625	0.26868	105.92818739778\\
73.625	0.27234	109.560735800576\\
73.625	0.276	113.259674140535\\
74	0.093	9.81310444762239\\
74	0.09666	10.2593530132271\\
74	0.10032	10.7719915159948\\
74	0.10398	11.3510199559253\\
74	0.10764	11.9964383330186\\
74	0.1113	12.7082466472749\\
74	0.11496	13.486444898694\\
74	0.11862	14.331033087276\\
74	0.12228	15.2420112130209\\
74	0.12594	16.2193792759286\\
74	0.1296	17.2631372759992\\
74	0.13326	18.3732852132328\\
74	0.13692	19.5498230876291\\
74	0.14058	20.7927508991883\\
74	0.14424	22.1020686479105\\
74	0.1479	23.4777763337955\\
74	0.15156	24.9198739568433\\
74	0.15522	26.4283615170541\\
74	0.15888	28.0032390144277\\
74	0.16254	29.6445064489642\\
74	0.1662	31.3521638206636\\
74	0.16986	33.1262111295258\\
74	0.17352	34.9666483755509\\
74	0.17718	36.8734755587389\\
74	0.18084	38.8466926790898\\
74	0.1845	40.8862997366035\\
74	0.18816	42.9922967312801\\
74	0.19182	45.1646836631196\\
74	0.19548	47.403460532122\\
74	0.19914	49.7086273382872\\
74	0.2028	52.0801840816153\\
74	0.20646	54.5181307621063\\
74	0.21012	57.0224673797602\\
74	0.21378	59.5931939345769\\
74	0.21744	62.2303104265565\\
74	0.2211	64.933816855699\\
74	0.22476	67.7037132220043\\
74	0.22842	70.5399995254726\\
74	0.23208	73.4426757661036\\
74	0.23574	76.4117419438977\\
74	0.2394	79.4471980588545\\
74	0.24306	82.5490441109742\\
74	0.24672	85.7172801002568\\
74	0.25038	88.9519060267024\\
74	0.25404	92.2529218903107\\
74	0.2577	95.6203276910819\\
74	0.26136	99.054123429016\\
74	0.26502	102.554309104113\\
74	0.26868	106.120884716373\\
74	0.27234	109.753850265796\\
74	0.276	113.453205752381\\
};
\end{axis}

\begin{axis}[%
width=4.527496cm,
height=3.870968cm,
at={(0cm,5.376344cm)},
scale only axis,
xmin=56,
xmax=74,
tick align=outside,
xlabel={$L_{cut}$},
xmajorgrids,
ymin=0.093,
ymax=0.276,
ylabel={$D_{rlx}$},
ymajorgrids,
zmin=-3.90013578541794,
zmax=52.2197059680429,
zlabel={$x_2$},
zmajorgrids,
view={-140}{50},
legend style={at={(1.03,1)},anchor=north west,legend cell align=left,align=left,draw=white!15!black}
]
\addplot3[only marks,mark=*,mark options={},mark size=1.5000pt,color=mycolor1] plot table[row sep=crcr,]{%
74	0.123	0.706917161850943\\
72	0.113	-0.725188933625476\\
61	0.095	-2.23969084196957\\
56	0.093	-0.663995623424443\\
};
\addplot3[only marks,mark=*,mark options={},mark size=1.5000pt,color=mycolor2] plot table[row sep=crcr,]{%
67	0.276	45.9418866067325\\
66	0.255	36.5734993512285\\
62	0.209	18.3955084875498\\
57	0.193	12.6775369279536\\
};
\addplot3[only marks,mark=*,mark options={},mark size=1.5000pt,color=black] plot table[row sep=crcr,]{%
69	0.104	-2.29954839560715\\
};
\addplot3[only marks,mark=*,mark options={},mark size=1.5000pt,color=black] plot table[row sep=crcr,]{%
64	0.23	25.9519553252518\\
};

\addplot3[%
surf,
opacity=0.7,
shader=interp,
colormap={mymap}{[1pt] rgb(0pt)=(0.0901961,0.239216,0.0745098); rgb(1pt)=(0.0945149,0.242058,0.0739522); rgb(2pt)=(0.0988592,0.244894,0.0733566); rgb(3pt)=(0.103229,0.247724,0.0727241); rgb(4pt)=(0.107623,0.250549,0.0720557); rgb(5pt)=(0.112043,0.253367,0.0713525); rgb(6pt)=(0.116487,0.25618,0.0706154); rgb(7pt)=(0.120956,0.258986,0.0698456); rgb(8pt)=(0.125449,0.261787,0.0690441); rgb(9pt)=(0.129967,0.264581,0.0682118); rgb(10pt)=(0.134508,0.26737,0.06735); rgb(11pt)=(0.139074,0.270152,0.0664596); rgb(12pt)=(0.143663,0.272929,0.0655416); rgb(13pt)=(0.148275,0.275699,0.0645971); rgb(14pt)=(0.152911,0.278463,0.0636271); rgb(15pt)=(0.15757,0.281221,0.0626328); rgb(16pt)=(0.162252,0.283973,0.0616151); rgb(17pt)=(0.166957,0.286719,0.060575); rgb(18pt)=(0.171685,0.289458,0.0595136); rgb(19pt)=(0.176434,0.292191,0.0584321); rgb(20pt)=(0.181207,0.294918,0.0573313); rgb(21pt)=(0.186001,0.297639,0.0562123); rgb(22pt)=(0.190817,0.300353,0.0550763); rgb(23pt)=(0.195655,0.303061,0.0539242); rgb(24pt)=(0.200514,0.305763,0.052757); rgb(25pt)=(0.205395,0.308459,0.0515759); rgb(26pt)=(0.210296,0.311149,0.0503624); rgb(27pt)=(0.215212,0.313846,0.0490067); rgb(28pt)=(0.220142,0.316548,0.0475043); rgb(29pt)=(0.22509,0.319254,0.0458704); rgb(30pt)=(0.230056,0.321962,0.0441205); rgb(31pt)=(0.235042,0.324671,0.04227); rgb(32pt)=(0.240048,0.327379,0.0403343); rgb(33pt)=(0.245078,0.330085,0.0383287); rgb(34pt)=(0.250131,0.332786,0.0362688); rgb(35pt)=(0.25521,0.335482,0.0341698); rgb(36pt)=(0.260317,0.33817,0.0320472); rgb(37pt)=(0.265451,0.340849,0.0299163); rgb(38pt)=(0.270616,0.343517,0.0277927); rgb(39pt)=(0.275813,0.346172,0.0256916); rgb(40pt)=(0.281043,0.348814,0.0236284); rgb(41pt)=(0.286307,0.35144,0.0216186); rgb(42pt)=(0.291607,0.354048,0.0196776); rgb(43pt)=(0.296945,0.356637,0.0178207); rgb(44pt)=(0.302322,0.359206,0.0160634); rgb(45pt)=(0.307739,0.361753,0.0144211); rgb(46pt)=(0.313198,0.364275,0.0129091); rgb(47pt)=(0.318701,0.366772,0.0115428); rgb(48pt)=(0.324249,0.369242,0.0103377); rgb(49pt)=(0.329843,0.371682,0.00930909); rgb(50pt)=(0.335485,0.374093,0.00847245); rgb(51pt)=(0.341176,0.376471,0.00784314); rgb(52pt)=(0.346925,0.378826,0.00732741); rgb(53pt)=(0.352735,0.381168,0.00682184); rgb(54pt)=(0.358605,0.383497,0.00632729); rgb(55pt)=(0.364532,0.385812,0.00584464); rgb(56pt)=(0.370516,0.388113,0.00537476); rgb(57pt)=(0.376552,0.390399,0.00491852); rgb(58pt)=(0.38264,0.39267,0.00447681); rgb(59pt)=(0.388777,0.394925,0.00405048); rgb(60pt)=(0.394962,0.397164,0.00364042); rgb(61pt)=(0.401191,0.399386,0.00324749); rgb(62pt)=(0.407464,0.401592,0.00287258); rgb(63pt)=(0.413777,0.40378,0.00251655); rgb(64pt)=(0.420129,0.40595,0.00218028); rgb(65pt)=(0.426518,0.408102,0.00186463); rgb(66pt)=(0.432942,0.410234,0.00157049); rgb(67pt)=(0.439399,0.412348,0.00129873); rgb(68pt)=(0.445885,0.414441,0.00105022); rgb(69pt)=(0.452401,0.416515,0.000825833); rgb(70pt)=(0.458942,0.418567,0.000626441); rgb(71pt)=(0.465508,0.420599,0.00045292); rgb(72pt)=(0.472096,0.422609,0.000306141); rgb(73pt)=(0.478704,0.424596,0.000186979); rgb(74pt)=(0.485331,0.426562,9.63073e-05); rgb(75pt)=(0.491973,0.428504,3.49981e-05); rgb(76pt)=(0.498628,0.430422,3.92506e-06); rgb(77pt)=(0.505323,0.432315,0); rgb(78pt)=(0.512206,0.434168,0); rgb(79pt)=(0.519282,0.435983,0); rgb(80pt)=(0.526529,0.437764,0); rgb(81pt)=(0.533922,0.439512,0); rgb(82pt)=(0.54144,0.441232,0); rgb(83pt)=(0.549059,0.442927,0); rgb(84pt)=(0.556756,0.444599,0); rgb(85pt)=(0.564508,0.446252,0); rgb(86pt)=(0.572292,0.447889,0); rgb(87pt)=(0.580084,0.449514,0); rgb(88pt)=(0.587863,0.451129,0); rgb(89pt)=(0.595604,0.452737,0); rgb(90pt)=(0.603284,0.454343,0); rgb(91pt)=(0.610882,0.455948,0); rgb(92pt)=(0.618373,0.457556,0); rgb(93pt)=(0.625734,0.459171,0); rgb(94pt)=(0.632943,0.460795,0); rgb(95pt)=(0.639976,0.462432,0); rgb(96pt)=(0.64681,0.464084,0); rgb(97pt)=(0.653423,0.465756,0); rgb(98pt)=(0.659791,0.46745,0); rgb(99pt)=(0.665891,0.469169,0); rgb(100pt)=(0.6717,0.470916,0); rgb(101pt)=(0.677195,0.472696,0); rgb(102pt)=(0.682353,0.47451,0); rgb(103pt)=(0.687242,0.476355,0); rgb(104pt)=(0.691952,0.478225,0); rgb(105pt)=(0.696497,0.480118,0); rgb(106pt)=(0.700887,0.482033,0); rgb(107pt)=(0.705134,0.483968,0); rgb(108pt)=(0.709251,0.485921,0); rgb(109pt)=(0.713249,0.487891,0); rgb(110pt)=(0.71714,0.489876,0); rgb(111pt)=(0.720936,0.491875,0); rgb(112pt)=(0.724649,0.493887,0); rgb(113pt)=(0.72829,0.495909,0); rgb(114pt)=(0.731872,0.49794,0); rgb(115pt)=(0.735406,0.499979,0); rgb(116pt)=(0.738904,0.502025,0); rgb(117pt)=(0.742378,0.504075,0); rgb(118pt)=(0.74584,0.506128,0); rgb(119pt)=(0.749302,0.508182,0); rgb(120pt)=(0.752775,0.510237,0); rgb(121pt)=(0.756272,0.51229,0); rgb(122pt)=(0.759804,0.514339,0); rgb(123pt)=(0.763384,0.516385,0); rgb(124pt)=(0.767022,0.518424,0); rgb(125pt)=(0.770731,0.520455,0); rgb(126pt)=(0.774523,0.522478,0); rgb(127pt)=(0.77841,0.524489,0); rgb(128pt)=(0.782391,0.526491,0); rgb(129pt)=(0.786402,0.528496,0); rgb(130pt)=(0.790431,0.530506,0); rgb(131pt)=(0.794478,0.532521,0); rgb(132pt)=(0.798541,0.534539,0); rgb(133pt)=(0.802619,0.53656,0); rgb(134pt)=(0.806712,0.538584,0); rgb(135pt)=(0.81082,0.540609,0); rgb(136pt)=(0.81494,0.542635,0); rgb(137pt)=(0.819074,0.54466,0); rgb(138pt)=(0.823219,0.546686,0); rgb(139pt)=(0.827374,0.548709,0); rgb(140pt)=(0.831541,0.55073,0); rgb(141pt)=(0.835716,0.552749,0); rgb(142pt)=(0.8399,0.554763,0); rgb(143pt)=(0.844092,0.556774,0); rgb(144pt)=(0.848292,0.558779,0); rgb(145pt)=(0.852497,0.560778,0); rgb(146pt)=(0.856708,0.562771,0); rgb(147pt)=(0.860924,0.564756,0); rgb(148pt)=(0.865143,0.566733,0); rgb(149pt)=(0.869366,0.568701,0); rgb(150pt)=(0.873592,0.57066,0); rgb(151pt)=(0.877819,0.572608,0); rgb(152pt)=(0.882047,0.574545,0); rgb(153pt)=(0.886275,0.576471,0); rgb(154pt)=(0.890659,0.578362,0); rgb(155pt)=(0.895333,0.580203,0); rgb(156pt)=(0.900258,0.581999,0); rgb(157pt)=(0.905397,0.583755,0); rgb(158pt)=(0.910711,0.585479,0); rgb(159pt)=(0.916164,0.587176,0); rgb(160pt)=(0.921717,0.588852,0); rgb(161pt)=(0.927333,0.590513,0); rgb(162pt)=(0.932974,0.592166,0); rgb(163pt)=(0.938602,0.593815,0); rgb(164pt)=(0.94418,0.595468,0); rgb(165pt)=(0.949669,0.59713,0); rgb(166pt)=(0.955033,0.598808,0); rgb(167pt)=(0.960233,0.600507,0); rgb(168pt)=(0.965232,0.602233,0); rgb(169pt)=(0.969992,0.603992,0); rgb(170pt)=(0.974475,0.605791,0); rgb(171pt)=(0.978643,0.607636,0); rgb(172pt)=(0.98246,0.609532,0); rgb(173pt)=(0.985886,0.611486,0); rgb(174pt)=(0.988885,0.613503,0); rgb(175pt)=(0.991419,0.61559,0); rgb(176pt)=(0.99345,0.617753,0); rgb(177pt)=(0.99494,0.619997,0); rgb(178pt)=(0.995851,0.622329,0); rgb(179pt)=(0.996226,0.624763,0); rgb(180pt)=(0.996512,0.627352,0); rgb(181pt)=(0.996788,0.630095,0); rgb(182pt)=(0.997053,0.632982,0); rgb(183pt)=(0.997308,0.636004,0); rgb(184pt)=(0.997552,0.639152,0); rgb(185pt)=(0.997785,0.642416,0); rgb(186pt)=(0.998006,0.645786,0); rgb(187pt)=(0.998217,0.649253,0); rgb(188pt)=(0.998416,0.652807,0); rgb(189pt)=(0.998605,0.656439,0); rgb(190pt)=(0.998781,0.660138,0); rgb(191pt)=(0.998946,0.663897,0); rgb(192pt)=(0.9991,0.667704,0); rgb(193pt)=(0.999242,0.67155,0); rgb(194pt)=(0.999372,0.675427,0); rgb(195pt)=(0.99949,0.679323,0); rgb(196pt)=(0.999596,0.68323,0); rgb(197pt)=(0.99969,0.687139,0); rgb(198pt)=(0.999771,0.691039,0); rgb(199pt)=(0.999841,0.694921,0); rgb(200pt)=(0.999898,0.698775,0); rgb(201pt)=(0.999942,0.702592,0); rgb(202pt)=(0.999974,0.706363,0); rgb(203pt)=(0.999994,0.710077,0); rgb(204pt)=(1,0.713725,0); rgb(205pt)=(1,0.717341,0); rgb(206pt)=(1,0.720963,0); rgb(207pt)=(1,0.724591,0); rgb(208pt)=(1,0.728226,0); rgb(209pt)=(1,0.731867,0); rgb(210pt)=(1,0.735514,0); rgb(211pt)=(1,0.739167,0); rgb(212pt)=(1,0.742827,0); rgb(213pt)=(1,0.746493,0); rgb(214pt)=(1,0.750165,0); rgb(215pt)=(1,0.753843,0); rgb(216pt)=(1,0.757527,0); rgb(217pt)=(1,0.761217,0); rgb(218pt)=(1,0.764913,0); rgb(219pt)=(1,0.768615,0); rgb(220pt)=(1,0.772324,0); rgb(221pt)=(1,0.776038,0); rgb(222pt)=(1,0.779758,0); rgb(223pt)=(1,0.783484,0); rgb(224pt)=(1,0.787215,0); rgb(225pt)=(1,0.790953,0); rgb(226pt)=(1,0.794696,0); rgb(227pt)=(1,0.798445,0); rgb(228pt)=(1,0.8022,0); rgb(229pt)=(1,0.805961,0); rgb(230pt)=(1,0.809727,0); rgb(231pt)=(1,0.8135,0); rgb(232pt)=(1,0.817278,0); rgb(233pt)=(1,0.821063,0); rgb(234pt)=(1,0.824854,0); rgb(235pt)=(1,0.828652,0); rgb(236pt)=(1,0.832455,0); rgb(237pt)=(1,0.836265,0); rgb(238pt)=(1,0.840081,0); rgb(239pt)=(1,0.843903,0); rgb(240pt)=(1,0.847732,0); rgb(241pt)=(1,0.851566,0); rgb(242pt)=(1,0.855406,0); rgb(243pt)=(1,0.859253,0); rgb(244pt)=(1,0.863106,0); rgb(245pt)=(1,0.866964,0); rgb(246pt)=(1,0.870829,0); rgb(247pt)=(1,0.8747,0); rgb(248pt)=(1,0.878577,0); rgb(249pt)=(1,0.88246,0); rgb(250pt)=(1,0.886349,0); rgb(251pt)=(1,0.890243,0); rgb(252pt)=(1,0.894144,0); rgb(253pt)=(1,0.898051,0); rgb(254pt)=(1,0.901964,0); rgb(255pt)=(1,0.905882,0)},
mesh/rows=49]
table[row sep=crcr,header=false] {%
%
56	0.093	-0.774334427159586\\
56	0.09666	-0.624731683335815\\
56	0.10032	-0.449584953080116\\
56	0.10398	-0.248894236392514\\
56	0.10764	-0.0226595332730053\\
56	0.1113	0.229119156278419\\
56	0.11496	0.506441832261782\\
56	0.11862	0.80930849467703\\
56	0.12228	1.1377191435242\\
56	0.12594	1.49167377880329\\
56	0.1296	1.87117240051427\\
56	0.13326	2.27621500865718\\
56	0.13692	2.706801603232\\
56	0.14058	3.16293218423874\\
56	0.14424	3.64460675167738\\
56	0.1479	4.15182530554794\\
56	0.15156	4.68458784585042\\
56	0.15522	5.24289437258479\\
56	0.15888	5.82674488575109\\
56	0.16254	6.43613938534931\\
56	0.1662	7.07107787137943\\
56	0.16986	7.73156034384146\\
56	0.17352	8.41758680273542\\
56	0.17718	9.12915724806128\\
56	0.18084	9.86627167981906\\
56	0.1845	10.6289300980087\\
56	0.18816	11.4171325026303\\
56	0.19182	12.2308788936839\\
56	0.19548	13.0701692711693\\
56	0.19914	13.9350036350866\\
56	0.2028	14.8253819854359\\
56	0.20646	15.7413043222171\\
56	0.21012	16.6827706454301\\
56	0.21378	17.6497809550751\\
56	0.21744	18.642335251152\\
56	0.2211	19.6604335336609\\
56	0.22476	20.7040758026016\\
56	0.22842	21.7732620579742\\
56	0.23208	22.8679922997788\\
56	0.23574	23.9882665280153\\
56	0.2394	25.1340847426837\\
56	0.24306	26.3054469437839\\
56	0.24672	27.5023531313161\\
56	0.25038	28.7248033052803\\
56	0.25404	29.9727974656763\\
56	0.2577	31.2463356125043\\
56	0.26136	32.5454177457641\\
56	0.26502	33.8700438654559\\
56	0.26868	35.2202139715796\\
56	0.27234	36.5959280641352\\
56	0.276	37.9971861431227\\
56.375	0.093	-0.89511921239191\\
56.375	0.09666	-0.738288001408483\\
56.375	0.10032	-0.555912803993127\\
56.375	0.10398	-0.347993620145862\\
56.375	0.10764	-0.114530449866697\\
56.375	0.1113	0.144476706844383\\
56.375	0.11496	0.429027849987403\\
56.375	0.11862	0.739122979562307\\
56.375	0.12228	1.07476209556914\\
56.375	0.12594	1.43594519800788\\
56.375	0.1296	1.82267228687852\\
56.375	0.13326	2.23494336218108\\
56.375	0.13692	2.67275842391557\\
56.375	0.14058	3.13611747208196\\
56.375	0.14424	3.62502050668026\\
56.375	0.1479	4.13946752771048\\
56.375	0.15156	4.67945853517261\\
56.375	0.15522	5.24499352906665\\
56.375	0.15888	5.83607250939261\\
56.375	0.16254	6.45269547615048\\
56.375	0.1662	7.09486242934026\\
56.375	0.16986	7.76257336896195\\
56.375	0.17352	8.45582829501556\\
56.375	0.17718	9.17462720750109\\
56.375	0.18084	9.91897010641852\\
56.375	0.1845	10.6888569917679\\
56.375	0.18816	11.4842878635491\\
56.375	0.19182	12.3052627217623\\
56.375	0.19548	13.1517815664074\\
56.375	0.19914	14.0238443974844\\
56.375	0.2028	14.9214512149933\\
56.375	0.20646	15.8446020189341\\
56.375	0.21012	16.7932968093069\\
56.375	0.21378	17.7675355861115\\
56.375	0.21744	18.7673183493481\\
56.375	0.2211	19.7926450990166\\
56.375	0.22476	20.8435158351169\\
56.375	0.22842	21.9199305576492\\
56.375	0.23208	23.0218892666134\\
56.375	0.23574	24.1493919620096\\
56.375	0.2394	25.3024386438376\\
56.375	0.24306	26.4810293120976\\
56.375	0.24672	27.6851639667895\\
56.375	0.25038	28.9148426079132\\
56.375	0.25404	30.1700652354689\\
56.375	0.2577	31.4508318494565\\
56.375	0.26136	32.757142449876\\
56.375	0.26502	34.0889970367275\\
56.375	0.26868	35.4463956100108\\
56.375	0.27234	36.8293381697261\\
56.375	0.276	38.2378247158733\\
56.75	0.093	-1.01353532002414\\
56.75	0.09666	-0.849475641881041\\
56.75	0.10032	-0.659871977306043\\
56.75	0.10398	-0.444724326299129\\
56.75	0.10764	-0.204032688860279\\
56.75	0.1113	0.0622029350104576\\
56.75	0.11496	0.353982545313105\\
56.75	0.11862	0.67130614204768\\
56.75	0.12228	1.01417372521417\\
56.75	0.12594	1.38258529481255\\
56.75	0.1296	1.77654085084288\\
56.75	0.13326	2.1960403933051\\
56.75	0.13692	2.64108392219922\\
56.75	0.14058	3.11167143752527\\
56.75	0.14424	3.60780293928326\\
56.75	0.1479	4.12947842747312\\
56.75	0.15156	4.67669790209491\\
56.75	0.15522	5.2494613631486\\
56.75	0.15888	5.84776881063422\\
56.75	0.16254	6.47162024455175\\
56.75	0.1662	7.12101566490119\\
56.75	0.16986	7.79595507168253\\
56.75	0.17352	8.4964384648958\\
56.75	0.17718	9.22246584454098\\
56.75	0.18084	9.97403721061808\\
56.75	0.1845	10.7511525631271\\
56.75	0.18816	11.553811902068\\
56.75	0.19182	12.3820152274408\\
56.75	0.19548	13.2357625392456\\
56.75	0.19914	14.1150538374822\\
56.75	0.2028	15.0198891221508\\
56.75	0.20646	15.9502683932513\\
56.75	0.21012	16.9061916507837\\
56.75	0.21378	17.887658894748\\
56.75	0.21744	18.8946701251442\\
56.75	0.2211	19.9272253419723\\
56.75	0.22476	20.9853245452324\\
56.75	0.22842	22.0689677349243\\
56.75	0.23208	23.1781549110482\\
56.75	0.23574	24.312886073604\\
56.75	0.2394	25.4731612225917\\
56.75	0.24306	26.6589803580113\\
56.75	0.24672	27.8703434798628\\
56.75	0.25038	29.1072505881463\\
56.75	0.25404	30.3697016828617\\
56.75	0.2577	31.6576967640089\\
56.75	0.26136	32.9712358315881\\
56.75	0.26502	34.3103188855992\\
56.75	0.26868	35.6749459260421\\
56.75	0.27234	37.0651169529171\\
56.75	0.276	38.4808319662239\\
57.125	0.093	-1.12958275005631\\
57.125	0.09666	-0.958294604753545\\
57.125	0.10032	-0.761462473018884\\
57.125	0.10398	-0.539086354852321\\
57.125	0.10764	-0.291166250253815\\
57.125	0.1113	-0.0177021592234148\\
57.125	0.11496	0.281305918238889\\
57.125	0.11862	0.605857982133127\\
57.125	0.12228	0.955954032459271\\
57.125	0.12594	1.33159406921731\\
57.125	0.1296	1.73277809240729\\
57.125	0.13326	2.15950610202917\\
57.125	0.13692	2.61177809808295\\
57.125	0.14058	3.08959408056866\\
57.125	0.14424	3.59295404948631\\
57.125	0.1479	4.12185800483582\\
57.125	0.15156	4.67630594661727\\
57.125	0.15522	5.25629787483062\\
57.125	0.15888	5.8618337894759\\
57.125	0.16254	6.49291369055308\\
57.125	0.1662	7.14953757806218\\
57.125	0.16986	7.83170545200318\\
57.125	0.17352	8.53941731237611\\
57.125	0.17718	9.27267315918095\\
57.125	0.18084	10.0314729924177\\
57.125	0.1845	10.8158168120864\\
57.125	0.18816	11.6257046181869\\
57.125	0.19182	12.4611364107194\\
57.125	0.19548	13.3221121896838\\
57.125	0.19914	14.2086319550801\\
57.125	0.2028	15.1206957069084\\
57.125	0.20646	16.0583034451685\\
57.125	0.21012	17.0214551698606\\
57.125	0.21378	18.0101508809845\\
57.125	0.21744	19.0243905785404\\
57.125	0.2211	20.0641742625282\\
57.125	0.22476	21.1295019329479\\
57.125	0.22842	22.2203735897995\\
57.125	0.23208	23.336789233083\\
57.125	0.23574	24.4787488627985\\
57.125	0.2394	25.6462524789459\\
57.125	0.24306	26.8393000815251\\
57.125	0.24672	28.0578916705363\\
57.125	0.25038	29.3020272459794\\
57.125	0.25404	30.5717068078544\\
57.125	0.2577	31.8669303561614\\
57.125	0.26136	33.1876978909002\\
57.125	0.26502	34.5340094120709\\
57.125	0.26868	35.9058649196736\\
57.125	0.27234	37.3032644137081\\
57.125	0.276	38.7262078941746\\
57.5	0.093	-1.24326150248837\\
57.5	0.09666	-1.06474489002598\\
57.5	0.10032	-0.860684291131644\\
57.5	0.10398	-0.631079705805403\\
57.5	0.10764	-0.375931134047269\\
57.5	0.1113	-0.0952385758572127\\
57.5	0.11496	0.210997968764783\\
57.5	0.11862	0.542778499818656\\
57.5	0.12228	0.900103017304456\\
57.5	0.12594	1.28297152122218\\
57.5	0.1296	1.69138401157178\\
57.5	0.13326	2.12534048835333\\
57.5	0.13692	2.58484095156678\\
57.5	0.14058	3.06988540121215\\
57.5	0.14424	3.58047383728942\\
57.5	0.1479	4.11660625979862\\
57.5	0.15156	4.67828266873972\\
57.5	0.15522	5.26550306411272\\
57.5	0.15888	5.87826744591766\\
57.5	0.16254	6.5165758141545\\
57.5	0.1662	7.18042816882327\\
57.5	0.16986	7.86982450992392\\
57.5	0.17352	8.58476483745651\\
57.5	0.17718	9.325249151421\\
57.5	0.18084	10.0912774518174\\
57.5	0.1845	10.8828497386457\\
57.5	0.18816	11.699966011906\\
57.5	0.19182	12.5426262715981\\
57.5	0.19548	13.4108305177222\\
57.5	0.19914	14.3045787502781\\
57.5	0.2028	15.223870969266\\
57.5	0.20646	16.1687071746858\\
57.5	0.21012	17.1390873665375\\
57.5	0.21378	18.1350115448212\\
57.5	0.21744	19.1564797095367\\
57.5	0.2211	20.2034918606841\\
57.5	0.22476	21.2760479982635\\
57.5	0.22842	22.3741481222748\\
57.5	0.23208	23.497792232718\\
57.5	0.23574	24.6469803295931\\
57.5	0.2394	25.8217124129001\\
57.5	0.24306	27.021988482639\\
57.5	0.24672	28.2478085388099\\
57.5	0.25038	29.4991725814126\\
57.5	0.25404	30.7760806104473\\
57.5	0.2577	32.0785326259139\\
57.5	0.26136	33.4065286278123\\
57.5	0.26502	34.7600686161428\\
57.5	0.26868	36.1391525909051\\
57.5	0.27234	37.5437805520993\\
57.5	0.276	38.9739524997255\\
57.875	0.093	-1.35457157732037\\
57.875	0.09666	-1.16882649769831\\
57.875	0.10032	-0.957537431644319\\
57.875	0.10398	-0.720704379158429\\
57.875	0.10764	-0.458327340240631\\
57.875	0.1113	-0.170406314890919\\
57.875	0.11496	0.143058696890733\\
57.875	0.11862	0.482067695104263\\
57.875	0.12228	0.846620679749726\\
57.875	0.12594	1.2367176508271\\
57.875	0.1296	1.65235860833637\\
57.875	0.13326	2.09354355227757\\
57.875	0.13692	2.56027248265068\\
57.875	0.14058	3.05254539945571\\
57.875	0.14424	3.57036230269264\\
57.875	0.1479	4.11372319236148\\
57.875	0.15156	4.68262806846225\\
57.875	0.15522	5.27707693099492\\
57.875	0.15888	5.89706977995951\\
57.875	0.16254	6.54260661535601\\
57.875	0.1662	7.21368743718443\\
57.875	0.16986	7.91031224544474\\
57.875	0.17352	8.63248104013698\\
57.875	0.17718	9.38019382126114\\
57.875	0.18084	10.1534505888172\\
57.875	0.1845	10.9522513428052\\
57.875	0.18816	11.7765960832251\\
57.875	0.19182	12.6264848100769\\
57.875	0.19548	13.5019175233606\\
57.875	0.19914	14.4028942230762\\
57.875	0.2028	15.3294149092238\\
57.875	0.20646	16.2814795818032\\
57.875	0.21012	17.2590882408146\\
57.875	0.21378	18.2622408862579\\
57.875	0.21744	19.2909375181331\\
57.875	0.2211	20.3451781364402\\
57.875	0.22476	21.4249627411792\\
57.875	0.22842	22.5302913323501\\
57.875	0.23208	23.661163909953\\
57.875	0.23574	24.8175804739877\\
57.875	0.2394	25.9995410244544\\
57.875	0.24306	27.207045561353\\
57.875	0.24672	28.4400940846835\\
57.875	0.25038	29.6986865944459\\
57.875	0.25404	30.9828230906402\\
57.875	0.2577	32.2925035732665\\
57.875	0.26136	33.6277280423246\\
57.875	0.26502	34.9884964978147\\
57.875	0.26868	36.3748089397366\\
57.875	0.27234	37.7866653680905\\
57.875	0.276	39.2240657828764\\
58.25	0.093	-1.46351297455227\\
58.25	0.09666	-1.27053942777054\\
58.25	0.10032	-1.0520218945569\\
58.25	0.10398	-0.807960374911373\\
58.25	0.10764	-0.538354868833883\\
58.25	0.1113	-0.243205376324514\\
58.25	0.11496	0.0774881026167655\\
58.25	0.11862	0.423725567989973\\
58.25	0.12228	0.795507019795085\\
58.25	0.12594	1.1928324580321\\
58.25	0.1296	1.61570188270106\\
58.25	0.13326	2.0641152938019\\
58.25	0.13692	2.53807269133466\\
58.25	0.14058	3.03757407529935\\
58.25	0.14424	3.56261944569596\\
58.25	0.1479	4.11320880252445\\
58.25	0.15156	4.68934214578487\\
58.25	0.15522	5.2910194754772\\
58.25	0.15888	5.91824079160144\\
58.25	0.16254	6.57100609415761\\
58.25	0.1662	7.24931538314568\\
58.25	0.16986	7.95316865856566\\
58.25	0.17352	8.68256592041755\\
58.25	0.17718	9.43750716870136\\
58.25	0.18084	10.2179924034171\\
58.25	0.1845	11.0240216245647\\
58.25	0.18816	11.8555948321443\\
58.25	0.19182	12.7127120261557\\
58.25	0.19548	13.5953732065991\\
58.25	0.19914	14.5035783734744\\
58.25	0.2028	15.4373275267816\\
58.25	0.20646	16.3966206665207\\
58.25	0.21012	17.3814577926918\\
58.25	0.21378	18.3918389052947\\
58.25	0.21744	19.4277640043295\\
58.25	0.2211	20.4892330897963\\
58.25	0.22476	21.576246161695\\
58.25	0.22842	22.6888032200256\\
58.25	0.23208	23.8269042647881\\
58.25	0.23574	24.9905492959825\\
58.25	0.2394	26.1797383136088\\
58.25	0.24306	27.394471317667\\
58.25	0.24672	28.6347483081572\\
58.25	0.25038	29.9005692850793\\
58.25	0.25404	31.1919342484333\\
58.25	0.2577	32.5088431982192\\
58.25	0.26136	33.851296134437\\
58.25	0.26502	35.2192930570867\\
58.25	0.26868	36.6128339661683\\
58.25	0.27234	38.0319188616819\\
58.25	0.276	39.4765477436273\\
58.625	0.093	-1.57008569418409\\
58.625	0.09666	-1.36988368024271\\
58.625	0.10032	-1.14413767986942\\
58.625	0.10398	-0.892847693064224\\
58.625	0.10764	-0.616013719827086\\
58.625	0.1113	-0.313635760158054\\
58.625	0.11496	0.0142861859428827\\
58.625	0.11862	0.367752118475746\\
58.625	0.12228	0.746762037440522\\
58.625	0.12594	1.1513159428372\\
58.625	0.1296	1.58141383466581\\
58.625	0.13326	2.03705571292631\\
58.625	0.13692	2.51824157761872\\
58.625	0.14058	3.02497142874306\\
58.625	0.14424	3.55724526629934\\
58.625	0.1479	4.11506309028749\\
58.625	0.15156	4.69842490070756\\
58.625	0.15522	5.30733069755955\\
58.625	0.15888	5.94178048084346\\
58.625	0.16254	6.60177425055928\\
58.625	0.1662	7.28731200670701\\
58.625	0.16986	7.99839374928664\\
58.625	0.17352	8.7350194782982\\
58.625	0.17718	9.49718919374167\\
58.625	0.18084	10.284902895617\\
58.625	0.1845	11.0981605839243\\
58.625	0.18816	11.9369622586635\\
58.625	0.19182	12.8013079198347\\
58.625	0.19548	13.6911975674377\\
58.625	0.19914	14.6066312014726\\
58.625	0.2028	15.5476088219395\\
58.625	0.20646	16.5141304288383\\
58.625	0.21012	17.506196022169\\
58.625	0.21378	18.5238056019316\\
58.625	0.21744	19.5669591681261\\
58.625	0.2211	20.6356567207525\\
58.625	0.22476	21.7298982598108\\
58.625	0.22842	22.8496837853011\\
58.625	0.23208	23.9950132972232\\
58.625	0.23574	25.1658867955773\\
58.625	0.2394	26.3623042803633\\
58.625	0.24306	27.5842657515812\\
58.625	0.24672	28.831771209231\\
58.625	0.25038	30.1048206533127\\
58.625	0.25404	31.4034140838264\\
58.625	0.2577	32.7275515007719\\
58.625	0.26136	34.0772329041494\\
58.625	0.26502	35.4524582939588\\
58.625	0.26868	36.8532276702\\
58.625	0.27234	38.2795410328733\\
58.625	0.276	39.7313983819784\\
59	0.093	-1.67428973621583\\
59	0.09666	-1.46685925511481\\
59	0.10032	-1.23388478758184\\
59	0.10398	-0.97536633361697\\
59	0.10764	-0.691303893220203\\
59	0.1113	-0.381697466391515\\
59	0.11496	-0.0465470531308938\\
59	0.11862	0.314147346561612\\
59	0.12228	0.700385732686044\\
59	0.12594	1.1121681052424\\
59	0.1296	1.54949446423064\\
59	0.13326	2.01236480965081\\
59	0.13692	2.50077914150289\\
59	0.14058	3.0147374597869\\
59	0.14424	3.5542397645028\\
59	0.1479	4.11928605565062\\
59	0.15156	4.70987633323036\\
59	0.15522	5.326010597242\\
59	0.15888	5.96768884768556\\
59	0.16254	6.63491108456103\\
59	0.1662	7.32767730786843\\
59	0.16986	8.04598751760771\\
59	0.17352	8.78984171377893\\
59	0.17718	9.55923989638205\\
59	0.18084	10.3541820654171\\
59	0.1845	11.174668220884\\
59	0.18816	12.0206983627829\\
59	0.19182	12.8922724911137\\
59	0.19548	13.7893906058764\\
59	0.19914	14.712052707071\\
59	0.2028	15.6602587946975\\
59	0.20646	16.6340088687559\\
59	0.21012	17.6333029292463\\
59	0.21378	18.6581409761685\\
59	0.21744	19.7085230095227\\
59	0.2211	20.7844490293088\\
59	0.22476	21.8859190355268\\
59	0.22842	23.0129330281767\\
59	0.23208	24.1654910072585\\
59	0.23574	25.3435929727722\\
59	0.2394	26.5472389247179\\
59	0.24306	27.7764288630954\\
59	0.24672	29.0311627879049\\
59	0.25038	30.3114406991463\\
59	0.25404	31.6172625968196\\
59	0.2577	32.9486284809248\\
59	0.26136	34.3055383514619\\
59	0.26502	35.687992208431\\
59	0.26868	37.0959900518319\\
59	0.27234	38.5295318816648\\
59	0.276	39.9886176979295\\
59.375	0.093	-1.7761251006475\\
59.375	0.09666	-1.56146615238682\\
59.375	0.10032	-1.3212632176942\\
59.375	0.10398	-1.05551629656967\\
59.375	0.10764	-0.764225389013236\\
59.375	0.1113	-0.447390495024891\\
59.375	0.11496	-0.105011614604614\\
59.375	0.11862	0.262911252247555\\
59.375	0.12228	0.656378105531644\\
59.375	0.12594	1.07538894524765\\
59.375	0.1296	1.51994377139555\\
59.375	0.13326	1.99004258397538\\
59.375	0.13692	2.48568538298712\\
59.375	0.14058	3.00687216843078\\
59.375	0.14424	3.55360294030634\\
59.375	0.1479	4.12587769861382\\
59.375	0.15156	4.72369644335322\\
59.375	0.15522	5.34705917452451\\
59.375	0.15888	5.99596589212774\\
59.375	0.16254	6.67041659616287\\
59.375	0.1662	7.37041128662992\\
59.375	0.16986	8.09594996352886\\
59.375	0.17352	8.84703262685974\\
59.375	0.17718	9.62365927662252\\
59.375	0.18084	10.4258299128172\\
59.375	0.1845	11.2535445354438\\
59.375	0.18816	12.1068031445023\\
59.375	0.19182	12.9856057399928\\
59.375	0.19548	13.8899523219151\\
59.375	0.19914	14.8198428902694\\
59.375	0.2028	15.7752774450556\\
59.375	0.20646	16.7562559862737\\
59.375	0.21012	17.7627785139237\\
59.375	0.21378	18.7948450280056\\
59.375	0.21744	19.8524555285194\\
59.375	0.2211	20.9356100154651\\
59.375	0.22476	22.0443084888428\\
59.375	0.22842	23.1785509486523\\
59.375	0.23208	24.3383373948938\\
59.375	0.23574	25.5236678275672\\
59.375	0.2394	26.7345422466725\\
59.375	0.24306	27.9709606522097\\
59.375	0.24672	29.2329230441789\\
59.375	0.25038	30.5204294225799\\
59.375	0.25404	31.8334797874129\\
59.375	0.2577	33.1720741386777\\
59.375	0.26136	34.5362124763745\\
59.375	0.26502	35.9258948005032\\
59.375	0.26868	37.3411211110638\\
59.375	0.27234	38.7818914080563\\
59.375	0.276	40.2482056914808\\
59.75	0.093	-1.87559178747907\\
59.75	0.09666	-1.65370437205871\\
59.75	0.10032	-1.40627297020644\\
59.75	0.10398	-1.13329758192227\\
59.75	0.10764	-0.834778207206165\\
59.75	0.1113	-0.510714846058157\\
59.75	0.11496	-0.161107498478252\\
59.75	0.11862	0.214043835533587\\
59.75	0.12228	0.614739155977333\\
59.75	0.12594	1.04097846285298\\
59.75	0.1296	1.49276175616057\\
59.75	0.13326	1.97008903590004\\
59.75	0.13692	2.47296030207143\\
59.75	0.14058	3.00137555467475\\
59.75	0.14424	3.55533479371\\
59.75	0.1479	4.13483801917711\\
59.75	0.15156	4.73988523107617\\
59.75	0.15522	5.37047642940712\\
59.75	0.15888	6.02661161417\\
59.75	0.16254	6.7082907853648\\
59.75	0.1662	7.4155139429915\\
59.75	0.16986	8.1482810870501\\
59.75	0.17352	8.90659221754063\\
59.75	0.17718	9.69044733446308\\
59.75	0.18084	10.4998464378174\\
59.75	0.1845	11.3347895276037\\
59.75	0.18816	12.1952766038219\\
59.75	0.19182	13.081307666472\\
59.75	0.19548	13.992882715554\\
59.75	0.19914	14.9300017510679\\
59.75	0.2028	15.8926647730137\\
59.75	0.20646	16.8808717813915\\
59.75	0.21012	17.8946227762011\\
59.75	0.21378	18.9339177574427\\
59.75	0.21744	19.9987567251162\\
59.75	0.2211	21.0891396792216\\
59.75	0.22476	22.2050666197589\\
59.75	0.22842	23.3465375467281\\
59.75	0.23208	24.5135524601292\\
59.75	0.23574	25.7061113599623\\
59.75	0.2394	26.9242142462273\\
59.75	0.24306	28.1678611189241\\
59.75	0.24672	29.4370519780529\\
59.75	0.25038	30.7317868236136\\
59.75	0.25404	32.0520656556063\\
59.75	0.2577	33.3978884740308\\
59.75	0.26136	34.7692552788872\\
59.75	0.26502	36.1661660701756\\
59.75	0.26868	37.5886208478958\\
59.75	0.27234	39.036619612048\\
59.75	0.276	40.5101623626321\\
60.125	0.093	-1.97268979671057\\
60.125	0.09666	-1.74357391413055\\
60.125	0.10032	-1.48891404511862\\
60.125	0.10398	-1.2087101896748\\
60.125	0.10764	-0.902962347799031\\
60.125	0.1113	-0.571670519491374\\
60.125	0.11496	-0.214834704751805\\
60.125	0.11862	0.167545096419691\\
60.125	0.12228	0.575468884023099\\
60.125	0.12594	1.00893665805841\\
60.125	0.1296	1.46794841852565\\
60.125	0.13326	1.95250416542478\\
60.125	0.13692	2.46260389875582\\
60.125	0.14058	2.9982476185188\\
60.125	0.14424	3.5594353247137\\
60.125	0.1479	4.14616701734048\\
60.125	0.15156	4.75844269639919\\
60.125	0.15522	5.39626236188981\\
60.125	0.15888	6.05962601381234\\
60.125	0.16254	6.7485336521668\\
60.125	0.1662	7.46298527695316\\
60.125	0.16986	8.20298088817141\\
60.125	0.17352	8.9685204858216\\
60.125	0.17718	9.75960406990371\\
60.125	0.18084	10.5762316404177\\
60.125	0.1845	11.4184031973636\\
60.125	0.18816	12.2861187407415\\
60.125	0.19182	13.1793782705512\\
60.125	0.19548	14.0981817867929\\
60.125	0.19914	15.0425292894665\\
60.125	0.2028	16.012420778572\\
60.125	0.20646	17.0078562541094\\
60.125	0.21012	18.0288357160787\\
60.125	0.21378	19.0753591644799\\
60.125	0.21744	20.147426599313\\
60.125	0.2211	21.2450380205781\\
60.125	0.22476	22.3681934282751\\
60.125	0.22842	23.516892822404\\
60.125	0.23208	24.6911362029647\\
60.125	0.23574	25.8909235699575\\
60.125	0.2394	27.1162549233821\\
60.125	0.24306	28.3671302632386\\
60.125	0.24672	29.643549589527\\
60.125	0.25038	30.9455129022474\\
60.125	0.25404	32.2730202013997\\
60.125	0.2577	33.6260714869839\\
60.125	0.26136	35.0046667589999\\
60.125	0.26502	36.4088060174479\\
60.125	0.26868	37.8384892623279\\
60.125	0.27234	39.2937164936397\\
60.125	0.276	40.7744877113835\\
60.5	0.093	-2.06741912834199\\
60.5	0.09666	-1.83107477860231\\
60.5	0.10032	-1.56918644243074\\
60.5	0.10398	-1.28175411982725\\
60.5	0.10764	-0.968777810791819\\
60.5	0.1113	-0.630257515324505\\
60.5	0.11496	-0.26619323342528\\
60.5	0.11862	0.123415034905872\\
60.5	0.12228	0.538567289668936\\
60.5	0.12594	0.979263530863907\\
60.5	0.1296	1.44550375849081\\
60.5	0.13326	1.9372879725496\\
60.5	0.13692	2.4546161730403\\
60.5	0.14058	2.99748835996293\\
60.5	0.14424	3.56590453331749\\
60.5	0.1479	4.15986469310393\\
60.5	0.15156	4.7793688393223\\
60.5	0.15522	5.42441697197257\\
60.5	0.15888	6.09500909105476\\
60.5	0.16254	6.79114519656887\\
60.5	0.1662	7.51282528851489\\
60.5	0.16986	8.26004936689281\\
60.5	0.17352	9.03281743170266\\
60.5	0.17718	9.83112948294442\\
60.5	0.18084	10.6549855206181\\
60.5	0.1845	11.5043855447237\\
60.5	0.18816	12.3793295552612\\
60.5	0.19182	13.2798175522306\\
60.5	0.19548	14.2058495356319\\
60.5	0.19914	15.1574255054651\\
60.5	0.2028	16.1345454617303\\
60.5	0.20646	17.1372094044273\\
60.5	0.21012	18.1654173335563\\
60.5	0.21378	19.2191692491172\\
60.5	0.21744	20.29846515111\\
60.5	0.2211	21.4033050395347\\
60.5	0.22476	22.5336889143913\\
60.5	0.22842	23.6896167756798\\
60.5	0.23208	24.8710886234003\\
60.5	0.23574	26.0781044575527\\
60.5	0.2394	27.310664278137\\
60.5	0.24306	28.5687680851532\\
60.5	0.24672	29.8524158786013\\
60.5	0.25038	31.1616076584813\\
60.5	0.25404	32.4963434247932\\
60.5	0.2577	33.856623177537\\
60.5	0.26136	35.2424469167128\\
60.5	0.26502	36.6538146423205\\
60.5	0.26868	38.09072635436\\
60.5	0.27234	39.5531820528315\\
60.5	0.276	41.041181737735\\
60.875	0.093	-2.15977978237331\\
60.875	0.09666	-1.91620696547399\\
60.875	0.10032	-1.64709016214274\\
60.875	0.10398	-1.35242937237958\\
60.875	0.10764	-1.03222459618453\\
60.875	0.1113	-0.686475833557552\\
60.875	0.11496	-0.315183084498642\\
60.875	0.11862	0.081653650992159\\
60.875	0.12228	0.50403437291488\\
60.875	0.12594	0.951959081269514\\
60.875	0.1296	1.42542777605605\\
60.875	0.13326	1.92444045727451\\
60.875	0.13692	2.44899712492488\\
60.875	0.14058	2.99909777900718\\
60.875	0.14424	3.57474241952136\\
60.875	0.1479	4.17593104646747\\
60.875	0.15156	4.8026636598455\\
60.875	0.15522	5.45494025965543\\
60.875	0.15888	6.13276084589728\\
60.875	0.16254	6.83612541857104\\
60.875	0.1662	7.56503397767673\\
60.875	0.16986	8.3194865232143\\
60.875	0.17352	9.0994830551838\\
60.875	0.17718	9.90502357358522\\
60.875	0.18084	10.7361080784186\\
60.875	0.1845	11.5927365696838\\
60.875	0.18816	12.4749090473809\\
60.875	0.19182	13.38262551151\\
60.875	0.19548	14.315885962071\\
60.875	0.19914	15.2746903990639\\
60.875	0.2028	16.2590388224887\\
60.875	0.20646	17.2689312323454\\
60.875	0.21012	18.304367628634\\
60.875	0.21378	19.3653480113546\\
60.875	0.21744	20.451872380507\\
60.875	0.2211	21.5639407360914\\
60.875	0.22476	22.7015530781077\\
60.875	0.22842	23.8647094065559\\
60.875	0.23208	25.053409721436\\
60.875	0.23574	26.267654022748\\
60.875	0.2394	27.507442310492\\
60.875	0.24306	28.7727745846678\\
60.875	0.24672	30.0636508452755\\
60.875	0.25038	31.3800710923152\\
60.875	0.25404	32.7220353257868\\
60.875	0.2577	34.0895435456903\\
60.875	0.26136	35.4825957520257\\
60.875	0.26502	36.9011919447931\\
60.875	0.26868	38.3453321239923\\
60.875	0.27234	39.8150162896234\\
60.875	0.276	41.3102444416865\\
61.25	0.093	-2.24977175880457\\
61.25	0.09666	-1.9989704747456\\
61.25	0.10032	-1.72262520425469\\
61.25	0.10398	-1.42073594733187\\
61.25	0.10764	-1.09330270397716\\
61.25	0.1113	-0.740325474190524\\
61.25	0.11496	-0.361804257971951\\
61.25	0.11862	0.0422609446784996\\
61.25	0.12228	0.471870133760884\\
61.25	0.12594	0.927023309275175\\
61.25	0.1296	1.40772047122136\\
61.25	0.13326	1.91396161959948\\
61.25	0.13692	2.44574675440951\\
61.25	0.14058	3.00307587565146\\
61.25	0.14424	3.58594898332531\\
61.25	0.1479	4.19436607743108\\
61.25	0.15156	4.82832715796876\\
61.25	0.15522	5.48783222493835\\
61.25	0.15888	6.17288127833985\\
61.25	0.16254	6.88347431817328\\
61.25	0.1662	7.61961134443861\\
61.25	0.16986	8.38129235713585\\
61.25	0.17352	9.16851735626501\\
61.25	0.17718	9.98128634182608\\
61.25	0.18084	10.8195993138191\\
61.25	0.1845	11.683456272244\\
61.25	0.18816	12.5728572171008\\
61.25	0.19182	13.4878021483895\\
61.25	0.19548	14.4282910661101\\
61.25	0.19914	15.3943239702627\\
61.25	0.2028	16.3859008608471\\
61.25	0.20646	17.4030217378635\\
61.25	0.21012	18.4456866013118\\
61.25	0.21378	19.513895451192\\
61.25	0.21744	20.6076482875041\\
61.25	0.2211	21.7269451102482\\
61.25	0.22476	22.8717859194241\\
61.25	0.22842	24.0421707150319\\
61.25	0.23208	25.2380994970717\\
61.25	0.23574	26.4595722655434\\
61.25	0.2394	27.706589020447\\
61.25	0.24306	28.9791497617825\\
61.25	0.24672	30.2772544895499\\
61.25	0.25038	31.6009032037492\\
61.25	0.25404	32.9500959043805\\
61.25	0.2577	34.3248325914437\\
61.25	0.26136	35.7251132649387\\
61.25	0.26502	37.1509379248657\\
61.25	0.26868	38.6023065712246\\
61.25	0.27234	40.0792192040154\\
61.25	0.276	41.5816758232381\\
61.625	0.093	-2.33739505763572\\
61.625	0.09666	-2.07936530641708\\
61.625	0.10032	-1.79579156876652\\
61.625	0.10398	-1.48667384468407\\
61.625	0.10764	-1.15201213416966\\
61.625	0.1113	-0.791806437223375\\
61.625	0.11496	-0.406056753845174\\
61.625	0.11862	0.00523691596494658\\
61.625	0.12228	0.442074572206987\\
61.625	0.12594	0.904456214880927\\
61.625	0.1296	1.3923818439868\\
61.625	0.13326	1.90585145952456\\
61.625	0.13692	2.44486506149424\\
61.625	0.14058	3.00942264989584\\
61.625	0.14424	3.59952422472938\\
61.625	0.1479	4.21516978599479\\
61.625	0.15156	4.85635933369213\\
61.625	0.15522	5.52309286782137\\
61.625	0.15888	6.21537038838254\\
61.625	0.16254	6.93319189537563\\
61.625	0.1662	7.67655738880061\\
61.625	0.16986	8.44546686865751\\
61.625	0.17352	9.23992033494632\\
61.625	0.17718	10.0599177876671\\
61.625	0.18084	10.9054592268197\\
61.625	0.1845	11.7765446524043\\
61.625	0.18816	12.6731740644207\\
61.625	0.19182	13.5953474628691\\
61.625	0.19548	14.5430648477494\\
61.625	0.19914	15.5163262190616\\
61.625	0.2028	16.5151315768057\\
61.625	0.20646	17.5394809209818\\
61.625	0.21012	18.5893742515897\\
61.625	0.21378	19.6648115686296\\
61.625	0.21744	20.7657928721013\\
61.625	0.2211	21.892318162005\\
61.625	0.22476	23.0443874383406\\
61.625	0.22842	24.2220007011081\\
61.625	0.23208	25.4251579503075\\
61.625	0.23574	26.6538591859389\\
61.625	0.2394	27.9081044080021\\
61.625	0.24306	29.1878936164973\\
61.625	0.24672	30.4932268114244\\
61.625	0.25038	31.8241039927834\\
61.625	0.25404	33.1805251605743\\
61.625	0.2577	34.5624903147971\\
61.625	0.26136	35.9699994554518\\
61.625	0.26502	37.4030525825385\\
61.625	0.26868	38.861649696057\\
61.625	0.27234	40.3457907960075\\
61.625	0.276	41.8554758823899\\
62	0.093	-2.42264967886682\\
62	0.09666	-2.15739146048851\\
62	0.10032	-1.8665892556783\\
62	0.10398	-1.55024306443618\\
62	0.10764	-1.20835288676212\\
62	0.1113	-0.840918722656177\\
62	0.11496	-0.44794057211832\\
62	0.11862	-0.0294184351485356\\
62	0.12228	0.414647688253162\\
62	0.12594	0.88425779808675\\
62	0.1296	1.37941189435228\\
62	0.13326	1.90010997704971\\
62	0.13692	2.44635204617904\\
62	0.14058	3.0181381017403\\
62	0.14424	3.61546814373349\\
62	0.1479	4.23834217215856\\
62	0.15156	4.88676018701556\\
62	0.15522	5.56072218830446\\
62	0.15888	6.26022817602529\\
62	0.16254	6.98527815017803\\
62	0.1662	7.73587211076268\\
62	0.16986	8.51201005777923\\
62	0.17352	9.31369199122771\\
62	0.17718	10.1409179111081\\
62	0.18084	10.9936878174204\\
62	0.1845	11.8720017101646\\
62	0.18816	12.7758595893407\\
62	0.19182	13.7052614549488\\
62	0.19548	14.6602073069887\\
62	0.19914	15.6406971454606\\
62	0.2028	16.6467309703644\\
62	0.20646	17.6783087817001\\
62	0.21012	18.7354305794677\\
62	0.21378	19.8180963636672\\
62	0.21744	20.9263061342986\\
62	0.2211	22.060059891362\\
62	0.22476	23.2193576348572\\
62	0.22842	24.4041993647844\\
62	0.23208	25.6145850811434\\
62	0.23574	26.8505147839345\\
62	0.2394	28.1119884731574\\
62	0.24306	29.3990061488122\\
62	0.24672	30.7115678108989\\
62	0.25038	32.0496734594176\\
62	0.25404	33.4133230943681\\
62	0.2577	34.8025167157506\\
62	0.26136	36.217254323565\\
62	0.26502	37.6575359178113\\
62	0.26868	39.1233614984895\\
62	0.27234	40.6147310655996\\
62	0.276	42.1316446191416\\
62.375	0.093	-2.5055356224978\\
62.375	0.09666	-2.23304893695986\\
62.375	0.10032	-1.93501826498998\\
62.375	0.10398	-1.61144360658819\\
62.375	0.10764	-1.2623249617545\\
62.375	0.1113	-0.887662330488894\\
62.375	0.11496	-0.487455712791352\\
62.375	0.11862	-0.0617051086619256\\
62.375	0.12228	0.389589481899428\\
62.375	0.12594	0.866428058892694\\
62.375	0.1296	1.36881062231786\\
62.375	0.13326	1.89673717217495\\
62.375	0.13692	2.45020770846395\\
62.375	0.14058	3.02922223118488\\
62.375	0.14424	3.6337807403377\\
62.375	0.1479	4.26388323592244\\
62.375	0.15156	4.9195297179391\\
62.375	0.15522	5.60072018638765\\
62.375	0.15888	6.30745464126814\\
62.375	0.16254	7.03973308258053\\
62.375	0.1662	7.79755551032484\\
62.375	0.16986	8.58092192450105\\
62.375	0.17352	9.38983232510918\\
62.375	0.17718	10.2242867121492\\
62.375	0.18084	11.0842850856212\\
62.375	0.1845	11.9698274455251\\
62.375	0.18816	12.8809137918608\\
62.375	0.19182	13.8175441246285\\
62.375	0.19548	14.7797184438282\\
62.375	0.19914	15.7674367494597\\
62.375	0.2028	16.7806990415231\\
62.375	0.20646	17.8195053200185\\
62.375	0.21012	18.8838555849457\\
62.375	0.21378	19.9737498363049\\
62.375	0.21744	21.089188074096\\
62.375	0.2211	22.230170298319\\
62.375	0.22476	23.3966965089739\\
62.375	0.22842	24.5887667060607\\
62.375	0.23208	25.8063808895794\\
62.375	0.23574	27.0495390595301\\
62.375	0.2394	28.3182412159127\\
62.375	0.24306	29.6124873587272\\
62.375	0.24672	30.9322774879736\\
62.375	0.25038	32.2776116036519\\
62.375	0.25404	33.6484897057621\\
62.375	0.2577	35.0449117943042\\
62.375	0.26136	36.4668778692782\\
62.375	0.26502	37.9143879306842\\
62.375	0.26868	39.3874419785221\\
62.375	0.27234	40.8860400127918\\
62.375	0.276	42.4101820334935\\
62.75	0.093	-2.58605288852874\\
62.75	0.09666	-2.30633773583113\\
62.75	0.10032	-2.00107859670159\\
62.75	0.10398	-1.67027547114014\\
62.75	0.10764	-1.31392835914679\\
62.75	0.1113	-0.932037260721529\\
62.75	0.11496	-0.524602175864331\\
62.75	0.11862	-0.0916231045752482\\
62.75	0.12228	0.366899953145769\\
62.75	0.12594	0.850966997298691\\
62.75	0.1296	1.36057802788351\\
62.75	0.13326	1.89573304490026\\
62.75	0.13692	2.45643204834893\\
62.75	0.14058	3.0426750382295\\
62.75	0.14424	3.65446201454198\\
62.75	0.1479	4.29179297728638\\
62.75	0.15156	4.95466792646269\\
62.75	0.15522	5.64308686207091\\
62.75	0.15888	6.35704978411106\\
62.75	0.16254	7.09655669258311\\
62.75	0.1662	7.86160758748707\\
62.75	0.16986	8.65220246882294\\
62.75	0.17352	9.46834133659073\\
62.75	0.17718	10.3100241907904\\
62.75	0.18084	11.1772510314221\\
62.75	0.1845	12.0700218584856\\
62.75	0.18816	12.988336671981\\
62.75	0.19182	13.9321954719084\\
62.75	0.19548	14.9015982582676\\
62.75	0.19914	15.8965450310588\\
62.75	0.2028	16.9170357902819\\
62.75	0.20646	17.9630705359369\\
62.75	0.21012	19.0346492680238\\
62.75	0.21378	20.1317719865427\\
62.75	0.21744	21.2544386914934\\
62.75	0.2211	22.4026493828761\\
62.75	0.22476	23.5764040606907\\
62.75	0.22842	24.7757027249371\\
62.75	0.23208	26.0005453756155\\
62.75	0.23574	27.2509320127258\\
62.75	0.2394	28.5268626362681\\
62.75	0.24306	29.8283372462422\\
62.75	0.24672	31.1553558426482\\
62.75	0.25038	32.5079184254862\\
62.75	0.25404	33.8860249947561\\
62.75	0.2577	35.2896755504579\\
62.75	0.26136	36.7188700925916\\
62.75	0.26502	38.1736086211572\\
62.75	0.26868	39.6538911361547\\
62.75	0.27234	41.1597176375842\\
62.75	0.276	42.6910881254455\\
63.125	0.093	-2.66420147695956\\
63.125	0.09666	-2.37725785710228\\
63.125	0.10032	-2.0647702508131\\
63.125	0.10398	-1.72673865809201\\
63.125	0.10764	-1.36316307893898\\
63.125	0.1113	-0.974043513354054\\
63.125	0.11496	-0.559379961337228\\
63.125	0.11862	-0.119172422888468\\
63.125	0.12228	0.346579101992198\\
63.125	0.12594	0.83787461330477\\
63.125	0.1296	1.35471411104927\\
63.125	0.13326	1.89709759522566\\
63.125	0.13692	2.46502506583397\\
63.125	0.14058	3.05849652287421\\
63.125	0.14424	3.67751196634637\\
63.125	0.1479	4.32207139625041\\
63.125	0.15156	4.99217481258639\\
63.125	0.15522	5.68782221535426\\
63.125	0.15888	6.40901360455406\\
63.125	0.16254	7.15574898018578\\
63.125	0.1662	7.9280283422494\\
63.125	0.16986	8.72585169074492\\
63.125	0.17352	9.54921902567237\\
63.125	0.17718	10.3981303470317\\
63.125	0.18084	11.272585654823\\
63.125	0.1845	12.1725849490462\\
63.125	0.18816	13.0981282297013\\
63.125	0.19182	14.0492154967883\\
63.125	0.19548	15.0258467503072\\
63.125	0.19914	16.0280219902581\\
63.125	0.2028	17.0557412166408\\
63.125	0.20646	18.1090044294555\\
63.125	0.21012	19.1878116287021\\
63.125	0.21378	20.2921628143806\\
63.125	0.21744	21.422057986491\\
63.125	0.2211	22.5774971450333\\
63.125	0.22476	23.7584802900075\\
63.125	0.22842	24.9650074214137\\
63.125	0.23208	26.1970785392517\\
63.125	0.23574	27.4546936435217\\
63.125	0.2394	28.7378527342236\\
63.125	0.24306	30.0465558113573\\
63.125	0.24672	31.380802874923\\
63.125	0.25038	32.7405939249207\\
63.125	0.25404	34.1259289613502\\
63.125	0.2577	35.5368079842117\\
63.125	0.26136	36.973230993505\\
63.125	0.26502	38.4351979892303\\
63.125	0.26868	39.9227089713875\\
63.125	0.27234	41.4357639399766\\
63.125	0.276	42.9743628949976\\
63.5	0.093	-2.73998138779032\\
63.5	0.09666	-2.44580930077338\\
63.5	0.10032	-2.12609322732454\\
63.5	0.10398	-1.78083316744379\\
63.5	0.10764	-1.4100291211311\\
63.5	0.1113	-1.01368108838652\\
63.5	0.11496	-0.591789069210037\\
63.5	0.11862	-0.144353063601621\\
63.5	0.12228	0.328626928438709\\
63.5	0.12594	0.82715090691093\\
63.5	0.1296	1.3512188718151\\
63.5	0.13326	1.90083082315115\\
63.5	0.13692	2.47598676091911\\
63.5	0.14058	3.076686685119\\
63.5	0.14424	3.70293059575083\\
63.5	0.1479	4.35471849281452\\
63.5	0.15156	5.03205037631016\\
63.5	0.15522	5.73492624623769\\
63.5	0.15888	6.46334610259715\\
63.5	0.16254	7.21730994538851\\
63.5	0.1662	7.9968177746118\\
63.5	0.16986	8.80186959026698\\
63.5	0.17352	9.63246539235409\\
63.5	0.17718	10.4886051808731\\
63.5	0.18084	11.370288955824\\
63.5	0.1845	12.2775167172069\\
63.5	0.18816	13.2102884650216\\
63.5	0.19182	14.1686041992683\\
63.5	0.19548	15.1524639199469\\
63.5	0.19914	16.1618676270574\\
63.5	0.2028	17.1968153205998\\
63.5	0.20646	18.2573070005741\\
63.5	0.21012	19.3433426669804\\
63.5	0.21378	20.4549223198185\\
63.5	0.21744	21.5920459590885\\
63.5	0.2211	22.7547135847905\\
63.5	0.22476	23.9429251969244\\
63.5	0.22842	25.1566807954902\\
63.5	0.23208	26.3959803804879\\
63.5	0.23574	27.6608239519176\\
63.5	0.2394	28.9512115097791\\
63.5	0.24306	30.2671430540726\\
63.5	0.24672	31.6086185847979\\
63.5	0.25038	32.9756381019552\\
63.5	0.25404	34.3682016055444\\
63.5	0.2577	35.7863090955655\\
63.5	0.26136	37.2299605720185\\
63.5	0.26502	38.6991560349035\\
63.5	0.26868	40.1938954842202\\
63.5	0.27234	41.714178919969\\
63.5	0.276	43.2600063421497\\
63.875	0.093	-2.81339262102098\\
63.875	0.09666	-2.5119920668444\\
63.875	0.10032	-2.18504752623589\\
63.875	0.10398	-1.83255899919547\\
63.875	0.10764	-1.45452648572315\\
63.875	0.1113	-1.05094998581891\\
63.875	0.11496	-0.621829499482743\\
63.875	0.11862	-0.167165026714684\\
63.875	0.12228	0.313043432485301\\
63.875	0.12594	0.8187958781172\\
63.875	0.1296	1.35009231018099\\
63.875	0.13326	1.90693272867672\\
63.875	0.13692	2.48931713360436\\
63.875	0.14058	3.09724552496391\\
63.875	0.14424	3.73071790275536\\
63.875	0.1479	4.38973426697872\\
63.875	0.15156	5.07429461763401\\
63.875	0.15522	5.7843989547212\\
63.875	0.15888	6.52004727824032\\
63.875	0.16254	7.28123958819135\\
63.875	0.1662	8.06797588457429\\
63.875	0.16986	8.88025616738913\\
63.875	0.17352	9.71808043663589\\
63.875	0.17718	10.5814486923146\\
63.875	0.18084	11.4703609344252\\
63.875	0.1845	12.3848171629677\\
63.875	0.18816	13.3248173779421\\
63.875	0.19182	14.2903615793484\\
63.875	0.19548	15.2814497671867\\
63.875	0.19914	16.2980819414568\\
63.875	0.2028	17.3402581021589\\
63.875	0.20646	18.4079782492929\\
63.875	0.21012	19.5012423828587\\
63.875	0.21378	20.6200505028565\\
63.875	0.21744	21.7644026092863\\
63.875	0.2211	22.9342987021479\\
63.875	0.22476	24.1297387814414\\
63.875	0.22842	25.3507228471669\\
63.875	0.23208	26.5972508993243\\
63.875	0.23574	27.8693229379136\\
63.875	0.2394	29.1669389629348\\
63.875	0.24306	30.4900989743878\\
63.875	0.24672	31.8388029722729\\
63.875	0.25038	33.2130509565898\\
63.875	0.25404	34.6128429273387\\
63.875	0.2577	36.0381788845194\\
63.875	0.26136	37.4890588281321\\
63.875	0.26502	38.9654827581767\\
63.875	0.26868	40.4674506746532\\
63.875	0.27234	41.9949625775616\\
63.875	0.276	43.5480184669019\\
64.25	0.093	-2.88443517665157\\
64.25	0.09666	-2.57580615531534\\
64.25	0.10032	-2.24163314754716\\
64.25	0.10398	-1.88191615334709\\
64.25	0.10764	-1.49665517271511\\
64.25	0.1113	-1.08585020565122\\
64.25	0.11496	-0.649501252155389\\
64.25	0.11862	-0.187608312227667\\
64.25	0.12228	0.299828614131975\\
64.25	0.12594	0.81280952692353\\
64.25	0.1296	1.35133442614698\\
64.25	0.13326	1.91540331180236\\
64.25	0.13692	2.50501618388965\\
64.25	0.14058	3.12017304240886\\
64.25	0.14424	3.76087388735997\\
64.25	0.1479	4.427118718743\\
64.25	0.15156	5.11890753655794\\
64.25	0.15522	5.83624034080479\\
64.25	0.15888	6.57911713148357\\
64.25	0.16254	7.34753790859425\\
64.25	0.1662	8.14150267213685\\
64.25	0.16986	8.96101142211135\\
64.25	0.17352	9.80606415851777\\
64.25	0.17718	10.6766608813561\\
64.25	0.18084	11.5728015906264\\
64.25	0.1845	12.4944862863285\\
64.25	0.18816	13.4417149684626\\
64.25	0.19182	14.4144876370286\\
64.25	0.19548	15.4128042920265\\
64.25	0.19914	16.4366649334563\\
64.25	0.2028	17.486069561318\\
64.25	0.20646	18.5610181756117\\
64.25	0.21012	19.6615107763372\\
64.25	0.21378	20.7875473634947\\
64.25	0.21744	21.939127937084\\
64.25	0.2211	23.1162524971053\\
64.25	0.22476	24.3189210435585\\
64.25	0.22842	25.5471335764436\\
64.25	0.23208	26.8008900957606\\
64.25	0.23574	28.0801906015096\\
64.25	0.2394	29.3850350936905\\
64.25	0.24306	30.7154235723032\\
64.25	0.24672	32.0713560373479\\
64.25	0.25038	33.4528324888245\\
64.25	0.25404	34.859852926733\\
64.25	0.2577	36.2924173510734\\
64.25	0.26136	37.7505257618458\\
64.25	0.26502	39.23417815905\\
64.25	0.26868	40.7433745426862\\
64.25	0.27234	42.2781149127542\\
64.25	0.276	43.8383992692542\\
64.625	0.093	-2.95310905468209\\
64.625	0.09666	-2.6372515661862\\
64.625	0.10032	-2.29585009125837\\
64.625	0.10398	-1.92890462989863\\
64.625	0.10764	-1.53641518210699\\
64.625	0.1113	-1.11838174788345\\
64.625	0.11496	-0.674804327227953\\
64.625	0.11862	-0.205682920140582\\
64.625	0.12228	0.288982473378717\\
64.625	0.12594	0.809191853329928\\
64.625	0.1296	1.35494521971304\\
64.625	0.13326	1.92624257252808\\
64.625	0.13692	2.52308391177503\\
64.625	0.14058	3.1454692374539\\
64.625	0.14424	3.79339854956466\\
64.625	0.1479	4.46687184810735\\
64.625	0.15156	5.16588913308195\\
64.625	0.15522	5.89045040448846\\
64.625	0.15888	6.64055566232689\\
64.625	0.16254	7.41620490659723\\
64.625	0.1662	8.21739813729949\\
64.625	0.16986	9.04413535443364\\
64.625	0.17352	9.89641655799972\\
64.625	0.17718	10.7742417479977\\
64.625	0.18084	11.6776109244276\\
64.625	0.1845	12.6065240872894\\
64.625	0.18816	13.5609812365832\\
64.625	0.19182	14.5409823723088\\
64.625	0.19548	15.5465274944664\\
64.625	0.19914	16.5776166030558\\
64.625	0.2028	17.6342496980772\\
64.625	0.20646	18.7164267795305\\
64.625	0.21012	19.8241478474157\\
64.625	0.21378	20.9574129017328\\
64.625	0.21744	22.1162219424819\\
64.625	0.2211	23.3005749696628\\
64.625	0.22476	24.5104719832757\\
64.625	0.22842	25.7459129833205\\
64.625	0.23208	27.0068979697972\\
64.625	0.23574	28.2934269427057\\
64.625	0.2394	29.6054999020463\\
64.625	0.24306	30.9431168478187\\
64.625	0.24672	32.306277780023\\
64.625	0.25038	33.6949826986593\\
64.625	0.25404	35.1092316037274\\
64.625	0.2577	36.5490244952275\\
64.625	0.26136	38.0143613731595\\
64.625	0.26502	39.5052422375234\\
64.625	0.26868	41.0216670883192\\
64.625	0.27234	42.5636359255469\\
64.625	0.276	44.1311487492066\\
65	0.093	-3.0194142551125\\
65	0.09666	-2.69632829945694\\
65	0.10032	-2.34769835736947\\
65	0.10398	-1.97352442885009\\
65	0.10764	-1.57380651389877\\
65	0.1113	-1.14854461251556\\
65	0.11496	-0.697738724700436\\
65	0.11862	-0.221388850453394\\
65	0.12228	0.280505010225568\\
65	0.12594	0.807942857336421\\
65	0.1296	1.36092469087922\\
65	0.13326	1.9394505108539\\
65	0.13692	2.54352031726049\\
65	0.14058	3.17313411009902\\
65	0.14424	3.82829188936947\\
65	0.1479	4.5089936550718\\
65	0.15156	5.21523940720606\\
65	0.15522	5.94702914577223\\
65	0.15888	6.70436287077032\\
65	0.16254	7.48724058220031\\
65	0.1662	8.29566228006223\\
65	0.16986	9.12962796435604\\
65	0.17352	9.98913763508178\\
65	0.17718	10.8741912922394\\
65	0.18084	11.784788935829\\
65	0.1845	12.7209305658505\\
65	0.18816	13.6826161823039\\
65	0.19182	14.6698457851891\\
65	0.19548	15.6826193745064\\
65	0.19914	16.7209369502555\\
65	0.2028	17.7847985124365\\
65	0.20646	18.8742040610495\\
65	0.21012	19.9891535960944\\
65	0.21378	21.1296471175711\\
65	0.21744	22.2956846254798\\
65	0.2211	23.4872661198204\\
65	0.22476	24.7043916005929\\
65	0.22842	25.9470610677974\\
65	0.23208	27.2152745214337\\
65	0.23574	28.509031961502\\
65	0.2394	29.8283333880022\\
65	0.24306	31.1731788009342\\
65	0.24672	32.5435682002982\\
65	0.25038	33.9395015860941\\
65	0.25404	35.360978958322\\
65	0.2577	36.8080003169817\\
65	0.26136	38.2805656620733\\
65	0.26502	39.7786749935969\\
65	0.26868	41.3023283115524\\
65	0.27234	42.8515256159397\\
65	0.276	44.426266906759\\
65.375	0.093	-3.08335077794285\\
65.375	0.09666	-2.75303635512763\\
65.375	0.10032	-2.39717794588049\\
65.375	0.10398	-2.01577555020146\\
65.375	0.10764	-1.60882916809048\\
65.375	0.1113	-1.17633879954762\\
65.375	0.11496	-0.718304444572837\\
65.375	0.11862	-0.234726103166132\\
65.375	0.12228	0.274396224672479\\
65.375	0.12594	0.809062538942996\\
65.375	0.1296	1.36927283964544\\
65.375	0.13326	1.95502712677979\\
65.375	0.13692	2.56632540034604\\
65.375	0.14058	3.20316766034422\\
65.375	0.14424	3.86555390677433\\
65.375	0.1479	4.55348413963632\\
65.375	0.15156	5.26695835893024\\
65.375	0.15522	6.00597656465606\\
65.375	0.15888	6.77053875681381\\
65.375	0.16254	7.56064493540347\\
65.375	0.1662	8.37629510042504\\
65.375	0.16986	9.2174892518785\\
65.375	0.17352	10.0842273897639\\
65.375	0.17718	10.9765095140812\\
65.375	0.18084	11.8943356248304\\
65.375	0.1845	12.8377057220116\\
65.375	0.18816	13.8066198056246\\
65.375	0.19182	14.8010778756696\\
65.375	0.19548	15.8210799321464\\
65.375	0.19914	16.8666259750552\\
65.375	0.2028	17.9377160043959\\
65.375	0.20646	19.0343500201685\\
65.375	0.21012	20.1565280223731\\
65.375	0.21378	21.3042500110095\\
65.375	0.21744	22.4775159860778\\
65.375	0.2211	23.6763259475781\\
65.375	0.22476	24.9006798955103\\
65.375	0.22842	26.1505778298744\\
65.375	0.23208	27.4260197506704\\
65.375	0.23574	28.7270056578983\\
65.375	0.2394	30.0535355515581\\
65.375	0.24306	31.4056094316499\\
65.375	0.24672	32.7832272981735\\
65.375	0.25038	34.1863891511291\\
65.375	0.25404	35.6150949905166\\
65.375	0.2577	37.0693448163359\\
65.375	0.26136	38.5491386285872\\
65.375	0.26502	40.0544764272705\\
65.375	0.26868	41.5853582123856\\
65.375	0.27234	43.1417839839326\\
65.375	0.276	44.7237537419116\\
65.75	0.093	-3.1449186231731\\
65.75	0.09666	-2.80737573319824\\
65.75	0.10032	-2.44428885679143\\
65.75	0.10398	-2.05565799395272\\
65.75	0.10764	-1.64148314468211\\
65.75	0.1113	-1.20176430897959\\
65.75	0.11496	-0.736501486845128\\
65.75	0.11862	-0.245694678278781\\
65.75	0.12228	0.270656116719493\\
65.75	0.12594	0.812550898149674\\
65.75	0.1296	1.37998966601176\\
65.75	0.13326	1.97297242030577\\
65.75	0.13692	2.5914991610317\\
65.75	0.14058	3.23556988818953\\
65.75	0.14424	3.90518460177928\\
65.75	0.1479	4.60034330180094\\
65.75	0.15156	5.32104598825451\\
65.75	0.15522	6.06729266113999\\
65.75	0.15888	6.8390833204574\\
65.75	0.16254	7.63641796620671\\
65.75	0.1662	8.45929659838794\\
65.75	0.16986	9.30771921700106\\
65.75	0.17352	10.1816858220461\\
65.75	0.17718	11.0811964135231\\
65.75	0.18084	12.006250991432\\
65.75	0.1845	12.9568495557728\\
65.75	0.18816	13.9329921065455\\
65.75	0.19182	14.9346786437501\\
65.75	0.19548	15.9619091673866\\
65.75	0.19914	17.014683677455\\
65.75	0.2028	18.0930021739554\\
65.75	0.20646	19.1968646568877\\
65.75	0.21012	20.3262711262519\\
65.75	0.21378	21.4812215820479\\
65.75	0.21744	22.661716024276\\
65.75	0.2211	23.8677544529359\\
65.75	0.22476	25.0993368680277\\
65.75	0.22842	26.3564632695515\\
65.75	0.23208	27.6391336575071\\
65.75	0.23574	28.9473480318947\\
65.75	0.2394	30.2811063927142\\
65.75	0.24306	31.6404087399656\\
65.75	0.24672	33.0252550736489\\
65.75	0.25038	34.4356453937641\\
65.75	0.25404	35.8715797003113\\
65.75	0.2577	37.3330579932903\\
65.75	0.26136	38.8200802727013\\
65.75	0.26502	40.3326465385441\\
65.75	0.26868	41.8707567908189\\
65.75	0.27234	43.4344110295256\\
65.75	0.276	45.0236092546642\\
66.125	0.093	-3.20411779080329\\
66.125	0.09666	-2.85934643366876\\
66.125	0.10032	-2.4890310901023\\
66.125	0.10398	-2.09317176010394\\
66.125	0.10764	-1.67176844367367\\
66.125	0.1113	-1.22482114081149\\
66.125	0.11496	-0.752329851517366\\
66.125	0.11862	-0.254294575791363\\
66.125	0.12228	0.269284686366568\\
66.125	0.12594	0.818407934956412\\
66.125	0.1296	1.39307516997815\\
66.125	0.13326	1.99328639143182\\
66.125	0.13692	2.6190415993174\\
66.125	0.14058	3.2703407936349\\
66.125	0.14424	3.9471839743843\\
66.125	0.1479	4.64957114156562\\
66.125	0.15156	5.37750229517885\\
66.125	0.15522	6.13097743522399\\
66.125	0.15888	6.90999656170105\\
66.125	0.16254	7.71455967461002\\
66.125	0.1662	8.54466677395091\\
66.125	0.16986	9.4003178597237\\
66.125	0.17352	10.2815129319284\\
66.125	0.17718	11.188251990565\\
66.125	0.18084	12.1205350356336\\
66.125	0.1845	13.078362067134\\
66.125	0.18816	14.0617330850664\\
66.125	0.19182	15.0706480894307\\
66.125	0.19548	16.1051070802268\\
66.125	0.19914	17.1651100574549\\
66.125	0.2028	18.250657021115\\
66.125	0.20646	19.3617479712069\\
66.125	0.21012	20.4983829077307\\
66.125	0.21378	21.6605618306865\\
66.125	0.21744	22.8482847400741\\
66.125	0.2211	24.0615516358937\\
66.125	0.22476	25.3003625181452\\
66.125	0.22842	26.5647173868286\\
66.125	0.23208	27.8546162419439\\
66.125	0.23574	29.1700590834912\\
66.125	0.2394	30.5110459114703\\
66.125	0.24306	31.8775767258813\\
66.125	0.24672	33.2696515267243\\
66.125	0.25038	34.6872703139992\\
66.125	0.25404	36.130433087706\\
66.125	0.2577	37.5991398478447\\
66.125	0.26136	39.0933905944153\\
66.125	0.26502	40.6131853274179\\
66.125	0.26868	42.1585240468523\\
66.125	0.27234	43.7294067527186\\
66.125	0.276	45.3258334450169\\
66.5	0.093	-3.26094828083337\\
66.5	0.09666	-2.90894845653917\\
66.5	0.10032	-2.53140464581307\\
66.5	0.10398	-2.12831684865506\\
66.5	0.10764	-1.69968506506511\\
66.5	0.1113	-1.24550929504326\\
66.5	0.11496	-0.765789538589516\\
66.5	0.11862	-0.260525795703842\\
66.5	0.12228	0.270281933613752\\
66.5	0.12594	0.826633649363238\\
66.5	0.1296	1.40852935154466\\
66.5	0.13326	2.01596904015797\\
66.5	0.13692	2.6489527152032\\
66.5	0.14058	3.30748037668036\\
66.5	0.14424	3.99155202458945\\
66.5	0.1479	4.70116765893041\\
66.5	0.15156	5.4363272797033\\
66.5	0.15522	6.19703088690809\\
66.5	0.15888	6.98327848054481\\
66.5	0.16254	7.79507006061344\\
66.5	0.1662	8.63240562711398\\
66.5	0.16986	9.49528518004643\\
66.5	0.17352	10.3837087194108\\
66.5	0.17718	11.2976762452071\\
66.5	0.18084	12.2371877574353\\
66.5	0.1845	13.2022432560954\\
66.5	0.18816	14.1928427411874\\
66.5	0.19182	15.2089862127113\\
66.5	0.19548	16.2506736706672\\
66.5	0.19914	17.3179051150549\\
66.5	0.2028	18.4106805458746\\
66.5	0.20646	19.5289999631262\\
66.5	0.21012	20.6728633668097\\
66.5	0.21378	21.8422707569251\\
66.5	0.21744	23.0372221334724\\
66.5	0.2211	24.2577174964517\\
66.5	0.22476	25.5037568458628\\
66.5	0.22842	26.7753401817058\\
66.5	0.23208	28.0724675039808\\
66.5	0.23574	29.3951388126877\\
66.5	0.2394	30.7433541078265\\
66.5	0.24306	32.1171133893972\\
66.5	0.24672	33.5164166573999\\
66.5	0.25038	34.9412639118344\\
66.5	0.25404	36.3916551527009\\
66.5	0.2577	37.8675903799992\\
66.5	0.26136	39.3690695937295\\
66.5	0.26502	40.8960927938917\\
66.5	0.26868	42.4486599804858\\
66.5	0.27234	44.0267711535118\\
66.5	0.276	45.6304263129698\\
66.875	0.093	-3.31541009326339\\
66.875	0.09666	-2.95618180180953\\
66.875	0.10032	-2.57140952392377\\
66.875	0.10398	-2.16109325960611\\
66.875	0.10764	-1.72523300885649\\
66.875	0.1113	-1.26382877167499\\
66.875	0.11496	-0.776880548061591\\
66.875	0.11862	-0.264388338016253\\
66.875	0.12228	0.27364785846099\\
66.875	0.12594	0.837228041370139\\
66.875	0.1296	1.42635221071122\\
66.875	0.13326	2.04102036648419\\
66.875	0.13692	2.68123250868907\\
66.875	0.14058	3.34698863732589\\
66.875	0.14424	4.03828875239463\\
66.875	0.1479	4.75513285389525\\
66.875	0.15156	5.4975209418278\\
66.875	0.15522	6.26545301619225\\
66.875	0.15888	7.05892907698863\\
66.875	0.16254	7.87794912421692\\
66.875	0.1662	8.72251315787712\\
66.875	0.16986	9.59262117796922\\
66.875	0.17352	10.4882731844932\\
66.875	0.17718	11.4094691774492\\
66.875	0.18084	12.356209156837\\
66.875	0.1845	13.3284931226568\\
66.875	0.18816	14.3263210749085\\
66.875	0.19182	15.3496930135921\\
66.875	0.19548	16.3986089387076\\
66.875	0.19914	17.473068850255\\
66.875	0.2028	18.5730727482343\\
66.875	0.20646	19.6986206326456\\
66.875	0.21012	20.8497125034887\\
66.875	0.21378	22.0263483607638\\
66.875	0.21744	23.2285282044708\\
66.875	0.2211	24.4562520346096\\
66.875	0.22476	25.7095198511804\\
66.875	0.22842	26.9883316541832\\
66.875	0.23208	28.2926874436178\\
66.875	0.23574	29.6225872194844\\
66.875	0.2394	30.9780309817828\\
66.875	0.24306	32.3590187305132\\
66.875	0.24672	33.7655504656755\\
66.875	0.25038	35.1976261872697\\
66.875	0.25404	36.6552458952958\\
66.875	0.2577	38.1384095897538\\
66.875	0.26136	39.6471172706437\\
66.875	0.26502	41.1813689379656\\
66.875	0.26868	42.7411645917193\\
66.875	0.27234	44.326504231905\\
66.875	0.276	45.9373878585226\\
67.25	0.093	-3.36750322809331\\
67.25	0.09666	-3.0010464694798\\
67.25	0.10032	-2.60904572443437\\
67.25	0.10398	-2.19150099295703\\
67.25	0.10764	-1.7484122750478\\
67.25	0.1113	-1.27977957070664\\
67.25	0.11496	-0.785602879933549\\
67.25	0.11862	-0.265882202728569\\
67.25	0.12228	0.279382460908337\\
67.25	0.12594	0.85019111097715\\
67.25	0.1296	1.44654374747786\\
67.25	0.13326	2.06844037041051\\
67.25	0.13692	2.71588097977506\\
67.25	0.14058	3.38886557557153\\
67.25	0.14424	4.08739415779991\\
67.25	0.1479	4.8114667264602\\
67.25	0.15156	5.5610832815524\\
67.25	0.15522	6.33624382307652\\
67.25	0.15888	7.13694835103255\\
67.25	0.16254	7.9631968654205\\
67.25	0.1662	8.81498936624035\\
67.25	0.16986	9.69232585349211\\
67.25	0.17352	10.5952063271758\\
67.25	0.17718	11.5236307872914\\
67.25	0.18084	12.4775992338389\\
67.25	0.1845	13.4571116668183\\
67.25	0.18816	14.4621680862297\\
67.25	0.19182	15.4927684920729\\
67.25	0.19548	16.5489128843481\\
67.25	0.19914	17.6306012630551\\
67.25	0.2028	18.7378336281941\\
67.25	0.20646	19.870609979765\\
67.25	0.21012	21.0289303177678\\
67.25	0.21378	22.2127946422026\\
67.25	0.21744	23.4222029530692\\
67.25	0.2211	24.6571552503677\\
67.25	0.22476	25.9176515340982\\
67.25	0.22842	27.2036918042606\\
67.25	0.23208	28.5152760608549\\
67.25	0.23574	29.8524043038811\\
67.25	0.2394	31.2150765333392\\
67.25	0.24306	32.6032927492292\\
67.25	0.24672	34.0170529515512\\
67.25	0.25038	35.456357140305\\
67.25	0.25404	36.9212053154908\\
67.25	0.2577	38.4115974771085\\
67.25	0.26136	39.9275336251581\\
67.25	0.26502	41.4690137596396\\
67.25	0.26868	43.036037880553\\
67.25	0.27234	44.6286059878983\\
67.25	0.276	46.2467180816756\\
67.625	0.093	-3.41722768532316\\
67.625	0.09666	-3.04354245955\\
67.625	0.10032	-2.64431324734491\\
67.625	0.10398	-2.21954004870791\\
67.625	0.10764	-1.76922286363902\\
67.625	0.1113	-1.2933616921382\\
67.625	0.11496	-0.791956534205454\\
67.625	0.11862	-0.265007389840818\\
67.625	0.12228	0.287485740955745\\
67.625	0.12594	0.865522858184221\\
67.625	0.1296	1.46910396184459\\
67.625	0.13326	2.09822905193689\\
67.625	0.13692	2.75289812846111\\
67.625	0.14058	3.43311119141724\\
67.625	0.14424	4.13886824080527\\
67.625	0.1479	4.87016927662522\\
67.625	0.15156	5.62701429887708\\
67.625	0.15522	6.40940330756084\\
67.625	0.15888	7.21733630267654\\
67.625	0.16254	8.05081328422414\\
67.625	0.1662	8.90983425220366\\
67.625	0.16986	9.79439920661508\\
67.625	0.17352	10.7045081474584\\
67.625	0.17718	11.6401610747337\\
67.625	0.18084	12.6013579884408\\
67.625	0.1845	13.5880988885799\\
67.625	0.18816	14.6003837751509\\
67.625	0.19182	15.6382126481538\\
67.625	0.19548	16.7015855075886\\
67.625	0.19914	17.7905023534554\\
67.625	0.2028	18.904963185754\\
67.625	0.20646	20.0449680044846\\
67.625	0.21012	21.2105168096471\\
67.625	0.21378	22.4016096012414\\
67.625	0.21744	23.6182463792677\\
67.625	0.2211	24.8604271437259\\
67.625	0.22476	26.128151894616\\
67.625	0.22842	27.4214206319381\\
67.625	0.23208	28.740233355692\\
67.625	0.23574	30.0845900658779\\
67.625	0.2394	31.4544907624957\\
67.625	0.24306	32.8499354455454\\
67.625	0.24672	34.2709241150269\\
67.625	0.25038	35.7174567709405\\
67.625	0.25404	37.1895334132859\\
67.625	0.2577	38.6871540420632\\
67.625	0.26136	40.2103186572725\\
67.625	0.26502	41.7590272589136\\
67.625	0.26868	43.3332798469867\\
67.625	0.27234	44.9330764214917\\
67.625	0.276	46.5584169824286\\
68	0.093	-3.46458346495291\\
68	0.09666	-3.08366977202008\\
68	0.10032	-2.67721209265535\\
68	0.10398	-2.24521042685871\\
68	0.10764	-1.78766477463012\\
68	0.1113	-1.30457513596965\\
68	0.11496	-0.79594151087727\\
68	0.11862	-0.261763899352964\\
68	0.12228	0.297957698603255\\
68	0.12594	0.883223282991374\\
68	0.1296	1.49403285381143\\
68	0.13326	2.13038641106337\\
68	0.13692	2.79228395474723\\
68	0.14058	3.47972548486301\\
68	0.14424	4.19271100141074\\
68	0.1479	4.93124050439033\\
68	0.15156	5.69531399380185\\
68	0.15522	6.48493146964528\\
68	0.15888	7.30009293192062\\
68	0.16254	8.14079838062788\\
68	0.1662	9.00704781576706\\
68	0.16986	9.89884123733813\\
68	0.17352	10.8161786453411\\
68	0.17718	11.759060039776\\
68	0.18084	12.7274854206429\\
68	0.1845	13.7214547879416\\
68	0.18816	14.7409681416723\\
68	0.19182	15.7860254818348\\
68	0.19548	16.8566268084293\\
68	0.19914	17.9527721214557\\
68	0.2028	19.074461420914\\
68	0.20646	20.2216947068042\\
68	0.21012	21.3944719791263\\
68	0.21378	22.5927932378804\\
68	0.21744	23.8166584830663\\
68	0.2211	25.0660677146842\\
68	0.22476	26.341020932734\\
68	0.22842	27.6415181372157\\
68	0.23208	28.9675593281293\\
68	0.23574	30.3191445054748\\
68	0.2394	31.6962736692522\\
68	0.24306	33.0989468194616\\
68	0.24672	34.5271639561028\\
68	0.25038	35.980925079176\\
68	0.25404	37.4602301886811\\
68	0.2577	38.9650792846181\\
68	0.26136	40.495472366987\\
68	0.26502	42.0514094357878\\
68	0.26868	43.6328904910205\\
68	0.27234	45.2399155326852\\
68	0.276	46.8724845607817\\
68.375	0.093	-3.5095705669826\\
68.375	0.09666	-3.12142840689012\\
68.375	0.10032	-2.70774226036573\\
68.375	0.10398	-2.26851212740943\\
68.375	0.10764	-1.80373800802119\\
68.375	0.1113	-1.31341990220105\\
68.375	0.11496	-0.797557809949019\\
68.375	0.11862	-0.256151731265057\\
68.375	0.12228	0.310798333850819\\
68.375	0.12594	0.903292385398593\\
68.375	0.1296	1.52133042337831\\
68.375	0.13326	2.16491244778991\\
68.375	0.13692	2.83403845863343\\
68.375	0.14058	3.52870845590887\\
68.375	0.14424	4.24892243961624\\
68.375	0.1479	4.99468040975549\\
68.375	0.15156	5.76598236632668\\
68.375	0.15522	6.56282830932976\\
68.375	0.15888	7.38521823876476\\
68.375	0.16254	8.23315215463169\\
68.375	0.1662	9.10663005693052\\
68.375	0.16986	10.0056519456612\\
68.375	0.17352	10.9302178208239\\
68.375	0.17718	11.8803276824185\\
68.375	0.18084	12.855981530445\\
68.375	0.1845	13.8571793649034\\
68.375	0.18816	14.8839211857937\\
68.375	0.19182	15.9362069931159\\
68.375	0.19548	17.01403678687\\
68.375	0.19914	18.1174105670561\\
68.375	0.2028	19.246328333674\\
68.375	0.20646	20.4007900867239\\
68.375	0.21012	21.5807958262057\\
68.375	0.21378	22.7863455521194\\
68.375	0.21744	24.017439264465\\
68.375	0.2211	25.2740769632425\\
68.375	0.22476	26.556258648452\\
68.375	0.22842	27.8639843200933\\
68.375	0.23208	29.1972539781666\\
68.375	0.23574	30.5560676226718\\
68.375	0.2394	31.9404252536088\\
68.375	0.24306	33.3503268709778\\
68.375	0.24672	34.7857724747787\\
68.375	0.25038	36.2467620650116\\
68.375	0.25404	37.7332956416763\\
68.375	0.2577	39.245373204773\\
68.375	0.26136	40.7829947543015\\
68.375	0.26502	42.346160290262\\
68.375	0.26868	43.9348698126544\\
68.375	0.27234	45.5491233214787\\
68.375	0.276	47.188920816735\\
68.75	0.093	-3.55218899141221\\
68.75	0.09666	-3.15681836416007\\
68.75	0.10032	-2.73590375047602\\
68.75	0.10398	-2.28944515036006\\
68.75	0.10764	-1.81744256381217\\
68.75	0.1113	-1.31989599083238\\
68.75	0.11496	-0.79680543142068\\
68.75	0.11862	-0.248170885577061\\
68.75	0.12228	0.326007646698478\\
68.75	0.12594	0.925730165405909\\
68.75	0.1296	1.55099667054528\\
68.75	0.13326	2.20180716211654\\
68.75	0.13692	2.87816164011972\\
68.75	0.14058	3.58006010455481\\
68.75	0.14424	4.30750255542185\\
68.75	0.1479	5.06048899272075\\
68.75	0.15156	5.8390194164516\\
68.75	0.15522	6.64309382661433\\
68.75	0.15888	7.472712223209\\
68.75	0.16254	8.32787460623558\\
68.75	0.1662	9.20858097569407\\
68.75	0.16986	10.1148313315845\\
68.75	0.17352	11.0466256739068\\
68.75	0.17718	12.003964002661\\
68.75	0.18084	12.9868463178471\\
68.75	0.1845	13.9952726194652\\
68.75	0.18816	15.0292429075152\\
68.75	0.19182	16.088757181997\\
68.75	0.19548	17.1738154429108\\
68.75	0.19914	18.2844176902565\\
68.75	0.2028	19.4205639240342\\
68.75	0.20646	20.5822541442437\\
68.75	0.21012	21.7694883508851\\
68.75	0.21378	22.9822665439585\\
68.75	0.21744	24.2205887234638\\
68.75	0.2211	25.4844548894009\\
68.75	0.22476	26.77386504177\\
68.75	0.22842	28.0888191805711\\
68.75	0.23208	29.4293173058039\\
68.75	0.23574	30.7953594174688\\
68.75	0.2394	32.1869455155656\\
68.75	0.24306	33.6040756000942\\
68.75	0.24672	35.0467496710548\\
68.75	0.25038	36.5149677284473\\
68.75	0.25404	38.0087297722716\\
68.75	0.2577	39.528035802528\\
68.75	0.26136	41.0728858192162\\
68.75	0.26502	42.6432798223363\\
68.75	0.26868	44.2392178118884\\
68.75	0.27234	45.8606997878723\\
68.75	0.276	47.5077257502882\\
69.125	0.093	-3.59243873824172\\
69.125	0.09666	-3.18983964382994\\
69.125	0.10032	-2.76169656298621\\
69.125	0.10398	-2.30800949571059\\
69.125	0.10764	-1.82877844200306\\
69.125	0.1113	-1.32400340186361\\
69.125	0.11496	-0.79368437529223\\
69.125	0.11862	-0.237821362288962\\
69.125	0.12228	0.343585637146226\\
69.125	0.12594	0.950536623013335\\
69.125	0.1296	1.58303159531234\\
69.125	0.13326	2.24107055404327\\
69.125	0.13692	2.92465349920611\\
69.125	0.14058	3.63378043080087\\
69.125	0.14424	4.36845134882753\\
69.125	0.1479	5.12866625328611\\
69.125	0.15156	5.91442514417661\\
69.125	0.15522	6.72572802149901\\
69.125	0.15888	7.56257488525333\\
69.125	0.16254	8.42496573543956\\
69.125	0.1662	9.31290057205771\\
69.125	0.16986	10.2263793951078\\
69.125	0.17352	11.1654022045897\\
69.125	0.17718	12.1299690005036\\
69.125	0.18084	13.1200797828494\\
69.125	0.1845	14.1357345516271\\
69.125	0.18816	15.1769333068367\\
69.125	0.19182	16.2436760484783\\
69.125	0.19548	17.3359627765518\\
69.125	0.19914	18.4537934910571\\
69.125	0.2028	19.5971681919944\\
69.125	0.20646	20.7660868793636\\
69.125	0.21012	21.9605495531647\\
69.125	0.21378	23.1805562133977\\
69.125	0.21744	24.4261068600626\\
69.125	0.2211	25.6972014931594\\
69.125	0.22476	26.9938401126882\\
69.125	0.22842	28.3160227186489\\
69.125	0.23208	29.6637493110414\\
69.125	0.23574	31.037019889866\\
69.125	0.2394	32.4358344551223\\
69.125	0.24306	33.8601930068106\\
69.125	0.24672	35.3100955449309\\
69.125	0.25038	36.785542069483\\
69.125	0.25404	38.2865325804671\\
69.125	0.2577	39.8130670778831\\
69.125	0.26136	41.3651455617309\\
69.125	0.26502	42.9427680320107\\
69.125	0.26868	44.5459344887224\\
69.125	0.27234	46.174644931866\\
69.125	0.276	47.8288993614416\\
69.5	0.093	-3.63031980747115\\
69.5	0.09666	-3.22049224589971\\
69.5	0.10032	-2.78512069789633\\
69.5	0.10398	-2.32420516346105\\
69.5	0.10764	-1.83774564259386\\
69.5	0.1113	-1.32574213529476\\
69.5	0.11496	-0.78819464156372\\
69.5	0.11862	-0.225103161400796\\
69.5	0.12228	0.363532305194056\\
69.5	0.12594	0.977711758220821\\
69.5	0.1296	1.61743519767947\\
69.5	0.13326	2.28270262357006\\
69.5	0.13692	2.97351403589257\\
69.5	0.14058	3.68986943464698\\
69.5	0.14424	4.4317688198333\\
69.5	0.1479	5.19921219145154\\
69.5	0.15156	5.9921995495017\\
69.5	0.15522	6.81073089398375\\
69.5	0.15888	7.65480622489773\\
69.5	0.16254	8.52442554224363\\
69.5	0.1662	9.41958884602143\\
69.5	0.16986	10.3402961362311\\
69.5	0.17352	11.2865474128728\\
69.5	0.17718	12.2583426759463\\
69.5	0.18084	13.2556819254518\\
69.5	0.1845	14.2785651613891\\
69.5	0.18816	15.3269923837584\\
69.5	0.19182	16.4009635925596\\
69.5	0.19548	17.5004787877927\\
69.5	0.19914	18.6255379694577\\
69.5	0.2028	19.7761411375547\\
69.5	0.20646	20.9522882920835\\
69.5	0.21012	22.1539794330443\\
69.5	0.21378	23.3812145604369\\
69.5	0.21744	24.6339936742615\\
69.5	0.2211	25.912316774518\\
69.5	0.22476	27.2161838612064\\
69.5	0.22842	28.5455949343268\\
69.5	0.23208	29.900549993879\\
69.5	0.23574	31.2810490398631\\
69.5	0.2394	32.6870920722792\\
69.5	0.24306	34.1186790911272\\
69.5	0.24672	35.5758100964071\\
69.5	0.25038	37.0584850881189\\
69.5	0.25404	38.5667040662626\\
69.5	0.2577	40.1004670308382\\
69.5	0.26136	41.6597739818458\\
69.5	0.26502	43.2446249192852\\
69.5	0.26868	44.8550198431566\\
69.5	0.27234	46.4909587534599\\
69.5	0.276	48.1524416501951\\
69.875	0.093	-3.66583219910051\\
69.875	0.09666	-3.24877617036939\\
69.875	0.10032	-2.80617615520637\\
69.875	0.10398	-2.33803215361144\\
69.875	0.10764	-1.84434416558457\\
69.875	0.1113	-1.3251121911258\\
69.875	0.11496	-0.780336230235136\\
69.875	0.11862	-0.210016282912541\\
69.875	0.12228	0.385847650841967\\
69.875	0.12594	1.00725557102837\\
69.875	0.1296	1.65420747764672\\
69.875	0.13326	2.32670337069695\\
69.875	0.13692	3.0247432501791\\
69.875	0.14058	3.74832711609318\\
69.875	0.14424	4.49745496843918\\
69.875	0.1479	5.27212680721706\\
69.875	0.15156	6.07234263242687\\
69.875	0.15522	6.89810244406858\\
69.875	0.15888	7.74940624214223\\
69.875	0.16254	8.62625402664778\\
69.875	0.1662	9.52864579758524\\
69.875	0.16986	10.4565815549546\\
69.875	0.17352	11.4100612987559\\
69.875	0.17718	12.3890850289891\\
69.875	0.18084	13.3936527456542\\
69.875	0.1845	14.4237644487512\\
69.875	0.18816	15.4794201382802\\
69.875	0.19182	16.560619814241\\
69.875	0.19548	17.6673634766338\\
69.875	0.19914	18.7996511254585\\
69.875	0.2028	19.9574827607151\\
69.875	0.20646	21.1408583824036\\
69.875	0.21012	22.349777990524\\
69.875	0.21378	23.5842415850763\\
69.875	0.21744	24.8442491660606\\
69.875	0.2211	26.1298007334767\\
69.875	0.22476	27.4408962873248\\
69.875	0.22842	28.7775358276048\\
69.875	0.23208	30.1397193543166\\
69.875	0.23574	31.5274468674605\\
69.875	0.2394	32.9407183670362\\
69.875	0.24306	34.3795338530438\\
69.875	0.24672	35.8438933254834\\
69.875	0.25038	37.3337967843548\\
69.875	0.25404	38.8492442296582\\
69.875	0.2577	40.3902356613935\\
69.875	0.26136	41.9567710795607\\
69.875	0.26502	43.5488504841598\\
69.875	0.26868	45.1664738751908\\
69.875	0.27234	46.8096412526537\\
69.875	0.276	48.4783526165486\\
70.25	0.093	-3.69897591312978\\
70.25	0.09666	-3.274691417239\\
70.25	0.10032	-2.82486293491632\\
70.25	0.10398	-2.34949046616173\\
70.25	0.10764	-1.84857401097521\\
70.25	0.1113	-1.32211356935679\\
70.25	0.11496	-0.770109141306463\\
70.25	0.11862	-0.192560726824205\\
70.25	0.12228	0.41053167408996\\
70.25	0.12594	1.03916806143602\\
70.25	0.1296	1.69334843521402\\
70.25	0.13326	2.37307279542392\\
70.25	0.13692	3.07834114206572\\
70.25	0.14058	3.80915347513946\\
70.25	0.14424	4.56550979464512\\
70.25	0.1479	5.34741010058265\\
70.25	0.15156	6.15485439295212\\
70.25	0.15522	6.9878426717535\\
70.25	0.15888	7.84637493698679\\
70.25	0.16254	8.730451188652\\
70.25	0.1662	9.64007142674912\\
70.25	0.16986	10.5752356512781\\
70.25	0.17352	11.5359438622391\\
70.25	0.17718	12.5221960596319\\
70.25	0.18084	13.5339922434567\\
70.25	0.1845	14.5713324137134\\
70.25	0.18816	15.634216570402\\
70.25	0.19182	16.7226447135225\\
70.25	0.19548	17.8366168430749\\
70.25	0.19914	18.9761329590593\\
70.25	0.2028	20.1411930614755\\
70.25	0.20646	21.3317971503237\\
70.25	0.21012	22.5479452256038\\
70.25	0.21378	23.7896372873157\\
70.25	0.21744	25.0568733354596\\
70.25	0.2211	26.3496533700354\\
70.25	0.22476	27.6679773910432\\
70.25	0.22842	29.0118453984828\\
70.25	0.23208	30.3812573923544\\
70.25	0.23574	31.7762133726578\\
70.25	0.2394	33.1967133393932\\
70.25	0.24306	34.6427572925605\\
70.25	0.24672	36.1143452321597\\
70.25	0.25038	37.6114771581908\\
70.25	0.25404	39.1341530706539\\
70.25	0.2577	40.6823729695488\\
70.25	0.26136	42.2561368548756\\
70.25	0.26502	43.8554447266344\\
70.25	0.26868	45.4802965848251\\
70.25	0.27234	47.1306924294477\\
70.25	0.276	48.8066322605022\\
70.625	0.093	-3.72975094955896\\
70.625	0.09666	-3.29823798650853\\
70.625	0.10032	-2.84118103702618\\
70.625	0.10398	-2.35858010111193\\
70.625	0.10764	-1.85043517876577\\
70.625	0.1113	-1.31674626998769\\
70.625	0.11496	-0.75751337477768\\
70.625	0.11862	-0.17273649313578\\
70.625	0.12228	0.437584374938041\\
70.625	0.12594	1.07344922944378\\
70.625	0.1296	1.73485807038141\\
70.625	0.13326	2.42181089775098\\
70.625	0.13692	3.13430771155245\\
70.625	0.14058	3.87234851178584\\
70.625	0.14424	4.63593329845114\\
70.625	0.1479	5.42506207154834\\
70.625	0.15156	6.23973483107747\\
70.625	0.15522	7.0799515770385\\
70.625	0.15888	7.94571230943146\\
70.625	0.16254	8.83701702825632\\
70.625	0.1662	9.7538657335131\\
70.625	0.16986	10.6962584252018\\
70.625	0.17352	11.6641951033224\\
70.625	0.17718	12.6576757678749\\
70.625	0.18084	13.6767004188593\\
70.625	0.1845	14.7212690562757\\
70.625	0.18816	15.7913816801239\\
70.625	0.19182	16.8870382904041\\
70.625	0.19548	18.0082388871162\\
70.625	0.19914	19.1549834702602\\
70.625	0.2028	20.3272720398361\\
70.625	0.20646	21.5251045958439\\
70.625	0.21012	22.7484811382837\\
70.625	0.21378	23.9974016671552\\
70.625	0.21744	25.2718661824588\\
70.625	0.2211	26.5718746841943\\
70.625	0.22476	27.8974271723617\\
70.625	0.22842	29.248523646961\\
70.625	0.23208	30.6251641079922\\
70.625	0.23574	32.0273485554553\\
70.625	0.2394	33.4550769893504\\
70.625	0.24306	34.9083494096773\\
70.625	0.24672	36.3871658164361\\
70.625	0.25038	37.891526209627\\
70.625	0.25404	39.4214305892496\\
70.625	0.2577	40.9768789553042\\
70.625	0.26136	42.5578713077907\\
70.625	0.26502	44.1644076467092\\
70.625	0.26868	45.7964879720595\\
70.625	0.27234	47.4541122838417\\
70.625	0.276	49.1372805820559\\
71	0.093	-3.75815730838807\\
71	0.09666	-3.31941587817799\\
71	0.10032	-2.85513046153598\\
71	0.10398	-2.36530105846206\\
71	0.10764	-1.84992766895625\\
71	0.1113	-1.30901029301851\\
71	0.11496	-0.742548930648844\\
71	0.11862	-0.150543581847288\\
71	0.12228	0.467005753386196\\
71	0.12594	1.11009907505159\\
71	0.1296	1.77873638314888\\
71	0.13326	2.4729176776781\\
71	0.13692	3.19264295863923\\
71	0.14058	3.93791222603229\\
71	0.14424	4.70872547985723\\
71	0.1479	5.5050827201141\\
71	0.15156	6.32698394680289\\
71	0.15522	7.17442915992357\\
71	0.15888	8.04741835947618\\
71	0.16254	8.94595154546071\\
71	0.1662	9.87002871787715\\
71	0.16986	10.8196498767255\\
71	0.17352	11.7948150220057\\
71	0.17718	12.7955241537179\\
71	0.18084	13.821777271862\\
71	0.1845	14.873574376438\\
71	0.18816	15.9509154674459\\
71	0.19182	17.0538005448857\\
71	0.19548	18.1822296087575\\
71	0.19914	19.3362026590611\\
71	0.2028	20.5157196957967\\
71	0.20646	21.7207807189642\\
71	0.21012	22.9513857285636\\
71	0.21378	24.2075347245949\\
71	0.21744	25.4892277070581\\
71	0.2211	26.7964646759532\\
71	0.22476	28.1292456312803\\
71	0.22842	29.4875705730392\\
71	0.23208	30.87143950123\\
71	0.23574	32.2808524158529\\
71	0.2394	33.7158093169076\\
71	0.24306	35.1763102043941\\
71	0.24672	36.6623550783127\\
71	0.25038	38.1739439386631\\
71	0.25404	39.7110767854455\\
71	0.2577	41.2737536186597\\
71	0.26136	42.8619744383059\\
71	0.26502	44.475739244384\\
71	0.26868	46.1150480368939\\
71	0.27234	47.7799008158358\\
71	0.276	49.4702975812097\\
71.375	0.093	-3.78419498961708\\
71.375	0.09666	-3.33822509224733\\
71.375	0.10032	-2.86671120844568\\
71.375	0.10398	-2.36965333821212\\
71.375	0.10764	-1.84705148154662\\
71.375	0.1113	-1.29890563844922\\
71.375	0.11496	-0.72521580891992\\
71.375	0.11862	-0.1259819929587\\
71.375	0.12228	0.498795809434441\\
71.375	0.12594	1.14911759825948\\
71.375	0.1296	1.82498337351646\\
71.375	0.13326	2.52639313520532\\
71.375	0.13692	3.2533468833261\\
71.375	0.14058	4.0058446178788\\
71.375	0.14424	4.78388633886344\\
71.375	0.1479	5.58747204627995\\
71.375	0.15156	6.4166017401284\\
71.375	0.15522	7.27127542040874\\
71.375	0.15888	8.15149308712101\\
71.375	0.16254	9.05725474026519\\
71.375	0.1662	9.98856037984129\\
71.375	0.16986	10.9454100058493\\
71.375	0.17352	11.9278036182892\\
71.375	0.17718	12.935741217161\\
71.375	0.18084	13.9692228024648\\
71.375	0.1845	15.0282483742004\\
71.375	0.18816	16.112817932368\\
71.375	0.19182	17.2229314769675\\
71.375	0.19548	18.3585890079989\\
71.375	0.19914	19.5197905254622\\
71.375	0.2028	20.7065360293574\\
71.375	0.20646	21.9188255196845\\
71.375	0.21012	23.1566589964436\\
71.375	0.21378	24.4200364596346\\
71.375	0.21744	25.7089579092574\\
71.375	0.2211	27.0234233453122\\
71.375	0.22476	28.3634327677989\\
71.375	0.22842	29.7289861767175\\
71.375	0.23208	31.120083572068\\
71.375	0.23574	32.5367249538505\\
71.375	0.2394	33.9789103220648\\
71.375	0.24306	35.4466396767111\\
71.375	0.24672	36.9399130177893\\
71.375	0.25038	38.4587303452994\\
71.375	0.25404	40.0030916592414\\
71.375	0.2577	41.5729969596153\\
71.375	0.26136	43.1684462464211\\
71.375	0.26502	44.7894395196589\\
71.375	0.26868	46.4359767793285\\
71.375	0.27234	48.1080580254301\\
71.375	0.276	49.8056832579636\\
71.75	0.093	-3.80786399324603\\
71.75	0.09666	-3.35466562871662\\
71.75	0.10032	-2.87592327775531\\
71.75	0.10398	-2.37163694036209\\
71.75	0.10764	-1.84180661653693\\
71.75	0.1113	-1.28643230627988\\
71.75	0.11496	-0.705514009590921\\
71.75	0.11862	-0.0990517264700372\\
71.75	0.12228	0.532954543082759\\
71.75	0.12594	1.19050479906745\\
71.75	0.1296	1.87359904148409\\
71.75	0.13326	2.58223727033261\\
71.75	0.13692	3.31641948561304\\
71.75	0.14058	4.07614568732541\\
71.75	0.14424	4.8614158754697\\
71.75	0.1479	5.67223005004587\\
71.75	0.15156	6.50858821105397\\
71.75	0.15522	7.37049035849397\\
71.75	0.15888	8.25793649236591\\
71.75	0.16254	9.17092661266975\\
71.75	0.1662	10.1094607194055\\
71.75	0.16986	11.0735388125731\\
71.75	0.17352	12.0631608921727\\
71.75	0.17718	13.0783269582042\\
71.75	0.18084	14.1190370106676\\
71.75	0.1845	15.1852910495629\\
71.75	0.18816	16.2770890748902\\
71.75	0.19182	17.3944310866493\\
71.75	0.19548	18.5373170848404\\
71.75	0.19914	19.7057470694633\\
71.75	0.2028	20.8997210405182\\
71.75	0.20646	22.119238998005\\
71.75	0.21012	23.3643009419237\\
71.75	0.21378	24.6349068722743\\
71.75	0.21744	25.9310567890569\\
71.75	0.2211	27.2527506922713\\
71.75	0.22476	28.5999885819176\\
71.75	0.22842	29.9727704579959\\
71.75	0.23208	31.3710963205061\\
71.75	0.23574	32.7949661694482\\
71.75	0.2394	34.2443800048222\\
71.75	0.24306	35.7193378266281\\
71.75	0.24672	37.219839634866\\
71.75	0.25038	38.7458854295357\\
71.75	0.25404	40.2974752106374\\
71.75	0.2577	41.874608978171\\
71.75	0.26136	43.4772867321364\\
71.75	0.26502	45.1055084725338\\
71.75	0.26868	46.7592741993631\\
71.75	0.27234	48.4385839126244\\
71.75	0.276	50.1434376123175\\
72.125	0.093	-3.82916431927489\\
72.125	0.09666	-3.36873748758583\\
72.125	0.10032	-2.88276666946486\\
72.125	0.10398	-2.37125186491198\\
72.125	0.10764	-1.83419307392716\\
72.125	0.1113	-1.27159029651045\\
72.125	0.11496	-0.68344353266184\\
72.125	0.11862	-0.0697527823813004\\
72.125	0.12228	0.569481954331152\\
72.125	0.12594	1.23426067747551\\
72.125	0.1296	1.9245833870518\\
72.125	0.13326	2.64045008305998\\
72.125	0.13692	3.38186076550007\\
72.125	0.14058	4.14881543437209\\
72.125	0.14424	4.94131408967604\\
72.125	0.1479	5.75935673141187\\
72.125	0.15156	6.60294335957963\\
72.125	0.15522	7.47207397417929\\
72.125	0.15888	8.36674857521088\\
72.125	0.16254	9.28696716267437\\
72.125	0.1662	10.2327297365698\\
72.125	0.16986	11.2040362968971\\
72.125	0.17352	12.2008868436563\\
72.125	0.17718	13.2232813768475\\
72.125	0.18084	14.2712198964705\\
72.125	0.1845	15.3447024025255\\
72.125	0.18816	16.4437288950124\\
72.125	0.19182	17.5682993739312\\
72.125	0.19548	18.7184138392819\\
72.125	0.19914	19.8940722910645\\
72.125	0.2028	21.0952747292791\\
72.125	0.20646	22.3220211539255\\
72.125	0.21012	23.5743115650039\\
72.125	0.21378	24.8521459625142\\
72.125	0.21744	26.1555243464563\\
72.125	0.2211	27.4844467168304\\
72.125	0.22476	28.8389130736365\\
72.125	0.22842	30.2189234168744\\
72.125	0.23208	31.6244777465442\\
72.125	0.23574	33.055576062646\\
72.125	0.2394	34.5122183651796\\
72.125	0.24306	35.9944046541452\\
72.125	0.24672	37.5021349295427\\
72.125	0.25038	39.0354091913721\\
72.125	0.25404	40.5942274396335\\
72.125	0.2577	42.1785896743267\\
72.125	0.26136	43.7884958954518\\
72.125	0.26502	45.4239461030089\\
72.125	0.26868	47.0849402969978\\
72.125	0.27234	48.7714784774187\\
72.125	0.276	50.4835606442715\\
72.5	0.093	-3.84809596770366\\
72.5	0.09666	-3.38044066885495\\
72.5	0.10032	-2.88724138357431\\
72.5	0.10398	-2.36849811186176\\
72.5	0.10764	-1.82421085371732\\
72.5	0.1113	-1.25437960914095\\
72.5	0.11496	-0.65900437813265\\
72.5	0.11862	-0.0380851606924679\\
72.5	0.12228	0.608378043179648\\
72.5	0.12594	1.28038523348367\\
72.5	0.1296	1.97793641021959\\
72.5	0.13326	2.70103157338744\\
72.5	0.13692	3.44967072298721\\
72.5	0.14058	4.22385385901889\\
72.5	0.14424	5.02358098148247\\
72.5	0.1479	5.84885209037797\\
72.5	0.15156	6.69966718570538\\
72.5	0.15522	7.5760262674647\\
72.5	0.15888	8.47792933565594\\
72.5	0.16254	9.4053763902791\\
72.5	0.1662	10.3583674313342\\
72.5	0.16986	11.3369024588211\\
72.5	0.17352	12.34098147274\\
72.5	0.17718	13.3706044730908\\
72.5	0.18084	14.4257714598736\\
72.5	0.1845	15.5064824330882\\
72.5	0.18816	16.6127373927347\\
72.5	0.19182	17.7445363388132\\
72.5	0.19548	18.9018792713235\\
72.5	0.19914	20.0847661902658\\
72.5	0.2028	21.29319709564\\
72.5	0.20646	22.5271719874461\\
72.5	0.21012	23.7866908656842\\
72.5	0.21378	25.0717537303541\\
72.5	0.21744	26.3823605814559\\
72.5	0.2211	27.7185114189897\\
72.5	0.22476	29.0802062429554\\
72.5	0.22842	30.4674450533529\\
72.5	0.23208	31.8802278501825\\
72.5	0.23574	33.3185546334439\\
72.5	0.2394	34.7824254031372\\
72.5	0.24306	36.2718401592624\\
72.5	0.24672	37.7867989018196\\
72.5	0.25038	39.3273016308087\\
72.5	0.25404	40.8933483462296\\
72.5	0.2577	42.4849390480825\\
72.5	0.26136	44.1020737363673\\
72.5	0.26502	45.744752411084\\
72.5	0.26868	47.4129750722326\\
72.5	0.27234	49.1067417198132\\
72.5	0.276	50.8260523538256\\
72.875	0.093	-3.86465893853234\\
72.875	0.09666	-3.38977517252396\\
72.875	0.10032	-2.88934742008367\\
72.875	0.10398	-2.36337568121149\\
72.875	0.10764	-1.81185995590735\\
72.875	0.1113	-1.23480024417133\\
72.875	0.11496	-0.632196546003399\\
72.875	0.11862	-0.00404886140353966\\
72.875	0.12228	0.649642809628226\\
72.875	0.12594	1.3288784670919\\
72.875	0.1296	2.03365811098751\\
72.875	0.13326	2.763981741315\\
72.875	0.13692	3.51984935807441\\
72.875	0.14058	4.30126096126574\\
72.875	0.14424	5.10821655088901\\
72.875	0.1479	5.94071612694415\\
72.875	0.15156	6.79875968943123\\
72.875	0.15522	7.6823472383502\\
72.875	0.15888	8.59147877370111\\
72.875	0.16254	9.52615429548392\\
72.875	0.1662	10.4863738036986\\
72.875	0.16986	11.4721372983453\\
72.875	0.17352	12.4834447794238\\
72.875	0.17718	13.5202962469343\\
72.875	0.18084	14.5826917008767\\
72.875	0.1845	15.6706311412509\\
72.875	0.18816	16.7841145680571\\
72.875	0.19182	17.9231419812952\\
72.875	0.19548	19.0877133809653\\
72.875	0.19914	20.2778287670672\\
72.875	0.2028	21.4934881396011\\
72.875	0.20646	22.7346914985669\\
72.875	0.21012	24.0014388439645\\
72.875	0.21378	25.2937301757941\\
72.875	0.21744	26.6115654940556\\
72.875	0.2211	27.954944798749\\
72.875	0.22476	29.3238680898743\\
72.875	0.22842	30.7183353674316\\
72.875	0.23208	32.1383466314208\\
72.875	0.23574	33.5839018818418\\
72.875	0.2394	35.0550011186948\\
72.875	0.24306	36.5516443419797\\
72.875	0.24672	38.0738315516965\\
72.875	0.25038	39.6215627478452\\
72.875	0.25404	41.1948379304259\\
72.875	0.2577	42.7936570994384\\
72.875	0.26136	44.4180202548829\\
72.875	0.26502	46.0679273967592\\
72.875	0.26868	47.7433785250675\\
72.875	0.27234	49.4443736398077\\
72.875	0.276	51.1709127409798\\
73.25	0.093	-3.87885323176096\\
73.25	0.09666	-3.39674099859292\\
73.25	0.10032	-2.88908477899297\\
73.25	0.10398	-2.35588457296113\\
73.25	0.10764	-1.79714038049733\\
73.25	0.1113	-1.21285220160166\\
73.25	0.11496	-0.603020036274067\\
73.25	0.11862	0.0323561154854488\\
73.25	0.12228	0.693276253676878\\
73.25	0.12594	1.37974037830021\\
73.25	0.1296	2.09174848935547\\
73.25	0.13326	2.82930058684262\\
73.25	0.13692	3.59239667076168\\
73.25	0.14058	4.38103674111268\\
73.25	0.14424	5.19522079789561\\
73.25	0.1479	6.0349488411104\\
73.25	0.15156	6.90022087075713\\
73.25	0.15522	7.79103688683577\\
73.25	0.15888	8.70739688934633\\
73.25	0.16254	9.6493008782888\\
73.25	0.1662	10.6167488536632\\
73.25	0.16986	11.6097408154695\\
73.25	0.17352	12.6282767637077\\
73.25	0.17718	13.6723566983778\\
73.25	0.18084	14.7419806194798\\
73.25	0.1845	15.8371485270138\\
73.25	0.18816	16.9578604209796\\
73.25	0.19182	18.1041163013774\\
73.25	0.19548	19.2759161682071\\
73.25	0.19914	20.4732600214687\\
73.25	0.2028	21.6961478611622\\
73.25	0.20646	22.9445796872876\\
73.25	0.21012	24.218555499845\\
73.25	0.21378	25.5180752988342\\
73.25	0.21744	26.8431390842554\\
73.25	0.2211	28.1937468561084\\
73.25	0.22476	29.5698986143934\\
73.25	0.22842	30.9715943591103\\
73.25	0.23208	32.3988340902591\\
73.25	0.23574	33.8516178078399\\
73.25	0.2394	35.3299455118525\\
73.25	0.24306	36.8338172022971\\
73.25	0.24672	38.3632328791735\\
73.25	0.25038	39.9181925424819\\
73.25	0.25404	41.4986961922222\\
73.25	0.2577	43.1047438283944\\
73.25	0.26136	44.7363354509985\\
73.25	0.26502	46.3934710600346\\
73.25	0.26868	48.0761506555025\\
73.25	0.27234	49.7843742374023\\
73.25	0.276	51.5181418057341\\
73.625	0.093	-3.89067884738949\\
73.625	0.09666	-3.40133814706179\\
73.625	0.10032	-2.88645346030219\\
73.625	0.10398	-2.34602478711069\\
73.625	0.10764	-1.78005212748724\\
73.625	0.1113	-1.1885354814319\\
73.625	0.11496	-0.571474848944654\\
73.625	0.11862	0.0711297699745188\\
73.625	0.12228	0.739278375325604\\
73.625	0.12594	1.43297096710859\\
73.625	0.1296	2.15220754532352\\
73.625	0.13326	2.89698810997033\\
73.625	0.13692	3.66731266104905\\
73.625	0.14058	4.4631811985597\\
73.625	0.14424	5.28459372250228\\
73.625	0.1479	6.13155023287674\\
73.625	0.15156	7.00405072968313\\
73.625	0.15522	7.90209521292142\\
73.625	0.15888	8.82568368259164\\
73.625	0.16254	9.77481613869377\\
73.625	0.1662	10.7494925812278\\
73.625	0.16986	11.7497130101937\\
73.625	0.17352	12.7754774255916\\
73.625	0.17718	13.8267858274214\\
73.625	0.18084	14.9036382156831\\
73.625	0.1845	16.0060345903767\\
73.625	0.18816	17.1339749515022\\
73.625	0.19182	18.2874592990596\\
73.625	0.19548	19.466487633049\\
73.625	0.19914	20.6710599534702\\
73.625	0.2028	21.9011762603234\\
73.625	0.20646	23.1568365536085\\
73.625	0.21012	24.4380408333255\\
73.625	0.21378	25.7447890994744\\
73.625	0.21744	27.0770813520552\\
73.625	0.2211	28.4349175910679\\
73.625	0.22476	29.8182978165126\\
73.625	0.22842	31.2272220283891\\
73.625	0.23208	32.6616902266976\\
73.625	0.23574	34.121702411438\\
73.625	0.2394	35.6072585826103\\
73.625	0.24306	37.1183587402145\\
73.625	0.24672	38.6550028842506\\
73.625	0.25038	40.2171910147187\\
73.625	0.25404	41.8049231316186\\
73.625	0.2577	43.4181992349505\\
73.625	0.26136	45.0570193247142\\
73.625	0.26502	46.7213834009099\\
73.625	0.26868	48.4112914635375\\
73.625	0.27234	50.126743512597\\
73.625	0.276	51.8677395480885\\
74	0.093	-3.90013578541794\\
74	0.09666	-3.40356661793058\\
74	0.10032	-2.88145346401132\\
74	0.10398	-2.33379632366016\\
74	0.10764	-1.76059519687706\\
74	0.1113	-1.16185008366206\\
74	0.11496	-0.537560984015158\\
74	0.11862	0.112272102063677\\
74	0.12228	0.787649174574419\\
74	0.12594	1.48857023351706\\
74	0.1296	2.21503527889164\\
74	0.13326	2.96704431069811\\
74	0.13692	3.74459732893649\\
74	0.14058	4.5476943336068\\
74	0.14424	5.37633532470904\\
74	0.1479	6.23052030224315\\
74	0.15156	7.11024926620921\\
74	0.15522	8.01552221660715\\
74	0.15888	8.94633915343703\\
74	0.16254	9.90270007669881\\
74	0.1662	10.8846049863925\\
74	0.16986	11.8920538825181\\
74	0.17352	12.9250467650756\\
74	0.17718	13.983583634065\\
74	0.18084	15.0676644894864\\
74	0.1845	16.1772893313397\\
74	0.18816	17.3124581596249\\
74	0.19182	18.473170974342\\
74	0.19548	19.6594277754909\\
74	0.19914	20.8712285630718\\
74	0.2028	22.1085733370847\\
74	0.20646	23.3714620975294\\
74	0.21012	24.6598948444061\\
74	0.21378	25.9738715777146\\
74	0.21744	27.3133922974551\\
74	0.2211	28.6784570036275\\
74	0.22476	30.0690656962318\\
74	0.22842	31.485218375268\\
74	0.23208	32.9269150407361\\
74	0.23574	34.3941556926362\\
74	0.2394	35.8869403309681\\
74	0.24306	37.405268955732\\
74	0.24672	38.9491415669278\\
74	0.25038	40.5185581645555\\
74	0.25404	42.1135187486151\\
74	0.2577	43.7340233191066\\
74	0.26136	45.3800718760301\\
74	0.26502	47.0516644193854\\
74	0.26868	48.7488009491726\\
74	0.27234	50.4714814653918\\
74	0.276	52.2197059680429\\
};
\end{axis}

\begin{axis}[%
width=4.527496cm,
height=3.870968cm,
at={(6.483547cm,16.129032cm)},
scale only axis,
xmin=56,
xmax=74,
tick align=outside,
xlabel={$L_{cut}$},
xmajorgrids,
ymin=0.093,
ymax=0.276,
ylabel={$D_{rlx}$},
ymajorgrids,
zmin=0,
zmax=517.903625289077,
zlabel={$x_3$},
zmajorgrids,
view={-140}{50},
legend style={at={(1.03,1)},anchor=north west,legend cell align=left,align=left,draw=white!15!black}
]
\addplot3[only marks,mark=*,mark options={},mark size=1.5000pt,color=mycolor1] plot table[row sep=crcr,]{%
74	0.123	75.4245651900459\\
72	0.113	59.5782863482935\\
61	0.095	33.0003406266977\\
56	0.093	26.0569470415911\\
};
\addplot3[only marks,mark=*,mark options={},mark size=1.5000pt,color=mycolor2] plot table[row sep=crcr,]{%
67	0.276	508.93731875208\\
66	0.255	419.53506953587\\
62	0.209	242.35439857544\\
57	0.193	195.15231965962\\
};
\addplot3[only marks,mark=*,mark options={},mark size=1.5000pt,color=black] plot table[row sep=crcr,]{%
69	0.104	48.1087944305374\\
};
\addplot3[only marks,mark=*,mark options={},mark size=1.5000pt,color=black] plot table[row sep=crcr,]{%
64	0.23	317.238786352579\\
};

\addplot3[%
surf,
opacity=0.7,
shader=interp,
colormap={mymap}{[1pt] rgb(0pt)=(0.0901961,0.239216,0.0745098); rgb(1pt)=(0.0945149,0.242058,0.0739522); rgb(2pt)=(0.0988592,0.244894,0.0733566); rgb(3pt)=(0.103229,0.247724,0.0727241); rgb(4pt)=(0.107623,0.250549,0.0720557); rgb(5pt)=(0.112043,0.253367,0.0713525); rgb(6pt)=(0.116487,0.25618,0.0706154); rgb(7pt)=(0.120956,0.258986,0.0698456); rgb(8pt)=(0.125449,0.261787,0.0690441); rgb(9pt)=(0.129967,0.264581,0.0682118); rgb(10pt)=(0.134508,0.26737,0.06735); rgb(11pt)=(0.139074,0.270152,0.0664596); rgb(12pt)=(0.143663,0.272929,0.0655416); rgb(13pt)=(0.148275,0.275699,0.0645971); rgb(14pt)=(0.152911,0.278463,0.0636271); rgb(15pt)=(0.15757,0.281221,0.0626328); rgb(16pt)=(0.162252,0.283973,0.0616151); rgb(17pt)=(0.166957,0.286719,0.060575); rgb(18pt)=(0.171685,0.289458,0.0595136); rgb(19pt)=(0.176434,0.292191,0.0584321); rgb(20pt)=(0.181207,0.294918,0.0573313); rgb(21pt)=(0.186001,0.297639,0.0562123); rgb(22pt)=(0.190817,0.300353,0.0550763); rgb(23pt)=(0.195655,0.303061,0.0539242); rgb(24pt)=(0.200514,0.305763,0.052757); rgb(25pt)=(0.205395,0.308459,0.0515759); rgb(26pt)=(0.210296,0.311149,0.0503624); rgb(27pt)=(0.215212,0.313846,0.0490067); rgb(28pt)=(0.220142,0.316548,0.0475043); rgb(29pt)=(0.22509,0.319254,0.0458704); rgb(30pt)=(0.230056,0.321962,0.0441205); rgb(31pt)=(0.235042,0.324671,0.04227); rgb(32pt)=(0.240048,0.327379,0.0403343); rgb(33pt)=(0.245078,0.330085,0.0383287); rgb(34pt)=(0.250131,0.332786,0.0362688); rgb(35pt)=(0.25521,0.335482,0.0341698); rgb(36pt)=(0.260317,0.33817,0.0320472); rgb(37pt)=(0.265451,0.340849,0.0299163); rgb(38pt)=(0.270616,0.343517,0.0277927); rgb(39pt)=(0.275813,0.346172,0.0256916); rgb(40pt)=(0.281043,0.348814,0.0236284); rgb(41pt)=(0.286307,0.35144,0.0216186); rgb(42pt)=(0.291607,0.354048,0.0196776); rgb(43pt)=(0.296945,0.356637,0.0178207); rgb(44pt)=(0.302322,0.359206,0.0160634); rgb(45pt)=(0.307739,0.361753,0.0144211); rgb(46pt)=(0.313198,0.364275,0.0129091); rgb(47pt)=(0.318701,0.366772,0.0115428); rgb(48pt)=(0.324249,0.369242,0.0103377); rgb(49pt)=(0.329843,0.371682,0.00930909); rgb(50pt)=(0.335485,0.374093,0.00847245); rgb(51pt)=(0.341176,0.376471,0.00784314); rgb(52pt)=(0.346925,0.378826,0.00732741); rgb(53pt)=(0.352735,0.381168,0.00682184); rgb(54pt)=(0.358605,0.383497,0.00632729); rgb(55pt)=(0.364532,0.385812,0.00584464); rgb(56pt)=(0.370516,0.388113,0.00537476); rgb(57pt)=(0.376552,0.390399,0.00491852); rgb(58pt)=(0.38264,0.39267,0.00447681); rgb(59pt)=(0.388777,0.394925,0.00405048); rgb(60pt)=(0.394962,0.397164,0.00364042); rgb(61pt)=(0.401191,0.399386,0.00324749); rgb(62pt)=(0.407464,0.401592,0.00287258); rgb(63pt)=(0.413777,0.40378,0.00251655); rgb(64pt)=(0.420129,0.40595,0.00218028); rgb(65pt)=(0.426518,0.408102,0.00186463); rgb(66pt)=(0.432942,0.410234,0.00157049); rgb(67pt)=(0.439399,0.412348,0.00129873); rgb(68pt)=(0.445885,0.414441,0.00105022); rgb(69pt)=(0.452401,0.416515,0.000825833); rgb(70pt)=(0.458942,0.418567,0.000626441); rgb(71pt)=(0.465508,0.420599,0.00045292); rgb(72pt)=(0.472096,0.422609,0.000306141); rgb(73pt)=(0.478704,0.424596,0.000186979); rgb(74pt)=(0.485331,0.426562,9.63073e-05); rgb(75pt)=(0.491973,0.428504,3.49981e-05); rgb(76pt)=(0.498628,0.430422,3.92506e-06); rgb(77pt)=(0.505323,0.432315,0); rgb(78pt)=(0.512206,0.434168,0); rgb(79pt)=(0.519282,0.435983,0); rgb(80pt)=(0.526529,0.437764,0); rgb(81pt)=(0.533922,0.439512,0); rgb(82pt)=(0.54144,0.441232,0); rgb(83pt)=(0.549059,0.442927,0); rgb(84pt)=(0.556756,0.444599,0); rgb(85pt)=(0.564508,0.446252,0); rgb(86pt)=(0.572292,0.447889,0); rgb(87pt)=(0.580084,0.449514,0); rgb(88pt)=(0.587863,0.451129,0); rgb(89pt)=(0.595604,0.452737,0); rgb(90pt)=(0.603284,0.454343,0); rgb(91pt)=(0.610882,0.455948,0); rgb(92pt)=(0.618373,0.457556,0); rgb(93pt)=(0.625734,0.459171,0); rgb(94pt)=(0.632943,0.460795,0); rgb(95pt)=(0.639976,0.462432,0); rgb(96pt)=(0.64681,0.464084,0); rgb(97pt)=(0.653423,0.465756,0); rgb(98pt)=(0.659791,0.46745,0); rgb(99pt)=(0.665891,0.469169,0); rgb(100pt)=(0.6717,0.470916,0); rgb(101pt)=(0.677195,0.472696,0); rgb(102pt)=(0.682353,0.47451,0); rgb(103pt)=(0.687242,0.476355,0); rgb(104pt)=(0.691952,0.478225,0); rgb(105pt)=(0.696497,0.480118,0); rgb(106pt)=(0.700887,0.482033,0); rgb(107pt)=(0.705134,0.483968,0); rgb(108pt)=(0.709251,0.485921,0); rgb(109pt)=(0.713249,0.487891,0); rgb(110pt)=(0.71714,0.489876,0); rgb(111pt)=(0.720936,0.491875,0); rgb(112pt)=(0.724649,0.493887,0); rgb(113pt)=(0.72829,0.495909,0); rgb(114pt)=(0.731872,0.49794,0); rgb(115pt)=(0.735406,0.499979,0); rgb(116pt)=(0.738904,0.502025,0); rgb(117pt)=(0.742378,0.504075,0); rgb(118pt)=(0.74584,0.506128,0); rgb(119pt)=(0.749302,0.508182,0); rgb(120pt)=(0.752775,0.510237,0); rgb(121pt)=(0.756272,0.51229,0); rgb(122pt)=(0.759804,0.514339,0); rgb(123pt)=(0.763384,0.516385,0); rgb(124pt)=(0.767022,0.518424,0); rgb(125pt)=(0.770731,0.520455,0); rgb(126pt)=(0.774523,0.522478,0); rgb(127pt)=(0.77841,0.524489,0); rgb(128pt)=(0.782391,0.526491,0); rgb(129pt)=(0.786402,0.528496,0); rgb(130pt)=(0.790431,0.530506,0); rgb(131pt)=(0.794478,0.532521,0); rgb(132pt)=(0.798541,0.534539,0); rgb(133pt)=(0.802619,0.53656,0); rgb(134pt)=(0.806712,0.538584,0); rgb(135pt)=(0.81082,0.540609,0); rgb(136pt)=(0.81494,0.542635,0); rgb(137pt)=(0.819074,0.54466,0); rgb(138pt)=(0.823219,0.546686,0); rgb(139pt)=(0.827374,0.548709,0); rgb(140pt)=(0.831541,0.55073,0); rgb(141pt)=(0.835716,0.552749,0); rgb(142pt)=(0.8399,0.554763,0); rgb(143pt)=(0.844092,0.556774,0); rgb(144pt)=(0.848292,0.558779,0); rgb(145pt)=(0.852497,0.560778,0); rgb(146pt)=(0.856708,0.562771,0); rgb(147pt)=(0.860924,0.564756,0); rgb(148pt)=(0.865143,0.566733,0); rgb(149pt)=(0.869366,0.568701,0); rgb(150pt)=(0.873592,0.57066,0); rgb(151pt)=(0.877819,0.572608,0); rgb(152pt)=(0.882047,0.574545,0); rgb(153pt)=(0.886275,0.576471,0); rgb(154pt)=(0.890659,0.578362,0); rgb(155pt)=(0.895333,0.580203,0); rgb(156pt)=(0.900258,0.581999,0); rgb(157pt)=(0.905397,0.583755,0); rgb(158pt)=(0.910711,0.585479,0); rgb(159pt)=(0.916164,0.587176,0); rgb(160pt)=(0.921717,0.588852,0); rgb(161pt)=(0.927333,0.590513,0); rgb(162pt)=(0.932974,0.592166,0); rgb(163pt)=(0.938602,0.593815,0); rgb(164pt)=(0.94418,0.595468,0); rgb(165pt)=(0.949669,0.59713,0); rgb(166pt)=(0.955033,0.598808,0); rgb(167pt)=(0.960233,0.600507,0); rgb(168pt)=(0.965232,0.602233,0); rgb(169pt)=(0.969992,0.603992,0); rgb(170pt)=(0.974475,0.605791,0); rgb(171pt)=(0.978643,0.607636,0); rgb(172pt)=(0.98246,0.609532,0); rgb(173pt)=(0.985886,0.611486,0); rgb(174pt)=(0.988885,0.613503,0); rgb(175pt)=(0.991419,0.61559,0); rgb(176pt)=(0.99345,0.617753,0); rgb(177pt)=(0.99494,0.619997,0); rgb(178pt)=(0.995851,0.622329,0); rgb(179pt)=(0.996226,0.624763,0); rgb(180pt)=(0.996512,0.627352,0); rgb(181pt)=(0.996788,0.630095,0); rgb(182pt)=(0.997053,0.632982,0); rgb(183pt)=(0.997308,0.636004,0); rgb(184pt)=(0.997552,0.639152,0); rgb(185pt)=(0.997785,0.642416,0); rgb(186pt)=(0.998006,0.645786,0); rgb(187pt)=(0.998217,0.649253,0); rgb(188pt)=(0.998416,0.652807,0); rgb(189pt)=(0.998605,0.656439,0); rgb(190pt)=(0.998781,0.660138,0); rgb(191pt)=(0.998946,0.663897,0); rgb(192pt)=(0.9991,0.667704,0); rgb(193pt)=(0.999242,0.67155,0); rgb(194pt)=(0.999372,0.675427,0); rgb(195pt)=(0.99949,0.679323,0); rgb(196pt)=(0.999596,0.68323,0); rgb(197pt)=(0.99969,0.687139,0); rgb(198pt)=(0.999771,0.691039,0); rgb(199pt)=(0.999841,0.694921,0); rgb(200pt)=(0.999898,0.698775,0); rgb(201pt)=(0.999942,0.702592,0); rgb(202pt)=(0.999974,0.706363,0); rgb(203pt)=(0.999994,0.710077,0); rgb(204pt)=(1,0.713725,0); rgb(205pt)=(1,0.717341,0); rgb(206pt)=(1,0.720963,0); rgb(207pt)=(1,0.724591,0); rgb(208pt)=(1,0.728226,0); rgb(209pt)=(1,0.731867,0); rgb(210pt)=(1,0.735514,0); rgb(211pt)=(1,0.739167,0); rgb(212pt)=(1,0.742827,0); rgb(213pt)=(1,0.746493,0); rgb(214pt)=(1,0.750165,0); rgb(215pt)=(1,0.753843,0); rgb(216pt)=(1,0.757527,0); rgb(217pt)=(1,0.761217,0); rgb(218pt)=(1,0.764913,0); rgb(219pt)=(1,0.768615,0); rgb(220pt)=(1,0.772324,0); rgb(221pt)=(1,0.776038,0); rgb(222pt)=(1,0.779758,0); rgb(223pt)=(1,0.783484,0); rgb(224pt)=(1,0.787215,0); rgb(225pt)=(1,0.790953,0); rgb(226pt)=(1,0.794696,0); rgb(227pt)=(1,0.798445,0); rgb(228pt)=(1,0.8022,0); rgb(229pt)=(1,0.805961,0); rgb(230pt)=(1,0.809727,0); rgb(231pt)=(1,0.8135,0); rgb(232pt)=(1,0.817278,0); rgb(233pt)=(1,0.821063,0); rgb(234pt)=(1,0.824854,0); rgb(235pt)=(1,0.828652,0); rgb(236pt)=(1,0.832455,0); rgb(237pt)=(1,0.836265,0); rgb(238pt)=(1,0.840081,0); rgb(239pt)=(1,0.843903,0); rgb(240pt)=(1,0.847732,0); rgb(241pt)=(1,0.851566,0); rgb(242pt)=(1,0.855406,0); rgb(243pt)=(1,0.859253,0); rgb(244pt)=(1,0.863106,0); rgb(245pt)=(1,0.866964,0); rgb(246pt)=(1,0.870829,0); rgb(247pt)=(1,0.8747,0); rgb(248pt)=(1,0.878577,0); rgb(249pt)=(1,0.88246,0); rgb(250pt)=(1,0.886349,0); rgb(251pt)=(1,0.890243,0); rgb(252pt)=(1,0.894144,0); rgb(253pt)=(1,0.898051,0); rgb(254pt)=(1,0.901964,0); rgb(255pt)=(1,0.905882,0)},
mesh/rows=49]
table[row sep=crcr,header=false] {%
%
56	0.093	26.7123045686334\\
56	0.09666	28.8008271328639\\
56	0.10032	31.1960133736808\\
56	0.10398	33.897863291084\\
56	0.10764	36.9063768850736\\
56	0.1113	40.2215541556496\\
56	0.11496	43.8433951028119\\
56	0.11862	47.7718997265605\\
56	0.12228	52.0070680268955\\
56	0.12594	56.5489000038169\\
56	0.1296	61.3973956573246\\
56	0.13326	66.5525549874186\\
56	0.13692	72.014377994099\\
56	0.14058	77.7828646773658\\
56	0.14424	83.8580150372189\\
56	0.1479	90.2398290736583\\
56	0.15156	96.9283067866842\\
56	0.15522	103.923448176296\\
56	0.15888	111.225253242495\\
56	0.16254	118.83372198528\\
56	0.1662	126.748854404651\\
56	0.16986	134.970650500609\\
56	0.17352	143.499110273152\\
56	0.17718	152.334233722283\\
56	0.18084	161.476020847999\\
56	0.1845	170.924471650302\\
56	0.18816	180.679586129192\\
56	0.19182	190.741364284667\\
56	0.19548	201.109806116729\\
56	0.19914	211.784911625378\\
56	0.2028	222.766680810613\\
56	0.20646	234.055113672434\\
56	0.21012	245.650210210841\\
56	0.21378	257.551970425835\\
56	0.21744	269.760394317415\\
56	0.2211	282.275481885581\\
56	0.22476	295.097233130334\\
56	0.22842	308.225648051674\\
56	0.23208	321.660726649599\\
56	0.23574	335.402468924111\\
56	0.2394	349.450874875209\\
56	0.24306	363.805944502894\\
56	0.24672	378.467677807165\\
56	0.25038	393.436074788022\\
56	0.25404	408.711135445466\\
56	0.2577	424.292859779496\\
56	0.26136	440.181247790112\\
56	0.26502	456.376299477315\\
56	0.26868	472.878014841104\\
56	0.27234	489.686393881479\\
56	0.276	506.801436598441\\
56.375	0.093	26.9534015004779\\
56.375	0.09666	29.0365780221604\\
56.375	0.10032	31.4264182204294\\
56.375	0.10398	34.1229220952847\\
56.375	0.10764	37.1260896467263\\
56.375	0.1113	40.4359208747543\\
56.375	0.11496	44.0524157793686\\
56.375	0.11862	47.9755743605693\\
56.375	0.12228	52.2053966183563\\
56.375	0.12594	56.7418825527297\\
56.375	0.1296	61.5850321636895\\
56.375	0.13326	66.7348454512355\\
56.375	0.13692	72.191322415368\\
56.375	0.14058	77.9544630560869\\
56.375	0.14424	84.024267373392\\
56.375	0.1479	90.4007353672835\\
56.375	0.15156	97.0838670377614\\
56.375	0.15522	104.073662384826\\
56.375	0.15888	111.370121408476\\
56.375	0.16254	118.973244108713\\
56.375	0.1662	126.883030485536\\
56.375	0.16986	135.099480538946\\
56.375	0.17352	143.622594268942\\
56.375	0.17718	152.452371675524\\
56.375	0.18084	161.588812758693\\
56.375	0.1845	171.031917518448\\
56.375	0.18816	180.781685954789\\
56.375	0.19182	190.838118067717\\
56.375	0.19548	201.201213857231\\
56.375	0.19914	211.870973323332\\
56.375	0.2028	222.847396466018\\
56.375	0.20646	234.130483285291\\
56.375	0.21012	245.720233781151\\
56.375	0.21378	257.616647953597\\
56.375	0.21744	269.819725802629\\
56.375	0.2211	282.329467328247\\
56.375	0.22476	295.145872530452\\
56.375	0.22842	308.268941409244\\
56.375	0.23208	321.698673964621\\
56.375	0.23574	335.435070196585\\
56.375	0.2394	349.478130105136\\
56.375	0.24306	363.827853690272\\
56.375	0.24672	378.484240951995\\
56.375	0.25038	393.447291890305\\
56.375	0.25404	408.7170065052\\
56.375	0.2577	424.293384796682\\
56.375	0.26136	440.176426764751\\
56.375	0.26502	456.366132409405\\
56.375	0.26868	472.862501730646\\
56.375	0.27234	489.665534728474\\
56.375	0.276	506.775231402888\\
56.75	0.093	27.2054559128872\\
56.75	0.09666	29.2832863920218\\
56.75	0.10032	31.6677805477428\\
56.75	0.10398	34.3589383800501\\
56.75	0.10764	37.3567598889438\\
56.75	0.1113	40.6612450744238\\
56.75	0.11496	44.2723939364902\\
56.75	0.11862	48.1902064751429\\
56.75	0.12228	52.414682690382\\
56.75	0.12594	56.9458225822074\\
56.75	0.1296	61.7836261506192\\
56.75	0.13326	66.9280933956174\\
56.75	0.13692	72.3792243172018\\
56.75	0.14058	78.1370189153727\\
56.75	0.14424	84.20147719013\\
56.75	0.1479	90.5725991414735\\
56.75	0.15156	97.2503847694034\\
56.75	0.15522	104.23483407392\\
56.75	0.15888	111.525947055022\\
56.75	0.16254	119.123723712711\\
56.75	0.1662	127.028164046987\\
56.75	0.16986	135.239268057848\\
56.75	0.17352	143.757035745296\\
56.75	0.17718	152.581467109331\\
56.75	0.18084	161.712562149951\\
56.75	0.1845	171.150320867158\\
56.75	0.18816	180.894743260952\\
56.75	0.19182	190.945829331331\\
56.75	0.19548	201.303579078298\\
56.75	0.19914	211.96799250185\\
56.75	0.2028	222.939069601989\\
56.75	0.20646	234.216810378714\\
56.75	0.21012	245.801214832026\\
56.75	0.21378	257.692282961923\\
56.75	0.21744	269.890014768408\\
56.75	0.2211	282.394410251478\\
56.75	0.22476	295.205469411135\\
56.75	0.22842	308.323192247379\\
56.75	0.23208	321.747578760208\\
56.75	0.23574	335.478628949624\\
56.75	0.2394	349.516342815627\\
56.75	0.24306	363.860720358215\\
56.75	0.24672	378.51176157739\\
56.75	0.25038	393.469466473152\\
56.75	0.25404	408.733835045499\\
56.75	0.2577	424.304867294434\\
56.75	0.26136	440.182563219954\\
56.75	0.26502	456.366922822061\\
56.75	0.26868	472.857946100754\\
56.75	0.27234	489.655633056033\\
56.75	0.276	506.759983687899\\
57.125	0.093	27.4684678058615\\
57.125	0.09666	29.5409522424481\\
57.125	0.10032	31.9201003556211\\
57.125	0.10398	34.6059121453805\\
57.125	0.10764	37.5983876117262\\
57.125	0.1113	40.8975267546583\\
57.125	0.11496	44.5033295741767\\
57.125	0.11862	48.4157960702815\\
57.125	0.12228	52.6349262429726\\
57.125	0.12594	57.1607200922501\\
57.125	0.1296	61.9931776181139\\
57.125	0.13326	67.1322988205641\\
57.125	0.13692	72.5780836996007\\
57.125	0.14058	78.3305322552235\\
57.125	0.14424	84.3896444874328\\
57.125	0.1479	90.7554203962284\\
57.125	0.15156	97.4278599816103\\
57.125	0.15522	104.406963243579\\
57.125	0.15888	111.692730182133\\
57.125	0.16254	119.285160797274\\
57.125	0.1662	127.184255089002\\
57.125	0.16986	135.390013057315\\
57.125	0.17352	143.902434702215\\
57.125	0.17718	152.721520023702\\
57.125	0.18084	161.847269021774\\
57.125	0.1845	171.279681696434\\
57.125	0.18816	181.018758047679\\
57.125	0.19182	191.064498075511\\
57.125	0.19548	201.416901779929\\
57.125	0.19914	212.075969160934\\
57.125	0.2028	223.041700218524\\
57.125	0.20646	234.314094952702\\
57.125	0.21012	245.893153363465\\
57.125	0.21378	257.778875450815\\
57.125	0.21744	269.971261214751\\
57.125	0.2211	282.470310655274\\
57.125	0.22476	295.276023772383\\
57.125	0.22842	308.388400566078\\
57.125	0.23208	321.80744103636\\
57.125	0.23574	335.533145183228\\
57.125	0.2394	349.565513006683\\
57.125	0.24306	363.904544506723\\
57.125	0.24672	378.55023968335\\
57.125	0.25038	393.502598536564\\
57.125	0.25404	408.761621066364\\
57.125	0.2577	424.32730727275\\
57.125	0.26136	440.199657155722\\
57.125	0.26502	456.378670715281\\
57.125	0.26868	472.864347951426\\
57.125	0.27234	489.656688864158\\
57.125	0.276	506.755693453476\\
57.5	0.093	27.7424371794006\\
57.5	0.09666	29.8095755734393\\
57.5	0.10032	32.1833776440644\\
57.5	0.10398	34.8638433912757\\
57.5	0.10764	37.8509728150735\\
57.5	0.1113	41.1447659154576\\
57.5	0.11496	44.7452226924281\\
57.5	0.11862	48.6523431459849\\
57.5	0.12228	52.8661272761281\\
57.5	0.12594	57.3865750828576\\
57.5	0.1296	62.2136865661734\\
57.5	0.13326	67.3474617260757\\
57.5	0.13692	72.7879005625643\\
57.5	0.14058	78.5350030756393\\
57.5	0.14424	84.5887692653005\\
57.5	0.1479	90.9491991315482\\
57.5	0.15156	97.6162926743822\\
57.5	0.15522	104.590049893802\\
57.5	0.15888	111.870470789809\\
57.5	0.16254	119.457555362402\\
57.5	0.1662	127.351303611582\\
57.5	0.16986	135.551715537347\\
57.5	0.17352	144.0587911397\\
57.5	0.17718	152.872530418638\\
57.5	0.18084	161.992933374163\\
57.5	0.1845	171.420000006274\\
57.5	0.18816	181.153730314971\\
57.5	0.19182	191.194124300255\\
57.5	0.19548	201.541181962125\\
57.5	0.19914	212.194903300582\\
57.5	0.2028	223.155288315625\\
57.5	0.20646	234.422337007254\\
57.5	0.21012	245.99604937547\\
57.5	0.21378	257.876425420272\\
57.5	0.21744	270.06346514166\\
57.5	0.2211	282.557168539635\\
57.5	0.22476	295.357535614196\\
57.5	0.22842	308.464566365343\\
57.5	0.23208	321.878260793077\\
57.5	0.23574	335.598618897397\\
57.5	0.2394	349.625640678303\\
57.5	0.24306	363.959326135796\\
57.5	0.24672	378.599675269875\\
57.5	0.25038	393.546688080541\\
57.5	0.25404	408.800364567793\\
57.5	0.2577	424.360704731631\\
57.5	0.26136	440.227708572055\\
57.5	0.26502	456.401376089066\\
57.5	0.26868	472.881707282663\\
57.5	0.27234	489.668702152847\\
57.5	0.276	506.762360699617\\
57.875	0.093	28.0273640335046\\
57.875	0.09666	30.0891563849953\\
57.875	0.10032	32.4576124130725\\
57.875	0.10398	35.1327321177359\\
57.875	0.10764	38.1145154989857\\
57.875	0.1113	41.4029625568218\\
57.875	0.11496	44.9980732912444\\
57.875	0.11862	48.8998477022533\\
57.875	0.12228	53.1082857898485\\
57.875	0.12594	57.62338755403\\
57.875	0.1296	62.445152994798\\
57.875	0.13326	67.5735821121522\\
57.875	0.13692	73.0086749060928\\
57.875	0.14058	78.7504313766198\\
57.875	0.14424	84.7988515237332\\
57.875	0.1479	91.1539353474328\\
57.875	0.15156	97.8156828477188\\
57.875	0.15522	104.784094024591\\
57.875	0.15888	112.05916887805\\
57.875	0.16254	119.640907408095\\
57.875	0.1662	127.529309614727\\
57.875	0.16986	135.724375497944\\
57.875	0.17352	144.226105057748\\
57.875	0.17718	153.034498294139\\
57.875	0.18084	162.149555207116\\
57.875	0.1845	171.571275796679\\
57.875	0.18816	181.299660062829\\
57.875	0.19182	191.334708005564\\
57.875	0.19548	201.676419624887\\
57.875	0.19914	212.324794920795\\
57.875	0.2028	223.27983389329\\
57.875	0.20646	234.541536542372\\
57.875	0.21012	246.109902868039\\
57.875	0.21378	257.984932870293\\
57.875	0.21744	270.166626549133\\
57.875	0.2211	282.65498390456\\
57.875	0.22476	295.450004936573\\
57.875	0.22842	308.551689645173\\
57.875	0.23208	321.960038030358\\
57.875	0.23574	335.675050092131\\
57.875	0.2394	349.696725830489\\
57.875	0.24306	364.025065245434\\
57.875	0.24672	378.660068336965\\
57.875	0.25038	393.601735105083\\
57.875	0.25404	408.850065549787\\
57.875	0.2577	424.405059671077\\
57.875	0.26136	440.266717468953\\
57.875	0.26502	456.435038943416\\
57.875	0.26868	472.910024094466\\
57.875	0.27234	489.691672922101\\
57.875	0.276	506.779985426323\\
58.25	0.093	28.3232483681735\\
58.25	0.09666	30.3796946771163\\
58.25	0.10032	32.7428046626454\\
58.25	0.10398	35.4125783247609\\
58.25	0.10764	38.3890156634628\\
58.25	0.1113	41.672116678751\\
58.25	0.11496	45.2618813706256\\
58.25	0.11862	49.1583097390864\\
58.25	0.12228	53.3614017841337\\
58.25	0.12594	57.8711575057673\\
58.25	0.1296	62.6875769039872\\
58.25	0.13326	67.8106599787936\\
58.25	0.13692	73.2404067301862\\
58.25	0.14058	78.9768171581653\\
58.25	0.14424	85.0198912627307\\
58.25	0.1479	91.3696290438824\\
58.25	0.15156	98.0260305016205\\
58.25	0.15522	104.989095635945\\
58.25	0.15888	112.258824446856\\
58.25	0.16254	119.835216934353\\
58.25	0.1662	127.718273098436\\
58.25	0.16986	135.907992939106\\
58.25	0.17352	144.404376456362\\
58.25	0.17718	153.207423650205\\
58.25	0.18084	162.317134520634\\
58.25	0.1845	171.733509067649\\
58.25	0.18816	181.456547291251\\
58.25	0.19182	191.486249191438\\
58.25	0.19548	201.822614768213\\
58.25	0.19914	212.465644021573\\
58.25	0.2028	223.41533695152\\
58.25	0.20646	234.671693558054\\
58.25	0.21012	246.234713841174\\
58.25	0.21378	258.104397800879\\
58.25	0.21744	270.280745437172\\
58.25	0.2211	282.763756750051\\
58.25	0.22476	295.553431739516\\
58.25	0.22842	308.649770405567\\
58.25	0.23208	322.052772748205\\
58.25	0.23574	335.762438767429\\
58.25	0.2394	349.77876846324\\
58.25	0.24306	364.101761835637\\
58.25	0.24672	378.73141888462\\
58.25	0.25038	393.66773961019\\
58.25	0.25404	408.910724012345\\
58.25	0.2577	424.460372091088\\
58.25	0.26136	440.316683846416\\
58.25	0.26502	456.479659278331\\
58.25	0.26868	472.949298386833\\
58.25	0.27234	489.72560117192\\
58.25	0.276	506.808567633594\\
58.625	0.093	28.6300901834073\\
58.625	0.09666	30.6811904498022\\
58.625	0.10032	33.0389543927833\\
58.625	0.10398	35.7033820123509\\
58.625	0.10764	38.6744733085048\\
58.625	0.1113	41.952228281245\\
58.625	0.11496	45.5366469305716\\
58.625	0.11862	49.4277292564846\\
58.625	0.12228	53.6254752589839\\
58.625	0.12594	58.1298849380695\\
58.625	0.1296	62.9409582937415\\
58.625	0.13326	68.0586953259999\\
58.625	0.13692	73.4830960348446\\
58.625	0.14058	79.2141604202757\\
58.625	0.14424	85.2518884822931\\
58.625	0.1479	91.5962802208968\\
58.625	0.15156	98.2473356360869\\
58.625	0.15522	105.205054727863\\
58.625	0.15888	112.469437496226\\
58.625	0.16254	120.040483941175\\
58.625	0.1662	127.918194062711\\
58.625	0.16986	136.102567860833\\
58.625	0.17352	144.593605335541\\
58.625	0.17718	153.391306486836\\
58.625	0.18084	162.495671314717\\
58.625	0.1845	171.906699819184\\
58.625	0.18816	181.624392000237\\
58.625	0.19182	191.648747857877\\
58.625	0.19548	201.979767392104\\
58.625	0.19914	212.617450602917\\
58.625	0.2028	223.561797490316\\
58.625	0.20646	234.812808054301\\
58.625	0.21012	246.370482294873\\
58.625	0.21378	258.234820212031\\
58.625	0.21744	270.405821805775\\
58.625	0.2211	282.883487076106\\
58.625	0.22476	295.667816023023\\
58.625	0.22842	308.758808646527\\
58.625	0.23208	322.156464946617\\
58.625	0.23574	335.860784923293\\
58.625	0.2394	349.871768576555\\
58.625	0.24306	364.189415906404\\
58.625	0.24672	378.813726912839\\
58.625	0.25038	393.744701595861\\
58.625	0.25404	408.982339955469\\
58.625	0.2577	424.526641991664\\
58.625	0.26136	440.377607704444\\
58.625	0.26502	456.535237093811\\
58.625	0.26868	472.999530159764\\
58.625	0.27234	489.770486902304\\
58.625	0.276	506.84810732143\\
59	0.093	28.947889479206\\
59	0.09666	30.9936437030529\\
59	0.10032	33.3460616034861\\
59	0.10398	36.0051431805057\\
59	0.10764	38.9708884341117\\
59	0.1113	42.2432973643039\\
59	0.11496	45.8223699710826\\
59	0.11862	49.7081062544476\\
59	0.12228	53.9005062143989\\
59	0.12594	58.3995698509366\\
59	0.1296	63.2052971640606\\
59	0.13326	68.3176881537711\\
59	0.13692	73.7367428200678\\
59	0.14058	79.462461162951\\
59	0.14424	85.4948431824204\\
59	0.1479	91.8338888784762\\
59	0.15156	98.4795982511184\\
59	0.15522	105.431971300347\\
59	0.15888	112.691008026162\\
59	0.16254	120.256708428563\\
59	0.1662	128.129072507551\\
59	0.16986	136.308100263124\\
59	0.17352	144.793791695285\\
59	0.17718	153.586146804031\\
59	0.18084	162.685165589364\\
59	0.1845	172.090848051284\\
59	0.18816	181.803194189789\\
59	0.19182	191.822204004881\\
59	0.19548	202.14787749656\\
59	0.19914	212.780214664824\\
59	0.2028	223.719215509676\\
59	0.20646	234.964880031113\\
59	0.21012	246.517208229137\\
59	0.21378	258.376200103747\\
59	0.21744	270.541855654943\\
59	0.2211	283.014174882726\\
59	0.22476	295.793157787095\\
59	0.22842	308.878804368051\\
59	0.23208	322.271114625593\\
59	0.23574	335.970088559721\\
59	0.2394	349.975726170436\\
59	0.24306	364.288027457737\\
59	0.24672	378.906992421624\\
59	0.25038	393.832621062098\\
59	0.25404	409.064913379158\\
59	0.2577	424.603869372804\\
59	0.26136	440.449489043037\\
59	0.26502	456.601772389856\\
59	0.26868	473.060719413261\\
59	0.27234	489.826330113253\\
59	0.276	506.898604489831\\
59.375	0.093	29.2766462555696\\
59.375	0.09666	31.3170544368685\\
59.375	0.10032	33.6641262947538\\
59.375	0.10398	36.3178618292254\\
59.375	0.10764	39.2782610402834\\
59.375	0.1113	42.5453239279277\\
59.375	0.11496	46.1190504921584\\
59.375	0.11862	49.9994407329754\\
59.375	0.12228	54.1864946503788\\
59.375	0.12594	58.6802122443686\\
59.375	0.1296	63.4805935149446\\
59.375	0.13326	68.5876384621071\\
59.375	0.13692	74.0013470858559\\
59.375	0.14058	79.7217193861911\\
59.375	0.14424	85.7487553631126\\
59.375	0.1479	92.0824550166204\\
59.375	0.15156	98.7228183467146\\
59.375	0.15522	105.669845353395\\
59.375	0.15888	112.923536036662\\
59.375	0.16254	120.483890396515\\
59.375	0.1662	128.350908432955\\
59.375	0.16986	136.524590145981\\
59.375	0.17352	145.004935535593\\
59.375	0.17718	153.791944601792\\
59.375	0.18084	162.885617344577\\
59.375	0.1845	172.285953763948\\
59.375	0.18816	181.992953859906\\
59.375	0.19182	192.00661763245\\
59.375	0.19548	202.326945081581\\
59.375	0.19914	212.953936207297\\
59.375	0.2028	223.887591009601\\
59.375	0.20646	235.12790948849\\
59.375	0.21012	246.674891643966\\
59.375	0.21378	258.528537476028\\
59.375	0.21744	270.688846984677\\
59.375	0.2211	283.155820169911\\
59.375	0.22476	295.929457031733\\
59.375	0.22842	309.00975757014\\
59.375	0.23208	322.396721785134\\
59.375	0.23574	336.090349676715\\
59.375	0.2394	350.090641244881\\
59.375	0.24306	364.397596489634\\
59.375	0.24672	379.011215410973\\
59.375	0.25038	393.931498008899\\
59.375	0.25404	409.158444283411\\
59.375	0.2577	424.69205423451\\
59.375	0.26136	440.532327862195\\
59.375	0.26502	456.679265166465\\
59.375	0.26868	473.132866147323\\
59.375	0.27234	489.893130804767\\
59.375	0.276	506.960059138797\\
59.75	0.093	29.6163605124981\\
59.75	0.09666	31.651422651249\\
59.75	0.10032	33.9931484665864\\
59.75	0.10398	36.6415379585101\\
59.75	0.10764	39.5965911270201\\
59.75	0.1113	42.8583079721164\\
59.75	0.11496	46.4266884937992\\
59.75	0.11862	50.3017326920683\\
59.75	0.12228	54.4834405669237\\
59.75	0.12594	58.9718121183655\\
59.75	0.1296	63.7668473463936\\
59.75	0.13326	68.8685462510081\\
59.75	0.13692	74.2769088322089\\
59.75	0.14058	79.9919350899962\\
59.75	0.14424	86.0136250243697\\
59.75	0.1479	92.3419786353296\\
59.75	0.15156	98.9769959228759\\
59.75	0.15522	105.918676887008\\
59.75	0.15888	113.167021527728\\
59.75	0.16254	120.722029845033\\
59.75	0.1662	128.583701838924\\
59.75	0.16986	136.752037509402\\
59.75	0.17352	145.227036856467\\
59.75	0.17718	154.008699880117\\
59.75	0.18084	163.097026580354\\
59.75	0.1845	172.492016957178\\
59.75	0.18816	182.193671010588\\
59.75	0.19182	192.201988740584\\
59.75	0.19548	202.516970147166\\
59.75	0.19914	213.138615230335\\
59.75	0.2028	224.06692399009\\
59.75	0.20646	235.301896426432\\
59.75	0.21012	246.84353253936\\
59.75	0.21378	258.691832328874\\
59.75	0.21744	270.846795794974\\
59.75	0.2211	283.308422937661\\
59.75	0.22476	296.076713756935\\
59.75	0.22842	309.151668252794\\
59.75	0.23208	322.53328642524\\
59.75	0.23574	336.221568274273\\
59.75	0.2394	350.216513799891\\
59.75	0.24306	364.518123002096\\
59.75	0.24672	379.126395880888\\
59.75	0.25038	394.041332436266\\
59.75	0.25404	409.26293266823\\
59.75	0.2577	424.79119657678\\
59.75	0.26136	440.626124161917\\
59.75	0.26502	456.76771542364\\
59.75	0.26868	473.215970361949\\
59.75	0.27234	489.970888976845\\
59.75	0.276	507.032471268328\\
60.125	0.093	29.9670322499914\\
60.125	0.09666	31.9967483461945\\
60.125	0.10032	34.3331281189838\\
60.125	0.10398	36.9761715683595\\
60.125	0.10764	39.9258786943216\\
60.125	0.1113	43.18224949687\\
60.125	0.11496	46.7452839760048\\
60.125	0.11862	50.6149821317259\\
60.125	0.12228	54.7913439640334\\
60.125	0.12594	59.2743694729272\\
60.125	0.1296	64.0640586584074\\
60.125	0.13326	69.1604115204739\\
60.125	0.13692	74.5634280591268\\
60.125	0.14058	80.2731082743661\\
60.125	0.14424	86.2894521661917\\
60.125	0.1479	92.6124597346036\\
60.125	0.15156	99.2421309796019\\
60.125	0.15522	106.178465901187\\
60.125	0.15888	113.421464499358\\
60.125	0.16254	120.971126774115\\
60.125	0.1662	128.827452725459\\
60.125	0.16986	136.990442353389\\
60.125	0.17352	145.460095657905\\
60.125	0.17718	154.236412639008\\
60.125	0.18084	163.319393296697\\
60.125	0.1845	172.709037630972\\
60.125	0.18816	182.405345641834\\
60.125	0.19182	192.408317329282\\
60.125	0.19548	202.717952693317\\
60.125	0.19914	213.334251733938\\
60.125	0.2028	224.257214451145\\
60.125	0.20646	235.486840844939\\
60.125	0.21012	247.023130915318\\
60.125	0.21378	258.866084662285\\
60.125	0.21744	271.015702085837\\
60.125	0.2211	283.471983185976\\
60.125	0.22476	296.234927962702\\
60.125	0.22842	309.304536416013\\
60.125	0.23208	322.680808545911\\
60.125	0.23574	336.363744352396\\
60.125	0.2394	350.353343835466\\
60.125	0.24306	364.649606995124\\
60.125	0.24672	379.252533831367\\
60.125	0.25038	394.162124344197\\
60.125	0.25404	409.378378533613\\
60.125	0.2577	424.901296399616\\
60.125	0.26136	440.730877942204\\
60.125	0.26502	456.867123161379\\
60.125	0.26868	473.310032057141\\
60.125	0.27234	490.059604629489\\
60.125	0.276	507.115840878423\\
60.5	0.093	30.3286614680497\\
60.5	0.09666	32.3530315217048\\
60.5	0.10032	34.6840652519462\\
60.5	0.10398	37.3217626587739\\
60.5	0.10764	40.2661237421881\\
60.5	0.1113	43.5171485021885\\
60.5	0.11496	47.0748369387753\\
60.5	0.11862	50.9391890519485\\
60.5	0.12228	55.1102048417081\\
60.5	0.12594	59.5878843080539\\
60.5	0.1296	64.3722274509861\\
60.5	0.13326	69.4632342705046\\
60.5	0.13692	74.8609047666096\\
60.5	0.14058	80.5652389393009\\
60.5	0.14424	86.5762367885786\\
60.5	0.1479	92.8938983144425\\
60.5	0.15156	99.5182235168928\\
60.5	0.15522	106.44921239593\\
60.5	0.15888	113.686864951553\\
60.5	0.16254	121.231181183762\\
60.5	0.1662	129.082161092558\\
60.5	0.16986	137.23980467794\\
60.5	0.17352	145.704111939908\\
60.5	0.17718	154.475082878463\\
60.5	0.18084	163.552717493604\\
60.5	0.1845	172.937015785332\\
60.5	0.18816	182.627977753646\\
60.5	0.19182	192.625603398546\\
60.5	0.19548	202.929892720032\\
60.5	0.19914	213.540845718105\\
60.5	0.2028	224.458462392765\\
60.5	0.20646	235.68274274401\\
60.5	0.21012	247.213686771842\\
60.5	0.21378	259.05129447626\\
60.5	0.21744	271.195565857265\\
60.5	0.2211	283.646500914856\\
60.5	0.22476	296.404099649033\\
60.5	0.22842	309.468362059797\\
60.5	0.23208	322.839288147147\\
60.5	0.23574	336.516877911084\\
60.5	0.2394	350.501131351606\\
60.5	0.24306	364.792048468716\\
60.5	0.24672	379.389629262411\\
60.5	0.25038	394.293873732693\\
60.5	0.25404	409.504781879561\\
60.5	0.2577	425.022353703016\\
60.5	0.26136	440.846589203057\\
60.5	0.26502	456.977488379684\\
60.5	0.26868	473.415051232897\\
60.5	0.27234	490.159277762697\\
60.5	0.276	507.210167969083\\
60.875	0.093	30.7012481666729\\
60.875	0.09666	32.72027217778\\
60.875	0.10032	35.0459598654734\\
60.875	0.10398	37.6783112297532\\
60.875	0.10764	40.6173262706194\\
60.875	0.1113	43.8630049880719\\
60.875	0.11496	47.4153473821108\\
60.875	0.11862	51.274353452736\\
60.875	0.12228	55.4400231999475\\
60.875	0.12594	59.9123566237454\\
60.875	0.1296	64.6913537241297\\
60.875	0.13326	69.7770145011003\\
60.875	0.13692	75.1693389546573\\
60.875	0.14058	80.8683270848006\\
60.875	0.14424	86.8739788915304\\
60.875	0.1479	93.1862943748463\\
60.875	0.15156	99.8052735347487\\
60.875	0.15522	106.730916371237\\
60.875	0.15888	113.963222884313\\
60.875	0.16254	121.502193073974\\
60.875	0.1662	129.347826940222\\
60.875	0.16986	137.500124483056\\
60.875	0.17352	145.959085702476\\
60.875	0.17718	154.724710598483\\
60.875	0.18084	163.796999171076\\
60.875	0.1845	173.175951420256\\
60.875	0.18816	182.861567346022\\
60.875	0.19182	192.853846948374\\
60.875	0.19548	203.152790227313\\
60.875	0.19914	213.758397182838\\
60.875	0.2028	224.670667814949\\
60.875	0.20646	235.889602123647\\
60.875	0.21012	247.415200108931\\
60.875	0.21378	259.247461770801\\
60.875	0.21744	271.386387109258\\
60.875	0.2211	283.831976124301\\
60.875	0.22476	296.58422881593\\
60.875	0.22842	309.643145184146\\
60.875	0.23208	323.008725228948\\
60.875	0.23574	336.680968950337\\
60.875	0.2394	350.659876348311\\
60.875	0.24306	364.945447422872\\
60.875	0.24672	379.53768217402\\
60.875	0.25038	394.436580601754\\
60.875	0.25404	409.642142706074\\
60.875	0.2577	425.154368486981\\
60.875	0.26136	440.973257944474\\
60.875	0.26502	457.098811078553\\
60.875	0.26868	473.531027889219\\
60.875	0.27234	490.269908376471\\
60.875	0.276	507.315452540309\\
61.25	0.093	31.0847923458609\\
61.25	0.09666	33.0984703144201\\
61.25	0.10032	35.4188119595656\\
61.25	0.10398	38.0458172812974\\
61.25	0.10764	40.9794862796156\\
61.25	0.1113	44.2198189545201\\
61.25	0.11496	47.7668153060111\\
61.25	0.11862	51.6204753340883\\
61.25	0.12228	55.7807990387519\\
61.25	0.12594	60.2477864200019\\
61.25	0.1296	65.0214374778382\\
61.25	0.13326	70.1017522122608\\
61.25	0.13692	75.4887306232698\\
61.25	0.14058	81.1823727108653\\
61.25	0.14424	87.182678475047\\
61.25	0.1479	93.489647915815\\
61.25	0.15156	100.103281033169\\
61.25	0.15522	107.02357782711\\
61.25	0.15888	114.250538297637\\
61.25	0.16254	121.784162444751\\
61.25	0.1662	129.624450268451\\
61.25	0.16986	137.771401768737\\
61.25	0.17352	146.225016945609\\
61.25	0.17718	154.985295799068\\
61.25	0.18084	164.052238329113\\
61.25	0.1845	173.425844535745\\
61.25	0.18816	183.106114418963\\
61.25	0.19182	193.093047978767\\
61.25	0.19548	203.386645215158\\
61.25	0.19914	213.986906128135\\
61.25	0.2028	224.893830717698\\
61.25	0.20646	236.107418983848\\
61.25	0.21012	247.627670926584\\
61.25	0.21378	259.454586545906\\
61.25	0.21744	271.588165841815\\
61.25	0.2211	284.02840881431\\
61.25	0.22476	296.775315463392\\
61.25	0.22842	309.82888578906\\
61.25	0.23208	323.189119791314\\
61.25	0.23574	336.856017470154\\
61.25	0.2394	350.829578825581\\
61.25	0.24306	365.109803857594\\
61.25	0.24672	379.696692566194\\
61.25	0.25038	394.59024495138\\
61.25	0.25404	409.790461013152\\
61.25	0.2577	425.297340751511\\
61.25	0.26136	441.110884166456\\
61.25	0.26502	457.231091257987\\
61.25	0.26868	473.657962026105\\
61.25	0.27234	490.391496470809\\
61.25	0.276	507.431694592099\\
61.625	0.093	31.4792940056139\\
61.625	0.09666	33.487625931625\\
61.625	0.10032	35.8026215342226\\
61.625	0.10398	38.4242808134065\\
61.625	0.10764	41.3526037691767\\
61.625	0.1113	44.5875904015333\\
61.625	0.11496	48.1292407104763\\
61.625	0.11862	51.9775546960055\\
61.625	0.12228	56.1325323581212\\
61.625	0.12594	60.5941736968232\\
61.625	0.1296	65.3624787121115\\
61.625	0.13326	70.4374474039862\\
61.625	0.13692	75.8190797724473\\
61.625	0.14058	81.5073758174948\\
61.625	0.14424	87.5023355391285\\
61.625	0.1479	93.8039589373486\\
61.625	0.15156	100.412246012155\\
61.625	0.15522	107.327196763548\\
61.625	0.15888	114.548811191527\\
61.625	0.16254	122.077089296093\\
61.625	0.1662	129.912031077245\\
61.625	0.16986	138.053636534983\\
61.625	0.17352	146.501905669307\\
61.625	0.17718	155.256838480218\\
61.625	0.18084	164.318434967715\\
61.625	0.1845	173.686695131799\\
61.625	0.18816	183.361618972469\\
61.625	0.19182	193.343206489725\\
61.625	0.19548	203.631457683568\\
61.625	0.19914	214.226372553997\\
61.625	0.2028	225.127951101013\\
61.625	0.20646	236.336193324614\\
61.625	0.21012	247.851099224802\\
61.625	0.21378	259.672668801577\\
61.625	0.21744	271.800902054938\\
61.625	0.2211	284.235798984885\\
61.625	0.22476	296.977359591418\\
61.625	0.22842	310.025583874538\\
61.625	0.23208	323.380471834244\\
61.625	0.23574	337.042023470537\\
61.625	0.2394	351.010238783416\\
61.625	0.24306	365.285117772881\\
61.625	0.24672	379.866660438933\\
61.625	0.25038	394.754866781571\\
61.625	0.25404	409.949736800795\\
61.625	0.2577	425.451270496606\\
61.625	0.26136	441.259467869003\\
61.625	0.26502	457.374328917986\\
61.625	0.26868	473.795853643556\\
61.625	0.27234	490.524042045712\\
61.625	0.276	507.558894124454\\
62	0.093	31.8847531459317\\
62	0.09666	33.8877390293949\\
62	0.10032	36.1973885894445\\
62	0.10398	38.8137018260804\\
62	0.10764	41.7366787393027\\
62	0.1113	44.9663193291113\\
62	0.11496	48.5026235955064\\
62	0.11862	52.3455915384877\\
62	0.12228	56.4952231580554\\
62	0.12594	60.9515184542095\\
62	0.1296	65.7144774269498\\
62	0.13326	70.7841000762766\\
62	0.13692	76.1603864021897\\
62	0.14058	81.8433364046892\\
62	0.14424	87.832950083775\\
62	0.1479	94.1292274394471\\
62	0.15156	100.732168471706\\
62	0.15522	107.641773180551\\
62	0.15888	114.858041565982\\
62	0.16254	122.380973627999\\
62	0.1662	130.210569366603\\
62	0.16986	138.346828781793\\
62	0.17352	146.78975187357\\
62	0.17718	155.539338641933\\
62	0.18084	164.595589086882\\
62	0.1845	173.958503208418\\
62	0.18816	183.62808100654\\
62	0.19182	193.604322481248\\
62	0.19548	203.887227632543\\
62	0.19914	214.476796460424\\
62	0.2028	225.373028964892\\
62	0.20646	236.575925145946\\
62	0.21012	248.085485003586\\
62	0.21378	259.901708537812\\
62	0.21744	272.024595748625\\
62	0.2211	284.454146636024\\
62	0.22476	297.19036120001\\
62	0.22842	310.233239440582\\
62	0.23208	323.58278135774\\
62	0.23574	337.238986951485\\
62	0.2394	351.201856221815\\
62	0.24306	365.471389168733\\
62	0.24672	380.047585792236\\
62	0.25038	394.930446092326\\
62	0.25404	410.119970069003\\
62	0.2577	425.616157722266\\
62	0.26136	441.419009052115\\
62	0.26502	457.52852405855\\
62	0.26868	473.944702741572\\
62	0.27234	490.66754510118\\
62	0.276	507.697051137374\\
62.375	0.093	32.3011697668144\\
62.375	0.09666	34.2988096077297\\
62.375	0.10032	36.6031131252313\\
62.375	0.10398	39.2140803193193\\
62.375	0.10764	42.1317111899936\\
62.375	0.1113	45.3560057372543\\
62.375	0.11496	48.8869639611013\\
62.375	0.11862	52.7245858615347\\
62.375	0.12228	56.8688714385544\\
62.375	0.12594	61.3198206921606\\
62.375	0.1296	66.077433622353\\
62.375	0.13326	71.1417102291317\\
62.375	0.13692	76.5126505124969\\
62.375	0.14058	82.1902544724485\\
62.375	0.14424	88.1745221089863\\
62.375	0.1479	94.4654534221105\\
62.375	0.15156	101.063048411821\\
62.375	0.15522	107.967307078118\\
62.375	0.15888	115.178229421001\\
62.375	0.16254	122.695815440471\\
62.375	0.1662	130.520065136527\\
62.375	0.16986	138.650978509169\\
62.375	0.17352	147.088555558398\\
62.375	0.17718	155.832796284213\\
62.375	0.18084	164.883700686614\\
62.375	0.1845	174.241268765602\\
62.375	0.18816	183.905500521176\\
62.375	0.19182	193.876395953336\\
62.375	0.19548	204.153955062083\\
62.375	0.19914	214.738177847416\\
62.375	0.2028	225.629064309336\\
62.375	0.20646	236.826614447842\\
62.375	0.21012	248.330828262934\\
62.375	0.21378	260.141705754612\\
62.375	0.21744	272.259246922877\\
62.375	0.2211	284.683451767728\\
62.375	0.22476	297.414320289166\\
62.375	0.22842	310.45185248719\\
62.375	0.23208	323.7960483618\\
62.375	0.23574	337.446907912997\\
62.375	0.2394	351.40443114078\\
62.375	0.24306	365.668618045149\\
62.375	0.24672	380.239468626105\\
62.375	0.25038	395.116982883647\\
62.375	0.25404	410.301160817775\\
62.375	0.2577	425.79200242849\\
62.375	0.26136	441.589507715791\\
62.375	0.26502	457.693676679679\\
62.375	0.26868	474.104509320152\\
62.375	0.27234	490.822005637213\\
62.375	0.276	507.846165630859\\
62.75	0.093	32.728543868262\\
62.75	0.09666	34.7208376666293\\
62.75	0.10032	37.019795141583\\
62.75	0.10398	39.625416293123\\
62.75	0.10764	42.5377011212494\\
62.75	0.1113	45.7566496259621\\
62.75	0.11496	49.2822618072612\\
62.75	0.11862	53.1145376651466\\
62.75	0.12228	57.2534771996184\\
62.75	0.12594	61.6990804106765\\
62.75	0.1296	66.451347298321\\
62.75	0.13326	71.5102778625518\\
62.75	0.13692	76.875872103369\\
62.75	0.14058	82.5481300207726\\
62.75	0.14424	88.5270516147625\\
62.75	0.1479	94.8126368853387\\
62.75	0.15156	101.404885832501\\
62.75	0.15522	108.30379845625\\
62.75	0.15888	115.509374756586\\
62.75	0.16254	123.021614733507\\
62.75	0.1662	130.840518387015\\
62.75	0.16986	138.96608571711\\
62.75	0.17352	147.39831672379\\
62.75	0.17718	156.137211407057\\
62.75	0.18084	165.182769766911\\
62.75	0.1845	174.534991803351\\
62.75	0.18816	184.193877516377\\
62.75	0.19182	194.159426905989\\
62.75	0.19548	204.431639972188\\
62.75	0.19914	215.010516714973\\
62.75	0.2028	225.896057134345\\
62.75	0.20646	237.088261230303\\
62.75	0.21012	248.587129002847\\
62.75	0.21378	260.392660451977\\
62.75	0.21744	272.504855577694\\
62.75	0.2211	284.923714379997\\
62.75	0.22476	297.649236858887\\
62.75	0.22842	310.681423014363\\
62.75	0.23208	324.020272846425\\
62.75	0.23574	337.665786355074\\
62.75	0.2394	351.617963540309\\
62.75	0.24306	365.876804402131\\
62.75	0.24672	380.442308940538\\
62.75	0.25038	395.314477155533\\
62.75	0.25404	410.493309047113\\
62.75	0.2577	425.97880461528\\
62.75	0.26136	441.770963860033\\
62.75	0.26502	457.869786781372\\
62.75	0.26868	474.275273379298\\
62.75	0.27234	490.98742365381\\
62.75	0.276	508.006237604909\\
63.125	0.093	33.1668754502745\\
63.125	0.09666	35.1538232060939\\
63.125	0.10032	37.4474346384996\\
63.125	0.10398	40.0477097474917\\
63.125	0.10764	42.9546485330701\\
63.125	0.1113	46.1682509952349\\
63.125	0.11496	49.688517133986\\
63.125	0.11862	53.5154469493235\\
63.125	0.12228	57.6490404412473\\
63.125	0.12594	62.0892976097575\\
63.125	0.1296	66.836218454854\\
63.125	0.13326	71.8898029765369\\
63.125	0.13692	77.2500511748061\\
63.125	0.14058	82.9169630496617\\
63.125	0.14424	88.8905386011037\\
63.125	0.1479	95.1707778291319\\
63.125	0.15156	101.757680733747\\
63.125	0.15522	108.651247314948\\
63.125	0.15888	115.851477572735\\
63.125	0.16254	123.358371507109\\
63.125	0.1662	131.171929118069\\
63.125	0.16986	139.292150405615\\
63.125	0.17352	147.719035369748\\
63.125	0.17718	156.452584010467\\
63.125	0.18084	165.492796327772\\
63.125	0.1845	174.839672321664\\
63.125	0.18816	184.493211992142\\
63.125	0.19182	194.453415339207\\
63.125	0.19548	204.720282362858\\
63.125	0.19914	215.293813063095\\
63.125	0.2028	226.174007439918\\
63.125	0.20646	237.360865493328\\
63.125	0.21012	248.854387223325\\
63.125	0.21378	260.654572629907\\
63.125	0.21744	272.761421713076\\
63.125	0.2211	285.174934472831\\
63.125	0.22476	297.895110909173\\
63.125	0.22842	310.921951022101\\
63.125	0.23208	324.255454811616\\
63.125	0.23574	337.895622277716\\
63.125	0.2394	351.842453420403\\
63.125	0.24306	366.095948239677\\
63.125	0.24672	380.656106735537\\
63.125	0.25038	395.522928907983\\
63.125	0.25404	410.696414757015\\
63.125	0.2577	426.176564282634\\
63.125	0.26136	441.963377484839\\
63.125	0.26502	458.056854363631\\
63.125	0.26868	474.456994919009\\
63.125	0.27234	491.163799150973\\
63.125	0.276	508.177267059524\\
63.5	0.093	33.616164512852\\
63.5	0.09666	35.5977662261234\\
63.5	0.10032	37.8860316159811\\
63.5	0.10398	40.4809606824252\\
63.5	0.10764	43.3825534254557\\
63.5	0.1113	46.5908098450725\\
63.5	0.11496	50.1057299412757\\
63.5	0.11862	53.9273137140651\\
63.5	0.12228	58.0555611634411\\
63.5	0.12594	62.4904722894033\\
63.5	0.1296	67.2320470919518\\
63.5	0.13326	72.2802855710867\\
63.5	0.13692	77.635187726808\\
63.5	0.14058	83.2967535591157\\
63.5	0.14424	89.2649830680097\\
63.5	0.1479	95.53987625349\\
63.5	0.15156	102.121433115557\\
63.5	0.15522	109.00965365421\\
63.5	0.15888	116.204537869449\\
63.5	0.16254	123.706085761275\\
63.5	0.1662	131.514297329687\\
63.5	0.16986	139.629172574685\\
63.5	0.17352	148.05071149627\\
63.5	0.17718	156.778914094441\\
63.5	0.18084	165.813780369199\\
63.5	0.1845	175.155310320543\\
63.5	0.18816	184.803503948473\\
63.5	0.19182	194.758361252989\\
63.5	0.19548	205.019882234092\\
63.5	0.19914	215.588066891782\\
63.5	0.2028	226.462915226057\\
63.5	0.20646	237.644427236919\\
63.5	0.21012	249.132602924367\\
63.5	0.21378	260.927442288402\\
63.5	0.21744	273.028945329023\\
63.5	0.2211	285.43711204623\\
63.5	0.22476	298.151942440024\\
63.5	0.22842	311.173436510404\\
63.5	0.23208	324.501594257371\\
63.5	0.23574	338.136415680924\\
63.5	0.2394	352.077900781063\\
63.5	0.24306	366.326049557788\\
63.5	0.24672	380.8808620111\\
63.5	0.25038	395.742338140998\\
63.5	0.25404	410.910477947483\\
63.5	0.2577	426.385281430554\\
63.5	0.26136	442.166748590211\\
63.5	0.26502	458.254879426454\\
63.5	0.26868	474.649673939284\\
63.5	0.27234	491.351132128701\\
63.5	0.276	508.359253994703\\
63.875	0.093	34.0764110559942\\
63.875	0.09666	36.0526667267177\\
63.875	0.10032	38.3355860740275\\
63.875	0.10398	40.9251690979236\\
63.875	0.10764	43.8214157984061\\
63.875	0.1113	47.024326175475\\
63.875	0.11496	50.5339002291302\\
63.875	0.11862	54.3501379593718\\
63.875	0.12228	58.4730393661997\\
63.875	0.12594	62.9026044496139\\
63.875	0.1296	67.6388332096145\\
63.875	0.13326	72.6817256462014\\
63.875	0.13692	78.0312817593748\\
63.875	0.14058	83.6875015491345\\
63.875	0.14424	89.6503850154806\\
63.875	0.1479	95.9199321584129\\
63.875	0.15156	102.496142977932\\
63.875	0.15522	109.379017474037\\
63.875	0.15888	116.568555646728\\
63.875	0.16254	124.064757496006\\
63.875	0.1662	131.86762302187\\
63.875	0.16986	139.977152224321\\
63.875	0.17352	148.393345103357\\
63.875	0.17718	157.116201658981\\
63.875	0.18084	166.14572189119\\
63.875	0.1845	175.481905799986\\
63.875	0.18816	185.124753385368\\
63.875	0.19182	195.074264647337\\
63.875	0.19548	205.330439585892\\
63.875	0.19914	215.893278201033\\
63.875	0.2028	226.762780492761\\
63.875	0.20646	237.938946461075\\
63.875	0.21012	249.421776105975\\
63.875	0.21378	261.211269427462\\
63.875	0.21744	273.307426425535\\
63.875	0.2211	285.710247100194\\
63.875	0.22476	298.41973145144\\
63.875	0.22842	311.435879479272\\
63.875	0.23208	324.758691183691\\
63.875	0.23574	338.388166564696\\
63.875	0.2394	352.324305622287\\
63.875	0.24306	366.567108356464\\
63.875	0.24672	381.116574767228\\
63.875	0.25038	395.972704854578\\
63.875	0.25404	411.135498618515\\
63.875	0.2577	426.604956059038\\
63.875	0.26136	442.381077176147\\
63.875	0.26502	458.463861969843\\
63.875	0.26868	474.853310440125\\
63.875	0.27234	491.549422586993\\
63.875	0.276	508.552198410448\\
64.25	0.093	34.5476150797014\\
64.25	0.09666	36.5185247078769\\
64.25	0.10032	38.7960980126388\\
64.25	0.10398	41.3803349939869\\
64.25	0.10764	44.2712356519215\\
64.25	0.1113	47.4687999864424\\
64.25	0.11496	50.9730279975497\\
64.25	0.11862	54.7839196852433\\
64.25	0.12228	58.9014750495232\\
64.25	0.12594	63.3256940903896\\
64.25	0.1296	68.0565768078422\\
64.25	0.13326	73.0941232018812\\
64.25	0.13692	78.4383332725066\\
64.25	0.14058	84.0892070197183\\
64.25	0.14424	90.0467444435164\\
64.25	0.1479	96.3109455439007\\
64.25	0.15156	102.881810320872\\
64.25	0.15522	109.759338774429\\
64.25	0.15888	116.943530904572\\
64.25	0.16254	124.434386711302\\
64.25	0.1662	132.231906194618\\
64.25	0.16986	140.336089354521\\
64.25	0.17352	148.74693619101\\
64.25	0.17718	157.464446704085\\
64.25	0.18084	166.488620893746\\
64.25	0.1845	175.819458759994\\
64.25	0.18816	185.456960302829\\
64.25	0.19182	195.401125522249\\
64.25	0.19548	205.651954418256\\
64.25	0.19914	216.20944699085\\
64.25	0.2028	227.073603240029\\
64.25	0.20646	238.244423165795\\
64.25	0.21012	249.721906768148\\
64.25	0.21378	261.506054047086\\
64.25	0.21744	273.596865002612\\
64.25	0.2211	285.994339634723\\
64.25	0.22476	298.698477943421\\
64.25	0.22842	311.709279928705\\
64.25	0.23208	325.026745590576\\
64.25	0.23574	338.650874929033\\
64.25	0.2394	352.581667944076\\
64.25	0.24306	366.819124635705\\
64.25	0.24672	381.363245003921\\
64.25	0.25038	396.214029048723\\
64.25	0.25404	411.371476770112\\
64.25	0.2577	426.835588168087\\
64.25	0.26136	442.606363242648\\
64.25	0.26502	458.683801993796\\
64.25	0.26868	475.06790442153\\
64.25	0.27234	491.75867052585\\
64.25	0.276	508.756100306757\\
64.625	0.093	35.0297765839735\\
64.625	0.09666	36.995340169601\\
64.625	0.10032	39.2675674318149\\
64.625	0.10398	41.8464583706151\\
64.625	0.10764	44.7320129860017\\
64.625	0.1113	47.9242312779747\\
64.625	0.11496	51.423113246534\\
64.625	0.11862	55.2286588916796\\
64.625	0.12228	59.3408682134116\\
64.625	0.12594	63.75974121173\\
64.625	0.1296	68.4852778866346\\
64.625	0.13326	73.5174782381257\\
64.625	0.13692	78.8563422662031\\
64.625	0.14058	84.501869970867\\
64.625	0.14424	90.454061352117\\
64.625	0.1479	96.7129164099535\\
64.625	0.15156	103.278435144376\\
64.625	0.15522	110.150617555386\\
64.625	0.15888	117.329463642981\\
64.625	0.16254	124.814973407163\\
64.625	0.1662	132.607146847931\\
64.625	0.16986	140.705983965286\\
64.625	0.17352	149.111484759227\\
64.625	0.17718	157.823649229754\\
64.625	0.18084	166.842477376867\\
64.625	0.1845	176.167969200567\\
64.625	0.18816	185.800124700854\\
64.625	0.19182	195.738943877726\\
64.625	0.19548	205.984426731186\\
64.625	0.19914	216.536573261231\\
64.625	0.2028	227.395383467863\\
64.625	0.20646	238.560857351081\\
64.625	0.21012	250.032994910885\\
64.625	0.21378	261.811796147276\\
64.625	0.21744	273.897261060253\\
64.625	0.2211	286.289389649817\\
64.625	0.22476	298.988181915966\\
64.625	0.22842	311.993637858703\\
64.625	0.23208	325.305757478025\\
64.625	0.23574	338.924540773934\\
64.625	0.2394	352.849987746429\\
64.625	0.24306	367.082098395511\\
64.625	0.24672	381.620872721179\\
64.625	0.25038	396.466310723433\\
64.625	0.25404	411.618412402274\\
64.625	0.2577	427.077177757701\\
64.625	0.26136	442.842606789714\\
64.625	0.26502	458.914699498314\\
64.625	0.26868	475.2934558835\\
64.625	0.27234	491.978875945273\\
64.625	0.276	508.970959683631\\
65	0.093	35.5228955688105\\
65	0.09666	37.48311311189\\
65	0.10032	39.749994331556\\
65	0.10398	42.3235392278083\\
65	0.10764	45.2037478006469\\
65	0.1113	48.3906200500719\\
65	0.11496	51.8841559760832\\
65	0.11862	55.6843555786809\\
65	0.12228	59.7912188578649\\
65	0.12594	64.2047458136353\\
65	0.1296	68.9249364459921\\
65	0.13326	73.9517907549352\\
65	0.13692	79.2853087404646\\
65	0.14058	84.9254904025805\\
65	0.14424	90.8723357412826\\
65	0.1479	97.1258447565712\\
65	0.15156	103.686017448446\\
65	0.15522	110.552853816907\\
65	0.15888	117.726353861955\\
65	0.16254	125.206517583589\\
65	0.1662	132.993344981809\\
65	0.16986	141.086836056616\\
65	0.17352	149.486990808009\\
65	0.17718	158.193809235988\\
65	0.18084	167.207291340553\\
65	0.1845	176.527437121706\\
65	0.18816	186.154246579444\\
65	0.19182	196.087719713769\\
65	0.19548	206.32785652468\\
65	0.19914	216.874657012177\\
65	0.2028	227.728121176261\\
65	0.20646	238.888249016931\\
65	0.21012	250.355040534188\\
65	0.21378	262.12849572803\\
65	0.21744	274.20861459846\\
65	0.2211	286.595397145475\\
65	0.22476	299.288843369077\\
65	0.22842	312.288953269265\\
65	0.23208	325.59572684604\\
65	0.23574	339.209164099401\\
65	0.2394	353.129265029348\\
65	0.24306	367.356029635882\\
65	0.24672	381.889457919002\\
65	0.25038	396.729549878708\\
65	0.25404	411.876305515001\\
65	0.2577	427.32972482788\\
65	0.26136	443.089807817345\\
65	0.26502	459.156554483397\\
65	0.26868	475.529964826035\\
65	0.27234	492.21003884526\\
65	0.276	509.19677654107\\
65.375	0.093	36.0269720342123\\
65.375	0.09666	37.9818435347439\\
65.375	0.10032	40.2433787118619\\
65.375	0.10398	42.8115775655662\\
65.375	0.10764	45.6864400958569\\
65.375	0.1113	48.8679663027339\\
65.375	0.11496	52.3561561861974\\
65.375	0.11862	56.1510097462471\\
65.375	0.12228	60.2525269828832\\
65.375	0.12594	64.6607078961056\\
65.375	0.1296	69.3755524859143\\
65.375	0.13326	74.3970607523095\\
65.375	0.13692	79.725232695291\\
65.375	0.14058	85.3600683148589\\
65.375	0.14424	91.301567611013\\
65.375	0.1479	97.5497305837536\\
65.375	0.15156	104.104557233081\\
65.375	0.15522	110.966047558994\\
65.375	0.15888	118.134201561493\\
65.375	0.16254	125.609019240579\\
65.375	0.1662	133.390500596252\\
65.375	0.16986	141.47864562851\\
65.375	0.17352	149.873454337355\\
65.375	0.17718	158.574926722787\\
65.375	0.18084	167.583062784804\\
65.375	0.1845	176.897862523408\\
65.375	0.18816	186.519325938599\\
65.375	0.19182	196.447453030376\\
65.375	0.19548	206.682243798739\\
65.375	0.19914	217.223698243688\\
65.375	0.2028	228.071816365224\\
65.375	0.20646	239.226598163346\\
65.375	0.21012	250.688043638055\\
65.375	0.21378	262.45615278935\\
65.375	0.21744	274.530925617231\\
65.375	0.2211	286.912362121698\\
65.375	0.22476	299.600462302752\\
65.375	0.22842	312.595226160393\\
65.375	0.23208	325.896653694619\\
65.375	0.23574	339.504744905432\\
65.375	0.2394	353.419499792832\\
65.375	0.24306	367.640918356817\\
65.375	0.24672	382.169000597389\\
65.375	0.25038	397.003746514548\\
65.375	0.25404	412.145156108293\\
65.375	0.2577	427.593229378624\\
65.375	0.26136	443.347966325541\\
65.375	0.26502	459.409366949045\\
65.375	0.26868	475.777431249135\\
65.375	0.27234	492.452159225812\\
65.375	0.276	509.433550879074\\
65.75	0.093	36.5420059801791\\
65.75	0.09666	38.4915314381627\\
65.75	0.10032	40.7477205727327\\
65.75	0.10398	43.3105733838891\\
65.75	0.10764	46.1800898716318\\
65.75	0.1113	49.3562700359609\\
65.75	0.11496	52.8391138768764\\
65.75	0.11862	56.6286213943781\\
65.75	0.12228	60.7247925884662\\
65.75	0.12594	65.1276274591407\\
65.75	0.1296	69.8371260064016\\
65.75	0.13326	74.8532882302487\\
65.75	0.13692	80.1761141306822\\
65.75	0.14058	85.8056037077021\\
65.75	0.14424	91.7417569613085\\
65.75	0.1479	97.9845738915011\\
65.75	0.15156	104.53405449828\\
65.75	0.15522	111.390198781645\\
65.75	0.15888	118.553006741597\\
65.75	0.16254	126.022478378135\\
65.75	0.1662	133.798613691259\\
65.75	0.16986	141.88141268097\\
65.75	0.17352	150.270875347267\\
65.75	0.17718	158.96700169015\\
65.75	0.18084	167.96979170962\\
65.75	0.1845	177.279245405676\\
65.75	0.18816	186.895362778319\\
65.75	0.19182	196.818143827548\\
65.75	0.19548	207.047588553363\\
65.75	0.19914	217.583696955764\\
65.75	0.2028	228.426469034752\\
65.75	0.20646	239.575904790326\\
65.75	0.21012	251.032004222487\\
65.75	0.21378	262.794767331234\\
65.75	0.21744	274.864194116567\\
65.75	0.2211	287.240284578487\\
65.75	0.22476	299.923038716993\\
65.75	0.22842	312.912456532085\\
65.75	0.23208	326.208538023764\\
65.75	0.23574	339.811283192029\\
65.75	0.2394	353.72069203688\\
65.75	0.24306	367.936764558318\\
65.75	0.24672	382.459500756342\\
65.75	0.25038	397.288900630953\\
65.75	0.25404	412.424964182149\\
65.75	0.2577	427.867691409932\\
65.75	0.26136	443.617082314302\\
65.75	0.26502	459.673136895258\\
65.75	0.26868	476.0358551528\\
65.75	0.27234	492.705237086928\\
65.75	0.276	509.681282697643\\
66.125	0.093	37.0679974067107\\
66.125	0.09666	39.0121768221464\\
66.125	0.10032	41.2630199141685\\
66.125	0.10398	43.8205266827769\\
66.125	0.10764	46.6846971279717\\
66.125	0.1113	49.8555312497528\\
66.125	0.11496	53.3330290481203\\
66.125	0.11862	57.1171905230741\\
66.125	0.12228	61.2080156746142\\
66.125	0.12594	65.6055045027408\\
66.125	0.1296	70.3096570074536\\
66.125	0.13326	75.3204731887528\\
66.125	0.13692	80.6379530466385\\
66.125	0.14058	86.2620965811104\\
66.125	0.14424	92.1929037921688\\
66.125	0.1479	98.4303746798133\\
66.125	0.15156	104.974509244044\\
66.125	0.15522	111.825307484862\\
66.125	0.15888	118.982769402265\\
66.125	0.16254	126.446894996255\\
66.125	0.1662	134.217684266832\\
66.125	0.16986	142.295137213995\\
66.125	0.17352	150.679253837744\\
66.125	0.17718	159.370034138079\\
66.125	0.18084	168.367478115001\\
66.125	0.1845	177.671585768509\\
66.125	0.18816	187.282357098604\\
66.125	0.19182	197.199792105284\\
66.125	0.19548	207.423890788552\\
66.125	0.19914	217.954653148405\\
66.125	0.2028	228.792079184845\\
66.125	0.20646	239.936168897871\\
66.125	0.21012	251.386922287484\\
66.125	0.21378	263.144339353683\\
66.125	0.21744	275.208420096468\\
66.125	0.2211	287.57916451584\\
66.125	0.22476	300.256572611798\\
66.125	0.22842	313.240644384342\\
66.125	0.23208	326.531379833473\\
66.125	0.23574	340.12877895919\\
66.125	0.2394	354.032841761494\\
66.125	0.24306	368.243568240383\\
66.125	0.24672	382.760958395859\\
66.125	0.25038	397.585012227922\\
66.125	0.25404	412.715729736571\\
66.125	0.2577	428.153110921806\\
66.125	0.26136	443.897155783628\\
66.125	0.26502	459.947864322036\\
66.125	0.26868	476.30523653703\\
66.125	0.27234	492.96927242861\\
66.125	0.276	509.939971996777\\
66.5	0.093	37.6049463138072\\
66.5	0.09666	39.543779686695\\
66.5	0.10032	41.7892767361691\\
66.5	0.10398	44.3414374622295\\
66.5	0.10764	47.2002618648764\\
66.5	0.1113	50.3657499441095\\
66.5	0.11496	53.8379016999291\\
66.5	0.11862	57.6167171323349\\
66.5	0.12228	61.7021962413271\\
66.5	0.12594	66.0943390269057\\
66.5	0.1296	70.7931454890706\\
66.5	0.13326	75.7986156278218\\
66.5	0.13692	81.1107494431595\\
66.5	0.14058	86.7295469350835\\
66.5	0.14424	92.6550081035939\\
66.5	0.1479	98.8871329486905\\
66.5	0.15156	105.425921470373\\
66.5	0.15522	112.271373668643\\
66.5	0.15888	119.423489543499\\
66.5	0.16254	126.882269094941\\
66.5	0.1662	134.647712322969\\
66.5	0.16986	142.719819227584\\
66.5	0.17352	151.098589808785\\
66.5	0.17718	159.784024066573\\
66.5	0.18084	168.776122000946\\
66.5	0.1845	178.074883611907\\
66.5	0.18816	187.680308899453\\
66.5	0.19182	197.592397863586\\
66.5	0.19548	207.811150504305\\
66.5	0.19914	218.336566821611\\
66.5	0.2028	229.168646815503\\
66.5	0.20646	240.307390485981\\
66.5	0.21012	251.752797833046\\
66.5	0.21378	263.504868856697\\
66.5	0.21744	275.563603556934\\
66.5	0.2211	287.929001933758\\
66.5	0.22476	300.601063987168\\
66.5	0.22842	313.579789717165\\
66.5	0.23208	326.865179123747\\
66.5	0.23574	340.457232206917\\
66.5	0.2394	354.355948966672\\
66.5	0.24306	368.561329403014\\
66.5	0.24672	383.073373515942\\
66.5	0.25038	397.892081305456\\
66.5	0.25404	413.017452771557\\
66.5	0.2577	428.449487914245\\
66.5	0.26136	444.188186733518\\
66.5	0.26502	460.233549229378\\
66.5	0.26868	476.585575401824\\
66.5	0.27234	493.244265250857\\
66.5	0.276	510.209618776476\\
66.875	0.093	38.1528527014687\\
66.875	0.09666	40.0863400318085\\
66.875	0.10032	42.3264910387347\\
66.875	0.10398	44.8733057222471\\
66.875	0.10764	47.726784082346\\
66.875	0.1113	50.8869261190312\\
66.875	0.11496	54.3537318323028\\
66.875	0.11862	58.1272012221607\\
66.875	0.12228	62.2073342886049\\
66.875	0.12594	66.5941310316355\\
66.875	0.1296	71.2875914512525\\
66.875	0.13326	76.2877155474558\\
66.875	0.13692	81.5945033202455\\
66.875	0.14058	87.2079547696216\\
66.875	0.14424	93.1280698955839\\
66.875	0.1479	99.3548486981326\\
66.875	0.15156	105.888291177268\\
66.875	0.15522	112.728397332989\\
66.875	0.15888	119.875167165297\\
66.875	0.16254	127.328600674191\\
66.875	0.1662	135.088697859672\\
66.875	0.16986	143.155458721738\\
66.875	0.17352	151.528883260392\\
66.875	0.17718	160.208971475631\\
66.875	0.18084	169.195723367457\\
66.875	0.1845	178.489138935869\\
66.875	0.18816	188.089218180868\\
66.875	0.19182	197.995961102453\\
66.875	0.19548	208.209367700624\\
66.875	0.19914	218.729437975382\\
66.875	0.2028	229.556171926726\\
66.875	0.20646	240.689569554656\\
66.875	0.21012	252.129630859173\\
66.875	0.21378	263.876355840276\\
66.875	0.21744	275.929744497965\\
66.875	0.2211	288.289796832241\\
66.875	0.22476	300.956512843103\\
66.875	0.22842	313.929892530552\\
66.875	0.23208	327.209935894586\\
66.875	0.23574	340.796642935208\\
66.875	0.2394	354.690013652415\\
66.875	0.24306	368.890048046209\\
66.875	0.24672	383.396746116589\\
66.875	0.25038	398.210107863556\\
66.875	0.25404	413.330133287109\\
66.875	0.2577	428.756822387248\\
66.875	0.26136	444.490175163974\\
66.875	0.26502	460.530191617286\\
66.875	0.26868	476.876871747184\\
66.875	0.27234	493.530215553669\\
66.875	0.276	510.49022303674\\
67.25	0.093	38.711716569695\\
67.25	0.09666	40.6398578574868\\
67.25	0.10032	42.874662821865\\
67.25	0.10398	45.4161314628295\\
67.25	0.10764	48.2642637803804\\
67.25	0.1113	51.4190597745177\\
67.25	0.11496	54.8805194452413\\
67.25	0.11862	58.6486427925513\\
67.25	0.12228	62.7234298164476\\
67.25	0.12594	67.1048805169302\\
67.25	0.1296	71.7929948939992\\
67.25	0.13326	76.7877729476546\\
67.25	0.13692	82.0892146778963\\
67.25	0.14058	87.6973200847244\\
67.25	0.14424	93.6120891681388\\
67.25	0.1479	99.8335219281395\\
67.25	0.15156	106.361618364727\\
67.25	0.15522	113.1963784779\\
67.25	0.15888	120.33780226766\\
67.25	0.16254	127.785889734006\\
67.25	0.1662	135.540640876939\\
67.25	0.16986	143.602055696458\\
67.25	0.17352	151.970134192563\\
67.25	0.17718	160.644876365254\\
67.25	0.18084	169.626282214532\\
67.25	0.1845	178.914351740397\\
67.25	0.18816	188.509084942847\\
67.25	0.19182	198.410481821884\\
67.25	0.19548	208.618542377508\\
67.25	0.19914	219.133266609717\\
67.25	0.2028	229.954654518513\\
67.25	0.20646	241.082706103896\\
67.25	0.21012	252.517421365864\\
67.25	0.21378	264.258800304419\\
67.25	0.21744	276.306842919561\\
67.25	0.2211	288.661549211289\\
67.25	0.22476	301.322919179603\\
67.25	0.22842	314.290952824503\\
67.25	0.23208	327.56565014599\\
67.25	0.23574	341.147011144064\\
67.25	0.2394	355.035035818723\\
67.25	0.24306	369.229724169969\\
67.25	0.24672	383.731076197801\\
67.25	0.25038	398.53909190222\\
67.25	0.25404	413.653771283225\\
67.25	0.2577	429.075114340816\\
67.25	0.26136	444.803121074994\\
67.25	0.26502	460.837791485758\\
67.25	0.26868	477.179125573108\\
67.25	0.27234	493.827123337045\\
67.25	0.276	510.781784777568\\
67.625	0.093	39.2815379184862\\
67.625	0.09666	41.2043331637301\\
67.625	0.10032	43.4337920855603\\
67.625	0.10398	45.9699146839769\\
67.625	0.10764	48.8127009589799\\
67.625	0.1113	51.9621509105691\\
67.625	0.11496	55.4182645387448\\
67.625	0.11862	59.1810418435068\\
67.625	0.12228	63.2504828248551\\
67.625	0.12594	67.6265874827898\\
67.625	0.1296	72.3093558173109\\
67.625	0.13326	77.2987878284183\\
67.625	0.13692	82.594883516112\\
67.625	0.14058	88.1976428803922\\
67.625	0.14424	94.1070659212586\\
67.625	0.1479	100.323152638711\\
67.625	0.15156	106.845903032751\\
67.625	0.15522	113.675317103376\\
67.625	0.15888	120.811394850588\\
67.625	0.16254	128.254136274386\\
67.625	0.1662	136.003541374771\\
67.625	0.16986	144.059610151742\\
67.625	0.17352	152.422342605299\\
67.625	0.17718	161.091738735443\\
67.625	0.18084	170.067798542172\\
67.625	0.1845	179.350522025489\\
67.625	0.18816	188.939909185392\\
67.625	0.19182	198.835960021881\\
67.625	0.19548	209.038674534956\\
67.625	0.19914	219.548052724618\\
67.625	0.2028	230.364094590866\\
67.625	0.20646	241.4868001337\\
67.625	0.21012	252.916169353121\\
67.625	0.21378	264.652202249128\\
67.625	0.21744	276.694898821722\\
67.625	0.2211	289.044259070901\\
67.625	0.22476	301.700282996668\\
67.625	0.22842	314.66297059902\\
67.625	0.23208	327.932321877959\\
67.625	0.23574	341.508336833484\\
67.625	0.2394	355.391015465596\\
67.625	0.24306	369.580357774294\\
67.625	0.24672	384.076363759578\\
67.625	0.25038	398.879033421449\\
67.625	0.25404	413.988366759906\\
67.625	0.2577	429.40436377495\\
67.625	0.26136	445.127024466579\\
67.625	0.26502	461.156348834795\\
67.625	0.26868	477.492336879597\\
67.625	0.27234	494.134988600986\\
67.625	0.276	511.084303998961\\
68	0.093	39.8623167478423\\
68	0.09666	41.7797659505383\\
68	0.10032	44.0038788298206\\
68	0.10398	46.5346553856892\\
68	0.10764	49.3720956181442\\
68	0.1113	52.5161995271854\\
68	0.11496	55.9669671128132\\
68	0.11862	59.7243983750272\\
68	0.12228	63.7884933138276\\
68	0.12594	68.1592519292143\\
68	0.1296	72.8366742211875\\
68	0.13326	77.8207601897468\\
68	0.13692	83.1115098348926\\
68	0.14058	88.7089231566248\\
68	0.14424	94.6130001549434\\
68	0.1479	100.823740829848\\
68	0.15156	107.341145181339\\
68	0.15522	114.165213209417\\
68	0.15888	121.295944914081\\
68	0.16254	128.733340295331\\
68	0.1662	136.477399353168\\
68	0.16986	144.528122087591\\
68	0.17352	152.8855084986\\
68	0.17718	161.549558586196\\
68	0.18084	170.520272350378\\
68	0.1845	179.797649791146\\
68	0.18816	189.381690908501\\
68	0.19182	199.272395702442\\
68	0.19548	209.469764172969\\
68	0.19914	219.973796320083\\
68	0.2028	230.784492143783\\
68	0.20646	241.90185164407\\
68	0.21012	253.325874820943\\
68	0.21378	265.056561674402\\
68	0.21744	277.093912204447\\
68	0.2211	289.437926411079\\
68	0.22476	302.088604294297\\
68	0.22842	315.045945854102\\
68	0.23208	328.309951090493\\
68	0.23574	341.88062000347\\
68	0.2394	355.757952593034\\
68	0.24306	369.941948859184\\
68	0.24672	384.43260880192\\
68	0.25038	399.229932421243\\
68	0.25404	414.333919717152\\
68	0.2577	429.744570689648\\
68	0.26136	445.461885338729\\
68	0.26502	461.485863664397\\
68	0.26868	477.816505666652\\
68	0.27234	494.453811345493\\
68	0.276	511.39778070092\\
68.375	0.093	40.4540530577633\\
68.375	0.09666	42.3661562179113\\
68.375	0.10032	44.5849230546456\\
68.375	0.10398	47.1103535679663\\
68.375	0.10764	49.9424477578733\\
68.375	0.1113	53.0812056243667\\
68.375	0.11496	56.5266271674464\\
68.375	0.11862	60.2787123871125\\
68.375	0.12228	64.3374612833649\\
68.375	0.12594	68.7028738562037\\
68.375	0.1296	73.3749501056288\\
68.375	0.13326	78.3536900316403\\
68.375	0.13692	83.6390936342381\\
68.375	0.14058	89.2311609134224\\
68.375	0.14424	95.129891869193\\
68.375	0.1479	101.33528650155\\
68.375	0.15156	107.847344810493\\
68.375	0.15522	114.666066796023\\
68.375	0.15888	121.791452458139\\
68.375	0.16254	129.223501796841\\
68.375	0.1662	136.96221481213\\
68.375	0.16986	145.007591504005\\
68.375	0.17352	153.359631872466\\
68.375	0.17718	162.018335917514\\
68.375	0.18084	170.983703639148\\
68.375	0.1845	180.255735037368\\
68.375	0.18816	189.834430112175\\
68.375	0.19182	199.719788863568\\
68.375	0.19548	209.911811291548\\
68.375	0.19914	220.410497396113\\
68.375	0.2028	231.215847177266\\
68.375	0.20646	242.327860635004\\
68.375	0.21012	253.746537769329\\
68.375	0.21378	265.47187858024\\
68.375	0.21744	277.503883067738\\
68.375	0.2211	289.842551231822\\
68.375	0.22476	302.487883072492\\
68.375	0.22842	315.439878589749\\
68.375	0.23208	328.698537783592\\
68.375	0.23574	342.263860654021\\
68.375	0.2394	356.135847201037\\
68.375	0.24306	370.314497424639\\
68.375	0.24672	384.799811324827\\
68.375	0.25038	399.591788901602\\
68.375	0.25404	414.690430154963\\
68.375	0.2577	430.09573508491\\
68.375	0.26136	445.807703691444\\
68.375	0.26502	461.826335974564\\
68.375	0.26868	478.151631934271\\
68.375	0.27234	494.783591570564\\
68.375	0.276	511.722214883443\\
68.75	0.093	41.0567468482492\\
68.75	0.09666	42.9635039658492\\
68.75	0.10032	45.1769247600356\\
68.75	0.10398	47.6970092308083\\
68.75	0.10764	50.5237573781674\\
68.75	0.1113	53.6571692021128\\
68.75	0.11496	57.0972447026446\\
68.75	0.11862	60.8439838797627\\
68.75	0.12228	64.8973867334672\\
68.75	0.12594	69.2574532637581\\
68.75	0.1296	73.9241834706352\\
68.75	0.13326	78.8975773540987\\
68.75	0.13692	84.1776349141486\\
68.75	0.14058	89.7643561507849\\
68.75	0.14424	95.6577410640075\\
68.75	0.1479	101.857789653816\\
68.75	0.15156	108.364501920212\\
68.75	0.15522	115.177877863193\\
68.75	0.15888	122.297917482761\\
68.75	0.16254	129.724620778916\\
68.75	0.1662	137.457987751656\\
68.75	0.16986	145.498018400983\\
68.75	0.17352	153.844712726897\\
68.75	0.17718	162.498070729397\\
68.75	0.18084	171.458092408483\\
68.75	0.1845	180.724777764155\\
68.75	0.18816	190.298126796414\\
68.75	0.19182	200.178139505259\\
68.75	0.19548	210.364815890691\\
68.75	0.19914	220.858155952709\\
68.75	0.2028	231.658159691313\\
68.75	0.20646	242.764827106503\\
68.75	0.21012	254.17815819828\\
68.75	0.21378	265.898152966643\\
68.75	0.21744	277.924811411593\\
68.75	0.2211	290.258133533129\\
68.75	0.22476	302.898119331251\\
68.75	0.22842	315.84476880596\\
68.75	0.23208	329.098081957255\\
68.75	0.23574	342.658058785137\\
68.75	0.2394	356.524699289604\\
68.75	0.24306	370.698003470658\\
68.75	0.24672	385.177971328299\\
68.75	0.25038	399.964602862526\\
68.75	0.25404	415.057898073339\\
68.75	0.2577	430.457856960738\\
68.75	0.26136	446.164479524724\\
68.75	0.26502	462.177765765296\\
68.75	0.26868	478.497715682455\\
68.75	0.27234	495.1243292762\\
68.75	0.276	512.057606546531\\
69.125	0.093	41.6703981193\\
69.125	0.09666	43.571809194352\\
69.125	0.10032	45.7798839459904\\
69.125	0.10398	48.2946223742152\\
69.125	0.10764	51.1160244790263\\
69.125	0.1113	54.2440902604238\\
69.125	0.11496	57.6788197184076\\
69.125	0.11862	61.4202128529778\\
69.125	0.12228	65.4682696641343\\
69.125	0.12594	69.8229901518772\\
69.125	0.1296	74.4843743162064\\
69.125	0.13326	79.4524221571219\\
69.125	0.13692	84.7271336746239\\
69.125	0.14058	90.3085088687122\\
69.125	0.14424	96.1965477393869\\
69.125	0.1479	102.391250286648\\
69.125	0.15156	108.892616510495\\
69.125	0.15522	115.700646410929\\
69.125	0.15888	122.815339987949\\
69.125	0.16254	130.236697241555\\
69.125	0.1662	137.964718171748\\
69.125	0.16986	145.999402778527\\
69.125	0.17352	154.340751061893\\
69.125	0.17718	162.988763021844\\
69.125	0.18084	171.943438658382\\
69.125	0.1845	181.204777971507\\
69.125	0.18816	190.772780961218\\
69.125	0.19182	200.647447627515\\
69.125	0.19548	210.828777970399\\
69.125	0.19914	221.316771989869\\
69.125	0.2028	232.111429685925\\
69.125	0.20646	243.212751058567\\
69.125	0.21012	254.620736107796\\
69.125	0.21378	266.335384833612\\
69.125	0.21744	278.356697236013\\
69.125	0.2211	290.684673315001\\
69.125	0.22476	303.319313070576\\
69.125	0.22842	316.260616502737\\
69.125	0.23208	329.508583611484\\
69.125	0.23574	343.063214396817\\
69.125	0.2394	356.924508858737\\
69.125	0.24306	371.092466997243\\
69.125	0.24672	385.567088812335\\
69.125	0.25038	400.348374304014\\
69.125	0.25404	415.436323472279\\
69.125	0.2577	430.830936317131\\
69.125	0.26136	446.532212838569\\
69.125	0.26502	462.540153036593\\
69.125	0.26868	478.854756911204\\
69.125	0.27234	495.476024462401\\
69.125	0.276	512.403955690184\\
69.5	0.093	42.2950068709157\\
69.5	0.09666	44.1910719034198\\
69.5	0.10032	46.3938006125102\\
69.5	0.10398	48.903192998187\\
69.5	0.10764	51.7192490604502\\
69.5	0.1113	54.8419687992997\\
69.5	0.11496	58.2713522147356\\
69.5	0.11862	62.0073993067578\\
69.5	0.12228	66.0501100753664\\
69.5	0.12594	70.3994845205613\\
69.5	0.1296	75.0555226423425\\
69.5	0.13326	80.0182244407101\\
69.5	0.13692	85.2875899156641\\
69.5	0.14058	90.8636190672045\\
69.5	0.14424	96.7463118953312\\
69.5	0.1479	102.935668400044\\
69.5	0.15156	109.431688581344\\
69.5	0.15522	116.234372439229\\
69.5	0.15888	123.343719973701\\
69.5	0.16254	130.75973118476\\
69.5	0.1662	138.482406072405\\
69.5	0.16986	146.511744636636\\
69.5	0.17352	154.847746877453\\
69.5	0.17718	163.490412794857\\
69.5	0.18084	172.439742388847\\
69.5	0.1845	181.695735659424\\
69.5	0.18816	191.258392606587\\
69.5	0.19182	201.127713230336\\
69.5	0.19548	211.303697530672\\
69.5	0.19914	221.786345507594\\
69.5	0.2028	232.575657161102\\
69.5	0.20646	243.671632491197\\
69.5	0.21012	255.074271497878\\
69.5	0.21378	266.783574181145\\
69.5	0.21744	278.799540540999\\
69.5	0.2211	291.122170577439\\
69.5	0.22476	303.751464290465\\
69.5	0.22842	316.687421680078\\
69.5	0.23208	329.930042746277\\
69.5	0.23574	343.479327489062\\
69.5	0.2394	357.335275908434\\
69.5	0.24306	371.497888004392\\
69.5	0.24672	385.967163776937\\
69.5	0.25038	400.743103226068\\
69.5	0.25404	415.825706351785\\
69.5	0.2577	431.214973154089\\
69.5	0.26136	446.910903632979\\
69.5	0.26502	462.913497788455\\
69.5	0.26868	479.222755620517\\
69.5	0.27234	495.838677129166\\
69.5	0.276	512.761262314402\\
69.875	0.093	42.9305731030962\\
69.875	0.09666	44.8212920930523\\
69.875	0.10032	47.0186747595949\\
69.875	0.10398	49.5227211027237\\
69.875	0.10764	52.3334311224389\\
69.875	0.1113	55.4508048187404\\
69.875	0.11496	58.8748421916284\\
69.875	0.11862	62.6055432411026\\
69.875	0.12228	66.6429079671632\\
69.875	0.12594	70.9869363698102\\
69.875	0.1296	75.6376284490434\\
69.875	0.13326	80.5949842048631\\
69.875	0.13692	85.8590036372692\\
69.875	0.14058	91.4296867462616\\
69.875	0.14424	97.3070335318403\\
69.875	0.1479	103.491043994005\\
69.875	0.15156	109.981718132757\\
69.875	0.15522	116.779055948095\\
69.875	0.15888	123.883057440019\\
69.875	0.16254	131.293722608529\\
69.875	0.1662	139.011051453626\\
69.875	0.16986	147.035043975309\\
69.875	0.17352	155.365700173579\\
69.875	0.17718	164.003020048435\\
69.875	0.18084	172.947003599877\\
69.875	0.1845	182.197650827905\\
69.875	0.18816	191.75496173252\\
69.875	0.19182	201.618936313722\\
69.875	0.19548	211.789574571509\\
69.875	0.19914	222.266876505883\\
69.875	0.2028	233.050842116844\\
69.875	0.20646	244.14147140439\\
69.875	0.21012	255.538764368523\\
69.875	0.21378	267.242721009243\\
69.875	0.21744	279.253341326549\\
69.875	0.2211	291.570625320441\\
69.875	0.22476	304.194572990919\\
69.875	0.22842	317.125184337984\\
69.875	0.23208	330.362459361635\\
69.875	0.23574	343.906398061873\\
69.875	0.2394	357.757000438696\\
69.875	0.24306	371.914266492107\\
69.875	0.24672	386.378196222103\\
69.875	0.25038	401.148789628686\\
69.875	0.25404	416.226046711855\\
69.875	0.2577	431.609967471611\\
69.875	0.26136	447.300551907953\\
69.875	0.26502	463.297800020881\\
69.875	0.26868	479.601711810396\\
69.875	0.27234	496.212287276497\\
69.875	0.276	513.129526419184\\
70.25	0.093	43.5770968158417\\
70.25	0.09666	45.4624697632499\\
70.25	0.10032	47.6545063872444\\
70.25	0.10398	50.1532066878253\\
70.25	0.10764	52.9585706649925\\
70.25	0.1113	56.0705983187461\\
70.25	0.11496	59.4892896490861\\
70.25	0.11862	63.2146446560124\\
70.25	0.12228	67.246663339525\\
70.25	0.12594	71.585345699624\\
70.25	0.1296	76.2306917363094\\
70.25	0.13326	81.1827014495811\\
70.25	0.13692	86.4413748394391\\
70.25	0.14058	92.0067119058836\\
70.25	0.14424	97.8787126489143\\
70.25	0.1479	104.057377068531\\
70.25	0.15156	110.542705164735\\
70.25	0.15522	117.334696937525\\
70.25	0.15888	124.433352386901\\
70.25	0.16254	131.838671512863\\
70.25	0.1662	139.550654315412\\
70.25	0.16986	147.569300794548\\
70.25	0.17352	155.894610950269\\
70.25	0.17718	164.526584782577\\
70.25	0.18084	173.465222291471\\
70.25	0.1845	182.710523476952\\
70.25	0.18816	192.262488339019\\
70.25	0.19182	202.121116877672\\
70.25	0.19548	212.286409092912\\
70.25	0.19914	222.758364984738\\
70.25	0.2028	233.53698455315\\
70.25	0.20646	244.622267798149\\
70.25	0.21012	256.014214719734\\
70.25	0.21378	267.712825317906\\
70.25	0.21744	279.718099592663\\
70.25	0.2211	292.030037544008\\
70.25	0.22476	304.648639171938\\
70.25	0.22842	317.573904476455\\
70.25	0.23208	330.805833457558\\
70.25	0.23574	344.344426115248\\
70.25	0.2394	358.189682449524\\
70.25	0.24306	372.341602460386\\
70.25	0.24672	386.800186147835\\
70.25	0.25038	401.56543351187\\
70.25	0.25404	416.637344552491\\
70.25	0.2577	432.015919269699\\
70.25	0.26136	447.701157663493\\
70.25	0.26502	463.693059733873\\
70.25	0.26868	479.99162548084\\
70.25	0.27234	496.596854904393\\
70.25	0.276	513.508748004532\\
70.625	0.093	44.234578009152\\
70.625	0.09666	46.1146049140123\\
70.625	0.10032	48.3012954954589\\
70.625	0.10398	50.7946497534918\\
70.625	0.10764	53.5946676881111\\
70.625	0.1113	56.7013492993167\\
70.625	0.11496	60.1146945871087\\
70.625	0.11862	63.8347035514871\\
70.625	0.12228	67.8613761924518\\
70.625	0.12594	72.1947125100028\\
70.625	0.1296	76.8347125041402\\
70.625	0.13326	81.7813761748639\\
70.625	0.13692	87.034703522174\\
70.625	0.14058	92.5946945460705\\
70.625	0.14424	98.4613492465534\\
70.625	0.1479	104.634667623623\\
70.625	0.15156	111.114649677278\\
70.625	0.15522	117.90129540752\\
70.625	0.15888	124.994604814348\\
70.625	0.16254	132.394577897763\\
70.625	0.1662	140.101214657764\\
70.625	0.16986	148.114515094351\\
70.625	0.17352	156.434479207524\\
70.625	0.17718	165.061106997284\\
70.625	0.18084	173.994398463631\\
70.625	0.1845	183.234353606563\\
70.625	0.18816	192.780972426082\\
70.625	0.19182	202.634254922188\\
70.625	0.19548	212.79420109488\\
70.625	0.19914	223.260810944158\\
70.625	0.2028	234.034084470022\\
70.625	0.20646	245.114021672473\\
70.625	0.21012	256.50062255151\\
70.625	0.21378	268.193887107133\\
70.625	0.21744	280.193815339343\\
70.625	0.2211	292.500407248139\\
70.625	0.22476	305.113662833522\\
70.625	0.22842	318.033582095491\\
70.625	0.23208	331.260165034046\\
70.625	0.23574	344.793411649188\\
70.625	0.2394	358.633321940916\\
70.625	0.24306	372.77989590923\\
70.625	0.24672	387.233133554131\\
70.625	0.25038	401.993034875618\\
70.625	0.25404	417.059599873691\\
70.625	0.2577	432.432828548351\\
70.625	0.26136	448.112720899597\\
70.625	0.26502	464.099276927429\\
70.625	0.26868	480.392496631848\\
70.625	0.27234	496.992380012853\\
70.625	0.276	513.898927070445\\
71	0.093	44.9030166830273\\
71	0.09666	46.7776975453395\\
71	0.10032	48.9590420842382\\
71	0.10398	51.4470502997231\\
71	0.10764	54.2417221917945\\
71	0.1113	57.3430577604521\\
71	0.11496	60.7510570056962\\
71	0.11862	64.4657199275266\\
71	0.12228	68.4870465259433\\
71	0.12594	72.8150368009464\\
71	0.1296	77.4496907525358\\
71	0.13326	82.3910083807116\\
71	0.13692	87.6389896854737\\
71	0.14058	93.1936346668223\\
71	0.14424	99.0549433247572\\
71	0.1479	105.222915659278\\
71	0.15156	111.697551670386\\
71	0.15522	118.47885135808\\
71	0.15888	125.56681472236\\
71	0.16254	132.961441763227\\
71	0.1662	140.66273248068\\
71	0.16986	148.670686874719\\
71	0.17352	156.985304945345\\
71	0.17718	165.606586692557\\
71	0.18084	174.534532116355\\
71	0.1845	183.76914121674\\
71	0.18816	193.310413993711\\
71	0.19182	203.158350447268\\
71	0.19548	213.312950577412\\
71	0.19914	223.774214384142\\
71	0.2028	234.542141867459\\
71	0.20646	245.616733027361\\
71	0.21012	256.997987863851\\
71	0.21378	268.685906376926\\
71	0.21744	280.680488566588\\
71	0.2211	292.981734432836\\
71	0.22476	305.589643975671\\
71	0.22842	318.504217195092\\
71	0.23208	331.725454091099\\
71	0.23574	345.253354663693\\
71	0.2394	359.087918912873\\
71	0.24306	373.229146838639\\
71	0.24672	387.677038440992\\
71	0.25038	402.431593719931\\
71	0.25404	417.492812675456\\
71	0.2577	432.860695307568\\
71	0.26136	448.535241616266\\
71	0.26502	464.516451601551\\
71	0.26868	480.804325263421\\
71	0.27234	497.398862601878\\
71	0.276	514.300063616922\\
71.375	0.093	45.5824128374674\\
71.375	0.09666	47.4517476572317\\
71.375	0.10032	49.6277461535824\\
71.375	0.10398	52.1104083265194\\
71.375	0.10764	54.8997341760428\\
71.375	0.1113	57.9957237021525\\
71.375	0.11496	61.3983769048486\\
71.375	0.11862	65.107693784131\\
71.375	0.12228	69.1236743399998\\
71.375	0.12594	73.4463185724549\\
71.375	0.1296	78.0756264814964\\
71.375	0.13326	83.0115980671242\\
71.375	0.13692	88.2542333293385\\
71.375	0.14058	93.803532268139\\
71.375	0.14424	99.659494883526\\
71.375	0.1479	105.822121175499\\
71.375	0.15156	112.291411144059\\
71.375	0.15522	119.067364789205\\
71.375	0.15888	126.149982110937\\
71.375	0.16254	133.539263109256\\
71.375	0.1662	141.235207784161\\
71.375	0.16986	149.237816135652\\
71.375	0.17352	157.54708816373\\
71.375	0.17718	166.163023868394\\
71.375	0.18084	175.085623249644\\
71.375	0.1845	184.314886307481\\
71.375	0.18816	193.850813041904\\
71.375	0.19182	203.693403452914\\
71.375	0.19548	213.842657540509\\
71.375	0.19914	224.298575304692\\
71.375	0.2028	235.06115674546\\
71.375	0.20646	246.130401862815\\
71.375	0.21012	257.506310656756\\
71.375	0.21378	269.188883127284\\
71.375	0.21744	281.178119274398\\
71.375	0.2211	293.474019098098\\
71.375	0.22476	306.076582598385\\
71.375	0.22842	318.985809775258\\
71.375	0.23208	332.201700628717\\
71.375	0.23574	345.724255158763\\
71.375	0.2394	359.553473365395\\
71.375	0.24306	373.689355248613\\
71.375	0.24672	388.131900808418\\
71.375	0.25038	402.881110044809\\
71.375	0.25404	417.936982957786\\
71.375	0.2577	433.29951954735\\
71.375	0.26136	448.9687198135\\
71.375	0.26502	464.944583756237\\
71.375	0.26868	481.22711137556\\
71.375	0.27234	497.816302671469\\
71.375	0.276	514.712157643964\\
71.75	0.093	46.2727664724724\\
71.75	0.09666	48.1367552496888\\
71.75	0.10032	50.3074077034915\\
71.75	0.10398	52.7847238338806\\
71.75	0.10764	55.568703640856\\
71.75	0.1113	58.6593471244177\\
71.75	0.11496	62.0566542845659\\
71.75	0.11862	65.7606251213003\\
71.75	0.12228	69.7712596346212\\
71.75	0.12594	74.0885578245284\\
71.75	0.1296	78.7125196910218\\
71.75	0.13326	83.6431452341017\\
71.75	0.13692	88.880434453768\\
71.75	0.14058	94.4243873500206\\
71.75	0.14424	100.27500392286\\
71.75	0.1479	106.432284172285\\
71.75	0.15156	112.896228098296\\
71.75	0.15522	119.666835700894\\
71.75	0.15888	126.744106980079\\
71.75	0.16254	134.128041935849\\
71.75	0.1662	141.818640568207\\
71.75	0.16986	149.81590287715\\
71.75	0.17352	158.11982886268\\
71.75	0.17718	166.730418524796\\
71.75	0.18084	175.647671863498\\
71.75	0.1845	184.871588878787\\
71.75	0.18816	194.402169570662\\
71.75	0.19182	204.239413939124\\
71.75	0.19548	214.383321984172\\
71.75	0.19914	224.833893705806\\
71.75	0.2028	235.591129104026\\
71.75	0.20646	246.655028178833\\
71.75	0.21012	258.025590930227\\
71.75	0.21378	269.702817358206\\
71.75	0.21744	281.686707462772\\
71.75	0.2211	293.977261243924\\
71.75	0.22476	306.574478701663\\
71.75	0.22842	319.478359835988\\
71.75	0.23208	332.6889046469\\
71.75	0.23574	346.206113134397\\
71.75	0.2394	360.029985298481\\
71.75	0.24306	374.160521139152\\
71.75	0.24672	388.597720656409\\
71.75	0.25038	403.341583850252\\
71.75	0.25404	418.392110720681\\
71.75	0.2577	433.749301267697\\
71.75	0.26136	449.413155491299\\
71.75	0.26502	465.383673391488\\
71.75	0.26868	481.660854968263\\
71.75	0.27234	498.244700221624\\
71.75	0.276	515.135209151571\\
72.125	0.093	46.9740775880423\\
72.125	0.09666	48.8327203227107\\
72.125	0.10032	50.9980267339655\\
72.125	0.10398	53.4699968218066\\
72.125	0.10764	56.2486305862341\\
72.125	0.1113	59.3339280272479\\
72.125	0.11496	62.7258891448481\\
72.125	0.11862	66.4245139390346\\
72.125	0.12228	70.4298024098075\\
72.125	0.12594	74.7417545571667\\
72.125	0.1296	79.3603703811122\\
72.125	0.13326	84.2856498816441\\
72.125	0.13692	89.5175930587625\\
72.125	0.14058	95.0561999124671\\
72.125	0.14424	100.901470442758\\
72.125	0.1479	107.053404649635\\
72.125	0.15156	113.512002533099\\
72.125	0.15522	120.277264093149\\
72.125	0.15888	127.349189329786\\
72.125	0.16254	134.727778243008\\
72.125	0.1662	142.413030832817\\
72.125	0.16986	150.404947099213\\
72.125	0.17352	158.703527042195\\
72.125	0.17718	167.308770661763\\
72.125	0.18084	176.220677957917\\
72.125	0.1845	185.439248930658\\
72.125	0.18816	194.964483579985\\
72.125	0.19182	204.796381905899\\
72.125	0.19548	214.934943908399\\
72.125	0.19914	225.380169587485\\
72.125	0.2028	236.132058943158\\
72.125	0.20646	247.190611975417\\
72.125	0.21012	258.555828684262\\
72.125	0.21378	270.227709069693\\
72.125	0.21744	282.206253131712\\
72.125	0.2211	294.491460870316\\
72.125	0.22476	307.083332285507\\
72.125	0.22842	319.981867377284\\
72.125	0.23208	333.187066145647\\
72.125	0.23574	346.698928590597\\
72.125	0.2394	360.517454712133\\
72.125	0.24306	374.642644510255\\
72.125	0.24672	389.074497984964\\
72.125	0.25038	403.81301513626\\
72.125	0.25404	418.858195964141\\
72.125	0.2577	434.210040468609\\
72.125	0.26136	449.868548649663\\
72.125	0.26502	465.833720507304\\
72.125	0.26868	482.105556041531\\
72.125	0.27234	498.684055252344\\
72.125	0.276	515.569218139744\\
72.5	0.093	47.6863461841771\\
72.5	0.09666	49.5396428762976\\
72.5	0.10032	51.6996032450044\\
72.5	0.10398	54.1662272902975\\
72.5	0.10764	56.9395150121771\\
72.5	0.1113	60.0194664106429\\
72.5	0.11496	63.4060814856951\\
72.5	0.11862	67.0993602373337\\
72.5	0.12228	71.0993026655586\\
72.5	0.12594	75.4059087703698\\
72.5	0.1296	80.0191785517674\\
72.5	0.13326	84.9391120097514\\
72.5	0.13692	90.1657091443217\\
72.5	0.14058	95.6989699554785\\
72.5	0.14424	101.538894443221\\
72.5	0.1479	107.685482607551\\
72.5	0.15156	114.138734448467\\
72.5	0.15522	120.898649965969\\
72.5	0.15888	127.965229160057\\
72.5	0.16254	135.338472030732\\
72.5	0.1662	143.018378577993\\
72.5	0.16986	151.00494880184\\
72.5	0.17352	159.298182702274\\
72.5	0.17718	167.898080279295\\
72.5	0.18084	176.804641532901\\
72.5	0.1845	186.017866463094\\
72.5	0.18816	195.537755069873\\
72.5	0.19182	205.364307353239\\
72.5	0.19548	215.497523313191\\
72.5	0.19914	225.937402949729\\
72.5	0.2028	236.683946262854\\
72.5	0.20646	247.737153252565\\
72.5	0.21012	259.097023918862\\
72.5	0.21378	270.763558261746\\
72.5	0.21744	282.736756281216\\
72.5	0.2211	295.016617977272\\
72.5	0.22476	307.603143349915\\
72.5	0.22842	320.496332399144\\
72.5	0.23208	333.69618512496\\
72.5	0.23574	347.202701527362\\
72.5	0.2394	361.01588160635\\
72.5	0.24306	375.135725361924\\
72.5	0.24672	389.562232794085\\
72.5	0.25038	404.295403902832\\
72.5	0.25404	419.335238688166\\
72.5	0.2577	434.681737150086\\
72.5	0.26136	450.334899288592\\
72.5	0.26502	466.294725103685\\
72.5	0.26868	482.561214595364\\
72.5	0.27234	499.134367763629\\
72.5	0.276	516.014184608481\\
72.875	0.093	48.4095722608768\\
72.875	0.09666	50.2575229104493\\
72.875	0.10032	52.4121372366082\\
72.875	0.10398	54.8734152393533\\
72.875	0.10764	57.6413569186849\\
72.875	0.1113	60.7159622746028\\
72.875	0.11496	64.0972313071071\\
72.875	0.11862	67.7851640161977\\
72.875	0.12228	71.7797604018747\\
72.875	0.12594	76.0810204641379\\
72.875	0.1296	80.6889442029876\\
72.875	0.13326	85.6035316184236\\
72.875	0.13692	90.824782710446\\
72.875	0.14058	96.3526974790547\\
72.875	0.14424	102.18727592425\\
72.875	0.1479	108.328518046031\\
72.875	0.15156	114.776423844399\\
72.875	0.15522	121.530993319353\\
72.875	0.15888	128.592226470894\\
72.875	0.16254	135.96012329902\\
72.875	0.1662	143.634683803734\\
72.875	0.16986	151.615907985033\\
72.875	0.17352	159.903795842919\\
72.875	0.17718	168.498347377391\\
72.875	0.18084	177.39956258845\\
72.875	0.1845	186.607441476095\\
72.875	0.18816	196.121984040326\\
72.875	0.19182	205.943190281144\\
72.875	0.19548	216.071060198548\\
72.875	0.19914	226.505593792538\\
72.875	0.2028	237.246791063115\\
72.875	0.20646	248.294652010278\\
72.875	0.21012	259.649176634027\\
72.875	0.21378	271.310364934363\\
72.875	0.21744	283.278216911285\\
72.875	0.2211	295.552732564793\\
72.875	0.22476	308.133911894888\\
72.875	0.22842	321.021754901569\\
72.875	0.23208	334.216261584837\\
72.875	0.23574	347.717431944691\\
72.875	0.2394	361.525265981131\\
72.875	0.24306	375.639763694158\\
72.875	0.24672	390.06092508377\\
72.875	0.25038	404.78875014997\\
72.875	0.25404	419.823238892755\\
72.875	0.2577	435.164391312127\\
72.875	0.26136	450.812207408086\\
72.875	0.26502	466.76668718063\\
72.875	0.26868	483.027830629761\\
72.875	0.27234	499.595637755479\\
72.875	0.276	516.470108557782\\
73.25	0.093	49.1437558181414\\
73.25	0.09666	50.986360425166\\
73.25	0.10032	53.1356287087768\\
73.25	0.10398	55.5915606689741\\
73.25	0.10764	58.3541563057577\\
73.25	0.1113	61.4234156191276\\
73.25	0.11496	64.799338609084\\
73.25	0.11862	68.4819252756266\\
73.25	0.12228	72.4711756187556\\
73.25	0.12594	76.767089638471\\
73.25	0.1296	81.3696673347727\\
73.25	0.13326	86.2789087076607\\
73.25	0.13692	91.4948137571351\\
73.25	0.14058	97.0173824831959\\
73.25	0.14424	102.846614885843\\
73.25	0.1479	108.982510965076\\
73.25	0.15156	115.425070720896\\
73.25	0.15522	122.174294153302\\
73.25	0.15888	129.230181262295\\
73.25	0.16254	136.592732047874\\
73.25	0.1662	144.261946510039\\
73.25	0.16986	152.237824648791\\
73.25	0.17352	160.520366464129\\
73.25	0.17718	169.109571956053\\
73.25	0.18084	178.005441124563\\
73.25	0.1845	187.20797396966\\
73.25	0.18816	196.717170491344\\
73.25	0.19182	206.533030689613\\
73.25	0.19548	216.655554564469\\
73.25	0.19914	227.084742115912\\
73.25	0.2028	237.820593343941\\
73.25	0.20646	248.863108248556\\
73.25	0.21012	260.212286829757\\
73.25	0.21378	271.868129087545\\
73.25	0.21744	283.830635021919\\
73.25	0.2211	296.099804632879\\
73.25	0.22476	308.675637920427\\
73.25	0.22842	321.55813488456\\
73.25	0.23208	334.747295525279\\
73.25	0.23574	348.243119842585\\
73.25	0.2394	362.045607836477\\
73.25	0.24306	376.154759506956\\
73.25	0.24672	390.570574854021\\
73.25	0.25038	405.293053877672\\
73.25	0.25404	420.32219657791\\
73.25	0.2577	435.658002954734\\
73.25	0.26136	451.300473008144\\
73.25	0.26502	467.249606738141\\
73.25	0.26868	483.505404144724\\
73.25	0.27234	500.067865227893\\
73.25	0.276	516.936989987649\\
73.625	0.093	49.8888968559709\\
73.625	0.09666	51.7261554204475\\
73.625	0.10032	53.8700776615104\\
73.625	0.10398	56.3206635791597\\
73.625	0.10764	59.0779131733954\\
73.625	0.1113	62.1418264442173\\
73.625	0.11496	65.5124033916257\\
73.625	0.11862	69.1896440156204\\
73.625	0.12228	73.1735483162014\\
73.625	0.12594	77.4641162933688\\
73.625	0.1296	82.0613479471226\\
73.625	0.13326	86.9652432774626\\
73.625	0.13692	92.1758022843891\\
73.625	0.14058	97.6930249679019\\
73.625	0.14424	103.516911328001\\
73.625	0.1479	109.647461364687\\
73.625	0.15156	116.084675077958\\
73.625	0.15522	122.828552467817\\
73.625	0.15888	129.879093534261\\
73.625	0.16254	137.236298277292\\
73.625	0.1662	144.900166696909\\
73.625	0.16986	152.870698793113\\
73.625	0.17352	161.147894565903\\
73.625	0.17718	169.731754015279\\
73.625	0.18084	178.622277141242\\
73.625	0.1845	187.819463943791\\
73.625	0.18816	197.323314422926\\
73.625	0.19182	207.133828578648\\
73.625	0.19548	217.251006410956\\
73.625	0.19914	227.674847919851\\
73.625	0.2028	238.405353105331\\
73.625	0.20646	249.442521967399\\
73.625	0.21012	260.786354506052\\
73.625	0.21378	272.436850721292\\
73.625	0.21744	284.394010613118\\
73.625	0.2211	296.65783418153\\
73.625	0.22476	309.22832142653\\
73.625	0.22842	322.105472348115\\
73.625	0.23208	335.289286946286\\
73.625	0.23574	348.779765221044\\
73.625	0.2394	362.576907172389\\
73.625	0.24306	376.680712800319\\
73.625	0.24672	391.091182104836\\
73.625	0.25038	405.80831508594\\
73.625	0.25404	420.832111743629\\
73.625	0.2577	436.162572077905\\
73.625	0.26136	451.799696088768\\
73.625	0.26502	467.743483776217\\
73.625	0.26868	483.993935140252\\
73.625	0.27234	500.551050180873\\
73.625	0.276	517.414828898081\\
74	0.093	50.6449953743653\\
74	0.09666	52.4769078962939\\
74	0.10032	54.6154840948089\\
74	0.10398	57.0607239699102\\
74	0.10764	59.8126275215979\\
74	0.1113	62.871194749872\\
74	0.11496	66.2364256547323\\
74	0.11862	69.908320236179\\
74	0.12228	73.8868784942122\\
74	0.12594	78.1721004288316\\
74	0.1296	82.7639860400373\\
74	0.13326	87.6625353278295\\
74	0.13692	92.867748292208\\
74	0.14058	98.3796249331729\\
74	0.14424	104.198165250724\\
74	0.1479	110.323369244862\\
74	0.15156	116.755236915585\\
74	0.15522	123.493768262896\\
74	0.15888	130.538963286792\\
74	0.16254	137.890821987275\\
74	0.1662	145.549344364345\\
74	0.16986	153.514530418\\
74	0.17352	161.786380148242\\
74	0.17718	170.364893555071\\
74	0.18084	179.250070638485\\
74	0.1845	188.441911398486\\
74	0.18816	197.940415835074\\
74	0.19182	207.745583948248\\
74	0.19548	217.857415738008\\
74	0.19914	228.275911204354\\
74	0.2028	239.001070347287\\
74	0.20646	250.032893166806\\
74	0.21012	261.371379662912\\
74	0.21378	273.016529835604\\
74	0.21744	284.968343684882\\
74	0.2211	297.226821210746\\
74	0.22476	309.791962413197\\
74	0.22842	322.663767292235\\
74	0.23208	335.842235847858\\
74	0.23574	349.327368080068\\
74	0.2394	363.119163988865\\
74	0.24306	377.217623574247\\
74	0.24672	391.622746836216\\
74	0.25038	406.334533774772\\
74	0.25404	421.352984389914\\
74	0.2577	436.678098681642\\
74	0.26136	452.309876649956\\
74	0.26502	468.248318294857\\
74	0.26868	484.493423616344\\
74	0.27234	501.045192614418\\
74	0.276	517.903625289077\\
};
\end{axis}

\begin{axis}[%
width=4.527496cm,
height=3.870968cm,
at={(0cm,10.752688cm)},
scale only axis,
xmin=56,
xmax=74,
tick align=outside,
xlabel={$L_{cut}$},
xmajorgrids,
ymin=0.093,
ymax=0.276,
ylabel={$D_{rlx}$},
ymajorgrids,
zmin=-0.468526216891017,
zmax=12.0724461325189,
zlabel={$x_1,x_4$},
zmajorgrids,
view={-140}{50},
legend style={at={(1.03,1)},anchor=north west,legend cell align=left,align=left,draw=white!15!black}
]
\addplot3[only marks,mark=*,mark options={},mark size=1.5000pt,color=mycolor1] plot table[row sep=crcr,]{%
74	0.123	0.616147082677301\\
72	0.113	0.756244217181634\\
61	0.095	0.315136730386088\\
56	0.093	0.479225561787365\\
};
\addplot3[only marks,mark=*,mark options={},mark size=1.5000pt,color=mycolor2] plot table[row sep=crcr,]{%
67	0.276	8.54895561081077\\
66	0.255	7.8265872893112\\
62	0.209	2.65659919911432\\
57	0.193	1.15263735041756\\
};
\addplot3[only marks,mark=*,mark options={},mark size=1.5000pt,color=black] plot table[row sep=crcr,]{%
69	0.104	0.308894496859802\\
};
\addplot3[only marks,mark=*,mark options={},mark size=1.5000pt,color=black] plot table[row sep=crcr,]{%
64	0.23	4.75084671442933\\
};

\addplot3[%
surf,
opacity=0.7,
shader=interp,
colormap={mymap}{[1pt] rgb(0pt)=(0.0901961,0.239216,0.0745098); rgb(1pt)=(0.0945149,0.242058,0.0739522); rgb(2pt)=(0.0988592,0.244894,0.0733566); rgb(3pt)=(0.103229,0.247724,0.0727241); rgb(4pt)=(0.107623,0.250549,0.0720557); rgb(5pt)=(0.112043,0.253367,0.0713525); rgb(6pt)=(0.116487,0.25618,0.0706154); rgb(7pt)=(0.120956,0.258986,0.0698456); rgb(8pt)=(0.125449,0.261787,0.0690441); rgb(9pt)=(0.129967,0.264581,0.0682118); rgb(10pt)=(0.134508,0.26737,0.06735); rgb(11pt)=(0.139074,0.270152,0.0664596); rgb(12pt)=(0.143663,0.272929,0.0655416); rgb(13pt)=(0.148275,0.275699,0.0645971); rgb(14pt)=(0.152911,0.278463,0.0636271); rgb(15pt)=(0.15757,0.281221,0.0626328); rgb(16pt)=(0.162252,0.283973,0.0616151); rgb(17pt)=(0.166957,0.286719,0.060575); rgb(18pt)=(0.171685,0.289458,0.0595136); rgb(19pt)=(0.176434,0.292191,0.0584321); rgb(20pt)=(0.181207,0.294918,0.0573313); rgb(21pt)=(0.186001,0.297639,0.0562123); rgb(22pt)=(0.190817,0.300353,0.0550763); rgb(23pt)=(0.195655,0.303061,0.0539242); rgb(24pt)=(0.200514,0.305763,0.052757); rgb(25pt)=(0.205395,0.308459,0.0515759); rgb(26pt)=(0.210296,0.311149,0.0503624); rgb(27pt)=(0.215212,0.313846,0.0490067); rgb(28pt)=(0.220142,0.316548,0.0475043); rgb(29pt)=(0.22509,0.319254,0.0458704); rgb(30pt)=(0.230056,0.321962,0.0441205); rgb(31pt)=(0.235042,0.324671,0.04227); rgb(32pt)=(0.240048,0.327379,0.0403343); rgb(33pt)=(0.245078,0.330085,0.0383287); rgb(34pt)=(0.250131,0.332786,0.0362688); rgb(35pt)=(0.25521,0.335482,0.0341698); rgb(36pt)=(0.260317,0.33817,0.0320472); rgb(37pt)=(0.265451,0.340849,0.0299163); rgb(38pt)=(0.270616,0.343517,0.0277927); rgb(39pt)=(0.275813,0.346172,0.0256916); rgb(40pt)=(0.281043,0.348814,0.0236284); rgb(41pt)=(0.286307,0.35144,0.0216186); rgb(42pt)=(0.291607,0.354048,0.0196776); rgb(43pt)=(0.296945,0.356637,0.0178207); rgb(44pt)=(0.302322,0.359206,0.0160634); rgb(45pt)=(0.307739,0.361753,0.0144211); rgb(46pt)=(0.313198,0.364275,0.0129091); rgb(47pt)=(0.318701,0.366772,0.0115428); rgb(48pt)=(0.324249,0.369242,0.0103377); rgb(49pt)=(0.329843,0.371682,0.00930909); rgb(50pt)=(0.335485,0.374093,0.00847245); rgb(51pt)=(0.341176,0.376471,0.00784314); rgb(52pt)=(0.346925,0.378826,0.00732741); rgb(53pt)=(0.352735,0.381168,0.00682184); rgb(54pt)=(0.358605,0.383497,0.00632729); rgb(55pt)=(0.364532,0.385812,0.00584464); rgb(56pt)=(0.370516,0.388113,0.00537476); rgb(57pt)=(0.376552,0.390399,0.00491852); rgb(58pt)=(0.38264,0.39267,0.00447681); rgb(59pt)=(0.388777,0.394925,0.00405048); rgb(60pt)=(0.394962,0.397164,0.00364042); rgb(61pt)=(0.401191,0.399386,0.00324749); rgb(62pt)=(0.407464,0.401592,0.00287258); rgb(63pt)=(0.413777,0.40378,0.00251655); rgb(64pt)=(0.420129,0.40595,0.00218028); rgb(65pt)=(0.426518,0.408102,0.00186463); rgb(66pt)=(0.432942,0.410234,0.00157049); rgb(67pt)=(0.439399,0.412348,0.00129873); rgb(68pt)=(0.445885,0.414441,0.00105022); rgb(69pt)=(0.452401,0.416515,0.000825833); rgb(70pt)=(0.458942,0.418567,0.000626441); rgb(71pt)=(0.465508,0.420599,0.00045292); rgb(72pt)=(0.472096,0.422609,0.000306141); rgb(73pt)=(0.478704,0.424596,0.000186979); rgb(74pt)=(0.485331,0.426562,9.63073e-05); rgb(75pt)=(0.491973,0.428504,3.49981e-05); rgb(76pt)=(0.498628,0.430422,3.92506e-06); rgb(77pt)=(0.505323,0.432315,0); rgb(78pt)=(0.512206,0.434168,0); rgb(79pt)=(0.519282,0.435983,0); rgb(80pt)=(0.526529,0.437764,0); rgb(81pt)=(0.533922,0.439512,0); rgb(82pt)=(0.54144,0.441232,0); rgb(83pt)=(0.549059,0.442927,0); rgb(84pt)=(0.556756,0.444599,0); rgb(85pt)=(0.564508,0.446252,0); rgb(86pt)=(0.572292,0.447889,0); rgb(87pt)=(0.580084,0.449514,0); rgb(88pt)=(0.587863,0.451129,0); rgb(89pt)=(0.595604,0.452737,0); rgb(90pt)=(0.603284,0.454343,0); rgb(91pt)=(0.610882,0.455948,0); rgb(92pt)=(0.618373,0.457556,0); rgb(93pt)=(0.625734,0.459171,0); rgb(94pt)=(0.632943,0.460795,0); rgb(95pt)=(0.639976,0.462432,0); rgb(96pt)=(0.64681,0.464084,0); rgb(97pt)=(0.653423,0.465756,0); rgb(98pt)=(0.659791,0.46745,0); rgb(99pt)=(0.665891,0.469169,0); rgb(100pt)=(0.6717,0.470916,0); rgb(101pt)=(0.677195,0.472696,0); rgb(102pt)=(0.682353,0.47451,0); rgb(103pt)=(0.687242,0.476355,0); rgb(104pt)=(0.691952,0.478225,0); rgb(105pt)=(0.696497,0.480118,0); rgb(106pt)=(0.700887,0.482033,0); rgb(107pt)=(0.705134,0.483968,0); rgb(108pt)=(0.709251,0.485921,0); rgb(109pt)=(0.713249,0.487891,0); rgb(110pt)=(0.71714,0.489876,0); rgb(111pt)=(0.720936,0.491875,0); rgb(112pt)=(0.724649,0.493887,0); rgb(113pt)=(0.72829,0.495909,0); rgb(114pt)=(0.731872,0.49794,0); rgb(115pt)=(0.735406,0.499979,0); rgb(116pt)=(0.738904,0.502025,0); rgb(117pt)=(0.742378,0.504075,0); rgb(118pt)=(0.74584,0.506128,0); rgb(119pt)=(0.749302,0.508182,0); rgb(120pt)=(0.752775,0.510237,0); rgb(121pt)=(0.756272,0.51229,0); rgb(122pt)=(0.759804,0.514339,0); rgb(123pt)=(0.763384,0.516385,0); rgb(124pt)=(0.767022,0.518424,0); rgb(125pt)=(0.770731,0.520455,0); rgb(126pt)=(0.774523,0.522478,0); rgb(127pt)=(0.77841,0.524489,0); rgb(128pt)=(0.782391,0.526491,0); rgb(129pt)=(0.786402,0.528496,0); rgb(130pt)=(0.790431,0.530506,0); rgb(131pt)=(0.794478,0.532521,0); rgb(132pt)=(0.798541,0.534539,0); rgb(133pt)=(0.802619,0.53656,0); rgb(134pt)=(0.806712,0.538584,0); rgb(135pt)=(0.81082,0.540609,0); rgb(136pt)=(0.81494,0.542635,0); rgb(137pt)=(0.819074,0.54466,0); rgb(138pt)=(0.823219,0.546686,0); rgb(139pt)=(0.827374,0.548709,0); rgb(140pt)=(0.831541,0.55073,0); rgb(141pt)=(0.835716,0.552749,0); rgb(142pt)=(0.8399,0.554763,0); rgb(143pt)=(0.844092,0.556774,0); rgb(144pt)=(0.848292,0.558779,0); rgb(145pt)=(0.852497,0.560778,0); rgb(146pt)=(0.856708,0.562771,0); rgb(147pt)=(0.860924,0.564756,0); rgb(148pt)=(0.865143,0.566733,0); rgb(149pt)=(0.869366,0.568701,0); rgb(150pt)=(0.873592,0.57066,0); rgb(151pt)=(0.877819,0.572608,0); rgb(152pt)=(0.882047,0.574545,0); rgb(153pt)=(0.886275,0.576471,0); rgb(154pt)=(0.890659,0.578362,0); rgb(155pt)=(0.895333,0.580203,0); rgb(156pt)=(0.900258,0.581999,0); rgb(157pt)=(0.905397,0.583755,0); rgb(158pt)=(0.910711,0.585479,0); rgb(159pt)=(0.916164,0.587176,0); rgb(160pt)=(0.921717,0.588852,0); rgb(161pt)=(0.927333,0.590513,0); rgb(162pt)=(0.932974,0.592166,0); rgb(163pt)=(0.938602,0.593815,0); rgb(164pt)=(0.94418,0.595468,0); rgb(165pt)=(0.949669,0.59713,0); rgb(166pt)=(0.955033,0.598808,0); rgb(167pt)=(0.960233,0.600507,0); rgb(168pt)=(0.965232,0.602233,0); rgb(169pt)=(0.969992,0.603992,0); rgb(170pt)=(0.974475,0.605791,0); rgb(171pt)=(0.978643,0.607636,0); rgb(172pt)=(0.98246,0.609532,0); rgb(173pt)=(0.985886,0.611486,0); rgb(174pt)=(0.988885,0.613503,0); rgb(175pt)=(0.991419,0.61559,0); rgb(176pt)=(0.99345,0.617753,0); rgb(177pt)=(0.99494,0.619997,0); rgb(178pt)=(0.995851,0.622329,0); rgb(179pt)=(0.996226,0.624763,0); rgb(180pt)=(0.996512,0.627352,0); rgb(181pt)=(0.996788,0.630095,0); rgb(182pt)=(0.997053,0.632982,0); rgb(183pt)=(0.997308,0.636004,0); rgb(184pt)=(0.997552,0.639152,0); rgb(185pt)=(0.997785,0.642416,0); rgb(186pt)=(0.998006,0.645786,0); rgb(187pt)=(0.998217,0.649253,0); rgb(188pt)=(0.998416,0.652807,0); rgb(189pt)=(0.998605,0.656439,0); rgb(190pt)=(0.998781,0.660138,0); rgb(191pt)=(0.998946,0.663897,0); rgb(192pt)=(0.9991,0.667704,0); rgb(193pt)=(0.999242,0.67155,0); rgb(194pt)=(0.999372,0.675427,0); rgb(195pt)=(0.99949,0.679323,0); rgb(196pt)=(0.999596,0.68323,0); rgb(197pt)=(0.99969,0.687139,0); rgb(198pt)=(0.999771,0.691039,0); rgb(199pt)=(0.999841,0.694921,0); rgb(200pt)=(0.999898,0.698775,0); rgb(201pt)=(0.999942,0.702592,0); rgb(202pt)=(0.999974,0.706363,0); rgb(203pt)=(0.999994,0.710077,0); rgb(204pt)=(1,0.713725,0); rgb(205pt)=(1,0.717341,0); rgb(206pt)=(1,0.720963,0); rgb(207pt)=(1,0.724591,0); rgb(208pt)=(1,0.728226,0); rgb(209pt)=(1,0.731867,0); rgb(210pt)=(1,0.735514,0); rgb(211pt)=(1,0.739167,0); rgb(212pt)=(1,0.742827,0); rgb(213pt)=(1,0.746493,0); rgb(214pt)=(1,0.750165,0); rgb(215pt)=(1,0.753843,0); rgb(216pt)=(1,0.757527,0); rgb(217pt)=(1,0.761217,0); rgb(218pt)=(1,0.764913,0); rgb(219pt)=(1,0.768615,0); rgb(220pt)=(1,0.772324,0); rgb(221pt)=(1,0.776038,0); rgb(222pt)=(1,0.779758,0); rgb(223pt)=(1,0.783484,0); rgb(224pt)=(1,0.787215,0); rgb(225pt)=(1,0.790953,0); rgb(226pt)=(1,0.794696,0); rgb(227pt)=(1,0.798445,0); rgb(228pt)=(1,0.8022,0); rgb(229pt)=(1,0.805961,0); rgb(230pt)=(1,0.809727,0); rgb(231pt)=(1,0.8135,0); rgb(232pt)=(1,0.817278,0); rgb(233pt)=(1,0.821063,0); rgb(234pt)=(1,0.824854,0); rgb(235pt)=(1,0.828652,0); rgb(236pt)=(1,0.832455,0); rgb(237pt)=(1,0.836265,0); rgb(238pt)=(1,0.840081,0); rgb(239pt)=(1,0.843903,0); rgb(240pt)=(1,0.847732,0); rgb(241pt)=(1,0.851566,0); rgb(242pt)=(1,0.855406,0); rgb(243pt)=(1,0.859253,0); rgb(244pt)=(1,0.863106,0); rgb(245pt)=(1,0.866964,0); rgb(246pt)=(1,0.870829,0); rgb(247pt)=(1,0.8747,0); rgb(248pt)=(1,0.878577,0); rgb(249pt)=(1,0.88246,0); rgb(250pt)=(1,0.886349,0); rgb(251pt)=(1,0.890243,0); rgb(252pt)=(1,0.894144,0); rgb(253pt)=(1,0.898051,0); rgb(254pt)=(1,0.901964,0); rgb(255pt)=(1,0.905882,0)},
mesh/rows=49]
table[row sep=crcr,header=false] {%
%
56	0.093	0.414666213324685\\
56	0.09666	0.37013123446904\\
56	0.10032	0.32991630850448\\
56	0.10398	0.294021435431006\\
56	0.10764	0.262446615248612\\
56	0.1113	0.235191847957301\\
56	0.11496	0.212257133557077\\
56	0.11862	0.193642472047932\\
56	0.12228	0.179347863429871\\
56	0.12594	0.169373307702895\\
56	0.1296	0.163718804867\\
56	0.13326	0.162384354922187\\
56	0.13692	0.165369957868462\\
56	0.14058	0.172675613705816\\
56	0.14424	0.184301322434254\\
56	0.1479	0.200247084053774\\
56	0.15156	0.22051289856438\\
56	0.15522	0.245098765966067\\
56	0.15888	0.274004686258834\\
56	0.16254	0.307230659442694\\
56	0.1662	0.34477668551763\\
56	0.16986	0.386642764483645\\
56	0.17352	0.432828896340753\\
56	0.17718	0.483335081088939\\
56	0.18084	0.538161318728203\\
56	0.1845	0.597307609258562\\
56	0.18816	0.660773952679993\\
56	0.19182	0.728560348992515\\
56	0.19548	0.800666798196115\\
56	0.19914	0.877093300290799\\
56	0.2028	0.957839855276571\\
56	0.20646	1.04290646315342\\
56	0.21012	1.13229312392135\\
56	0.21378	1.22599983758037\\
56	0.21744	1.32402660413047\\
56	0.2211	1.42637342357165\\
56	0.22476	1.53304029590393\\
56	0.22842	1.64402722112727\\
56	0.23208	1.7593341992417\\
56	0.23574	1.87896123024722\\
56	0.2394	2.00290831414382\\
56	0.24306	2.1311754509315\\
56	0.24672	2.26376264061027\\
56	0.25038	2.40066988318012\\
56	0.25404	2.54189717864105\\
56	0.2577	2.68744452699306\\
56	0.26136	2.83731192823616\\
56	0.26502	2.99149938237034\\
56	0.26868	3.1500068893956\\
56	0.27234	3.31283444931195\\
56	0.276	3.47998206211939\\
56.375	0.093	0.420828723367943\\
56.375	0.09666	0.380241934720889\\
56.375	0.10032	0.343975198964918\\
56.375	0.10398	0.312028516100034\\
56.375	0.10764	0.284401886126226\\
56.375	0.1113	0.261095309043508\\
56.375	0.11496	0.24210878485187\\
56.375	0.11862	0.227442313551318\\
56.375	0.12228	0.217095895141846\\
56.375	0.12594	0.211069529623457\\
56.375	0.1296	0.209363216996155\\
56.375	0.13326	0.211976957259932\\
56.375	0.13692	0.218910750414792\\
56.375	0.14058	0.230164596460737\\
56.375	0.14424	0.245738495397767\\
56.375	0.1479	0.265632447225877\\
56.375	0.15156	0.289846451945071\\
56.375	0.15522	0.318380509555348\\
56.375	0.15888	0.351234620056704\\
56.375	0.16254	0.388408783449154\\
56.375	0.1662	0.429902999732676\\
56.375	0.16986	0.475717268907283\\
56.375	0.17352	0.525851590972982\\
56.375	0.17718	0.580305965929753\\
56.375	0.18084	0.63908039377761\\
56.375	0.1845	0.702174874516559\\
56.375	0.18816	0.769589408146577\\
56.375	0.19182	0.841323994667691\\
56.375	0.19548	0.917378634079881\\
56.375	0.19914	0.997753326383155\\
56.375	0.2028	1.08244807157752\\
56.375	0.20646	1.17146286966295\\
56.375	0.21012	1.26479772063948\\
56.375	0.21378	1.36245262450709\\
56.375	0.21744	1.46442758126577\\
56.375	0.2211	1.57072259091555\\
56.375	0.22476	1.68133765345641\\
56.375	0.22842	1.79627276888834\\
56.375	0.23208	1.91552793721137\\
56.375	0.23574	2.03910315842547\\
56.375	0.2394	2.16699843253066\\
56.375	0.24306	2.29921375952694\\
56.375	0.24672	2.43574913941429\\
56.375	0.25038	2.57660457219273\\
56.375	0.25404	2.72178005786225\\
56.375	0.2577	2.87127559642285\\
56.375	0.26136	3.02509118787454\\
56.375	0.26502	3.18322683221731\\
56.375	0.26868	3.34568252945116\\
56.375	0.27234	3.5124582795761\\
56.375	0.276	3.68355408259212\\
56.75	0.093	0.425946026928677\\
56.75	0.09666	0.389307428490212\\
56.75	0.10032	0.356988882942833\\
56.75	0.10398	0.328990390286535\\
56.75	0.10764	0.30531195052132\\
56.75	0.1113	0.285953563647189\\
56.75	0.11496	0.270915229664144\\
56.75	0.11862	0.260196948572181\\
56.75	0.12228	0.253798720371296\\
56.75	0.12594	0.2517205450615\\
56.75	0.1296	0.253962422642787\\
56.75	0.13326	0.26052435311515\\
56.75	0.13692	0.271406336478603\\
56.75	0.14058	0.286608372733137\\
56.75	0.14424	0.306130461878757\\
56.75	0.1479	0.329972603915457\\
56.75	0.15156	0.358134798843239\\
56.75	0.15522	0.390617046662109\\
56.75	0.15888	0.427419347372055\\
56.75	0.16254	0.468541700973094\\
56.75	0.1662	0.513984107465205\\
56.75	0.16986	0.563746566848403\\
56.75	0.17352	0.617829079122691\\
56.75	0.17718	0.676231644288052\\
56.75	0.18084	0.738954262344499\\
56.75	0.1845	0.805996933292037\\
56.75	0.18816	0.877359657130648\\
56.75	0.19182	0.953042433860348\\
56.75	0.19548	1.03304526348113\\
56.75	0.19914	1.11736814599299\\
56.75	0.2028	1.20601108139594\\
56.75	0.20646	1.29897406968997\\
56.75	0.21012	1.39625711087509\\
56.75	0.21378	1.49786020495128\\
56.75	0.21744	1.60378335191856\\
56.75	0.2211	1.71402655177693\\
56.75	0.22476	1.82858980452637\\
56.75	0.22842	1.9474731101669\\
56.75	0.23208	2.07067646869852\\
56.75	0.23574	2.1981998801212\\
56.75	0.2394	2.33004334443498\\
56.75	0.24306	2.46620686163985\\
56.75	0.24672	2.60669043173579\\
56.75	0.25038	2.75149405472283\\
56.75	0.25404	2.90061773060094\\
56.75	0.2577	3.05406145937013\\
56.75	0.26136	3.2118252410304\\
56.75	0.26502	3.37390907558176\\
56.75	0.26868	3.5403129630242\\
56.75	0.27234	3.71103690335774\\
56.75	0.276	3.88608089658235\\
57.125	0.093	0.430018124006879\\
57.125	0.09666	0.397327715777004\\
57.125	0.10032	0.368957360438213\\
57.125	0.10398	0.344907057990508\\
57.125	0.10764	0.325176808433883\\
57.125	0.1113	0.309766611768341\\
57.125	0.11496	0.298676467993885\\
57.125	0.11862	0.291906377110509\\
57.125	0.12228	0.289456339118217\\
57.125	0.12594	0.29132635401701\\
57.125	0.1296	0.297516421806883\\
57.125	0.13326	0.308026542487839\\
57.125	0.13692	0.322856716059883\\
57.125	0.14058	0.342006942523006\\
57.125	0.14424	0.365477221877216\\
57.125	0.1479	0.393267554122501\\
57.125	0.15156	0.425377939258875\\
57.125	0.15522	0.461808377286338\\
57.125	0.15888	0.502558868204874\\
57.125	0.16254	0.547629412014499\\
57.125	0.1662	0.597020008715203\\
57.125	0.16986	0.650730658306991\\
57.125	0.17352	0.708761360789865\\
57.125	0.17718	0.771112116163819\\
57.125	0.18084	0.837782924428855\\
57.125	0.1845	0.908773785584979\\
57.125	0.18816	0.98408469963218\\
57.125	0.19182	1.06371566657047\\
57.125	0.19548	1.14766668639984\\
57.125	0.19914	1.23593775912029\\
57.125	0.2028	1.32852888473183\\
57.125	0.20646	1.42544006323446\\
57.125	0.21012	1.52667129462816\\
57.125	0.21378	1.63222257891294\\
57.125	0.21744	1.74209391608881\\
57.125	0.2211	1.85628530615577\\
57.125	0.22476	1.9747967491138\\
57.125	0.22842	2.09762824496292\\
57.125	0.23208	2.22477979370312\\
57.125	0.23574	2.35625139533441\\
57.125	0.2394	2.49204304985677\\
57.125	0.24306	2.63215475727023\\
57.125	0.24672	2.77658651757476\\
57.125	0.25038	2.92533833077038\\
57.125	0.25404	3.07841019685708\\
57.125	0.2577	3.23580211583486\\
57.125	0.26136	3.39751408770373\\
57.125	0.26502	3.56354611246368\\
57.125	0.26868	3.73389819011471\\
57.125	0.27234	3.90857032065683\\
57.125	0.276	4.08756250409002\\
57.5	0.093	0.433045014602555\\
57.5	0.09666	0.40430279658127\\
57.5	0.10032	0.37988063145107\\
57.5	0.10398	0.35977851921195\\
57.5	0.10764	0.343996459863915\\
57.5	0.1113	0.332534453406966\\
57.5	0.11496	0.325392499841097\\
57.5	0.11862	0.322570599166314\\
57.5	0.12228	0.324068751382611\\
57.5	0.12594	0.32988695648999\\
57.5	0.1296	0.340025214488457\\
57.5	0.13326	0.354483525378003\\
57.5	0.13692	0.373261889158635\\
57.5	0.14058	0.396360305830349\\
57.5	0.14424	0.423778775393144\\
57.5	0.1479	0.455517297847023\\
57.5	0.15156	0.491575873191988\\
57.5	0.15522	0.531954501428037\\
57.5	0.15888	0.576653182555162\\
57.5	0.16254	0.625671916573377\\
57.5	0.1662	0.679010703482671\\
57.5	0.16986	0.736669543283048\\
57.5	0.17352	0.798648435974512\\
57.5	0.17718	0.864947381557055\\
57.5	0.18084	0.935566380030681\\
57.5	0.1845	1.0105054313954\\
57.5	0.18816	1.08976453565119\\
57.5	0.19182	1.17334369279807\\
57.5	0.19548	1.26124290283603\\
57.5	0.19914	1.35346216576507\\
57.5	0.2028	1.4500014815852\\
57.5	0.20646	1.55086085029641\\
57.5	0.21012	1.65604027189871\\
57.5	0.21378	1.76553974639208\\
57.5	0.21744	1.87935927377653\\
57.5	0.2211	1.99749885405208\\
57.5	0.22476	2.1199584872187\\
57.5	0.22842	2.24673817327641\\
57.5	0.23208	2.37783791222521\\
57.5	0.23574	2.51325770406508\\
57.5	0.2394	2.65299754879603\\
57.5	0.24306	2.79705744641808\\
57.5	0.24672	2.9454373969312\\
57.5	0.25038	3.09813740033541\\
57.5	0.25404	3.2551574566307\\
57.5	0.2577	3.41649756581707\\
57.5	0.26136	3.58215772789453\\
57.5	0.26502	3.75213794286307\\
57.5	0.26868	3.92643821072269\\
57.5	0.27234	4.1050585314734\\
57.5	0.276	4.28799890511518\\
57.875	0.093	0.435026698715704\\
57.875	0.09666	0.410232670903006\\
57.875	0.10032	0.389758695981394\\
57.875	0.10398	0.373604773950868\\
57.875	0.10764	0.361770904811422\\
57.875	0.1113	0.35425708856306\\
57.875	0.11496	0.351063325205784\\
57.875	0.11862	0.35218961473959\\
57.875	0.12228	0.357635957164473\\
57.875	0.12594	0.367402352480446\\
57.875	0.1296	0.381488800687502\\
57.875	0.13326	0.399895301785637\\
57.875	0.13692	0.422621855774856\\
57.875	0.14058	0.449668462655159\\
57.875	0.14424	0.481035122426548\\
57.875	0.1479	0.516721835089016\\
57.875	0.15156	0.556728600642571\\
57.875	0.15522	0.601055419087206\\
57.875	0.15888	0.649702290422921\\
57.875	0.16254	0.702669214649728\\
57.875	0.1662	0.759956191767612\\
57.875	0.16986	0.821563221776575\\
57.875	0.17352	0.887490304676632\\
57.875	0.17718	0.957737440467765\\
57.875	0.18084	1.03230462914998\\
57.875	0.1845	1.11119187072328\\
57.875	0.18816	1.19439916518766\\
57.875	0.19182	1.28192651254314\\
57.875	0.19548	1.37377391278968\\
57.875	0.19914	1.46994136592732\\
57.875	0.2028	1.57042887195604\\
57.875	0.20646	1.67523643087583\\
57.875	0.21012	1.78436404268672\\
57.875	0.21378	1.89781170738868\\
57.875	0.21744	2.01557942498173\\
57.875	0.2211	2.13766719546587\\
57.875	0.22476	2.26407501884108\\
57.875	0.22842	2.39480289510737\\
57.875	0.23208	2.52985082426477\\
57.875	0.23574	2.66921880631322\\
57.875	0.2394	2.81290684125276\\
57.875	0.24306	2.9609149290834\\
57.875	0.24672	3.11324306980512\\
57.875	0.25038	3.26989126341792\\
57.875	0.25404	3.43085950992179\\
57.875	0.2577	3.59614780931676\\
57.875	0.26136	3.7657561616028\\
57.875	0.26502	3.93968456677992\\
57.875	0.26868	4.11793302484814\\
57.875	0.27234	4.30050153580744\\
57.875	0.276	4.48739009965782\\
58.25	0.093	0.435963176346323\\
58.25	0.09666	0.415117338742217\\
58.25	0.10032	0.398591554029194\\
58.25	0.10398	0.386385822207258\\
58.25	0.10764	0.378500143276398\\
58.25	0.1113	0.374934517236629\\
58.25	0.11496	0.375688944087942\\
58.25	0.11862	0.380763423830334\\
58.25	0.12228	0.390157956463811\\
58.25	0.12594	0.403872541988373\\
58.25	0.1296	0.421907180404018\\
58.25	0.13326	0.44426187171074\\
58.25	0.13692	0.470936615908552\\
58.25	0.14058	0.501931412997445\\
58.25	0.14424	0.537246262977423\\
58.25	0.1479	0.576881165848481\\
58.25	0.15156	0.620836121610622\\
58.25	0.15522	0.669111130263851\\
58.25	0.15888	0.721706191808156\\
58.25	0.16254	0.778621306243553\\
58.25	0.1662	0.839856473570027\\
58.25	0.16986	0.905411693787583\\
58.25	0.17352	0.975286966896226\\
58.25	0.17718	1.04948229289595\\
58.25	0.18084	1.12799767178675\\
58.25	0.1845	1.21083310356865\\
58.25	0.18816	1.29798858824162\\
58.25	0.19182	1.38946412580568\\
58.25	0.19548	1.48525971626082\\
58.25	0.19914	1.58537535960703\\
58.25	0.2028	1.68981105584434\\
58.25	0.20646	1.79856680497274\\
58.25	0.21012	1.91164260699221\\
58.25	0.21378	2.02903846190277\\
58.25	0.21744	2.1507543697044\\
58.25	0.2211	2.27679033039712\\
58.25	0.22476	2.40714634398093\\
58.25	0.22842	2.54182241045581\\
58.25	0.23208	2.68081852982179\\
58.25	0.23574	2.82413470207885\\
58.25	0.2394	2.97177092722698\\
58.25	0.24306	3.1237272052662\\
58.25	0.24672	3.2800035361965\\
58.25	0.25038	3.4405999200179\\
58.25	0.25404	3.60551635673036\\
58.25	0.2577	3.77475284633391\\
58.25	0.26136	3.94830938882855\\
58.25	0.26502	4.12618598421427\\
58.25	0.26868	4.30838263249107\\
58.25	0.27234	4.49489933365896\\
58.25	0.276	4.68573608771792\\
58.625	0.093	0.435854447494409\\
58.625	0.09666	0.418956800098891\\
58.625	0.10032	0.406379205594462\\
58.625	0.10398	0.398121663981112\\
58.625	0.10764	0.394184175258845\\
58.625	0.1113	0.394566739427665\\
58.625	0.11496	0.399269356487565\\
58.625	0.11862	0.40829202643855\\
58.625	0.12228	0.421634749280616\\
58.625	0.12594	0.439297525013768\\
58.625	0.1296	0.461280353638\\
58.625	0.13326	0.487583235153314\\
58.625	0.13692	0.518206169559716\\
58.625	0.14058	0.553149156857198\\
58.625	0.14424	0.592412197045766\\
58.625	0.1479	0.63599529012541\\
58.625	0.15156	0.683898436096144\\
58.625	0.15522	0.736121634957962\\
58.625	0.15888	0.792664886710856\\
58.625	0.16254	0.853528191354843\\
58.625	0.1662	0.918711548889906\\
58.625	0.16986	0.988214959316048\\
58.625	0.17352	1.06203842263328\\
58.625	0.17718	1.1401819388416\\
58.625	0.18084	1.22264550794099\\
58.625	0.1845	1.30942912993147\\
58.625	0.18816	1.40053280481303\\
58.625	0.19182	1.49595653258569\\
58.625	0.19548	1.59570031324941\\
58.625	0.19914	1.69976414680422\\
58.625	0.2028	1.80814803325012\\
58.625	0.20646	1.9208519725871\\
58.625	0.21012	2.03787596481516\\
58.625	0.21378	2.1592200099343\\
58.625	0.21744	2.28488410794453\\
58.625	0.2211	2.41486825884585\\
58.625	0.22476	2.54917246263824\\
58.625	0.22842	2.68779671932172\\
58.625	0.23208	2.83074102889628\\
58.625	0.23574	2.97800539136192\\
58.625	0.2394	3.12958980671865\\
58.625	0.24306	3.28549427496646\\
58.625	0.24672	3.44571879610536\\
58.625	0.25038	3.61026337013534\\
58.625	0.25404	3.77912799705639\\
58.625	0.2577	3.95231267686853\\
58.625	0.26136	4.12981740957176\\
58.625	0.26502	4.31164219516606\\
58.625	0.26868	4.49778703365145\\
58.625	0.27234	4.68825192502793\\
58.625	0.276	4.88303686929549\\
59	0.093	0.434700512159973\\
59	0.09666	0.421751054973047\\
59	0.10032	0.413121650677204\\
59	0.10398	0.408812299272446\\
59	0.10764	0.408823000758769\\
59	0.1113	0.413153755136175\\
59	0.11496	0.421804562404668\\
59	0.11862	0.434775422564243\\
59	0.12228	0.452066335614895\\
59	0.12594	0.473677301556637\\
59	0.1296	0.499608320389461\\
59	0.13326	0.529859392113366\\
59	0.13692	0.564430516728357\\
59	0.14058	0.603321694234429\\
59	0.14424	0.646532924631587\\
59	0.1479	0.69406420791982\\
59	0.15156	0.745915544099144\\
59	0.15522	0.802086933169549\\
59	0.15888	0.862578375131033\\
59	0.16254	0.927389869983609\\
59	0.1662	0.996521417727262\\
59	0.16986	1.06997301836199\\
59	0.17352	1.14774467188782\\
59	0.17718	1.22983637830472\\
59	0.18084	1.31624813761271\\
59	0.1845	1.40697994981178\\
59	0.18816	1.50203181490193\\
59	0.19182	1.60140373288317\\
59	0.19548	1.70509570375549\\
59	0.19914	1.81310772751889\\
59	0.2028	1.92543980417338\\
59	0.20646	2.04209193371894\\
59	0.21012	2.1630641161556\\
59	0.21378	2.28835635148333\\
59	0.21744	2.41796863970215\\
59	0.2211	2.55190098081205\\
59	0.22476	2.69015337481303\\
59	0.22842	2.8327258217051\\
59	0.23208	2.97961832148825\\
59	0.23574	3.13083087416248\\
59	0.2394	3.2863634797278\\
59	0.24306	3.4462161381842\\
59	0.24672	3.61038884953168\\
59	0.25038	3.77888161377025\\
59	0.25404	3.9516944308999\\
59	0.2577	4.12882730092063\\
59	0.26136	4.31028022383245\\
59	0.26502	4.49605319963534\\
59	0.26868	4.68614622832932\\
59	0.27234	4.88055930991439\\
59	0.276	5.07929244439054\\
59.375	0.093	0.432501370343004\\
59.375	0.09666	0.423500103364665\\
59.375	0.10032	0.418818889277415\\
59.375	0.10398	0.418457728081247\\
59.375	0.10764	0.422416619776156\\
59.375	0.1113	0.430695564362156\\
59.375	0.11496	0.443294561839238\\
59.375	0.11862	0.460213612207399\\
59.375	0.12228	0.481452715466644\\
59.375	0.12594	0.507011871616975\\
59.375	0.1296	0.53689108065839\\
59.375	0.13326	0.571090342590884\\
59.375	0.13692	0.609609657414461\\
59.375	0.14058	0.652449025129122\\
59.375	0.14424	0.699608445734869\\
59.375	0.1479	0.751087919231696\\
59.375	0.15156	0.806887445619609\\
59.375	0.15522	0.867007024898606\\
59.375	0.15888	0.93144665706868\\
59.375	0.16254	1.00020634212984\\
59.375	0.1662	1.07328608008208\\
59.375	0.16986	1.15068587092541\\
59.375	0.17352	1.23240571465982\\
59.375	0.17718	1.31844561128532\\
59.375	0.18084	1.40880556080189\\
59.375	0.1845	1.50348556320955\\
59.375	0.18816	1.60248561850829\\
59.375	0.19182	1.70580572669812\\
59.375	0.19548	1.81344588777903\\
59.375	0.19914	1.92540610175102\\
59.375	0.2028	2.0416863686141\\
59.375	0.20646	2.16228668836825\\
59.375	0.21012	2.2872070610135\\
59.375	0.21378	2.41644748654982\\
59.375	0.21744	2.55000796497722\\
59.375	0.2211	2.68788849629572\\
59.375	0.22476	2.83008908050529\\
59.375	0.22842	2.97660971760595\\
59.375	0.23208	3.12745040759769\\
59.375	0.23574	3.28261115048051\\
59.375	0.2394	3.44209194625442\\
59.375	0.24306	3.60589279491941\\
59.375	0.24672	3.77401369647547\\
59.375	0.25038	3.94645465092263\\
59.375	0.25404	4.12321565826087\\
59.375	0.2577	4.30429671849019\\
59.375	0.26136	4.48969783161061\\
59.375	0.26502	4.67941899762209\\
59.375	0.26868	4.87346021652466\\
59.375	0.27234	5.07182148831831\\
59.375	0.276	5.27450281300305\\
59.75	0.093	0.429257022043504\\
59.75	0.09666	0.424203945273756\\
59.75	0.10032	0.423470921395092\\
59.75	0.10398	0.427057950407514\\
59.75	0.10764	0.434965032311016\\
59.75	0.1113	0.447192167105605\\
59.75	0.11496	0.463739354791274\\
59.75	0.11862	0.484606595368028\\
59.75	0.12228	0.509793888835863\\
59.75	0.12594	0.539301235194784\\
59.75	0.1296	0.573128634444784\\
59.75	0.13326	0.611276086585868\\
59.75	0.13692	0.653743591618038\\
59.75	0.14058	0.700531149541289\\
59.75	0.14424	0.751638760355626\\
59.75	0.1479	0.807066424061042\\
59.75	0.15156	0.866814140657545\\
59.75	0.15522	0.930881910145129\\
59.75	0.15888	0.999269732523793\\
59.75	0.16254	1.07197760779355\\
59.75	0.1662	1.14900553595438\\
59.75	0.16986	1.23035351700629\\
59.75	0.17352	1.3160215509493\\
59.75	0.17718	1.40600963778338\\
59.75	0.18084	1.50031777750854\\
59.75	0.1845	1.5989459701248\\
59.75	0.18816	1.70189421563213\\
59.75	0.19182	1.80916251403054\\
59.75	0.19548	1.92075086532004\\
59.75	0.19914	2.03665926950062\\
59.75	0.2028	2.15688772657229\\
59.75	0.20646	2.28143623653503\\
59.75	0.21012	2.41030479938887\\
59.75	0.21378	2.54349341513378\\
59.75	0.21744	2.68100208376977\\
59.75	0.2211	2.82283080529686\\
59.75	0.22476	2.96897957971502\\
59.75	0.22842	3.11944840702427\\
59.75	0.23208	3.2742372872246\\
59.75	0.23574	3.43334622031601\\
59.75	0.2394	3.5967752062985\\
59.75	0.24306	3.76452424517209\\
59.75	0.24672	3.93659333693674\\
59.75	0.25038	4.11298248159249\\
59.75	0.25404	4.29369167913932\\
59.75	0.2577	4.47872092957723\\
59.75	0.26136	4.66807023290622\\
59.75	0.26502	4.8617395891263\\
59.75	0.26868	5.05972899823746\\
59.75	0.27234	5.26203846023971\\
59.75	0.276	5.46866797513303\\
60.125	0.093	0.424967467261482\\
60.125	0.09666	0.423862580700321\\
60.125	0.10032	0.42707774703025\\
60.125	0.10398	0.434612966251262\\
60.125	0.10764	0.446468238363354\\
60.125	0.1113	0.462643563366529\\
60.125	0.11496	0.48313894126079\\
60.125	0.11862	0.507954372046134\\
60.125	0.12228	0.537089855722559\\
60.125	0.12594	0.570545392290065\\
60.125	0.1296	0.608320981748659\\
60.125	0.13326	0.650416624098332\\
60.125	0.13692	0.696832319339092\\
60.125	0.14058	0.747568067470933\\
60.125	0.14424	0.802623868493859\\
60.125	0.1479	0.861999722407865\\
60.125	0.15156	0.925695629212954\\
60.125	0.15522	0.99371158890913\\
60.125	0.15888	1.06604760149638\\
60.125	0.16254	1.14270366697473\\
60.125	0.1662	1.22367978534415\\
60.125	0.16986	1.30897595660465\\
60.125	0.17352	1.39859218075625\\
60.125	0.17718	1.49252845779892\\
60.125	0.18084	1.59078478773267\\
60.125	0.1845	1.69336117055751\\
60.125	0.18816	1.80025760627343\\
60.125	0.19182	1.91147409488044\\
60.125	0.19548	2.02701063637853\\
60.125	0.19914	2.14686723076769\\
60.125	0.2028	2.27104387804795\\
60.125	0.20646	2.39954057821929\\
60.125	0.21012	2.53235733128171\\
60.125	0.21378	2.66949413723522\\
60.125	0.21744	2.8109509960798\\
60.125	0.2211	2.95672790781547\\
60.125	0.22476	3.10682487244223\\
60.125	0.22842	3.26124188996006\\
60.125	0.23208	3.41997896036898\\
60.125	0.23574	3.58303608366898\\
60.125	0.2394	3.75041325986006\\
60.125	0.24306	3.92211048894224\\
60.125	0.24672	4.09812777091549\\
60.125	0.25038	4.27846510577983\\
60.125	0.25404	4.46312249353524\\
60.125	0.2577	4.65209993418174\\
60.125	0.26136	4.84539742771933\\
60.125	0.26502	5.043014974148\\
60.125	0.26868	5.24495257346774\\
60.125	0.27234	5.45121022567858\\
60.125	0.276	5.66178793078049\\
60.5	0.093	0.419632705996928\\
60.5	0.09666	0.422476009644359\\
60.5	0.10032	0.429639366182878\\
60.5	0.10398	0.441122775612476\\
60.5	0.10764	0.456926237933157\\
60.5	0.1113	0.477049753144925\\
60.5	0.11496	0.501493321247776\\
60.5	0.11862	0.530256942241706\\
60.5	0.12228	0.56334061612672\\
60.5	0.12594	0.600744342902821\\
60.5	0.1296	0.642468122570004\\
60.5	0.13326	0.688511955128266\\
60.5	0.13692	0.738875840577616\\
60.5	0.14058	0.793559778918046\\
60.5	0.14424	0.852563770149559\\
60.5	0.1479	0.915887814272154\\
60.5	0.15156	0.983531911285836\\
60.5	0.15522	1.0554960611906\\
60.5	0.15888	1.13178026398644\\
60.5	0.16254	1.21238451967338\\
60.5	0.1662	1.29730882825139\\
60.5	0.16986	1.38655318972048\\
60.5	0.17352	1.48011760408066\\
60.5	0.17718	1.57800207133192\\
60.5	0.18084	1.68020659147427\\
60.5	0.1845	1.7867311645077\\
60.5	0.18816	1.89757579043221\\
60.5	0.19182	2.01274046924781\\
60.5	0.19548	2.13222520095449\\
60.5	0.19914	2.25602998555224\\
60.5	0.2028	2.38415482304109\\
60.5	0.20646	2.51659971342102\\
60.5	0.21012	2.65336465669203\\
60.5	0.21378	2.79444965285412\\
60.5	0.21744	2.93985470190729\\
60.5	0.2211	3.08957980385156\\
60.5	0.22476	3.2436249586869\\
60.5	0.22842	3.40199016641332\\
60.5	0.23208	3.56467542703084\\
60.5	0.23574	3.73168074053943\\
60.5	0.2394	3.9030061069391\\
60.5	0.24306	4.07865152622985\\
60.5	0.24672	4.2586169984117\\
60.5	0.25038	4.44290252348462\\
60.5	0.25404	4.63150810144863\\
60.5	0.2577	4.82443373230372\\
60.5	0.26136	5.0216794160499\\
60.5	0.26502	5.22324515268716\\
60.5	0.26868	5.42913094221549\\
60.5	0.27234	5.63933678463491\\
60.5	0.276	5.85386267994542\\
60.875	0.093	0.413252738249848\\
60.875	0.09666	0.420044232105869\\
60.875	0.10032	0.431155778852974\\
60.875	0.10398	0.446587378491165\\
60.875	0.10764	0.466339031020436\\
60.875	0.1113	0.49041073644079\\
60.875	0.11496	0.518802494752231\\
60.875	0.11862	0.551514305954754\\
60.875	0.12228	0.588546170048358\\
60.875	0.12594	0.629898087033047\\
60.875	0.1296	0.67557005690882\\
60.875	0.13326	0.725562079675669\\
60.875	0.13692	0.779874155333608\\
60.875	0.14058	0.838506283882631\\
60.875	0.14424	0.901458465322734\\
60.875	0.1479	0.968730699653919\\
60.875	0.15156	1.04032298687619\\
60.875	0.15522	1.11623532698955\\
60.875	0.15888	1.19646771999398\\
60.875	0.16254	1.2810201658895\\
60.875	0.1662	1.3698926646761\\
60.875	0.16986	1.46308521635379\\
60.875	0.17352	1.56059782092256\\
60.875	0.17718	1.66243047838241\\
60.875	0.18084	1.76858318873334\\
60.875	0.1845	1.87905595197537\\
60.875	0.18816	1.99384876810846\\
60.875	0.19182	2.11296163713265\\
60.875	0.19548	2.23639455904791\\
60.875	0.19914	2.36414753385426\\
60.875	0.2028	2.4962205615517\\
60.875	0.20646	2.63261364214021\\
60.875	0.21012	2.77332677561982\\
60.875	0.21378	2.9183599619905\\
60.875	0.21744	3.06771320125227\\
60.875	0.2211	3.22138649340512\\
60.875	0.22476	3.37937983844905\\
60.875	0.22842	3.54169323638406\\
60.875	0.23208	3.70832668721016\\
60.875	0.23574	3.87928019092734\\
60.875	0.2394	4.05455374753561\\
60.875	0.24306	4.23414735703496\\
60.875	0.24672	4.41806101942539\\
60.875	0.25038	4.60629473470691\\
60.875	0.25404	4.7988485028795\\
60.875	0.2577	4.99572232394318\\
60.875	0.26136	5.19691619789795\\
60.875	0.26502	5.40243012474379\\
60.875	0.26868	5.61226410448072\\
60.875	0.27234	5.82641813710873\\
60.875	0.276	6.04489222262783\\
61.25	0.093	0.405827564020234\\
61.25	0.09666	0.416567248084845\\
61.25	0.10032	0.431626985040543\\
61.25	0.10398	0.451006774887324\\
61.25	0.10764	0.474706617625184\\
61.25	0.1113	0.502726513254128\\
61.25	0.11496	0.535066461774159\\
61.25	0.11862	0.571726463185271\\
61.25	0.12228	0.612706517487465\\
61.25	0.12594	0.65800662468074\\
61.25	0.1296	0.707626784765103\\
61.25	0.13326	0.761566997740545\\
61.25	0.13692	0.819827263607074\\
61.25	0.14058	0.882407582364683\\
61.25	0.14424	0.949307954013378\\
61.25	0.1479	1.02052837855315\\
61.25	0.15156	1.09606885598401\\
61.25	0.15522	1.17592938630596\\
61.25	0.15888	1.26010996951898\\
61.25	0.16254	1.34861060562309\\
61.25	0.1662	1.44143129461829\\
61.25	0.16986	1.53857203650455\\
61.25	0.17352	1.64003283128192\\
61.25	0.17718	1.74581367895036\\
61.25	0.18084	1.85591457950988\\
61.25	0.1845	1.97033553296049\\
61.25	0.18816	2.08907653930218\\
61.25	0.19182	2.21213759853496\\
61.25	0.19548	2.33951871065881\\
61.25	0.19914	2.47121987567375\\
61.25	0.2028	2.60724109357978\\
61.25	0.20646	2.74758236437688\\
61.25	0.21012	2.89224368806508\\
61.25	0.21378	3.04122506464434\\
61.25	0.21744	3.1945264941147\\
61.25	0.2211	3.35214797647614\\
61.25	0.22476	3.51408951172866\\
61.25	0.22842	3.68035109987227\\
61.25	0.23208	3.85093274090696\\
61.25	0.23574	4.02583443483272\\
61.25	0.2394	4.20505618164958\\
61.25	0.24306	4.38859798135752\\
61.25	0.24672	4.57645983395654\\
61.25	0.25038	4.76864173944665\\
61.25	0.25404	4.96514369782783\\
61.25	0.2577	5.1659657091001\\
61.25	0.26136	5.37110777326346\\
61.25	0.26502	5.58056989031789\\
61.25	0.26868	5.79435206026341\\
61.25	0.27234	6.01245428310001\\
61.25	0.276	6.2348765588277\\
61.625	0.093	0.397357183308098\\
61.625	0.09666	0.412045057581299\\
61.625	0.10032	0.431052984745583\\
61.625	0.10398	0.454380964800953\\
61.625	0.10764	0.482028997747403\\
61.625	0.1113	0.51399708358494\\
61.625	0.11496	0.55028522231356\\
61.625	0.11862	0.590893413933259\\
61.625	0.12228	0.635821658444041\\
61.625	0.12594	0.68506995584591\\
61.625	0.1296	0.738638306138862\\
61.625	0.13326	0.796526709322894\\
61.625	0.13692	0.858735165398012\\
61.625	0.14058	0.925263674364211\\
61.625	0.14424	0.996112236221496\\
61.625	0.1479	1.07128085096986\\
61.625	0.15156	1.15076951860931\\
61.625	0.15522	1.23457823913984\\
61.625	0.15888	1.32270701256146\\
61.625	0.16254	1.41515583887416\\
61.625	0.1662	1.51192471807794\\
61.625	0.16986	1.6130136501728\\
61.625	0.17352	1.71842263515875\\
61.625	0.17718	1.82815167303578\\
61.625	0.18084	1.9422007638039\\
61.625	0.1845	2.06056990746309\\
61.625	0.18816	2.18325910401337\\
61.625	0.19182	2.31026835345474\\
61.625	0.19548	2.44159765578718\\
61.625	0.19914	2.57724701101071\\
61.625	0.2028	2.71721641912533\\
61.625	0.20646	2.86150588013102\\
61.625	0.21012	3.0101153940278\\
61.625	0.21378	3.16304496081567\\
61.625	0.21744	3.32029458049461\\
61.625	0.2211	3.48186425306464\\
61.625	0.22476	3.64775397852575\\
61.625	0.22842	3.81796375687795\\
61.625	0.23208	3.99249358812123\\
61.625	0.23574	4.17134347225558\\
61.625	0.2394	4.35451340928103\\
61.625	0.24306	4.54200339919756\\
61.625	0.24672	4.73381344200517\\
61.625	0.25038	4.92994353770387\\
61.625	0.25404	5.13039368629364\\
61.625	0.2577	5.3351638877745\\
61.625	0.26136	5.54425414214644\\
61.625	0.26502	5.75766444940946\\
61.625	0.26868	5.97539480956357\\
61.625	0.27234	6.19744522260877\\
61.625	0.276	6.42381568854504\\
62	0.093	0.38784159611343\\
62	0.09666	0.406477660595216\\
62	0.10032	0.429433777968093\\
62	0.10398	0.456709948232053\\
62	0.10764	0.488306171387093\\
62	0.1113	0.524222447433216\\
62	0.11496	0.564458776370425\\
62	0.11862	0.609015158198718\\
62	0.12228	0.65789159291809\\
62	0.12594	0.711088080528548\\
62	0.1296	0.76860462103009\\
62	0.13326	0.830441214422711\\
62	0.13692	0.896597860706416\\
62	0.14058	0.967074559881208\\
62	0.14424	1.04187131194708\\
62	0.1479	1.12098811690403\\
62	0.15156	1.20442497475207\\
62	0.15522	1.2921818854912\\
62	0.15888	1.3842588491214\\
62	0.16254	1.48065586564269\\
62	0.1662	1.58137293505506\\
62	0.16986	1.68641005735851\\
62	0.17352	1.79576723255306\\
62	0.17718	1.90944446063867\\
62	0.18084	2.02744174161537\\
62	0.1845	2.14975907548317\\
62	0.18816	2.27639646224203\\
62	0.19182	2.40735390189199\\
62	0.19548	2.54263139443303\\
62	0.19914	2.68222893986514\\
62	0.2028	2.82614653818835\\
62	0.20646	2.97438418940263\\
62	0.21012	3.126941893508\\
62	0.21378	3.28381965050446\\
62	0.21744	3.44501746039199\\
62	0.2211	3.61053532317061\\
62	0.22476	3.78037323884031\\
62	0.22842	3.95453120740109\\
62	0.23208	4.13300922885296\\
62	0.23574	4.31580730319591\\
62	0.2394	4.50292543042994\\
62	0.24306	4.69436361055507\\
62	0.24672	4.89012184357126\\
62	0.25038	5.09020012947856\\
62	0.25404	5.29459846827692\\
62	0.2577	5.50331685996636\\
62	0.26136	5.71635530454689\\
62	0.26502	5.9337138020185\\
62	0.26868	6.1553923523812\\
62	0.27234	6.38139095563499\\
62	0.276	6.61170961177986\\
62.375	0.093	0.377280802436231\\
62.375	0.09666	0.399865057126611\\
62.375	0.10032	0.426769364708078\\
62.375	0.10398	0.457993725180627\\
62.375	0.10764	0.493538138544253\\
62.375	0.1113	0.533402604798969\\
62.375	0.11496	0.577587123944768\\
62.375	0.11862	0.62609169598165\\
62.375	0.12228	0.678916320909612\\
62.375	0.12594	0.73606099872866\\
62.375	0.1296	0.797525729438787\\
62.375	0.13326	0.863310513039998\\
62.375	0.13692	0.933415349532296\\
62.375	0.14058	1.00784023891567\\
62.375	0.14424	1.08658518119014\\
62.375	0.1479	1.16965017635568\\
62.375	0.15156	1.25703522441231\\
62.375	0.15522	1.34874032536003\\
62.375	0.15888	1.44476547919882\\
62.375	0.16254	1.5451106859287\\
62.375	0.1662	1.64977594554966\\
62.375	0.16986	1.7587612580617\\
62.375	0.17352	1.87206662346483\\
62.375	0.17718	1.98969204175904\\
62.375	0.18084	2.11163751294433\\
62.375	0.1845	2.23790303702072\\
62.375	0.18816	2.36848861398817\\
62.375	0.19182	2.50339424384672\\
62.375	0.19548	2.64261992659634\\
62.375	0.19914	2.78616566223705\\
62.375	0.2028	2.93403145076884\\
62.375	0.20646	3.08621729219172\\
62.375	0.21012	3.24272318650568\\
62.375	0.21378	3.40354913371072\\
62.375	0.21744	3.56869513380684\\
62.375	0.2211	3.73816118679405\\
62.375	0.22476	3.91194729267234\\
62.375	0.22842	4.09005345144171\\
62.375	0.23208	4.27247966310218\\
62.375	0.23574	4.45922592765371\\
62.375	0.2394	4.65029224509633\\
62.375	0.24306	4.84567861543005\\
62.375	0.24672	5.04538503865483\\
62.375	0.25038	5.24941151477071\\
62.375	0.25404	5.45775804377766\\
62.375	0.2577	5.6704246256757\\
62.375	0.26136	5.88741126046483\\
62.375	0.26502	6.10871794814503\\
62.375	0.26868	6.33434468871632\\
62.375	0.27234	6.56429148217869\\
62.375	0.276	6.79855832853214\\
62.75	0.093	0.365674802276511\\
62.75	0.09666	0.392207247175481\\
62.75	0.10032	0.423059744965534\\
62.75	0.10398	0.458232295646672\\
62.75	0.10764	0.497724899218891\\
62.75	0.1113	0.541537555682197\\
62.75	0.11496	0.589670265036586\\
62.75	0.11862	0.642123027282054\\
62.75	0.12228	0.698895842418605\\
62.75	0.12594	0.759988710446243\\
62.75	0.1296	0.825401631364964\\
62.75	0.13326	0.895134605174764\\
62.75	0.13692	0.969187631875652\\
62.75	0.14058	1.04756071146762\\
62.75	0.14424	1.13025384395067\\
62.75	0.1479	1.21726702932481\\
62.75	0.15156	1.30860026759003\\
62.75	0.15522	1.40425355874633\\
62.75	0.15888	1.50422690279371\\
62.75	0.16254	1.60852029973218\\
62.75	0.1662	1.71713374956173\\
62.75	0.16986	1.83006725228236\\
62.75	0.17352	1.94732080789408\\
62.75	0.17718	2.06889441639688\\
62.75	0.18084	2.19478807779076\\
62.75	0.1845	2.32500179207573\\
62.75	0.18816	2.45953555925177\\
62.75	0.19182	2.59838937931892\\
62.75	0.19548	2.74156325227712\\
62.75	0.19914	2.88905717812642\\
62.75	0.2028	3.04087115686681\\
62.75	0.20646	3.19700518849827\\
62.75	0.21012	3.35745927302083\\
62.75	0.21378	3.52223341043445\\
62.75	0.21744	3.69132760073917\\
62.75	0.2211	3.86474184393497\\
62.75	0.22476	4.04247614002185\\
62.75	0.22842	4.22453048899981\\
62.75	0.23208	4.41090489086886\\
62.75	0.23574	4.60159934562899\\
62.75	0.2394	4.7966138532802\\
62.75	0.24306	4.9959484138225\\
62.75	0.24672	5.19960302725587\\
62.75	0.25038	5.40757769358034\\
62.75	0.25404	5.61987241279589\\
62.75	0.2577	5.83648718490251\\
62.75	0.26136	6.05742200990023\\
62.75	0.26502	6.28267688778902\\
62.75	0.26868	6.5122518185689\\
62.75	0.27234	6.74614680223986\\
62.75	0.276	6.9843618388019\\
63.125	0.093	0.353023595634254\\
63.125	0.09666	0.383504230741813\\
63.125	0.10032	0.418304918740459\\
63.125	0.10398	0.457425659630188\\
63.125	0.10764	0.500866453410996\\
63.125	0.1113	0.548627300082888\\
63.125	0.11496	0.600708199645867\\
63.125	0.11862	0.657109152099927\\
63.125	0.12228	0.717830157445069\\
63.125	0.12594	0.782871215681296\\
63.125	0.1296	0.852232326808606\\
63.125	0.13326	0.925913490826996\\
63.125	0.13692	1.00391470773647\\
63.125	0.14058	1.08623597753703\\
63.125	0.14424	1.17287730022867\\
63.125	0.1479	1.2638386758114\\
63.125	0.15156	1.35912010428521\\
63.125	0.15522	1.4587215856501\\
63.125	0.15888	1.56264311990607\\
63.125	0.16254	1.67088470705313\\
63.125	0.1662	1.78344634709127\\
63.125	0.16986	1.90032804002049\\
63.125	0.17352	2.0215297858408\\
63.125	0.17718	2.14705158455219\\
63.125	0.18084	2.27689343615466\\
63.125	0.1845	2.41105534064822\\
63.125	0.18816	2.54953729803285\\
63.125	0.19182	2.69233930830858\\
63.125	0.19548	2.83946137147538\\
63.125	0.19914	2.99090348753327\\
63.125	0.2028	3.14666565648225\\
63.125	0.20646	3.3067478783223\\
63.125	0.21012	3.47115015305344\\
63.125	0.21378	3.63987248067566\\
63.125	0.21744	3.81291486118896\\
63.125	0.2211	3.99027729459335\\
63.125	0.22476	4.17195978088882\\
63.125	0.22842	4.35796232007537\\
63.125	0.23208	4.54828491215301\\
63.125	0.23574	4.74292755712173\\
63.125	0.2394	4.94189025498152\\
63.125	0.24306	5.14517300573243\\
63.125	0.24672	5.35277580937439\\
63.125	0.25038	5.56469866590744\\
63.125	0.25404	5.78094157533158\\
63.125	0.2577	6.00150453764679\\
63.125	0.26136	6.2263875528531\\
63.125	0.26502	6.45559062095047\\
63.125	0.26868	6.68911374193894\\
63.125	0.27234	6.9269569158185\\
63.125	0.276	7.16912014258913\\
63.5	0.093	0.339327182509474\\
63.5	0.09666	0.373756007825623\\
63.5	0.10032	0.412504886032855\\
63.5	0.10398	0.455573817131173\\
63.5	0.10764	0.502962801120571\\
63.5	0.1113	0.554671838001056\\
63.5	0.11496	0.610700927772624\\
63.5	0.11862	0.671050070435275\\
63.5	0.12228	0.735719265989005\\
63.5	0.12594	0.804708514433822\\
63.5	0.1296	0.878017815769722\\
63.5	0.13326	0.955647169996702\\
63.5	0.13692	1.03759657711477\\
63.5	0.14058	1.12386603712392\\
63.5	0.14424	1.21445555002415\\
63.5	0.1479	1.30936511581546\\
63.5	0.15156	1.40859473449786\\
63.5	0.15522	1.51214440607134\\
63.5	0.15888	1.6200141305359\\
63.5	0.16254	1.73220390789155\\
63.5	0.1662	1.84871373813828\\
63.5	0.16986	1.96954362127609\\
63.5	0.17352	2.09469355730499\\
63.5	0.17718	2.22416354622497\\
63.5	0.18084	2.35795358803603\\
63.5	0.1845	2.49606368273818\\
63.5	0.18816	2.63849383033141\\
63.5	0.19182	2.78524403081572\\
63.5	0.19548	2.93631428419112\\
63.5	0.19914	3.09170459045759\\
63.5	0.2028	3.25141494961516\\
63.5	0.20646	3.4154453616638\\
63.5	0.21012	3.58379582660353\\
63.5	0.21378	3.75646634443434\\
63.5	0.21744	3.93345691515623\\
63.5	0.2211	4.11476753876921\\
63.5	0.22476	4.30039821527327\\
63.5	0.22842	4.49034894466841\\
63.5	0.23208	4.68461972695464\\
63.5	0.23574	4.88321056213194\\
63.5	0.2394	5.08612145020034\\
63.5	0.24306	5.29335239115981\\
63.5	0.24672	5.50490338501037\\
63.5	0.25038	5.72077443175202\\
63.5	0.25404	5.94096553138474\\
63.5	0.2577	6.16547668390855\\
63.5	0.26136	6.39430788932344\\
63.5	0.26502	6.62745914762941\\
63.5	0.26868	6.86493045882647\\
63.5	0.27234	7.10672182291461\\
63.5	0.276	7.35283323989383\\
63.875	0.093	0.32458556290216\\
63.875	0.09666	0.362962578426899\\
63.875	0.10032	0.405659646842724\\
63.875	0.10398	0.452676768149632\\
63.875	0.10764	0.504013942347619\\
63.875	0.1113	0.559671169436694\\
63.875	0.11496	0.619648449416851\\
63.875	0.11862	0.683945782288088\\
63.875	0.12228	0.752563168050409\\
63.875	0.12594	0.825500606703815\\
63.875	0.1296	0.902758098248305\\
63.875	0.13326	0.984335642683874\\
63.875	0.13692	1.07023324001053\\
63.875	0.14058	1.16045089022827\\
63.875	0.14424	1.25498859333709\\
63.875	0.1479	1.35384634933699\\
63.875	0.15156	1.45702415822798\\
63.875	0.15522	1.56452202001005\\
63.875	0.15888	1.6763399346832\\
63.875	0.16254	1.79247790224745\\
63.875	0.1662	1.91293592270276\\
63.875	0.16986	2.03771399604916\\
63.875	0.17352	2.16681212228665\\
63.875	0.17718	2.30023030141522\\
63.875	0.18084	2.43796853343487\\
63.875	0.1845	2.58002681834561\\
63.875	0.18816	2.72640515614742\\
63.875	0.19182	2.87710354684033\\
63.875	0.19548	3.03212199042431\\
63.875	0.19914	3.19146048689938\\
63.875	0.2028	3.35511903626553\\
63.875	0.20646	3.52309763852277\\
63.875	0.21012	3.69539629367108\\
63.875	0.21378	3.87201500171048\\
63.875	0.21744	4.05295376264096\\
63.875	0.2211	4.23821257646254\\
63.875	0.22476	4.42779144317518\\
63.875	0.22842	4.62169036277891\\
63.875	0.23208	4.81990933527373\\
63.875	0.23574	5.02244836065963\\
63.875	0.2394	5.22930743893661\\
63.875	0.24306	5.44048657010468\\
63.875	0.24672	5.65598575416383\\
63.875	0.25038	5.87580499111406\\
63.875	0.25404	6.09994428095537\\
63.875	0.2577	6.32840362368777\\
63.875	0.26136	6.56118301931125\\
63.875	0.26502	6.79828246782581\\
63.875	0.26868	7.03970196923146\\
63.875	0.27234	7.28544152352819\\
63.875	0.276	7.535501130716\\
64.25	0.093	0.308798736812324\\
64.25	0.09666	0.351123942545651\\
64.25	0.10032	0.397769201170066\\
64.25	0.10398	0.44873451268556\\
64.25	0.10764	0.504019877092137\\
64.25	0.1113	0.563625294389802\\
64.25	0.11496	0.627550764578549\\
64.25	0.11862	0.695796287658379\\
64.25	0.12228	0.768361863629289\\
64.25	0.12594	0.845247492491285\\
64.25	0.1296	0.926453174244364\\
64.25	0.13326	1.01197890888852\\
64.25	0.13692	1.10182469642377\\
64.25	0.14058	1.19599053685009\\
64.25	0.14424	1.29447643016751\\
64.25	0.1479	1.397282376376\\
64.25	0.15156	1.50440837547558\\
64.25	0.15522	1.61585442746624\\
64.25	0.15888	1.73162053234798\\
64.25	0.16254	1.85170669012081\\
64.25	0.1662	1.97611290078472\\
64.25	0.16986	2.10483916433971\\
64.25	0.17352	2.23788548078579\\
64.25	0.17718	2.37525185012294\\
64.25	0.18084	2.51693827235118\\
64.25	0.1845	2.66294474747051\\
64.25	0.18816	2.81327127548091\\
64.25	0.19182	2.96791785638241\\
64.25	0.19548	3.12688449017498\\
64.25	0.19914	3.29017117685864\\
64.25	0.2028	3.45777791643338\\
64.25	0.20646	3.62970470889921\\
64.25	0.21012	3.80595155425612\\
64.25	0.21378	3.98651845250411\\
64.25	0.21744	4.17140540364317\\
64.25	0.2211	4.36061240767333\\
64.25	0.22476	4.55413946459457\\
64.25	0.22842	4.75198657440689\\
64.25	0.23208	4.9541537371103\\
64.25	0.23574	5.16064095270479\\
64.25	0.2394	5.37144822119036\\
64.25	0.24306	5.58657554256702\\
64.25	0.24672	5.80602291683476\\
64.25	0.25038	6.02979034399358\\
64.25	0.25404	6.25787782404348\\
64.25	0.2577	6.49028535698447\\
64.25	0.26136	6.72701294281654\\
64.25	0.26502	6.96806058153969\\
64.25	0.26868	7.21342827315392\\
64.25	0.27234	7.46311601765925\\
64.25	0.276	7.71712381505565\\
64.625	0.093	0.291966704239957\\
64.625	0.09666	0.338240100181874\\
64.625	0.10032	0.388833549014875\\
64.625	0.10398	0.443747050738962\\
64.625	0.10764	0.502980605354129\\
64.625	0.1113	0.566534212860383\\
64.625	0.11496	0.634407873257719\\
64.625	0.11862	0.706601586546139\\
64.625	0.12228	0.783115352725638\\
64.625	0.12594	0.863949171796224\\
64.625	0.1296	0.949103043757893\\
64.625	0.13326	1.03857696861064\\
64.625	0.13692	1.13237094635448\\
64.625	0.14058	1.23048497698939\\
64.625	0.14424	1.33291906051539\\
64.625	0.1479	1.43967319693248\\
64.625	0.15156	1.55074738624064\\
64.625	0.15522	1.66614162843989\\
64.625	0.15888	1.78585592353022\\
64.625	0.16254	1.90989027151164\\
64.625	0.1662	2.03824467238414\\
64.625	0.16986	2.17091912614772\\
64.625	0.17352	2.30791363280239\\
64.625	0.17718	2.44922819234814\\
64.625	0.18084	2.59486280478497\\
64.625	0.1845	2.74481747011288\\
64.625	0.18816	2.89909218833188\\
64.625	0.19182	3.05768695944197\\
64.625	0.19548	3.22060178344312\\
64.625	0.19914	3.38783666033537\\
64.625	0.2028	3.55939159011871\\
64.625	0.20646	3.73526657279311\\
64.625	0.21012	3.91546160835862\\
64.625	0.21378	4.09997669681519\\
64.625	0.21744	4.28881183816285\\
64.625	0.2211	4.4819670324016\\
64.625	0.22476	4.67944227953143\\
64.625	0.22842	4.88123757955234\\
64.625	0.23208	5.08735293246434\\
64.625	0.23574	5.29778833826742\\
64.625	0.2394	5.51254379696158\\
64.625	0.24306	5.73161930854682\\
64.625	0.24672	5.95501487302314\\
64.625	0.25038	6.18273049039056\\
64.625	0.25404	6.41476616064905\\
64.625	0.2577	6.65112188379863\\
64.625	0.26136	6.8917976598393\\
64.625	0.26502	7.13679348877104\\
64.625	0.26868	7.38610937059386\\
64.625	0.27234	7.63974530530777\\
64.625	0.276	7.89770129291276\\
65	0.093	0.274089465185058\\
65	0.09666	0.324311051335565\\
65	0.10032	0.378852690377159\\
65	0.10398	0.437714382309835\\
65	0.10764	0.500896127133592\\
65	0.1113	0.568397924848435\\
65	0.11496	0.640219775454358\\
65	0.11862	0.716361678951367\\
65	0.12228	0.796823635339456\\
65	0.12594	0.881605644618632\\
65	0.1296	0.97070770678889\\
65	0.13326	1.06412982185023\\
65	0.13692	1.16187198980265\\
65	0.14058	1.26393421064616\\
65	0.14424	1.37031648438075\\
65	0.1479	1.48101881100642\\
65	0.15156	1.59604119052318\\
65	0.15522	1.71538362293102\\
65	0.15888	1.83904610822994\\
65	0.16254	1.96702864641995\\
65	0.1662	2.09933123750104\\
65	0.16986	2.23595388147321\\
65	0.17352	2.37689657833646\\
65	0.17718	2.5221593280908\\
65	0.18084	2.67174213073622\\
65	0.1845	2.82564498627273\\
65	0.18816	2.98386789470031\\
65	0.19182	3.14641085601899\\
65	0.19548	3.31327387022874\\
65	0.19914	3.48445693732958\\
65	0.2028	3.6599600573215\\
65	0.20646	3.8397832302045\\
65	0.21012	4.02392645597859\\
65	0.21378	4.21238973464375\\
65	0.21744	4.40517306620001\\
65	0.2211	4.60227645064734\\
65	0.22476	4.80369988798576\\
65	0.22842	5.00944337821526\\
65	0.23208	5.21950692133585\\
65	0.23574	5.43389051734752\\
65	0.2394	5.65259416625026\\
65	0.24306	5.8756178680441\\
65	0.24672	6.10296162272901\\
65	0.25038	6.33462543030502\\
65	0.25404	6.5706092907721\\
65	0.2577	6.81091320413027\\
65	0.26136	7.05553717037952\\
65	0.26502	7.30448118951985\\
65	0.26868	7.55774526155126\\
65	0.27234	7.81532938647376\\
65	0.276	8.07723356428735\\
65.375	0.093	0.255167019647636\\
65.375	0.09666	0.309336796006732\\
65.375	0.10032	0.367826625256912\\
65.375	0.10398	0.430636507398179\\
65.375	0.10764	0.497766442430525\\
65.375	0.1113	0.569216430353958\\
65.375	0.11496	0.644986471168474\\
65.375	0.11862	0.725076564874072\\
65.375	0.12228	0.809486711470751\\
65.375	0.12594	0.898216910958516\\
65.375	0.1296	0.991267163337364\\
65.375	0.13326	1.08863746860729\\
65.375	0.13692	1.19032782676831\\
65.375	0.14058	1.29633823782041\\
65.375	0.14424	1.40666870176359\\
65.375	0.1479	1.52131921859785\\
65.375	0.15156	1.64028978832319\\
65.375	0.15522	1.76358041093962\\
65.375	0.15888	1.89119108644713\\
65.375	0.16254	2.02312181484573\\
65.375	0.1662	2.15937259613541\\
65.375	0.16986	2.29994343031617\\
65.375	0.17352	2.44483431738801\\
65.375	0.17718	2.59404525735094\\
65.375	0.18084	2.74757625020495\\
65.375	0.1845	2.90542729595005\\
65.375	0.18816	3.06759839458622\\
65.375	0.19182	3.23408954611348\\
65.375	0.19548	3.40490075053183\\
65.375	0.19914	3.58003200784125\\
65.375	0.2028	3.75948331804176\\
65.375	0.20646	3.94325468113336\\
65.375	0.21012	4.13134609711603\\
65.375	0.21378	4.32375756598979\\
65.375	0.21744	4.52048908775463\\
65.375	0.2211	4.72154066241056\\
65.375	0.22476	4.92691228995756\\
65.375	0.22842	5.13660397039565\\
65.375	0.23208	5.35061570372483\\
65.375	0.23574	5.56894748994509\\
65.375	0.2394	5.79159932905643\\
65.375	0.24306	6.01857122105886\\
65.375	0.24672	6.24986316595236\\
65.375	0.25038	6.48547516373695\\
65.375	0.25404	6.72540721441263\\
65.375	0.2577	6.96965931797938\\
65.375	0.26136	7.21823147443722\\
65.375	0.26502	7.47112368378614\\
65.375	0.26868	7.72833594602615\\
65.375	0.27234	7.98986826115724\\
65.375	0.276	8.25572062917941\\
65.75	0.093	0.23519936762768\\
65.75	0.09666	0.293317334195366\\
65.75	0.10032	0.35575535365414\\
65.75	0.10398	0.422513426003995\\
65.75	0.10764	0.493591551244931\\
65.75	0.1113	0.568989729376954\\
65.75	0.11496	0.648707960400059\\
65.75	0.11862	0.732746244314248\\
65.75	0.12228	0.821104581119516\\
65.75	0.12594	0.913782970815871\\
65.75	0.1296	1.01078141340331\\
65.75	0.13326	1.11209990888183\\
65.75	0.13692	1.21773845725143\\
65.75	0.14058	1.32769705851211\\
65.75	0.14424	1.44197571266389\\
65.75	0.1479	1.56057441970674\\
65.75	0.15156	1.68349317964067\\
65.75	0.15522	1.81073199246569\\
65.75	0.15888	1.94229085818179\\
65.75	0.16254	2.07816977678898\\
65.75	0.1662	2.21836874828725\\
65.75	0.16986	2.36288777267659\\
65.75	0.17352	2.51172684995703\\
65.75	0.17718	2.66488598012855\\
65.75	0.18084	2.82236516319115\\
65.75	0.1845	2.98416439914483\\
65.75	0.18816	3.1502836879896\\
65.75	0.19182	3.32072302972545\\
65.75	0.19548	3.49548242435239\\
65.75	0.19914	3.6745618718704\\
65.75	0.2028	3.8579613722795\\
65.75	0.20646	4.04568092557968\\
65.75	0.21012	4.23772053177095\\
65.75	0.21378	4.4340801908533\\
65.75	0.21744	4.63475990282672\\
65.75	0.2211	4.83975966769124\\
65.75	0.22476	5.04907948544684\\
65.75	0.22842	5.26271935609352\\
65.75	0.23208	5.48067927963128\\
65.75	0.23574	5.70295925606013\\
65.75	0.2394	5.92955928538006\\
65.75	0.24306	6.16047936759108\\
65.75	0.24672	6.39571950269317\\
65.75	0.25038	6.63527969068635\\
65.75	0.25404	6.87915993157061\\
65.75	0.2577	7.12736022534595\\
65.75	0.26136	7.3798805720124\\
65.75	0.26502	7.6367209715699\\
65.75	0.26868	7.8978814240185\\
65.75	0.27234	8.16336192935817\\
65.75	0.276	8.43316248758893\\
66.125	0.093	0.214186509125203\\
66.125	0.09666	0.276252665901479\\
66.125	0.10032	0.342638875568842\\
66.125	0.10398	0.413345138127287\\
66.125	0.10764	0.488371453576813\\
66.125	0.1113	0.567717821917421\\
66.125	0.11496	0.651384243149117\\
66.125	0.11862	0.739370717271894\\
66.125	0.12228	0.831677244285753\\
66.125	0.12594	0.9283038241907\\
66.125	0.1296	1.02925045698673\\
66.125	0.13326	1.13451714267383\\
66.125	0.13692	1.24410388125203\\
66.125	0.14058	1.3580106727213\\
66.125	0.14424	1.47623751708166\\
66.125	0.1479	1.5987844143331\\
66.125	0.15156	1.72565136447563\\
66.125	0.15522	1.85683836750924\\
66.125	0.15888	1.99234542343393\\
66.125	0.16254	2.13217253224971\\
66.125	0.1662	2.27631969395656\\
66.125	0.16986	2.4247869085545\\
66.125	0.17352	2.57757417604352\\
66.125	0.17718	2.73468149642363\\
66.125	0.18084	2.89610886969482\\
66.125	0.1845	3.0618562958571\\
66.125	0.18816	3.23192377491045\\
66.125	0.19182	3.4063113068549\\
66.125	0.19548	3.58501889169041\\
66.125	0.19914	3.76804652941702\\
66.125	0.2028	3.95539422003471\\
66.125	0.20646	4.14706196354349\\
66.125	0.21012	4.34304975994334\\
66.125	0.21378	4.54335760923428\\
66.125	0.21744	4.74798551141629\\
66.125	0.2211	4.95693346648941\\
66.125	0.22476	5.17020147445359\\
66.125	0.22842	5.38778953530886\\
66.125	0.23208	5.60969764905522\\
66.125	0.23574	5.83592581569265\\
66.125	0.2394	6.06647403522117\\
66.125	0.24306	6.30134230764077\\
66.125	0.24672	6.54053063295146\\
66.125	0.25038	6.78403901115323\\
66.125	0.25404	7.03186744224609\\
66.125	0.2577	7.28401592623002\\
66.125	0.26136	7.54048446310504\\
66.125	0.26502	7.80127305287114\\
66.125	0.26868	8.06638169552832\\
66.125	0.27234	8.33581039107659\\
66.125	0.276	8.60955913951594\\
66.5	0.093	0.192128444140185\\
66.5	0.09666	0.258142791125051\\
66.5	0.10032	0.328477191001003\\
66.5	0.10398	0.403131643768038\\
66.5	0.10764	0.482106149426153\\
66.5	0.1113	0.565400707975355\\
66.5	0.11496	0.65301531941564\\
66.5	0.11862	0.744949983747007\\
66.5	0.12228	0.841204700969455\\
66.5	0.12594	0.941779471082989\\
66.5	0.1296	1.04667429408761\\
66.5	0.13326	1.1558891699833\\
66.5	0.13692	1.26942409877009\\
66.5	0.14058	1.38727908044795\\
66.5	0.14424	1.5094541150169\\
66.5	0.1479	1.63594920247693\\
66.5	0.15156	1.76676434282805\\
66.5	0.15522	1.90189953607025\\
66.5	0.15888	2.04135478220352\\
66.5	0.16254	2.18513008122789\\
66.5	0.1662	2.33322543314334\\
66.5	0.16986	2.48564083794987\\
66.5	0.17352	2.64237629564748\\
66.5	0.17718	2.80343180623618\\
66.5	0.18084	2.96880736971595\\
66.5	0.1845	3.13850298608682\\
66.5	0.18816	3.31251865534877\\
66.5	0.19182	3.4908543775018\\
66.5	0.19548	3.67351015254591\\
66.5	0.19914	3.8604859804811\\
66.5	0.2028	4.05178186130739\\
66.5	0.20646	4.24739779502474\\
66.5	0.21012	4.44733378163319\\
66.5	0.21378	4.65158982113272\\
66.5	0.21744	4.86016591352333\\
66.5	0.2211	5.07306205880502\\
66.5	0.22476	5.2902782569778\\
66.5	0.22842	5.51181450804166\\
66.5	0.23208	5.73767081199661\\
66.5	0.23574	5.96784716884263\\
66.5	0.2394	6.20234357857974\\
66.5	0.24306	6.44116004120793\\
66.5	0.24672	6.68429655672721\\
66.5	0.25038	6.93175312513757\\
66.5	0.25404	7.18352974643901\\
66.5	0.2577	7.43962642063153\\
66.5	0.26136	7.70004314771515\\
66.5	0.26502	7.96477992768983\\
66.5	0.26868	8.23383676055561\\
66.5	0.27234	8.50721364631247\\
66.5	0.276	8.78491058496041\\
66.875	0.093	0.169025172672648\\
66.875	0.09666	0.238987709866103\\
66.875	0.10032	0.313270299950645\\
66.875	0.10398	0.39187294292627\\
66.875	0.10764	0.474795638792975\\
66.875	0.1113	0.562038387550766\\
66.875	0.11496	0.65360118919964\\
66.875	0.11862	0.749484043739598\\
66.875	0.12228	0.849686951170635\\
66.875	0.12594	0.954209911492758\\
66.875	0.1296	1.06305292470596\\
66.875	0.13326	1.17621599081025\\
66.875	0.13692	1.29369910980562\\
66.875	0.14058	1.41550228169208\\
66.875	0.14424	1.54162550646962\\
66.875	0.1479	1.67206878413824\\
66.875	0.15156	1.80683211469794\\
66.875	0.15522	1.94591549814873\\
66.875	0.15888	2.0893189344906\\
66.875	0.16254	2.23704242372356\\
66.875	0.1662	2.3890859658476\\
66.875	0.16986	2.54544956086271\\
66.875	0.17352	2.70613320876892\\
66.875	0.17718	2.8711369095662\\
66.875	0.18084	3.04046066325457\\
66.875	0.1845	3.21410446983402\\
66.875	0.18816	3.39206832930456\\
66.875	0.19182	3.57435224166618\\
66.875	0.19548	3.76095620691888\\
66.875	0.19914	3.95188022506266\\
66.875	0.2028	4.14712429609754\\
66.875	0.20646	4.34668842002349\\
66.875	0.21012	4.55057259684053\\
66.875	0.21378	4.75877682654864\\
66.875	0.21744	4.97130110914784\\
66.875	0.2211	5.18814544463812\\
66.875	0.22476	5.40930983301949\\
66.875	0.22842	5.63479427429194\\
66.875	0.23208	5.86459876845548\\
66.875	0.23574	6.09872331551009\\
66.875	0.2394	6.33716791545579\\
66.875	0.24306	6.57993256829257\\
66.875	0.24672	6.82701727402044\\
66.875	0.25038	7.07842203263939\\
66.875	0.25404	7.33414684414942\\
66.875	0.2577	7.59419170855053\\
66.875	0.26136	7.85855662584274\\
66.875	0.26502	8.12724159602601\\
66.875	0.26868	8.40024661910038\\
66.875	0.27234	8.67757169506583\\
66.875	0.276	8.95921682392236\\
67.25	0.093	0.144876694722584\\
67.25	0.09666	0.218787422124629\\
67.25	0.10032	0.297018202417757\\
67.25	0.10398	0.379569035601971\\
67.25	0.10764	0.466439921677269\\
67.25	0.1113	0.557630860643647\\
67.25	0.11496	0.653141852501111\\
67.25	0.11862	0.752972897249657\\
67.25	0.12228	0.857123994889288\\
67.25	0.12594	0.965595145420001\\
67.25	0.1296	1.0783863488418\\
67.25	0.13326	1.19549760515467\\
67.25	0.13692	1.31692891435864\\
67.25	0.14058	1.44268027645368\\
67.25	0.14424	1.57275169143981\\
67.25	0.1479	1.70714315931702\\
67.25	0.15156	1.84585468008531\\
67.25	0.15522	1.98888625374469\\
67.25	0.15888	2.13623788029515\\
67.25	0.16254	2.2879095597367\\
67.25	0.1662	2.44390129206932\\
67.25	0.16986	2.60421307729303\\
67.25	0.17352	2.76884491540783\\
67.25	0.17718	2.9377968064137\\
67.25	0.18084	3.11106875031066\\
67.25	0.1845	3.2886607470987\\
67.25	0.18816	3.47057279677783\\
67.25	0.19182	3.65680489934804\\
67.25	0.19548	3.84735705480933\\
67.25	0.19914	4.0422292631617\\
67.25	0.2028	4.24142152440516\\
67.25	0.20646	4.4449338385397\\
67.25	0.21012	4.65276620556533\\
67.25	0.21378	4.86491862548203\\
67.25	0.21744	5.08139109828982\\
67.25	0.2211	5.3021836239887\\
67.25	0.22476	5.52729620257866\\
67.25	0.22842	5.75672883405969\\
67.25	0.23208	5.99048151843182\\
67.25	0.23574	6.22855425569502\\
67.25	0.2394	6.47094704584931\\
67.25	0.24306	6.71765988889469\\
67.25	0.24672	6.96869278483114\\
67.25	0.25038	7.22404573365868\\
67.25	0.25404	7.4837187353773\\
67.25	0.2577	7.747711789987\\
67.25	0.26136	8.01602489748779\\
67.25	0.26502	8.28865805787966\\
67.25	0.26868	8.56561127116261\\
67.25	0.27234	8.84688453733666\\
67.25	0.276	9.13247785640178\\
67.625	0.093	0.119683010289987\\
67.625	0.09666	0.197541927900621\\
67.625	0.10032	0.279720898402342\\
67.625	0.10398	0.366219921795146\\
67.625	0.10764	0.45703899807903\\
67.625	0.1113	0.552178127254001\\
67.625	0.11496	0.651637309320054\\
67.625	0.11862	0.755416544277191\\
67.625	0.12228	0.863515832125407\\
67.625	0.12594	0.97593517286471\\
67.625	0.1296	1.0926745664951\\
67.625	0.13326	1.21373401301656\\
67.625	0.13692	1.33911351242912\\
67.625	0.14058	1.46881306473275\\
67.625	0.14424	1.60283266992747\\
67.625	0.1479	1.74117232801327\\
67.625	0.15156	1.88383203899015\\
67.625	0.15522	2.03081180285812\\
67.625	0.15888	2.18211161961717\\
67.625	0.16254	2.3377314892673\\
67.625	0.1662	2.49767141180852\\
67.625	0.16986	2.66193138724082\\
67.625	0.17352	2.8305114155642\\
67.625	0.17718	3.00341149677867\\
67.625	0.18084	3.18063163088421\\
67.625	0.1845	3.36217181788085\\
67.625	0.18816	3.54803205776856\\
67.625	0.19182	3.73821235054737\\
67.625	0.19548	3.93271269621724\\
67.625	0.19914	4.13153309477821\\
67.625	0.2028	4.33467354623026\\
67.625	0.20646	4.54213405057339\\
67.625	0.21012	4.7539146078076\\
67.625	0.21378	4.9700152179329\\
67.625	0.21744	5.19043588094927\\
67.625	0.2211	5.41517659685674\\
67.625	0.22476	5.64423736565528\\
67.625	0.22842	5.87761818734491\\
67.625	0.23208	6.11531906192563\\
67.625	0.23574	6.35733998939743\\
67.625	0.2394	6.6036809697603\\
67.625	0.24306	6.85434200301427\\
67.625	0.24672	7.10932308915931\\
67.625	0.25038	7.36862422819544\\
67.625	0.25404	7.63224542012265\\
67.625	0.2577	7.90018666494094\\
67.625	0.26136	8.17244796265032\\
67.625	0.26502	8.44902931325078\\
67.625	0.26868	8.72993071674232\\
67.625	0.27234	9.01515217312495\\
67.625	0.276	9.30469368239866\\
68	0.093	0.0934441193748627\\
68	0.09666	0.175251227194087\\
68	0.10032	0.261378387904398\\
68	0.10398	0.351825601505791\\
68	0.10764	0.446592867998264\\
68	0.1113	0.545680187381825\\
68	0.11496	0.649087559656468\\
68	0.11862	0.756814984822194\\
68	0.12228	0.868862462879\\
68	0.12594	0.985229993826892\\
68	0.1296	1.10591757766587\\
68	0.13326	1.23092521439592\\
68	0.13692	1.36025290401706\\
68	0.14058	1.49390064652929\\
68	0.14424	1.6318684419326\\
68	0.1479	1.77415629022699\\
68	0.15156	1.92076419141246\\
68	0.15522	2.07169214548902\\
68	0.15888	2.22694015245666\\
68	0.16254	2.38650821231538\\
68	0.1662	2.55039632506519\\
68	0.16986	2.71860449070608\\
68	0.17352	2.89113270923805\\
68	0.17718	3.0679809806611\\
68	0.18084	3.24914930497524\\
68	0.1845	3.43463768218047\\
68	0.18816	3.62444611227677\\
68	0.19182	3.81857459526416\\
68	0.19548	4.01702313114263\\
68	0.19914	4.21979171991218\\
68	0.2028	4.42688036157282\\
68	0.20646	4.63828905612454\\
68	0.21012	4.85401780356734\\
68	0.21378	5.07406660390123\\
68	0.21744	5.2984354571262\\
68	0.2211	5.52712436324226\\
68	0.22476	5.76013332224939\\
68	0.22842	5.9974623341476\\
68	0.23208	6.23911139893691\\
68	0.23574	6.4850805166173\\
68	0.2394	6.73536968718876\\
68	0.24306	6.98997891065132\\
68	0.24672	7.24890818700494\\
68	0.25038	7.51215751624967\\
68	0.25404	7.77972689838547\\
68	0.2577	8.05161633341235\\
68	0.26136	8.32782582133032\\
68	0.26502	8.60835536213937\\
68	0.26868	8.8932049558395\\
68	0.27234	9.18237460243072\\
68	0.276	9.47586430191301\\
68.375	0.093	0.0661600219772049\\
68.375	0.09666	0.151915320005019\\
68.375	0.10032	0.241990670923919\\
68.375	0.10398	0.336386074733902\\
68.375	0.10764	0.435101531434965\\
68.375	0.1113	0.538137041027115\\
68.375	0.11496	0.645492603510348\\
68.375	0.11862	0.757168218884663\\
68.375	0.12228	0.873163887150063\\
68.375	0.12594	0.993479608306544\\
68.375	0.1296	1.11811538235411\\
68.375	0.13326	1.24707120929275\\
68.375	0.13692	1.38034708912249\\
68.375	0.14058	1.5179430218433\\
68.375	0.14424	1.6598590074552\\
68.375	0.1479	1.80609504595818\\
68.375	0.15156	1.95665113735224\\
68.375	0.15522	2.11152728163739\\
68.375	0.15888	2.27072347881361\\
68.375	0.16254	2.43423972888093\\
68.375	0.1662	2.60207603183932\\
68.375	0.16986	2.7742323876888\\
68.375	0.17352	2.95070879642937\\
68.375	0.17718	3.13150525806101\\
68.375	0.18084	3.31662177258373\\
68.375	0.1845	3.50605833999755\\
68.375	0.18816	3.69981496030244\\
68.375	0.19182	3.89789163349842\\
68.375	0.19548	4.10028835958548\\
68.375	0.19914	4.30700513856362\\
68.375	0.2028	4.51804197043285\\
68.375	0.20646	4.73339885519316\\
68.375	0.21012	4.95307579284456\\
68.375	0.21378	5.17707278338704\\
68.375	0.21744	5.40538982682059\\
68.375	0.2211	5.63802692314524\\
68.375	0.22476	5.87498407236096\\
68.375	0.22842	6.11626127446777\\
68.375	0.23208	6.36185852946566\\
68.375	0.23574	6.61177583735464\\
68.375	0.2394	6.86601319813468\\
68.375	0.24306	7.12457061180584\\
68.375	0.24672	7.38744807836806\\
68.375	0.25038	7.65464559782137\\
68.375	0.25404	7.92616317016576\\
68.375	0.2577	8.20200079540123\\
68.375	0.26136	8.48215847352778\\
68.375	0.26502	8.76663620454542\\
68.375	0.26868	9.05543398845414\\
68.375	0.27234	9.34855182525396\\
68.375	0.276	9.64598971494484\\
68.75	0.093	0.0378307180970312\\
68.75	0.09666	0.127534206333434\\
68.75	0.10032	0.221557747460924\\
68.75	0.10398	0.319901341479497\\
68.75	0.10764	0.42256498838915\\
68.75	0.1113	0.529548688189889\\
68.75	0.11496	0.640852440881712\\
68.75	0.11862	0.756476246464617\\
68.75	0.12228	0.876420104938602\\
68.75	0.12594	1.00068401630367\\
68.75	0.1296	1.12926798055983\\
68.75	0.13326	1.26217199770707\\
68.75	0.13692	1.39939606774539\\
68.75	0.14058	1.54094019067479\\
68.75	0.14424	1.68680436649528\\
68.75	0.1479	1.83698859520685\\
68.75	0.15156	1.9914928768095\\
68.75	0.15522	2.15031721130324\\
68.75	0.15888	2.31346159868805\\
68.75	0.16254	2.48092603896396\\
68.75	0.1662	2.65271053213094\\
68.75	0.16986	2.82881507818901\\
68.75	0.17352	3.00923967713817\\
68.75	0.17718	3.1939843289784\\
68.75	0.18084	3.38304903370971\\
68.75	0.1845	3.57643379133212\\
68.75	0.18816	3.7741386018456\\
68.75	0.19182	3.97616346525017\\
68.75	0.19548	4.18250838154582\\
68.75	0.19914	4.39317335073255\\
68.75	0.2028	4.60815837281037\\
68.75	0.20646	4.82746344777927\\
68.75	0.21012	5.05108857563926\\
68.75	0.21378	5.27903375639031\\
68.75	0.21744	5.51129899003246\\
68.75	0.2211	5.7478842765657\\
68.75	0.22476	5.98878961599001\\
68.75	0.22842	6.2340150083054\\
68.75	0.23208	6.48356045351189\\
68.75	0.23574	6.73742595160945\\
68.75	0.2394	6.9956115025981\\
68.75	0.24306	7.25811710647784\\
68.75	0.24672	7.52494276324865\\
68.75	0.25038	7.79608847291055\\
68.75	0.25404	8.07155423546353\\
68.75	0.2577	8.35134005090759\\
68.75	0.26136	8.63544591924274\\
68.75	0.26502	8.92387184046897\\
68.75	0.26868	9.21661781458627\\
68.75	0.27234	9.51368384159468\\
68.75	0.276	9.81506992149415\\
69.125	0.093	0.0084562077343131\\
69.125	0.09666	0.102107886179309\\
69.125	0.10032	0.200079617515386\\
69.125	0.10398	0.302371401742551\\
69.125	0.10764	0.408983238860794\\
69.125	0.1113	0.519915128870123\\
69.125	0.11496	0.635167071770535\\
69.125	0.11862	0.75473906756203\\
69.125	0.12228	0.878631116244605\\
69.125	0.12594	1.00684321781827\\
69.125	0.1296	1.13937537228301\\
69.125	0.13326	1.27622757963883\\
69.125	0.13692	1.41739983988574\\
69.125	0.14058	1.56289215302374\\
69.125	0.14424	1.71270451905282\\
69.125	0.1479	1.86683693797297\\
69.125	0.15156	2.02528940978422\\
69.125	0.15522	2.18806193448655\\
69.125	0.15888	2.35515451207995\\
69.125	0.16254	2.52656714256445\\
69.125	0.1662	2.70229982594002\\
69.125	0.16986	2.88235256220668\\
69.125	0.17352	3.06672535136442\\
69.125	0.17718	3.25541819341324\\
69.125	0.18084	3.44843108835315\\
69.125	0.1845	3.64576403618414\\
69.125	0.18816	3.84741703690621\\
69.125	0.19182	4.05339009051937\\
69.125	0.19548	4.26368319702361\\
69.125	0.19914	4.47829635641893\\
69.125	0.2028	4.69722956870534\\
69.125	0.20646	4.92048283388283\\
69.125	0.21012	5.14805615195141\\
69.125	0.21378	5.37994952291106\\
69.125	0.21744	5.6161629467618\\
69.125	0.2211	5.85669642350362\\
69.125	0.22476	6.10154995313652\\
69.125	0.22842	6.3507235356605\\
69.125	0.23208	6.60421717107558\\
69.125	0.23574	6.86203085938173\\
69.125	0.2394	7.12416460057897\\
69.125	0.24306	7.39061839466729\\
69.125	0.24672	7.6613922416467\\
69.125	0.25038	7.93648614151719\\
69.125	0.25404	8.21590009427876\\
69.125	0.2577	8.4996340999314\\
69.125	0.26136	8.78768815847515\\
69.125	0.26502	9.08006226990996\\
69.125	0.26868	9.37675643423586\\
69.125	0.27234	9.67777065145285\\
69.125	0.276	9.98310492156092\\
69.5	0.093	-0.0219635091109245\\
69.5	0.09666	0.0756363595426579\\
69.5	0.10032	0.177556281087327\\
69.5	0.10398	0.283796255523079\\
69.5	0.10764	0.394356282849911\\
69.5	0.1113	0.50923636306783\\
69.5	0.11496	0.628436496176831\\
69.5	0.11862	0.751956682176916\\
69.5	0.12228	0.879796921068084\\
69.5	0.12594	1.01195721285033\\
69.5	0.1296	1.14843755752367\\
69.5	0.13326	1.28923795508808\\
69.5	0.13692	1.43435840554358\\
69.5	0.14058	1.58379890889017\\
69.5	0.14424	1.73755946512783\\
69.5	0.1479	1.89564007425658\\
69.5	0.15156	2.05804073627641\\
69.5	0.15522	2.22476145118733\\
69.5	0.15888	2.39580221898932\\
69.5	0.16254	2.57116303968241\\
69.5	0.1662	2.75084391326657\\
69.5	0.16986	2.93484483974182\\
69.5	0.17352	3.12316581910815\\
69.5	0.17718	3.31580685136557\\
69.5	0.18084	3.51276793651406\\
69.5	0.1845	3.71404907455365\\
69.5	0.18816	3.9196502654843\\
69.5	0.19182	4.12957150930606\\
69.5	0.19548	4.34381280601888\\
69.5	0.19914	4.56237415562279\\
69.5	0.2028	4.78525555811779\\
69.5	0.20646	5.01245701350387\\
69.5	0.21012	5.24397852178103\\
69.5	0.21378	5.47982008294928\\
69.5	0.21744	5.7199816970086\\
69.5	0.2211	5.96446336395902\\
69.5	0.22476	6.21326508380052\\
69.5	0.22842	6.46638685653309\\
69.5	0.23208	6.72382868215676\\
69.5	0.23574	6.9855905606715\\
69.5	0.2394	7.25167249207732\\
69.5	0.24306	7.52207447637423\\
69.5	0.24672	7.79679651356223\\
69.5	0.25038	8.07583860364131\\
69.5	0.25404	8.35920074661146\\
69.5	0.2577	8.6468829424727\\
69.5	0.26136	8.93888519122503\\
69.5	0.26502	9.23520749286843\\
69.5	0.26868	9.53584984740293\\
69.5	0.27234	9.84081225482851\\
69.5	0.276	10.1500947151452\\
69.875	0.093	-0.0534284324386887\\
69.875	0.09666	0.0481196264234833\\
69.875	0.10032	0.153987738176742\\
69.875	0.10398	0.264175902821084\\
69.875	0.10764	0.378684120356505\\
69.875	0.1113	0.497512390783014\\
69.875	0.11496	0.620660714100605\\
69.875	0.11862	0.748129090309279\\
69.875	0.12228	0.879917519409033\\
69.875	0.12594	1.01602600139987\\
69.875	0.1296	1.1564545362818\\
69.875	0.13326	1.3012031240548\\
69.875	0.13692	1.45027176471889\\
69.875	0.14058	1.60366045827407\\
69.875	0.14424	1.76136920472032\\
69.875	0.1479	1.92339800405766\\
69.875	0.15156	2.08974685628608\\
69.875	0.15522	2.26041576140559\\
69.875	0.15888	2.43540471941617\\
69.875	0.16254	2.61471373031785\\
69.875	0.1662	2.7983427941106\\
69.875	0.16986	2.98629191079443\\
69.875	0.17352	3.17856108036936\\
69.875	0.17718	3.37515030283536\\
69.875	0.18084	3.57605957819244\\
69.875	0.1845	3.78128890644062\\
69.875	0.18816	3.99083828757987\\
69.875	0.19182	4.20470772161021\\
69.875	0.19548	4.42289720853163\\
69.875	0.19914	4.64540674834413\\
69.875	0.2028	4.87223634104772\\
69.875	0.20646	5.10338598664239\\
69.875	0.21012	5.33885568512814\\
69.875	0.21378	5.57864543650497\\
69.875	0.21744	5.82275524077288\\
69.875	0.2211	6.07118509793189\\
69.875	0.22476	6.32393500798197\\
69.875	0.22842	6.58100497092314\\
69.875	0.23208	6.8423949867554\\
69.875	0.23574	7.10810505547873\\
69.875	0.2394	7.37813517709314\\
69.875	0.24306	7.65248535159864\\
69.875	0.24672	7.93115557899522\\
69.875	0.25038	8.21414585928289\\
69.875	0.25404	8.50145619246164\\
69.875	0.2577	8.79308657853147\\
69.875	0.26136	9.08903701749239\\
69.875	0.26502	9.38930750934439\\
69.875	0.26868	9.69389805408747\\
69.875	0.27234	10.0028086517216\\
69.875	0.276	10.3160393022469\\
70.25	0.093	-0.085938562248983\\
70.25	0.09666	0.0195576868217786\\
70.25	0.10032	0.129373988783627\\
70.25	0.10398	0.243510343636558\\
70.25	0.10764	0.361966751380569\\
70.25	0.1113	0.484743212015667\\
70.25	0.11496	0.611839725541852\\
70.25	0.11862	0.743256291959115\\
70.25	0.12228	0.878992911267459\\
70.25	0.12594	1.01904958346689\\
70.25	0.1296	1.1634263085574\\
70.25	0.13326	1.312123086539\\
70.25	0.13692	1.46513991741168\\
70.25	0.14058	1.62247680117544\\
70.25	0.14424	1.78413373783029\\
70.25	0.1479	1.95011072737621\\
70.25	0.15156	2.12040776981322\\
70.25	0.15522	2.29502486514132\\
70.25	0.15888	2.4739620133605\\
70.25	0.16254	2.65721921447076\\
70.25	0.1662	2.8447964684721\\
70.25	0.16986	3.03669377536453\\
70.25	0.17352	3.23291113514804\\
70.25	0.17718	3.43344854782263\\
70.25	0.18084	3.63830601338831\\
70.25	0.1845	3.84748353184507\\
70.25	0.18816	4.06098110319291\\
70.25	0.19182	4.27879872743184\\
70.25	0.19548	4.50093640456185\\
70.25	0.19914	4.72739413458293\\
70.25	0.2028	4.95817191749512\\
70.25	0.20646	5.19326975329837\\
70.25	0.21012	5.43268764199272\\
70.25	0.21378	5.67642558357814\\
70.25	0.21744	5.92448357805464\\
70.25	0.2211	6.17686162542224\\
70.25	0.22476	6.43355972568091\\
70.25	0.22842	6.69457787883067\\
70.25	0.23208	6.95991608487151\\
70.25	0.23574	7.22957434380343\\
70.25	0.2394	7.50355265562643\\
70.25	0.24306	7.78185102034053\\
70.25	0.24672	8.0644694379457\\
70.25	0.25038	8.35140790844196\\
70.25	0.25404	8.6426664318293\\
70.25	0.2577	8.93824500810771\\
70.25	0.26136	9.23814363727723\\
70.25	0.26502	9.54236231933781\\
70.25	0.26868	9.85090105428948\\
70.25	0.27234	10.1637598421322\\
70.25	0.276	10.4809386828661\\
70.625	0.093	-0.119493898541811\\
70.625	0.09666	-0.0100494592624598\\
70.625	0.10032	0.103715032907978\\
70.625	0.10398	0.221799577969499\\
70.625	0.10764	0.3442041759221\\
70.625	0.1113	0.470928826765787\\
70.625	0.11496	0.601973530500558\\
70.625	0.11862	0.737338287126411\\
70.625	0.12228	0.877023096643348\\
70.625	0.12594	1.02102795905137\\
70.625	0.1296	1.16935287435047\\
70.625	0.13326	1.32199784254065\\
70.625	0.13692	1.47896286362192\\
70.625	0.14058	1.64024793759427\\
70.625	0.14424	1.80585306445771\\
70.625	0.1479	1.97577824421223\\
70.625	0.15156	2.15002347685783\\
70.625	0.15522	2.32858876239451\\
70.625	0.15888	2.51147410082228\\
70.625	0.16254	2.69867949214113\\
70.625	0.1662	2.89020493635106\\
70.625	0.16986	3.08605043345208\\
70.625	0.17352	3.28621598344418\\
70.625	0.17718	3.49070158632736\\
70.625	0.18084	3.69950724210162\\
70.625	0.1845	3.91263295076698\\
70.625	0.18816	4.13007871232341\\
70.625	0.19182	4.35184452677093\\
70.625	0.19548	4.57793039410953\\
70.625	0.19914	4.8083363143392\\
70.625	0.2028	5.04306228745998\\
70.625	0.20646	5.28210831347182\\
70.625	0.21012	5.52547439237476\\
70.625	0.21378	5.77316052416877\\
70.625	0.21744	6.02516670885386\\
70.625	0.2211	6.28149294643004\\
70.625	0.22476	6.54213923689731\\
70.625	0.22842	6.80710558025565\\
70.625	0.23208	7.07639197650508\\
70.625	0.23574	7.3499984256456\\
70.625	0.2394	7.62792492767719\\
70.625	0.24306	7.91017148259987\\
70.625	0.24672	8.19673809041363\\
70.625	0.25038	8.48762475111849\\
70.625	0.25404	8.78283146471441\\
70.625	0.2577	9.08235823120142\\
70.625	0.26136	9.38620505057952\\
70.625	0.26502	9.69437192284869\\
70.625	0.26868	10.006858848009\\
70.625	0.27234	10.3236658260603\\
70.625	0.276	10.6447928570027\\
71	0.093	-0.154094441317162\\
71	0.09666	-0.0407018118292213\\
71	0.10032	0.0770108705498065\\
71	0.10398	0.199043605819917\\
71	0.10764	0.325396393981111\\
71	0.1113	0.456069235033388\\
71	0.11496	0.591062128976748\\
71	0.11862	0.730375075811191\\
71	0.12228	0.874008075536714\\
71	0.12594	1.02196112815333\\
71	0.1296	1.17423423366102\\
71	0.13326	1.33082739205979\\
71	0.13692	1.49174060334965\\
71	0.14058	1.65697386753059\\
71	0.14424	1.82652718460262\\
71	0.1479	2.00040055456572\\
71	0.15156	2.17859397741992\\
71	0.15522	2.36110745316519\\
71	0.15888	2.54794098180154\\
71	0.16254	2.73909456332899\\
71	0.1662	2.93456819774751\\
71	0.16986	3.13436188505711\\
71	0.17352	3.33847562525781\\
71	0.17718	3.54690941834957\\
71	0.18084	3.75966326433243\\
71	0.1845	3.97673716320638\\
71	0.18816	4.19813111497139\\
71	0.19182	4.42384511962751\\
71	0.19548	4.65387917717469\\
71	0.19914	4.88823328761296\\
71	0.2028	5.12690745094232\\
71	0.20646	5.36990166716275\\
71	0.21012	5.61721593627427\\
71	0.21378	5.86885025827688\\
71	0.21744	6.12480463317057\\
71	0.2211	6.38507906095534\\
71	0.22476	6.64967354163119\\
71	0.22842	6.91858807519812\\
71	0.23208	7.19182266165615\\
71	0.23574	7.46937730100524\\
71	0.2394	7.75125199324543\\
71	0.24306	8.0374467383767\\
71	0.24672	8.32796153639905\\
71	0.25038	8.62279638731249\\
71	0.25404	8.921951291117\\
71	0.2577	9.2254262478126\\
71	0.26136	9.5332212573993\\
71	0.26502	9.84533631987706\\
71	0.26868	10.1617714352459\\
71	0.27234	10.4825266035058\\
71	0.276	10.8076018246569\\
71.375	0.093	-0.189740190575043\\
71.375	0.09666	-0.0723993708785129\\
71.375	0.10032	0.0492615017091045\\
71.375	0.10398	0.175242427187804\\
71.375	0.10764	0.305543405557584\\
71.375	0.1113	0.440164436818451\\
71.375	0.11496	0.579105520970404\\
71.375	0.11862	0.722366658013437\\
71.375	0.12228	0.869947847947549\\
71.375	0.12594	1.02184909077275\\
71.375	0.1296	1.17807038648903\\
71.375	0.13326	1.33861173509639\\
71.375	0.13692	1.50347313659484\\
71.375	0.14058	1.67265459098438\\
71.375	0.14424	1.84615609826499\\
71.375	0.1479	2.02397765843668\\
71.375	0.15156	2.20611927149946\\
71.375	0.15522	2.39258093745333\\
71.375	0.15888	2.58336265629827\\
71.375	0.16254	2.77846442803431\\
71.375	0.1662	2.97788625266142\\
71.375	0.16986	3.18162813017961\\
71.375	0.17352	3.3896900605889\\
71.375	0.17718	3.60207204388925\\
71.375	0.18084	3.8187740800807\\
71.375	0.1845	4.03979616916323\\
71.375	0.18816	4.26513831113684\\
71.375	0.19182	4.49480050600154\\
71.375	0.19548	4.72878275375732\\
71.375	0.19914	4.96708505440418\\
71.375	0.2028	5.20970740794212\\
71.375	0.20646	5.45664981437115\\
71.375	0.21012	5.70791227369126\\
71.375	0.21378	5.96349478590245\\
71.375	0.21744	6.22339735100473\\
71.375	0.2211	6.48761996899809\\
71.375	0.22476	6.75616263988253\\
71.375	0.22842	7.02902536365805\\
71.375	0.23208	7.30620814032467\\
71.375	0.23574	7.58771096988236\\
71.375	0.2394	7.87353385233113\\
71.375	0.24306	8.163676787671\\
71.375	0.24672	8.45813977590193\\
71.375	0.25038	8.75692281702397\\
71.375	0.25404	9.06002591103707\\
71.375	0.2577	9.36744905794126\\
71.375	0.26136	9.67919225773653\\
71.375	0.26502	9.99525551042289\\
71.375	0.26868	10.3156388160003\\
71.375	0.27234	10.6403421744689\\
71.375	0.276	10.9693655858285\\
71.75	0.093	-0.226431146315458\\
71.75	0.09666	-0.105142136410335\\
71.75	0.10032	0.0204669263858723\\
71.75	0.10398	0.150396042073162\\
71.75	0.10764	0.284645210651531\\
71.75	0.1113	0.423214432120988\\
71.75	0.11496	0.566103706481527\\
71.75	0.11862	0.713313033733153\\
71.75	0.12228	0.864842413875855\\
71.75	0.12594	1.02069184690964\\
71.75	0.1296	1.18086133283451\\
71.75	0.13326	1.34535087165047\\
71.75	0.13692	1.51416046335751\\
71.75	0.14058	1.68729010795563\\
71.75	0.14424	1.86473980544483\\
71.75	0.1479	2.04650955582512\\
71.75	0.15156	2.23259935909649\\
71.75	0.15522	2.42300921525894\\
71.75	0.15888	2.61773912431248\\
71.75	0.16254	2.8167890862571\\
71.75	0.1662	3.0201591010928\\
71.75	0.16986	3.22784916881958\\
71.75	0.17352	3.43985928943745\\
71.75	0.17718	3.6561894629464\\
71.75	0.18084	3.87683968934644\\
71.75	0.1845	4.10180996863756\\
71.75	0.18816	4.33110030081976\\
71.75	0.19182	4.56471068589305\\
71.75	0.19548	4.80264112385741\\
71.75	0.19914	5.04489161471286\\
71.75	0.2028	5.2914621584594\\
71.75	0.20646	5.54235275509702\\
71.75	0.21012	5.79756340462572\\
71.75	0.21378	6.0570941070455\\
71.75	0.21744	6.32094486235636\\
71.75	0.2211	6.58911567055832\\
71.75	0.22476	6.86160653165135\\
71.75	0.22842	7.13841744563546\\
71.75	0.23208	7.41954841251066\\
71.75	0.23574	7.70499943227694\\
71.75	0.2394	7.9947705049343\\
71.75	0.24306	8.28886163048275\\
71.75	0.24672	8.58727280892228\\
71.75	0.25038	8.8900040402529\\
71.75	0.25404	9.1970553244746\\
71.75	0.2577	9.50842666158738\\
71.75	0.26136	9.82411805159124\\
71.75	0.26502	10.1441294944862\\
71.75	0.26868	10.4684609902722\\
71.75	0.27234	10.7971125389493\\
71.75	0.276	11.1300841405175\\
72.125	0.093	-0.264167308538396\\
72.125	0.09666	-0.138930108424687\\
72.125	0.10032	-0.00937285541988997\\
72.125	0.10398	0.124504450475989\\
72.125	0.10764	0.262701809262952\\
72.125	0.1113	0.405219220940998\\
72.125	0.11496	0.552056685510127\\
72.125	0.11862	0.703214202970338\\
72.125	0.12228	0.85869177332163\\
72.125	0.12594	1.01848939656401\\
72.125	0.1296	1.18260707269747\\
72.125	0.13326	1.35104480172201\\
72.125	0.13692	1.52380258363764\\
72.125	0.14058	1.70088041844435\\
72.125	0.14424	1.88227830614215\\
72.125	0.1479	2.06799624673102\\
72.125	0.15156	2.25803424021098\\
72.125	0.15522	2.45239228658203\\
72.125	0.15888	2.65107038584415\\
72.125	0.16254	2.85406853799736\\
72.125	0.1662	3.06138674304166\\
72.125	0.16986	3.27302500097702\\
72.125	0.17352	3.48898331180349\\
72.125	0.17718	3.70926167552103\\
72.125	0.18084	3.93386009212965\\
72.125	0.1845	4.16277856162936\\
72.125	0.18816	4.39601708402015\\
72.125	0.19182	4.63357565930203\\
72.125	0.19548	4.87545428747499\\
72.125	0.19914	5.12165296853902\\
72.125	0.2028	5.37217170249415\\
72.125	0.20646	5.62701048934035\\
72.125	0.21012	5.88616932907765\\
72.125	0.21378	6.14964822170602\\
72.125	0.21744	6.41744716722547\\
72.125	0.2211	6.68956616563602\\
72.125	0.22476	6.96600521693763\\
72.125	0.22842	7.24676432113034\\
72.125	0.23208	7.53184347821413\\
72.125	0.23574	7.821242688189\\
72.125	0.2394	8.11496195105495\\
72.125	0.24306	8.41300126681199\\
72.125	0.24672	8.71536063546011\\
72.125	0.25038	9.02204005699932\\
72.125	0.25404	9.33303953142961\\
72.125	0.2577	9.64835905875098\\
72.125	0.26136	9.96799863896343\\
72.125	0.26502	10.291958272067\\
72.125	0.26868	10.6202379580616\\
72.125	0.27234	10.9528376969473\\
72.125	0.276	11.2897574887241\\
72.5	0.093	-0.302948677243864\\
72.5	0.09666	-0.173763286921565\\
72.5	0.10032	-0.0402578437081789\\
72.5	0.10398	0.0975676523962898\\
72.5	0.10764	0.239713201391839\\
72.5	0.1113	0.386178803278474\\
72.5	0.11496	0.536964458056196\\
72.5	0.11862	0.692070165724997\\
72.5	0.12228	0.851495926284879\\
72.5	0.12594	1.01524173973585\\
72.5	0.1296	1.1833076060779\\
72.5	0.13326	1.35569352531103\\
72.5	0.13692	1.53239949743525\\
72.5	0.14058	1.71342552245055\\
72.5	0.14424	1.89877160035693\\
72.5	0.1479	2.0884377311544\\
72.5	0.15156	2.28242391484295\\
72.5	0.15522	2.48073015142258\\
72.5	0.15888	2.68335644089329\\
72.5	0.16254	2.8903027832551\\
72.5	0.1662	3.10156917850798\\
72.5	0.16986	3.31715562665194\\
72.5	0.17352	3.53706212768699\\
72.5	0.17718	3.76128868161312\\
72.5	0.18084	3.98983528843033\\
72.5	0.1845	4.22270194813864\\
72.5	0.18816	4.45988866073801\\
72.5	0.19182	4.70139542622848\\
72.5	0.19548	4.94722224461002\\
72.5	0.19914	5.19736911588265\\
72.5	0.2028	5.45183604004637\\
72.5	0.20646	5.71062301710116\\
72.5	0.21012	5.97373004704705\\
72.5	0.21378	6.241157129884\\
72.5	0.21744	6.51290426561205\\
72.5	0.2211	6.78897145423119\\
72.5	0.22476	7.06935869574139\\
72.5	0.22842	7.35406599014268\\
72.5	0.23208	7.64309333743506\\
72.5	0.23574	7.93644073761853\\
72.5	0.2394	8.23410819069307\\
72.5	0.24306	8.5360956966587\\
72.5	0.24672	8.84240325551541\\
72.5	0.25038	9.15303086726321\\
72.5	0.25404	9.46797853190208\\
72.5	0.2577	9.78724624943204\\
72.5	0.26136	10.1108340198531\\
72.5	0.26502	10.4387418431652\\
72.5	0.26868	10.7709697193684\\
72.5	0.27234	11.1075176484627\\
72.5	0.276	11.4483856304481\\
72.875	0.093	-0.342775252431855\\
72.875	0.09666	-0.209641671900967\\
72.875	0.10032	-0.0721880384789908\\
72.875	0.10398	0.0695856478340711\\
72.875	0.10764	0.215679387038209\\
72.875	0.1113	0.366093179133435\\
72.875	0.11496	0.520827024119743\\
72.875	0.11862	0.679880921997134\\
72.875	0.12228	0.843254872765608\\
72.875	0.12594	1.01094887642517\\
72.875	0.1296	1.18296293297581\\
72.875	0.13326	1.35929704241753\\
72.875	0.13692	1.53995120475034\\
72.875	0.14058	1.72492541997423\\
72.875	0.14424	1.9142196880892\\
72.875	0.1479	2.10783400909525\\
72.875	0.15156	2.30576838299239\\
72.875	0.15522	2.50802280978061\\
72.875	0.15888	2.71459728945992\\
72.875	0.16254	2.92549182203032\\
72.875	0.1662	3.14070640749178\\
72.875	0.16986	3.36024104584433\\
72.875	0.17352	3.58409573708797\\
72.875	0.17718	3.81227048122269\\
72.875	0.18084	4.0447652782485\\
72.875	0.1845	4.28158012816538\\
72.875	0.18816	4.52271503097336\\
72.875	0.19182	4.76816998667241\\
72.875	0.19548	5.01794499526255\\
72.875	0.19914	5.27204005674377\\
72.875	0.2028	5.53045517111607\\
72.875	0.20646	5.79319033837946\\
72.875	0.21012	6.06024555853392\\
72.875	0.21378	6.33162083157948\\
72.875	0.21744	6.60731615751611\\
72.875	0.2211	6.88733153634383\\
72.875	0.22476	7.17166696806263\\
72.875	0.22842	7.46032245267252\\
72.875	0.23208	7.75329799017348\\
72.875	0.23574	8.05059358056554\\
72.875	0.2394	8.35220922384866\\
72.875	0.24306	8.65814492002289\\
72.875	0.24672	8.96840066908819\\
72.875	0.25038	9.28297647104457\\
72.875	0.25404	9.60187232589204\\
72.875	0.2577	9.92508823363058\\
72.875	0.26136	10.2526241942602\\
72.875	0.26502	10.5844802077809\\
72.875	0.26868	10.9206562741927\\
72.875	0.27234	11.2611523934956\\
72.875	0.276	11.6059685656896\\
73.25	0.093	-0.383647034102384\\
73.25	0.09666	-0.246565263362905\\
73.25	0.10032	-0.10516343973234\\
73.25	0.10398	0.0405584367893079\\
73.25	0.10764	0.190600366202039\\
73.25	0.1113	0.344962348505854\\
73.25	0.11496	0.503644383700752\\
73.25	0.11862	0.666646471786732\\
73.25	0.12228	0.833968612763793\\
73.25	0.12594	1.00561080663194\\
73.25	0.1296	1.18157305339117\\
73.25	0.13326	1.36185535304148\\
73.25	0.13692	1.54645770558288\\
73.25	0.14058	1.73538011101536\\
73.25	0.14424	1.92862256933893\\
73.25	0.1479	2.12618508055357\\
73.25	0.15156	2.3280676446593\\
73.25	0.15522	2.53427026165611\\
73.25	0.15888	2.744792931544\\
73.25	0.16254	2.95963565432298\\
73.25	0.1662	3.17879842999304\\
73.25	0.16986	3.40228125855419\\
73.25	0.17352	3.63008414000642\\
73.25	0.17718	3.86220707434973\\
73.25	0.18084	4.09865006158412\\
73.25	0.1845	4.3394131017096\\
73.25	0.18816	4.58449619472616\\
73.25	0.19182	4.8338993406338\\
73.25	0.19548	5.08762253943253\\
73.25	0.19914	5.34566579112234\\
73.25	0.2028	5.60802909570323\\
73.25	0.20646	5.87471245317521\\
73.25	0.21012	6.14571586353826\\
73.25	0.21378	6.42103932679241\\
73.25	0.21744	6.70068284293763\\
73.25	0.2211	6.98464641197394\\
73.25	0.22476	7.27293003390133\\
73.25	0.22842	7.5655337087198\\
73.25	0.23208	7.86245743642936\\
73.25	0.23574	8.16370121703\\
73.25	0.2394	8.46926505052172\\
73.25	0.24306	8.77914893690453\\
73.25	0.24672	9.09335287617842\\
73.25	0.25038	9.4118768683434\\
73.25	0.25404	9.73472091339945\\
73.25	0.2577	10.0618850113466\\
73.25	0.26136	10.3933691621848\\
73.25	0.26502	10.7291733659141\\
73.25	0.26868	11.0692976225345\\
73.25	0.27234	11.413741932046\\
73.25	0.276	11.7625062944485\\
73.625	0.093	-0.425564022255435\\
73.625	0.09666	-0.284534061307367\\
73.625	0.10032	-0.139184047468212\\
73.625	0.10398	0.0104860192620289\\
73.625	0.10764	0.164476138883346\\
73.625	0.1113	0.322786311395751\\
73.625	0.11496	0.485416536799242\\
73.625	0.11862	0.652366815093812\\
73.625	0.12228	0.823637146279462\\
73.625	0.12594	0.999227530356198\\
73.625	0.1296	1.17913796732402\\
73.625	0.13326	1.36336845718292\\
73.625	0.13692	1.55191899993291\\
73.625	0.14058	1.74478959557398\\
73.625	0.14424	1.94198024410613\\
73.625	0.1479	2.14349094552936\\
73.625	0.15156	2.34932169984368\\
73.625	0.15522	2.55947250704909\\
73.625	0.15888	2.77394336714557\\
73.625	0.16254	2.99273428013314\\
73.625	0.1662	3.21584524601179\\
73.625	0.16986	3.44327626478152\\
73.625	0.17352	3.67502733644234\\
73.625	0.17718	3.91109846099424\\
73.625	0.18084	4.15148963843722\\
73.625	0.1845	4.39620086877129\\
73.625	0.18816	4.64523215199644\\
73.625	0.19182	4.89858348811268\\
73.625	0.19548	5.15625487711999\\
73.625	0.19914	5.41824631901838\\
73.625	0.2028	5.68455781380787\\
73.625	0.20646	5.95518936148843\\
73.625	0.21012	6.23014096206009\\
73.625	0.21378	6.50941261552282\\
73.625	0.21744	6.79300432187663\\
73.625	0.2211	7.08091608112153\\
73.625	0.22476	7.37314789325751\\
73.625	0.22842	7.66969975828457\\
73.625	0.23208	7.97057167620272\\
73.625	0.23574	8.27576364701195\\
73.625	0.2394	8.58527567071226\\
73.625	0.24306	8.89910774730366\\
73.625	0.24672	9.21725987678614\\
73.625	0.25038	9.53973205915971\\
73.625	0.25404	9.86652429442436\\
73.625	0.2577	10.1976365825801\\
73.625	0.26136	10.5330689236269\\
73.625	0.26502	10.8728213175648\\
73.625	0.26868	11.2168937643938\\
73.625	0.27234	11.5652862641138\\
73.625	0.276	11.917998816725\\
74	0.093	-0.468526216891017\\
74	0.09666	-0.323548065734359\\
74	0.10032	-0.174249861686615\\
74	0.10398	-0.0206316047477875\\
74	0.10764	0.137306705082123\\
74	0.1113	0.299565067803117\\
74	0.11496	0.466143483415194\\
74	0.11862	0.637041951918357\\
74	0.12228	0.812260473312597\\
74	0.12594	0.991799047597923\\
74	0.1296	1.17565767477433\\
74	0.13326	1.36383635484182\\
74	0.13692	1.5563350878004\\
74	0.14058	1.75315387365006\\
74	0.14424	1.9542927123908\\
74	0.1479	2.15975160402263\\
74	0.15156	2.36953054854553\\
74	0.15522	2.58362954595953\\
74	0.15888	2.8020485962646\\
74	0.16254	3.02478769946076\\
74	0.1662	3.251846855548\\
74	0.16986	3.48322606452632\\
74	0.17352	3.71892532639573\\
74	0.17718	3.95894464115622\\
74	0.18084	4.20328400880779\\
74	0.1845	4.45194342935045\\
74	0.18816	4.70492290278418\\
74	0.19182	4.96222242910901\\
74	0.19548	5.22384200832491\\
74	0.19914	5.4897816404319\\
74	0.2028	5.76004132542998\\
74	0.20646	6.03462106331913\\
74	0.21012	6.31352085409937\\
74	0.21378	6.59674069777069\\
74	0.21744	6.88428059433309\\
74	0.2211	7.17614054378658\\
74	0.22476	7.47232054613116\\
74	0.22842	7.7728206013668\\
74	0.23208	8.07764070949355\\
74	0.23574	8.38678087051136\\
74	0.2394	8.70024108442026\\
74	0.24306	9.01802135122026\\
74	0.24672	9.34012167091132\\
74	0.25038	9.66654204349348\\
74	0.25404	9.99728246896671\\
74	0.2577	10.332342947331\\
74	0.26136	10.6717234785864\\
74	0.26502	11.0154240627329\\
74	0.26868	11.3634446997705\\
74	0.27234	11.7157853896991\\
74	0.276	12.0724461325189\\
};
\end{axis}

\begin{axis}[%
width=4.527496cm,
height=3.870968cm,
at={(0cm,16.129032cm)},
scale only axis,
xmin=56,
xmax=74,
tick align=outside,
xlabel={$L_{cut}$},
xmajorgrids,
ymin=0.093,
ymax=0.276,
ylabel={$D_{rlx}$},
ymajorgrids,
zmin=-200.569447084128,
zmax=0,
zlabel={$x_4,x_4$},
zmajorgrids,
view={-140}{50},
legend style={at={(1.03,1)},anchor=north west,legend cell align=left,align=left,draw=white!15!black}
]
\addplot3[only marks,mark=*,mark options={},mark size=1.5000pt,color=mycolor1] plot table[row sep=crcr,]{%
74	0.123	-26.0353957891804\\
72	0.113	-20.9879322169279\\
61	0.095	-10.6920630070547\\
56	0.093	-10.9569379219045\\
};
\addplot3[only marks,mark=*,mark options={},mark size=1.5000pt,color=mycolor2] plot table[row sep=crcr,]{%
67	0.276	-191.779551108501\\
66	0.255	-157.643535964989\\
62	0.209	-87.4196213968237\\
57	0.193	-69.8013569503948\\
};
\addplot3[only marks,mark=*,mark options={},mark size=1.5000pt,color=black] plot table[row sep=crcr,]{%
69	0.104	-15.5770582620907\\
};
\addplot3[only marks,mark=*,mark options={},mark size=1.5000pt,color=black] plot table[row sep=crcr,]{%
64	0.23	-116.694090391038\\
};

\addplot3[%
surf,
opacity=0.7,
shader=interp,
colormap={mymap}{[1pt] rgb(0pt)=(0.0901961,0.239216,0.0745098); rgb(1pt)=(0.0945149,0.242058,0.0739522); rgb(2pt)=(0.0988592,0.244894,0.0733566); rgb(3pt)=(0.103229,0.247724,0.0727241); rgb(4pt)=(0.107623,0.250549,0.0720557); rgb(5pt)=(0.112043,0.253367,0.0713525); rgb(6pt)=(0.116487,0.25618,0.0706154); rgb(7pt)=(0.120956,0.258986,0.0698456); rgb(8pt)=(0.125449,0.261787,0.0690441); rgb(9pt)=(0.129967,0.264581,0.0682118); rgb(10pt)=(0.134508,0.26737,0.06735); rgb(11pt)=(0.139074,0.270152,0.0664596); rgb(12pt)=(0.143663,0.272929,0.0655416); rgb(13pt)=(0.148275,0.275699,0.0645971); rgb(14pt)=(0.152911,0.278463,0.0636271); rgb(15pt)=(0.15757,0.281221,0.0626328); rgb(16pt)=(0.162252,0.283973,0.0616151); rgb(17pt)=(0.166957,0.286719,0.060575); rgb(18pt)=(0.171685,0.289458,0.0595136); rgb(19pt)=(0.176434,0.292191,0.0584321); rgb(20pt)=(0.181207,0.294918,0.0573313); rgb(21pt)=(0.186001,0.297639,0.0562123); rgb(22pt)=(0.190817,0.300353,0.0550763); rgb(23pt)=(0.195655,0.303061,0.0539242); rgb(24pt)=(0.200514,0.305763,0.052757); rgb(25pt)=(0.205395,0.308459,0.0515759); rgb(26pt)=(0.210296,0.311149,0.0503624); rgb(27pt)=(0.215212,0.313846,0.0490067); rgb(28pt)=(0.220142,0.316548,0.0475043); rgb(29pt)=(0.22509,0.319254,0.0458704); rgb(30pt)=(0.230056,0.321962,0.0441205); rgb(31pt)=(0.235042,0.324671,0.04227); rgb(32pt)=(0.240048,0.327379,0.0403343); rgb(33pt)=(0.245078,0.330085,0.0383287); rgb(34pt)=(0.250131,0.332786,0.0362688); rgb(35pt)=(0.25521,0.335482,0.0341698); rgb(36pt)=(0.260317,0.33817,0.0320472); rgb(37pt)=(0.265451,0.340849,0.0299163); rgb(38pt)=(0.270616,0.343517,0.0277927); rgb(39pt)=(0.275813,0.346172,0.0256916); rgb(40pt)=(0.281043,0.348814,0.0236284); rgb(41pt)=(0.286307,0.35144,0.0216186); rgb(42pt)=(0.291607,0.354048,0.0196776); rgb(43pt)=(0.296945,0.356637,0.0178207); rgb(44pt)=(0.302322,0.359206,0.0160634); rgb(45pt)=(0.307739,0.361753,0.0144211); rgb(46pt)=(0.313198,0.364275,0.0129091); rgb(47pt)=(0.318701,0.366772,0.0115428); rgb(48pt)=(0.324249,0.369242,0.0103377); rgb(49pt)=(0.329843,0.371682,0.00930909); rgb(50pt)=(0.335485,0.374093,0.00847245); rgb(51pt)=(0.341176,0.376471,0.00784314); rgb(52pt)=(0.346925,0.378826,0.00732741); rgb(53pt)=(0.352735,0.381168,0.00682184); rgb(54pt)=(0.358605,0.383497,0.00632729); rgb(55pt)=(0.364532,0.385812,0.00584464); rgb(56pt)=(0.370516,0.388113,0.00537476); rgb(57pt)=(0.376552,0.390399,0.00491852); rgb(58pt)=(0.38264,0.39267,0.00447681); rgb(59pt)=(0.388777,0.394925,0.00405048); rgb(60pt)=(0.394962,0.397164,0.00364042); rgb(61pt)=(0.401191,0.399386,0.00324749); rgb(62pt)=(0.407464,0.401592,0.00287258); rgb(63pt)=(0.413777,0.40378,0.00251655); rgb(64pt)=(0.420129,0.40595,0.00218028); rgb(65pt)=(0.426518,0.408102,0.00186463); rgb(66pt)=(0.432942,0.410234,0.00157049); rgb(67pt)=(0.439399,0.412348,0.00129873); rgb(68pt)=(0.445885,0.414441,0.00105022); rgb(69pt)=(0.452401,0.416515,0.000825833); rgb(70pt)=(0.458942,0.418567,0.000626441); rgb(71pt)=(0.465508,0.420599,0.00045292); rgb(72pt)=(0.472096,0.422609,0.000306141); rgb(73pt)=(0.478704,0.424596,0.000186979); rgb(74pt)=(0.485331,0.426562,9.63073e-05); rgb(75pt)=(0.491973,0.428504,3.49981e-05); rgb(76pt)=(0.498628,0.430422,3.92506e-06); rgb(77pt)=(0.505323,0.432315,0); rgb(78pt)=(0.512206,0.434168,0); rgb(79pt)=(0.519282,0.435983,0); rgb(80pt)=(0.526529,0.437764,0); rgb(81pt)=(0.533922,0.439512,0); rgb(82pt)=(0.54144,0.441232,0); rgb(83pt)=(0.549059,0.442927,0); rgb(84pt)=(0.556756,0.444599,0); rgb(85pt)=(0.564508,0.446252,0); rgb(86pt)=(0.572292,0.447889,0); rgb(87pt)=(0.580084,0.449514,0); rgb(88pt)=(0.587863,0.451129,0); rgb(89pt)=(0.595604,0.452737,0); rgb(90pt)=(0.603284,0.454343,0); rgb(91pt)=(0.610882,0.455948,0); rgb(92pt)=(0.618373,0.457556,0); rgb(93pt)=(0.625734,0.459171,0); rgb(94pt)=(0.632943,0.460795,0); rgb(95pt)=(0.639976,0.462432,0); rgb(96pt)=(0.64681,0.464084,0); rgb(97pt)=(0.653423,0.465756,0); rgb(98pt)=(0.659791,0.46745,0); rgb(99pt)=(0.665891,0.469169,0); rgb(100pt)=(0.6717,0.470916,0); rgb(101pt)=(0.677195,0.472696,0); rgb(102pt)=(0.682353,0.47451,0); rgb(103pt)=(0.687242,0.476355,0); rgb(104pt)=(0.691952,0.478225,0); rgb(105pt)=(0.696497,0.480118,0); rgb(106pt)=(0.700887,0.482033,0); rgb(107pt)=(0.705134,0.483968,0); rgb(108pt)=(0.709251,0.485921,0); rgb(109pt)=(0.713249,0.487891,0); rgb(110pt)=(0.71714,0.489876,0); rgb(111pt)=(0.720936,0.491875,0); rgb(112pt)=(0.724649,0.493887,0); rgb(113pt)=(0.72829,0.495909,0); rgb(114pt)=(0.731872,0.49794,0); rgb(115pt)=(0.735406,0.499979,0); rgb(116pt)=(0.738904,0.502025,0); rgb(117pt)=(0.742378,0.504075,0); rgb(118pt)=(0.74584,0.506128,0); rgb(119pt)=(0.749302,0.508182,0); rgb(120pt)=(0.752775,0.510237,0); rgb(121pt)=(0.756272,0.51229,0); rgb(122pt)=(0.759804,0.514339,0); rgb(123pt)=(0.763384,0.516385,0); rgb(124pt)=(0.767022,0.518424,0); rgb(125pt)=(0.770731,0.520455,0); rgb(126pt)=(0.774523,0.522478,0); rgb(127pt)=(0.77841,0.524489,0); rgb(128pt)=(0.782391,0.526491,0); rgb(129pt)=(0.786402,0.528496,0); rgb(130pt)=(0.790431,0.530506,0); rgb(131pt)=(0.794478,0.532521,0); rgb(132pt)=(0.798541,0.534539,0); rgb(133pt)=(0.802619,0.53656,0); rgb(134pt)=(0.806712,0.538584,0); rgb(135pt)=(0.81082,0.540609,0); rgb(136pt)=(0.81494,0.542635,0); rgb(137pt)=(0.819074,0.54466,0); rgb(138pt)=(0.823219,0.546686,0); rgb(139pt)=(0.827374,0.548709,0); rgb(140pt)=(0.831541,0.55073,0); rgb(141pt)=(0.835716,0.552749,0); rgb(142pt)=(0.8399,0.554763,0); rgb(143pt)=(0.844092,0.556774,0); rgb(144pt)=(0.848292,0.558779,0); rgb(145pt)=(0.852497,0.560778,0); rgb(146pt)=(0.856708,0.562771,0); rgb(147pt)=(0.860924,0.564756,0); rgb(148pt)=(0.865143,0.566733,0); rgb(149pt)=(0.869366,0.568701,0); rgb(150pt)=(0.873592,0.57066,0); rgb(151pt)=(0.877819,0.572608,0); rgb(152pt)=(0.882047,0.574545,0); rgb(153pt)=(0.886275,0.576471,0); rgb(154pt)=(0.890659,0.578362,0); rgb(155pt)=(0.895333,0.580203,0); rgb(156pt)=(0.900258,0.581999,0); rgb(157pt)=(0.905397,0.583755,0); rgb(158pt)=(0.910711,0.585479,0); rgb(159pt)=(0.916164,0.587176,0); rgb(160pt)=(0.921717,0.588852,0); rgb(161pt)=(0.927333,0.590513,0); rgb(162pt)=(0.932974,0.592166,0); rgb(163pt)=(0.938602,0.593815,0); rgb(164pt)=(0.94418,0.595468,0); rgb(165pt)=(0.949669,0.59713,0); rgb(166pt)=(0.955033,0.598808,0); rgb(167pt)=(0.960233,0.600507,0); rgb(168pt)=(0.965232,0.602233,0); rgb(169pt)=(0.969992,0.603992,0); rgb(170pt)=(0.974475,0.605791,0); rgb(171pt)=(0.978643,0.607636,0); rgb(172pt)=(0.98246,0.609532,0); rgb(173pt)=(0.985886,0.611486,0); rgb(174pt)=(0.988885,0.613503,0); rgb(175pt)=(0.991419,0.61559,0); rgb(176pt)=(0.99345,0.617753,0); rgb(177pt)=(0.99494,0.619997,0); rgb(178pt)=(0.995851,0.622329,0); rgb(179pt)=(0.996226,0.624763,0); rgb(180pt)=(0.996512,0.627352,0); rgb(181pt)=(0.996788,0.630095,0); rgb(182pt)=(0.997053,0.632982,0); rgb(183pt)=(0.997308,0.636004,0); rgb(184pt)=(0.997552,0.639152,0); rgb(185pt)=(0.997785,0.642416,0); rgb(186pt)=(0.998006,0.645786,0); rgb(187pt)=(0.998217,0.649253,0); rgb(188pt)=(0.998416,0.652807,0); rgb(189pt)=(0.998605,0.656439,0); rgb(190pt)=(0.998781,0.660138,0); rgb(191pt)=(0.998946,0.663897,0); rgb(192pt)=(0.9991,0.667704,0); rgb(193pt)=(0.999242,0.67155,0); rgb(194pt)=(0.999372,0.675427,0); rgb(195pt)=(0.99949,0.679323,0); rgb(196pt)=(0.999596,0.68323,0); rgb(197pt)=(0.99969,0.687139,0); rgb(198pt)=(0.999771,0.691039,0); rgb(199pt)=(0.999841,0.694921,0); rgb(200pt)=(0.999898,0.698775,0); rgb(201pt)=(0.999942,0.702592,0); rgb(202pt)=(0.999974,0.706363,0); rgb(203pt)=(0.999994,0.710077,0); rgb(204pt)=(1,0.713725,0); rgb(205pt)=(1,0.717341,0); rgb(206pt)=(1,0.720963,0); rgb(207pt)=(1,0.724591,0); rgb(208pt)=(1,0.728226,0); rgb(209pt)=(1,0.731867,0); rgb(210pt)=(1,0.735514,0); rgb(211pt)=(1,0.739167,0); rgb(212pt)=(1,0.742827,0); rgb(213pt)=(1,0.746493,0); rgb(214pt)=(1,0.750165,0); rgb(215pt)=(1,0.753843,0); rgb(216pt)=(1,0.757527,0); rgb(217pt)=(1,0.761217,0); rgb(218pt)=(1,0.764913,0); rgb(219pt)=(1,0.768615,0); rgb(220pt)=(1,0.772324,0); rgb(221pt)=(1,0.776038,0); rgb(222pt)=(1,0.779758,0); rgb(223pt)=(1,0.783484,0); rgb(224pt)=(1,0.787215,0); rgb(225pt)=(1,0.790953,0); rgb(226pt)=(1,0.794696,0); rgb(227pt)=(1,0.798445,0); rgb(228pt)=(1,0.8022,0); rgb(229pt)=(1,0.805961,0); rgb(230pt)=(1,0.809727,0); rgb(231pt)=(1,0.8135,0); rgb(232pt)=(1,0.817278,0); rgb(233pt)=(1,0.821063,0); rgb(234pt)=(1,0.824854,0); rgb(235pt)=(1,0.828652,0); rgb(236pt)=(1,0.832455,0); rgb(237pt)=(1,0.836265,0); rgb(238pt)=(1,0.840081,0); rgb(239pt)=(1,0.843903,0); rgb(240pt)=(1,0.847732,0); rgb(241pt)=(1,0.851566,0); rgb(242pt)=(1,0.855406,0); rgb(243pt)=(1,0.859253,0); rgb(244pt)=(1,0.863106,0); rgb(245pt)=(1,0.866964,0); rgb(246pt)=(1,0.870829,0); rgb(247pt)=(1,0.8747,0); rgb(248pt)=(1,0.878577,0); rgb(249pt)=(1,0.88246,0); rgb(250pt)=(1,0.886349,0); rgb(251pt)=(1,0.890243,0); rgb(252pt)=(1,0.894144,0); rgb(253pt)=(1,0.898051,0); rgb(254pt)=(1,0.901964,0); rgb(255pt)=(1,0.905882,0)},
mesh/rows=49]
table[row sep=crcr,header=false] {%
%
56	0.093	-10.9049618708685\\
56	0.09666	-11.4423373147976\\
56	0.10032	-12.1019444097205\\
56	0.10398	-12.8837831556373\\
56	0.10764	-13.7878535525481\\
56	0.1113	-14.8141556004528\\
56	0.11496	-15.9626892993514\\
56	0.11862	-17.2334546492439\\
56	0.12228	-18.6264516501303\\
56	0.12594	-20.1416803020107\\
56	0.1296	-21.779140604885\\
56	0.13326	-23.5388325587532\\
56	0.13692	-25.4207561636153\\
56	0.14058	-27.4249114194713\\
56	0.14424	-29.5512983263213\\
56	0.1479	-31.7999168841651\\
56	0.15156	-34.1707670930029\\
56	0.15522	-36.6638489528346\\
56	0.15888	-39.2791624636602\\
56	0.16254	-42.0167076254798\\
56	0.1662	-44.8764844382932\\
56	0.16986	-47.8584929021006\\
56	0.17352	-50.9627330169019\\
56	0.17718	-54.189204782697\\
56	0.18084	-57.5379081994862\\
56	0.1845	-61.0088432672693\\
56	0.18816	-64.6020099860462\\
56	0.19182	-68.3174083558171\\
56	0.19548	-72.1550383765819\\
56	0.19914	-76.1149000483406\\
56	0.2028	-80.1969933710933\\
56	0.20646	-84.4013183448399\\
56	0.21012	-88.7278749695803\\
56	0.21378	-93.1766632453147\\
56	0.21744	-97.747683172043\\
56	0.2211	-102.440934749765\\
56	0.22476	-107.256417978481\\
56	0.22842	-112.194132858191\\
56	0.23208	-117.254079388895\\
56	0.23574	-122.436257570593\\
56	0.2394	-127.740667403285\\
56	0.24306	-133.167308886971\\
56	0.24672	-138.716182021651\\
56	0.25038	-144.387286807324\\
56	0.25404	-150.180623243992\\
56	0.2577	-156.096191331653\\
56	0.26136	-162.133991070308\\
56	0.26502	-168.294022459957\\
56	0.26868	-174.576285500601\\
56	0.27234	-180.980780192238\\
56	0.276	-187.507506534869\\
56.375	0.093	-10.8126291856249\\
56.375	0.09666	-11.3527834706413\\
56.375	0.10032	-12.0151694066517\\
56.375	0.10398	-12.7997869936559\\
56.375	0.10764	-13.7066362316541\\
56.375	0.1113	-14.7357171206462\\
56.375	0.11496	-15.8870296606322\\
56.375	0.11862	-17.1605738516121\\
56.375	0.12228	-18.556349693586\\
56.375	0.12594	-20.0743571865538\\
56.375	0.1296	-21.7145963305155\\
56.375	0.13326	-23.4770671254711\\
56.375	0.13692	-25.3617695714206\\
56.375	0.14058	-27.3687036683641\\
56.375	0.14424	-29.4978694163015\\
56.375	0.1479	-31.7492668152327\\
56.375	0.15156	-34.1228958651579\\
56.375	0.15522	-36.6187565660771\\
56.375	0.15888	-39.2368489179901\\
56.375	0.16254	-41.9771729208971\\
56.375	0.1662	-44.839728574798\\
56.375	0.16986	-47.8245158796927\\
56.375	0.17352	-50.9315348355815\\
56.375	0.17718	-54.1607854424641\\
56.375	0.18084	-57.5122677003406\\
56.375	0.1845	-60.9859816092111\\
56.375	0.18816	-64.5819271690755\\
56.375	0.19182	-68.3001043799338\\
56.375	0.19548	-72.140513241786\\
56.375	0.19914	-76.1031537546321\\
56.375	0.2028	-80.1880259184722\\
56.375	0.20646	-84.3951297333062\\
56.375	0.21012	-88.7244651991341\\
56.375	0.21378	-93.1760323159559\\
56.375	0.21744	-97.7498310837716\\
56.375	0.2211	-102.445861502581\\
56.375	0.22476	-107.264123572385\\
56.375	0.22842	-112.204617293182\\
56.375	0.23208	-117.267342664974\\
56.375	0.23574	-122.452299687759\\
56.375	0.2394	-127.759488361538\\
56.375	0.24306	-133.188908686311\\
56.375	0.24672	-138.740560662079\\
56.375	0.25038	-144.41444428884\\
56.375	0.25404	-150.210559566594\\
56.375	0.2577	-156.128906495343\\
56.375	0.26136	-162.169485075086\\
56.375	0.26502	-168.332295305823\\
56.375	0.26868	-174.617337187553\\
56.375	0.27234	-181.024610720278\\
56.375	0.276	-187.554115903996\\
56.75	0.093	-10.729892857501\\
56.75	0.09666	-11.2728259836048\\
56.75	0.10032	-11.9379907607026\\
56.75	0.10398	-12.7253871887943\\
56.75	0.10764	-13.6350152678799\\
56.75	0.1113	-14.6668749979594\\
56.75	0.11496	-15.8209663790329\\
56.75	0.11862	-17.0972894111002\\
56.75	0.12228	-18.4958440941615\\
56.75	0.12594	-20.0166304282167\\
56.75	0.1296	-21.6596484132658\\
56.75	0.13326	-23.4248980493089\\
56.75	0.13692	-25.3123793363458\\
56.75	0.14058	-27.3220922743767\\
56.75	0.14424	-29.4540368634015\\
56.75	0.1479	-31.7082131034202\\
56.75	0.15156	-34.0846209944328\\
56.75	0.15522	-36.5832605364393\\
56.75	0.15888	-39.2041317294398\\
56.75	0.16254	-41.9472345734342\\
56.75	0.1662	-44.8125690684225\\
56.75	0.16986	-47.8001352144047\\
56.75	0.17352	-50.9099330113808\\
56.75	0.17718	-54.1419624593509\\
56.75	0.18084	-57.4962235583148\\
56.75	0.1845	-60.9727163082728\\
56.75	0.18816	-64.5714407092246\\
56.75	0.19182	-68.2923967611703\\
56.75	0.19548	-72.1355844641099\\
56.75	0.19914	-76.1010038180435\\
56.75	0.2028	-80.188654822971\\
56.75	0.20646	-84.3985374788924\\
56.75	0.21012	-88.7306517858077\\
56.75	0.21378	-93.1849977437169\\
56.75	0.21744	-97.76157535262\\
56.75	0.2211	-102.460384612517\\
56.75	0.22476	-107.281425523408\\
56.75	0.22842	-112.224698085293\\
56.75	0.23208	-117.290202298172\\
56.75	0.23574	-122.477938162045\\
56.75	0.2394	-127.787905676911\\
56.75	0.24306	-133.220104842772\\
56.75	0.24672	-138.774535659626\\
56.75	0.25038	-144.451198127475\\
56.75	0.25404	-150.250092246317\\
56.75	0.2577	-156.171218016153\\
56.75	0.26136	-162.214575436984\\
56.75	0.26502	-168.380164508808\\
56.75	0.26868	-174.667985231626\\
56.75	0.27234	-181.078037605437\\
56.75	0.276	-187.610321630243\\
57.125	0.093	-10.6567528864969\\
57.125	0.09666	-11.2024648536881\\
57.125	0.10032	-11.8704084718733\\
57.125	0.10398	-12.6605837410525\\
57.125	0.10764	-13.5729906612255\\
57.125	0.1113	-14.6076292323924\\
57.125	0.11496	-15.7644994545533\\
57.125	0.11862	-17.0436013277081\\
57.125	0.12228	-18.4449348518568\\
57.125	0.12594	-19.9685000269994\\
57.125	0.1296	-21.614296853136\\
57.125	0.13326	-23.3823253302664\\
57.125	0.13692	-25.2725854583908\\
57.125	0.14058	-27.2850772375091\\
57.125	0.14424	-29.4198006676213\\
57.125	0.1479	-31.6767557487274\\
57.125	0.15156	-34.0559424808275\\
57.125	0.15522	-36.5573608639214\\
57.125	0.15888	-39.1810108980094\\
57.125	0.16254	-41.9268925830912\\
57.125	0.1662	-44.7950059191669\\
57.125	0.16986	-47.7853509062365\\
57.125	0.17352	-50.8979275443\\
57.125	0.17718	-54.1327358333575\\
57.125	0.18084	-57.4897757734089\\
57.125	0.1845	-60.9690473644542\\
57.125	0.18816	-64.5705506064935\\
57.125	0.19182	-68.2942854995266\\
57.125	0.19548	-72.1402520435537\\
57.125	0.19914	-76.1084502385746\\
57.125	0.2028	-80.1988800845895\\
57.125	0.20646	-84.4115415815984\\
57.125	0.21012	-88.7464347296011\\
57.125	0.21378	-93.2035595285978\\
57.125	0.21744	-97.7829159785883\\
57.125	0.2211	-102.484504079573\\
57.125	0.22476	-107.308323831551\\
57.125	0.22842	-112.254375234524\\
57.125	0.23208	-117.32265828849\\
57.125	0.23574	-122.51317299345\\
57.125	0.2394	-127.825919349404\\
57.125	0.24306	-133.260897356352\\
57.125	0.24672	-138.818107014294\\
57.125	0.25038	-144.49754832323\\
57.125	0.25404	-150.29922128316\\
57.125	0.2577	-156.223125894083\\
57.125	0.26136	-162.269262156001\\
57.125	0.26502	-168.437630068912\\
57.125	0.26868	-174.728229632818\\
57.125	0.27234	-181.141060847717\\
57.125	0.276	-187.67612371361\\
57.5	0.093	-10.5932092726126\\
57.5	0.09666	-11.1417000808913\\
57.5	0.10032	-11.8124225401639\\
57.5	0.10398	-12.6053766504304\\
57.5	0.10764	-13.5205624116909\\
57.5	0.1113	-14.5579798239453\\
57.5	0.11496	-15.7176288871936\\
57.5	0.11862	-16.9995096014358\\
57.5	0.12228	-18.4036219666719\\
57.5	0.12594	-19.929965982902\\
57.5	0.1296	-21.5785416501259\\
57.5	0.13326	-23.3493489683438\\
57.5	0.13692	-25.2423879375556\\
57.5	0.14058	-27.2576585577613\\
57.5	0.14424	-29.3951608289609\\
57.5	0.1479	-31.6548947511545\\
57.5	0.15156	-34.036860324342\\
57.5	0.15522	-36.5410575485233\\
57.5	0.15888	-39.1674864236987\\
57.5	0.16254	-41.9161469498679\\
57.5	0.1662	-44.787039127031\\
57.5	0.16986	-47.7801629551881\\
57.5	0.17352	-50.8955184343391\\
57.5	0.17718	-54.1331055644839\\
57.5	0.18084	-57.4929243456227\\
57.5	0.1845	-60.9749747777555\\
57.5	0.18816	-64.5792568608822\\
57.5	0.19182	-68.3057705950027\\
57.5	0.19548	-72.1545159801172\\
57.5	0.19914	-76.1254930162256\\
57.5	0.2028	-80.2187017033279\\
57.5	0.20646	-84.4341420414242\\
57.5	0.21012	-88.7718140305143\\
57.5	0.21378	-93.2317176705984\\
57.5	0.21744	-97.8138529616764\\
57.5	0.2211	-102.518219903748\\
57.5	0.22476	-107.344818496814\\
57.5	0.22842	-112.293648740874\\
57.5	0.23208	-117.364710635927\\
57.5	0.23574	-122.558004181975\\
57.5	0.2394	-127.873529379017\\
57.5	0.24306	-133.311286227052\\
57.5	0.24672	-138.871274726081\\
57.5	0.25038	-144.553494876105\\
57.5	0.25404	-150.357946677122\\
57.5	0.2577	-156.284630129133\\
57.5	0.26136	-162.333545232138\\
57.5	0.26502	-168.504691986137\\
57.5	0.26868	-174.79807039113\\
57.5	0.27234	-181.213680447116\\
57.5	0.276	-187.751522154097\\
57.875	0.093	-10.5392620158482\\
57.875	0.09666	-11.0905316652143\\
57.875	0.10032	-11.7640329655743\\
57.875	0.10398	-12.5597659169282\\
57.875	0.10764	-13.4777305192761\\
57.875	0.1113	-14.5179267726179\\
57.875	0.11496	-15.6803546769536\\
57.875	0.11862	-16.9650142322833\\
57.875	0.12228	-18.3719054386068\\
57.875	0.12594	-19.9010282959243\\
57.875	0.1296	-21.5523828042356\\
57.875	0.13326	-23.325968963541\\
57.875	0.13692	-25.2217867738402\\
57.875	0.14058	-27.2398362351333\\
57.875	0.14424	-29.3801173474204\\
57.875	0.1479	-31.6426301107013\\
57.875	0.15156	-34.0273745249762\\
57.875	0.15522	-36.5343505902451\\
57.875	0.15888	-39.1635583065078\\
57.875	0.16254	-41.9149976737644\\
57.875	0.1662	-44.788668692015\\
57.875	0.16986	-47.7845713612595\\
57.875	0.17352	-50.9027056814979\\
57.875	0.17718	-54.1430716527301\\
57.875	0.18084	-57.5056692749564\\
57.875	0.1845	-60.9904985481766\\
57.875	0.18816	-64.5975594723906\\
57.875	0.19182	-68.3268520475986\\
57.875	0.19548	-72.1783762738005\\
57.875	0.19914	-76.1521321509963\\
57.875	0.2028	-80.2481196791861\\
57.875	0.20646	-84.4663388583698\\
57.875	0.21012	-88.8067896885474\\
57.875	0.21378	-93.2694721697188\\
57.875	0.21744	-97.8543863018843\\
57.875	0.2211	-102.561532085044\\
57.875	0.22476	-107.390909519197\\
57.875	0.22842	-112.342518604344\\
57.875	0.23208	-117.416359340485\\
57.875	0.23574	-122.61243172762\\
57.875	0.2394	-127.930735765749\\
57.875	0.24306	-133.371271454872\\
57.875	0.24672	-138.934038794989\\
57.875	0.25038	-144.619037786099\\
57.875	0.25404	-150.426268428204\\
57.875	0.2577	-156.355730721302\\
57.875	0.26136	-162.407424665395\\
57.875	0.26502	-168.581350260481\\
57.875	0.26868	-174.877507506561\\
57.875	0.27234	-181.295896403636\\
57.875	0.276	-187.836516951704\\
58.25	0.093	-10.4949111162035\\
58.25	0.09666	-11.048959606657\\
58.25	0.10032	-11.7252397481045\\
58.25	0.10398	-12.5237515405458\\
58.25	0.10764	-13.4444949839811\\
58.25	0.1113	-14.4874700784104\\
58.25	0.11496	-15.6526768238335\\
58.25	0.11862	-16.9401152202505\\
58.25	0.12228	-18.3497852676615\\
58.25	0.12594	-19.8816869660664\\
58.25	0.1296	-21.5358203154652\\
58.25	0.13326	-23.3121853158579\\
58.25	0.13692	-25.2107819672446\\
58.25	0.14058	-27.2316102696251\\
58.25	0.14424	-29.3746702229996\\
58.25	0.1479	-31.639961827368\\
58.25	0.15156	-34.0274850827303\\
58.25	0.15522	-36.5372399890866\\
58.25	0.15888	-39.1692265464367\\
58.25	0.16254	-41.9234447547808\\
58.25	0.1662	-44.7998946141188\\
58.25	0.16986	-47.7985761244506\\
58.25	0.17352	-50.9194892857765\\
58.25	0.17718	-54.1626340980962\\
58.25	0.18084	-57.5280105614098\\
58.25	0.1845	-61.0156186757175\\
58.25	0.18816	-64.625458441019\\
58.25	0.19182	-68.3575298573143\\
58.25	0.19548	-72.2118329246037\\
58.25	0.19914	-76.1883676428869\\
58.25	0.2028	-80.2871340121641\\
58.25	0.20646	-84.5081320324352\\
58.25	0.21012	-88.8513617037002\\
58.25	0.21378	-93.3168230259591\\
58.25	0.21744	-97.904515999212\\
58.25	0.2211	-102.614440623459\\
58.25	0.22476	-107.446596898699\\
58.25	0.22842	-112.400984824934\\
58.25	0.23208	-117.477604402162\\
58.25	0.23574	-122.676455630385\\
58.25	0.2394	-127.997538509601\\
58.25	0.24306	-133.440853039812\\
58.25	0.24672	-139.006399221016\\
58.25	0.25038	-144.694177053214\\
58.25	0.25404	-150.504186536406\\
58.25	0.2577	-156.436427670592\\
58.25	0.26136	-162.490900455772\\
58.25	0.26502	-168.667604891945\\
58.25	0.26868	-174.966540979113\\
58.25	0.27234	-181.387708717275\\
58.25	0.276	-187.93110810643\\
58.625	0.093	-10.4601565736786\\
58.625	0.09666	-11.0169839052196\\
58.625	0.10032	-11.6960428877545\\
58.625	0.10398	-12.4973335212833\\
58.625	0.10764	-13.420855805806\\
58.625	0.1113	-14.4666097413226\\
58.625	0.11496	-15.6345953278332\\
58.625	0.11862	-16.9248125653376\\
58.625	0.12228	-18.337261453836\\
58.625	0.12594	-19.8719419933284\\
58.625	0.1296	-21.5288541838146\\
58.625	0.13326	-23.3079980252947\\
58.625	0.13692	-25.2093735177688\\
58.625	0.14058	-27.2329806612368\\
58.625	0.14424	-29.3788194556987\\
58.625	0.1479	-31.6468899011545\\
58.625	0.15156	-34.0371919976042\\
58.625	0.15522	-36.5497257450479\\
58.625	0.15888	-39.1844911434855\\
58.625	0.16254	-41.941488192917\\
58.625	0.1662	-44.8207168933424\\
58.625	0.16986	-47.8221772447617\\
58.625	0.17352	-50.9458692471749\\
58.625	0.17718	-54.1917929005821\\
58.625	0.18084	-57.5599482049831\\
58.625	0.1845	-61.0503351603781\\
58.625	0.18816	-64.6629537667671\\
58.625	0.19182	-68.3978040241499\\
58.625	0.19548	-72.2548859325266\\
58.625	0.19914	-76.2341994918973\\
58.625	0.2028	-80.3357447022619\\
58.625	0.20646	-84.5595215636204\\
58.625	0.21012	-88.9055300759729\\
58.625	0.21378	-93.3737702393192\\
58.625	0.21744	-97.9642420536594\\
58.625	0.2211	-102.676945518994\\
58.625	0.22476	-107.511880635322\\
58.625	0.22842	-112.469047402644\\
58.625	0.23208	-117.54844582096\\
58.625	0.23574	-122.75007589027\\
58.625	0.2394	-128.073937610573\\
58.625	0.24306	-133.520030981871\\
58.625	0.24672	-139.088356004163\\
58.625	0.25038	-144.778912677448\\
58.625	0.25404	-150.591701001728\\
58.625	0.2577	-156.526720977001\\
58.625	0.26136	-162.583972603268\\
58.625	0.26502	-168.763455880529\\
58.625	0.26868	-175.065170808784\\
58.625	0.27234	-181.489117388033\\
58.625	0.276	-188.035295618276\\
59	0.093	-10.4349983882736\\
59	0.09666	-10.994604560902\\
59	0.10032	-11.6764423845243\\
59	0.10398	-12.4805118591405\\
59	0.10764	-13.4068129847506\\
59	0.1113	-14.4553457613547\\
59	0.11496	-15.6261101889527\\
59	0.11862	-16.9191062675446\\
59	0.12228	-18.3343339971304\\
59	0.12594	-19.8717933777101\\
59	0.1296	-21.5314844092838\\
59	0.13326	-23.3134070918513\\
59	0.13692	-25.2175614254128\\
59	0.14058	-27.2439474099682\\
59	0.14424	-29.3925650455176\\
59	0.1479	-31.6634143320608\\
59	0.15156	-34.0564952695979\\
59	0.15522	-36.571807858129\\
59	0.15888	-39.209352097654\\
59	0.16254	-41.9691279881729\\
59	0.1662	-44.8511355296857\\
59	0.16986	-47.8553747221925\\
59	0.17352	-50.9818455656931\\
59	0.17718	-54.2305480601877\\
59	0.18084	-57.6014822056762\\
59	0.1845	-61.0946480021586\\
59	0.18816	-64.710045449635\\
59	0.19182	-68.4476745481052\\
59	0.19548	-72.3075352975694\\
59	0.19914	-76.2896276980275\\
59	0.2028	-80.3939517494795\\
59	0.20646	-84.6205074519255\\
59	0.21012	-88.9692948053654\\
59	0.21378	-93.4403138097991\\
59	0.21744	-98.0335644652267\\
59	0.2211	-102.749046771648\\
59	0.22476	-107.586760729064\\
59	0.22842	-112.546706337473\\
59	0.23208	-117.628883596877\\
59	0.23574	-122.833292507274\\
59	0.2394	-128.159933068665\\
59	0.24306	-133.60880528105\\
59	0.24672	-139.179909144429\\
59	0.25038	-144.873244658802\\
59	0.25404	-150.688811824169\\
59	0.2577	-156.62661064053\\
59	0.26136	-162.686641107885\\
59	0.26502	-168.868903226233\\
59	0.26868	-175.173396995576\\
59	0.27234	-181.600122415912\\
59	0.276	-188.149079487242\\
59.375	0.093	-10.4194365599884\\
59.375	0.09666	-10.9818215737042\\
59.375	0.10032	-11.6664382384139\\
59.375	0.10398	-12.4732865541175\\
59.375	0.10764	-13.4023665208151\\
59.375	0.1113	-14.4536781385066\\
59.375	0.11496	-15.627221407192\\
59.375	0.11862	-16.9229963268713\\
59.375	0.12228	-18.3410028975445\\
59.375	0.12594	-19.8812411192117\\
59.375	0.1296	-21.5437109918728\\
59.375	0.13326	-23.3284125155277\\
59.375	0.13692	-25.2353456901767\\
59.375	0.14058	-27.2645105158195\\
59.375	0.14424	-29.4159069924562\\
59.375	0.1479	-31.6895351200869\\
59.375	0.15156	-34.0853948987114\\
59.375	0.15522	-36.60348632833\\
59.375	0.15888	-39.2438094089424\\
59.375	0.16254	-42.0063641405487\\
59.375	0.1662	-44.891150523149\\
59.375	0.16986	-47.8981685567431\\
59.375	0.17352	-51.0274182413312\\
59.375	0.17718	-54.2788995769132\\
59.375	0.18084	-57.6526125634891\\
59.375	0.1845	-61.148557201059\\
59.375	0.18816	-64.7667334896228\\
59.375	0.19182	-68.5071414291804\\
59.375	0.19548	-72.369781019732\\
59.375	0.19914	-76.3546522612775\\
59.375	0.2028	-80.461755153817\\
59.375	0.20646	-84.6910896973503\\
59.375	0.21012	-89.0426558918776\\
59.375	0.21378	-93.5164537373988\\
59.375	0.21744	-98.1124832339139\\
59.375	0.2211	-102.830744381423\\
59.375	0.22476	-107.671237179926\\
59.375	0.22842	-112.633961629423\\
59.375	0.23208	-117.718917729913\\
59.375	0.23574	-122.926105481398\\
59.375	0.2394	-128.255524883877\\
59.375	0.24306	-133.707175937349\\
59.375	0.24672	-139.281058641816\\
59.375	0.25038	-144.977172997276\\
59.375	0.25404	-150.79551900373\\
59.375	0.2577	-156.736096661179\\
59.375	0.26136	-162.798905969621\\
59.375	0.26502	-168.983946929057\\
59.375	0.26868	-175.291219539487\\
59.375	0.27234	-181.720723800911\\
59.375	0.276	-188.272459713328\\
59.75	0.093	-10.413471088823\\
59.75	0.09666	-10.9786349436262\\
59.75	0.10032	-11.6660304494233\\
59.75	0.10398	-12.4756576062144\\
59.75	0.10764	-13.4075164139994\\
59.75	0.1113	-14.4616068727783\\
59.75	0.11496	-15.6379289825511\\
59.75	0.11862	-16.9364827433178\\
59.75	0.12228	-18.3572681550785\\
59.75	0.12594	-19.9002852178331\\
59.75	0.1296	-21.5655339315816\\
59.75	0.13326	-23.353014296324\\
59.75	0.13692	-25.2627263120603\\
59.75	0.14058	-27.2946699787905\\
59.75	0.14424	-29.4488452965147\\
59.75	0.1479	-31.7252522652328\\
59.75	0.15156	-34.1238908849448\\
59.75	0.15522	-36.6447611556507\\
59.75	0.15888	-39.2878630773506\\
59.75	0.16254	-42.0531966500443\\
59.75	0.1662	-44.940761873732\\
59.75	0.16986	-47.9505587484136\\
59.75	0.17352	-51.0825872740891\\
59.75	0.17718	-54.3368474507585\\
59.75	0.18084	-57.7133392784218\\
59.75	0.1845	-61.2120627570791\\
59.75	0.18816	-64.8330178867303\\
59.75	0.19182	-68.5762046673754\\
59.75	0.19548	-72.4416230990144\\
59.75	0.19914	-76.4292731816473\\
59.75	0.2028	-80.5391549152742\\
59.75	0.20646	-84.771268299895\\
59.75	0.21012	-89.1256133355097\\
59.75	0.21378	-93.6021900221183\\
59.75	0.21744	-98.2009983597208\\
59.75	0.2211	-102.922038348317\\
59.75	0.22476	-107.765309987908\\
59.75	0.22842	-112.730813278492\\
59.75	0.23208	-117.81854822007\\
59.75	0.23574	-123.028514812642\\
59.75	0.2394	-128.360713056208\\
59.75	0.24306	-133.815142950768\\
59.75	0.24672	-139.391804496322\\
59.75	0.25038	-145.09069769287\\
59.75	0.25404	-150.911822540412\\
59.75	0.2577	-156.855179038947\\
59.75	0.26136	-162.920767188477\\
59.75	0.26502	-169.108586989\\
59.75	0.26868	-175.418638440517\\
59.75	0.27234	-181.850921543029\\
59.75	0.276	-188.405436296534\\
60.125	0.093	-10.4171019747773\\
60.125	0.09666	-10.985044670668\\
60.125	0.10032	-11.6752190175526\\
60.125	0.10398	-12.487625015431\\
60.125	0.10764	-13.4222626643034\\
60.125	0.1113	-14.4791319641698\\
60.125	0.11496	-15.65823291503\\
60.125	0.11862	-16.9595655168842\\
60.125	0.12228	-18.3831297697322\\
60.125	0.12594	-19.9289256735743\\
60.125	0.1296	-21.5969532284102\\
60.125	0.13326	-23.38721243424\\
60.125	0.13692	-25.2997032910637\\
60.125	0.14058	-27.3344257988814\\
60.125	0.14424	-29.491379957693\\
60.125	0.1479	-31.7705657674985\\
60.125	0.15156	-34.1719832282979\\
60.125	0.15522	-36.6956323400913\\
60.125	0.15888	-39.3415131028785\\
60.125	0.16254	-42.1096255166597\\
60.125	0.1662	-44.9999695814348\\
60.125	0.16986	-48.0125452972038\\
60.125	0.17352	-51.1473526639667\\
60.125	0.17718	-54.4043916817236\\
60.125	0.18084	-57.7836623504743\\
60.125	0.1845	-61.285164670219\\
60.125	0.18816	-64.9088986409576\\
60.125	0.19182	-68.6548642626901\\
60.125	0.19548	-72.5230615354166\\
60.125	0.19914	-76.5134904591369\\
60.125	0.2028	-80.6261510338512\\
60.125	0.20646	-84.8610432595595\\
60.125	0.21012	-89.2181671362616\\
60.125	0.21378	-93.6975226639576\\
60.125	0.21744	-98.2991098426475\\
60.125	0.2211	-103.022928672331\\
60.125	0.22476	-107.868979153009\\
60.125	0.22842	-112.837261284681\\
60.125	0.23208	-117.927775067346\\
60.125	0.23574	-123.140520501006\\
60.125	0.2394	-128.475497585659\\
60.125	0.24306	-133.932706321307\\
60.125	0.24672	-139.512146707948\\
60.125	0.25038	-145.213818745583\\
60.125	0.25404	-151.037722434212\\
60.125	0.2577	-156.983857773836\\
60.125	0.26136	-163.052224764453\\
60.125	0.26502	-169.242823406063\\
60.125	0.26868	-175.555653698668\\
60.125	0.27234	-181.990715642267\\
60.125	0.276	-188.548009236859\\
60.5	0.093	-10.4303292178515\\
60.5	0.09666	-11.0010507548296\\
60.5	0.10032	-11.6940039428016\\
60.5	0.10398	-12.5091887817675\\
60.5	0.10764	-13.4466052717273\\
60.5	0.1113	-14.5062534126811\\
60.5	0.11496	-15.6881332046288\\
60.5	0.11862	-16.9922446475703\\
60.5	0.12228	-18.4185877415058\\
60.5	0.12594	-19.9671624864353\\
60.5	0.1296	-21.6379688823586\\
60.5	0.13326	-23.4310069292758\\
60.5	0.13692	-25.346276627187\\
60.5	0.14058	-27.3837779760921\\
60.5	0.14424	-29.5435109759911\\
60.5	0.1479	-31.825475626884\\
60.5	0.15156	-34.2296719287709\\
60.5	0.15522	-36.7560998816516\\
60.5	0.15888	-39.4047594855263\\
60.5	0.16254	-42.1756507403949\\
60.5	0.1662	-45.0687736462574\\
60.5	0.16986	-48.0841282031138\\
60.5	0.17352	-51.2217144109642\\
60.5	0.17718	-54.4815322698084\\
60.5	0.18084	-57.8635817796466\\
60.5	0.1845	-61.3678629404788\\
60.5	0.18816	-64.9943757523048\\
60.5	0.19182	-68.7431202151247\\
60.5	0.19548	-72.6140963289386\\
60.5	0.19914	-76.6073040937464\\
60.5	0.2028	-80.7227435095481\\
60.5	0.20646	-84.9604145763437\\
60.5	0.21012	-89.3203172941333\\
60.5	0.21378	-93.8024516629167\\
60.5	0.21744	-98.406817682694\\
60.5	0.2211	-103.133415353465\\
60.5	0.22476	-107.982244675231\\
60.5	0.22842	-112.95330564799\\
60.5	0.23208	-118.046598271743\\
60.5	0.23574	-123.26212254649\\
60.5	0.2394	-128.599878472231\\
60.5	0.24306	-134.059866048965\\
60.5	0.24672	-139.642085276694\\
60.5	0.25038	-145.346536155417\\
60.5	0.25404	-151.173218685133\\
60.5	0.2577	-157.122132865844\\
60.5	0.26136	-163.193278697548\\
60.5	0.26502	-169.386656180246\\
60.5	0.26868	-175.702265313939\\
60.5	0.27234	-182.140106098625\\
60.5	0.276	-188.700178534305\\
60.875	0.093	-10.4531528180456\\
60.875	0.09666	-11.026653196111\\
60.875	0.10032	-11.7223852251704\\
60.875	0.10398	-12.5403489052238\\
60.875	0.10764	-13.480544236271\\
60.875	0.1113	-14.5429712183122\\
60.875	0.11496	-15.7276298513473\\
60.875	0.11862	-17.0345201353763\\
60.875	0.12228	-18.4636420703992\\
60.875	0.12594	-20.014995656416\\
60.875	0.1296	-21.6885808934268\\
60.875	0.13326	-23.4843977814315\\
60.875	0.13692	-25.4024463204301\\
60.875	0.14058	-27.4427265104226\\
60.875	0.14424	-29.605238351409\\
60.875	0.1479	-31.8899818433894\\
60.875	0.15156	-34.2969569863636\\
60.875	0.15522	-36.8261637803318\\
60.875	0.15888	-39.4776022252939\\
60.875	0.16254	-42.2512723212499\\
60.875	0.1662	-45.1471740681999\\
60.875	0.16986	-48.1653074661437\\
60.875	0.17352	-51.3056725150815\\
60.875	0.17718	-54.5682692150131\\
60.875	0.18084	-57.9530975659387\\
60.875	0.1845	-61.4601575678583\\
60.875	0.18816	-65.0894492207718\\
60.875	0.19182	-68.8409725246791\\
60.875	0.19548	-72.7147274795804\\
60.875	0.19914	-76.7107140854756\\
60.875	0.2028	-80.8289323423648\\
60.875	0.20646	-85.0693822502478\\
60.875	0.21012	-89.4320638091248\\
60.875	0.21378	-93.9169770189956\\
60.875	0.21744	-98.5241218798604\\
60.875	0.2211	-103.253498391719\\
60.875	0.22476	-108.105106554572\\
60.875	0.22842	-113.078946368418\\
60.875	0.23208	-118.175017833259\\
60.875	0.23574	-123.393320949093\\
60.875	0.2394	-128.733855715921\\
60.875	0.24306	-134.196622133744\\
60.875	0.24672	-139.78162020256\\
60.875	0.25038	-145.48884992237\\
60.875	0.25404	-151.318311293174\\
60.875	0.2577	-157.270004314972\\
60.875	0.26136	-163.343928987764\\
60.875	0.26502	-169.540085311549\\
60.875	0.26868	-175.858473286329\\
60.875	0.27234	-182.299092912102\\
60.875	0.276	-188.86194418887\\
61.25	0.093	-10.4855727753594\\
61.25	0.09666	-11.0618519945123\\
61.25	0.10032	-11.7603628646591\\
61.25	0.10398	-12.5811053857998\\
61.25	0.10764	-13.5240795579345\\
61.25	0.1113	-14.5892853810631\\
61.25	0.11496	-15.7767228551856\\
61.25	0.11862	-17.086391980302\\
61.25	0.12228	-18.5182927564124\\
61.25	0.12594	-20.0724251835166\\
61.25	0.1296	-21.7487892616148\\
61.25	0.13326	-23.5473849907069\\
61.25	0.13692	-25.4682123707929\\
61.25	0.14058	-27.5112714018729\\
61.25	0.14424	-29.6765620839467\\
61.25	0.1479	-31.9640844170145\\
61.25	0.15156	-34.3738384010762\\
61.25	0.15522	-36.9058240361318\\
61.25	0.15888	-39.5600413221813\\
61.25	0.16254	-42.3364902592248\\
61.25	0.1662	-45.2351708472621\\
61.25	0.16986	-48.2560830862934\\
61.25	0.17352	-51.3992269763186\\
61.25	0.17718	-54.6646025173377\\
61.25	0.18084	-58.0522097093507\\
61.25	0.1845	-61.5620485523577\\
61.25	0.18816	-65.1941190463585\\
61.25	0.19182	-68.9484211913533\\
61.25	0.19548	-72.824954987342\\
61.25	0.19914	-76.8237204343247\\
61.25	0.2028	-80.9447175323012\\
61.25	0.20646	-85.1879462812717\\
61.25	0.21012	-89.5534066812361\\
61.25	0.21378	-94.0410987321943\\
61.25	0.21744	-98.6510224341465\\
61.25	0.2211	-103.383177787093\\
61.25	0.22476	-108.237564791033\\
61.25	0.22842	-113.214183445967\\
61.25	0.23208	-118.313033751895\\
61.25	0.23574	-123.534115708816\\
61.25	0.2394	-128.877429316732\\
61.25	0.24306	-134.342974575642\\
61.25	0.24672	-139.930751485545\\
61.25	0.25038	-145.640760046443\\
61.25	0.25404	-151.473000258334\\
61.25	0.2577	-157.427472121219\\
61.25	0.26136	-163.504175635099\\
61.25	0.26502	-169.703110799972\\
61.25	0.26868	-176.024277615839\\
61.25	0.27234	-182.4676760827\\
61.25	0.276	-189.033306200555\\
61.625	0.093	-10.527589089793\\
61.625	0.09666	-11.1066471500333\\
61.625	0.10032	-11.8079368612676\\
61.625	0.10398	-12.6314582234957\\
61.625	0.10764	-13.5772112367179\\
61.625	0.1113	-14.6451959009339\\
61.625	0.11496	-15.8354122161438\\
61.625	0.11862	-17.1478601823476\\
61.625	0.12228	-18.5825397995454\\
61.625	0.12594	-20.1394510677371\\
61.625	0.1296	-21.8185939869227\\
61.625	0.13326	-23.6199685571022\\
61.625	0.13692	-25.5435747782756\\
61.625	0.14058	-27.589412650443\\
61.625	0.14424	-29.7574821736043\\
61.625	0.1479	-32.0477833477595\\
61.625	0.15156	-34.4603161729086\\
61.625	0.15522	-36.9950806490516\\
61.625	0.15888	-39.6520767761886\\
61.625	0.16254	-42.4313045543194\\
61.625	0.1662	-45.3327639834442\\
61.625	0.16986	-48.3564550635629\\
61.625	0.17352	-51.5023777946755\\
61.625	0.17718	-54.770532176782\\
61.625	0.18084	-58.1609182098824\\
61.625	0.1845	-61.6735358939769\\
61.625	0.18816	-65.3083852290652\\
61.625	0.19182	-69.0654662151473\\
61.625	0.19548	-72.9447788522235\\
61.625	0.19914	-76.9463231402935\\
61.625	0.2028	-81.0700990793575\\
61.625	0.20646	-85.3161066694154\\
61.625	0.21012	-89.6843459104672\\
61.625	0.21378	-94.1748168025129\\
61.625	0.21744	-98.7875193455525\\
61.625	0.2211	-103.522453539586\\
61.625	0.22476	-108.379619384614\\
61.625	0.22842	-113.359016880635\\
61.625	0.23208	-118.46064602765\\
61.625	0.23574	-123.684506825659\\
61.625	0.2394	-129.030599274663\\
61.625	0.24306	-134.49892337466\\
61.625	0.24672	-140.089479125651\\
61.625	0.25038	-145.802266527636\\
61.625	0.25404	-151.637285580614\\
61.625	0.2577	-157.594536284587\\
61.625	0.26136	-163.674018639554\\
61.625	0.26502	-169.875732645514\\
61.625	0.26868	-176.199678302469\\
61.625	0.27234	-182.645855610417\\
61.625	0.276	-189.214264569359\\
62	0.093	-10.5792017613464\\
62	0.09666	-11.1610386626742\\
62	0.10032	-11.8651072149959\\
62	0.10398	-12.6914074183114\\
62	0.10764	-13.6399392726209\\
62	0.1113	-14.7107027779244\\
62	0.11496	-15.9036979342217\\
62	0.11862	-17.218924741513\\
62	0.12228	-18.6563831997982\\
62	0.12594	-20.2160733090773\\
62	0.1296	-21.8979950693503\\
62	0.13326	-23.7021484806173\\
62	0.13692	-25.6285335428781\\
62	0.14058	-27.6771502561329\\
62	0.14424	-29.8479986203816\\
62	0.1479	-32.1410786356242\\
62	0.15156	-34.5563903018607\\
62	0.15522	-37.0939336190912\\
62	0.15888	-39.7537085873156\\
62	0.16254	-42.5357152065338\\
62	0.1662	-45.439953476746\\
62	0.16986	-48.4664233979522\\
62	0.17352	-51.6151249701522\\
62	0.17718	-54.8860581933461\\
62	0.18084	-58.279223067534\\
62	0.1845	-61.7946195927158\\
62	0.18816	-65.4322477688915\\
62	0.19182	-69.1921075960611\\
62	0.19548	-73.0741990742247\\
62	0.19914	-77.0785222033822\\
62	0.2028	-81.2050769835336\\
62	0.20646	-85.4538634146789\\
62	0.21012	-89.8248814968181\\
62	0.21378	-94.3181312299512\\
62	0.21744	-98.9336126140783\\
62	0.2211	-103.671325649199\\
62	0.22476	-108.531270335314\\
62	0.22842	-113.513446672423\\
62	0.23208	-118.617854660526\\
62	0.23574	-123.844494299622\\
62	0.2394	-129.193365589713\\
62	0.24306	-134.664468530797\\
62	0.24672	-140.257803122876\\
62	0.25038	-145.973369365948\\
62	0.25404	-151.811167260014\\
62	0.2577	-157.771196805075\\
62	0.26136	-163.853458001129\\
62	0.26502	-170.057950848176\\
62	0.26868	-176.384675346218\\
62	0.27234	-182.833631495254\\
62	0.276	-189.404819295284\\
62.375	0.093	-10.6404107900197\\
62.375	0.09666	-11.2250265324349\\
62.375	0.10032	-11.931873925844\\
62.375	0.10398	-12.760952970247\\
62.375	0.10764	-13.7122636656439\\
62.375	0.1113	-14.7858060120348\\
62.375	0.11496	-15.9815800094195\\
62.375	0.11862	-17.2995856577982\\
62.375	0.12228	-18.7398229571708\\
62.375	0.12594	-20.3022919075374\\
62.375	0.1296	-21.9869925088978\\
62.375	0.13326	-23.7939247612521\\
62.375	0.13692	-25.7230886646005\\
62.375	0.14058	-27.7744842189426\\
62.375	0.14424	-29.9481114242788\\
62.375	0.1479	-32.2439702806088\\
62.375	0.15156	-34.6620607879327\\
62.375	0.15522	-37.2023829462506\\
62.375	0.15888	-39.8649367555624\\
62.375	0.16254	-42.6497222158681\\
62.375	0.1662	-45.5567393271677\\
62.375	0.16986	-48.5859880894613\\
62.375	0.17352	-51.7374685027487\\
62.375	0.17718	-55.0111805670301\\
62.375	0.18084	-58.4071242823054\\
62.375	0.1845	-61.9252996485746\\
62.375	0.18816	-65.5657066658378\\
62.375	0.19182	-69.3283453340948\\
62.375	0.19548	-73.2132156533458\\
62.375	0.19914	-77.2203176235907\\
62.375	0.2028	-81.3496512448295\\
62.375	0.20646	-85.6012165170623\\
62.375	0.21012	-89.9750134402889\\
62.375	0.21378	-94.4710420145094\\
62.375	0.21744	-99.0893022397238\\
62.375	0.2211	-103.829794115932\\
62.375	0.22476	-108.692517643135\\
62.375	0.22842	-113.677472821331\\
62.375	0.23208	-118.784659650521\\
62.375	0.23574	-124.014078130705\\
62.375	0.2394	-129.365728261883\\
62.375	0.24306	-134.839610044055\\
62.375	0.24672	-140.435723477221\\
62.375	0.25038	-146.154068561381\\
62.375	0.25404	-151.994645296534\\
62.375	0.2577	-157.957453682682\\
62.375	0.26136	-164.042493719823\\
62.375	0.26502	-170.249765407959\\
62.375	0.26868	-176.579268747088\\
62.375	0.27234	-183.031003737211\\
62.375	0.276	-189.604970378328\\
62.75	0.093	-10.7112161758128\\
62.75	0.09666	-11.2986107593154\\
62.75	0.10032	-12.0082369938119\\
62.75	0.10398	-12.8400948793023\\
62.75	0.10764	-13.7941844157867\\
62.75	0.1113	-14.870505603265\\
62.75	0.11496	-16.0690584417372\\
62.75	0.11862	-17.3898429312033\\
62.75	0.12228	-18.8328590716633\\
62.75	0.12594	-20.3981068631172\\
62.75	0.1296	-22.0855863055651\\
62.75	0.13326	-23.8952973990069\\
62.75	0.13692	-25.8272401434426\\
62.75	0.14058	-27.8814145388722\\
62.75	0.14424	-30.0578205852958\\
62.75	0.1479	-32.3564582827132\\
62.75	0.15156	-34.7773276311246\\
62.75	0.15522	-37.3204286305299\\
62.75	0.15888	-39.9857612809291\\
62.75	0.16254	-42.7733255823222\\
62.75	0.1662	-45.6831215347093\\
62.75	0.16986	-48.7151491380902\\
62.75	0.17352	-51.8694083924651\\
62.75	0.17718	-55.1458992978339\\
62.75	0.18084	-58.5446218541966\\
62.75	0.1845	-62.0655760615533\\
62.75	0.18816	-65.7087619199038\\
62.75	0.19182	-69.4741794292482\\
62.75	0.19548	-73.3618285895866\\
62.75	0.19914	-77.3717094009189\\
62.75	0.2028	-81.5038218632452\\
62.75	0.20646	-85.7581659765654\\
62.75	0.21012	-90.1347417408794\\
62.75	0.21378	-94.6335491561874\\
62.75	0.21744	-99.2545882224892\\
62.75	0.2211	-103.997858939785\\
62.75	0.22476	-108.863361308075\\
62.75	0.22842	-113.851095327359\\
62.75	0.23208	-118.961060997636\\
62.75	0.23574	-124.193258318908\\
62.75	0.2394	-129.547687291173\\
62.75	0.24306	-135.024347914432\\
62.75	0.24672	-140.623240188686\\
62.75	0.25038	-146.344364113933\\
62.75	0.25404	-152.187719690174\\
62.75	0.2577	-158.153306917409\\
62.75	0.26136	-164.241125795638\\
62.75	0.26502	-170.451176324861\\
62.75	0.26868	-176.783458505077\\
62.75	0.27234	-183.237972336288\\
62.75	0.276	-189.814717818493\\
63.125	0.093	-10.7916179187256\\
63.125	0.09666	-11.3817913433157\\
63.125	0.10032	-12.0941964188996\\
63.125	0.10398	-12.9288331454774\\
63.125	0.10764	-13.8857015230492\\
63.125	0.1113	-14.9648015516149\\
63.125	0.11496	-16.1661332311746\\
63.125	0.11862	-17.4896965617281\\
63.125	0.12228	-18.9354915432755\\
63.125	0.12594	-20.5035181758169\\
63.125	0.1296	-22.1937764593522\\
63.125	0.13326	-24.0062663938814\\
63.125	0.13692	-25.9409879794045\\
63.125	0.14058	-27.9979412159216\\
63.125	0.14424	-30.1771261034325\\
63.125	0.1479	-32.4785426419374\\
63.125	0.15156	-34.9021908314362\\
63.125	0.15522	-37.4480706719289\\
63.125	0.15888	-40.1161821634156\\
63.125	0.16254	-42.9065253058961\\
63.125	0.1662	-45.8191000993706\\
63.125	0.16986	-48.8539065438389\\
63.125	0.17352	-52.0109446393012\\
63.125	0.17718	-55.2902143857575\\
63.125	0.18084	-58.6917157832076\\
63.125	0.1845	-62.2154488316517\\
63.125	0.18816	-65.8614135310897\\
63.125	0.19182	-69.6296098815215\\
63.125	0.19548	-73.5200378829473\\
63.125	0.19914	-77.532697535367\\
63.125	0.2028	-81.6675888387807\\
63.125	0.20646	-85.9247117931883\\
63.125	0.21012	-90.3040663985898\\
63.125	0.21378	-94.8056526549852\\
63.125	0.21744	-99.4294705623745\\
63.125	0.2211	-104.175520120758\\
63.125	0.22476	-109.043801330135\\
63.125	0.22842	-114.034314190506\\
63.125	0.23208	-119.147058701871\\
63.125	0.23574	-124.38203486423\\
63.125	0.2394	-129.739242677583\\
63.125	0.24306	-135.218682141929\\
63.125	0.24672	-140.82035325727\\
63.125	0.25038	-146.544256023605\\
63.125	0.25404	-152.390390440933\\
63.125	0.2577	-158.358756509256\\
63.125	0.26136	-164.449354228572\\
63.125	0.26502	-170.662183598882\\
63.125	0.26868	-176.997244620186\\
63.125	0.27234	-183.454537292484\\
63.125	0.276	-190.034061615776\\
63.5	0.093	-10.8816160187583\\
63.5	0.09666	-11.4745682844357\\
63.5	0.10032	-12.1897522011071\\
63.5	0.10398	-13.0271677687724\\
63.5	0.10764	-13.9868149874316\\
63.5	0.1113	-15.0686938570847\\
63.5	0.11496	-16.2728043777317\\
63.5	0.11862	-17.5991465493727\\
63.5	0.12228	-19.0477203720076\\
63.5	0.12594	-20.6185258456364\\
63.5	0.1296	-22.3115629702591\\
63.5	0.13326	-24.1268317458757\\
63.5	0.13692	-26.0643321724862\\
63.5	0.14058	-28.1240642500907\\
63.5	0.14424	-30.3060279786891\\
63.5	0.1479	-32.6102233582814\\
63.5	0.15156	-35.0366503888676\\
63.5	0.15522	-37.5853090704477\\
63.5	0.15888	-40.2561994030218\\
63.5	0.16254	-43.0493213865898\\
63.5	0.1662	-45.9646750211517\\
63.5	0.16986	-49.0022603067075\\
63.5	0.17352	-52.1620772432572\\
63.5	0.17718	-55.4441258308008\\
63.5	0.18084	-58.8484060693384\\
63.5	0.1845	-62.3749179588699\\
63.5	0.18816	-66.0236614993953\\
63.5	0.19182	-69.7946366909145\\
63.5	0.19548	-73.6878435334278\\
63.5	0.19914	-77.7032820269349\\
63.5	0.2028	-81.840952171436\\
63.5	0.20646	-86.100853966931\\
63.5	0.21012	-90.4829874134199\\
63.5	0.21378	-94.9873525109027\\
63.5	0.21744	-99.6139492593795\\
63.5	0.2211	-104.36277765885\\
63.5	0.22476	-109.233837709315\\
63.5	0.22842	-114.227129410773\\
63.5	0.23208	-119.342652763226\\
63.5	0.23574	-124.580407766672\\
63.5	0.2394	-129.940394421112\\
63.5	0.24306	-135.422612726546\\
63.5	0.24672	-141.027062682975\\
63.5	0.25038	-146.753744290396\\
63.5	0.25404	-152.602657548812\\
63.5	0.2577	-158.573802458222\\
63.5	0.26136	-164.667179018626\\
63.5	0.26502	-170.882787230024\\
63.5	0.26868	-177.220627092415\\
63.5	0.27234	-183.680698605801\\
63.5	0.276	-190.26300177018\\
63.875	0.093	-10.9812104759108\\
63.875	0.09666	-11.5769415826757\\
63.875	0.10032	-12.2949043404345\\
63.875	0.10398	-13.1350987491871\\
63.875	0.10764	-14.0975248089338\\
63.875	0.1113	-15.1821825196743\\
63.875	0.11496	-16.3890718814088\\
63.875	0.11862	-17.7181928941371\\
63.875	0.12228	-19.1695455578594\\
63.875	0.12594	-20.7431298725757\\
63.875	0.1296	-22.4389458382858\\
63.875	0.13326	-24.2569934549898\\
63.875	0.13692	-26.1972727226878\\
63.875	0.14058	-28.2597836413797\\
63.875	0.14424	-30.4445262110655\\
63.875	0.1479	-32.7515004317452\\
63.875	0.15156	-35.1807063034188\\
63.875	0.15522	-37.7321438260864\\
63.875	0.15888	-40.4058129997479\\
63.875	0.16254	-43.2017138244033\\
63.875	0.1662	-46.1198463000526\\
63.875	0.16986	-49.1602104266958\\
63.875	0.17352	-52.3228062043329\\
63.875	0.17718	-55.607633632964\\
63.875	0.18084	-59.014692712589\\
63.875	0.1845	-62.5439834432079\\
63.875	0.18816	-66.1955058248207\\
63.875	0.19182	-69.9692598574274\\
63.875	0.19548	-73.8652455410281\\
63.875	0.19914	-77.8834628756227\\
63.875	0.2028	-82.0239118612112\\
63.875	0.20646	-86.2865924977936\\
63.875	0.21012	-90.6715047853699\\
63.875	0.21378	-95.1786487239401\\
63.875	0.21744	-99.8080243135043\\
63.875	0.2211	-104.559631554062\\
63.875	0.22476	-109.433470445614\\
63.875	0.22842	-114.42954098816\\
63.875	0.23208	-119.5478431817\\
63.875	0.23574	-124.788377026234\\
63.875	0.2394	-130.151142521762\\
63.875	0.24306	-135.636139668283\\
63.875	0.24672	-141.243368465799\\
63.875	0.25038	-146.972828914308\\
63.875	0.25404	-152.824521013811\\
63.875	0.2577	-158.798444764309\\
63.875	0.26136	-164.8946001658\\
63.875	0.26502	-171.112987218285\\
63.875	0.26868	-177.453605921764\\
63.875	0.27234	-183.916456276237\\
63.875	0.276	-190.501538281704\\
64.25	0.093	-11.0904012901831\\
64.25	0.09666	-11.6889112380354\\
64.25	0.10032	-12.4096528368816\\
64.25	0.10398	-13.2526260867217\\
64.25	0.10764	-14.2178309875557\\
64.25	0.1113	-15.3052675393837\\
64.25	0.11496	-16.5149357422056\\
64.25	0.11862	-17.8468355960214\\
64.25	0.12228	-19.3009671008311\\
64.25	0.12594	-20.8773302566347\\
64.25	0.1296	-22.5759250634323\\
64.25	0.13326	-24.3967515212237\\
64.25	0.13692	-26.3398096300091\\
64.25	0.14058	-28.4050993897884\\
64.25	0.14424	-30.5926208005617\\
64.25	0.1479	-32.9023738623288\\
64.25	0.15156	-35.3343585750899\\
64.25	0.15522	-37.8885749388449\\
64.25	0.15888	-40.5650229535938\\
64.25	0.16254	-43.3637026193366\\
64.25	0.1662	-46.2846139360733\\
64.25	0.16986	-49.3277569038039\\
64.25	0.17352	-52.4931315225285\\
64.25	0.17718	-55.780737792247\\
64.25	0.18084	-59.1905757129594\\
64.25	0.1845	-62.7226452846657\\
64.25	0.18816	-66.3769465073659\\
64.25	0.19182	-70.1534793810601\\
64.25	0.19548	-74.0522439057482\\
64.25	0.19914	-78.0732400814302\\
64.25	0.2028	-82.2164679081061\\
64.25	0.20646	-86.4819273857759\\
64.25	0.21012	-90.8696185144397\\
64.25	0.21378	-95.3795412940973\\
64.25	0.21744	-100.011695724749\\
64.25	0.2211	-104.766081806394\\
64.25	0.22476	-109.642699539034\\
64.25	0.22842	-114.641548922667\\
64.25	0.23208	-119.762629957294\\
64.25	0.23574	-125.005942642916\\
64.25	0.2394	-130.371486979531\\
64.25	0.24306	-135.85926296714\\
64.25	0.24672	-141.469270605743\\
64.25	0.25038	-147.20150989534\\
64.25	0.25404	-153.05598083593\\
64.25	0.2577	-159.032683427515\\
64.25	0.26136	-165.131617670094\\
64.25	0.26502	-171.352783563666\\
64.25	0.26868	-177.696181108233\\
64.25	0.27234	-184.161810303793\\
64.25	0.276	-190.749671150347\\
64.625	0.093	-11.2091884615752\\
64.625	0.09666	-11.8104772505149\\
64.625	0.10032	-12.5339976904485\\
64.625	0.10398	-13.379749781376\\
64.625	0.10764	-14.3477335232975\\
64.625	0.1113	-15.4379489162129\\
64.625	0.11496	-16.6503959601222\\
64.625	0.11862	-17.9850746550254\\
64.625	0.12228	-19.4419850009225\\
64.625	0.12594	-21.0211269978136\\
64.625	0.1296	-22.7225006456986\\
64.625	0.13326	-24.5461059445775\\
64.625	0.13692	-26.4919428944503\\
64.625	0.14058	-28.560011495317\\
64.625	0.14424	-30.7503117471777\\
64.625	0.1479	-33.0628436500322\\
64.625	0.15156	-35.4976072038807\\
64.625	0.15522	-38.0546024087231\\
64.625	0.15888	-40.7338292645595\\
64.625	0.16254	-43.5352877713897\\
64.625	0.1662	-46.4589779292138\\
64.625	0.16986	-49.5048997380319\\
64.625	0.17352	-52.6730531978439\\
64.625	0.17718	-55.9634383086498\\
64.625	0.18084	-59.3760550704496\\
64.625	0.1845	-62.9109034832434\\
64.625	0.18816	-66.567983547031\\
64.625	0.19182	-70.3472952618125\\
64.625	0.19548	-74.2488386275881\\
64.625	0.19914	-78.2726136443575\\
64.625	0.2028	-82.4186203121209\\
64.625	0.20646	-86.6868586308781\\
64.625	0.21012	-91.0773286006293\\
64.625	0.21378	-95.5900302213743\\
64.625	0.21744	-100.224963493113\\
64.625	0.2211	-104.982128415846\\
64.625	0.22476	-109.861524989573\\
64.625	0.22842	-114.863153214294\\
64.625	0.23208	-119.987013090009\\
64.625	0.23574	-125.233104616717\\
64.625	0.2394	-130.60142779442\\
64.625	0.24306	-136.091982623116\\
64.625	0.24672	-141.704769102806\\
64.625	0.25038	-147.439787233491\\
64.625	0.25404	-153.297037015169\\
64.625	0.2577	-159.276518447841\\
64.625	0.26136	-165.378231531507\\
64.625	0.26502	-171.602176266167\\
64.625	0.26868	-177.948352651821\\
64.625	0.27234	-184.416760688469\\
64.625	0.276	-191.00740037611\\
65	0.093	-11.3375719900871\\
65	0.09666	-11.9416396201142\\
65	0.10032	-12.6679389011353\\
65	0.10398	-13.5164698331502\\
65	0.10764	-14.4872324161591\\
65	0.1113	-15.5802266501619\\
65	0.11496	-16.7954525351586\\
65	0.11862	-18.1329100711493\\
65	0.12228	-19.5925992581338\\
65	0.12594	-21.1745200961123\\
65	0.1296	-22.8786725850847\\
65	0.13326	-24.705056725051\\
65	0.13692	-26.6536725160113\\
65	0.14058	-28.7245199579654\\
65	0.14424	-30.9175990509135\\
65	0.1479	-33.2329097948555\\
65	0.15156	-35.6704521897914\\
65	0.15522	-38.2302262357212\\
65	0.15888	-40.9122319326449\\
65	0.16254	-43.7164692805626\\
65	0.1662	-46.6429382794742\\
65	0.16986	-49.6916389293796\\
65	0.17352	-52.8625712302791\\
65	0.17718	-56.1557351821724\\
65	0.18084	-59.5711307850596\\
65	0.1845	-63.1087580389408\\
65	0.18816	-66.7686169438159\\
65	0.19182	-70.5507074996848\\
65	0.19548	-74.4550297065478\\
65	0.19914	-78.4815835644046\\
65	0.2028	-82.6303690732554\\
65	0.20646	-86.9013862331001\\
65	0.21012	-91.2946350439387\\
65	0.21378	-95.8101155057712\\
65	0.21744	-100.447827618598\\
65	0.2211	-105.207771382418\\
65	0.22476	-110.089946797232\\
65	0.22842	-115.09435386304\\
65	0.23208	-120.220992579842\\
65	0.23574	-125.469862947639\\
65	0.2394	-130.840964966428\\
65	0.24306	-136.334298636212\\
65	0.24672	-141.94986395699\\
65	0.25038	-147.687660928762\\
65	0.25404	-153.547689551527\\
65	0.2577	-159.529949825287\\
65	0.26136	-165.63444175004\\
65	0.26502	-171.861165325788\\
65	0.26868	-178.210120552529\\
65	0.27234	-184.681307430264\\
65	0.276	-191.274725958993\\
65.375	0.093	-11.4755518757188\\
65.375	0.09666	-12.0823983468333\\
65.375	0.10032	-12.8114764689418\\
65.375	0.10398	-13.6627862420442\\
65.375	0.10764	-14.6363276661405\\
65.375	0.1113	-15.7321007412307\\
65.375	0.11496	-16.9501054673149\\
65.375	0.11862	-18.290341844393\\
65.375	0.12228	-19.7528098724649\\
65.375	0.12594	-21.3375095515309\\
65.375	0.1296	-23.0444408815907\\
65.375	0.13326	-24.8736038626444\\
65.375	0.13692	-26.824998494692\\
65.375	0.14058	-28.8986247777336\\
65.375	0.14424	-31.0944827117691\\
65.375	0.1479	-33.4125722967985\\
65.375	0.15156	-35.8528935328218\\
65.375	0.15522	-38.4154464198391\\
65.375	0.15888	-41.1002309578503\\
65.375	0.16254	-43.9072471468554\\
65.375	0.1662	-46.8364949868543\\
65.375	0.16986	-49.8879744778473\\
65.375	0.17352	-53.0616856198341\\
65.375	0.17718	-56.3576284128148\\
65.375	0.18084	-59.7758028567894\\
65.375	0.1845	-63.3162089517581\\
65.375	0.18816	-66.9788466977206\\
65.375	0.19182	-70.763716094677\\
65.375	0.19548	-74.6708171426273\\
65.375	0.19914	-78.7001498415716\\
65.375	0.2028	-82.8517141915098\\
65.375	0.20646	-87.1255101924419\\
65.375	0.21012	-91.5215378443679\\
65.375	0.21378	-96.0397971472878\\
65.375	0.21744	-100.680288101202\\
65.375	0.2211	-105.443010706109\\
65.375	0.22476	-110.327964962011\\
65.375	0.22842	-115.335150868907\\
65.375	0.23208	-120.464568426796\\
65.375	0.23574	-125.71621763568\\
65.375	0.2394	-131.090098495557\\
65.375	0.24306	-136.586211006428\\
65.375	0.24672	-142.204555168294\\
65.375	0.25038	-147.945130981153\\
65.375	0.25404	-153.807938445006\\
65.375	0.2577	-159.792977559853\\
65.375	0.26136	-165.900248325694\\
65.375	0.26502	-172.129750742528\\
65.375	0.26868	-178.481484810357\\
65.375	0.27234	-184.95545052918\\
65.375	0.276	-191.551647898996\\
65.75	0.093	-11.6231281184703\\
65.75	0.09666	-12.2327534306723\\
65.75	0.10032	-12.9646103938682\\
65.75	0.10398	-13.818699008058\\
65.75	0.10764	-14.7950192732417\\
65.75	0.1113	-15.8935711894194\\
65.75	0.11496	-17.114354756591\\
65.75	0.11862	-18.4573699747564\\
65.75	0.12228	-19.9226168439158\\
65.75	0.12594	-21.5100953640692\\
65.75	0.1296	-23.2198055352164\\
65.75	0.13326	-25.0517473573575\\
65.75	0.13692	-27.0059208304926\\
65.75	0.14058	-29.0823259546216\\
65.75	0.14424	-31.2809627297446\\
65.75	0.1479	-33.6018311558614\\
65.75	0.15156	-36.0449312329721\\
65.75	0.15522	-38.6102629610768\\
65.75	0.15888	-41.2978263401754\\
65.75	0.16254	-44.1076213702679\\
65.75	0.1662	-47.0396480513543\\
65.75	0.16986	-50.0939063834346\\
65.75	0.17352	-53.2703963665089\\
65.75	0.17718	-56.569118000577\\
65.75	0.18084	-59.9900712856391\\
65.75	0.1845	-63.5332562216952\\
65.75	0.18816	-67.1986728087451\\
65.75	0.19182	-70.9863210467889\\
65.75	0.19548	-74.8962009358266\\
65.75	0.19914	-78.9283124758583\\
65.75	0.2028	-83.082655666884\\
65.75	0.20646	-87.3592305089035\\
65.75	0.21012	-91.7580370019169\\
65.75	0.21378	-96.2790751459242\\
65.75	0.21744	-100.922344940926\\
65.75	0.2211	-105.687846386921\\
65.75	0.22476	-110.57557948391\\
65.75	0.22842	-115.585544231893\\
65.75	0.23208	-120.71774063087\\
65.75	0.23574	-125.972168680841\\
65.75	0.2394	-131.348828381805\\
65.75	0.24306	-136.847719733764\\
65.75	0.24672	-142.468842736717\\
65.75	0.25038	-148.212197390663\\
65.75	0.25404	-154.077783695604\\
65.75	0.2577	-160.065601651538\\
65.75	0.26136	-166.175651258467\\
65.75	0.26502	-172.407932516389\\
65.75	0.26868	-178.762445425305\\
65.75	0.27234	-185.239189985215\\
65.75	0.276	-191.838166196119\\
66.125	0.093	-11.7803007183416\\
66.125	0.09666	-12.392704871631\\
66.125	0.10032	-13.1273406759143\\
66.125	0.10398	-13.9842081311916\\
66.125	0.10764	-14.9633072374627\\
66.125	0.1113	-16.0646379947278\\
66.125	0.11496	-17.2882004029868\\
66.125	0.11862	-18.6339944622397\\
66.125	0.12228	-20.1020201724865\\
66.125	0.12594	-21.6922775337273\\
66.125	0.1296	-23.4047665459619\\
66.125	0.13326	-25.2394872091905\\
66.125	0.13692	-27.196439523413\\
66.125	0.14058	-29.2756234886294\\
66.125	0.14424	-31.4770391048398\\
66.125	0.1479	-33.800686372044\\
66.125	0.15156	-36.2465652902422\\
66.125	0.15522	-38.8146758594342\\
66.125	0.15888	-41.5050180796203\\
66.125	0.16254	-44.3175919508002\\
66.125	0.1662	-47.252397472974\\
66.125	0.16986	-50.3094346461418\\
66.125	0.17352	-53.4887034703034\\
66.125	0.17718	-56.790203945459\\
66.125	0.18084	-60.2139360716085\\
66.125	0.1845	-63.759899848752\\
66.125	0.18816	-67.4280952768893\\
66.125	0.19182	-71.2185223560206\\
66.125	0.19548	-75.1311810861458\\
66.125	0.19914	-79.1660714672649\\
66.125	0.2028	-83.3231934993779\\
66.125	0.20646	-87.6025471824849\\
66.125	0.21012	-92.0041325165857\\
66.125	0.21378	-96.5279495016805\\
66.125	0.21744	-101.173998137769\\
66.125	0.2211	-105.942278424852\\
66.125	0.22476	-110.832790362928\\
66.125	0.22842	-115.845533951999\\
66.125	0.23208	-120.980509192063\\
66.125	0.23574	-126.237716083121\\
66.125	0.2394	-131.617154625174\\
66.125	0.24306	-137.11882481822\\
66.125	0.24672	-142.74272666226\\
66.125	0.25038	-148.488860157294\\
66.125	0.25404	-154.357225303322\\
66.125	0.2577	-160.347822100343\\
66.125	0.26136	-166.460650548359\\
66.125	0.26502	-172.695710647369\\
66.125	0.26868	-179.053002397372\\
66.125	0.27234	-185.53252579837\\
66.125	0.276	-192.134280850361\\
66.5	0.093	-11.9470696753328\\
66.5	0.09666	-12.5622526697096\\
66.5	0.10032	-13.2996673150804\\
66.5	0.10398	-14.159313611445\\
66.5	0.10764	-15.1411915588036\\
66.5	0.1113	-16.2453011571561\\
66.5	0.11496	-17.4716424065025\\
66.5	0.11862	-18.8202153068428\\
66.5	0.12228	-20.291019858177\\
66.5	0.12594	-21.8840560605052\\
66.5	0.1296	-23.5993239138273\\
66.5	0.13326	-25.4368234181433\\
66.5	0.13692	-27.3965545734532\\
66.5	0.14058	-29.478517379757\\
66.5	0.14424	-31.6827118370548\\
66.5	0.1479	-34.0091379453465\\
66.5	0.15156	-36.4577957046321\\
66.5	0.15522	-39.0286851149116\\
66.5	0.15888	-41.721806176185\\
66.5	0.16254	-44.5371588884524\\
66.5	0.1662	-47.4747432517136\\
66.5	0.16986	-50.5345592659688\\
66.5	0.17352	-53.7166069312179\\
66.5	0.17718	-57.0208862474609\\
66.5	0.18084	-60.4473972146978\\
66.5	0.1845	-63.9961398329287\\
66.5	0.18816	-67.6671141021534\\
66.5	0.19182	-71.4603200223721\\
66.5	0.19548	-75.3757575935848\\
66.5	0.19914	-79.4134268157912\\
66.5	0.2028	-83.5733276889917\\
66.5	0.20646	-87.8554602131861\\
66.5	0.21012	-92.2598243883744\\
66.5	0.21378	-96.7864202145566\\
66.5	0.21744	-101.435247691733\\
66.5	0.2211	-106.206306819903\\
66.5	0.22476	-111.099597599067\\
66.5	0.22842	-116.115120029225\\
66.5	0.23208	-121.252874110376\\
66.5	0.23574	-126.512859842522\\
66.5	0.2394	-131.895077225662\\
66.5	0.24306	-137.399526259795\\
66.5	0.24672	-143.026206944923\\
66.5	0.25038	-148.775119281044\\
66.5	0.25404	-154.646263268159\\
66.5	0.2577	-160.639638906269\\
66.5	0.26136	-166.755246195372\\
66.5	0.26502	-172.993085135469\\
66.5	0.26868	-179.35315572656\\
66.5	0.27234	-185.835457968645\\
66.5	0.276	-192.439991861723\\
66.875	0.093	-12.1234349894437\\
66.875	0.09666	-12.741396824908\\
66.875	0.10032	-13.4815903113661\\
66.875	0.10398	-14.3440154488182\\
66.875	0.10764	-15.3286722372642\\
66.875	0.1113	-16.4355606767041\\
66.875	0.11496	-17.664680767138\\
66.875	0.11862	-19.0160325085657\\
66.875	0.12228	-20.4896159009873\\
66.875	0.12594	-22.085430944403\\
66.875	0.1296	-23.8034776388125\\
66.875	0.13326	-25.6437559842159\\
66.875	0.13692	-27.6062659806132\\
66.875	0.14058	-29.6910076280045\\
66.875	0.14424	-31.8979809263897\\
66.875	0.1479	-34.2271858757688\\
66.875	0.15156	-36.6786224761418\\
66.875	0.15522	-39.2522907275087\\
66.875	0.15888	-41.9481906298696\\
66.875	0.16254	-44.7663221832243\\
66.875	0.1662	-47.706685387573\\
66.875	0.16986	-50.7692802429156\\
66.875	0.17352	-53.9541067492521\\
66.875	0.17718	-57.2611649065825\\
66.875	0.18084	-60.6904547149068\\
66.875	0.1845	-64.2419761742252\\
66.875	0.18816	-67.9157292845374\\
66.875	0.19182	-71.7117140458435\\
66.875	0.19548	-75.6299304581435\\
66.875	0.19914	-79.6703785214374\\
66.875	0.2028	-83.8330582357253\\
66.875	0.20646	-88.1179696010071\\
66.875	0.21012	-92.5251126172828\\
66.875	0.21378	-97.0544872845524\\
66.875	0.21744	-101.706093602816\\
66.875	0.2211	-106.479931572073\\
66.875	0.22476	-111.376001192325\\
66.875	0.22842	-116.39430246357\\
66.875	0.23208	-121.534835385809\\
66.875	0.23574	-126.797599959042\\
66.875	0.2394	-132.182596183269\\
66.875	0.24306	-137.68982405849\\
66.875	0.24672	-143.319283584705\\
66.875	0.25038	-149.070974761914\\
66.875	0.25404	-154.944897590117\\
66.875	0.2577	-160.941052069313\\
66.875	0.26136	-167.059438199504\\
66.875	0.26502	-173.300055980688\\
66.875	0.26868	-179.662905412867\\
66.875	0.27234	-186.147986496039\\
66.875	0.276	-192.755299230205\\
67.25	0.093	-12.3093966606745\\
67.25	0.09666	-12.9301373372262\\
67.25	0.10032	-13.6731096647718\\
67.25	0.10398	-14.5383136433112\\
67.25	0.10764	-15.5257492728446\\
67.25	0.1113	-16.635416553372\\
67.25	0.11496	-17.8673154848933\\
67.25	0.11862	-19.2214460674084\\
67.25	0.12228	-20.6978083009175\\
67.25	0.12594	-22.2964021854205\\
67.25	0.1296	-24.0172277209174\\
67.25	0.13326	-25.8602849074083\\
67.25	0.13692	-27.8255737448931\\
67.25	0.14058	-29.9130942333717\\
67.25	0.14424	-32.1228463728443\\
67.25	0.1479	-34.4548301633108\\
67.25	0.15156	-36.9090456047713\\
67.25	0.15522	-39.4854926972256\\
67.25	0.15888	-42.1841714406739\\
67.25	0.16254	-45.0050818351161\\
67.25	0.1662	-47.9482238805522\\
67.25	0.16986	-51.0135975769822\\
67.25	0.17352	-54.2012029244061\\
67.25	0.17718	-57.511039922824\\
67.25	0.18084	-60.9431085722357\\
67.25	0.1845	-64.4974088726415\\
67.25	0.18816	-68.1739408240411\\
67.25	0.19182	-71.9727044264346\\
67.25	0.19548	-75.8936996798221\\
67.25	0.19914	-79.9369265842034\\
67.25	0.2028	-84.1023851395787\\
67.25	0.20646	-88.390075345948\\
67.25	0.21012	-92.7999972033111\\
67.25	0.21378	-97.3321507116681\\
67.25	0.21744	-101.986535871019\\
67.25	0.2211	-106.763152681364\\
67.25	0.22476	-111.662001142703\\
67.25	0.22842	-116.683081255035\\
67.25	0.23208	-121.826393018362\\
67.25	0.23574	-127.091936432683\\
67.25	0.2394	-132.479711497997\\
67.25	0.24306	-137.989718214306\\
67.25	0.24672	-143.621956581608\\
67.25	0.25038	-149.376426599904\\
67.25	0.25404	-155.253128269194\\
67.25	0.2577	-161.252061589478\\
67.25	0.26136	-167.373226560756\\
67.25	0.26502	-173.616623183028\\
67.25	0.26868	-179.982251456294\\
67.25	0.27234	-186.470111380554\\
67.25	0.276	-193.080202955807\\
67.625	0.093	-12.504954689025\\
67.625	0.09666	-13.1284742066641\\
67.625	0.10032	-13.8742253752971\\
67.625	0.10398	-14.742208194924\\
67.625	0.10764	-15.7324226655449\\
67.625	0.1113	-16.8448687871596\\
67.625	0.11496	-18.0795465597683\\
67.625	0.11862	-19.4364559833709\\
67.625	0.12228	-20.9155970579674\\
67.625	0.12594	-22.5169697835579\\
67.625	0.1296	-24.2405741601422\\
67.625	0.13326	-26.0864101877205\\
67.625	0.13692	-28.0544778662927\\
67.625	0.14058	-30.1447771958587\\
67.625	0.14424	-32.3573081764188\\
67.625	0.1479	-34.6920708079727\\
67.625	0.15156	-37.1490650905206\\
67.625	0.15522	-39.7282910240624\\
67.625	0.15888	-42.4297486085981\\
67.625	0.16254	-45.2534378441277\\
67.625	0.1662	-48.1993587306512\\
67.625	0.16986	-51.2675112681686\\
67.625	0.17352	-54.45789545668\\
67.625	0.17718	-57.7705112961852\\
67.625	0.18084	-61.2053587866844\\
67.625	0.1845	-64.7624379281776\\
67.625	0.18816	-68.4417487206646\\
67.625	0.19182	-72.2432911641455\\
67.625	0.19548	-76.1670652586204\\
67.625	0.19914	-80.2130710040892\\
67.625	0.2028	-84.381308400552\\
67.625	0.20646	-88.6717774480086\\
67.625	0.21012	-93.0844781464591\\
67.625	0.21378	-97.6194104959036\\
67.625	0.21744	-102.276574496342\\
67.625	0.2211	-107.055970147774\\
67.625	0.22476	-111.957597450201\\
67.625	0.22842	-116.981456403621\\
67.625	0.23208	-122.127547008035\\
67.625	0.23574	-127.395869263443\\
67.625	0.2394	-132.786423169845\\
67.625	0.24306	-138.29920872724\\
67.625	0.24672	-143.93422593563\\
67.625	0.25038	-149.691474795014\\
67.625	0.25404	-155.570955305391\\
67.625	0.2577	-161.572667466763\\
67.625	0.26136	-167.696611279128\\
67.625	0.26502	-173.942786742487\\
67.625	0.26868	-180.311193856841\\
67.625	0.27234	-186.801832622188\\
67.625	0.276	-193.414703038529\\
68	0.093	-12.7101090744954\\
68	0.09666	-13.3364074332219\\
68	0.10032	-14.0849374429423\\
68	0.10398	-14.9556991036567\\
68	0.10764	-15.9486924153649\\
68	0.1113	-17.0639173780671\\
68	0.11496	-18.3013739917632\\
68	0.11862	-19.6610622564532\\
68	0.12228	-21.1429821721372\\
68	0.12594	-22.747133738815\\
68	0.1296	-24.4735169564868\\
68	0.13326	-26.3221318251525\\
68	0.13692	-28.2929783448121\\
68	0.14058	-30.3860565154656\\
68	0.14424	-32.6013663371131\\
68	0.1479	-34.9389078097544\\
68	0.15156	-37.3986809333897\\
68	0.15522	-39.9806857080189\\
68	0.15888	-42.684922133642\\
68	0.16254	-45.5113902102591\\
68	0.1662	-48.46008993787\\
68	0.16986	-51.5310213164748\\
68	0.17352	-54.7241843460736\\
68	0.17718	-58.0395790266663\\
68	0.18084	-61.4772053582529\\
68	0.1845	-65.0370633408335\\
68	0.18816	-68.719152974408\\
68	0.19182	-72.5234742589763\\
68	0.19548	-76.4500271945386\\
68	0.19914	-80.4988117810948\\
68	0.2028	-84.669828018645\\
68	0.20646	-88.9630759071891\\
68	0.21012	-93.378555446727\\
68	0.21378	-97.9162666372589\\
68	0.21744	-102.576209478785\\
68	0.2211	-107.358383971304\\
68	0.22476	-112.262790114818\\
68	0.22842	-117.289427909326\\
68	0.23208	-122.438297354827\\
68	0.23574	-127.709398451322\\
68	0.2394	-133.102731198812\\
68	0.24306	-138.618295597295\\
68	0.24672	-144.256091646772\\
68	0.25038	-150.016119347243\\
68	0.25404	-155.898378698708\\
68	0.2577	-161.902869701167\\
68	0.26136	-168.02959235462\\
68	0.26502	-174.278546659067\\
68	0.26868	-180.649732614507\\
68	0.27234	-187.143150220942\\
68	0.276	-193.75879947837\\
68.375	0.093	-12.9248598170856\\
68.375	0.09666	-13.5539370168995\\
68.375	0.10032	-14.3052458677074\\
68.375	0.10398	-15.1787863695091\\
68.375	0.10764	-16.1745585223048\\
68.375	0.1113	-17.2925623260944\\
68.375	0.11496	-18.532797780878\\
68.375	0.11862	-19.8952648866554\\
68.375	0.12228	-21.3799636434267\\
68.375	0.12594	-22.986894051192\\
68.375	0.1296	-24.7160561099512\\
68.375	0.13326	-26.5674498197043\\
68.375	0.13692	-28.5410751804513\\
68.375	0.14058	-30.6369321921923\\
68.375	0.14424	-32.8550208549272\\
68.375	0.1479	-35.1953411686559\\
68.375	0.15156	-37.6578931333786\\
68.375	0.15522	-40.2426767490953\\
68.375	0.15888	-42.9496920158058\\
68.375	0.16254	-45.7789389335102\\
68.375	0.1662	-48.7304175022086\\
68.375	0.16986	-51.8041277219009\\
68.375	0.17352	-55.0000695925871\\
68.375	0.17718	-58.3182431142672\\
68.375	0.18084	-61.7586482869412\\
68.375	0.1845	-65.3212851106092\\
68.375	0.18816	-69.0061535852711\\
68.375	0.19182	-72.8132537109269\\
68.375	0.19548	-76.7425854875766\\
68.375	0.19914	-80.7941489152202\\
68.375	0.2028	-84.9679439938578\\
68.375	0.20646	-89.2639707234893\\
68.375	0.21012	-93.6822291041147\\
68.375	0.21378	-98.2227191357339\\
68.375	0.21744	-102.885440818347\\
68.375	0.2211	-107.670394151954\\
68.375	0.22476	-112.577579136555\\
68.375	0.22842	-117.60699577215\\
68.375	0.23208	-122.758644058739\\
68.375	0.23574	-128.032523996322\\
68.375	0.2394	-133.428635584899\\
68.375	0.24306	-138.94697882447\\
68.375	0.24672	-144.587553715034\\
68.375	0.25038	-150.350360256593\\
68.375	0.25404	-156.235398449145\\
68.375	0.2577	-162.242668292691\\
68.375	0.26136	-168.372169787232\\
68.375	0.26502	-174.623902932766\\
68.375	0.26868	-180.997867729294\\
68.375	0.27234	-187.494064176816\\
68.375	0.276	-194.112492275332\\
68.75	0.093	-13.1492069167956\\
68.75	0.09666	-13.781062957697\\
68.75	0.10032	-14.5351506495922\\
68.75	0.10398	-15.4114699924814\\
68.75	0.10764	-16.4100209863645\\
68.75	0.1113	-17.5308036312415\\
68.75	0.11496	-18.7738179271125\\
68.75	0.11862	-20.1390638739773\\
68.75	0.12228	-21.6265414718361\\
68.75	0.12594	-23.2362507206888\\
68.75	0.1296	-24.9681916205354\\
68.75	0.13326	-26.822364171376\\
68.75	0.13692	-28.7987683732104\\
68.75	0.14058	-30.8974042260388\\
68.75	0.14424	-33.1182717298611\\
68.75	0.1479	-35.4613708846773\\
68.75	0.15156	-37.9267016904874\\
68.75	0.15522	-40.5142641472914\\
68.75	0.15888	-43.2240582550894\\
68.75	0.16254	-46.0560840138812\\
68.75	0.1662	-49.0103414236671\\
68.75	0.16986	-52.0868304844468\\
68.75	0.17352	-55.2855511962204\\
68.75	0.17718	-58.6065035589879\\
68.75	0.18084	-62.0496875727493\\
68.75	0.1845	-65.6151032375048\\
68.75	0.18816	-69.3027505532541\\
68.75	0.19182	-73.1126295199973\\
68.75	0.19548	-77.0447401377344\\
68.75	0.19914	-81.0990824064655\\
68.75	0.2028	-85.2756563261904\\
68.75	0.20646	-89.5744618969094\\
68.75	0.21012	-93.9954991186222\\
68.75	0.21378	-98.5387679913289\\
68.75	0.21744	-103.20426851503\\
68.75	0.2211	-107.992000689724\\
68.75	0.22476	-112.901964515413\\
68.75	0.22842	-117.934159992095\\
68.75	0.23208	-123.088587119771\\
68.75	0.23574	-128.365245898442\\
68.75	0.2394	-133.764136328106\\
68.75	0.24306	-139.285258408764\\
68.75	0.24672	-144.928612140416\\
68.75	0.25038	-150.694197523062\\
68.75	0.25404	-156.582014556701\\
68.75	0.2577	-162.592063241335\\
68.75	0.26136	-168.724343576963\\
68.75	0.26502	-174.978855563584\\
68.75	0.26868	-181.3555992012\\
68.75	0.27234	-187.854574489809\\
68.75	0.276	-194.475781429413\\
69.125	0.093	-13.3831503736254\\
69.125	0.09666	-14.0177852556142\\
69.125	0.10032	-14.7746517885969\\
69.125	0.10398	-15.6537499725735\\
69.125	0.10764	-16.655079807544\\
69.125	0.1113	-17.7786412935085\\
69.125	0.11496	-19.0244344304668\\
69.125	0.11862	-20.3924592184191\\
69.125	0.12228	-21.8827156573653\\
69.125	0.12594	-23.4952037473054\\
69.125	0.1296	-25.2299234882395\\
69.125	0.13326	-27.0868748801674\\
69.125	0.13692	-29.0660579230893\\
69.125	0.14058	-31.167472617005\\
69.125	0.14424	-33.3911189619148\\
69.125	0.1479	-35.7369969578184\\
69.125	0.15156	-38.2051066047159\\
69.125	0.15522	-40.7954479026074\\
69.125	0.15888	-43.5080208514928\\
69.125	0.16254	-46.3428254513721\\
69.125	0.1662	-49.2998617022453\\
69.125	0.16986	-52.3791296041124\\
69.125	0.17352	-55.5806291569735\\
69.125	0.17718	-58.9043603608284\\
69.125	0.18084	-62.3503232156773\\
69.125	0.1845	-65.9185177215201\\
69.125	0.18816	-69.6089438783569\\
69.125	0.19182	-73.4216016861875\\
69.125	0.19548	-77.356491145012\\
69.125	0.19914	-81.4136122548305\\
69.125	0.2028	-85.5929650156429\\
69.125	0.20646	-89.8945494274492\\
69.125	0.21012	-94.3183654902495\\
69.125	0.21378	-98.8644132040436\\
69.125	0.21744	-103.532692568832\\
69.125	0.2211	-108.323203584614\\
69.125	0.22476	-113.23594625139\\
69.125	0.22842	-118.270920569159\\
69.125	0.23208	-123.428126537923\\
69.125	0.23574	-128.707564157681\\
69.125	0.2394	-134.109233428432\\
69.125	0.24306	-139.633134350178\\
69.125	0.24672	-145.279266922917\\
69.125	0.25038	-151.047631146651\\
69.125	0.25404	-156.938227021378\\
69.125	0.2577	-162.951054547099\\
69.125	0.26136	-169.086113723814\\
69.125	0.26502	-175.343404551523\\
69.125	0.26868	-181.722927030226\\
69.125	0.27234	-188.224681159923\\
69.125	0.276	-194.848666940614\\
69.5	0.093	-13.626690187575\\
69.5	0.09666	-14.2641039106512\\
69.5	0.10032	-15.0237492847213\\
69.5	0.10398	-15.9056263097854\\
69.5	0.10764	-16.9097349858433\\
69.5	0.1113	-18.0360753128952\\
69.5	0.11496	-19.284647290941\\
69.5	0.11862	-20.6554509199807\\
69.5	0.12228	-22.1484862000143\\
69.5	0.12594	-23.7637531310418\\
69.5	0.1296	-25.5012517130633\\
69.5	0.13326	-27.3609819460787\\
69.5	0.13692	-29.342943830088\\
69.5	0.14058	-31.4471373650911\\
69.5	0.14424	-33.6735625510883\\
69.5	0.1479	-36.0222193880793\\
69.5	0.15156	-38.4931078760643\\
69.5	0.15522	-41.0862280150432\\
69.5	0.15888	-43.801579805016\\
69.5	0.16254	-46.6391632459827\\
69.5	0.1662	-49.5989783379433\\
69.5	0.16986	-52.6810250808979\\
69.5	0.17352	-55.8853034748464\\
69.5	0.17718	-59.2118135197887\\
69.5	0.18084	-62.660555215725\\
69.5	0.1845	-66.2315285626553\\
69.5	0.18816	-69.9247335605795\\
69.5	0.19182	-73.7401702094975\\
69.5	0.19548	-77.6778385094095\\
69.5	0.19914	-81.7377384603153\\
69.5	0.2028	-85.9198700622152\\
69.5	0.20646	-90.2242333151089\\
69.5	0.21012	-94.6508282189966\\
69.5	0.21378	-99.1996547738782\\
69.5	0.21744	-103.870712979754\\
69.5	0.2211	-108.664002836623\\
69.5	0.22476	-113.579524344486\\
69.5	0.22842	-118.617277503344\\
69.5	0.23208	-123.777262313195\\
69.5	0.23574	-129.05947877404\\
69.5	0.2394	-134.463926885879\\
69.5	0.24306	-139.990606648712\\
69.5	0.24672	-145.639518062539\\
69.5	0.25038	-151.410661127359\\
69.5	0.25404	-157.304035843174\\
69.5	0.2577	-163.319642209983\\
69.5	0.26136	-169.457480227785\\
69.5	0.26502	-175.717549896582\\
69.5	0.26868	-182.099851216372\\
69.5	0.27234	-188.604384187156\\
69.5	0.276	-195.231148808934\\
69.875	0.093	-13.8798263586445\\
69.875	0.09666	-14.520018922808\\
69.875	0.10032	-15.2824431379656\\
69.875	0.10398	-16.167099004117\\
69.875	0.10764	-17.1739865212624\\
69.875	0.1113	-18.3031056894017\\
69.875	0.11496	-19.5544565085349\\
69.875	0.11862	-20.928038978662\\
69.875	0.12228	-22.4238530997831\\
69.875	0.12594	-24.041898871898\\
69.875	0.1296	-25.7821762950069\\
69.875	0.13326	-27.6446853691097\\
69.875	0.13692	-29.6294260942064\\
69.875	0.14058	-31.736398470297\\
69.875	0.14424	-33.9656024973816\\
69.875	0.1479	-36.3170381754601\\
69.875	0.15156	-38.7907055045325\\
69.875	0.15522	-41.3866044845987\\
69.875	0.15888	-44.104735115659\\
69.875	0.16254	-46.9450973977131\\
69.875	0.1662	-49.9076913307612\\
69.875	0.16986	-52.9925169148031\\
69.875	0.17352	-56.199574149839\\
69.875	0.17718	-59.5288630358688\\
69.875	0.18084	-62.9803835728925\\
69.875	0.1845	-66.5541357609102\\
69.875	0.18816	-70.2501195999218\\
69.875	0.19182	-74.0683350899273\\
69.875	0.19548	-78.0087822309267\\
69.875	0.19914	-82.0714610229199\\
69.875	0.2028	-86.2563714659073\\
69.875	0.20646	-90.5635135598884\\
69.875	0.21012	-94.9928873048635\\
69.875	0.21378	-99.5444927008324\\
69.875	0.21744	-104.218329747795\\
69.875	0.2211	-109.014398445752\\
69.875	0.22476	-113.932698794703\\
69.875	0.22842	-118.973230794648\\
69.875	0.23208	-124.135994445586\\
69.875	0.23574	-129.420989747519\\
69.875	0.2394	-134.828216700445\\
69.875	0.24306	-140.357675304366\\
69.875	0.24672	-146.00936555928\\
69.875	0.25038	-151.783287465188\\
69.875	0.25404	-157.67944102209\\
69.875	0.2577	-163.697826229986\\
69.875	0.26136	-169.838443088876\\
69.875	0.26502	-176.10129159876\\
69.875	0.26868	-182.486371759638\\
69.875	0.27234	-188.993683571509\\
69.875	0.276	-195.623227034375\\
70.25	0.093	-14.1425588868337\\
70.25	0.09666	-14.7855302920847\\
70.25	0.10032	-15.5507333483297\\
70.25	0.10398	-16.4381680555685\\
70.25	0.10764	-17.4478344138013\\
70.25	0.1113	-18.579732423028\\
70.25	0.11496	-19.8338620832487\\
70.25	0.11862	-21.2102233944632\\
70.25	0.12228	-22.7088163566717\\
70.25	0.12594	-24.3296409698741\\
70.25	0.1296	-26.0726972340703\\
70.25	0.13326	-27.9379851492606\\
70.25	0.13692	-29.9255047154447\\
70.25	0.14058	-32.0352559326227\\
70.25	0.14424	-34.2672388007947\\
70.25	0.1479	-36.6214533199606\\
70.25	0.15156	-39.0978994901204\\
70.25	0.15522	-41.6965773112741\\
70.25	0.15888	-44.4174867834218\\
70.25	0.16254	-47.2606279065634\\
70.25	0.1662	-50.2260006806988\\
70.25	0.16986	-53.3136051058282\\
70.25	0.17352	-56.5234411819515\\
70.25	0.17718	-59.8555089090687\\
70.25	0.18084	-63.3098082871799\\
70.25	0.1845	-66.886339316285\\
70.25	0.18816	-70.585101996384\\
70.25	0.19182	-74.4060963274769\\
70.25	0.19548	-78.3493223095637\\
70.25	0.19914	-82.4147799426444\\
70.25	0.2028	-86.6024692267191\\
70.25	0.20646	-90.9123901617878\\
70.25	0.21012	-95.3445427478502\\
70.25	0.21378	-99.8989269849066\\
70.25	0.21744	-104.575542872957\\
70.25	0.2211	-109.374390412001\\
70.25	0.22476	-114.295469602039\\
70.25	0.22842	-119.338780443071\\
70.25	0.23208	-124.504322935097\\
70.25	0.23574	-129.792097078117\\
70.25	0.2394	-135.202102872131\\
70.25	0.24306	-140.734340317139\\
70.25	0.24672	-146.388809413141\\
70.25	0.25038	-152.165510160136\\
70.25	0.25404	-158.064442558126\\
70.25	0.2577	-164.085606607109\\
70.25	0.26136	-170.229002307087\\
70.25	0.26502	-176.494629658058\\
70.25	0.26868	-182.882488660023\\
70.25	0.27234	-189.392579312982\\
70.25	0.276	-196.024901616935\\
70.625	0.093	-14.4148877721427\\
70.625	0.09666	-15.0606380184812\\
70.625	0.10032	-15.8286199158135\\
70.625	0.10398	-16.7188334641398\\
70.625	0.10764	-17.73127866346\\
70.625	0.1113	-18.8659555137742\\
70.625	0.11496	-20.1228640150822\\
70.625	0.11862	-21.5020041673842\\
70.625	0.12228	-23.0033759706801\\
70.625	0.12594	-24.6269794249699\\
70.625	0.1296	-26.3728145302536\\
70.625	0.13326	-28.2408812865312\\
70.625	0.13692	-30.2311796938028\\
70.625	0.14058	-32.3437097520683\\
70.625	0.14424	-34.5784714613277\\
70.625	0.1479	-36.935464821581\\
70.625	0.15156	-39.4146898328282\\
70.625	0.15522	-42.0161464950693\\
70.625	0.15888	-44.7398348083044\\
70.625	0.16254	-47.5857547725334\\
70.625	0.1662	-50.5539063877563\\
70.625	0.16986	-53.6442896539731\\
70.625	0.17352	-56.8569045711839\\
70.625	0.17718	-60.1917511393885\\
70.625	0.18084	-63.6488293585871\\
70.625	0.1845	-67.2281392287796\\
70.625	0.18816	-70.929680749966\\
70.625	0.19182	-74.7534539221463\\
70.625	0.19548	-78.6994587453206\\
70.625	0.19914	-82.7676952194887\\
70.625	0.2028	-86.9581633446508\\
70.625	0.20646	-91.2708631208069\\
70.625	0.21012	-95.7057945479567\\
70.625	0.21378	-100.262957626101\\
70.625	0.21744	-104.942352355238\\
70.625	0.2211	-109.74397873537\\
70.625	0.22476	-114.667836766496\\
70.625	0.22842	-119.713926448615\\
70.625	0.23208	-124.882247781728\\
70.625	0.23574	-130.172800765836\\
70.625	0.2394	-135.585585400937\\
70.625	0.24306	-141.120601687032\\
70.625	0.24672	-146.777849624121\\
70.625	0.25038	-152.557329212204\\
70.625	0.25404	-158.459040451281\\
70.625	0.2577	-164.482983341352\\
70.625	0.26136	-170.629157882417\\
70.625	0.26502	-176.897564074476\\
70.625	0.26868	-183.288201917528\\
70.625	0.27234	-189.801071411575\\
70.625	0.276	-196.436172556615\\
71	0.093	-14.6968130145715\\
71	0.09666	-15.3453421019974\\
71	0.10032	-16.1161028404172\\
71	0.10398	-17.0090952298309\\
71	0.10764	-18.0243192702386\\
71	0.1113	-19.1617749616401\\
71	0.11496	-20.4214623040356\\
71	0.11862	-21.803381297425\\
71	0.12228	-23.3075319418083\\
71	0.12594	-24.9339142371855\\
71	0.1296	-26.6825281835567\\
71	0.13326	-28.5533737809217\\
71	0.13692	-30.5464510292807\\
71	0.14058	-32.6617599286336\\
71	0.14424	-34.8993004789804\\
71	0.1479	-37.2590726803212\\
71	0.15156	-39.7410765326558\\
71	0.15522	-42.3453120359843\\
71	0.15888	-45.0717791903069\\
71	0.16254	-47.9204779956233\\
71	0.1662	-50.8914084519336\\
71	0.16986	-53.9845705592378\\
71	0.17352	-57.199964317536\\
71	0.17718	-60.537589726828\\
71	0.18084	-63.997446787114\\
71	0.1845	-67.579535498394\\
71	0.18816	-71.2838558606678\\
71	0.19182	-75.1104078739355\\
71	0.19548	-79.0591915381972\\
71	0.19914	-83.1302068534528\\
71	0.2028	-87.3234538197023\\
71	0.20646	-91.6389324369458\\
71	0.21012	-96.0766427051831\\
71	0.21378	-100.636584624414\\
71	0.21744	-105.318758194639\\
71	0.2211	-110.123163415859\\
71	0.22476	-115.049800288072\\
71	0.22842	-120.098668811279\\
71	0.23208	-125.269768985479\\
71	0.23574	-130.563100810674\\
71	0.2394	-135.978664286863\\
71	0.24306	-141.516459414045\\
71	0.24672	-147.176486192222\\
71	0.25038	-152.958744621392\\
71	0.25404	-158.863234701557\\
71	0.2577	-164.889956432715\\
71	0.26136	-171.038909814867\\
71	0.26502	-177.310094848013\\
71	0.26868	-183.703511532153\\
71	0.27234	-190.219159867287\\
71	0.276	-196.857039853415\\
71.375	0.093	-14.9883346141203\\
71.375	0.09666	-15.6396425426336\\
71.375	0.10032	-16.4131821221408\\
71.375	0.10398	-17.3089533526419\\
71.375	0.10764	-18.326956234137\\
71.375	0.1113	-19.467190766626\\
71.375	0.11496	-20.7296569501089\\
71.375	0.11862	-22.1143547845857\\
71.375	0.12228	-23.6212842700564\\
71.375	0.12594	-25.2504454065211\\
71.375	0.1296	-27.0018381939796\\
71.375	0.13326	-28.8754626324321\\
71.375	0.13692	-30.8713187218785\\
71.375	0.14058	-32.9894064623188\\
71.375	0.14424	-35.2297258537531\\
71.375	0.1479	-37.5922768961812\\
71.375	0.15156	-40.0770595896033\\
71.375	0.15522	-42.6840739340193\\
71.375	0.15888	-45.4133199294292\\
71.375	0.16254	-48.2647975758331\\
71.375	0.1662	-51.2385068732308\\
71.375	0.16986	-54.3344478216225\\
71.375	0.17352	-57.552620421008\\
71.375	0.17718	-60.8930246713875\\
71.375	0.18084	-64.3556605727609\\
71.375	0.1845	-67.9405281251283\\
71.375	0.18816	-71.6476273284895\\
71.375	0.19182	-75.4769581828447\\
71.375	0.19548	-79.4285206881938\\
71.375	0.19914	-83.5023148445368\\
71.375	0.2028	-87.6983406518738\\
71.375	0.20646	-92.0165981102046\\
71.375	0.21012	-96.4570872195294\\
71.375	0.21378	-101.019807979848\\
71.375	0.21744	-105.704760391161\\
71.375	0.2211	-110.511944453467\\
71.375	0.22476	-115.441360166768\\
71.375	0.22842	-120.493007531062\\
71.375	0.23208	-125.66688654635\\
71.375	0.23574	-130.962997212632\\
71.375	0.2394	-136.381339529909\\
71.375	0.24306	-141.921913498178\\
71.375	0.24672	-147.584719117442\\
71.375	0.25038	-153.3697563877\\
71.375	0.25404	-159.277025308952\\
71.375	0.2577	-165.306525881198\\
71.375	0.26136	-171.458258104437\\
71.375	0.26502	-177.732221978671\\
71.375	0.26868	-184.128417503898\\
71.375	0.27234	-190.64684468012\\
71.375	0.276	-197.287503507335\\
71.75	0.093	-15.2894525707887\\
71.75	0.09666	-15.9435393403894\\
71.75	0.10032	-16.7198577609841\\
71.75	0.10398	-17.6184078325726\\
71.75	0.10764	-18.6391895551551\\
71.75	0.1113	-19.7822029287315\\
71.75	0.11496	-21.0474479533019\\
71.75	0.11862	-22.4349246288661\\
71.75	0.12228	-23.9446329554242\\
71.75	0.12594	-25.5765729329763\\
71.75	0.1296	-27.3307445615223\\
71.75	0.13326	-29.2071478410622\\
71.75	0.13692	-31.205782771596\\
71.75	0.14058	-33.3266493531237\\
71.75	0.14424	-35.5697475856454\\
71.75	0.1479	-37.935077469161\\
71.75	0.15156	-40.4226390036705\\
71.75	0.15522	-43.0324321891739\\
71.75	0.15888	-45.7644570256713\\
71.75	0.16254	-48.6187135131625\\
71.75	0.1662	-51.5952016516477\\
71.75	0.16986	-54.6939214411267\\
71.75	0.17352	-57.9148728815998\\
71.75	0.17718	-61.2580559730667\\
71.75	0.18084	-64.7234707155275\\
71.75	0.1845	-68.3111171089823\\
71.75	0.18816	-72.020995153431\\
71.75	0.19182	-75.8531048488735\\
71.75	0.19548	-79.80744619531\\
71.75	0.19914	-83.8840191927405\\
71.75	0.2028	-88.0828238411648\\
71.75	0.20646	-92.4038601405831\\
71.75	0.21012	-96.8471280909953\\
71.75	0.21378	-101.412627692401\\
71.75	0.21744	-106.100358944801\\
71.75	0.2211	-110.910321848195\\
71.75	0.22476	-115.842516402583\\
71.75	0.22842	-120.896942607965\\
71.75	0.23208	-126.073600464341\\
71.75	0.23574	-131.37248997171\\
71.75	0.2394	-136.793611130074\\
71.75	0.24306	-142.336963939431\\
71.75	0.24672	-148.002548399783\\
71.75	0.25038	-153.790364511128\\
71.75	0.25404	-159.700412273467\\
71.75	0.2577	-165.7326916868\\
71.75	0.26136	-171.887202751127\\
71.75	0.26502	-178.163945466448\\
71.75	0.26868	-184.562919832763\\
71.75	0.27234	-191.084125850072\\
71.75	0.276	-197.727563518375\\
72.125	0.093	-15.600166884577\\
72.125	0.09666	-16.2570324952651\\
72.125	0.10032	-17.0361297569472\\
72.125	0.10398	-17.9374586696232\\
72.125	0.10764	-18.9610192332931\\
72.125	0.1113	-20.1068114479569\\
72.125	0.11496	-21.3748353136147\\
72.125	0.11862	-22.7650908302663\\
72.125	0.12228	-24.2775779979119\\
72.125	0.12594	-25.9122968165514\\
72.125	0.1296	-27.6692472861848\\
72.125	0.13326	-29.5484294068121\\
72.125	0.13692	-31.5498431784333\\
72.125	0.14058	-33.6734886010485\\
72.125	0.14424	-35.9193656746576\\
72.125	0.1479	-38.2874743992606\\
72.125	0.15156	-40.7778147748575\\
72.125	0.15522	-43.3903868014484\\
72.125	0.15888	-46.1251904790332\\
72.125	0.16254	-48.9822258076118\\
72.125	0.1662	-51.9614927871844\\
72.125	0.16986	-55.0629914177509\\
72.125	0.17352	-58.2867216993113\\
72.125	0.17718	-61.6326836318657\\
72.125	0.18084	-65.1008772154139\\
72.125	0.1845	-68.6913024499561\\
72.125	0.18816	-72.4039593354922\\
72.125	0.19182	-76.2388478720222\\
72.125	0.19548	-80.1959680595461\\
72.125	0.19914	-84.275319898064\\
72.125	0.2028	-88.4769033875758\\
72.125	0.20646	-92.8007185280815\\
72.125	0.21012	-97.2467653195811\\
72.125	0.21378	-101.815043762075\\
72.125	0.21744	-106.505553855562\\
72.125	0.2211	-111.318295600043\\
72.125	0.22476	-116.253268995519\\
72.125	0.22842	-121.310474041988\\
72.125	0.23208	-126.489910739451\\
72.125	0.23574	-131.791579087908\\
72.125	0.2394	-137.215479087359\\
72.125	0.24306	-142.761610737804\\
72.125	0.24672	-148.429974039243\\
72.125	0.25038	-154.220568991675\\
72.125	0.25404	-160.133395595102\\
72.125	0.2577	-166.168453849523\\
72.125	0.26136	-172.325743754937\\
72.125	0.26502	-178.605265311345\\
72.125	0.26868	-185.007018518748\\
72.125	0.27234	-191.531003377144\\
72.125	0.276	-198.177219886534\\
72.5	0.093	-15.9204775554851\\
72.5	0.09666	-16.5801220072606\\
72.5	0.10032	-17.3619981100301\\
72.5	0.10398	-18.2661058637935\\
72.5	0.10764	-19.2924452685509\\
72.5	0.1113	-20.4410163243021\\
72.5	0.11496	-21.7118190310473\\
72.5	0.11862	-23.1048533887863\\
72.5	0.12228	-24.6201193975193\\
72.5	0.12594	-26.2576170572463\\
72.5	0.1296	-28.0173463679671\\
72.5	0.13326	-29.8993073296818\\
72.5	0.13692	-31.9034999423905\\
72.5	0.14058	-34.0299242060931\\
72.5	0.14424	-36.2785801207896\\
72.5	0.1479	-38.64946768648\\
72.5	0.15156	-41.1425869031644\\
72.5	0.15522	-43.7579377708426\\
72.5	0.15888	-46.4955202895148\\
72.5	0.16254	-49.3553344591809\\
72.5	0.1662	-52.3373802798409\\
72.5	0.16986	-55.4416577514948\\
72.5	0.17352	-58.6681668741427\\
72.5	0.17718	-62.0169076477844\\
72.5	0.18084	-65.4878800724201\\
72.5	0.1845	-69.0810841480497\\
72.5	0.18816	-72.7965198746733\\
72.5	0.19182	-76.6341872522906\\
72.5	0.19548	-80.594086280902\\
72.5	0.19914	-84.6762169605073\\
72.5	0.2028	-88.8805792911065\\
72.5	0.20646	-93.2071732726996\\
72.5	0.21012	-97.6559989052867\\
72.5	0.21378	-102.227056188868\\
72.5	0.21744	-106.920345123442\\
72.5	0.2211	-111.735865709011\\
72.5	0.22476	-116.673617945574\\
72.5	0.22842	-121.733601833131\\
72.5	0.23208	-126.915817371681\\
72.5	0.23574	-132.220264561226\\
72.5	0.2394	-137.646943401764\\
72.5	0.24306	-143.195853893296\\
72.5	0.24672	-148.866996035822\\
72.5	0.25038	-154.660369829343\\
72.5	0.25404	-160.575975273857\\
72.5	0.2577	-166.613812369365\\
72.5	0.26136	-172.773881115866\\
72.5	0.26502	-179.056181513362\\
72.5	0.26868	-185.460713561852\\
72.5	0.27234	-191.987477261336\\
72.5	0.276	-198.636472611813\\
72.875	0.093	-16.250384583513\\
72.875	0.09666	-16.912807876376\\
72.875	0.10032	-17.6974628202329\\
72.875	0.10398	-18.6043494150837\\
72.875	0.10764	-19.6334676609285\\
72.875	0.1113	-20.7848175577671\\
72.875	0.11496	-22.0583991055997\\
72.875	0.11862	-23.4542123044262\\
72.875	0.12228	-24.9722571542466\\
72.875	0.12594	-26.6125336550609\\
72.875	0.1296	-28.3750418068692\\
72.875	0.13326	-30.2597816096714\\
72.875	0.13692	-32.2667530634675\\
72.875	0.14058	-34.3959561682574\\
72.875	0.14424	-36.6473909240414\\
72.875	0.1479	-39.0210573308192\\
72.875	0.15156	-41.516955388591\\
72.875	0.15522	-44.1350850973567\\
72.875	0.15888	-46.8754464571163\\
72.875	0.16254	-49.7380394678698\\
72.875	0.1662	-52.7228641296172\\
72.875	0.16986	-55.8299204423586\\
72.875	0.17352	-59.0592084060939\\
72.875	0.17718	-62.410728020823\\
72.875	0.18084	-65.8844792865461\\
72.875	0.1845	-69.4804622032632\\
72.875	0.18816	-73.1986767709741\\
72.875	0.19182	-77.0391229896789\\
72.875	0.19548	-81.0018008593778\\
72.875	0.19914	-85.0867103800704\\
72.875	0.2028	-89.2938515517571\\
72.875	0.20646	-93.6232243744376\\
72.875	0.21012	-98.0748288481121\\
72.875	0.21378	-102.64866497278\\
72.875	0.21744	-107.344732748443\\
72.875	0.2211	-112.163032175099\\
72.875	0.22476	-117.103563252749\\
72.875	0.22842	-122.166325981393\\
72.875	0.23208	-127.351320361031\\
72.875	0.23574	-132.658546391663\\
72.875	0.2394	-138.088004073289\\
72.875	0.24306	-143.639693405908\\
72.875	0.24672	-149.313614389522\\
72.875	0.25038	-155.10976702413\\
72.875	0.25404	-161.028151309731\\
72.875	0.2577	-167.068767246326\\
72.875	0.26136	-173.231614833916\\
72.875	0.26502	-179.516694072499\\
72.875	0.26868	-185.924004962076\\
72.875	0.27234	-192.453547502647\\
72.875	0.276	-199.105321694212\\
73.25	0.093	-16.5898879686607\\
73.25	0.09666	-17.2550901026111\\
73.25	0.10032	-18.0425238875554\\
73.25	0.10398	-18.9521893234936\\
73.25	0.10764	-19.9840864104258\\
73.25	0.1113	-21.1382151483519\\
73.25	0.11496	-22.4145755372719\\
73.25	0.11862	-23.8131675771858\\
73.25	0.12228	-25.3339912680937\\
73.25	0.12594	-26.9770466099954\\
73.25	0.1296	-28.7423336028911\\
73.25	0.13326	-30.6298522467807\\
73.25	0.13692	-32.6396025416642\\
73.25	0.14058	-34.7715844875416\\
73.25	0.14424	-37.025798084413\\
73.25	0.1479	-39.4022433322783\\
73.25	0.15156	-41.9009202311375\\
73.25	0.15522	-44.5218287809906\\
73.25	0.15888	-47.2649689818376\\
73.25	0.16254	-50.1303408336785\\
73.25	0.1662	-53.1179443365134\\
73.25	0.16986	-56.2277794903421\\
73.25	0.17352	-59.4598462951648\\
73.25	0.17718	-62.8141447509814\\
73.25	0.18084	-66.2906748577919\\
73.25	0.1845	-69.8894366155964\\
73.25	0.18816	-73.6104300243948\\
73.25	0.19182	-77.453655084187\\
73.25	0.19548	-81.4191117949732\\
73.25	0.19914	-85.5068001567533\\
73.25	0.2028	-89.7167201695274\\
73.25	0.20646	-94.0488718332954\\
73.25	0.21012	-98.5032551480573\\
73.25	0.21378	-103.079870113813\\
73.25	0.21744	-107.778716730563\\
73.25	0.2211	-112.599794998306\\
73.25	0.22476	-117.543104917044\\
73.25	0.22842	-122.608646486775\\
73.25	0.23208	-127.796419707501\\
73.25	0.23574	-133.10642457922\\
73.25	0.2394	-138.538661101933\\
73.25	0.24306	-144.09312927564\\
73.25	0.24672	-149.769829100341\\
73.25	0.25038	-155.568760576036\\
73.25	0.25404	-161.489923702725\\
73.25	0.2577	-167.533318480408\\
73.25	0.26136	-173.698944909085\\
73.25	0.26502	-179.986802988755\\
73.25	0.26868	-186.39689271942\\
73.25	0.27234	-192.929214101078\\
73.25	0.276	-199.583767133731\\
73.625	0.093	-16.9389877109282\\
73.625	0.09666	-17.606968685966\\
73.625	0.10032	-18.3971813119978\\
73.625	0.10398	-19.3096255890234\\
73.625	0.10764	-20.344301517043\\
73.625	0.1113	-21.5012090960565\\
73.625	0.11496	-22.7803483260639\\
73.625	0.11862	-24.1817192070653\\
73.625	0.12228	-25.7053217390605\\
73.625	0.12594	-27.3511559220497\\
73.625	0.1296	-29.1192217560328\\
73.625	0.13326	-31.0095192410098\\
73.625	0.13692	-33.0220483769808\\
73.625	0.14058	-35.1568091639456\\
73.625	0.14424	-37.4138016019044\\
73.625	0.1479	-39.7930256908571\\
73.625	0.15156	-42.2944814308037\\
73.625	0.15522	-44.9181688217442\\
73.625	0.15888	-47.6640878636787\\
73.625	0.16254	-50.532238556607\\
73.625	0.1662	-53.5226209005293\\
73.625	0.16986	-56.6352348954455\\
73.625	0.17352	-59.8700805413556\\
73.625	0.17718	-63.2271578382596\\
73.625	0.18084	-66.7064667861576\\
73.625	0.1845	-70.3080073850495\\
73.625	0.18816	-74.0317796349353\\
73.625	0.19182	-77.8777835358149\\
73.625	0.19548	-81.8460190876885\\
73.625	0.19914	-85.9364862905561\\
73.625	0.2028	-90.1491851444176\\
73.625	0.20646	-94.484115649273\\
73.625	0.21012	-98.9412778051223\\
73.625	0.21378	-103.520671611965\\
73.625	0.21744	-108.222297069803\\
73.625	0.2211	-113.046154178634\\
73.625	0.22476	-117.992242938459\\
73.625	0.22842	-123.060563349277\\
73.625	0.23208	-128.25111541109\\
73.625	0.23574	-133.563899123897\\
73.625	0.2394	-138.998914487698\\
73.625	0.24306	-144.556161502492\\
73.625	0.24672	-150.235640168281\\
73.625	0.25038	-156.037350485063\\
73.625	0.25404	-161.961292452839\\
73.625	0.2577	-168.00746607161\\
73.625	0.26136	-174.175871341374\\
73.625	0.26502	-180.466508262132\\
73.625	0.26868	-186.879376833884\\
73.625	0.27234	-193.41447705663\\
73.625	0.276	-200.071808930369\\
74	0.093	-17.2976838103155\\
74	0.09666	-17.9684436264408\\
74	0.10032	-18.7614350935599\\
74	0.10398	-19.676658211673\\
74	0.10764	-20.71411298078\\
74	0.1113	-21.873799400881\\
74	0.11496	-23.1557174719758\\
74	0.11862	-24.5598671940646\\
74	0.12228	-26.0862485671472\\
74	0.12594	-27.7348615912239\\
74	0.1296	-29.5057062662944\\
74	0.13326	-31.3987825923588\\
74	0.13692	-33.4140905694171\\
74	0.14058	-35.5516301974694\\
74	0.14424	-37.8114014765157\\
74	0.1479	-40.1934044065557\\
74	0.15156	-42.6976389875898\\
74	0.15522	-45.3241052196177\\
74	0.15888	-48.0728031026396\\
74	0.16254	-50.9437326366554\\
74	0.1662	-53.9368938216651\\
74	0.16986	-57.0522866576686\\
74	0.17352	-60.2899111446662\\
74	0.17718	-63.6497672826576\\
74	0.18084	-67.131855071643\\
74	0.1845	-70.7361745116223\\
74	0.18816	-74.4627256025956\\
74	0.19182	-78.3115083445626\\
74	0.19548	-82.2825227375237\\
74	0.19914	-86.3757687814787\\
74	0.2028	-90.5912464764275\\
74	0.20646	-94.9289558223704\\
74	0.21012	-99.3888968193071\\
74	0.21378	-103.971069467238\\
74	0.21744	-108.675473766162\\
74	0.2211	-113.502109716081\\
74	0.22476	-118.450977316993\\
74	0.22842	-123.522076568899\\
74	0.23208	-128.7154074718\\
74	0.23574	-134.030970025694\\
74	0.2394	-139.468764230582\\
74	0.24306	-145.028790086464\\
74	0.24672	-150.71104759334\\
74	0.25038	-156.515536751209\\
74	0.25404	-162.442257560073\\
74	0.2577	-168.491210019931\\
74	0.26136	-174.662394130782\\
74	0.26502	-180.955809892628\\
74	0.26868	-187.371457305467\\
74	0.27234	-193.909336369301\\
74	0.276	-200.569447084128\\
};\label{tikz:theta_surf}
\end{axis}
\end{tikzpicture}%
	\caption{The quantifiable relationship between external parameters ($L_{cut}, D_{rlx}$) and internal parameters $\theta$ is obtained by fitting a polynomial surface to  the internal parameter values estimated for the set F1 (\ref{tikz:thetas1}) and set F2 (\ref{tikz:thetas2}). The fitted surfaces (\ref{tikz:theta_surf}) are then used to compute the internal model parameters corresponding to the settings of choice (\ref{tikz:thetaidentified}). }
\end{figure}

\end{document}