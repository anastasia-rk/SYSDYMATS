\documentclass[a4paper,11pt,twoside]{article}
%%%%%%%%%%%%%%%%%%%%%%%%%%%%%%%%%%%%%%%%%%%%%%%%%%%%%%%%%%%%%%%%%%%%%%%%%%%
% Packages
\usepackage[final]{pdfpages}
\usepackage{verbatim}
\usepackage{inputenc}
\usepackage{graphicx} 
\usepackage{amsmath,amssymb,mathrsfs,amsfonts}
\usepackage{mathtools}
\usepackage{amsthm}
\usepackage{mathtools}
\usepackage{calrsfs}
\usepackage{graphicx}
\usepackage{subfig}
\usepackage{eucal}    
\usepackage{amssymb}  
\usepackage{pifont}
\usepackage{color} 
\usepackage{cancel}
\usepackage[toc,page]{appendix}
%\usepackage[usenames,dvipsnames,table]{xcolor}
\usepackage{pgfplots}
\pgfplotsset{every axis/.append style={line width=0.5pt},label style={font=\scriptsize},tick label style={font=\scriptsize},x tick label style={/pgf/number format/.cd,fixed,precision=3, set thousands separator={}}}
\usetikzlibrary{shapes,shadows,arrows,backgrounds,patterns,positioning,automata,calc,decorations.markings,decorations.pathreplacing,bayesnet,arrows.meta}
\usepackage{varwidth}
\usepackage{lscape}
\usepackage{array} 
\usepackage[colorlinks=false,pdfborder={0 0 0}]{hyperref}
\usepackage{tabularx}
\usepackage{textcomp}
\usepackage{multicol} 
\usepackage{booktabs}
\usepackage{multirow}
\usepackage[font=small,labelfont=bf]{caption}                                                           
\usepackage{textcase}
\usepackage{bbm} 
\usepackage{fancyhdr}
\usepackage{enumitem}
\usepackage{soul}
%\usepackage[british]{babel}
\usepackage{wrapfig}
%\usepackage{glossaries}
%%%%%%%%%%%%%%%%%%%%%%%%%%%%%%%%%%%%%%%%%%%%%%%%%%%%%%%%%%%%%%%%%%%%%%%%%%%
\setlength{\parindent}{2em}
\setlength{\parskip}{0.5em}
\renewcommand{\baselinestretch}{1.2}
\usepackage[left=3cm, right=2cm, top=2.5cm, bottom=3cm, headheight=13.6pt]{geometry}
\allowdisplaybreaks 
% Bibliography
%\usepackage[backend=bibtex,style=ieee,sorting=none]{biblatex} 
%\bibliography{Bibliography-thesis}
%\renewcommand*{\bibfont}{\scriptsize}
\makeatletter
\newcommand*{\rom}[1]{\expandafter\@slowromancap\romannumeral #1@}
\newcommand{\ie}{\textit{i.e.} }
\newcommand{\eg}{\textit{e.g.} }
\makeatother
\newcommand\id{\ensuremath{\mathbbm{1}}} 
\DeclareMathOperator{\E}{\mathbb{E}}
\DeclareMathOperator{\eye}{\mathbb{I}}
\DeclareMathOperator{\zeros}{\mathbb{O}}
\DeclareMathOperator{\tr}{\textrm{tr}}
\DeclareMathOperator{\vvec}{\textrm{vec}}
\DeclareMathOperator{\ik}{\mathrm{k}}
\DeclareMathOperator{\ip}{\mathrm{p}}
\DeclareMathOperator{\inn}{\mathrm{n}}
\DeclareMathOperator{\im}{\mathrm{m}}
\DeclareMathOperator{\td}{\mathrm{t}}
\DeclareMathOperator{\kd}{\mathrm{k}}
\DeclareMathOperator{\T}{\mathrm{T}}
\DeclareMathOperator{\K}{\mathrm{K}}
\DeclareSymbolFontAlphabet{\mathcal} {symbols}
\DeclareSymbolFont{symbols}{OMS}{cm}{m}{n}
\DeclareMathAlphabet{\mathbfit}{OML}{cmm}{b}{it}

% Number equations
%\numberwithin{equation}{section}

%%%%%%%%%%%%%%%%%%%%%%%%%%%%%%%%%%%%%%%%%%%%%%%%%%%%%%%%%%%%%%%%%%%%%%%%%%%
%Theorems
\newtheoremstyle{mytheoremstyle} % name
{.5em}                    % Space above
{.8em}                    % Space below
{\itshape}                % Body font
{1em}                           % Indent amount
{\bfseries}                   % Theorem head font
{:}                          % Punctuation after theorem head
{.5em}                       % Space after theorem head
{}  % Theorem head spec (can be left empty, meaning ‘normal’)

\theoremstyle{mytheoremstyle}
\newtheorem{theorem}{Theorem}[section]
\newtheorem{remark}{Remark}[section]
\newtheorem{assumption}{Assumption}[section]
\newtheorem{lemma}{Lemma}[section]
\newtheorem{condition}{Condition}[section]
\newtheorem{definition}{Definition}[section]
\newtheorem{property}{Property}[section]
\newtheorem{corollary}{Corollary}[section]
\renewcommand\qedsymbol{$\blacksquare$}
%%%%%%%%%%%%%%%%%%%%%%%%%%%%%%%%%%%%%%%%%%%%%%%%%%%%%%%%%%%%%%%%%%%%%%%%%%%
% Nomenclature
\usepackage[intoc]{nomencl}
\makenomenclature

%\usepackage[ruled,chapter]{algorithm}
%\usepackage{float}

%\usepackage{algorithmic}
%\algsetup{linenosize=\scriptsize}
%\usepackage{etoolbox}
%\AtBeginEnvironment{algorithmic}{\scriptsize}
%\renewcommand{\thealgorithm}{\thechapter.\arabic{algorithm}} 
%\usepackage{chngcntr}
%\counterwithin{algorithm}{section}

% correct bad hyphenation here
\hyphenation{op-tical net-works semi-conduc-tor}
%%%%%%%%%%%%%%%%%%%%%%%%%%%%%%%%%%%%%%%%%%%%%%%%%%%%%%%%%%%%%%%%%%%%%%%%%%%%%
% Captions
%\newcommand{\xLanguage}{british}          % <-- Added this command
%
%\usepackage[\xLanguage]{babel}             % <-- Implemented here
%
%\expandafter\addto\csname captions\xLanguage\endcsname{% <-- and here
%	\renewcommand{\tableshortname}{Table}%
%	\renewcommand{\figureshortname}{Figure}%
%}
%
%\captionsetup[table]{format=plain,indention=1.15cm,justification=justified}
%\captionsetup[figure]{format=plain,indention=1.25cm,justification=justified}

%%%%%%%%%%%%%%%%%%%%%%%%%%%%%%%%%%%%%%%%%%%%%%%%%%%%%%%%%%%%%%%%%%%%%%%%%%%%%
\usepackage[explicit]{titlesec}
\usepackage{titletoc}
%\titleformat{\section}[block]{\normalfont\Large\rm\filright\bfseries}{\thesection}{1em}{#1}
%\titlecontents{chapter}[1.5em]{}{\scshape\contentslabel{2.3em}}{}{\titlerule*[1pc]{}\contentspage}
%\definecolor{gray75}{gray}{0.5}
%\newcommand{\hsp}{\hspace{20pt}}
%\titleformat{\chapter}[hang]{\huge\scshape\filright\bfseries}{\color{gray75}\thechapter}{20pt}{\begin{tabular}[t]{@{\color{gray75}\vrule width 2pt\hsp}p{0.85\textwidth}}\raggedright#1\end{tabular}}
%\titleformat{name=\chapter,numberless}[display]{}{}{0pt}{\normalfont\huge\bfseries #1} % format for numberless chapters
%\titleformat{\subsection}[block]{\normalfont\large\rm\filright\bfseries}{\thesubsection}{1em}{#1}
%\renewcommand{\sectionmark}[1]{\markright{\thesection ~ \ #1}}
%%\renewcommand{\chaptermark}[1]{\markboth{\chaptername\ \thechapter ~ \ #1}{}} 
%\pagestyle{fancy}
%%\fancyhf{}
%\fancyhead[RO,LE]{\thepage}
%\fancyhead[RE]{\itshape \nouppercase \rightmark}       % chaptertitle left
%\fancyhead[LO]{\itshape \nouppercase \leftmark}       % sectiontitle right
%\renewcommand{\headrulewidth}{0.5pt} % no rule
%\cfoot{}
\interfootnotelinepenalty=10000

\title{Structure and parameter identification of au}


\begin{document}
	\maketitle
%	\section{Experimental setup}
%	
	\section{Experimental data}
		\begin{figure}[!h]
		\centering
		\scalebox{0.8}{% This file was created by matlab2tikz.
% Minimal pgfplots version: 1.3
%
\definecolor{mycolor1}{rgb}{0.00000,0.44700,0.74100}%
%
\begin{tikzpicture}

\begin{axis}[%
width=5.5cm,
height=2.1cm,
at={(7.5cm,7.627119cm)},
scale only axis,
xmin=1000,
xmax=1500,
xlabel={\small Time, sec},
ymin=-60,
ymax=0,
ylabel={\small Load, kN},
legend style={legend cell align=left,align=left,draw=white!15!black}
]
\addplot [color=mycolor1,line width=0.5pt,solid,forget plot]
  table[row sep=crcr]{%
1000	-17.09\\
1001	-19.531\\
1002	-14.648\\
1003	-14.648\\
1004	-19.531\\
1005	-18.311\\
1006	-23.193\\
1007	-18.311\\
1008	-9.766\\
1009	-15.869\\
1010	-12.207\\
1011	-14.648\\
1012	-10.986\\
1013	-3.662\\
1014	-2.441\\
1015	-4.883\\
1016	-6.104\\
1017	-14.648\\
1018	-17.09\\
1019	-17.09\\
1020	-13.428\\
1021	-9.766\\
1022	-18.311\\
1023	-14.648\\
1024	-10.986\\
1025	-13.428\\
1026	-12.207\\
1027	-10.986\\
1028	-20.752\\
1029	-19.531\\
1030	-12.207\\
1031	-20.752\\
1032	-20.752\\
1033	-15.869\\
1034	-13.428\\
1035	-9.766\\
1036	-14.648\\
1037	-14.648\\
1038	-14.648\\
1039	-14.648\\
1040	-14.648\\
1041	-15.869\\
1042	-21.973\\
1043	-20.752\\
1044	-13.428\\
1045	-13.428\\
1046	-8.545\\
1047	-12.207\\
1048	-12.207\\
1049	-13.428\\
1050	-10.986\\
1051	-13.428\\
1052	-14.648\\
1053	-13.428\\
1054	-20.752\\
1055	-13.428\\
1056	-9.766\\
1057	-7.324\\
1058	-8.545\\
1059	-10.986\\
1060	-6.104\\
1061	-9.766\\
1062	-10.986\\
1063	-6.104\\
1064	-6.104\\
1065	-9.766\\
1066	-9.766\\
1067	-15.869\\
1068	-14.648\\
1069	-17.09\\
1070	-15.869\\
1071	-18.311\\
1072	-13.428\\
1073	-14.648\\
1074	-12.207\\
1075	-10.986\\
1076	-12.207\\
1077	-23.193\\
1078	-28.076\\
1079	-30.518\\
1080	-29.297\\
1081	-18.311\\
1082	-28.076\\
1083	-32.959\\
1084	-31.738\\
1085	-20.752\\
1086	-34.18\\
1087	-40.283\\
1088	-29.297\\
1089	-24.414\\
1090	-19.531\\
1091	-15.869\\
1092	-12.207\\
1093	-14.648\\
1094	-9.766\\
1095	-7.324\\
1096	-7.324\\
1097	-10.986\\
1098	-13.428\\
1099	-12.207\\
1100	-12.207\\
1101	-15.869\\
1102	-18.311\\
1103	-19.531\\
1104	-15.869\\
1105	-21.973\\
1106	-15.869\\
1107	-15.869\\
1108	-17.09\\
1109	-12.207\\
1110	-12.207\\
1111	-15.869\\
1112	-14.648\\
1113	-8.545\\
1114	-12.207\\
1115	-8.545\\
1116	-9.766\\
1117	-9.766\\
1118	-7.324\\
1119	-9.766\\
1120	-12.207\\
1121	-17.09\\
1122	-17.09\\
1123	-17.09\\
1124	-10.986\\
1125	-12.207\\
1126	-23.193\\
1127	-15.869\\
1128	-20.752\\
1129	-21.973\\
1130	-12.207\\
1131	-9.766\\
1132	-12.207\\
1133	-13.428\\
1134	-21.973\\
1135	-25.635\\
1136	-26.855\\
1137	-19.531\\
1138	-20.752\\
1139	-18.311\\
1140	-17.09\\
1141	-18.311\\
1142	-13.428\\
1143	-12.207\\
1144	-9.766\\
1145	-9.766\\
1146	-10.986\\
1147	-15.869\\
1148	-23.193\\
1149	-17.09\\
1150	-13.428\\
1151	-10.986\\
1152	-10.986\\
1153	-7.324\\
1154	-6.104\\
1155	-6.104\\
1156	-12.207\\
1157	-7.324\\
1158	-8.545\\
1159	-8.545\\
1160	-8.545\\
1161	-7.324\\
1162	-4.883\\
1163	-3.662\\
1164	-8.545\\
1165	-15.869\\
1166	-18.311\\
1167	-20.752\\
1168	-15.869\\
1169	-10.986\\
1170	-8.545\\
1171	-4.883\\
1172	-9.766\\
1173	-8.545\\
1174	-15.869\\
1175	-17.09\\
1176	-29.297\\
1177	-32.959\\
1178	-25.635\\
1179	-25.635\\
1180	-18.311\\
1181	-23.193\\
1182	-21.973\\
1183	-24.414\\
1184	-15.869\\
1185	-17.09\\
1186	-17.09\\
1187	-15.869\\
1188	-15.869\\
1189	-13.428\\
1190	-23.193\\
1191	-25.635\\
1192	-19.531\\
1193	-13.428\\
1194	-14.648\\
1195	-23.193\\
1196	-24.414\\
1197	-28.076\\
1198	-30.518\\
1199	-29.297\\
1200	-23.193\\
1201	-21.973\\
1202	-25.635\\
1203	-28.076\\
1204	-18.311\\
1205	-12.207\\
1206	-19.531\\
1207	-14.648\\
1208	-9.766\\
1209	-13.428\\
1210	-14.648\\
1211	-10.986\\
1212	-10.986\\
1213	-10.986\\
1214	-8.545\\
1215	-10.986\\
1216	-18.311\\
1217	-17.09\\
1218	-12.207\\
1219	-12.207\\
1220	-18.311\\
1221	-26.855\\
1222	-17.09\\
1223	-14.648\\
1224	-10.986\\
1225	-13.428\\
1226	-10.986\\
1227	-8.545\\
1228	-8.545\\
1229	-10.986\\
1230	-13.428\\
1231	-14.648\\
1232	-12.207\\
1233	-15.869\\
1234	-20.752\\
1235	-17.09\\
1236	-12.207\\
1237	-15.869\\
1238	-20.752\\
1239	-13.428\\
1240	-7.324\\
1241	-13.428\\
1242	-10.986\\
1243	-12.207\\
1244	-9.766\\
1245	-8.545\\
1246	-13.428\\
1247	-13.428\\
1248	-9.766\\
1249	-12.207\\
1250	-9.766\\
1251	-7.324\\
1252	-12.207\\
1253	-8.545\\
1254	-9.766\\
1255	-7.324\\
1256	-4.883\\
1257	-12.207\\
1258	-15.869\\
1259	-17.09\\
1260	-23.193\\
1261	-15.869\\
1262	-9.766\\
1263	-8.545\\
1264	-8.545\\
1265	-10.986\\
1266	-8.545\\
1267	-4.883\\
1268	-10.986\\
1269	-15.869\\
1270	-13.428\\
1271	-20.752\\
1272	-15.869\\
1273	-14.648\\
1274	-8.545\\
1275	-10.986\\
1276	-4.883\\
1277	-3.662\\
1278	-4.883\\
1279	-9.766\\
1280	-10.986\\
1281	-12.207\\
1282	-13.428\\
1283	-20.752\\
1284	-17.09\\
1285	-14.648\\
1286	-17.09\\
1287	-19.531\\
1288	-20.752\\
1289	-17.09\\
1290	-13.428\\
1291	-7.324\\
1292	-6.104\\
1293	-4.883\\
1294	-7.324\\
1295	-7.324\\
1296	-4.883\\
1297	-8.545\\
1298	-12.207\\
1299	-10.986\\
1300	-12.207\\
1301	-12.207\\
1302	-8.545\\
1303	-4.883\\
1304	-9.766\\
1305	-15.869\\
1306	-13.428\\
1307	-15.869\\
1308	-13.428\\
1309	-9.766\\
1310	-14.648\\
1311	-18.311\\
1312	-18.311\\
1313	-13.428\\
1314	-20.752\\
1315	-15.869\\
1316	-17.09\\
1317	-18.311\\
1318	-19.531\\
1319	-13.428\\
1320	-13.428\\
1321	-18.311\\
1322	-26.855\\
1323	-21.973\\
1324	-13.428\\
1325	-13.428\\
1326	-13.428\\
1327	-15.869\\
1328	-17.09\\
1329	-12.207\\
1330	-14.648\\
1331	-18.311\\
1332	-20.752\\
1333	-13.428\\
1334	-12.207\\
1335	-15.869\\
1336	-23.193\\
1337	-20.752\\
1338	-21.973\\
1339	-14.648\\
1340	-14.648\\
1341	-12.207\\
1342	-9.766\\
1343	-4.883\\
1344	-3.662\\
1345	-3.662\\
1346	-7.324\\
1347	-13.428\\
1348	-14.648\\
1349	-9.766\\
1350	-12.207\\
1351	-13.428\\
1352	-9.766\\
1353	-8.545\\
1354	-8.545\\
1355	-6.104\\
1356	-9.766\\
1357	-14.648\\
1358	-15.869\\
1359	-10.986\\
1360	-8.545\\
1361	-8.545\\
1362	-8.545\\
1363	-7.324\\
1364	-10.986\\
1365	-20.752\\
1366	-18.311\\
1367	-18.311\\
1368	-24.414\\
1369	-23.193\\
1370	-18.311\\
1371	-14.648\\
1372	-14.648\\
1373	-14.648\\
1374	-17.09\\
1375	-19.531\\
1376	-19.531\\
1377	-25.635\\
1378	-26.855\\
1379	-31.738\\
1380	-24.414\\
1381	-25.635\\
1382	-26.855\\
1383	-21.973\\
1384	-24.414\\
1385	-20.752\\
1386	-13.428\\
1387	-9.766\\
1388	-9.766\\
1389	-12.207\\
1390	-9.766\\
1391	-7.324\\
1392	-10.986\\
1393	-8.545\\
1394	-6.104\\
1395	-7.324\\
1396	-9.766\\
1397	-6.104\\
1398	-12.207\\
1399	-14.648\\
1400	-10.986\\
1401	-17.09\\
1402	-24.414\\
1403	-24.414\\
1404	-28.076\\
1405	-23.193\\
1406	-21.973\\
1407	-17.09\\
1408	-9.766\\
1409	-10.986\\
1410	-10.986\\
1411	-7.324\\
1412	-10.986\\
1413	-9.766\\
1414	-2.441\\
1415	-6.104\\
1416	-9.766\\
1417	-12.207\\
1418	-14.648\\
1419	-17.09\\
1420	-14.648\\
1421	-17.09\\
1422	-14.648\\
1423	-13.428\\
1424	-12.207\\
1425	-10.986\\
1426	-14.648\\
1427	-13.428\\
1428	-10.986\\
1429	-13.428\\
1430	-18.311\\
1431	-13.428\\
1432	-10.986\\
1433	-10.986\\
1434	-10.986\\
1435	-8.545\\
1436	-7.324\\
1437	-10.986\\
1438	-13.428\\
1439	-13.428\\
1440	-10.986\\
1441	-13.428\\
1442	-13.428\\
1443	-8.545\\
1444	-7.324\\
1445	-15.869\\
1446	-19.531\\
1447	-18.311\\
1448	-18.311\\
1449	-15.869\\
1450	-13.428\\
1451	-10.986\\
1452	-9.766\\
1453	-13.428\\
1454	-13.428\\
1455	-8.545\\
1456	-12.207\\
1457	-14.648\\
1458	-14.648\\
1459	-17.09\\
1460	-17.09\\
1461	-20.752\\
1462	-28.076\\
1463	-30.518\\
1464	-20.752\\
1465	-13.428\\
1466	-8.545\\
1467	-6.104\\
1468	-8.545\\
1469	-7.324\\
1470	-10.986\\
1471	-12.207\\
1472	-12.207\\
1473	-13.428\\
1474	-15.869\\
1475	-19.531\\
1476	-19.531\\
1477	-28.076\\
1478	-23.193\\
1479	-18.311\\
1480	-14.648\\
1481	-14.648\\
1482	-14.648\\
1483	-17.09\\
1484	-14.648\\
1485	-12.207\\
1486	-13.428\\
1487	-8.545\\
1488	-7.324\\
1489	-17.09\\
1490	-14.648\\
1491	-12.207\\
1492	-23.193\\
1493	-18.311\\
1494	-15.869\\
1495	-21.973\\
1496	-23.193\\
1497	-14.648\\
1498	-8.545\\
1499	-9.766\\
1500	-15.869\\
};
\end{axis}

\begin{axis}[%
width=5.5cm,
height=2.1cm,
at={(7.5cm,11.440678cm)},
scale only axis,
xmin=1000,
xmax=1500,
xlabel={\small Time, sec},
ymin=-60,
ymax=0,
ylabel={\small Load, kN},
legend style={legend cell align=left,align=left,draw=white!15!black}
]
\addplot [color=mycolor1,line width=0.5pt,solid,forget plot]
  table[row sep=crcr]{%
1000	-19.531\\
1001	-24.414\\
1002	-19.531\\
1003	-20.752\\
1004	-25.635\\
1005	-24.414\\
1006	-28.076\\
1007	-23.193\\
1008	-13.428\\
1009	-17.09\\
1010	-17.09\\
1011	-18.311\\
1012	-15.869\\
1013	-8.545\\
1014	-6.104\\
1015	-7.324\\
1016	-9.766\\
1017	-21.973\\
1018	-23.193\\
1019	-23.193\\
1020	-19.531\\
1021	-10.986\\
1022	-17.09\\
1023	-19.531\\
1024	-13.428\\
1025	-18.311\\
1026	-19.531\\
1027	-14.648\\
1028	-24.414\\
1029	-21.973\\
1030	-15.869\\
1031	-23.193\\
1032	-25.635\\
1033	-19.531\\
1034	-17.09\\
1035	-14.648\\
1036	-17.09\\
1037	-20.752\\
1038	-18.311\\
1039	-18.311\\
1040	-19.531\\
1041	-20.752\\
1042	-28.076\\
1043	-25.635\\
1044	-17.09\\
1045	-15.869\\
1046	-12.207\\
1047	-10.986\\
1048	-15.869\\
1049	-12.207\\
1050	-12.207\\
1051	-15.869\\
1052	-18.311\\
1053	-15.869\\
1054	-25.635\\
1055	-21.973\\
1056	-12.207\\
1057	-9.766\\
1058	-13.428\\
1059	-15.869\\
1060	-10.986\\
1061	-10.986\\
1062	-12.207\\
1063	-9.766\\
1064	-8.545\\
1065	-12.207\\
1066	-14.648\\
1067	-17.09\\
1068	-18.311\\
1069	-23.193\\
1070	-21.973\\
1071	-21.973\\
1072	-17.09\\
1073	-17.09\\
1074	-17.09\\
1075	-15.869\\
1076	-17.09\\
1077	-24.414\\
1078	-35.4\\
1079	-36.621\\
1080	-34.18\\
1081	-24.414\\
1082	-29.297\\
1083	-36.621\\
1084	-40.283\\
1085	-31.738\\
1086	-37.842\\
1087	-48.828\\
1088	-39.063\\
1089	-30.518\\
1090	-25.635\\
1091	-19.531\\
1092	-14.648\\
1093	-19.531\\
1094	-15.869\\
1095	-9.766\\
1096	-9.766\\
1097	-12.207\\
1098	-17.09\\
1099	-15.869\\
1100	-15.869\\
1101	-19.531\\
1102	-19.531\\
1103	-28.076\\
1104	-23.193\\
1105	-25.635\\
1106	-23.193\\
1107	-18.311\\
1108	-20.752\\
1109	-17.09\\
1110	-14.648\\
1111	-18.311\\
1112	-17.09\\
1113	-15.869\\
1114	-13.428\\
1115	-10.986\\
1116	-12.207\\
1117	-13.428\\
1118	-10.986\\
1119	-9.766\\
1120	-14.648\\
1121	-19.531\\
1122	-23.193\\
1123	-23.193\\
1124	-17.09\\
1125	-18.311\\
1126	-30.518\\
1127	-23.193\\
1128	-24.414\\
1129	-26.855\\
1130	-17.09\\
1131	-10.986\\
1132	-17.09\\
1133	-18.311\\
1134	-26.855\\
1135	-34.18\\
1136	-31.738\\
1137	-25.635\\
1138	-24.414\\
1139	-23.193\\
1140	-23.193\\
1141	-21.973\\
1142	-18.311\\
1143	-14.648\\
1144	-13.428\\
1145	-12.207\\
1146	-13.428\\
1147	-19.531\\
1148	-25.635\\
1149	-21.973\\
1150	-14.648\\
1151	-13.428\\
1152	-14.648\\
1153	-12.207\\
1154	-8.545\\
1155	-8.545\\
1156	-13.428\\
1157	-12.207\\
1158	-12.207\\
1159	-12.207\\
1160	-12.207\\
1161	-8.545\\
1162	-7.324\\
1163	-4.883\\
1164	-7.324\\
1165	-17.09\\
1166	-24.414\\
1167	-24.414\\
1168	-19.531\\
1169	-12.207\\
1170	-12.207\\
1171	-8.545\\
1172	-10.986\\
1173	-14.648\\
1174	-14.648\\
1175	-24.414\\
1176	-34.18\\
1177	-40.283\\
1178	-32.959\\
1179	-31.738\\
1180	-23.193\\
1181	-25.635\\
1182	-26.855\\
1183	-28.076\\
1184	-21.973\\
1185	-20.752\\
1186	-19.531\\
1187	-19.531\\
1188	-21.973\\
1189	-19.531\\
1190	-23.193\\
1191	-29.297\\
1192	-24.414\\
1193	-14.648\\
1194	-15.869\\
1195	-25.635\\
1196	-32.959\\
1197	-35.4\\
1198	-39.063\\
1199	-37.842\\
1200	-30.518\\
1201	-26.855\\
1202	-30.518\\
1203	-35.4\\
1204	-25.635\\
1205	-14.648\\
1206	-19.531\\
1207	-20.752\\
1208	-12.207\\
1209	-13.428\\
1210	-18.311\\
1211	-14.648\\
1212	-12.207\\
1213	-18.311\\
1214	-10.986\\
1215	-14.648\\
1216	-24.414\\
1217	-21.973\\
1218	-14.648\\
1219	-14.648\\
1220	-20.752\\
1221	-34.18\\
1222	-28.076\\
1223	-15.869\\
1224	-14.648\\
1225	-14.648\\
1226	-12.207\\
1227	-10.986\\
1228	-10.986\\
1229	-13.428\\
1230	-17.09\\
1231	-19.531\\
1232	-14.648\\
1233	-19.531\\
1234	-26.855\\
1235	-23.193\\
1236	-17.09\\
1237	-21.973\\
1238	-25.635\\
1239	-21.973\\
1240	-8.545\\
1241	-12.207\\
1242	-13.428\\
1243	-15.869\\
1244	-15.869\\
1245	-10.986\\
1246	-17.09\\
1247	-15.869\\
1248	-10.986\\
1249	-14.648\\
1250	-14.648\\
1251	-12.207\\
1252	-14.648\\
1253	-10.986\\
1254	-9.766\\
1255	-12.207\\
1256	-8.545\\
1257	-17.09\\
1258	-18.311\\
1259	-20.752\\
1260	-29.297\\
1261	-20.752\\
1262	-10.986\\
1263	-10.986\\
1264	-8.545\\
1265	-13.428\\
1266	-12.207\\
1267	-9.766\\
1268	-15.869\\
1269	-19.531\\
1270	-20.752\\
1271	-23.193\\
1272	-23.193\\
1273	-17.09\\
1274	-15.869\\
1275	-10.986\\
1276	-10.986\\
1277	-7.324\\
1278	-8.545\\
1279	-10.986\\
1280	-17.09\\
1281	-18.311\\
1282	-17.09\\
1283	-24.414\\
1284	-25.635\\
1285	-15.869\\
1286	-19.531\\
1287	-23.193\\
1288	-26.855\\
1289	-25.635\\
1290	-17.09\\
1291	-10.986\\
1292	-7.324\\
1293	-6.104\\
1294	-8.545\\
1295	-9.766\\
1296	-8.545\\
1297	-10.986\\
1298	-17.09\\
1299	-15.869\\
1300	-13.428\\
1301	-15.869\\
1302	-12.207\\
1303	-6.104\\
1304	-9.766\\
1305	-20.752\\
1306	-17.09\\
1307	-18.311\\
1308	-15.869\\
1309	-10.986\\
1310	-17.09\\
1311	-23.193\\
1312	-23.193\\
1313	-17.09\\
1314	-26.855\\
1315	-25.635\\
1316	-18.311\\
1317	-23.193\\
1318	-25.635\\
1319	-19.531\\
1320	-15.869\\
1321	-24.414\\
1322	-35.4\\
1323	-29.297\\
1324	-17.09\\
1325	-17.09\\
1326	-18.311\\
1327	-20.752\\
1328	-23.193\\
1329	-19.531\\
1330	-17.09\\
1331	-24.414\\
1332	-25.635\\
1333	-18.311\\
1334	-15.869\\
1335	-18.311\\
1336	-28.076\\
1337	-26.855\\
1338	-24.414\\
1339	-19.531\\
1340	-17.09\\
1341	-18.311\\
1342	-13.428\\
1343	-6.104\\
1344	-6.104\\
1345	-6.104\\
1346	-9.766\\
1347	-19.531\\
1348	-15.869\\
1349	-14.648\\
1350	-13.428\\
1351	-17.09\\
1352	-13.428\\
1353	-9.766\\
1354	-12.207\\
1355	-9.766\\
1356	-8.545\\
1357	-17.09\\
1358	-20.752\\
1359	-17.09\\
1360	-10.986\\
1361	-10.986\\
1362	-13.428\\
1363	-9.766\\
1364	-10.986\\
1365	-28.076\\
1366	-20.752\\
1367	-25.635\\
1368	-35.4\\
1369	-30.518\\
1370	-24.414\\
1371	-18.311\\
1372	-18.311\\
1373	-19.531\\
1374	-21.973\\
1375	-25.635\\
1376	-25.635\\
1377	-31.738\\
1378	-34.18\\
1379	-41.504\\
1380	-32.959\\
1381	-28.076\\
1382	-35.4\\
1383	-26.855\\
1384	-29.297\\
1385	-28.076\\
1386	-17.09\\
1387	-10.986\\
1388	-12.207\\
1389	-17.09\\
1390	-14.648\\
1391	-9.766\\
1392	-13.428\\
1393	-13.428\\
1394	-7.324\\
1395	-8.545\\
1396	-12.207\\
1397	-7.324\\
1398	-14.648\\
1399	-20.752\\
1400	-13.428\\
1401	-21.973\\
1402	-30.518\\
1403	-29.297\\
1404	-35.4\\
1405	-29.297\\
1406	-28.076\\
1407	-23.193\\
1408	-12.207\\
1409	-14.648\\
1410	-15.869\\
1411	-10.986\\
1412	-12.207\\
1413	-13.428\\
1414	-3.662\\
1415	-8.545\\
1416	-17.09\\
1417	-19.531\\
1418	-20.752\\
1419	-23.193\\
1420	-20.752\\
1421	-23.193\\
1422	-18.311\\
1423	-15.869\\
1424	-15.869\\
1425	-12.207\\
1426	-18.311\\
1427	-17.09\\
1428	-12.207\\
1429	-17.09\\
1430	-23.193\\
1431	-17.09\\
1432	-13.428\\
1433	-12.207\\
1434	-14.648\\
1435	-7.324\\
1436	-8.545\\
1437	-12.207\\
1438	-17.09\\
1439	-17.09\\
1440	-14.648\\
1441	-15.869\\
1442	-18.311\\
1443	-12.207\\
1444	-9.766\\
1445	-18.311\\
1446	-26.855\\
1447	-26.855\\
1448	-21.973\\
1449	-19.531\\
1450	-17.09\\
1451	-14.648\\
1452	-13.428\\
1453	-17.09\\
1454	-17.09\\
1455	-10.986\\
1456	-14.648\\
1457	-18.311\\
1458	-19.531\\
1459	-21.973\\
1460	-21.973\\
1461	-29.297\\
1462	-35.4\\
1463	-36.621\\
1464	-26.855\\
1465	-14.648\\
1466	-10.986\\
1467	-7.324\\
1468	-10.986\\
1469	-10.986\\
1470	-14.648\\
1471	-13.428\\
1472	-12.207\\
1473	-15.869\\
1474	-20.752\\
1475	-20.752\\
1476	-24.414\\
1477	-31.738\\
1478	-29.297\\
1479	-26.855\\
1480	-18.311\\
1481	-17.09\\
1482	-20.752\\
1483	-20.752\\
1484	-17.09\\
1485	-14.648\\
1486	-18.311\\
1487	-12.207\\
1488	-13.428\\
1489	-20.752\\
1490	-17.09\\
1491	-18.311\\
1492	-32.959\\
1493	-25.635\\
1494	-20.752\\
1495	-28.076\\
1496	-28.076\\
1497	-19.531\\
1498	-13.428\\
1499	-13.428\\
1500	-19.531\\
};
\end{axis}

\begin{axis}[%
width=5.5cm,
height=2.1cm,
at={(0cm,11.440678cm)},
scale only axis,
xmin=1000,
xmax=1500,
xlabel={\small Time, sec},
ymin=-7,
ymax=-4.999,
ylabel={\small $\Delta \mathrm{x}$, mm},
legend style={legend cell align=left,align=left,draw=white!15!black}
]
\addplot [color=mycolor1,line width=0.5pt,solid,forget plot]
  table[row sep=crcr]{%
1000	-6.134\\
1001	-6.244\\
1002	-6.171\\
1003	-6.134\\
1004	-6.244\\
1005	-6.262\\
1006	-6.317\\
1007	-6.262\\
1008	-5.969\\
1009	-5.933\\
1010	-5.951\\
1011	-5.969\\
1012	-5.951\\
1013	-5.603\\
1014	-5.273\\
1015	-5.145\\
1016	-5.511\\
1017	-5.914\\
1018	-6.024\\
1019	-6.061\\
1020	-5.951\\
1021	-5.731\\
1022	-5.988\\
1023	-5.933\\
1024	-5.75\\
1025	-5.933\\
1026	-5.951\\
1027	-5.878\\
1028	-6.042\\
1029	-6.116\\
1030	-5.988\\
1031	-6.116\\
1032	-6.152\\
1033	-6.061\\
1034	-5.969\\
1035	-5.859\\
1036	-5.878\\
1037	-5.969\\
1038	-5.969\\
1039	-5.988\\
1040	-5.988\\
1041	-6.024\\
1042	-6.189\\
1043	-6.207\\
1044	-6.042\\
1045	-5.969\\
1046	-5.75\\
1047	-5.64\\
1048	-5.75\\
1049	-5.695\\
1050	-5.658\\
1051	-5.786\\
1052	-5.878\\
1053	-5.878\\
1054	-6.061\\
1055	-6.042\\
1056	-5.804\\
1057	-5.658\\
1058	-5.713\\
1059	-5.768\\
1060	-5.585\\
1061	-5.493\\
1062	-5.603\\
1063	-5.585\\
1064	-5.475\\
1065	-5.548\\
1066	-5.621\\
1067	-5.823\\
1068	-5.896\\
1069	-5.951\\
1070	-5.969\\
1071	-6.061\\
1072	-6.024\\
1073	-5.951\\
1074	-5.914\\
1075	-5.896\\
1076	-5.878\\
1077	-6.097\\
1078	-6.39\\
1079	-6.5\\
1080	-6.537\\
1081	-6.335\\
1082	-6.427\\
1083	-6.592\\
1084	-6.647\\
1085	-6.555\\
1086	-6.647\\
1087	-6.866\\
1088	-6.775\\
1089	-6.665\\
1090	-6.445\\
1091	-6.244\\
1092	-6.116\\
1093	-6.134\\
1094	-6.061\\
1095	-5.804\\
1096	-5.676\\
1097	-5.713\\
1098	-5.878\\
1099	-5.896\\
1100	-5.896\\
1101	-6.006\\
1102	-6.079\\
1103	-6.226\\
1104	-6.226\\
1105	-6.281\\
1106	-6.226\\
1107	-6.116\\
1108	-6.152\\
1109	-6.061\\
1110	-5.969\\
1111	-6.024\\
1112	-6.024\\
1113	-5.914\\
1114	-5.878\\
1115	-5.768\\
1116	-5.75\\
1117	-5.75\\
1118	-5.64\\
1119	-5.658\\
1120	-5.75\\
1121	-5.951\\
1122	-6.042\\
1123	-6.097\\
1124	-5.933\\
1125	-6.006\\
1126	-6.244\\
1127	-6.226\\
1128	-6.244\\
1129	-6.299\\
1130	-6.061\\
1131	-5.859\\
1132	-5.896\\
1133	-5.969\\
1134	-6.226\\
1135	-6.409\\
1136	-6.464\\
1137	-6.372\\
1138	-6.335\\
1139	-6.281\\
1140	-6.262\\
1141	-6.262\\
1142	-6.116\\
1143	-5.988\\
1144	-5.914\\
1145	-5.841\\
1146	-5.859\\
1147	-5.988\\
1148	-6.226\\
1149	-6.207\\
1150	-6.042\\
1151	-5.951\\
1152	-5.896\\
1153	-5.695\\
1154	-5.475\\
1155	-5.438\\
1156	-5.64\\
1157	-5.621\\
1158	-5.621\\
1159	-5.64\\
1160	-5.64\\
1161	-5.53\\
1162	-5.383\\
1163	-5.219\\
1164	-5.31\\
1165	-5.695\\
1166	-6.006\\
1167	-6.134\\
1168	-6.024\\
1169	-5.823\\
1170	-5.64\\
1171	-5.438\\
1172	-5.658\\
1173	-5.713\\
1174	-5.841\\
1175	-6.024\\
1176	-6.372\\
1177	-6.61\\
1178	-6.592\\
1179	-6.573\\
1180	-6.354\\
1181	-6.354\\
1182	-6.372\\
1183	-6.409\\
1184	-6.335\\
1185	-6.244\\
1186	-6.226\\
1187	-6.171\\
1188	-6.207\\
1189	-6.116\\
1190	-6.281\\
1191	-6.445\\
1192	-6.335\\
1193	-6.097\\
1194	-6.024\\
1195	-6.244\\
1196	-6.409\\
1197	-6.555\\
1198	-6.647\\
1199	-6.683\\
1200	-6.573\\
1201	-6.5\\
1202	-6.537\\
1203	-6.628\\
1204	-6.464\\
1205	-6.171\\
1206	-6.226\\
1207	-6.189\\
1208	-5.988\\
1209	-5.988\\
1210	-6.061\\
1211	-5.988\\
1212	-5.878\\
1213	-5.896\\
1214	-5.878\\
1215	-5.933\\
1216	-6.116\\
1217	-6.134\\
1218	-6.006\\
1219	-5.951\\
1220	-6.097\\
1221	-6.445\\
1222	-6.39\\
1223	-6.152\\
1224	-5.969\\
1225	-5.988\\
1226	-5.951\\
1227	-5.841\\
1228	-5.804\\
1229	-5.859\\
1230	-5.951\\
1231	-6.042\\
1232	-5.914\\
1233	-6.061\\
1234	-6.207\\
1235	-6.189\\
1236	-6.134\\
1237	-6.116\\
1238	-6.262\\
1239	-6.134\\
1240	-5.75\\
1241	-5.695\\
1242	-5.768\\
1243	-5.896\\
1244	-5.896\\
1245	-5.786\\
1246	-5.914\\
1247	-5.951\\
1248	-5.841\\
1249	-5.859\\
1250	-5.859\\
1251	-5.786\\
1252	-5.823\\
1253	-5.676\\
1254	-5.64\\
1255	-5.548\\
1256	-5.53\\
1257	-5.713\\
1258	-5.841\\
1259	-6.006\\
1260	-6.244\\
1261	-6.152\\
1262	-5.878\\
1263	-5.713\\
1264	-5.64\\
1265	-5.75\\
1266	-5.621\\
1267	-5.585\\
1268	-5.676\\
1269	-5.969\\
1270	-6.024\\
1271	-6.152\\
1272	-6.061\\
1273	-5.969\\
1274	-5.713\\
1275	-5.566\\
1276	-5.42\\
1277	-5.347\\
1278	-5.402\\
1279	-5.585\\
1280	-5.731\\
1281	-5.768\\
1282	-5.823\\
1283	-6.097\\
1284	-6.152\\
1285	-6.024\\
1286	-6.061\\
1287	-6.097\\
1288	-6.244\\
1289	-6.189\\
1290	-6.006\\
1291	-5.676\\
1292	-5.42\\
1293	-5.365\\
1294	-5.383\\
1295	-5.402\\
1296	-5.383\\
1297	-5.457\\
1298	-5.676\\
1299	-5.713\\
1300	-5.731\\
1301	-5.786\\
1302	-5.64\\
1303	-5.365\\
1304	-5.511\\
1305	-5.768\\
1306	-5.823\\
1307	-5.878\\
1308	-5.859\\
1309	-5.75\\
1310	-5.841\\
1311	-6.006\\
1312	-6.079\\
1313	-5.933\\
1314	-6.116\\
1315	-6.134\\
1316	-6.061\\
1317	-6.134\\
1318	-6.152\\
1319	-6.061\\
1320	-5.969\\
1321	-6.171\\
1322	-6.409\\
1323	-6.372\\
1324	-6.116\\
1325	-6.024\\
1326	-6.006\\
1327	-6.079\\
1328	-6.171\\
1329	-6.079\\
1330	-6.061\\
1331	-6.171\\
1332	-6.244\\
1333	-6.116\\
1334	-5.969\\
1335	-6.024\\
1336	-6.299\\
1337	-6.317\\
1338	-6.299\\
1339	-6.134\\
1340	-5.988\\
1341	-5.951\\
1342	-5.786\\
1343	-5.53\\
1344	-5.347\\
1345	-5.237\\
1346	-5.457\\
1347	-5.676\\
1348	-5.841\\
1349	-5.768\\
1350	-5.768\\
1351	-5.841\\
1352	-5.676\\
1353	-5.621\\
1354	-5.621\\
1355	-5.511\\
1356	-5.566\\
1357	-5.786\\
1358	-5.951\\
1359	-5.823\\
1360	-5.658\\
1361	-5.585\\
1362	-5.585\\
1363	-5.475\\
1364	-5.713\\
1365	-6.097\\
1366	-6.079\\
1367	-6.189\\
1368	-6.335\\
1369	-6.39\\
1370	-6.299\\
1371	-6.134\\
1372	-6.079\\
1373	-6.079\\
1374	-6.134\\
1375	-6.207\\
1376	-6.226\\
1377	-6.372\\
1378	-6.519\\
1379	-6.647\\
1380	-6.537\\
1381	-6.5\\
1382	-6.573\\
1383	-6.5\\
1384	-6.519\\
1385	-6.482\\
1386	-6.207\\
1387	-5.914\\
1388	-5.823\\
1389	-5.933\\
1390	-5.896\\
1391	-5.731\\
1392	-5.786\\
1393	-5.713\\
1394	-5.566\\
1395	-5.585\\
1396	-5.621\\
1397	-5.548\\
1398	-5.695\\
1399	-5.896\\
1400	-5.823\\
1401	-6.061\\
1402	-6.372\\
1403	-6.39\\
1404	-6.5\\
1405	-6.427\\
1406	-6.39\\
1407	-6.262\\
1408	-5.988\\
1409	-5.969\\
1410	-5.896\\
1411	-5.768\\
1412	-5.804\\
1413	-5.621\\
1414	-5.402\\
1415	-5.347\\
1416	-5.64\\
1417	-5.841\\
1418	-5.969\\
1419	-6.024\\
1420	-6.024\\
1421	-6.079\\
1422	-6.042\\
1423	-5.933\\
1424	-5.933\\
1425	-5.859\\
1426	-5.988\\
1427	-5.896\\
1428	-5.786\\
1429	-5.878\\
1430	-6.061\\
1431	-6.006\\
1432	-5.878\\
1433	-5.804\\
1434	-5.823\\
1435	-5.713\\
1436	-5.621\\
1437	-5.731\\
1438	-5.823\\
1439	-5.878\\
1440	-5.823\\
1441	-5.896\\
1442	-5.914\\
1443	-5.713\\
1444	-5.658\\
1445	-5.914\\
1446	-6.152\\
1447	-6.189\\
1448	-6.134\\
1449	-6.097\\
1450	-6.024\\
1451	-5.951\\
1452	-5.878\\
1453	-5.933\\
1454	-5.933\\
1455	-5.786\\
1456	-5.859\\
1457	-5.933\\
1458	-6.006\\
1459	-6.079\\
1460	-6.097\\
1461	-6.244\\
1462	-6.445\\
1463	-6.592\\
1464	-6.39\\
1465	-6.116\\
1466	-5.841\\
1467	-5.566\\
1468	-5.603\\
1469	-5.603\\
1470	-5.823\\
1471	-5.859\\
1472	-5.804\\
1473	-5.896\\
1474	-5.969\\
1475	-6.079\\
1476	-6.152\\
1477	-6.372\\
1478	-6.427\\
1479	-6.299\\
1480	-6.134\\
1481	-6.061\\
1482	-6.061\\
1483	-6.134\\
1484	-6.061\\
1485	-5.988\\
1486	-5.969\\
1487	-5.768\\
1488	-5.878\\
1489	-6.079\\
1490	-6.006\\
1491	-5.969\\
1492	-6.262\\
1493	-6.244\\
1494	-6.171\\
1495	-6.317\\
1496	-6.354\\
1497	-6.152\\
1498	-5.914\\
1499	-5.841\\
1500	-6.024\\
};
\end{axis}

\begin{axis}[%
width=5.5cm,
height=2.1cm,
at={(0cm,3.813559cm)},
scale only axis,
xmin=1000,
xmax=1500,
xlabel={\small Time, sec},
ymin=-7,
ymax=-4.999,
ylabel={\small $\Delta \mathrm{x}$, mm},
legend style={legend cell align=left,align=left,draw=white!15!black}
]
\addplot [color=mycolor1,line width=0.5pt,solid,forget plot]
  table[row sep=crcr]{%
1000	-6.134\\
1001	-6.226\\
1002	-6.189\\
1003	-6.134\\
1004	-6.226\\
1005	-6.262\\
1006	-6.299\\
1007	-6.244\\
1008	-5.951\\
1009	-5.914\\
1010	-5.914\\
1011	-5.988\\
1012	-5.933\\
1013	-5.603\\
1014	-5.31\\
1015	-5.145\\
1016	-5.511\\
1017	-5.914\\
1018	-6.024\\
1019	-6.061\\
1020	-5.969\\
1021	-5.75\\
1022	-5.969\\
1023	-5.896\\
1024	-5.731\\
1025	-5.933\\
1026	-5.969\\
1027	-5.896\\
1028	-6.061\\
1029	-6.097\\
1030	-5.988\\
1031	-6.116\\
1032	-6.152\\
1033	-6.079\\
1034	-5.969\\
1035	-5.878\\
1036	-5.878\\
1037	-5.969\\
1038	-5.969\\
1039	-5.988\\
1040	-6.006\\
1041	-6.024\\
1042	-6.189\\
1043	-6.207\\
1044	-6.042\\
1045	-5.951\\
1046	-5.731\\
1047	-5.64\\
1048	-5.731\\
1049	-5.676\\
1050	-5.64\\
1051	-5.768\\
1052	-5.859\\
1053	-5.859\\
1054	-6.061\\
1055	-6.042\\
1056	-5.804\\
1057	-5.658\\
1058	-5.676\\
1059	-5.75\\
1060	-5.585\\
1061	-5.475\\
1062	-5.603\\
1063	-5.566\\
1064	-5.475\\
1065	-5.53\\
1066	-5.603\\
1067	-5.859\\
1068	-5.914\\
1069	-6.006\\
1070	-6.024\\
1071	-6.079\\
1072	-6.024\\
1073	-5.988\\
1074	-5.951\\
1075	-5.896\\
1076	-5.896\\
1077	-6.097\\
1078	-6.39\\
1079	-6.519\\
1080	-6.537\\
1081	-6.335\\
1082	-6.445\\
1083	-6.61\\
1084	-6.647\\
1085	-6.592\\
1086	-6.647\\
1087	-6.866\\
1088	-6.757\\
1089	-6.665\\
1090	-6.445\\
1091	-6.226\\
1092	-6.097\\
1093	-6.152\\
1094	-6.042\\
1095	-5.804\\
1096	-5.695\\
1097	-5.713\\
1098	-5.878\\
1099	-5.914\\
1100	-5.896\\
1101	-6.024\\
1102	-6.061\\
1103	-6.244\\
1104	-6.226\\
1105	-6.299\\
1106	-6.207\\
1107	-6.116\\
1108	-6.152\\
1109	-6.042\\
1110	-5.969\\
1111	-6.024\\
1112	-6.042\\
1113	-5.896\\
1114	-5.878\\
1115	-5.75\\
1116	-5.713\\
1117	-5.75\\
1118	-5.64\\
1119	-5.658\\
1120	-5.731\\
1121	-5.951\\
1122	-6.061\\
1123	-6.116\\
1124	-5.914\\
1125	-5.969\\
1126	-6.244\\
1127	-6.207\\
1128	-6.207\\
1129	-6.281\\
1130	-6.061\\
1131	-5.859\\
1132	-5.914\\
1133	-5.988\\
1134	-6.207\\
1135	-6.427\\
1136	-6.482\\
1137	-6.335\\
1138	-6.317\\
1139	-6.299\\
1140	-6.262\\
1141	-6.244\\
1142	-6.079\\
1143	-5.951\\
1144	-5.896\\
1145	-5.841\\
1146	-5.859\\
1147	-5.988\\
1148	-6.226\\
1149	-6.207\\
1150	-6.042\\
1151	-5.933\\
1152	-5.896\\
1153	-5.695\\
1154	-5.475\\
1155	-5.438\\
1156	-5.64\\
1157	-5.585\\
1158	-5.566\\
1159	-5.621\\
1160	-5.621\\
1161	-5.53\\
1162	-5.365\\
1163	-5.2\\
1164	-5.292\\
1165	-5.658\\
1166	-5.988\\
1167	-6.134\\
1168	-6.024\\
1169	-5.841\\
1170	-5.658\\
1171	-5.457\\
1172	-5.621\\
1173	-5.731\\
1174	-5.859\\
1175	-6.024\\
1176	-6.372\\
1177	-6.628\\
1178	-6.61\\
1179	-6.592\\
1180	-6.39\\
1181	-6.372\\
1182	-6.427\\
1183	-6.427\\
1184	-6.335\\
1185	-6.262\\
1186	-6.262\\
1187	-6.189\\
1188	-6.226\\
1189	-6.152\\
1190	-6.299\\
1191	-6.464\\
1192	-6.335\\
1193	-6.097\\
1194	-6.042\\
1195	-6.262\\
1196	-6.409\\
1197	-6.573\\
1198	-6.647\\
1199	-6.683\\
1200	-6.573\\
1201	-6.5\\
1202	-6.537\\
1203	-6.61\\
1204	-6.464\\
1205	-6.171\\
1206	-6.226\\
1207	-6.189\\
1208	-5.988\\
1209	-5.988\\
1210	-6.061\\
1211	-5.969\\
1212	-5.878\\
1213	-5.914\\
1214	-5.878\\
1215	-5.951\\
1216	-6.134\\
1217	-6.152\\
1218	-6.006\\
1219	-5.951\\
1220	-6.116\\
1221	-6.464\\
1222	-6.445\\
1223	-6.189\\
1224	-6.024\\
1225	-6.006\\
1226	-5.988\\
1227	-5.859\\
1228	-5.823\\
1229	-5.878\\
1230	-5.969\\
1231	-6.024\\
1232	-5.933\\
1233	-6.042\\
1234	-6.189\\
1235	-6.189\\
1236	-6.116\\
1237	-6.134\\
1238	-6.244\\
1239	-6.116\\
1240	-5.75\\
1241	-5.695\\
1242	-5.75\\
1243	-5.878\\
1244	-5.878\\
1245	-5.75\\
1246	-5.878\\
1247	-5.914\\
1248	-5.804\\
1249	-5.841\\
1250	-5.823\\
1251	-5.786\\
1252	-5.823\\
1253	-5.676\\
1254	-5.64\\
1255	-5.548\\
1256	-5.493\\
1257	-5.731\\
1258	-5.841\\
1259	-6.024\\
1260	-6.244\\
1261	-6.134\\
1262	-5.878\\
1263	-5.713\\
1264	-5.64\\
1265	-5.768\\
1266	-5.658\\
1267	-5.64\\
1268	-5.695\\
1269	-5.988\\
1270	-6.042\\
1271	-6.171\\
1272	-6.061\\
1273	-5.969\\
1274	-5.713\\
1275	-5.585\\
1276	-5.42\\
1277	-5.347\\
1278	-5.383\\
1279	-5.603\\
1280	-5.695\\
1281	-5.768\\
1282	-5.823\\
1283	-6.097\\
1284	-6.152\\
1285	-6.024\\
1286	-6.079\\
1287	-6.116\\
1288	-6.226\\
1289	-6.189\\
1290	-6.024\\
1291	-5.658\\
1292	-5.42\\
1293	-5.347\\
1294	-5.365\\
1295	-5.402\\
1296	-5.383\\
1297	-5.475\\
1298	-5.676\\
1299	-5.713\\
1300	-5.713\\
1301	-5.786\\
1302	-5.621\\
1303	-5.365\\
1304	-5.53\\
1305	-5.768\\
1306	-5.823\\
1307	-5.878\\
1308	-5.878\\
1309	-5.731\\
1310	-5.804\\
1311	-6.042\\
1312	-6.079\\
1313	-5.969\\
1314	-6.134\\
1315	-6.134\\
1316	-6.079\\
1317	-6.134\\
1318	-6.152\\
1319	-6.061\\
1320	-5.988\\
1321	-6.134\\
1322	-6.372\\
1323	-6.335\\
1324	-6.097\\
1325	-6.006\\
1326	-5.988\\
1327	-6.079\\
1328	-6.152\\
1329	-6.061\\
1330	-6.061\\
1331	-6.152\\
1332	-6.226\\
1333	-6.061\\
1334	-5.951\\
1335	-6.061\\
1336	-6.281\\
1337	-6.317\\
1338	-6.335\\
1339	-6.116\\
1340	-5.988\\
1341	-5.988\\
1342	-5.786\\
1343	-5.548\\
1344	-5.347\\
1345	-5.219\\
1346	-5.457\\
1347	-5.695\\
1348	-5.841\\
1349	-5.768\\
1350	-5.75\\
1351	-5.823\\
1352	-5.64\\
1353	-5.603\\
1354	-5.603\\
1355	-5.493\\
1356	-5.53\\
1357	-5.75\\
1358	-5.933\\
1359	-5.804\\
1360	-5.658\\
1361	-5.548\\
1362	-5.566\\
1363	-5.457\\
1364	-5.713\\
1365	-6.079\\
1366	-6.097\\
1367	-6.207\\
1368	-6.354\\
1369	-6.409\\
1370	-6.299\\
1371	-6.134\\
1372	-6.079\\
1373	-6.079\\
1374	-6.134\\
1375	-6.207\\
1376	-6.244\\
1377	-6.409\\
1378	-6.537\\
1379	-6.683\\
1380	-6.573\\
1381	-6.537\\
1382	-6.61\\
1383	-6.482\\
1384	-6.519\\
1385	-6.482\\
1386	-6.226\\
1387	-5.951\\
1388	-5.841\\
1389	-5.969\\
1390	-5.933\\
1391	-5.768\\
1392	-5.804\\
1393	-5.731\\
1394	-5.585\\
1395	-5.585\\
1396	-5.621\\
1397	-5.566\\
1398	-5.713\\
1399	-5.896\\
1400	-5.841\\
1401	-6.061\\
1402	-6.335\\
1403	-6.39\\
1404	-6.519\\
1405	-6.409\\
1406	-6.39\\
1407	-6.226\\
1408	-5.969\\
1409	-5.969\\
1410	-5.914\\
1411	-5.75\\
1412	-5.823\\
1413	-5.64\\
1414	-5.383\\
1415	-5.365\\
1416	-5.64\\
1417	-5.841\\
1418	-5.988\\
1419	-6.024\\
1420	-6.024\\
1421	-6.097\\
1422	-6.061\\
1423	-5.951\\
1424	-5.933\\
1425	-5.878\\
1426	-6.006\\
1427	-5.896\\
1428	-5.786\\
1429	-5.878\\
1430	-6.079\\
1431	-6.024\\
1432	-5.896\\
1433	-5.804\\
1434	-5.823\\
1435	-5.695\\
1436	-5.603\\
1437	-5.695\\
1438	-5.804\\
1439	-5.878\\
1440	-5.804\\
1441	-5.859\\
1442	-5.878\\
1443	-5.658\\
1444	-5.621\\
1445	-5.859\\
1446	-6.116\\
1447	-6.171\\
1448	-6.116\\
1449	-6.079\\
1450	-6.006\\
1451	-5.933\\
1452	-5.859\\
1453	-5.933\\
1454	-5.951\\
1455	-5.768\\
1456	-5.841\\
1457	-5.951\\
1458	-6.006\\
1459	-6.061\\
1460	-6.097\\
1461	-6.262\\
1462	-6.445\\
1463	-6.592\\
1464	-6.464\\
1465	-6.116\\
1466	-5.859\\
1467	-5.585\\
1468	-5.621\\
1469	-5.621\\
1470	-5.823\\
1471	-5.859\\
1472	-5.804\\
1473	-5.896\\
1474	-5.988\\
1475	-6.079\\
1476	-6.152\\
1477	-6.39\\
1478	-6.427\\
1479	-6.299\\
1480	-6.116\\
1481	-6.042\\
1482	-6.061\\
1483	-6.134\\
1484	-6.024\\
1485	-5.969\\
1486	-5.969\\
1487	-5.713\\
1488	-5.768\\
1489	-6.006\\
1490	-5.933\\
1491	-5.933\\
1492	-6.226\\
1493	-6.226\\
1494	-6.171\\
1495	-6.299\\
1496	-6.335\\
1497	-6.189\\
1498	-5.933\\
1499	-5.859\\
1500	-6.042\\
};
\end{axis}

\begin{axis}[%
width=5.5cm,
height=2.1cm,
at={(0cm,7.627119cm)},
scale only axis,
xmin=1000,
xmax=1500,
xlabel={\small Time, sec},
ymin=-7,
ymax=-4.999,
ylabel={\small $\Delta \mathrm{x}$, mm},
legend style={legend cell align=left,align=left,draw=white!15!black}
]
\addplot [color=mycolor1,line width=0.5pt,solid,forget plot]
  table[row sep=crcr]{%
1000	-6.116\\
1001	-6.207\\
1002	-6.189\\
1003	-6.134\\
1004	-6.244\\
1005	-6.281\\
1006	-6.335\\
1007	-6.262\\
1008	-5.988\\
1009	-5.933\\
1010	-5.933\\
1011	-5.988\\
1012	-5.951\\
1013	-5.603\\
1014	-5.292\\
1015	-5.109\\
1016	-5.475\\
1017	-5.878\\
1018	-5.988\\
1019	-6.042\\
1020	-5.933\\
1021	-5.713\\
1022	-5.951\\
1023	-5.896\\
1024	-5.75\\
1025	-5.878\\
1026	-5.933\\
1027	-5.841\\
1028	-6.024\\
1029	-6.079\\
1030	-5.969\\
1031	-6.097\\
1032	-6.152\\
1033	-6.061\\
1034	-5.969\\
1035	-5.878\\
1036	-5.878\\
1037	-5.969\\
1038	-5.988\\
1039	-5.969\\
1040	-5.988\\
1041	-6.024\\
1042	-6.189\\
1043	-6.226\\
1044	-6.061\\
1045	-5.969\\
1046	-5.75\\
1047	-5.64\\
1048	-5.731\\
1049	-5.695\\
1050	-5.676\\
1051	-5.786\\
1052	-5.878\\
1053	-5.878\\
1054	-6.079\\
1055	-6.061\\
1056	-5.804\\
1057	-5.658\\
1058	-5.713\\
1059	-5.786\\
1060	-5.621\\
1061	-5.493\\
1062	-5.603\\
1063	-5.585\\
1064	-5.493\\
1065	-5.548\\
1066	-5.621\\
1067	-5.823\\
1068	-5.896\\
1069	-5.969\\
1070	-5.969\\
1071	-6.042\\
1072	-6.006\\
1073	-5.951\\
1074	-5.914\\
1075	-5.896\\
1076	-5.878\\
1077	-6.097\\
1078	-6.409\\
1079	-6.519\\
1080	-6.537\\
1081	-6.317\\
1082	-6.427\\
1083	-6.592\\
1084	-6.647\\
1085	-6.555\\
1086	-6.647\\
1087	-6.848\\
1088	-6.793\\
1089	-6.665\\
1090	-6.445\\
1091	-6.262\\
1092	-6.134\\
1093	-6.134\\
1094	-6.061\\
1095	-5.823\\
1096	-5.676\\
1097	-5.713\\
1098	-5.878\\
1099	-5.914\\
1100	-5.914\\
1101	-6.006\\
1102	-6.061\\
1103	-6.262\\
1104	-6.262\\
1105	-6.299\\
1106	-6.226\\
1107	-6.116\\
1108	-6.171\\
1109	-6.079\\
1110	-5.969\\
1111	-6.024\\
1112	-6.042\\
1113	-5.951\\
1114	-5.896\\
1115	-5.804\\
1116	-5.75\\
1117	-5.804\\
1118	-5.658\\
1119	-5.658\\
1120	-5.768\\
1121	-5.969\\
1122	-6.042\\
1123	-6.079\\
1124	-5.933\\
1125	-6.006\\
1126	-6.262\\
1127	-6.207\\
1128	-6.244\\
1129	-6.262\\
1130	-6.079\\
1131	-5.878\\
1132	-5.896\\
1133	-5.969\\
1134	-6.207\\
1135	-6.409\\
1136	-6.445\\
1137	-6.354\\
1138	-6.317\\
1139	-6.299\\
1140	-6.262\\
1141	-6.244\\
1142	-6.097\\
1143	-5.969\\
1144	-5.914\\
1145	-5.859\\
1146	-5.859\\
1147	-6.006\\
1148	-6.244\\
1149	-6.244\\
1150	-6.061\\
1151	-5.933\\
1152	-5.896\\
1153	-5.695\\
1154	-5.493\\
1155	-5.457\\
1156	-5.658\\
1157	-5.64\\
1158	-5.621\\
1159	-5.658\\
1160	-5.64\\
1161	-5.548\\
1162	-5.365\\
1163	-5.219\\
1164	-5.31\\
1165	-5.676\\
1166	-5.988\\
1167	-6.134\\
1168	-5.988\\
1169	-5.804\\
1170	-5.621\\
1171	-5.402\\
1172	-5.621\\
1173	-5.695\\
1174	-5.841\\
1175	-6.024\\
1176	-6.372\\
1177	-6.61\\
1178	-6.573\\
1179	-6.555\\
1180	-6.409\\
1181	-6.354\\
1182	-6.39\\
1183	-6.427\\
1184	-6.335\\
1185	-6.262\\
1186	-6.244\\
1187	-6.189\\
1188	-6.244\\
1189	-6.152\\
1190	-6.281\\
1191	-6.445\\
1192	-6.335\\
1193	-6.097\\
1194	-6.042\\
1195	-6.244\\
1196	-6.409\\
1197	-6.555\\
1198	-6.647\\
1199	-6.683\\
1200	-6.573\\
1201	-6.5\\
1202	-6.537\\
1203	-6.61\\
1204	-6.445\\
1205	-6.171\\
1206	-6.207\\
1207	-6.171\\
1208	-5.969\\
1209	-5.988\\
1210	-6.061\\
1211	-5.988\\
1212	-5.859\\
1213	-5.896\\
1214	-5.878\\
1215	-5.951\\
1216	-6.134\\
1217	-6.134\\
1218	-6.006\\
1219	-5.951\\
1220	-6.116\\
1221	-6.427\\
1222	-6.409\\
1223	-6.171\\
1224	-6.006\\
1225	-5.969\\
1226	-5.969\\
1227	-5.841\\
1228	-5.823\\
1229	-5.859\\
1230	-5.969\\
1231	-6.042\\
1232	-5.896\\
1233	-6.006\\
1234	-6.152\\
1235	-6.171\\
1236	-6.097\\
1237	-6.097\\
1238	-6.244\\
1239	-6.116\\
1240	-5.75\\
1241	-5.695\\
1242	-5.75\\
1243	-5.896\\
1244	-5.896\\
1245	-5.786\\
1246	-5.896\\
1247	-5.933\\
1248	-5.823\\
1249	-5.859\\
1250	-5.841\\
1251	-5.804\\
1252	-5.823\\
1253	-5.676\\
1254	-5.64\\
1255	-5.548\\
1256	-5.53\\
1257	-5.75\\
1258	-5.878\\
1259	-6.042\\
1260	-6.262\\
1261	-6.152\\
1262	-5.896\\
1263	-5.713\\
1264	-5.64\\
1265	-5.75\\
1266	-5.621\\
1267	-5.585\\
1268	-5.676\\
1269	-5.914\\
1270	-5.988\\
1271	-6.116\\
1272	-6.024\\
1273	-5.933\\
1274	-5.695\\
1275	-5.566\\
1276	-5.365\\
1277	-5.292\\
1278	-5.347\\
1279	-5.621\\
1280	-5.695\\
1281	-5.786\\
1282	-5.841\\
1283	-6.097\\
1284	-6.171\\
1285	-6.024\\
1286	-6.061\\
1287	-6.097\\
1288	-6.244\\
1289	-6.207\\
1290	-6.024\\
1291	-5.731\\
1292	-5.475\\
1293	-5.402\\
1294	-5.42\\
1295	-5.42\\
1296	-5.402\\
1297	-5.475\\
1298	-5.676\\
1299	-5.731\\
1300	-5.731\\
1301	-5.804\\
1302	-5.658\\
1303	-5.383\\
1304	-5.548\\
1305	-5.768\\
1306	-5.841\\
1307	-5.878\\
1308	-5.859\\
1309	-5.75\\
1310	-5.823\\
1311	-6.006\\
1312	-6.079\\
1313	-5.951\\
1314	-6.152\\
1315	-6.134\\
1316	-6.079\\
1317	-6.152\\
1318	-6.152\\
1319	-6.061\\
1320	-5.969\\
1321	-6.116\\
1322	-6.39\\
1323	-6.335\\
1324	-6.079\\
1325	-5.969\\
1326	-5.988\\
1327	-6.061\\
1328	-6.152\\
1329	-6.061\\
1330	-6.061\\
1331	-6.171\\
1332	-6.244\\
1333	-6.116\\
1334	-5.988\\
1335	-6.042\\
1336	-6.299\\
1337	-6.317\\
1338	-6.317\\
1339	-6.116\\
1340	-5.988\\
1341	-5.969\\
1342	-5.75\\
1343	-5.548\\
1344	-5.347\\
1345	-5.2\\
1346	-5.438\\
1347	-5.676\\
1348	-5.823\\
1349	-5.75\\
1350	-5.75\\
1351	-5.804\\
1352	-5.676\\
1353	-5.621\\
1354	-5.621\\
1355	-5.53\\
1356	-5.548\\
1357	-5.731\\
1358	-5.914\\
1359	-5.786\\
1360	-5.64\\
1361	-5.548\\
1362	-5.585\\
1363	-5.457\\
1364	-5.731\\
1365	-6.079\\
1366	-6.061\\
1367	-6.189\\
1368	-6.354\\
1369	-6.39\\
1370	-6.299\\
1371	-6.134\\
1372	-6.079\\
1373	-6.079\\
1374	-6.134\\
1375	-6.207\\
1376	-6.244\\
1377	-6.354\\
1378	-6.482\\
1379	-6.628\\
1380	-6.519\\
1381	-6.482\\
1382	-6.573\\
1383	-6.5\\
1384	-6.5\\
1385	-6.464\\
1386	-6.171\\
1387	-5.914\\
1388	-5.841\\
1389	-5.969\\
1390	-5.933\\
1391	-5.768\\
1392	-5.804\\
1393	-5.713\\
1394	-5.566\\
1395	-5.585\\
1396	-5.621\\
1397	-5.53\\
1398	-5.695\\
1399	-5.896\\
1400	-5.841\\
1401	-6.006\\
1402	-6.299\\
1403	-6.354\\
1404	-6.482\\
1405	-6.39\\
1406	-6.354\\
1407	-6.226\\
1408	-5.969\\
1409	-5.969\\
1410	-5.878\\
1411	-5.731\\
1412	-5.786\\
1413	-5.603\\
1414	-5.365\\
1415	-5.347\\
1416	-5.566\\
1417	-5.804\\
1418	-5.951\\
1419	-6.006\\
1420	-6.006\\
1421	-6.061\\
1422	-6.024\\
1423	-5.933\\
1424	-5.933\\
1425	-5.859\\
1426	-5.988\\
1427	-5.896\\
1428	-5.768\\
1429	-5.878\\
1430	-6.079\\
1431	-6.042\\
1432	-5.896\\
1433	-5.804\\
1434	-5.823\\
1435	-5.695\\
1436	-5.603\\
1437	-5.695\\
1438	-5.786\\
1439	-5.878\\
1440	-5.823\\
1441	-5.878\\
1442	-5.896\\
1443	-5.713\\
1444	-5.64\\
1445	-5.859\\
1446	-6.116\\
1447	-6.171\\
1448	-6.116\\
1449	-6.097\\
1450	-6.006\\
1451	-5.951\\
1452	-5.859\\
1453	-5.933\\
1454	-5.933\\
1455	-5.75\\
1456	-5.823\\
1457	-5.896\\
1458	-6.006\\
1459	-6.061\\
1460	-6.079\\
1461	-6.244\\
1462	-6.427\\
1463	-6.573\\
1464	-6.427\\
1465	-6.134\\
1466	-5.878\\
1467	-5.621\\
1468	-5.64\\
1469	-5.658\\
1470	-5.841\\
1471	-5.878\\
1472	-5.804\\
1473	-5.896\\
1474	-5.988\\
1475	-6.079\\
1476	-6.152\\
1477	-6.409\\
1478	-6.445\\
1479	-6.335\\
1480	-6.152\\
1481	-6.061\\
1482	-6.079\\
1483	-6.152\\
1484	-6.061\\
1485	-6.006\\
1486	-5.988\\
1487	-5.768\\
1488	-5.841\\
1489	-6.042\\
1490	-5.969\\
1491	-5.951\\
1492	-6.226\\
1493	-6.207\\
1494	-6.152\\
1495	-6.299\\
1496	-6.335\\
1497	-6.171\\
1498	-5.896\\
1499	-5.841\\
1500	-6.006\\
};
\end{axis}

\begin{axis}[%
width=5.5cm,
height=2.1cm,
at={(7.5cm,15.254237cm)},
scale only axis,
xmin=1000,
xmax=1500,
xlabel={\small Time, sec},
ymin=-60,
ymax=0,
ylabel={\small Load, kN},
legend style={legend cell align=left,align=left,draw=white!15!black}
]
\addplot [color=mycolor1,line width=0.5pt,solid,forget plot]
  table[row sep=crcr]{%
1000	-23.193\\
1001	-28.076\\
1002	-26.855\\
1003	-20.752\\
1004	-29.297\\
1005	-28.076\\
1006	-32.959\\
1007	-25.635\\
1008	-14.648\\
1009	-20.752\\
1010	-17.09\\
1011	-18.311\\
1012	-20.752\\
1013	-7.324\\
1014	-4.883\\
1015	-4.883\\
1016	-13.428\\
1017	-23.193\\
1018	-24.414\\
1019	-25.635\\
1020	-19.531\\
1021	-12.207\\
1022	-24.414\\
1023	-19.531\\
1024	-14.648\\
1025	-20.752\\
1026	-18.311\\
1027	-17.09\\
1028	-29.297\\
1029	-25.635\\
1030	-18.311\\
1031	-28.076\\
1032	-28.076\\
1033	-21.973\\
1034	-18.311\\
1035	-15.869\\
1036	-15.869\\
1037	-21.973\\
1038	-21.973\\
1039	-21.973\\
1040	-21.973\\
1041	-23.193\\
1042	-30.518\\
1043	-29.297\\
1044	-19.531\\
1045	-18.311\\
1046	-10.986\\
1047	-14.648\\
1048	-15.869\\
1049	-12.207\\
1050	-13.428\\
1051	-18.311\\
1052	-19.531\\
1053	-18.311\\
1054	-29.297\\
1055	-21.973\\
1056	-12.207\\
1057	-12.207\\
1058	-13.428\\
1059	-17.09\\
1060	-9.766\\
1061	-10.986\\
1062	-14.648\\
1063	-9.766\\
1064	-7.324\\
1065	-13.428\\
1066	-13.428\\
1067	-21.973\\
1068	-20.752\\
1069	-25.635\\
1070	-21.973\\
1071	-25.635\\
1072	-20.752\\
1073	-20.752\\
1074	-18.311\\
1075	-18.311\\
1076	-17.09\\
1077	-30.518\\
1078	-41.504\\
1079	-41.504\\
1080	-42.725\\
1081	-28.076\\
1082	-37.842\\
1083	-46.387\\
1084	-46.387\\
1085	-35.4\\
1086	-47.607\\
1087	-58.594\\
1088	-43.945\\
1089	-34.18\\
1090	-24.414\\
1091	-20.752\\
1092	-17.09\\
1093	-21.973\\
1094	-17.09\\
1095	-12.207\\
1096	-9.766\\
1097	-12.207\\
1098	-18.311\\
1099	-18.311\\
1100	-18.311\\
1101	-21.973\\
1102	-23.193\\
1103	-30.518\\
1104	-28.076\\
1105	-30.518\\
1106	-25.635\\
1107	-20.752\\
1108	-24.414\\
1109	-18.311\\
1110	-13.428\\
1111	-19.531\\
1112	-20.752\\
1113	-12.207\\
1114	-15.869\\
1115	-12.207\\
1116	-13.428\\
1117	-13.428\\
1118	-9.766\\
1119	-10.986\\
1120	-17.09\\
1121	-23.193\\
1122	-25.635\\
1123	-26.855\\
1124	-15.869\\
1125	-20.752\\
1126	-32.959\\
1127	-24.414\\
1128	-28.076\\
1129	-29.297\\
1130	-18.311\\
1131	-12.207\\
1132	-17.09\\
1133	-18.311\\
1134	-30.518\\
1135	-36.621\\
1136	-36.621\\
1137	-28.076\\
1138	-28.076\\
1139	-25.635\\
1140	-25.635\\
1141	-25.635\\
1142	-20.752\\
1143	-15.869\\
1144	-14.648\\
1145	-13.428\\
1146	-13.428\\
1147	-20.752\\
1148	-31.738\\
1149	-26.855\\
1150	-17.09\\
1151	-15.869\\
1152	-15.869\\
1153	-9.766\\
1154	-7.324\\
1155	-7.324\\
1156	-15.869\\
1157	-10.986\\
1158	-14.648\\
1159	-14.648\\
1160	-13.428\\
1161	-9.766\\
1162	-6.104\\
1163	-4.883\\
1164	-7.324\\
1165	-18.311\\
1166	-26.855\\
1167	-28.076\\
1168	-20.752\\
1169	-13.428\\
1170	-10.986\\
1171	-6.104\\
1172	-12.207\\
1173	-14.648\\
1174	-18.311\\
1175	-26.855\\
1176	-39.063\\
1177	-47.607\\
1178	-42.725\\
1179	-37.842\\
1180	-28.076\\
1181	-30.518\\
1182	-31.738\\
1183	-34.18\\
1184	-24.414\\
1185	-23.193\\
1186	-23.193\\
1187	-21.973\\
1188	-23.193\\
1189	-18.311\\
1190	-30.518\\
1191	-37.842\\
1192	-28.076\\
1193	-18.311\\
1194	-18.311\\
1195	-30.518\\
1196	-35.4\\
1197	-40.283\\
1198	-45.166\\
1199	-45.166\\
1200	-34.18\\
1201	-32.959\\
1202	-36.621\\
1203	-41.504\\
1204	-29.297\\
1205	-17.09\\
1206	-23.193\\
1207	-20.752\\
1208	-13.428\\
1209	-18.311\\
1210	-19.531\\
1211	-14.648\\
1212	-13.428\\
1213	-15.869\\
1214	-13.428\\
1215	-17.09\\
1216	-25.635\\
1217	-25.635\\
1218	-15.869\\
1219	-14.648\\
1220	-24.414\\
1221	-40.283\\
1222	-29.297\\
1223	-20.752\\
1224	-14.648\\
1225	-18.311\\
1226	-15.869\\
1227	-12.207\\
1228	-13.428\\
1229	-15.869\\
1230	-18.311\\
1231	-20.752\\
1232	-15.869\\
1233	-23.193\\
1234	-29.297\\
1235	-25.635\\
1236	-20.752\\
1237	-23.193\\
1238	-30.518\\
1239	-21.973\\
1240	-8.545\\
1241	-13.428\\
1242	-13.428\\
1243	-17.09\\
1244	-17.09\\
1245	-13.428\\
1246	-17.09\\
1247	-19.531\\
1248	-13.428\\
1249	-15.869\\
1250	-15.869\\
1251	-12.207\\
1252	-15.869\\
1253	-9.766\\
1254	-10.986\\
1255	-8.545\\
1256	-8.545\\
1257	-14.648\\
1258	-20.752\\
1259	-24.414\\
1260	-35.4\\
1261	-24.414\\
1262	-13.428\\
1263	-12.207\\
1264	-10.986\\
1265	-13.428\\
1266	-10.986\\
1267	-12.207\\
1268	-14.648\\
1269	-24.414\\
1270	-21.973\\
1271	-26.855\\
1272	-23.193\\
1273	-17.09\\
1274	-13.428\\
1275	-10.986\\
1276	-8.545\\
1277	-6.104\\
1278	-9.766\\
1279	-14.648\\
1280	-18.311\\
1281	-18.311\\
1282	-19.531\\
1283	-29.297\\
1284	-25.635\\
1285	-18.311\\
1286	-24.414\\
1287	-24.414\\
1288	-31.738\\
1289	-26.855\\
1290	-17.09\\
1291	-9.766\\
1292	-6.104\\
1293	-7.324\\
1294	-9.766\\
1295	-8.545\\
1296	-8.545\\
1297	-12.207\\
1298	-17.09\\
1299	-15.869\\
1300	-15.869\\
1301	-19.531\\
1302	-13.428\\
1303	-4.883\\
1304	-9.766\\
1305	-21.973\\
1306	-19.531\\
1307	-21.973\\
1308	-18.311\\
1309	-14.648\\
1310	-19.531\\
1311	-25.635\\
1312	-25.635\\
1313	-19.531\\
1314	-26.855\\
1315	-26.855\\
1316	-23.193\\
1317	-28.076\\
1318	-26.855\\
1319	-20.752\\
1320	-19.531\\
1321	-29.297\\
1322	-40.283\\
1323	-30.518\\
1324	-18.311\\
1325	-17.09\\
1326	-18.311\\
1327	-21.973\\
1328	-26.855\\
1329	-19.531\\
1330	-19.531\\
1331	-26.855\\
1332	-28.076\\
1333	-20.752\\
1334	-14.648\\
1335	-20.752\\
1336	-34.18\\
1337	-30.518\\
1338	-31.738\\
1339	-21.973\\
1340	-17.09\\
1341	-17.09\\
1342	-10.986\\
1343	-6.104\\
1344	-6.104\\
1345	-4.883\\
1346	-10.986\\
1347	-19.531\\
1348	-18.311\\
1349	-15.869\\
1350	-14.648\\
1351	-20.752\\
1352	-14.648\\
1353	-9.766\\
1354	-13.428\\
1355	-8.545\\
1356	-10.986\\
1357	-18.311\\
1358	-23.193\\
1359	-17.09\\
1360	-9.766\\
1361	-10.986\\
1362	-12.207\\
1363	-9.766\\
1364	-13.428\\
1365	-31.738\\
1366	-25.635\\
1367	-31.738\\
1368	-37.842\\
1369	-36.621\\
1370	-26.855\\
1371	-20.752\\
1372	-18.311\\
1373	-21.973\\
1374	-23.193\\
1375	-29.297\\
1376	-29.297\\
1377	-36.621\\
1378	-42.725\\
1379	-46.387\\
1380	-37.842\\
1381	-35.4\\
1382	-40.283\\
1383	-31.738\\
1384	-34.18\\
1385	-31.738\\
1386	-18.311\\
1387	-12.207\\
1388	-13.428\\
1389	-18.311\\
1390	-17.09\\
1391	-8.545\\
1392	-14.648\\
1393	-13.428\\
1394	-6.104\\
1395	-10.986\\
1396	-12.207\\
1397	-7.324\\
1398	-18.311\\
1399	-20.752\\
1400	-14.648\\
1401	-23.193\\
1402	-37.842\\
1403	-32.959\\
1404	-37.842\\
1405	-32.959\\
1406	-29.297\\
1407	-20.752\\
1408	-14.648\\
1409	-17.09\\
1410	-13.428\\
1411	-10.986\\
1412	-14.648\\
1413	-13.428\\
1414	-3.662\\
1415	-7.324\\
1416	-15.869\\
1417	-20.752\\
1418	-21.973\\
1419	-23.193\\
1420	-21.973\\
1421	-26.855\\
1422	-21.973\\
1423	-17.09\\
1424	-17.09\\
1425	-14.648\\
1426	-20.752\\
1427	-18.311\\
1428	-13.428\\
1429	-18.311\\
1430	-26.855\\
1431	-20.752\\
1432	-15.869\\
1433	-13.428\\
1434	-14.648\\
1435	-12.207\\
1436	-8.545\\
1437	-14.648\\
1438	-19.531\\
1439	-18.311\\
1440	-15.869\\
1441	-19.531\\
1442	-18.311\\
1443	-10.986\\
1444	-10.986\\
1445	-21.973\\
1446	-30.518\\
1447	-29.297\\
1448	-23.193\\
1449	-21.973\\
1450	-19.531\\
1451	-18.311\\
1452	-15.869\\
1453	-19.531\\
1454	-17.09\\
1455	-13.428\\
1456	-17.09\\
1457	-19.531\\
1458	-21.973\\
1459	-25.635\\
1460	-25.635\\
1461	-32.959\\
1462	-41.504\\
1463	-45.166\\
1464	-31.738\\
1465	-17.09\\
1466	-13.428\\
1467	-8.545\\
1468	-10.986\\
1469	-13.428\\
1470	-17.09\\
1471	-18.311\\
1472	-13.428\\
1473	-19.531\\
1474	-24.414\\
1475	-25.635\\
1476	-26.855\\
1477	-39.063\\
1478	-35.4\\
1479	-25.635\\
1480	-20.752\\
1481	-20.752\\
1482	-19.531\\
1483	-23.193\\
1484	-19.531\\
1485	-15.869\\
1486	-17.09\\
1487	-10.986\\
1488	-14.648\\
1489	-28.076\\
1490	-19.531\\
1491	-15.869\\
1492	-34.18\\
1493	-28.076\\
1494	-21.973\\
1495	-34.18\\
1496	-31.738\\
1497	-20.752\\
1498	-15.869\\
1499	-13.428\\
1500	-23.193\\
};
\end{axis}

\begin{axis}[%
width=5.5cm,
height=2.1cm,
at={(7.5cm,0cm)},
scale only axis,
xmin=1000,
xmax=1500,
xlabel={\small Time, sec},
ymin=-60,
ymax=0,
ylabel={\small Load, kN},
legend style={legend cell align=left,align=left,draw=white!15!black}
]
\addplot [color=mycolor1,line width=0.5pt,solid,forget plot]
  table[row sep=crcr]{%
1000	-10.986\\
1001	-15.869\\
1002	-10.986\\
1003	-12.207\\
1004	-13.428\\
1005	-15.869\\
1006	-14.648\\
1007	-14.648\\
1008	-7.324\\
1009	-8.545\\
1010	-10.986\\
1011	-9.766\\
1012	-9.766\\
1013	-6.104\\
1014	-2.441\\
1015	-3.662\\
1016	-2.441\\
1017	-12.207\\
1018	-13.428\\
1019	-12.207\\
1020	-10.986\\
1021	-6.104\\
1022	-6.104\\
1023	-1.221\\
1024	-6.104\\
1025	-10.986\\
1026	-12.207\\
1027	-8.545\\
1028	-14.648\\
1029	-14.648\\
1030	-9.766\\
1031	-13.428\\
1032	-12.207\\
1033	-9.766\\
1034	-9.766\\
1035	-8.545\\
1036	-8.545\\
1037	-12.207\\
1038	-10.986\\
1039	-10.986\\
1040	-10.986\\
1041	-10.986\\
1042	-14.648\\
1043	-15.869\\
1044	-9.766\\
1045	-8.545\\
1046	-8.545\\
1047	-6.104\\
1048	-7.324\\
1049	-7.324\\
1050	-4.883\\
1051	-9.766\\
1052	-10.986\\
1053	-8.545\\
1054	-13.428\\
1055	-15.869\\
1056	-8.545\\
1057	-6.104\\
1058	-6.104\\
1059	-9.766\\
1060	-6.104\\
1061	-4.883\\
1062	-7.324\\
1063	-7.324\\
1064	-3.662\\
1065	-6.104\\
1066	-8.545\\
1067	-10.986\\
1068	-10.986\\
1069	-9.766\\
1070	-13.428\\
1071	-10.986\\
1072	-12.207\\
1073	-10.986\\
1074	-10.986\\
1075	-7.324\\
1076	-8.545\\
1077	-14.648\\
1078	-20.752\\
1079	-20.752\\
1080	-19.531\\
1081	-13.428\\
1082	-15.869\\
1083	-24.414\\
1084	-23.193\\
1085	-15.869\\
1086	-21.973\\
1087	-28.076\\
1088	-23.193\\
1089	-17.09\\
1090	-14.648\\
1091	-13.428\\
1092	-9.766\\
1093	-10.986\\
1094	-12.207\\
1095	-4.883\\
1096	-4.883\\
1097	-7.324\\
1098	-10.986\\
1099	-9.766\\
1100	-9.766\\
1101	-10.986\\
1102	-13.428\\
1103	-14.648\\
1104	-15.869\\
1105	-14.648\\
1106	-17.09\\
1107	-12.207\\
1108	-12.207\\
1109	-10.986\\
1110	-8.545\\
1111	-9.766\\
1112	-12.207\\
1113	-8.545\\
1114	-9.766\\
1115	-7.324\\
1116	-6.104\\
1117	-8.545\\
1118	-6.104\\
1119	-4.883\\
1120	-8.545\\
1121	-13.428\\
1122	-12.207\\
1123	-10.986\\
1124	-7.324\\
1125	-8.545\\
1126	-19.531\\
1127	-14.648\\
1128	-10.986\\
1129	-17.09\\
1130	-12.207\\
1131	-6.104\\
1132	-9.766\\
1133	-8.545\\
1134	-14.648\\
1135	-15.869\\
1136	-19.531\\
1137	-14.648\\
1138	-14.648\\
1139	-13.428\\
1140	-13.428\\
1141	-14.648\\
1142	-10.986\\
1143	-7.324\\
1144	-7.324\\
1145	-6.104\\
1146	-8.545\\
1147	-10.986\\
1148	-15.869\\
1149	-14.648\\
1150	-8.545\\
1151	-8.545\\
1152	-9.766\\
1153	-6.104\\
1154	-2.441\\
1155	-4.883\\
1156	-7.324\\
1157	-8.545\\
1158	-6.104\\
1159	-7.324\\
1160	-6.104\\
1161	-4.883\\
1162	-4.883\\
1163	-1.221\\
1164	-3.662\\
1165	-12.207\\
1166	-14.648\\
1167	-15.869\\
1168	-12.207\\
1169	-7.324\\
1170	-6.104\\
1171	-3.662\\
1172	-7.324\\
1173	-8.545\\
1174	-9.766\\
1175	-12.207\\
1176	-18.311\\
1177	-18.311\\
1178	-19.531\\
1179	-18.311\\
1180	-15.869\\
1181	-15.869\\
1182	-17.09\\
1183	-18.311\\
1184	-12.207\\
1185	-10.986\\
1186	-12.207\\
1187	-10.986\\
1188	-12.207\\
1189	-9.766\\
1190	-15.869\\
1191	-18.311\\
1192	-12.207\\
1193	-7.324\\
1194	-9.766\\
1195	-17.09\\
1196	-18.311\\
1197	-18.311\\
1198	-21.973\\
1199	-20.752\\
1200	-17.09\\
1201	-15.869\\
1202	-18.311\\
1203	-20.752\\
1204	-17.09\\
1205	-8.545\\
1206	-14.648\\
1207	-12.207\\
1208	-7.324\\
1209	-10.986\\
1210	-9.766\\
1211	-8.545\\
1212	-7.324\\
1213	-8.545\\
1214	-8.545\\
1215	-9.766\\
1216	-13.428\\
1217	-14.648\\
1218	-7.324\\
1219	-8.545\\
1220	-12.207\\
1221	-17.09\\
1222	-15.869\\
1223	-8.545\\
1224	-8.545\\
1225	-9.766\\
1226	-8.545\\
1227	-7.324\\
1228	-6.104\\
1229	-8.545\\
1230	-9.766\\
1231	-9.766\\
1232	-9.766\\
1233	-12.207\\
1234	-14.648\\
1235	-15.869\\
1236	-10.986\\
1237	-13.428\\
1238	-17.09\\
1239	-12.207\\
1240	-3.662\\
1241	-9.766\\
1242	-8.545\\
1243	-9.766\\
1244	-8.545\\
1245	-8.545\\
1246	-8.545\\
1247	-10.986\\
1248	-7.324\\
1249	-7.324\\
1250	-9.766\\
1251	-6.104\\
1252	-8.545\\
1253	-6.104\\
1254	-3.662\\
1255	-6.104\\
1256	-4.883\\
1257	-10.986\\
1258	-12.207\\
1259	-12.207\\
1260	-18.311\\
1261	-10.986\\
1262	-4.883\\
1263	-6.104\\
1264	-6.104\\
1265	-8.545\\
1266	-6.104\\
1267	-8.545\\
1268	-7.324\\
1269	-13.428\\
1270	-9.766\\
1271	-14.648\\
1272	-10.986\\
1273	-9.766\\
1274	-6.104\\
1275	-3.662\\
1276	-3.662\\
1277	-1.221\\
1278	-2.441\\
1279	-8.545\\
1280	-8.545\\
1281	-7.324\\
1282	-9.766\\
1283	-15.869\\
1284	-10.986\\
1285	-9.766\\
1286	-9.766\\
1287	-13.428\\
1288	-14.648\\
1289	-12.207\\
1290	-12.207\\
1291	-6.104\\
1292	-2.441\\
1293	-4.883\\
1294	-4.883\\
1295	-6.104\\
1296	-3.662\\
1297	-4.883\\
1298	-9.766\\
1299	-7.324\\
1300	-7.324\\
1301	-10.986\\
1302	-7.324\\
1303	-4.883\\
1304	-9.766\\
1305	-12.207\\
1306	-9.766\\
1307	-10.986\\
1308	-7.324\\
1309	-7.324\\
1310	-10.986\\
1311	-14.648\\
1312	-13.428\\
1313	-8.545\\
1314	-14.648\\
1315	-13.428\\
1316	-13.428\\
1317	-13.428\\
1318	-13.428\\
1319	-10.986\\
1320	-10.986\\
1321	-12.207\\
1322	-18.311\\
1323	-17.09\\
1324	-10.986\\
1325	-10.986\\
1326	-10.986\\
1327	-13.428\\
1328	-13.428\\
1329	-9.766\\
1330	-12.207\\
1331	-14.648\\
1332	-14.648\\
1333	-9.766\\
1334	-8.545\\
1335	-10.986\\
1336	-18.311\\
1337	-14.648\\
1338	-13.428\\
1339	-9.766\\
1340	-10.986\\
1341	-9.766\\
1342	-6.104\\
1343	-4.883\\
1344	-3.662\\
1345	-3.662\\
1346	-7.324\\
1347	-10.986\\
1348	-9.766\\
1349	-6.104\\
1350	-7.324\\
1351	-9.766\\
1352	-6.104\\
1353	-6.104\\
1354	-9.766\\
1355	-6.104\\
1356	-7.324\\
1357	-12.207\\
1358	-12.207\\
1359	-8.545\\
1360	-4.883\\
1361	-6.104\\
1362	-7.324\\
1363	-6.104\\
1364	-7.324\\
1365	-14.648\\
1366	-9.766\\
1367	-17.09\\
1368	-20.752\\
1369	-18.311\\
1370	-13.428\\
1371	-10.986\\
1372	-10.986\\
1373	-12.207\\
1374	-13.428\\
1375	-14.648\\
1376	-15.869\\
1377	-20.752\\
1378	-20.752\\
1379	-24.414\\
1380	-15.869\\
1381	-20.752\\
1382	-20.752\\
1383	-14.648\\
1384	-17.09\\
1385	-18.311\\
1386	-9.766\\
1387	-7.324\\
1388	-8.545\\
1389	-9.766\\
1390	-7.324\\
1391	-6.104\\
1392	-7.324\\
1393	-6.104\\
1394	-3.662\\
1395	-4.883\\
1396	-7.324\\
1397	-2.441\\
1398	-10.986\\
1399	-8.545\\
1400	-7.324\\
1401	-15.869\\
1402	-19.531\\
1403	-19.531\\
1404	-19.531\\
1405	-14.648\\
1406	-17.09\\
1407	-12.207\\
1408	-7.324\\
1409	-10.986\\
1410	-7.324\\
1411	-3.662\\
1412	-6.104\\
1413	-4.883\\
1414	-2.441\\
1415	-4.883\\
1416	-10.986\\
1417	-9.766\\
1418	-10.986\\
1419	-10.986\\
1420	-12.207\\
1421	-14.648\\
1422	-10.986\\
1423	-10.986\\
1424	-8.545\\
1425	-7.324\\
1426	-13.428\\
1427	-9.766\\
1428	-6.104\\
1429	-10.986\\
1430	-13.428\\
1431	-10.986\\
1432	-7.324\\
1433	-7.324\\
1434	-7.324\\
1435	-4.883\\
1436	-6.104\\
1437	-7.324\\
1438	-10.986\\
1439	-8.545\\
1440	-7.324\\
1441	-10.986\\
1442	-8.545\\
1443	-6.104\\
1444	-7.324\\
1445	-13.428\\
1446	-14.648\\
1447	-13.428\\
1448	-12.207\\
1449	-12.207\\
1450	-8.545\\
1451	-9.766\\
1452	-8.545\\
1453	-12.207\\
1454	-7.324\\
1455	-4.883\\
1456	-9.766\\
1457	-9.766\\
1458	-10.986\\
1459	-12.207\\
1460	-10.986\\
1461	-17.09\\
1462	-20.752\\
1463	-23.193\\
1464	-15.869\\
1465	-9.766\\
1466	-6.104\\
1467	-6.104\\
1468	-9.766\\
1469	-6.104\\
1470	-9.766\\
1471	-6.104\\
1472	-7.324\\
1473	-8.545\\
1474	-10.986\\
1475	-13.428\\
1476	-14.648\\
1477	-19.531\\
1478	-14.648\\
1479	-13.428\\
1480	-10.986\\
1481	-10.986\\
1482	-10.986\\
1483	-12.207\\
1484	-9.766\\
1485	-10.986\\
1486	-9.766\\
1487	-7.324\\
1488	-12.207\\
1489	-13.428\\
1490	-8.545\\
1491	-9.766\\
1492	-18.311\\
1493	-12.207\\
1494	-12.207\\
1495	-17.09\\
1496	-14.648\\
1497	-10.986\\
1498	-7.324\\
1499	-7.324\\
1500	-12.207\\
};
\end{axis}

\begin{axis}[%
width=5.5cm,
height=2.1cm,
at={(0cm,15.254237cm)},
scale only axis,
xmin=1000,
xmax=1500,
xlabel={\small Time, sec},
ymin=-7,
ymax=-4.999,
ylabel={\small $\Delta \mathrm{x}$, mm},
legend style={legend cell align=left,align=left,draw=white!15!black}
]
\addplot [color=mycolor1,line width=0.5pt,solid,forget plot]
  table[row sep=crcr]{%
1000	-6.152\\
1001	-6.226\\
1002	-6.207\\
1003	-6.152\\
1004	-6.262\\
1005	-6.281\\
1006	-6.335\\
1007	-6.281\\
1008	-5.969\\
1009	-5.933\\
1010	-5.933\\
1011	-5.988\\
1012	-5.969\\
1013	-5.621\\
1014	-5.292\\
1015	-5.127\\
1016	-5.511\\
1017	-5.914\\
1018	-6.024\\
1019	-6.079\\
1020	-5.969\\
1021	-5.731\\
1022	-5.988\\
1023	-5.933\\
1024	-5.75\\
1025	-5.896\\
1026	-5.933\\
1027	-5.859\\
1028	-6.042\\
1029	-6.079\\
1030	-5.988\\
1031	-6.097\\
1032	-6.152\\
1033	-6.079\\
1034	-5.988\\
1035	-5.859\\
1036	-5.896\\
1037	-5.969\\
1038	-5.969\\
1039	-5.969\\
1040	-6.006\\
1041	-6.024\\
1042	-6.207\\
1043	-6.207\\
1044	-6.061\\
1045	-5.988\\
1046	-5.786\\
1047	-5.658\\
1048	-5.768\\
1049	-5.695\\
1050	-5.676\\
1051	-5.786\\
1052	-5.878\\
1053	-5.878\\
1054	-6.061\\
1055	-6.024\\
1056	-5.804\\
1057	-5.658\\
1058	-5.731\\
1059	-5.786\\
1060	-5.585\\
1061	-5.511\\
1062	-5.603\\
1063	-5.566\\
1064	-5.475\\
1065	-5.548\\
1066	-5.603\\
1067	-5.841\\
1068	-5.896\\
1069	-5.969\\
1070	-5.988\\
1071	-6.024\\
1072	-5.988\\
1073	-5.951\\
1074	-5.933\\
1075	-5.878\\
1076	-5.896\\
1077	-6.097\\
1078	-6.409\\
1079	-6.519\\
1080	-6.519\\
1081	-6.317\\
1082	-6.427\\
1083	-6.61\\
1084	-6.665\\
1085	-6.573\\
1086	-6.647\\
1087	-6.866\\
1088	-6.793\\
1089	-6.647\\
1090	-6.482\\
1091	-6.281\\
1092	-6.152\\
1093	-6.171\\
1094	-6.061\\
1095	-5.823\\
1096	-5.695\\
1097	-5.731\\
1098	-5.878\\
1099	-5.896\\
1100	-5.896\\
1101	-6.006\\
1102	-6.061\\
1103	-6.244\\
1104	-6.244\\
1105	-6.281\\
1106	-6.207\\
1107	-6.097\\
1108	-6.152\\
1109	-6.042\\
1110	-5.969\\
1111	-6.006\\
1112	-6.024\\
1113	-5.933\\
1114	-5.896\\
1115	-5.786\\
1116	-5.75\\
1117	-5.786\\
1118	-5.676\\
1119	-5.676\\
1120	-5.75\\
1121	-5.969\\
1122	-6.061\\
1123	-6.116\\
1124	-5.896\\
1125	-5.951\\
1126	-6.244\\
1127	-6.189\\
1128	-6.207\\
1129	-6.244\\
1130	-6.079\\
1131	-5.841\\
1132	-5.896\\
1133	-5.969\\
1134	-6.171\\
1135	-6.39\\
1136	-6.445\\
1137	-6.317\\
1138	-6.299\\
1139	-6.262\\
1140	-6.244\\
1141	-6.244\\
1142	-6.079\\
1143	-5.951\\
1144	-5.914\\
1145	-5.841\\
1146	-5.859\\
1147	-5.988\\
1148	-6.262\\
1149	-6.226\\
1150	-6.042\\
1151	-5.933\\
1152	-5.896\\
1153	-5.713\\
1154	-5.493\\
1155	-5.457\\
1156	-5.658\\
1157	-5.64\\
1158	-5.621\\
1159	-5.658\\
1160	-5.64\\
1161	-5.548\\
1162	-5.402\\
1163	-5.237\\
1164	-5.31\\
1165	-5.695\\
1166	-6.006\\
1167	-6.152\\
1168	-5.988\\
1169	-5.804\\
1170	-5.64\\
1171	-5.42\\
1172	-5.621\\
1173	-5.695\\
1174	-5.841\\
1175	-6.024\\
1176	-6.354\\
1177	-6.628\\
1178	-6.592\\
1179	-6.555\\
1180	-6.39\\
1181	-6.354\\
1182	-6.39\\
1183	-6.409\\
1184	-6.335\\
1185	-6.244\\
1186	-6.244\\
1187	-6.171\\
1188	-6.207\\
1189	-6.116\\
1190	-6.317\\
1191	-6.464\\
1192	-6.354\\
1193	-6.097\\
1194	-6.042\\
1195	-6.244\\
1196	-6.409\\
1197	-6.573\\
1198	-6.647\\
1199	-6.683\\
1200	-6.592\\
1201	-6.519\\
1202	-6.555\\
1203	-6.628\\
1204	-6.464\\
1205	-6.171\\
1206	-6.244\\
1207	-6.189\\
1208	-5.969\\
1209	-6.024\\
1210	-6.079\\
1211	-6.006\\
1212	-5.878\\
1213	-5.914\\
1214	-5.878\\
1215	-5.933\\
1216	-6.134\\
1217	-6.116\\
1218	-5.988\\
1219	-5.951\\
1220	-6.134\\
1221	-6.427\\
1222	-6.39\\
1223	-6.189\\
1224	-6.006\\
1225	-5.969\\
1226	-5.951\\
1227	-5.823\\
1228	-5.804\\
1229	-5.878\\
1230	-5.969\\
1231	-6.024\\
1232	-5.896\\
1233	-6.061\\
1234	-6.207\\
1235	-6.207\\
1236	-6.116\\
1237	-6.134\\
1238	-6.262\\
1239	-6.116\\
1240	-5.75\\
1241	-5.713\\
1242	-5.768\\
1243	-5.878\\
1244	-5.914\\
1245	-5.786\\
1246	-5.896\\
1247	-5.933\\
1248	-5.823\\
1249	-5.859\\
1250	-5.823\\
1251	-5.786\\
1252	-5.841\\
1253	-5.64\\
1254	-5.585\\
1255	-5.511\\
1256	-5.493\\
1257	-5.731\\
1258	-5.859\\
1259	-6.042\\
1260	-6.244\\
1261	-6.134\\
1262	-5.878\\
1263	-5.713\\
1264	-5.64\\
1265	-5.731\\
1266	-5.621\\
1267	-5.585\\
1268	-5.676\\
1269	-5.914\\
1270	-5.988\\
1271	-6.134\\
1272	-6.024\\
1273	-5.951\\
1274	-5.695\\
1275	-5.566\\
1276	-5.402\\
1277	-5.347\\
1278	-5.365\\
1279	-5.603\\
1280	-5.713\\
1281	-5.786\\
1282	-5.823\\
1283	-6.097\\
1284	-6.171\\
1285	-6.024\\
1286	-6.079\\
1287	-6.116\\
1288	-6.226\\
1289	-6.171\\
1290	-6.006\\
1291	-5.676\\
1292	-5.42\\
1293	-5.347\\
1294	-5.383\\
1295	-5.402\\
1296	-5.383\\
1297	-5.493\\
1298	-5.658\\
1299	-5.695\\
1300	-5.731\\
1301	-5.786\\
1302	-5.621\\
1303	-5.383\\
1304	-5.53\\
1305	-5.786\\
1306	-5.823\\
1307	-5.878\\
1308	-5.878\\
1309	-5.75\\
1310	-5.823\\
1311	-6.006\\
1312	-6.079\\
1313	-5.951\\
1314	-6.152\\
1315	-6.134\\
1316	-6.061\\
1317	-6.134\\
1318	-6.134\\
1319	-6.061\\
1320	-5.969\\
1321	-6.116\\
1322	-6.409\\
1323	-6.372\\
1324	-6.097\\
1325	-6.006\\
1326	-5.988\\
1327	-6.079\\
1328	-6.171\\
1329	-6.061\\
1330	-6.061\\
1331	-6.171\\
1332	-6.244\\
1333	-6.097\\
1334	-5.969\\
1335	-6.024\\
1336	-6.262\\
1337	-6.335\\
1338	-6.299\\
1339	-6.116\\
1340	-5.988\\
1341	-5.969\\
1342	-5.804\\
1343	-5.548\\
1344	-5.347\\
1345	-5.2\\
1346	-5.438\\
1347	-5.695\\
1348	-5.823\\
1349	-5.768\\
1350	-5.75\\
1351	-5.823\\
1352	-5.658\\
1353	-5.621\\
1354	-5.621\\
1355	-5.53\\
1356	-5.548\\
1357	-5.768\\
1358	-5.951\\
1359	-5.804\\
1360	-5.658\\
1361	-5.548\\
1362	-5.585\\
1363	-5.475\\
1364	-5.713\\
1365	-6.097\\
1366	-6.061\\
1367	-6.171\\
1368	-6.335\\
1369	-6.372\\
1370	-6.299\\
1371	-6.116\\
1372	-6.079\\
1373	-6.079\\
1374	-6.134\\
1375	-6.207\\
1376	-6.226\\
1377	-6.372\\
1378	-6.519\\
1379	-6.647\\
1380	-6.537\\
1381	-6.5\\
1382	-6.592\\
1383	-6.482\\
1384	-6.519\\
1385	-6.482\\
1386	-6.226\\
1387	-5.933\\
1388	-5.841\\
1389	-5.969\\
1390	-5.914\\
1391	-5.75\\
1392	-5.786\\
1393	-5.731\\
1394	-5.566\\
1395	-5.585\\
1396	-5.621\\
1397	-5.548\\
1398	-5.676\\
1399	-5.859\\
1400	-5.823\\
1401	-6.006\\
1402	-6.317\\
1403	-6.372\\
1404	-6.5\\
1405	-6.39\\
1406	-6.372\\
1407	-6.207\\
1408	-5.969\\
1409	-5.951\\
1410	-5.859\\
1411	-5.713\\
1412	-5.786\\
1413	-5.603\\
1414	-5.365\\
1415	-5.328\\
1416	-5.603\\
1417	-5.841\\
1418	-5.969\\
1419	-6.024\\
1420	-6.006\\
1421	-6.061\\
1422	-6.042\\
1423	-5.933\\
1424	-5.933\\
1425	-5.859\\
1426	-5.988\\
1427	-5.896\\
1428	-5.768\\
1429	-5.896\\
1430	-6.079\\
1431	-5.988\\
1432	-5.878\\
1433	-5.804\\
1434	-5.823\\
1435	-5.713\\
1436	-5.603\\
1437	-5.713\\
1438	-5.804\\
1439	-5.878\\
1440	-5.841\\
1441	-5.878\\
1442	-5.914\\
1443	-5.713\\
1444	-5.658\\
1445	-5.896\\
1446	-6.152\\
1447	-6.189\\
1448	-6.152\\
1449	-6.097\\
1450	-6.024\\
1451	-5.951\\
1452	-5.896\\
1453	-5.969\\
1454	-5.933\\
1455	-5.768\\
1456	-5.841\\
1457	-5.933\\
1458	-6.006\\
1459	-6.079\\
1460	-6.079\\
1461	-6.244\\
1462	-6.427\\
1463	-6.592\\
1464	-6.427\\
1465	-6.134\\
1466	-5.878\\
1467	-5.621\\
1468	-5.64\\
1469	-5.64\\
1470	-5.841\\
1471	-5.878\\
1472	-5.823\\
1473	-5.896\\
1474	-5.988\\
1475	-6.079\\
1476	-6.152\\
1477	-6.372\\
1478	-6.445\\
1479	-6.317\\
1480	-6.116\\
1481	-6.061\\
1482	-6.061\\
1483	-6.152\\
1484	-6.042\\
1485	-6.006\\
1486	-5.988\\
1487	-5.768\\
1488	-5.841\\
1489	-6.042\\
1490	-5.969\\
1491	-5.951\\
1492	-6.244\\
1493	-6.226\\
1494	-6.171\\
1495	-6.354\\
1496	-6.372\\
1497	-6.171\\
1498	-5.896\\
1499	-5.859\\
1500	-6.024\\
};
\end{axis}

\begin{axis}[%
width=5.5cm,
height=2.1cm,
at={(0cm,0cm)},
scale only axis,
xmin=1000,
xmax=1500,
xlabel={\small Time, sec},
ymin=-7,
ymax=-4.999,
ylabel={\small $\Delta \mathrm{x}$, mm},
legend style={legend cell align=left,align=left,draw=white!15!black}
]
\addplot [color=mycolor1,line width=0.5pt,solid,forget plot]
  table[row sep=crcr]{%
1000	-6.134\\
1001	-6.226\\
1002	-6.189\\
1003	-6.134\\
1004	-6.262\\
1005	-6.262\\
1006	-6.335\\
1007	-6.262\\
1008	-5.988\\
1009	-5.933\\
1010	-5.951\\
1011	-5.988\\
1012	-5.933\\
1013	-5.585\\
1014	-5.273\\
1015	-5.127\\
1016	-5.493\\
1017	-5.914\\
1018	-6.024\\
1019	-6.061\\
1020	-5.969\\
1021	-5.731\\
1022	-5.951\\
1023	-5.951\\
1024	-5.768\\
1025	-5.969\\
1026	-5.969\\
1027	-5.896\\
1028	-6.061\\
1029	-6.097\\
1030	-5.988\\
1031	-6.116\\
1032	-6.152\\
1033	-6.079\\
1034	-5.988\\
1035	-5.878\\
1036	-5.878\\
1037	-5.951\\
1038	-5.951\\
1039	-5.969\\
1040	-5.988\\
1041	-6.042\\
1042	-6.189\\
1043	-6.207\\
1044	-6.042\\
1045	-5.969\\
1046	-5.731\\
1047	-5.658\\
1048	-5.731\\
1049	-5.695\\
1050	-5.658\\
1051	-5.768\\
1052	-5.859\\
1053	-5.859\\
1054	-6.061\\
1055	-6.042\\
1056	-5.786\\
1057	-5.621\\
1058	-5.695\\
1059	-5.75\\
1060	-5.585\\
1061	-5.493\\
1062	-5.603\\
1063	-5.548\\
1064	-5.493\\
1065	-5.548\\
1066	-5.603\\
1067	-5.786\\
1068	-5.878\\
1069	-5.951\\
1070	-5.988\\
1071	-6.042\\
1072	-5.988\\
1073	-5.951\\
1074	-5.914\\
1075	-5.896\\
1076	-5.878\\
1077	-6.061\\
1078	-6.372\\
1079	-6.482\\
1080	-6.5\\
1081	-6.317\\
1082	-6.427\\
1083	-6.592\\
1084	-6.647\\
1085	-6.555\\
1086	-6.647\\
1087	-6.866\\
1088	-6.793\\
1089	-6.647\\
1090	-6.464\\
1091	-6.262\\
1092	-6.152\\
1093	-6.152\\
1094	-6.061\\
1095	-5.823\\
1096	-5.695\\
1097	-5.731\\
1098	-5.878\\
1099	-5.914\\
1100	-5.914\\
1101	-6.006\\
1102	-6.079\\
1103	-6.244\\
1104	-6.244\\
1105	-6.281\\
1106	-6.226\\
1107	-6.116\\
1108	-6.152\\
1109	-6.061\\
1110	-5.969\\
1111	-6.006\\
1112	-6.024\\
1113	-5.896\\
1114	-5.878\\
1115	-5.768\\
1116	-5.75\\
1117	-5.768\\
1118	-5.658\\
1119	-5.658\\
1120	-5.768\\
1121	-5.933\\
1122	-6.042\\
1123	-6.116\\
1124	-5.933\\
1125	-5.988\\
1126	-6.244\\
1127	-6.207\\
1128	-6.244\\
1129	-6.281\\
1130	-6.061\\
1131	-5.859\\
1132	-5.914\\
1133	-5.969\\
1134	-6.207\\
1135	-6.409\\
1136	-6.445\\
1137	-6.335\\
1138	-6.299\\
1139	-6.299\\
1140	-6.244\\
1141	-6.244\\
1142	-6.097\\
1143	-5.988\\
1144	-5.914\\
1145	-5.841\\
1146	-5.859\\
1147	-6.006\\
1148	-6.244\\
1149	-6.226\\
1150	-6.042\\
1151	-5.933\\
1152	-5.896\\
1153	-5.695\\
1154	-5.475\\
1155	-5.438\\
1156	-5.64\\
1157	-5.621\\
1158	-5.603\\
1159	-5.658\\
1160	-5.621\\
1161	-5.53\\
1162	-5.365\\
1163	-5.237\\
1164	-5.31\\
1165	-5.676\\
1166	-5.988\\
1167	-6.152\\
1168	-6.006\\
1169	-5.841\\
1170	-5.658\\
1171	-5.438\\
1172	-5.621\\
1173	-5.713\\
1174	-5.841\\
1175	-6.024\\
1176	-6.372\\
1177	-6.61\\
1178	-6.573\\
1179	-6.555\\
1180	-6.39\\
1181	-6.354\\
1182	-6.39\\
1183	-6.409\\
1184	-6.335\\
1185	-6.244\\
1186	-6.226\\
1187	-6.207\\
1188	-6.226\\
1189	-6.097\\
1190	-6.244\\
1191	-6.427\\
1192	-6.317\\
1193	-6.097\\
1194	-6.006\\
1195	-6.226\\
1196	-6.409\\
1197	-6.573\\
1198	-6.647\\
1199	-6.665\\
1200	-6.573\\
1201	-6.482\\
1202	-6.537\\
1203	-6.628\\
1204	-6.445\\
1205	-6.189\\
1206	-6.207\\
1207	-6.207\\
1208	-5.969\\
1209	-5.988\\
1210	-6.061\\
1211	-5.988\\
1212	-5.859\\
1213	-5.914\\
1214	-5.878\\
1215	-5.951\\
1216	-6.116\\
1217	-6.134\\
1218	-6.006\\
1219	-5.969\\
1220	-6.116\\
1221	-6.409\\
1222	-6.372\\
1223	-6.134\\
1224	-5.951\\
1225	-5.969\\
1226	-5.951\\
1227	-5.804\\
1228	-5.786\\
1229	-5.859\\
1230	-5.969\\
1231	-6.024\\
1232	-5.896\\
1233	-6.097\\
1234	-6.244\\
1235	-6.226\\
1236	-6.152\\
1237	-6.152\\
1238	-6.262\\
1239	-6.134\\
1240	-5.768\\
1241	-5.731\\
1242	-5.768\\
1243	-5.896\\
1244	-5.896\\
1245	-5.786\\
1246	-5.896\\
1247	-5.951\\
1248	-5.823\\
1249	-5.859\\
1250	-5.841\\
1251	-5.786\\
1252	-5.823\\
1253	-5.676\\
1254	-5.658\\
1255	-5.566\\
1256	-5.53\\
1257	-5.75\\
1258	-5.878\\
1259	-6.042\\
1260	-6.262\\
1261	-6.152\\
1262	-5.896\\
1263	-5.713\\
1264	-5.658\\
1265	-5.75\\
1266	-5.621\\
1267	-5.566\\
1268	-5.676\\
1269	-5.914\\
1270	-6.006\\
1271	-6.134\\
1272	-6.042\\
1273	-5.951\\
1274	-5.64\\
1275	-5.53\\
1276	-5.365\\
1277	-5.31\\
1278	-5.365\\
1279	-5.566\\
1280	-5.713\\
1281	-5.768\\
1282	-5.841\\
1283	-6.079\\
1284	-6.152\\
1285	-6.042\\
1286	-6.061\\
1287	-6.116\\
1288	-6.226\\
1289	-6.244\\
1290	-6.024\\
1291	-5.713\\
1292	-5.457\\
1293	-5.365\\
1294	-5.402\\
1295	-5.42\\
1296	-5.402\\
1297	-5.475\\
1298	-5.695\\
1299	-5.731\\
1300	-5.75\\
1301	-5.804\\
1302	-5.658\\
1303	-5.383\\
1304	-5.53\\
1305	-5.768\\
1306	-5.841\\
1307	-5.896\\
1308	-5.859\\
1309	-5.75\\
1310	-5.823\\
1311	-6.024\\
1312	-6.079\\
1313	-5.969\\
1314	-6.152\\
1315	-6.152\\
1316	-6.079\\
1317	-6.152\\
1318	-6.152\\
1319	-6.079\\
1320	-5.988\\
1321	-6.207\\
1322	-6.445\\
1323	-6.39\\
1324	-6.134\\
1325	-6.024\\
1326	-6.024\\
1327	-6.097\\
1328	-6.171\\
1329	-6.097\\
1330	-6.061\\
1331	-6.189\\
1332	-6.226\\
1333	-6.152\\
1334	-5.969\\
1335	-6.042\\
1336	-6.281\\
1337	-6.335\\
1338	-6.317\\
1339	-6.134\\
1340	-5.988\\
1341	-5.988\\
1342	-5.768\\
1343	-5.493\\
1344	-5.328\\
1345	-5.2\\
1346	-5.383\\
1347	-5.658\\
1348	-5.804\\
1349	-5.768\\
1350	-5.75\\
1351	-5.823\\
1352	-5.658\\
1353	-5.603\\
1354	-5.603\\
1355	-5.511\\
1356	-5.548\\
1357	-5.75\\
1358	-5.933\\
1359	-5.804\\
1360	-5.658\\
1361	-5.566\\
1362	-5.566\\
1363	-5.475\\
1364	-5.695\\
1365	-6.097\\
1366	-6.097\\
1367	-6.207\\
1368	-6.335\\
1369	-6.409\\
1370	-6.317\\
1371	-6.152\\
1372	-6.079\\
1373	-6.079\\
1374	-6.116\\
1375	-6.189\\
1376	-6.226\\
1377	-6.372\\
1378	-6.5\\
1379	-6.647\\
1380	-6.537\\
1381	-6.5\\
1382	-6.573\\
1383	-6.5\\
1384	-6.519\\
1385	-6.5\\
1386	-6.244\\
1387	-5.914\\
1388	-5.823\\
1389	-5.951\\
1390	-5.914\\
1391	-5.75\\
1392	-5.804\\
1393	-5.731\\
1394	-5.566\\
1395	-5.585\\
1396	-5.621\\
1397	-5.548\\
1398	-5.695\\
1399	-5.878\\
1400	-5.823\\
1401	-6.024\\
1402	-6.317\\
1403	-6.372\\
1404	-6.5\\
1405	-6.39\\
1406	-6.372\\
1407	-6.226\\
1408	-5.951\\
1409	-5.951\\
1410	-5.878\\
1411	-5.731\\
1412	-5.786\\
1413	-5.621\\
1414	-5.365\\
1415	-5.347\\
1416	-5.603\\
1417	-5.823\\
1418	-5.969\\
1419	-6.042\\
1420	-6.042\\
1421	-6.079\\
1422	-6.061\\
1423	-5.951\\
1424	-5.933\\
1425	-5.878\\
1426	-5.969\\
1427	-5.896\\
1428	-5.768\\
1429	-5.878\\
1430	-6.042\\
1431	-5.969\\
1432	-5.859\\
1433	-5.804\\
1434	-5.804\\
1435	-5.695\\
1436	-5.585\\
1437	-5.695\\
1438	-5.804\\
1439	-5.878\\
1440	-5.823\\
1441	-5.878\\
1442	-5.896\\
1443	-5.75\\
1444	-5.658\\
1445	-5.914\\
1446	-6.134\\
1447	-6.189\\
1448	-6.134\\
1449	-6.097\\
1450	-6.006\\
1451	-5.951\\
1452	-5.859\\
1453	-5.933\\
1454	-5.914\\
1455	-5.75\\
1456	-5.823\\
1457	-5.933\\
1458	-5.988\\
1459	-6.079\\
1460	-6.079\\
1461	-6.244\\
1462	-6.445\\
1463	-6.555\\
1464	-6.409\\
1465	-6.097\\
1466	-5.841\\
1467	-5.585\\
1468	-5.603\\
1469	-5.621\\
1470	-5.823\\
1471	-5.859\\
1472	-5.804\\
1473	-5.878\\
1474	-5.969\\
1475	-6.097\\
1476	-6.152\\
1477	-6.39\\
1478	-6.427\\
1479	-6.317\\
1480	-6.116\\
1481	-6.061\\
1482	-6.061\\
1483	-6.152\\
1484	-6.042\\
1485	-5.988\\
1486	-5.988\\
1487	-5.786\\
1488	-5.859\\
1489	-6.079\\
1490	-6.006\\
1491	-5.951\\
1492	-6.244\\
1493	-6.244\\
1494	-6.171\\
1495	-6.317\\
1496	-6.354\\
1497	-6.171\\
1498	-5.914\\
1499	-5.841\\
1500	-6.024\\
};
\end{axis}

\begin{axis}[%
width=5.5cm,
height=2.1cm,
at={(7.5cm,3.813559cm)},
scale only axis,
xmin=1000,
xmax=1500,
xlabel={\small Time, sec},
ymin=-60,
ymax=0,
ylabel={\small Load, kN},
legend style={legend cell align=left,align=left,draw=white!15!black}
]
\addplot [color=mycolor1,line width=0.5pt,solid,forget plot]
  table[row sep=crcr]{%
1000	-10.986\\
1001	-15.869\\
1002	-10.986\\
1003	-10.986\\
1004	-14.648\\
1005	-14.648\\
1006	-17.09\\
1007	-10.986\\
1008	-7.324\\
1009	-14.648\\
1010	-8.545\\
1011	-9.766\\
1012	-6.104\\
1013	-3.662\\
1014	-3.662\\
1015	-3.662\\
1016	-2.441\\
1017	-13.428\\
1018	-13.428\\
1019	-13.428\\
1020	-9.766\\
1021	-7.324\\
1022	-10.986\\
1023	-12.207\\
1024	-6.104\\
1025	-9.766\\
1026	-13.428\\
1027	-7.324\\
1028	-15.869\\
1029	-13.428\\
1030	-10.986\\
1031	-17.09\\
1032	-13.428\\
1033	-10.986\\
1034	-8.545\\
1035	-8.545\\
1036	-10.986\\
1037	-12.207\\
1038	-10.986\\
1039	-9.766\\
1040	-10.986\\
1041	-10.986\\
1042	-15.869\\
1043	-14.648\\
1044	-10.986\\
1045	-10.986\\
1046	-6.104\\
1047	-4.883\\
1048	-8.545\\
1049	-7.324\\
1050	-7.324\\
1051	-10.986\\
1052	-9.766\\
1053	-9.766\\
1054	-15.869\\
1055	-8.545\\
1056	-6.104\\
1057	-7.324\\
1058	-8.545\\
1059	-9.766\\
1060	-4.883\\
1061	-7.324\\
1062	-7.324\\
1063	-7.324\\
1064	-6.104\\
1065	-7.324\\
1066	-8.545\\
1067	-9.766\\
1068	-10.986\\
1069	-10.986\\
1070	-12.207\\
1071	-13.428\\
1072	-10.986\\
1073	-12.207\\
1074	-9.766\\
1075	-9.766\\
1076	-9.766\\
1077	-17.09\\
1078	-20.752\\
1079	-19.531\\
1080	-19.531\\
1081	-15.869\\
1082	-20.752\\
1083	-23.193\\
1084	-23.193\\
1085	-14.648\\
1086	-23.193\\
1087	-26.855\\
1088	-21.973\\
1089	-15.869\\
1090	-14.648\\
1091	-10.986\\
1092	-8.545\\
1093	-10.986\\
1094	-8.545\\
1095	-6.104\\
1096	-6.104\\
1097	-8.545\\
1098	-9.766\\
1099	-10.986\\
1100	-9.766\\
1101	-12.207\\
1102	-10.986\\
1103	-15.869\\
1104	-13.428\\
1105	-18.311\\
1106	-12.207\\
1107	-10.986\\
1108	-12.207\\
1109	-9.766\\
1110	-8.545\\
1111	-10.986\\
1112	-10.986\\
1113	-8.545\\
1114	-8.545\\
1115	-7.324\\
1116	-7.324\\
1117	-7.324\\
1118	-4.883\\
1119	-6.104\\
1120	-8.545\\
1121	-13.428\\
1122	-12.207\\
1123	-13.428\\
1124	-8.545\\
1125	-15.869\\
1126	-17.09\\
1127	-13.428\\
1128	-14.648\\
1129	-14.648\\
1130	-10.986\\
1131	-7.324\\
1132	-8.545\\
1133	-8.545\\
1134	-17.09\\
1135	-20.752\\
1136	-19.531\\
1137	-14.648\\
1138	-15.869\\
1139	-13.428\\
1140	-12.207\\
1141	-12.207\\
1142	-9.766\\
1143	-8.545\\
1144	-8.545\\
1145	-9.766\\
1146	-8.545\\
1147	-13.428\\
1148	-15.869\\
1149	-10.986\\
1150	-8.545\\
1151	-9.766\\
1152	-8.545\\
1153	-6.104\\
1154	-3.662\\
1155	-4.883\\
1156	-8.545\\
1157	-9.766\\
1158	-4.883\\
1159	-7.324\\
1160	-7.324\\
1161	-3.662\\
1162	-3.662\\
1163	-2.441\\
1164	-4.883\\
1165	-10.986\\
1166	-12.207\\
1167	-14.648\\
1168	-13.428\\
1169	-8.545\\
1170	-7.324\\
1171	-4.883\\
1172	-7.324\\
1173	-6.104\\
1174	-10.986\\
1175	-12.207\\
1176	-21.973\\
1177	-23.193\\
1178	-21.973\\
1179	-18.311\\
1180	-13.428\\
1181	-17.09\\
1182	-15.869\\
1183	-19.531\\
1184	-12.207\\
1185	-13.428\\
1186	-12.207\\
1187	-13.428\\
1188	-13.428\\
1189	-10.986\\
1190	-18.311\\
1191	-17.09\\
1192	-13.428\\
1193	-7.324\\
1194	-12.207\\
1195	-15.869\\
1196	-17.09\\
1197	-19.531\\
1198	-21.973\\
1199	-20.752\\
1200	-15.869\\
1201	-14.648\\
1202	-18.311\\
1203	-19.531\\
1204	-13.428\\
1205	-9.766\\
1206	-13.428\\
1207	-9.766\\
1208	-7.324\\
1209	-9.766\\
1210	-10.986\\
1211	-7.324\\
1212	-7.324\\
1213	-9.766\\
1214	-6.104\\
1215	-10.986\\
1216	-13.428\\
1217	-13.428\\
1218	-8.545\\
1219	-9.766\\
1220	-12.207\\
1221	-18.311\\
1222	-15.869\\
1223	-9.766\\
1224	-8.545\\
1225	-9.766\\
1226	-7.324\\
1227	-8.545\\
1228	-7.324\\
1229	-8.545\\
1230	-10.986\\
1231	-10.986\\
1232	-6.104\\
1233	-10.986\\
1234	-13.428\\
1235	-13.428\\
1236	-9.766\\
1237	-12.207\\
1238	-13.428\\
1239	-10.986\\
1240	-4.883\\
1241	-10.986\\
1242	-8.545\\
1243	-8.545\\
1244	-6.104\\
1245	-6.104\\
1246	-10.986\\
1247	-10.986\\
1248	-6.104\\
1249	-8.545\\
1250	-8.545\\
1251	-4.883\\
1252	-8.545\\
1253	-4.883\\
1254	-7.324\\
1255	-4.883\\
1256	-4.883\\
1257	-10.986\\
1258	-12.207\\
1259	-12.207\\
1260	-15.869\\
1261	-10.986\\
1262	-6.104\\
1263	-7.324\\
1264	-4.883\\
1265	-7.324\\
1266	-6.104\\
1267	-3.662\\
1268	-8.545\\
1269	-9.766\\
1270	-9.766\\
1271	-17.09\\
1272	-10.986\\
1273	-13.428\\
1274	-7.324\\
1275	-8.545\\
1276	-3.662\\
1277	-1.221\\
1278	-6.104\\
1279	-9.766\\
1280	-8.545\\
1281	-9.766\\
1282	-9.766\\
1283	-17.09\\
1284	-10.986\\
1285	-10.986\\
1286	-10.986\\
1287	-13.428\\
1288	-15.869\\
1289	-15.869\\
1290	-10.986\\
1291	-6.104\\
1292	-3.662\\
1293	-4.883\\
1294	-6.104\\
1295	-4.883\\
1296	-6.104\\
1297	-6.104\\
1298	-9.766\\
1299	-9.766\\
1300	-8.545\\
1301	-9.766\\
1302	-6.104\\
1303	-3.662\\
1304	-7.324\\
1305	-10.986\\
1306	-8.545\\
1307	-12.207\\
1308	-8.545\\
1309	-7.324\\
1310	-8.545\\
1311	-12.207\\
1312	-14.648\\
1313	-10.986\\
1314	-17.09\\
1315	-9.766\\
1316	-12.207\\
1317	-13.428\\
1318	-13.428\\
1319	-9.766\\
1320	-9.766\\
1321	-12.207\\
1322	-18.311\\
1323	-15.869\\
1324	-8.545\\
1325	-12.207\\
1326	-9.766\\
1327	-12.207\\
1328	-13.428\\
1329	-9.766\\
1330	-9.766\\
1331	-14.648\\
1332	-13.428\\
1333	-12.207\\
1334	-8.545\\
1335	-9.766\\
1336	-15.869\\
1337	-14.648\\
1338	-15.869\\
1339	-12.207\\
1340	-10.986\\
1341	-9.766\\
1342	-8.545\\
1343	-4.883\\
1344	-3.662\\
1345	-3.662\\
1346	-7.324\\
1347	-10.986\\
1348	-9.766\\
1349	-8.545\\
1350	-10.986\\
1351	-9.766\\
1352	-8.545\\
1353	-3.662\\
1354	-6.104\\
1355	-6.104\\
1356	-6.104\\
1357	-8.545\\
1358	-10.986\\
1359	-7.324\\
1360	-6.104\\
1361	-7.324\\
1362	-7.324\\
1363	-6.104\\
1364	-8.545\\
1365	-15.869\\
1366	-14.648\\
1367	-17.09\\
1368	-17.09\\
1369	-17.09\\
1370	-13.428\\
1371	-10.986\\
1372	-9.766\\
1373	-10.986\\
1374	-12.207\\
1375	-14.648\\
1376	-14.648\\
1377	-17.09\\
1378	-19.531\\
1379	-21.973\\
1380	-18.311\\
1381	-21.973\\
1382	-19.531\\
1383	-14.648\\
1384	-17.09\\
1385	-14.648\\
1386	-10.986\\
1387	-8.545\\
1388	-8.545\\
1389	-10.986\\
1390	-7.324\\
1391	-7.324\\
1392	-7.324\\
1393	-7.324\\
1394	-3.662\\
1395	-7.324\\
1396	-7.324\\
1397	-6.104\\
1398	-8.545\\
1399	-12.207\\
1400	-8.545\\
1401	-12.207\\
1402	-18.311\\
1403	-17.09\\
1404	-19.531\\
1405	-17.09\\
1406	-14.648\\
1407	-14.648\\
1408	-7.324\\
1409	-9.766\\
1410	-8.545\\
1411	-6.104\\
1412	-7.324\\
1413	-8.545\\
1414	-2.441\\
1415	-4.883\\
1416	-8.545\\
1417	-9.766\\
1418	-12.207\\
1419	-12.207\\
1420	-10.986\\
1421	-12.207\\
1422	-14.648\\
1423	-10.986\\
1424	-9.766\\
1425	-7.324\\
1426	-10.986\\
1427	-10.986\\
1428	-7.324\\
1429	-10.986\\
1430	-13.428\\
1431	-10.986\\
1432	-8.545\\
1433	-8.545\\
1434	-8.545\\
1435	-6.104\\
1436	-7.324\\
1437	-6.104\\
1438	-9.766\\
1439	-8.545\\
1440	-10.986\\
1441	-10.986\\
1442	-8.545\\
1443	-6.104\\
1444	-6.104\\
1445	-10.986\\
1446	-14.648\\
1447	-14.648\\
1448	-12.207\\
1449	-10.986\\
1450	-9.766\\
1451	-8.545\\
1452	-8.545\\
1453	-9.766\\
1454	-12.207\\
1455	-7.324\\
1456	-7.324\\
1457	-10.986\\
1458	-10.986\\
1459	-13.428\\
1460	-13.428\\
1461	-14.648\\
1462	-20.752\\
1463	-21.973\\
1464	-14.648\\
1465	-9.766\\
1466	-7.324\\
1467	-6.104\\
1468	-7.324\\
1469	-4.883\\
1470	-7.324\\
1471	-7.324\\
1472	-7.324\\
1473	-10.986\\
1474	-12.207\\
1475	-13.428\\
1476	-13.428\\
1477	-18.311\\
1478	-17.09\\
1479	-12.207\\
1480	-12.207\\
1481	-9.766\\
1482	-9.766\\
1483	-13.428\\
1484	-9.766\\
1485	-8.545\\
1486	-10.986\\
1487	-6.104\\
1488	-10.986\\
1489	-14.648\\
1490	-8.545\\
1491	-8.545\\
1492	-15.869\\
1493	-13.428\\
1494	-10.986\\
1495	-17.09\\
1496	-17.09\\
1497	-10.986\\
1498	-8.545\\
1499	-9.766\\
1500	-12.207\\
};
\end{axis}
\end{tikzpicture}%}
		\caption{Experimental data.}
	\end{figure}	
\begin{figure}[!h]
	\centering
	\scalebox{0.8}{% This file was created by matlab2tikz.
% Minimal pgfplots version: 1.3
%
\definecolor{mycolor1}{rgb}{0.00000,0.44700,0.74100}%
%
\begin{tikzpicture}

\begin{axis}[%
width=5.5cm,
height=2.1cm,
at={(7.5cm,7.627119cm)},
scale only axis,
xmin=1000,
xmax=1500,
xlabel={\small Time, sec},
ymin=-405,
ymax=0,
ylabel={\small Load, kN},
legend style={legend cell align=left,align=left,draw=white!15!black}
]
\addplot [color=mycolor1,line width=0.5pt,solid,forget plot]
  table[row sep=crcr]{%
1000	-96.436\\
1001	-122.07\\
1002	-100.098\\
1003	-93.994\\
1004	-130.615\\
1005	-125.732\\
1006	-153.809\\
1007	-115.967\\
1008	-61.035\\
1009	-74.463\\
1010	-73.242\\
1011	-86.67\\
1012	-73.242\\
1013	-34.18\\
1014	-20.752\\
1015	-23.193\\
1016	-58.594\\
1017	-100.098\\
1018	-109.863\\
1019	-114.746\\
1020	-83.008\\
1021	-52.49\\
1022	-104.98\\
1023	-81.787\\
1024	-59.814\\
1025	-97.656\\
1026	-83.008\\
1027	-72.021\\
1028	-118.408\\
1029	-107.422\\
1030	-83.008\\
1031	-119.629\\
1032	-123.291\\
1033	-95.215\\
1034	-80.566\\
1035	-62.256\\
1036	-75.684\\
1037	-95.215\\
1038	-87.891\\
1039	-91.553\\
1040	-96.436\\
1041	-102.539\\
1042	-141.602\\
1043	-128.174\\
1044	-85.449\\
1045	-79.346\\
1046	-47.607\\
1047	-48.828\\
1048	-68.359\\
1049	-52.49\\
1050	-57.373\\
1051	-75.684\\
1052	-89.111\\
1053	-81.787\\
1054	-128.174\\
1055	-98.877\\
1056	-57.373\\
1057	-45.166\\
1058	-62.256\\
1059	-72.021\\
1060	-40.283\\
1061	-42.725\\
1062	-56.152\\
1063	-46.387\\
1064	-40.283\\
1065	-52.49\\
1066	-62.256\\
1067	-92.773\\
1068	-90.332\\
1069	-107.422\\
1070	-98.877\\
1071	-115.967\\
1072	-91.553\\
1073	-91.553\\
1074	-79.346\\
1075	-75.684\\
1076	-76.904\\
1077	-133.057\\
1078	-189.209\\
1079	-194.092\\
1080	-189.209\\
1081	-119.629\\
1082	-170.898\\
1083	-213.623\\
1084	-219.727\\
1085	-169.678\\
1086	-219.727\\
1087	-279.541\\
1088	-197.754\\
1089	-161.133\\
1090	-109.863\\
1091	-85.449\\
1092	-73.242\\
1093	-91.553\\
1094	-62.256\\
1095	-46.387\\
1096	-42.725\\
1097	-54.932\\
1098	-79.346\\
1099	-74.463\\
1100	-75.684\\
1101	-100.098\\
1102	-103.76\\
1103	-141.602\\
1104	-114.746\\
1105	-142.822\\
1106	-104.98\\
1107	-90.332\\
1108	-103.76\\
1109	-76.904\\
1110	-69.58\\
1111	-84.229\\
1112	-86.67\\
1113	-59.814\\
1114	-64.697\\
1115	-48.828\\
1116	-53.711\\
1117	-57.373\\
1118	-41.504\\
1119	-52.49\\
1120	-65.918\\
1121	-101.318\\
1122	-104.98\\
1123	-114.746\\
1124	-68.359\\
1125	-93.994\\
1126	-150.146\\
1127	-114.746\\
1128	-125.732\\
1129	-128.174\\
1130	-80.566\\
1131	-53.711\\
1132	-75.684\\
1133	-85.449\\
1134	-137.939\\
1135	-172.119\\
1136	-170.898\\
1137	-123.291\\
1138	-130.615\\
1139	-115.967\\
1140	-113.525\\
1141	-111.084\\
1142	-76.904\\
1143	-68.359\\
1144	-61.035\\
1145	-57.373\\
1146	-62.256\\
1147	-91.553\\
1148	-140.381\\
1149	-111.084\\
1150	-79.346\\
1151	-64.697\\
1152	-62.256\\
1153	-40.283\\
1154	-29.297\\
1155	-36.621\\
1156	-63.477\\
1157	-51.27\\
1158	-52.49\\
1159	-58.594\\
1160	-54.932\\
1161	-37.842\\
1162	-28.076\\
1163	-23.193\\
1164	-39.063\\
1165	-83.008\\
1166	-117.188\\
1167	-130.615\\
1168	-86.67\\
1169	-58.594\\
1170	-45.166\\
1171	-32.959\\
1172	-67.139\\
1173	-65.918\\
1174	-91.553\\
1175	-117.188\\
1176	-186.768\\
1177	-216.064\\
1178	-177.002\\
1179	-168.457\\
1180	-109.863\\
1181	-122.07\\
1182	-131.836\\
1183	-146.484\\
1184	-108.643\\
1185	-100.098\\
1186	-98.877\\
1187	-89.111\\
1188	-106.201\\
1189	-78.125\\
1190	-130.615\\
1191	-159.912\\
1192	-113.525\\
1193	-70.801\\
1194	-73.242\\
1195	-123.291\\
1196	-153.809\\
1197	-189.209\\
1198	-202.637\\
1199	-202.637\\
1200	-153.809\\
1201	-137.939\\
1202	-158.691\\
1203	-187.988\\
1204	-109.863\\
1205	-72.021\\
1206	-101.318\\
1207	-86.67\\
1208	-53.711\\
1209	-74.463\\
1210	-84.229\\
1211	-67.139\\
1212	-51.27\\
1213	-70.801\\
1214	-58.594\\
1215	-79.346\\
1216	-107.422\\
1217	-100.098\\
1218	-69.58\\
1219	-70.801\\
1220	-107.422\\
1221	-177.002\\
1222	-128.174\\
1223	-79.346\\
1224	-61.035\\
1225	-70.801\\
1226	-64.697\\
1227	-48.828\\
1228	-53.711\\
1229	-65.918\\
1230	-83.008\\
1231	-91.553\\
1232	-63.477\\
1233	-104.98\\
1234	-124.512\\
1235	-118.408\\
1236	-91.553\\
1237	-100.098\\
1238	-135.498\\
1239	-87.891\\
1240	-42.725\\
1241	-57.373\\
1242	-62.256\\
1243	-81.787\\
1244	-72.021\\
1245	-52.49\\
1246	-79.346\\
1247	-80.566\\
1248	-57.373\\
1249	-70.801\\
1250	-63.477\\
1251	-56.152\\
1252	-67.139\\
1253	-41.504\\
1254	-48.828\\
1255	-36.621\\
1256	-40.283\\
1257	-75.684\\
1258	-84.229\\
1259	-113.525\\
1260	-146.484\\
1261	-98.877\\
1262	-58.594\\
1263	-46.387\\
1264	-43.945\\
1265	-63.477\\
1266	-42.725\\
1267	-47.607\\
1268	-61.035\\
1269	-102.539\\
1270	-93.994\\
1271	-134.277\\
1272	-89.111\\
1273	-81.787\\
1274	-46.387\\
1275	-45.166\\
1276	-28.076\\
1277	-35.4\\
1278	-37.842\\
1279	-67.139\\
1280	-70.801\\
1281	-79.346\\
1282	-85.449\\
1283	-125.732\\
1284	-112.305\\
1285	-84.229\\
1286	-102.539\\
1287	-109.863\\
1288	-144.043\\
1289	-115.967\\
1290	-75.684\\
1291	-40.283\\
1292	-29.297\\
1293	-29.297\\
1294	-36.621\\
1295	-39.063\\
1296	-37.842\\
1297	-50.049\\
1298	-74.463\\
1299	-68.359\\
1300	-70.801\\
1301	-83.008\\
1302	-51.27\\
1303	-30.518\\
1304	-58.594\\
1305	-90.332\\
1306	-87.891\\
1307	-97.656\\
1308	-83.008\\
1309	-61.035\\
1310	-85.449\\
1311	-118.408\\
1312	-118.408\\
1313	-84.229\\
1314	-136.719\\
1315	-100.098\\
1316	-101.318\\
1317	-117.188\\
1318	-115.967\\
1319	-86.67\\
1320	-76.904\\
1321	-128.174\\
1322	-178.223\\
1323	-139.16\\
1324	-79.346\\
1325	-76.904\\
1326	-79.346\\
1327	-98.877\\
1328	-114.746\\
1329	-85.449\\
1330	-90.332\\
1331	-120.85\\
1332	-131.836\\
1333	-87.891\\
1334	-68.359\\
1335	-91.553\\
1336	-156.25\\
1337	-137.939\\
1338	-140.381\\
1339	-84.229\\
1340	-76.904\\
1341	-72.021\\
1342	-47.607\\
1343	-30.518\\
1344	-26.855\\
1345	-23.193\\
1346	-54.932\\
1347	-70.801\\
1348	-87.891\\
1349	-65.918\\
1350	-73.242\\
1351	-83.008\\
1352	-53.711\\
1353	-56.152\\
1354	-53.711\\
1355	-40.283\\
1356	-51.27\\
1357	-84.229\\
1358	-106.201\\
1359	-63.477\\
1360	-48.828\\
1361	-40.283\\
1362	-50.049\\
1363	-35.4\\
1364	-76.904\\
1365	-130.615\\
1366	-104.98\\
1367	-139.16\\
1368	-162.354\\
1369	-164.795\\
1370	-124.512\\
1371	-90.332\\
1372	-89.111\\
1373	-89.111\\
1374	-106.201\\
1375	-125.732\\
1376	-125.732\\
1377	-168.457\\
1378	-183.105\\
1379	-219.727\\
1380	-147.705\\
1381	-166.016\\
1382	-184.326\\
1383	-140.381\\
1384	-162.354\\
1385	-139.16\\
1386	-80.566\\
1387	-52.49\\
1388	-56.152\\
1389	-81.787\\
1390	-65.918\\
1391	-43.945\\
1392	-62.256\\
1393	-47.607\\
1394	-36.621\\
1395	-46.387\\
1396	-52.49\\
1397	-36.621\\
1398	-70.801\\
1399	-92.773\\
1400	-68.359\\
1401	-114.746\\
1402	-164.795\\
1403	-150.146\\
1404	-196.533\\
1405	-131.836\\
1406	-144.043\\
1407	-96.436\\
1408	-63.477\\
1409	-76.904\\
1410	-59.814\\
1411	-45.166\\
1412	-64.697\\
1413	-36.621\\
1414	-28.076\\
1415	-34.18\\
1416	-70.801\\
1417	-86.67\\
1418	-107.422\\
1419	-103.76\\
1420	-100.098\\
1421	-114.746\\
1422	-93.994\\
1423	-76.904\\
1424	-76.904\\
1425	-62.256\\
1426	-97.656\\
1427	-68.359\\
1428	-56.152\\
1429	-81.787\\
1430	-113.525\\
1431	-86.67\\
1432	-65.918\\
1433	-56.152\\
1434	-65.918\\
1435	-45.166\\
1436	-45.166\\
1437	-59.814\\
1438	-75.684\\
1439	-84.229\\
1440	-65.918\\
1441	-86.67\\
1442	-79.346\\
1443	-47.607\\
1444	-50.049\\
1445	-95.215\\
1446	-131.836\\
1447	-131.836\\
1448	-107.422\\
1449	-103.76\\
1450	-79.346\\
1451	-76.904\\
1452	-61.035\\
1453	-89.111\\
1454	-75.684\\
1455	-50.049\\
1456	-74.463\\
1457	-85.449\\
1458	-101.318\\
1459	-109.863\\
1460	-109.863\\
1461	-148.926\\
1462	-181.885\\
1463	-205.078\\
1464	-129.395\\
1465	-73.242\\
1466	-47.607\\
1467	-34.18\\
1468	-54.932\\
1469	-46.387\\
1470	-80.566\\
1471	-72.021\\
1472	-64.697\\
1473	-85.449\\
1474	-98.877\\
1475	-117.188\\
1476	-123.291\\
1477	-175.781\\
1478	-150.146\\
1479	-117.188\\
1480	-80.566\\
1481	-80.566\\
1482	-83.008\\
1483	-106.201\\
1484	-75.684\\
1485	-79.346\\
1486	-74.463\\
1487	-42.725\\
1488	-74.463\\
1489	-107.422\\
1490	-73.242\\
1491	-80.566\\
1492	-147.705\\
1493	-109.863\\
1494	-106.201\\
1495	-147.705\\
1496	-140.381\\
1497	-87.891\\
1498	-56.152\\
1499	-58.594\\
1500	-97.656\\
};
\end{axis}

\begin{axis}[%
width=5.5cm,
height=2.1cm,
at={(0cm,3.813559cm)},
scale only axis,
xmin=1000,
xmax=1500,
xlabel={\small Time, sec},
ymin=-7,
ymax=-4.999,
ylabel={\small $\Delta \mathrm{x}$, mm},
legend style={legend cell align=left,align=left,draw=white!15!black}
]
\addplot [color=mycolor1,line width=0.5pt,solid,forget plot]
  table[row sep=crcr]{%
1000	-6.116\\
1001	-6.226\\
1002	-6.171\\
1003	-6.134\\
1004	-6.226\\
1005	-6.262\\
1006	-6.354\\
1007	-6.281\\
1008	-5.988\\
1009	-5.933\\
1010	-5.933\\
1011	-5.988\\
1012	-5.969\\
1013	-5.603\\
1014	-5.273\\
1015	-5.127\\
1016	-5.511\\
1017	-5.878\\
1018	-6.006\\
1019	-6.042\\
1020	-5.951\\
1021	-5.713\\
1022	-5.951\\
1023	-5.914\\
1024	-5.75\\
1025	-5.914\\
1026	-5.933\\
1027	-5.859\\
1028	-6.061\\
1029	-6.116\\
1030	-5.988\\
1031	-6.097\\
1032	-6.171\\
1033	-6.061\\
1034	-5.988\\
1035	-5.859\\
1036	-5.878\\
1037	-5.951\\
1038	-5.969\\
1039	-5.969\\
1040	-5.988\\
1041	-6.042\\
1042	-6.189\\
1043	-6.226\\
1044	-6.042\\
1045	-5.988\\
1046	-5.75\\
1047	-5.64\\
1048	-5.731\\
1049	-5.676\\
1050	-5.676\\
1051	-5.768\\
1052	-5.878\\
1053	-5.878\\
1054	-6.061\\
1055	-6.042\\
1056	-5.823\\
1057	-5.676\\
1058	-5.695\\
1059	-5.768\\
1060	-5.585\\
1061	-5.511\\
1062	-5.603\\
1063	-5.566\\
1064	-5.475\\
1065	-5.548\\
1066	-5.621\\
1067	-5.823\\
1068	-5.896\\
1069	-5.969\\
1070	-5.988\\
1071	-6.024\\
1072	-5.969\\
1073	-5.933\\
1074	-5.896\\
1075	-5.878\\
1076	-5.878\\
1077	-6.097\\
1078	-6.39\\
1079	-6.5\\
1080	-6.519\\
1081	-6.335\\
1082	-6.409\\
1083	-6.592\\
1084	-6.647\\
1085	-6.537\\
1086	-6.647\\
1087	-6.848\\
1088	-6.793\\
1089	-6.647\\
1090	-6.445\\
1091	-6.244\\
1092	-6.134\\
1093	-6.152\\
1094	-6.024\\
1095	-5.786\\
1096	-5.676\\
1097	-5.695\\
1098	-5.878\\
1099	-5.896\\
1100	-5.914\\
1101	-6.006\\
1102	-6.061\\
1103	-6.226\\
1104	-6.262\\
1105	-6.299\\
1106	-6.244\\
1107	-6.116\\
1108	-6.171\\
1109	-6.061\\
1110	-5.969\\
1111	-6.024\\
1112	-6.042\\
1113	-5.896\\
1114	-5.878\\
1115	-5.786\\
1116	-5.75\\
1117	-5.768\\
1118	-5.658\\
1119	-5.658\\
1120	-5.75\\
1121	-5.969\\
1122	-6.061\\
1123	-6.097\\
1124	-5.951\\
1125	-5.988\\
1126	-6.262\\
1127	-6.226\\
1128	-6.262\\
1129	-6.281\\
1130	-6.079\\
1131	-5.859\\
1132	-5.914\\
1133	-5.988\\
1134	-6.207\\
1135	-6.409\\
1136	-6.445\\
1137	-6.372\\
1138	-6.335\\
1139	-6.317\\
1140	-6.281\\
1141	-6.262\\
1142	-6.097\\
1143	-6.006\\
1144	-5.933\\
1145	-5.859\\
1146	-5.878\\
1147	-5.988\\
1148	-6.226\\
1149	-6.207\\
1150	-6.042\\
1151	-5.933\\
1152	-5.896\\
1153	-5.695\\
1154	-5.475\\
1155	-5.438\\
1156	-5.64\\
1157	-5.603\\
1158	-5.603\\
1159	-5.621\\
1160	-5.621\\
1161	-5.548\\
1162	-5.383\\
1163	-5.219\\
1164	-5.31\\
1165	-5.676\\
1166	-5.988\\
1167	-6.116\\
1168	-5.988\\
1169	-5.804\\
1170	-5.64\\
1171	-5.457\\
1172	-5.64\\
1173	-5.713\\
1174	-5.841\\
1175	-6.024\\
1176	-6.372\\
1177	-6.61\\
1178	-6.573\\
1179	-6.555\\
1180	-6.354\\
1181	-6.317\\
1182	-6.354\\
1183	-6.39\\
1184	-6.317\\
1185	-6.226\\
1186	-6.207\\
1187	-6.171\\
1188	-6.207\\
1189	-6.116\\
1190	-6.262\\
1191	-6.427\\
1192	-6.335\\
1193	-6.097\\
1194	-6.042\\
1195	-6.244\\
1196	-6.409\\
1197	-6.573\\
1198	-6.647\\
1199	-6.683\\
1200	-6.592\\
1201	-6.482\\
1202	-6.537\\
1203	-6.61\\
1204	-6.409\\
1205	-6.152\\
1206	-6.189\\
1207	-6.171\\
1208	-5.969\\
1209	-5.988\\
1210	-6.042\\
1211	-5.988\\
1212	-5.859\\
1213	-5.914\\
1214	-5.878\\
1215	-5.914\\
1216	-6.097\\
1217	-6.116\\
1218	-5.988\\
1219	-5.951\\
1220	-6.097\\
1221	-6.427\\
1222	-6.409\\
1223	-6.152\\
1224	-5.951\\
1225	-5.951\\
1226	-5.914\\
1227	-5.804\\
1228	-5.786\\
1229	-5.841\\
1230	-5.933\\
1231	-6.006\\
1232	-5.896\\
1233	-6.061\\
1234	-6.207\\
1235	-6.207\\
1236	-6.134\\
1237	-6.116\\
1238	-6.262\\
1239	-6.152\\
1240	-5.768\\
1241	-5.713\\
1242	-5.768\\
1243	-5.914\\
1244	-5.896\\
1245	-5.768\\
1246	-5.878\\
1247	-5.914\\
1248	-5.786\\
1249	-5.841\\
1250	-5.823\\
1251	-5.786\\
1252	-5.804\\
1253	-5.658\\
1254	-5.621\\
1255	-5.53\\
1256	-5.511\\
1257	-5.731\\
1258	-5.859\\
1259	-6.006\\
1260	-6.226\\
1261	-6.134\\
1262	-5.878\\
1263	-5.695\\
1264	-5.621\\
1265	-5.731\\
1266	-5.621\\
1267	-5.585\\
1268	-5.676\\
1269	-5.896\\
1270	-5.951\\
1271	-6.097\\
1272	-6.006\\
1273	-5.914\\
1274	-5.676\\
1275	-5.53\\
1276	-5.383\\
1277	-5.31\\
1278	-5.365\\
1279	-5.566\\
1280	-5.695\\
1281	-5.75\\
1282	-5.823\\
1283	-6.079\\
1284	-6.152\\
1285	-6.024\\
1286	-6.061\\
1287	-6.097\\
1288	-6.226\\
1289	-6.207\\
1290	-6.006\\
1291	-5.676\\
1292	-5.438\\
1293	-5.365\\
1294	-5.383\\
1295	-5.402\\
1296	-5.383\\
1297	-5.457\\
1298	-5.676\\
1299	-5.713\\
1300	-5.713\\
1301	-5.786\\
1302	-5.658\\
1303	-5.402\\
1304	-5.53\\
1305	-5.786\\
1306	-5.841\\
1307	-5.896\\
1308	-5.878\\
1309	-5.75\\
1310	-5.841\\
1311	-6.042\\
1312	-6.097\\
1313	-5.969\\
1314	-6.189\\
1315	-6.171\\
1316	-6.116\\
1317	-6.171\\
1318	-6.189\\
1319	-6.079\\
1320	-6.006\\
1321	-6.171\\
1322	-6.427\\
1323	-6.372\\
1324	-6.116\\
1325	-6.006\\
1326	-5.988\\
1327	-6.079\\
1328	-6.171\\
1329	-6.061\\
1330	-6.079\\
1331	-6.189\\
1332	-6.226\\
1333	-6.097\\
1334	-5.951\\
1335	-6.024\\
1336	-6.281\\
1337	-6.335\\
1338	-6.317\\
1339	-6.134\\
1340	-5.988\\
1341	-5.988\\
1342	-5.786\\
1343	-5.53\\
1344	-5.347\\
1345	-5.219\\
1346	-5.457\\
1347	-5.695\\
1348	-5.823\\
1349	-5.768\\
1350	-5.75\\
1351	-5.823\\
1352	-5.676\\
1353	-5.621\\
1354	-5.621\\
1355	-5.53\\
1356	-5.53\\
1357	-5.75\\
1358	-5.933\\
1359	-5.804\\
1360	-5.64\\
1361	-5.548\\
1362	-5.585\\
1363	-5.457\\
1364	-5.713\\
1365	-6.097\\
1366	-6.079\\
1367	-6.189\\
1368	-6.335\\
1369	-6.39\\
1370	-6.299\\
1371	-6.134\\
1372	-6.079\\
1373	-6.079\\
1374	-6.134\\
1375	-6.207\\
1376	-6.244\\
1377	-6.39\\
1378	-6.5\\
1379	-6.628\\
1380	-6.5\\
1381	-6.464\\
1382	-6.537\\
1383	-6.464\\
1384	-6.5\\
1385	-6.445\\
1386	-6.189\\
1387	-5.896\\
1388	-5.823\\
1389	-5.933\\
1390	-5.914\\
1391	-5.768\\
1392	-5.786\\
1393	-5.731\\
1394	-5.566\\
1395	-5.585\\
1396	-5.621\\
1397	-5.548\\
1398	-5.695\\
1399	-5.896\\
1400	-5.859\\
1401	-6.061\\
1402	-6.372\\
1403	-6.409\\
1404	-6.5\\
1405	-6.409\\
1406	-6.372\\
1407	-6.244\\
1408	-5.969\\
1409	-5.988\\
1410	-5.896\\
1411	-5.731\\
1412	-5.804\\
1413	-5.658\\
1414	-5.42\\
1415	-5.383\\
1416	-5.64\\
1417	-5.878\\
1418	-6.006\\
1419	-6.042\\
1420	-6.042\\
1421	-6.097\\
1422	-6.024\\
1423	-5.951\\
1424	-5.951\\
1425	-5.859\\
1426	-5.988\\
1427	-5.896\\
1428	-5.786\\
1429	-5.878\\
1430	-6.061\\
1431	-6.006\\
1432	-5.878\\
1433	-5.804\\
1434	-5.823\\
1435	-5.695\\
1436	-5.603\\
1437	-5.695\\
1438	-5.804\\
1439	-5.878\\
1440	-5.823\\
1441	-5.878\\
1442	-5.896\\
1443	-5.676\\
1444	-5.64\\
1445	-5.878\\
1446	-6.152\\
1447	-6.189\\
1448	-6.134\\
1449	-6.097\\
1450	-6.006\\
1451	-5.969\\
1452	-5.878\\
1453	-5.951\\
1454	-5.933\\
1455	-5.75\\
1456	-5.841\\
1457	-5.933\\
1458	-6.006\\
1459	-6.061\\
1460	-6.097\\
1461	-6.244\\
1462	-6.445\\
1463	-6.573\\
1464	-6.409\\
1465	-6.097\\
1466	-5.841\\
1467	-5.585\\
1468	-5.64\\
1469	-5.658\\
1470	-5.841\\
1471	-5.878\\
1472	-5.804\\
1473	-5.896\\
1474	-5.988\\
1475	-6.097\\
1476	-6.171\\
1477	-6.372\\
1478	-6.445\\
1479	-6.317\\
1480	-6.116\\
1481	-6.061\\
1482	-6.061\\
1483	-6.134\\
1484	-6.042\\
1485	-5.988\\
1486	-5.988\\
1487	-5.731\\
1488	-5.786\\
1489	-6.024\\
1490	-5.914\\
1491	-5.933\\
1492	-6.226\\
1493	-6.189\\
1494	-6.134\\
1495	-6.299\\
1496	-6.335\\
1497	-6.152\\
1498	-5.914\\
1499	-5.859\\
1500	-6.006\\
};
\end{axis}

\begin{axis}[%
width=5.5cm,
height=2.1cm,
at={(0cm,15.254237cm)},
scale only axis,
xmin=1000,
xmax=1500,
xlabel={\small Time, sec},
ymin=-7,
ymax=-4.999,
ylabel={\small $\Delta \mathrm{x}$, mm},
legend style={legend cell align=left,align=left,draw=white!15!black}
]
\addplot [color=mycolor1,line width=0.5pt,solid,forget plot]
  table[row sep=crcr]{%
1000	-6.134\\
1001	-6.207\\
1002	-6.189\\
1003	-6.134\\
1004	-6.262\\
1005	-6.262\\
1006	-6.335\\
1007	-6.281\\
1008	-5.969\\
1009	-5.933\\
1010	-5.951\\
1011	-6.006\\
1012	-5.951\\
1013	-5.603\\
1014	-5.31\\
1015	-5.164\\
1016	-5.511\\
1017	-5.914\\
1018	-6.042\\
1019	-6.061\\
1020	-5.969\\
1021	-5.731\\
1022	-5.969\\
1023	-5.933\\
1024	-5.75\\
1025	-5.933\\
1026	-5.933\\
1027	-5.859\\
1028	-6.042\\
1029	-6.097\\
1030	-5.969\\
1031	-6.097\\
1032	-6.152\\
1033	-6.061\\
1034	-5.969\\
1035	-5.859\\
1036	-5.859\\
1037	-5.951\\
1038	-5.951\\
1039	-5.969\\
1040	-6.006\\
1041	-6.042\\
1042	-6.189\\
1043	-6.207\\
1044	-6.042\\
1045	-5.969\\
1046	-5.75\\
1047	-5.658\\
1048	-5.75\\
1049	-5.695\\
1050	-5.658\\
1051	-5.768\\
1052	-5.878\\
1053	-5.878\\
1054	-6.061\\
1055	-6.042\\
1056	-5.804\\
1057	-5.658\\
1058	-5.695\\
1059	-5.768\\
1060	-5.585\\
1061	-5.475\\
1062	-5.603\\
1063	-5.566\\
1064	-5.475\\
1065	-5.548\\
1066	-5.621\\
1067	-5.804\\
1068	-5.896\\
1069	-5.969\\
1070	-5.969\\
1071	-6.042\\
1072	-5.988\\
1073	-5.969\\
1074	-5.914\\
1075	-5.878\\
1076	-5.878\\
1077	-6.116\\
1078	-6.409\\
1079	-6.519\\
1080	-6.519\\
1081	-6.335\\
1082	-6.409\\
1083	-6.592\\
1084	-6.647\\
1085	-6.555\\
1086	-6.628\\
1087	-6.848\\
1088	-6.793\\
1089	-6.665\\
1090	-6.445\\
1091	-6.244\\
1092	-6.134\\
1093	-6.134\\
1094	-6.042\\
1095	-5.823\\
1096	-5.676\\
1097	-5.713\\
1098	-5.878\\
1099	-5.914\\
1100	-5.914\\
1101	-6.024\\
1102	-6.079\\
1103	-6.244\\
1104	-6.244\\
1105	-6.281\\
1106	-6.207\\
1107	-6.097\\
1108	-6.152\\
1109	-6.042\\
1110	-5.969\\
1111	-6.006\\
1112	-6.042\\
1113	-5.914\\
1114	-5.896\\
1115	-5.786\\
1116	-5.768\\
1117	-5.786\\
1118	-5.676\\
1119	-5.658\\
1120	-5.75\\
1121	-5.969\\
1122	-6.061\\
1123	-6.116\\
1124	-5.933\\
1125	-5.969\\
1126	-6.226\\
1127	-6.189\\
1128	-6.226\\
1129	-6.262\\
1130	-6.061\\
1131	-5.859\\
1132	-5.896\\
1133	-5.969\\
1134	-6.207\\
1135	-6.409\\
1136	-6.464\\
1137	-6.354\\
1138	-6.317\\
1139	-6.299\\
1140	-6.281\\
1141	-6.262\\
1142	-6.097\\
1143	-5.969\\
1144	-5.914\\
1145	-5.859\\
1146	-5.859\\
1147	-5.988\\
1148	-6.244\\
1149	-6.226\\
1150	-6.042\\
1151	-5.951\\
1152	-5.896\\
1153	-5.695\\
1154	-5.493\\
1155	-5.438\\
1156	-5.64\\
1157	-5.64\\
1158	-5.64\\
1159	-5.658\\
1160	-5.64\\
1161	-5.548\\
1162	-5.383\\
1163	-5.219\\
1164	-5.31\\
1165	-5.676\\
1166	-5.988\\
1167	-6.134\\
1168	-6.042\\
1169	-5.841\\
1170	-5.676\\
1171	-5.457\\
1172	-5.64\\
1173	-5.713\\
1174	-5.841\\
1175	-6.042\\
1176	-6.372\\
1177	-6.628\\
1178	-6.592\\
1179	-6.555\\
1180	-6.372\\
1181	-6.317\\
1182	-6.39\\
1183	-6.409\\
1184	-6.317\\
1185	-6.226\\
1186	-6.226\\
1187	-6.171\\
1188	-6.207\\
1189	-6.116\\
1190	-6.281\\
1191	-6.427\\
1192	-6.354\\
1193	-6.116\\
1194	-6.042\\
1195	-6.262\\
1196	-6.427\\
1197	-6.555\\
1198	-6.647\\
1199	-6.683\\
1200	-6.573\\
1201	-6.5\\
1202	-6.537\\
1203	-6.628\\
1204	-6.464\\
1205	-6.171\\
1206	-6.226\\
1207	-6.171\\
1208	-5.988\\
1209	-5.988\\
1210	-6.061\\
1211	-5.988\\
1212	-5.878\\
1213	-5.914\\
1214	-5.878\\
1215	-5.951\\
1216	-6.152\\
1217	-6.134\\
1218	-6.006\\
1219	-5.969\\
1220	-6.116\\
1221	-6.427\\
1222	-6.39\\
1223	-6.152\\
1224	-5.969\\
1225	-5.951\\
1226	-5.951\\
1227	-5.823\\
1228	-5.786\\
1229	-5.859\\
1230	-5.951\\
1231	-6.024\\
1232	-5.896\\
1233	-6.061\\
1234	-6.207\\
1235	-6.189\\
1236	-6.152\\
1237	-6.134\\
1238	-6.281\\
1239	-6.134\\
1240	-5.75\\
1241	-5.713\\
1242	-5.768\\
1243	-5.896\\
1244	-5.896\\
1245	-5.786\\
1246	-5.896\\
1247	-5.933\\
1248	-5.823\\
1249	-5.841\\
1250	-5.823\\
1251	-5.786\\
1252	-5.823\\
1253	-5.658\\
1254	-5.64\\
1255	-5.548\\
1256	-5.511\\
1257	-5.75\\
1258	-5.859\\
1259	-6.042\\
1260	-6.244\\
1261	-6.152\\
1262	-5.896\\
1263	-5.695\\
1264	-5.658\\
1265	-5.75\\
1266	-5.621\\
1267	-5.585\\
1268	-5.676\\
1269	-5.933\\
1270	-5.988\\
1271	-6.134\\
1272	-6.024\\
1273	-5.951\\
1274	-5.695\\
1275	-5.566\\
1276	-5.402\\
1277	-5.347\\
1278	-5.365\\
1279	-5.585\\
1280	-5.695\\
1281	-5.768\\
1282	-5.841\\
1283	-6.079\\
1284	-6.171\\
1285	-6.042\\
1286	-6.061\\
1287	-6.116\\
1288	-6.244\\
1289	-6.189\\
1290	-6.006\\
1291	-5.695\\
1292	-5.438\\
1293	-5.347\\
1294	-5.383\\
1295	-5.42\\
1296	-5.383\\
1297	-5.457\\
1298	-5.676\\
1299	-5.731\\
1300	-5.713\\
1301	-5.786\\
1302	-5.64\\
1303	-5.365\\
1304	-5.53\\
1305	-5.786\\
1306	-5.823\\
1307	-5.878\\
1308	-5.878\\
1309	-5.75\\
1310	-5.823\\
1311	-6.024\\
1312	-6.079\\
1313	-5.969\\
1314	-6.152\\
1315	-6.134\\
1316	-6.079\\
1317	-6.152\\
1318	-6.152\\
1319	-6.061\\
1320	-5.988\\
1321	-6.152\\
1322	-6.409\\
1323	-6.372\\
1324	-6.134\\
1325	-6.042\\
1326	-6.024\\
1327	-6.097\\
1328	-6.171\\
1329	-6.097\\
1330	-6.079\\
1331	-6.189\\
1332	-6.244\\
1333	-6.097\\
1334	-5.969\\
1335	-6.042\\
1336	-6.262\\
1337	-6.317\\
1338	-6.317\\
1339	-6.116\\
1340	-5.988\\
1341	-5.988\\
1342	-5.786\\
1343	-5.53\\
1344	-5.347\\
1345	-5.219\\
1346	-5.457\\
1347	-5.695\\
1348	-5.823\\
1349	-5.768\\
1350	-5.768\\
1351	-5.823\\
1352	-5.676\\
1353	-5.621\\
1354	-5.621\\
1355	-5.511\\
1356	-5.566\\
1357	-5.75\\
1358	-5.933\\
1359	-5.804\\
1360	-5.676\\
1361	-5.566\\
1362	-5.585\\
1363	-5.475\\
1364	-5.731\\
1365	-6.079\\
1366	-6.097\\
1367	-6.207\\
1368	-6.354\\
1369	-6.409\\
1370	-6.317\\
1371	-6.152\\
1372	-6.079\\
1373	-6.079\\
1374	-6.134\\
1375	-6.207\\
1376	-6.226\\
1377	-6.372\\
1378	-6.5\\
1379	-6.647\\
1380	-6.519\\
1381	-6.5\\
1382	-6.592\\
1383	-6.482\\
1384	-6.519\\
1385	-6.464\\
1386	-6.207\\
1387	-5.914\\
1388	-5.823\\
1389	-5.969\\
1390	-5.951\\
1391	-5.786\\
1392	-5.804\\
1393	-5.731\\
1394	-5.566\\
1395	-5.603\\
1396	-5.621\\
1397	-5.548\\
1398	-5.695\\
1399	-5.914\\
1400	-5.841\\
1401	-6.024\\
1402	-6.335\\
1403	-6.39\\
1404	-6.519\\
1405	-6.409\\
1406	-6.372\\
1407	-6.244\\
1408	-5.951\\
1409	-5.951\\
1410	-5.914\\
1411	-5.75\\
1412	-5.804\\
1413	-5.64\\
1414	-5.402\\
1415	-5.347\\
1416	-5.64\\
1417	-5.841\\
1418	-5.988\\
1419	-6.042\\
1420	-6.042\\
1421	-6.079\\
1422	-6.042\\
1423	-5.933\\
1424	-5.914\\
1425	-5.859\\
1426	-5.988\\
1427	-5.896\\
1428	-5.768\\
1429	-5.878\\
1430	-6.061\\
1431	-5.988\\
1432	-5.878\\
1433	-5.804\\
1434	-5.804\\
1435	-5.713\\
1436	-5.603\\
1437	-5.695\\
1438	-5.804\\
1439	-5.859\\
1440	-5.841\\
1441	-5.878\\
1442	-5.896\\
1443	-5.658\\
1444	-5.603\\
1445	-5.878\\
1446	-6.116\\
1447	-6.152\\
1448	-6.116\\
1449	-6.097\\
1450	-5.988\\
1451	-5.933\\
1452	-5.859\\
1453	-5.933\\
1454	-5.951\\
1455	-5.768\\
1456	-5.859\\
1457	-5.914\\
1458	-6.006\\
1459	-6.079\\
1460	-6.097\\
1461	-6.244\\
1462	-6.427\\
1463	-6.592\\
1464	-6.427\\
1465	-6.116\\
1466	-5.823\\
1467	-5.566\\
1468	-5.621\\
1469	-5.603\\
1470	-5.804\\
1471	-5.841\\
1472	-5.804\\
1473	-5.896\\
1474	-5.988\\
1475	-6.097\\
1476	-6.171\\
1477	-6.39\\
1478	-6.445\\
1479	-6.317\\
1480	-6.134\\
1481	-6.061\\
1482	-6.061\\
1483	-6.152\\
1484	-6.061\\
1485	-5.988\\
1486	-5.988\\
1487	-5.713\\
1488	-5.804\\
1489	-6.024\\
1490	-5.969\\
1491	-5.933\\
1492	-6.226\\
1493	-6.207\\
1494	-6.152\\
1495	-6.299\\
1496	-6.354\\
1497	-6.152\\
1498	-5.933\\
1499	-5.859\\
1500	-6.042\\
};
\end{axis}

\begin{axis}[%
width=5.5cm,
height=2.1cm,
at={(7.5cm,11.440678cm)},
scale only axis,
xmin=1000,
xmax=1500,
xlabel={\small Time, sec},
ymin=-405,
ymax=0,
ylabel={\small Load, kN},
legend style={legend cell align=left,align=left,draw=white!15!black}
]
\addplot [color=mycolor1,line width=0.5pt,solid,forget plot]
  table[row sep=crcr]{%
1000	-112.305\\
1001	-144.043\\
1002	-117.188\\
1003	-111.084\\
1004	-156.25\\
1005	-147.705\\
1006	-175.781\\
1007	-134.277\\
1008	-68.359\\
1009	-80.566\\
1010	-80.566\\
1011	-100.098\\
1012	-81.787\\
1013	-34.18\\
1014	-24.414\\
1015	-23.193\\
1016	-73.242\\
1017	-128.174\\
1018	-133.057\\
1019	-137.939\\
1020	-92.773\\
1021	-56.152\\
1022	-125.732\\
1023	-86.67\\
1024	-68.359\\
1025	-114.746\\
1026	-100.098\\
1027	-85.449\\
1028	-142.822\\
1029	-133.057\\
1030	-102.539\\
1031	-150.146\\
1032	-145.264\\
1033	-113.525\\
1034	-92.773\\
1035	-72.021\\
1036	-86.67\\
1037	-112.305\\
1038	-103.76\\
1039	-109.863\\
1040	-112.305\\
1041	-122.07\\
1042	-175.781\\
1043	-155.029\\
1044	-102.539\\
1045	-92.773\\
1046	-53.711\\
1047	-58.594\\
1048	-79.346\\
1049	-59.814\\
1050	-62.256\\
1051	-89.111\\
1052	-102.539\\
1053	-95.215\\
1054	-151.367\\
1055	-111.084\\
1056	-64.697\\
1057	-51.27\\
1058	-69.58\\
1059	-81.787\\
1060	-43.945\\
1061	-45.166\\
1062	-67.139\\
1063	-51.27\\
1064	-42.725\\
1065	-61.035\\
1066	-70.801\\
1067	-109.863\\
1068	-104.98\\
1069	-129.395\\
1070	-120.85\\
1071	-137.939\\
1072	-109.863\\
1073	-107.422\\
1074	-92.773\\
1075	-91.553\\
1076	-87.891\\
1077	-157.471\\
1078	-224.609\\
1079	-231.934\\
1080	-225.83\\
1081	-140.381\\
1082	-205.078\\
1083	-252.686\\
1084	-261.23\\
1085	-200.195\\
1086	-270.996\\
1087	-335.693\\
1088	-235.596\\
1089	-191.65\\
1090	-128.174\\
1091	-98.877\\
1092	-84.229\\
1093	-104.98\\
1094	-74.463\\
1095	-50.049\\
1096	-48.828\\
1097	-63.477\\
1098	-95.215\\
1099	-86.67\\
1100	-89.111\\
1101	-119.629\\
1102	-123.291\\
1103	-167.236\\
1104	-134.277\\
1105	-169.678\\
1106	-122.07\\
1107	-108.643\\
1108	-125.732\\
1109	-89.111\\
1110	-80.566\\
1111	-97.656\\
1112	-98.877\\
1113	-68.359\\
1114	-73.242\\
1115	-54.932\\
1116	-61.035\\
1117	-64.697\\
1118	-45.166\\
1119	-58.594\\
1120	-72.021\\
1121	-117.188\\
1122	-122.07\\
1123	-136.719\\
1124	-76.904\\
1125	-109.863\\
1126	-173.34\\
1127	-131.836\\
1128	-157.471\\
1129	-157.471\\
1130	-85.449\\
1131	-59.814\\
1132	-85.449\\
1133	-98.877\\
1134	-161.133\\
1135	-202.637\\
1136	-198.975\\
1137	-145.264\\
1138	-150.146\\
1139	-135.498\\
1140	-131.836\\
1141	-126.953\\
1142	-85.449\\
1143	-76.904\\
1144	-69.58\\
1145	-62.256\\
1146	-70.801\\
1147	-107.422\\
1148	-164.795\\
1149	-125.732\\
1150	-86.67\\
1151	-73.242\\
1152	-70.801\\
1153	-42.725\\
1154	-31.738\\
1155	-39.063\\
1156	-72.021\\
1157	-57.373\\
1158	-59.814\\
1159	-67.139\\
1160	-63.477\\
1161	-43.945\\
1162	-29.297\\
1163	-25.635\\
1164	-43.945\\
1165	-96.436\\
1166	-140.381\\
1167	-155.029\\
1168	-101.318\\
1169	-69.58\\
1170	-50.049\\
1171	-34.18\\
1172	-83.008\\
1173	-81.787\\
1174	-119.629\\
1175	-145.264\\
1176	-230.713\\
1177	-262.451\\
1178	-211.182\\
1179	-202.637\\
1180	-136.719\\
1181	-151.367\\
1182	-159.912\\
1183	-172.119\\
1184	-129.395\\
1185	-118.408\\
1186	-115.967\\
1187	-103.76\\
1188	-123.291\\
1189	-87.891\\
1190	-153.809\\
1191	-189.209\\
1192	-133.057\\
1193	-78.125\\
1194	-85.449\\
1195	-146.484\\
1196	-181.885\\
1197	-229.492\\
1198	-241.699\\
1199	-240.479\\
1200	-180.664\\
1201	-163.574\\
1202	-187.988\\
1203	-222.168\\
1204	-139.16\\
1205	-81.787\\
1206	-122.07\\
1207	-102.539\\
1208	-65.918\\
1209	-87.891\\
1210	-98.877\\
1211	-76.904\\
1212	-58.594\\
1213	-80.566\\
1214	-65.918\\
1215	-91.553\\
1216	-133.057\\
1217	-122.07\\
1218	-79.346\\
1219	-83.008\\
1220	-126.953\\
1221	-209.961\\
1222	-150.146\\
1223	-92.773\\
1224	-69.58\\
1225	-83.008\\
1226	-74.463\\
1227	-56.152\\
1228	-62.256\\
1229	-74.463\\
1230	-96.436\\
1231	-107.422\\
1232	-72.021\\
1233	-122.07\\
1234	-144.043\\
1235	-137.939\\
1236	-106.201\\
1237	-118.408\\
1238	-158.691\\
1239	-101.318\\
1240	-41.504\\
1241	-58.594\\
1242	-65.918\\
1243	-91.553\\
1244	-81.787\\
1245	-59.814\\
1246	-96.436\\
1247	-91.553\\
1248	-65.918\\
1249	-83.008\\
1250	-76.904\\
1251	-67.139\\
1252	-79.346\\
1253	-46.387\\
1254	-53.711\\
1255	-40.283\\
1256	-43.945\\
1257	-86.67\\
1258	-100.098\\
1259	-137.939\\
1260	-177.002\\
1261	-115.967\\
1262	-68.359\\
1263	-48.828\\
1264	-48.828\\
1265	-73.242\\
1266	-46.387\\
1267	-52.49\\
1268	-70.801\\
1269	-119.629\\
1270	-107.422\\
1271	-150.146\\
1272	-102.539\\
1273	-91.553\\
1274	-47.607\\
1275	-47.607\\
1276	-31.738\\
1277	-34.18\\
1278	-41.504\\
1279	-75.684\\
1280	-83.008\\
1281	-92.773\\
1282	-97.656\\
1283	-152.588\\
1284	-130.615\\
1285	-97.656\\
1286	-119.629\\
1287	-129.395\\
1288	-167.236\\
1289	-133.057\\
1290	-84.229\\
1291	-43.945\\
1292	-31.738\\
1293	-34.18\\
1294	-41.504\\
1295	-43.945\\
1296	-40.283\\
1297	-56.152\\
1298	-87.891\\
1299	-79.346\\
1300	-81.787\\
1301	-96.436\\
1302	-58.594\\
1303	-29.297\\
1304	-64.697\\
1305	-97.656\\
1306	-96.436\\
1307	-109.863\\
1308	-93.994\\
1309	-69.58\\
1310	-97.656\\
1311	-137.939\\
1312	-139.16\\
1313	-97.656\\
1314	-162.354\\
1315	-128.174\\
1316	-118.408\\
1317	-137.939\\
1318	-137.939\\
1319	-103.76\\
1320	-90.332\\
1321	-153.809\\
1322	-214.844\\
1323	-164.795\\
1324	-92.773\\
1325	-92.773\\
1326	-91.553\\
1327	-113.525\\
1328	-136.719\\
1329	-100.098\\
1330	-104.98\\
1331	-140.381\\
1332	-153.809\\
1333	-103.76\\
1334	-79.346\\
1335	-104.98\\
1336	-175.781\\
1337	-159.912\\
1338	-162.354\\
1339	-95.215\\
1340	-84.229\\
1341	-83.008\\
1342	-52.49\\
1343	-34.18\\
1344	-28.076\\
1345	-23.193\\
1346	-59.814\\
1347	-86.67\\
1348	-104.98\\
1349	-76.904\\
1350	-87.891\\
1351	-97.656\\
1352	-59.814\\
1353	-63.477\\
1354	-61.035\\
1355	-45.166\\
1356	-57.373\\
1357	-97.656\\
1358	-124.512\\
1359	-78.125\\
1360	-56.152\\
1361	-46.387\\
1362	-59.814\\
1363	-41.504\\
1364	-91.553\\
1365	-158.691\\
1366	-122.07\\
1367	-167.236\\
1368	-194.092\\
1369	-197.754\\
1370	-140.381\\
1371	-106.201\\
1372	-106.201\\
1373	-104.98\\
1374	-120.85\\
1375	-150.146\\
1376	-147.705\\
1377	-200.195\\
1378	-220.947\\
1379	-263.672\\
1380	-178.223\\
1381	-190.43\\
1382	-212.402\\
1383	-163.574\\
1384	-189.209\\
1385	-163.574\\
1386	-91.553\\
1387	-58.594\\
1388	-62.256\\
1389	-95.215\\
1390	-75.684\\
1391	-51.27\\
1392	-72.021\\
1393	-52.49\\
1394	-40.283\\
1395	-48.828\\
1396	-56.152\\
1397	-40.283\\
1398	-76.904\\
1399	-109.863\\
1400	-78.125\\
1401	-134.277\\
1402	-195.313\\
1403	-178.223\\
1404	-222.168\\
1405	-150.146\\
1406	-161.133\\
1407	-111.084\\
1408	-67.139\\
1409	-85.449\\
1410	-67.139\\
1411	-48.828\\
1412	-72.021\\
1413	-41.504\\
1414	-25.635\\
1415	-36.621\\
1416	-79.346\\
1417	-104.98\\
1418	-128.174\\
1419	-124.512\\
1420	-117.188\\
1421	-136.719\\
1422	-111.084\\
1423	-90.332\\
1424	-90.332\\
1425	-73.242\\
1426	-115.967\\
1427	-78.125\\
1428	-67.139\\
1429	-100.098\\
1430	-137.939\\
1431	-103.76\\
1432	-76.904\\
1433	-64.697\\
1434	-74.463\\
1435	-51.27\\
1436	-50.049\\
1437	-72.021\\
1438	-86.67\\
1439	-98.877\\
1440	-79.346\\
1441	-101.318\\
1442	-92.773\\
1443	-52.49\\
1444	-54.932\\
1445	-112.305\\
1446	-158.691\\
1447	-153.809\\
1448	-129.395\\
1449	-122.07\\
1450	-92.773\\
1451	-89.111\\
1452	-70.801\\
1453	-102.539\\
1454	-90.332\\
1455	-54.932\\
1456	-84.229\\
1457	-101.318\\
1458	-118.408\\
1459	-129.395\\
1460	-129.395\\
1461	-175.781\\
1462	-216.064\\
1463	-244.141\\
1464	-153.809\\
1465	-85.449\\
1466	-54.932\\
1467	-37.842\\
1468	-57.373\\
1469	-53.711\\
1470	-95.215\\
1471	-85.449\\
1472	-75.684\\
1473	-102.539\\
1474	-118.408\\
1475	-141.602\\
1476	-148.926\\
1477	-208.74\\
1478	-181.885\\
1479	-136.719\\
1480	-90.332\\
1481	-92.773\\
1482	-95.215\\
1483	-123.291\\
1484	-87.891\\
1485	-89.111\\
1486	-87.891\\
1487	-47.607\\
1488	-81.787\\
1489	-128.174\\
1490	-90.332\\
1491	-96.436\\
1492	-170.898\\
1493	-129.395\\
1494	-122.07\\
1495	-177.002\\
1496	-166.016\\
1497	-102.539\\
1498	-64.697\\
1499	-65.918\\
1500	-114.746\\
};
\end{axis}

\begin{axis}[%
width=5.5cm,
height=2.1cm,
at={(7.5cm,15.254237cm)},
scale only axis,
xmin=1000,
xmax=1500,
xlabel={\small Time, sec},
ymin=-405,
ymax=0,
ylabel={\small Load, kN},
legend style={legend cell align=left,align=left,draw=white!15!black}
]
\addplot [color=mycolor1,line width=0.5pt,solid,forget plot]
  table[row sep=crcr]{%
1000	-140.381\\
1001	-178.223\\
1002	-147.705\\
1003	-139.16\\
1004	-192.871\\
1005	-185.547\\
1006	-219.727\\
1007	-169.678\\
1008	-86.67\\
1009	-107.422\\
1010	-102.539\\
1011	-126.953\\
1012	-103.76\\
1013	-41.504\\
1014	-31.738\\
1015	-28.076\\
1016	-89.111\\
1017	-157.471\\
1018	-167.236\\
1019	-174.561\\
1020	-124.512\\
1021	-74.463\\
1022	-158.691\\
1023	-111.084\\
1024	-83.008\\
1025	-142.822\\
1026	-123.291\\
1027	-103.76\\
1028	-173.34\\
1029	-164.795\\
1030	-124.512\\
1031	-180.664\\
1032	-179.443\\
1033	-142.822\\
1034	-117.188\\
1035	-89.111\\
1036	-108.643\\
1037	-137.939\\
1038	-130.615\\
1039	-136.719\\
1040	-140.381\\
1041	-151.367\\
1042	-213.623\\
1043	-191.65\\
1044	-126.953\\
1045	-115.967\\
1046	-64.697\\
1047	-69.58\\
1048	-96.436\\
1049	-74.463\\
1050	-78.125\\
1051	-111.084\\
1052	-128.174\\
1053	-119.629\\
1054	-190.43\\
1055	-139.16\\
1056	-81.787\\
1057	-63.477\\
1058	-85.449\\
1059	-101.318\\
1060	-51.27\\
1061	-58.594\\
1062	-81.787\\
1063	-65.918\\
1064	-56.152\\
1065	-74.463\\
1066	-86.67\\
1067	-139.16\\
1068	-130.615\\
1069	-163.574\\
1070	-146.484\\
1071	-170.898\\
1072	-139.16\\
1073	-133.057\\
1074	-115.967\\
1075	-111.084\\
1076	-111.084\\
1077	-196.533\\
1078	-280.762\\
1079	-291.748\\
1080	-280.762\\
1081	-175.781\\
1082	-240.479\\
1083	-303.955\\
1084	-317.383\\
1085	-239.258\\
1086	-313.721\\
1087	-404.053\\
1088	-289.307\\
1089	-238.037\\
1090	-158.691\\
1091	-122.07\\
1092	-102.539\\
1093	-129.395\\
1094	-92.773\\
1095	-61.035\\
1096	-59.814\\
1097	-75.684\\
1098	-118.408\\
1099	-107.422\\
1100	-111.084\\
1101	-146.484\\
1102	-152.588\\
1103	-207.52\\
1104	-167.236\\
1105	-208.74\\
1106	-148.926\\
1107	-130.615\\
1108	-151.367\\
1109	-108.643\\
1110	-98.877\\
1111	-122.07\\
1112	-123.291\\
1113	-85.449\\
1114	-95.215\\
1115	-67.139\\
1116	-76.904\\
1117	-81.787\\
1118	-57.373\\
1119	-75.684\\
1120	-93.994\\
1121	-151.367\\
1122	-152.588\\
1123	-170.898\\
1124	-97.656\\
1125	-140.381\\
1126	-211.182\\
1127	-161.133\\
1128	-186.768\\
1129	-191.65\\
1130	-112.305\\
1131	-75.684\\
1132	-107.422\\
1133	-123.291\\
1134	-205.078\\
1135	-252.686\\
1136	-252.686\\
1137	-184.326\\
1138	-189.209\\
1139	-167.236\\
1140	-163.574\\
1141	-162.354\\
1142	-108.643\\
1143	-96.436\\
1144	-86.67\\
1145	-78.125\\
1146	-87.891\\
1147	-133.057\\
1148	-203.857\\
1149	-161.133\\
1150	-109.863\\
1151	-92.773\\
1152	-89.111\\
1153	-52.49\\
1154	-41.504\\
1155	-47.607\\
1156	-90.332\\
1157	-67.139\\
1158	-78.125\\
1159	-85.449\\
1160	-80.566\\
1161	-54.932\\
1162	-37.842\\
1163	-30.518\\
1164	-54.932\\
1165	-119.629\\
1166	-172.119\\
1167	-194.092\\
1168	-136.719\\
1169	-91.553\\
1170	-64.697\\
1171	-43.945\\
1172	-98.877\\
1173	-91.553\\
1174	-137.939\\
1175	-168.457\\
1176	-275.879\\
1177	-316.162\\
1178	-260.01\\
1179	-249.023\\
1180	-159.912\\
1181	-185.547\\
1182	-195.313\\
1183	-212.402\\
1184	-158.691\\
1185	-144.043\\
1186	-142.822\\
1187	-129.395\\
1188	-156.25\\
1189	-109.863\\
1190	-195.313\\
1191	-234.375\\
1192	-175.781\\
1193	-104.98\\
1194	-111.084\\
1195	-192.871\\
1196	-234.375\\
1197	-289.307\\
1198	-303.955\\
1199	-302.734\\
1200	-228.271\\
1201	-205.078\\
1202	-235.596\\
1203	-275.879\\
1204	-170.898\\
1205	-104.98\\
1206	-152.588\\
1207	-122.07\\
1208	-79.346\\
1209	-109.863\\
1210	-122.07\\
1211	-95.215\\
1212	-75.684\\
1213	-101.318\\
1214	-80.566\\
1215	-115.967\\
1216	-162.354\\
1217	-147.705\\
1218	-100.098\\
1219	-101.318\\
1220	-157.471\\
1221	-261.23\\
1222	-185.547\\
1223	-118.408\\
1224	-85.449\\
1225	-101.318\\
1226	-90.332\\
1227	-67.139\\
1228	-76.904\\
1229	-92.773\\
1230	-120.85\\
1231	-134.277\\
1232	-89.111\\
1233	-156.25\\
1234	-179.443\\
1235	-170.898\\
1236	-140.381\\
1237	-150.146\\
1238	-202.637\\
1239	-125.732\\
1240	-53.711\\
1241	-78.125\\
1242	-85.449\\
1243	-119.629\\
1244	-101.318\\
1245	-74.463\\
1246	-118.408\\
1247	-112.305\\
1248	-79.346\\
1249	-102.539\\
1250	-90.332\\
1251	-80.566\\
1252	-95.215\\
1253	-54.932\\
1254	-68.359\\
1255	-47.607\\
1256	-51.27\\
1257	-106.201\\
1258	-122.07\\
1259	-167.236\\
1260	-219.727\\
1261	-142.822\\
1262	-83.008\\
1263	-63.477\\
1264	-62.256\\
1265	-91.553\\
1266	-54.932\\
1267	-70.801\\
1268	-85.449\\
1269	-150.146\\
1270	-133.057\\
1271	-190.43\\
1272	-124.512\\
1273	-117.188\\
1274	-56.152\\
1275	-62.256\\
1276	-34.18\\
1277	-46.387\\
1278	-51.27\\
1279	-95.215\\
1280	-100.098\\
1281	-117.188\\
1282	-123.291\\
1283	-197.754\\
1284	-170.898\\
1285	-128.174\\
1286	-153.809\\
1287	-163.574\\
1288	-213.623\\
1289	-161.133\\
1290	-104.98\\
1291	-53.711\\
1292	-40.283\\
1293	-41.504\\
1294	-52.49\\
1295	-54.932\\
1296	-51.27\\
1297	-70.801\\
1298	-108.643\\
1299	-98.877\\
1300	-103.76\\
1301	-118.408\\
1302	-69.58\\
1303	-36.621\\
1304	-81.787\\
1305	-125.732\\
1306	-125.732\\
1307	-142.822\\
1308	-118.408\\
1309	-87.891\\
1310	-123.291\\
1311	-172.119\\
1312	-172.119\\
1313	-120.85\\
1314	-207.52\\
1315	-155.029\\
1316	-151.367\\
1317	-173.34\\
1318	-172.119\\
1319	-130.615\\
1320	-112.305\\
1321	-192.871\\
1322	-266.113\\
1323	-202.637\\
1324	-124.512\\
1325	-118.408\\
1326	-115.967\\
1327	-150.146\\
1328	-173.34\\
1329	-125.732\\
1330	-134.277\\
1331	-177.002\\
1332	-192.871\\
1333	-120.85\\
1334	-97.656\\
1335	-130.615\\
1336	-218.506\\
1337	-195.313\\
1338	-203.857\\
1339	-115.967\\
1340	-108.643\\
1341	-101.318\\
1342	-63.477\\
1343	-41.504\\
1344	-34.18\\
1345	-29.297\\
1346	-78.125\\
1347	-108.643\\
1348	-131.836\\
1349	-93.994\\
1350	-107.422\\
1351	-119.629\\
1352	-72.021\\
1353	-79.346\\
1354	-73.242\\
1355	-54.932\\
1356	-75.684\\
1357	-120.85\\
1358	-156.25\\
1359	-95.215\\
1360	-73.242\\
1361	-58.594\\
1362	-74.463\\
1363	-45.166\\
1364	-117.188\\
1365	-195.313\\
1366	-163.574\\
1367	-216.064\\
1368	-241.699\\
1369	-249.023\\
1370	-181.885\\
1371	-131.836\\
1372	-130.615\\
1373	-129.395\\
1374	-155.029\\
1375	-184.326\\
1376	-181.885\\
1377	-247.803\\
1378	-270.996\\
1379	-328.369\\
1380	-217.285\\
1381	-238.037\\
1382	-264.893\\
1383	-205.078\\
1384	-239.258\\
1385	-200.195\\
1386	-113.525\\
1387	-74.463\\
1388	-79.346\\
1389	-117.188\\
1390	-97.656\\
1391	-64.697\\
1392	-90.332\\
1393	-63.477\\
1394	-48.828\\
1395	-64.697\\
1396	-72.021\\
1397	-48.828\\
1398	-100.098\\
1399	-135.498\\
1400	-97.656\\
1401	-173.34\\
1402	-241.699\\
1403	-220.947\\
1404	-279.541\\
1405	-187.988\\
1406	-206.299\\
1407	-134.277\\
1408	-85.449\\
1409	-108.643\\
1410	-87.891\\
1411	-63.477\\
1412	-92.773\\
1413	-50.049\\
1414	-32.959\\
1415	-43.945\\
1416	-98.877\\
1417	-125.732\\
1418	-157.471\\
1419	-152.588\\
1420	-146.484\\
1421	-168.457\\
1422	-135.498\\
1423	-108.643\\
1424	-112.305\\
1425	-89.111\\
1426	-146.484\\
1427	-96.436\\
1428	-80.566\\
1429	-120.85\\
1430	-164.795\\
1431	-123.291\\
1432	-93.994\\
1433	-80.566\\
1434	-91.553\\
1435	-61.035\\
1436	-61.035\\
1437	-87.891\\
1438	-109.863\\
1439	-124.512\\
1440	-96.436\\
1441	-128.174\\
1442	-114.746\\
1443	-61.035\\
1444	-69.58\\
1445	-135.498\\
1446	-190.43\\
1447	-186.768\\
1448	-157.471\\
1449	-152.588\\
1450	-114.746\\
1451	-109.863\\
1452	-87.891\\
1453	-125.732\\
1454	-117.188\\
1455	-70.801\\
1456	-107.422\\
1457	-126.953\\
1458	-150.146\\
1459	-164.795\\
1460	-163.574\\
1461	-222.168\\
1462	-267.334\\
1463	-302.734\\
1464	-190.43\\
1465	-106.201\\
1466	-68.359\\
1467	-45.166\\
1468	-70.801\\
1469	-64.697\\
1470	-119.629\\
1471	-102.539\\
1472	-93.994\\
1473	-124.512\\
1474	-145.264\\
1475	-172.119\\
1476	-183.105\\
1477	-258.789\\
1478	-231.934\\
1479	-177.002\\
1480	-120.85\\
1481	-120.85\\
1482	-123.291\\
1483	-157.471\\
1484	-108.643\\
1485	-113.525\\
1486	-107.422\\
1487	-58.594\\
1488	-102.539\\
1489	-152.588\\
1490	-107.422\\
1491	-115.967\\
1492	-216.064\\
1493	-159.912\\
1494	-156.25\\
1495	-219.727\\
1496	-206.299\\
1497	-129.395\\
1498	-85.449\\
1499	-85.449\\
1500	-146.484\\
};
\end{axis}

\begin{axis}[%
width=5.5cm,
height=2.1cm,
at={(0cm,11.440678cm)},
scale only axis,
xmin=1000,
xmax=1500,
xlabel={\small Time, sec},
ymin=-7,
ymax=-4.999,
ylabel={\small $\Delta \mathrm{x}$, mm},
legend style={legend cell align=left,align=left,draw=white!15!black}
]
\addplot [color=mycolor1,line width=0.5pt,solid,forget plot]
  table[row sep=crcr]{%
1000	-6.134\\
1001	-6.226\\
1002	-6.171\\
1003	-6.152\\
1004	-6.244\\
1005	-6.281\\
1006	-6.335\\
1007	-6.262\\
1008	-5.969\\
1009	-5.914\\
1010	-5.933\\
1011	-5.988\\
1012	-5.951\\
1013	-5.603\\
1014	-5.273\\
1015	-5.127\\
1016	-5.511\\
1017	-5.914\\
1018	-6.006\\
1019	-6.079\\
1020	-5.933\\
1021	-5.713\\
1022	-5.969\\
1023	-5.914\\
1024	-5.75\\
1025	-5.914\\
1026	-5.933\\
1027	-5.878\\
1028	-6.042\\
1029	-6.116\\
1030	-5.988\\
1031	-6.116\\
1032	-6.171\\
1033	-6.079\\
1034	-5.988\\
1035	-5.859\\
1036	-5.878\\
1037	-5.969\\
1038	-5.951\\
1039	-5.969\\
1040	-5.988\\
1041	-6.024\\
1042	-6.207\\
1043	-6.226\\
1044	-6.079\\
1045	-5.988\\
1046	-5.75\\
1047	-5.658\\
1048	-5.75\\
1049	-5.713\\
1050	-5.658\\
1051	-5.786\\
1052	-5.878\\
1053	-5.878\\
1054	-6.079\\
1055	-6.042\\
1056	-5.804\\
1057	-5.658\\
1058	-5.695\\
1059	-5.786\\
1060	-5.603\\
1061	-5.493\\
1062	-5.603\\
1063	-5.566\\
1064	-5.475\\
1065	-5.53\\
1066	-5.603\\
1067	-5.823\\
1068	-5.878\\
1069	-5.969\\
1070	-5.988\\
1071	-6.042\\
1072	-6.006\\
1073	-5.951\\
1074	-5.933\\
1075	-5.896\\
1076	-5.878\\
1077	-6.116\\
1078	-6.409\\
1079	-6.519\\
1080	-6.519\\
1081	-6.335\\
1082	-6.427\\
1083	-6.61\\
1084	-6.665\\
1085	-6.592\\
1086	-6.683\\
1087	-6.903\\
1088	-6.812\\
1089	-6.683\\
1090	-6.464\\
1091	-6.262\\
1092	-6.134\\
1093	-6.171\\
1094	-6.061\\
1095	-5.841\\
1096	-5.713\\
1097	-5.731\\
1098	-5.878\\
1099	-5.914\\
1100	-5.896\\
1101	-6.024\\
1102	-6.061\\
1103	-6.262\\
1104	-6.226\\
1105	-6.299\\
1106	-6.226\\
1107	-6.134\\
1108	-6.171\\
1109	-6.079\\
1110	-5.988\\
1111	-6.024\\
1112	-6.042\\
1113	-5.914\\
1114	-5.878\\
1115	-5.786\\
1116	-5.75\\
1117	-5.768\\
1118	-5.676\\
1119	-5.658\\
1120	-5.75\\
1121	-5.969\\
1122	-6.042\\
1123	-6.097\\
1124	-5.914\\
1125	-5.988\\
1126	-6.244\\
1127	-6.207\\
1128	-6.262\\
1129	-6.281\\
1130	-6.061\\
1131	-5.859\\
1132	-5.878\\
1133	-5.969\\
1134	-6.207\\
1135	-6.409\\
1136	-6.445\\
1137	-6.335\\
1138	-6.299\\
1139	-6.281\\
1140	-6.244\\
1141	-6.262\\
1142	-6.097\\
1143	-5.969\\
1144	-5.914\\
1145	-5.841\\
1146	-5.859\\
1147	-5.988\\
1148	-6.244\\
1149	-6.207\\
1150	-6.042\\
1151	-5.933\\
1152	-5.878\\
1153	-5.695\\
1154	-5.475\\
1155	-5.438\\
1156	-5.658\\
1157	-5.621\\
1158	-5.603\\
1159	-5.658\\
1160	-5.64\\
1161	-5.53\\
1162	-5.383\\
1163	-5.237\\
1164	-5.31\\
1165	-5.695\\
1166	-5.988\\
1167	-6.134\\
1168	-6.024\\
1169	-5.823\\
1170	-5.658\\
1171	-5.438\\
1172	-5.676\\
1173	-5.768\\
1174	-5.914\\
1175	-6.079\\
1176	-6.409\\
1177	-6.647\\
1178	-6.61\\
1179	-6.555\\
1180	-6.39\\
1181	-6.372\\
1182	-6.409\\
1183	-6.445\\
1184	-6.354\\
1185	-6.281\\
1186	-6.244\\
1187	-6.207\\
1188	-6.226\\
1189	-6.134\\
1190	-6.262\\
1191	-6.445\\
1192	-6.317\\
1193	-6.097\\
1194	-5.988\\
1195	-6.244\\
1196	-6.39\\
1197	-6.555\\
1198	-6.628\\
1199	-6.665\\
1200	-6.573\\
1201	-6.482\\
1202	-6.555\\
1203	-6.61\\
1204	-6.427\\
1205	-6.171\\
1206	-6.207\\
1207	-6.207\\
1208	-5.988\\
1209	-6.006\\
1210	-6.061\\
1211	-5.988\\
1212	-5.878\\
1213	-5.914\\
1214	-5.878\\
1215	-5.951\\
1216	-6.134\\
1217	-6.171\\
1218	-5.988\\
1219	-5.969\\
1220	-6.097\\
1221	-6.445\\
1222	-6.409\\
1223	-6.171\\
1224	-5.988\\
1225	-5.969\\
1226	-5.951\\
1227	-5.841\\
1228	-5.786\\
1229	-5.859\\
1230	-5.969\\
1231	-6.006\\
1232	-5.896\\
1233	-6.079\\
1234	-6.207\\
1235	-6.189\\
1236	-6.116\\
1237	-6.134\\
1238	-6.262\\
1239	-6.116\\
1240	-5.713\\
1241	-5.676\\
1242	-5.75\\
1243	-5.878\\
1244	-5.878\\
1245	-5.768\\
1246	-5.896\\
1247	-5.914\\
1248	-5.804\\
1249	-5.841\\
1250	-5.841\\
1251	-5.786\\
1252	-5.823\\
1253	-5.676\\
1254	-5.621\\
1255	-5.566\\
1256	-5.53\\
1257	-5.768\\
1258	-5.878\\
1259	-6.061\\
1260	-6.262\\
1261	-6.152\\
1262	-5.859\\
1263	-5.695\\
1264	-5.64\\
1265	-5.75\\
1266	-5.621\\
1267	-5.585\\
1268	-5.676\\
1269	-5.951\\
1270	-5.988\\
1271	-6.116\\
1272	-6.042\\
1273	-5.933\\
1274	-5.695\\
1275	-5.566\\
1276	-5.402\\
1277	-5.347\\
1278	-5.402\\
1279	-5.621\\
1280	-5.695\\
1281	-5.786\\
1282	-5.841\\
1283	-6.061\\
1284	-6.116\\
1285	-6.006\\
1286	-6.042\\
1287	-6.079\\
1288	-6.207\\
1289	-6.189\\
1290	-6.006\\
1291	-5.676\\
1292	-5.438\\
1293	-5.365\\
1294	-5.383\\
1295	-5.402\\
1296	-5.383\\
1297	-5.457\\
1298	-5.676\\
1299	-5.713\\
1300	-5.713\\
1301	-5.804\\
1302	-5.64\\
1303	-5.347\\
1304	-5.475\\
1305	-5.731\\
1306	-5.786\\
1307	-5.859\\
1308	-5.841\\
1309	-5.713\\
1310	-5.804\\
1311	-6.006\\
1312	-6.079\\
1313	-5.951\\
1314	-6.134\\
1315	-6.134\\
1316	-6.079\\
1317	-6.134\\
1318	-6.134\\
1319	-6.061\\
1320	-5.969\\
1321	-6.171\\
1322	-6.409\\
1323	-6.372\\
1324	-6.116\\
1325	-6.006\\
1326	-6.006\\
1327	-6.079\\
1328	-6.171\\
1329	-6.079\\
1330	-6.061\\
1331	-6.171\\
1332	-6.226\\
1333	-6.097\\
1334	-5.951\\
1335	-6.042\\
1336	-6.281\\
1337	-6.335\\
1338	-6.317\\
1339	-6.116\\
1340	-5.969\\
1341	-5.988\\
1342	-5.768\\
1343	-5.53\\
1344	-5.328\\
1345	-5.2\\
1346	-5.438\\
1347	-5.713\\
1348	-5.823\\
1349	-5.75\\
1350	-5.768\\
1351	-5.841\\
1352	-5.695\\
1353	-5.621\\
1354	-5.603\\
1355	-5.511\\
1356	-5.548\\
1357	-5.75\\
1358	-5.933\\
1359	-5.823\\
1360	-5.64\\
1361	-5.585\\
1362	-5.585\\
1363	-5.475\\
1364	-5.713\\
1365	-6.097\\
1366	-6.079\\
1367	-6.189\\
1368	-6.335\\
1369	-6.409\\
1370	-6.281\\
1371	-6.134\\
1372	-6.097\\
1373	-6.097\\
1374	-6.134\\
1375	-6.207\\
1376	-6.244\\
1377	-6.372\\
1378	-6.5\\
1379	-6.647\\
1380	-6.555\\
1381	-6.5\\
1382	-6.592\\
1383	-6.482\\
1384	-6.519\\
1385	-6.482\\
1386	-6.207\\
1387	-5.914\\
1388	-5.823\\
1389	-5.951\\
1390	-5.914\\
1391	-5.75\\
1392	-5.768\\
1393	-5.731\\
1394	-5.548\\
1395	-5.566\\
1396	-5.603\\
1397	-5.511\\
1398	-5.695\\
1399	-5.878\\
1400	-5.823\\
1401	-6.024\\
1402	-6.317\\
1403	-6.372\\
1404	-6.464\\
1405	-6.372\\
1406	-6.354\\
1407	-6.244\\
1408	-5.951\\
1409	-5.951\\
1410	-5.878\\
1411	-5.75\\
1412	-5.786\\
1413	-5.603\\
1414	-5.365\\
1415	-5.328\\
1416	-5.658\\
1417	-5.878\\
1418	-6.006\\
1419	-6.042\\
1420	-6.024\\
1421	-6.097\\
1422	-6.042\\
1423	-5.951\\
1424	-5.951\\
1425	-5.878\\
1426	-5.988\\
1427	-5.933\\
1428	-5.786\\
1429	-5.914\\
1430	-6.079\\
1431	-6.042\\
1432	-5.896\\
1433	-5.823\\
1434	-5.804\\
1435	-5.713\\
1436	-5.621\\
1437	-5.695\\
1438	-5.804\\
1439	-5.878\\
1440	-5.823\\
1441	-5.878\\
1442	-5.896\\
1443	-5.695\\
1444	-5.64\\
1445	-5.896\\
1446	-6.134\\
1447	-6.189\\
1448	-6.134\\
1449	-6.116\\
1450	-6.024\\
1451	-5.951\\
1452	-5.878\\
1453	-5.933\\
1454	-5.951\\
1455	-5.768\\
1456	-5.823\\
1457	-5.933\\
1458	-5.988\\
1459	-6.079\\
1460	-6.079\\
1461	-6.244\\
1462	-6.427\\
1463	-6.573\\
1464	-6.445\\
1465	-6.097\\
1466	-5.859\\
1467	-5.603\\
1468	-5.621\\
1469	-5.621\\
1470	-5.804\\
1471	-5.878\\
1472	-5.823\\
1473	-5.914\\
1474	-6.006\\
1475	-6.097\\
1476	-6.171\\
1477	-6.39\\
1478	-6.445\\
1479	-6.317\\
1480	-6.116\\
1481	-6.042\\
1482	-6.042\\
1483	-6.116\\
1484	-6.024\\
1485	-5.988\\
1486	-5.969\\
1487	-5.768\\
1488	-5.841\\
1489	-6.042\\
1490	-5.988\\
1491	-5.951\\
1492	-6.244\\
1493	-6.226\\
1494	-6.152\\
1495	-6.299\\
1496	-6.335\\
1497	-6.171\\
1498	-5.914\\
1499	-5.841\\
1500	-6.042\\
};
\end{axis}

\begin{axis}[%
width=5.5cm,
height=2.1cm,
at={(0cm,7.627119cm)},
scale only axis,
xmin=1000,
xmax=1500,
xlabel={\small Time, sec},
ymin=-7,
ymax=-4.999,
ylabel={\small $\Delta \mathrm{x}$, mm},
legend style={legend cell align=left,align=left,draw=white!15!black}
]
\addplot [color=mycolor1,line width=0.5pt,solid,forget plot]
  table[row sep=crcr]{%
1000	-6.116\\
1001	-6.226\\
1002	-6.189\\
1003	-6.134\\
1004	-6.262\\
1005	-6.262\\
1006	-6.354\\
1007	-6.299\\
1008	-5.988\\
1009	-5.933\\
1010	-5.933\\
1011	-6.006\\
1012	-5.951\\
1013	-5.603\\
1014	-5.292\\
1015	-5.145\\
1016	-5.493\\
1017	-5.878\\
1018	-6.006\\
1019	-6.042\\
1020	-5.951\\
1021	-5.713\\
1022	-5.951\\
1023	-5.914\\
1024	-5.75\\
1025	-5.933\\
1026	-5.951\\
1027	-5.859\\
1028	-6.042\\
1029	-6.079\\
1030	-5.969\\
1031	-6.079\\
1032	-6.134\\
1033	-6.061\\
1034	-5.969\\
1035	-5.859\\
1036	-5.878\\
1037	-5.951\\
1038	-5.969\\
1039	-5.969\\
1040	-5.988\\
1041	-6.006\\
1042	-6.171\\
1043	-6.207\\
1044	-6.061\\
1045	-5.969\\
1046	-5.75\\
1047	-5.64\\
1048	-5.75\\
1049	-5.695\\
1050	-5.658\\
1051	-5.786\\
1052	-5.878\\
1053	-5.878\\
1054	-6.079\\
1055	-6.042\\
1056	-5.804\\
1057	-5.658\\
1058	-5.695\\
1059	-5.786\\
1060	-5.585\\
1061	-5.493\\
1062	-5.603\\
1063	-5.566\\
1064	-5.475\\
1065	-5.548\\
1066	-5.603\\
1067	-5.841\\
1068	-5.878\\
1069	-5.969\\
1070	-5.988\\
1071	-6.024\\
1072	-5.988\\
1073	-5.951\\
1074	-5.914\\
1075	-5.896\\
1076	-5.878\\
1077	-6.079\\
1078	-6.372\\
1079	-6.5\\
1080	-6.519\\
1081	-6.335\\
1082	-6.427\\
1083	-6.592\\
1084	-6.665\\
1085	-6.592\\
1086	-6.665\\
1087	-6.885\\
1088	-6.812\\
1089	-6.683\\
1090	-6.464\\
1091	-6.262\\
1092	-6.134\\
1093	-6.152\\
1094	-6.024\\
1095	-5.786\\
1096	-5.658\\
1097	-5.713\\
1098	-5.878\\
1099	-5.896\\
1100	-5.896\\
1101	-6.006\\
1102	-6.079\\
1103	-6.244\\
1104	-6.262\\
1105	-6.299\\
1106	-6.226\\
1107	-6.116\\
1108	-6.152\\
1109	-6.061\\
1110	-5.969\\
1111	-6.024\\
1112	-6.042\\
1113	-5.933\\
1114	-5.896\\
1115	-5.786\\
1116	-5.731\\
1117	-5.768\\
1118	-5.676\\
1119	-5.676\\
1120	-5.768\\
1121	-5.988\\
1122	-6.042\\
1123	-6.116\\
1124	-5.933\\
1125	-5.988\\
1126	-6.244\\
1127	-6.226\\
1128	-6.244\\
1129	-6.262\\
1130	-6.079\\
1131	-5.841\\
1132	-5.896\\
1133	-5.969\\
1134	-6.207\\
1135	-6.409\\
1136	-6.464\\
1137	-6.335\\
1138	-6.335\\
1139	-6.299\\
1140	-6.262\\
1141	-6.262\\
1142	-6.097\\
1143	-5.988\\
1144	-5.914\\
1145	-5.841\\
1146	-5.859\\
1147	-6.006\\
1148	-6.226\\
1149	-6.244\\
1150	-6.061\\
1151	-5.951\\
1152	-5.896\\
1153	-5.695\\
1154	-5.475\\
1155	-5.457\\
1156	-5.64\\
1157	-5.621\\
1158	-5.621\\
1159	-5.658\\
1160	-5.658\\
1161	-5.53\\
1162	-5.383\\
1163	-5.219\\
1164	-5.292\\
1165	-5.676\\
1166	-5.988\\
1167	-6.116\\
1168	-6.006\\
1169	-5.804\\
1170	-5.64\\
1171	-5.457\\
1172	-5.64\\
1173	-5.713\\
1174	-5.841\\
1175	-6.024\\
1176	-6.372\\
1177	-6.61\\
1178	-6.592\\
1179	-6.537\\
1180	-6.354\\
1181	-6.317\\
1182	-6.372\\
1183	-6.409\\
1184	-6.335\\
1185	-6.226\\
1186	-6.226\\
1187	-6.189\\
1188	-6.207\\
1189	-6.116\\
1190	-6.262\\
1191	-6.445\\
1192	-6.335\\
1193	-6.116\\
1194	-6.024\\
1195	-6.207\\
1196	-6.39\\
1197	-6.519\\
1198	-6.61\\
1199	-6.665\\
1200	-6.555\\
1201	-6.482\\
1202	-6.519\\
1203	-6.61\\
1204	-6.409\\
1205	-6.134\\
1206	-6.207\\
1207	-6.171\\
1208	-5.951\\
1209	-5.969\\
1210	-6.042\\
1211	-5.969\\
1212	-5.878\\
1213	-5.914\\
1214	-5.878\\
1215	-5.914\\
1216	-6.116\\
1217	-6.134\\
1218	-6.006\\
1219	-5.951\\
1220	-6.097\\
1221	-6.409\\
1222	-6.409\\
1223	-6.152\\
1224	-5.988\\
1225	-5.969\\
1226	-5.951\\
1227	-5.804\\
1228	-5.786\\
1229	-5.841\\
1230	-5.933\\
1231	-6.024\\
1232	-5.914\\
1233	-6.042\\
1234	-6.207\\
1235	-6.207\\
1236	-6.116\\
1237	-6.116\\
1238	-6.244\\
1239	-6.171\\
1240	-5.786\\
1241	-5.713\\
1242	-5.786\\
1243	-5.914\\
1244	-5.914\\
1245	-5.804\\
1246	-5.914\\
1247	-5.933\\
1248	-5.804\\
1249	-5.859\\
1250	-5.841\\
1251	-5.768\\
1252	-5.823\\
1253	-5.676\\
1254	-5.64\\
1255	-5.548\\
1256	-5.511\\
1257	-5.75\\
1258	-5.859\\
1259	-6.006\\
1260	-6.226\\
1261	-6.134\\
1262	-5.878\\
1263	-5.695\\
1264	-5.621\\
1265	-5.75\\
1266	-5.64\\
1267	-5.585\\
1268	-5.676\\
1269	-5.933\\
1270	-5.988\\
1271	-6.152\\
1272	-6.079\\
1273	-5.951\\
1274	-5.713\\
1275	-5.585\\
1276	-5.42\\
1277	-5.365\\
1278	-5.402\\
1279	-5.603\\
1280	-5.713\\
1281	-5.768\\
1282	-5.823\\
1283	-6.042\\
1284	-6.134\\
1285	-6.006\\
1286	-6.024\\
1287	-6.097\\
1288	-6.226\\
1289	-6.189\\
1290	-6.006\\
1291	-5.676\\
1292	-5.42\\
1293	-5.347\\
1294	-5.383\\
1295	-5.402\\
1296	-5.383\\
1297	-5.438\\
1298	-5.676\\
1299	-5.731\\
1300	-5.713\\
1301	-5.786\\
1302	-5.658\\
1303	-5.402\\
1304	-5.53\\
1305	-5.786\\
1306	-5.841\\
1307	-5.914\\
1308	-5.878\\
1309	-5.75\\
1310	-5.823\\
1311	-6.006\\
1312	-6.079\\
1313	-5.969\\
1314	-6.134\\
1315	-6.116\\
1316	-6.042\\
1317	-6.134\\
1318	-6.152\\
1319	-6.061\\
1320	-5.988\\
1321	-6.152\\
1322	-6.409\\
1323	-6.372\\
1324	-6.116\\
1325	-5.988\\
1326	-6.006\\
1327	-6.061\\
1328	-6.134\\
1329	-6.061\\
1330	-6.061\\
1331	-6.171\\
1332	-6.226\\
1333	-6.097\\
1334	-5.988\\
1335	-6.042\\
1336	-6.317\\
1337	-6.354\\
1338	-6.335\\
1339	-6.152\\
1340	-6.006\\
1341	-5.969\\
1342	-5.786\\
1343	-5.53\\
1344	-5.347\\
1345	-5.219\\
1346	-5.438\\
1347	-5.676\\
1348	-5.823\\
1349	-5.768\\
1350	-5.75\\
1351	-5.823\\
1352	-5.676\\
1353	-5.621\\
1354	-5.621\\
1355	-5.511\\
1356	-5.548\\
1357	-5.75\\
1358	-5.933\\
1359	-5.786\\
1360	-5.621\\
1361	-5.548\\
1362	-5.585\\
1363	-5.457\\
1364	-5.695\\
1365	-6.061\\
1366	-6.079\\
1367	-6.171\\
1368	-6.317\\
1369	-6.39\\
1370	-6.281\\
1371	-6.116\\
1372	-6.079\\
1373	-6.061\\
1374	-6.116\\
1375	-6.207\\
1376	-6.226\\
1377	-6.372\\
1378	-6.5\\
1379	-6.628\\
1380	-6.519\\
1381	-6.519\\
1382	-6.573\\
1383	-6.5\\
1384	-6.519\\
1385	-6.482\\
1386	-6.226\\
1387	-5.933\\
1388	-5.841\\
1389	-5.951\\
1390	-5.914\\
1391	-5.75\\
1392	-5.804\\
1393	-5.731\\
1394	-5.585\\
1395	-5.603\\
1396	-5.621\\
1397	-5.548\\
1398	-5.695\\
1399	-5.896\\
1400	-5.841\\
1401	-6.042\\
1402	-6.354\\
1403	-6.409\\
1404	-6.537\\
1405	-6.427\\
1406	-6.409\\
1407	-6.262\\
1408	-5.988\\
1409	-5.969\\
1410	-5.914\\
1411	-5.75\\
1412	-5.786\\
1413	-5.676\\
1414	-5.402\\
1415	-5.365\\
1416	-5.64\\
1417	-5.859\\
1418	-5.988\\
1419	-6.042\\
1420	-6.024\\
1421	-6.079\\
1422	-6.042\\
1423	-5.951\\
1424	-5.933\\
1425	-5.859\\
1426	-5.988\\
1427	-5.896\\
1428	-5.768\\
1429	-5.896\\
1430	-6.061\\
1431	-6.006\\
1432	-5.878\\
1433	-5.804\\
1434	-5.823\\
1435	-5.695\\
1436	-5.603\\
1437	-5.695\\
1438	-5.786\\
1439	-5.859\\
1440	-5.823\\
1441	-5.878\\
1442	-5.878\\
1443	-5.695\\
1444	-5.64\\
1445	-5.878\\
1446	-6.152\\
1447	-6.207\\
1448	-6.134\\
1449	-6.097\\
1450	-6.024\\
1451	-5.951\\
1452	-5.878\\
1453	-5.933\\
1454	-5.951\\
1455	-5.768\\
1456	-5.841\\
1457	-5.914\\
1458	-5.988\\
1459	-6.079\\
1460	-6.079\\
1461	-6.262\\
1462	-6.445\\
1463	-6.573\\
1464	-6.445\\
1465	-6.116\\
1466	-5.823\\
1467	-5.603\\
1468	-5.64\\
1469	-5.621\\
1470	-5.804\\
1471	-5.841\\
1472	-5.786\\
1473	-5.878\\
1474	-5.969\\
1475	-6.079\\
1476	-6.152\\
1477	-6.372\\
1478	-6.427\\
1479	-6.299\\
1480	-6.134\\
1481	-6.042\\
1482	-6.061\\
1483	-6.134\\
1484	-6.042\\
1485	-5.988\\
1486	-6.006\\
1487	-5.75\\
1488	-5.823\\
1489	-6.042\\
1490	-5.969\\
1491	-5.914\\
1492	-6.244\\
1493	-6.207\\
1494	-6.152\\
1495	-6.299\\
1496	-6.335\\
1497	-6.171\\
1498	-5.933\\
1499	-5.841\\
1500	-6.024\\
};
\end{axis}

\begin{axis}[%
width=5.5cm,
height=2.1cm,
at={(0cm,0cm)},
scale only axis,
xmin=1000,
xmax=1500,
xlabel={\small Time, sec},
ymin=-7,
ymax=-4.999,
ylabel={\small $\Delta \mathrm{x}$, mm},
legend style={legend cell align=left,align=left,draw=white!15!black}
]
\addplot [color=mycolor1,line width=0.5pt,solid,forget plot]
  table[row sep=crcr]{%
1000	-6.134\\
1001	-6.244\\
1002	-6.171\\
1003	-6.134\\
1004	-6.262\\
1005	-6.281\\
1006	-6.335\\
1007	-6.244\\
1008	-5.951\\
1009	-5.914\\
1010	-5.951\\
1011	-5.969\\
1012	-5.951\\
1013	-5.585\\
1014	-5.273\\
1015	-5.145\\
1016	-5.493\\
1017	-5.896\\
1018	-6.006\\
1019	-6.061\\
1020	-5.969\\
1021	-5.731\\
1022	-5.988\\
1023	-5.914\\
1024	-5.75\\
1025	-5.933\\
1026	-5.914\\
1027	-5.878\\
1028	-6.042\\
1029	-6.079\\
1030	-5.969\\
1031	-6.079\\
1032	-6.134\\
1033	-6.061\\
1034	-5.969\\
1035	-5.878\\
1036	-5.878\\
1037	-5.969\\
1038	-5.969\\
1039	-5.969\\
1040	-5.988\\
1041	-6.006\\
1042	-6.189\\
1043	-6.207\\
1044	-6.042\\
1045	-5.969\\
1046	-5.75\\
1047	-5.658\\
1048	-5.731\\
1049	-5.676\\
1050	-5.658\\
1051	-5.804\\
1052	-5.878\\
1053	-5.859\\
1054	-6.079\\
1055	-6.042\\
1056	-5.804\\
1057	-5.658\\
1058	-5.695\\
1059	-5.786\\
1060	-5.603\\
1061	-5.493\\
1062	-5.603\\
1063	-5.566\\
1064	-5.493\\
1065	-5.548\\
1066	-5.603\\
1067	-5.841\\
1068	-5.896\\
1069	-5.969\\
1070	-6.006\\
1071	-6.024\\
1072	-5.988\\
1073	-5.951\\
1074	-5.914\\
1075	-5.878\\
1076	-5.896\\
1077	-6.116\\
1078	-6.39\\
1079	-6.519\\
1080	-6.519\\
1081	-6.335\\
1082	-6.445\\
1083	-6.61\\
1084	-6.665\\
1085	-6.555\\
1086	-6.665\\
1087	-6.866\\
1088	-6.793\\
1089	-6.665\\
1090	-6.464\\
1091	-6.244\\
1092	-6.134\\
1093	-6.152\\
1094	-6.042\\
1095	-5.786\\
1096	-5.658\\
1097	-5.695\\
1098	-5.878\\
1099	-5.896\\
1100	-5.896\\
1101	-6.006\\
1102	-6.061\\
1103	-6.226\\
1104	-6.244\\
1105	-6.281\\
1106	-6.189\\
1107	-6.097\\
1108	-6.134\\
1109	-6.042\\
1110	-5.969\\
1111	-6.006\\
1112	-6.024\\
1113	-5.914\\
1114	-5.878\\
1115	-5.804\\
1116	-5.768\\
1117	-5.768\\
1118	-5.676\\
1119	-5.676\\
1120	-5.768\\
1121	-5.969\\
1122	-6.061\\
1123	-6.116\\
1124	-5.933\\
1125	-5.988\\
1126	-6.262\\
1127	-6.226\\
1128	-6.244\\
1129	-6.281\\
1130	-6.061\\
1131	-5.859\\
1132	-5.896\\
1133	-5.969\\
1134	-6.226\\
1135	-6.409\\
1136	-6.464\\
1137	-6.354\\
1138	-6.335\\
1139	-6.281\\
1140	-6.262\\
1141	-6.262\\
1142	-6.097\\
1143	-5.969\\
1144	-5.914\\
1145	-5.841\\
1146	-5.859\\
1147	-6.006\\
1148	-6.244\\
1149	-6.207\\
1150	-6.061\\
1151	-5.951\\
1152	-5.896\\
1153	-5.695\\
1154	-5.475\\
1155	-5.457\\
1156	-5.658\\
1157	-5.64\\
1158	-5.621\\
1159	-5.658\\
1160	-5.64\\
1161	-5.53\\
1162	-5.365\\
1163	-5.219\\
1164	-5.31\\
1165	-5.676\\
1166	-5.988\\
1167	-6.134\\
1168	-6.006\\
1169	-5.804\\
1170	-5.64\\
1171	-5.438\\
1172	-5.64\\
1173	-5.713\\
1174	-5.841\\
1175	-6.024\\
1176	-6.372\\
1177	-6.61\\
1178	-6.573\\
1179	-6.555\\
1180	-6.354\\
1181	-6.335\\
1182	-6.354\\
1183	-6.427\\
1184	-6.317\\
1185	-6.244\\
1186	-6.226\\
1187	-6.171\\
1188	-6.207\\
1189	-6.116\\
1190	-6.299\\
1191	-6.427\\
1192	-6.317\\
1193	-6.079\\
1194	-6.006\\
1195	-6.226\\
1196	-6.39\\
1197	-6.537\\
1198	-6.628\\
1199	-6.683\\
1200	-6.555\\
1201	-6.5\\
1202	-6.537\\
1203	-6.61\\
1204	-6.445\\
1205	-6.152\\
1206	-6.189\\
1207	-6.171\\
1208	-5.951\\
1209	-5.969\\
1210	-6.024\\
1211	-5.969\\
1212	-5.859\\
1213	-5.896\\
1214	-5.878\\
1215	-5.914\\
1216	-6.116\\
1217	-6.134\\
1218	-6.006\\
1219	-5.969\\
1220	-6.097\\
1221	-6.427\\
1222	-6.39\\
1223	-6.152\\
1224	-5.951\\
1225	-5.951\\
1226	-5.933\\
1227	-5.804\\
1228	-5.768\\
1229	-5.859\\
1230	-5.933\\
1231	-6.006\\
1232	-5.878\\
1233	-6.042\\
1234	-6.207\\
1235	-6.207\\
1236	-6.134\\
1237	-6.134\\
1238	-6.281\\
1239	-6.152\\
1240	-5.786\\
1241	-5.731\\
1242	-5.768\\
1243	-5.896\\
1244	-5.896\\
1245	-5.786\\
1246	-5.896\\
1247	-5.933\\
1248	-5.823\\
1249	-5.878\\
1250	-5.859\\
1251	-5.804\\
1252	-5.823\\
1253	-5.676\\
1254	-5.658\\
1255	-5.548\\
1256	-5.511\\
1257	-5.768\\
1258	-5.878\\
1259	-6.024\\
1260	-6.244\\
1261	-6.152\\
1262	-5.878\\
1263	-5.695\\
1264	-5.621\\
1265	-5.75\\
1266	-5.621\\
1267	-5.585\\
1268	-5.676\\
1269	-5.896\\
1270	-5.969\\
1271	-6.134\\
1272	-6.042\\
1273	-5.951\\
1274	-5.713\\
1275	-5.566\\
1276	-5.42\\
1277	-5.328\\
1278	-5.383\\
1279	-5.603\\
1280	-5.695\\
1281	-5.768\\
1282	-5.823\\
1283	-6.097\\
1284	-6.152\\
1285	-6.024\\
1286	-6.061\\
1287	-6.116\\
1288	-6.244\\
1289	-6.226\\
1290	-6.006\\
1291	-5.695\\
1292	-5.457\\
1293	-5.383\\
1294	-5.402\\
1295	-5.42\\
1296	-5.383\\
1297	-5.457\\
1298	-5.676\\
1299	-5.731\\
1300	-5.713\\
1301	-5.786\\
1302	-5.64\\
1303	-5.402\\
1304	-5.53\\
1305	-5.804\\
1306	-5.841\\
1307	-5.914\\
1308	-5.878\\
1309	-5.75\\
1310	-5.823\\
1311	-6.024\\
1312	-6.097\\
1313	-5.988\\
1314	-6.152\\
1315	-6.134\\
1316	-6.061\\
1317	-6.134\\
1318	-6.152\\
1319	-6.061\\
1320	-5.988\\
1321	-6.171\\
1322	-6.409\\
1323	-6.39\\
1324	-6.116\\
1325	-6.006\\
1326	-6.006\\
1327	-6.079\\
1328	-6.152\\
1329	-6.079\\
1330	-6.042\\
1331	-6.171\\
1332	-6.244\\
1333	-6.116\\
1334	-5.969\\
1335	-6.024\\
1336	-6.262\\
1337	-6.299\\
1338	-6.299\\
1339	-6.097\\
1340	-5.988\\
1341	-5.969\\
1342	-5.768\\
1343	-5.511\\
1344	-5.328\\
1345	-5.2\\
1346	-5.438\\
1347	-5.676\\
1348	-5.823\\
1349	-5.768\\
1350	-5.768\\
1351	-5.841\\
1352	-5.676\\
1353	-5.621\\
1354	-5.621\\
1355	-5.511\\
1356	-5.548\\
1357	-5.768\\
1358	-5.933\\
1359	-5.768\\
1360	-5.621\\
1361	-5.548\\
1362	-5.585\\
1363	-5.457\\
1364	-5.695\\
1365	-6.079\\
1366	-6.061\\
1367	-6.189\\
1368	-6.335\\
1369	-6.39\\
1370	-6.299\\
1371	-6.134\\
1372	-6.097\\
1373	-6.061\\
1374	-6.116\\
1375	-6.207\\
1376	-6.226\\
1377	-6.39\\
1378	-6.519\\
1379	-6.628\\
1380	-6.519\\
1381	-6.519\\
1382	-6.592\\
1383	-6.482\\
1384	-6.519\\
1385	-6.482\\
1386	-6.207\\
1387	-5.933\\
1388	-5.823\\
1389	-5.969\\
1390	-5.914\\
1391	-5.75\\
1392	-5.768\\
1393	-5.713\\
1394	-5.566\\
1395	-5.566\\
1396	-5.621\\
1397	-5.53\\
1398	-5.713\\
1399	-5.896\\
1400	-5.823\\
1401	-6.024\\
1402	-6.335\\
1403	-6.354\\
1404	-6.5\\
1405	-6.39\\
1406	-6.372\\
1407	-6.226\\
1408	-5.951\\
1409	-5.969\\
1410	-5.878\\
1411	-5.731\\
1412	-5.786\\
1413	-5.676\\
1414	-5.42\\
1415	-5.365\\
1416	-5.64\\
1417	-5.859\\
1418	-5.988\\
1419	-6.042\\
1420	-6.024\\
1421	-6.097\\
1422	-6.024\\
1423	-5.951\\
1424	-5.933\\
1425	-5.841\\
1426	-5.988\\
1427	-5.914\\
1428	-5.768\\
1429	-5.878\\
1430	-6.061\\
1431	-6.006\\
1432	-5.896\\
1433	-5.804\\
1434	-5.823\\
1435	-5.713\\
1436	-5.603\\
1437	-5.676\\
1438	-5.804\\
1439	-5.878\\
1440	-5.823\\
1441	-5.878\\
1442	-5.878\\
1443	-5.695\\
1444	-5.64\\
1445	-5.896\\
1446	-6.134\\
1447	-6.207\\
1448	-6.134\\
1449	-6.116\\
1450	-6.006\\
1451	-5.951\\
1452	-5.859\\
1453	-5.951\\
1454	-5.933\\
1455	-5.768\\
1456	-5.841\\
1457	-5.914\\
1458	-6.006\\
1459	-6.079\\
1460	-6.097\\
1461	-6.262\\
1462	-6.445\\
1463	-6.592\\
1464	-6.427\\
1465	-6.097\\
1466	-5.823\\
1467	-5.603\\
1468	-5.603\\
1469	-5.621\\
1470	-5.823\\
1471	-5.841\\
1472	-5.786\\
1473	-5.896\\
1474	-5.969\\
1475	-6.079\\
1476	-6.152\\
1477	-6.372\\
1478	-6.409\\
1479	-6.317\\
1480	-6.152\\
1481	-6.061\\
1482	-6.061\\
1483	-6.152\\
1484	-6.061\\
1485	-6.006\\
1486	-5.988\\
1487	-5.75\\
1488	-5.823\\
1489	-6.061\\
1490	-5.969\\
1491	-5.969\\
1492	-6.207\\
1493	-6.189\\
1494	-6.134\\
1495	-6.299\\
1496	-6.299\\
1497	-6.134\\
1498	-5.914\\
1499	-5.823\\
1500	-6.006\\
};
\end{axis}

\begin{axis}[%
width=5.5cm,
height=2.1cm,
at={(7.5cm,3.813559cm)},
scale only axis,
xmin=1000,
xmax=1500,
xlabel={\small Time, sec},
ymin=-405,
ymax=0,
ylabel={\small Load, kN},
legend style={legend cell align=left,align=left,draw=white!15!black}
]
\addplot [color=mycolor1,line width=0.5pt,solid,forget plot]
  table[row sep=crcr]{%
1000	-68.359\\
1001	-85.449\\
1002	-69.58\\
1003	-65.918\\
1004	-90.332\\
1005	-86.67\\
1006	-103.76\\
1007	-79.346\\
1008	-45.166\\
1009	-57.373\\
1010	-51.27\\
1011	-63.477\\
1012	-48.828\\
1013	-24.414\\
1014	-17.09\\
1015	-17.09\\
1016	-43.945\\
1017	-73.242\\
1018	-76.904\\
1019	-79.346\\
1020	-58.594\\
1021	-36.621\\
1022	-76.904\\
1023	-53.711\\
1024	-43.945\\
1025	-69.58\\
1026	-61.035\\
1027	-48.828\\
1028	-84.229\\
1029	-79.346\\
1030	-59.814\\
1031	-87.891\\
1032	-85.449\\
1033	-67.139\\
1034	-54.932\\
1035	-46.387\\
1036	-56.152\\
1037	-64.697\\
1038	-64.697\\
1039	-67.139\\
1040	-68.359\\
1041	-73.242\\
1042	-100.098\\
1043	-87.891\\
1044	-61.035\\
1045	-56.152\\
1046	-34.18\\
1047	-34.18\\
1048	-48.828\\
1049	-37.842\\
1050	-39.063\\
1051	-56.152\\
1052	-62.256\\
1053	-58.594\\
1054	-90.332\\
1055	-64.697\\
1056	-41.504\\
1057	-34.18\\
1058	-45.166\\
1059	-51.27\\
1060	-28.076\\
1061	-25.635\\
1062	-41.504\\
1063	-34.18\\
1064	-29.297\\
1065	-40.283\\
1066	-43.945\\
1067	-68.359\\
1068	-63.477\\
1069	-79.346\\
1070	-69.58\\
1071	-79.346\\
1072	-63.477\\
1073	-62.256\\
1074	-54.932\\
1075	-53.711\\
1076	-53.711\\
1077	-91.553\\
1078	-126.953\\
1079	-131.836\\
1080	-129.395\\
1081	-83.008\\
1082	-114.746\\
1083	-141.602\\
1084	-147.705\\
1085	-111.084\\
1086	-146.484\\
1087	-185.547\\
1088	-131.836\\
1089	-112.305\\
1090	-75.684\\
1091	-61.035\\
1092	-52.49\\
1093	-64.697\\
1094	-46.387\\
1095	-31.738\\
1096	-31.738\\
1097	-40.283\\
1098	-57.373\\
1099	-52.49\\
1100	-53.711\\
1101	-72.021\\
1102	-72.021\\
1103	-97.656\\
1104	-79.346\\
1105	-101.318\\
1106	-72.021\\
1107	-64.697\\
1108	-73.242\\
1109	-54.932\\
1110	-51.27\\
1111	-59.814\\
1112	-62.256\\
1113	-43.945\\
1114	-47.607\\
1115	-34.18\\
1116	-40.283\\
1117	-42.725\\
1118	-30.518\\
1119	-39.063\\
1120	-47.607\\
1121	-72.021\\
1122	-74.463\\
1123	-81.787\\
1124	-48.828\\
1125	-70.801\\
1126	-101.318\\
1127	-84.229\\
1128	-93.994\\
1129	-91.553\\
1130	-54.932\\
1131	-40.283\\
1132	-56.152\\
1133	-61.035\\
1134	-97.656\\
1135	-119.629\\
1136	-117.188\\
1137	-89.111\\
1138	-92.773\\
1139	-81.787\\
1140	-80.566\\
1141	-79.346\\
1142	-54.932\\
1143	-48.828\\
1144	-45.166\\
1145	-40.283\\
1146	-45.166\\
1147	-64.697\\
1148	-96.436\\
1149	-74.463\\
1150	-52.49\\
1151	-45.166\\
1152	-46.387\\
1153	-29.297\\
1154	-23.193\\
1155	-26.855\\
1156	-46.387\\
1157	-35.4\\
1158	-41.504\\
1159	-40.283\\
1160	-39.063\\
1161	-28.076\\
1162	-21.973\\
1163	-17.09\\
1164	-29.297\\
1165	-58.594\\
1166	-80.566\\
1167	-91.553\\
1168	-59.814\\
1169	-43.945\\
1170	-32.959\\
1171	-23.193\\
1172	-47.607\\
1173	-46.387\\
1174	-67.139\\
1175	-81.787\\
1176	-131.836\\
1177	-147.705\\
1178	-120.85\\
1179	-117.188\\
1180	-78.125\\
1181	-81.787\\
1182	-91.553\\
1183	-98.877\\
1184	-73.242\\
1185	-68.359\\
1186	-69.58\\
1187	-62.256\\
1188	-75.684\\
1189	-53.711\\
1190	-93.994\\
1191	-108.643\\
1192	-80.566\\
1193	-51.27\\
1194	-54.932\\
1195	-92.773\\
1196	-109.863\\
1197	-134.277\\
1198	-140.381\\
1199	-140.381\\
1200	-106.201\\
1201	-96.436\\
1202	-111.084\\
1203	-129.395\\
1204	-78.125\\
1205	-50.049\\
1206	-73.242\\
1207	-57.373\\
1208	-39.063\\
1209	-56.152\\
1210	-61.035\\
1211	-46.387\\
1212	-37.842\\
1213	-50.049\\
1214	-39.063\\
1215	-53.711\\
1216	-76.904\\
1217	-72.021\\
1218	-51.27\\
1219	-51.27\\
1220	-78.125\\
1221	-123.291\\
1222	-86.67\\
1223	-56.152\\
1224	-43.945\\
1225	-48.828\\
1226	-43.945\\
1227	-34.18\\
1228	-37.842\\
1229	-47.607\\
1230	-59.814\\
1231	-64.697\\
1232	-46.387\\
1233	-74.463\\
1234	-85.449\\
1235	-83.008\\
1236	-65.918\\
1237	-73.242\\
1238	-96.436\\
1239	-61.035\\
1240	-29.297\\
1241	-42.725\\
1242	-43.945\\
1243	-58.594\\
1244	-50.049\\
1245	-40.283\\
1246	-58.594\\
1247	-54.932\\
1248	-39.063\\
1249	-48.828\\
1250	-43.945\\
1251	-39.063\\
1252	-50.049\\
1253	-29.297\\
1254	-36.621\\
1255	-26.855\\
1256	-31.738\\
1257	-53.711\\
1258	-61.035\\
1259	-76.904\\
1260	-101.318\\
1261	-67.139\\
1262	-42.725\\
1263	-32.959\\
1264	-31.738\\
1265	-47.607\\
1266	-30.518\\
1267	-36.621\\
1268	-43.945\\
1269	-65.918\\
1270	-62.256\\
1271	-86.67\\
1272	-59.814\\
1273	-56.152\\
1274	-31.738\\
1275	-32.959\\
1276	-20.752\\
1277	-24.414\\
1278	-26.855\\
1279	-47.607\\
1280	-47.607\\
1281	-56.152\\
1282	-59.814\\
1283	-93.994\\
1284	-79.346\\
1285	-62.256\\
1286	-73.242\\
1287	-79.346\\
1288	-101.318\\
1289	-80.566\\
1290	-52.49\\
1291	-30.518\\
1292	-21.973\\
1293	-23.193\\
1294	-29.297\\
1295	-29.297\\
1296	-26.855\\
1297	-36.621\\
1298	-53.711\\
1299	-47.607\\
1300	-51.27\\
1301	-57.373\\
1302	-37.842\\
1303	-20.752\\
1304	-42.725\\
1305	-63.477\\
1306	-61.035\\
1307	-69.58\\
1308	-58.594\\
1309	-45.166\\
1310	-59.814\\
1311	-84.229\\
1312	-85.449\\
1313	-59.814\\
1314	-102.539\\
1315	-76.904\\
1316	-74.463\\
1317	-84.229\\
1318	-83.008\\
1319	-63.477\\
1320	-54.932\\
1321	-90.332\\
1322	-124.512\\
1323	-95.215\\
1324	-57.373\\
1325	-53.711\\
1326	-54.932\\
1327	-72.021\\
1328	-83.008\\
1329	-59.814\\
1330	-64.697\\
1331	-83.008\\
1332	-90.332\\
1333	-61.035\\
1334	-50.049\\
1335	-63.477\\
1336	-104.98\\
1337	-92.773\\
1338	-95.215\\
1339	-58.594\\
1340	-53.711\\
1341	-51.27\\
1342	-34.18\\
1343	-23.193\\
1344	-20.752\\
1345	-17.09\\
1346	-40.283\\
1347	-54.932\\
1348	-62.256\\
1349	-47.607\\
1350	-52.49\\
1351	-57.373\\
1352	-37.842\\
1353	-39.063\\
1354	-39.063\\
1355	-28.076\\
1356	-39.063\\
1357	-57.373\\
1358	-73.242\\
1359	-48.828\\
1360	-35.4\\
1361	-31.738\\
1362	-36.621\\
1363	-26.855\\
1364	-54.932\\
1365	-93.994\\
1366	-72.021\\
1367	-97.656\\
1368	-113.525\\
1369	-114.746\\
1370	-85.449\\
1371	-63.477\\
1372	-63.477\\
1373	-63.477\\
1374	-75.684\\
1375	-89.111\\
1376	-86.67\\
1377	-115.967\\
1378	-125.732\\
1379	-150.146\\
1380	-96.436\\
1381	-109.863\\
1382	-122.07\\
1383	-92.773\\
1384	-107.422\\
1385	-92.773\\
1386	-54.932\\
1387	-39.063\\
1388	-40.283\\
1389	-57.373\\
1390	-42.725\\
1391	-32.959\\
1392	-45.166\\
1393	-34.18\\
1394	-26.855\\
1395	-34.18\\
1396	-37.842\\
1397	-25.635\\
1398	-48.828\\
1399	-64.697\\
1400	-46.387\\
1401	-81.787\\
1402	-115.967\\
1403	-106.201\\
1404	-133.057\\
1405	-92.773\\
1406	-97.656\\
1407	-69.58\\
1408	-43.945\\
1409	-53.711\\
1410	-42.725\\
1411	-34.18\\
1412	-45.166\\
1413	-26.855\\
1414	-20.752\\
1415	-25.635\\
1416	-50.049\\
1417	-61.035\\
1418	-78.125\\
1419	-74.463\\
1420	-69.58\\
1421	-80.566\\
1422	-64.697\\
1423	-53.711\\
1424	-54.932\\
1425	-43.945\\
1426	-69.58\\
1427	-50.049\\
1428	-40.283\\
1429	-58.594\\
1430	-80.566\\
1431	-59.814\\
1432	-46.387\\
1433	-41.504\\
1434	-47.607\\
1435	-32.959\\
1436	-31.738\\
1437	-43.945\\
1438	-53.711\\
1439	-59.814\\
1440	-48.828\\
1441	-61.035\\
1442	-57.373\\
1443	-34.18\\
1444	-37.842\\
1445	-65.918\\
1446	-91.553\\
1447	-90.332\\
1448	-75.684\\
1449	-73.242\\
1450	-57.373\\
1451	-53.711\\
1452	-43.945\\
1453	-61.035\\
1454	-54.932\\
1455	-36.621\\
1456	-52.49\\
1457	-61.035\\
1458	-70.801\\
1459	-76.904\\
1460	-76.904\\
1461	-102.539\\
1462	-124.512\\
1463	-140.381\\
1464	-91.553\\
1465	-52.49\\
1466	-34.18\\
1467	-25.635\\
1468	-36.621\\
1469	-32.959\\
1470	-58.594\\
1471	-53.711\\
1472	-46.387\\
1473	-62.256\\
1474	-72.021\\
1475	-81.787\\
1476	-86.67\\
1477	-122.07\\
1478	-101.318\\
1479	-81.787\\
1480	-58.594\\
1481	-58.594\\
1482	-59.814\\
1483	-75.684\\
1484	-54.932\\
1485	-56.152\\
1486	-53.711\\
1487	-30.518\\
1488	-54.932\\
1489	-70.801\\
1490	-50.049\\
1491	-53.711\\
1492	-100.098\\
1493	-75.684\\
1494	-72.021\\
1495	-102.539\\
1496	-98.877\\
1497	-62.256\\
1498	-40.283\\
1499	-42.725\\
1500	-70.801\\
};
\end{axis}

\begin{axis}[%
width=5.5cm,
height=2.1cm,
at={(7.5cm,0cm)},
scale only axis,
xmin=1000,
xmax=1500,
xlabel={\small Time, sec},
ymin=-405,
ymax=0,
ylabel={\small Load, kN},
legend style={legend cell align=left,align=left,draw=white!15!black}
]
\addplot [color=mycolor1,line width=0.5pt,solid,forget plot]
  table[row sep=crcr]{%
1000	-63.477\\
1001	-75.684\\
1002	-62.256\\
1003	-59.814\\
1004	-80.566\\
1005	-76.904\\
1006	-90.332\\
1007	-68.359\\
1008	-40.283\\
1009	-48.828\\
1010	-46.387\\
1011	-54.932\\
1012	-45.166\\
1013	-21.973\\
1014	-17.09\\
1015	-18.311\\
1016	-40.283\\
1017	-62.256\\
1018	-67.139\\
1019	-69.58\\
1020	-52.49\\
1021	-34.18\\
1022	-64.697\\
1023	-52.49\\
1024	-39.063\\
1025	-61.035\\
1026	-53.711\\
1027	-45.166\\
1028	-73.242\\
1029	-67.139\\
1030	-52.49\\
1031	-75.684\\
1032	-74.463\\
1033	-58.594\\
1034	-50.049\\
1035	-41.504\\
1036	-47.607\\
1037	-58.594\\
1038	-54.932\\
1039	-59.814\\
1040	-59.814\\
1041	-65.918\\
1042	-85.449\\
1043	-76.904\\
1044	-54.932\\
1045	-50.049\\
1046	-35.4\\
1047	-34.18\\
1048	-45.166\\
1049	-34.18\\
1050	-36.621\\
1051	-51.27\\
1052	-54.932\\
1053	-51.27\\
1054	-78.125\\
1055	-62.256\\
1056	-36.621\\
1057	-30.518\\
1058	-40.283\\
1059	-46.387\\
1060	-29.297\\
1061	-29.297\\
1062	-36.621\\
1063	-31.738\\
1064	-26.855\\
1065	-35.4\\
1066	-40.283\\
1067	-58.594\\
1068	-56.152\\
1069	-65.918\\
1070	-61.035\\
1071	-70.801\\
1072	-57.373\\
1073	-58.594\\
1074	-51.27\\
1075	-50.049\\
1076	-48.828\\
1077	-80.566\\
1078	-111.084\\
1079	-114.746\\
1080	-112.305\\
1081	-74.463\\
1082	-100.098\\
1083	-125.732\\
1084	-128.174\\
1085	-96.436\\
1086	-124.512\\
1087	-158.691\\
1088	-115.967\\
1089	-97.656\\
1090	-68.359\\
1091	-53.711\\
1092	-47.607\\
1093	-56.152\\
1094	-45.166\\
1095	-30.518\\
1096	-30.518\\
1097	-36.621\\
1098	-50.049\\
1099	-48.828\\
1100	-47.607\\
1101	-62.256\\
1102	-64.697\\
1103	-85.449\\
1104	-72.021\\
1105	-85.449\\
1106	-63.477\\
1107	-54.932\\
1108	-64.697\\
1109	-48.828\\
1110	-43.945\\
1111	-53.711\\
1112	-54.932\\
1113	-40.283\\
1114	-43.945\\
1115	-32.959\\
1116	-36.621\\
1117	-40.283\\
1118	-30.518\\
1119	-34.18\\
1120	-43.945\\
1121	-63.477\\
1122	-65.918\\
1123	-72.021\\
1124	-45.166\\
1125	-56.152\\
1126	-89.111\\
1127	-70.801\\
1128	-81.787\\
1129	-80.566\\
1130	-50.049\\
1131	-36.621\\
1132	-47.607\\
1133	-52.49\\
1134	-81.787\\
1135	-103.76\\
1136	-102.539\\
1137	-76.904\\
1138	-80.566\\
1139	-72.021\\
1140	-69.58\\
1141	-68.359\\
1142	-51.27\\
1143	-45.166\\
1144	-39.063\\
1145	-37.842\\
1146	-41.504\\
1147	-56.152\\
1148	-84.229\\
1149	-70.801\\
1150	-50.049\\
1151	-42.725\\
1152	-40.283\\
1153	-28.076\\
1154	-23.193\\
1155	-24.414\\
1156	-41.504\\
1157	-32.959\\
1158	-34.18\\
1159	-37.842\\
1160	-35.4\\
1161	-26.855\\
1162	-21.973\\
1163	-18.311\\
1164	-25.635\\
1165	-51.27\\
1166	-73.242\\
1167	-78.125\\
1168	-54.932\\
1169	-39.063\\
1170	-31.738\\
1171	-24.414\\
1172	-41.504\\
1173	-42.725\\
1174	-54.932\\
1175	-73.242\\
1176	-108.643\\
1177	-125.732\\
1178	-103.76\\
1179	-101.318\\
1180	-67.139\\
1181	-75.684\\
1182	-79.346\\
1183	-86.67\\
1184	-69.58\\
1185	-62.256\\
1186	-61.035\\
1187	-56.152\\
1188	-67.139\\
1189	-52.49\\
1190	-76.904\\
1191	-97.656\\
1192	-72.021\\
1193	-45.166\\
1194	-47.607\\
1195	-78.125\\
1196	-93.994\\
1197	-112.305\\
1198	-119.629\\
1199	-119.629\\
1200	-91.553\\
1201	-84.229\\
1202	-96.436\\
1203	-111.084\\
1204	-74.463\\
1205	-46.387\\
1206	-62.256\\
1207	-54.932\\
1208	-34.18\\
1209	-47.607\\
1210	-53.711\\
1211	-42.725\\
1212	-34.18\\
1213	-43.945\\
1214	-40.283\\
1215	-52.49\\
1216	-65.918\\
1217	-62.256\\
1218	-45.166\\
1219	-45.166\\
1220	-67.139\\
1221	-104.98\\
1222	-79.346\\
1223	-51.27\\
1224	-40.283\\
1225	-47.607\\
1226	-40.283\\
1227	-31.738\\
1228	-35.4\\
1229	-42.725\\
1230	-51.27\\
1231	-57.373\\
1232	-42.725\\
1233	-63.477\\
1234	-75.684\\
1235	-70.801\\
1236	-57.373\\
1237	-62.256\\
1238	-81.787\\
1239	-52.49\\
1240	-28.076\\
1241	-36.621\\
1242	-42.725\\
1243	-51.27\\
1244	-46.387\\
1245	-36.621\\
1246	-52.49\\
1247	-52.49\\
1248	-37.842\\
1249	-47.607\\
1250	-41.504\\
1251	-37.842\\
1252	-45.166\\
1253	-29.297\\
1254	-31.738\\
1255	-29.297\\
1256	-26.855\\
1257	-47.607\\
1258	-53.711\\
1259	-69.58\\
1260	-89.111\\
1261	-62.256\\
1262	-36.621\\
1263	-31.738\\
1264	-29.297\\
1265	-40.283\\
1266	-31.738\\
1267	-30.518\\
1268	-39.063\\
1269	-61.035\\
1270	-53.711\\
1271	-79.346\\
1272	-54.932\\
1273	-48.828\\
1274	-32.959\\
1275	-29.297\\
1276	-23.193\\
1277	-23.193\\
1278	-25.635\\
1279	-40.283\\
1280	-46.387\\
1281	-48.828\\
1282	-51.27\\
1283	-79.346\\
1284	-73.242\\
1285	-53.711\\
1286	-63.477\\
1287	-69.58\\
1288	-85.449\\
1289	-72.021\\
1290	-47.607\\
1291	-28.076\\
1292	-21.973\\
1293	-21.973\\
1294	-26.855\\
1295	-26.855\\
1296	-25.635\\
1297	-32.959\\
1298	-47.607\\
1299	-43.945\\
1300	-45.166\\
1301	-52.49\\
1302	-35.4\\
1303	-20.752\\
1304	-35.4\\
1305	-56.152\\
1306	-54.932\\
1307	-59.814\\
1308	-53.711\\
1309	-40.283\\
1310	-51.27\\
1311	-72.021\\
1312	-70.801\\
1313	-51.27\\
1314	-81.787\\
1315	-62.256\\
1316	-63.477\\
1317	-73.242\\
1318	-70.801\\
1319	-54.932\\
1320	-47.607\\
1321	-78.125\\
1322	-108.643\\
1323	-85.449\\
1324	-50.049\\
1325	-51.27\\
1326	-51.27\\
1327	-61.035\\
1328	-72.021\\
1329	-54.932\\
1330	-56.152\\
1331	-73.242\\
1332	-79.346\\
1333	-56.152\\
1334	-45.166\\
1335	-57.373\\
1336	-87.891\\
1337	-78.125\\
1338	-80.566\\
1339	-54.932\\
1340	-45.166\\
1341	-47.607\\
1342	-31.738\\
1343	-20.752\\
1344	-20.752\\
1345	-17.09\\
1346	-31.738\\
1347	-48.828\\
1348	-54.932\\
1349	-40.283\\
1350	-45.166\\
1351	-52.49\\
1352	-35.4\\
1353	-35.4\\
1354	-35.4\\
1355	-29.297\\
1356	-31.738\\
1357	-51.27\\
1358	-64.697\\
1359	-41.504\\
1360	-32.959\\
1361	-28.076\\
1362	-32.959\\
1363	-26.855\\
1364	-45.166\\
1365	-80.566\\
1366	-63.477\\
1367	-81.787\\
1368	-98.877\\
1369	-97.656\\
1370	-76.904\\
1371	-58.594\\
1372	-54.932\\
1373	-57.373\\
1374	-65.918\\
1375	-76.904\\
1376	-76.904\\
1377	-100.098\\
1378	-108.643\\
1379	-130.615\\
1380	-91.553\\
1381	-98.877\\
1382	-108.643\\
1383	-85.449\\
1384	-96.436\\
1385	-85.449\\
1386	-51.27\\
1387	-34.18\\
1388	-36.621\\
1389	-51.27\\
1390	-43.945\\
1391	-31.738\\
1392	-41.504\\
1393	-32.959\\
1394	-23.193\\
1395	-31.738\\
1396	-34.18\\
1397	-25.635\\
1398	-42.725\\
1399	-58.594\\
1400	-45.166\\
1401	-68.359\\
1402	-100.098\\
1403	-87.891\\
1404	-109.863\\
1405	-81.787\\
1406	-81.787\\
1407	-62.256\\
1408	-37.842\\
1409	-46.387\\
1410	-42.725\\
1411	-29.297\\
1412	-40.283\\
1413	-21.973\\
1414	-19.531\\
1415	-24.414\\
1416	-45.166\\
1417	-57.373\\
1418	-65.918\\
1419	-65.918\\
1420	-62.256\\
1421	-69.58\\
1422	-57.373\\
1423	-48.828\\
1424	-50.049\\
1425	-43.945\\
1426	-59.814\\
1427	-46.387\\
1428	-36.621\\
1429	-51.27\\
1430	-69.58\\
1431	-54.932\\
1432	-41.504\\
1433	-37.842\\
1434	-41.504\\
1435	-32.959\\
1436	-30.518\\
1437	-41.504\\
1438	-47.607\\
1439	-52.49\\
1440	-42.725\\
1441	-53.711\\
1442	-51.27\\
1443	-32.959\\
1444	-32.959\\
1445	-58.594\\
1446	-80.566\\
1447	-78.125\\
1448	-67.139\\
1449	-62.256\\
1450	-50.049\\
1451	-48.828\\
1452	-41.504\\
1453	-54.932\\
1454	-50.049\\
1455	-32.959\\
1456	-45.166\\
1457	-52.49\\
1458	-61.035\\
1459	-68.359\\
1460	-68.359\\
1461	-87.891\\
1462	-107.422\\
1463	-119.629\\
1464	-80.566\\
1465	-46.387\\
1466	-32.959\\
1467	-24.414\\
1468	-34.18\\
1469	-29.297\\
1470	-51.27\\
1471	-48.828\\
1472	-40.283\\
1473	-52.49\\
1474	-63.477\\
1475	-70.801\\
1476	-75.684\\
1477	-103.76\\
1478	-92.773\\
1479	-72.021\\
1480	-51.27\\
1481	-52.49\\
1482	-54.932\\
1483	-65.918\\
1484	-51.27\\
1485	-50.049\\
1486	-48.828\\
1487	-30.518\\
1488	-45.166\\
1489	-68.359\\
1490	-48.828\\
1491	-52.49\\
1492	-86.67\\
1493	-63.477\\
1494	-63.477\\
1495	-85.449\\
1496	-81.787\\
1497	-53.711\\
1498	-35.4\\
1499	-36.621\\
1500	-61.035\\
};
\end{axis}
\end{tikzpicture}%}
	\caption{Experimental data.}
\end{figure}
	\section{Structure identification}
	\par The following model structure is assumed. The output of the NARX model $\mathbfit{y}(t)$ is the measured load. The input vector is composed as
	\begin{equation}
	\mathbfit{x}(t) = \{x_i(t)\}^{d}_{i=1} = \left[\{y(t - k + 1)\}^{n_y}_{k=1} \quad \{u(t - k + n_y + 1)\}^{n_y + n_u}_{k= n_y + 1} \right] \top,
	\end{equation}
	where $n_u$ is the length of the input lag and $n_y$ is the length of the output lag in discrete time, and where $d = n_u + n_y$. In this case, the identification is performed under the following assumptions:
	\begin{itemize}[noitemsep,topsep=0.3pt,parsep=0.3pt,partopsep=0.2pt,labelindent=1cm] 
		\item only the input signal affects the output ($n_y = 0$).
		\item the input signal has a lag of length $n_u = 4$.
	\end{itemize}
  The unknown model is approximated with a sum of polynomial basis functions up to second degree ($\lambda = 2$), rendering the following structure
	\begin{equation}
	\mathbfit{y}(t) = \theta^0 + \sum_{i=1}^{d} \theta_i x_i(t) + \sum_{i=1}^{d} \sum_{j=1}^{d} \theta_{i,j} x_i(t) x_j(t) + e(t).
	\end{equation}
	The number and order of significant terms are identified within the EFOR-CMSS algorithm based on the data from 8 out of 10 datasets. Figure \ref{fig:aamdl} illustrates the relationship between the number of model terms and the selected criterion of significance, AAMDL.
	\begin{figure}[!h]
		\centering
		% This file was created by matlab2tikz.
% Minimal pgfplots version: 1.3
%
\definecolor{mycolor1}{rgb}{0.00000,0.44700,0.74100}%
\definecolor{mycolor2}{rgb}{0.85000,0.32500,0.09800}%
%
\begin{tikzpicture}

\begin{axis}[%
width=7.762432cm,
height=6cm,
at={(0cm,0cm)},
scale only axis,
xmin=1,
xmax=14,
xlabel={Number of terms},
ymin=-1.7,
ymax=-1,
ylabel={AAMDL},
legend style={legend cell align=left,align=left,draw=white!15!black}
]
\addplot [color=mycolor1,only marks,mark=o,mark options={solid},forget plot]
  table[row sep=crcr]{%
1	-1.09649186435475\\
2	-1.18645912215366\\
3	-1.22569443353968\\
4	-1.25346802192774\\
5	-1.62657496695785\\
6	-1.65781165265103\\
7	-1.66105488382319\\
8	-1.66375965757176\\
9	-1.66115241812931\\
10	-1.65778401897266\\
11	-1.65450046162883\\
12	-1.653656835792\\
13	-1.65060854102856\\
14	-1.64791867997776\\
};
\addplot [color=mycolor2,line width=5.0pt,only marks,mark=asterisk,mark options={solid},forget plot]
  table[row sep=crcr]{%
8	-1.66375965757176\\
};
\end{axis}
\end{tikzpicture}%
		\caption{AAMDL evolution with the growing number of terms.}\label{fig:aamdl}
	\end{figure}	
		
	\section{Parameter estimation}
	\begin{table}[!h]
	\centering
	\caption{Tuning parameters of the dynamical model.}
	\begin{tabular}{lllll}
Iteration & Terms & Parameters & AEERmax & AAMDL \\ 
\hline 
1 & $x_4$ & -124.7        -99.04             0        -59.03        -49.98       -961.09       -799.64             0       -460.45       -335.35 & 0 & -1.096 \\ 
2 & $x_3$ & 82.76         66.04             0         61.26          24.7        670.41        549.96             0        336.32        204.96 & 0 & -1.186 \\ 
3 & $x_2$ & -28.77        -34.71             0        -46.65         -7.71       -252.83        -203.3             0       -131.84        -48.29 & 0 & -1.226 \\ 
4 & $x_1$ & 53.78         54.59             0         36.88         24.52        431.37         362.1             0        205.64         138.7 & 0 & -1.253 \\ 
5 & $x_4 \times x_4$ & -25        -19.95             0        -11.47        -10.22       -192.41       -159.57             0        -91.27        -67.87 & 0 & -1.627 \\ 
6 & $x_3 \times x_4$ & 16.89         13.52             0         11.12          5.54        137.99        113.52             0         67.92         43.63 & 0 & -1.658 \\ 
7 & $x_2 \times x_4$ & -5.06        -5.95            0        -7.38        -1.28        -49.8        -40.6            0       -25.14       -10.39 & 0 & -1.661 \\ 
8 & $x_1 \times x_4$ & 9.75         9.65            0         6.16         4.25        81.62        68.36            0        38.21        26.28 & 0 & -1.664 \\ 
\hline 
\end{tabular}
	\end{table}
\end{document}