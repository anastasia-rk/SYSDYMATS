\documentclass[a4paper,11pt,twoside]{article}
%%%%%%%%%%%%%%%%%%%%%%%%%%%%%%%%%%%%%%%%%%%%%%%%%%%%%%%%%%%%%%%%%%%%%%%%%%%
% Packages
\usepackage[final]{pdfpages}
\usepackage{verbatim}
\usepackage{inputenc}
\usepackage{graphicx} 
\usepackage{amsmath,amssymb,mathrsfs,amsfonts}
\usepackage{mathtools}
\usepackage{amsthm}
\usepackage{mathtools}
\usepackage{calrsfs}
\usepackage{graphicx}
\usepackage{subfig}
\usepackage{eucal}    
\usepackage{amssymb}  
\usepackage{pifont}
\usepackage{color} 
\usepackage{cancel}
\usepackage[toc,page]{appendix}
%\usepackage[usenames,dvipsnames,table]{xcolor}
\usepackage{pgfplots}
\pgfplotsset{every axis/.append style={line width=0.5pt},label style={font=\scriptsize},tick label style={font=\scriptsize},x tick label style={/pgf/number format/.cd,fixed,precision=3, set thousands separator={}},z tick label style={/pgf/number format/.cd,fixed,precision=3, set thousands separator={}}}
\usetikzlibrary{shapes,shadows,arrows,backgrounds,patterns,positioning,automata,calc,decorations.markings,decorations.pathreplacing,bayesnet,arrows.meta}
\usepackage{varwidth}
\usepackage{lscape}
\usepackage{array} 
\usepackage[colorlinks=false,pdfborder={0 0 0}]{hyperref}
\usepackage{tabularx}
\usepackage{textcomp}
\usepackage{multicol} 
\usepackage{booktabs}
\usepackage{multirow}
\usepackage[font=small,labelfont=bf]{caption}                                                           
\usepackage{textcase}
\usepackage{bbm} 
\usepackage{fancyhdr}
\usepackage{enumitem}
\usepackage{soul}
%\usepackage[british]{babel}
\usepackage{wrapfig}
%\usepackage{glossaries}
%%%%%%%%%%%%%%%%%%%%%%%%%%%%%%%%%%%%%%%%%%%%%%%%%%%%%%%%%%%%%%%%%%%%%%%%%%%
\setlength{\parindent}{2em}
\setlength{\parskip}{0.5em}
\renewcommand{\baselinestretch}{1.2}
\usepackage[left=2cm, right=2cm, top=2.5cm, bottom=3cm, headheight=13.6pt]{geometry}
\allowdisplaybreaks 
% Bibliography
%\usepackage[backend=bibtex,style=ieee,sorting=none]{biblatex} 
%\bibliography{Bibliography-thesis}
%\renewcommand*{\bibfont}{\scriptsize}
\makeatletter
\newcommand*{\rom}[1]{\expandafter\@slowromancap\romannumeral #1@}
\newcommand{\ie}{\textit{i.e.} }
\newcommand{\eg}{\textit{e.g.} }
\makeatother
\newcommand\id{\ensuremath{\mathbbm{1}}} 
\DeclareMathOperator{\E}{\mathbb{E}}
\DeclareMathOperator{\eye}{\mathbb{I}}
\DeclareMathOperator{\zeros}{\mathbb{O}}
\DeclareMathOperator{\tr}{\textrm{tr}}
\DeclareMathOperator{\vvec}{\textrm{vec}}
\DeclareMathOperator{\ik}{\mathrm{k}}
\DeclareMathOperator{\ip}{\mathrm{p}}
\DeclareMathOperator{\inn}{\mathrm{n}}
\DeclareMathOperator{\im}{\mathrm{m}}
\DeclareMathOperator{\td}{\mathrm{t}}
\DeclareMathOperator{\kd}{\mathrm{k}}
\DeclareMathOperator{\T}{\mathrm{T}}
\DeclareMathOperator{\K}{\mathrm{K}}
\DeclareSymbolFontAlphabet{\mathcal} {symbols}
\DeclareSymbolFont{symbols}{OMS}{cm}{m}{n}
\DeclareMathAlphabet{\mathbfit}{OML}{cmm}{b}{it}

% Number equations
%\numberwithin{equation}{section}

%%%%%%%%%%%%%%%%%%%%%%%%%%%%%%%%%%%%%%%%%%%%%%%%%%%%%%%%%%%%%%%%%%%%%%%%%%%
%Theorems
\newtheoremstyle{mytheoremstyle} % name
{.5em}                    % Space above
{.8em}                    % Space below
{\itshape}                % Body font
{1em}                           % Indent amount
{\bfseries}                   % Theorem head font
{:}                          % Punctuation after theorem head
{.5em}                       % Space after theorem head
{}  % Theorem head spec (can be left empty, meaning ‘normal’)

\theoremstyle{mytheoremstyle}
\newtheorem{theorem}{Theorem}[section]
\newtheorem{remark}{Remark}[section]
\newtheorem{assumption}{Assumption}[section]
\newtheorem{lemma}{Lemma}[section]
\newtheorem{condition}{Condition}[section]
\newtheorem{definition}{Definition}[section]
\newtheorem{property}{Property}[section]
\newtheorem{corollary}{Corollary}[section]
\renewcommand\qedsymbol{$\blacksquare$}
%%%%%%%%%%%%%%%%%%%%%%%%%%%%%%%%%%%%%%%%%%%%%%%%%%%%%%%%%%%%%%%%%%%%%%%%%%%
% Nomenclature
\usepackage[intoc]{nomencl}
\makenomenclature

%\usepackage[ruled,chapter]{algorithm}
%\usepackage{float}

%\usepackage{algorithmic}
%\algsetup{linenosize=\scriptsize}
%\usepackage{etoolbox}
%\AtBeginEnvironment{algorithmic}{\scriptsize}
%\renewcommand{\thealgorithm}{\thechapter.\arabic{algorithm}} 
%\usepackage{chngcntr}
%\counterwithin{algorithm}{section}

% correct bad hyphenation here
\hyphenation{op-tical net-works semi-conduc-tor}
%%%%%%%%%%%%%%%%%%%%%%%%%%%%%%%%%%%%%%%%%%%%%%%%%%%%%%%%%%%%%%%%%%%%%%%%%%%%%
% Captions
%\newcommand{\xLanguage}{british}          % <-- Added this command
%
%\usepackage[\xLanguage]{babel}             % <-- Implemented here
%
%\expandafter\addto\csname captions\xLanguage\endcsname{% <-- and here
%	\renewcommand{\tableshortname}{Table}%
%	\renewcommand{\figureshortname}{Figure}%
%}
%
%\captionsetup[table]{format=plain,indention=1.15cm,justification=justified}
%\captionsetup[figure]{format=plain,indention=1.25cm,justification=justified}

%%%%%%%%%%%%%%%%%%%%%%%%%%%%%%%%%%%%%%%%%%%%%%%%%%%%%%%%%%%%%%%%%%%%%%%%%%%%%
\usepackage[explicit]{titlesec}
\usepackage{titletoc}
%\titleformat{\section}[block]{\normalfont\Large\rm\filright\bfseries}{\thesection}{1em}{#1}
%\titlecontents{chapter}[1.5em]{}{\scshape\contentslabel{2.3em}}{}{\titlerule*[1pc]{}\contentspage}
%\definecolor{gray75}{gray}{0.5}
%\newcommand{\hsp}{\hspace{20pt}}
%\titleformat{\chapter}[hang]{\huge\scshape\filright\bfseries}{\color{gray75}\thechapter}{20pt}{\begin{tabular}[t]{@{\color{gray75}\vrule width 2pt\hsp}p{0.85\textwidth}}\raggedright#1\end{tabular}}
%\titleformat{name=\chapter,numberless}[display]{}{}{0pt}{\normalfont\huge\bfseries #1} % format for numberless chapters
%\titleformat{\subsection}[block]{\normalfont\large\rm\filright\bfseries}{\thesubsection}{1em}{#1}
%\renewcommand{\sectionmark}[1]{\markright{\thesection ~ \ #1}}
%%\renewcommand{\chaptermark}[1]{\markboth{\chaptername\ \thechapter ~ \ #1}{}} 
%\pagestyle{fancy}
%%\fancyhf{}
%\fancyhead[RO,LE]{\thepage}
%\fancyhead[RE]{\itshape \nouppercase \rightmark}      % chaptertitle left
%\fancyhead[LO]{\itshape \nouppercase \leftmark}       % sectiontitle right
%\renewcommand{\headrulewidth}{0.5pt} 				   % no rule
%\cfoot{}
\interfootnotelinepenalty=10000

\title{Structure and parameter identification of au}


\begin{document}
	\maketitle
%	\section{Experimental setup}
%	
\section{Experimental data}
\begin{figure}[!h]
		\centering
		% This file was created by matlab2tikz.
%
\definecolor{mycolor1}{rgb}{0.00000,0.44700,0.74100}%
%
\begin{tikzpicture}

\begin{axis}[%
width=11.411cm,
height=5cm,
at={(0cm,0cm)},
scale only axis,
xmin=1,
xmax=1500,
xlabel style={font=\color{white!15!black}},
xlabel={Sample index},
ymin=-7,
ymax=-4.999,
ylabel style={font=\color{white!15!black}},
ylabel={$\Delta x$, mm},
axis background/.style={fill=white},
legend style={legend cell align=left, align=left, draw=white!15!black}
]
\addplot [color=mycolor1]
  table[row sep=crcr]{%
1	-5.951\\
3	-5.933\\
5	-5.933\\
7	-5.914\\
9	-5.933\\
11	-5.933\\
13	-5.914\\
15	-5.914\\
17	-5.933\\
19	-5.933\\
21	-5.933\\
23	-5.933\\
25	-5.914\\
27	-5.914\\
29	-5.933\\
31	-5.933\\
33	-5.933\\
35	-5.914\\
37	-5.933\\
39	-5.933\\
41	-5.933\\
43	-5.933\\
45	-5.933\\
47	-5.933\\
49	-5.933\\
51	-5.933\\
53	-5.914\\
55	-5.914\\
57	-5.914\\
59	-5.933\\
61	-5.933\\
63	-5.933\\
65	-5.933\\
67	-5.914\\
69	-5.914\\
71	-5.914\\
73	-5.933\\
75	-5.914\\
77	-5.914\\
79	-5.933\\
81	-5.933\\
83	-5.951\\
85	-5.933\\
87	-5.933\\
89	-5.914\\
91	-5.933\\
93	-5.933\\
95	-5.933\\
97	-5.933\\
99	-5.914\\
101	-5.914\\
103	-5.933\\
105	-5.933\\
107	-5.933\\
109	-5.933\\
111	-5.914\\
113	-5.896\\
115	-5.878\\
117	-5.841\\
119	-5.859\\
121	-6.061\\
123	-6.226\\
125	-6.116\\
127	-6.171\\
129	-6.281\\
131	-6.061\\
133	-5.878\\
135	-6.226\\
137	-6.061\\
139	-6.226\\
141	-5.859\\
143	-5.933\\
145	-6.354\\
147	-6.079\\
149	-5.896\\
151	-5.64\\
153	-5.64\\
155	-5.365\\
157	-5.621\\
159	-5.878\\
161	-5.786\\
163	-5.548\\
165	-5.988\\
167	-6.207\\
169	-5.933\\
171	-6.097\\
173	-5.713\\
175	-5.511\\
177	-5.548\\
179	-5.42\\
181	-5.511\\
183	-5.2\\
185	-4.999\\
187	-5.219\\
189	-5.731\\
191	-5.566\\
193	-5.585\\
195	-5.621\\
197	-5.823\\
199	-6.079\\
201	-6.006\\
203	-5.475\\
205	-5.64\\
207	-5.786\\
209	-5.951\\
211	-5.841\\
213	-6.171\\
215	-6.207\\
217	-6.006\\
219	-5.731\\
221	-5.64\\
223	-5.969\\
225	-5.768\\
227	-5.841\\
229	-5.933\\
231	-6.299\\
233	-5.859\\
235	-5.878\\
237	-5.75\\
239	-5.493\\
241	-5.768\\
243	-5.951\\
245	-5.896\\
247	-5.933\\
249	-6.207\\
251	-5.914\\
253	-6.042\\
255	-5.64\\
257	-5.896\\
259	-5.823\\
261	-5.658\\
263	-5.585\\
265	-6.097\\
267	-5.768\\
269	-5.658\\
271	-5.695\\
273	-5.676\\
275	-5.914\\
277	-6.299\\
279	-6.299\\
281	-5.859\\
283	-5.804\\
285	-5.878\\
287	-6.024\\
289	-6.207\\
291	-5.969\\
293	-6.024\\
295	-6.061\\
297	-6.244\\
299	-6.097\\
301	-5.878\\
303	-5.804\\
305	-5.475\\
307	-5.328\\
309	-5.621\\
311	-5.621\\
313	-5.64\\
315	-5.804\\
317	-5.988\\
319	-6.335\\
321	-6.647\\
323	-6.354\\
325	-5.933\\
327	-6.042\\
329	-5.951\\
331	-5.896\\
333	-6.152\\
335	-6.793\\
337	-6.573\\
339	-6.519\\
341	-6.061\\
343	-5.75\\
345	-5.951\\
347	-5.64\\
349	-5.2\\
351	-5.585\\
353	-6.024\\
355	-6.464\\
357	-6.445\\
359	-6.299\\
361	-6.207\\
363	-6.354\\
365	-6.097\\
367	-6.171\\
369	-5.914\\
371	-5.859\\
373	-5.969\\
375	-6.537\\
377	-6.409\\
379	-6.061\\
381	-5.42\\
383	-5.383\\
385	-5.676\\
387	-6.024\\
389	-6.244\\
391	-6.72\\
393	-6.555\\
395	-6.152\\
397	-6.628\\
399	-6.39\\
401	-6.042\\
403	-5.804\\
405	-5.804\\
407	-5.493\\
409	-5.53\\
411	-5.878\\
413	-5.75\\
415	-6.152\\
417	-6.281\\
419	-6.354\\
421	-6.079\\
423	-5.969\\
425	-5.896\\
427	-5.969\\
429	-5.841\\
431	-5.951\\
433	-5.548\\
435	-6.006\\
437	-6.573\\
439	-6.592\\
441	-6.647\\
443	-6.281\\
445	-6.171\\
447	-5.878\\
449	-5.53\\
451	-5.53\\
453	-5.878\\
455	-6.226\\
457	-6.024\\
459	-6.317\\
461	-6.354\\
463	-6.024\\
465	-5.878\\
467	-5.768\\
469	-6.006\\
471	-5.933\\
473	-5.969\\
475	-6.024\\
477	-6.189\\
479	-6.189\\
481	-6.317\\
483	-6.006\\
485	-5.804\\
487	-5.896\\
489	-6.042\\
491	-6.042\\
493	-5.988\\
495	-5.75\\
497	-6.061\\
499	-5.914\\
501	-5.695\\
503	-5.585\\
505	-5.621\\
507	-5.933\\
509	-6.226\\
511	-6.354\\
513	-6.464\\
515	-6.573\\
517	-6.134\\
519	-5.841\\
521	-6.079\\
523	-5.713\\
525	-5.53\\
527	-5.548\\
529	-5.731\\
531	-5.933\\
533	-6.226\\
535	-6.061\\
537	-6.024\\
539	-6.006\\
541	-5.933\\
543	-6.116\\
545	-6.427\\
547	-6.244\\
549	-6.281\\
551	-5.969\\
553	-6.39\\
555	-5.914\\
557	-6.116\\
559	-6.207\\
561	-5.75\\
563	-5.713\\
565	-5.585\\
567	-5.658\\
569	-5.402\\
571	-5.219\\
573	-5.273\\
575	-5.566\\
577	-5.878\\
579	-6.097\\
581	-5.713\\
583	-5.493\\
585	-5.566\\
587	-5.237\\
589	-5.164\\
591	-5.493\\
593	-5.603\\
595	-5.475\\
597	-5.823\\
599	-6.079\\
601	-6.152\\
603	-6.097\\
605	-5.969\\
607	-5.786\\
609	-5.603\\
611	-5.786\\
613	-5.548\\
615	-5.951\\
617	-6.079\\
619	-6.079\\
621	-6.189\\
623	-6.042\\
625	-5.841\\
627	-5.804\\
629	-6.189\\
631	-6.207\\
633	-6.207\\
635	-5.878\\
637	-6.024\\
639	-6.244\\
641	-6.006\\
643	-5.878\\
645	-5.621\\
647	-5.914\\
649	-5.896\\
651	-5.493\\
653	-5.585\\
655	-5.878\\
657	-5.804\\
659	-5.951\\
661	-5.603\\
663	-5.53\\
665	-5.713\\
667	-5.823\\
669	-6.171\\
671	-6.262\\
673	-6.024\\
675	-6.116\\
677	-6.006\\
679	-6.006\\
681	-5.988\\
683	-6.189\\
685	-6.079\\
687	-6.189\\
689	-6.335\\
691	-6.372\\
693	-6.189\\
695	-6.39\\
697	-6.702\\
699	-6.262\\
701	-6.171\\
703	-6.134\\
705	-6.5\\
707	-6.226\\
709	-6.171\\
711	-6.024\\
713	-6.207\\
715	-6.317\\
717	-6.006\\
719	-5.878\\
721	-5.75\\
723	-5.75\\
725	-6.079\\
727	-6.281\\
729	-6.702\\
731	-6.39\\
733	-6.116\\
735	-5.786\\
737	-5.988\\
739	-6.189\\
741	-5.896\\
743	-6.042\\
745	-6.134\\
747	-6.189\\
749	-5.841\\
751	-5.969\\
753	-5.603\\
755	-5.951\\
757	-5.695\\
759	-5.768\\
761	-5.859\\
763	-6.354\\
765	-6.152\\
767	-6.042\\
769	-6.171\\
771	-6.079\\
773	-5.603\\
775	-5.713\\
777	-5.878\\
779	-5.988\\
781	-6.152\\
783	-5.933\\
785	-6.262\\
787	-6.207\\
789	-6.207\\
791	-6.207\\
793	-6.116\\
795	-5.75\\
797	-6.061\\
799	-6.006\\
801	-5.823\\
803	-5.896\\
805	-5.878\\
807	-5.786\\
809	-5.713\\
811	-5.64\\
813	-5.585\\
815	-5.841\\
817	-5.75\\
819	-6.134\\
821	-5.988\\
823	-6.061\\
825	-5.53\\
827	-5.676\\
829	-6.134\\
831	-6.061\\
833	-6.024\\
835	-6.207\\
837	-5.988\\
839	-5.786\\
841	-5.75\\
843	-5.768\\
845	-5.75\\
847	-6.244\\
849	-5.878\\
851	-5.457\\
853	-5.42\\
855	-5.493\\
857	-5.566\\
859	-6.024\\
861	-6.372\\
863	-6.244\\
865	-6.409\\
867	-6.042\\
869	-5.731\\
871	-5.713\\
873	-5.621\\
875	-5.768\\
877	-5.896\\
879	-5.988\\
881	-6.189\\
883	-6.061\\
885	-6.061\\
887	-6.226\\
889	-6.281\\
891	-6.39\\
893	-6.061\\
895	-5.951\\
897	-5.658\\
899	-5.621\\
901	-5.328\\
903	-5.731\\
905	-5.621\\
907	-5.511\\
909	-5.53\\
911	-5.786\\
913	-5.566\\
915	-6.134\\
917	-6.097\\
919	-6.372\\
921	-6.39\\
923	-6.226\\
925	-6.262\\
927	-5.768\\
929	-5.658\\
931	-5.676\\
933	-5.969\\
935	-5.676\\
937	-5.731\\
939	-5.383\\
941	-5.859\\
943	-5.676\\
945	-5.823\\
947	-5.804\\
949	-6.189\\
951	-6.244\\
953	-6.244\\
955	-6.061\\
957	-6.281\\
959	-6.262\\
961	-6.006\\
963	-5.988\\
965	-5.786\\
967	-5.64\\
969	-5.676\\
971	-5.823\\
973	-5.695\\
975	-5.896\\
977	-5.676\\
979	-5.695\\
981	-5.841\\
983	-5.804\\
985	-6.317\\
987	-5.933\\
989	-6.152\\
991	-6.189\\
993	-6.299\\
995	-6.537\\
997	-6.39\\
999	-6.152\\
1001	-6.244\\
1003	-6.134\\
1005	-6.281\\
1007	-6.244\\
1009	-5.914\\
1011	-5.969\\
1013	-5.585\\
1015	-5.145\\
1017	-5.896\\
1019	-6.061\\
1021	-5.731\\
1023	-5.914\\
1025	-5.933\\
1027	-5.878\\
1029	-6.079\\
1031	-6.079\\
1033	-6.061\\
1035	-5.878\\
1037	-5.969\\
1039	-5.969\\
1041	-6.006\\
1043	-6.207\\
1045	-5.969\\
1047	-5.658\\
1049	-5.676\\
1051	-5.804\\
1053	-5.859\\
1055	-6.042\\
1057	-5.658\\
1059	-5.786\\
1061	-5.493\\
1063	-5.566\\
1065	-5.548\\
1067	-5.841\\
1069	-5.969\\
1071	-6.024\\
1073	-5.951\\
1075	-5.878\\
1077	-6.116\\
1079	-6.519\\
1081	-6.335\\
1083	-6.61\\
1085	-6.555\\
1087	-6.866\\
1089	-6.665\\
1091	-6.244\\
1093	-6.152\\
1095	-5.786\\
1097	-5.695\\
1099	-5.896\\
1101	-6.006\\
1103	-6.226\\
1105	-6.281\\
1107	-6.097\\
1109	-6.042\\
1111	-6.006\\
1113	-5.914\\
1115	-5.804\\
1117	-5.768\\
1119	-5.676\\
1121	-5.969\\
1123	-6.116\\
1125	-5.988\\
1127	-6.226\\
1129	-6.281\\
1131	-5.859\\
1133	-5.969\\
1135	-6.409\\
1137	-6.354\\
1139	-6.281\\
1141	-6.262\\
1143	-5.969\\
1145	-5.841\\
1147	-6.006\\
1149	-6.207\\
1151	-5.951\\
1153	-5.695\\
1155	-5.457\\
1157	-5.64\\
1159	-5.658\\
1161	-5.53\\
1163	-5.219\\
1165	-5.676\\
1167	-6.134\\
1169	-5.804\\
1171	-5.438\\
1173	-5.713\\
1175	-6.024\\
1177	-6.61\\
1179	-6.555\\
1181	-6.335\\
1183	-6.427\\
1185	-6.244\\
1187	-6.171\\
1189	-6.116\\
1191	-6.427\\
1193	-6.079\\
1195	-6.226\\
1197	-6.537\\
1199	-6.683\\
1201	-6.5\\
1203	-6.61\\
1205	-6.152\\
1207	-6.171\\
1209	-5.969\\
1211	-5.969\\
1213	-5.896\\
1215	-5.914\\
1217	-6.134\\
1219	-5.969\\
1221	-6.427\\
1223	-6.152\\
1225	-5.951\\
1227	-5.804\\
1229	-5.859\\
1231	-6.006\\
1233	-6.042\\
1235	-6.207\\
1237	-6.134\\
1239	-6.152\\
1241	-5.731\\
1243	-5.896\\
1245	-5.786\\
1247	-5.933\\
1249	-5.878\\
1251	-5.804\\
1253	-5.676\\
1255	-5.548\\
1257	-5.768\\
1259	-6.024\\
1261	-6.152\\
1263	-5.695\\
1265	-5.75\\
1267	-5.585\\
1269	-5.896\\
1271	-6.134\\
1273	-5.951\\
1275	-5.566\\
1277	-5.328\\
1279	-5.603\\
1281	-5.768\\
1283	-6.097\\
1285	-6.024\\
1287	-6.116\\
1289	-6.226\\
1291	-5.695\\
1293	-5.383\\
1295	-5.42\\
1297	-5.457\\
1299	-5.731\\
1301	-5.786\\
1303	-5.402\\
1305	-5.804\\
1307	-5.914\\
1309	-5.75\\
1311	-6.024\\
1313	-5.988\\
1315	-6.134\\
1317	-6.134\\
1319	-6.061\\
1321	-6.171\\
1323	-6.39\\
1325	-6.006\\
1327	-6.079\\
1329	-6.079\\
1331	-6.171\\
1333	-6.116\\
1335	-6.024\\
1337	-6.299\\
1339	-6.097\\
1341	-5.969\\
1343	-5.511\\
1345	-5.2\\
1347	-5.676\\
1349	-5.768\\
1351	-5.841\\
1353	-5.621\\
1355	-5.511\\
1357	-5.768\\
1359	-5.768\\
1361	-5.548\\
1363	-5.457\\
1365	-6.079\\
1367	-6.189\\
1369	-6.39\\
1371	-6.134\\
1373	-6.061\\
1375	-6.207\\
1377	-6.39\\
1379	-6.628\\
1381	-6.519\\
1383	-6.482\\
1385	-6.482\\
1387	-5.933\\
1389	-5.969\\
1391	-5.75\\
1393	-5.713\\
1395	-5.566\\
1397	-5.53\\
1399	-5.896\\
1401	-6.024\\
1403	-6.354\\
1405	-6.39\\
1407	-6.226\\
1409	-5.969\\
1411	-5.731\\
1413	-5.676\\
1415	-5.365\\
1417	-5.859\\
1419	-6.042\\
1421	-6.097\\
1423	-5.951\\
1425	-5.841\\
1427	-5.914\\
1429	-5.878\\
1431	-6.006\\
1433	-5.804\\
1435	-5.713\\
1437	-5.676\\
1439	-5.878\\
1441	-5.878\\
1443	-5.695\\
1445	-5.896\\
1447	-6.207\\
1449	-6.116\\
1451	-5.951\\
1453	-5.951\\
1455	-5.768\\
1457	-5.914\\
1459	-6.079\\
1461	-6.262\\
1463	-6.592\\
1465	-6.097\\
1467	-5.603\\
1469	-5.621\\
1471	-5.841\\
1473	-5.896\\
1475	-6.079\\
1477	-6.372\\
1479	-6.317\\
1481	-6.061\\
1483	-6.152\\
1485	-6.006\\
1487	-5.75\\
1489	-6.061\\
1491	-5.969\\
1493	-6.189\\
1495	-6.299\\
1497	-6.134\\
1499	-5.823\\
1501	-6.281\\
};
\addlegendentry{data1}

\end{axis}
\end{tikzpicture}%
		\caption{The input for the set C.}
	\end{figure}	
\begin{figure}[!h]
	\centering
	% This file was created by matlab2tikz.
% Minimal pgfplots version: 1.3
%
\definecolor{mycolor1}{rgb}{0.00000,0.44700,0.74100}%
%
\begin{tikzpicture}

\begin{axis}[%
width=5.090835cm,
height=2.240107cm,
at={(0cm,0cm)},
scale only axis,
xmin=1000,
xmax=1500,
xlabel={Sample index},
ymin=-200,
ymax=0,
ylabel={C9 load, kN},
legend style={legend cell align=left,align=left,draw=white!15!black}
]
\addplot [color=mycolor1,solid,forget plot]
  table[row sep=crcr]{%
1000	-68.359\\
1001	-85.449\\
1002	-69.58\\
1003	-65.918\\
1004	-90.332\\
1005	-86.67\\
1006	-103.76\\
1007	-79.346\\
1008	-45.166\\
1009	-57.373\\
1010	-51.27\\
1011	-63.477\\
1012	-48.828\\
1013	-24.414\\
1014	-17.09\\
1015	-17.09\\
1016	-43.945\\
1017	-73.242\\
1018	-76.904\\
1019	-79.346\\
1020	-58.594\\
1021	-36.621\\
1022	-76.904\\
1023	-53.711\\
1024	-43.945\\
1025	-69.58\\
1026	-61.035\\
1027	-48.828\\
1028	-84.229\\
1029	-79.346\\
1030	-59.814\\
1031	-87.891\\
1032	-85.449\\
1033	-67.139\\
1034	-54.932\\
1035	-46.387\\
1036	-56.152\\
1037	-64.697\\
1038	-64.697\\
1039	-67.139\\
1040	-68.359\\
1041	-73.242\\
1042	-100.098\\
1043	-87.891\\
1044	-61.035\\
1045	-56.152\\
1046	-34.18\\
1047	-34.18\\
1048	-48.828\\
1049	-37.842\\
1050	-39.063\\
1051	-56.152\\
1052	-62.256\\
1053	-58.594\\
1054	-90.332\\
1055	-64.697\\
1056	-41.504\\
1057	-34.18\\
1058	-45.166\\
1059	-51.27\\
1060	-28.076\\
1061	-25.635\\
1062	-41.504\\
1063	-34.18\\
1064	-29.297\\
1065	-40.283\\
1066	-43.945\\
1067	-68.359\\
1068	-63.477\\
1069	-79.346\\
1070	-69.58\\
1071	-79.346\\
1072	-63.477\\
1073	-62.256\\
1074	-54.932\\
1075	-53.711\\
1076	-53.711\\
1077	-91.553\\
1078	-126.953\\
1079	-131.836\\
1080	-129.395\\
1081	-83.008\\
1082	-114.746\\
1083	-141.602\\
1084	-147.705\\
1085	-111.084\\
1086	-146.484\\
1087	-185.547\\
1088	-131.836\\
1089	-112.305\\
1090	-75.684\\
1091	-61.035\\
1092	-52.49\\
1093	-64.697\\
1094	-46.387\\
1095	-31.738\\
1096	-31.738\\
1097	-40.283\\
1098	-57.373\\
1099	-52.49\\
1100	-53.711\\
1101	-72.021\\
1102	-72.021\\
1103	-97.656\\
1104	-79.346\\
1105	-101.318\\
1106	-72.021\\
1107	-64.697\\
1108	-73.242\\
1109	-54.932\\
1110	-51.27\\
1111	-59.814\\
1112	-62.256\\
1113	-43.945\\
1114	-47.607\\
1115	-34.18\\
1116	-40.283\\
1117	-42.725\\
1118	-30.518\\
1119	-39.063\\
1120	-47.607\\
1121	-72.021\\
1122	-74.463\\
1123	-81.787\\
1124	-48.828\\
1125	-70.801\\
1126	-101.318\\
1127	-84.229\\
1128	-93.994\\
1129	-91.553\\
1130	-54.932\\
1131	-40.283\\
1132	-56.152\\
1133	-61.035\\
1134	-97.656\\
1135	-119.629\\
1136	-117.188\\
1137	-89.111\\
1138	-92.773\\
1139	-81.787\\
1140	-80.566\\
1141	-79.346\\
1142	-54.932\\
1143	-48.828\\
1144	-45.166\\
1145	-40.283\\
1146	-45.166\\
1147	-64.697\\
1148	-96.436\\
1149	-74.463\\
1150	-52.49\\
1151	-45.166\\
1152	-46.387\\
1153	-29.297\\
1154	-23.193\\
1155	-26.855\\
1156	-46.387\\
1157	-35.4\\
1158	-41.504\\
1159	-40.283\\
1160	-39.063\\
1161	-28.076\\
1162	-21.973\\
1163	-17.09\\
1164	-29.297\\
1165	-58.594\\
1166	-80.566\\
1167	-91.553\\
1168	-59.814\\
1169	-43.945\\
1170	-32.959\\
1171	-23.193\\
1172	-47.607\\
1173	-46.387\\
1174	-67.139\\
1175	-81.787\\
1176	-131.836\\
1177	-147.705\\
1178	-120.85\\
1179	-117.188\\
1180	-78.125\\
1181	-81.787\\
1182	-91.553\\
1183	-98.877\\
1184	-73.242\\
1185	-68.359\\
1186	-69.58\\
1187	-62.256\\
1188	-75.684\\
1189	-53.711\\
1190	-93.994\\
1191	-108.643\\
1192	-80.566\\
1193	-51.27\\
1194	-54.932\\
1195	-92.773\\
1196	-109.863\\
1197	-134.277\\
1198	-140.381\\
1199	-140.381\\
1200	-106.201\\
1201	-96.436\\
1202	-111.084\\
1203	-129.395\\
1204	-78.125\\
1205	-50.049\\
1206	-73.242\\
1207	-57.373\\
1208	-39.063\\
1209	-56.152\\
1210	-61.035\\
1211	-46.387\\
1212	-37.842\\
1213	-50.049\\
1214	-39.063\\
1215	-53.711\\
1216	-76.904\\
1217	-72.021\\
1218	-51.27\\
1219	-51.27\\
1220	-78.125\\
1221	-123.291\\
1222	-86.67\\
1223	-56.152\\
1224	-43.945\\
1225	-48.828\\
1226	-43.945\\
1227	-34.18\\
1228	-37.842\\
1229	-47.607\\
1230	-59.814\\
1231	-64.697\\
1232	-46.387\\
1233	-74.463\\
1234	-85.449\\
1235	-83.008\\
1236	-65.918\\
1237	-73.242\\
1238	-96.436\\
1239	-61.035\\
1240	-29.297\\
1241	-42.725\\
1242	-43.945\\
1243	-58.594\\
1244	-50.049\\
1245	-40.283\\
1246	-58.594\\
1247	-54.932\\
1248	-39.063\\
1249	-48.828\\
1250	-43.945\\
1251	-39.063\\
1252	-50.049\\
1253	-29.297\\
1254	-36.621\\
1255	-26.855\\
1256	-31.738\\
1257	-53.711\\
1258	-61.035\\
1259	-76.904\\
1260	-101.318\\
1261	-67.139\\
1262	-42.725\\
1263	-32.959\\
1264	-31.738\\
1265	-47.607\\
1266	-30.518\\
1267	-36.621\\
1268	-43.945\\
1269	-65.918\\
1270	-62.256\\
1271	-86.67\\
1272	-59.814\\
1273	-56.152\\
1274	-31.738\\
1275	-32.959\\
1276	-20.752\\
1277	-24.414\\
1278	-26.855\\
1279	-47.607\\
1280	-47.607\\
1281	-56.152\\
1282	-59.814\\
1283	-93.994\\
1284	-79.346\\
1285	-62.256\\
1286	-73.242\\
1287	-79.346\\
1288	-101.318\\
1289	-80.566\\
1290	-52.49\\
1291	-30.518\\
1292	-21.973\\
1293	-23.193\\
1294	-29.297\\
1295	-29.297\\
1296	-26.855\\
1297	-36.621\\
1298	-53.711\\
1299	-47.607\\
1300	-51.27\\
1301	-57.373\\
1302	-37.842\\
1303	-20.752\\
1304	-42.725\\
1305	-63.477\\
1306	-61.035\\
1307	-69.58\\
1308	-58.594\\
1309	-45.166\\
1310	-59.814\\
1311	-84.229\\
1312	-85.449\\
1313	-59.814\\
1314	-102.539\\
1315	-76.904\\
1316	-74.463\\
1317	-84.229\\
1318	-83.008\\
1319	-63.477\\
1320	-54.932\\
1321	-90.332\\
1322	-124.512\\
1323	-95.215\\
1324	-57.373\\
1325	-53.711\\
1326	-54.932\\
1327	-72.021\\
1328	-83.008\\
1329	-59.814\\
1330	-64.697\\
1331	-83.008\\
1332	-90.332\\
1333	-61.035\\
1334	-50.049\\
1335	-63.477\\
1336	-104.98\\
1337	-92.773\\
1338	-95.215\\
1339	-58.594\\
1340	-53.711\\
1341	-51.27\\
1342	-34.18\\
1343	-23.193\\
1344	-20.752\\
1345	-17.09\\
1346	-40.283\\
1347	-54.932\\
1348	-62.256\\
1349	-47.607\\
1350	-52.49\\
1351	-57.373\\
1352	-37.842\\
1353	-39.063\\
1354	-39.063\\
1355	-28.076\\
1356	-39.063\\
1357	-57.373\\
1358	-73.242\\
1359	-48.828\\
1360	-35.4\\
1361	-31.738\\
1362	-36.621\\
1363	-26.855\\
1364	-54.932\\
1365	-93.994\\
1366	-72.021\\
1367	-97.656\\
1368	-113.525\\
1369	-114.746\\
1370	-85.449\\
1371	-63.477\\
1372	-63.477\\
1373	-63.477\\
1374	-75.684\\
1375	-89.111\\
1376	-86.67\\
1377	-115.967\\
1378	-125.732\\
1379	-150.146\\
1380	-96.436\\
1381	-109.863\\
1382	-122.07\\
1383	-92.773\\
1384	-107.422\\
1385	-92.773\\
1386	-54.932\\
1387	-39.063\\
1388	-40.283\\
1389	-57.373\\
1390	-42.725\\
1391	-32.959\\
1392	-45.166\\
1393	-34.18\\
1394	-26.855\\
1395	-34.18\\
1396	-37.842\\
1397	-25.635\\
1398	-48.828\\
1399	-64.697\\
1400	-46.387\\
1401	-81.787\\
1402	-115.967\\
1403	-106.201\\
1404	-133.057\\
1405	-92.773\\
1406	-97.656\\
1407	-69.58\\
1408	-43.945\\
1409	-53.711\\
1410	-42.725\\
1411	-34.18\\
1412	-45.166\\
1413	-26.855\\
1414	-20.752\\
1415	-25.635\\
1416	-50.049\\
1417	-61.035\\
1418	-78.125\\
1419	-74.463\\
1420	-69.58\\
1421	-80.566\\
1422	-64.697\\
1423	-53.711\\
1424	-54.932\\
1425	-43.945\\
1426	-69.58\\
1427	-50.049\\
1428	-40.283\\
1429	-58.594\\
1430	-80.566\\
1431	-59.814\\
1432	-46.387\\
1433	-41.504\\
1434	-47.607\\
1435	-32.959\\
1436	-31.738\\
1437	-43.945\\
1438	-53.711\\
1439	-59.814\\
1440	-48.828\\
1441	-61.035\\
1442	-57.373\\
1443	-34.18\\
1444	-37.842\\
1445	-65.918\\
1446	-91.553\\
1447	-90.332\\
1448	-75.684\\
1449	-73.242\\
1450	-57.373\\
1451	-53.711\\
1452	-43.945\\
1453	-61.035\\
1454	-54.932\\
1455	-36.621\\
1456	-52.49\\
1457	-61.035\\
1458	-70.801\\
1459	-76.904\\
1460	-76.904\\
1461	-102.539\\
1462	-124.512\\
1463	-140.381\\
1464	-91.553\\
1465	-52.49\\
1466	-34.18\\
1467	-25.635\\
1468	-36.621\\
1469	-32.959\\
1470	-58.594\\
1471	-53.711\\
1472	-46.387\\
1473	-62.256\\
1474	-72.021\\
1475	-81.787\\
1476	-86.67\\
1477	-122.07\\
1478	-101.318\\
1479	-81.787\\
1480	-58.594\\
1481	-58.594\\
1482	-59.814\\
1483	-75.684\\
1484	-54.932\\
1485	-56.152\\
1486	-53.711\\
1487	-30.518\\
1488	-54.932\\
1489	-70.801\\
1490	-50.049\\
1491	-53.711\\
1492	-100.098\\
1493	-75.684\\
1494	-72.021\\
1495	-102.539\\
1496	-98.877\\
1497	-62.256\\
1498	-40.283\\
1499	-42.725\\
1500	-70.801\\
};
\end{axis}

\begin{axis}[%
width=5.090835cm,
height=2.240107cm,
at={(6.698467cm,15.759893cm)},
scale only axis,
xmin=1000,
xmax=1500,
xlabel={Sample index},
ymin=-50,
ymax=0,
ylabel={C2 load, kN},
legend style={legend cell align=left,align=left,draw=white!15!black}
]
\addplot [color=mycolor1,solid,forget plot]
  table[row sep=crcr]{%
1000	-19.531\\
1001	-24.414\\
1002	-19.531\\
1003	-20.752\\
1004	-25.635\\
1005	-24.414\\
1006	-28.076\\
1007	-23.193\\
1008	-13.428\\
1009	-17.09\\
1010	-17.09\\
1011	-18.311\\
1012	-15.869\\
1013	-8.545\\
1014	-6.104\\
1015	-7.324\\
1016	-9.766\\
1017	-21.973\\
1018	-23.193\\
1019	-23.193\\
1020	-19.531\\
1021	-10.986\\
1022	-17.09\\
1023	-19.531\\
1024	-13.428\\
1025	-18.311\\
1026	-19.531\\
1027	-14.648\\
1028	-24.414\\
1029	-21.973\\
1030	-15.869\\
1031	-23.193\\
1032	-25.635\\
1033	-19.531\\
1034	-17.09\\
1035	-14.648\\
1036	-17.09\\
1037	-20.752\\
1038	-18.311\\
1039	-18.311\\
1040	-19.531\\
1041	-20.752\\
1042	-28.076\\
1043	-25.635\\
1044	-17.09\\
1045	-15.869\\
1046	-12.207\\
1047	-10.986\\
1048	-15.869\\
1049	-12.207\\
1050	-12.207\\
1051	-15.869\\
1052	-18.311\\
1053	-15.869\\
1054	-25.635\\
1055	-21.973\\
1056	-12.207\\
1057	-9.766\\
1058	-13.428\\
1059	-15.869\\
1060	-10.986\\
1061	-10.986\\
1062	-12.207\\
1063	-9.766\\
1064	-8.545\\
1065	-12.207\\
1066	-14.648\\
1067	-17.09\\
1068	-18.311\\
1069	-23.193\\
1070	-21.973\\
1071	-21.973\\
1072	-17.09\\
1073	-17.09\\
1074	-17.09\\
1075	-15.869\\
1076	-17.09\\
1077	-24.414\\
1078	-35.4\\
1079	-36.621\\
1080	-34.18\\
1081	-24.414\\
1082	-29.297\\
1083	-36.621\\
1084	-40.283\\
1085	-31.738\\
1086	-37.842\\
1087	-48.828\\
1088	-39.063\\
1089	-30.518\\
1090	-25.635\\
1091	-19.531\\
1092	-14.648\\
1093	-19.531\\
1094	-15.869\\
1095	-9.766\\
1096	-9.766\\
1097	-12.207\\
1098	-17.09\\
1099	-15.869\\
1100	-15.869\\
1101	-19.531\\
1102	-19.531\\
1103	-28.076\\
1104	-23.193\\
1105	-25.635\\
1106	-23.193\\
1107	-18.311\\
1108	-20.752\\
1109	-17.09\\
1110	-14.648\\
1111	-18.311\\
1112	-17.09\\
1113	-15.869\\
1114	-13.428\\
1115	-10.986\\
1116	-12.207\\
1117	-13.428\\
1118	-10.986\\
1119	-9.766\\
1120	-14.648\\
1121	-19.531\\
1122	-23.193\\
1123	-23.193\\
1124	-17.09\\
1125	-18.311\\
1126	-30.518\\
1127	-23.193\\
1128	-24.414\\
1129	-26.855\\
1130	-17.09\\
1131	-10.986\\
1132	-17.09\\
1133	-18.311\\
1134	-26.855\\
1135	-34.18\\
1136	-31.738\\
1137	-25.635\\
1138	-24.414\\
1139	-23.193\\
1140	-23.193\\
1141	-21.973\\
1142	-18.311\\
1143	-14.648\\
1144	-13.428\\
1145	-12.207\\
1146	-13.428\\
1147	-19.531\\
1148	-25.635\\
1149	-21.973\\
1150	-14.648\\
1151	-13.428\\
1152	-14.648\\
1153	-12.207\\
1154	-8.545\\
1155	-8.545\\
1156	-13.428\\
1157	-12.207\\
1158	-12.207\\
1159	-12.207\\
1160	-12.207\\
1161	-8.545\\
1162	-7.324\\
1163	-4.883\\
1164	-7.324\\
1165	-17.09\\
1166	-24.414\\
1167	-24.414\\
1168	-19.531\\
1169	-12.207\\
1170	-12.207\\
1171	-8.545\\
1172	-10.986\\
1173	-14.648\\
1174	-14.648\\
1175	-24.414\\
1176	-34.18\\
1177	-40.283\\
1178	-32.959\\
1179	-31.738\\
1180	-23.193\\
1181	-25.635\\
1182	-26.855\\
1183	-28.076\\
1184	-21.973\\
1185	-20.752\\
1186	-19.531\\
1187	-19.531\\
1188	-21.973\\
1189	-19.531\\
1190	-23.193\\
1191	-29.297\\
1192	-24.414\\
1193	-14.648\\
1194	-15.869\\
1195	-25.635\\
1196	-32.959\\
1197	-35.4\\
1198	-39.063\\
1199	-37.842\\
1200	-30.518\\
1201	-26.855\\
1202	-30.518\\
1203	-35.4\\
1204	-25.635\\
1205	-14.648\\
1206	-19.531\\
1207	-20.752\\
1208	-12.207\\
1209	-13.428\\
1210	-18.311\\
1211	-14.648\\
1212	-12.207\\
1213	-18.311\\
1214	-10.986\\
1215	-14.648\\
1216	-24.414\\
1217	-21.973\\
1218	-14.648\\
1219	-14.648\\
1220	-20.752\\
1221	-34.18\\
1222	-28.076\\
1223	-15.869\\
1224	-14.648\\
1225	-14.648\\
1226	-12.207\\
1227	-10.986\\
1228	-10.986\\
1229	-13.428\\
1230	-17.09\\
1231	-19.531\\
1232	-14.648\\
1233	-19.531\\
1234	-26.855\\
1235	-23.193\\
1236	-17.09\\
1237	-21.973\\
1238	-25.635\\
1239	-21.973\\
1240	-8.545\\
1241	-12.207\\
1242	-13.428\\
1243	-15.869\\
1244	-15.869\\
1245	-10.986\\
1246	-17.09\\
1247	-15.869\\
1248	-10.986\\
1249	-14.648\\
1250	-14.648\\
1251	-12.207\\
1252	-14.648\\
1253	-10.986\\
1254	-9.766\\
1255	-12.207\\
1256	-8.545\\
1257	-17.09\\
1258	-18.311\\
1259	-20.752\\
1260	-29.297\\
1261	-20.752\\
1262	-10.986\\
1263	-10.986\\
1264	-8.545\\
1265	-13.428\\
1266	-12.207\\
1267	-9.766\\
1268	-15.869\\
1269	-19.531\\
1270	-20.752\\
1271	-23.193\\
1272	-23.193\\
1273	-17.09\\
1274	-15.869\\
1275	-10.986\\
1276	-10.986\\
1277	-7.324\\
1278	-8.545\\
1279	-10.986\\
1280	-17.09\\
1281	-18.311\\
1282	-17.09\\
1283	-24.414\\
1284	-25.635\\
1285	-15.869\\
1286	-19.531\\
1287	-23.193\\
1288	-26.855\\
1289	-25.635\\
1290	-17.09\\
1291	-10.986\\
1292	-7.324\\
1293	-6.104\\
1294	-8.545\\
1295	-9.766\\
1296	-8.545\\
1297	-10.986\\
1298	-17.09\\
1299	-15.869\\
1300	-13.428\\
1301	-15.869\\
1302	-12.207\\
1303	-6.104\\
1304	-9.766\\
1305	-20.752\\
1306	-17.09\\
1307	-18.311\\
1308	-15.869\\
1309	-10.986\\
1310	-17.09\\
1311	-23.193\\
1312	-23.193\\
1313	-17.09\\
1314	-26.855\\
1315	-25.635\\
1316	-18.311\\
1317	-23.193\\
1318	-25.635\\
1319	-19.531\\
1320	-15.869\\
1321	-24.414\\
1322	-35.4\\
1323	-29.297\\
1324	-17.09\\
1325	-17.09\\
1326	-18.311\\
1327	-20.752\\
1328	-23.193\\
1329	-19.531\\
1330	-17.09\\
1331	-24.414\\
1332	-25.635\\
1333	-18.311\\
1334	-15.869\\
1335	-18.311\\
1336	-28.076\\
1337	-26.855\\
1338	-24.414\\
1339	-19.531\\
1340	-17.09\\
1341	-18.311\\
1342	-13.428\\
1343	-6.104\\
1344	-6.104\\
1345	-6.104\\
1346	-9.766\\
1347	-19.531\\
1348	-15.869\\
1349	-14.648\\
1350	-13.428\\
1351	-17.09\\
1352	-13.428\\
1353	-9.766\\
1354	-12.207\\
1355	-9.766\\
1356	-8.545\\
1357	-17.09\\
1358	-20.752\\
1359	-17.09\\
1360	-10.986\\
1361	-10.986\\
1362	-13.428\\
1363	-9.766\\
1364	-10.986\\
1365	-28.076\\
1366	-20.752\\
1367	-25.635\\
1368	-35.4\\
1369	-30.518\\
1370	-24.414\\
1371	-18.311\\
1372	-18.311\\
1373	-19.531\\
1374	-21.973\\
1375	-25.635\\
1376	-25.635\\
1377	-31.738\\
1378	-34.18\\
1379	-41.504\\
1380	-32.959\\
1381	-28.076\\
1382	-35.4\\
1383	-26.855\\
1384	-29.297\\
1385	-28.076\\
1386	-17.09\\
1387	-10.986\\
1388	-12.207\\
1389	-17.09\\
1390	-14.648\\
1391	-9.766\\
1392	-13.428\\
1393	-13.428\\
1394	-7.324\\
1395	-8.545\\
1396	-12.207\\
1397	-7.324\\
1398	-14.648\\
1399	-20.752\\
1400	-13.428\\
1401	-21.973\\
1402	-30.518\\
1403	-29.297\\
1404	-35.4\\
1405	-29.297\\
1406	-28.076\\
1407	-23.193\\
1408	-12.207\\
1409	-14.648\\
1410	-15.869\\
1411	-10.986\\
1412	-12.207\\
1413	-13.428\\
1414	-3.662\\
1415	-8.545\\
1416	-17.09\\
1417	-19.531\\
1418	-20.752\\
1419	-23.193\\
1420	-20.752\\
1421	-23.193\\
1422	-18.311\\
1423	-15.869\\
1424	-15.869\\
1425	-12.207\\
1426	-18.311\\
1427	-17.09\\
1428	-12.207\\
1429	-17.09\\
1430	-23.193\\
1431	-17.09\\
1432	-13.428\\
1433	-12.207\\
1434	-14.648\\
1435	-7.324\\
1436	-8.545\\
1437	-12.207\\
1438	-17.09\\
1439	-17.09\\
1440	-14.648\\
1441	-15.869\\
1442	-18.311\\
1443	-12.207\\
1444	-9.766\\
1445	-18.311\\
1446	-26.855\\
1447	-26.855\\
1448	-21.973\\
1449	-19.531\\
1450	-17.09\\
1451	-14.648\\
1452	-13.428\\
1453	-17.09\\
1454	-17.09\\
1455	-10.986\\
1456	-14.648\\
1457	-18.311\\
1458	-19.531\\
1459	-21.973\\
1460	-21.973\\
1461	-29.297\\
1462	-35.4\\
1463	-36.621\\
1464	-26.855\\
1465	-14.648\\
1466	-10.986\\
1467	-7.324\\
1468	-10.986\\
1469	-10.986\\
1470	-14.648\\
1471	-13.428\\
1472	-12.207\\
1473	-15.869\\
1474	-20.752\\
1475	-20.752\\
1476	-24.414\\
1477	-31.738\\
1478	-29.297\\
1479	-26.855\\
1480	-18.311\\
1481	-17.09\\
1482	-20.752\\
1483	-20.752\\
1484	-17.09\\
1485	-14.648\\
1486	-18.311\\
1487	-12.207\\
1488	-13.428\\
1489	-20.752\\
1490	-17.09\\
1491	-18.311\\
1492	-32.959\\
1493	-25.635\\
1494	-20.752\\
1495	-28.076\\
1496	-28.076\\
1497	-19.531\\
1498	-13.428\\
1499	-13.428\\
1500	-19.531\\
};
\end{axis}

\begin{axis}[%
width=5.090835cm,
height=2.240107cm,
at={(0cm,15.759893cm)},
scale only axis,
xmin=1000,
xmax=1500,
xlabel={Sample index},
ymin=-58.594,
ymax=0,
ylabel={C1 load, kN},
legend style={legend cell align=left,align=left,draw=white!15!black}
]
\addplot [color=mycolor1,solid,forget plot]
  table[row sep=crcr]{%
1000	-23.193\\
1001	-28.076\\
1002	-26.855\\
1003	-20.752\\
1004	-29.297\\
1005	-28.076\\
1006	-32.959\\
1007	-25.635\\
1008	-14.648\\
1009	-20.752\\
1010	-17.09\\
1011	-18.311\\
1012	-20.752\\
1013	-7.324\\
1014	-4.883\\
1015	-4.883\\
1016	-13.428\\
1017	-23.193\\
1018	-24.414\\
1019	-25.635\\
1020	-19.531\\
1021	-12.207\\
1022	-24.414\\
1023	-19.531\\
1024	-14.648\\
1025	-20.752\\
1026	-18.311\\
1027	-17.09\\
1028	-29.297\\
1029	-25.635\\
1030	-18.311\\
1031	-28.076\\
1032	-28.076\\
1033	-21.973\\
1034	-18.311\\
1035	-15.869\\
1036	-15.869\\
1037	-21.973\\
1038	-21.973\\
1039	-21.973\\
1040	-21.973\\
1041	-23.193\\
1042	-30.518\\
1043	-29.297\\
1044	-19.531\\
1045	-18.311\\
1046	-10.986\\
1047	-14.648\\
1048	-15.869\\
1049	-12.207\\
1050	-13.428\\
1051	-18.311\\
1052	-19.531\\
1053	-18.311\\
1054	-29.297\\
1055	-21.973\\
1056	-12.207\\
1057	-12.207\\
1058	-13.428\\
1059	-17.09\\
1060	-9.766\\
1061	-10.986\\
1062	-14.648\\
1063	-9.766\\
1064	-7.324\\
1065	-13.428\\
1066	-13.428\\
1067	-21.973\\
1068	-20.752\\
1069	-25.635\\
1070	-21.973\\
1071	-25.635\\
1072	-20.752\\
1073	-20.752\\
1074	-18.311\\
1075	-18.311\\
1076	-17.09\\
1077	-30.518\\
1078	-41.504\\
1079	-41.504\\
1080	-42.725\\
1081	-28.076\\
1082	-37.842\\
1083	-46.387\\
1084	-46.387\\
1085	-35.4\\
1086	-47.607\\
1087	-58.594\\
1088	-43.945\\
1089	-34.18\\
1090	-24.414\\
1091	-20.752\\
1092	-17.09\\
1093	-21.973\\
1094	-17.09\\
1095	-12.207\\
1096	-9.766\\
1097	-12.207\\
1098	-18.311\\
1099	-18.311\\
1100	-18.311\\
1101	-21.973\\
1102	-23.193\\
1103	-30.518\\
1104	-28.076\\
1105	-30.518\\
1106	-25.635\\
1107	-20.752\\
1108	-24.414\\
1109	-18.311\\
1110	-13.428\\
1111	-19.531\\
1112	-20.752\\
1113	-12.207\\
1114	-15.869\\
1115	-12.207\\
1116	-13.428\\
1117	-13.428\\
1118	-9.766\\
1119	-10.986\\
1120	-17.09\\
1121	-23.193\\
1122	-25.635\\
1123	-26.855\\
1124	-15.869\\
1125	-20.752\\
1126	-32.959\\
1127	-24.414\\
1128	-28.076\\
1129	-29.297\\
1130	-18.311\\
1131	-12.207\\
1132	-17.09\\
1133	-18.311\\
1134	-30.518\\
1135	-36.621\\
1136	-36.621\\
1137	-28.076\\
1138	-28.076\\
1139	-25.635\\
1140	-25.635\\
1141	-25.635\\
1142	-20.752\\
1143	-15.869\\
1144	-14.648\\
1145	-13.428\\
1146	-13.428\\
1147	-20.752\\
1148	-31.738\\
1149	-26.855\\
1150	-17.09\\
1151	-15.869\\
1152	-15.869\\
1153	-9.766\\
1154	-7.324\\
1155	-7.324\\
1156	-15.869\\
1157	-10.986\\
1158	-14.648\\
1159	-14.648\\
1160	-13.428\\
1161	-9.766\\
1162	-6.104\\
1163	-4.883\\
1164	-7.324\\
1165	-18.311\\
1166	-26.855\\
1167	-28.076\\
1168	-20.752\\
1169	-13.428\\
1170	-10.986\\
1171	-6.104\\
1172	-12.207\\
1173	-14.648\\
1174	-18.311\\
1175	-26.855\\
1176	-39.063\\
1177	-47.607\\
1178	-42.725\\
1179	-37.842\\
1180	-28.076\\
1181	-30.518\\
1182	-31.738\\
1183	-34.18\\
1184	-24.414\\
1185	-23.193\\
1186	-23.193\\
1187	-21.973\\
1188	-23.193\\
1189	-18.311\\
1190	-30.518\\
1191	-37.842\\
1192	-28.076\\
1193	-18.311\\
1194	-18.311\\
1195	-30.518\\
1196	-35.4\\
1197	-40.283\\
1198	-45.166\\
1199	-45.166\\
1200	-34.18\\
1201	-32.959\\
1202	-36.621\\
1203	-41.504\\
1204	-29.297\\
1205	-17.09\\
1206	-23.193\\
1207	-20.752\\
1208	-13.428\\
1209	-18.311\\
1210	-19.531\\
1211	-14.648\\
1212	-13.428\\
1213	-15.869\\
1214	-13.428\\
1215	-17.09\\
1216	-25.635\\
1217	-25.635\\
1218	-15.869\\
1219	-14.648\\
1220	-24.414\\
1221	-40.283\\
1222	-29.297\\
1223	-20.752\\
1224	-14.648\\
1225	-18.311\\
1226	-15.869\\
1227	-12.207\\
1228	-13.428\\
1229	-15.869\\
1230	-18.311\\
1231	-20.752\\
1232	-15.869\\
1233	-23.193\\
1234	-29.297\\
1235	-25.635\\
1236	-20.752\\
1237	-23.193\\
1238	-30.518\\
1239	-21.973\\
1240	-8.545\\
1241	-13.428\\
1242	-13.428\\
1243	-17.09\\
1244	-17.09\\
1245	-13.428\\
1246	-17.09\\
1247	-19.531\\
1248	-13.428\\
1249	-15.869\\
1250	-15.869\\
1251	-12.207\\
1252	-15.869\\
1253	-9.766\\
1254	-10.986\\
1255	-8.545\\
1256	-8.545\\
1257	-14.648\\
1258	-20.752\\
1259	-24.414\\
1260	-35.4\\
1261	-24.414\\
1262	-13.428\\
1263	-12.207\\
1264	-10.986\\
1265	-13.428\\
1266	-10.986\\
1267	-12.207\\
1268	-14.648\\
1269	-24.414\\
1270	-21.973\\
1271	-26.855\\
1272	-23.193\\
1273	-17.09\\
1274	-13.428\\
1275	-10.986\\
1276	-8.545\\
1277	-6.104\\
1278	-9.766\\
1279	-14.648\\
1280	-18.311\\
1281	-18.311\\
1282	-19.531\\
1283	-29.297\\
1284	-25.635\\
1285	-18.311\\
1286	-24.414\\
1287	-24.414\\
1288	-31.738\\
1289	-26.855\\
1290	-17.09\\
1291	-9.766\\
1292	-6.104\\
1293	-7.324\\
1294	-9.766\\
1295	-8.545\\
1296	-8.545\\
1297	-12.207\\
1298	-17.09\\
1299	-15.869\\
1300	-15.869\\
1301	-19.531\\
1302	-13.428\\
1303	-4.883\\
1304	-9.766\\
1305	-21.973\\
1306	-19.531\\
1307	-21.973\\
1308	-18.311\\
1309	-14.648\\
1310	-19.531\\
1311	-25.635\\
1312	-25.635\\
1313	-19.531\\
1314	-26.855\\
1315	-26.855\\
1316	-23.193\\
1317	-28.076\\
1318	-26.855\\
1319	-20.752\\
1320	-19.531\\
1321	-29.297\\
1322	-40.283\\
1323	-30.518\\
1324	-18.311\\
1325	-17.09\\
1326	-18.311\\
1327	-21.973\\
1328	-26.855\\
1329	-19.531\\
1330	-19.531\\
1331	-26.855\\
1332	-28.076\\
1333	-20.752\\
1334	-14.648\\
1335	-20.752\\
1336	-34.18\\
1337	-30.518\\
1338	-31.738\\
1339	-21.973\\
1340	-17.09\\
1341	-17.09\\
1342	-10.986\\
1343	-6.104\\
1344	-6.104\\
1345	-4.883\\
1346	-10.986\\
1347	-19.531\\
1348	-18.311\\
1349	-15.869\\
1350	-14.648\\
1351	-20.752\\
1352	-14.648\\
1353	-9.766\\
1354	-13.428\\
1355	-8.545\\
1356	-10.986\\
1357	-18.311\\
1358	-23.193\\
1359	-17.09\\
1360	-9.766\\
1361	-10.986\\
1362	-12.207\\
1363	-9.766\\
1364	-13.428\\
1365	-31.738\\
1366	-25.635\\
1367	-31.738\\
1368	-37.842\\
1369	-36.621\\
1370	-26.855\\
1371	-20.752\\
1372	-18.311\\
1373	-21.973\\
1374	-23.193\\
1375	-29.297\\
1376	-29.297\\
1377	-36.621\\
1378	-42.725\\
1379	-46.387\\
1380	-37.842\\
1381	-35.4\\
1382	-40.283\\
1383	-31.738\\
1384	-34.18\\
1385	-31.738\\
1386	-18.311\\
1387	-12.207\\
1388	-13.428\\
1389	-18.311\\
1390	-17.09\\
1391	-8.545\\
1392	-14.648\\
1393	-13.428\\
1394	-6.104\\
1395	-10.986\\
1396	-12.207\\
1397	-7.324\\
1398	-18.311\\
1399	-20.752\\
1400	-14.648\\
1401	-23.193\\
1402	-37.842\\
1403	-32.959\\
1404	-37.842\\
1405	-32.959\\
1406	-29.297\\
1407	-20.752\\
1408	-14.648\\
1409	-17.09\\
1410	-13.428\\
1411	-10.986\\
1412	-14.648\\
1413	-13.428\\
1414	-3.662\\
1415	-7.324\\
1416	-15.869\\
1417	-20.752\\
1418	-21.973\\
1419	-23.193\\
1420	-21.973\\
1421	-26.855\\
1422	-21.973\\
1423	-17.09\\
1424	-17.09\\
1425	-14.648\\
1426	-20.752\\
1427	-18.311\\
1428	-13.428\\
1429	-18.311\\
1430	-26.855\\
1431	-20.752\\
1432	-15.869\\
1433	-13.428\\
1434	-14.648\\
1435	-12.207\\
1436	-8.545\\
1437	-14.648\\
1438	-19.531\\
1439	-18.311\\
1440	-15.869\\
1441	-19.531\\
1442	-18.311\\
1443	-10.986\\
1444	-10.986\\
1445	-21.973\\
1446	-30.518\\
1447	-29.297\\
1448	-23.193\\
1449	-21.973\\
1450	-19.531\\
1451	-18.311\\
1452	-15.869\\
1453	-19.531\\
1454	-17.09\\
1455	-13.428\\
1456	-17.09\\
1457	-19.531\\
1458	-21.973\\
1459	-25.635\\
1460	-25.635\\
1461	-32.959\\
1462	-41.504\\
1463	-45.166\\
1464	-31.738\\
1465	-17.09\\
1466	-13.428\\
1467	-8.545\\
1468	-10.986\\
1469	-13.428\\
1470	-17.09\\
1471	-18.311\\
1472	-13.428\\
1473	-19.531\\
1474	-24.414\\
1475	-25.635\\
1476	-26.855\\
1477	-39.063\\
1478	-35.4\\
1479	-25.635\\
1480	-20.752\\
1481	-20.752\\
1482	-19.531\\
1483	-23.193\\
1484	-19.531\\
1485	-15.869\\
1486	-17.09\\
1487	-10.986\\
1488	-14.648\\
1489	-28.076\\
1490	-19.531\\
1491	-15.869\\
1492	-34.18\\
1493	-28.076\\
1494	-21.973\\
1495	-34.18\\
1496	-31.738\\
1497	-20.752\\
1498	-15.869\\
1499	-13.428\\
1500	-23.193\\
};
\end{axis}

\begin{axis}[%
width=5.090835cm,
height=2.240107cm,
at={(0cm,3.939973cm)},
scale only axis,
xmin=1000,
xmax=1500,
xlabel={Sample index},
ymin=-400,
ymax=0,
ylabel={C7 load, kN},
legend style={legend cell align=left,align=left,draw=white!15!black}
]
\addplot [color=mycolor1,solid,forget plot]
  table[row sep=crcr]{%
1000	-112.305\\
1001	-144.043\\
1002	-117.188\\
1003	-111.084\\
1004	-156.25\\
1005	-147.705\\
1006	-175.781\\
1007	-134.277\\
1008	-68.359\\
1009	-80.566\\
1010	-80.566\\
1011	-100.098\\
1012	-81.787\\
1013	-34.18\\
1014	-24.414\\
1015	-23.193\\
1016	-73.242\\
1017	-128.174\\
1018	-133.057\\
1019	-137.939\\
1020	-92.773\\
1021	-56.152\\
1022	-125.732\\
1023	-86.67\\
1024	-68.359\\
1025	-114.746\\
1026	-100.098\\
1027	-85.449\\
1028	-142.822\\
1029	-133.057\\
1030	-102.539\\
1031	-150.146\\
1032	-145.264\\
1033	-113.525\\
1034	-92.773\\
1035	-72.021\\
1036	-86.67\\
1037	-112.305\\
1038	-103.76\\
1039	-109.863\\
1040	-112.305\\
1041	-122.07\\
1042	-175.781\\
1043	-155.029\\
1044	-102.539\\
1045	-92.773\\
1046	-53.711\\
1047	-58.594\\
1048	-79.346\\
1049	-59.814\\
1050	-62.256\\
1051	-89.111\\
1052	-102.539\\
1053	-95.215\\
1054	-151.367\\
1055	-111.084\\
1056	-64.697\\
1057	-51.27\\
1058	-69.58\\
1059	-81.787\\
1060	-43.945\\
1061	-45.166\\
1062	-67.139\\
1063	-51.27\\
1064	-42.725\\
1065	-61.035\\
1066	-70.801\\
1067	-109.863\\
1068	-104.98\\
1069	-129.395\\
1070	-120.85\\
1071	-137.939\\
1072	-109.863\\
1073	-107.422\\
1074	-92.773\\
1075	-91.553\\
1076	-87.891\\
1077	-157.471\\
1078	-224.609\\
1079	-231.934\\
1080	-225.83\\
1081	-140.381\\
1082	-205.078\\
1083	-252.686\\
1084	-261.23\\
1085	-200.195\\
1086	-270.996\\
1087	-335.693\\
1088	-235.596\\
1089	-191.65\\
1090	-128.174\\
1091	-98.877\\
1092	-84.229\\
1093	-104.98\\
1094	-74.463\\
1095	-50.049\\
1096	-48.828\\
1097	-63.477\\
1098	-95.215\\
1099	-86.67\\
1100	-89.111\\
1101	-119.629\\
1102	-123.291\\
1103	-167.236\\
1104	-134.277\\
1105	-169.678\\
1106	-122.07\\
1107	-108.643\\
1108	-125.732\\
1109	-89.111\\
1110	-80.566\\
1111	-97.656\\
1112	-98.877\\
1113	-68.359\\
1114	-73.242\\
1115	-54.932\\
1116	-61.035\\
1117	-64.697\\
1118	-45.166\\
1119	-58.594\\
1120	-72.021\\
1121	-117.188\\
1122	-122.07\\
1123	-136.719\\
1124	-76.904\\
1125	-109.863\\
1126	-173.34\\
1127	-131.836\\
1128	-157.471\\
1129	-157.471\\
1130	-85.449\\
1131	-59.814\\
1132	-85.449\\
1133	-98.877\\
1134	-161.133\\
1135	-202.637\\
1136	-198.975\\
1137	-145.264\\
1138	-150.146\\
1139	-135.498\\
1140	-131.836\\
1141	-126.953\\
1142	-85.449\\
1143	-76.904\\
1144	-69.58\\
1145	-62.256\\
1146	-70.801\\
1147	-107.422\\
1148	-164.795\\
1149	-125.732\\
1150	-86.67\\
1151	-73.242\\
1152	-70.801\\
1153	-42.725\\
1154	-31.738\\
1155	-39.063\\
1156	-72.021\\
1157	-57.373\\
1158	-59.814\\
1159	-67.139\\
1160	-63.477\\
1161	-43.945\\
1162	-29.297\\
1163	-25.635\\
1164	-43.945\\
1165	-96.436\\
1166	-140.381\\
1167	-155.029\\
1168	-101.318\\
1169	-69.58\\
1170	-50.049\\
1171	-34.18\\
1172	-83.008\\
1173	-81.787\\
1174	-119.629\\
1175	-145.264\\
1176	-230.713\\
1177	-262.451\\
1178	-211.182\\
1179	-202.637\\
1180	-136.719\\
1181	-151.367\\
1182	-159.912\\
1183	-172.119\\
1184	-129.395\\
1185	-118.408\\
1186	-115.967\\
1187	-103.76\\
1188	-123.291\\
1189	-87.891\\
1190	-153.809\\
1191	-189.209\\
1192	-133.057\\
1193	-78.125\\
1194	-85.449\\
1195	-146.484\\
1196	-181.885\\
1197	-229.492\\
1198	-241.699\\
1199	-240.479\\
1200	-180.664\\
1201	-163.574\\
1202	-187.988\\
1203	-222.168\\
1204	-139.16\\
1205	-81.787\\
1206	-122.07\\
1207	-102.539\\
1208	-65.918\\
1209	-87.891\\
1210	-98.877\\
1211	-76.904\\
1212	-58.594\\
1213	-80.566\\
1214	-65.918\\
1215	-91.553\\
1216	-133.057\\
1217	-122.07\\
1218	-79.346\\
1219	-83.008\\
1220	-126.953\\
1221	-209.961\\
1222	-150.146\\
1223	-92.773\\
1224	-69.58\\
1225	-83.008\\
1226	-74.463\\
1227	-56.152\\
1228	-62.256\\
1229	-74.463\\
1230	-96.436\\
1231	-107.422\\
1232	-72.021\\
1233	-122.07\\
1234	-144.043\\
1235	-137.939\\
1236	-106.201\\
1237	-118.408\\
1238	-158.691\\
1239	-101.318\\
1240	-41.504\\
1241	-58.594\\
1242	-65.918\\
1243	-91.553\\
1244	-81.787\\
1245	-59.814\\
1246	-96.436\\
1247	-91.553\\
1248	-65.918\\
1249	-83.008\\
1250	-76.904\\
1251	-67.139\\
1252	-79.346\\
1253	-46.387\\
1254	-53.711\\
1255	-40.283\\
1256	-43.945\\
1257	-86.67\\
1258	-100.098\\
1259	-137.939\\
1260	-177.002\\
1261	-115.967\\
1262	-68.359\\
1263	-48.828\\
1264	-48.828\\
1265	-73.242\\
1266	-46.387\\
1267	-52.49\\
1268	-70.801\\
1269	-119.629\\
1270	-107.422\\
1271	-150.146\\
1272	-102.539\\
1273	-91.553\\
1274	-47.607\\
1275	-47.607\\
1276	-31.738\\
1277	-34.18\\
1278	-41.504\\
1279	-75.684\\
1280	-83.008\\
1281	-92.773\\
1282	-97.656\\
1283	-152.588\\
1284	-130.615\\
1285	-97.656\\
1286	-119.629\\
1287	-129.395\\
1288	-167.236\\
1289	-133.057\\
1290	-84.229\\
1291	-43.945\\
1292	-31.738\\
1293	-34.18\\
1294	-41.504\\
1295	-43.945\\
1296	-40.283\\
1297	-56.152\\
1298	-87.891\\
1299	-79.346\\
1300	-81.787\\
1301	-96.436\\
1302	-58.594\\
1303	-29.297\\
1304	-64.697\\
1305	-97.656\\
1306	-96.436\\
1307	-109.863\\
1308	-93.994\\
1309	-69.58\\
1310	-97.656\\
1311	-137.939\\
1312	-139.16\\
1313	-97.656\\
1314	-162.354\\
1315	-128.174\\
1316	-118.408\\
1317	-137.939\\
1318	-137.939\\
1319	-103.76\\
1320	-90.332\\
1321	-153.809\\
1322	-214.844\\
1323	-164.795\\
1324	-92.773\\
1325	-92.773\\
1326	-91.553\\
1327	-113.525\\
1328	-136.719\\
1329	-100.098\\
1330	-104.98\\
1331	-140.381\\
1332	-153.809\\
1333	-103.76\\
1334	-79.346\\
1335	-104.98\\
1336	-175.781\\
1337	-159.912\\
1338	-162.354\\
1339	-95.215\\
1340	-84.229\\
1341	-83.008\\
1342	-52.49\\
1343	-34.18\\
1344	-28.076\\
1345	-23.193\\
1346	-59.814\\
1347	-86.67\\
1348	-104.98\\
1349	-76.904\\
1350	-87.891\\
1351	-97.656\\
1352	-59.814\\
1353	-63.477\\
1354	-61.035\\
1355	-45.166\\
1356	-57.373\\
1357	-97.656\\
1358	-124.512\\
1359	-78.125\\
1360	-56.152\\
1361	-46.387\\
1362	-59.814\\
1363	-41.504\\
1364	-91.553\\
1365	-158.691\\
1366	-122.07\\
1367	-167.236\\
1368	-194.092\\
1369	-197.754\\
1370	-140.381\\
1371	-106.201\\
1372	-106.201\\
1373	-104.98\\
1374	-120.85\\
1375	-150.146\\
1376	-147.705\\
1377	-200.195\\
1378	-220.947\\
1379	-263.672\\
1380	-178.223\\
1381	-190.43\\
1382	-212.402\\
1383	-163.574\\
1384	-189.209\\
1385	-163.574\\
1386	-91.553\\
1387	-58.594\\
1388	-62.256\\
1389	-95.215\\
1390	-75.684\\
1391	-51.27\\
1392	-72.021\\
1393	-52.49\\
1394	-40.283\\
1395	-48.828\\
1396	-56.152\\
1397	-40.283\\
1398	-76.904\\
1399	-109.863\\
1400	-78.125\\
1401	-134.277\\
1402	-195.313\\
1403	-178.223\\
1404	-222.168\\
1405	-150.146\\
1406	-161.133\\
1407	-111.084\\
1408	-67.139\\
1409	-85.449\\
1410	-67.139\\
1411	-48.828\\
1412	-72.021\\
1413	-41.504\\
1414	-25.635\\
1415	-36.621\\
1416	-79.346\\
1417	-104.98\\
1418	-128.174\\
1419	-124.512\\
1420	-117.188\\
1421	-136.719\\
1422	-111.084\\
1423	-90.332\\
1424	-90.332\\
1425	-73.242\\
1426	-115.967\\
1427	-78.125\\
1428	-67.139\\
1429	-100.098\\
1430	-137.939\\
1431	-103.76\\
1432	-76.904\\
1433	-64.697\\
1434	-74.463\\
1435	-51.27\\
1436	-50.049\\
1437	-72.021\\
1438	-86.67\\
1439	-98.877\\
1440	-79.346\\
1441	-101.318\\
1442	-92.773\\
1443	-52.49\\
1444	-54.932\\
1445	-112.305\\
1446	-158.691\\
1447	-153.809\\
1448	-129.395\\
1449	-122.07\\
1450	-92.773\\
1451	-89.111\\
1452	-70.801\\
1453	-102.539\\
1454	-90.332\\
1455	-54.932\\
1456	-84.229\\
1457	-101.318\\
1458	-118.408\\
1459	-129.395\\
1460	-129.395\\
1461	-175.781\\
1462	-216.064\\
1463	-244.141\\
1464	-153.809\\
1465	-85.449\\
1466	-54.932\\
1467	-37.842\\
1468	-57.373\\
1469	-53.711\\
1470	-95.215\\
1471	-85.449\\
1472	-75.684\\
1473	-102.539\\
1474	-118.408\\
1475	-141.602\\
1476	-148.926\\
1477	-208.74\\
1478	-181.885\\
1479	-136.719\\
1480	-90.332\\
1481	-92.773\\
1482	-95.215\\
1483	-123.291\\
1484	-87.891\\
1485	-89.111\\
1486	-87.891\\
1487	-47.607\\
1488	-81.787\\
1489	-128.174\\
1490	-90.332\\
1491	-96.436\\
1492	-170.898\\
1493	-129.395\\
1494	-122.07\\
1495	-177.002\\
1496	-166.016\\
1497	-102.539\\
1498	-64.697\\
1499	-65.918\\
1500	-114.746\\
};
\end{axis}

\begin{axis}[%
width=5.090835cm,
height=2.240107cm,
at={(6.698467cm,7.879947cm)},
scale only axis,
xmin=1000,
xmax=1500,
xlabel={Sample index},
ymin=-404.053,
ymax=0,
ylabel={C6 load, kN},
legend style={legend cell align=left,align=left,draw=white!15!black}
]
\addplot [color=mycolor1,solid,forget plot]
  table[row sep=crcr]{%
1000	-140.381\\
1001	-178.223\\
1002	-147.705\\
1003	-139.16\\
1004	-192.871\\
1005	-185.547\\
1006	-219.727\\
1007	-169.678\\
1008	-86.67\\
1009	-107.422\\
1010	-102.539\\
1011	-126.953\\
1012	-103.76\\
1013	-41.504\\
1014	-31.738\\
1015	-28.076\\
1016	-89.111\\
1017	-157.471\\
1018	-167.236\\
1019	-174.561\\
1020	-124.512\\
1021	-74.463\\
1022	-158.691\\
1023	-111.084\\
1024	-83.008\\
1025	-142.822\\
1026	-123.291\\
1027	-103.76\\
1028	-173.34\\
1029	-164.795\\
1030	-124.512\\
1031	-180.664\\
1032	-179.443\\
1033	-142.822\\
1034	-117.188\\
1035	-89.111\\
1036	-108.643\\
1037	-137.939\\
1038	-130.615\\
1039	-136.719\\
1040	-140.381\\
1041	-151.367\\
1042	-213.623\\
1043	-191.65\\
1044	-126.953\\
1045	-115.967\\
1046	-64.697\\
1047	-69.58\\
1048	-96.436\\
1049	-74.463\\
1050	-78.125\\
1051	-111.084\\
1052	-128.174\\
1053	-119.629\\
1054	-190.43\\
1055	-139.16\\
1056	-81.787\\
1057	-63.477\\
1058	-85.449\\
1059	-101.318\\
1060	-51.27\\
1061	-58.594\\
1062	-81.787\\
1063	-65.918\\
1064	-56.152\\
1065	-74.463\\
1066	-86.67\\
1067	-139.16\\
1068	-130.615\\
1069	-163.574\\
1070	-146.484\\
1071	-170.898\\
1072	-139.16\\
1073	-133.057\\
1074	-115.967\\
1075	-111.084\\
1076	-111.084\\
1077	-196.533\\
1078	-280.762\\
1079	-291.748\\
1080	-280.762\\
1081	-175.781\\
1082	-240.479\\
1083	-303.955\\
1084	-317.383\\
1085	-239.258\\
1086	-313.721\\
1087	-404.053\\
1088	-289.307\\
1089	-238.037\\
1090	-158.691\\
1091	-122.07\\
1092	-102.539\\
1093	-129.395\\
1094	-92.773\\
1095	-61.035\\
1096	-59.814\\
1097	-75.684\\
1098	-118.408\\
1099	-107.422\\
1100	-111.084\\
1101	-146.484\\
1102	-152.588\\
1103	-207.52\\
1104	-167.236\\
1105	-208.74\\
1106	-148.926\\
1107	-130.615\\
1108	-151.367\\
1109	-108.643\\
1110	-98.877\\
1111	-122.07\\
1112	-123.291\\
1113	-85.449\\
1114	-95.215\\
1115	-67.139\\
1116	-76.904\\
1117	-81.787\\
1118	-57.373\\
1119	-75.684\\
1120	-93.994\\
1121	-151.367\\
1122	-152.588\\
1123	-170.898\\
1124	-97.656\\
1125	-140.381\\
1126	-211.182\\
1127	-161.133\\
1128	-186.768\\
1129	-191.65\\
1130	-112.305\\
1131	-75.684\\
1132	-107.422\\
1133	-123.291\\
1134	-205.078\\
1135	-252.686\\
1136	-252.686\\
1137	-184.326\\
1138	-189.209\\
1139	-167.236\\
1140	-163.574\\
1141	-162.354\\
1142	-108.643\\
1143	-96.436\\
1144	-86.67\\
1145	-78.125\\
1146	-87.891\\
1147	-133.057\\
1148	-203.857\\
1149	-161.133\\
1150	-109.863\\
1151	-92.773\\
1152	-89.111\\
1153	-52.49\\
1154	-41.504\\
1155	-47.607\\
1156	-90.332\\
1157	-67.139\\
1158	-78.125\\
1159	-85.449\\
1160	-80.566\\
1161	-54.932\\
1162	-37.842\\
1163	-30.518\\
1164	-54.932\\
1165	-119.629\\
1166	-172.119\\
1167	-194.092\\
1168	-136.719\\
1169	-91.553\\
1170	-64.697\\
1171	-43.945\\
1172	-98.877\\
1173	-91.553\\
1174	-137.939\\
1175	-168.457\\
1176	-275.879\\
1177	-316.162\\
1178	-260.01\\
1179	-249.023\\
1180	-159.912\\
1181	-185.547\\
1182	-195.313\\
1183	-212.402\\
1184	-158.691\\
1185	-144.043\\
1186	-142.822\\
1187	-129.395\\
1188	-156.25\\
1189	-109.863\\
1190	-195.313\\
1191	-234.375\\
1192	-175.781\\
1193	-104.98\\
1194	-111.084\\
1195	-192.871\\
1196	-234.375\\
1197	-289.307\\
1198	-303.955\\
1199	-302.734\\
1200	-228.271\\
1201	-205.078\\
1202	-235.596\\
1203	-275.879\\
1204	-170.898\\
1205	-104.98\\
1206	-152.588\\
1207	-122.07\\
1208	-79.346\\
1209	-109.863\\
1210	-122.07\\
1211	-95.215\\
1212	-75.684\\
1213	-101.318\\
1214	-80.566\\
1215	-115.967\\
1216	-162.354\\
1217	-147.705\\
1218	-100.098\\
1219	-101.318\\
1220	-157.471\\
1221	-261.23\\
1222	-185.547\\
1223	-118.408\\
1224	-85.449\\
1225	-101.318\\
1226	-90.332\\
1227	-67.139\\
1228	-76.904\\
1229	-92.773\\
1230	-120.85\\
1231	-134.277\\
1232	-89.111\\
1233	-156.25\\
1234	-179.443\\
1235	-170.898\\
1236	-140.381\\
1237	-150.146\\
1238	-202.637\\
1239	-125.732\\
1240	-53.711\\
1241	-78.125\\
1242	-85.449\\
1243	-119.629\\
1244	-101.318\\
1245	-74.463\\
1246	-118.408\\
1247	-112.305\\
1248	-79.346\\
1249	-102.539\\
1250	-90.332\\
1251	-80.566\\
1252	-95.215\\
1253	-54.932\\
1254	-68.359\\
1255	-47.607\\
1256	-51.27\\
1257	-106.201\\
1258	-122.07\\
1259	-167.236\\
1260	-219.727\\
1261	-142.822\\
1262	-83.008\\
1263	-63.477\\
1264	-62.256\\
1265	-91.553\\
1266	-54.932\\
1267	-70.801\\
1268	-85.449\\
1269	-150.146\\
1270	-133.057\\
1271	-190.43\\
1272	-124.512\\
1273	-117.188\\
1274	-56.152\\
1275	-62.256\\
1276	-34.18\\
1277	-46.387\\
1278	-51.27\\
1279	-95.215\\
1280	-100.098\\
1281	-117.188\\
1282	-123.291\\
1283	-197.754\\
1284	-170.898\\
1285	-128.174\\
1286	-153.809\\
1287	-163.574\\
1288	-213.623\\
1289	-161.133\\
1290	-104.98\\
1291	-53.711\\
1292	-40.283\\
1293	-41.504\\
1294	-52.49\\
1295	-54.932\\
1296	-51.27\\
1297	-70.801\\
1298	-108.643\\
1299	-98.877\\
1300	-103.76\\
1301	-118.408\\
1302	-69.58\\
1303	-36.621\\
1304	-81.787\\
1305	-125.732\\
1306	-125.732\\
1307	-142.822\\
1308	-118.408\\
1309	-87.891\\
1310	-123.291\\
1311	-172.119\\
1312	-172.119\\
1313	-120.85\\
1314	-207.52\\
1315	-155.029\\
1316	-151.367\\
1317	-173.34\\
1318	-172.119\\
1319	-130.615\\
1320	-112.305\\
1321	-192.871\\
1322	-266.113\\
1323	-202.637\\
1324	-124.512\\
1325	-118.408\\
1326	-115.967\\
1327	-150.146\\
1328	-173.34\\
1329	-125.732\\
1330	-134.277\\
1331	-177.002\\
1332	-192.871\\
1333	-120.85\\
1334	-97.656\\
1335	-130.615\\
1336	-218.506\\
1337	-195.313\\
1338	-203.857\\
1339	-115.967\\
1340	-108.643\\
1341	-101.318\\
1342	-63.477\\
1343	-41.504\\
1344	-34.18\\
1345	-29.297\\
1346	-78.125\\
1347	-108.643\\
1348	-131.836\\
1349	-93.994\\
1350	-107.422\\
1351	-119.629\\
1352	-72.021\\
1353	-79.346\\
1354	-73.242\\
1355	-54.932\\
1356	-75.684\\
1357	-120.85\\
1358	-156.25\\
1359	-95.215\\
1360	-73.242\\
1361	-58.594\\
1362	-74.463\\
1363	-45.166\\
1364	-117.188\\
1365	-195.313\\
1366	-163.574\\
1367	-216.064\\
1368	-241.699\\
1369	-249.023\\
1370	-181.885\\
1371	-131.836\\
1372	-130.615\\
1373	-129.395\\
1374	-155.029\\
1375	-184.326\\
1376	-181.885\\
1377	-247.803\\
1378	-270.996\\
1379	-328.369\\
1380	-217.285\\
1381	-238.037\\
1382	-264.893\\
1383	-205.078\\
1384	-239.258\\
1385	-200.195\\
1386	-113.525\\
1387	-74.463\\
1388	-79.346\\
1389	-117.188\\
1390	-97.656\\
1391	-64.697\\
1392	-90.332\\
1393	-63.477\\
1394	-48.828\\
1395	-64.697\\
1396	-72.021\\
1397	-48.828\\
1398	-100.098\\
1399	-135.498\\
1400	-97.656\\
1401	-173.34\\
1402	-241.699\\
1403	-220.947\\
1404	-279.541\\
1405	-187.988\\
1406	-206.299\\
1407	-134.277\\
1408	-85.449\\
1409	-108.643\\
1410	-87.891\\
1411	-63.477\\
1412	-92.773\\
1413	-50.049\\
1414	-32.959\\
1415	-43.945\\
1416	-98.877\\
1417	-125.732\\
1418	-157.471\\
1419	-152.588\\
1420	-146.484\\
1421	-168.457\\
1422	-135.498\\
1423	-108.643\\
1424	-112.305\\
1425	-89.111\\
1426	-146.484\\
1427	-96.436\\
1428	-80.566\\
1429	-120.85\\
1430	-164.795\\
1431	-123.291\\
1432	-93.994\\
1433	-80.566\\
1434	-91.553\\
1435	-61.035\\
1436	-61.035\\
1437	-87.891\\
1438	-109.863\\
1439	-124.512\\
1440	-96.436\\
1441	-128.174\\
1442	-114.746\\
1443	-61.035\\
1444	-69.58\\
1445	-135.498\\
1446	-190.43\\
1447	-186.768\\
1448	-157.471\\
1449	-152.588\\
1450	-114.746\\
1451	-109.863\\
1452	-87.891\\
1453	-125.732\\
1454	-117.188\\
1455	-70.801\\
1456	-107.422\\
1457	-126.953\\
1458	-150.146\\
1459	-164.795\\
1460	-163.574\\
1461	-222.168\\
1462	-267.334\\
1463	-302.734\\
1464	-190.43\\
1465	-106.201\\
1466	-68.359\\
1467	-45.166\\
1468	-70.801\\
1469	-64.697\\
1470	-119.629\\
1471	-102.539\\
1472	-93.994\\
1473	-124.512\\
1474	-145.264\\
1475	-172.119\\
1476	-183.105\\
1477	-258.789\\
1478	-231.934\\
1479	-177.002\\
1480	-120.85\\
1481	-120.85\\
1482	-123.291\\
1483	-157.471\\
1484	-108.643\\
1485	-113.525\\
1486	-107.422\\
1487	-58.594\\
1488	-102.539\\
1489	-152.588\\
1490	-107.422\\
1491	-115.967\\
1492	-216.064\\
1493	-159.912\\
1494	-156.25\\
1495	-219.727\\
1496	-206.299\\
1497	-129.395\\
1498	-85.449\\
1499	-85.449\\
1500	-146.484\\
};
\end{axis}

\begin{axis}[%
width=5.090835cm,
height=2.240107cm,
at={(0cm,7.879947cm)},
scale only axis,
xmin=1000,
xmax=1500,
xlabel={Sample index},
ymin=-28.076,
ymax=0,
ylabel={C5 load, kN},
legend style={legend cell align=left,align=left,draw=white!15!black}
]
\addplot [color=mycolor1,solid,forget plot]
  table[row sep=crcr]{%
1000	-10.986\\
1001	-15.869\\
1002	-10.986\\
1003	-12.207\\
1004	-13.428\\
1005	-15.869\\
1006	-14.648\\
1007	-14.648\\
1008	-7.324\\
1009	-8.545\\
1010	-10.986\\
1011	-9.766\\
1012	-9.766\\
1013	-6.104\\
1014	-2.441\\
1015	-3.662\\
1016	-2.441\\
1017	-12.207\\
1018	-13.428\\
1019	-12.207\\
1020	-10.986\\
1021	-6.104\\
1022	-6.104\\
1023	-1.221\\
1024	-6.104\\
1025	-10.986\\
1026	-12.207\\
1027	-8.545\\
1028	-14.648\\
1029	-14.648\\
1030	-9.766\\
1031	-13.428\\
1032	-12.207\\
1033	-9.766\\
1034	-9.766\\
1035	-8.545\\
1036	-8.545\\
1037	-12.207\\
1038	-10.986\\
1039	-10.986\\
1040	-10.986\\
1041	-10.986\\
1042	-14.648\\
1043	-15.869\\
1044	-9.766\\
1045	-8.545\\
1046	-8.545\\
1047	-6.104\\
1048	-7.324\\
1049	-7.324\\
1050	-4.883\\
1051	-9.766\\
1052	-10.986\\
1053	-8.545\\
1054	-13.428\\
1055	-15.869\\
1056	-8.545\\
1057	-6.104\\
1058	-6.104\\
1059	-9.766\\
1060	-6.104\\
1061	-4.883\\
1062	-7.324\\
1063	-7.324\\
1064	-3.662\\
1065	-6.104\\
1066	-8.545\\
1067	-10.986\\
1068	-10.986\\
1069	-9.766\\
1070	-13.428\\
1071	-10.986\\
1072	-12.207\\
1073	-10.986\\
1074	-10.986\\
1075	-7.324\\
1076	-8.545\\
1077	-14.648\\
1078	-20.752\\
1079	-20.752\\
1080	-19.531\\
1081	-13.428\\
1082	-15.869\\
1083	-24.414\\
1084	-23.193\\
1085	-15.869\\
1086	-21.973\\
1087	-28.076\\
1088	-23.193\\
1089	-17.09\\
1090	-14.648\\
1091	-13.428\\
1092	-9.766\\
1093	-10.986\\
1094	-12.207\\
1095	-4.883\\
1096	-4.883\\
1097	-7.324\\
1098	-10.986\\
1099	-9.766\\
1100	-9.766\\
1101	-10.986\\
1102	-13.428\\
1103	-14.648\\
1104	-15.869\\
1105	-14.648\\
1106	-17.09\\
1107	-12.207\\
1108	-12.207\\
1109	-10.986\\
1110	-8.545\\
1111	-9.766\\
1112	-12.207\\
1113	-8.545\\
1114	-9.766\\
1115	-7.324\\
1116	-6.104\\
1117	-8.545\\
1118	-6.104\\
1119	-4.883\\
1120	-8.545\\
1121	-13.428\\
1122	-12.207\\
1123	-10.986\\
1124	-7.324\\
1125	-8.545\\
1126	-19.531\\
1127	-14.648\\
1128	-10.986\\
1129	-17.09\\
1130	-12.207\\
1131	-6.104\\
1132	-9.766\\
1133	-8.545\\
1134	-14.648\\
1135	-15.869\\
1136	-19.531\\
1137	-14.648\\
1138	-14.648\\
1139	-13.428\\
1140	-13.428\\
1141	-14.648\\
1142	-10.986\\
1143	-7.324\\
1144	-7.324\\
1145	-6.104\\
1146	-8.545\\
1147	-10.986\\
1148	-15.869\\
1149	-14.648\\
1150	-8.545\\
1151	-8.545\\
1152	-9.766\\
1153	-6.104\\
1154	-2.441\\
1155	-4.883\\
1156	-7.324\\
1157	-8.545\\
1158	-6.104\\
1159	-7.324\\
1160	-6.104\\
1161	-4.883\\
1162	-4.883\\
1163	-1.221\\
1164	-3.662\\
1165	-12.207\\
1166	-14.648\\
1167	-15.869\\
1168	-12.207\\
1169	-7.324\\
1170	-6.104\\
1171	-3.662\\
1172	-7.324\\
1173	-8.545\\
1174	-9.766\\
1175	-12.207\\
1176	-18.311\\
1177	-18.311\\
1178	-19.531\\
1179	-18.311\\
1180	-15.869\\
1181	-15.869\\
1182	-17.09\\
1183	-18.311\\
1184	-12.207\\
1185	-10.986\\
1186	-12.207\\
1187	-10.986\\
1188	-12.207\\
1189	-9.766\\
1190	-15.869\\
1191	-18.311\\
1192	-12.207\\
1193	-7.324\\
1194	-9.766\\
1195	-17.09\\
1196	-18.311\\
1197	-18.311\\
1198	-21.973\\
1199	-20.752\\
1200	-17.09\\
1201	-15.869\\
1202	-18.311\\
1203	-20.752\\
1204	-17.09\\
1205	-8.545\\
1206	-14.648\\
1207	-12.207\\
1208	-7.324\\
1209	-10.986\\
1210	-9.766\\
1211	-8.545\\
1212	-7.324\\
1213	-8.545\\
1214	-8.545\\
1215	-9.766\\
1216	-13.428\\
1217	-14.648\\
1218	-7.324\\
1219	-8.545\\
1220	-12.207\\
1221	-17.09\\
1222	-15.869\\
1223	-8.545\\
1224	-8.545\\
1225	-9.766\\
1226	-8.545\\
1227	-7.324\\
1228	-6.104\\
1229	-8.545\\
1230	-9.766\\
1231	-9.766\\
1232	-9.766\\
1233	-12.207\\
1234	-14.648\\
1235	-15.869\\
1236	-10.986\\
1237	-13.428\\
1238	-17.09\\
1239	-12.207\\
1240	-3.662\\
1241	-9.766\\
1242	-8.545\\
1243	-9.766\\
1244	-8.545\\
1245	-8.545\\
1246	-8.545\\
1247	-10.986\\
1248	-7.324\\
1249	-7.324\\
1250	-9.766\\
1251	-6.104\\
1252	-8.545\\
1253	-6.104\\
1254	-3.662\\
1255	-6.104\\
1256	-4.883\\
1257	-10.986\\
1258	-12.207\\
1259	-12.207\\
1260	-18.311\\
1261	-10.986\\
1262	-4.883\\
1263	-6.104\\
1264	-6.104\\
1265	-8.545\\
1266	-6.104\\
1267	-8.545\\
1268	-7.324\\
1269	-13.428\\
1270	-9.766\\
1271	-14.648\\
1272	-10.986\\
1273	-9.766\\
1274	-6.104\\
1275	-3.662\\
1276	-3.662\\
1277	-1.221\\
1278	-2.441\\
1279	-8.545\\
1280	-8.545\\
1281	-7.324\\
1282	-9.766\\
1283	-15.869\\
1284	-10.986\\
1285	-9.766\\
1286	-9.766\\
1287	-13.428\\
1288	-14.648\\
1289	-12.207\\
1290	-12.207\\
1291	-6.104\\
1292	-2.441\\
1293	-4.883\\
1294	-4.883\\
1295	-6.104\\
1296	-3.662\\
1297	-4.883\\
1298	-9.766\\
1299	-7.324\\
1300	-7.324\\
1301	-10.986\\
1302	-7.324\\
1303	-4.883\\
1304	-9.766\\
1305	-12.207\\
1306	-9.766\\
1307	-10.986\\
1308	-7.324\\
1309	-7.324\\
1310	-10.986\\
1311	-14.648\\
1312	-13.428\\
1313	-8.545\\
1314	-14.648\\
1315	-13.428\\
1316	-13.428\\
1317	-13.428\\
1318	-13.428\\
1319	-10.986\\
1320	-10.986\\
1321	-12.207\\
1322	-18.311\\
1323	-17.09\\
1324	-10.986\\
1325	-10.986\\
1326	-10.986\\
1327	-13.428\\
1328	-13.428\\
1329	-9.766\\
1330	-12.207\\
1331	-14.648\\
1332	-14.648\\
1333	-9.766\\
1334	-8.545\\
1335	-10.986\\
1336	-18.311\\
1337	-14.648\\
1338	-13.428\\
1339	-9.766\\
1340	-10.986\\
1341	-9.766\\
1342	-6.104\\
1343	-4.883\\
1344	-3.662\\
1345	-3.662\\
1346	-7.324\\
1347	-10.986\\
1348	-9.766\\
1349	-6.104\\
1350	-7.324\\
1351	-9.766\\
1352	-6.104\\
1353	-6.104\\
1354	-9.766\\
1355	-6.104\\
1356	-7.324\\
1357	-12.207\\
1358	-12.207\\
1359	-8.545\\
1360	-4.883\\
1361	-6.104\\
1362	-7.324\\
1363	-6.104\\
1364	-7.324\\
1365	-14.648\\
1366	-9.766\\
1367	-17.09\\
1368	-20.752\\
1369	-18.311\\
1370	-13.428\\
1371	-10.986\\
1372	-10.986\\
1373	-12.207\\
1374	-13.428\\
1375	-14.648\\
1376	-15.869\\
1377	-20.752\\
1378	-20.752\\
1379	-24.414\\
1380	-15.869\\
1381	-20.752\\
1382	-20.752\\
1383	-14.648\\
1384	-17.09\\
1385	-18.311\\
1386	-9.766\\
1387	-7.324\\
1388	-8.545\\
1389	-9.766\\
1390	-7.324\\
1391	-6.104\\
1392	-7.324\\
1393	-6.104\\
1394	-3.662\\
1395	-4.883\\
1396	-7.324\\
1397	-2.441\\
1398	-10.986\\
1399	-8.545\\
1400	-7.324\\
1401	-15.869\\
1402	-19.531\\
1403	-19.531\\
1404	-19.531\\
1405	-14.648\\
1406	-17.09\\
1407	-12.207\\
1408	-7.324\\
1409	-10.986\\
1410	-7.324\\
1411	-3.662\\
1412	-6.104\\
1413	-4.883\\
1414	-2.441\\
1415	-4.883\\
1416	-10.986\\
1417	-9.766\\
1418	-10.986\\
1419	-10.986\\
1420	-12.207\\
1421	-14.648\\
1422	-10.986\\
1423	-10.986\\
1424	-8.545\\
1425	-7.324\\
1426	-13.428\\
1427	-9.766\\
1428	-6.104\\
1429	-10.986\\
1430	-13.428\\
1431	-10.986\\
1432	-7.324\\
1433	-7.324\\
1434	-7.324\\
1435	-4.883\\
1436	-6.104\\
1437	-7.324\\
1438	-10.986\\
1439	-8.545\\
1440	-7.324\\
1441	-10.986\\
1442	-8.545\\
1443	-6.104\\
1444	-7.324\\
1445	-13.428\\
1446	-14.648\\
1447	-13.428\\
1448	-12.207\\
1449	-12.207\\
1450	-8.545\\
1451	-9.766\\
1452	-8.545\\
1453	-12.207\\
1454	-7.324\\
1455	-4.883\\
1456	-9.766\\
1457	-9.766\\
1458	-10.986\\
1459	-12.207\\
1460	-10.986\\
1461	-17.09\\
1462	-20.752\\
1463	-23.193\\
1464	-15.869\\
1465	-9.766\\
1466	-6.104\\
1467	-6.104\\
1468	-9.766\\
1469	-6.104\\
1470	-9.766\\
1471	-6.104\\
1472	-7.324\\
1473	-8.545\\
1474	-10.986\\
1475	-13.428\\
1476	-14.648\\
1477	-19.531\\
1478	-14.648\\
1479	-13.428\\
1480	-10.986\\
1481	-10.986\\
1482	-10.986\\
1483	-12.207\\
1484	-9.766\\
1485	-10.986\\
1486	-9.766\\
1487	-7.324\\
1488	-12.207\\
1489	-13.428\\
1490	-8.545\\
1491	-9.766\\
1492	-18.311\\
1493	-12.207\\
1494	-12.207\\
1495	-17.09\\
1496	-14.648\\
1497	-10.986\\
1498	-7.324\\
1499	-7.324\\
1500	-12.207\\
};
\end{axis}

\begin{axis}[%
width=5.090835cm,
height=2.240107cm,
at={(0cm,11.81992cm)},
scale only axis,
xmin=1000,
xmax=1500,
xlabel={Sample index},
ymin=-40.283,
ymax=0,
ylabel={C3 load, kN},
legend style={legend cell align=left,align=left,draw=white!15!black}
]
\addplot [color=mycolor1,solid,forget plot]
  table[row sep=crcr]{%
1000	-17.09\\
1001	-19.531\\
1002	-14.648\\
1003	-14.648\\
1004	-19.531\\
1005	-18.311\\
1006	-23.193\\
1007	-18.311\\
1008	-9.766\\
1009	-15.869\\
1010	-12.207\\
1011	-14.648\\
1012	-10.986\\
1013	-3.662\\
1014	-2.441\\
1015	-4.883\\
1016	-6.104\\
1017	-14.648\\
1018	-17.09\\
1019	-17.09\\
1020	-13.428\\
1021	-9.766\\
1022	-18.311\\
1023	-14.648\\
1024	-10.986\\
1025	-13.428\\
1026	-12.207\\
1027	-10.986\\
1028	-20.752\\
1029	-19.531\\
1030	-12.207\\
1031	-20.752\\
1032	-20.752\\
1033	-15.869\\
1034	-13.428\\
1035	-9.766\\
1036	-14.648\\
1037	-14.648\\
1038	-14.648\\
1039	-14.648\\
1040	-14.648\\
1041	-15.869\\
1042	-21.973\\
1043	-20.752\\
1044	-13.428\\
1045	-13.428\\
1046	-8.545\\
1047	-12.207\\
1048	-12.207\\
1049	-13.428\\
1050	-10.986\\
1051	-13.428\\
1052	-14.648\\
1053	-13.428\\
1054	-20.752\\
1055	-13.428\\
1056	-9.766\\
1057	-7.324\\
1058	-8.545\\
1059	-10.986\\
1060	-6.104\\
1061	-9.766\\
1062	-10.986\\
1063	-6.104\\
1064	-6.104\\
1065	-9.766\\
1066	-9.766\\
1067	-15.869\\
1068	-14.648\\
1069	-17.09\\
1070	-15.869\\
1071	-18.311\\
1072	-13.428\\
1073	-14.648\\
1074	-12.207\\
1075	-10.986\\
1076	-12.207\\
1077	-23.193\\
1078	-28.076\\
1079	-30.518\\
1080	-29.297\\
1081	-18.311\\
1082	-28.076\\
1083	-32.959\\
1084	-31.738\\
1085	-20.752\\
1086	-34.18\\
1087	-40.283\\
1088	-29.297\\
1089	-24.414\\
1090	-19.531\\
1091	-15.869\\
1092	-12.207\\
1093	-14.648\\
1094	-9.766\\
1095	-7.324\\
1096	-7.324\\
1097	-10.986\\
1098	-13.428\\
1099	-12.207\\
1100	-12.207\\
1101	-15.869\\
1102	-18.311\\
1103	-19.531\\
1104	-15.869\\
1105	-21.973\\
1106	-15.869\\
1107	-15.869\\
1108	-17.09\\
1109	-12.207\\
1110	-12.207\\
1111	-15.869\\
1112	-14.648\\
1113	-8.545\\
1114	-12.207\\
1115	-8.545\\
1116	-9.766\\
1117	-9.766\\
1118	-7.324\\
1119	-9.766\\
1120	-12.207\\
1121	-17.09\\
1122	-17.09\\
1123	-17.09\\
1124	-10.986\\
1125	-12.207\\
1126	-23.193\\
1127	-15.869\\
1128	-20.752\\
1129	-21.973\\
1130	-12.207\\
1131	-9.766\\
1132	-12.207\\
1133	-13.428\\
1134	-21.973\\
1135	-25.635\\
1136	-26.855\\
1137	-19.531\\
1138	-20.752\\
1139	-18.311\\
1140	-17.09\\
1141	-18.311\\
1142	-13.428\\
1143	-12.207\\
1144	-9.766\\
1145	-9.766\\
1146	-10.986\\
1147	-15.869\\
1148	-23.193\\
1149	-17.09\\
1150	-13.428\\
1151	-10.986\\
1152	-10.986\\
1153	-7.324\\
1154	-6.104\\
1155	-6.104\\
1156	-12.207\\
1157	-7.324\\
1158	-8.545\\
1159	-8.545\\
1160	-8.545\\
1161	-7.324\\
1162	-4.883\\
1163	-3.662\\
1164	-8.545\\
1165	-15.869\\
1166	-18.311\\
1167	-20.752\\
1168	-15.869\\
1169	-10.986\\
1170	-8.545\\
1171	-4.883\\
1172	-9.766\\
1173	-8.545\\
1174	-15.869\\
1175	-17.09\\
1176	-29.297\\
1177	-32.959\\
1178	-25.635\\
1179	-25.635\\
1180	-18.311\\
1181	-23.193\\
1182	-21.973\\
1183	-24.414\\
1184	-15.869\\
1185	-17.09\\
1186	-17.09\\
1187	-15.869\\
1188	-15.869\\
1189	-13.428\\
1190	-23.193\\
1191	-25.635\\
1192	-19.531\\
1193	-13.428\\
1194	-14.648\\
1195	-23.193\\
1196	-24.414\\
1197	-28.076\\
1198	-30.518\\
1199	-29.297\\
1200	-23.193\\
1201	-21.973\\
1202	-25.635\\
1203	-28.076\\
1204	-18.311\\
1205	-12.207\\
1206	-19.531\\
1207	-14.648\\
1208	-9.766\\
1209	-13.428\\
1210	-14.648\\
1211	-10.986\\
1212	-10.986\\
1213	-10.986\\
1214	-8.545\\
1215	-10.986\\
1216	-18.311\\
1217	-17.09\\
1218	-12.207\\
1219	-12.207\\
1220	-18.311\\
1221	-26.855\\
1222	-17.09\\
1223	-14.648\\
1224	-10.986\\
1225	-13.428\\
1226	-10.986\\
1227	-8.545\\
1228	-8.545\\
1229	-10.986\\
1230	-13.428\\
1231	-14.648\\
1232	-12.207\\
1233	-15.869\\
1234	-20.752\\
1235	-17.09\\
1236	-12.207\\
1237	-15.869\\
1238	-20.752\\
1239	-13.428\\
1240	-7.324\\
1241	-13.428\\
1242	-10.986\\
1243	-12.207\\
1244	-9.766\\
1245	-8.545\\
1246	-13.428\\
1247	-13.428\\
1248	-9.766\\
1249	-12.207\\
1250	-9.766\\
1251	-7.324\\
1252	-12.207\\
1253	-8.545\\
1254	-9.766\\
1255	-7.324\\
1256	-4.883\\
1257	-12.207\\
1258	-15.869\\
1259	-17.09\\
1260	-23.193\\
1261	-15.869\\
1262	-9.766\\
1263	-8.545\\
1264	-8.545\\
1265	-10.986\\
1266	-8.545\\
1267	-4.883\\
1268	-10.986\\
1269	-15.869\\
1270	-13.428\\
1271	-20.752\\
1272	-15.869\\
1273	-14.648\\
1274	-8.545\\
1275	-10.986\\
1276	-4.883\\
1277	-3.662\\
1278	-4.883\\
1279	-9.766\\
1280	-10.986\\
1281	-12.207\\
1282	-13.428\\
1283	-20.752\\
1284	-17.09\\
1285	-14.648\\
1286	-17.09\\
1287	-19.531\\
1288	-20.752\\
1289	-17.09\\
1290	-13.428\\
1291	-7.324\\
1292	-6.104\\
1293	-4.883\\
1294	-7.324\\
1295	-7.324\\
1296	-4.883\\
1297	-8.545\\
1298	-12.207\\
1299	-10.986\\
1300	-12.207\\
1301	-12.207\\
1302	-8.545\\
1303	-4.883\\
1304	-9.766\\
1305	-15.869\\
1306	-13.428\\
1307	-15.869\\
1308	-13.428\\
1309	-9.766\\
1310	-14.648\\
1311	-18.311\\
1312	-18.311\\
1313	-13.428\\
1314	-20.752\\
1315	-15.869\\
1316	-17.09\\
1317	-18.311\\
1318	-19.531\\
1319	-13.428\\
1320	-13.428\\
1321	-18.311\\
1322	-26.855\\
1323	-21.973\\
1324	-13.428\\
1325	-13.428\\
1326	-13.428\\
1327	-15.869\\
1328	-17.09\\
1329	-12.207\\
1330	-14.648\\
1331	-18.311\\
1332	-20.752\\
1333	-13.428\\
1334	-12.207\\
1335	-15.869\\
1336	-23.193\\
1337	-20.752\\
1338	-21.973\\
1339	-14.648\\
1340	-14.648\\
1341	-12.207\\
1342	-9.766\\
1343	-4.883\\
1344	-3.662\\
1345	-3.662\\
1346	-7.324\\
1347	-13.428\\
1348	-14.648\\
1349	-9.766\\
1350	-12.207\\
1351	-13.428\\
1352	-9.766\\
1353	-8.545\\
1354	-8.545\\
1355	-6.104\\
1356	-9.766\\
1357	-14.648\\
1358	-15.869\\
1359	-10.986\\
1360	-8.545\\
1361	-8.545\\
1362	-8.545\\
1363	-7.324\\
1364	-10.986\\
1365	-20.752\\
1366	-18.311\\
1367	-18.311\\
1368	-24.414\\
1369	-23.193\\
1370	-18.311\\
1371	-14.648\\
1372	-14.648\\
1373	-14.648\\
1374	-17.09\\
1375	-19.531\\
1376	-19.531\\
1377	-25.635\\
1378	-26.855\\
1379	-31.738\\
1380	-24.414\\
1381	-25.635\\
1382	-26.855\\
1383	-21.973\\
1384	-24.414\\
1385	-20.752\\
1386	-13.428\\
1387	-9.766\\
1388	-9.766\\
1389	-12.207\\
1390	-9.766\\
1391	-7.324\\
1392	-10.986\\
1393	-8.545\\
1394	-6.104\\
1395	-7.324\\
1396	-9.766\\
1397	-6.104\\
1398	-12.207\\
1399	-14.648\\
1400	-10.986\\
1401	-17.09\\
1402	-24.414\\
1403	-24.414\\
1404	-28.076\\
1405	-23.193\\
1406	-21.973\\
1407	-17.09\\
1408	-9.766\\
1409	-10.986\\
1410	-10.986\\
1411	-7.324\\
1412	-10.986\\
1413	-9.766\\
1414	-2.441\\
1415	-6.104\\
1416	-9.766\\
1417	-12.207\\
1418	-14.648\\
1419	-17.09\\
1420	-14.648\\
1421	-17.09\\
1422	-14.648\\
1423	-13.428\\
1424	-12.207\\
1425	-10.986\\
1426	-14.648\\
1427	-13.428\\
1428	-10.986\\
1429	-13.428\\
1430	-18.311\\
1431	-13.428\\
1432	-10.986\\
1433	-10.986\\
1434	-10.986\\
1435	-8.545\\
1436	-7.324\\
1437	-10.986\\
1438	-13.428\\
1439	-13.428\\
1440	-10.986\\
1441	-13.428\\
1442	-13.428\\
1443	-8.545\\
1444	-7.324\\
1445	-15.869\\
1446	-19.531\\
1447	-18.311\\
1448	-18.311\\
1449	-15.869\\
1450	-13.428\\
1451	-10.986\\
1452	-9.766\\
1453	-13.428\\
1454	-13.428\\
1455	-8.545\\
1456	-12.207\\
1457	-14.648\\
1458	-14.648\\
1459	-17.09\\
1460	-17.09\\
1461	-20.752\\
1462	-28.076\\
1463	-30.518\\
1464	-20.752\\
1465	-13.428\\
1466	-8.545\\
1467	-6.104\\
1468	-8.545\\
1469	-7.324\\
1470	-10.986\\
1471	-12.207\\
1472	-12.207\\
1473	-13.428\\
1474	-15.869\\
1475	-19.531\\
1476	-19.531\\
1477	-28.076\\
1478	-23.193\\
1479	-18.311\\
1480	-14.648\\
1481	-14.648\\
1482	-14.648\\
1483	-17.09\\
1484	-14.648\\
1485	-12.207\\
1486	-13.428\\
1487	-8.545\\
1488	-7.324\\
1489	-17.09\\
1490	-14.648\\
1491	-12.207\\
1492	-23.193\\
1493	-18.311\\
1494	-15.869\\
1495	-21.973\\
1496	-23.193\\
1497	-14.648\\
1498	-8.545\\
1499	-9.766\\
1500	-15.869\\
};
\end{axis}

\begin{axis}[%
width=5.090835cm,
height=2.240107cm,
at={(6.698467cm,11.81992cm)},
scale only axis,
xmin=1000,
xmax=1500,
xlabel={Sample index},
ymin=-26.855,
ymax=0,
ylabel={C4 load, kN},
legend style={legend cell align=left,align=left,draw=white!15!black}
]
\addplot [color=mycolor1,solid,forget plot]
  table[row sep=crcr]{%
1000	-10.986\\
1001	-15.869\\
1002	-10.986\\
1003	-10.986\\
1004	-14.648\\
1005	-14.648\\
1006	-17.09\\
1007	-10.986\\
1008	-7.324\\
1009	-14.648\\
1010	-8.545\\
1011	-9.766\\
1012	-6.104\\
1013	-3.662\\
1014	-3.662\\
1015	-3.662\\
1016	-2.441\\
1017	-13.428\\
1018	-13.428\\
1019	-13.428\\
1020	-9.766\\
1021	-7.324\\
1022	-10.986\\
1023	-12.207\\
1024	-6.104\\
1025	-9.766\\
1026	-13.428\\
1027	-7.324\\
1028	-15.869\\
1029	-13.428\\
1030	-10.986\\
1031	-17.09\\
1032	-13.428\\
1033	-10.986\\
1034	-8.545\\
1035	-8.545\\
1036	-10.986\\
1037	-12.207\\
1038	-10.986\\
1039	-9.766\\
1040	-10.986\\
1041	-10.986\\
1042	-15.869\\
1043	-14.648\\
1044	-10.986\\
1045	-10.986\\
1046	-6.104\\
1047	-4.883\\
1048	-8.545\\
1049	-7.324\\
1050	-7.324\\
1051	-10.986\\
1052	-9.766\\
1053	-9.766\\
1054	-15.869\\
1055	-8.545\\
1056	-6.104\\
1057	-7.324\\
1058	-8.545\\
1059	-9.766\\
1060	-4.883\\
1061	-7.324\\
1062	-7.324\\
1063	-7.324\\
1064	-6.104\\
1065	-7.324\\
1066	-8.545\\
1067	-9.766\\
1068	-10.986\\
1069	-10.986\\
1070	-12.207\\
1071	-13.428\\
1072	-10.986\\
1073	-12.207\\
1074	-9.766\\
1075	-9.766\\
1076	-9.766\\
1077	-17.09\\
1078	-20.752\\
1079	-19.531\\
1080	-19.531\\
1081	-15.869\\
1082	-20.752\\
1083	-23.193\\
1084	-23.193\\
1085	-14.648\\
1086	-23.193\\
1087	-26.855\\
1088	-21.973\\
1089	-15.869\\
1090	-14.648\\
1091	-10.986\\
1092	-8.545\\
1093	-10.986\\
1094	-8.545\\
1095	-6.104\\
1096	-6.104\\
1097	-8.545\\
1098	-9.766\\
1099	-10.986\\
1100	-9.766\\
1101	-12.207\\
1102	-10.986\\
1103	-15.869\\
1104	-13.428\\
1105	-18.311\\
1106	-12.207\\
1107	-10.986\\
1108	-12.207\\
1109	-9.766\\
1110	-8.545\\
1111	-10.986\\
1112	-10.986\\
1113	-8.545\\
1114	-8.545\\
1115	-7.324\\
1116	-7.324\\
1117	-7.324\\
1118	-4.883\\
1119	-6.104\\
1120	-8.545\\
1121	-13.428\\
1122	-12.207\\
1123	-13.428\\
1124	-8.545\\
1125	-15.869\\
1126	-17.09\\
1127	-13.428\\
1128	-14.648\\
1129	-14.648\\
1130	-10.986\\
1131	-7.324\\
1132	-8.545\\
1133	-8.545\\
1134	-17.09\\
1135	-20.752\\
1136	-19.531\\
1137	-14.648\\
1138	-15.869\\
1139	-13.428\\
1140	-12.207\\
1141	-12.207\\
1142	-9.766\\
1143	-8.545\\
1144	-8.545\\
1145	-9.766\\
1146	-8.545\\
1147	-13.428\\
1148	-15.869\\
1149	-10.986\\
1150	-8.545\\
1151	-9.766\\
1152	-8.545\\
1153	-6.104\\
1154	-3.662\\
1155	-4.883\\
1156	-8.545\\
1157	-9.766\\
1158	-4.883\\
1159	-7.324\\
1160	-7.324\\
1161	-3.662\\
1162	-3.662\\
1163	-2.441\\
1164	-4.883\\
1165	-10.986\\
1166	-12.207\\
1167	-14.648\\
1168	-13.428\\
1169	-8.545\\
1170	-7.324\\
1171	-4.883\\
1172	-7.324\\
1173	-6.104\\
1174	-10.986\\
1175	-12.207\\
1176	-21.973\\
1177	-23.193\\
1178	-21.973\\
1179	-18.311\\
1180	-13.428\\
1181	-17.09\\
1182	-15.869\\
1183	-19.531\\
1184	-12.207\\
1185	-13.428\\
1186	-12.207\\
1187	-13.428\\
1188	-13.428\\
1189	-10.986\\
1190	-18.311\\
1191	-17.09\\
1192	-13.428\\
1193	-7.324\\
1194	-12.207\\
1195	-15.869\\
1196	-17.09\\
1197	-19.531\\
1198	-21.973\\
1199	-20.752\\
1200	-15.869\\
1201	-14.648\\
1202	-18.311\\
1203	-19.531\\
1204	-13.428\\
1205	-9.766\\
1206	-13.428\\
1207	-9.766\\
1208	-7.324\\
1209	-9.766\\
1210	-10.986\\
1211	-7.324\\
1212	-7.324\\
1213	-9.766\\
1214	-6.104\\
1215	-10.986\\
1216	-13.428\\
1217	-13.428\\
1218	-8.545\\
1219	-9.766\\
1220	-12.207\\
1221	-18.311\\
1222	-15.869\\
1223	-9.766\\
1224	-8.545\\
1225	-9.766\\
1226	-7.324\\
1227	-8.545\\
1228	-7.324\\
1229	-8.545\\
1230	-10.986\\
1231	-10.986\\
1232	-6.104\\
1233	-10.986\\
1234	-13.428\\
1235	-13.428\\
1236	-9.766\\
1237	-12.207\\
1238	-13.428\\
1239	-10.986\\
1240	-4.883\\
1241	-10.986\\
1242	-8.545\\
1243	-8.545\\
1244	-6.104\\
1245	-6.104\\
1246	-10.986\\
1247	-10.986\\
1248	-6.104\\
1249	-8.545\\
1250	-8.545\\
1251	-4.883\\
1252	-8.545\\
1253	-4.883\\
1254	-7.324\\
1255	-4.883\\
1256	-4.883\\
1257	-10.986\\
1258	-12.207\\
1259	-12.207\\
1260	-15.869\\
1261	-10.986\\
1262	-6.104\\
1263	-7.324\\
1264	-4.883\\
1265	-7.324\\
1266	-6.104\\
1267	-3.662\\
1268	-8.545\\
1269	-9.766\\
1270	-9.766\\
1271	-17.09\\
1272	-10.986\\
1273	-13.428\\
1274	-7.324\\
1275	-8.545\\
1276	-3.662\\
1277	-1.221\\
1278	-6.104\\
1279	-9.766\\
1280	-8.545\\
1281	-9.766\\
1282	-9.766\\
1283	-17.09\\
1284	-10.986\\
1285	-10.986\\
1286	-10.986\\
1287	-13.428\\
1288	-15.869\\
1289	-15.869\\
1290	-10.986\\
1291	-6.104\\
1292	-3.662\\
1293	-4.883\\
1294	-6.104\\
1295	-4.883\\
1296	-6.104\\
1297	-6.104\\
1298	-9.766\\
1299	-9.766\\
1300	-8.545\\
1301	-9.766\\
1302	-6.104\\
1303	-3.662\\
1304	-7.324\\
1305	-10.986\\
1306	-8.545\\
1307	-12.207\\
1308	-8.545\\
1309	-7.324\\
1310	-8.545\\
1311	-12.207\\
1312	-14.648\\
1313	-10.986\\
1314	-17.09\\
1315	-9.766\\
1316	-12.207\\
1317	-13.428\\
1318	-13.428\\
1319	-9.766\\
1320	-9.766\\
1321	-12.207\\
1322	-18.311\\
1323	-15.869\\
1324	-8.545\\
1325	-12.207\\
1326	-9.766\\
1327	-12.207\\
1328	-13.428\\
1329	-9.766\\
1330	-9.766\\
1331	-14.648\\
1332	-13.428\\
1333	-12.207\\
1334	-8.545\\
1335	-9.766\\
1336	-15.869\\
1337	-14.648\\
1338	-15.869\\
1339	-12.207\\
1340	-10.986\\
1341	-9.766\\
1342	-8.545\\
1343	-4.883\\
1344	-3.662\\
1345	-3.662\\
1346	-7.324\\
1347	-10.986\\
1348	-9.766\\
1349	-8.545\\
1350	-10.986\\
1351	-9.766\\
1352	-8.545\\
1353	-3.662\\
1354	-6.104\\
1355	-6.104\\
1356	-6.104\\
1357	-8.545\\
1358	-10.986\\
1359	-7.324\\
1360	-6.104\\
1361	-7.324\\
1362	-7.324\\
1363	-6.104\\
1364	-8.545\\
1365	-15.869\\
1366	-14.648\\
1367	-17.09\\
1368	-17.09\\
1369	-17.09\\
1370	-13.428\\
1371	-10.986\\
1372	-9.766\\
1373	-10.986\\
1374	-12.207\\
1375	-14.648\\
1376	-14.648\\
1377	-17.09\\
1378	-19.531\\
1379	-21.973\\
1380	-18.311\\
1381	-21.973\\
1382	-19.531\\
1383	-14.648\\
1384	-17.09\\
1385	-14.648\\
1386	-10.986\\
1387	-8.545\\
1388	-8.545\\
1389	-10.986\\
1390	-7.324\\
1391	-7.324\\
1392	-7.324\\
1393	-7.324\\
1394	-3.662\\
1395	-7.324\\
1396	-7.324\\
1397	-6.104\\
1398	-8.545\\
1399	-12.207\\
1400	-8.545\\
1401	-12.207\\
1402	-18.311\\
1403	-17.09\\
1404	-19.531\\
1405	-17.09\\
1406	-14.648\\
1407	-14.648\\
1408	-7.324\\
1409	-9.766\\
1410	-8.545\\
1411	-6.104\\
1412	-7.324\\
1413	-8.545\\
1414	-2.441\\
1415	-4.883\\
1416	-8.545\\
1417	-9.766\\
1418	-12.207\\
1419	-12.207\\
1420	-10.986\\
1421	-12.207\\
1422	-14.648\\
1423	-10.986\\
1424	-9.766\\
1425	-7.324\\
1426	-10.986\\
1427	-10.986\\
1428	-7.324\\
1429	-10.986\\
1430	-13.428\\
1431	-10.986\\
1432	-8.545\\
1433	-8.545\\
1434	-8.545\\
1435	-6.104\\
1436	-7.324\\
1437	-6.104\\
1438	-9.766\\
1439	-8.545\\
1440	-10.986\\
1441	-10.986\\
1442	-8.545\\
1443	-6.104\\
1444	-6.104\\
1445	-10.986\\
1446	-14.648\\
1447	-14.648\\
1448	-12.207\\
1449	-10.986\\
1450	-9.766\\
1451	-8.545\\
1452	-8.545\\
1453	-9.766\\
1454	-12.207\\
1455	-7.324\\
1456	-7.324\\
1457	-10.986\\
1458	-10.986\\
1459	-13.428\\
1460	-13.428\\
1461	-14.648\\
1462	-20.752\\
1463	-21.973\\
1464	-14.648\\
1465	-9.766\\
1466	-7.324\\
1467	-6.104\\
1468	-7.324\\
1469	-4.883\\
1470	-7.324\\
1471	-7.324\\
1472	-7.324\\
1473	-10.986\\
1474	-12.207\\
1475	-13.428\\
1476	-13.428\\
1477	-18.311\\
1478	-17.09\\
1479	-12.207\\
1480	-12.207\\
1481	-9.766\\
1482	-9.766\\
1483	-13.428\\
1484	-9.766\\
1485	-8.545\\
1486	-10.986\\
1487	-6.104\\
1488	-10.986\\
1489	-14.648\\
1490	-8.545\\
1491	-8.545\\
1492	-15.869\\
1493	-13.428\\
1494	-10.986\\
1495	-17.09\\
1496	-17.09\\
1497	-10.986\\
1498	-8.545\\
1499	-9.766\\
1500	-12.207\\
};
\end{axis}

\begin{axis}[%
width=5.090835cm,
height=2.240107cm,
at={(6.698467cm,3.939973cm)},
scale only axis,
xmin=1000,
xmax=1500,
xlabel={Sample index},
ymin=-279.541,
ymax=0,
ylabel={C8 load, kN},
legend style={legend cell align=left,align=left,draw=white!15!black}
]
\addplot [color=mycolor1,solid,forget plot]
  table[row sep=crcr]{%
1000	-96.436\\
1001	-122.07\\
1002	-100.098\\
1003	-93.994\\
1004	-130.615\\
1005	-125.732\\
1006	-153.809\\
1007	-115.967\\
1008	-61.035\\
1009	-74.463\\
1010	-73.242\\
1011	-86.67\\
1012	-73.242\\
1013	-34.18\\
1014	-20.752\\
1015	-23.193\\
1016	-58.594\\
1017	-100.098\\
1018	-109.863\\
1019	-114.746\\
1020	-83.008\\
1021	-52.49\\
1022	-104.98\\
1023	-81.787\\
1024	-59.814\\
1025	-97.656\\
1026	-83.008\\
1027	-72.021\\
1028	-118.408\\
1029	-107.422\\
1030	-83.008\\
1031	-119.629\\
1032	-123.291\\
1033	-95.215\\
1034	-80.566\\
1035	-62.256\\
1036	-75.684\\
1037	-95.215\\
1038	-87.891\\
1039	-91.553\\
1040	-96.436\\
1041	-102.539\\
1042	-141.602\\
1043	-128.174\\
1044	-85.449\\
1045	-79.346\\
1046	-47.607\\
1047	-48.828\\
1048	-68.359\\
1049	-52.49\\
1050	-57.373\\
1051	-75.684\\
1052	-89.111\\
1053	-81.787\\
1054	-128.174\\
1055	-98.877\\
1056	-57.373\\
1057	-45.166\\
1058	-62.256\\
1059	-72.021\\
1060	-40.283\\
1061	-42.725\\
1062	-56.152\\
1063	-46.387\\
1064	-40.283\\
1065	-52.49\\
1066	-62.256\\
1067	-92.773\\
1068	-90.332\\
1069	-107.422\\
1070	-98.877\\
1071	-115.967\\
1072	-91.553\\
1073	-91.553\\
1074	-79.346\\
1075	-75.684\\
1076	-76.904\\
1077	-133.057\\
1078	-189.209\\
1079	-194.092\\
1080	-189.209\\
1081	-119.629\\
1082	-170.898\\
1083	-213.623\\
1084	-219.727\\
1085	-169.678\\
1086	-219.727\\
1087	-279.541\\
1088	-197.754\\
1089	-161.133\\
1090	-109.863\\
1091	-85.449\\
1092	-73.242\\
1093	-91.553\\
1094	-62.256\\
1095	-46.387\\
1096	-42.725\\
1097	-54.932\\
1098	-79.346\\
1099	-74.463\\
1100	-75.684\\
1101	-100.098\\
1102	-103.76\\
1103	-141.602\\
1104	-114.746\\
1105	-142.822\\
1106	-104.98\\
1107	-90.332\\
1108	-103.76\\
1109	-76.904\\
1110	-69.58\\
1111	-84.229\\
1112	-86.67\\
1113	-59.814\\
1114	-64.697\\
1115	-48.828\\
1116	-53.711\\
1117	-57.373\\
1118	-41.504\\
1119	-52.49\\
1120	-65.918\\
1121	-101.318\\
1122	-104.98\\
1123	-114.746\\
1124	-68.359\\
1125	-93.994\\
1126	-150.146\\
1127	-114.746\\
1128	-125.732\\
1129	-128.174\\
1130	-80.566\\
1131	-53.711\\
1132	-75.684\\
1133	-85.449\\
1134	-137.939\\
1135	-172.119\\
1136	-170.898\\
1137	-123.291\\
1138	-130.615\\
1139	-115.967\\
1140	-113.525\\
1141	-111.084\\
1142	-76.904\\
1143	-68.359\\
1144	-61.035\\
1145	-57.373\\
1146	-62.256\\
1147	-91.553\\
1148	-140.381\\
1149	-111.084\\
1150	-79.346\\
1151	-64.697\\
1152	-62.256\\
1153	-40.283\\
1154	-29.297\\
1155	-36.621\\
1156	-63.477\\
1157	-51.27\\
1158	-52.49\\
1159	-58.594\\
1160	-54.932\\
1161	-37.842\\
1162	-28.076\\
1163	-23.193\\
1164	-39.063\\
1165	-83.008\\
1166	-117.188\\
1167	-130.615\\
1168	-86.67\\
1169	-58.594\\
1170	-45.166\\
1171	-32.959\\
1172	-67.139\\
1173	-65.918\\
1174	-91.553\\
1175	-117.188\\
1176	-186.768\\
1177	-216.064\\
1178	-177.002\\
1179	-168.457\\
1180	-109.863\\
1181	-122.07\\
1182	-131.836\\
1183	-146.484\\
1184	-108.643\\
1185	-100.098\\
1186	-98.877\\
1187	-89.111\\
1188	-106.201\\
1189	-78.125\\
1190	-130.615\\
1191	-159.912\\
1192	-113.525\\
1193	-70.801\\
1194	-73.242\\
1195	-123.291\\
1196	-153.809\\
1197	-189.209\\
1198	-202.637\\
1199	-202.637\\
1200	-153.809\\
1201	-137.939\\
1202	-158.691\\
1203	-187.988\\
1204	-109.863\\
1205	-72.021\\
1206	-101.318\\
1207	-86.67\\
1208	-53.711\\
1209	-74.463\\
1210	-84.229\\
1211	-67.139\\
1212	-51.27\\
1213	-70.801\\
1214	-58.594\\
1215	-79.346\\
1216	-107.422\\
1217	-100.098\\
1218	-69.58\\
1219	-70.801\\
1220	-107.422\\
1221	-177.002\\
1222	-128.174\\
1223	-79.346\\
1224	-61.035\\
1225	-70.801\\
1226	-64.697\\
1227	-48.828\\
1228	-53.711\\
1229	-65.918\\
1230	-83.008\\
1231	-91.553\\
1232	-63.477\\
1233	-104.98\\
1234	-124.512\\
1235	-118.408\\
1236	-91.553\\
1237	-100.098\\
1238	-135.498\\
1239	-87.891\\
1240	-42.725\\
1241	-57.373\\
1242	-62.256\\
1243	-81.787\\
1244	-72.021\\
1245	-52.49\\
1246	-79.346\\
1247	-80.566\\
1248	-57.373\\
1249	-70.801\\
1250	-63.477\\
1251	-56.152\\
1252	-67.139\\
1253	-41.504\\
1254	-48.828\\
1255	-36.621\\
1256	-40.283\\
1257	-75.684\\
1258	-84.229\\
1259	-113.525\\
1260	-146.484\\
1261	-98.877\\
1262	-58.594\\
1263	-46.387\\
1264	-43.945\\
1265	-63.477\\
1266	-42.725\\
1267	-47.607\\
1268	-61.035\\
1269	-102.539\\
1270	-93.994\\
1271	-134.277\\
1272	-89.111\\
1273	-81.787\\
1274	-46.387\\
1275	-45.166\\
1276	-28.076\\
1277	-35.4\\
1278	-37.842\\
1279	-67.139\\
1280	-70.801\\
1281	-79.346\\
1282	-85.449\\
1283	-125.732\\
1284	-112.305\\
1285	-84.229\\
1286	-102.539\\
1287	-109.863\\
1288	-144.043\\
1289	-115.967\\
1290	-75.684\\
1291	-40.283\\
1292	-29.297\\
1293	-29.297\\
1294	-36.621\\
1295	-39.063\\
1296	-37.842\\
1297	-50.049\\
1298	-74.463\\
1299	-68.359\\
1300	-70.801\\
1301	-83.008\\
1302	-51.27\\
1303	-30.518\\
1304	-58.594\\
1305	-90.332\\
1306	-87.891\\
1307	-97.656\\
1308	-83.008\\
1309	-61.035\\
1310	-85.449\\
1311	-118.408\\
1312	-118.408\\
1313	-84.229\\
1314	-136.719\\
1315	-100.098\\
1316	-101.318\\
1317	-117.188\\
1318	-115.967\\
1319	-86.67\\
1320	-76.904\\
1321	-128.174\\
1322	-178.223\\
1323	-139.16\\
1324	-79.346\\
1325	-76.904\\
1326	-79.346\\
1327	-98.877\\
1328	-114.746\\
1329	-85.449\\
1330	-90.332\\
1331	-120.85\\
1332	-131.836\\
1333	-87.891\\
1334	-68.359\\
1335	-91.553\\
1336	-156.25\\
1337	-137.939\\
1338	-140.381\\
1339	-84.229\\
1340	-76.904\\
1341	-72.021\\
1342	-47.607\\
1343	-30.518\\
1344	-26.855\\
1345	-23.193\\
1346	-54.932\\
1347	-70.801\\
1348	-87.891\\
1349	-65.918\\
1350	-73.242\\
1351	-83.008\\
1352	-53.711\\
1353	-56.152\\
1354	-53.711\\
1355	-40.283\\
1356	-51.27\\
1357	-84.229\\
1358	-106.201\\
1359	-63.477\\
1360	-48.828\\
1361	-40.283\\
1362	-50.049\\
1363	-35.4\\
1364	-76.904\\
1365	-130.615\\
1366	-104.98\\
1367	-139.16\\
1368	-162.354\\
1369	-164.795\\
1370	-124.512\\
1371	-90.332\\
1372	-89.111\\
1373	-89.111\\
1374	-106.201\\
1375	-125.732\\
1376	-125.732\\
1377	-168.457\\
1378	-183.105\\
1379	-219.727\\
1380	-147.705\\
1381	-166.016\\
1382	-184.326\\
1383	-140.381\\
1384	-162.354\\
1385	-139.16\\
1386	-80.566\\
1387	-52.49\\
1388	-56.152\\
1389	-81.787\\
1390	-65.918\\
1391	-43.945\\
1392	-62.256\\
1393	-47.607\\
1394	-36.621\\
1395	-46.387\\
1396	-52.49\\
1397	-36.621\\
1398	-70.801\\
1399	-92.773\\
1400	-68.359\\
1401	-114.746\\
1402	-164.795\\
1403	-150.146\\
1404	-196.533\\
1405	-131.836\\
1406	-144.043\\
1407	-96.436\\
1408	-63.477\\
1409	-76.904\\
1410	-59.814\\
1411	-45.166\\
1412	-64.697\\
1413	-36.621\\
1414	-28.076\\
1415	-34.18\\
1416	-70.801\\
1417	-86.67\\
1418	-107.422\\
1419	-103.76\\
1420	-100.098\\
1421	-114.746\\
1422	-93.994\\
1423	-76.904\\
1424	-76.904\\
1425	-62.256\\
1426	-97.656\\
1427	-68.359\\
1428	-56.152\\
1429	-81.787\\
1430	-113.525\\
1431	-86.67\\
1432	-65.918\\
1433	-56.152\\
1434	-65.918\\
1435	-45.166\\
1436	-45.166\\
1437	-59.814\\
1438	-75.684\\
1439	-84.229\\
1440	-65.918\\
1441	-86.67\\
1442	-79.346\\
1443	-47.607\\
1444	-50.049\\
1445	-95.215\\
1446	-131.836\\
1447	-131.836\\
1448	-107.422\\
1449	-103.76\\
1450	-79.346\\
1451	-76.904\\
1452	-61.035\\
1453	-89.111\\
1454	-75.684\\
1455	-50.049\\
1456	-74.463\\
1457	-85.449\\
1458	-101.318\\
1459	-109.863\\
1460	-109.863\\
1461	-148.926\\
1462	-181.885\\
1463	-205.078\\
1464	-129.395\\
1465	-73.242\\
1466	-47.607\\
1467	-34.18\\
1468	-54.932\\
1469	-46.387\\
1470	-80.566\\
1471	-72.021\\
1472	-64.697\\
1473	-85.449\\
1474	-98.877\\
1475	-117.188\\
1476	-123.291\\
1477	-175.781\\
1478	-150.146\\
1479	-117.188\\
1480	-80.566\\
1481	-80.566\\
1482	-83.008\\
1483	-106.201\\
1484	-75.684\\
1485	-79.346\\
1486	-74.463\\
1487	-42.725\\
1488	-74.463\\
1489	-107.422\\
1490	-73.242\\
1491	-80.566\\
1492	-147.705\\
1493	-109.863\\
1494	-106.201\\
1495	-147.705\\
1496	-140.381\\
1497	-87.891\\
1498	-56.152\\
1499	-58.594\\
1500	-97.656\\
};
\end{axis}

\begin{axis}[%
width=5.090835cm,
height=2.240107cm,
at={(6.698467cm,0cm)},
scale only axis,
xmin=1000,
xmax=1500,
xlabel={Sample index},
ymin=-158.691,
ymax=-17.09,
ylabel={C10 load, kN},
legend style={legend cell align=left,align=left,draw=white!15!black}
]
\addplot [color=mycolor1,solid,forget plot]
  table[row sep=crcr]{%
1000	-63.477\\
1001	-75.684\\
1002	-62.256\\
1003	-59.814\\
1004	-80.566\\
1005	-76.904\\
1006	-90.332\\
1007	-68.359\\
1008	-40.283\\
1009	-48.828\\
1010	-46.387\\
1011	-54.932\\
1012	-45.166\\
1013	-21.973\\
1014	-17.09\\
1015	-18.311\\
1016	-40.283\\
1017	-62.256\\
1018	-67.139\\
1019	-69.58\\
1020	-52.49\\
1021	-34.18\\
1022	-64.697\\
1023	-52.49\\
1024	-39.063\\
1025	-61.035\\
1026	-53.711\\
1027	-45.166\\
1028	-73.242\\
1029	-67.139\\
1030	-52.49\\
1031	-75.684\\
1032	-74.463\\
1033	-58.594\\
1034	-50.049\\
1035	-41.504\\
1036	-47.607\\
1037	-58.594\\
1038	-54.932\\
1039	-59.814\\
1040	-59.814\\
1041	-65.918\\
1042	-85.449\\
1043	-76.904\\
1044	-54.932\\
1045	-50.049\\
1046	-35.4\\
1047	-34.18\\
1048	-45.166\\
1049	-34.18\\
1050	-36.621\\
1051	-51.27\\
1052	-54.932\\
1053	-51.27\\
1054	-78.125\\
1055	-62.256\\
1056	-36.621\\
1057	-30.518\\
1058	-40.283\\
1059	-46.387\\
1060	-29.297\\
1061	-29.297\\
1062	-36.621\\
1063	-31.738\\
1064	-26.855\\
1065	-35.4\\
1066	-40.283\\
1067	-58.594\\
1068	-56.152\\
1069	-65.918\\
1070	-61.035\\
1071	-70.801\\
1072	-57.373\\
1073	-58.594\\
1074	-51.27\\
1075	-50.049\\
1076	-48.828\\
1077	-80.566\\
1078	-111.084\\
1079	-114.746\\
1080	-112.305\\
1081	-74.463\\
1082	-100.098\\
1083	-125.732\\
1084	-128.174\\
1085	-96.436\\
1086	-124.512\\
1087	-158.691\\
1088	-115.967\\
1089	-97.656\\
1090	-68.359\\
1091	-53.711\\
1092	-47.607\\
1093	-56.152\\
1094	-45.166\\
1095	-30.518\\
1096	-30.518\\
1097	-36.621\\
1098	-50.049\\
1099	-48.828\\
1100	-47.607\\
1101	-62.256\\
1102	-64.697\\
1103	-85.449\\
1104	-72.021\\
1105	-85.449\\
1106	-63.477\\
1107	-54.932\\
1108	-64.697\\
1109	-48.828\\
1110	-43.945\\
1111	-53.711\\
1112	-54.932\\
1113	-40.283\\
1114	-43.945\\
1115	-32.959\\
1116	-36.621\\
1117	-40.283\\
1118	-30.518\\
1119	-34.18\\
1120	-43.945\\
1121	-63.477\\
1122	-65.918\\
1123	-72.021\\
1124	-45.166\\
1125	-56.152\\
1126	-89.111\\
1127	-70.801\\
1128	-81.787\\
1129	-80.566\\
1130	-50.049\\
1131	-36.621\\
1132	-47.607\\
1133	-52.49\\
1134	-81.787\\
1135	-103.76\\
1136	-102.539\\
1137	-76.904\\
1138	-80.566\\
1139	-72.021\\
1140	-69.58\\
1141	-68.359\\
1142	-51.27\\
1143	-45.166\\
1144	-39.063\\
1145	-37.842\\
1146	-41.504\\
1147	-56.152\\
1148	-84.229\\
1149	-70.801\\
1150	-50.049\\
1151	-42.725\\
1152	-40.283\\
1153	-28.076\\
1154	-23.193\\
1155	-24.414\\
1156	-41.504\\
1157	-32.959\\
1158	-34.18\\
1159	-37.842\\
1160	-35.4\\
1161	-26.855\\
1162	-21.973\\
1163	-18.311\\
1164	-25.635\\
1165	-51.27\\
1166	-73.242\\
1167	-78.125\\
1168	-54.932\\
1169	-39.063\\
1170	-31.738\\
1171	-24.414\\
1172	-41.504\\
1173	-42.725\\
1174	-54.932\\
1175	-73.242\\
1176	-108.643\\
1177	-125.732\\
1178	-103.76\\
1179	-101.318\\
1180	-67.139\\
1181	-75.684\\
1182	-79.346\\
1183	-86.67\\
1184	-69.58\\
1185	-62.256\\
1186	-61.035\\
1187	-56.152\\
1188	-67.139\\
1189	-52.49\\
1190	-76.904\\
1191	-97.656\\
1192	-72.021\\
1193	-45.166\\
1194	-47.607\\
1195	-78.125\\
1196	-93.994\\
1197	-112.305\\
1198	-119.629\\
1199	-119.629\\
1200	-91.553\\
1201	-84.229\\
1202	-96.436\\
1203	-111.084\\
1204	-74.463\\
1205	-46.387\\
1206	-62.256\\
1207	-54.932\\
1208	-34.18\\
1209	-47.607\\
1210	-53.711\\
1211	-42.725\\
1212	-34.18\\
1213	-43.945\\
1214	-40.283\\
1215	-52.49\\
1216	-65.918\\
1217	-62.256\\
1218	-45.166\\
1219	-45.166\\
1220	-67.139\\
1221	-104.98\\
1222	-79.346\\
1223	-51.27\\
1224	-40.283\\
1225	-47.607\\
1226	-40.283\\
1227	-31.738\\
1228	-35.4\\
1229	-42.725\\
1230	-51.27\\
1231	-57.373\\
1232	-42.725\\
1233	-63.477\\
1234	-75.684\\
1235	-70.801\\
1236	-57.373\\
1237	-62.256\\
1238	-81.787\\
1239	-52.49\\
1240	-28.076\\
1241	-36.621\\
1242	-42.725\\
1243	-51.27\\
1244	-46.387\\
1245	-36.621\\
1246	-52.49\\
1247	-52.49\\
1248	-37.842\\
1249	-47.607\\
1250	-41.504\\
1251	-37.842\\
1252	-45.166\\
1253	-29.297\\
1254	-31.738\\
1255	-29.297\\
1256	-26.855\\
1257	-47.607\\
1258	-53.711\\
1259	-69.58\\
1260	-89.111\\
1261	-62.256\\
1262	-36.621\\
1263	-31.738\\
1264	-29.297\\
1265	-40.283\\
1266	-31.738\\
1267	-30.518\\
1268	-39.063\\
1269	-61.035\\
1270	-53.711\\
1271	-79.346\\
1272	-54.932\\
1273	-48.828\\
1274	-32.959\\
1275	-29.297\\
1276	-23.193\\
1277	-23.193\\
1278	-25.635\\
1279	-40.283\\
1280	-46.387\\
1281	-48.828\\
1282	-51.27\\
1283	-79.346\\
1284	-73.242\\
1285	-53.711\\
1286	-63.477\\
1287	-69.58\\
1288	-85.449\\
1289	-72.021\\
1290	-47.607\\
1291	-28.076\\
1292	-21.973\\
1293	-21.973\\
1294	-26.855\\
1295	-26.855\\
1296	-25.635\\
1297	-32.959\\
1298	-47.607\\
1299	-43.945\\
1300	-45.166\\
1301	-52.49\\
1302	-35.4\\
1303	-20.752\\
1304	-35.4\\
1305	-56.152\\
1306	-54.932\\
1307	-59.814\\
1308	-53.711\\
1309	-40.283\\
1310	-51.27\\
1311	-72.021\\
1312	-70.801\\
1313	-51.27\\
1314	-81.787\\
1315	-62.256\\
1316	-63.477\\
1317	-73.242\\
1318	-70.801\\
1319	-54.932\\
1320	-47.607\\
1321	-78.125\\
1322	-108.643\\
1323	-85.449\\
1324	-50.049\\
1325	-51.27\\
1326	-51.27\\
1327	-61.035\\
1328	-72.021\\
1329	-54.932\\
1330	-56.152\\
1331	-73.242\\
1332	-79.346\\
1333	-56.152\\
1334	-45.166\\
1335	-57.373\\
1336	-87.891\\
1337	-78.125\\
1338	-80.566\\
1339	-54.932\\
1340	-45.166\\
1341	-47.607\\
1342	-31.738\\
1343	-20.752\\
1344	-20.752\\
1345	-17.09\\
1346	-31.738\\
1347	-48.828\\
1348	-54.932\\
1349	-40.283\\
1350	-45.166\\
1351	-52.49\\
1352	-35.4\\
1353	-35.4\\
1354	-35.4\\
1355	-29.297\\
1356	-31.738\\
1357	-51.27\\
1358	-64.697\\
1359	-41.504\\
1360	-32.959\\
1361	-28.076\\
1362	-32.959\\
1363	-26.855\\
1364	-45.166\\
1365	-80.566\\
1366	-63.477\\
1367	-81.787\\
1368	-98.877\\
1369	-97.656\\
1370	-76.904\\
1371	-58.594\\
1372	-54.932\\
1373	-57.373\\
1374	-65.918\\
1375	-76.904\\
1376	-76.904\\
1377	-100.098\\
1378	-108.643\\
1379	-130.615\\
1380	-91.553\\
1381	-98.877\\
1382	-108.643\\
1383	-85.449\\
1384	-96.436\\
1385	-85.449\\
1386	-51.27\\
1387	-34.18\\
1388	-36.621\\
1389	-51.27\\
1390	-43.945\\
1391	-31.738\\
1392	-41.504\\
1393	-32.959\\
1394	-23.193\\
1395	-31.738\\
1396	-34.18\\
1397	-25.635\\
1398	-42.725\\
1399	-58.594\\
1400	-45.166\\
1401	-68.359\\
1402	-100.098\\
1403	-87.891\\
1404	-109.863\\
1405	-81.787\\
1406	-81.787\\
1407	-62.256\\
1408	-37.842\\
1409	-46.387\\
1410	-42.725\\
1411	-29.297\\
1412	-40.283\\
1413	-21.973\\
1414	-19.531\\
1415	-24.414\\
1416	-45.166\\
1417	-57.373\\
1418	-65.918\\
1419	-65.918\\
1420	-62.256\\
1421	-69.58\\
1422	-57.373\\
1423	-48.828\\
1424	-50.049\\
1425	-43.945\\
1426	-59.814\\
1427	-46.387\\
1428	-36.621\\
1429	-51.27\\
1430	-69.58\\
1431	-54.932\\
1432	-41.504\\
1433	-37.842\\
1434	-41.504\\
1435	-32.959\\
1436	-30.518\\
1437	-41.504\\
1438	-47.607\\
1439	-52.49\\
1440	-42.725\\
1441	-53.711\\
1442	-51.27\\
1443	-32.959\\
1444	-32.959\\
1445	-58.594\\
1446	-80.566\\
1447	-78.125\\
1448	-67.139\\
1449	-62.256\\
1450	-50.049\\
1451	-48.828\\
1452	-41.504\\
1453	-54.932\\
1454	-50.049\\
1455	-32.959\\
1456	-45.166\\
1457	-52.49\\
1458	-61.035\\
1459	-68.359\\
1460	-68.359\\
1461	-87.891\\
1462	-107.422\\
1463	-119.629\\
1464	-80.566\\
1465	-46.387\\
1466	-32.959\\
1467	-24.414\\
1468	-34.18\\
1469	-29.297\\
1470	-51.27\\
1471	-48.828\\
1472	-40.283\\
1473	-52.49\\
1474	-63.477\\
1475	-70.801\\
1476	-75.684\\
1477	-103.76\\
1478	-92.773\\
1479	-72.021\\
1480	-51.27\\
1481	-52.49\\
1482	-54.932\\
1483	-65.918\\
1484	-51.27\\
1485	-50.049\\
1486	-48.828\\
1487	-30.518\\
1488	-45.166\\
1489	-68.359\\
1490	-48.828\\
1491	-52.49\\
1492	-86.67\\
1493	-63.477\\
1494	-63.477\\
1495	-85.449\\
1496	-81.787\\
1497	-53.711\\
1498	-35.4\\
1499	-36.621\\
1500	-61.035\\
};
\end{axis}
\end{tikzpicture}%
	\caption{Experimental data.}
\end{figure}
\section{Structure identification}
\par The following model structure is assumed. The output of the NARX model $\mathbfit{y}(t)$ is the measured load. The input vector is composed as
\begin{equation}
	\mathbfit{x}(t) = \{x_i(t)\}^{d}_{i=1} = \left[\{y(t - k + 1)\}^{n_y}_{k=1} \quad \{u(t - k + n_y + 1)\}^{n_y + n_u}_{k= n_y + 1} \right]^{\top},
\end{equation}
where $n_u$ is the length of the input lag and $n_y$ is the length of the output lag in discrete time, and where $d = n_u + n_y$. In this case, the identification is performed under the following assumptions:
\begin{itemize}[noitemsep,topsep=0.3pt,parsep=0.3pt,partopsep=0.2pt,labelindent=1cm] 
	\item only the input signal affects the output ($n_y = 0$).
	\item the input signal has a lag of length $n_u = 4$.
\end{itemize}
The unknown model is approximated with a sum of polynomial basis functions up to second degree ($\lambda = 2$), rendering the following structure
\begin{equation}
	\mathbfit{y}(t) = \theta^0 + \sum_{i=1}^{d} \theta_i x_i(t) + \sum_{i=1}^{d} \sum_{j=1}^{d} \theta_{i,j} x_i(t) x_j(t) + e(t).
\end{equation}
The number and order of significant terms are identified within the EFOR-CMSS algorithm based on the data from 8 out of 10 datasets. Figure \ref{fig:aamdl} illustrates the relationship between the number of model terms and the selected criterion of significance, AAMDL.
\begin{figure}[!h]
	\centering
	\subfloat[Sample size 2000.]{% This file was created by matlab2tikz.
% Minimal pgfplots version: 1.3
%
\definecolor{mycolor1}{rgb}{0.00000,0.44700,0.74100}%
\definecolor{mycolor2}{rgb}{0.85000,0.32500,0.09800}%
%
\begin{tikzpicture}

\begin{axis}[%
width=6cm,
height=6cm,
at={(0cm,0cm)},
scale only axis,
xmin=1,
xmax=15,
xlabel={Number of terms},
ymin=-2.21881347579609,
ymax=-1.56472398176224,
ylabel={AAMDL},
legend style={legend cell align=left,align=left,draw=white!15!black}
]
\addplot [color=mycolor1,only marks,mark=o,mark options={solid},forget plot]
  table[row sep=crcr]{%
1	-1.56472398176224\\
2	-1.97701270064064\\
3	-2.12029297016719\\
4	-2.18293755649416\\
5	-2.19697120568313\\
6	-2.19619682060115\\
7	-2.20287253171251\\
8	-2.21881347579609\\
9	-2.21849462617766\\
10	-2.21775809851537\\
11	-2.21624604641005\\
12	-2.21472746636226\\
13	-2.21467778403799\\
14	-2.2138532692636\\
15	-2.21310607093081\\
};
\addplot [color=mycolor2,line width=5.0pt,only marks,mark=asterisk,mark options={solid},forget plot]
  table[row sep=crcr]
	\subfloat[Sample size 4000.]{% This file was created by matlab2tikz.
% Minimal pgfplots version: 1.3
%
\definecolor{mycolor1}{rgb}{0.00000,0.44700,0.74100}%
\definecolor{mycolor2}{rgb}{0.85000,0.32500,0.09800}%
%
\begin{tikzpicture}

\begin{axis}[%
width=5.979382cm,
height=6cm,
at={(0cm,0cm)},
scale only axis,
xmin=1,
xmax=15,
xlabel={Number of terms},
ymin=-2.8,
ymax=-1.8,
ylabel={AAMDL},
legend style={legend cell align=left,align=left,draw=white!15!black}
]
\addplot [color=mycolor1,only marks,mark=o,mark options={solid},forget plot]
  table[row sep=crcr]{%
1	-1.85238639353984\\
2	-2.51661709688035\\
3	-2.54325120703526\\
4	-2.55128539901149\\
5	-2.55878900600603\\
6	-2.55830335858106\\
7	-2.65674313484709\\
8	-2.69442575885501\\
9	-2.69503767764109\\
10	-2.69462917143197\\
11	-2.69378515117799\\
12	-2.69252058304669\\
13	-2.69122725766994\\
14	-2.68930444709195\\
15	-2.68735035007295\\
};
\addplot [color=mycolor2,line width=5.0pt,only marks,mark=asterisk,mark options={solid},forget plot]
  table[row sep=crcr]
	\caption{AAMDL evolution with the growing number of terms for samples of different size.}\label{fig:aamdl}
\end{figure}	
\section{Parameter estimation}
\par	Results of internal parameter estimation via EFOR-CMSS for different sample sizes are presented in Tables 2 and 3
\begin{table}[!h]
	\centering
	\caption{Estimated parameters for the sample length 2000.}
	\small
	\begin{tabular}{rrrrrrrrrrr}
Step & Terms & C1 & C2 & C4 & C5 & C6 & C7 & C9 & C10 & AEER($\%$) \\ 
\hline 
1 & $x_4,x_4$ & -26.04 & -20.99 & -10.69 & -10.96 & -191.78 & -157.64 & -87.42 & -69.8 & 89.511 \\ 
2 & $x_3$ & 75.42 & 59.58 & 33 & 26.06 & 508.94 & 419.54 & 242.35 & 195.15 & 8.849 \\ 
3 & $x_1,x_4$ & 0.62 & 0.76 & 0.32 & 0.48 & 8.55 & 7.83 & 2.66 & 1.15 & 0.139 \\ 
4 & $x_1,x_1$ & 0.01 & -0.19 & -0.15 & -0.22 & 0.05 & -0.48 & 0.44 & 0.76 & 0.045 \\ 
5 & $x_2$ & 0.71 & -0.73 & -2.24 & -0.66 & 45.94 & 36.57 & 18.4 & 12.68 & 0.032 \\ 
6 & $x_4$ & -171.24 & -139.22 & -69.61 & -73.69 & -1273.02 & -1046.38 & -579.72 & -465.59 & 0.006 \\ 
7 & $c$ & -233.16 & -200.83 & -93.74 & -119.7 & -1805.9 & -1488.55 & -803.7 & -648.8 & 0.308 \\ 
8 & $x_3,x_4$ & 15.47 & 12.1 & 6.36 & 5.68 & 110.13 & 90.43 & 51.77 & 41.43 & 0.093 \\ 
\hline 
\end{tabular}
\end{table}
\begin{table}[!h]
	\centering
	\caption{Estimated parameters for the sample length 4000.}
	\small
	\begin{tabular}{llllllllllllll}
Step & Terms & C1 & C2 & C3 & C4 & C5 & C6 & C7 & C8 & C9 & C10 & AEER & AAMDL \\ 
\hline 
1 & $c$ & -165.57 & -140.83 & -106.8 & -57.88 & -93.13 & -1037.61 & -964.13 & -843.17 & -559.51 & -370.04 & 0 & -1.665 \\ 
2 & $x_4$ & -156.14 & -123.14 & -93.42 & -60.44 & -67.23 & -1089.72 & -931.89 & -801 & -530.57 & -398.02 & 0 & -2.209 \\ 
3 & $x_3$ & 77.52 & 56.15 & 39.44 & 33.75 & 22.39 & 466.78 & 421.82 & 365.29 & 244.52 & 171.74 & 0 & -2.389 \\ 
4 & $x_4,x_4$ & -19.62 & -12.42 & -9.15 & -4.16 & -7.3 & -103.98 & -95.74 & -83.16 & -55.36 & -35.81 & 0 & -2.477 \\ 
5 & $x_3,x_4$ & -0.61 & -10.91 & -10.1 & -11.57 & -7.03 & -126.87 & -80.74 & -60.96 & -39.74 & -50.79 & 0 & -2.499 \\ 
6 & $x_3,x_3$ & 16.49 & 22.71 & 16.85 & 13.74 & 12.56 & 259.29 & 184.73 & 152.18 & 99.4 & 95.7 & 0 & -2.499 \\ 
7 & $x_2$ & -19.39 & -14.83 & -7.99 & -13.43 & -5.4 & -54.34 & -71.89 & -59.62 & -35.85 & -9.18 & 0 & -2.507 \\ 
8 & $x_2,x_4$ & 1.31 & 5.3 & 5.13 & 4.35 & 5.21 & 55.56 & 39.19 & 29.86 & 20.47 & 21.91 & 0 & -2.506 \\ 
9 & $x_2,x_3$ & -16.39 & -22.92 & -15.24 & -9.28 & -13.26 & -288.73 & -197.82 & -165.46 & -106.81 & -103.02 & 0 & -2.504 \\ 
10 & $x_2,x_2$ & 5.77 & 7.59 & 4.45 & 1.49 & 3.72 & 108.29 & 70.29 & 60.34 & 38.62 & 38.72 & 0 & -2.506 \\ 
11 & $x_1$ & 25.77 & 21.68 & 15.22 & 13.42 & 10.92 & 220.93 & 169.96 & 139.84 & 85.56 & 72.68 & 0 & -2.526 \\ 
12 & $x_1,x_4$ & 5.03 & 4.03 & 2.82 & 2.28 & 1.83 & 45.48 & 35.21 & 28.83 & 17.78 & 14.85 & 0 & -2.527 \\ 
\hline 
\end{tabular}
\end{table}
Visualised
\begin{figure}[!h]
	\centering
	% This file was created by matlab2tikz.
%
\definecolor{mycolor1}{rgb}{0.00000,0.44700,0.74100}%
%
\begin{tikzpicture}

\begin{axis}[%
width=2.306cm,
height=2.512cm,
at={(0cm,3.488cm)},
scale only axis,
xmin=0,
xmax=10,
xlabel style={font=\color{white!15!black}},
xlabel={Dataset index},
ymin=-1570.60720264055,
ymax=0,
ylabel style={font=\color{white!15!black}},
ylabel={$c$},
axis background/.style={fill=white},
legend style={legend cell align=left, align=left, draw=white!15!black}
]
\addplot [color=mycolor1, line width=2.0pt, draw=none, mark=o, mark options={solid, mycolor1}]
  table[row sep=crcr]{%
1	-234.539140897964\\
2	-204.620222005464\\
4	-69.3183215856622\\
5	-119.469410422321\\
6	-1570.60720264055\\
7	-1369.94256508144\\
9	-810.157687342949\\
10	-578.871880489588\\
};
\addlegendentry{data1}

\end{axis}

\begin{axis}[%
width=2.306cm,
height=2.512cm,
at={(3.035cm,3.488cm)},
scale only axis,
xmin=0,
xmax=10,
xlabel style={font=\color{white!15!black}},
xlabel={Dataset index},
ymin=-1500,
ymax=0,
ylabel style={font=\color{white!15!black}},
ylabel={$x_4$},
axis background/.style={fill=white},
legend style={legend cell align=left, align=left, draw=white!15!black}
]
\addplot [color=mycolor1, line width=2.0pt, draw=none, mark=o, mark options={solid, mycolor1}]
  table[row sep=crcr]{%
1	-177.027842821286\\
2	-144.091783081883\\
4	-68.5172968049575\\
5	-76.681359016238\\
6	-1355.31197379512\\
7	-1111.81339717774\\
9	-630.593824253468\\
10	-490.279895335194\\
};
\addlegendentry{data1}

\end{axis}

\begin{axis}[%
width=2.306cm,
height=2.512cm,
at={(6.07cm,3.488cm)},
scale only axis,
xmin=0,
xmax=10,
xlabel style={font=\color{white!15!black}},
xlabel={Dataset index},
ymin=0,
ymax=725.984005688338,
ylabel style={font=\color{white!15!black}},
ylabel={$x_3$},
axis background/.style={fill=white},
legend style={legend cell align=left, align=left, draw=white!15!black}
]
\addplot [color=mycolor1, line width=2.0pt, draw=none, mark=o, mark options={solid, mycolor1}]
  table[row sep=crcr]{%
1	84.0784415628057\\
2	64.2993799213961\\
4	40.1997153300434\\
5	28.8321342630344\\
6	725.984005688338\\
7	566.327570839676\\
9	311.636312916187\\
10	259.452196820681\\
};
\addlegendentry{data1}

\end{axis}

\begin{axis}[%
width=2.306cm,
height=2.512cm,
at={(9.105cm,3.488cm)},
scale only axis,
xmin=0,
xmax=10,
xlabel style={font=\color{white!15!black}},
xlabel={Dataset index},
ymin=-150.420523327553,
ymax=0,
ylabel style={font=\color{white!15!black}},
ylabel={$x_4,x_4$},
axis background/.style={fill=white},
legend style={legend cell align=left, align=left, draw=white!15!black}
]
\addplot [color=mycolor1, line width=2.0pt, draw=none, mark=o, mark options={solid, mycolor1}]
  table[row sep=crcr]{%
1	-23.3162164993923\\
2	-17.551126535841\\
4	-8.57889852771006\\
5	-11.1249029147396\\
6	-150.420523327553\\
7	-144.424085070836\\
9	-83.7837850877327\\
10	-56.0717545660385\\
};
\addlegendentry{data1}

\end{axis}

\begin{axis}[%
width=2.306cm,
height=2.512cm,
at={(0cm,0cm)},
scale only axis,
xmin=0,
xmax=10,
xlabel style={font=\color{white!15!black}},
xlabel={Dataset index},
ymin=0,
ymax=60,
ylabel style={font=\color{white!15!black}},
ylabel={$x_3,x_4$},
axis background/.style={fill=white},
legend style={legend cell align=left, align=left, draw=white!15!black}
]
\addplot [color=mycolor1, line width=2.0pt, draw=none, mark=o, mark options={solid, mycolor1}]
  table[row sep=crcr]{%
1	9.41071532819576\\
2	4.85648939202435\\
4	2.52525319022002\\
5	5.82144512126454\\
6	19.5828294495475\\
7	58.5359665260704\\
9	37.4478718322197\\
10	9.84958308987871\\
};
\addlegendentry{data1}

\end{axis}

\begin{axis}[%
width=2.306cm,
height=2.512cm,
at={(3.035cm,0cm)},
scale only axis,
xmin=0,
xmax=10,
xlabel style={font=\color{white!15!black}},
xlabel={Dataset index},
ymin=0,
ymax=63.8645034703178,
ylabel style={font=\color{white!15!black}},
ylabel={$x_3,x_3$},
axis background/.style={fill=white},
legend style={legend cell align=left, align=left, draw=white!15!black}
]
\addplot [color=mycolor1, line width=2.0pt, draw=none, mark=o, mark options={solid, mycolor1}]
  table[row sep=crcr]{%
1	3.84735837114033\\
2	4.18589479318199\\
4	2.5253492774925\\
5	0.202004646323669\\
6	63.8645034703178\\
7	28.8636620018487\\
9	13.1339134575995\\
10	21.4862675861994\\
};
\addlegendentry{data1}

\end{axis}

\begin{axis}[%
width=2.306cm,
height=2.512cm,
at={(6.07cm,0cm)},
scale only axis,
xmin=0,
xmax=10,
xlabel style={font=\color{white!15!black}},
xlabel={Dataset index},
ymin=-2.6962358835339,
ymax=43.652313591371,
ylabel style={font=\color{white!15!black}},
ylabel={$x_2$},
axis background/.style={fill=white},
legend style={legend cell align=left, align=left, draw=white!15!black}
]
\addplot [color=mycolor1, line width=2.0pt, draw=none, mark=o, mark options={solid, mycolor1}]
  table[row sep=crcr]{%
1	1.42162596296746\\
2	0.746168274194868\\
4	-2.6962358835339\\
5	-0.0502271623069443\\
6	43.652313591371\\
7	38.6573138130893\\
9	17.6606791673818\\
10	13.6793355027508\\
};
\addlegendentry{data1}

\end{axis}

\begin{axis}[%
width=2.306cm,
height=2.512cm,
at={(9.105cm,0cm)},
scale only axis,
xmin=0,
xmax=10,
xlabel style={font=\color{white!15!black}},
xlabel={Dataset index},
ymin=-54.5712613439143,
ymax=0.18652536446538,
ylabel style={font=\color{white!15!black}},
ylabel={$x_1$},
axis background/.style={fill=white},
legend style={legend cell align=left, align=left, draw=white!15!black}
]
\addplot [color=mycolor1, line width=2.0pt, draw=none, mark=o, mark options={solid, mycolor1}]
  table[row sep=crcr]{%
1	-4.315102766666\\
2	-2.89970791020909\\
4	0.18652536446538\\
5	-0.786268351112874\\
6	-54.5712613439143\\
7	-45.0319465291698\\
9	-21.7631864713255\\
10	-17.6985827852262\\
};
\addlegendentry{data1}

\end{axis}
\end{tikzpicture}%
	\caption{Estimated values of internal parameters.}
\end{figure}
\par In order to link the external and internal parameters an arbitrary polynomial function is selected for two arguments
\begin{equation}
content...
\end{equation}
Curve fitting results are presented in Tablels 4 and 5.
\begin{table}[!h]
	\centering
	\caption{Estimated polynomial coefficients for the sample length 2000.}
	\small
	\input{betas_C_ny_0_nu_4_size_2000.tex}
\end{table}

\begin{figure}[!h]
	\centering
	% This file was created by matlab2tikz.
% Minimal pgfplots version: 1.3
%
\definecolor{mycolor1}{rgb}{0.00000,0.44700,0.74100}%
\definecolor{mycolor2}{rgb}{0.85000,0.32500,0.09800}%
%
\begin{tikzpicture}

\begin{axis}[%
width=4.527496cm,
height=3.870968cm,
at={(0cm,0cm)},
scale only axis,
xmin=56,
xmax=74,
tick align=outside,
xlabel={$L_{cut}$},
xmajorgrids,
ymin=0.093,
ymax=0.276,
ylabel={$D_{rlx}$},
ymajorgrids,
zmin=-2000,
zmax=0,
zlabel={$c$},
zmajorgrids,
view={-140}{50},
legend style={at={(1.03,1)},anchor=north west,legend cell align=left,align=left,draw=white!15!black}
]
\addplot3[only marks,mark=*,mark options={},mark size=1.5000pt,color=mycolor1] plot table[row sep=crcr,]{%
74	0.123	-233.157966898601\\
72	0.113	-200.830593420783\\
61	0.095	-93.7391747783605\\
56	0.093	-119.696332564413\\
};\label{tikz:thetas1}
\addplot3[only marks,mark=*,mark options={},mark size=1.5000pt,color=mycolor2] plot table[row sep=crcr,]{%
67	0.276	-1805.89673785913\\
66	0.255	-1488.55090215304\\
62	0.209	-803.703476355143\\
57	0.193	-648.796609601896\\
};\label{tikz:thetas2}
\addplot3[only marks,mark=*,mark options={},mark size=1.5000pt,color=black] plot table[row sep=crcr,]{%
69	0.104	-138.628962092727\\
};\label{tikz:thetaidentified}
\addplot3[only marks,mark=*,mark options={},mark size=1.5000pt,color=black] plot table[row sep=crcr,]{%
64	0.23	-1085.45181546038\\
};

\addplot3[%
surf,
opacity=0.7,
shader=interp,
colormap={mymap}{[1pt] rgb(0pt)=(0.0901961,0.239216,0.0745098); rgb(1pt)=(0.0945149,0.242058,0.0739522); rgb(2pt)=(0.0988592,0.244894,0.0733566); rgb(3pt)=(0.103229,0.247724,0.0727241); rgb(4pt)=(0.107623,0.250549,0.0720557); rgb(5pt)=(0.112043,0.253367,0.0713525); rgb(6pt)=(0.116487,0.25618,0.0706154); rgb(7pt)=(0.120956,0.258986,0.0698456); rgb(8pt)=(0.125449,0.261787,0.0690441); rgb(9pt)=(0.129967,0.264581,0.0682118); rgb(10pt)=(0.134508,0.26737,0.06735); rgb(11pt)=(0.139074,0.270152,0.0664596); rgb(12pt)=(0.143663,0.272929,0.0655416); rgb(13pt)=(0.148275,0.275699,0.0645971); rgb(14pt)=(0.152911,0.278463,0.0636271); rgb(15pt)=(0.15757,0.281221,0.0626328); rgb(16pt)=(0.162252,0.283973,0.0616151); rgb(17pt)=(0.166957,0.286719,0.060575); rgb(18pt)=(0.171685,0.289458,0.0595136); rgb(19pt)=(0.176434,0.292191,0.0584321); rgb(20pt)=(0.181207,0.294918,0.0573313); rgb(21pt)=(0.186001,0.297639,0.0562123); rgb(22pt)=(0.190817,0.300353,0.0550763); rgb(23pt)=(0.195655,0.303061,0.0539242); rgb(24pt)=(0.200514,0.305763,0.052757); rgb(25pt)=(0.205395,0.308459,0.0515759); rgb(26pt)=(0.210296,0.311149,0.0503624); rgb(27pt)=(0.215212,0.313846,0.0490067); rgb(28pt)=(0.220142,0.316548,0.0475043); rgb(29pt)=(0.22509,0.319254,0.0458704); rgb(30pt)=(0.230056,0.321962,0.0441205); rgb(31pt)=(0.235042,0.324671,0.04227); rgb(32pt)=(0.240048,0.327379,0.0403343); rgb(33pt)=(0.245078,0.330085,0.0383287); rgb(34pt)=(0.250131,0.332786,0.0362688); rgb(35pt)=(0.25521,0.335482,0.0341698); rgb(36pt)=(0.260317,0.33817,0.0320472); rgb(37pt)=(0.265451,0.340849,0.0299163); rgb(38pt)=(0.270616,0.343517,0.0277927); rgb(39pt)=(0.275813,0.346172,0.0256916); rgb(40pt)=(0.281043,0.348814,0.0236284); rgb(41pt)=(0.286307,0.35144,0.0216186); rgb(42pt)=(0.291607,0.354048,0.0196776); rgb(43pt)=(0.296945,0.356637,0.0178207); rgb(44pt)=(0.302322,0.359206,0.0160634); rgb(45pt)=(0.307739,0.361753,0.0144211); rgb(46pt)=(0.313198,0.364275,0.0129091); rgb(47pt)=(0.318701,0.366772,0.0115428); rgb(48pt)=(0.324249,0.369242,0.0103377); rgb(49pt)=(0.329843,0.371682,0.00930909); rgb(50pt)=(0.335485,0.374093,0.00847245); rgb(51pt)=(0.341176,0.376471,0.00784314); rgb(52pt)=(0.346925,0.378826,0.00732741); rgb(53pt)=(0.352735,0.381168,0.00682184); rgb(54pt)=(0.358605,0.383497,0.00632729); rgb(55pt)=(0.364532,0.385812,0.00584464); rgb(56pt)=(0.370516,0.388113,0.00537476); rgb(57pt)=(0.376552,0.390399,0.00491852); rgb(58pt)=(0.38264,0.39267,0.00447681); rgb(59pt)=(0.388777,0.394925,0.00405048); rgb(60pt)=(0.394962,0.397164,0.00364042); rgb(61pt)=(0.401191,0.399386,0.00324749); rgb(62pt)=(0.407464,0.401592,0.00287258); rgb(63pt)=(0.413777,0.40378,0.00251655); rgb(64pt)=(0.420129,0.40595,0.00218028); rgb(65pt)=(0.426518,0.408102,0.00186463); rgb(66pt)=(0.432942,0.410234,0.00157049); rgb(67pt)=(0.439399,0.412348,0.00129873); rgb(68pt)=(0.445885,0.414441,0.00105022); rgb(69pt)=(0.452401,0.416515,0.000825833); rgb(70pt)=(0.458942,0.418567,0.000626441); rgb(71pt)=(0.465508,0.420599,0.00045292); rgb(72pt)=(0.472096,0.422609,0.000306141); rgb(73pt)=(0.478704,0.424596,0.000186979); rgb(74pt)=(0.485331,0.426562,9.63073e-05); rgb(75pt)=(0.491973,0.428504,3.49981e-05); rgb(76pt)=(0.498628,0.430422,3.92506e-06); rgb(77pt)=(0.505323,0.432315,0); rgb(78pt)=(0.512206,0.434168,0); rgb(79pt)=(0.519282,0.435983,0); rgb(80pt)=(0.526529,0.437764,0); rgb(81pt)=(0.533922,0.439512,0); rgb(82pt)=(0.54144,0.441232,0); rgb(83pt)=(0.549059,0.442927,0); rgb(84pt)=(0.556756,0.444599,0); rgb(85pt)=(0.564508,0.446252,0); rgb(86pt)=(0.572292,0.447889,0); rgb(87pt)=(0.580084,0.449514,0); rgb(88pt)=(0.587863,0.451129,0); rgb(89pt)=(0.595604,0.452737,0); rgb(90pt)=(0.603284,0.454343,0); rgb(91pt)=(0.610882,0.455948,0); rgb(92pt)=(0.618373,0.457556,0); rgb(93pt)=(0.625734,0.459171,0); rgb(94pt)=(0.632943,0.460795,0); rgb(95pt)=(0.639976,0.462432,0); rgb(96pt)=(0.64681,0.464084,0); rgb(97pt)=(0.653423,0.465756,0); rgb(98pt)=(0.659791,0.46745,0); rgb(99pt)=(0.665891,0.469169,0); rgb(100pt)=(0.6717,0.470916,0); rgb(101pt)=(0.677195,0.472696,0); rgb(102pt)=(0.682353,0.47451,0); rgb(103pt)=(0.687242,0.476355,0); rgb(104pt)=(0.691952,0.478225,0); rgb(105pt)=(0.696497,0.480118,0); rgb(106pt)=(0.700887,0.482033,0); rgb(107pt)=(0.705134,0.483968,0); rgb(108pt)=(0.709251,0.485921,0); rgb(109pt)=(0.713249,0.487891,0); rgb(110pt)=(0.71714,0.489876,0); rgb(111pt)=(0.720936,0.491875,0); rgb(112pt)=(0.724649,0.493887,0); rgb(113pt)=(0.72829,0.495909,0); rgb(114pt)=(0.731872,0.49794,0); rgb(115pt)=(0.735406,0.499979,0); rgb(116pt)=(0.738904,0.502025,0); rgb(117pt)=(0.742378,0.504075,0); rgb(118pt)=(0.74584,0.506128,0); rgb(119pt)=(0.749302,0.508182,0); rgb(120pt)=(0.752775,0.510237,0); rgb(121pt)=(0.756272,0.51229,0); rgb(122pt)=(0.759804,0.514339,0); rgb(123pt)=(0.763384,0.516385,0); rgb(124pt)=(0.767022,0.518424,0); rgb(125pt)=(0.770731,0.520455,0); rgb(126pt)=(0.774523,0.522478,0); rgb(127pt)=(0.77841,0.524489,0); rgb(128pt)=(0.782391,0.526491,0); rgb(129pt)=(0.786402,0.528496,0); rgb(130pt)=(0.790431,0.530506,0); rgb(131pt)=(0.794478,0.532521,0); rgb(132pt)=(0.798541,0.534539,0); rgb(133pt)=(0.802619,0.53656,0); rgb(134pt)=(0.806712,0.538584,0); rgb(135pt)=(0.81082,0.540609,0); rgb(136pt)=(0.81494,0.542635,0); rgb(137pt)=(0.819074,0.54466,0); rgb(138pt)=(0.823219,0.546686,0); rgb(139pt)=(0.827374,0.548709,0); rgb(140pt)=(0.831541,0.55073,0); rgb(141pt)=(0.835716,0.552749,0); rgb(142pt)=(0.8399,0.554763,0); rgb(143pt)=(0.844092,0.556774,0); rgb(144pt)=(0.848292,0.558779,0); rgb(145pt)=(0.852497,0.560778,0); rgb(146pt)=(0.856708,0.562771,0); rgb(147pt)=(0.860924,0.564756,0); rgb(148pt)=(0.865143,0.566733,0); rgb(149pt)=(0.869366,0.568701,0); rgb(150pt)=(0.873592,0.57066,0); rgb(151pt)=(0.877819,0.572608,0); rgb(152pt)=(0.882047,0.574545,0); rgb(153pt)=(0.886275,0.576471,0); rgb(154pt)=(0.890659,0.578362,0); rgb(155pt)=(0.895333,0.580203,0); rgb(156pt)=(0.900258,0.581999,0); rgb(157pt)=(0.905397,0.583755,0); rgb(158pt)=(0.910711,0.585479,0); rgb(159pt)=(0.916164,0.587176,0); rgb(160pt)=(0.921717,0.588852,0); rgb(161pt)=(0.927333,0.590513,0); rgb(162pt)=(0.932974,0.592166,0); rgb(163pt)=(0.938602,0.593815,0); rgb(164pt)=(0.94418,0.595468,0); rgb(165pt)=(0.949669,0.59713,0); rgb(166pt)=(0.955033,0.598808,0); rgb(167pt)=(0.960233,0.600507,0); rgb(168pt)=(0.965232,0.602233,0); rgb(169pt)=(0.969992,0.603992,0); rgb(170pt)=(0.974475,0.605791,0); rgb(171pt)=(0.978643,0.607636,0); rgb(172pt)=(0.98246,0.609532,0); rgb(173pt)=(0.985886,0.611486,0); rgb(174pt)=(0.988885,0.613503,0); rgb(175pt)=(0.991419,0.61559,0); rgb(176pt)=(0.99345,0.617753,0); rgb(177pt)=(0.99494,0.619997,0); rgb(178pt)=(0.995851,0.622329,0); rgb(179pt)=(0.996226,0.624763,0); rgb(180pt)=(0.996512,0.627352,0); rgb(181pt)=(0.996788,0.630095,0); rgb(182pt)=(0.997053,0.632982,0); rgb(183pt)=(0.997308,0.636004,0); rgb(184pt)=(0.997552,0.639152,0); rgb(185pt)=(0.997785,0.642416,0); rgb(186pt)=(0.998006,0.645786,0); rgb(187pt)=(0.998217,0.649253,0); rgb(188pt)=(0.998416,0.652807,0); rgb(189pt)=(0.998605,0.656439,0); rgb(190pt)=(0.998781,0.660138,0); rgb(191pt)=(0.998946,0.663897,0); rgb(192pt)=(0.9991,0.667704,0); rgb(193pt)=(0.999242,0.67155,0); rgb(194pt)=(0.999372,0.675427,0); rgb(195pt)=(0.99949,0.679323,0); rgb(196pt)=(0.999596,0.68323,0); rgb(197pt)=(0.99969,0.687139,0); rgb(198pt)=(0.999771,0.691039,0); rgb(199pt)=(0.999841,0.694921,0); rgb(200pt)=(0.999898,0.698775,0); rgb(201pt)=(0.999942,0.702592,0); rgb(202pt)=(0.999974,0.706363,0); rgb(203pt)=(0.999994,0.710077,0); rgb(204pt)=(1,0.713725,0); rgb(205pt)=(1,0.717341,0); rgb(206pt)=(1,0.720963,0); rgb(207pt)=(1,0.724591,0); rgb(208pt)=(1,0.728226,0); rgb(209pt)=(1,0.731867,0); rgb(210pt)=(1,0.735514,0); rgb(211pt)=(1,0.739167,0); rgb(212pt)=(1,0.742827,0); rgb(213pt)=(1,0.746493,0); rgb(214pt)=(1,0.750165,0); rgb(215pt)=(1,0.753843,0); rgb(216pt)=(1,0.757527,0); rgb(217pt)=(1,0.761217,0); rgb(218pt)=(1,0.764913,0); rgb(219pt)=(1,0.768615,0); rgb(220pt)=(1,0.772324,0); rgb(221pt)=(1,0.776038,0); rgb(222pt)=(1,0.779758,0); rgb(223pt)=(1,0.783484,0); rgb(224pt)=(1,0.787215,0); rgb(225pt)=(1,0.790953,0); rgb(226pt)=(1,0.794696,0); rgb(227pt)=(1,0.798445,0); rgb(228pt)=(1,0.8022,0); rgb(229pt)=(1,0.805961,0); rgb(230pt)=(1,0.809727,0); rgb(231pt)=(1,0.8135,0); rgb(232pt)=(1,0.817278,0); rgb(233pt)=(1,0.821063,0); rgb(234pt)=(1,0.824854,0); rgb(235pt)=(1,0.828652,0); rgb(236pt)=(1,0.832455,0); rgb(237pt)=(1,0.836265,0); rgb(238pt)=(1,0.840081,0); rgb(239pt)=(1,0.843903,0); rgb(240pt)=(1,0.847732,0); rgb(241pt)=(1,0.851566,0); rgb(242pt)=(1,0.855406,0); rgb(243pt)=(1,0.859253,0); rgb(244pt)=(1,0.863106,0); rgb(245pt)=(1,0.866964,0); rgb(246pt)=(1,0.870829,0); rgb(247pt)=(1,0.8747,0); rgb(248pt)=(1,0.878577,0); rgb(249pt)=(1,0.88246,0); rgb(250pt)=(1,0.886349,0); rgb(251pt)=(1,0.890243,0); rgb(252pt)=(1,0.894144,0); rgb(253pt)=(1,0.898051,0); rgb(254pt)=(1,0.901964,0); rgb(255pt)=(1,0.905882,0)},
mesh/rows=49]
table[row sep=crcr,header=false] {%
%
56	0.093	-115.68341116986\\
56	0.09666	-119.178384949931\\
56	0.10032	-123.892976330784\\
56	0.10398	-129.827185312418\\
56	0.10764	-136.981011894832\\
56	0.1113	-145.354456078027\\
56	0.11496	-154.947517862003\\
56	0.11862	-165.76019724676\\
56	0.12228	-177.792494232298\\
56	0.12594	-191.044408818617\\
56	0.1296	-205.515941005717\\
56	0.13326	-221.207090793597\\
56	0.13692	-238.117858182259\\
56	0.14058	-256.248243171702\\
56	0.14424	-275.598245761925\\
56	0.1479	-296.16786595293\\
56	0.15156	-317.957103744714\\
56	0.15522	-340.965959137281\\
56	0.15888	-365.194432130628\\
56	0.16254	-390.642522724757\\
56	0.1662	-417.310230919665\\
56	0.16986	-445.197556715355\\
56	0.17352	-474.304500111826\\
56	0.17718	-504.631061109078\\
56	0.18084	-536.177239707109\\
56	0.1845	-568.943035905924\\
56	0.18816	-602.928449705518\\
56	0.19182	-638.133481105893\\
56	0.19548	-674.558130107049\\
56	0.19914	-712.202396708986\\
56	0.2028	-751.066280911705\\
56	0.20646	-791.149782715203\\
56	0.21012	-832.452902119484\\
56	0.21378	-874.975639124543\\
56	0.21744	-918.717993730386\\
56	0.2211	-963.679965937008\\
56	0.22476	-1009.86155574441\\
56	0.22842	-1057.2627631526\\
56	0.23208	-1105.88358816156\\
56	0.23574	-1155.72403077131\\
56	0.2394	-1206.78409098184\\
56	0.24306	-1259.06376879314\\
56	0.24672	-1312.56306420523\\
56	0.25038	-1367.2819772181\\
56	0.25404	-1423.22050783175\\
56	0.2577	-1480.37865604619\\
56	0.26136	-1538.7564218614\\
56	0.26502	-1598.35380527739\\
56	0.26868	-1659.17080629417\\
56	0.27234	-1721.20742491172\\
56	0.276	-1784.46366113006\\
56.375	0.093	-113.514447153983\\
56.375	0.09666	-117.038923411893\\
56.375	0.10032	-121.783017270585\\
56.375	0.10398	-127.746728730057\\
56.375	0.10764	-134.93005779031\\
56.375	0.1113	-143.333004451344\\
56.375	0.11496	-152.955568713159\\
56.375	0.11862	-163.797750575755\\
56.375	0.12228	-175.859550039132\\
56.375	0.12594	-189.140967103289\\
56.375	0.1296	-203.642001768228\\
56.375	0.13326	-219.362654033947\\
56.375	0.13692	-236.302923900447\\
56.375	0.14058	-254.462811367729\\
56.375	0.14424	-273.842316435791\\
56.375	0.1479	-294.441439104634\\
56.375	0.15156	-316.260179374258\\
56.375	0.15522	-339.298537244663\\
56.375	0.15888	-363.556512715849\\
56.375	0.16254	-389.034105787816\\
56.375	0.1662	-415.731316460563\\
56.375	0.16986	-443.648144734093\\
56.375	0.17352	-472.784590608401\\
56.375	0.17718	-503.140654083492\\
56.375	0.18084	-534.716335159363\\
56.375	0.1845	-567.511633836016\\
56.375	0.18816	-601.526550113449\\
56.375	0.19182	-636.761083991663\\
56.375	0.19548	-673.215235470658\\
56.375	0.19914	-710.889004550434\\
56.375	0.2028	-749.782391230991\\
56.375	0.20646	-789.895395512328\\
56.375	0.21012	-831.228017394447\\
56.375	0.21378	-873.780256877346\\
56.375	0.21744	-917.552113961027\\
56.375	0.2211	-962.543588645488\\
56.375	0.22476	-1008.75468093073\\
56.375	0.22842	-1056.18539081675\\
56.375	0.23208	-1104.83571830356\\
56.375	0.23574	-1154.70566339114\\
56.375	0.2394	-1205.79522607951\\
56.375	0.24306	-1258.10440636866\\
56.375	0.24672	-1311.63320425858\\
56.375	0.25038	-1366.38161974929\\
56.375	0.25404	-1422.34965284078\\
56.375	0.2577	-1479.53730353305\\
56.375	0.26136	-1537.9445718261\\
56.375	0.26502	-1597.57145771994\\
56.375	0.26868	-1658.41796121455\\
56.375	0.27234	-1720.48408230994\\
56.375	0.276	-1783.76982100612\\
56.75	0.093	-111.484086745226\\
56.75	0.09666	-115.038065480975\\
56.75	0.10032	-119.811661817505\\
56.75	0.10398	-125.804875754816\\
56.75	0.10764	-133.017707292908\\
56.75	0.1113	-141.45015643178\\
56.75	0.11496	-151.102223171434\\
56.75	0.11862	-161.973907511869\\
56.75	0.12228	-174.065209453084\\
56.75	0.12594	-187.376128995081\\
56.75	0.1296	-201.906666137858\\
56.75	0.13326	-217.656820881416\\
56.75	0.13692	-234.626593225755\\
56.75	0.14058	-252.815983170875\\
56.75	0.14424	-272.224990716777\\
56.75	0.1479	-292.853615863458\\
56.75	0.15156	-314.701858610921\\
56.75	0.15522	-337.769718959164\\
56.75	0.15888	-362.05719690819\\
56.75	0.16254	-387.564292457995\\
56.75	0.1662	-414.291005608582\\
56.75	0.16986	-442.237336359949\\
56.75	0.17352	-471.403284712098\\
56.75	0.17718	-501.788850665026\\
56.75	0.18084	-533.394034218737\\
56.75	0.1845	-566.218835373227\\
56.75	0.18816	-600.2632541285\\
56.75	0.19182	-635.527290484552\\
56.75	0.19548	-672.010944441386\\
56.75	0.19914	-709.714215999001\\
56.75	0.2028	-748.637105157396\\
56.75	0.20646	-788.779611916573\\
56.75	0.21012	-830.14173627653\\
56.75	0.21378	-872.723478237269\\
56.75	0.21744	-916.524837798787\\
56.75	0.2211	-961.545814961088\\
56.75	0.22476	-1007.78640972417\\
56.75	0.22842	-1055.24662208803\\
56.75	0.23208	-1103.92645205267\\
56.75	0.23574	-1153.8258996181\\
56.75	0.2394	-1204.9449647843\\
56.75	0.24306	-1257.28364755129\\
56.75	0.24672	-1310.84194791905\\
56.75	0.25038	-1365.6198658876\\
56.75	0.25404	-1421.61740145693\\
56.75	0.2577	-1478.83455462704\\
56.75	0.26136	-1537.27132539793\\
56.75	0.26502	-1596.9277137696\\
56.75	0.26868	-1657.80371974205\\
56.75	0.27234	-1719.89934331529\\
56.75	0.276	-1783.2145844893\\
57.125	0.093	-109.592329943587\\
57.125	0.09666	-113.175811157175\\
57.125	0.10032	-117.978909971543\\
57.125	0.10398	-124.001626386693\\
57.125	0.10764	-131.243960402624\\
57.125	0.1113	-139.705912019335\\
57.125	0.11496	-149.387481236828\\
57.125	0.11862	-160.288668055101\\
57.125	0.12228	-172.409472474155\\
57.125	0.12594	-185.749894493991\\
57.125	0.1296	-200.309934114607\\
57.125	0.13326	-216.089591336004\\
57.125	0.13692	-233.088866158182\\
57.125	0.14058	-251.30775858114\\
57.125	0.14424	-270.746268604881\\
57.125	0.1479	-291.404396229401\\
57.125	0.15156	-313.282141454703\\
57.125	0.15522	-336.379504280785\\
57.125	0.15888	-360.696484707648\\
57.125	0.16254	-386.233082735293\\
57.125	0.1662	-412.989298363718\\
57.125	0.16986	-440.965131592924\\
57.125	0.17352	-470.160582422912\\
57.125	0.17718	-500.575650853679\\
57.125	0.18084	-532.210336885228\\
57.125	0.1845	-565.064640517558\\
57.125	0.18816	-599.138561750668\\
57.125	0.19182	-634.43210058456\\
57.125	0.19548	-670.945257019233\\
57.125	0.19914	-708.678031054686\\
57.125	0.2028	-747.630422690921\\
57.125	0.20646	-787.802431927936\\
57.125	0.21012	-829.194058765732\\
57.125	0.21378	-871.805303204309\\
57.125	0.21744	-915.636165243666\\
57.125	0.2211	-960.686644883806\\
57.125	0.22476	-1006.95674212473\\
57.125	0.22842	-1054.44645696643\\
57.125	0.23208	-1103.15578940891\\
57.125	0.23574	-1153.08473945217\\
57.125	0.2394	-1204.23330709621\\
57.125	0.24306	-1256.60149234104\\
57.125	0.24672	-1310.18929518664\\
57.125	0.25038	-1364.99671563303\\
57.125	0.25404	-1421.0237536802\\
57.125	0.2577	-1478.27040932815\\
57.125	0.26136	-1536.73668257687\\
57.125	0.26502	-1596.42257342638\\
57.125	0.26868	-1657.32808187668\\
57.125	0.27234	-1719.45320792775\\
57.125	0.276	-1782.7979515796\\
57.5	0.093	-107.839176749067\\
57.5	0.09666	-111.452160440494\\
57.5	0.10032	-116.284761732701\\
57.5	0.10398	-122.33698062569\\
57.5	0.10764	-129.608817119459\\
57.5	0.1113	-138.10027121401\\
57.5	0.11496	-147.811342909341\\
57.5	0.11862	-158.742032205453\\
57.5	0.12228	-170.892339102346\\
57.5	0.12594	-184.26226360002\\
57.5	0.1296	-198.851805698475\\
57.5	0.13326	-214.66096539771\\
57.5	0.13692	-231.689742697727\\
57.5	0.14058	-249.938137598525\\
57.5	0.14424	-269.406150100104\\
57.5	0.1479	-290.093780202463\\
57.5	0.15156	-312.001027905603\\
57.5	0.15522	-335.127893209524\\
57.5	0.15888	-359.474376114227\\
57.5	0.16254	-385.04047661971\\
57.5	0.1662	-411.826194725974\\
57.5	0.16986	-439.831530433019\\
57.5	0.17352	-469.056483740845\\
57.5	0.17718	-499.501054649451\\
57.5	0.18084	-531.165243158839\\
57.5	0.1845	-564.049049269008\\
57.5	0.18816	-598.152472979957\\
57.5	0.19182	-633.475514291687\\
57.5	0.19548	-670.018173204199\\
57.5	0.19914	-707.780449717491\\
57.5	0.2028	-746.762343831564\\
57.5	0.20646	-786.963855546418\\
57.5	0.21012	-828.384984862053\\
57.5	0.21378	-871.025731778469\\
57.5	0.21744	-914.886096295665\\
57.5	0.2211	-959.966078413643\\
57.5	0.22476	-1006.2656781324\\
57.5	0.22842	-1053.78489545194\\
57.5	0.23208	-1102.52373037226\\
57.5	0.23574	-1152.48218289336\\
57.5	0.2394	-1203.66025301525\\
57.5	0.24306	-1256.05794073791\\
57.5	0.24672	-1309.67524606135\\
57.5	0.25038	-1364.51216898558\\
57.5	0.25404	-1420.56870951058\\
57.5	0.2577	-1477.84486763637\\
57.5	0.26136	-1536.34064336294\\
57.5	0.26502	-1596.05603669029\\
57.5	0.26868	-1656.99104761842\\
57.5	0.27234	-1719.14567614733\\
57.5	0.276	-1782.51992227702\\
57.875	0.093	-106.224627161667\\
57.875	0.09666	-109.867113330933\\
57.875	0.10032	-114.729217100979\\
57.875	0.10398	-120.810938471806\\
57.875	0.10764	-128.112277443414\\
57.875	0.1113	-136.633234015804\\
57.875	0.11496	-146.373808188973\\
57.875	0.11862	-157.333999962924\\
57.875	0.12228	-169.513809337656\\
57.875	0.12594	-182.913236313169\\
57.875	0.1296	-197.532280889462\\
57.875	0.13326	-213.370943066537\\
57.875	0.13692	-230.429222844393\\
57.875	0.14058	-248.707120223029\\
57.875	0.14424	-268.204635202446\\
57.875	0.1479	-288.921767782645\\
57.875	0.15156	-310.858517963623\\
57.875	0.15522	-334.014885745384\\
57.875	0.15888	-358.390871127924\\
57.875	0.16254	-383.986474111247\\
57.875	0.1662	-410.801694695349\\
57.875	0.16986	-438.836532880234\\
57.875	0.17352	-468.090988665897\\
57.875	0.17718	-498.565062052343\\
57.875	0.18084	-530.258753039569\\
57.875	0.1845	-563.172061627577\\
57.875	0.18816	-597.304987816365\\
57.875	0.19182	-632.657531605935\\
57.875	0.19548	-669.229692996284\\
57.875	0.19914	-707.021471987415\\
57.875	0.2028	-746.032868579328\\
57.875	0.20646	-786.26388277202\\
57.875	0.21012	-827.714514565494\\
57.875	0.21378	-870.384763959748\\
57.875	0.21744	-914.274630954784\\
57.875	0.2211	-959.3841155506\\
57.875	0.22476	-1005.7132177472\\
57.875	0.22842	-1053.26193754458\\
57.875	0.23208	-1102.03027494274\\
57.875	0.23574	-1152.01822994168\\
57.875	0.2394	-1203.2258025414\\
57.875	0.24306	-1255.6529927419\\
57.875	0.24672	-1309.29980054318\\
57.875	0.25038	-1364.16622594525\\
57.875	0.25404	-1420.25226894809\\
57.875	0.2577	-1477.55792955172\\
57.875	0.26136	-1536.08320775612\\
57.875	0.26502	-1595.82810356131\\
57.875	0.26868	-1656.79261696728\\
57.875	0.27234	-1718.97674797403\\
57.875	0.276	-1782.38049658156\\
58.25	0.093	-104.748681181386\\
58.25	0.09666	-108.42066982849\\
58.25	0.10032	-113.312276076375\\
58.25	0.10398	-119.423499925041\\
58.25	0.10764	-126.754341374488\\
58.25	0.1113	-135.304800424716\\
58.25	0.11496	-145.074877075725\\
58.25	0.11862	-156.064571327514\\
58.25	0.12228	-168.273883180085\\
58.25	0.12594	-181.702812633437\\
58.25	0.1296	-196.351359687569\\
58.25	0.13326	-212.219524342482\\
58.25	0.13692	-229.307306598176\\
58.25	0.14058	-247.614706454652\\
58.25	0.14424	-267.141723911908\\
58.25	0.1479	-287.888358969945\\
58.25	0.15156	-309.854611628762\\
58.25	0.15522	-333.040481888362\\
58.25	0.15888	-357.445969748741\\
58.25	0.16254	-383.071075209902\\
58.25	0.1662	-409.915798271843\\
58.25	0.16986	-437.980138934566\\
58.25	0.17352	-467.264097198069\\
58.25	0.17718	-497.767673062354\\
58.25	0.18084	-529.490866527418\\
58.25	0.1845	-562.433677593265\\
58.25	0.18816	-596.596106259892\\
58.25	0.19182	-631.9781525273\\
58.25	0.19548	-668.579816395489\\
58.25	0.19914	-706.401097864459\\
58.25	0.2028	-745.44199693421\\
58.25	0.20646	-785.702513604741\\
58.25	0.21012	-827.182647876054\\
58.25	0.21378	-869.882399748146\\
58.25	0.21744	-913.801769221021\\
58.25	0.2211	-958.940756294676\\
58.25	0.22476	-1005.29936096911\\
58.25	0.22842	-1052.87758324433\\
58.25	0.23208	-1101.67542312033\\
58.25	0.23574	-1151.69288059711\\
58.25	0.2394	-1202.92995567467\\
58.25	0.24306	-1255.38664835301\\
58.25	0.24672	-1309.06295863213\\
58.25	0.25038	-1363.95888651203\\
58.25	0.25404	-1420.07443199271\\
58.25	0.2577	-1477.40959507418\\
58.25	0.26136	-1535.96437575642\\
58.25	0.26502	-1595.73877403945\\
58.25	0.26868	-1656.73278992326\\
58.25	0.27234	-1718.94642340785\\
58.25	0.276	-1782.37967449321\\
58.625	0.093	-103.411338808224\\
58.625	0.09666	-107.112829933167\\
58.625	0.10032	-112.033938658891\\
58.625	0.10398	-118.174664985396\\
58.625	0.10764	-125.535008912681\\
58.625	0.1113	-134.114970440748\\
58.625	0.11496	-143.914549569595\\
58.625	0.11862	-154.933746299224\\
58.625	0.12228	-167.172560629633\\
58.625	0.12594	-180.630992560824\\
58.625	0.1296	-195.309042092795\\
58.625	0.13326	-211.206709225547\\
58.625	0.13692	-228.32399395908\\
58.625	0.14058	-246.660896293393\\
58.625	0.14424	-266.217416228489\\
58.625	0.1479	-286.993553764364\\
58.625	0.15156	-308.989308901021\\
58.625	0.15522	-332.204681638459\\
58.625	0.15888	-356.639671976677\\
58.625	0.16254	-382.294279915676\\
58.625	0.1662	-409.168505455457\\
58.625	0.16986	-437.262348596018\\
58.625	0.17352	-466.575809337361\\
58.625	0.17718	-497.108887679483\\
58.625	0.18084	-528.861583622388\\
58.625	0.1845	-561.833897166072\\
58.625	0.18816	-596.025828310538\\
58.625	0.19182	-631.437377055785\\
58.625	0.19548	-668.068543401812\\
58.625	0.19914	-705.919327348621\\
58.625	0.2028	-744.98972889621\\
58.625	0.20646	-785.279748044581\\
58.625	0.21012	-826.789384793732\\
58.625	0.21378	-869.518639143664\\
58.625	0.21744	-913.467511094377\\
58.625	0.2211	-958.636000645871\\
58.625	0.22476	-1005.02410779815\\
58.625	0.22842	-1052.6318325512\\
58.625	0.23208	-1101.45917490504\\
58.625	0.23574	-1151.50613485966\\
58.625	0.2394	-1202.77271241506\\
58.625	0.24306	-1255.25890757123\\
58.625	0.24672	-1308.96472032819\\
58.625	0.25038	-1363.89015068594\\
58.625	0.25404	-1420.03519864446\\
58.625	0.2577	-1477.39986420376\\
58.625	0.26136	-1535.98414736385\\
58.625	0.26502	-1595.78804812471\\
58.625	0.26868	-1656.81156648636\\
58.625	0.27234	-1719.05470244878\\
58.625	0.276	-1782.51745601199\\
59	0.093	-102.212600042181\\
59	0.09666	-105.943593644963\\
59	0.10032	-110.894204848525\\
59	0.10398	-117.064433652869\\
59	0.10764	-124.454280057993\\
59	0.1113	-133.063744063899\\
59	0.11496	-142.892825670585\\
59	0.11862	-153.941524878052\\
59	0.12228	-166.2098416863\\
59	0.12594	-179.697776095329\\
59	0.1296	-194.405328105139\\
59	0.13326	-210.33249771573\\
59	0.13692	-227.479284927102\\
59	0.14058	-245.845689739254\\
59	0.14424	-265.431712152189\\
59	0.1479	-286.237352165903\\
59	0.15156	-308.262609780399\\
59	0.15522	-331.507484995674\\
59	0.15888	-355.971977811732\\
59	0.16254	-381.65608822857\\
59	0.1662	-408.559816246189\\
59	0.16986	-436.683161864589\\
59	0.17352	-466.02612508377\\
59	0.17718	-496.588705903732\\
59	0.18084	-528.370904324475\\
59	0.1845	-561.372720345998\\
59	0.18816	-595.594153968303\\
59	0.19182	-631.035205191389\\
59	0.19548	-667.695874015255\\
59	0.19914	-705.576160439902\\
59	0.2028	-744.67606446533\\
59	0.20646	-784.99558609154\\
59	0.21012	-826.534725318529\\
59	0.21378	-869.293482146301\\
59	0.21744	-913.271856574852\\
59	0.2211	-958.469848604185\\
59	0.22476	-1004.8874582343\\
59	0.22842	-1052.52468546519\\
59	0.23208	-1101.38153029687\\
59	0.23574	-1151.45799272933\\
59	0.2394	-1202.75407276256\\
59	0.24306	-1255.26977039658\\
59	0.24672	-1309.00508563138\\
59	0.25038	-1363.96001846696\\
59	0.25404	-1420.13456890332\\
59	0.2577	-1477.52873694046\\
59	0.26136	-1536.14252257839\\
59	0.26502	-1595.97592581709\\
59	0.26868	-1657.02894665657\\
59	0.27234	-1719.30158509684\\
59	0.276	-1782.79384113789\\
59.375	0.093	-101.152464883257\\
59.375	0.09666	-104.912960963878\\
59.375	0.10032	-109.893074645279\\
59.375	0.10398	-116.092805927461\\
59.375	0.10764	-123.512154810424\\
59.375	0.1113	-132.151121294169\\
59.375	0.11496	-142.009705378694\\
59.375	0.11862	-153.087907064\\
59.375	0.12228	-165.385726350087\\
59.375	0.12594	-178.903163236955\\
59.375	0.1296	-193.640217724603\\
59.375	0.13326	-209.596889813033\\
59.375	0.13692	-226.773179502243\\
59.375	0.14058	-245.169086792235\\
59.375	0.14424	-264.784611683007\\
59.375	0.1479	-285.619754174561\\
59.375	0.15156	-307.674514266894\\
59.375	0.15522	-330.94889196001\\
59.375	0.15888	-355.442887253906\\
59.375	0.16254	-381.156500148584\\
59.375	0.1662	-408.089730644041\\
59.375	0.16986	-436.24257874028\\
59.375	0.17352	-465.615044437299\\
59.375	0.17718	-496.207127735101\\
59.375	0.18084	-528.018828633681\\
59.375	0.1845	-561.050147133044\\
59.375	0.18816	-595.301083233187\\
59.375	0.19182	-630.771636934112\\
59.375	0.19548	-667.461808235817\\
59.375	0.19914	-705.371597138303\\
59.375	0.2028	-744.50100364157\\
59.375	0.20646	-784.850027745618\\
59.375	0.21012	-826.418669450447\\
59.375	0.21378	-869.206928756056\\
59.375	0.21744	-913.214805662447\\
59.375	0.2211	-958.442300169618\\
59.375	0.22476	-1004.88941227757\\
59.375	0.22842	-1052.5561419863\\
59.375	0.23208	-1101.44248929582\\
59.375	0.23574	-1151.54845420611\\
59.375	0.2394	-1202.87403671719\\
59.375	0.24306	-1255.41923682905\\
59.375	0.24672	-1309.18405454168\\
59.375	0.25038	-1364.1684898551\\
59.375	0.25404	-1420.3725427693\\
59.375	0.2577	-1477.79621328428\\
59.375	0.26136	-1536.43950140005\\
59.375	0.26502	-1596.30240711659\\
59.375	0.26868	-1657.38493043391\\
59.375	0.27234	-1719.68707135202\\
59.375	0.276	-1783.2088298709\\
59.75	0.093	-100.230933331452\\
59.75	0.09666	-104.020931889911\\
59.75	0.10032	-109.030548049152\\
59.75	0.10398	-115.259781809173\\
59.75	0.10764	-122.708633169975\\
59.75	0.1113	-131.377102131558\\
59.75	0.11496	-141.265188693922\\
59.75	0.11862	-152.372892857067\\
59.75	0.12228	-164.700214620992\\
59.75	0.12594	-178.247153985699\\
59.75	0.1296	-193.013710951186\\
59.75	0.13326	-208.999885517455\\
59.75	0.13692	-226.205677684504\\
59.75	0.14058	-244.631087452335\\
59.75	0.14424	-264.276114820945\\
59.75	0.1479	-285.140759790338\\
59.75	0.15156	-307.22502236051\\
59.75	0.15522	-330.528902531465\\
59.75	0.15888	-355.052400303199\\
59.75	0.16254	-380.795515675716\\
59.75	0.1662	-407.758248649012\\
59.75	0.16986	-435.94059922309\\
59.75	0.17352	-465.342567397948\\
59.75	0.17718	-495.964153173587\\
59.75	0.18084	-527.805356550007\\
59.75	0.1845	-560.866177527209\\
59.75	0.18816	-595.146616105191\\
59.75	0.19182	-630.646672283954\\
59.75	0.19548	-667.366346063498\\
59.75	0.19914	-705.305637443822\\
59.75	0.2028	-744.464546424929\\
59.75	0.20646	-784.843073006815\\
59.75	0.21012	-826.441217189483\\
59.75	0.21378	-869.258978972931\\
59.75	0.21744	-913.296358357161\\
59.75	0.2211	-958.55335534217\\
59.75	0.22476	-1005.02996992796\\
59.75	0.22842	-1052.72620211453\\
59.75	0.23208	-1101.64205190189\\
59.75	0.23574	-1151.77751929002\\
59.75	0.2394	-1203.13260427894\\
59.75	0.24306	-1255.70730686863\\
59.75	0.24672	-1309.50162705911\\
59.75	0.25038	-1364.51556485037\\
59.75	0.25404	-1420.74912024241\\
59.75	0.2577	-1478.20229323522\\
59.75	0.26136	-1536.87508382882\\
59.75	0.26502	-1596.76749202321\\
59.75	0.26868	-1657.87951781837\\
59.75	0.27234	-1720.21116121431\\
59.75	0.276	-1783.76242221104\\
60.125	0.093	-99.448005386767\\
60.125	0.09666	-103.267506423065\\
60.125	0.10032	-108.306625060144\\
60.125	0.10398	-114.565361298004\\
60.125	0.10764	-122.043715136645\\
60.125	0.1113	-130.741686576066\\
60.125	0.11496	-140.659275616269\\
60.125	0.11862	-151.796482257253\\
60.125	0.12228	-164.153306499017\\
60.125	0.12594	-177.729748341563\\
60.125	0.1296	-192.525807784889\\
60.125	0.13326	-208.541484828996\\
60.125	0.13692	-225.776779473884\\
60.125	0.14058	-244.231691719553\\
60.125	0.14424	-263.906221566003\\
60.125	0.1479	-284.800369013234\\
60.125	0.15156	-306.914134061246\\
60.125	0.15522	-330.247516710038\\
60.125	0.15888	-354.800516959612\\
60.125	0.16254	-380.573134809966\\
60.125	0.1662	-407.565370261102\\
60.125	0.16986	-435.777223313018\\
60.125	0.17352	-465.208693965716\\
60.125	0.17718	-495.859782219193\\
60.125	0.18084	-527.730488073453\\
60.125	0.1845	-560.820811528492\\
60.125	0.18816	-595.130752584313\\
60.125	0.19182	-630.660311240915\\
60.125	0.19548	-667.409487498298\\
60.125	0.19914	-705.378281356461\\
60.125	0.2028	-744.566692815406\\
60.125	0.20646	-784.974721875132\\
60.125	0.21012	-826.602368535638\\
60.125	0.21378	-869.449632796925\\
60.125	0.21744	-913.516514658993\\
60.125	0.2211	-958.803014121842\\
60.125	0.22476	-1005.30913118547\\
60.125	0.22842	-1053.03486584988\\
60.125	0.23208	-1101.98021811507\\
60.125	0.23574	-1152.14518798105\\
60.125	0.2394	-1203.5297754478\\
60.125	0.24306	-1256.13398051534\\
60.125	0.24672	-1309.95780318365\\
60.125	0.25038	-1365.00124345275\\
60.125	0.25404	-1421.26430132263\\
60.125	0.2577	-1478.74697679328\\
60.125	0.26136	-1537.44926986472\\
60.125	0.26502	-1597.37118053694\\
60.125	0.26868	-1658.51270880994\\
60.125	0.27234	-1720.87385468373\\
60.125	0.276	-1784.45461815829\\
60.5	0.093	-98.8036810492002\\
60.5	0.09666	-102.652684563337\\
60.5	0.10032	-107.721305678255\\
60.5	0.10398	-114.009544393953\\
60.5	0.10764	-121.517400710433\\
60.5	0.1113	-130.244874627694\\
60.5	0.11496	-140.191966145735\\
60.5	0.11862	-151.358675264557\\
60.5	0.12228	-163.74500198416\\
60.5	0.12594	-177.350946304545\\
60.5	0.1296	-192.17650822571\\
60.5	0.13326	-208.221687747656\\
60.5	0.13692	-225.486484870382\\
60.5	0.14058	-243.97089959389\\
60.5	0.14424	-263.674931918179\\
60.5	0.1479	-284.598581843248\\
60.5	0.15156	-306.741849369099\\
60.5	0.15522	-330.10473449573\\
60.5	0.15888	-354.687237223143\\
60.5	0.16254	-380.489357551336\\
60.5	0.1662	-407.511095480311\\
60.5	0.16986	-435.752451010066\\
60.5	0.17352	-465.213424140602\\
60.5	0.17718	-495.894014871918\\
60.5	0.18084	-527.794223204016\\
60.5	0.1845	-560.914049136895\\
60.5	0.18816	-595.253492670555\\
60.5	0.19182	-630.812553804995\\
60.5	0.19548	-667.591232540217\\
60.5	0.19914	-705.589528876219\\
60.5	0.2028	-744.807442813003\\
60.5	0.20646	-785.244974350567\\
60.5	0.21012	-826.902123488912\\
60.5	0.21378	-869.778890228038\\
60.5	0.21744	-913.875274567944\\
60.5	0.2211	-959.191276508633\\
60.5	0.22476	-1005.7268960501\\
60.5	0.22842	-1053.48213319235\\
60.5	0.23208	-1102.45698793538\\
60.5	0.23574	-1152.65146027919\\
60.5	0.2394	-1204.06555022379\\
60.5	0.24306	-1256.69925776916\\
60.5	0.24672	-1310.55258291531\\
60.5	0.25038	-1365.62552566225\\
60.5	0.25404	-1421.91808600996\\
60.5	0.2577	-1479.43026395846\\
60.5	0.26136	-1538.16205950774\\
60.5	0.26502	-1598.1134726578\\
60.5	0.26868	-1659.28450340864\\
60.5	0.27234	-1721.67515176026\\
60.5	0.276	-1785.28541771266\\
60.875	0.093	-98.2979603187528\\
60.875	0.09666	-102.176466310728\\
60.875	0.10032	-107.274589903485\\
60.875	0.10398	-113.592331097022\\
60.875	0.10764	-121.129689891341\\
60.875	0.1113	-129.88666628644\\
60.875	0.11496	-139.86326028232\\
60.875	0.11862	-151.059471878981\\
60.875	0.12228	-163.475301076423\\
60.875	0.12594	-177.110747874646\\
60.875	0.1296	-191.96581227365\\
60.875	0.13326	-208.040494273434\\
60.875	0.13692	-225.334793874\\
60.875	0.14058	-243.848711075346\\
60.875	0.14424	-263.582245877475\\
60.875	0.1479	-284.535398280383\\
60.875	0.15156	-306.708168284072\\
60.875	0.15522	-330.100555888542\\
60.875	0.15888	-354.712561093794\\
60.875	0.16254	-380.544183899826\\
60.875	0.1662	-407.595424306639\\
60.875	0.16986	-435.866282314232\\
60.875	0.17352	-465.356757922607\\
60.875	0.17718	-496.066851131763\\
60.875	0.18084	-527.996561941699\\
60.875	0.1845	-561.145890352417\\
60.875	0.18816	-595.514836363915\\
60.875	0.19182	-631.103399976195\\
60.875	0.19548	-667.911581189255\\
60.875	0.19914	-705.939380003096\\
60.875	0.2028	-745.186796417718\\
60.875	0.20646	-785.653830433122\\
60.875	0.21012	-827.340482049305\\
60.875	0.21378	-870.24675126627\\
60.875	0.21744	-914.372638084015\\
60.875	0.2211	-959.718142502542\\
60.875	0.22476	-1006.28326452185\\
60.875	0.22842	-1054.06800414194\\
60.875	0.23208	-1103.07236136281\\
60.875	0.23574	-1153.29633618446\\
60.875	0.2394	-1204.73992860689\\
60.875	0.24306	-1257.4031386301\\
60.875	0.24672	-1311.28596625409\\
60.875	0.25038	-1366.38841147887\\
60.875	0.25404	-1422.71047430442\\
60.875	0.2577	-1480.25215473076\\
60.875	0.26136	-1539.01345275788\\
60.875	0.26502	-1598.99436838577\\
60.875	0.26868	-1660.19490161445\\
60.875	0.27234	-1722.61505244391\\
60.875	0.276	-1786.25482087415\\
61.25	0.093	-97.9308431954249\\
61.25	0.09666	-101.838851665239\\
61.25	0.10032	-106.966477735835\\
61.25	0.10398	-113.31372140721\\
61.25	0.10764	-120.880582679368\\
61.25	0.1113	-129.667061552306\\
61.25	0.11496	-139.673158026025\\
61.25	0.11862	-150.898872100525\\
61.25	0.12228	-163.344203775805\\
61.25	0.12594	-177.009153051867\\
61.25	0.1296	-191.89371992871\\
61.25	0.13326	-207.997904406333\\
61.25	0.13692	-225.321706484738\\
61.25	0.14058	-243.865126163923\\
61.25	0.14424	-263.628163443889\\
61.25	0.1479	-284.610818324637\\
61.25	0.15156	-306.813090806164\\
61.25	0.15522	-330.234980888474\\
61.25	0.15888	-354.876488571563\\
61.25	0.16254	-380.737613855435\\
61.25	0.1662	-407.818356740086\\
61.25	0.16986	-436.118717225519\\
61.25	0.17352	-465.638695311732\\
61.25	0.17718	-496.378290998727\\
61.25	0.18084	-528.337504286502\\
61.25	0.1845	-561.516335175059\\
61.25	0.18816	-595.914783664396\\
61.25	0.19182	-631.532849754514\\
61.25	0.19548	-668.370533445413\\
61.25	0.19914	-706.427834737093\\
61.25	0.2028	-745.704753629554\\
61.25	0.20646	-786.201290122795\\
61.25	0.21012	-827.917444216818\\
61.25	0.21378	-870.853215911621\\
61.25	0.21744	-915.008605207206\\
61.25	0.2211	-960.383612103571\\
61.25	0.22476	-1006.97823660072\\
61.25	0.22842	-1054.79247869865\\
61.25	0.23208	-1103.82633839735\\
61.25	0.23574	-1154.07981569684\\
61.25	0.2394	-1205.55291059711\\
61.25	0.24306	-1258.24562309816\\
61.25	0.24672	-1312.15795319999\\
61.25	0.25038	-1367.28990090261\\
61.25	0.25404	-1423.641466206\\
61.25	0.2577	-1481.21264911018\\
61.25	0.26136	-1540.00344961513\\
61.25	0.26502	-1600.01386772087\\
61.25	0.26868	-1661.24390342738\\
61.25	0.27234	-1723.69355673468\\
61.25	0.276	-1787.36282764276\\
61.625	0.093	-97.7023296792157\\
61.625	0.09666	-101.639840626868\\
61.625	0.10032	-106.796969175303\\
61.625	0.10398	-113.173715324518\\
61.625	0.10764	-120.770079074514\\
61.625	0.1113	-129.586060425291\\
61.625	0.11496	-139.621659376848\\
61.625	0.11862	-150.876875929187\\
61.625	0.12228	-163.351710082306\\
61.625	0.12594	-177.046161836207\\
61.625	0.1296	-191.960231190888\\
61.625	0.13326	-208.09391814635\\
61.625	0.13692	-225.447222702594\\
61.625	0.14058	-244.020144859618\\
61.625	0.14424	-263.812684617423\\
61.625	0.1479	-284.824841976009\\
61.625	0.15156	-307.056616935375\\
61.625	0.15522	-330.508009495524\\
61.625	0.15888	-355.179019656452\\
61.625	0.16254	-381.069647418162\\
61.625	0.1662	-408.179892780652\\
61.625	0.16986	-436.509755743924\\
61.625	0.17352	-466.059236307976\\
61.625	0.17718	-496.82833447281\\
61.625	0.18084	-528.817050238423\\
61.625	0.1845	-562.025383604818\\
61.625	0.18816	-596.453334571994\\
61.625	0.19182	-632.100903139951\\
61.625	0.19548	-668.968089308689\\
61.625	0.19914	-707.054893078208\\
61.625	0.2028	-746.361314448508\\
61.625	0.20646	-786.887353419588\\
61.625	0.21012	-828.63300999145\\
61.625	0.21378	-871.598284164091\\
61.625	0.21744	-915.783175937515\\
61.625	0.2211	-961.187685311718\\
61.625	0.22476	-1007.8118122867\\
61.625	0.22842	-1055.65555686247\\
61.625	0.23208	-1104.71891903902\\
61.625	0.23574	-1155.00189881635\\
61.625	0.2394	-1206.50449619445\\
61.625	0.24306	-1259.22671117334\\
61.625	0.24672	-1313.16854375301\\
61.625	0.25038	-1368.32999393347\\
61.625	0.25404	-1424.7110617147\\
61.625	0.2577	-1482.31174709671\\
61.625	0.26136	-1541.13205007951\\
61.625	0.26502	-1601.17197066308\\
61.625	0.26868	-1662.43150884744\\
61.625	0.27234	-1724.91066463257\\
61.625	0.276	-1788.60943801849\\
62	0.093	-97.6124197701259\\
62	0.09666	-101.579433195618\\
62	0.10032	-106.766064221891\\
62	0.10398	-113.172312848944\\
62	0.10764	-120.798179076779\\
62	0.1113	-129.643662905395\\
62	0.11496	-139.708764334791\\
62	0.11862	-150.993483364968\\
62	0.12228	-163.497819995927\\
62	0.12594	-177.221774227666\\
62	0.1296	-192.165346060186\\
62	0.13326	-208.328535493487\\
62	0.13692	-225.711342527569\\
62	0.14058	-244.313767162432\\
62	0.14424	-264.135809398076\\
62	0.1479	-285.1774692345\\
62	0.15156	-307.438746671706\\
62	0.15522	-330.919641709693\\
62	0.15888	-355.62015434846\\
62	0.16254	-381.540284588008\\
62	0.1662	-408.680032428338\\
62	0.16986	-437.039397869448\\
62	0.17352	-466.618380911339\\
62	0.17718	-497.416981554011\\
62	0.18084	-529.435199797464\\
62	0.1845	-562.673035641698\\
62	0.18816	-597.130489086713\\
62	0.19182	-632.807560132508\\
62	0.19548	-669.704248779085\\
62	0.19914	-707.820555026442\\
62	0.2028	-747.15647887458\\
62	0.20646	-787.7120203235\\
62	0.21012	-829.4871793732\\
62	0.21378	-872.481956023681\\
62	0.21744	-916.696350274943\\
62	0.2211	-962.130362126986\\
62	0.22476	-1008.78399157981\\
62	0.22842	-1056.65723863342\\
62	0.23208	-1105.7501032878\\
62	0.23574	-1156.06258554297\\
62	0.2394	-1207.59468539892\\
62	0.24306	-1260.34640285564\\
62	0.24672	-1314.31773791315\\
62	0.25038	-1369.50869057144\\
62	0.25404	-1425.91926083051\\
62	0.2577	-1483.54944869037\\
62	0.26136	-1542.399254151\\
62	0.26502	-1602.46867721241\\
62	0.26868	-1663.75771787461\\
62	0.27234	-1726.26637613758\\
62	0.276	-1789.99465200134\\
62.375	0.093	-97.6611134681548\\
62.375	0.09666	-101.657629371485\\
62.375	0.10032	-106.873762875597\\
62.375	0.10398	-113.309513980489\\
62.375	0.10764	-120.964882686163\\
62.375	0.1113	-129.839868992617\\
62.375	0.11496	-139.934472899853\\
62.375	0.11862	-151.248694407869\\
62.375	0.12228	-163.782533516666\\
62.375	0.12594	-177.535990226244\\
62.375	0.1296	-192.509064536603\\
62.375	0.13326	-208.701756447742\\
62.375	0.13692	-226.114065959663\\
62.375	0.14058	-244.745993072365\\
62.375	0.14424	-264.597537785848\\
62.375	0.1479	-285.668700100111\\
62.375	0.15156	-307.959480015156\\
62.375	0.15522	-331.46987753098\\
62.375	0.15888	-356.199892647587\\
62.375	0.16254	-382.149525364974\\
62.375	0.1662	-409.318775683143\\
62.375	0.16986	-437.707643602091\\
62.375	0.17352	-467.316129121821\\
62.375	0.17718	-498.144232242332\\
62.375	0.18084	-530.191952963623\\
62.375	0.1845	-563.459291285696\\
62.375	0.18816	-597.946247208549\\
62.375	0.19182	-633.652820732184\\
62.375	0.19548	-670.579011856599\\
62.375	0.19914	-708.724820581796\\
62.375	0.2028	-748.090246907773\\
62.375	0.20646	-788.675290834531\\
62.375	0.21012	-830.47995236207\\
62.375	0.21378	-873.50423149039\\
62.375	0.21744	-917.74812821949\\
62.375	0.2211	-963.211642549372\\
62.375	0.22476	-1009.89477448003\\
62.375	0.22842	-1057.79752401148\\
62.375	0.23208	-1106.9198911437\\
62.375	0.23574	-1157.26187587671\\
62.375	0.2394	-1208.8234782105\\
62.375	0.24306	-1261.60469814506\\
62.375	0.24672	-1315.60553568041\\
62.375	0.25038	-1370.82599081654\\
62.375	0.25404	-1427.26606355345\\
62.375	0.2577	-1484.92575389114\\
62.375	0.26136	-1543.80506182961\\
62.375	0.26502	-1603.90398736886\\
62.375	0.26868	-1665.2225305089\\
62.375	0.27234	-1727.76069124971\\
62.375	0.276	-1791.51846959131\\
62.75	0.093	-97.848410773303\\
62.75	0.09666	-101.874429154473\\
62.75	0.10032	-107.120065136423\\
62.75	0.10398	-113.585318719155\\
62.75	0.10764	-121.270189902667\\
62.75	0.1113	-130.17467868696\\
62.75	0.11496	-140.298785072034\\
62.75	0.11862	-151.642509057889\\
62.75	0.12228	-164.205850644525\\
62.75	0.12594	-177.988809831941\\
62.75	0.1296	-192.991386620139\\
62.75	0.13326	-209.213581009118\\
62.75	0.13692	-226.655392998877\\
62.75	0.14058	-245.316822589418\\
62.75	0.14424	-265.197869780739\\
62.75	0.1479	-286.298534572841\\
62.75	0.15156	-308.618816965724\\
62.75	0.15522	-332.158716959389\\
62.75	0.15888	-356.918234553834\\
62.75	0.16254	-382.89736974906\\
62.75	0.1662	-410.096122545066\\
62.75	0.16986	-438.514492941855\\
62.75	0.17352	-468.152480939422\\
62.75	0.17718	-499.010086537772\\
62.75	0.18084	-531.087309736902\\
62.75	0.1845	-564.384150536814\\
62.75	0.18816	-598.900608937506\\
62.75	0.19182	-634.636684938979\\
62.75	0.19548	-671.592378541233\\
62.75	0.19914	-709.767689744269\\
62.75	0.2028	-749.162618548085\\
62.75	0.20646	-789.777164952681\\
62.75	0.21012	-831.611328958059\\
62.75	0.21378	-874.665110564217\\
62.75	0.21744	-918.938509771157\\
62.75	0.2211	-964.431526578877\\
62.75	0.22476	-1011.14416098738\\
62.75	0.22842	-1059.07641299666\\
62.75	0.23208	-1108.22828260672\\
62.75	0.23574	-1158.59976981757\\
62.75	0.2394	-1210.19087462919\\
62.75	0.24306	-1263.0015970416\\
62.75	0.24672	-1317.03193705479\\
62.75	0.25038	-1372.28189466876\\
62.75	0.25404	-1428.7514698835\\
62.75	0.2577	-1486.44066269903\\
62.75	0.26136	-1545.34947311534\\
62.75	0.26502	-1605.47790113244\\
62.75	0.26868	-1666.82594675031\\
62.75	0.27234	-1729.39360996896\\
62.75	0.276	-1793.18089078839\\
63.125	0.093	-98.1743116855703\\
63.125	0.09666	-102.229832544579\\
63.125	0.10032	-107.504971004368\\
63.125	0.10398	-113.999727064938\\
63.125	0.10764	-121.714100726289\\
63.125	0.1113	-130.648091988421\\
63.125	0.11496	-140.801700851334\\
63.125	0.11862	-152.174927315027\\
63.125	0.12228	-164.767771379502\\
63.125	0.12594	-178.580233044758\\
63.125	0.1296	-193.612312310794\\
63.125	0.13326	-209.864009177611\\
63.125	0.13692	-227.335323645209\\
63.125	0.14058	-246.026255713589\\
63.125	0.14424	-265.936805382749\\
63.125	0.1479	-287.06697265269\\
63.125	0.15156	-309.416757523412\\
63.125	0.15522	-332.986159994915\\
63.125	0.15888	-357.775180067199\\
63.125	0.16254	-383.783817740264\\
63.125	0.1662	-411.012073014109\\
63.125	0.16986	-439.459945888736\\
63.125	0.17352	-469.127436364143\\
63.125	0.17718	-500.014544440331\\
63.125	0.18084	-532.1212701173\\
63.125	0.1845	-565.447613395051\\
63.125	0.18816	-599.993574273582\\
63.125	0.19182	-635.759152752893\\
63.125	0.19548	-672.744348832987\\
63.125	0.19914	-710.94916251386\\
63.125	0.2028	-750.373593795515\\
63.125	0.20646	-791.01764267795\\
63.125	0.21012	-832.881309161168\\
63.125	0.21378	-875.964593245164\\
63.125	0.21744	-920.267494929943\\
63.125	0.2211	-965.790014215502\\
63.125	0.22476	-1012.53215110184\\
63.125	0.22842	-1060.49390558896\\
63.125	0.23208	-1109.67527767687\\
63.125	0.23574	-1160.07626736555\\
63.125	0.2394	-1211.69687465501\\
63.125	0.24306	-1264.53709954526\\
63.125	0.24672	-1318.59694203628\\
63.125	0.25038	-1373.87640212809\\
63.125	0.25404	-1430.37547982068\\
63.125	0.2577	-1488.09417511405\\
63.125	0.26136	-1547.03248800819\\
63.125	0.26502	-1607.19041850313\\
63.125	0.26868	-1668.56796659884\\
63.125	0.27234	-1731.16513229533\\
63.125	0.276	-1794.9819155926\\
63.5	0.093	-98.6388162049566\\
63.5	0.09666	-102.723839541804\\
63.5	0.10032	-108.028480479432\\
63.5	0.10398	-114.552739017841\\
63.5	0.10764	-122.29661515703\\
63.5	0.1113	-131.260108897001\\
63.5	0.11496	-141.443220237753\\
63.5	0.11862	-152.845949179285\\
63.5	0.12228	-165.468295721598\\
63.5	0.12594	-179.310259864693\\
63.5	0.1296	-194.371841608568\\
63.5	0.13326	-210.653040953224\\
63.5	0.13692	-228.153857898661\\
63.5	0.14058	-246.874292444879\\
63.5	0.14424	-266.814344591878\\
63.5	0.1479	-287.974014339658\\
63.5	0.15156	-310.353301688218\\
63.5	0.15522	-333.952206637561\\
63.5	0.15888	-358.770729187683\\
63.5	0.16254	-384.808869338587\\
63.5	0.1662	-412.066627090271\\
63.5	0.16986	-440.544002442736\\
63.5	0.17352	-470.240995395982\\
63.5	0.17718	-501.157605950009\\
63.5	0.18084	-533.293834104817\\
63.5	0.1845	-566.649679860406\\
63.5	0.18816	-601.225143216776\\
63.5	0.19182	-637.020224173927\\
63.5	0.19548	-674.034922731858\\
63.5	0.19914	-712.269238890571\\
63.5	0.2028	-751.723172650065\\
63.5	0.20646	-792.396724010339\\
63.5	0.21012	-834.289892971394\\
63.5	0.21378	-877.40267953323\\
63.5	0.21744	-921.735083695848\\
63.5	0.2211	-967.287105459245\\
63.5	0.22476	-1014.05874482342\\
63.5	0.22842	-1062.05000178838\\
63.5	0.23208	-1111.26087635412\\
63.5	0.23574	-1161.69136852065\\
63.5	0.2394	-1213.34147828795\\
63.5	0.24306	-1266.21120565603\\
63.5	0.24672	-1320.3005506249\\
63.5	0.25038	-1375.60951319454\\
63.5	0.25404	-1432.13809336497\\
63.5	0.2577	-1489.88629113618\\
63.5	0.26136	-1548.85410650816\\
63.5	0.26502	-1609.04153948093\\
63.5	0.26868	-1670.44859005448\\
63.5	0.27234	-1733.07525822881\\
63.5	0.276	-1796.92154400393\\
63.875	0.093	-99.2419243314626\\
63.875	0.09666	-103.356450146148\\
63.875	0.10032	-108.690593561615\\
63.875	0.10398	-115.244354577863\\
63.875	0.10764	-123.017733194891\\
63.875	0.1113	-132.010729412701\\
63.875	0.11496	-142.223343231291\\
63.875	0.11862	-153.655574650662\\
63.875	0.12228	-166.307423670815\\
63.875	0.12594	-180.178890291748\\
63.875	0.1296	-195.269974513462\\
63.875	0.13326	-211.580676335956\\
63.875	0.13692	-229.110995759232\\
63.875	0.14058	-247.860932783289\\
63.875	0.14424	-267.830487408127\\
63.875	0.1479	-289.019659633745\\
63.875	0.15156	-311.428449460145\\
63.875	0.15522	-335.056856887325\\
63.875	0.15888	-359.904881915287\\
63.875	0.16254	-385.972524544029\\
63.875	0.1662	-413.259784773552\\
63.875	0.16986	-441.766662603856\\
63.875	0.17352	-471.493158034941\\
63.875	0.17718	-502.439271066807\\
63.875	0.18084	-534.605001699454\\
63.875	0.1845	-567.990349932881\\
63.875	0.18816	-602.59531576709\\
63.875	0.19182	-638.419899202079\\
63.875	0.19548	-675.46410023785\\
63.875	0.19914	-713.727918874401\\
63.875	0.2028	-753.211355111733\\
63.875	0.20646	-793.914408949847\\
63.875	0.21012	-835.837080388741\\
63.875	0.21378	-878.979369428416\\
63.875	0.21744	-923.341276068871\\
63.875	0.2211	-968.922800310108\\
63.875	0.22476	-1015.72394215213\\
63.875	0.22842	-1063.74470159493\\
63.875	0.23208	-1112.9850786385\\
63.875	0.23574	-1163.44507328286\\
63.875	0.2394	-1215.12468552801\\
63.875	0.24306	-1268.02391537393\\
63.875	0.24672	-1322.14276282063\\
63.875	0.25038	-1377.48122786812\\
63.875	0.25404	-1434.03931051638\\
63.875	0.2577	-1491.81701076543\\
63.875	0.26136	-1550.81432861525\\
63.875	0.26502	-1611.03126406586\\
63.875	0.26868	-1672.46781711725\\
63.875	0.27234	-1735.12398776942\\
63.875	0.276	-1798.99977602237\\
64.25	0.093	-99.9836360650873\\
64.25	0.09666	-104.127664357612\\
64.25	0.10032	-109.491310250918\\
64.25	0.10398	-116.074573745004\\
64.25	0.10764	-123.877454839871\\
64.25	0.1113	-132.89995353552\\
64.25	0.11496	-143.142069831948\\
64.25	0.11862	-154.603803729158\\
64.25	0.12228	-167.28515522715\\
64.25	0.12594	-181.186124325921\\
64.25	0.1296	-196.306711025474\\
64.25	0.13326	-212.646915325808\\
64.25	0.13692	-230.206737226923\\
64.25	0.14058	-248.986176728817\\
64.25	0.14424	-268.985233831494\\
64.25	0.1479	-290.203908534951\\
64.25	0.15156	-312.64220083919\\
64.25	0.15522	-336.300110744209\\
64.25	0.15888	-361.177638250009\\
64.25	0.16254	-387.27478335659\\
64.25	0.1662	-414.591546063952\\
64.25	0.16986	-443.127926372095\\
64.25	0.17352	-472.883924281019\\
64.25	0.17718	-503.859539790723\\
64.25	0.18084	-536.054772901209\\
64.25	0.1845	-569.469623612475\\
64.25	0.18816	-604.104091924523\\
64.25	0.19182	-639.958177837351\\
64.25	0.19548	-677.03188135096\\
64.25	0.19914	-715.32520246535\\
64.25	0.2028	-754.838141180521\\
64.25	0.20646	-795.570697496474\\
64.25	0.21012	-837.522871413206\\
64.25	0.21378	-880.69466293072\\
64.25	0.21744	-925.086072049014\\
64.25	0.2211	-970.69709876809\\
64.25	0.22476	-1017.52774308795\\
64.25	0.22842	-1065.57800500858\\
64.25	0.23208	-1114.84788453\\
64.25	0.23574	-1165.3373816522\\
64.25	0.2394	-1217.04649637518\\
64.25	0.24306	-1269.97522869894\\
64.25	0.24672	-1324.12357862348\\
64.25	0.25038	-1379.49154614881\\
64.25	0.25404	-1436.07913127491\\
64.25	0.2577	-1493.8863340018\\
64.25	0.26136	-1552.91315432946\\
64.25	0.26502	-1613.15959225791\\
64.25	0.26868	-1674.62564778714\\
64.25	0.27234	-1737.31132091714\\
64.25	0.276	-1801.21661164793\\
64.625	0.093	-100.863951405831\\
64.625	0.09666	-105.037482176195\\
64.625	0.10032	-110.430630547339\\
64.625	0.10398	-117.043396519264\\
64.625	0.10764	-124.87578009197\\
64.625	0.1113	-133.927781265457\\
64.625	0.11496	-144.199400039726\\
64.625	0.11862	-155.690636414774\\
64.625	0.12228	-168.401490390604\\
64.625	0.12594	-182.331961967215\\
64.625	0.1296	-197.482051144606\\
64.625	0.13326	-213.851757922778\\
64.625	0.13692	-231.441082301732\\
64.625	0.14058	-250.250024281466\\
64.625	0.14424	-270.278583861981\\
64.625	0.1479	-291.526761043277\\
64.625	0.15156	-313.994555825354\\
64.625	0.15522	-337.681968208213\\
64.625	0.15888	-362.588998191851\\
64.625	0.16254	-388.715645776271\\
64.625	0.1662	-416.061910961472\\
64.625	0.16986	-444.627793747454\\
64.625	0.17352	-474.413294134215\\
64.625	0.17718	-505.41841212176\\
64.625	0.18084	-537.643147710083\\
64.625	0.1845	-571.087500899189\\
64.625	0.18816	-605.751471689075\\
64.625	0.19182	-641.635060079742\\
64.625	0.19548	-678.73826607119\\
64.625	0.19914	-717.061089663419\\
64.625	0.2028	-756.603530856429\\
64.625	0.20646	-797.365589650219\\
64.625	0.21012	-839.347266044791\\
64.625	0.21378	-882.548560040143\\
64.625	0.21744	-926.969471636277\\
64.625	0.2211	-972.610000833191\\
64.625	0.22476	-1019.47014763089\\
64.625	0.22842	-1067.54991202936\\
64.625	0.23208	-1116.84929402862\\
64.625	0.23574	-1167.36829362866\\
64.625	0.2394	-1219.10691082948\\
64.625	0.24306	-1272.06514563108\\
64.625	0.24672	-1326.24299803346\\
64.625	0.25038	-1381.64046803662\\
64.625	0.25404	-1438.25755564056\\
64.625	0.2577	-1496.09426084529\\
64.625	0.26136	-1555.15058365079\\
64.625	0.26502	-1615.42652405707\\
64.625	0.26868	-1676.92208206414\\
64.625	0.27234	-1739.63725767199\\
64.625	0.276	-1803.57205088062\\
65	0.093	-101.882870353695\\
65	0.09666	-106.085903601897\\
65	0.10032	-111.50855445088\\
65	0.10398	-118.150822900644\\
65	0.10764	-126.012708951189\\
65	0.1113	-135.094212602515\\
65	0.11496	-145.395333854621\\
65	0.11862	-156.916072707509\\
65	0.12228	-169.656429161177\\
65	0.12594	-183.616403215627\\
65	0.1296	-198.795994870857\\
65	0.13326	-215.195204126868\\
65	0.13692	-232.81403098366\\
65	0.14058	-251.652475441233\\
65	0.14424	-271.710537499588\\
65	0.1479	-292.988217158722\\
65	0.15156	-315.485514418638\\
65	0.15522	-339.202429279334\\
65	0.15888	-364.138961740812\\
65	0.16254	-390.295111803071\\
65	0.1662	-417.670879466111\\
65	0.16986	-446.266264729931\\
65	0.17352	-476.081267594532\\
65	0.17718	-507.115888059914\\
65	0.18084	-539.370126126077\\
65	0.1845	-572.843981793021\\
65	0.18816	-607.537455060746\\
65	0.19182	-643.450545929252\\
65	0.19548	-680.583254398539\\
65	0.19914	-718.935580468606\\
65	0.2028	-758.507524139455\\
65	0.20646	-799.299085411085\\
65	0.21012	-841.310264283495\\
65	0.21378	-884.541060756686\\
65	0.21744	-928.991474830658\\
65	0.2211	-974.661506505412\\
65	0.22476	-1021.55115578095\\
65	0.22842	-1069.66042265726\\
65	0.23208	-1118.98930713436\\
65	0.23574	-1169.53780921223\\
65	0.2394	-1221.30592889089\\
65	0.24306	-1274.29366617033\\
65	0.24672	-1328.50102105055\\
65	0.25038	-1383.92799353155\\
65	0.25404	-1440.57458361333\\
65	0.2577	-1498.44079129589\\
65	0.26136	-1557.52661657924\\
65	0.26502	-1617.83205946336\\
65	0.26868	-1679.35711994827\\
65	0.27234	-1742.10179803395\\
65	0.276	-1806.06609372042\\
65.375	0.093	-103.040392908677\\
65.375	0.09666	-107.272928634718\\
65.375	0.10032	-112.725081961539\\
65.375	0.10398	-119.396852889142\\
65.375	0.10764	-127.288241417526\\
65.375	0.1113	-136.39924754669\\
65.375	0.11496	-146.729871276636\\
65.375	0.11862	-158.280112607362\\
65.375	0.12228	-171.049971538869\\
65.375	0.12594	-185.039448071157\\
65.375	0.1296	-200.248542204227\\
65.375	0.13326	-216.677253938076\\
65.375	0.13692	-234.325583272707\\
65.375	0.14058	-253.193530208119\\
65.375	0.14424	-273.281094744312\\
65.375	0.1479	-294.588276881285\\
65.375	0.15156	-317.11507661904\\
65.375	0.15522	-340.861493957575\\
65.375	0.15888	-365.827528896892\\
65.375	0.16254	-392.013181436989\\
65.375	0.1662	-419.418451577868\\
65.375	0.16986	-448.043339319527\\
65.375	0.17352	-477.887844661967\\
65.375	0.17718	-508.951967605188\\
65.375	0.18084	-541.23570814919\\
65.375	0.1845	-574.739066293972\\
65.375	0.18816	-609.462042039536\\
65.375	0.19182	-645.404635385881\\
65.375	0.19548	-682.566846333006\\
65.375	0.19914	-720.948674880912\\
65.375	0.2028	-760.5501210296\\
65.375	0.20646	-801.371184779068\\
65.375	0.21012	-843.411866129317\\
65.375	0.21378	-886.672165080347\\
65.375	0.21744	-931.152081632158\\
65.375	0.2211	-976.85161578475\\
65.375	0.22476	-1023.77076753812\\
65.375	0.22842	-1071.90953689228\\
65.375	0.23208	-1121.26792384721\\
65.375	0.23574	-1171.84592840293\\
65.375	0.2394	-1223.64355055942\\
65.375	0.24306	-1276.6607903167\\
65.375	0.24672	-1330.89764767476\\
65.375	0.25038	-1386.3541226336\\
65.375	0.25404	-1443.03021519322\\
65.375	0.2577	-1500.92592535362\\
65.375	0.26136	-1560.0412531148\\
65.375	0.26502	-1620.37619847676\\
65.375	0.26868	-1681.93076143951\\
65.375	0.27234	-1744.70494200303\\
65.375	0.276	-1808.69874016734\\
65.75	0.093	-104.336519070778\\
65.75	0.09666	-108.598557274657\\
65.75	0.10032	-114.080213079318\\
65.75	0.10398	-120.78148648476\\
65.75	0.10764	-128.702377490982\\
65.75	0.1113	-137.842886097985\\
65.75	0.11496	-148.20301230577\\
65.75	0.11862	-159.782756114334\\
65.75	0.12228	-172.582117523681\\
65.75	0.12594	-186.601096533808\\
65.75	0.1296	-201.839693144715\\
65.75	0.13326	-218.297907356404\\
65.75	0.13692	-235.975739168874\\
65.75	0.14058	-254.873188582124\\
65.75	0.14424	-274.990255596156\\
65.75	0.1479	-296.326940210968\\
65.75	0.15156	-318.883242426562\\
65.75	0.15522	-342.659162242936\\
65.75	0.15888	-367.654699660091\\
65.75	0.16254	-393.869854678027\\
65.75	0.1662	-421.304627296745\\
65.75	0.16986	-449.959017516242\\
65.75	0.17352	-479.833025336521\\
65.75	0.17718	-510.926650757581\\
65.75	0.18084	-543.239893779421\\
65.75	0.1845	-576.772754402043\\
65.75	0.18816	-611.525232625446\\
65.75	0.19182	-647.497328449629\\
65.75	0.19548	-684.689041874593\\
65.75	0.19914	-723.100372900338\\
65.75	0.2028	-762.731321526864\\
65.75	0.20646	-803.581887754172\\
65.75	0.21012	-845.652071582259\\
65.75	0.21378	-888.941873011128\\
65.75	0.21744	-933.451292040777\\
65.75	0.2211	-979.180328671209\\
65.75	0.22476	-1026.12898290242\\
65.75	0.22842	-1074.29725473441\\
65.75	0.23208	-1123.68514416719\\
65.75	0.23574	-1174.29265120074\\
65.75	0.2394	-1226.11977583508\\
65.75	0.24306	-1279.16651807019\\
65.75	0.24672	-1333.43287790609\\
65.75	0.25038	-1388.91885534277\\
65.75	0.25404	-1445.62445038023\\
65.75	0.2577	-1503.54966301847\\
65.75	0.26136	-1562.69449325749\\
65.75	0.26502	-1623.05894109729\\
65.75	0.26868	-1684.64300653787\\
65.75	0.27234	-1747.44668957923\\
65.75	0.276	-1811.46999022138\\
66.125	0.093	-105.771248839998\\
66.125	0.09666	-110.062789521716\\
66.125	0.10032	-115.573947804216\\
66.125	0.10398	-122.304723687496\\
66.125	0.10764	-130.255117171557\\
66.125	0.1113	-139.425128256399\\
66.125	0.11496	-149.814756942022\\
66.125	0.11862	-161.424003228426\\
66.125	0.12228	-174.252867115611\\
66.125	0.12594	-188.301348603577\\
66.125	0.1296	-203.569447692323\\
66.125	0.13326	-220.057164381851\\
66.125	0.13692	-237.764498672159\\
66.125	0.14058	-256.691450563249\\
66.125	0.14424	-276.838020055119\\
66.125	0.1479	-298.20420714777\\
66.125	0.15156	-320.790011841202\\
66.125	0.15522	-344.595434135415\\
66.125	0.15888	-369.620474030409\\
66.125	0.16254	-395.865131526184\\
66.125	0.1662	-423.32940662274\\
66.125	0.16986	-452.013299320077\\
66.125	0.17352	-481.916809618194\\
66.125	0.17718	-513.039937517092\\
66.125	0.18084	-545.382683016772\\
66.125	0.1845	-578.945046117232\\
66.125	0.18816	-613.727026818474\\
66.125	0.19182	-649.728625120496\\
66.125	0.19548	-686.949841023299\\
66.125	0.19914	-725.390674526883\\
66.125	0.2028	-765.051125631248\\
66.125	0.20646	-805.931194336394\\
66.125	0.21012	-848.03088064232\\
66.125	0.21378	-891.350184549028\\
66.125	0.21744	-935.889106056516\\
66.125	0.2211	-981.647645164786\\
66.125	0.22476	-1028.62580187384\\
66.125	0.22842	-1076.82357618367\\
66.125	0.23208	-1126.24096809428\\
66.125	0.23574	-1176.87797760567\\
66.125	0.2394	-1228.73460471785\\
66.125	0.24306	-1281.8108494308\\
66.125	0.24672	-1336.10671174454\\
66.125	0.25038	-1391.62219165905\\
66.125	0.25404	-1448.35728917435\\
66.125	0.2577	-1506.31200429043\\
66.125	0.26136	-1565.48633700729\\
66.125	0.26502	-1625.88028732493\\
66.125	0.26868	-1687.49385524335\\
66.125	0.27234	-1750.32704076255\\
66.125	0.276	-1814.37984388254\\
66.5	0.093	-107.344582216338\\
66.5	0.09666	-111.665625375895\\
66.5	0.10032	-117.206286136233\\
66.5	0.10398	-123.966564497352\\
66.5	0.10764	-131.946460459252\\
66.5	0.1113	-141.145974021933\\
66.5	0.11496	-151.565105185394\\
66.5	0.11862	-163.203853949637\\
66.5	0.12228	-176.062220314661\\
66.5	0.12594	-190.140204280465\\
66.5	0.1296	-205.437805847051\\
66.5	0.13326	-221.955025014417\\
66.5	0.13692	-239.691861782564\\
66.5	0.14058	-258.648316151492\\
66.5	0.14424	-278.824388121202\\
66.5	0.1479	-300.220077691691\\
66.5	0.15156	-322.835384862962\\
66.5	0.15522	-346.670309635014\\
66.5	0.15888	-371.724852007847\\
66.5	0.16254	-397.99901198146\\
66.5	0.1662	-425.492789555855\\
66.5	0.16986	-454.20618473103\\
66.5	0.17352	-484.139197506987\\
66.5	0.17718	-515.291827883724\\
66.5	0.18084	-547.664075861242\\
66.5	0.1845	-581.255941439541\\
66.5	0.18816	-616.067424618622\\
66.5	0.19182	-652.098525398482\\
66.5	0.19548	-689.349243779124\\
66.5	0.19914	-727.819579760547\\
66.5	0.2028	-767.509533342751\\
66.5	0.20646	-808.419104525736\\
66.5	0.21012	-850.5482933095\\
66.5	0.21378	-893.897099694047\\
66.5	0.21744	-938.465523679374\\
66.5	0.2211	-984.253565265482\\
66.5	0.22476	-1031.26122445237\\
66.5	0.22842	-1079.48850124004\\
66.5	0.23208	-1128.93539562849\\
66.5	0.23574	-1179.60190761772\\
66.5	0.2394	-1231.48803720774\\
66.5	0.24306	-1284.59378439853\\
66.5	0.24672	-1338.91914919011\\
66.5	0.25038	-1394.46413158246\\
66.5	0.25404	-1451.2287315756\\
66.5	0.2577	-1509.21294916952\\
66.5	0.26136	-1568.41678436421\\
66.5	0.26502	-1628.84023715969\\
66.5	0.26868	-1690.48330755595\\
66.5	0.27234	-1753.34599555299\\
66.5	0.276	-1817.42830115082\\
66.875	0.093	-109.056519199797\\
66.875	0.09666	-113.407064837193\\
66.875	0.10032	-118.97722807537\\
66.875	0.10398	-125.767008914328\\
66.875	0.10764	-133.776407354067\\
66.875	0.1113	-143.005423394586\\
66.875	0.11496	-153.454057035887\\
66.875	0.11862	-165.122308277968\\
66.875	0.12228	-178.01017712083\\
66.875	0.12594	-192.117663564474\\
66.875	0.1296	-207.444767608898\\
66.875	0.13326	-223.991489254103\\
66.875	0.13692	-241.757828500089\\
66.875	0.14058	-260.743785346856\\
66.875	0.14424	-280.949359794404\\
66.875	0.1479	-302.374551842732\\
66.875	0.15156	-325.019361491842\\
66.875	0.15522	-348.883788741733\\
66.875	0.15888	-373.967833592404\\
66.875	0.16254	-400.271496043857\\
66.875	0.1662	-427.79477609609\\
66.875	0.16986	-456.537673749104\\
66.875	0.17352	-486.500189002899\\
66.875	0.17718	-517.682321857475\\
66.875	0.18084	-550.084072312832\\
66.875	0.1845	-583.70544036897\\
66.875	0.18816	-618.546426025889\\
66.875	0.19182	-654.607029283589\\
66.875	0.19548	-691.887250142069\\
66.875	0.19914	-730.387088601331\\
66.875	0.2028	-770.106544661373\\
66.875	0.20646	-811.045618322197\\
66.875	0.21012	-853.204309583801\\
66.875	0.21378	-896.582618446186\\
66.875	0.21744	-941.180544909352\\
66.875	0.2211	-986.998088973299\\
66.875	0.22476	-1034.03525063803\\
66.875	0.22842	-1082.29202990354\\
66.875	0.23208	-1131.76842676983\\
66.875	0.23574	-1182.4644412369\\
66.875	0.2394	-1234.38007330475\\
66.875	0.24306	-1287.51532297338\\
66.875	0.24672	-1341.87019024279\\
66.875	0.25038	-1397.44467511299\\
66.875	0.25404	-1454.23877758396\\
66.875	0.2577	-1512.25249765572\\
66.875	0.26136	-1571.48583532826\\
66.875	0.26502	-1631.93879060157\\
66.875	0.26868	-1693.61136347567\\
66.875	0.27234	-1756.50355395055\\
66.875	0.276	-1820.61536202621\\
67.25	0.093	-110.907059790375\\
67.25	0.09666	-115.28710790561\\
67.25	0.10032	-120.886773621625\\
67.25	0.10398	-127.706056938422\\
67.25	0.10764	-135.744957855999\\
67.25	0.1113	-145.003476374358\\
67.25	0.11496	-155.481612493497\\
67.25	0.11862	-167.179366213417\\
67.25	0.12228	-180.096737534118\\
67.25	0.12594	-194.2337264556\\
67.25	0.1296	-209.590332977863\\
67.25	0.13326	-226.166557100907\\
67.25	0.13692	-243.962398824732\\
67.25	0.14058	-262.977858149337\\
67.25	0.14424	-283.212935074724\\
67.25	0.1479	-304.667629600891\\
67.25	0.15156	-327.34194172784\\
67.25	0.15522	-351.235871455569\\
67.25	0.15888	-376.349418784079\\
67.25	0.16254	-402.682583713371\\
67.25	0.1662	-430.235366243443\\
67.25	0.16986	-459.007766374296\\
67.25	0.17352	-488.99978410593\\
67.25	0.17718	-520.211419438344\\
67.25	0.18084	-552.64267237154\\
67.25	0.1845	-586.293542905517\\
67.25	0.18816	-621.164031040274\\
67.25	0.19182	-657.254136775813\\
67.25	0.19548	-694.563860112132\\
67.25	0.19914	-733.093201049233\\
67.25	0.2028	-772.842159587114\\
67.25	0.20646	-813.810735725777\\
67.25	0.21012	-855.998929465219\\
67.25	0.21378	-899.406740805443\\
67.25	0.21744	-944.034169746447\\
67.25	0.2211	-989.881216288234\\
67.25	0.22476	-1036.9478804308\\
67.25	0.22842	-1085.23416217415\\
67.25	0.23208	-1134.74006151828\\
67.25	0.23574	-1185.46557846319\\
67.25	0.2394	-1237.41071300888\\
67.25	0.24306	-1290.57546515535\\
67.25	0.24672	-1344.9598349026\\
67.25	0.25038	-1400.56382225063\\
67.25	0.25404	-1457.38742719945\\
67.25	0.2577	-1515.43064974904\\
67.25	0.26136	-1574.69348989942\\
67.25	0.26502	-1635.17594765057\\
67.25	0.26868	-1696.87802300251\\
67.25	0.27234	-1759.79971595523\\
67.25	0.276	-1823.94102650873\\
67.625	0.093	-112.896203988071\\
67.625	0.09666	-117.305754581144\\
67.625	0.10032	-122.934922774999\\
67.625	0.10398	-129.783708569635\\
67.625	0.10764	-137.852111965051\\
67.625	0.1113	-147.140132961248\\
67.625	0.11496	-157.647771558226\\
67.625	0.11862	-169.375027755985\\
67.625	0.12228	-182.321901554525\\
67.625	0.12594	-196.488392953846\\
67.625	0.1296	-211.874501953947\\
67.625	0.13326	-228.48022855483\\
67.625	0.13692	-246.305572756494\\
67.625	0.14058	-265.350534558938\\
67.625	0.14424	-285.615113962164\\
67.625	0.1479	-307.099310966169\\
67.625	0.15156	-329.803125570957\\
67.625	0.15522	-353.726557776525\\
67.625	0.15888	-378.869607582874\\
67.625	0.16254	-405.232274990004\\
67.625	0.1662	-432.814559997915\\
67.625	0.16986	-461.616462606607\\
67.625	0.17352	-491.637982816079\\
67.625	0.17718	-522.879120626333\\
67.625	0.18084	-555.339876037368\\
67.625	0.1845	-589.020249049182\\
67.625	0.18816	-623.920239661779\\
67.625	0.19182	-660.039847875157\\
67.625	0.19548	-697.379073689314\\
67.625	0.19914	-735.937917104254\\
67.625	0.2028	-775.716378119974\\
67.625	0.20646	-816.714456736475\\
67.625	0.21012	-858.932152953756\\
67.625	0.21378	-902.369466771819\\
67.625	0.21744	-947.026398190662\\
67.625	0.2211	-992.902947210287\\
67.625	0.22476	-1039.99911383069\\
67.625	0.22842	-1088.31489805188\\
67.625	0.23208	-1137.85029987385\\
67.625	0.23574	-1188.60531929659\\
67.625	0.2394	-1240.57995632012\\
67.625	0.24306	-1293.77421094443\\
67.625	0.24672	-1348.18808316952\\
67.625	0.25038	-1403.8215729954\\
67.625	0.25404	-1460.67468042205\\
67.625	0.2577	-1518.74740544948\\
67.625	0.26136	-1578.0397480777\\
67.625	0.26502	-1638.55170830669\\
67.625	0.26868	-1700.28328613647\\
67.625	0.27234	-1763.23448156703\\
67.625	0.276	-1827.40529459836\\
68	0.093	-115.023951792887\\
68	0.09666	-119.463004863799\\
68	0.10032	-125.121675535493\\
68	0.10398	-131.999963807967\\
68	0.10764	-140.097869681222\\
68	0.1113	-149.415393155258\\
68	0.11496	-159.952534230075\\
68	0.11862	-171.709292905672\\
68	0.12228	-184.685669182051\\
68	0.12594	-198.88166305921\\
68	0.1296	-214.297274537151\\
68	0.13326	-230.932503615873\\
68	0.13692	-248.787350295375\\
68	0.14058	-267.861814575658\\
68	0.14424	-288.155896456722\\
68	0.1479	-309.669595938567\\
68	0.15156	-332.402913021193\\
68	0.15522	-356.3558477046\\
68	0.15888	-381.528399988788\\
68	0.16254	-407.920569873757\\
68	0.1662	-435.532357359506\\
68	0.16986	-464.363762446037\\
68	0.17352	-494.414785133348\\
68	0.17718	-525.68542542144\\
68	0.18084	-558.175683310314\\
68	0.1845	-591.885558799968\\
68	0.18816	-626.815051890403\\
68	0.19182	-662.964162581619\\
68	0.19548	-700.332890873616\\
68	0.19914	-738.921236766394\\
68	0.2028	-778.729200259953\\
68	0.20646	-819.756781354293\\
68	0.21012	-862.003980049413\\
68	0.21378	-905.470796345314\\
68	0.21744	-950.157230241996\\
68	0.2211	-996.06328173946\\
68	0.22476	-1043.1889508377\\
68	0.22842	-1091.53423753673\\
68	0.23208	-1141.09914183654\\
68	0.23574	-1191.88366373712\\
68	0.2394	-1243.88780323849\\
68	0.24306	-1297.11156034064\\
68	0.24672	-1351.55493504357\\
68	0.25038	-1407.21792734728\\
68	0.25404	-1464.10053725177\\
68	0.2577	-1522.20276475704\\
68	0.26136	-1581.5246098631\\
68	0.26502	-1642.06607256993\\
68	0.26868	-1703.82715287755\\
68	0.27234	-1766.80785078594\\
68	0.276	-1831.00816629512\\
68.375	0.093	-117.290303204823\\
68.375	0.09666	-121.758858753574\\
68.375	0.10032	-127.447031903106\\
68.375	0.10398	-134.354822653419\\
68.375	0.10764	-142.482231004513\\
68.375	0.1113	-151.829256956388\\
68.375	0.11496	-162.395900509043\\
68.375	0.11862	-174.182161662479\\
68.375	0.12228	-187.188040416697\\
68.375	0.12594	-201.413536771695\\
68.375	0.1296	-216.858650727475\\
68.375	0.13326	-233.523382284035\\
68.375	0.13692	-251.407731441376\\
68.375	0.14058	-270.511698199498\\
68.375	0.14424	-290.835282558401\\
68.375	0.1479	-312.378484518084\\
68.375	0.15156	-335.141304078549\\
68.375	0.15522	-359.123741239795\\
68.375	0.15888	-384.325796001822\\
68.375	0.16254	-410.747468364629\\
68.375	0.1662	-438.388758328218\\
68.375	0.16986	-467.249665892587\\
68.375	0.17352	-497.330191057737\\
68.375	0.17718	-528.630333823668\\
68.375	0.18084	-561.15009419038\\
68.375	0.1845	-594.889472157873\\
68.375	0.18816	-629.848467726147\\
68.375	0.19182	-666.027080895202\\
68.375	0.19548	-703.425311665037\\
68.375	0.19914	-742.043160035654\\
68.375	0.2028	-781.880626007052\\
68.375	0.20646	-822.93770957923\\
68.375	0.21012	-865.214410752189\\
68.375	0.21378	-908.71072952593\\
68.375	0.21744	-953.42666590045\\
68.375	0.2211	-999.362219875753\\
68.375	0.22476	-1046.51739145184\\
68.375	0.22842	-1094.8921806287\\
68.375	0.23208	-1144.48658740634\\
68.375	0.23574	-1195.30061178477\\
68.375	0.2394	-1247.33425376398\\
68.375	0.24306	-1300.58751334396\\
68.375	0.24672	-1355.06039052473\\
68.375	0.25038	-1410.75288530628\\
68.375	0.25404	-1467.66499768861\\
68.375	0.2577	-1525.79672767172\\
68.375	0.26136	-1585.14807525562\\
68.375	0.26502	-1645.71904044029\\
68.375	0.26868	-1707.50962322574\\
68.375	0.27234	-1770.51982361198\\
68.375	0.276	-1834.74964159899\\
68.75	0.093	-119.695258223877\\
68.75	0.09666	-124.193316250467\\
68.75	0.10032	-129.910991877838\\
68.75	0.10398	-136.848285105989\\
68.75	0.10764	-145.005195934922\\
68.75	0.1113	-154.381724364636\\
68.75	0.11496	-164.97787039513\\
68.75	0.11862	-176.793634026405\\
68.75	0.12228	-189.829015258462\\
68.75	0.12594	-204.084014091299\\
68.75	0.1296	-219.558630524916\\
68.75	0.13326	-236.252864559315\\
68.75	0.13692	-254.166716194495\\
68.75	0.14058	-273.300185430456\\
68.75	0.14424	-293.653272267198\\
68.75	0.1479	-315.22597670472\\
68.75	0.15156	-338.018298743024\\
68.75	0.15522	-362.030238382108\\
68.75	0.15888	-387.261795621974\\
68.75	0.16254	-413.71297046262\\
68.75	0.1662	-441.383762904048\\
68.75	0.16986	-470.274172946255\\
68.75	0.17352	-500.384200589244\\
68.75	0.17718	-531.713845833014\\
68.75	0.18084	-564.263108677565\\
68.75	0.1845	-598.031989122896\\
68.75	0.18816	-633.020487169009\\
68.75	0.19182	-669.228602815903\\
68.75	0.19548	-706.656336063577\\
68.75	0.19914	-745.303686912033\\
68.75	0.2028	-785.170655361269\\
68.75	0.20646	-826.257241411287\\
68.75	0.21012	-868.563445062084\\
68.75	0.21378	-912.089266313663\\
68.75	0.21744	-956.834705166023\\
68.75	0.2211	-1002.79976161916\\
68.75	0.22476	-1049.98443567309\\
68.75	0.22842	-1098.38872732779\\
68.75	0.23208	-1148.01263658327\\
68.75	0.23574	-1198.85616343954\\
68.75	0.2394	-1250.91930789658\\
68.75	0.24306	-1304.20206995441\\
68.75	0.24672	-1358.70444961302\\
68.75	0.25038	-1414.4264468724\\
68.75	0.25404	-1471.36806173257\\
68.75	0.2577	-1529.52929419352\\
68.75	0.26136	-1588.91014425525\\
68.75	0.26502	-1649.51061191777\\
68.75	0.26868	-1711.33069718106\\
68.75	0.27234	-1774.37040004513\\
68.75	0.276	-1838.62972050999\\
69.125	0.093	-122.238816850051\\
69.125	0.09666	-126.766377354479\\
69.125	0.10032	-132.513555459689\\
69.125	0.10398	-139.480351165679\\
69.125	0.10764	-147.66676447245\\
69.125	0.1113	-157.072795380003\\
69.125	0.11496	-167.698443888336\\
69.125	0.11862	-179.54370999745\\
69.125	0.12228	-192.608593707345\\
69.125	0.12594	-206.893095018021\\
69.125	0.1296	-222.397213929478\\
69.125	0.13326	-239.120950441715\\
69.125	0.13692	-257.064304554734\\
69.125	0.14058	-276.227276268533\\
69.125	0.14424	-296.609865583114\\
69.125	0.1479	-318.212072498475\\
69.125	0.15156	-341.033897014618\\
69.125	0.15522	-365.075339131541\\
69.125	0.15888	-390.336398849245\\
69.125	0.16254	-416.81707616773\\
69.125	0.1662	-444.517371086996\\
69.125	0.16986	-473.437283607043\\
69.125	0.17352	-503.576813727871\\
69.125	0.17718	-534.935961449479\\
69.125	0.18084	-567.514726771869\\
69.125	0.1845	-601.313109695039\\
69.125	0.18816	-636.331110218991\\
69.125	0.19182	-672.568728343723\\
69.125	0.19548	-710.025964069236\\
69.125	0.19914	-748.70281739553\\
69.125	0.2028	-788.599288322606\\
69.125	0.20646	-829.715376850462\\
69.125	0.21012	-872.051082979098\\
69.125	0.21378	-915.606406708516\\
69.125	0.21744	-960.381348038714\\
69.125	0.2211	-1006.37590696969\\
69.125	0.22476	-1053.59008350145\\
69.125	0.22842	-1102.023877634\\
69.125	0.23208	-1151.67728936732\\
69.125	0.23574	-1202.55031870142\\
69.125	0.2394	-1254.64296563631\\
69.125	0.24306	-1307.95523017197\\
69.125	0.24672	-1362.48711230842\\
69.125	0.25038	-1418.23861204565\\
69.125	0.25404	-1475.20972938365\\
69.125	0.2577	-1533.40046432244\\
69.125	0.26136	-1592.81081686201\\
69.125	0.26502	-1653.44078700236\\
69.125	0.26868	-1715.29037474349\\
69.125	0.27234	-1778.3595800854\\
69.125	0.276	-1842.6484030281\\
69.5	0.093	-124.920979083343\\
69.5	0.09666	-129.47804206561\\
69.5	0.10032	-135.254722648658\\
69.5	0.10398	-142.251020832488\\
69.5	0.10764	-150.466936617098\\
69.5	0.1113	-159.902470002489\\
69.5	0.11496	-170.557620988661\\
69.5	0.11862	-182.432389575614\\
69.5	0.12228	-195.526775763347\\
69.5	0.12594	-209.840779551862\\
69.5	0.1296	-225.374400941157\\
69.5	0.13326	-242.127639931234\\
69.5	0.13692	-260.100496522091\\
69.5	0.14058	-279.292970713729\\
69.5	0.14424	-299.705062506149\\
69.5	0.1479	-321.336771899349\\
69.5	0.15156	-344.18809889333\\
69.5	0.15522	-368.259043488092\\
69.5	0.15888	-393.549605683635\\
69.5	0.16254	-420.059785479959\\
69.5	0.1662	-447.789582877064\\
69.5	0.16986	-476.738997874949\\
69.5	0.17352	-506.908030473616\\
69.5	0.17718	-538.296680673063\\
69.5	0.18084	-570.904948473291\\
69.5	0.1845	-604.7328338743\\
69.5	0.18816	-639.780336876091\\
69.5	0.19182	-676.047457478662\\
69.5	0.19548	-713.534195682014\\
69.5	0.19914	-752.240551486147\\
69.5	0.2028	-792.166524891061\\
69.5	0.20646	-833.312115896756\\
69.5	0.21012	-875.677324503231\\
69.5	0.21378	-919.262150710488\\
69.5	0.21744	-964.066594518525\\
69.5	0.2211	-1010.09065592734\\
69.5	0.22476	-1057.33433493694\\
69.5	0.22842	-1105.79763154732\\
69.5	0.23208	-1155.48054575848\\
69.5	0.23574	-1206.38307757043\\
69.5	0.2394	-1258.50522698315\\
69.5	0.24306	-1311.84699399665\\
69.5	0.24672	-1366.40837861094\\
69.5	0.25038	-1422.189380826\\
69.5	0.25404	-1479.19000064185\\
69.5	0.2577	-1537.41023805848\\
69.5	0.26136	-1596.85009307589\\
69.5	0.26502	-1657.50956569408\\
69.5	0.26868	-1719.38865591305\\
69.5	0.27234	-1782.4873637328\\
69.5	0.276	-1846.80568915333\\
69.875	0.093	-127.741744923754\\
69.875	0.09666	-132.32831038386\\
69.875	0.10032	-138.134493444747\\
69.875	0.10398	-145.160294106415\\
69.875	0.10764	-153.405712368864\\
69.875	0.1113	-162.870748232094\\
69.875	0.11496	-173.555401696105\\
69.875	0.11862	-185.459672760896\\
69.875	0.12228	-198.583561426469\\
69.875	0.12594	-212.927067692822\\
69.875	0.1296	-228.490191559956\\
69.875	0.13326	-245.272933027872\\
69.875	0.13692	-263.275292096568\\
69.875	0.14058	-282.497268766045\\
69.875	0.14424	-302.938863036303\\
69.875	0.1479	-324.600074907341\\
69.875	0.15156	-347.480904379162\\
69.875	0.15522	-371.581351451762\\
69.875	0.15888	-396.901416125144\\
69.875	0.16254	-423.441098399307\\
69.875	0.1662	-451.200398274251\\
69.875	0.16986	-480.179315749975\\
69.875	0.17352	-510.37785082648\\
69.875	0.17718	-541.796003503766\\
69.875	0.18084	-574.433773781833\\
69.875	0.1845	-608.291161660681\\
69.875	0.18816	-643.36816714031\\
69.875	0.19182	-679.66479022072\\
69.875	0.19548	-717.181030901911\\
69.875	0.19914	-755.916889183883\\
69.875	0.2028	-795.872365066635\\
69.875	0.20646	-837.047458550169\\
69.875	0.21012	-879.442169634483\\
69.875	0.21378	-923.056498319579\\
69.875	0.21744	-967.890444605454\\
69.875	0.2211	-1013.94400849211\\
69.875	0.22476	-1061.21718997955\\
69.875	0.22842	-1109.70998906777\\
69.875	0.23208	-1159.42240575677\\
69.875	0.23574	-1210.35444004655\\
69.875	0.2394	-1262.50609193711\\
69.875	0.24306	-1315.87736142845\\
69.875	0.24672	-1370.46824852058\\
69.875	0.25038	-1426.27875321348\\
69.875	0.25404	-1483.30887550717\\
69.875	0.2577	-1541.55861540163\\
69.875	0.26136	-1601.02797289688\\
69.875	0.26502	-1661.71694799291\\
69.875	0.26868	-1723.62554068972\\
69.875	0.27234	-1786.75375098731\\
69.875	0.276	-1851.10157888568\\
70.25	0.093	-130.701114371285\\
70.25	0.09666	-135.31718230923\\
70.25	0.10032	-141.152867847956\\
70.25	0.10398	-148.208170987463\\
70.25	0.10764	-156.483091727751\\
70.25	0.1113	-165.977630068819\\
70.25	0.11496	-176.691786010669\\
70.25	0.11862	-188.625559553299\\
70.25	0.12228	-201.77895069671\\
70.25	0.12594	-216.151959440903\\
70.25	0.1296	-231.744585785876\\
70.25	0.13326	-248.556829731629\\
70.25	0.13692	-266.588691278165\\
70.25	0.14058	-285.84017042548\\
70.25	0.14424	-306.311267173578\\
70.25	0.1479	-328.001981522455\\
70.25	0.15156	-350.912313472114\\
70.25	0.15522	-375.042263022553\\
70.25	0.15888	-400.391830173774\\
70.25	0.16254	-426.961014925775\\
70.25	0.1662	-454.749817278558\\
70.25	0.16986	-483.758237232121\\
70.25	0.17352	-513.986274786464\\
70.25	0.17718	-545.433929941589\\
70.25	0.18084	-578.101202697495\\
70.25	0.1845	-611.988093054182\\
70.25	0.18816	-647.09460101165\\
70.25	0.19182	-683.420726569899\\
70.25	0.19548	-720.966469728928\\
70.25	0.19914	-759.731830488739\\
70.25	0.2028	-799.71680884933\\
70.25	0.20646	-840.921404810703\\
70.25	0.21012	-883.345618372855\\
70.25	0.21378	-926.98944953579\\
70.25	0.21744	-971.852898299504\\
70.25	0.2211	-1017.935964664\\
70.25	0.22476	-1065.23864862928\\
70.25	0.22842	-1113.76095019534\\
70.25	0.23208	-1163.50286936217\\
70.25	0.23574	-1214.46440612979\\
70.25	0.2394	-1266.64556049819\\
70.25	0.24306	-1320.04633246738\\
70.25	0.24672	-1374.66672203734\\
70.25	0.25038	-1430.50672920808\\
70.25	0.25404	-1487.56635397961\\
70.25	0.2577	-1545.84559635191\\
70.25	0.26136	-1605.344456325\\
70.25	0.26502	-1666.06293389886\\
70.25	0.26868	-1728.00102907351\\
70.25	0.27234	-1791.15874184894\\
70.25	0.276	-1855.53607222515\\
70.625	0.093	-133.799087425935\\
70.625	0.09666	-138.444657841719\\
70.625	0.10032	-144.309845858284\\
70.625	0.10398	-151.394651475629\\
70.625	0.10764	-159.699074693756\\
70.625	0.1113	-169.223115512663\\
70.625	0.11496	-179.966773932352\\
70.625	0.11862	-191.930049952821\\
70.625	0.12228	-205.112943574071\\
70.625	0.12594	-219.515454796102\\
70.625	0.1296	-235.137583618913\\
70.625	0.13326	-251.979330042506\\
70.625	0.13692	-270.04069406688\\
70.625	0.14058	-289.321675692034\\
70.625	0.14424	-309.822274917971\\
70.625	0.1479	-331.542491744686\\
70.625	0.15156	-354.482326172184\\
70.625	0.15522	-378.641778200462\\
70.625	0.15888	-404.020847829521\\
70.625	0.16254	-430.619535059362\\
70.625	0.1662	-458.437839889983\\
70.625	0.16986	-487.475762321385\\
70.625	0.17352	-517.733302353568\\
70.625	0.17718	-549.210459986531\\
70.625	0.18084	-581.907235220276\\
70.625	0.1845	-615.823628054801\\
70.625	0.18816	-650.959638490108\\
70.625	0.19182	-687.315266526195\\
70.625	0.19548	-724.890512163064\\
70.625	0.19914	-763.685375400713\\
70.625	0.2028	-803.699856239143\\
70.625	0.20646	-844.933954678355\\
70.625	0.21012	-887.387670718346\\
70.625	0.21378	-931.061004359119\\
70.625	0.21744	-975.953955600673\\
70.625	0.2211	-1022.06652444301\\
70.625	0.22476	-1069.39871088612\\
70.625	0.22842	-1117.95051493002\\
70.625	0.23208	-1167.7219365747\\
70.625	0.23574	-1218.71297582016\\
70.625	0.2394	-1270.92363266639\\
70.625	0.24306	-1324.35390711342\\
70.625	0.24672	-1379.00379916122\\
70.625	0.25038	-1434.8733088098\\
70.625	0.25404	-1491.96243605916\\
70.625	0.2577	-1550.27118090931\\
70.625	0.26136	-1609.79954336023\\
70.625	0.26502	-1670.54752341193\\
70.625	0.26868	-1732.51512106442\\
70.625	0.27234	-1795.70233631769\\
70.625	0.276	-1860.10916917174\\
71	0.093	-137.035664087704\\
71	0.09666	-141.710736981327\\
71	0.10032	-147.60542747573\\
71	0.10398	-154.719735570914\\
71	0.10764	-163.053661266879\\
71	0.1113	-172.607204563626\\
71	0.11496	-183.380365461153\\
71	0.11862	-195.373143959461\\
71	0.12228	-208.585540058549\\
71	0.12594	-223.017553758419\\
71	0.1296	-238.66918505907\\
71	0.13326	-255.540433960501\\
71	0.13692	-273.631300462714\\
71	0.14058	-292.941784565707\\
71	0.14424	-313.471886269482\\
71	0.1479	-335.221605574036\\
71	0.15156	-358.190942479373\\
71	0.15522	-382.37989698549\\
71	0.15888	-407.788469092388\\
71	0.16254	-434.416658800067\\
71	0.1662	-462.264466108527\\
71	0.16986	-491.331891017767\\
71	0.17352	-521.618933527789\\
71	0.17718	-553.125593638591\\
71	0.18084	-585.851871350175\\
71	0.1845	-619.797766662539\\
71	0.18816	-654.963279575685\\
71	0.19182	-691.348410089611\\
71	0.19548	-728.953158204318\\
71	0.19914	-767.777523919806\\
71	0.2028	-807.821507236075\\
71	0.20646	-849.085108153125\\
71	0.21012	-891.568326670955\\
71	0.21378	-935.271162789567\\
71	0.21744	-980.193616508959\\
71	0.2211	-1026.33568782913\\
71	0.22476	-1073.69737675009\\
71	0.22842	-1122.27868327182\\
71	0.23208	-1172.07960739434\\
71	0.23574	-1223.10014911764\\
71	0.2394	-1275.34030844171\\
71	0.24306	-1328.80008536657\\
71	0.24672	-1383.47947989221\\
71	0.25038	-1439.37849201863\\
71	0.25404	-1496.49712174584\\
71	0.2577	-1554.83536907382\\
71	0.26136	-1614.39323400258\\
71	0.26502	-1675.17071653213\\
71	0.26868	-1737.16781666245\\
71	0.27234	-1800.38453439356\\
71	0.276	-1864.82086972544\\
71.375	0.093	-140.410844356592\\
71.375	0.09666	-145.115419728054\\
71.375	0.10032	-151.039612700296\\
71.375	0.10398	-158.183423273318\\
71.375	0.10764	-166.546851447123\\
71.375	0.1113	-176.129897221708\\
71.375	0.11496	-186.932560597073\\
71.375	0.11862	-198.95484157322\\
71.375	0.12228	-212.196740150148\\
71.375	0.12594	-226.658256327856\\
71.375	0.1296	-242.339390106345\\
71.375	0.13326	-259.240141485616\\
71.375	0.13692	-277.360510465667\\
71.375	0.14058	-296.700497046499\\
71.375	0.14424	-317.260101228113\\
71.375	0.1479	-339.039323010506\\
71.375	0.15156	-362.038162393681\\
71.375	0.15522	-386.256619377637\\
71.375	0.15888	-411.694693962374\\
71.375	0.16254	-438.352386147892\\
71.375	0.1662	-466.229695934191\\
71.375	0.16986	-495.32662332127\\
71.375	0.17352	-525.64316830913\\
71.375	0.17718	-557.179330897771\\
71.375	0.18084	-589.935111087193\\
71.375	0.1845	-623.910508877396\\
71.375	0.18816	-659.105524268381\\
71.375	0.19182	-695.520157260146\\
71.375	0.19548	-733.154407852692\\
71.375	0.19914	-772.008276046018\\
71.375	0.2028	-812.081761840126\\
71.375	0.20646	-853.374865235015\\
71.375	0.21012	-895.887586230684\\
71.375	0.21378	-939.619924827135\\
71.375	0.21744	-984.571881024365\\
71.375	0.2211	-1030.74345482238\\
71.375	0.22476	-1078.13464622117\\
71.375	0.22842	-1126.74545522075\\
71.375	0.23208	-1176.5758818211\\
71.375	0.23574	-1227.62592602224\\
71.375	0.2394	-1279.89558782415\\
71.375	0.24306	-1333.38486722685\\
71.375	0.24672	-1388.09376423033\\
71.375	0.25038	-1444.02227883459\\
71.375	0.25404	-1501.17041103963\\
71.375	0.2577	-1559.53816084545\\
71.375	0.26136	-1619.12552825205\\
71.375	0.26502	-1679.93251325944\\
71.375	0.26868	-1741.9591158676\\
71.375	0.27234	-1805.20533607655\\
71.375	0.276	-1869.67117388627\\
71.75	0.093	-143.924628232599\\
71.75	0.09666	-148.658706081899\\
71.75	0.10032	-154.612401531981\\
71.75	0.10398	-161.785714582842\\
71.75	0.10764	-170.178645234485\\
71.75	0.1113	-179.791193486909\\
71.75	0.11496	-190.623359340113\\
71.75	0.11862	-202.675142794099\\
71.75	0.12228	-215.946543848865\\
71.75	0.12594	-230.437562504413\\
71.75	0.1296	-246.148198760741\\
71.75	0.13326	-263.07845261785\\
71.75	0.13692	-281.228324075739\\
71.75	0.14058	-300.597813134411\\
71.75	0.14424	-321.186919793863\\
71.75	0.1479	-342.995644054095\\
71.75	0.15156	-366.023985915109\\
71.75	0.15522	-390.271945376903\\
71.75	0.15888	-415.739522439479\\
71.75	0.16254	-442.426717102836\\
71.75	0.1662	-470.333529366973\\
71.75	0.16986	-499.459959231891\\
71.75	0.17352	-529.80600669759\\
71.75	0.17718	-561.37167176407\\
71.75	0.18084	-594.156954431332\\
71.75	0.1845	-628.161854699373\\
71.75	0.18816	-663.386372568196\\
71.75	0.19182	-699.8305080378\\
71.75	0.19548	-737.494261108185\\
71.75	0.19914	-776.37763177935\\
71.75	0.2028	-816.480620051296\\
71.75	0.20646	-857.803225924024\\
71.75	0.21012	-900.345449397532\\
71.75	0.21378	-944.107290471822\\
71.75	0.21744	-989.088749146891\\
71.75	0.2211	-1035.28982542274\\
71.75	0.22476	-1082.71051929937\\
71.75	0.22842	-1131.35083077679\\
71.75	0.23208	-1181.21075985498\\
71.75	0.23574	-1232.29030653396\\
71.75	0.2394	-1284.58947081371\\
71.75	0.24306	-1338.10825269425\\
71.75	0.24672	-1392.84665217556\\
71.75	0.25038	-1448.80466925766\\
71.75	0.25404	-1505.98230394054\\
71.75	0.2577	-1564.3795562242\\
71.75	0.26136	-1623.99642610864\\
71.75	0.26502	-1684.83291359387\\
71.75	0.26868	-1746.88901867987\\
71.75	0.27234	-1810.16474136665\\
71.75	0.276	-1874.66008165422\\
72.125	0.093	-147.577015715726\\
72.125	0.09666	-152.340596042865\\
72.125	0.10032	-158.323793970785\\
72.125	0.10398	-165.526609499485\\
72.125	0.10764	-173.949042628967\\
72.125	0.1113	-183.59109335923\\
72.125	0.11496	-194.452761690273\\
72.125	0.11862	-206.534047622097\\
72.125	0.12228	-219.834951154702\\
72.125	0.12594	-234.355472288088\\
72.125	0.1296	-250.095611022255\\
72.125	0.13326	-267.055367357203\\
72.125	0.13692	-285.234741292932\\
72.125	0.14058	-304.633732829441\\
72.125	0.14424	-325.252341966733\\
72.125	0.1479	-347.090568704803\\
72.125	0.15156	-370.148413043656\\
72.125	0.15522	-394.425874983289\\
72.125	0.15888	-419.922954523704\\
72.125	0.16254	-446.639651664899\\
72.125	0.1662	-474.575966406876\\
72.125	0.16986	-503.731898749633\\
72.125	0.17352	-534.10744869317\\
72.125	0.17718	-565.702616237489\\
72.125	0.18084	-598.517401382589\\
72.125	0.1845	-632.551804128469\\
72.125	0.18816	-667.805824475131\\
72.125	0.19182	-704.279462422574\\
72.125	0.19548	-741.972717970797\\
72.125	0.19914	-780.885591119801\\
72.125	0.2028	-821.018081869586\\
72.125	0.20646	-862.370190220153\\
72.125	0.21012	-904.9419161715\\
72.125	0.21378	-948.733259723628\\
72.125	0.21744	-993.744220876536\\
72.125	0.2211	-1039.97479963023\\
72.125	0.22476	-1087.4249959847\\
72.125	0.22842	-1136.09480993995\\
72.125	0.23208	-1185.98424149598\\
72.125	0.23574	-1237.0932906528\\
72.125	0.2394	-1289.42195741039\\
72.125	0.24306	-1342.97024176876\\
72.125	0.24672	-1397.73814372792\\
72.125	0.25038	-1453.72566328786\\
72.125	0.25404	-1510.93280044858\\
72.125	0.2577	-1569.35955521007\\
72.125	0.26136	-1629.00592757235\\
72.125	0.26502	-1689.87191753542\\
72.125	0.26868	-1751.95752509926\\
72.125	0.27234	-1815.26275026388\\
72.125	0.276	-1879.78759302928\\
72.5	0.093	-151.368006805972\\
72.5	0.09666	-156.161089610949\\
72.5	0.10032	-162.173790016708\\
72.5	0.10398	-169.406108023247\\
72.5	0.10764	-177.858043630568\\
72.5	0.1113	-187.529596838669\\
72.5	0.11496	-198.420767647551\\
72.5	0.11862	-210.531556057214\\
72.5	0.12228	-223.861962067658\\
72.5	0.12594	-238.411985678882\\
72.5	0.1296	-254.181626890888\\
72.5	0.13326	-271.170885703675\\
72.5	0.13692	-289.379762117242\\
72.5	0.14058	-308.808256131591\\
72.5	0.14424	-329.456367746721\\
72.5	0.1479	-351.32409696263\\
72.5	0.15156	-374.411443779322\\
72.5	0.15522	-398.718408196794\\
72.5	0.15888	-424.244990215047\\
72.5	0.16254	-450.991189834081\\
72.5	0.1662	-478.957007053897\\
72.5	0.16986	-508.142441874492\\
72.5	0.17352	-538.547494295869\\
72.5	0.17718	-570.172164318026\\
72.5	0.18084	-603.016451940965\\
72.5	0.1845	-637.080357164684\\
72.5	0.18816	-672.363879989184\\
72.5	0.19182	-708.867020414466\\
72.5	0.19548	-746.589778440528\\
72.5	0.19914	-785.532154067371\\
72.5	0.2028	-825.694147294995\\
72.5	0.20646	-867.0757581234\\
72.5	0.21012	-909.676986552586\\
72.5	0.21378	-953.497832582553\\
72.5	0.21744	-998.5382962133\\
72.5	0.2211	-1044.79837744483\\
72.5	0.22476	-1092.27807627714\\
72.5	0.22842	-1140.97739271023\\
72.5	0.23208	-1190.8963267441\\
72.5	0.23574	-1242.03487837875\\
72.5	0.2394	-1294.39304761419\\
72.5	0.24306	-1347.9708344504\\
72.5	0.24672	-1402.76823888739\\
72.5	0.25038	-1458.78526092517\\
72.5	0.25404	-1516.02190056373\\
72.5	0.2577	-1574.47815780306\\
72.5	0.26136	-1634.15403264318\\
72.5	0.26502	-1695.04952508408\\
72.5	0.26868	-1757.16463512576\\
72.5	0.27234	-1820.49936276822\\
72.5	0.276	-1885.05370801147\\
72.875	0.093	-155.297601503336\\
72.875	0.09666	-160.120186786153\\
72.875	0.10032	-166.16238966975\\
72.875	0.10398	-173.424210154128\\
72.875	0.10764	-181.905648239287\\
72.875	0.1113	-191.606703925227\\
72.875	0.11496	-202.527377211948\\
72.875	0.11862	-214.66766809945\\
72.875	0.12228	-228.027576587732\\
72.875	0.12594	-242.607102676796\\
72.875	0.1296	-258.406246366641\\
72.875	0.13326	-275.425007657266\\
72.875	0.13692	-293.663386548672\\
72.875	0.14058	-313.121383040859\\
72.875	0.14424	-333.798997133828\\
72.875	0.1479	-355.696228827577\\
72.875	0.15156	-378.813078122107\\
72.875	0.15522	-403.149545017417\\
72.875	0.15888	-428.70562951351\\
72.875	0.16254	-455.481331610383\\
72.875	0.1662	-483.476651308037\\
72.875	0.16986	-512.691588606471\\
72.875	0.17352	-543.126143505686\\
72.875	0.17718	-574.780316005682\\
72.875	0.18084	-607.65410610646\\
72.875	0.1845	-641.747513808018\\
72.875	0.18816	-677.060539110357\\
72.875	0.19182	-713.593182013477\\
72.875	0.19548	-751.345442517378\\
72.875	0.19914	-790.31732062206\\
72.875	0.2028	-830.508816327523\\
72.875	0.20646	-871.919929633767\\
72.875	0.21012	-914.550660540791\\
72.875	0.21378	-958.401009048597\\
72.875	0.21744	-1003.47097515718\\
72.875	0.2211	-1049.76055886655\\
72.875	0.22476	-1097.2697601767\\
72.875	0.22842	-1145.99857908763\\
72.875	0.23208	-1195.94701559934\\
72.875	0.23574	-1247.11506971183\\
72.875	0.2394	-1299.5027414251\\
72.875	0.24306	-1353.11003073915\\
72.875	0.24672	-1407.93693765399\\
72.875	0.25038	-1463.9834621696\\
72.875	0.25404	-1521.249604286\\
72.875	0.2577	-1579.73536400317\\
72.875	0.26136	-1639.44074132113\\
72.875	0.26502	-1700.36573623987\\
72.875	0.26868	-1762.51034875939\\
72.875	0.27234	-1825.87457887969\\
72.875	0.276	-1890.45842660077\\
73.25	0.093	-159.36579980782\\
73.25	0.09666	-164.217887568475\\
73.25	0.10032	-170.289592929911\\
73.25	0.10398	-177.580915892128\\
73.25	0.10764	-186.091856455126\\
73.25	0.1113	-195.822414618904\\
73.25	0.11496	-206.772590383464\\
73.25	0.11862	-218.942383748805\\
73.25	0.12228	-232.331794714926\\
73.25	0.12594	-246.940823281829\\
73.25	0.1296	-262.769469449512\\
73.25	0.13326	-279.817733217976\\
73.25	0.13692	-298.085614587221\\
73.25	0.14058	-317.573113557247\\
73.25	0.14424	-338.280230128054\\
73.25	0.1479	-360.206964299642\\
73.25	0.15156	-383.353316072011\\
73.25	0.15522	-407.71928544516\\
73.25	0.15888	-433.304872419091\\
73.25	0.16254	-460.110076993803\\
73.25	0.1662	-488.134899169295\\
73.25	0.16986	-517.379338945568\\
73.25	0.17352	-547.843396322623\\
73.25	0.17718	-579.527071300457\\
73.25	0.18084	-612.430363879074\\
73.25	0.1845	-646.55327405847\\
73.25	0.18816	-681.895801838649\\
73.25	0.19182	-718.457947219608\\
73.25	0.19548	-756.239710201347\\
73.25	0.19914	-795.241090783868\\
73.25	0.2028	-835.462088967169\\
73.25	0.20646	-876.902704751252\\
73.25	0.21012	-919.562938136115\\
73.25	0.21378	-963.44278912176\\
73.25	0.21744	-1008.54225770818\\
73.25	0.2211	-1054.86134389539\\
73.25	0.22476	-1102.40004768338\\
73.25	0.22842	-1151.15836907215\\
73.25	0.23208	-1201.13630806169\\
73.25	0.23574	-1252.33386465203\\
73.25	0.2394	-1304.75103884314\\
73.25	0.24306	-1358.38783063503\\
73.25	0.24672	-1413.2442400277\\
73.25	0.25038	-1469.32026702115\\
73.25	0.25404	-1526.61591161539\\
73.25	0.2577	-1585.1311738104\\
73.25	0.26136	-1644.8660536062\\
73.25	0.26502	-1705.82055100278\\
73.25	0.26868	-1767.99466600013\\
73.25	0.27234	-1831.38839859827\\
73.25	0.276	-1896.00174879719\\
73.625	0.093	-163.572601719423\\
73.625	0.09666	-168.454191957917\\
73.625	0.10032	-174.555399797192\\
73.625	0.10398	-181.876225237248\\
73.625	0.10764	-190.416668278084\\
73.625	0.1113	-200.176728919702\\
73.625	0.11496	-211.1564071621\\
73.625	0.11862	-223.35570300528\\
73.625	0.12228	-236.77461644924\\
73.625	0.12594	-251.413147493981\\
73.625	0.1296	-267.271296139503\\
73.625	0.13326	-284.349062385806\\
73.625	0.13692	-302.64644623289\\
73.625	0.14058	-322.163447680755\\
73.625	0.14424	-342.900066729401\\
73.625	0.1479	-364.856303378827\\
73.625	0.15156	-388.032157629035\\
73.625	0.15522	-412.427629480023\\
73.625	0.15888	-438.042718931793\\
73.625	0.16254	-464.877425984343\\
73.625	0.1662	-492.931750637675\\
73.625	0.16986	-522.205692891786\\
73.625	0.17352	-552.699252746679\\
73.625	0.17718	-584.412430202353\\
73.625	0.18084	-617.345225258808\\
73.625	0.1845	-651.497637916043\\
73.625	0.18816	-686.869668174061\\
73.625	0.19182	-723.461316032858\\
73.625	0.19548	-761.272581492437\\
73.625	0.19914	-800.303464552796\\
73.625	0.2028	-840.553965213936\\
73.625	0.20646	-882.024083475858\\
73.625	0.21012	-924.713819338559\\
73.625	0.21378	-968.623172802043\\
73.625	0.21744	-1013.75214386631\\
73.625	0.2211	-1060.10073253135\\
73.625	0.22476	-1107.66893879718\\
73.625	0.22842	-1156.45676266378\\
73.625	0.23208	-1206.46420413117\\
73.625	0.23574	-1257.69126319934\\
73.625	0.2394	-1310.13793986829\\
73.625	0.24306	-1363.80423413802\\
73.625	0.24672	-1418.69014600853\\
73.625	0.25038	-1474.79567547982\\
73.625	0.25404	-1532.1208225519\\
73.625	0.2577	-1590.66558722475\\
73.625	0.26136	-1650.42996949839\\
73.625	0.26502	-1711.4139693728\\
73.625	0.26868	-1773.617586848\\
73.625	0.27234	-1837.04082192398\\
73.625	0.276	-1901.68367460073\\
74	0.093	-167.918007238146\\
74	0.09666	-172.829099954478\\
74	0.10032	-178.959810271592\\
74	0.10398	-186.310138189486\\
74	0.10764	-194.880083708161\\
74	0.1113	-204.669646827618\\
74	0.11496	-215.678827547855\\
74	0.11862	-227.907625868873\\
74	0.12228	-241.356041790672\\
74	0.12594	-256.024075313252\\
74	0.1296	-271.911726436613\\
74	0.13326	-289.018995160754\\
74	0.13692	-307.345881485677\\
74	0.14058	-326.892385411381\\
74	0.14424	-347.658506937865\\
74	0.1479	-369.644246065131\\
74	0.15156	-392.849602793177\\
74	0.15522	-417.274577122004\\
74	0.15888	-442.919169051612\\
74	0.16254	-469.783378582002\\
74	0.1662	-497.867205713172\\
74	0.16986	-527.170650445122\\
74	0.17352	-557.693712777855\\
74	0.17718	-589.436392711367\\
74	0.18084	-622.39869024566\\
74	0.1845	-656.580605380735\\
74	0.18816	-691.98213811659\\
74	0.19182	-728.603288453227\\
74	0.19548	-766.444056390644\\
74	0.19914	-805.504441928842\\
74	0.2028	-845.784445067821\\
74	0.20646	-887.284065807582\\
74	0.21012	-930.003304148122\\
74	0.21378	-973.942160089444\\
74	0.21744	-1019.10063363155\\
74	0.2211	-1065.47872477443\\
74	0.22476	-1113.0764335181\\
74	0.22842	-1161.89375986254\\
74	0.23208	-1211.93070380777\\
74	0.23574	-1263.18726535377\\
74	0.2394	-1315.66344450056\\
74	0.24306	-1369.35924124813\\
74	0.24672	-1424.27465559648\\
74	0.25038	-1480.40968754561\\
74	0.25404	-1537.76433709553\\
74	0.2577	-1596.33860424622\\
74	0.26136	-1656.13248899769\\
74	0.26502	-1717.14599134995\\
74	0.26868	-1779.37911130298\\
74	0.27234	-1842.8318488568\\
74	0.276	-1907.50420401139\\
};
\end{axis}

\begin{axis}[%
width=4.527496cm,
height=3.870968cm,
at={(6.483547cm,5.376344cm)},
scale only axis,
xmin=56,
xmax=74,
tick align=outside,
xlabel={$L_{cut}$},
xmajorgrids,
ymin=0.093,
ymax=0.276,
ylabel={$D_{rlx}$},
ymajorgrids,
zmin=-1328.01931269086,
zmax=0,
zlabel={$x_4$},
zmajorgrids,
view={-140}{50},
legend style={at={(1.03,1)},anchor=north west,legend cell align=left,align=left,draw=white!15!black}
]
\addplot3[only marks,mark=*,mark options={},mark size=1.5000pt,color=mycolor1] plot table[row sep=crcr,]{%
74	0.123	-171.244880190644\\
72	0.113	-139.217460396701\\
61	0.095	-69.6051966848396\\
56	0.093	-73.6870961160989\\
};
\addplot3[only marks,mark=*,mark options={},mark size=1.5000pt,color=mycolor2] plot table[row sep=crcr,]{%
67	0.276	-1273.01999820883\\
66	0.255	-1046.38133574319\\
62	0.209	-579.722288788632\\
57	0.193	-465.590859275979\\
};
\addplot3[only marks,mark=*,mark options={},mark size=1.5000pt,color=black] plot table[row sep=crcr,]{%
69	0.104	-102.299172261903\\
};
\addplot3[only marks,mark=*,mark options={},mark size=1.5000pt,color=black] plot table[row sep=crcr,]{%
64	0.23	-773.899909377579\\
};

\addplot3[%
surf,
opacity=0.7,
shader=interp,
colormap={mymap}{[1pt] rgb(0pt)=(0.0901961,0.239216,0.0745098); rgb(1pt)=(0.0945149,0.242058,0.0739522); rgb(2pt)=(0.0988592,0.244894,0.0733566); rgb(3pt)=(0.103229,0.247724,0.0727241); rgb(4pt)=(0.107623,0.250549,0.0720557); rgb(5pt)=(0.112043,0.253367,0.0713525); rgb(6pt)=(0.116487,0.25618,0.0706154); rgb(7pt)=(0.120956,0.258986,0.0698456); rgb(8pt)=(0.125449,0.261787,0.0690441); rgb(9pt)=(0.129967,0.264581,0.0682118); rgb(10pt)=(0.134508,0.26737,0.06735); rgb(11pt)=(0.139074,0.270152,0.0664596); rgb(12pt)=(0.143663,0.272929,0.0655416); rgb(13pt)=(0.148275,0.275699,0.0645971); rgb(14pt)=(0.152911,0.278463,0.0636271); rgb(15pt)=(0.15757,0.281221,0.0626328); rgb(16pt)=(0.162252,0.283973,0.0616151); rgb(17pt)=(0.166957,0.286719,0.060575); rgb(18pt)=(0.171685,0.289458,0.0595136); rgb(19pt)=(0.176434,0.292191,0.0584321); rgb(20pt)=(0.181207,0.294918,0.0573313); rgb(21pt)=(0.186001,0.297639,0.0562123); rgb(22pt)=(0.190817,0.300353,0.0550763); rgb(23pt)=(0.195655,0.303061,0.0539242); rgb(24pt)=(0.200514,0.305763,0.052757); rgb(25pt)=(0.205395,0.308459,0.0515759); rgb(26pt)=(0.210296,0.311149,0.0503624); rgb(27pt)=(0.215212,0.313846,0.0490067); rgb(28pt)=(0.220142,0.316548,0.0475043); rgb(29pt)=(0.22509,0.319254,0.0458704); rgb(30pt)=(0.230056,0.321962,0.0441205); rgb(31pt)=(0.235042,0.324671,0.04227); rgb(32pt)=(0.240048,0.327379,0.0403343); rgb(33pt)=(0.245078,0.330085,0.0383287); rgb(34pt)=(0.250131,0.332786,0.0362688); rgb(35pt)=(0.25521,0.335482,0.0341698); rgb(36pt)=(0.260317,0.33817,0.0320472); rgb(37pt)=(0.265451,0.340849,0.0299163); rgb(38pt)=(0.270616,0.343517,0.0277927); rgb(39pt)=(0.275813,0.346172,0.0256916); rgb(40pt)=(0.281043,0.348814,0.0236284); rgb(41pt)=(0.286307,0.35144,0.0216186); rgb(42pt)=(0.291607,0.354048,0.0196776); rgb(43pt)=(0.296945,0.356637,0.0178207); rgb(44pt)=(0.302322,0.359206,0.0160634); rgb(45pt)=(0.307739,0.361753,0.0144211); rgb(46pt)=(0.313198,0.364275,0.0129091); rgb(47pt)=(0.318701,0.366772,0.0115428); rgb(48pt)=(0.324249,0.369242,0.0103377); rgb(49pt)=(0.329843,0.371682,0.00930909); rgb(50pt)=(0.335485,0.374093,0.00847245); rgb(51pt)=(0.341176,0.376471,0.00784314); rgb(52pt)=(0.346925,0.378826,0.00732741); rgb(53pt)=(0.352735,0.381168,0.00682184); rgb(54pt)=(0.358605,0.383497,0.00632729); rgb(55pt)=(0.364532,0.385812,0.00584464); rgb(56pt)=(0.370516,0.388113,0.00537476); rgb(57pt)=(0.376552,0.390399,0.00491852); rgb(58pt)=(0.38264,0.39267,0.00447681); rgb(59pt)=(0.388777,0.394925,0.00405048); rgb(60pt)=(0.394962,0.397164,0.00364042); rgb(61pt)=(0.401191,0.399386,0.00324749); rgb(62pt)=(0.407464,0.401592,0.00287258); rgb(63pt)=(0.413777,0.40378,0.00251655); rgb(64pt)=(0.420129,0.40595,0.00218028); rgb(65pt)=(0.426518,0.408102,0.00186463); rgb(66pt)=(0.432942,0.410234,0.00157049); rgb(67pt)=(0.439399,0.412348,0.00129873); rgb(68pt)=(0.445885,0.414441,0.00105022); rgb(69pt)=(0.452401,0.416515,0.000825833); rgb(70pt)=(0.458942,0.418567,0.000626441); rgb(71pt)=(0.465508,0.420599,0.00045292); rgb(72pt)=(0.472096,0.422609,0.000306141); rgb(73pt)=(0.478704,0.424596,0.000186979); rgb(74pt)=(0.485331,0.426562,9.63073e-05); rgb(75pt)=(0.491973,0.428504,3.49981e-05); rgb(76pt)=(0.498628,0.430422,3.92506e-06); rgb(77pt)=(0.505323,0.432315,0); rgb(78pt)=(0.512206,0.434168,0); rgb(79pt)=(0.519282,0.435983,0); rgb(80pt)=(0.526529,0.437764,0); rgb(81pt)=(0.533922,0.439512,0); rgb(82pt)=(0.54144,0.441232,0); rgb(83pt)=(0.549059,0.442927,0); rgb(84pt)=(0.556756,0.444599,0); rgb(85pt)=(0.564508,0.446252,0); rgb(86pt)=(0.572292,0.447889,0); rgb(87pt)=(0.580084,0.449514,0); rgb(88pt)=(0.587863,0.451129,0); rgb(89pt)=(0.595604,0.452737,0); rgb(90pt)=(0.603284,0.454343,0); rgb(91pt)=(0.610882,0.455948,0); rgb(92pt)=(0.618373,0.457556,0); rgb(93pt)=(0.625734,0.459171,0); rgb(94pt)=(0.632943,0.460795,0); rgb(95pt)=(0.639976,0.462432,0); rgb(96pt)=(0.64681,0.464084,0); rgb(97pt)=(0.653423,0.465756,0); rgb(98pt)=(0.659791,0.46745,0); rgb(99pt)=(0.665891,0.469169,0); rgb(100pt)=(0.6717,0.470916,0); rgb(101pt)=(0.677195,0.472696,0); rgb(102pt)=(0.682353,0.47451,0); rgb(103pt)=(0.687242,0.476355,0); rgb(104pt)=(0.691952,0.478225,0); rgb(105pt)=(0.696497,0.480118,0); rgb(106pt)=(0.700887,0.482033,0); rgb(107pt)=(0.705134,0.483968,0); rgb(108pt)=(0.709251,0.485921,0); rgb(109pt)=(0.713249,0.487891,0); rgb(110pt)=(0.71714,0.489876,0); rgb(111pt)=(0.720936,0.491875,0); rgb(112pt)=(0.724649,0.493887,0); rgb(113pt)=(0.72829,0.495909,0); rgb(114pt)=(0.731872,0.49794,0); rgb(115pt)=(0.735406,0.499979,0); rgb(116pt)=(0.738904,0.502025,0); rgb(117pt)=(0.742378,0.504075,0); rgb(118pt)=(0.74584,0.506128,0); rgb(119pt)=(0.749302,0.508182,0); rgb(120pt)=(0.752775,0.510237,0); rgb(121pt)=(0.756272,0.51229,0); rgb(122pt)=(0.759804,0.514339,0); rgb(123pt)=(0.763384,0.516385,0); rgb(124pt)=(0.767022,0.518424,0); rgb(125pt)=(0.770731,0.520455,0); rgb(126pt)=(0.774523,0.522478,0); rgb(127pt)=(0.77841,0.524489,0); rgb(128pt)=(0.782391,0.526491,0); rgb(129pt)=(0.786402,0.528496,0); rgb(130pt)=(0.790431,0.530506,0); rgb(131pt)=(0.794478,0.532521,0); rgb(132pt)=(0.798541,0.534539,0); rgb(133pt)=(0.802619,0.53656,0); rgb(134pt)=(0.806712,0.538584,0); rgb(135pt)=(0.81082,0.540609,0); rgb(136pt)=(0.81494,0.542635,0); rgb(137pt)=(0.819074,0.54466,0); rgb(138pt)=(0.823219,0.546686,0); rgb(139pt)=(0.827374,0.548709,0); rgb(140pt)=(0.831541,0.55073,0); rgb(141pt)=(0.835716,0.552749,0); rgb(142pt)=(0.8399,0.554763,0); rgb(143pt)=(0.844092,0.556774,0); rgb(144pt)=(0.848292,0.558779,0); rgb(145pt)=(0.852497,0.560778,0); rgb(146pt)=(0.856708,0.562771,0); rgb(147pt)=(0.860924,0.564756,0); rgb(148pt)=(0.865143,0.566733,0); rgb(149pt)=(0.869366,0.568701,0); rgb(150pt)=(0.873592,0.57066,0); rgb(151pt)=(0.877819,0.572608,0); rgb(152pt)=(0.882047,0.574545,0); rgb(153pt)=(0.886275,0.576471,0); rgb(154pt)=(0.890659,0.578362,0); rgb(155pt)=(0.895333,0.580203,0); rgb(156pt)=(0.900258,0.581999,0); rgb(157pt)=(0.905397,0.583755,0); rgb(158pt)=(0.910711,0.585479,0); rgb(159pt)=(0.916164,0.587176,0); rgb(160pt)=(0.921717,0.588852,0); rgb(161pt)=(0.927333,0.590513,0); rgb(162pt)=(0.932974,0.592166,0); rgb(163pt)=(0.938602,0.593815,0); rgb(164pt)=(0.94418,0.595468,0); rgb(165pt)=(0.949669,0.59713,0); rgb(166pt)=(0.955033,0.598808,0); rgb(167pt)=(0.960233,0.600507,0); rgb(168pt)=(0.965232,0.602233,0); rgb(169pt)=(0.969992,0.603992,0); rgb(170pt)=(0.974475,0.605791,0); rgb(171pt)=(0.978643,0.607636,0); rgb(172pt)=(0.98246,0.609532,0); rgb(173pt)=(0.985886,0.611486,0); rgb(174pt)=(0.988885,0.613503,0); rgb(175pt)=(0.991419,0.61559,0); rgb(176pt)=(0.99345,0.617753,0); rgb(177pt)=(0.99494,0.619997,0); rgb(178pt)=(0.995851,0.622329,0); rgb(179pt)=(0.996226,0.624763,0); rgb(180pt)=(0.996512,0.627352,0); rgb(181pt)=(0.996788,0.630095,0); rgb(182pt)=(0.997053,0.632982,0); rgb(183pt)=(0.997308,0.636004,0); rgb(184pt)=(0.997552,0.639152,0); rgb(185pt)=(0.997785,0.642416,0); rgb(186pt)=(0.998006,0.645786,0); rgb(187pt)=(0.998217,0.649253,0); rgb(188pt)=(0.998416,0.652807,0); rgb(189pt)=(0.998605,0.656439,0); rgb(190pt)=(0.998781,0.660138,0); rgb(191pt)=(0.998946,0.663897,0); rgb(192pt)=(0.9991,0.667704,0); rgb(193pt)=(0.999242,0.67155,0); rgb(194pt)=(0.999372,0.675427,0); rgb(195pt)=(0.99949,0.679323,0); rgb(196pt)=(0.999596,0.68323,0); rgb(197pt)=(0.99969,0.687139,0); rgb(198pt)=(0.999771,0.691039,0); rgb(199pt)=(0.999841,0.694921,0); rgb(200pt)=(0.999898,0.698775,0); rgb(201pt)=(0.999942,0.702592,0); rgb(202pt)=(0.999974,0.706363,0); rgb(203pt)=(0.999994,0.710077,0); rgb(204pt)=(1,0.713725,0); rgb(205pt)=(1,0.717341,0); rgb(206pt)=(1,0.720963,0); rgb(207pt)=(1,0.724591,0); rgb(208pt)=(1,0.728226,0); rgb(209pt)=(1,0.731867,0); rgb(210pt)=(1,0.735514,0); rgb(211pt)=(1,0.739167,0); rgb(212pt)=(1,0.742827,0); rgb(213pt)=(1,0.746493,0); rgb(214pt)=(1,0.750165,0); rgb(215pt)=(1,0.753843,0); rgb(216pt)=(1,0.757527,0); rgb(217pt)=(1,0.761217,0); rgb(218pt)=(1,0.764913,0); rgb(219pt)=(1,0.768615,0); rgb(220pt)=(1,0.772324,0); rgb(221pt)=(1,0.776038,0); rgb(222pt)=(1,0.779758,0); rgb(223pt)=(1,0.783484,0); rgb(224pt)=(1,0.787215,0); rgb(225pt)=(1,0.790953,0); rgb(226pt)=(1,0.794696,0); rgb(227pt)=(1,0.798445,0); rgb(228pt)=(1,0.8022,0); rgb(229pt)=(1,0.805961,0); rgb(230pt)=(1,0.809727,0); rgb(231pt)=(1,0.8135,0); rgb(232pt)=(1,0.817278,0); rgb(233pt)=(1,0.821063,0); rgb(234pt)=(1,0.824854,0); rgb(235pt)=(1,0.828652,0); rgb(236pt)=(1,0.832455,0); rgb(237pt)=(1,0.836265,0); rgb(238pt)=(1,0.840081,0); rgb(239pt)=(1,0.843903,0); rgb(240pt)=(1,0.847732,0); rgb(241pt)=(1,0.851566,0); rgb(242pt)=(1,0.855406,0); rgb(243pt)=(1,0.859253,0); rgb(244pt)=(1,0.863106,0); rgb(245pt)=(1,0.866964,0); rgb(246pt)=(1,0.870829,0); rgb(247pt)=(1,0.8747,0); rgb(248pt)=(1,0.878577,0); rgb(249pt)=(1,0.88246,0); rgb(250pt)=(1,0.886349,0); rgb(251pt)=(1,0.890243,0); rgb(252pt)=(1,0.894144,0); rgb(253pt)=(1,0.898051,0); rgb(254pt)=(1,0.901964,0); rgb(255pt)=(1,0.905882,0)},
mesh/rows=49]
table[row sep=crcr,header=false] {%
%
56	0.093	-73.0247201609133\\
56	0.09666	-76.5488775261567\\
56	0.10032	-80.8947757668137\\
56	0.10398	-86.0624148828836\\
56	0.10764	-92.0517948743666\\
56	0.1113	-98.8629157412631\\
56	0.11496	-106.495777483573\\
56	0.11862	-114.950380101296\\
56	0.12228	-124.226723594431\\
56	0.12594	-134.324807962981\\
56	0.1296	-145.244633206943\\
56	0.13326	-156.986199326319\\
56	0.13692	-169.549506321107\\
56	0.14058	-182.934554191309\\
56	0.14424	-197.141342936925\\
56	0.1479	-212.169872557953\\
56	0.15156	-228.020143054394\\
56	0.15522	-244.692154426249\\
56	0.15888	-262.185906673517\\
56	0.16254	-280.501399796198\\
56	0.1662	-299.638633794293\\
56	0.16986	-319.5976086678\\
56	0.17352	-340.378324416721\\
56	0.17718	-361.980781041055\\
56	0.18084	-384.404978540802\\
56	0.1845	-407.650916915963\\
56	0.18816	-431.718596166536\\
56	0.19182	-456.608016292523\\
56	0.19548	-482.319177293923\\
56	0.19914	-508.852079170736\\
56	0.2028	-536.206721922963\\
56	0.20646	-564.383105550602\\
56	0.21012	-593.381230053655\\
56	0.21378	-623.201095432121\\
56	0.21744	-653.842701686\\
56	0.2211	-685.306048815292\\
56	0.22476	-717.591136819998\\
56	0.22842	-750.697965700116\\
56	0.23208	-784.626535455648\\
56	0.23574	-819.376846086594\\
56	0.2394	-854.948897592952\\
56	0.24306	-891.342689974723\\
56	0.24672	-928.558223231908\\
56	0.25038	-966.595497364506\\
56	0.25404	-1005.45451237252\\
56	0.2577	-1045.13526825594\\
56	0.26136	-1085.63776501478\\
56	0.26502	-1126.96200264903\\
56	0.26868	-1169.10798115869\\
56	0.27234	-1212.07570054377\\
56	0.276	-1255.86516080426\\
56.375	0.093	-72.2605225046986\\
56.375	0.09666	-75.7966422256901\\
56.375	0.10032	-80.1545028220945\\
56.375	0.10398	-85.3341042939122\\
56.375	0.10764	-91.3354466411432\\
56.375	0.1113	-98.1585298637872\\
56.375	0.11496	-105.803353961845\\
56.375	0.11862	-114.269918935315\\
56.375	0.12228	-123.558224784199\\
56.375	0.12594	-133.668271508496\\
56.375	0.1296	-144.600059108206\\
56.375	0.13326	-156.353587583329\\
56.375	0.13692	-168.928856933866\\
56.375	0.14058	-182.325867159816\\
56.375	0.14424	-196.544618261178\\
56.375	0.1479	-211.585110237955\\
56.375	0.15156	-227.447343090144\\
56.375	0.15522	-244.131316817747\\
56.375	0.15888	-261.637031420762\\
56.375	0.16254	-279.964486899192\\
56.375	0.1662	-299.113683253033\\
56.375	0.16986	-319.084620482289\\
56.375	0.17352	-339.877298586957\\
56.375	0.17718	-361.491717567039\\
56.375	0.18084	-383.927877422534\\
56.375	0.1845	-407.185778153442\\
56.375	0.18816	-431.265419759763\\
56.375	0.19182	-456.166802241498\\
56.375	0.19548	-481.889925598646\\
56.375	0.19914	-508.434789831206\\
56.375	0.2028	-535.801394939181\\
56.375	0.20646	-563.989740922568\\
56.375	0.21012	-592.999827781368\\
56.375	0.21378	-622.831655515582\\
56.375	0.21744	-653.485224125209\\
56.375	0.2211	-684.960533610249\\
56.375	0.22476	-717.257583970702\\
56.375	0.22842	-750.376375206569\\
56.375	0.23208	-784.316907317849\\
56.375	0.23574	-819.079180304542\\
56.375	0.2394	-854.663194166648\\
56.375	0.24306	-891.068948904167\\
56.375	0.24672	-928.296444517099\\
56.375	0.25038	-966.345681005445\\
56.375	0.25404	-1005.2166583692\\
56.375	0.2577	-1044.90937660838\\
56.375	0.26136	-1085.42383572296\\
56.375	0.26502	-1126.76003571296\\
56.375	0.26868	-1168.91797657837\\
56.375	0.27234	-1211.8976583192\\
56.375	0.276	-1255.69908093543\\
56.75	0.093	-71.5673585236526\\
56.75	0.09666	-75.1154406003918\\
56.75	0.10032	-79.485263552544\\
56.75	0.10398	-84.6768273801095\\
56.75	0.10764	-90.6901320830883\\
56.75	0.1113	-97.52517766148\\
56.75	0.11496	-105.181964115285\\
56.75	0.11862	-113.660491444503\\
56.75	0.12228	-122.960759649135\\
56.75	0.12594	-133.08276872918\\
56.75	0.1296	-144.026518684638\\
56.75	0.13326	-155.792009515509\\
56.75	0.13692	-168.379241221793\\
56.75	0.14058	-181.788213803491\\
56.75	0.14424	-196.018927260601\\
56.75	0.1479	-211.071381593125\\
56.75	0.15156	-226.945576801062\\
56.75	0.15522	-243.641512884412\\
56.75	0.15888	-261.159189843176\\
56.75	0.16254	-279.498607677353\\
56.75	0.1662	-298.659766386942\\
56.75	0.16986	-318.642665971946\\
56.75	0.17352	-339.447306432362\\
56.75	0.17718	-361.073687768191\\
56.75	0.18084	-383.521809979434\\
56.75	0.1845	-406.79167306609\\
56.75	0.18816	-430.883277028159\\
56.75	0.19182	-455.796621865641\\
56.75	0.19548	-481.531707578537\\
56.75	0.19914	-508.088534166845\\
56.75	0.2028	-535.467101630567\\
56.75	0.20646	-563.667409969703\\
56.75	0.21012	-592.689459184251\\
56.75	0.21378	-622.533249274212\\
56.75	0.21744	-653.198780239587\\
56.75	0.2211	-684.686052080375\\
56.75	0.22476	-716.995064796576\\
56.75	0.22842	-750.12581838819\\
56.75	0.23208	-784.078312855218\\
56.75	0.23574	-818.852548197658\\
56.75	0.2394	-854.448524415512\\
56.75	0.24306	-890.866241508779\\
56.75	0.24672	-928.105699477459\\
56.75	0.25038	-966.166898321553\\
56.75	0.25404	-1005.04983804106\\
56.75	0.2577	-1044.75451863598\\
56.75	0.26136	-1085.28094010631\\
56.75	0.26502	-1126.62910245206\\
56.75	0.26868	-1168.79900567322\\
56.75	0.27234	-1211.79064976979\\
56.75	0.276	-1255.60403474178\\
57.125	0.093	-70.9452282177755\\
57.125	0.09666	-74.5052726502622\\
57.125	0.10032	-78.8870579581624\\
57.125	0.10398	-84.0905841414756\\
57.125	0.10764	-90.1158512002021\\
57.125	0.1113	-96.9628591343419\\
57.125	0.11496	-104.631607943894\\
57.125	0.11862	-113.122097628861\\
57.125	0.12228	-122.43432818924\\
57.125	0.12594	-132.568299625032\\
57.125	0.1296	-143.524011936238\\
57.125	0.13326	-155.301465122857\\
57.125	0.13692	-167.900659184889\\
57.125	0.14058	-181.321594122334\\
57.125	0.14424	-195.564269935193\\
57.125	0.1479	-210.628686623464\\
57.125	0.15156	-226.514844187149\\
57.125	0.15522	-243.222742626247\\
57.125	0.15888	-260.752381940759\\
57.125	0.16254	-279.103762130683\\
57.125	0.1662	-298.276883196021\\
57.125	0.16986	-318.271745136771\\
57.125	0.17352	-339.088347952935\\
57.125	0.17718	-360.726691644512\\
57.125	0.18084	-383.186776211503\\
57.125	0.1845	-406.468601653907\\
57.125	0.18816	-430.572167971724\\
57.125	0.19182	-455.497475164954\\
57.125	0.19548	-481.244523233597\\
57.125	0.19914	-507.813312177653\\
57.125	0.2028	-535.203841997123\\
57.125	0.20646	-563.416112692006\\
57.125	0.21012	-592.450124262302\\
57.125	0.21378	-622.305876708011\\
57.125	0.21744	-652.983370029133\\
57.125	0.2211	-684.482604225669\\
57.125	0.22476	-716.803579297618\\
57.125	0.22842	-749.94629524498\\
57.125	0.23208	-783.910752067755\\
57.125	0.23574	-818.696949765944\\
57.125	0.2394	-854.304888339545\\
57.125	0.24306	-890.73456778856\\
57.125	0.24672	-927.985988112988\\
57.125	0.25038	-966.059149312829\\
57.125	0.25404	-1004.95405138808\\
57.125	0.2577	-1044.67069433875\\
57.125	0.26136	-1085.20907816483\\
57.125	0.26502	-1126.56920286633\\
57.125	0.26868	-1168.75106844323\\
57.125	0.27234	-1211.75467489555\\
57.125	0.276	-1255.58002222329\\
57.5	0.093	-70.394131587067\\
57.5	0.09666	-73.9661383753014\\
57.5	0.10032	-78.3598860389494\\
57.5	0.10398	-83.5753745780103\\
57.5	0.10764	-89.6126039924844\\
57.5	0.1113	-96.4715742823719\\
57.5	0.11496	-104.152285447672\\
57.5	0.11862	-112.654737488386\\
57.5	0.12228	-121.978930404513\\
57.5	0.12594	-132.124864196054\\
57.5	0.1296	-143.092538863007\\
57.5	0.13326	-154.881954405374\\
57.5	0.13692	-167.493110823153\\
57.5	0.14058	-180.926008116346\\
57.5	0.14424	-195.180646284953\\
57.5	0.1479	-210.257025328972\\
57.5	0.15156	-226.155145248405\\
57.5	0.15522	-242.87500604325\\
57.5	0.15888	-260.41660771351\\
57.5	0.16254	-278.779950259182\\
57.5	0.1662	-297.965033680267\\
57.5	0.16986	-317.971857976766\\
57.5	0.17352	-338.800423148678\\
57.5	0.17718	-360.450729196002\\
57.5	0.18084	-382.92277611874\\
57.5	0.1845	-406.216563916892\\
57.5	0.18816	-430.332092590457\\
57.5	0.19182	-455.269362139434\\
57.5	0.19548	-481.028372563825\\
57.5	0.19914	-507.60912386363\\
57.5	0.2028	-535.011616038847\\
57.5	0.20646	-563.235849089478\\
57.5	0.21012	-592.281823015521\\
57.5	0.21378	-622.149537816978\\
57.5	0.21744	-652.838993493849\\
57.5	0.2211	-684.350190046132\\
57.5	0.22476	-716.683127473828\\
57.5	0.22842	-749.837805776938\\
57.5	0.23208	-783.814224955461\\
57.5	0.23574	-818.612385009398\\
57.5	0.2394	-854.232285938747\\
57.5	0.24306	-890.673927743509\\
57.5	0.24672	-927.937310423685\\
57.5	0.25038	-966.022433979274\\
57.5	0.25404	-1004.92929841028\\
57.5	0.2577	-1044.65790371669\\
57.5	0.26136	-1085.20824989852\\
57.5	0.26502	-1126.58033695576\\
57.5	0.26868	-1168.77416488842\\
57.5	0.27234	-1211.78973369648\\
57.5	0.276	-1255.62704337997\\
57.875	0.093	-69.914068631527\\
57.875	0.09666	-73.4980377755096\\
57.875	0.10032	-77.9037477949049\\
57.875	0.10398	-83.1311986897136\\
57.875	0.10764	-89.1803904599356\\
57.875	0.1113	-96.0513231055706\\
57.875	0.11496	-103.743996626619\\
57.875	0.11862	-112.258411023081\\
57.875	0.12228	-121.594566294956\\
57.875	0.12594	-131.752462442243\\
57.875	0.1296	-142.732099464945\\
57.875	0.13326	-154.533477363059\\
57.875	0.13692	-167.156596136587\\
57.875	0.14058	-180.601455785527\\
57.875	0.14424	-194.868056309881\\
57.875	0.1479	-209.956397709649\\
57.875	0.15156	-225.866479984829\\
57.875	0.15522	-242.598303135422\\
57.875	0.15888	-260.151867161429\\
57.875	0.16254	-278.527172062849\\
57.875	0.1662	-297.724217839682\\
57.875	0.16986	-317.743004491929\\
57.875	0.17352	-338.583532019588\\
57.875	0.17718	-360.245800422661\\
57.875	0.18084	-382.729809701147\\
57.875	0.1845	-406.035559855046\\
57.875	0.18816	-430.163050884359\\
57.875	0.19182	-455.112282789084\\
57.875	0.19548	-480.883255569223\\
57.875	0.19914	-507.475969224775\\
57.875	0.2028	-534.89042375574\\
57.875	0.20646	-563.126619162118\\
57.875	0.21012	-592.18455544391\\
57.875	0.21378	-622.064232601115\\
57.875	0.21744	-652.765650633733\\
57.875	0.2211	-684.288809541764\\
57.875	0.22476	-716.633709325208\\
57.875	0.22842	-749.800349984065\\
57.875	0.23208	-783.788731518336\\
57.875	0.23574	-818.59885392802\\
57.875	0.2394	-854.230717213117\\
57.875	0.24306	-890.684321373628\\
57.875	0.24672	-927.959666409551\\
57.875	0.25038	-966.056752320888\\
57.875	0.25404	-1004.97557910764\\
57.875	0.2577	-1044.7161467698\\
57.875	0.26136	-1085.27845530738\\
57.875	0.26502	-1126.66250472037\\
57.875	0.26868	-1168.86829500877\\
57.875	0.27234	-1211.89582617259\\
57.875	0.276	-1255.74509821181\\
58.25	0.093	-69.5050393511555\\
58.25	0.09666	-73.1009708508859\\
58.25	0.10032	-77.5186432260292\\
58.25	0.10398	-82.7580564765856\\
58.25	0.10764	-88.8192106025554\\
58.25	0.1113	-95.7021056039382\\
58.25	0.11496	-103.406741480735\\
58.25	0.11862	-111.933118232944\\
58.25	0.12228	-121.281235860567\\
58.25	0.12594	-131.451094363602\\
58.25	0.1296	-142.442693742051\\
58.25	0.13326	-154.256033995913\\
58.25	0.13692	-166.891115125188\\
58.25	0.14058	-180.347937129877\\
58.25	0.14424	-194.626500009979\\
58.25	0.1479	-209.726803765494\\
58.25	0.15156	-225.648848396422\\
58.25	0.15522	-242.392633902763\\
58.25	0.15888	-259.958160284517\\
58.25	0.16254	-278.345427541685\\
58.25	0.1662	-297.554435674266\\
58.25	0.16986	-317.585184682261\\
58.25	0.17352	-338.437674565668\\
58.25	0.17718	-360.111905324488\\
58.25	0.18084	-382.607876958722\\
58.25	0.1845	-405.925589468369\\
58.25	0.18816	-430.065042853429\\
58.25	0.19182	-455.026237113902\\
58.25	0.19548	-480.809172249789\\
58.25	0.19914	-507.413848261088\\
58.25	0.2028	-534.840265147801\\
58.25	0.20646	-563.088422909928\\
58.25	0.21012	-592.158321547467\\
58.25	0.21378	-622.049961060419\\
58.25	0.21744	-652.763341448785\\
58.25	0.2211	-684.298462712564\\
58.25	0.22476	-716.655324851756\\
58.25	0.22842	-749.833927866361\\
58.25	0.23208	-783.83427175638\\
58.25	0.23574	-818.656356521812\\
58.25	0.2394	-854.300182162656\\
58.25	0.24306	-890.765748678914\\
58.25	0.24672	-928.053056070586\\
58.25	0.25038	-966.16210433767\\
58.25	0.25404	-1005.09289348017\\
58.25	0.2577	-1044.84542349808\\
58.25	0.26136	-1085.4196943914\\
58.25	0.26502	-1126.81570616014\\
58.25	0.26868	-1169.03345880429\\
58.25	0.27234	-1212.07295232385\\
58.25	0.276	-1255.93418671883\\
58.625	0.093	-69.1670437459534\\
58.625	0.09666	-72.7749376014311\\
58.625	0.10032	-77.2045723323224\\
58.625	0.10398	-82.4559479386266\\
58.625	0.10764	-88.5290644203441\\
58.625	0.1113	-95.4239217774749\\
58.625	0.11496	-103.140520010019\\
58.625	0.11862	-111.678859117976\\
58.625	0.12228	-121.038939101346\\
58.625	0.12594	-131.22075996013\\
58.625	0.1296	-142.224321694326\\
58.625	0.13326	-154.049624303936\\
58.625	0.13692	-166.696667788959\\
58.625	0.14058	-180.165452149395\\
58.625	0.14424	-194.455977385245\\
58.625	0.1479	-209.568243496508\\
58.625	0.15156	-225.502250483184\\
58.625	0.15522	-242.257998345272\\
58.625	0.15888	-259.835487082775\\
58.625	0.16254	-278.23471669569\\
58.625	0.1662	-297.455687184019\\
58.625	0.16986	-317.498398547761\\
58.625	0.17352	-338.362850786916\\
58.625	0.17718	-360.049043901484\\
58.625	0.18084	-382.556977891466\\
58.625	0.1845	-405.88665275686\\
58.625	0.18816	-430.038068497668\\
58.625	0.19182	-455.01122511389\\
58.625	0.19548	-480.806122605523\\
58.625	0.19914	-507.422760972571\\
58.625	0.2028	-534.861140215032\\
58.625	0.20646	-563.121260332906\\
58.625	0.21012	-592.203121326193\\
58.625	0.21378	-622.106723194893\\
58.625	0.21744	-652.832065939006\\
58.625	0.2211	-684.379149558533\\
58.625	0.22476	-716.747974053473\\
58.625	0.22842	-749.938539423826\\
58.625	0.23208	-783.950845669592\\
58.625	0.23574	-818.784892790772\\
58.625	0.2394	-854.440680787364\\
58.625	0.24306	-890.91820965937\\
58.625	0.24672	-928.217479406789\\
58.625	0.25038	-966.338490029622\\
58.625	0.25404	-1005.28124152787\\
58.625	0.2577	-1045.04573390153\\
58.625	0.26136	-1085.6319671506\\
58.625	0.26502	-1127.03994127508\\
58.625	0.26868	-1169.26965627498\\
58.625	0.27234	-1212.32111215029\\
58.625	0.276	-1256.19430890102\\
59	0.093	-68.9000818159197\\
59	0.09666	-72.5199380271451\\
59	0.10032	-76.9615351137841\\
59	0.10398	-82.2248730758361\\
59	0.10764	-88.3099519133012\\
59	0.1113	-95.2167716261797\\
59	0.11496	-102.945332214471\\
59	0.11862	-111.495633678176\\
59	0.12228	-120.867676017294\\
59	0.12594	-131.061459231826\\
59	0.1296	-142.07698332177\\
59	0.13326	-153.914248287128\\
59	0.13692	-166.573254127898\\
59	0.14058	-180.054000844083\\
59	0.14424	-194.35648843568\\
59	0.1479	-209.48071690269\\
59	0.15156	-225.426686245114\\
59	0.15522	-242.194396462951\\
59	0.15888	-259.783847556201\\
59	0.16254	-278.195039524864\\
59	0.1662	-297.42797236894\\
59	0.16986	-317.48264608843\\
59	0.17352	-338.359060683333\\
59	0.17718	-360.057216153649\\
59	0.18084	-382.577112499378\\
59	0.1845	-405.918749720521\\
59	0.18816	-430.082127817076\\
59	0.19182	-455.067246789045\\
59	0.19548	-480.874106636427\\
59	0.19914	-507.502707359222\\
59	0.2028	-534.953048957431\\
59	0.20646	-563.225131431052\\
59	0.21012	-592.318954780087\\
59	0.21378	-622.234519004535\\
59	0.21744	-652.971824104396\\
59	0.2211	-684.530870079671\\
59	0.22476	-716.911656930358\\
59	0.22842	-750.114184656459\\
59	0.23208	-784.138453257973\\
59	0.23574	-818.984462734901\\
59	0.2394	-854.652213087241\\
59	0.24306	-891.141704314994\\
59	0.24672	-928.452936418161\\
59	0.25038	-966.585909396741\\
59	0.25404	-1005.54062325073\\
59	0.2577	-1045.31707798014\\
59	0.26136	-1085.91527358496\\
59	0.26502	-1127.33521006519\\
59	0.26868	-1169.57688742084\\
59	0.27234	-1212.6403056519\\
59	0.276	-1256.52546475837\\
59.375	0.093	-68.7041535610545\\
59.375	0.09666	-72.3359721280279\\
59.375	0.10032	-76.7895315704145\\
59.375	0.10398	-82.0648318882142\\
59.375	0.10764	-88.1618730814272\\
59.375	0.1113	-95.0806551500533\\
59.375	0.11496	-102.821178094093\\
59.375	0.11862	-111.383441913545\\
59.375	0.12228	-120.767446608411\\
59.375	0.12594	-130.97319217869\\
59.375	0.1296	-142.000678624383\\
59.375	0.13326	-153.849905945488\\
59.375	0.13692	-166.520874142007\\
59.375	0.14058	-180.013583213938\\
59.375	0.14424	-194.328033161283\\
59.375	0.1479	-209.464223984042\\
59.375	0.15156	-225.422155682213\\
59.375	0.15522	-242.201828255797\\
59.375	0.15888	-259.803241704795\\
59.375	0.16254	-278.226396029207\\
59.375	0.1662	-297.47129122903\\
59.375	0.16986	-317.537927304268\\
59.375	0.17352	-338.426304254918\\
59.375	0.17718	-360.136422080982\\
59.375	0.18084	-382.668280782459\\
59.375	0.1845	-406.021880359349\\
59.375	0.18816	-430.197220811653\\
59.375	0.19182	-455.19430213937\\
59.375	0.19548	-481.013124342499\\
59.375	0.19914	-507.653687421042\\
59.375	0.2028	-535.115991374998\\
59.375	0.20646	-563.400036204368\\
59.375	0.21012	-592.50582190915\\
59.375	0.21378	-622.433348489346\\
59.375	0.21744	-653.182615944955\\
59.375	0.2211	-684.753624275977\\
59.375	0.22476	-717.146373482413\\
59.375	0.22842	-750.360863564261\\
59.375	0.23208	-784.397094521523\\
59.375	0.23574	-819.255066354198\\
59.375	0.2394	-854.934779062286\\
59.375	0.24306	-891.436232645788\\
59.375	0.24672	-928.759427104702\\
59.375	0.25038	-966.90436243903\\
59.375	0.25404	-1005.87103864877\\
59.375	0.2577	-1045.65945573393\\
59.375	0.26136	-1086.26961369449\\
59.375	0.26502	-1127.70151253047\\
59.375	0.26868	-1169.95515224187\\
59.375	0.27234	-1213.03053282867\\
59.375	0.276	-1256.92765429089\\
59.75	0.093	-68.5792589813578\\
59.75	0.09666	-72.2230399040792\\
59.75	0.10032	-76.6885617022135\\
59.75	0.10398	-81.975824375761\\
59.75	0.10764	-88.0848279247218\\
59.75	0.1113	-95.0155723490956\\
59.75	0.11496	-102.768057648883\\
59.75	0.11862	-111.342283824083\\
59.75	0.12228	-120.738250874697\\
59.75	0.12594	-130.955958800724\\
59.75	0.1296	-141.995407602164\\
59.75	0.13326	-153.856597279017\\
59.75	0.13692	-166.539527831283\\
59.75	0.14058	-180.044199258963\\
59.75	0.14424	-194.370611562056\\
59.75	0.1479	-209.518764740561\\
59.75	0.15156	-225.48865879448\\
59.75	0.15522	-242.280293723813\\
59.75	0.15888	-259.893669528558\\
59.75	0.16254	-278.328786208717\\
59.75	0.1662	-297.585643764289\\
59.75	0.16986	-317.664242195274\\
59.75	0.17352	-338.564581501672\\
59.75	0.17718	-360.286661683484\\
59.75	0.18084	-382.830482740709\\
59.75	0.1845	-406.196044673347\\
59.75	0.18816	-430.383347481398\\
59.75	0.19182	-455.392391164862\\
59.75	0.19548	-481.22317572374\\
59.75	0.19914	-507.875701158031\\
59.75	0.2028	-535.349967467735\\
59.75	0.20646	-563.645974652852\\
59.75	0.21012	-592.763722713382\\
59.75	0.21378	-622.703211649325\\
59.75	0.21744	-653.464441460682\\
59.75	0.2211	-685.047412147452\\
59.75	0.22476	-717.452123709635\\
59.75	0.22842	-750.678576147232\\
59.75	0.23208	-784.726769460241\\
59.75	0.23574	-819.596703648664\\
59.75	0.2394	-855.2883787125\\
59.75	0.24306	-891.801794651749\\
59.75	0.24672	-929.136951466411\\
59.75	0.25038	-967.293849156487\\
59.75	0.25404	-1006.27248772198\\
59.75	0.2577	-1046.07286716288\\
59.75	0.26136	-1086.69498747919\\
59.75	0.26502	-1128.13884867092\\
59.75	0.26868	-1170.40445073806\\
59.75	0.27234	-1213.49179368062\\
59.75	0.276	-1257.40087749859\\
60.125	0.093	-68.5253980768301\\
60.125	0.09666	-72.1811413552992\\
60.125	0.10032	-76.6586255091813\\
60.125	0.10398	-81.9578505384765\\
60.125	0.10764	-88.0788164431851\\
60.125	0.1113	-95.0215232233066\\
60.125	0.11496	-102.785970878842\\
60.125	0.11862	-111.37215940979\\
60.125	0.12228	-120.780088816151\\
60.125	0.12594	-131.009759097926\\
60.125	0.1296	-142.061170255113\\
60.125	0.13326	-153.934322287714\\
60.125	0.13692	-166.629215195728\\
60.125	0.14058	-180.145848979156\\
60.125	0.14424	-194.484223637996\\
60.125	0.1479	-209.64433917225\\
60.125	0.15156	-225.626195581917\\
60.125	0.15522	-242.429792866997\\
60.125	0.15888	-260.05513102749\\
60.125	0.16254	-278.502210063397\\
60.125	0.1662	-297.771029974716\\
60.125	0.16986	-317.86159076145\\
60.125	0.17352	-338.773892423595\\
60.125	0.17718	-360.507934961155\\
60.125	0.18084	-383.063718374127\\
60.125	0.1845	-406.441242662513\\
60.125	0.18816	-430.640507826312\\
60.125	0.19182	-455.661513865524\\
60.125	0.19548	-481.504260780149\\
60.125	0.19914	-508.168748570188\\
60.125	0.2028	-535.65497723564\\
60.125	0.20646	-563.962946776505\\
60.125	0.21012	-593.092657192783\\
60.125	0.21378	-623.044108484474\\
60.125	0.21744	-653.817300651578\\
60.125	0.2211	-685.412233694096\\
60.125	0.22476	-717.828907612027\\
60.125	0.22842	-751.067322405371\\
60.125	0.23208	-785.127478074128\\
60.125	0.23574	-820.009374618299\\
60.125	0.2394	-855.713012037883\\
60.125	0.24306	-892.238390332879\\
60.125	0.24672	-929.585509503289\\
60.125	0.25038	-967.754369549113\\
60.125	0.25404	-1006.74497047035\\
60.125	0.2577	-1046.557312267\\
60.125	0.26136	-1087.19139493906\\
60.125	0.26502	-1128.64721848654\\
60.125	0.26868	-1170.92478290943\\
60.125	0.27234	-1214.02408820773\\
60.125	0.276	-1257.94513438145\\
60.5	0.093	-68.5425708474715\\
60.5	0.09666	-72.210276481688\\
60.5	0.10032	-76.699722991318\\
60.5	0.10398	-82.010910376361\\
60.5	0.10764	-88.1438386368173\\
60.5	0.1113	-95.0985077726868\\
60.5	0.11496	-102.874917783969\\
60.5	0.11862	-111.473068670665\\
60.5	0.12228	-120.892960432774\\
60.5	0.12594	-131.134593070297\\
60.5	0.1296	-142.197966583232\\
60.5	0.13326	-154.083080971581\\
60.5	0.13692	-166.789936235342\\
60.5	0.14058	-180.318532374518\\
60.5	0.14424	-194.668869389106\\
60.5	0.1479	-209.840947279107\\
60.5	0.15156	-225.834766044522\\
60.5	0.15522	-242.65032568535\\
60.5	0.15888	-260.287626201591\\
60.5	0.16254	-278.746667593245\\
60.5	0.1662	-298.027449860313\\
60.5	0.16986	-318.129973002793\\
60.5	0.17352	-339.054237020687\\
60.5	0.17718	-360.800241913994\\
60.5	0.18084	-383.367987682714\\
60.5	0.1845	-406.757474326848\\
60.5	0.18816	-430.968701846395\\
60.5	0.19182	-456.001670241355\\
60.5	0.19548	-481.856379511728\\
60.5	0.19914	-508.532829657514\\
60.5	0.2028	-536.031020678714\\
60.5	0.20646	-564.350952575326\\
60.5	0.21012	-593.492625347352\\
60.5	0.21378	-623.456038994791\\
60.5	0.21744	-654.241193517643\\
60.5	0.2211	-685.848088915909\\
60.5	0.22476	-718.276725189587\\
60.5	0.22842	-751.527102338679\\
60.5	0.23208	-785.599220363184\\
60.5	0.23574	-820.493079263103\\
60.5	0.2394	-856.208679038434\\
60.5	0.24306	-892.746019689178\\
60.5	0.24672	-930.105101215336\\
60.5	0.25038	-968.285923616907\\
60.5	0.25404	-1007.28848689389\\
60.5	0.2577	-1047.11279104629\\
60.5	0.26136	-1087.7588360741\\
60.5	0.26502	-1129.22662197732\\
60.5	0.26868	-1171.51614875596\\
60.5	0.27234	-1214.62741641001\\
60.5	0.276	-1258.56042493947\\
60.875	0.093	-68.6307772932809\\
60.875	0.09666	-72.3104452832451\\
60.875	0.10032	-76.8118541486229\\
60.875	0.10398	-82.1350038894137\\
60.875	0.10764	-88.2798945056177\\
60.875	0.1113	-95.246525997235\\
60.875	0.11496	-103.034898364265\\
60.875	0.11862	-111.645011606709\\
60.875	0.12228	-121.076865724566\\
60.875	0.12594	-131.330460717836\\
60.875	0.1296	-142.405796586519\\
60.875	0.13326	-154.302873330616\\
60.875	0.13692	-167.021690950125\\
60.875	0.14058	-180.562249445048\\
60.875	0.14424	-194.924548815384\\
60.875	0.1479	-210.108589061133\\
60.875	0.15156	-226.114370182296\\
60.875	0.15522	-242.941892178871\\
60.875	0.15888	-260.59115505086\\
60.875	0.16254	-279.062158798262\\
60.875	0.1662	-298.354903421078\\
60.875	0.16986	-318.469388919306\\
60.875	0.17352	-339.405615292948\\
60.875	0.17718	-361.163582542002\\
60.875	0.18084	-383.74329066647\\
60.875	0.1845	-407.144739666352\\
60.875	0.18816	-431.367929541646\\
60.875	0.19182	-456.412860292354\\
60.875	0.19548	-482.279531918475\\
60.875	0.19914	-508.967944420009\\
60.875	0.2028	-536.478097796956\\
60.875	0.20646	-564.809992049316\\
60.875	0.21012	-593.96362717709\\
60.875	0.21378	-623.939003180277\\
60.875	0.21744	-654.736120058877\\
60.875	0.2211	-686.35497781289\\
60.875	0.22476	-718.795576442316\\
60.875	0.22842	-752.057915947156\\
60.875	0.23208	-786.141996327409\\
60.875	0.23574	-821.047817583075\\
60.875	0.2394	-856.775379714154\\
60.875	0.24306	-893.324682720646\\
60.875	0.24672	-930.695726602552\\
60.875	0.25038	-968.888511359871\\
60.875	0.25404	-1007.9030369926\\
60.875	0.2577	-1047.73930350075\\
60.875	0.26136	-1088.39731088431\\
60.875	0.26502	-1129.87705914328\\
60.875	0.26868	-1172.17854827766\\
60.875	0.27234	-1215.30177828746\\
60.875	0.276	-1259.24674917267\\
61.25	0.093	-68.7900174142592\\
61.25	0.09666	-72.4816477599717\\
61.25	0.10032	-76.995018981097\\
61.25	0.10398	-82.3301310776355\\
61.25	0.10764	-88.4869840495873\\
61.25	0.1113	-95.4655778969521\\
61.25	0.11496	-103.265912619731\\
61.25	0.11862	-111.887988217922\\
61.25	0.12228	-121.331804691527\\
61.25	0.12594	-131.597362040544\\
61.25	0.1296	-142.684660264975\\
61.25	0.13326	-154.593699364819\\
61.25	0.13692	-167.324479340077\\
61.25	0.14058	-180.877000190747\\
61.25	0.14424	-195.251261916831\\
61.25	0.1479	-210.447264518328\\
61.25	0.15156	-226.465007995238\\
61.25	0.15522	-243.304492347562\\
61.25	0.15888	-260.965717575298\\
61.25	0.16254	-279.448683678448\\
61.25	0.1662	-298.753390657011\\
61.25	0.16986	-318.879838510988\\
61.25	0.17352	-339.828027240377\\
61.25	0.17718	-361.597956845179\\
61.25	0.18084	-384.189627325395\\
61.25	0.1845	-407.603038681024\\
61.25	0.18816	-431.838190912066\\
61.25	0.19182	-456.895084018522\\
61.25	0.19548	-482.77371800039\\
61.25	0.19914	-509.474092857672\\
61.25	0.2028	-536.996208590367\\
61.25	0.20646	-565.340065198475\\
61.25	0.21012	-594.505662681997\\
61.25	0.21378	-624.493001040931\\
61.25	0.21744	-655.302080275279\\
61.25	0.2211	-686.93290038504\\
61.25	0.22476	-719.385461370214\\
61.25	0.22842	-752.659763230801\\
61.25	0.23208	-786.755805966802\\
61.25	0.23574	-821.673589578216\\
61.25	0.2394	-857.413114065043\\
61.25	0.24306	-893.974379427283\\
61.25	0.24672	-931.357385664936\\
61.25	0.25038	-969.562132778003\\
61.25	0.25404	-1008.58862076648\\
61.25	0.2577	-1048.43684963038\\
61.25	0.26136	-1089.10681936968\\
61.25	0.26502	-1130.5985299844\\
61.25	0.26868	-1172.91198147453\\
61.25	0.27234	-1216.04717384008\\
61.25	0.276	-1260.00410708104\\
61.625	0.093	-69.0202912104062\\
61.625	0.09666	-72.7238839118664\\
61.625	0.10032	-77.2492174887395\\
61.625	0.10398	-82.5962919410258\\
61.625	0.10764	-88.7651072687253\\
61.625	0.1113	-95.7556634718379\\
61.625	0.11496	-103.567960550364\\
61.625	0.11862	-112.201998504303\\
61.625	0.12228	-121.657777333656\\
61.625	0.12594	-131.935297038421\\
61.625	0.1296	-143.0345576186\\
61.625	0.13326	-154.955559074192\\
61.625	0.13692	-167.698301405197\\
61.625	0.14058	-181.262784611615\\
61.625	0.14424	-195.649008693447\\
61.625	0.1479	-210.856973650692\\
61.625	0.15156	-226.886679483349\\
61.625	0.15522	-243.738126191421\\
61.625	0.15888	-261.411313774905\\
61.625	0.16254	-279.906242233803\\
61.625	0.1662	-299.222911568113\\
61.625	0.16986	-319.361321777837\\
61.625	0.17352	-340.321472862974\\
61.625	0.17718	-362.103364823525\\
61.625	0.18084	-384.706997659488\\
61.625	0.1845	-408.132371370865\\
61.625	0.18816	-432.379485957655\\
61.625	0.19182	-457.448341419858\\
61.625	0.19548	-483.338937757474\\
61.625	0.19914	-510.051274970504\\
61.625	0.2028	-537.585353058947\\
61.625	0.20646	-565.941172022803\\
61.625	0.21012	-595.118731862072\\
61.625	0.21378	-625.118032576754\\
61.625	0.21744	-655.93907416685\\
61.625	0.2211	-687.581856632358\\
61.625	0.22476	-720.04637997328\\
61.625	0.22842	-753.332644189615\\
61.625	0.23208	-787.440649281364\\
61.625	0.23574	-822.370395248525\\
61.625	0.2394	-858.1218820911\\
61.625	0.24306	-894.695109809088\\
61.625	0.24672	-932.090078402489\\
61.625	0.25038	-970.306787871303\\
61.625	0.25404	-1009.34523821553\\
61.625	0.2577	-1049.20542943517\\
61.625	0.26136	-1089.88736153023\\
61.625	0.26502	-1131.39103450069\\
61.625	0.26868	-1173.71644834657\\
61.625	0.27234	-1216.86360306787\\
61.625	0.276	-1260.83249866457\\
62	0.093	-69.3215986817225\\
62	0.09666	-73.03715373893\\
62	0.10032	-77.574449671551\\
62	0.10398	-82.933486479585\\
62	0.10764	-89.1142641630324\\
62	0.1113	-96.1167827218929\\
62	0.11496	-103.941042156166\\
62	0.11862	-112.587042465854\\
62	0.12228	-122.054783650953\\
62	0.12594	-132.344265711467\\
62	0.1296	-143.455488647393\\
62	0.13326	-155.388452458733\\
62	0.13692	-168.143157145486\\
62	0.14058	-181.719602707652\\
62	0.14424	-196.117789145231\\
62	0.1479	-211.337716458224\\
62	0.15156	-227.37938464663\\
62	0.15522	-244.242793710448\\
62	0.15888	-261.927943649681\\
62	0.16254	-280.434834464326\\
62	0.1662	-299.763466154385\\
62	0.16986	-319.913838719856\\
62	0.17352	-340.885952160741\\
62	0.17718	-362.679806477039\\
62	0.18084	-385.29540166875\\
62	0.1845	-408.732737735875\\
62	0.18816	-432.991814678412\\
62	0.19182	-458.072632496364\\
62	0.19548	-483.975191189728\\
62	0.19914	-510.699490758505\\
62	0.2028	-538.245531202695\\
62	0.20646	-566.613312522299\\
62	0.21012	-595.802834717316\\
62	0.21378	-625.814097787746\\
62	0.21744	-656.647101733589\\
62	0.2211	-688.301846554846\\
62	0.22476	-720.778332251515\\
62	0.22842	-754.076558823598\\
62	0.23208	-788.196526271094\\
62	0.23574	-823.138234594004\\
62	0.2394	-858.901683792326\\
62	0.24306	-895.486873866062\\
62	0.24672	-932.893804815211\\
62	0.25038	-971.122476639773\\
62	0.25404	-1010.17288933975\\
62	0.2577	-1050.04504291514\\
62	0.26136	-1090.73893736594\\
62	0.26502	-1132.25457269215\\
62	0.26868	-1174.59194889378\\
62	0.27234	-1217.75106597082\\
62	0.276	-1261.73192392328\\
62.375	0.093	-69.6939398282066\\
62.375	0.09666	-73.4214572411619\\
62.375	0.10032	-77.9707155295307\\
62.375	0.10398	-83.3417146933124\\
62.375	0.10764	-89.5344547325075\\
62.375	0.1113	-96.5489356471159\\
62.375	0.11496	-104.385157437137\\
62.375	0.11862	-113.043120102572\\
62.375	0.12228	-122.52282364342\\
62.375	0.12594	-132.824268059681\\
62.375	0.1296	-143.947453351355\\
62.375	0.13326	-155.892379518443\\
62.375	0.13692	-168.659046560943\\
62.375	0.14058	-182.247454478857\\
62.375	0.14424	-196.657603272184\\
62.375	0.1479	-211.889492940925\\
62.375	0.15156	-227.943123485078\\
62.375	0.15522	-244.818494904645\\
62.375	0.15888	-262.515607199625\\
62.375	0.16254	-281.034460370018\\
62.375	0.1662	-300.375054415824\\
62.375	0.16986	-320.537389337043\\
62.375	0.17352	-341.521465133676\\
62.375	0.17718	-363.327281805722\\
62.375	0.18084	-385.954839353181\\
62.375	0.1845	-409.404137776053\\
62.375	0.18816	-433.675177074338\\
62.375	0.19182	-458.767957248037\\
62.375	0.19548	-484.682478297149\\
62.375	0.19914	-511.418740221674\\
62.375	0.2028	-538.976743021612\\
62.375	0.20646	-567.356486696964\\
62.375	0.21012	-596.557971247728\\
62.375	0.21378	-626.581196673906\\
62.375	0.21744	-657.426162975497\\
62.375	0.2211	-689.092870152501\\
62.375	0.22476	-721.581318204919\\
62.375	0.22842	-754.89150713275\\
62.375	0.23208	-789.023436935993\\
62.375	0.23574	-823.977107614651\\
62.375	0.2394	-859.752519168721\\
62.375	0.24306	-896.349671598204\\
62.375	0.24672	-933.768564903101\\
62.375	0.25038	-972.009199083411\\
62.375	0.25404	-1011.07157413913\\
62.375	0.2577	-1050.95569007027\\
62.375	0.26136	-1091.66154687682\\
62.375	0.26502	-1133.18914455878\\
62.375	0.26868	-1175.53848311616\\
62.375	0.27234	-1218.70956254895\\
62.375	0.276	-1262.70238285715\\
62.75	0.093	-70.1373146498598\\
62.75	0.09666	-73.8767944185632\\
62.75	0.10032	-78.4380150626795\\
62.75	0.10398	-83.8209765822091\\
62.75	0.10764	-90.0256789771519\\
62.75	0.1113	-97.0521222475078\\
62.75	0.11496	-104.900306393277\\
62.75	0.11862	-113.570231414459\\
62.75	0.12228	-123.061897311055\\
62.75	0.12594	-133.375304083064\\
62.75	0.1296	-144.510451730486\\
62.75	0.13326	-156.467340253321\\
62.75	0.13692	-169.24596965157\\
62.75	0.14058	-182.846339925232\\
62.75	0.14424	-197.268451074306\\
62.75	0.1479	-212.512303098794\\
62.75	0.15156	-228.577895998695\\
62.75	0.15522	-245.46522977401\\
62.75	0.15888	-263.174304424738\\
62.75	0.16254	-281.705119950879\\
62.75	0.1662	-301.057676352432\\
62.75	0.16986	-321.2319736294\\
62.75	0.17352	-342.22801178178\\
62.75	0.17718	-364.045790809573\\
62.75	0.18084	-386.68531071278\\
62.75	0.1845	-410.1465714914\\
62.75	0.18816	-434.429573145434\\
62.75	0.19182	-459.53431567488\\
62.75	0.19548	-485.46079907974\\
62.75	0.19914	-512.209023360012\\
62.75	0.2028	-539.778988515698\\
62.75	0.20646	-568.170694546798\\
62.75	0.21012	-597.38414145331\\
62.75	0.21378	-627.419329235236\\
62.75	0.21744	-658.276257892574\\
62.75	0.2211	-689.954927425326\\
62.75	0.22476	-722.455337833492\\
62.75	0.22842	-755.77748911707\\
62.75	0.23208	-789.921381276062\\
62.75	0.23574	-824.887014310467\\
62.75	0.2394	-860.674388220285\\
62.75	0.24306	-897.283503005516\\
62.75	0.24672	-934.71435866616\\
62.75	0.25038	-972.966955202218\\
62.75	0.25404	-1012.04129261369\\
62.75	0.2577	-1051.93737090057\\
62.75	0.26136	-1092.65519006287\\
62.75	0.26502	-1134.19475010058\\
62.75	0.26868	-1176.5560510137\\
62.75	0.27234	-1219.73909280224\\
62.75	0.276	-1263.74387546619\\
63.125	0.093	-70.6517231466816\\
63.125	0.09666	-74.4031652711328\\
63.125	0.10032	-78.9763482709969\\
63.125	0.10398	-84.3712721462742\\
63.125	0.10764	-90.5879368969648\\
63.125	0.1113	-97.6263425230684\\
63.125	0.11496	-105.486489024586\\
63.125	0.11862	-114.168376401516\\
63.125	0.12228	-123.672004653859\\
63.125	0.12594	-133.997373781616\\
63.125	0.1296	-145.144483784786\\
63.125	0.13326	-157.113334663368\\
63.125	0.13692	-169.903926417365\\
63.125	0.14058	-183.516259046774\\
63.125	0.14424	-197.950332551596\\
63.125	0.1479	-213.206146931832\\
63.125	0.15156	-229.283702187481\\
63.125	0.15522	-246.182998318544\\
63.125	0.15888	-263.904035325019\\
63.125	0.16254	-282.446813206908\\
63.125	0.1662	-301.811331964209\\
63.125	0.16986	-321.997591596925\\
63.125	0.17352	-343.005592105052\\
63.125	0.17718	-364.835333488594\\
63.125	0.18084	-387.486815747548\\
63.125	0.1845	-410.960038881916\\
63.125	0.18816	-435.255002891697\\
63.125	0.19182	-460.371707776891\\
63.125	0.19548	-486.310153537499\\
63.125	0.19914	-513.070340173519\\
63.125	0.2028	-540.652267684953\\
63.125	0.20646	-569.0559360718\\
63.125	0.21012	-598.28134533406\\
63.125	0.21378	-628.328495471734\\
63.125	0.21744	-659.19738648482\\
63.125	0.2211	-690.88801837332\\
63.125	0.22476	-723.400391137233\\
63.125	0.22842	-756.734504776559\\
63.125	0.23208	-790.890359291298\\
63.125	0.23574	-825.867954681451\\
63.125	0.2394	-861.667290947017\\
63.125	0.24306	-898.288368087996\\
63.125	0.24672	-935.731186104387\\
63.125	0.25038	-973.995744996193\\
63.125	0.25404	-1013.08204476341\\
63.125	0.2577	-1052.99008540604\\
63.125	0.26136	-1093.71986692409\\
63.125	0.26502	-1135.27138931755\\
63.125	0.26868	-1177.64465258642\\
63.125	0.27234	-1220.8396567307\\
63.125	0.276	-1264.8564017504\\
63.5	0.093	-71.2371653186723\\
63.5	0.09666	-75.0005697988711\\
63.5	0.10032	-79.5857151544831\\
63.5	0.10398	-84.9926013855082\\
63.5	0.10764	-91.2212284919465\\
63.5	0.1113	-98.2715964737981\\
63.5	0.11496	-106.143705331063\\
63.5	0.11862	-114.837555063741\\
63.5	0.12228	-124.353145671832\\
63.5	0.12594	-134.690477155336\\
63.5	0.1296	-145.849549514254\\
63.5	0.13326	-157.830362748584\\
63.5	0.13692	-170.632916858328\\
63.5	0.14058	-184.257211843485\\
63.5	0.14424	-198.703247704056\\
63.5	0.1479	-213.971024440039\\
63.5	0.15156	-230.060542051436\\
63.5	0.15522	-246.971800538246\\
63.5	0.15888	-264.70479990047\\
63.5	0.16254	-283.259540138106\\
63.5	0.1662	-302.636021251155\\
63.5	0.16986	-322.834243239618\\
63.5	0.17352	-343.854206103494\\
63.5	0.17718	-365.695909842783\\
63.5	0.18084	-388.359354457485\\
63.5	0.1845	-411.844539947601\\
63.5	0.18816	-436.151466313129\\
63.5	0.19182	-461.280133554071\\
63.5	0.19548	-487.230541670426\\
63.5	0.19914	-514.002690662195\\
63.5	0.2028	-541.596580529376\\
63.5	0.20646	-570.012211271971\\
63.5	0.21012	-599.249582889979\\
63.5	0.21378	-629.3086953834\\
63.5	0.21744	-660.189548752234\\
63.5	0.2211	-691.892142996482\\
63.5	0.22476	-724.416478116143\\
63.5	0.22842	-757.762554111217\\
63.5	0.23208	-791.930370981704\\
63.5	0.23574	-826.919928727604\\
63.5	0.2394	-862.731227348918\\
63.5	0.24306	-899.364266845644\\
63.5	0.24672	-936.819047217784\\
63.5	0.25038	-975.095568465337\\
63.5	0.25404	-1014.1938305883\\
63.5	0.2577	-1054.11383358668\\
63.5	0.26136	-1094.85557746048\\
63.5	0.26502	-1136.41906220968\\
63.5	0.26868	-1178.8042878343\\
63.5	0.27234	-1222.01125433433\\
63.5	0.276	-1266.03996170978\\
63.875	0.093	-71.8936411658314\\
63.875	0.09666	-75.6690080017777\\
63.875	0.10032	-80.2661157131375\\
63.875	0.10398	-85.6849642999103\\
63.875	0.10764	-91.9255537620966\\
63.875	0.1113	-98.987884099696\\
63.875	0.11496	-106.871955312708\\
63.875	0.11862	-115.577767401134\\
63.875	0.12228	-125.105320364973\\
63.875	0.12594	-135.454614204225\\
63.875	0.1296	-146.62564891889\\
63.875	0.13326	-158.618424508969\\
63.875	0.13692	-171.43294097446\\
63.875	0.14058	-185.069198315365\\
63.875	0.14424	-199.527196531683\\
63.875	0.1479	-214.806935623415\\
63.875	0.15156	-230.908415590559\\
63.875	0.15522	-247.831636433117\\
63.875	0.15888	-265.576598151088\\
63.875	0.16254	-284.143300744472\\
63.875	0.1662	-303.531744213269\\
63.875	0.16986	-323.74192855748\\
63.875	0.17352	-344.773853777104\\
63.875	0.17718	-366.62751987214\\
63.875	0.18084	-389.30292684259\\
63.875	0.1845	-412.800074688454\\
63.875	0.18816	-437.11896340973\\
63.875	0.19182	-462.25959300642\\
63.875	0.19548	-488.221963478523\\
63.875	0.19914	-515.006074826039\\
63.875	0.2028	-542.611927048968\\
63.875	0.20646	-571.039520147311\\
63.875	0.21012	-600.288854121066\\
63.875	0.21378	-630.359928970235\\
63.875	0.21744	-661.252744694817\\
63.875	0.2211	-692.967301294812\\
63.875	0.22476	-725.503598770221\\
63.875	0.22842	-758.861637121043\\
63.875	0.23208	-793.041416347277\\
63.875	0.23574	-828.042936448926\\
63.875	0.2394	-863.866197425987\\
63.875	0.24306	-900.511199278461\\
63.875	0.24672	-937.977942006348\\
63.875	0.25038	-976.26642560965\\
63.875	0.25404	-1015.37665008836\\
63.875	0.2577	-1055.30861544249\\
63.875	0.26136	-1096.06232167203\\
63.875	0.26502	-1137.63776877698\\
63.875	0.26868	-1180.03495675735\\
63.875	0.27234	-1223.25388561313\\
63.875	0.276	-1267.29455534433\\
64.25	0.093	-72.6211506881593\\
64.25	0.09666	-76.4084798798538\\
64.25	0.10032	-81.0175499469611\\
64.25	0.10398	-86.4483608894816\\
64.25	0.10764	-92.7009127074155\\
64.25	0.1113	-99.7752054007624\\
64.25	0.11496	-107.671238969523\\
64.25	0.11862	-116.389013413696\\
64.25	0.12228	-125.928528733283\\
64.25	0.12594	-136.289784928283\\
64.25	0.1296	-147.472781998696\\
64.25	0.13326	-159.477519944522\\
64.25	0.13692	-172.303998765762\\
64.25	0.14058	-185.952218462414\\
64.25	0.14424	-200.42217903448\\
64.25	0.1479	-215.713880481959\\
64.25	0.15156	-231.827322804851\\
64.25	0.15522	-248.762506003157\\
64.25	0.15888	-266.519430076876\\
64.25	0.16254	-285.098095026008\\
64.25	0.1662	-304.498500850552\\
64.25	0.16986	-324.720647550511\\
64.25	0.17352	-345.764535125882\\
64.25	0.17718	-367.630163576667\\
64.25	0.18084	-390.317532902864\\
64.25	0.1845	-413.826643104476\\
64.25	0.18816	-438.1574941815\\
64.25	0.19182	-463.310086133937\\
64.25	0.19548	-489.284418961788\\
64.25	0.19914	-516.080492665052\\
64.25	0.2028	-543.698307243729\\
64.25	0.20646	-572.137862697819\\
64.25	0.21012	-601.399159027323\\
64.25	0.21378	-631.482196232239\\
64.25	0.21744	-662.386974312569\\
64.25	0.2211	-694.113493268312\\
64.25	0.22476	-726.661753099468\\
64.25	0.22842	-760.031753806038\\
64.25	0.23208	-794.22349538802\\
64.25	0.23574	-829.236977845417\\
64.25	0.2394	-865.072201178225\\
64.25	0.24306	-901.729165386447\\
64.25	0.24672	-939.207870470083\\
64.25	0.25038	-977.508316429132\\
64.25	0.25404	-1016.63050326359\\
64.25	0.2577	-1056.57443097347\\
64.25	0.26136	-1097.34009955876\\
64.25	0.26502	-1138.92750901946\\
64.25	0.26868	-1181.33665935557\\
64.25	0.27234	-1224.5675505671\\
64.25	0.276	-1268.62018265404\\
64.625	0.093	-73.4196938856559\\
64.625	0.09666	-77.2189854330982\\
64.625	0.10032	-81.8400178559532\\
64.625	0.10398	-87.2827911542216\\
64.625	0.10764	-93.5473053279032\\
64.625	0.1113	-100.633560376998\\
64.625	0.11496	-108.541556301506\\
64.625	0.11862	-117.271293101427\\
64.625	0.12228	-126.822770776762\\
64.625	0.12594	-137.195989327509\\
64.625	0.1296	-148.39094875367\\
64.625	0.13326	-160.407649055244\\
64.625	0.13692	-173.246090232231\\
64.625	0.14058	-186.906272284632\\
64.625	0.14424	-201.388195212445\\
64.625	0.1479	-216.691859015672\\
64.625	0.15156	-232.817263694312\\
64.625	0.15522	-249.764409248365\\
64.625	0.15888	-267.533295677832\\
64.625	0.16254	-286.123922982712\\
64.625	0.1662	-305.536291163004\\
64.625	0.16986	-325.770400218711\\
64.625	0.17352	-346.826250149829\\
64.625	0.17718	-368.703840956362\\
64.625	0.18084	-391.403172638307\\
64.625	0.1845	-414.924245195666\\
64.625	0.18816	-439.267058628438\\
64.625	0.19182	-464.431612936623\\
64.625	0.19548	-490.417908120222\\
64.625	0.19914	-517.225944179233\\
64.625	0.2028	-544.855721113658\\
64.625	0.20646	-573.307238923496\\
64.625	0.21012	-602.580497608747\\
64.625	0.21378	-632.675497169412\\
64.625	0.21744	-663.592237605489\\
64.625	0.2211	-695.33071891698\\
64.625	0.22476	-727.890941103884\\
64.625	0.22842	-761.272904166201\\
64.625	0.23208	-795.476608103932\\
64.625	0.23574	-830.502052917076\\
64.625	0.2394	-866.349238605632\\
64.625	0.24306	-903.018165169602\\
64.625	0.24672	-940.508832608985\\
64.625	0.25038	-978.821240923782\\
64.625	0.25404	-1017.95539011399\\
64.625	0.2577	-1057.91128017961\\
64.625	0.26136	-1098.68891112065\\
64.625	0.26502	-1140.2882829371\\
64.625	0.26868	-1182.70939562896\\
64.625	0.27234	-1225.95224919624\\
64.625	0.276	-1270.01684363893\\
65	0.093	-74.2892707583216\\
65	0.09666	-78.1005246615111\\
65	0.10032	-82.7335194401144\\
65	0.10398	-88.1882550941305\\
65	0.10764	-94.4647316235599\\
65	0.1113	-101.562949028402\\
65	0.11496	-109.482907308658\\
65	0.11862	-118.224606464327\\
65	0.12228	-127.788046495409\\
65	0.12594	-138.173227401905\\
65	0.1296	-149.380149183813\\
65	0.13326	-161.408811841135\\
65	0.13692	-174.25921537387\\
65	0.14058	-187.931359782018\\
65	0.14424	-202.42524506558\\
65	0.1479	-217.740871224554\\
65	0.15156	-233.878238258942\\
65	0.15522	-250.837346168743\\
65	0.15888	-268.618194953957\\
65	0.16254	-287.220784614585\\
65	0.1662	-306.645115150625\\
65	0.16986	-326.891186562079\\
65	0.17352	-347.958998848946\\
65	0.17718	-369.848552011226\\
65	0.18084	-392.559846048919\\
65	0.1845	-416.092880962026\\
65	0.18816	-440.447656750545\\
65	0.19182	-465.624173414478\\
65	0.19548	-491.622430953825\\
65	0.19914	-518.442429368584\\
65	0.2028	-546.084168658757\\
65	0.20646	-574.547648824343\\
65	0.21012	-603.832869865341\\
65	0.21378	-633.939831781753\\
65	0.21744	-664.868534573579\\
65	0.2211	-696.618978240817\\
65	0.22476	-729.191162783469\\
65	0.22842	-762.585088201534\\
65	0.23208	-796.800754495012\\
65	0.23574	-831.838161663904\\
65	0.2394	-867.697309708208\\
65	0.24306	-904.378198627926\\
65	0.24672	-941.880828423057\\
65	0.25038	-980.205199093601\\
65	0.25404	-1019.35131063956\\
65	0.2577	-1059.31916306093\\
65	0.26136	-1100.10875635771\\
65	0.26502	-1141.72009052991\\
65	0.26868	-1184.15316557752\\
65	0.27234	-1227.40798150054\\
65	0.276	-1271.48453829898\\
65.375	0.093	-75.2298813061554\\
65.375	0.09666	-79.0530975650927\\
65.375	0.10032	-83.6980546994435\\
65.375	0.10398	-89.1647527092073\\
65.375	0.10764	-95.4531915943847\\
65.375	0.1113	-102.563371354975\\
65.375	0.11496	-110.495291990978\\
65.375	0.11862	-119.248953502395\\
65.375	0.12228	-128.824355889225\\
65.375	0.12594	-139.221499151468\\
65.375	0.1296	-150.440383289124\\
65.375	0.13326	-162.481008302194\\
65.375	0.13692	-175.343374190677\\
65.375	0.14058	-189.027480954573\\
65.375	0.14424	-203.533328593882\\
65.375	0.1479	-218.860917108604\\
65.375	0.15156	-235.01024649874\\
65.375	0.15522	-251.981316764288\\
65.375	0.15888	-269.774127905251\\
65.375	0.16254	-288.388679921625\\
65.375	0.1662	-307.824972813414\\
65.375	0.16986	-328.083006580615\\
65.375	0.17352	-349.16278122323\\
65.375	0.17718	-371.064296741258\\
65.375	0.18084	-393.787553134699\\
65.375	0.1845	-417.332550403553\\
65.375	0.18816	-441.699288547821\\
65.375	0.19182	-466.887767567502\\
65.375	0.19548	-492.897987462596\\
65.375	0.19914	-519.729948233103\\
65.375	0.2028	-547.383649879023\\
65.375	0.20646	-575.859092400357\\
65.375	0.21012	-605.156275797103\\
65.375	0.21378	-635.275200069263\\
65.375	0.21744	-666.215865216836\\
65.375	0.2211	-697.978271239822\\
65.375	0.22476	-730.562418138222\\
65.375	0.22842	-763.968305912035\\
65.375	0.23208	-798.195934561261\\
65.375	0.23574	-833.2453040859\\
65.375	0.2394	-869.116414485952\\
65.375	0.24306	-905.809265761417\\
65.375	0.24672	-943.323857912296\\
65.375	0.25038	-981.660190938589\\
65.375	0.25404	-1020.81826484029\\
65.375	0.2577	-1060.79807961741\\
65.375	0.26136	-1101.59963526994\\
65.375	0.26502	-1143.22293179789\\
65.375	0.26868	-1185.66796920125\\
65.375	0.27234	-1228.93474748002\\
65.375	0.276	-1273.0232666342\\
65.75	0.093	-76.2415255291582\\
65.75	0.09666	-80.0767041438437\\
65.75	0.10032	-84.733623633942\\
65.75	0.10398	-90.2122839994536\\
65.75	0.10764	-96.5126852403785\\
65.75	0.1113	-103.634827356717\\
65.75	0.11496	-111.578710348468\\
65.75	0.11862	-120.344334215632\\
65.75	0.12228	-129.93169895821\\
65.75	0.12594	-140.340804576201\\
65.75	0.1296	-151.571651069605\\
65.75	0.13326	-163.624238438422\\
65.75	0.13692	-176.498566682653\\
65.75	0.14058	-190.194635802297\\
65.75	0.14424	-204.712445797354\\
65.75	0.1479	-220.051996667824\\
65.75	0.15156	-236.213288413707\\
65.75	0.15522	-253.196321035003\\
65.75	0.15888	-271.001094531713\\
65.75	0.16254	-289.627608903836\\
65.75	0.1662	-309.075864151372\\
65.75	0.16986	-329.345860274322\\
65.75	0.17352	-350.437597272684\\
65.75	0.17718	-372.351075146459\\
65.75	0.18084	-395.086293895648\\
65.75	0.1845	-418.64325352025\\
65.75	0.18816	-443.021954020266\\
65.75	0.19182	-468.222395395694\\
65.75	0.19548	-494.244577646536\\
65.75	0.19914	-521.088500772791\\
65.75	0.2028	-548.754164774459\\
65.75	0.20646	-577.24156965154\\
65.75	0.21012	-606.550715404034\\
65.75	0.21378	-636.681602031942\\
65.75	0.21744	-667.634229535263\\
65.75	0.2211	-699.408597913997\\
65.75	0.22476	-732.004707168144\\
65.75	0.22842	-765.422557297705\\
65.75	0.23208	-799.662148302678\\
65.75	0.23574	-834.723480183065\\
65.75	0.2394	-870.606552938865\\
65.75	0.24306	-907.311366570079\\
65.75	0.24672	-944.837921076705\\
65.75	0.25038	-983.186216458745\\
65.75	0.25404	-1022.3562527162\\
65.75	0.2577	-1062.34802984906\\
65.75	0.26136	-1103.16154785734\\
65.75	0.26502	-1144.79680674104\\
65.75	0.26868	-1187.25380650014\\
65.75	0.27234	-1230.53254713466\\
65.75	0.276	-1274.63302864459\\
66.125	0.093	-77.3242034273296\\
66.125	0.09666	-81.1713443977628\\
66.125	0.10032	-85.8402262436089\\
66.125	0.10398	-91.3308489648683\\
66.125	0.10764	-97.6432125615409\\
66.125	0.1113	-104.777317033627\\
66.125	0.11496	-112.733162381126\\
66.125	0.11862	-121.510748604038\\
66.125	0.12228	-131.110075702364\\
66.125	0.12594	-141.531143676102\\
66.125	0.1296	-152.773952525254\\
66.125	0.13326	-164.838502249819\\
66.125	0.13692	-177.724792849797\\
66.125	0.14058	-191.432824325189\\
66.125	0.14424	-205.962596675993\\
66.125	0.1479	-221.314109902211\\
66.125	0.15156	-237.487364003842\\
66.125	0.15522	-254.482358980886\\
66.125	0.15888	-272.299094833344\\
66.125	0.16254	-290.937571561215\\
66.125	0.1662	-310.397789164498\\
66.125	0.16986	-330.679747643196\\
66.125	0.17352	-351.783446997306\\
66.125	0.17718	-373.708887226829\\
66.125	0.18084	-396.456068331766\\
66.125	0.1845	-420.024990312116\\
66.125	0.18816	-444.415653167879\\
66.125	0.19182	-469.628056899055\\
66.125	0.19548	-495.662201505644\\
66.125	0.19914	-522.518086987647\\
66.125	0.2028	-550.195713345063\\
66.125	0.20646	-578.695080577892\\
66.125	0.21012	-608.016188686134\\
66.125	0.21378	-638.159037669789\\
66.125	0.21744	-669.123627528858\\
66.125	0.2211	-700.90995826334\\
66.125	0.22476	-733.518029873235\\
66.125	0.22842	-766.947842358543\\
66.125	0.23208	-801.199395719264\\
66.125	0.23574	-836.272689955399\\
66.125	0.2394	-872.167725066947\\
66.125	0.24306	-908.884501053908\\
66.125	0.24672	-946.423017916282\\
66.125	0.25038	-984.78327565407\\
66.125	0.25404	-1023.96527426727\\
66.125	0.2577	-1063.96901375588\\
66.125	0.26136	-1104.79449411991\\
66.125	0.26502	-1146.44171535935\\
66.125	0.26868	-1188.9106774742\\
66.125	0.27234	-1232.20138046447\\
66.125	0.276	-1276.31382433015\\
66.5	0.093	-78.4779150006693\\
66.5	0.09666	-82.3370183268503\\
66.5	0.10032	-87.0178625284444\\
66.5	0.10398	-92.5204476054515\\
66.5	0.10764	-98.8447735578719\\
66.5	0.1113	-105.990840385705\\
66.5	0.11496	-113.958648088952\\
66.5	0.11862	-122.748196667612\\
66.5	0.12228	-132.359486121686\\
66.5	0.12594	-142.792516451172\\
66.5	0.1296	-154.047287656072\\
66.5	0.13326	-166.123799736384\\
66.5	0.13692	-179.02205269211\\
66.5	0.14058	-192.742046523249\\
66.5	0.14424	-207.283781229802\\
66.5	0.1479	-222.647256811768\\
66.5	0.15156	-238.832473269146\\
66.5	0.15522	-255.839430601938\\
66.5	0.15888	-273.668128810143\\
66.5	0.16254	-292.318567893762\\
66.5	0.1662	-311.790747852793\\
66.5	0.16986	-332.084668687239\\
66.5	0.17352	-353.200330397096\\
66.5	0.17718	-375.137732982367\\
66.5	0.18084	-397.896876443052\\
66.5	0.1845	-421.477760779149\\
66.5	0.18816	-445.88038599066\\
66.5	0.19182	-471.104752077584\\
66.5	0.19548	-497.150859039921\\
66.5	0.19914	-524.018706877672\\
66.5	0.2028	-551.708295590835\\
66.5	0.20646	-580.219625179412\\
66.5	0.21012	-609.552695643402\\
66.5	0.21378	-639.707506982805\\
66.5	0.21744	-670.684059197622\\
66.5	0.2211	-702.482352287851\\
66.5	0.22476	-735.102386253494\\
66.5	0.22842	-768.54416109455\\
66.5	0.23208	-802.807676811019\\
66.5	0.23574	-837.892933402901\\
66.5	0.2394	-873.799930870197\\
66.5	0.24306	-910.528669212906\\
66.5	0.24672	-948.079148431028\\
66.5	0.25038	-986.451368524563\\
66.5	0.25404	-1025.64532949351\\
66.5	0.2577	-1065.66103133787\\
66.5	0.26136	-1106.49847405765\\
66.5	0.26502	-1148.15765765284\\
66.5	0.26868	-1190.63858212344\\
66.5	0.27234	-1233.94124746945\\
66.5	0.276	-1278.06565369088\\
66.875	0.093	-79.7026602491782\\
66.875	0.09666	-83.573725931107\\
66.875	0.10032	-88.2665324884486\\
66.875	0.10398	-93.7810799212035\\
66.875	0.10764	-100.117368229372\\
66.875	0.1113	-107.275397412953\\
66.875	0.11496	-115.255167471948\\
66.875	0.11862	-124.056678406355\\
66.875	0.12228	-133.679930216176\\
66.875	0.12594	-144.12492290141\\
66.875	0.1296	-155.391656462058\\
66.875	0.13326	-167.480130898118\\
66.875	0.13692	-180.390346209592\\
66.875	0.14058	-194.122302396479\\
66.875	0.14424	-208.675999458779\\
66.875	0.1479	-224.051437396493\\
66.875	0.15156	-240.248616209619\\
66.875	0.15522	-257.267535898159\\
66.875	0.15888	-275.108196462112\\
66.875	0.16254	-293.770597901478\\
66.875	0.1662	-313.254740216257\\
66.875	0.16986	-333.56062340645\\
66.875	0.17352	-354.688247472056\\
66.875	0.17718	-376.637612413075\\
66.875	0.18084	-399.408718229507\\
66.875	0.1845	-423.001564921352\\
66.875	0.18816	-447.416152488611\\
66.875	0.19182	-472.652480931283\\
66.875	0.19548	-498.710550249368\\
66.875	0.19914	-525.590360442866\\
66.875	0.2028	-553.291911511777\\
66.875	0.20646	-581.815203456102\\
66.875	0.21012	-611.160236275839\\
66.875	0.21378	-641.32700997099\\
66.875	0.21744	-672.315524541554\\
66.875	0.2211	-704.125779987532\\
66.875	0.22476	-736.757776308922\\
66.875	0.22842	-770.211513505726\\
66.875	0.23208	-804.486991577942\\
66.875	0.23574	-839.584210525573\\
66.875	0.2394	-875.503170348617\\
66.875	0.24306	-912.243871047073\\
66.875	0.24672	-949.806312620942\\
66.875	0.25038	-988.190495070226\\
66.875	0.25404	-1027.39641839492\\
66.875	0.2577	-1067.42408259503\\
66.875	0.26136	-1108.27348767055\\
66.875	0.26502	-1149.94463362149\\
66.875	0.26868	-1192.43752044784\\
66.875	0.27234	-1235.7521481496\\
66.875	0.276	-1279.88851672678\\
67.25	0.093	-80.998439172856\\
67.25	0.09666	-84.8814672105325\\
67.25	0.10032	-89.5862361236219\\
67.25	0.10398	-95.1127459121245\\
67.25	0.10764	-101.46099657604\\
67.25	0.1113	-108.63098811537\\
67.25	0.11496	-116.622720530112\\
67.25	0.11862	-125.436193820267\\
67.25	0.12228	-135.071407985836\\
67.25	0.12594	-145.528363026818\\
67.25	0.1296	-156.807058943213\\
67.25	0.13326	-168.907495735021\\
67.25	0.13692	-181.829673402243\\
67.25	0.14058	-195.573591944878\\
67.25	0.14424	-210.139251362926\\
67.25	0.1479	-225.526651656387\\
67.25	0.15156	-241.735792825261\\
67.25	0.15522	-258.766674869549\\
67.25	0.15888	-276.619297789249\\
67.25	0.16254	-295.293661584363\\
67.25	0.1662	-314.78976625489\\
67.25	0.16986	-335.107611800831\\
67.25	0.17352	-356.247198222184\\
67.25	0.17718	-378.208525518951\\
67.25	0.18084	-400.991593691131\\
67.25	0.1845	-424.596402738724\\
67.25	0.18816	-449.02295266173\\
67.25	0.19182	-474.27124346015\\
67.25	0.19548	-500.341275133982\\
67.25	0.19914	-527.233047683228\\
67.25	0.2028	-554.946561107887\\
67.25	0.20646	-583.48181540796\\
67.25	0.21012	-612.838810583445\\
67.25	0.21378	-643.017546634344\\
67.25	0.21744	-674.018023560656\\
67.25	0.2211	-705.840241362381\\
67.25	0.22476	-738.484200039519\\
67.25	0.22842	-771.949899592071\\
67.25	0.23208	-806.237340020036\\
67.25	0.23574	-841.346521323414\\
67.25	0.2394	-877.277443502205\\
67.25	0.24306	-914.030106556408\\
67.25	0.24672	-951.604510486026\\
67.25	0.25038	-990.000655291057\\
67.25	0.25404	-1029.2185409715\\
67.25	0.2577	-1069.25816752736\\
67.25	0.26136	-1110.11953495863\\
67.25	0.26502	-1151.80264326531\\
67.25	0.26868	-1194.30749244741\\
67.25	0.27234	-1237.63408250492\\
67.25	0.276	-1281.78241343784\\
67.625	0.093	-82.365251771702\\
67.625	0.09666	-86.2602421651266\\
67.625	0.10032	-90.9769734339637\\
67.625	0.10398	-96.515445578214\\
67.625	0.10764	-102.875658597878\\
67.625	0.1113	-110.057612492954\\
67.625	0.11496	-118.061307263445\\
67.625	0.11862	-126.886742909348\\
67.625	0.12228	-136.533919430665\\
67.625	0.12594	-147.002836827394\\
67.625	0.1296	-158.293495099537\\
67.625	0.13326	-170.405894247093\\
67.625	0.13692	-183.340034270062\\
67.625	0.14058	-197.095915168445\\
67.625	0.14424	-211.673536942241\\
67.625	0.1479	-227.072899591449\\
67.625	0.15156	-243.294003116072\\
67.625	0.15522	-260.336847516107\\
67.625	0.15888	-278.201432791555\\
67.625	0.16254	-296.887758942417\\
67.625	0.1662	-316.395825968692\\
67.625	0.16986	-336.72563387038\\
67.625	0.17352	-357.877182647481\\
67.625	0.17718	-379.850472299995\\
67.625	0.18084	-402.645502827923\\
67.625	0.1845	-426.262274231264\\
67.625	0.18816	-450.700786510018\\
67.625	0.19182	-475.961039664186\\
67.625	0.19548	-502.043033693766\\
67.625	0.19914	-528.946768598759\\
67.625	0.2028	-556.672244379166\\
67.625	0.20646	-585.219461034987\\
67.625	0.21012	-614.58841856622\\
67.625	0.21378	-644.779116972866\\
67.625	0.21744	-675.791556254926\\
67.625	0.2211	-707.625736412399\\
67.625	0.22476	-740.281657445284\\
67.625	0.22842	-773.759319353584\\
67.625	0.23208	-808.058722137297\\
67.625	0.23574	-843.179865796423\\
67.625	0.2394	-879.122750330961\\
67.625	0.24306	-915.887375740913\\
67.625	0.24672	-953.473742026278\\
67.625	0.25038	-991.881849187057\\
67.625	0.25404	-1031.11169722325\\
67.625	0.2577	-1071.16328613485\\
67.625	0.26136	-1112.03661592187\\
67.625	0.26502	-1153.7316865843\\
67.625	0.26868	-1196.24849812215\\
67.625	0.27234	-1239.5870505354\\
67.625	0.276	-1283.74734382407\\
68	0.093	-83.8030980457169\\
68	0.09666	-87.710050794889\\
68	0.10032	-92.4387444194741\\
68	0.10398	-97.9891789194722\\
68	0.10764	-104.361354294884\\
68	0.1113	-111.555270545708\\
68	0.11496	-119.570927671946\\
68	0.11862	-128.408325673597\\
68	0.12228	-138.067464550661\\
68	0.12594	-148.548344303139\\
68	0.1296	-159.850964931029\\
68	0.13326	-171.975326434333\\
68	0.13692	-184.92142881305\\
68	0.14058	-198.689272067181\\
68	0.14424	-213.278856196724\\
68	0.1479	-228.690181201681\\
68	0.15156	-244.92324708205\\
68	0.15522	-261.978053837833\\
68	0.15888	-279.85460146903\\
68	0.16254	-298.552889975639\\
68	0.1662	-318.072919357662\\
68	0.16986	-338.414689615098\\
68	0.17352	-359.578200747947\\
68	0.17718	-381.563452756209\\
68	0.18084	-404.370445639884\\
68	0.1845	-427.999179398973\\
68	0.18816	-452.449654033475\\
68	0.19182	-477.72186954339\\
68	0.19548	-503.815825928718\\
68	0.19914	-530.731523189459\\
68	0.2028	-558.468961325614\\
68	0.20646	-587.028140337182\\
68	0.21012	-616.409060224163\\
68	0.21378	-646.611720986557\\
68	0.21744	-677.636122624365\\
68	0.2211	-709.482265137585\\
68	0.22476	-742.150148526219\\
68	0.22842	-775.639772790266\\
68	0.23208	-809.951137929726\\
68	0.23574	-845.0842439446\\
68	0.2394	-881.039090834886\\
68	0.24306	-917.815678600586\\
68	0.24672	-955.414007241699\\
68	0.25038	-993.834076758225\\
68	0.25404	-1033.07588715016\\
68	0.2577	-1073.13943841752\\
68	0.26136	-1114.02473056028\\
68	0.26502	-1155.73176357846\\
68	0.26868	-1198.26053747205\\
68	0.27234	-1241.61105224106\\
68	0.276	-1285.78330788548\\
68.375	0.093	-85.3119779949004\\
68.375	0.09666	-89.2308930998202\\
68.375	0.10032	-93.9715490801528\\
68.375	0.10398	-99.5339459358989\\
68.375	0.10764	-105.918083667058\\
68.375	0.1113	-113.12396227363\\
68.375	0.11496	-121.151581755616\\
68.375	0.11862	-130.000942113015\\
68.375	0.12228	-139.672043345827\\
68.375	0.12594	-150.164885454052\\
68.375	0.1296	-161.47946843769\\
68.375	0.13326	-173.615792296742\\
68.375	0.13692	-186.573857031207\\
68.375	0.14058	-200.353662641085\\
68.375	0.14424	-214.955209126376\\
68.375	0.1479	-230.37849648708\\
68.375	0.15156	-246.623524723198\\
68.375	0.15522	-263.690293834729\\
68.375	0.15888	-281.578803821673\\
68.375	0.16254	-300.28905468403\\
68.375	0.1662	-319.8210464218\\
68.375	0.16986	-340.174779034984\\
68.375	0.17352	-361.350252523581\\
68.375	0.17718	-383.34746688759\\
68.375	0.18084	-406.166422127014\\
68.375	0.1845	-429.80711824185\\
68.375	0.18816	-454.2695552321\\
68.375	0.19182	-479.553733097763\\
68.375	0.19548	-505.659651838838\\
68.375	0.19914	-532.587311455328\\
68.375	0.2028	-560.33671194723\\
68.375	0.20646	-588.907853314546\\
68.375	0.21012	-618.300735557274\\
68.375	0.21378	-648.515358675416\\
68.375	0.21744	-679.551722668972\\
68.375	0.2211	-711.40982753794\\
68.375	0.22476	-744.089673282321\\
68.375	0.22842	-777.591259902116\\
68.375	0.23208	-811.914587397324\\
68.375	0.23574	-847.059655767946\\
68.375	0.2394	-883.02646501398\\
68.375	0.24306	-919.815015135427\\
68.375	0.24672	-957.425306132288\\
68.375	0.25038	-995.857338004562\\
68.375	0.25404	-1035.11111075225\\
68.375	0.2577	-1075.18662437535\\
68.375	0.26136	-1116.08387887386\\
68.375	0.26502	-1157.80287424779\\
68.375	0.26868	-1200.34361049713\\
68.375	0.27234	-1243.70608762188\\
68.375	0.276	-1287.89030562205\\
68.75	0.093	-86.891891619253\\
68.75	0.09666	-90.8227690799206\\
68.75	0.10032	-95.575387416001\\
68.75	0.10398	-101.149746627495\\
68.75	0.10764	-107.545846714402\\
68.75	0.1113	-114.763687676722\\
68.75	0.11496	-122.803269514455\\
68.75	0.11862	-131.664592227602\\
68.75	0.12228	-141.347655816161\\
68.75	0.12594	-151.852460280134\\
68.75	0.1296	-163.179005619521\\
68.75	0.13326	-175.32729183432\\
68.75	0.13692	-188.297318924532\\
68.75	0.14058	-202.089086890158\\
68.75	0.14424	-216.702595731197\\
68.75	0.1479	-232.137845447649\\
68.75	0.15156	-248.394836039515\\
68.75	0.15522	-265.473567506793\\
68.75	0.15888	-283.374039849485\\
68.75	0.16254	-302.09625306759\\
68.75	0.1662	-321.640207161108\\
68.75	0.16986	-342.00590213004\\
68.75	0.17352	-363.193337974384\\
68.75	0.17718	-385.202514694142\\
68.75	0.18084	-408.033432289312\\
68.75	0.1845	-431.686090759897\\
68.75	0.18816	-456.160490105894\\
68.75	0.19182	-481.456630327305\\
68.75	0.19548	-507.574511424128\\
68.75	0.19914	-534.514133396365\\
68.75	0.2028	-562.275496244016\\
68.75	0.20646	-590.858599967079\\
68.75	0.21012	-620.263444565556\\
68.75	0.21378	-650.490030039445\\
68.75	0.21744	-681.538356388748\\
68.75	0.2211	-713.408423613464\\
68.75	0.22476	-746.100231713594\\
68.75	0.22842	-779.613780689136\\
68.75	0.23208	-813.949070540091\\
68.75	0.23574	-849.106101266461\\
68.75	0.2394	-885.084872868243\\
68.75	0.24306	-921.885385345438\\
68.75	0.24672	-959.507638698046\\
68.75	0.25038	-997.951632926068\\
68.75	0.25404	-1037.2173680295\\
68.75	0.2577	-1077.30484400835\\
68.75	0.26136	-1118.21406086261\\
68.75	0.26502	-1159.94501859229\\
68.75	0.26868	-1202.49771719737\\
68.75	0.27234	-1245.87215667788\\
68.75	0.276	-1290.06833703379\\
69.125	0.093	-88.5428389187738\\
69.125	0.09666	-92.4856787351894\\
69.125	0.10032	-97.2502594270175\\
69.125	0.10398	-102.836580994259\\
69.125	0.10764	-109.244643436914\\
69.125	0.1113	-116.474446754981\\
69.125	0.11496	-124.525990948463\\
69.125	0.11862	-133.399276017357\\
69.125	0.12228	-143.094301961665\\
69.125	0.12594	-153.611068781385\\
69.125	0.1296	-164.949576476519\\
69.125	0.13326	-177.109825047066\\
69.125	0.13692	-190.091814493027\\
69.125	0.14058	-203.8955448144\\
69.125	0.14424	-218.521016011187\\
69.125	0.1479	-233.968228083387\\
69.125	0.15156	-250.237181031\\
69.125	0.15522	-267.327874854026\\
69.125	0.15888	-285.240309552466\\
69.125	0.16254	-303.974485126319\\
69.125	0.1662	-323.530401575584\\
69.125	0.16986	-343.908058900263\\
69.125	0.17352	-365.107457100356\\
69.125	0.17718	-387.128596175861\\
69.125	0.18084	-409.97147612678\\
69.125	0.1845	-433.636096953112\\
69.125	0.18816	-458.122458654857\\
69.125	0.19182	-483.430561232015\\
69.125	0.19548	-509.560404684587\\
69.125	0.19914	-536.511989012571\\
69.125	0.2028	-564.285314215969\\
69.125	0.20646	-592.88038029478\\
69.125	0.21012	-622.297187249005\\
69.125	0.21378	-652.535735078642\\
69.125	0.21744	-683.596023783693\\
69.125	0.2211	-715.478053364156\\
69.125	0.22476	-748.181823820034\\
69.125	0.22842	-781.707335151324\\
69.125	0.23208	-816.054587358027\\
69.125	0.23574	-851.223580440144\\
69.125	0.2394	-887.214314397674\\
69.125	0.24306	-924.026789230617\\
69.125	0.24672	-961.661004938973\\
69.125	0.25038	-1000.11696152274\\
69.125	0.25404	-1039.39465898193\\
69.125	0.2577	-1079.49409731652\\
69.125	0.26136	-1120.41527652653\\
69.125	0.26502	-1162.15819661195\\
69.125	0.26868	-1204.72285757279\\
69.125	0.27234	-1248.10925940904\\
69.125	0.276	-1292.3174021207\\
69.5	0.093	-90.2648198934634\\
69.5	0.09666	-94.2196220656265\\
69.5	0.10032	-98.9961651132026\\
69.5	0.10398	-104.594449036192\\
69.5	0.10764	-111.014473834594\\
69.5	0.1113	-118.25623950841\\
69.5	0.11496	-126.319746057639\\
69.5	0.11862	-135.204993482281\\
69.5	0.12228	-144.911981782336\\
69.5	0.12594	-155.440710957804\\
69.5	0.1296	-166.791181008686\\
69.5	0.13326	-178.963391934981\\
69.5	0.13692	-191.957343736689\\
69.5	0.14058	-205.77303641381\\
69.5	0.14424	-220.410469966345\\
69.5	0.1479	-235.869644394293\\
69.5	0.15156	-252.150559697653\\
69.5	0.15522	-269.253215876427\\
69.5	0.15888	-287.177612930615\\
69.5	0.16254	-305.923750860215\\
69.5	0.1662	-325.491629665229\\
69.5	0.16986	-345.881249345656\\
69.5	0.17352	-367.092609901496\\
69.5	0.17718	-389.125711332749\\
69.5	0.18084	-411.980553639415\\
69.5	0.1845	-435.657136821495\\
69.5	0.18816	-460.155460878988\\
69.5	0.19182	-485.475525811894\\
69.5	0.19548	-511.617331620213\\
69.5	0.19914	-538.580878303946\\
69.5	0.2028	-566.366165863092\\
69.5	0.20646	-594.97319429765\\
69.5	0.21012	-624.401963607623\\
69.5	0.21378	-654.652473793008\\
69.5	0.21744	-685.724724853806\\
69.5	0.2211	-717.618716790018\\
69.5	0.22476	-750.334449601643\\
69.5	0.22842	-783.871923288681\\
69.5	0.23208	-818.231137851132\\
69.5	0.23574	-853.412093288996\\
69.5	0.2394	-889.414789602274\\
69.5	0.24306	-926.239226790965\\
69.5	0.24672	-963.885404855068\\
69.5	0.25038	-1002.35332379459\\
69.5	0.25404	-1041.64298360952\\
69.5	0.2577	-1081.75438429986\\
69.5	0.26136	-1122.68752586562\\
69.5	0.26502	-1164.44240830679\\
69.5	0.26868	-1207.01903162337\\
69.5	0.27234	-1250.41739581537\\
69.5	0.276	-1294.63750088278\\
69.875	0.093	-92.0578345433218\\
69.875	0.09666	-96.0245990712326\\
69.875	0.10032	-100.813104474556\\
69.875	0.10398	-106.423350753293\\
69.875	0.10764	-112.855337907444\\
69.875	0.1113	-120.109065937007\\
69.875	0.11496	-128.184534841984\\
69.875	0.11862	-137.081744622373\\
69.875	0.12228	-146.800695278177\\
69.875	0.12594	-157.341386809393\\
69.875	0.1296	-168.703819216022\\
69.875	0.13326	-180.887992498065\\
69.875	0.13692	-193.893906655521\\
69.875	0.14058	-207.72156168839\\
69.875	0.14424	-222.370957596672\\
69.875	0.1479	-237.842094380367\\
69.875	0.15156	-254.134972039476\\
69.875	0.15522	-271.249590573998\\
69.875	0.15888	-289.185949983933\\
69.875	0.16254	-307.944050269281\\
69.875	0.1662	-327.523891430042\\
69.875	0.16986	-347.925473466217\\
69.875	0.17352	-369.148796377805\\
69.875	0.17718	-391.193860164806\\
69.875	0.18084	-414.06066482722\\
69.875	0.1845	-437.749210365047\\
69.875	0.18816	-462.259496778288\\
69.875	0.19182	-487.591524066942\\
69.875	0.19548	-513.745292231009\\
69.875	0.19914	-540.720801270489\\
69.875	0.2028	-568.518051185382\\
69.875	0.20646	-597.137041975689\\
69.875	0.21012	-626.577773641409\\
69.875	0.21378	-656.840246182542\\
69.875	0.21744	-687.924459599088\\
69.875	0.2211	-719.830413891047\\
69.875	0.22476	-752.55810905842\\
69.875	0.22842	-786.107545101206\\
69.875	0.23208	-820.478722019405\\
69.875	0.23574	-855.671639813017\\
69.875	0.2394	-891.686298482043\\
69.875	0.24306	-928.522698026481\\
69.875	0.24672	-966.180838446333\\
69.875	0.25038	-1004.6607197416\\
69.875	0.25404	-1043.96234191228\\
69.875	0.2577	-1084.08570495837\\
69.875	0.26136	-1125.03080887987\\
69.875	0.26502	-1166.79765367679\\
69.875	0.26868	-1209.38623934912\\
69.875	0.27234	-1252.79656589686\\
69.875	0.276	-1297.02863332002\\
70.25	0.093	-93.921882868349\\
70.25	0.09666	-97.9006097520078\\
70.25	0.10032	-102.701077511079\\
70.25	0.10398	-108.323286145564\\
70.25	0.10764	-114.767235655462\\
70.25	0.1113	-122.032926040773\\
70.25	0.11496	-130.120357301498\\
70.25	0.11862	-139.029529437635\\
70.25	0.12228	-148.760442449186\\
70.25	0.12594	-159.31309633615\\
70.25	0.1296	-170.687491098527\\
70.25	0.13326	-182.883626736317\\
70.25	0.13692	-195.901503249521\\
70.25	0.14058	-209.741120638138\\
70.25	0.14424	-224.402478902168\\
70.25	0.1479	-239.885578041611\\
70.25	0.15156	-256.190418056467\\
70.25	0.15522	-273.316998946737\\
70.25	0.15888	-291.26532071242\\
70.25	0.16254	-310.035383353516\\
70.25	0.1662	-329.627186870025\\
70.25	0.16986	-350.040731261947\\
70.25	0.17352	-371.276016529283\\
70.25	0.17718	-393.333042672031\\
70.25	0.18084	-416.211809690193\\
70.25	0.1845	-439.912317583769\\
70.25	0.18816	-464.434566352757\\
70.25	0.19182	-489.778555997159\\
70.25	0.19548	-515.944286516973\\
70.25	0.19914	-542.931757912201\\
70.25	0.2028	-570.740970182843\\
70.25	0.20646	-599.371923328897\\
70.25	0.21012	-628.824617350364\\
70.25	0.21378	-659.099052247245\\
70.25	0.21744	-690.195228019539\\
70.25	0.2211	-722.113144667246\\
70.25	0.22476	-754.852802190366\\
70.25	0.22842	-788.4142005889\\
70.25	0.23208	-822.797339862847\\
70.25	0.23574	-858.002220012207\\
70.25	0.2394	-894.02884103698\\
70.25	0.24306	-930.877202937166\\
70.25	0.24672	-968.547305712766\\
70.25	0.25038	-1007.03914936378\\
70.25	0.25404	-1046.3527338902\\
70.25	0.2577	-1086.48805929204\\
70.25	0.26136	-1127.4451255693\\
70.25	0.26502	-1169.22393272196\\
70.25	0.26868	-1211.82448075004\\
70.25	0.27234	-1255.24676965353\\
70.25	0.276	-1299.49079943244\\
70.625	0.093	-95.8569648685447\\
70.625	0.09666	-99.847654107951\\
70.625	0.10032	-104.66008422277\\
70.625	0.10398	-110.294255213003\\
70.625	0.10764	-116.750167078649\\
70.625	0.1113	-124.027819819707\\
70.625	0.11496	-132.12721343618\\
70.625	0.11862	-141.048347928065\\
70.625	0.12228	-150.791223295364\\
70.625	0.12594	-161.355839538075\\
70.625	0.1296	-172.7421966562\\
70.625	0.13326	-184.950294649738\\
70.625	0.13692	-197.98013351869\\
70.625	0.14058	-211.831713263054\\
70.625	0.14424	-226.505033882832\\
70.625	0.1479	-242.000095378023\\
70.625	0.15156	-258.316897748627\\
70.625	0.15522	-275.455440994644\\
70.625	0.15888	-293.415725116075\\
70.625	0.16254	-312.197750112919\\
70.625	0.1662	-331.801515985176\\
70.625	0.16986	-352.227022732846\\
70.625	0.17352	-373.474270355929\\
70.625	0.17718	-395.543258854426\\
70.625	0.18084	-418.433988228335\\
70.625	0.1845	-442.146458477658\\
70.625	0.18816	-466.680669602395\\
70.625	0.19182	-492.036621602544\\
70.625	0.19548	-518.214314478106\\
70.625	0.19914	-545.213748229082\\
70.625	0.2028	-573.034922855471\\
70.625	0.20646	-601.677838357273\\
70.625	0.21012	-631.142494734488\\
70.625	0.21378	-661.428891987117\\
70.625	0.21744	-692.537030115159\\
70.625	0.2211	-724.466909118613\\
70.625	0.22476	-757.218528997481\\
70.625	0.22842	-790.791889751763\\
70.625	0.23208	-825.186991381458\\
70.625	0.23574	-860.403833886566\\
70.625	0.2394	-896.442417267086\\
70.625	0.24306	-933.30274152302\\
70.625	0.24672	-970.984806654368\\
70.625	0.25038	-1009.48861266113\\
70.625	0.25404	-1048.8141595433\\
70.625	0.2577	-1088.96144730089\\
70.625	0.26136	-1129.93047593389\\
70.625	0.26502	-1171.7212454423\\
70.625	0.26868	-1214.33375582613\\
70.625	0.27234	-1257.76800708537\\
70.625	0.276	-1302.02399922002\\
71	0.093	-97.863080543909\\
71	0.09666	-101.865732139063\\
71	0.10032	-106.69012460963\\
71	0.10398	-112.33625795561\\
71	0.10764	-118.804132177004\\
71	0.1113	-126.09374727381\\
71	0.11496	-134.205103246031\\
71	0.11862	-143.138200093663\\
71	0.12228	-152.89303781671\\
71	0.12594	-163.469616415169\\
71	0.1296	-174.867935889042\\
71	0.13326	-187.087996238328\\
71	0.13692	-200.129797463027\\
71	0.14058	-213.993339563139\\
71	0.14424	-228.678622538665\\
71	0.1479	-244.185646389604\\
71	0.15156	-260.514411115955\\
71	0.15522	-277.664916717721\\
71	0.15888	-295.637163194899\\
71	0.16254	-314.431150547491\\
71	0.1662	-334.046878775495\\
71	0.16986	-354.484347878913\\
71	0.17352	-375.743557857744\\
71	0.17718	-397.824508711988\\
71	0.18084	-420.727200441646\\
71	0.1845	-444.451633046717\\
71	0.18816	-468.997806527201\\
71	0.19182	-494.365720883098\\
71	0.19548	-520.555376114408\\
71	0.19914	-547.566772221131\\
71	0.2028	-575.399909203268\\
71	0.20646	-604.054787060818\\
71	0.21012	-633.531405793781\\
71	0.21378	-663.829765402157\\
71	0.21744	-694.949865885947\\
71	0.2211	-726.891707245149\\
71	0.22476	-759.655289479765\\
71	0.22842	-793.240612589794\\
71	0.23208	-827.647676575237\\
71	0.23574	-862.876481436092\\
71	0.2394	-898.927027172361\\
71	0.24306	-935.799313784042\\
71	0.24672	-973.493341271138\\
71	0.25038	-1012.00910963365\\
71	0.25404	-1051.34661887157\\
71	0.2577	-1091.5058689849\\
71	0.26136	-1132.48685997365\\
71	0.26502	-1174.28959183781\\
71	0.26868	-1216.91406457739\\
71	0.27234	-1260.36027819237\\
71	0.276	-1304.62823268277\\
71.375	0.093	-99.9402298944421\\
71.375	0.09666	-103.954843845344\\
71.375	0.10032	-108.791198671659\\
71.375	0.10398	-114.449294373387\\
71.375	0.10764	-120.929130950528\\
71.375	0.1113	-128.230708403082\\
71.375	0.11496	-136.35402673105\\
71.375	0.11862	-145.299085934431\\
71.375	0.12228	-155.065886013225\\
71.375	0.12594	-165.654426967432\\
71.375	0.1296	-177.064708797053\\
71.375	0.13326	-189.296731502086\\
71.375	0.13692	-202.350495082533\\
71.375	0.14058	-216.225999538393\\
71.375	0.14424	-230.923244869667\\
71.375	0.1479	-246.442231076353\\
71.375	0.15156	-262.782958158453\\
71.375	0.15522	-279.945426115965\\
71.375	0.15888	-297.929634948891\\
71.375	0.16254	-316.735584657231\\
71.375	0.1662	-336.363275240983\\
71.375	0.16986	-356.812706700149\\
71.375	0.17352	-378.083879034728\\
71.375	0.17718	-400.17679224472\\
71.375	0.18084	-423.091446330125\\
71.375	0.1845	-446.827841290944\\
71.375	0.18816	-471.385977127175\\
71.375	0.19182	-496.76585383882\\
71.375	0.19548	-522.967471425878\\
71.375	0.19914	-549.990829888349\\
71.375	0.2028	-577.835929226234\\
71.375	0.20646	-606.502769439532\\
71.375	0.21012	-635.991350528242\\
71.375	0.21378	-666.301672492366\\
71.375	0.21744	-697.433735331904\\
71.375	0.2211	-729.387539046854\\
71.375	0.22476	-762.163083637217\\
71.375	0.22842	-795.760369102994\\
71.375	0.23208	-830.179395444184\\
71.375	0.23574	-865.420162660788\\
71.375	0.2394	-901.482670752804\\
71.375	0.24306	-938.366919720233\\
71.375	0.24672	-976.072909563076\\
71.375	0.25038	-1014.60064028133\\
71.375	0.25404	-1053.950111875\\
71.375	0.2577	-1094.12132434408\\
71.375	0.26136	-1135.11427768858\\
71.375	0.26502	-1176.92897190849\\
71.375	0.26868	-1219.56540700381\\
71.375	0.27234	-1263.02358297455\\
71.375	0.276	-1307.30349982069\\
71.75	0.093	-102.088412920144\\
71.75	0.09666	-106.114989226794\\
71.75	0.10032	-110.963306408856\\
71.75	0.10398	-116.633364466332\\
71.75	0.10764	-123.125163399221\\
71.75	0.1113	-130.438703207523\\
71.75	0.11496	-138.573983891239\\
71.75	0.11862	-147.531005450367\\
71.75	0.12228	-157.309767884909\\
71.75	0.12594	-167.910271194864\\
71.75	0.1296	-179.332515380233\\
71.75	0.13326	-191.576500441014\\
71.75	0.13692	-204.642226377209\\
71.75	0.14058	-218.529693188817\\
71.75	0.14424	-233.238900875837\\
71.75	0.1479	-248.769849438272\\
71.75	0.15156	-265.122538876119\\
71.75	0.15522	-282.29696918938\\
71.75	0.15888	-300.293140378053\\
71.75	0.16254	-319.111052442141\\
71.75	0.1662	-338.750705381641\\
71.75	0.16986	-359.212099196554\\
71.75	0.17352	-380.495233886881\\
71.75	0.17718	-402.60010945262\\
71.75	0.18084	-425.526725893773\\
71.75	0.1845	-449.27508321034\\
71.75	0.18816	-473.845181402319\\
71.75	0.19182	-499.237020469712\\
71.75	0.19548	-525.450600412518\\
71.75	0.19914	-552.485921230736\\
71.75	0.2028	-580.342982924368\\
71.75	0.20646	-609.021785493414\\
71.75	0.21012	-638.522328937873\\
71.75	0.21378	-668.844613257744\\
71.75	0.21744	-699.988638453029\\
71.75	0.2211	-731.954404523728\\
71.75	0.22476	-764.741911469839\\
71.75	0.22842	-798.351159291364\\
71.75	0.23208	-832.782147988301\\
71.75	0.23574	-868.034877560653\\
71.75	0.2394	-904.109348008417\\
71.75	0.24306	-941.005559331594\\
71.75	0.24672	-978.723511530184\\
71.75	0.25038	-1017.26320460419\\
71.75	0.25404	-1056.6246385536\\
71.75	0.2577	-1096.80781337844\\
71.75	0.26136	-1137.81272907868\\
71.75	0.26502	-1179.63938565434\\
71.75	0.26868	-1222.2877831054\\
71.75	0.27234	-1265.75792143189\\
71.75	0.276	-1310.04980063378\\
72.125	0.093	-104.307629621015\\
72.125	0.09666	-108.346168283412\\
72.125	0.10032	-113.206447821222\\
72.125	0.10398	-118.888468234446\\
72.125	0.10764	-125.392229523083\\
72.125	0.1113	-132.717731687133\\
72.125	0.11496	-140.864974726596\\
72.125	0.11862	-149.833958641472\\
72.125	0.12228	-159.624683431762\\
72.125	0.12594	-170.237149097464\\
72.125	0.1296	-181.671355638581\\
72.125	0.13326	-193.92730305511\\
72.125	0.13692	-207.004991347052\\
72.125	0.14058	-220.904420514408\\
72.125	0.14424	-235.625590557176\\
72.125	0.1479	-251.168501475358\\
72.125	0.15156	-267.533153268953\\
72.125	0.15522	-284.719545937962\\
72.125	0.15888	-302.727679482383\\
72.125	0.16254	-321.557553902219\\
72.125	0.1662	-341.209169197466\\
72.125	0.16986	-361.682525368128\\
72.125	0.17352	-382.977622414202\\
72.125	0.17718	-405.094460335689\\
72.125	0.18084	-428.03303913259\\
72.125	0.1845	-451.793358804904\\
72.125	0.18816	-476.375419352631\\
72.125	0.19182	-501.779220775772\\
72.125	0.19548	-528.004763074325\\
72.125	0.19914	-555.052046248292\\
72.125	0.2028	-582.921070297672\\
72.125	0.20646	-611.611835222465\\
72.125	0.21012	-641.124341022672\\
72.125	0.21378	-671.458587698291\\
72.125	0.21744	-702.614575249323\\
72.125	0.2211	-734.59230367577\\
72.125	0.22476	-767.391772977629\\
72.125	0.22842	-801.012983154901\\
72.125	0.23208	-835.455934207587\\
72.125	0.23574	-870.720626135685\\
72.125	0.2394	-906.807058939198\\
72.125	0.24306	-943.715232618123\\
72.125	0.24672	-981.445147172461\\
72.125	0.25038	-1019.99680260221\\
72.125	0.25404	-1059.37019890738\\
72.125	0.2577	-1099.56533608796\\
72.125	0.26136	-1140.58221414395\\
72.125	0.26502	-1182.42083307535\\
72.125	0.26868	-1225.08119288217\\
72.125	0.27234	-1268.5632935644\\
72.125	0.276	-1312.86713512204\\
72.5	0.093	-106.597879997054\\
72.5	0.09666	-110.648381015199\\
72.5	0.10032	-115.520622908757\\
72.5	0.10398	-121.214605677728\\
72.5	0.10764	-127.730329322113\\
72.5	0.1113	-135.067793841911\\
72.5	0.11496	-143.226999237122\\
72.5	0.11862	-152.207945507745\\
72.5	0.12228	-162.010632653783\\
72.5	0.12594	-172.635060675233\\
72.5	0.1296	-184.081229572097\\
72.5	0.13326	-196.349139344374\\
72.5	0.13692	-209.438789992064\\
72.5	0.14058	-223.350181515168\\
72.5	0.14424	-238.083313913684\\
72.5	0.1479	-253.638187187614\\
72.5	0.15156	-270.014801336957\\
72.5	0.15522	-287.213156361713\\
72.5	0.15888	-305.233252261882\\
72.5	0.16254	-324.075089037465\\
72.5	0.1662	-343.738666688461\\
72.5	0.16986	-364.22398521487\\
72.5	0.17352	-385.531044616692\\
72.5	0.17718	-407.659844893927\\
72.5	0.18084	-430.610386046575\\
72.5	0.1845	-454.382668074637\\
72.5	0.18816	-478.976690978112\\
72.5	0.19182	-504.392454757\\
72.5	0.19548	-530.629959411302\\
72.5	0.19914	-557.689204941016\\
72.5	0.2028	-585.570191346143\\
72.5	0.20646	-614.272918626685\\
72.5	0.21012	-643.797386782639\\
72.5	0.21378	-674.143595814006\\
72.5	0.21744	-705.311545720786\\
72.5	0.2211	-737.30123650298\\
72.5	0.22476	-770.112668160587\\
72.5	0.22842	-803.745840693607\\
72.5	0.23208	-838.20075410204\\
72.5	0.23574	-873.477408385887\\
72.5	0.2394	-909.575803545147\\
72.5	0.24306	-946.49593957982\\
72.5	0.24672	-984.237816489906\\
72.5	0.25038	-1022.80143427541\\
72.5	0.25404	-1062.18679293632\\
72.5	0.2577	-1102.39389247264\\
72.5	0.26136	-1143.42273288438\\
72.5	0.26502	-1185.27331417153\\
72.5	0.26868	-1227.9456363341\\
72.5	0.27234	-1271.43969937208\\
72.5	0.276	-1315.75550328547\\
72.875	0.093	-108.959164048261\\
72.875	0.09666	-113.021627422155\\
72.875	0.10032	-117.90583167146\\
72.875	0.10398	-123.611776796179\\
72.875	0.10764	-130.139462796312\\
72.875	0.1113	-137.488889671857\\
72.875	0.11496	-145.660057422816\\
72.875	0.11862	-154.652966049188\\
72.875	0.12228	-164.467615550973\\
72.875	0.12594	-175.104005928171\\
72.875	0.1296	-186.562137180783\\
72.875	0.13326	-198.842009308807\\
72.875	0.13692	-211.943622312245\\
72.875	0.14058	-225.866976191096\\
72.875	0.14424	-240.612070945361\\
72.875	0.1479	-256.178906575038\\
72.875	0.15156	-272.567483080129\\
72.875	0.15522	-289.777800460633\\
72.875	0.15888	-307.80985871655\\
72.875	0.16254	-326.66365784788\\
72.875	0.1662	-346.339197854623\\
72.875	0.16986	-366.83647873678\\
72.875	0.17352	-388.15550049435\\
72.875	0.17718	-410.296263127333\\
72.875	0.18084	-433.258766635729\\
72.875	0.1845	-457.043011019539\\
72.875	0.18816	-481.648996278762\\
72.875	0.19182	-507.076722413398\\
72.875	0.19548	-533.326189423447\\
72.875	0.19914	-560.397397308908\\
72.875	0.2028	-588.290346069784\\
72.875	0.20646	-617.005035706073\\
72.875	0.21012	-646.541466217775\\
72.875	0.21378	-676.89963760489\\
72.875	0.21744	-708.079549867418\\
72.875	0.2211	-740.08120300536\\
72.875	0.22476	-772.904597018714\\
72.875	0.22842	-806.549731907482\\
72.875	0.23208	-841.016607671663\\
72.875	0.23574	-876.305224311257\\
72.875	0.2394	-912.415581826265\\
72.875	0.24306	-949.347680216686\\
72.875	0.24672	-987.101519482519\\
72.875	0.25038	-1025.67709962377\\
72.875	0.25404	-1065.07442064043\\
72.875	0.2577	-1105.2934825325\\
72.875	0.26136	-1146.33428529999\\
72.875	0.26502	-1188.19682894289\\
72.875	0.26868	-1230.8811134612\\
72.875	0.27234	-1274.38713885493\\
72.875	0.276	-1318.71490512407\\
73.25	0.093	-111.391481774638\\
73.25	0.09666	-115.465907504279\\
73.25	0.10032	-120.362074109332\\
73.25	0.10398	-126.079981589799\\
73.25	0.10764	-132.619629945679\\
73.25	0.1113	-139.981019176972\\
73.25	0.11496	-148.164149283679\\
73.25	0.11862	-157.169020265798\\
73.25	0.12228	-166.995632123331\\
73.25	0.12594	-177.643984856277\\
73.25	0.1296	-189.114078464637\\
73.25	0.13326	-201.405912948409\\
73.25	0.13692	-214.519488307595\\
73.25	0.14058	-228.454804542194\\
73.25	0.14424	-243.211861652206\\
73.25	0.1479	-258.790659637631\\
73.25	0.15156	-275.191198498469\\
73.25	0.15522	-292.413478234721\\
73.25	0.15888	-310.457498846386\\
73.25	0.16254	-329.323260333464\\
73.25	0.1662	-349.010762695955\\
73.25	0.16986	-369.52000593386\\
73.25	0.17352	-390.850990047177\\
73.25	0.17718	-413.003715035908\\
73.25	0.18084	-435.978180900052\\
73.25	0.1845	-459.774387639609\\
73.25	0.18816	-484.39233525458\\
73.25	0.19182	-509.832023744964\\
73.25	0.19548	-536.09345311076\\
73.25	0.19914	-563.17662335197\\
73.25	0.2028	-591.081534468593\\
73.25	0.20646	-619.80818646063\\
73.25	0.21012	-649.35657932808\\
73.25	0.21378	-679.726713070942\\
73.25	0.21744	-710.918587689218\\
73.25	0.2211	-742.932203182907\\
73.25	0.22476	-775.76755955201\\
73.25	0.22842	-809.424656796526\\
73.25	0.23208	-843.903494916454\\
73.25	0.23574	-879.204073911797\\
73.25	0.2394	-915.326393782552\\
73.25	0.24306	-952.27045452872\\
73.25	0.24672	-990.036256150302\\
73.25	0.25038	-1028.6237986473\\
73.25	0.25404	-1068.0330820197\\
73.25	0.2577	-1108.26410626753\\
73.25	0.26136	-1149.31687139076\\
73.25	0.26502	-1191.19137738941\\
73.25	0.26868	-1233.88762426347\\
73.25	0.27234	-1277.40561201294\\
73.25	0.276	-1321.74534063783\\
73.625	0.093	-113.894833176184\\
73.625	0.09666	-117.981221261572\\
73.625	0.10032	-122.889350222374\\
73.625	0.10398	-128.619220058588\\
73.625	0.10764	-135.170830770216\\
73.625	0.1113	-142.544182357257\\
73.625	0.11496	-150.739274819711\\
73.625	0.11862	-159.756108157579\\
73.625	0.12228	-169.594682370859\\
73.625	0.12594	-180.254997459553\\
73.625	0.1296	-191.73705342366\\
73.625	0.13326	-204.04085026318\\
73.625	0.13692	-217.166387978114\\
73.625	0.14058	-231.11366656846\\
73.625	0.14424	-245.88268603422\\
73.625	0.1479	-261.473446375393\\
73.625	0.15156	-277.885947591979\\
73.625	0.15522	-295.120189683979\\
73.625	0.15888	-313.176172651391\\
73.625	0.16254	-332.053896494217\\
73.625	0.1662	-351.753361212456\\
73.625	0.16986	-372.274566806109\\
73.625	0.17352	-393.617513275174\\
73.625	0.17718	-415.782200619652\\
73.625	0.18084	-438.768628839544\\
73.625	0.1845	-462.576797934849\\
73.625	0.18816	-487.206707905567\\
73.625	0.19182	-512.658358751698\\
73.625	0.19548	-538.931750473243\\
73.625	0.19914	-566.026883070201\\
73.625	0.2028	-593.943756542572\\
73.625	0.20646	-622.682370890356\\
73.625	0.21012	-652.242726113554\\
73.625	0.21378	-682.624822212164\\
73.625	0.21744	-713.828659186188\\
73.625	0.2211	-745.854237035625\\
73.625	0.22476	-778.701555760475\\
73.625	0.22842	-812.370615360738\\
73.625	0.23208	-846.861415836415\\
73.625	0.23574	-882.173957187505\\
73.625	0.2394	-918.308239414008\\
73.625	0.24306	-955.264262515924\\
73.625	0.24672	-993.042026493253\\
73.625	0.25038	-1031.641531346\\
73.625	0.25404	-1071.06277707415\\
73.625	0.2577	-1111.30576367772\\
73.625	0.26136	-1152.3704911567\\
73.625	0.26502	-1194.2569595111\\
73.625	0.26868	-1236.96516874091\\
73.625	0.27234	-1280.49511884613\\
73.625	0.276	-1324.84680982676\\
74	0.093	-116.469218252898\\
74	0.09666	-120.567568694034\\
74	0.10032	-125.487660010583\\
74	0.10398	-131.229492202545\\
74	0.10764	-137.793065269921\\
74	0.1113	-145.17837921271\\
74	0.11496	-153.385434030912\\
74	0.11862	-162.414229724527\\
74	0.12228	-172.264766293555\\
74	0.12594	-182.937043737997\\
74	0.1296	-194.431062057852\\
74	0.13326	-206.746821253119\\
74	0.13692	-219.884321323801\\
74	0.14058	-233.843562269895\\
74	0.14424	-248.624544091403\\
74	0.1479	-264.227266788323\\
74	0.15156	-280.651730360657\\
74	0.15522	-297.897934808405\\
74	0.15888	-315.965880131565\\
74	0.16254	-334.855566330139\\
74	0.1662	-354.566993404125\\
74	0.16986	-375.100161353525\\
74	0.17352	-396.455070178338\\
74	0.17718	-418.631719878564\\
74	0.18084	-441.630110454204\\
74	0.1845	-465.450241905257\\
74	0.18816	-490.092114231723\\
74	0.19182	-515.555727433602\\
74	0.19548	-541.841081510894\\
74	0.19914	-568.9481764636\\
74	0.2028	-596.877012291719\\
74	0.20646	-625.627588995251\\
74	0.21012	-655.199906574196\\
74	0.21378	-685.593965028554\\
74	0.21744	-716.809764358326\\
74	0.2211	-748.84730456351\\
74	0.22476	-781.706585644108\\
74	0.22842	-815.38760760012\\
74	0.23208	-849.890370431544\\
74	0.23574	-885.214874138382\\
74	0.2394	-921.361118720632\\
74	0.24306	-958.329104178296\\
74	0.24672	-996.118830511373\\
74	0.25038	-1034.73029771986\\
74	0.25404	-1074.16350580377\\
74	0.2577	-1114.41845476308\\
74	0.26136	-1155.49514459781\\
74	0.26502	-1197.39357530796\\
74	0.26868	-1240.11374689351\\
74	0.27234	-1283.65565935448\\
74	0.276	-1328.01931269086\\
};
\end{axis}

\begin{axis}[%
width=4.527496cm,
height=3.870968cm,
at={(6.483547cm,10.752688cm)},
scale only axis,
xmin=56,
xmax=74,
tick align=outside,
xlabel={$L_{cut}$},
xmajorgrids,
ymin=0.093,
ymax=0.276,
ylabel={$D_{rlx}$},
ymajorgrids,
zmin=-1.52969162398006,
zmax=2.51190540588646,
zlabel={$x_1,x_1$},
zmajorgrids,
view={-140}{50},
legend style={at={(1.03,1)},anchor=north west,legend cell align=left,align=left,draw=white!15!black}
]
\addplot3[only marks,mark=*,mark options={},mark size=1.5000pt,color=mycolor1] plot table[row sep=crcr,]{%
74	0.123	0.00740370360905991\\
72	0.113	-0.186626587578763\\
61	0.095	-0.150607769785267\\
56	0.093	-0.216975707230133\\
};
\addplot3[only marks,mark=*,mark options={},mark size=1.5000pt,color=mycolor2] plot table[row sep=crcr,]{%
67	0.276	0.0527524251509314\\
66	0.255	-0.475927501616939\\
62	0.209	0.435130634728223\\
57	0.193	0.76117565090573\\
};
\addplot3[only marks,mark=*,mark options={},mark size=1.5000pt,color=black] plot table[row sep=crcr,]{%
69	0.104	-0.0638888311101987\\
};
\addplot3[only marks,mark=*,mark options={},mark size=1.5000pt,color=black] plot table[row sep=crcr,]{%
64	0.23	0.0838528346972987\\
};

\addplot3[%
surf,
opacity=0.7,
shader=interp,
colormap={mymap}{[1pt] rgb(0pt)=(0.0901961,0.239216,0.0745098); rgb(1pt)=(0.0945149,0.242058,0.0739522); rgb(2pt)=(0.0988592,0.244894,0.0733566); rgb(3pt)=(0.103229,0.247724,0.0727241); rgb(4pt)=(0.107623,0.250549,0.0720557); rgb(5pt)=(0.112043,0.253367,0.0713525); rgb(6pt)=(0.116487,0.25618,0.0706154); rgb(7pt)=(0.120956,0.258986,0.0698456); rgb(8pt)=(0.125449,0.261787,0.0690441); rgb(9pt)=(0.129967,0.264581,0.0682118); rgb(10pt)=(0.134508,0.26737,0.06735); rgb(11pt)=(0.139074,0.270152,0.0664596); rgb(12pt)=(0.143663,0.272929,0.0655416); rgb(13pt)=(0.148275,0.275699,0.0645971); rgb(14pt)=(0.152911,0.278463,0.0636271); rgb(15pt)=(0.15757,0.281221,0.0626328); rgb(16pt)=(0.162252,0.283973,0.0616151); rgb(17pt)=(0.166957,0.286719,0.060575); rgb(18pt)=(0.171685,0.289458,0.0595136); rgb(19pt)=(0.176434,0.292191,0.0584321); rgb(20pt)=(0.181207,0.294918,0.0573313); rgb(21pt)=(0.186001,0.297639,0.0562123); rgb(22pt)=(0.190817,0.300353,0.0550763); rgb(23pt)=(0.195655,0.303061,0.0539242); rgb(24pt)=(0.200514,0.305763,0.052757); rgb(25pt)=(0.205395,0.308459,0.0515759); rgb(26pt)=(0.210296,0.311149,0.0503624); rgb(27pt)=(0.215212,0.313846,0.0490067); rgb(28pt)=(0.220142,0.316548,0.0475043); rgb(29pt)=(0.22509,0.319254,0.0458704); rgb(30pt)=(0.230056,0.321962,0.0441205); rgb(31pt)=(0.235042,0.324671,0.04227); rgb(32pt)=(0.240048,0.327379,0.0403343); rgb(33pt)=(0.245078,0.330085,0.0383287); rgb(34pt)=(0.250131,0.332786,0.0362688); rgb(35pt)=(0.25521,0.335482,0.0341698); rgb(36pt)=(0.260317,0.33817,0.0320472); rgb(37pt)=(0.265451,0.340849,0.0299163); rgb(38pt)=(0.270616,0.343517,0.0277927); rgb(39pt)=(0.275813,0.346172,0.0256916); rgb(40pt)=(0.281043,0.348814,0.0236284); rgb(41pt)=(0.286307,0.35144,0.0216186); rgb(42pt)=(0.291607,0.354048,0.0196776); rgb(43pt)=(0.296945,0.356637,0.0178207); rgb(44pt)=(0.302322,0.359206,0.0160634); rgb(45pt)=(0.307739,0.361753,0.0144211); rgb(46pt)=(0.313198,0.364275,0.0129091); rgb(47pt)=(0.318701,0.366772,0.0115428); rgb(48pt)=(0.324249,0.369242,0.0103377); rgb(49pt)=(0.329843,0.371682,0.00930909); rgb(50pt)=(0.335485,0.374093,0.00847245); rgb(51pt)=(0.341176,0.376471,0.00784314); rgb(52pt)=(0.346925,0.378826,0.00732741); rgb(53pt)=(0.352735,0.381168,0.00682184); rgb(54pt)=(0.358605,0.383497,0.00632729); rgb(55pt)=(0.364532,0.385812,0.00584464); rgb(56pt)=(0.370516,0.388113,0.00537476); rgb(57pt)=(0.376552,0.390399,0.00491852); rgb(58pt)=(0.38264,0.39267,0.00447681); rgb(59pt)=(0.388777,0.394925,0.00405048); rgb(60pt)=(0.394962,0.397164,0.00364042); rgb(61pt)=(0.401191,0.399386,0.00324749); rgb(62pt)=(0.407464,0.401592,0.00287258); rgb(63pt)=(0.413777,0.40378,0.00251655); rgb(64pt)=(0.420129,0.40595,0.00218028); rgb(65pt)=(0.426518,0.408102,0.00186463); rgb(66pt)=(0.432942,0.410234,0.00157049); rgb(67pt)=(0.439399,0.412348,0.00129873); rgb(68pt)=(0.445885,0.414441,0.00105022); rgb(69pt)=(0.452401,0.416515,0.000825833); rgb(70pt)=(0.458942,0.418567,0.000626441); rgb(71pt)=(0.465508,0.420599,0.00045292); rgb(72pt)=(0.472096,0.422609,0.000306141); rgb(73pt)=(0.478704,0.424596,0.000186979); rgb(74pt)=(0.485331,0.426562,9.63073e-05); rgb(75pt)=(0.491973,0.428504,3.49981e-05); rgb(76pt)=(0.498628,0.430422,3.92506e-06); rgb(77pt)=(0.505323,0.432315,0); rgb(78pt)=(0.512206,0.434168,0); rgb(79pt)=(0.519282,0.435983,0); rgb(80pt)=(0.526529,0.437764,0); rgb(81pt)=(0.533922,0.439512,0); rgb(82pt)=(0.54144,0.441232,0); rgb(83pt)=(0.549059,0.442927,0); rgb(84pt)=(0.556756,0.444599,0); rgb(85pt)=(0.564508,0.446252,0); rgb(86pt)=(0.572292,0.447889,0); rgb(87pt)=(0.580084,0.449514,0); rgb(88pt)=(0.587863,0.451129,0); rgb(89pt)=(0.595604,0.452737,0); rgb(90pt)=(0.603284,0.454343,0); rgb(91pt)=(0.610882,0.455948,0); rgb(92pt)=(0.618373,0.457556,0); rgb(93pt)=(0.625734,0.459171,0); rgb(94pt)=(0.632943,0.460795,0); rgb(95pt)=(0.639976,0.462432,0); rgb(96pt)=(0.64681,0.464084,0); rgb(97pt)=(0.653423,0.465756,0); rgb(98pt)=(0.659791,0.46745,0); rgb(99pt)=(0.665891,0.469169,0); rgb(100pt)=(0.6717,0.470916,0); rgb(101pt)=(0.677195,0.472696,0); rgb(102pt)=(0.682353,0.47451,0); rgb(103pt)=(0.687242,0.476355,0); rgb(104pt)=(0.691952,0.478225,0); rgb(105pt)=(0.696497,0.480118,0); rgb(106pt)=(0.700887,0.482033,0); rgb(107pt)=(0.705134,0.483968,0); rgb(108pt)=(0.709251,0.485921,0); rgb(109pt)=(0.713249,0.487891,0); rgb(110pt)=(0.71714,0.489876,0); rgb(111pt)=(0.720936,0.491875,0); rgb(112pt)=(0.724649,0.493887,0); rgb(113pt)=(0.72829,0.495909,0); rgb(114pt)=(0.731872,0.49794,0); rgb(115pt)=(0.735406,0.499979,0); rgb(116pt)=(0.738904,0.502025,0); rgb(117pt)=(0.742378,0.504075,0); rgb(118pt)=(0.74584,0.506128,0); rgb(119pt)=(0.749302,0.508182,0); rgb(120pt)=(0.752775,0.510237,0); rgb(121pt)=(0.756272,0.51229,0); rgb(122pt)=(0.759804,0.514339,0); rgb(123pt)=(0.763384,0.516385,0); rgb(124pt)=(0.767022,0.518424,0); rgb(125pt)=(0.770731,0.520455,0); rgb(126pt)=(0.774523,0.522478,0); rgb(127pt)=(0.77841,0.524489,0); rgb(128pt)=(0.782391,0.526491,0); rgb(129pt)=(0.786402,0.528496,0); rgb(130pt)=(0.790431,0.530506,0); rgb(131pt)=(0.794478,0.532521,0); rgb(132pt)=(0.798541,0.534539,0); rgb(133pt)=(0.802619,0.53656,0); rgb(134pt)=(0.806712,0.538584,0); rgb(135pt)=(0.81082,0.540609,0); rgb(136pt)=(0.81494,0.542635,0); rgb(137pt)=(0.819074,0.54466,0); rgb(138pt)=(0.823219,0.546686,0); rgb(139pt)=(0.827374,0.548709,0); rgb(140pt)=(0.831541,0.55073,0); rgb(141pt)=(0.835716,0.552749,0); rgb(142pt)=(0.8399,0.554763,0); rgb(143pt)=(0.844092,0.556774,0); rgb(144pt)=(0.848292,0.558779,0); rgb(145pt)=(0.852497,0.560778,0); rgb(146pt)=(0.856708,0.562771,0); rgb(147pt)=(0.860924,0.564756,0); rgb(148pt)=(0.865143,0.566733,0); rgb(149pt)=(0.869366,0.568701,0); rgb(150pt)=(0.873592,0.57066,0); rgb(151pt)=(0.877819,0.572608,0); rgb(152pt)=(0.882047,0.574545,0); rgb(153pt)=(0.886275,0.576471,0); rgb(154pt)=(0.890659,0.578362,0); rgb(155pt)=(0.895333,0.580203,0); rgb(156pt)=(0.900258,0.581999,0); rgb(157pt)=(0.905397,0.583755,0); rgb(158pt)=(0.910711,0.585479,0); rgb(159pt)=(0.916164,0.587176,0); rgb(160pt)=(0.921717,0.588852,0); rgb(161pt)=(0.927333,0.590513,0); rgb(162pt)=(0.932974,0.592166,0); rgb(163pt)=(0.938602,0.593815,0); rgb(164pt)=(0.94418,0.595468,0); rgb(165pt)=(0.949669,0.59713,0); rgb(166pt)=(0.955033,0.598808,0); rgb(167pt)=(0.960233,0.600507,0); rgb(168pt)=(0.965232,0.602233,0); rgb(169pt)=(0.969992,0.603992,0); rgb(170pt)=(0.974475,0.605791,0); rgb(171pt)=(0.978643,0.607636,0); rgb(172pt)=(0.98246,0.609532,0); rgb(173pt)=(0.985886,0.611486,0); rgb(174pt)=(0.988885,0.613503,0); rgb(175pt)=(0.991419,0.61559,0); rgb(176pt)=(0.99345,0.617753,0); rgb(177pt)=(0.99494,0.619997,0); rgb(178pt)=(0.995851,0.622329,0); rgb(179pt)=(0.996226,0.624763,0); rgb(180pt)=(0.996512,0.627352,0); rgb(181pt)=(0.996788,0.630095,0); rgb(182pt)=(0.997053,0.632982,0); rgb(183pt)=(0.997308,0.636004,0); rgb(184pt)=(0.997552,0.639152,0); rgb(185pt)=(0.997785,0.642416,0); rgb(186pt)=(0.998006,0.645786,0); rgb(187pt)=(0.998217,0.649253,0); rgb(188pt)=(0.998416,0.652807,0); rgb(189pt)=(0.998605,0.656439,0); rgb(190pt)=(0.998781,0.660138,0); rgb(191pt)=(0.998946,0.663897,0); rgb(192pt)=(0.9991,0.667704,0); rgb(193pt)=(0.999242,0.67155,0); rgb(194pt)=(0.999372,0.675427,0); rgb(195pt)=(0.99949,0.679323,0); rgb(196pt)=(0.999596,0.68323,0); rgb(197pt)=(0.99969,0.687139,0); rgb(198pt)=(0.999771,0.691039,0); rgb(199pt)=(0.999841,0.694921,0); rgb(200pt)=(0.999898,0.698775,0); rgb(201pt)=(0.999942,0.702592,0); rgb(202pt)=(0.999974,0.706363,0); rgb(203pt)=(0.999994,0.710077,0); rgb(204pt)=(1,0.713725,0); rgb(205pt)=(1,0.717341,0); rgb(206pt)=(1,0.720963,0); rgb(207pt)=(1,0.724591,0); rgb(208pt)=(1,0.728226,0); rgb(209pt)=(1,0.731867,0); rgb(210pt)=(1,0.735514,0); rgb(211pt)=(1,0.739167,0); rgb(212pt)=(1,0.742827,0); rgb(213pt)=(1,0.746493,0); rgb(214pt)=(1,0.750165,0); rgb(215pt)=(1,0.753843,0); rgb(216pt)=(1,0.757527,0); rgb(217pt)=(1,0.761217,0); rgb(218pt)=(1,0.764913,0); rgb(219pt)=(1,0.768615,0); rgb(220pt)=(1,0.772324,0); rgb(221pt)=(1,0.776038,0); rgb(222pt)=(1,0.779758,0); rgb(223pt)=(1,0.783484,0); rgb(224pt)=(1,0.787215,0); rgb(225pt)=(1,0.790953,0); rgb(226pt)=(1,0.794696,0); rgb(227pt)=(1,0.798445,0); rgb(228pt)=(1,0.8022,0); rgb(229pt)=(1,0.805961,0); rgb(230pt)=(1,0.809727,0); rgb(231pt)=(1,0.8135,0); rgb(232pt)=(1,0.817278,0); rgb(233pt)=(1,0.821063,0); rgb(234pt)=(1,0.824854,0); rgb(235pt)=(1,0.828652,0); rgb(236pt)=(1,0.832455,0); rgb(237pt)=(1,0.836265,0); rgb(238pt)=(1,0.840081,0); rgb(239pt)=(1,0.843903,0); rgb(240pt)=(1,0.847732,0); rgb(241pt)=(1,0.851566,0); rgb(242pt)=(1,0.855406,0); rgb(243pt)=(1,0.859253,0); rgb(244pt)=(1,0.863106,0); rgb(245pt)=(1,0.866964,0); rgb(246pt)=(1,0.870829,0); rgb(247pt)=(1,0.8747,0); rgb(248pt)=(1,0.878577,0); rgb(249pt)=(1,0.88246,0); rgb(250pt)=(1,0.886349,0); rgb(251pt)=(1,0.890243,0); rgb(252pt)=(1,0.894144,0); rgb(253pt)=(1,0.898051,0); rgb(254pt)=(1,0.901964,0); rgb(255pt)=(1,0.905882,0)},
mesh/rows=49]
table[row sep=crcr,header=false] {%
%
56	0.093	-0.183093463550335\\
56	0.09666	-0.15506912383271\\
56	0.10032	-0.125988635638709\\
56	0.10398	-0.095851998968341\\
56	0.10764	-0.0646592138216007\\
56	0.1113	-0.032410280198485\\
56	0.11496	0.000894801900998998\\
56	0.11862	0.0352560324768565\\
56	0.12228	0.0706734115290841\\
56	0.12594	0.107146939057687\\
56	0.1296	0.14467661506266\\
56	0.13326	0.183262439544003\\
56	0.13692	0.22290441250172\\
56	0.14058	0.263602533935807\\
56	0.14424	0.305356803846267\\
56	0.1479	0.348167222233099\\
56	0.15156	0.392033789096303\\
56	0.15522	0.436956504435879\\
56	0.15888	0.482935368251827\\
56	0.16254	0.529970380544145\\
56	0.1662	0.578061541312836\\
56	0.16986	0.627208850557897\\
56	0.17352	0.677412308279332\\
56	0.17718	0.728671914477137\\
56	0.18084	0.780987669151318\\
56	0.1845	0.834359572301868\\
56	0.18816	0.888787623928789\\
56	0.19182	0.944271824032083\\
56	0.19548	1.00081217261175\\
56	0.19914	1.05840866966778\\
56	0.2028	1.11706131520019\\
56	0.20646	1.17677010920897\\
56	0.21012	1.23753505169413\\
56	0.21378	1.29935614265565\\
56	0.21744	1.36223338209355\\
56	0.2211	1.42616677000781\\
56	0.22476	1.49115630639846\\
56	0.22842	1.55720199126547\\
56	0.23208	1.62430382460885\\
56	0.23574	1.69246180642861\\
56	0.2394	1.76167593672473\\
56	0.24306	1.83194621549724\\
56	0.24672	1.90327264274611\\
56	0.25038	1.97565521847135\\
56	0.25404	2.04909394267296\\
56	0.2577	2.12358881535095\\
56	0.26136	2.19913983650531\\
56	0.26502	2.27574700613604\\
56	0.26868	2.35341032424314\\
56	0.27234	2.43212979082662\\
56	0.276	2.51190540588646\\
56.375	0.093	-0.18948708924544\\
56.375	0.09666	-0.163384241142382\\
56.375	0.10032	-0.136225244562956\\
56.375	0.10398	-0.108010099507157\\
56.375	0.10764	-0.0787388059749854\\
56.375	0.1113	-0.0484113639664421\\
56.375	0.11496	-0.0170277734815271\\
56.375	0.11862	0.0154119654797598\\
56.375	0.12228	0.0489078529174185\\
56.375	0.12594	0.0834598888314491\\
56.375	0.1296	0.119068073221851\\
56.375	0.13326	0.155732406088625\\
56.375	0.13692	0.193452887431771\\
56.375	0.14058	0.232229517251288\\
56.375	0.14424	0.272062295547178\\
56.375	0.1479	0.312951222319441\\
56.375	0.15156	0.354896297568072\\
56.375	0.15522	0.397897521293079\\
56.375	0.15888	0.441954893494456\\
56.375	0.16254	0.487068414172203\\
56.375	0.1662	0.533238083326324\\
56.375	0.16986	0.580463900956817\\
56.375	0.17352	0.628745867063681\\
56.375	0.17718	0.678083981646915\\
56.375	0.18084	0.728478244706525\\
56.375	0.1845	0.779928656242505\\
56.375	0.18816	0.832435216254855\\
56.375	0.19182	0.88599792474358\\
56.375	0.19548	0.940616781708675\\
56.375	0.19914	0.99629178715014\\
56.375	0.2028	1.05302294106798\\
56.375	0.20646	1.11081024346219\\
56.375	0.21012	1.16965369433278\\
56.375	0.21378	1.22955329367973\\
56.375	0.21744	1.29050904150305\\
56.375	0.2211	1.35252093780275\\
56.375	0.22476	1.41558898257882\\
56.375	0.22842	1.47971317583126\\
56.375	0.23208	1.54489351756008\\
56.375	0.23574	1.61113000776526\\
56.375	0.2394	1.67842264644682\\
56.375	0.24306	1.74677143360475\\
56.375	0.24672	1.81617636923905\\
56.375	0.25038	1.88663745334972\\
56.375	0.25404	1.95815468593677\\
56.375	0.2577	2.03072806700018\\
56.375	0.26136	2.10435759653997\\
56.375	0.26502	2.17904327455613\\
56.375	0.26868	2.25478510104866\\
56.375	0.27234	2.33158307601756\\
56.375	0.276	2.40943719946284\\
56.75	0.093	-0.195103341821336\\
56.75	0.09666	-0.170921985332851\\
56.75	0.10032	-0.145684480367992\\
56.75	0.10398	-0.119390826926765\\
56.75	0.10764	-0.0920410250091641\\
56.75	0.1113	-0.0636350746151897\\
56.75	0.11496	-0.0341729757448453\\
56.75	0.11862	-0.00365472839812919\\
56.75	0.12228	0.0279196674249589\\
56.75	0.12594	0.0605502117244188\\
56.75	0.1296	0.0942369045002521\\
56.75	0.13326	0.128979745752454\\
56.75	0.13692	0.164778735481031\\
56.75	0.14058	0.201633873685977\\
56.75	0.14424	0.239545160367297\\
56.75	0.1479	0.278512595524988\\
56.75	0.15156	0.318536179159051\\
56.75	0.15522	0.359615911269485\\
56.75	0.15888	0.401751791856294\\
56.75	0.16254	0.44494382091947\\
56.75	0.1662	0.489191998459022\\
56.75	0.16986	0.534496324474942\\
56.75	0.17352	0.580856798967237\\
56.75	0.17718	0.628273421935903\\
56.75	0.18084	0.67674619338094\\
56.75	0.1845	0.726275113302351\\
56.75	0.18816	0.77686018170013\\
56.75	0.19182	0.828501398574285\\
56.75	0.19548	0.881198763924809\\
56.75	0.19914	0.934952277751706\\
56.75	0.2028	0.989761940054972\\
56.75	0.20646	1.04562775083461\\
56.75	0.21012	1.10254971009063\\
56.75	0.21378	1.16052781782301\\
56.75	0.21744	1.21956207403177\\
56.75	0.2211	1.27965247871689\\
56.75	0.22476	1.34079903187839\\
56.75	0.22842	1.40300173351626\\
56.75	0.23208	1.46626058363051\\
56.75	0.23574	1.53057558222112\\
56.75	0.2394	1.59594672928811\\
56.75	0.24306	1.66237402483147\\
56.75	0.24672	1.7298574688512\\
56.75	0.25038	1.7983970613473\\
56.75	0.25404	1.86799280231977\\
56.75	0.2577	1.93864469176862\\
56.75	0.26136	2.01035272969384\\
56.75	0.26502	2.08311691609543\\
56.75	0.26868	2.15693725097339\\
56.75	0.27234	2.23181373432773\\
56.75	0.276	2.30774636615843\\
57.125	0.093	-0.199942221278023\\
57.125	0.09666	-0.17768235640411\\
57.125	0.10032	-0.154366343053822\\
57.125	0.10398	-0.129994181227162\\
57.125	0.10764	-0.104565870924133\\
57.125	0.1113	-0.0780814121447296\\
57.125	0.11496	-0.0505408048889558\\
57.125	0.11862	-0.0219440491568086\\
57.125	0.12228	0.00770885505170882\\
57.125	0.12594	0.038417907736598\\
57.125	0.1296	0.0701831088978607\\
57.125	0.13326	0.103004458535494\\
57.125	0.13692	0.1368819566495\\
57.125	0.14058	0.171815603239875\\
57.125	0.14424	0.207805398306625\\
57.125	0.1479	0.244851341849747\\
57.125	0.15156	0.282953433869239\\
57.125	0.15522	0.322111674365103\\
57.125	0.15888	0.362326063337341\\
57.125	0.16254	0.403596600785948\\
57.125	0.1662	0.445923286710927\\
57.125	0.16986	0.489306121112278\\
57.125	0.17352	0.533745103990003\\
57.125	0.17718	0.579240235344098\\
57.125	0.18084	0.625791515174564\\
57.125	0.1845	0.673398943481404\\
57.125	0.18816	0.722062520264615\\
57.125	0.19182	0.771782245524197\\
57.125	0.19548	0.822558119260153\\
57.125	0.19914	0.874390141472479\\
57.125	0.2028	0.927278312161178\\
57.125	0.20646	0.981222631326246\\
57.125	0.21012	1.03622309896769\\
57.125	0.21378	1.0922797150855\\
57.125	0.21744	1.14939247967969\\
57.125	0.2211	1.20756139275025\\
57.125	0.22476	1.26678645429718\\
57.125	0.22842	1.32706766432047\\
57.125	0.23208	1.38840502282015\\
57.125	0.23574	1.45079852979619\\
57.125	0.2394	1.51424818524861\\
57.125	0.24306	1.5787539891774\\
57.125	0.24672	1.64431594158256\\
57.125	0.25038	1.71093404246409\\
57.125	0.25404	1.77860829182199\\
57.125	0.2577	1.84733868965627\\
57.125	0.26136	1.91712523596692\\
57.125	0.26502	1.98796793075394\\
57.125	0.26868	2.05986677401732\\
57.125	0.27234	2.13282176575709\\
57.125	0.276	2.20683290597322\\
57.5	0.093	-0.204003727615505\\
57.5	0.09666	-0.183665354356161\\
57.5	0.10032	-0.162270832620443\\
57.5	0.10398	-0.139820162408354\\
57.5	0.10764	-0.116313343719895\\
57.5	0.1113	-0.0917503765550616\\
57.5	0.11496	-0.0661312609138586\\
57.5	0.11862	-0.039455996796282\\
57.5	0.12228	-0.0117245842023352\\
57.5	0.12594	0.0170629768679851\\
57.5	0.1296	0.0469066864146771\\
57.5	0.13326	0.0778065444377393\\
57.5	0.13692	0.109762550937175\\
57.5	0.14058	0.142774705912979\\
57.5	0.14424	0.176843009365159\\
57.5	0.1479	0.21196746129371\\
57.5	0.15156	0.248148061698631\\
57.5	0.15522	0.285384810579926\\
57.5	0.15888	0.323677707937593\\
57.5	0.16254	0.36302675377163\\
57.5	0.1662	0.403431948082039\\
57.5	0.16986	0.444893290868819\\
57.5	0.17352	0.487410782131973\\
57.5	0.17718	0.530984421871497\\
57.5	0.18084	0.575614210087395\\
57.5	0.1845	0.621300146779664\\
57.5	0.18816	0.668042231948304\\
57.5	0.19182	0.715840465593317\\
57.5	0.19548	0.764694847714702\\
57.5	0.19914	0.814605378312456\\
57.5	0.2028	0.865572057386585\\
57.5	0.20646	0.917594884937085\\
57.5	0.21012	0.970673860963958\\
57.5	0.21378	1.0248089854672\\
57.5	0.21744	1.08000025844681\\
57.5	0.2211	1.1362476799028\\
57.5	0.22476	1.19355124983516\\
57.5	0.22842	1.25191096824389\\
57.5	0.23208	1.31132683512899\\
57.5	0.23574	1.37179885049047\\
57.5	0.2394	1.43332701432831\\
57.5	0.24306	1.49591132664253\\
57.5	0.24672	1.55955178743312\\
57.5	0.25038	1.62424839670008\\
57.5	0.25404	1.69000115444341\\
57.5	0.2577	1.75681006066312\\
57.5	0.26136	1.8246751153592\\
57.5	0.26502	1.89359631853165\\
57.5	0.26868	1.96357367018047\\
57.5	0.27234	2.03460717030566\\
57.5	0.276	2.10669681890722\\
57.875	0.093	-0.207287860833774\\
57.875	0.09666	-0.188870979188999\\
57.875	0.10032	-0.169397949067852\\
57.875	0.10398	-0.148868770470335\\
57.875	0.10764	-0.127283443396446\\
57.875	0.1113	-0.104641967846182\\
57.875	0.11496	-0.08094434381955\\
57.875	0.11862	-0.056190571316544\\
57.875	0.12228	-0.030380650337168\\
57.875	0.12594	-0.00351458088141832\\
57.875	0.1296	0.024407637050703\\
57.875	0.13326	0.0533860034591945\\
57.875	0.13692	0.0834205183440596\\
57.875	0.14058	0.114511181705295\\
57.875	0.14424	0.146657993542904\\
57.875	0.1479	0.179860953856886\\
57.875	0.15156	0.214120062647237\\
57.875	0.15522	0.249435319913961\\
57.875	0.15888	0.285806725657057\\
57.875	0.16254	0.323234279876524\\
57.875	0.1662	0.361717982572363\\
57.875	0.16986	0.401257833744573\\
57.875	0.17352	0.441853833393156\\
57.875	0.17718	0.48350598151811\\
57.875	0.18084	0.526214278119437\\
57.875	0.1845	0.569978723197137\\
57.875	0.18816	0.614799316751205\\
57.875	0.19182	0.660676058781649\\
57.875	0.19548	0.707608949288463\\
57.875	0.19914	0.755597988271646\\
57.875	0.2028	0.804643175731204\\
57.875	0.20646	0.854744511667134\\
57.875	0.21012	0.905901996079436\\
57.875	0.21378	0.958115628968109\\
57.875	0.21744	1.01138541033315\\
57.875	0.2211	1.06571134017457\\
57.875	0.22476	1.12109341849236\\
57.875	0.22842	1.17753164528652\\
57.875	0.23208	1.23502602055705\\
57.875	0.23574	1.29357654430396\\
57.875	0.2394	1.35318321652723\\
57.875	0.24306	1.41384603722688\\
57.875	0.24672	1.4755650064029\\
57.875	0.25038	1.53834012405529\\
57.875	0.25404	1.60217139018405\\
57.875	0.2577	1.66705880478919\\
57.875	0.26136	1.73300236787069\\
57.875	0.26502	1.80000207942857\\
57.875	0.26868	1.86805793946282\\
57.875	0.27234	1.93716994797345\\
57.875	0.276	2.00733810496044\\
58.25	0.093	-0.209794620932834\\
58.25	0.09666	-0.193299230902629\\
58.25	0.10032	-0.175747692396053\\
58.25	0.10398	-0.157140005413105\\
58.25	0.10764	-0.137476169953787\\
58.25	0.1113	-0.116756186018094\\
58.25	0.11496	-0.0949800536060318\\
58.25	0.11862	-0.0721477727175948\\
58.25	0.12228	-0.0482593433527894\\
58.25	0.12594	-0.0233147655116104\\
58.25	0.1296	0.00268596080594208\\
58.25	0.13326	0.0297428355998629\\
58.25	0.13692	0.0578558588701573\\
58.25	0.14058	0.0870250306168221\\
58.25	0.14424	0.117250350839862\\
58.25	0.1479	0.148531819539272\\
58.25	0.15156	0.180869436715052\\
58.25	0.15522	0.214263202367207\\
58.25	0.15888	0.248713116495733\\
58.25	0.16254	0.284219179100628\\
58.25	0.1662	0.320781390181897\\
58.25	0.16986	0.358399749739538\\
58.25	0.17352	0.397074257773551\\
58.25	0.17718	0.436804914283934\\
58.25	0.18084	0.477591719270692\\
58.25	0.1845	0.51943467273382\\
58.25	0.18816	0.562333774673319\\
58.25	0.19182	0.60628902508919\\
58.25	0.19548	0.651300423981436\\
58.25	0.19914	0.697367971350051\\
58.25	0.2028	0.744491667195039\\
58.25	0.20646	0.792671511516395\\
58.25	0.21012	0.841907504314129\\
58.25	0.21378	0.892199645588232\\
58.25	0.21744	0.943547935338703\\
58.25	0.2211	0.995952373565549\\
58.25	0.22476	1.04941296026877\\
58.25	0.22842	1.10392969544836\\
58.25	0.23208	1.15950257910432\\
58.25	0.23574	1.21613161123665\\
58.25	0.2394	1.27381679184536\\
58.25	0.24306	1.33255812093044\\
58.25	0.24672	1.39235559849189\\
58.25	0.25038	1.45320922452971\\
58.25	0.25404	1.5151189990439\\
58.25	0.2577	1.57808492203447\\
58.25	0.26136	1.6421069935014\\
58.25	0.26502	1.70718521344471\\
58.25	0.26868	1.77331958186439\\
58.25	0.27234	1.84051009876044\\
58.25	0.276	1.90875676413287\\
58.625	0.093	-0.211524007912687\\
58.625	0.09666	-0.196950109497054\\
58.625	0.10032	-0.181320062605044\\
58.625	0.10398	-0.164633867236669\\
58.625	0.10764	-0.14689152339192\\
58.625	0.1113	-0.128093031070799\\
58.625	0.11496	-0.108238390273306\\
58.625	0.11862	-0.0873276009994395\\
58.625	0.12228	-0.0653606632492048\\
58.625	0.12594	-0.0423375770225947\\
58.625	0.1296	-0.0182583423196147\\
58.625	0.13326	0.00687704085973728\\
58.625	0.13692	0.033068572515461\\
58.625	0.14058	0.0603162526475551\\
58.625	0.14424	0.0886200812560244\\
58.625	0.1479	0.117980058340864\\
58.625	0.15156	0.148396183902075\\
58.625	0.15522	0.179868457939659\\
58.625	0.15888	0.212396880453614\\
58.625	0.16254	0.245981451443939\\
58.625	0.1662	0.280622170910639\\
58.625	0.16986	0.316319038853708\\
58.625	0.17352	0.353072055273151\\
58.625	0.17718	0.390881220168964\\
58.625	0.18084	0.429746533541151\\
58.625	0.1845	0.469667995389708\\
58.625	0.18816	0.510645605714638\\
58.625	0.19182	0.552679364515939\\
58.625	0.19548	0.595769271793614\\
58.625	0.19914	0.639915327547659\\
58.625	0.2028	0.685117531778076\\
58.625	0.20646	0.731375884484865\\
58.625	0.21012	0.778690385668025\\
58.625	0.21378	0.827061035327557\\
58.625	0.21744	0.876487833463461\\
58.625	0.2211	0.926970780075737\\
58.625	0.22476	0.978509875164384\\
58.625	0.22842	1.0311051187294\\
58.625	0.23208	1.08475651077079\\
58.625	0.23574	1.13946405128856\\
58.625	0.2394	1.19522774028269\\
58.625	0.24306	1.2520475777532\\
58.625	0.24672	1.30992356370008\\
58.625	0.25038	1.36885569812333\\
58.625	0.25404	1.42884398102295\\
58.625	0.2577	1.48988841239895\\
58.625	0.26136	1.55198899225132\\
58.625	0.26502	1.61514572058005\\
58.625	0.26868	1.67935859738516\\
58.625	0.27234	1.74462762266664\\
58.625	0.276	1.8109527964245\\
59	0.093	-0.212476021773331\\
59	0.09666	-0.199823614972266\\
59	0.10032	-0.186115059694831\\
59	0.10398	-0.171350355941025\\
59	0.10764	-0.155529503710846\\
59	0.1113	-0.138652503004294\\
59	0.11496	-0.120719353821372\\
59	0.11862	-0.101730056162078\\
59	0.12228	-0.0816846100264124\\
59	0.12594	-0.0605830154143729\\
59	0.1296	-0.0384252723259618\\
59	0.13326	-0.0152113807611823\\
59	0.13692	0.00905865927997251\\
59	0.14058	0.034384847797496\\
59	0.14424	0.0607671847913945\\
59	0.1479	0.0882056702616648\\
59	0.15156	0.116700304208305\\
59	0.15522	0.146251086631317\\
59	0.15888	0.176858017530704\\
59	0.16254	0.208521096906459\\
59	0.1662	0.241240324758587\\
59	0.16986	0.275015701087087\\
59	0.17352	0.30984722589196\\
59	0.17718	0.345734899173201\\
59	0.18084	0.382678720930818\\
59	0.1845	0.420678691164807\\
59	0.18816	0.459734809875164\\
59	0.19182	0.499847077061896\\
59	0.19548	0.541015492725\\
59	0.19914	0.583240056864474\\
59	0.2028	0.626520769480321\\
59	0.20646	0.670857630572539\\
59	0.21012	0.716250640141132\\
59	0.21378	0.762699798186093\\
59	0.21744	0.810205104707423\\
59	0.2211	0.858766559705128\\
59	0.22476	0.908384163179208\\
59	0.22842	0.959057915129657\\
59	0.23208	1.01078781555648\\
59	0.23574	1.06357386445967\\
59	0.2394	1.11741606183924\\
59	0.24306	1.17231440769517\\
59	0.24672	1.22826890202748\\
59	0.25038	1.28527954483616\\
59	0.25404	1.34334633612121\\
59	0.2577	1.40246927588264\\
59	0.26136	1.46264836412043\\
59	0.26502	1.5238836008346\\
59	0.26868	1.58617498602514\\
59	0.27234	1.64952251969205\\
59	0.276	1.71392620183533\\
59.375	0.093	-0.212650662514767\\
59.375	0.09666	-0.201919747328275\\
59.375	0.10032	-0.190132683665409\\
59.375	0.10398	-0.177289471526173\\
59.375	0.10764	-0.163390110910565\\
59.375	0.1113	-0.148434601818582\\
59.375	0.11496	-0.132422944250232\\
59.375	0.11862	-0.115355138205507\\
59.375	0.12228	-0.0972311836844122\\
59.375	0.12594	-0.0780510806869434\\
59.375	0.1296	-0.057814829213103\\
59.375	0.13326	-0.0365224292628924\\
59.375	0.13692	-0.0141738808363082\\
59.375	0.14058	0.00923081606664455\\
59.375	0.14424	0.0336916614459725\\
59.375	0.1479	0.0592086553016721\\
59.375	0.15156	0.0857817976337418\\
59.375	0.15522	0.113411088442185\\
59.375	0.15888	0.142096527727001\\
59.375	0.16254	0.171838115488184\\
59.375	0.1662	0.202635851725743\\
59.375	0.16986	0.234489736439672\\
59.375	0.17352	0.267399769629974\\
59.375	0.17718	0.301365951296647\\
59.375	0.18084	0.336388281439693\\
59.375	0.1845	0.372466760059111\\
59.375	0.18816	0.409601387154897\\
59.375	0.19182	0.447792162727059\\
59.375	0.19548	0.487039086775592\\
59.375	0.19914	0.527342159300496\\
59.375	0.2028	0.568701380301772\\
59.375	0.20646	0.611116749779419\\
59.375	0.21012	0.654588267733441\\
59.375	0.21378	0.699115934163832\\
59.375	0.21744	0.744699749070594\\
59.375	0.2211	0.791339712453729\\
59.375	0.22476	0.839035824313239\\
59.375	0.22842	0.887788084649117\\
59.375	0.23208	0.937596493461367\\
59.375	0.23574	0.988461050749989\\
59.375	0.2394	1.04038175651498\\
59.375	0.24306	1.09335861075635\\
59.375	0.24672	1.14739161347409\\
59.375	0.25038	1.2024807646682\\
59.375	0.25404	1.25862606433868\\
59.375	0.2577	1.31582751248554\\
59.375	0.26136	1.37408510910876\\
59.375	0.26502	1.43339885420836\\
59.375	0.26868	1.49376874778432\\
59.375	0.27234	1.55519478983667\\
59.375	0.276	1.61767698036538\\
59.75	0.093	-0.212047930136994\\
59.75	0.09666	-0.203238506565073\\
59.75	0.10032	-0.193372934516779\\
59.75	0.10398	-0.182451213992112\\
59.75	0.10764	-0.170473344991073\\
59.75	0.1113	-0.157439327513662\\
59.75	0.11496	-0.143349161559881\\
59.75	0.11862	-0.128202847129727\\
59.75	0.12228	-0.112000384223202\\
59.75	0.12594	-0.0947417728403044\\
59.75	0.1296	-0.0764270129810346\\
59.75	0.13326	-0.0570561046453929\\
59.75	0.13692	-0.0366290478333794\\
59.75	0.14058	-0.0151458425449973\\
59.75	0.14424	0.00739351121976173\\
59.75	0.1479	0.0309890134608907\\
59.75	0.15156	0.0556406641783898\\
59.75	0.15522	0.0813484633722624\\
59.75	0.15888	0.108112411042507\\
59.75	0.16254	0.135932507189122\\
59.75	0.1662	0.16480875181211\\
59.75	0.16986	0.194741144911468\\
59.75	0.17352	0.2257296864872\\
59.75	0.17718	0.257774376539302\\
59.75	0.18084	0.290875215067777\\
59.75	0.1845	0.325032202072624\\
59.75	0.18816	0.360245337553842\\
59.75	0.19182	0.396514621511433\\
59.75	0.19548	0.433840053945396\\
59.75	0.19914	0.472221634855729\\
59.75	0.2028	0.511659364242437\\
59.75	0.20646	0.552153242105514\\
59.75	0.21012	0.593703268444965\\
59.75	0.21378	0.636309443260785\\
59.75	0.21744	0.679971766552977\\
59.75	0.2211	0.724690238321541\\
59.75	0.22476	0.77046485856648\\
59.75	0.22842	0.817295627287788\\
59.75	0.23208	0.865182544485467\\
59.75	0.23574	0.914125610159519\\
59.75	0.2394	0.964124824309942\\
59.75	0.24306	1.01518018693674\\
59.75	0.24672	1.06729169803991\\
59.75	0.25038	1.12045935761945\\
59.75	0.25404	1.17468316567536\\
59.75	0.2577	1.22996312220764\\
59.75	0.26136	1.2862992272163\\
59.75	0.26502	1.34369148070132\\
59.75	0.26868	1.40213988266272\\
59.75	0.27234	1.46164443310049\\
59.75	0.276	1.52220513201464\\
60.125	0.093	-0.210667824640015\\
60.125	0.09666	-0.203779892682662\\
60.125	0.10032	-0.195835812248937\\
60.125	0.10398	-0.186835583338841\\
60.125	0.10764	-0.176779205952375\\
60.125	0.1113	-0.165666680089534\\
60.125	0.11496	-0.153498005750322\\
60.125	0.11862	-0.140273182934739\\
60.125	0.12228	-0.125992211642785\\
60.125	0.12594	-0.110655091874456\\
60.125	0.1296	-0.0942618236297567\\
60.125	0.13326	-0.0768124069086874\\
60.125	0.13692	-0.0583068417112428\\
60.125	0.14058	-0.0387451280374296\\
60.125	0.14424	-0.018127265887243\\
60.125	0.1479	0.00354674473931527\\
60.125	0.15156	0.0262769038422437\\
60.125	0.15522	0.0500632114215475\\
60.125	0.15888	0.0749056674772216\\
60.125	0.16254	0.100804272009265\\
60.125	0.1662	0.127759025017683\\
60.125	0.16986	0.155769926502472\\
60.125	0.17352	0.184836976463633\\
60.125	0.17718	0.214960174901164\\
60.125	0.18084	0.246139521815069\\
60.125	0.1845	0.278375017205348\\
60.125	0.18816	0.311666661071995\\
60.125	0.19182	0.346014453415015\\
60.125	0.19548	0.381418394234407\\
60.125	0.19914	0.417878483530169\\
60.125	0.2028	0.455394721302307\\
60.125	0.20646	0.493967107550813\\
60.125	0.21012	0.533595642275694\\
60.125	0.21378	0.574280325476943\\
60.125	0.21744	0.616021157154565\\
60.125	0.2211	0.658818137308561\\
60.125	0.22476	0.70267126593893\\
60.125	0.22842	0.747580543045666\\
60.125	0.23208	0.793545968628775\\
60.125	0.23574	0.840567542688256\\
60.125	0.2394	0.888645265224112\\
60.125	0.24306	0.937779136236339\\
60.125	0.24672	0.987969155724935\\
60.125	0.25038	1.0392153236899\\
60.125	0.25404	1.09151764013125\\
60.125	0.2577	1.14487610504896\\
60.125	0.26136	1.19929071844304\\
60.125	0.26502	1.2547614803135\\
60.125	0.26868	1.31128839066033\\
60.125	0.27234	1.36887144948353\\
60.125	0.276	1.4275106567831\\
60.5	0.093	-0.208510346023826\\
60.5	0.09666	-0.203543905681044\\
60.5	0.10032	-0.19752131686189\\
60.5	0.10398	-0.190442579566362\\
60.5	0.10764	-0.182307693794467\\
60.5	0.1113	-0.173116659546195\\
60.5	0.11496	-0.162869476821556\\
60.5	0.11862	-0.151566145620541\\
60.5	0.12228	-0.139206665943158\\
60.5	0.12594	-0.125791037789399\\
60.5	0.1296	-0.111319261159271\\
60.5	0.13326	-0.0957913360527706\\
60.5	0.13692	-0.0792072624698984\\
60.5	0.14058	-0.0615670404106541\\
60.5	0.14424	-0.0428706698750382\\
60.5	0.1479	-0.0231181508630488\\
60.5	0.15156	-0.00230948337469106\\
60.5	0.15522	0.0195553325900421\\
60.5	0.15888	0.0424762970311455\\
60.5	0.16254	0.0664534099486187\\
60.5	0.1662	0.0914866713424674\\
60.5	0.16986	0.117576081212684\\
60.5	0.17352	0.144721639559276\\
60.5	0.17718	0.172923346382237\\
60.5	0.18084	0.202181201681573\\
60.5	0.1845	0.232495205457279\\
60.5	0.18816	0.263865357709355\\
60.5	0.19182	0.296291658437806\\
60.5	0.19548	0.329774107642628\\
60.5	0.19914	0.364312705323819\\
60.5	0.2028	0.399907451481387\\
60.5	0.20646	0.436558346115322\\
60.5	0.21012	0.474265389225636\\
60.5	0.21378	0.513028580812314\\
60.5	0.21744	0.552847920875365\\
60.5	0.2211	0.593723409414787\\
60.5	0.22476	0.635655046430589\\
60.5	0.22842	0.678642831922755\\
60.5	0.23208	0.722686765891293\\
60.5	0.23574	0.767786848336207\\
60.5	0.2394	0.813943079257488\\
60.5	0.24306	0.861155458655145\\
60.5	0.24672	0.90942398652917\\
60.5	0.25038	0.958748662879571\\
60.5	0.25404	1.00912948770634\\
60.5	0.2577	1.06056646100948\\
60.5	0.26136	1.113059582789\\
60.5	0.26502	1.16660885304488\\
60.5	0.26868	1.22121427177714\\
60.5	0.27234	1.27687583898577\\
60.5	0.276	1.33359355467078\\
60.875	0.093	-0.205575494288422\\
60.875	0.09666	-0.202530545560211\\
60.875	0.10032	-0.198429448355626\\
60.875	0.10398	-0.193272202674671\\
60.875	0.10764	-0.187058808517345\\
60.875	0.1113	-0.179789265883645\\
60.875	0.11496	-0.171463574773574\\
60.875	0.11862	-0.16208173518713\\
60.875	0.12228	-0.151643747124316\\
60.875	0.12594	-0.14014961058513\\
60.875	0.1296	-0.12759932556957\\
60.875	0.13326	-0.113992892077641\\
60.875	0.13692	-0.0993303101093391\\
60.875	0.14058	-0.0836115796646655\\
60.875	0.14424	-0.0668367007436202\\
60.875	0.1479	-0.0490056733462015\\
60.875	0.15156	-0.0301184974724127\\
60.875	0.15522	-0.010175173122252\\
60.875	0.15888	0.0108242997042826\\
60.875	0.16254	0.0328799210071868\\
60.875	0.1662	0.055991690786463\\
60.875	0.16986	0.0801596090421111\\
60.875	0.17352	0.105383675774132\\
60.875	0.17718	0.131663890982522\\
60.875	0.18084	0.159000254667288\\
60.875	0.1845	0.187392766828425\\
60.875	0.18816	0.216841427465932\\
60.875	0.19182	0.247346236579811\\
60.875	0.19548	0.278907194170062\\
60.875	0.19914	0.311524300236686\\
60.875	0.2028	0.345197554779679\\
60.875	0.20646	0.379926957799047\\
60.875	0.21012	0.415712509294787\\
60.875	0.21378	0.452554209266895\\
60.875	0.21744	0.490452057715378\\
60.875	0.2211	0.52940605464023\\
60.875	0.22476	0.569416200041457\\
60.875	0.22842	0.610482493919056\\
60.875	0.23208	0.652604936273023\\
60.875	0.23574	0.695783527103367\\
60.875	0.2394	0.740018266410078\\
60.875	0.24306	0.785309154193164\\
60.875	0.24672	0.831656190452622\\
60.875	0.25038	0.879059375188449\\
60.875	0.25404	0.92751870840065\\
60.875	0.2577	0.977034190089224\\
60.875	0.26136	1.02760582025417\\
60.875	0.26502	1.07923359889548\\
60.875	0.26868	1.13191752601317\\
60.875	0.27234	1.18565760160723\\
60.875	0.276	1.24045382567766\\
61.25	0.093	-0.20186326943382\\
61.25	0.09666	-0.200739812320177\\
61.25	0.10032	-0.198560206730163\\
61.25	0.10398	-0.195324452663778\\
61.25	0.10764	-0.191032550121022\\
61.25	0.1113	-0.185684499101892\\
61.25	0.11496	-0.179280299606392\\
61.25	0.11862	-0.171819951634519\\
61.25	0.12228	-0.163303455186275\\
61.25	0.12594	-0.15373081026166\\
61.25	0.1296	-0.143102016860671\\
61.25	0.13326	-0.131417074983312\\
61.25	0.13692	-0.118675984629579\\
61.25	0.14058	-0.104878745799476\\
61.25	0.14424	-0.0900253584930016\\
61.25	0.1479	-0.0741158227101536\\
61.25	0.15156	-0.0571501384509354\\
61.25	0.15522	-0.0391283057153436\\
61.25	0.15888	-0.0200503245033797\\
61.25	0.16254	8.38051849521015e-05\\
61.25	0.1662	0.0212740833496594\\
61.25	0.16986	0.0435205099907368\\
61.25	0.17352	0.0668230851081875\\
61.25	0.17718	0.0911818087020084\\
61.25	0.18084	0.116596680772203\\
61.25	0.1845	0.143067701318769\\
61.25	0.18816	0.170594870341706\\
61.25	0.19182	0.199178187841014\\
61.25	0.19548	0.228817653816694\\
61.25	0.19914	0.259513268268748\\
61.25	0.2028	0.291265031197174\\
61.25	0.20646	0.324072942601968\\
61.25	0.21012	0.357937002483141\\
61.25	0.21378	0.392857210840678\\
61.25	0.21744	0.428833567674587\\
61.25	0.2211	0.465866072984872\\
61.25	0.22476	0.503954726771528\\
61.25	0.22842	0.543099529034556\\
61.25	0.23208	0.583300479773953\\
61.25	0.23574	0.624557578989726\\
61.25	0.2394	0.666870826681866\\
61.25	0.24306	0.710240222850385\\
61.25	0.24672	0.754665767495269\\
61.25	0.25038	0.800147460616528\\
61.25	0.25404	0.846685302214156\\
61.25	0.2577	0.894279292288163\\
61.25	0.26136	0.942929430838534\\
61.25	0.26502	0.99263571786528\\
61.25	0.26868	1.04339815336839\\
61.25	0.27234	1.09521673734789\\
61.25	0.276	1.14809146980375\\
61.625	0.093	-0.19737367146\\
61.625	0.09666	-0.19817170596093\\
61.625	0.10032	-0.197913591985488\\
61.625	0.10398	-0.196599329533673\\
61.625	0.10764	-0.194228918605487\\
61.625	0.1113	-0.190802359200926\\
61.625	0.11496	-0.186319651319997\\
61.625	0.11862	-0.180780794962694\\
61.625	0.12228	-0.174185790129021\\
61.625	0.12594	-0.166534636818975\\
61.625	0.1296	-0.157827335032556\\
61.625	0.13326	-0.148063884769768\\
61.625	0.13692	-0.137244286030606\\
61.625	0.14058	-0.125368538815074\\
61.625	0.14424	-0.112436643123168\\
61.625	0.1479	-0.0984485989548908\\
61.625	0.15156	-0.0834044063102433\\
61.625	0.15522	-0.0673040651892222\\
61.625	0.15888	-0.0501475755918289\\
61.625	0.16254	-0.031934937518066\\
61.625	0.1662	-0.0126661509679293\\
61.625	0.16986	0.00765878405857734\\
61.625	0.17352	0.0290398675614574\\
61.625	0.17718	0.0514770995407077\\
61.625	0.18084	0.0749704799963316\\
61.625	0.1845	0.0995200089283272\\
61.625	0.18816	0.125125686336692\\
61.625	0.19182	0.151787512221433\\
61.625	0.19548	0.179505486582542\\
61.625	0.19914	0.208279609420025\\
61.625	0.2028	0.23810988073388\\
61.625	0.20646	0.268996300524104\\
61.625	0.21012	0.300938868790706\\
61.625	0.21378	0.333937585533672\\
61.625	0.21744	0.367992450753014\\
61.625	0.2211	0.403103464448728\\
61.625	0.22476	0.439270626620814\\
61.625	0.22842	0.476493937269272\\
61.625	0.23208	0.514773396394097\\
61.625	0.23574	0.5541090039953\\
61.625	0.2394	0.594500760072873\\
61.625	0.24306	0.635948664626818\\
61.625	0.24672	0.678452717657134\\
61.625	0.25038	0.72201291916382\\
61.625	0.25404	0.76662926914688\\
61.625	0.2577	0.812301767606316\\
61.625	0.26136	0.859030414542116\\
61.625	0.26502	0.906815209954292\\
61.625	0.26868	0.955656153842836\\
61.625	0.27234	1.00555324620776\\
61.625	0.276	1.05650648704905\\
62	0.093	-0.192106700366977\\
62	0.09666	-0.194826226482477\\
62	0.10032	-0.196489604121602\\
62	0.10398	-0.197096833284359\\
62	0.10764	-0.196647913970744\\
62	0.1113	-0.195142846180754\\
62	0.11496	-0.192581629914396\\
62	0.11862	-0.188964265171662\\
62	0.12228	-0.184290751952559\\
62	0.12594	-0.178561090257084\\
62	0.1296	-0.171775280085236\\
62	0.13326	-0.163933321437017\\
62	0.13692	-0.155035214312425\\
62	0.14058	-0.145080958711464\\
62	0.14424	-0.134070554634129\\
62	0.1479	-0.122004002080422\\
62	0.15156	-0.108881301050343\\
62	0.15522	-0.0947024515438929\\
62	0.15888	-0.0794674535610703\\
62	0.16254	-0.063176307101878\\
62	0.1662	-0.0458290121663121\\
62	0.16986	-0.0274255687543743\\
62	0.17352	-0.00796597686606493\\
62	0.17718	0.0125497634986147\\
62	0.18084	0.0341216523396679\\
62	0.1845	0.0567496896570929\\
62	0.18816	0.0804338754508884\\
62	0.19182	0.105174209721059\\
62	0.19548	0.130970692467601\\
62	0.19914	0.15782332369051\\
62	0.2028	0.185732103389795\\
62	0.20646	0.214697031565451\\
62	0.21012	0.244718108217478\\
62	0.21378	0.275795333345878\\
62	0.21744	0.307928706950649\\
62	0.2211	0.341118229031789\\
62	0.22476	0.375363899589308\\
62	0.22842	0.410665718623195\\
62	0.23208	0.44702368613345\\
62	0.23574	0.484437802120081\\
62	0.2394	0.522908066583084\\
62	0.24306	0.562434479522458\\
62	0.24672	0.603017040938204\\
62	0.25038	0.644655750830322\\
62	0.25404	0.687350609198812\\
62	0.2577	0.731101616043674\\
62	0.26136	0.775908771364907\\
62	0.26502	0.821772075162508\\
62	0.26868	0.868691527436485\\
62	0.27234	0.916667128186838\\
62	0.276	0.965698877413555\\
62.375	0.093	-0.186062356154745\\
62.375	0.09666	-0.190703373884815\\
62.375	0.10032	-0.194288243138512\\
62.375	0.10398	-0.196816963915838\\
62.375	0.10764	-0.198289536216792\\
62.375	0.1113	-0.198705960041374\\
62.375	0.11496	-0.198066235389585\\
62.375	0.11862	-0.196370362261423\\
62.375	0.12228	-0.19361834065689\\
62.375	0.12594	-0.189810170575985\\
62.375	0.1296	-0.184945852018706\\
62.375	0.13326	-0.179025384985059\\
62.375	0.13692	-0.172048769475037\\
62.375	0.14058	-0.164016005488646\\
62.375	0.14424	-0.154927093025881\\
62.375	0.1479	-0.144782032086744\\
62.375	0.15156	-0.133580822671237\\
62.375	0.15522	-0.121323464779356\\
62.375	0.15888	-0.108009958411104\\
62.375	0.16254	-0.0936403035664823\\
62.375	0.1662	-0.0782145002454853\\
62.375	0.16986	-0.0617325484481199\\
62.375	0.17352	-0.0441944481743795\\
62.375	0.17718	-0.0256001994242705\\
62.375	0.18084	-0.00594980219778618\\
62.375	0.1845	0.0147567435050682\\
62.375	0.18816	0.036519437684293\\
62.375	0.19182	0.0593382803398927\\
62.375	0.19548	0.0832132714718643\\
62.375	0.19914	0.108144411080203\\
62.375	0.2028	0.134131699164916\\
62.375	0.20646	0.161175135726002\\
62.375	0.21012	0.189274720763462\\
62.375	0.21378	0.218430454277291\\
62.375	0.21744	0.248642336267489\\
62.375	0.2211	0.279910366734061\\
62.375	0.22476	0.312234545677009\\
62.375	0.22842	0.345614873096322\\
62.375	0.23208	0.38005134899201\\
62.375	0.23574	0.415543973364071\\
62.375	0.2394	0.452092746212502\\
62.375	0.24306	0.489697667537309\\
62.375	0.24672	0.528358737338481\\
62.375	0.25038	0.568075955616029\\
62.375	0.25404	0.608849322369948\\
62.375	0.2577	0.650678837600243\\
62.375	0.26136	0.693564501306905\\
62.375	0.26502	0.737506313489936\\
62.375	0.26868	0.782504274149342\\
62.375	0.27234	0.828558383285124\\
62.375	0.276	0.875668640897274\\
62.75	0.093	-0.179240638823306\\
62.75	0.09666	-0.185803148167946\\
62.75	0.10032	-0.191309509036215\\
62.75	0.10398	-0.195759721428111\\
62.75	0.10764	-0.199153785343636\\
62.75	0.1113	-0.201491700782787\\
62.75	0.11496	-0.202773467745568\\
62.75	0.11862	-0.202999086231977\\
62.75	0.12228	-0.202168556242014\\
62.75	0.12594	-0.200281877775678\\
62.75	0.1296	-0.197339050832972\\
62.75	0.13326	-0.193340075413894\\
62.75	0.13692	-0.188284951518444\\
62.75	0.14058	-0.182173679146622\\
62.75	0.14424	-0.175006258298428\\
62.75	0.1479	-0.166782688973861\\
62.75	0.15156	-0.157502971172924\\
62.75	0.15522	-0.147167104895615\\
62.75	0.15888	-0.135775090141932\\
62.75	0.16254	-0.123326926911881\\
62.75	0.1662	-0.109822615205454\\
62.75	0.16986	-0.0952621550226596\\
62.75	0.17352	-0.0796455463634897\\
62.75	0.17718	-0.0629727892279497\\
62.75	0.18084	-0.0452438836160378\\
62.75	0.1845	-0.0264588295277524\\
62.75	0.18816	-0.0066176269630982\\
62.75	0.19182	0.0142797240779309\\
62.75	0.19548	0.0362332235953318\\
62.75	0.19914	0.0592428715891029\\
62.75	0.2028	0.0833086680592461\\
62.75	0.20646	0.108430613005761\\
62.75	0.21012	0.134608706428647\\
62.75	0.21378	0.161842948327905\\
62.75	0.21744	0.190133338703536\\
62.75	0.2211	0.219479877555538\\
62.75	0.22476	0.249882564883915\\
62.75	0.22842	0.281341400688657\\
62.75	0.23208	0.313856384969774\\
62.75	0.23574	0.347427517727264\\
62.75	0.2394	0.382054798961125\\
62.75	0.24306	0.417738228671362\\
62.75	0.24672	0.454477806857966\\
62.75	0.25038	0.492273533520943\\
62.75	0.25404	0.531125408660292\\
62.75	0.2577	0.571033432276016\\
62.75	0.26136	0.611997604368104\\
62.75	0.26502	0.654017924936568\\
62.75	0.26868	0.697094393981403\\
62.75	0.27234	0.741227011502615\\
62.75	0.276	0.786415777500194\\
63.125	0.093	-0.171641548372656\\
63.125	0.09666	-0.180125549331865\\
63.125	0.10032	-0.187553401814702\\
63.125	0.10398	-0.193925105821169\\
63.125	0.10764	-0.199240661351266\\
63.125	0.1113	-0.203500068404986\\
63.125	0.11496	-0.20670332698234\\
63.125	0.11862	-0.208850437083318\\
63.125	0.12228	-0.209941398707926\\
63.125	0.12594	-0.20997621185616\\
63.125	0.1296	-0.208954876528023\\
63.125	0.13326	-0.206877392723515\\
63.125	0.13692	-0.203743760442636\\
63.125	0.14058	-0.199553979685385\\
63.125	0.14424	-0.19430805045176\\
63.125	0.1479	-0.188005972741765\\
63.125	0.15156	-0.180647746555399\\
63.125	0.15522	-0.172233371892658\\
63.125	0.15888	-0.162762848753546\\
63.125	0.16254	-0.152236177138066\\
63.125	0.1662	-0.14065335704621\\
63.125	0.16986	-0.128014388477984\\
63.125	0.17352	-0.114319271433385\\
63.125	0.17718	-0.0995680059124158\\
63.125	0.18084	-0.0837605919150728\\
63.125	0.1845	-0.066897029441358\\
63.125	0.18816	-0.0489773184912745\\
63.125	0.19182	-0.0300014590648161\\
63.125	0.19548	-0.00996945116198589\\
63.125	0.19914	0.0111187052172146\\
63.125	0.2028	0.0332630100727871\\
63.125	0.20646	0.0564634634047312\\
63.125	0.21012	0.0807200652130504\\
63.125	0.21378	0.106032815497738\\
63.125	0.21744	0.132401714258797\\
63.125	0.2211	0.159826761496229\\
63.125	0.22476	0.188307957210032\\
63.125	0.22842	0.217845301400207\\
63.125	0.23208	0.248438794066753\\
63.125	0.23574	0.280088435209673\\
63.125	0.2394	0.312794224828963\\
63.125	0.24306	0.346556162924629\\
63.125	0.24672	0.381374249496663\\
63.125	0.25038	0.417248484545069\\
63.125	0.25404	0.454178868069847\\
63.125	0.2577	0.492165400071\\
63.125	0.26136	0.531208080548522\\
63.125	0.26502	0.571306909502415\\
63.125	0.26868	0.612461886932679\\
63.125	0.27234	0.65467301283932\\
63.125	0.276	0.697940287222329\\
63.5	0.093	-0.163265084802796\\
63.5	0.09666	-0.173670577376576\\
63.5	0.10032	-0.183019921473983\\
63.5	0.10398	-0.191313117095019\\
63.5	0.10764	-0.198550164239685\\
63.5	0.1113	-0.204731062907978\\
63.5	0.11496	-0.2098558130999\\
63.5	0.11862	-0.213924414815449\\
63.5	0.12228	-0.216936868054628\\
63.5	0.12594	-0.218893172817433\\
63.5	0.1296	-0.219793329103866\\
63.5	0.13326	-0.219637336913929\\
63.5	0.13692	-0.218425196247619\\
63.5	0.14058	-0.216156907104938\\
63.5	0.14424	-0.212832469485884\\
63.5	0.1479	-0.208451883390458\\
63.5	0.15156	-0.203015148818662\\
63.5	0.15522	-0.196522265770493\\
63.5	0.15888	-0.188973234245951\\
63.5	0.16254	-0.18036805424504\\
63.5	0.1662	-0.170706725767755\\
63.5	0.16986	-0.159989248814099\\
63.5	0.17352	-0.148215623384071\\
63.5	0.17718	-0.135385849477672\\
63.5	0.18084	-0.121499927094902\\
63.5	0.1845	-0.106557856235758\\
63.5	0.18816	-0.0905596369002413\\
63.5	0.19182	-0.0735052690883535\\
63.5	0.19548	-0.055394752800094\\
63.5	0.19914	-0.0362280880354642\\
63.5	0.2028	-0.0160052747944623\\
63.5	0.20646	0.00527368692291108\\
63.5	0.21012	0.0276087971166596\\
63.5	0.21378	0.0510000557867765\\
63.5	0.21744	0.0754474629332653\\
63.5	0.2211	0.100951018556126\\
63.5	0.22476	0.127510722655362\\
63.5	0.22842	0.155126575230966\\
63.5	0.23208	0.183798576282942\\
63.5	0.23574	0.213526725811295\\
63.5	0.2394	0.244311023816014\\
63.5	0.24306	0.276151470297109\\
63.5	0.24672	0.309048065254573\\
63.5	0.25038	0.343000808688408\\
63.5	0.25404	0.378009700598615\\
63.5	0.2577	0.414074740985198\\
63.5	0.26136	0.451195929848148\\
63.5	0.26502	0.489373267187471\\
63.5	0.26868	0.528606753003165\\
63.5	0.27234	0.568896387295235\\
63.5	0.276	0.610242170063673\\
63.875	0.093	-0.154111248113726\\
63.875	0.09666	-0.166438232302078\\
63.875	0.10032	-0.177709068014059\\
63.875	0.10398	-0.187923755249664\\
63.875	0.10764	-0.1970822940089\\
63.875	0.1113	-0.205184684291763\\
63.875	0.11496	-0.212230926098255\\
63.875	0.11862	-0.218221019428374\\
63.875	0.12228	-0.223154964282123\\
63.875	0.12594	-0.227032760659499\\
63.875	0.1296	-0.229854408560501\\
63.875	0.13326	-0.231619907985135\\
63.875	0.13692	-0.232329258933396\\
63.875	0.14058	-0.231982461405285\\
63.875	0.14424	-0.230579515400802\\
63.875	0.1479	-0.228120420919947\\
63.875	0.15156	-0.22460517796272\\
63.875	0.15522	-0.220033786529121\\
63.875	0.15888	-0.21440624661915\\
63.875	0.16254	-0.207722558232809\\
63.875	0.1662	-0.199982721370095\\
63.875	0.16986	-0.19118673603101\\
63.875	0.17352	-0.181334602215551\\
63.875	0.17718	-0.170426319923723\\
63.875	0.18084	-0.158461889155521\\
63.875	0.1845	-0.145441309910948\\
63.875	0.18816	-0.131364582190002\\
63.875	0.19182	-0.116231705992685\\
63.875	0.19548	-0.100042681318996\\
63.875	0.19914	-0.0827975081689369\\
63.875	0.2028	-0.0644961865425058\\
63.875	0.20646	-0.045138716439703\\
63.875	0.21012	-0.0247250978605216\\
63.875	0.21378	-0.00325533080497542\\
63.875	0.21744	0.0192705847269428\\
63.875	0.2211	0.0428526487352328\\
63.875	0.22476	0.0674908612198981\\
63.875	0.22842	0.0931852221809315\\
63.875	0.23208	0.119935731618337\\
63.875	0.23574	0.147742389532119\\
63.875	0.2394	0.176605195922268\\
63.875	0.24306	0.206524150788792\\
63.875	0.24672	0.237499254131685\\
63.875	0.25038	0.26953050595095\\
63.875	0.25404	0.302617906246589\\
63.875	0.2577	0.336761455018602\\
63.875	0.26136	0.371961152266981\\
63.875	0.26502	0.408216997991733\\
63.875	0.26868	0.445528992192857\\
63.875	0.27234	0.483897134870356\\
63.875	0.276	0.523321426024223\\
64.25	0.093	-0.144180038305451\\
64.25	0.09666	-0.158428514108372\\
64.25	0.10032	-0.171620841434921\\
64.25	0.10398	-0.183757020285098\\
64.25	0.10764	-0.194837050658906\\
64.25	0.1113	-0.204860932556338\\
64.25	0.11496	-0.213828665977401\\
64.25	0.11862	-0.22174025092209\\
64.25	0.12228	-0.22859568739041\\
64.25	0.12594	-0.234394975382354\\
64.25	0.1296	-0.239138114897929\\
64.25	0.13326	-0.242825105937132\\
64.25	0.13692	-0.245455948499963\\
64.25	0.14058	-0.247030642586423\\
64.25	0.14424	-0.247549188196511\\
64.25	0.1479	-0.247011585330224\\
64.25	0.15156	-0.24541783398757\\
64.25	0.15522	-0.24276793416854\\
64.25	0.15888	-0.239061885873139\\
64.25	0.16254	-0.234299689101369\\
64.25	0.1662	-0.228481343853224\\
64.25	0.16986	-0.22160685012871\\
64.25	0.17352	-0.213676207927821\\
64.25	0.17718	-0.204689417250564\\
64.25	0.18084	-0.194646478096931\\
64.25	0.1845	-0.183547390466928\\
64.25	0.18816	-0.171392154360557\\
64.25	0.19182	-0.158180769777807\\
64.25	0.19548	-0.143913236718689\\
64.25	0.19914	-0.1285895551832\\
64.25	0.2028	-0.112209725171336\\
64.25	0.20646	-0.094773746683104\\
64.25	0.21012	-0.0762816197184968\\
64.25	0.21378	-0.0567333442775213\\
64.25	0.21744	-0.0361289203601738\\
64.25	0.2211	-0.0144683479664509\\
64.25	0.22476	0.00824837290364377\\
64.25	0.22842	0.0320212422501065\\
64.25	0.23208	0.0568502600729412\\
64.25	0.23574	0.0827354263721523\\
64.25	0.2394	0.109676741147731\\
64.25	0.24306	0.137674204399684\\
64.25	0.24672	0.166727816128006\\
64.25	0.25038	0.196837576332704\\
64.25	0.25404	0.22800348501377\\
64.25	0.2577	0.260225542171211\\
64.25	0.26136	0.293503747805024\\
64.25	0.26502	0.327838101915205\\
64.25	0.26868	0.363228604501758\\
64.25	0.27234	0.399675255564687\\
64.25	0.276	0.437178055103987\\
64.625	0.093	-0.133471455377967\\
64.625	0.09666	-0.149641422795459\\
64.625	0.10032	-0.164755241736579\\
64.625	0.10398	-0.178812912201327\\
64.625	0.10764	-0.191814434189703\\
64.625	0.1113	-0.203759807701706\\
64.625	0.11496	-0.21464903273734\\
64.625	0.11862	-0.224482109296599\\
64.625	0.12228	-0.23325903737949\\
64.625	0.12594	-0.240979816986005\\
64.625	0.1296	-0.247644448116149\\
64.625	0.13326	-0.253252930769924\\
64.625	0.13692	-0.257805264947324\\
64.625	0.14058	-0.261301450648355\\
64.625	0.14424	-0.263741487873011\\
64.625	0.1479	-0.265125376621297\\
64.625	0.15156	-0.265453116893212\\
64.625	0.15522	-0.264724708688754\\
64.625	0.15888	-0.262940152007923\\
64.625	0.16254	-0.260099446850723\\
64.625	0.1662	-0.256202593217149\\
64.625	0.16986	-0.251249591107205\\
64.625	0.17352	-0.245240440520887\\
64.625	0.17718	-0.238175141458201\\
64.625	0.18084	-0.230053693919137\\
64.625	0.1845	-0.220876097903705\\
64.625	0.18816	-0.210642353411904\\
64.625	0.19182	-0.199352460443724\\
64.625	0.19548	-0.187006418999177\\
64.625	0.19914	-0.173604229078259\\
64.625	0.2028	-0.159145890680966\\
64.625	0.20646	-0.143631403807304\\
64.625	0.21012	-0.127060768457264\\
64.625	0.21378	-0.109433984630859\\
64.625	0.21744	-0.0907510523280826\\
64.625	0.2211	-0.0710119715489339\\
64.625	0.22476	-0.0502167422934063\\
64.625	0.22842	-0.0283653645615143\\
64.625	0.23208	-0.0054578383532502\\
64.625	0.23574	0.0185058363313901\\
64.625	0.2394	0.0435256594923978\\
64.625	0.24306	0.0696016311297845\\
64.625	0.24672	0.0967337512435358\\
64.625	0.25038	0.124922019833659\\
64.625	0.25404	0.154166436900158\\
64.625	0.2577	0.184467002443029\\
64.625	0.26136	0.215823716462267\\
64.625	0.26502	0.248236578957881\\
64.625	0.26868	0.281705589929863\\
64.625	0.27234	0.316230749378221\\
64.625	0.276	0.351812057302951\\
65	0.093	-0.121985499331276\\
65	0.09666	-0.140076958363339\\
65	0.10032	-0.157112268919029\\
65	0.10398	-0.173091430998346\\
65	0.10764	-0.188014444601293\\
65	0.1113	-0.201881309727866\\
65	0.11496	-0.21469202637807\\
65	0.11862	-0.226446594551899\\
65	0.12228	-0.237145014249361\\
65	0.12594	-0.246787285470446\\
65	0.1296	-0.255373408215161\\
65	0.13326	-0.262903382483505\\
65	0.13692	-0.269377208275475\\
65	0.14058	-0.274794885591077\\
65	0.14424	-0.279156414430304\\
65	0.1479	-0.282461794793159\\
65	0.15156	-0.284711026679646\\
65	0.15522	-0.285904110089757\\
65	0.15888	-0.286041045023497\\
65	0.16254	-0.285121831480868\\
65	0.1662	-0.283146469461864\\
65	0.16986	-0.28011495896649\\
65	0.17352	-0.276027299994742\\
65	0.17718	-0.270883492546624\\
65	0.18084	-0.264683536622135\\
65	0.1845	-0.25742743222127\\
65	0.18816	-0.249115179344039\\
65	0.19182	-0.239746777990431\\
65	0.19548	-0.229322228160454\\
65	0.19914	-0.217841529854103\\
65	0.2028	-0.205304683071384\\
65	0.20646	-0.19171168781229\\
65	0.21012	-0.177062544076824\\
65	0.21378	-0.16135725186499\\
65	0.21744	-0.14459581117678\\
65	0.2211	-0.126778222012202\\
65	0.22476	-0.107904484371245\\
65	0.22842	-0.0879745982539237\\
65	0.23208	-0.0669885636602303\\
65	0.23574	-0.0449463805901607\\
65	0.2394	-0.0218480490437201\\
65	0.24306	0.00230643097909233\\
65	0.24672	0.027517059478273\\
65	0.25038	0.0537838364538294\\
65	0.25404	0.0811067619057537\\
65	0.2577	0.109485835834057\\
65	0.26136	0.138921058238725\\
65	0.26502	0.169412429119769\\
65	0.26868	0.20095994847718\\
65	0.27234	0.233563616310967\\
65	0.276	0.267223432621126\\
65.375	0.093	-0.109722170165376\\
65.375	0.09666	-0.129735120812007\\
65.375	0.10032	-0.148691922982266\\
65.375	0.10398	-0.166592576676156\\
65.375	0.10764	-0.183437081893673\\
65.375	0.1113	-0.199225438634815\\
65.375	0.11496	-0.213957646899589\\
65.375	0.11862	-0.22763370668799\\
65.375	0.12228	-0.24025361800002\\
65.375	0.12594	-0.251817380835677\\
65.375	0.1296	-0.262324995194961\\
65.375	0.13326	-0.271776461077876\\
65.375	0.13692	-0.280171778484417\\
65.375	0.14058	-0.28751094741459\\
65.375	0.14424	-0.293793967868388\\
65.375	0.1479	-0.299020839845813\\
65.375	0.15156	-0.303191563346869\\
65.375	0.15522	-0.306306138371551\\
65.375	0.15888	-0.308364564919861\\
65.375	0.16254	-0.309366842991801\\
65.375	0.1662	-0.309312972587368\\
65.375	0.16986	-0.308202953706564\\
65.375	0.17352	-0.306036786349387\\
65.375	0.17718	-0.302814470515842\\
65.375	0.18084	-0.298536006205919\\
65.375	0.1845	-0.293201393419628\\
65.375	0.18816	-0.286810632156965\\
65.375	0.19182	-0.279363722417928\\
65.375	0.19548	-0.270860664202521\\
65.375	0.19914	-0.261301457510741\\
65.375	0.2028	-0.250686102342593\\
65.375	0.20646	-0.239014598698069\\
65.375	0.21012	-0.226286946577174\\
65.375	0.21378	-0.212503145979907\\
65.375	0.21744	-0.197663196906271\\
65.375	0.2211	-0.181767099356261\\
65.375	0.22476	-0.164814853329878\\
65.375	0.22842	-0.146806458827124\\
65.375	0.23208	-0.127741915848001\\
65.375	0.23574	-0.107621224392502\\
65.375	0.2394	-0.086444384460632\\
65.375	0.24306	-0.0642113960523902\\
65.375	0.24672	-0.0409222591677767\\
65.375	0.25038	-0.0165769738067909\\
65.375	0.25404	0.0088244600305627\\
65.375	0.2577	0.0352820423442957\\
65.375	0.26136	0.0627957731343929\\
65.375	0.26502	0.0913656524008655\\
65.375	0.26868	0.120991680143706\\
65.375	0.27234	0.151673856362927\\
65.375	0.276	0.183412181058511\\
65.75	0.093	-0.0966814678802672\\
65.75	0.09666	-0.118615910141467\\
65.75	0.10032	-0.139494203926299\\
65.75	0.10398	-0.159316349234758\\
65.75	0.10764	-0.178082346066846\\
65.75	0.1113	-0.195792194422559\\
65.75	0.11496	-0.212445894301903\\
65.75	0.11862	-0.228043445704874\\
65.75	0.12228	-0.242584848631473\\
65.75	0.12594	-0.256070103081701\\
65.75	0.1296	-0.268499209055556\\
65.75	0.13326	-0.279872166553042\\
65.75	0.13692	-0.290188975574153\\
65.75	0.14058	-0.299449636118895\\
65.75	0.14424	-0.307654148187263\\
65.75	0.1479	-0.314802511779258\\
65.75	0.15156	-0.320894726894884\\
65.75	0.15522	-0.325930793534137\\
65.75	0.15888	-0.329910711697017\\
65.75	0.16254	-0.332834481383528\\
65.75	0.1662	-0.334702102593665\\
65.75	0.16986	-0.335513575327433\\
65.75	0.17352	-0.335268899584825\\
65.75	0.17718	-0.333968075365848\\
65.75	0.18084	-0.3316111026705\\
65.75	0.1845	-0.328197981498776\\
65.75	0.18816	-0.323728711850684\\
65.75	0.19182	-0.318203293726216\\
65.75	0.19548	-0.311621727125381\\
65.75	0.19914	-0.303984012048172\\
65.75	0.2028	-0.29529014849459\\
65.75	0.20646	-0.285540136464641\\
65.75	0.21012	-0.274733975958313\\
65.75	0.21378	-0.26287166697562\\
65.75	0.21744	-0.249953209516552\\
65.75	0.2211	-0.235978603581111\\
65.75	0.22476	-0.220947849169299\\
65.75	0.22842	-0.204860946281116\\
65.75	0.23208	-0.187717894916564\\
65.75	0.23574	-0.169518695075635\\
65.75	0.2394	-0.150263346758336\\
65.75	0.24306	-0.129951849964661\\
65.75	0.24672	-0.108584204694622\\
65.75	0.25038	-0.086160410948207\\
65.75	0.25404	-0.0626804687254205\\
65.75	0.2577	-0.0381443780262618\\
65.75	0.26136	-0.0125521388507317\\
65.75	0.26502	0.0140962488011667\\
65.75	0.26868	0.0418007849294404\\
65.75	0.27234	0.0705614695340899\\
65.75	0.276	0.100378302615104\\
66.125	0.093	-0.0828633924759457\\
66.125	0.09666	-0.106719326351716\\
66.125	0.10032	-0.129519111751119\\
66.125	0.10398	-0.151262748674146\\
66.125	0.10764	-0.171950237120805\\
66.125	0.1113	-0.191581577091089\\
66.125	0.11496	-0.210156768585004\\
66.125	0.11862	-0.227675811602545\\
66.125	0.12228	-0.244138706143715\\
66.125	0.12594	-0.259545452208513\\
66.125	0.1296	-0.273896049796939\\
66.125	0.13326	-0.287190498908994\\
66.125	0.13692	-0.299428799544676\\
66.125	0.14058	-0.310610951703989\\
66.125	0.14424	-0.320736955386927\\
66.125	0.1479	-0.329806810593493\\
66.125	0.15156	-0.33782051732369\\
66.125	0.15522	-0.344778075577511\\
66.125	0.15888	-0.350679485354963\\
66.125	0.16254	-0.355524746656044\\
66.125	0.1662	-0.359313859480752\\
66.125	0.16986	-0.362046823829088\\
66.125	0.17352	-0.363723639701052\\
66.125	0.17718	-0.364344307096645\\
66.125	0.18084	-0.363908826015867\\
66.125	0.1845	-0.362417196458714\\
66.125	0.18816	-0.359869418425192\\
66.125	0.19182	-0.356265491915296\\
66.125	0.19548	-0.351605416929031\\
66.125	0.19914	-0.345889193466392\\
66.125	0.2028	-0.339116821527382\\
66.125	0.20646	-0.331288301111999\\
66.125	0.21012	-0.322403632220245\\
66.125	0.21378	-0.31246281485212\\
66.125	0.21744	-0.301465849007622\\
66.125	0.2211	-0.289412734686756\\
66.125	0.22476	-0.276303471889511\\
66.125	0.22842	-0.262138060615898\\
66.125	0.23208	-0.246916500865913\\
66.125	0.23574	-0.230638792639559\\
66.125	0.2394	-0.213304935936831\\
66.125	0.24306	-0.194914930757727\\
66.125	0.24672	-0.175468777102255\\
66.125	0.25038	-0.15496647497041\\
66.125	0.25404	-0.133408024362198\\
66.125	0.2577	-0.110793425277606\\
66.125	0.26136	-0.0871226777166467\\
66.125	0.26502	-0.062395781679319\\
66.125	0.26868	-0.036612737165616\\
66.125	0.27234	-0.00977354417553711\\
66.125	0.276	0.018121797290906\\
66.5	0.093	-0.0682679439524163\\
66.5	0.09666	-0.0940453694427614\\
66.5	0.10032	-0.118766646456731\\
66.5	0.10398	-0.142431774994331\\
66.5	0.10764	-0.16504075505556\\
66.5	0.1113	-0.186593586640414\\
66.5	0.11496	-0.207090269748898\\
66.5	0.11862	-0.226530804381011\\
66.5	0.12228	-0.244915190536751\\
66.5	0.12594	-0.26224342821612\\
66.5	0.1296	-0.278515517419115\\
66.5	0.13326	-0.293731458145742\\
66.5	0.13692	-0.307891250395993\\
66.5	0.14058	-0.320994894169876\\
66.5	0.14424	-0.333042389467384\\
66.5	0.1479	-0.344033736288522\\
66.5	0.15156	-0.353968934633287\\
66.5	0.15522	-0.362847984501682\\
66.5	0.15888	-0.370670885893702\\
66.5	0.16254	-0.377437638809354\\
66.5	0.1662	-0.383148243248631\\
66.5	0.16986	-0.387802699211539\\
66.5	0.17352	-0.391401006698074\\
66.5	0.17718	-0.393943165708237\\
66.5	0.18084	-0.395429176242027\\
66.5	0.1845	-0.395859038299444\\
66.5	0.18816	-0.395232751880493\\
66.5	0.19182	-0.393550316985168\\
66.5	0.19548	-0.390811733613473\\
66.5	0.19914	-0.387017001765405\\
66.5	0.2028	-0.382166121440965\\
66.5	0.20646	-0.376259092640154\\
66.5	0.21012	-0.369295915362967\\
66.5	0.21378	-0.361276589609412\\
66.5	0.21744	-0.352201115379488\\
66.5	0.2211	-0.34206949267319\\
66.5	0.22476	-0.330881721490515\\
66.5	0.22842	-0.318637801831473\\
66.5	0.23208	-0.305337733696059\\
66.5	0.23574	-0.290981517084272\\
66.5	0.2394	-0.275569151996117\\
66.5	0.24306	-0.259100638431584\\
66.5	0.24672	-0.241575976390683\\
66.5	0.25038	-0.222995165873409\\
66.5	0.25404	-0.203358206879764\\
66.5	0.2577	-0.182665099409743\\
66.5	0.26136	-0.160915843463358\\
66.5	0.26502	-0.138110439040597\\
66.5	0.26868	-0.114248886141465\\
66.5	0.27234	-0.0893311847659564\\
66.5	0.276	-0.0633573349140804\\
66.875	0.093	-0.052895122309681\\
66.875	0.09666	-0.0805940394145949\\
66.875	0.10032	-0.107236808043137\\
66.875	0.10398	-0.132823428195306\\
66.875	0.10764	-0.157353899871106\\
66.875	0.1113	-0.180828223070529\\
66.875	0.11496	-0.203246397793585\\
66.875	0.11862	-0.224608424040267\\
66.875	0.12228	-0.244914301810578\\
66.875	0.12594	-0.264164031104515\\
66.875	0.1296	-0.282357611922083\\
66.875	0.13326	-0.299495044263279\\
66.875	0.13692	-0.315576328128101\\
66.875	0.14058	-0.330601463516554\\
66.875	0.14424	-0.344570450428633\\
66.875	0.1479	-0.357483288864339\\
66.875	0.15156	-0.369339978823676\\
66.875	0.15522	-0.38014052030664\\
66.875	0.15888	-0.389884913313231\\
66.875	0.16254	-0.398573157843454\\
66.875	0.1662	-0.406205253897302\\
66.875	0.16986	-0.412781201474779\\
66.875	0.17352	-0.418301000575883\\
66.875	0.17718	-0.422764651200617\\
66.875	0.18084	-0.426172153348977\\
66.875	0.1845	-0.428523507020965\\
66.875	0.18816	-0.429818712216585\\
66.875	0.19182	-0.43005776893583\\
66.875	0.19548	-0.429240677178706\\
66.875	0.19914	-0.427367436945209\\
66.875	0.2028	-0.424438048235339\\
66.875	0.20646	-0.420452511049098\\
66.875	0.21012	-0.415410825386482\\
66.875	0.21378	-0.409312991247498\\
66.875	0.21744	-0.402159008632142\\
66.875	0.2211	-0.393948877540413\\
66.875	0.22476	-0.38468259797231\\
66.875	0.22842	-0.374360169927838\\
66.875	0.23208	-0.362981593406995\\
66.875	0.23574	-0.350546868409778\\
66.875	0.2394	-0.337055994936191\\
66.875	0.24306	-0.322508972986232\\
66.875	0.24672	-0.306905802559901\\
66.875	0.25038	-0.290246483657195\\
66.875	0.25404	-0.272531016278124\\
66.875	0.2577	-0.253759400422673\\
66.875	0.26136	-0.233931636090855\\
66.875	0.26502	-0.213047723282665\\
66.875	0.26868	-0.191107661998104\\
66.875	0.27234	-0.168111452237166\\
66.875	0.276	-0.144059093999861\\
67.25	0.093	-0.0367449275477378\\
67.25	0.09666	-0.0663653362672207\\
67.25	0.10032	-0.0949295965103316\\
67.25	0.10398	-0.122437708277073\\
67.25	0.10764	-0.148889671567442\\
67.25	0.1113	-0.174285486381437\\
67.25	0.11496	-0.198625152719063\\
67.25	0.11862	-0.221908670580315\\
67.25	0.12228	-0.244136039965196\\
67.25	0.12594	-0.265307260873705\\
67.25	0.1296	-0.285422333305841\\
67.25	0.13326	-0.304481257261607\\
67.25	0.13692	-0.322484032741\\
67.25	0.14058	-0.339430659744024\\
67.25	0.14424	-0.355321138270673\\
67.25	0.1479	-0.370155468320951\\
67.25	0.15156	-0.383933649894858\\
67.25	0.15522	-0.396655682992392\\
67.25	0.15888	-0.408321567613553\\
67.25	0.16254	-0.418931303758345\\
67.25	0.1662	-0.428484891426763\\
67.25	0.16986	-0.436982330618811\\
67.25	0.17352	-0.444423621334487\\
67.25	0.17718	-0.450808763573792\\
67.25	0.18084	-0.456137757336723\\
67.25	0.1845	-0.460410602623282\\
67.25	0.18816	-0.463627299433472\\
67.25	0.19182	-0.465787847767287\\
67.25	0.19548	-0.466892247624731\\
67.25	0.19914	-0.466940499005804\\
67.25	0.2028	-0.465932601910502\\
67.25	0.20646	-0.463868556338832\\
67.25	0.21012	-0.460748362290786\\
67.25	0.21378	-0.456572019766373\\
67.25	0.21744	-0.451339528765587\\
67.25	0.2211	-0.44505088928843\\
67.25	0.22476	-0.437706101334897\\
67.25	0.22842	-0.429305164904996\\
67.25	0.23208	-0.419848079998723\\
67.25	0.23574	-0.409334846616077\\
67.25	0.2394	-0.397765464757061\\
67.25	0.24306	-0.385139934421669\\
67.25	0.24672	-0.371458255609908\\
67.25	0.25038	-0.356720428321776\\
67.25	0.25404	-0.340926452557272\\
67.25	0.2577	-0.324076328316393\\
67.25	0.26136	-0.306170055599145\\
67.25	0.26502	-0.287207634405526\\
67.25	0.26868	-0.267189064735535\\
67.25	0.27234	-0.246114346589168\\
67.25	0.276	-0.223983479966433\\
67.625	0.093	-0.0198173596665869\\
67.625	0.09666	-0.0513592600006404\\
67.625	0.10032	-0.081845011858322\\
67.625	0.10398	-0.111274615239632\\
67.625	0.10764	-0.139648070144572\\
67.625	0.1113	-0.166965376573138\\
67.625	0.11496	-0.193226534525332\\
67.625	0.11862	-0.218431544001155\\
67.625	0.12228	-0.242580405000607\\
67.625	0.12594	-0.265673117523686\\
67.625	0.1296	-0.287709681570393\\
67.625	0.13326	-0.308690097140729\\
67.625	0.13692	-0.328614364234692\\
67.625	0.14058	-0.347482482852287\\
67.625	0.14424	-0.365294452993505\\
67.625	0.1479	-0.382050274658353\\
67.625	0.15156	-0.397749947846831\\
67.625	0.15522	-0.412393472558935\\
67.625	0.15888	-0.425980848794667\\
67.625	0.16254	-0.43851207655403\\
67.625	0.1662	-0.449987155837018\\
67.625	0.16986	-0.460406086643635\\
67.625	0.17352	-0.469768868973881\\
67.625	0.17718	-0.478075502827756\\
67.625	0.18084	-0.485325988205257\\
67.625	0.1845	-0.491520325106387\\
67.625	0.18816	-0.496658513531148\\
67.625	0.19182	-0.500740553479534\\
67.625	0.19548	-0.503766444951548\\
67.625	0.19914	-0.505736187947189\\
67.625	0.2028	-0.506649782466461\\
67.625	0.20646	-0.506507228509361\\
67.625	0.21012	-0.505308526075886\\
67.625	0.21378	-0.503053675166043\\
67.625	0.21744	-0.499742675779828\\
67.625	0.2211	-0.495375527917241\\
67.625	0.22476	-0.489952231578279\\
67.625	0.22842	-0.483472786762945\\
67.625	0.23208	-0.475937193471243\\
67.625	0.23574	-0.467345451703168\\
67.625	0.2394	-0.457697561458722\\
67.625	0.24306	-0.446993522737901\\
67.625	0.24672	-0.435233335540711\\
67.625	0.25038	-0.422416999867146\\
67.625	0.25404	-0.408544515717213\\
67.625	0.2577	-0.393615883090904\\
67.625	0.26136	-0.377631101988227\\
67.625	0.26502	-0.360590172409179\\
67.625	0.26868	-0.342493094353758\\
67.625	0.27234	-0.323339867821962\\
67.625	0.276	-0.303130492813798\\
68	0.093	-0.00211241866622647\\
68	0.09666	-0.0355758106148507\\
68	0.10032	-0.0679830540871029\\
68	0.10398	-0.0993341490829835\\
68	0.10764	-0.129629095602494\\
68	0.1113	-0.158867893645629\\
68	0.11496	-0.187050543212396\\
68	0.11862	-0.214177044302787\\
68	0.12228	-0.24024739691681\\
68	0.12594	-0.26526160105446\\
68	0.1296	-0.289219656715736\\
68	0.13326	-0.312121563900644\\
68	0.13692	-0.333967322609178\\
68	0.14058	-0.354756932841342\\
68	0.14424	-0.37449039459713\\
68	0.1479	-0.393167707876549\\
68	0.15156	-0.410788872679597\\
68	0.15522	-0.427353889006271\\
68	0.15888	-0.442862756856573\\
68	0.16254	-0.457315476230507\\
68	0.1662	-0.470712047128066\\
68	0.16986	-0.483052469549255\\
68	0.17352	-0.494336743494071\\
68	0.17718	-0.504564868962514\\
68	0.18084	-0.513736845954586\\
68	0.1845	-0.521852674470286\\
68	0.18816	-0.528912354509618\\
68	0.19182	-0.534915886072571\\
68	0.19548	-0.539863269159156\\
68	0.19914	-0.543754503769371\\
68	0.2028	-0.546589589903213\\
68	0.20646	-0.548368527560684\\
68	0.21012	-0.549091316741777\\
68	0.21378	-0.548757957446504\\
68	0.21744	-0.54736844967486\\
68	0.2211	-0.544922793426844\\
68	0.22476	-0.541420988702452\\
68	0.22842	-0.536863035501689\\
68	0.23208	-0.531248933824557\\
68	0.23574	-0.524578683671053\\
68	0.2394	-0.516852285041178\\
68	0.24306	-0.508069737934927\\
68	0.24672	-0.498231042352305\\
68	0.25038	-0.487336198293314\\
68	0.25404	-0.475385205757951\\
68	0.2577	-0.462378064746213\\
68	0.26136	-0.448314775258103\\
68	0.26502	-0.433195337293625\\
68	0.26868	-0.417019750852776\\
68	0.27234	-0.39978801593555\\
68	0.276	-0.381500132541957\\
68.375	0.093	0.0163698954533436\\
68.375	0.09666	-0.0190149881098495\\
68.375	0.10032	-0.0533437231966706\\
68.375	0.10398	-0.0866163098071237\\
68.375	0.10764	-0.118832747941203\\
68.375	0.1113	-0.149993037598909\\
68.375	0.11496	-0.180097178780246\\
68.375	0.11862	-0.209145171485208\\
68.375	0.12228	-0.237137015713802\\
68.375	0.12594	-0.264072711466021\\
68.375	0.1296	-0.289952258741869\\
68.375	0.13326	-0.314775657541346\\
68.375	0.13692	-0.338542907864451\\
68.375	0.14058	-0.361254009711185\\
68.375	0.14424	-0.382908963081544\\
68.375	0.1479	-0.403507767975534\\
68.375	0.15156	-0.423050424393151\\
68.375	0.15522	-0.441536932334395\\
68.375	0.15888	-0.458967291799268\\
68.375	0.16254	-0.475341502787772\\
68.375	0.1662	-0.4906595652999\\
68.375	0.16986	-0.504921479335659\\
68.375	0.17352	-0.518127244895045\\
68.375	0.17718	-0.530276861978062\\
68.375	0.18084	-0.541370330584705\\
68.375	0.1845	-0.551407650714972\\
68.375	0.18816	-0.560388822368874\\
68.375	0.19182	-0.568313845546402\\
68.375	0.19548	-0.575182720247554\\
68.375	0.19914	-0.580995446472339\\
68.375	0.2028	-0.585752024220749\\
68.375	0.20646	-0.589452453492791\\
68.375	0.21012	-0.592096734288457\\
68.375	0.21378	-0.593684866607756\\
68.375	0.21744	-0.594216850450678\\
68.375	0.2211	-0.593692685817233\\
68.375	0.22476	-0.592112372707412\\
68.375	0.22842	-0.589475911121223\\
68.375	0.23208	-0.585783301058659\\
68.375	0.23574	-0.581034542519725\\
68.375	0.2394	-0.57522963550442\\
68.375	0.24306	-0.568368580012737\\
68.375	0.24672	-0.560451376044689\\
68.375	0.25038	-0.551478023600265\\
68.375	0.25404	-0.541448522679473\\
68.375	0.2577	-0.530362873282305\\
68.375	0.26136	-0.51822107540877\\
68.375	0.26502	-0.505023129058859\\
68.375	0.26868	-0.49076903423258\\
68.375	0.27234	-0.475458790929925\\
68.375	0.276	-0.459092399150903\\
68.75	0.093	0.0356295826921285\\
68.75	0.09666	-0.00167679248563707\\
68.75	0.10032	-0.0379270191870306\\
68.75	0.10398	-0.0731210974120526\\
68.75	0.10764	-0.107259027160703\\
68.75	0.1113	-0.140340808432979\\
68.75	0.11496	-0.172366441228885\\
68.75	0.11862	-0.203335925548418\\
68.75	0.12228	-0.233249261391583\\
68.75	0.12594	-0.262106448758372\\
68.75	0.1296	-0.289907487648789\\
68.75	0.13326	-0.316652378062838\\
68.75	0.13692	-0.342341120000512\\
68.75	0.14058	-0.366973713461817\\
68.75	0.14424	-0.390550158446747\\
68.75	0.1479	-0.413070454955305\\
68.75	0.15156	-0.434534602987493\\
68.75	0.15522	-0.45494260254331\\
68.75	0.15888	-0.474294453622752\\
68.75	0.16254	-0.492590156225826\\
68.75	0.1662	-0.509829710352525\\
68.75	0.16986	-0.526013116002854\\
68.75	0.17352	-0.541140373176812\\
68.75	0.17718	-0.555211481874395\\
68.75	0.18084	-0.568226442095609\\
68.75	0.1845	-0.58018525384045\\
68.75	0.18816	-0.59108791710892\\
68.75	0.19182	-0.600934431901018\\
68.75	0.19548	-0.609724798216741\\
68.75	0.19914	-0.617459016056097\\
68.75	0.2028	-0.624137085419077\\
68.75	0.20646	-0.629759006305689\\
68.75	0.21012	-0.634324778715927\\
68.75	0.21378	-0.637834402649792\\
68.75	0.21744	-0.640287878107289\\
68.75	0.2211	-0.641685205088411\\
68.75	0.22476	-0.642026383593161\\
68.75	0.22842	-0.641311413621542\\
68.75	0.23208	-0.639540295173548\\
68.75	0.23574	-0.636713028249186\\
68.75	0.2394	-0.632829612848448\\
68.75	0.24306	-0.627890048971339\\
68.75	0.24672	-0.621894336617861\\
68.75	0.25038	-0.614842475788008\\
68.75	0.25404	-0.606734466481787\\
68.75	0.2577	-0.597570308699186\\
68.75	0.26136	-0.587350002440222\\
68.75	0.26502	-0.576073547704882\\
68.75	0.26868	-0.563740944493173\\
68.75	0.27234	-0.550352192805089\\
68.75	0.276	-0.535907292640633\\
69.125	0.093	0.055666643050114\\
69.125	0.09666	0.0164387762577778\\
69.125	0.10032	-0.0217329420581864\\
69.125	0.10398	-0.0588485118977772\\
69.125	0.10764	-0.0949079332609979\\
69.125	0.1113	-0.129911206147843\\
69.125	0.11496	-0.163858330558322\\
69.125	0.11862	-0.196749306492426\\
69.125	0.12228	-0.228584133950159\\
69.125	0.12594	-0.259362812931519\\
69.125	0.1296	-0.289085343436507\\
69.125	0.13326	-0.317751725465125\\
69.125	0.13692	-0.345361959017369\\
69.125	0.14058	-0.371916044093243\\
69.125	0.14424	-0.397413980692745\\
69.125	0.1479	-0.421855768815874\\
69.125	0.15156	-0.445241408462633\\
69.125	0.15522	-0.467570899633018\\
69.125	0.15888	-0.488844242327031\\
69.125	0.16254	-0.509061436544675\\
69.125	0.1662	-0.528222482285944\\
69.125	0.16986	-0.546327379550846\\
69.125	0.17352	-0.56337612833937\\
69.125	0.17718	-0.579368728651528\\
69.125	0.18084	-0.594305180487309\\
69.125	0.1845	-0.608185483846721\\
69.125	0.18816	-0.621009638729761\\
69.125	0.19182	-0.63277764513643\\
69.125	0.19548	-0.643489503066723\\
69.125	0.19914	-0.65314521252065\\
69.125	0.2028	-0.661744773498201\\
69.125	0.20646	-0.669288185999384\\
69.125	0.21012	-0.675775450024188\\
69.125	0.21378	-0.681206565572628\\
69.125	0.21744	-0.685581532644692\\
69.125	0.2211	-0.688900351240388\\
69.125	0.22476	-0.691163021359705\\
69.125	0.22842	-0.692369543002657\\
69.125	0.23208	-0.692519916169234\\
69.125	0.23574	-0.691614140859442\\
69.125	0.2394	-0.689652217073275\\
69.125	0.24306	-0.686634144810736\\
69.125	0.24672	-0.682559924071826\\
69.125	0.25038	-0.677429554856543\\
69.125	0.25404	-0.671243037164893\\
69.125	0.2577	-0.664000370996863\\
69.125	0.26136	-0.655701556352469\\
69.125	0.26502	-0.646346593231703\\
69.125	0.26868	-0.635935481634562\\
69.125	0.27234	-0.624468221561048\\
69.125	0.276	-0.611944813011164\\
69.5	0.093	0.0764810765273092\\
69.5	0.09666	0.0353317181204023\\
69.5	0.10032	-0.00476149181012903\\
69.5	0.10398	-0.0437985532642923\\
69.5	0.10764	-0.0817794662420837\\
69.5	0.1113	-0.118704230743501\\
69.5	0.11496	-0.154572846768549\\
69.5	0.11862	-0.189385314317223\\
69.5	0.12228	-0.223141633389527\\
69.5	0.12594	-0.255841803985456\\
69.5	0.1296	-0.287485826105015\\
69.5	0.13326	-0.318073699748203\\
69.5	0.13692	-0.347605424915018\\
69.5	0.14058	-0.376081001605465\\
69.5	0.14424	-0.403500429819534\\
69.5	0.1479	-0.429863709557234\\
69.5	0.15156	-0.455170840818563\\
69.5	0.15522	-0.479421823603519\\
69.5	0.15888	-0.502616657912105\\
69.5	0.16254	-0.524755343744317\\
69.5	0.1662	-0.545837881100159\\
69.5	0.16986	-0.565864269979627\\
69.5	0.17352	-0.584834510382726\\
69.5	0.17718	-0.602748602309451\\
69.5	0.18084	-0.619606545759802\\
69.5	0.1845	-0.635408340733785\\
69.5	0.18816	-0.650153987231396\\
69.5	0.19182	-0.663843485252636\\
69.5	0.19548	-0.6764768347975\\
69.5	0.19914	-0.688054035865993\\
69.5	0.2028	-0.698575088458119\\
69.5	0.20646	-0.708039992573869\\
69.5	0.21012	-0.716448748213248\\
69.5	0.21378	-0.723801355376254\\
69.5	0.21744	-0.730097814062889\\
69.5	0.2211	-0.735338124273156\\
69.5	0.22476	-0.739522286007043\\
69.5	0.22842	-0.742650299264566\\
69.5	0.23208	-0.744722164045714\\
69.5	0.23574	-0.745737880350489\\
69.5	0.2394	-0.745697448178896\\
69.5	0.24306	-0.744600867530925\\
69.5	0.24672	-0.742448138406588\\
69.5	0.25038	-0.739239260805876\\
69.5	0.25404	-0.734974234728793\\
69.5	0.2577	-0.729653060175337\\
69.5	0.26136	-0.723275737145511\\
69.5	0.26502	-0.715842265639315\\
69.5	0.26868	-0.707352645656745\\
69.5	0.27234	-0.697806877197802\\
69.5	0.276	-0.687204960262488\\
69.875	0.093	0.0980728831237121\\
69.875	0.09666	0.0550020331022363\\
69.875	0.10032	0.0129873315571326\\
69.875	0.10398	-0.0279712215115996\\
69.875	0.10764	-0.0678736261039616\\
69.875	0.1113	-0.106719882219948\\
69.875	0.11496	-0.144509989859567\\
69.875	0.11862	-0.181243949022811\\
69.875	0.12228	-0.216921759709686\\
69.875	0.12594	-0.251543421920186\\
69.875	0.1296	-0.285108935654315\\
69.875	0.13326	-0.317618300912074\\
69.875	0.13692	-0.349071517693458\\
69.875	0.14058	-0.379468585998473\\
69.875	0.14424	-0.408809505827115\\
69.875	0.1479	-0.437094277179386\\
69.875	0.15156	-0.464322900055286\\
69.875	0.15522	-0.490495374454811\\
69.875	0.15888	-0.515611700377965\\
69.875	0.16254	-0.53967187782475\\
69.875	0.1662	-0.562675906795159\\
69.875	0.16986	-0.584623787289202\\
69.875	0.17352	-0.605515519306867\\
69.875	0.17718	-0.625351102848163\\
69.875	0.18084	-0.644130537913088\\
69.875	0.1845	-0.661853824501639\\
69.875	0.18816	-0.67852096261382\\
69.875	0.19182	-0.69413195224963\\
69.875	0.19548	-0.708686793409065\\
69.875	0.19914	-0.722185486092129\\
69.875	0.2028	-0.734628030298822\\
69.875	0.20646	-0.746014426029146\\
69.875	0.21012	-0.756344673283092\\
69.875	0.21378	-0.765618772060669\\
69.875	0.21744	-0.773836722361878\\
69.875	0.2211	-0.780998524186712\\
69.875	0.22476	-0.78710417753517\\
69.875	0.22842	-0.792153682407264\\
69.875	0.23208	-0.796147038802982\\
69.875	0.23574	-0.799084246722328\\
69.875	0.2394	-0.800965306165306\\
69.875	0.24306	-0.801790217131905\\
69.875	0.24672	-0.801558979622136\\
69.875	0.25038	-0.800271593635995\\
69.875	0.25404	-0.797928059173485\\
69.875	0.2577	-0.794528376234597\\
69.875	0.26136	-0.790072544819341\\
69.875	0.26502	-0.784560564927716\\
69.875	0.26868	-0.777992436559716\\
69.875	0.27234	-0.77036815971534\\
69.875	0.276	-0.7616877343946\\
70.25	0.093	0.120442062839325\\
70.25	0.09666	0.0754497212032799\\
70.25	0.10032	0.0315135280436037\\
70.25	0.10398	-0.0113665166396974\\
70.25	0.10764	-0.0531904128466301\\
70.25	0.1113	-0.0939581605771874\\
70.25	0.11496	-0.133669759831375\\
70.25	0.11862	-0.17232521060919\\
70.25	0.12228	-0.209924512910634\\
70.25	0.12594	-0.246467666735706\\
70.25	0.1296	-0.281954672084404\\
70.25	0.13326	-0.316385528956734\\
70.25	0.13692	-0.349760237352688\\
70.25	0.14058	-0.382078797272274\\
70.25	0.14424	-0.413341208715487\\
70.25	0.1479	-0.443547471682326\\
70.25	0.15156	-0.472697586172797\\
70.25	0.15522	-0.500791552186892\\
70.25	0.15888	-0.527829369724617\\
70.25	0.16254	-0.553811038785973\\
70.25	0.1662	-0.578736559370953\\
70.25	0.16986	-0.602605931479562\\
70.25	0.17352	-0.625419155111802\\
70.25	0.17718	-0.647176230267669\\
70.25	0.18084	-0.667877156947161\\
70.25	0.1845	-0.687521935150282\\
70.25	0.18816	-0.706110564877034\\
70.25	0.19182	-0.723643046127415\\
70.25	0.19548	-0.74011937890142\\
70.25	0.19914	-0.755539563199056\\
70.25	0.2028	-0.769903599020319\\
70.25	0.20646	-0.78321148636521\\
70.25	0.21012	-0.79546322523373\\
70.25	0.21378	-0.806658815625878\\
70.25	0.21744	-0.816798257541654\\
70.25	0.2211	-0.825881550981059\\
70.25	0.22476	-0.833908695944091\\
70.25	0.22842	-0.840879692430752\\
70.25	0.23208	-0.846794540441041\\
70.25	0.23574	-0.851653239974957\\
70.25	0.2394	-0.855455791032502\\
70.25	0.24306	-0.858202193613676\\
70.25	0.24672	-0.859892447718477\\
70.25	0.25038	-0.860526553346907\\
70.25	0.25404	-0.860104510498965\\
70.25	0.2577	-0.858626319174647\\
70.25	0.26136	-0.856091979373965\\
70.25	0.26502	-0.852501491096907\\
70.25	0.26868	-0.847854854343478\\
70.25	0.27234	-0.842152069113673\\
70.25	0.276	-0.835393135407504\\
70.625	0.093	0.143588615674148\\
70.625	0.09666	0.0966747824235295\\
70.625	0.10032	0.0508170976492862\\
70.625	0.10398	0.00601556135141262\\
70.625	0.10764	-0.0377298264700889\\
70.625	0.1113	-0.0804190658152169\\
70.625	0.11496	-0.122052156683977\\
70.625	0.11862	-0.162629099076361\\
70.625	0.12228	-0.202149892992375\\
70.625	0.12594	-0.240614538432018\\
70.625	0.1296	-0.278023035395287\\
70.625	0.13326	-0.314375383882186\\
70.625	0.13692	-0.349671583892713\\
70.625	0.14058	-0.383911635426867\\
70.625	0.14424	-0.417095538484651\\
70.625	0.1479	-0.449223293066061\\
70.625	0.15156	-0.4802948991711\\
70.625	0.15522	-0.510310356799768\\
70.625	0.15888	-0.539269665952062\\
70.625	0.16254	-0.567172826627988\\
70.625	0.1662	-0.594019838827539\\
70.625	0.16986	-0.619810702550719\\
70.625	0.17352	-0.644545417797526\\
70.625	0.17718	-0.668223984567963\\
70.625	0.18084	-0.69084640286203\\
70.625	0.1845	-0.712412672679722\\
70.625	0.18816	-0.732922794021044\\
70.625	0.19182	-0.752376766885992\\
70.625	0.19548	-0.770774591274568\\
70.625	0.19914	-0.788116267186774\\
70.625	0.2028	-0.804401794622608\\
70.625	0.20646	-0.81963117358207\\
70.625	0.21012	-0.833804404065157\\
70.625	0.21378	-0.846921486071876\\
70.625	0.21744	-0.858982419602226\\
70.625	0.2211	-0.869987204656201\\
70.625	0.22476	-0.879935841233801\\
70.625	0.22842	-0.888828329335032\\
70.625	0.23208	-0.896664668959892\\
70.625	0.23574	-0.903444860108379\\
70.625	0.2394	-0.909168902780495\\
70.625	0.24306	-0.913836796976235\\
70.625	0.24672	-0.917448542695607\\
70.625	0.25038	-0.920004139938607\\
70.625	0.25404	-0.92150358870524\\
70.625	0.2577	-0.921946888995492\\
70.625	0.26136	-0.921334040809378\\
70.625	0.26502	-0.919665044146891\\
70.625	0.26868	-0.916939899008032\\
70.625	0.27234	-0.913158605392798\\
70.625	0.276	-0.908321163301196\\
71	0.093	0.167512541628178\\
71	0.09666	0.11867721676299\\
71	0.10032	0.0708980403741746\\
71	0.10398	0.0241750124617322\\
71	0.10764	-0.02149186697434\\
71	0.1113	-0.0661025979340387\\
71	0.11496	-0.109657180417367\\
71	0.11862	-0.152155614424322\\
71	0.12228	-0.193597899954907\\
71	0.12594	-0.233984037009119\\
71	0.1296	-0.273314025586958\\
71	0.13326	-0.31158786568843\\
71	0.13692	-0.348805557313525\\
71	0.14058	-0.384967100462251\\
71	0.14424	-0.420072495134603\\
71	0.1479	-0.454121741330584\\
71	0.15156	-0.487114839050196\\
71	0.15522	-0.519051788293433\\
71	0.15888	-0.549932589060297\\
71	0.16254	-0.579757241350794\\
71	0.1662	-0.608525745164915\\
71	0.16986	-0.636238100502666\\
71	0.17352	-0.662894307364044\\
71	0.17718	-0.688494365749052\\
71	0.18084	-0.713038275657686\\
71	0.1845	-0.73652603708995\\
71	0.18816	-0.758957650045841\\
71	0.19182	-0.780333114525362\\
71	0.19548	-0.800652430528507\\
71	0.19914	-0.819915598055283\\
71	0.2028	-0.838122617105687\\
71	0.20646	-0.85527348767972\\
71	0.21012	-0.871368209777378\\
71	0.21378	-0.886406783398668\\
71	0.21744	-0.900389208543585\\
71	0.2211	-0.913315485212131\\
71	0.22476	-0.925185613404301\\
71	0.22842	-0.935999593120103\\
71	0.23208	-0.945757424359533\\
71	0.23574	-0.954459107122591\\
71	0.2394	-0.962104641409277\\
71	0.24306	-0.968694027219589\\
71	0.24672	-0.974227264553531\\
71	0.25038	-0.978704353411102\\
71	0.25404	-0.982125293792302\\
71	0.2577	-0.984490085697125\\
71	0.26136	-0.985798729125581\\
71	0.26502	-0.986051224077665\\
71	0.26868	-0.985247570553377\\
71	0.27234	-0.983387768552713\\
71	0.276	-0.980471818075682\\
71.375	0.093	0.192213840701419\\
71.375	0.09666	0.141457024221661\\
71.375	0.10032	0.0917563562182744\\
71.375	0.10398	0.0431118366912613\\
71.375	0.10764	-0.00447653435938156\\
71.375	0.1113	-0.0510087569336491\\
71.375	0.11496	-0.0964848310315484\\
71.375	0.11862	-0.140904756653074\\
71.375	0.12228	-0.18426853379823\\
71.375	0.12594	-0.226576162467012\\
71.375	0.1296	-0.267827642659422\\
71.375	0.13326	-0.308022974375462\\
71.375	0.13692	-0.347162157615129\\
71.375	0.14058	-0.385245192378425\\
71.375	0.14424	-0.422272078665348\\
71.375	0.1479	-0.458242816475899\\
71.375	0.15156	-0.49315740581008\\
71.375	0.15522	-0.527015846667887\\
71.375	0.15888	-0.559818139049325\\
71.375	0.16254	-0.591564282954388\\
71.375	0.1662	-0.62225427838308\\
71.375	0.16986	-0.651888125335402\\
71.375	0.17352	-0.680465823811351\\
71.375	0.17718	-0.707987373810927\\
71.375	0.18084	-0.734452775334133\\
71.375	0.1845	-0.759862028380968\\
71.375	0.18816	-0.78421513295143\\
71.375	0.19182	-0.807512089045521\\
71.375	0.19548	-0.829752896663237\\
71.375	0.19914	-0.850937555804584\\
71.375	0.2028	-0.871066066469556\\
71.375	0.20646	-0.890138428658159\\
71.375	0.21012	-0.908154642370388\\
71.375	0.21378	-0.925114707606248\\
71.375	0.21744	-0.941018624365736\\
71.375	0.2211	-0.955866392648852\\
71.375	0.22476	-0.969658012455593\\
71.375	0.22842	-0.982393483785966\\
71.375	0.23208	-0.994072806639967\\
71.375	0.23574	-1.0046959810176\\
71.375	0.2394	-1.01426300691885\\
71.375	0.24306	-1.02277388434373\\
71.375	0.24672	-1.03022861329225\\
71.375	0.25038	-1.03662719376439\\
71.375	0.25404	-1.04196962576016\\
71.375	0.2577	-1.04625590927955\\
71.375	0.26136	-1.04948604432258\\
71.375	0.26502	-1.05166003088923\\
71.375	0.26868	-1.05277786897951\\
71.375	0.27234	-1.05283955859342\\
71.375	0.276	-1.05184509973096\\
71.75	0.093	0.217692512893865\\
71.75	0.09666	0.165014204799537\\
71.75	0.10032	0.113392045181582\\
71.75	0.10398	0.0628260340399965\\
71.75	0.10764	0.0133161713747847\\
71.75	0.1113	-0.0351375428140553\\
71.75	0.11496	-0.0825351085265252\\
71.75	0.11862	-0.12887652576262\\
71.75	0.12228	-0.174161794522346\\
71.75	0.12594	-0.218390914805699\\
71.75	0.1296	-0.26156388661268\\
71.75	0.13326	-0.303680709943289\\
71.75	0.13692	-0.344741384797526\\
71.75	0.14058	-0.384745911175393\\
71.75	0.14424	-0.423694289076887\\
71.75	0.1479	-0.461586518502008\\
71.75	0.15156	-0.498422599450758\\
71.75	0.15522	-0.534202531923136\\
71.75	0.15888	-0.568926315919142\\
71.75	0.16254	-0.602593951438779\\
71.75	0.1662	-0.635205438482041\\
71.75	0.16986	-0.666760777048934\\
71.75	0.17352	-0.697259967139455\\
71.75	0.17718	-0.726703008753603\\
71.75	0.18084	-0.755089901891373\\
71.75	0.1845	-0.782420646552779\\
71.75	0.18816	-0.808695242737812\\
71.75	0.19182	-0.833913690446473\\
71.75	0.19548	-0.85807598967876\\
71.75	0.19914	-0.881182140434674\\
71.75	0.2028	-0.90323214271422\\
71.75	0.20646	-0.924225996517394\\
71.75	0.21012	-0.944163701844193\\
71.75	0.21378	-0.963045258694624\\
71.75	0.21744	-0.980870667068683\\
71.75	0.2211	-0.997639926966366\\
71.75	0.22476	-1.01335303838768\\
71.75	0.22842	-1.02801000133262\\
71.75	0.23208	-1.04161081580119\\
71.75	0.23574	-1.05415548179339\\
71.75	0.2394	-1.06564399930922\\
71.75	0.24306	-1.07607636834867\\
71.75	0.24672	-1.08545258891175\\
71.75	0.25038	-1.09377266099846\\
71.75	0.25404	-1.10103658460881\\
71.75	0.2577	-1.10724435974277\\
71.75	0.26136	-1.11239598640036\\
71.75	0.26502	-1.11649146458159\\
71.75	0.26868	-1.11953079428644\\
71.75	0.27234	-1.12151397551492\\
71.75	0.276	-1.12244100826703\\
72.125	0.093	0.24394855820552\\
72.125	0.09666	0.189348758496622\\
72.125	0.10032	0.135805107264096\\
72.125	0.10398	0.0833176045079412\\
72.125	0.10764	0.0318862502281587\\
72.125	0.1113	-0.0184889555752502\\
72.125	0.11496	-0.0678080129022908\\
72.125	0.11862	-0.116070921752958\\
72.125	0.12228	-0.163277682127253\\
72.125	0.12594	-0.209428294025176\\
72.125	0.1296	-0.254522757446726\\
72.125	0.13326	-0.298561072391908\\
72.125	0.13692	-0.341543238860714\\
72.125	0.14058	-0.383469256853151\\
72.125	0.14424	-0.424339126369216\\
72.125	0.1479	-0.464152847408906\\
72.125	0.15156	-0.502910419972229\\
72.125	0.15522	-0.540611844059176\\
72.125	0.15888	-0.577257119669752\\
72.125	0.16254	-0.612846246803961\\
72.125	0.1662	-0.647379225461792\\
72.125	0.16986	-0.680856055643252\\
72.125	0.17352	-0.71327673734834\\
72.125	0.17718	-0.744641270577063\\
72.125	0.18084	-0.774949655329408\\
72.125	0.1845	-0.80420189160538\\
72.125	0.18816	-0.832397979404984\\
72.125	0.19182	-0.859537918728216\\
72.125	0.19548	-0.885621709575073\\
72.125	0.19914	-0.910649351945558\\
72.125	0.2028	-0.934620845839674\\
72.125	0.20646	-0.957536191257419\\
72.125	0.21012	-0.979395388198785\\
72.125	0.21378	-1.00019843666379\\
72.125	0.21744	-1.01994533665242\\
72.125	0.2211	-1.03863608816467\\
72.125	0.22476	-1.05627069120055\\
72.125	0.22842	-1.07284914576007\\
72.125	0.23208	-1.08837145184321\\
72.125	0.23574	-1.10283760944998\\
72.125	0.2394	-1.11624761858037\\
72.125	0.24306	-1.1286014792344\\
72.125	0.24672	-1.13989919141205\\
72.125	0.25038	-1.15014075511333\\
72.125	0.25404	-1.15932617033824\\
72.125	0.2577	-1.16745543708678\\
72.125	0.26136	-1.17452855535895\\
72.125	0.26502	-1.18054552515474\\
72.125	0.26868	-1.18550634647416\\
72.125	0.27234	-1.18941101931721\\
72.125	0.276	-1.19225954368389\\
72.5	0.093	0.270981976636384\\
72.5	0.09666	0.214460685312917\\
72.5	0.10032	0.158995542465822\\
72.5	0.10398	0.104586548095095\\
72.5	0.10764	0.0512337022007423\\
72.5	0.1113	-0.00106299521723724\\
72.5	0.11496	-0.0523035441588467\\
72.5	0.11862	-0.102487944624085\\
72.5	0.12228	-0.15161619661295\\
72.5	0.12594	-0.199688300125443\\
72.5	0.1296	-0.246704255161565\\
72.5	0.13326	-0.292664061721315\\
72.5	0.13692	-0.337567719804692\\
72.5	0.14058	-0.3814152294117\\
72.5	0.14424	-0.424206590542333\\
72.5	0.1479	-0.465941803196597\\
72.5	0.15156	-0.506620867374488\\
72.5	0.15522	-0.546243783076006\\
72.5	0.15888	-0.584810550301151\\
72.5	0.16254	-0.622321169049929\\
72.5	0.1662	-0.658775639322331\\
72.5	0.16986	-0.694173961118365\\
72.5	0.17352	-0.728516134438023\\
72.5	0.17718	-0.761802159281314\\
72.5	0.18084	-0.794032035648228\\
72.5	0.1845	-0.825205763538772\\
72.5	0.18816	-0.855323342952946\\
72.5	0.19182	-0.884384773890749\\
72.5	0.19548	-0.912390056352177\\
72.5	0.19914	-0.939339190337232\\
72.5	0.2028	-0.965232175845919\\
72.5	0.20646	-0.990069012878231\\
72.5	0.21012	-1.01384970143417\\
72.5	0.21378	-1.03657424151374\\
72.5	0.21744	-1.05824263311694\\
72.5	0.2211	-1.07885487624377\\
72.5	0.22476	-1.09841097089422\\
72.5	0.22842	-1.1169109170683\\
72.5	0.23208	-1.13435471476602\\
72.5	0.23574	-1.15074236398735\\
72.5	0.2394	-1.16607386473232\\
72.5	0.24306	-1.18034921700092\\
72.5	0.24672	-1.19356842079314\\
72.5	0.25038	-1.20573147610899\\
72.5	0.25404	-1.21683838294847\\
72.5	0.2577	-1.22688914131158\\
72.5	0.26136	-1.23588375119831\\
72.5	0.26502	-1.24382221260868\\
72.5	0.26868	-1.25070452554267\\
72.5	0.27234	-1.25653069000029\\
72.5	0.276	-1.26130070598154\\
72.875	0.093	0.298792768186456\\
72.875	0.09666	0.240349985248419\\
72.875	0.10032	0.182963350786753\\
72.875	0.10398	0.126632864801457\\
72.875	0.10764	0.0713585272925337\\
72.875	0.1113	0.0171403382599835\\
72.875	0.11496	-0.0360217022961985\\
72.875	0.11862	-0.0881275943760051\\
72.875	0.12228	-0.139177337979442\\
72.875	0.12594	-0.189170933106505\\
72.875	0.1296	-0.238108379757196\\
72.875	0.13326	-0.285989677931517\\
72.875	0.13692	-0.332814827629464\\
72.875	0.14058	-0.378583828851043\\
72.875	0.14424	-0.423296681596247\\
72.875	0.1479	-0.466953385865079\\
72.875	0.15156	-0.509553941657543\\
72.875	0.15522	-0.551098348973631\\
72.875	0.15888	-0.591586607813351\\
72.875	0.16254	-0.631018718176696\\
72.875	0.1662	-0.669394680063668\\
72.875	0.16986	-0.706714493474273\\
72.875	0.17352	-0.742978158408499\\
72.875	0.17718	-0.778185674866363\\
72.875	0.18084	-0.812337042847845\\
72.875	0.1845	-0.845432262352962\\
72.875	0.18816	-0.877471333381704\\
72.875	0.19182	-0.908454255934074\\
72.875	0.19548	-0.938381030010072\\
72.875	0.19914	-0.967251655609702\\
72.875	0.2028	-0.995066132732956\\
72.875	0.20646	-1.02182446137984\\
72.875	0.21012	-1.04752664155035\\
72.875	0.21378	-1.07217267324449\\
72.875	0.21744	-1.09576255646226\\
72.875	0.2211	-1.11829629120366\\
72.875	0.22476	-1.13977387746868\\
72.875	0.22842	-1.16019531525733\\
72.875	0.23208	-1.17956060456962\\
72.875	0.23574	-1.19786974540553\\
72.875	0.2394	-1.21512273776507\\
72.875	0.24306	-1.23131958164823\\
72.875	0.24672	-1.24646027705502\\
72.875	0.25038	-1.26054482398544\\
72.875	0.25404	-1.2735732224395\\
72.875	0.2577	-1.28554547241717\\
72.875	0.26136	-1.29646157391848\\
72.875	0.26502	-1.30632152694342\\
72.875	0.26868	-1.31512533149198\\
72.875	0.27234	-1.32287298756417\\
72.875	0.276	-1.32956449515999\\
73.25	0.093	0.327380932855736\\
73.25	0.09666	0.26701665830313\\
73.25	0.10032	0.207708532226892\\
73.25	0.10398	0.149456554627027\\
73.25	0.10764	0.0922607255035328\\
73.25	0.1113	0.0361210448564119\\
73.25	0.11496	-0.0189624873143389\\
73.25	0.11862	-0.0729898710087162\\
73.25	0.12228	-0.125961106226723\\
73.25	0.12594	-0.177876192968357\\
73.25	0.1296	-0.228735131233619\\
73.25	0.13326	-0.278537921022511\\
73.25	0.13692	-0.327284562335029\\
73.25	0.14058	-0.374975055171177\\
73.25	0.14424	-0.421609399530949\\
73.25	0.1479	-0.467187595414356\\
73.25	0.15156	-0.511709642821386\\
73.25	0.15522	-0.555175541752045\\
73.25	0.15888	-0.597585292206336\\
73.25	0.16254	-0.638938894184248\\
73.25	0.1662	-0.679236347685795\\
73.25	0.16986	-0.718477652710966\\
73.25	0.17352	-0.756662809259766\\
73.25	0.17718	-0.793791817332198\\
73.25	0.18084	-0.829864676928254\\
73.25	0.1845	-0.864881388047938\\
73.25	0.18816	-0.898841950691251\\
73.25	0.19182	-0.931746364858191\\
73.25	0.19548	-0.96359463054876\\
73.25	0.19914	-0.994386747762961\\
73.25	0.2028	-1.02412271650079\\
73.25	0.20646	-1.05280253676224\\
73.25	0.21012	-1.08042620854732\\
73.25	0.21378	-1.10699373185603\\
73.25	0.21744	-1.13250510668837\\
73.25	0.2211	-1.15696033304434\\
73.25	0.22476	-1.18035941092393\\
73.25	0.22842	-1.20270234032716\\
73.25	0.23208	-1.22398912125401\\
73.25	0.23574	-1.24421975370449\\
73.25	0.2394	-1.2633942376786\\
73.25	0.24306	-1.28151257317633\\
73.25	0.24672	-1.2985747601977\\
73.25	0.25038	-1.31458079874269\\
73.25	0.25404	-1.32953068881131\\
73.25	0.2577	-1.34342443040356\\
73.25	0.26136	-1.35626202351943\\
73.25	0.26502	-1.36804346815894\\
73.25	0.26868	-1.37876876432207\\
73.25	0.27234	-1.38843791200883\\
73.25	0.276	-1.39705091121922\\
73.625	0.093	0.35674647064423\\
73.625	0.09666	0.294460704477049\\
73.625	0.10032	0.233231086786244\\
73.625	0.10398	0.173057617571807\\
73.625	0.10764	0.113940296833742\\
73.625	0.1113	0.0558791245720517\\
73.625	0.11496	-0.00112589921326978\\
73.625	0.11862	-0.0570747745222178\\
73.625	0.12228	-0.111967501354796\\
73.625	0.12594	-0.165804079710998\\
73.625	0.1296	-0.218584509590831\\
73.625	0.13326	-0.270308790994293\\
73.625	0.13692	-0.32097692392138\\
73.625	0.14058	-0.3705889083721\\
73.625	0.14424	-0.419144744346445\\
73.625	0.1479	-0.466644431844419\\
73.625	0.15156	-0.51308797086602\\
73.625	0.15522	-0.55847536141125\\
73.625	0.15888	-0.602806603480111\\
73.625	0.16254	-0.646081697072597\\
73.625	0.1662	-0.688300642188708\\
73.625	0.16986	-0.729463438828454\\
73.625	0.17352	-0.769570086991824\\
73.625	0.17718	-0.808620586678823\\
73.625	0.18084	-0.84661493788945\\
73.625	0.1845	-0.883553140623705\\
73.625	0.18816	-0.919435194881591\\
73.625	0.19182	-0.954261100663099\\
73.625	0.19548	-0.988030857968239\\
73.625	0.19914	-1.02074446679701\\
73.625	0.2028	-1.05240192714941\\
73.625	0.20646	-1.08300323902543\\
73.625	0.21012	-1.11254840242508\\
73.625	0.21378	-1.14103741734836\\
73.625	0.21744	-1.16847028379527\\
73.625	0.2211	-1.19484700176581\\
73.625	0.22476	-1.22016757125997\\
73.625	0.22842	-1.24443199227777\\
73.625	0.23208	-1.26764026481919\\
73.625	0.23574	-1.28979238888424\\
73.625	0.2394	-1.31088836447292\\
73.625	0.24306	-1.33092819158522\\
73.625	0.24672	-1.34991187022116\\
73.625	0.25038	-1.36783940038072\\
73.625	0.25404	-1.38471078206391\\
73.625	0.2577	-1.40052601527073\\
73.625	0.26136	-1.41528510000117\\
73.625	0.26502	-1.42898803625525\\
73.625	0.26868	-1.44163482403296\\
73.625	0.27234	-1.45322546333429\\
73.625	0.276	-1.46375995415925\\
74	0.093	0.386889381551931\\
74	0.09666	0.322682123770181\\
74	0.10032	0.259531014464803\\
74	0.10398	0.197436053635797\\
74	0.10764	0.136397241283162\\
74	0.1113	0.0764145774069011\\
74	0.11496	0.0174880620070089\\
74	0.11862	-0.040382304916508\\
74	0.12228	-0.0971965233636565\\
74	0.12594	-0.15295459333443\\
74	0.1296	-0.207656514828833\\
74	0.13326	-0.261302287846864\\
74	0.13692	-0.313891912388524\\
74	0.14058	-0.365425388453815\\
74	0.14424	-0.415902716042727\\
74	0.1479	-0.465323895155271\\
74	0.15156	-0.513688925791443\\
74	0.15522	-0.560997807951247\\
74	0.15888	-0.607250541634672\\
74	0.16254	-0.652447126841732\\
74	0.1662	-0.696587563572413\\
74	0.16986	-0.73967185182673\\
74	0.17352	-0.781699991604667\\
74	0.17718	-0.82267198290624\\
74	0.18084	-0.862587825731434\\
74	0.1845	-0.901447520080264\\
74	0.18816	-0.939251065952717\\
74	0.19182	-0.975998463348799\\
74	0.19548	-1.01168971226851\\
74	0.19914	-1.04632481271185\\
74	0.2028	-1.07990376467881\\
74	0.20646	-1.11242656816941\\
74	0.21012	-1.14389322318363\\
74	0.21378	-1.17430372972148\\
74	0.21744	-1.20365808778296\\
74	0.2211	-1.23195629736807\\
74	0.22476	-1.2591983584768\\
74	0.22842	-1.28538427110917\\
74	0.23208	-1.31051403526516\\
74	0.23574	-1.33458765094478\\
74	0.2394	-1.35760511814803\\
74	0.24306	-1.3795664368749\\
74	0.24672	-1.40047160712541\\
74	0.25038	-1.42032062889954\\
74	0.25404	-1.4391135021973\\
74	0.2577	-1.45685022701869\\
74	0.26136	-1.47353080336371\\
74	0.26502	-1.48915523123236\\
74	0.26868	-1.50372351062463\\
74	0.27234	-1.51723564154053\\
74	0.276	-1.52969162398006\\
};
\end{axis}

\begin{axis}[%
width=4.527496cm,
height=3.870968cm,
at={(6.483547cm,0cm)},
scale only axis,
xmin=56,
xmax=74,
tick align=outside,
xlabel={$L_{cut}$},
xmajorgrids,
ymin=0.093,
ymax=0.276,
ylabel={$D_{rlx}$},
ymajorgrids,
zmin=0,
zmax=113.453205752381,
zlabel={$x_3,x_4$},
zmajorgrids,
view={-140}{50},
legend style={at={(1.03,1)},anchor=north west,legend cell align=left,align=left,draw=white!15!black}
]
\addplot3[only marks,mark=*,mark options={},mark size=1.5000pt,color=mycolor1] plot table[row sep=crcr,]{%
74	0.123	15.4680684467597\\
72	0.113	12.0981268654473\\
61	0.095	6.35969255839677\\
56	0.093	5.67589105085895\\
};
\addplot3[only marks,mark=*,mark options={},mark size=1.5000pt,color=mycolor2] plot table[row sep=crcr,]{%
67	0.276	110.131370256376\\
66	0.255	90.4271696447197\\
62	0.209	51.7720891508705\\
57	0.193	41.4264697981385\\
};
\addplot3[only marks,mark=*,mark options={},mark size=1.5000pt,color=black] plot table[row sep=crcr,]{%
69	0.104	9.34303234279614\\
};
\addplot3[only marks,mark=*,mark options={},mark size=1.5000pt,color=black] plot table[row sep=crcr,]{%
64	0.23	68.0446111694493\\
};

\addplot3[%
surf,
opacity=0.7,
shader=interp,
colormap={mymap}{[1pt] rgb(0pt)=(0.0901961,0.239216,0.0745098); rgb(1pt)=(0.0945149,0.242058,0.0739522); rgb(2pt)=(0.0988592,0.244894,0.0733566); rgb(3pt)=(0.103229,0.247724,0.0727241); rgb(4pt)=(0.107623,0.250549,0.0720557); rgb(5pt)=(0.112043,0.253367,0.0713525); rgb(6pt)=(0.116487,0.25618,0.0706154); rgb(7pt)=(0.120956,0.258986,0.0698456); rgb(8pt)=(0.125449,0.261787,0.0690441); rgb(9pt)=(0.129967,0.264581,0.0682118); rgb(10pt)=(0.134508,0.26737,0.06735); rgb(11pt)=(0.139074,0.270152,0.0664596); rgb(12pt)=(0.143663,0.272929,0.0655416); rgb(13pt)=(0.148275,0.275699,0.0645971); rgb(14pt)=(0.152911,0.278463,0.0636271); rgb(15pt)=(0.15757,0.281221,0.0626328); rgb(16pt)=(0.162252,0.283973,0.0616151); rgb(17pt)=(0.166957,0.286719,0.060575); rgb(18pt)=(0.171685,0.289458,0.0595136); rgb(19pt)=(0.176434,0.292191,0.0584321); rgb(20pt)=(0.181207,0.294918,0.0573313); rgb(21pt)=(0.186001,0.297639,0.0562123); rgb(22pt)=(0.190817,0.300353,0.0550763); rgb(23pt)=(0.195655,0.303061,0.0539242); rgb(24pt)=(0.200514,0.305763,0.052757); rgb(25pt)=(0.205395,0.308459,0.0515759); rgb(26pt)=(0.210296,0.311149,0.0503624); rgb(27pt)=(0.215212,0.313846,0.0490067); rgb(28pt)=(0.220142,0.316548,0.0475043); rgb(29pt)=(0.22509,0.319254,0.0458704); rgb(30pt)=(0.230056,0.321962,0.0441205); rgb(31pt)=(0.235042,0.324671,0.04227); rgb(32pt)=(0.240048,0.327379,0.0403343); rgb(33pt)=(0.245078,0.330085,0.0383287); rgb(34pt)=(0.250131,0.332786,0.0362688); rgb(35pt)=(0.25521,0.335482,0.0341698); rgb(36pt)=(0.260317,0.33817,0.0320472); rgb(37pt)=(0.265451,0.340849,0.0299163); rgb(38pt)=(0.270616,0.343517,0.0277927); rgb(39pt)=(0.275813,0.346172,0.0256916); rgb(40pt)=(0.281043,0.348814,0.0236284); rgb(41pt)=(0.286307,0.35144,0.0216186); rgb(42pt)=(0.291607,0.354048,0.0196776); rgb(43pt)=(0.296945,0.356637,0.0178207); rgb(44pt)=(0.302322,0.359206,0.0160634); rgb(45pt)=(0.307739,0.361753,0.0144211); rgb(46pt)=(0.313198,0.364275,0.0129091); rgb(47pt)=(0.318701,0.366772,0.0115428); rgb(48pt)=(0.324249,0.369242,0.0103377); rgb(49pt)=(0.329843,0.371682,0.00930909); rgb(50pt)=(0.335485,0.374093,0.00847245); rgb(51pt)=(0.341176,0.376471,0.00784314); rgb(52pt)=(0.346925,0.378826,0.00732741); rgb(53pt)=(0.352735,0.381168,0.00682184); rgb(54pt)=(0.358605,0.383497,0.00632729); rgb(55pt)=(0.364532,0.385812,0.00584464); rgb(56pt)=(0.370516,0.388113,0.00537476); rgb(57pt)=(0.376552,0.390399,0.00491852); rgb(58pt)=(0.38264,0.39267,0.00447681); rgb(59pt)=(0.388777,0.394925,0.00405048); rgb(60pt)=(0.394962,0.397164,0.00364042); rgb(61pt)=(0.401191,0.399386,0.00324749); rgb(62pt)=(0.407464,0.401592,0.00287258); rgb(63pt)=(0.413777,0.40378,0.00251655); rgb(64pt)=(0.420129,0.40595,0.00218028); rgb(65pt)=(0.426518,0.408102,0.00186463); rgb(66pt)=(0.432942,0.410234,0.00157049); rgb(67pt)=(0.439399,0.412348,0.00129873); rgb(68pt)=(0.445885,0.414441,0.00105022); rgb(69pt)=(0.452401,0.416515,0.000825833); rgb(70pt)=(0.458942,0.418567,0.000626441); rgb(71pt)=(0.465508,0.420599,0.00045292); rgb(72pt)=(0.472096,0.422609,0.000306141); rgb(73pt)=(0.478704,0.424596,0.000186979); rgb(74pt)=(0.485331,0.426562,9.63073e-05); rgb(75pt)=(0.491973,0.428504,3.49981e-05); rgb(76pt)=(0.498628,0.430422,3.92506e-06); rgb(77pt)=(0.505323,0.432315,0); rgb(78pt)=(0.512206,0.434168,0); rgb(79pt)=(0.519282,0.435983,0); rgb(80pt)=(0.526529,0.437764,0); rgb(81pt)=(0.533922,0.439512,0); rgb(82pt)=(0.54144,0.441232,0); rgb(83pt)=(0.549059,0.442927,0); rgb(84pt)=(0.556756,0.444599,0); rgb(85pt)=(0.564508,0.446252,0); rgb(86pt)=(0.572292,0.447889,0); rgb(87pt)=(0.580084,0.449514,0); rgb(88pt)=(0.587863,0.451129,0); rgb(89pt)=(0.595604,0.452737,0); rgb(90pt)=(0.603284,0.454343,0); rgb(91pt)=(0.610882,0.455948,0); rgb(92pt)=(0.618373,0.457556,0); rgb(93pt)=(0.625734,0.459171,0); rgb(94pt)=(0.632943,0.460795,0); rgb(95pt)=(0.639976,0.462432,0); rgb(96pt)=(0.64681,0.464084,0); rgb(97pt)=(0.653423,0.465756,0); rgb(98pt)=(0.659791,0.46745,0); rgb(99pt)=(0.665891,0.469169,0); rgb(100pt)=(0.6717,0.470916,0); rgb(101pt)=(0.677195,0.472696,0); rgb(102pt)=(0.682353,0.47451,0); rgb(103pt)=(0.687242,0.476355,0); rgb(104pt)=(0.691952,0.478225,0); rgb(105pt)=(0.696497,0.480118,0); rgb(106pt)=(0.700887,0.482033,0); rgb(107pt)=(0.705134,0.483968,0); rgb(108pt)=(0.709251,0.485921,0); rgb(109pt)=(0.713249,0.487891,0); rgb(110pt)=(0.71714,0.489876,0); rgb(111pt)=(0.720936,0.491875,0); rgb(112pt)=(0.724649,0.493887,0); rgb(113pt)=(0.72829,0.495909,0); rgb(114pt)=(0.731872,0.49794,0); rgb(115pt)=(0.735406,0.499979,0); rgb(116pt)=(0.738904,0.502025,0); rgb(117pt)=(0.742378,0.504075,0); rgb(118pt)=(0.74584,0.506128,0); rgb(119pt)=(0.749302,0.508182,0); rgb(120pt)=(0.752775,0.510237,0); rgb(121pt)=(0.756272,0.51229,0); rgb(122pt)=(0.759804,0.514339,0); rgb(123pt)=(0.763384,0.516385,0); rgb(124pt)=(0.767022,0.518424,0); rgb(125pt)=(0.770731,0.520455,0); rgb(126pt)=(0.774523,0.522478,0); rgb(127pt)=(0.77841,0.524489,0); rgb(128pt)=(0.782391,0.526491,0); rgb(129pt)=(0.786402,0.528496,0); rgb(130pt)=(0.790431,0.530506,0); rgb(131pt)=(0.794478,0.532521,0); rgb(132pt)=(0.798541,0.534539,0); rgb(133pt)=(0.802619,0.53656,0); rgb(134pt)=(0.806712,0.538584,0); rgb(135pt)=(0.81082,0.540609,0); rgb(136pt)=(0.81494,0.542635,0); rgb(137pt)=(0.819074,0.54466,0); rgb(138pt)=(0.823219,0.546686,0); rgb(139pt)=(0.827374,0.548709,0); rgb(140pt)=(0.831541,0.55073,0); rgb(141pt)=(0.835716,0.552749,0); rgb(142pt)=(0.8399,0.554763,0); rgb(143pt)=(0.844092,0.556774,0); rgb(144pt)=(0.848292,0.558779,0); rgb(145pt)=(0.852497,0.560778,0); rgb(146pt)=(0.856708,0.562771,0); rgb(147pt)=(0.860924,0.564756,0); rgb(148pt)=(0.865143,0.566733,0); rgb(149pt)=(0.869366,0.568701,0); rgb(150pt)=(0.873592,0.57066,0); rgb(151pt)=(0.877819,0.572608,0); rgb(152pt)=(0.882047,0.574545,0); rgb(153pt)=(0.886275,0.576471,0); rgb(154pt)=(0.890659,0.578362,0); rgb(155pt)=(0.895333,0.580203,0); rgb(156pt)=(0.900258,0.581999,0); rgb(157pt)=(0.905397,0.583755,0); rgb(158pt)=(0.910711,0.585479,0); rgb(159pt)=(0.916164,0.587176,0); rgb(160pt)=(0.921717,0.588852,0); rgb(161pt)=(0.927333,0.590513,0); rgb(162pt)=(0.932974,0.592166,0); rgb(163pt)=(0.938602,0.593815,0); rgb(164pt)=(0.94418,0.595468,0); rgb(165pt)=(0.949669,0.59713,0); rgb(166pt)=(0.955033,0.598808,0); rgb(167pt)=(0.960233,0.600507,0); rgb(168pt)=(0.965232,0.602233,0); rgb(169pt)=(0.969992,0.603992,0); rgb(170pt)=(0.974475,0.605791,0); rgb(171pt)=(0.978643,0.607636,0); rgb(172pt)=(0.98246,0.609532,0); rgb(173pt)=(0.985886,0.611486,0); rgb(174pt)=(0.988885,0.613503,0); rgb(175pt)=(0.991419,0.61559,0); rgb(176pt)=(0.99345,0.617753,0); rgb(177pt)=(0.99494,0.619997,0); rgb(178pt)=(0.995851,0.622329,0); rgb(179pt)=(0.996226,0.624763,0); rgb(180pt)=(0.996512,0.627352,0); rgb(181pt)=(0.996788,0.630095,0); rgb(182pt)=(0.997053,0.632982,0); rgb(183pt)=(0.997308,0.636004,0); rgb(184pt)=(0.997552,0.639152,0); rgb(185pt)=(0.997785,0.642416,0); rgb(186pt)=(0.998006,0.645786,0); rgb(187pt)=(0.998217,0.649253,0); rgb(188pt)=(0.998416,0.652807,0); rgb(189pt)=(0.998605,0.656439,0); rgb(190pt)=(0.998781,0.660138,0); rgb(191pt)=(0.998946,0.663897,0); rgb(192pt)=(0.9991,0.667704,0); rgb(193pt)=(0.999242,0.67155,0); rgb(194pt)=(0.999372,0.675427,0); rgb(195pt)=(0.99949,0.679323,0); rgb(196pt)=(0.999596,0.68323,0); rgb(197pt)=(0.99969,0.687139,0); rgb(198pt)=(0.999771,0.691039,0); rgb(199pt)=(0.999841,0.694921,0); rgb(200pt)=(0.999898,0.698775,0); rgb(201pt)=(0.999942,0.702592,0); rgb(202pt)=(0.999974,0.706363,0); rgb(203pt)=(0.999994,0.710077,0); rgb(204pt)=(1,0.713725,0); rgb(205pt)=(1,0.717341,0); rgb(206pt)=(1,0.720963,0); rgb(207pt)=(1,0.724591,0); rgb(208pt)=(1,0.728226,0); rgb(209pt)=(1,0.731867,0); rgb(210pt)=(1,0.735514,0); rgb(211pt)=(1,0.739167,0); rgb(212pt)=(1,0.742827,0); rgb(213pt)=(1,0.746493,0); rgb(214pt)=(1,0.750165,0); rgb(215pt)=(1,0.753843,0); rgb(216pt)=(1,0.757527,0); rgb(217pt)=(1,0.761217,0); rgb(218pt)=(1,0.764913,0); rgb(219pt)=(1,0.768615,0); rgb(220pt)=(1,0.772324,0); rgb(221pt)=(1,0.776038,0); rgb(222pt)=(1,0.779758,0); rgb(223pt)=(1,0.783484,0); rgb(224pt)=(1,0.787215,0); rgb(225pt)=(1,0.790953,0); rgb(226pt)=(1,0.794696,0); rgb(227pt)=(1,0.798445,0); rgb(228pt)=(1,0.8022,0); rgb(229pt)=(1,0.805961,0); rgb(230pt)=(1,0.809727,0); rgb(231pt)=(1,0.8135,0); rgb(232pt)=(1,0.817278,0); rgb(233pt)=(1,0.821063,0); rgb(234pt)=(1,0.824854,0); rgb(235pt)=(1,0.828652,0); rgb(236pt)=(1,0.832455,0); rgb(237pt)=(1,0.836265,0); rgb(238pt)=(1,0.840081,0); rgb(239pt)=(1,0.843903,0); rgb(240pt)=(1,0.847732,0); rgb(241pt)=(1,0.851566,0); rgb(242pt)=(1,0.855406,0); rgb(243pt)=(1,0.859253,0); rgb(244pt)=(1,0.863106,0); rgb(245pt)=(1,0.866964,0); rgb(246pt)=(1,0.870829,0); rgb(247pt)=(1,0.8747,0); rgb(248pt)=(1,0.878577,0); rgb(249pt)=(1,0.88246,0); rgb(250pt)=(1,0.886349,0); rgb(251pt)=(1,0.890243,0); rgb(252pt)=(1,0.894144,0); rgb(253pt)=(1,0.898051,0); rgb(254pt)=(1,0.901964,0); rgb(255pt)=(1,0.905882,0)},
mesh/rows=49]
table[row sep=crcr,header=false] {%
%
56	0.093	5.74613142666568\\
56	0.09666	6.17235695419101\\
56	0.10032	6.66497241887921\\
56	0.10398	7.22397782073029\\
56	0.10764	7.84937315974424\\
56	0.1113	8.54115843592107\\
56	0.11496	9.29933364926077\\
56	0.11862	10.1238987997633\\
56	0.12228	11.0148538874288\\
56	0.12594	11.9721989122571\\
56	0.1296	12.9959338742483\\
56	0.13326	14.0860587734024\\
56	0.13692	15.2425736097193\\
56	0.14058	16.4654783831992\\
56	0.14424	17.7547730938419\\
56	0.1479	19.1104577416474\\
56	0.15156	20.5325323266159\\
56	0.15522	22.0209968487472\\
56	0.15888	23.5758513080414\\
56	0.16254	25.1970957044985\\
56	0.1662	26.8847300381184\\
56	0.16986	28.6387543089012\\
56	0.17352	30.4591685168469\\
56	0.17718	32.3459726619555\\
56	0.18084	34.2991667442269\\
56	0.1845	36.3187507636612\\
56	0.18816	38.4047247202584\\
56	0.19182	40.5570886140185\\
56	0.19548	42.7758424449414\\
56	0.19914	45.0609862130273\\
56	0.2028	47.412519918276\\
56	0.20646	49.8304435606875\\
56	0.21012	52.314757140262\\
56	0.21378	54.8654606569993\\
56	0.21744	57.4825541108994\\
56	0.2211	60.1660375019625\\
56	0.22476	62.9159108301884\\
56	0.22842	65.7321740955772\\
56	0.23208	68.6148272981289\\
56	0.23574	71.5638704378435\\
56	0.2394	74.5793035147209\\
56	0.24306	77.6611265287612\\
56	0.24672	80.8093394799644\\
56	0.25038	84.0239423683305\\
56	0.25404	87.3049351938594\\
56	0.2577	90.6523179565513\\
56	0.26136	94.0660906564059\\
56	0.26502	97.5462532934235\\
56	0.26868	101.092805867604\\
56	0.27234	104.705748378947\\
56	0.276	108.385080827453\\
56.375	0.093	5.74291435535881\\
56.375	0.09666	6.16955702951079\\
56.375	0.10032	6.66258964082565\\
56.375	0.10398	7.22201218930338\\
56.375	0.10764	7.847824674944\\
56.375	0.1113	8.54002709774747\\
56.375	0.11496	9.29861945771383\\
56.375	0.11862	10.1236017548431\\
56.375	0.12228	11.0149739891352\\
56.375	0.12594	11.9727361605902\\
56.375	0.1296	12.996888269208\\
56.375	0.13326	14.0874303149887\\
56.375	0.13692	15.2443622979323\\
56.375	0.14058	16.4676842180388\\
56.375	0.14424	17.7573960753082\\
56.375	0.1479	19.1134978697404\\
56.375	0.15156	20.5359896013355\\
56.375	0.15522	22.0248712700935\\
56.375	0.15888	23.5801428760143\\
56.375	0.16254	25.201804419098\\
56.375	0.1662	26.8898558993446\\
56.375	0.16986	28.6442973167541\\
56.375	0.17352	30.4651286713265\\
56.375	0.17718	32.3523499630617\\
56.375	0.18084	34.3059611919598\\
56.375	0.1845	36.3259623580207\\
56.375	0.18816	38.4123534612446\\
56.375	0.19182	40.5651345016313\\
56.375	0.19548	42.7843054791809\\
56.375	0.19914	45.0698663938934\\
56.375	0.2028	47.4218172457687\\
56.375	0.20646	49.840158034807\\
56.375	0.21012	52.324888761008\\
56.375	0.21378	54.876009424372\\
56.375	0.21744	57.4935200248988\\
56.375	0.2211	60.1774205625886\\
56.375	0.22476	62.9277110374411\\
56.375	0.22842	65.7443914494566\\
56.375	0.23208	68.6274617986349\\
56.375	0.23574	71.5769220849762\\
56.375	0.2394	74.5927723084802\\
56.375	0.24306	77.6750124691472\\
56.375	0.24672	80.823642566977\\
56.375	0.25038	84.0386626019698\\
56.375	0.25404	87.3200725741253\\
56.375	0.2577	90.6678724834438\\
56.375	0.26136	94.0820623299252\\
56.375	0.26502	97.5626421135694\\
56.375	0.26868	101.109611834376\\
56.375	0.27234	104.722971492346\\
56.375	0.276	108.402721087479\\
56.75	0.093	5.74343965323961\\
56.75	0.09666	6.17049947401825\\
56.75	0.10032	6.66394923195977\\
56.75	0.10398	7.22378892706414\\
56.75	0.10764	7.85001855933142\\
56.75	0.1113	8.54263812876155\\
56.75	0.11496	9.30164763535456\\
56.75	0.11862	10.1270470791104\\
56.75	0.12228	11.0188364600292\\
56.75	0.12594	11.9770157781109\\
56.75	0.1296	13.0015850333553\\
56.75	0.13326	14.0925442257627\\
56.75	0.13692	15.249893355333\\
56.75	0.14058	16.4736324220661\\
56.75	0.14424	17.7637614259621\\
56.75	0.1479	19.120280367021\\
56.75	0.15156	20.5431892452428\\
56.75	0.15522	22.0324880606274\\
56.75	0.15888	23.5881768131749\\
56.75	0.16254	25.2102555028853\\
56.75	0.1662	26.8987241297585\\
56.75	0.16986	28.6535826937947\\
56.75	0.17352	30.4748311949937\\
56.75	0.17718	32.3624696333555\\
56.75	0.18084	34.3164980088803\\
56.75	0.1845	36.3369163215679\\
56.75	0.18816	38.4237245714184\\
56.75	0.19182	40.5769227584318\\
56.75	0.19548	42.796510882608\\
56.75	0.19914	45.0824889439472\\
56.75	0.2028	47.4348569424492\\
56.75	0.20646	49.8536148781141\\
56.75	0.21012	52.3387627509418\\
56.75	0.21378	54.8903005609324\\
56.75	0.21744	57.5082283080859\\
56.75	0.2211	60.1925459924023\\
56.75	0.22476	62.9432536138815\\
56.75	0.22842	65.7603511725236\\
56.75	0.23208	68.6438386683286\\
56.75	0.23574	71.5937161012965\\
56.75	0.2394	74.6099834714272\\
56.75	0.24306	77.6926407787209\\
56.75	0.24672	80.8416880231773\\
56.75	0.25038	84.0571252047967\\
56.75	0.25404	87.338952323579\\
56.75	0.2577	90.6871693795241\\
56.75	0.26136	94.1017763726321\\
56.75	0.26502	97.5827733029029\\
56.75	0.26868	101.130160170337\\
56.75	0.27234	104.743936974933\\
56.75	0.276	108.424103716693\\
57.125	0.093	5.74770732030808\\
57.125	0.09666	6.17518428771337\\
57.125	0.10032	6.66905119228155\\
57.125	0.10398	7.22930803401258\\
57.125	0.10764	7.85595481290651\\
57.125	0.1113	8.54899152896329\\
57.125	0.11496	9.30841818218295\\
57.125	0.11862	10.1342347725655\\
57.125	0.12228	11.0264413001109\\
57.125	0.12594	11.9850377648192\\
57.125	0.1296	13.0100241666904\\
57.125	0.13326	14.1014005057244\\
57.125	0.13692	15.2591667819213\\
57.125	0.14058	16.4833229952811\\
57.125	0.14424	17.7738691458038\\
57.125	0.1479	19.1308052334893\\
57.125	0.15156	20.5541312583377\\
57.125	0.15522	22.043847220349\\
57.125	0.15888	23.5999531195232\\
57.125	0.16254	25.2224489558602\\
57.125	0.1662	26.9113347293601\\
57.125	0.16986	28.6666104400229\\
57.125	0.17352	30.4882760878485\\
57.125	0.17718	32.376331672837\\
57.125	0.18084	34.3307771949885\\
57.125	0.1845	36.3516126543027\\
57.125	0.18816	38.4388380507799\\
57.125	0.19182	40.5924533844199\\
57.125	0.19548	42.8124586552228\\
57.125	0.19914	45.0988538631886\\
57.125	0.2028	47.4516390083173\\
57.125	0.20646	49.8708140906088\\
57.125	0.21012	52.3563791100632\\
57.125	0.21378	54.9083340666805\\
57.125	0.21744	57.5266789604606\\
57.125	0.2211	60.2114137914037\\
57.125	0.22476	62.9625385595095\\
57.125	0.22842	65.7800532647783\\
57.125	0.23208	68.6639579072099\\
57.125	0.23574	71.6142524868045\\
57.125	0.2394	74.6309370035619\\
57.125	0.24306	77.7140114574822\\
57.125	0.24672	80.8634758485653\\
57.125	0.25038	84.0793301768114\\
57.125	0.25404	87.3615744422202\\
57.125	0.2577	90.710208644792\\
57.125	0.26136	94.1252327845267\\
57.125	0.26502	97.6066468614242\\
57.125	0.26868	101.154450875485\\
57.125	0.27234	104.768644826708\\
57.125	0.276	108.449228715094\\
57.5	0.093	5.75571735656422\\
57.5	0.09666	6.18361147059617\\
57.5	0.10032	6.677895521791\\
57.5	0.10398	7.23856951014869\\
57.5	0.10764	7.86563343566927\\
57.5	0.1113	8.55908729835271\\
57.5	0.11496	9.31893109819903\\
57.5	0.11862	10.1451648352082\\
57.5	0.12228	11.0377885093803\\
57.5	0.12594	11.9968021207152\\
57.5	0.1296	13.022205669213\\
57.5	0.13326	14.1139991548737\\
57.5	0.13692	15.2721825776973\\
57.5	0.14058	16.4967559376837\\
57.5	0.14424	17.7877192348331\\
57.5	0.1479	19.1450724691453\\
57.5	0.15156	20.5688156406203\\
57.5	0.15522	22.0589487492583\\
57.5	0.15888	23.6154717950591\\
57.5	0.16254	25.2383847780228\\
57.5	0.1662	26.9276876981493\\
57.5	0.16986	28.6833805554388\\
57.5	0.17352	30.5054633498911\\
57.5	0.17718	32.3939360815063\\
57.5	0.18084	34.3487987502843\\
57.5	0.1845	36.3700513562252\\
57.5	0.18816	38.4576938993291\\
57.5	0.19182	40.6117263795957\\
57.5	0.19548	42.8321487970253\\
57.5	0.19914	45.1189611516177\\
57.5	0.2028	47.4721634433731\\
57.5	0.20646	49.8917556722913\\
57.5	0.21012	52.3777378383723\\
57.5	0.21378	54.9301099416162\\
57.5	0.21744	57.548871982023\\
57.5	0.2211	60.2340239595927\\
57.5	0.22476	62.9855658743252\\
57.5	0.22842	65.8034977262207\\
57.5	0.23208	68.687819515279\\
57.5	0.23574	71.6385312415002\\
57.5	0.2394	74.6556329048842\\
57.5	0.24306	77.7391245054311\\
57.5	0.24672	80.8890060431409\\
57.5	0.25038	84.1052775180137\\
57.5	0.25404	87.3879389300492\\
57.5	0.2577	90.7369902792476\\
57.5	0.26136	94.1524315656089\\
57.5	0.26502	97.6342627891331\\
57.5	0.26868	101.18248394982\\
57.5	0.27234	104.79709504767\\
57.5	0.276	108.478096082683\\
57.875	0.093	5.76746976200801\\
57.875	0.09666	6.19578102266661\\
57.875	0.10032	6.69048222048808\\
57.875	0.10398	7.25157335547244\\
57.875	0.10764	7.87905442761965\\
57.875	0.1113	8.57292543692976\\
57.875	0.11496	9.33318638340273\\
57.875	0.11862	10.1598372670386\\
57.875	0.12228	11.0528780878373\\
57.875	0.12594	12.0123088457989\\
57.875	0.1296	13.0381295409234\\
57.875	0.13326	14.1303401732107\\
57.875	0.13692	15.2889407426609\\
57.875	0.14058	16.5139312492741\\
57.875	0.14424	17.80531169305\\
57.875	0.1479	19.1630820739889\\
57.875	0.15156	20.5872423920906\\
57.875	0.15522	22.0777926473552\\
57.875	0.15888	23.6347328397826\\
57.875	0.16254	25.258062969373\\
57.875	0.1662	26.9477830361262\\
57.875	0.16986	28.7038930400423\\
57.875	0.17352	30.5263929811213\\
57.875	0.17718	32.4152828593631\\
57.875	0.18084	34.3705626747678\\
57.875	0.1845	36.3922324273354\\
57.875	0.18816	38.4802921170659\\
57.875	0.19182	40.6347417439592\\
57.875	0.19548	42.8555813080154\\
57.875	0.19914	45.1428108092345\\
57.875	0.2028	47.4964302476165\\
57.875	0.20646	49.9164396231613\\
57.875	0.21012	52.402838935869\\
57.875	0.21378	54.9556281857396\\
57.875	0.21744	57.5748073727731\\
57.875	0.2211	60.2603764969694\\
57.875	0.22476	63.0123355583286\\
57.875	0.22842	65.8306845568507\\
57.875	0.23208	68.7154234925356\\
57.875	0.23574	71.6665523653835\\
57.875	0.2394	74.6840711753942\\
57.875	0.24306	77.7679799225678\\
57.875	0.24672	80.9182786069042\\
57.875	0.25038	84.1349672284036\\
57.875	0.25404	87.4180457870658\\
57.875	0.2577	90.7675142828909\\
57.875	0.26136	94.1833727158788\\
57.875	0.26502	97.6656210860296\\
57.875	0.26868	101.214259393343\\
57.875	0.27234	104.82928763782\\
57.875	0.276	108.510705819459\\
58.25	0.093	5.78296453663947\\
58.25	0.09666	6.21169294392473\\
58.25	0.10032	6.70681128837286\\
58.25	0.10398	7.26831956998387\\
58.25	0.10764	7.89621778875775\\
58.25	0.1113	8.59050594469451\\
58.25	0.11496	9.35118403779413\\
58.25	0.11862	10.1782520680566\\
58.25	0.12228	11.071710035482\\
58.25	0.12594	12.0315579400703\\
58.25	0.1296	13.0577957818214\\
58.25	0.13326	14.1504235607354\\
58.25	0.13692	15.3094412768123\\
58.25	0.14058	16.534848930052\\
58.25	0.14424	17.8266465204546\\
58.25	0.1479	19.1848340480201\\
58.25	0.15156	20.6094115127485\\
58.25	0.15522	22.1003789146398\\
58.25	0.15888	23.6577362536939\\
58.25	0.16254	25.2814835299109\\
58.25	0.1662	26.9716207432908\\
58.25	0.16986	28.7281478938335\\
58.25	0.17352	30.5510649815391\\
58.25	0.17718	32.4403720064076\\
58.25	0.18084	34.396068968439\\
58.25	0.1845	36.4181558676332\\
58.25	0.18816	38.5066327039904\\
58.25	0.19182	40.6614994775103\\
58.25	0.19548	42.8827561881932\\
58.25	0.19914	45.170402836039\\
58.25	0.2028	47.5244394210476\\
58.25	0.20646	49.9448659432191\\
58.25	0.21012	52.4316824025534\\
58.25	0.21378	54.9848887990507\\
58.25	0.21744	57.6044851327108\\
58.25	0.2211	60.2904714035338\\
58.25	0.22476	63.0428476115196\\
58.25	0.22842	65.8616137566684\\
58.25	0.23208	68.74676983898\\
58.25	0.23574	71.6983158584545\\
58.25	0.2394	74.7162518150918\\
58.25	0.24306	77.8005777088921\\
58.25	0.24672	80.9512935398552\\
58.25	0.25038	84.1683993079812\\
58.25	0.25404	87.45189501327\\
58.25	0.2577	90.8017806557218\\
58.25	0.26136	94.2180562353364\\
58.25	0.26502	97.7007217521139\\
58.25	0.26868	101.249777206054\\
58.25	0.27234	104.865222597157\\
58.25	0.276	108.547057925424\\
58.625	0.093	5.80220168045861\\
58.625	0.09666	6.23134723437052\\
58.625	0.10032	6.7268827254453\\
58.625	0.10398	7.28880815368297\\
58.625	0.10764	7.9171235190835\\
58.625	0.1113	8.61182882164692\\
58.625	0.11496	9.37292406137319\\
58.625	0.11862	10.2004092382623\\
58.625	0.12228	11.0942843523144\\
58.625	0.12594	12.0545494035293\\
58.625	0.1296	13.0812043919071\\
58.625	0.13326	14.1742493174477\\
58.625	0.13692	15.3336841801513\\
58.625	0.14058	16.5595089800177\\
58.625	0.14424	17.851723717047\\
58.625	0.1479	19.2103283912391\\
58.625	0.15156	20.6353230025941\\
58.625	0.15522	22.126707551112\\
58.625	0.15888	23.6844820367928\\
58.625	0.16254	25.3086464596365\\
58.625	0.1662	26.999200819643\\
58.625	0.16986	28.7561451168124\\
58.625	0.17352	30.5794793511447\\
58.625	0.17718	32.4692035226398\\
58.625	0.18084	34.4253176312978\\
58.625	0.1845	36.4478216771187\\
58.625	0.18816	38.5367156601025\\
58.625	0.19182	40.6919995802492\\
58.625	0.19548	42.9136734375587\\
58.625	0.19914	45.2017372320311\\
58.625	0.2028	47.5561909636664\\
58.625	0.20646	49.9770346324645\\
58.625	0.21012	52.4642682384255\\
58.625	0.21378	55.0178917815494\\
58.625	0.21744	57.6379052618362\\
58.625	0.2211	60.3243086792858\\
58.625	0.22476	63.0771020338983\\
58.625	0.22842	65.8962853256737\\
58.625	0.23208	68.781858554612\\
58.625	0.23574	71.7338217207131\\
58.625	0.2394	74.7521748239772\\
58.625	0.24306	77.8369178644041\\
58.625	0.24672	80.9880508419938\\
58.625	0.25038	84.2055737567465\\
58.625	0.25404	87.489486608662\\
58.625	0.2577	90.8397893977404\\
58.625	0.26136	94.2564821239816\\
58.625	0.26502	97.7395647873858\\
58.625	0.26868	101.289037387953\\
58.625	0.27234	104.904899925683\\
58.625	0.276	108.587152400575\\
59	0.093	5.8251811934654\\
59	0.09666	6.25474389400397\\
59	0.10032	6.7506965317054\\
59	0.10398	7.31303910656973\\
59	0.10764	7.94177161859691\\
59	0.1113	8.63689406778698\\
59	0.11496	9.39840645413992\\
59	0.11862	10.2263087776557\\
59	0.12228	11.1206010383344\\
59	0.12594	12.081283236176\\
59	0.1296	13.1083553711804\\
59	0.13326	14.2018174433477\\
59	0.13692	15.3616694526779\\
59	0.14058	16.587911399171\\
59	0.14424	17.8805432828269\\
59	0.1479	19.2395651036457\\
59	0.15156	20.6649768616274\\
59	0.15522	22.1567785567719\\
59	0.15888	23.7149701890794\\
59	0.16254	25.3395517585497\\
59	0.1662	27.0305232651829\\
59	0.16986	28.7878847089789\\
59	0.17352	30.6116360899379\\
59	0.17718	32.5017774080597\\
59	0.18084	34.4583086633443\\
59	0.1845	36.4812298557919\\
59	0.18816	38.5705409854023\\
59	0.19182	40.7262420521756\\
59	0.19548	42.9483330561118\\
59	0.19914	45.2368139972108\\
59	0.2028	47.5916848754728\\
59	0.20646	50.0129456908976\\
59	0.21012	52.5005964434853\\
59	0.21378	55.0546371332358\\
59	0.21744	57.6750677601492\\
59	0.2211	60.3618883242255\\
59	0.22476	63.1150988254647\\
59	0.22842	65.9346992638667\\
59	0.23208	68.8206896394316\\
59	0.23574	71.7730699521595\\
59	0.2394	74.7918402020501\\
59	0.24306	77.8770003891037\\
59	0.24672	81.0285505133201\\
59	0.25038	84.2464905746994\\
59	0.25404	87.5308205732416\\
59	0.2577	90.8815405089466\\
59	0.26136	94.2986503818145\\
59	0.26502	97.7821501918453\\
59	0.26868	101.332039939039\\
59	0.27234	104.948319623396\\
59	0.276	108.630989244915\\
59.375	0.093	5.85190307565984\\
59.375	0.09666	6.28188292282507\\
59.375	0.10032	6.77825270715317\\
59.375	0.10398	7.34101242864412\\
59.375	0.10764	7.97016208729798\\
59.375	0.1113	8.66570168311469\\
59.375	0.11496	9.42763121609428\\
59.375	0.11862	10.2559506862368\\
59.375	0.12228	11.1506600935421\\
59.375	0.12594	12.1117594380103\\
59.375	0.1296	13.1392487196414\\
59.375	0.13326	14.2331279384354\\
59.375	0.13692	15.3933970943922\\
59.375	0.14058	16.6200561875119\\
59.375	0.14424	17.9131052177945\\
59.375	0.1479	19.27254418524\\
59.375	0.15156	20.6983730898483\\
59.375	0.15522	22.1905919316195\\
59.375	0.15888	23.7492007105536\\
59.375	0.16254	25.3741994266506\\
59.375	0.1662	27.0655880799104\\
59.375	0.16986	28.8233666703331\\
59.375	0.17352	30.6475351979187\\
59.375	0.17718	32.5380936626672\\
59.375	0.18084	34.4950420645785\\
59.375	0.1845	36.5183804036527\\
59.375	0.18816	38.6081086798898\\
59.375	0.19182	40.7642268932897\\
59.375	0.19548	42.9867350438526\\
59.375	0.19914	45.2756331315783\\
59.375	0.2028	47.6309211564669\\
59.375	0.20646	50.0525991185183\\
59.375	0.21012	52.5406670177326\\
59.375	0.21378	55.0951248541098\\
59.375	0.21744	57.7159726276499\\
59.375	0.2211	60.4032103383529\\
59.375	0.22476	63.1568379862187\\
59.375	0.22842	65.9768555712474\\
59.375	0.23208	68.863263093439\\
59.375	0.23574	71.8160605527934\\
59.375	0.2394	74.8352479493108\\
59.375	0.24306	77.920825282991\\
59.375	0.24672	81.072792553834\\
59.375	0.25038	84.29114976184\\
59.375	0.25404	87.5758969070088\\
59.375	0.2577	90.9270339893405\\
59.375	0.26136	94.3445610088351\\
59.375	0.26502	97.8284779654925\\
59.375	0.26868	101.378784859313\\
59.375	0.27234	104.995481690296\\
59.375	0.276	108.678568458442\\
59.75	0.093	5.88236732704198\\
59.75	0.09666	6.31276432083386\\
59.75	0.10032	6.80955125178861\\
59.75	0.10398	7.37272811990622\\
59.75	0.10764	8.00229492518674\\
59.75	0.1113	8.6982516676301\\
59.75	0.11496	9.46059834723635\\
59.75	0.11862	10.2893349640055\\
59.75	0.12228	11.1844615179375\\
59.75	0.12594	12.1459780090323\\
59.75	0.1296	13.1738844372901\\
59.75	0.13326	14.2681808027107\\
59.75	0.13692	15.4288671052942\\
59.75	0.14058	16.6559433450406\\
59.75	0.14424	17.9494095219498\\
59.75	0.1479	19.3092656360219\\
59.75	0.15156	20.7355116872569\\
59.75	0.15522	22.2281476756548\\
59.75	0.15888	23.7871736012155\\
59.75	0.16254	25.4125894639391\\
59.75	0.1662	27.1043952638256\\
59.75	0.16986	28.862591000875\\
59.75	0.17352	30.6871766750873\\
59.75	0.17718	32.5781522864623\\
59.75	0.18084	34.5355178350003\\
59.75	0.1845	36.5592733207012\\
59.75	0.18816	38.6494187435649\\
59.75	0.19182	40.8059541035915\\
59.75	0.19548	43.028879400781\\
59.75	0.19914	45.3181946351334\\
59.75	0.2028	47.6738998066486\\
59.75	0.20646	50.0959949153268\\
59.75	0.21012	52.5844799611677\\
59.75	0.21378	55.1393549441716\\
59.75	0.21744	57.7606198643383\\
59.75	0.2211	60.4482747216679\\
59.75	0.22476	63.2023195161604\\
59.75	0.22842	66.0227542478158\\
59.75	0.23208	68.909578916634\\
59.75	0.23574	71.8627935226151\\
59.75	0.2394	74.8823980657591\\
59.75	0.24306	77.9683925460659\\
59.75	0.24672	81.1207769635357\\
59.75	0.25038	84.3395513181683\\
59.75	0.25404	87.6247156099637\\
59.75	0.2577	90.9762698389221\\
59.75	0.26136	94.3942140050434\\
59.75	0.26502	97.8785481083274\\
59.75	0.26868	101.429272148774\\
59.75	0.27234	105.046386126384\\
59.75	0.276	108.729890041157\\
60.125	0.093	5.91657394761176\\
60.125	0.09666	6.3473880880303\\
60.125	0.10032	6.8445921656117\\
60.125	0.10398	7.40818618035596\\
60.125	0.10764	8.03817013226314\\
60.125	0.1113	8.73454402133316\\
60.125	0.11496	9.49730784756606\\
60.125	0.11862	10.3264616109618\\
60.125	0.12228	11.2220053115205\\
60.125	0.12594	12.183938949242\\
60.125	0.1296	13.2122625241264\\
60.125	0.13326	14.3069760361737\\
60.125	0.13692	15.4680794853838\\
60.125	0.14058	16.6955728717569\\
60.125	0.14424	17.9894561952928\\
60.125	0.1479	19.3497294559915\\
60.125	0.15156	20.7763926538532\\
60.125	0.15522	22.2694457888777\\
60.125	0.15888	23.8288888610651\\
60.125	0.16254	25.4547218704154\\
60.125	0.1662	27.1469448169285\\
60.125	0.16986	28.9055577006045\\
60.125	0.17352	30.7305605214434\\
60.125	0.17718	32.6219532794452\\
60.125	0.18084	34.5797359746098\\
60.125	0.1845	36.6039086069373\\
60.125	0.18816	38.6944711764277\\
60.125	0.19182	40.851423683081\\
60.125	0.19548	43.0747661268971\\
60.125	0.19914	45.3644985078762\\
60.125	0.2028	47.7206208260181\\
60.125	0.20646	50.1431330813228\\
60.125	0.21012	52.6320352737905\\
60.125	0.21378	55.187327403421\\
60.125	0.21744	57.8090094702144\\
60.125	0.2211	60.4970814741706\\
60.125	0.22476	63.2515434152898\\
60.125	0.22842	66.0723952935718\\
60.125	0.23208	68.9596371090166\\
60.125	0.23574	71.9132688616244\\
60.125	0.2394	74.9332905513951\\
60.125	0.24306	78.0197021783286\\
60.125	0.24672	81.1725037424249\\
60.125	0.25038	84.3916952436842\\
60.125	0.25404	87.6772766821063\\
60.125	0.2577	91.0292480576914\\
60.125	0.26136	94.4476093704392\\
60.125	0.26502	97.93236062035\\
60.125	0.26868	101.483501807424\\
60.125	0.27234	105.10103293166\\
60.125	0.276	108.78495399306\\
60.5	0.093	5.95452293736921\\
60.5	0.09666	6.3857542244144\\
60.5	0.10032	6.88337544862247\\
60.5	0.10398	7.44738660999339\\
60.5	0.10764	8.07778770852721\\
60.5	0.1113	8.77457874422389\\
60.5	0.11496	9.53775971708345\\
60.5	0.11862	10.3673306271059\\
60.5	0.12228	11.2632914742912\\
60.5	0.12594	12.2256422586394\\
60.5	0.1296	13.2543829801504\\
60.5	0.13326	14.3495136388244\\
60.5	0.13692	15.5110342346611\\
60.5	0.14058	16.7389447676608\\
60.5	0.14424	18.0332452378234\\
60.5	0.1479	19.3939356451488\\
60.5	0.15156	20.8210159896371\\
60.5	0.15522	22.3144862712883\\
60.5	0.15888	23.8743464901023\\
60.5	0.16254	25.5005966460793\\
60.5	0.1662	27.1932367392191\\
60.5	0.16986	28.9522667695217\\
60.5	0.17352	30.7776867369873\\
60.5	0.17718	32.6694966416157\\
60.5	0.18084	34.627696483407\\
60.5	0.1845	36.6522862623612\\
60.5	0.18816	38.7432659784782\\
60.5	0.19182	40.9006356317581\\
60.5	0.19548	43.1243952222009\\
60.5	0.19914	45.4145447498066\\
60.5	0.2028	47.7710842145752\\
60.5	0.20646	50.1940136165066\\
60.5	0.21012	52.6833329556009\\
60.5	0.21378	55.239042231858\\
60.5	0.21744	57.8611414452781\\
60.5	0.2211	60.549630595861\\
60.5	0.22476	63.3045096836068\\
60.5	0.22842	66.1257787085154\\
60.5	0.23208	69.013437670587\\
60.5	0.23574	71.9674865698214\\
60.5	0.2394	74.9879254062187\\
60.5	0.24306	78.0747541797789\\
60.5	0.24672	81.2279728905019\\
60.5	0.25038	84.4475815383878\\
60.5	0.25404	87.7335801234366\\
60.5	0.2577	91.0859686456483\\
60.5	0.26136	94.5047471050228\\
60.5	0.26502	97.9899155015602\\
60.5	0.26868	101.54147383526\\
60.5	0.27234	105.159422106124\\
60.5	0.276	108.84376031415\\
60.875	0.093	5.99621429631434\\
60.875	0.09666	6.42786272998619\\
60.875	0.10032	6.92590110082091\\
60.875	0.10398	7.49032940881848\\
60.875	0.10764	8.12114765397896\\
60.875	0.1113	8.81835583630229\\
60.875	0.11496	9.58195395578849\\
60.875	0.11862	10.4119420124376\\
60.875	0.12228	11.3083200062495\\
60.875	0.12594	12.2710879372244\\
60.875	0.1296	13.3002458053621\\
60.875	0.13326	14.3957936106627\\
60.875	0.13692	15.5577313531261\\
60.875	0.14058	16.7860590327524\\
60.875	0.14424	18.0807766495417\\
60.875	0.1479	19.4418842034937\\
60.875	0.15156	20.8693816946087\\
60.875	0.15522	22.3632691228865\\
60.875	0.15888	23.9235464883272\\
60.875	0.16254	25.5502137909308\\
60.875	0.1662	27.2432710306973\\
60.875	0.16986	29.0027182076266\\
60.875	0.17352	30.8285553217188\\
60.875	0.17718	32.7207823729739\\
60.875	0.18084	34.6793993613918\\
60.875	0.1845	36.7044062869727\\
60.875	0.18816	38.7958031497164\\
60.875	0.19182	40.9535899496229\\
60.875	0.19548	43.1777666866924\\
60.875	0.19914	45.4683333609247\\
60.875	0.2028	47.8252899723199\\
60.875	0.20646	50.248636520878\\
60.875	0.21012	52.7383730065989\\
60.875	0.21378	55.2944994294828\\
60.875	0.21744	57.9170157895294\\
60.875	0.2211	60.605922086739\\
60.875	0.22476	63.3612183211115\\
60.875	0.22842	66.1829044926468\\
60.875	0.23208	69.070980601345\\
60.875	0.23574	72.0254466472061\\
60.875	0.2394	75.04630263023\\
60.875	0.24306	78.1335485504168\\
60.875	0.24672	81.2871844077665\\
60.875	0.25038	84.5072102022791\\
60.875	0.25404	87.7936259339545\\
60.875	0.2577	91.1464316027929\\
60.875	0.26136	94.565627208794\\
60.875	0.26502	98.0512127519581\\
60.875	0.26868	101.603188232285\\
60.875	0.27234	105.221553649775\\
60.875	0.276	108.906309004428\\
61.25	0.093	6.04164802444712\\
61.25	0.09666	6.47371360474562\\
61.25	0.10032	6.97216912220697\\
61.25	0.10398	7.53701457683123\\
61.25	0.10764	8.16824996861833\\
61.25	0.1113	8.86587529756834\\
61.25	0.11496	9.6298905636812\\
61.25	0.11862	10.4602957669569\\
61.25	0.12228	11.3570909073956\\
61.25	0.12594	12.320275984997\\
61.25	0.1296	13.3498509997614\\
61.25	0.13326	14.4458159516886\\
61.25	0.13692	15.6081708407787\\
61.25	0.14058	16.8369156670317\\
61.25	0.14424	18.1320504304476\\
61.25	0.1479	19.4935751310263\\
61.25	0.15156	20.9214897687679\\
61.25	0.15522	22.4157943436724\\
61.25	0.15888	23.9764888557398\\
61.25	0.16254	25.60357330497\\
61.25	0.1662	27.2970476913631\\
61.25	0.16986	29.0569120149191\\
61.25	0.17352	30.883166275638\\
61.25	0.17718	32.7758104735197\\
61.25	0.18084	34.7348446085643\\
61.25	0.1845	36.7602686807718\\
61.25	0.18816	38.8520826901422\\
61.25	0.19182	41.0102866366754\\
61.25	0.19548	43.2348805203715\\
61.25	0.19914	45.5258643412305\\
61.25	0.2028	47.8832380992523\\
61.25	0.20646	50.3070017944371\\
61.25	0.21012	52.7971554267847\\
61.25	0.21378	55.3536989962951\\
61.25	0.21744	57.9766325029685\\
61.25	0.2211	60.6659559468047\\
61.25	0.22476	63.4216693278038\\
61.25	0.22842	66.2437726459658\\
61.25	0.23208	69.1322659012906\\
61.25	0.23574	72.0871490937784\\
61.25	0.2394	75.108422223429\\
61.25	0.24306	78.1960852902424\\
61.25	0.24672	81.3501382942188\\
61.25	0.25038	84.570581235358\\
61.25	0.25404	87.8574141136601\\
61.25	0.2577	91.2106369291251\\
61.25	0.26136	94.6302496817529\\
61.25	0.26502	98.1162523715436\\
61.25	0.26868	101.668644998497\\
61.25	0.27234	105.287427562614\\
61.25	0.276	108.972600063893\\
61.625	0.093	6.09082412176756\\
61.625	0.09666	6.52330684869272\\
61.625	0.10032	7.02217951278073\\
61.625	0.10398	7.58744211403165\\
61.625	0.10764	8.21909465244541\\
61.625	0.1113	8.91713712802206\\
61.625	0.11496	9.68156954076157\\
61.625	0.11862	10.512391890664\\
61.625	0.12228	11.4096041777292\\
61.625	0.12594	12.3732064019574\\
61.625	0.1296	13.4031985633484\\
61.625	0.13326	14.4995806619023\\
61.625	0.13692	15.662352697619\\
61.625	0.14058	16.8915146704987\\
61.625	0.14424	18.1870665805412\\
61.625	0.1479	19.5490084277466\\
61.625	0.15156	20.9773402121149\\
61.625	0.15522	22.472061933646\\
61.625	0.15888	24.03317359234\\
61.625	0.16254	25.6606751881969\\
61.625	0.1662	27.3545667212167\\
61.625	0.16986	29.1148481913993\\
61.625	0.17352	30.9415195987448\\
61.625	0.17718	32.8345809432532\\
61.625	0.18084	34.7940322249245\\
61.625	0.1845	36.8198734437586\\
61.625	0.18816	38.9121045997556\\
61.625	0.19182	41.0707256929155\\
61.625	0.19548	43.2957367232383\\
61.625	0.19914	45.5871376907239\\
61.625	0.2028	47.9449285953724\\
61.625	0.20646	50.3691094371838\\
61.625	0.21012	52.8596802161581\\
61.625	0.21378	55.4166409322952\\
61.625	0.21744	58.0399915855952\\
61.625	0.2211	60.7297321760581\\
61.625	0.22476	63.4858627036838\\
61.625	0.22842	66.3083831684724\\
61.625	0.23208	69.1972935704239\\
61.625	0.23574	72.1525939095383\\
61.625	0.2394	75.1742841858156\\
61.625	0.24306	78.2623643992557\\
61.625	0.24672	81.4168345498587\\
61.625	0.25038	84.6376946376246\\
61.625	0.25404	87.9249446625533\\
61.625	0.2577	91.278584624645\\
61.625	0.26136	94.6986145238995\\
61.625	0.26502	98.1850343603169\\
61.625	0.26868	101.737844133897\\
61.625	0.27234	105.35704384464\\
61.625	0.276	109.042633492546\\
62	0.093	6.14374258827568\\
62	0.09666	6.57664246182749\\
62	0.10032	7.07593227254216\\
62	0.10398	7.64161202041971\\
62	0.10764	8.27368170546014\\
62	0.1113	8.97214132766344\\
62	0.11496	9.73699088702962\\
62	0.11862	10.5682303835586\\
62	0.12228	11.4658598172506\\
62	0.12594	12.4298791881054\\
62	0.1296	13.4602884961231\\
62	0.13326	14.5570877413036\\
62	0.13692	15.720276923647\\
62	0.14058	16.9498560431533\\
62	0.14424	18.2458250998225\\
62	0.1479	19.6081840936545\\
62	0.15156	21.0369330246495\\
62	0.15522	22.5320718928073\\
62	0.15888	24.0936006981279\\
62	0.16254	25.7215194406115\\
62	0.1662	27.4158281202579\\
62	0.16986	29.1765267370672\\
62	0.17352	31.0036152910394\\
62	0.17718	32.8970937821744\\
62	0.18084	34.8569622104723\\
62	0.1845	36.8832205759331\\
62	0.18816	38.9758688785568\\
62	0.19182	41.1349071183433\\
62	0.19548	43.3603352952927\\
62	0.19914	45.652153409405\\
62	0.2028	48.0103614606802\\
62	0.20646	50.4349594491182\\
62	0.21012	52.9259473747191\\
62	0.21378	55.4833252374829\\
62	0.21744	58.1070930374096\\
62	0.2211	60.7972507744991\\
62	0.22476	63.5537984487515\\
62	0.22842	66.3767360601668\\
62	0.23208	69.2660636087449\\
62	0.23574	72.221781094486\\
62	0.2394	75.2438885173899\\
62	0.24306	78.3323858774567\\
62	0.24672	81.4872731746863\\
62	0.25038	84.7085504090789\\
62	0.25404	87.9962175806343\\
62	0.2577	91.3502746893526\\
62	0.26136	94.7707217352337\\
62	0.26502	98.2575587182777\\
62	0.26868	101.810785638485\\
62	0.27234	105.430402495854\\
62	0.276	109.116409290387\\
62.375	0.093	6.20040342397147\\
62.375	0.09666	6.63372044414993\\
62.375	0.10032	7.13342740149125\\
62.375	0.10398	7.69952429599547\\
62.375	0.10764	8.33201112766254\\
62.375	0.1113	9.03088789649249\\
62.375	0.11496	9.79615460248532\\
62.375	0.11862	10.627811245641\\
62.375	0.12228	11.5258578259596\\
62.375	0.12594	12.4902943434411\\
62.375	0.1296	13.5211207980854\\
62.375	0.13326	14.6183371898926\\
62.375	0.13692	15.7819435188627\\
62.375	0.14058	17.0119397849956\\
62.375	0.14424	18.3083259882914\\
62.375	0.1479	19.6711021287501\\
62.375	0.15156	21.1002682063717\\
62.375	0.15522	22.5958242211562\\
62.375	0.15888	24.1577701731035\\
62.375	0.16254	25.7861060622137\\
62.375	0.1662	27.4808318884868\\
62.375	0.16986	29.2419476519227\\
62.375	0.17352	31.0694533525215\\
62.375	0.17718	32.9633489902832\\
62.375	0.18084	34.9236345652078\\
62.375	0.1845	36.9503100772952\\
62.375	0.18816	39.0433755265456\\
62.375	0.19182	41.2028309129588\\
62.375	0.19548	43.4286762365348\\
62.375	0.19914	45.7209114972738\\
62.375	0.2028	48.0795366951756\\
62.375	0.20646	50.5045518302403\\
62.375	0.21012	52.9959569024678\\
62.375	0.21378	55.5537519118583\\
62.375	0.21744	58.1779368584116\\
62.375	0.2211	60.8685117421278\\
62.375	0.22476	63.6254765630068\\
62.375	0.22842	66.4488313210488\\
62.375	0.23208	69.3385760162536\\
62.375	0.23574	72.2947106486213\\
62.375	0.2394	75.3172352181519\\
62.375	0.24306	78.4061497248453\\
62.375	0.24672	81.5614541687016\\
62.375	0.25038	84.7831485497208\\
62.375	0.25404	88.0712328679028\\
62.375	0.2577	91.4257071232478\\
62.375	0.26136	94.8465713157556\\
62.375	0.26502	98.3338254454263\\
62.375	0.26868	101.88746951226\\
62.375	0.27234	105.507503516256\\
62.375	0.276	109.193927457416\\
62.75	0.093	6.26080662885489\\
62.75	0.09666	6.69454079566001\\
62.75	0.10032	7.19466489962801\\
62.75	0.10398	7.76117894075885\\
62.75	0.10764	8.3940829190526\\
62.75	0.1113	9.0933768345092\\
62.75	0.11496	9.85906068712869\\
62.75	0.11862	10.6911344769111\\
62.75	0.12228	11.5895982038563\\
62.75	0.12594	12.5544518679644\\
62.75	0.1296	13.5856954692354\\
62.75	0.13326	14.6833290076692\\
62.75	0.13692	15.8473524832659\\
62.75	0.14058	17.0777658960256\\
62.75	0.14424	18.374569245948\\
62.75	0.1479	19.7377625330334\\
62.75	0.15156	21.1673457572816\\
62.75	0.15522	22.6633189186927\\
62.75	0.15888	24.2256820172667\\
62.75	0.16254	25.8544350530036\\
62.75	0.1662	27.5495780259033\\
62.75	0.16986	29.3111109359659\\
62.75	0.17352	31.1390337831914\\
62.75	0.17718	33.0333465675797\\
62.75	0.18084	34.9940492891309\\
62.75	0.1845	37.021141947845\\
62.75	0.18816	39.114624543722\\
62.75	0.19182	41.2744970767618\\
62.75	0.19548	43.5007595469646\\
62.75	0.19914	45.7934119543302\\
62.75	0.2028	48.1524542988587\\
62.75	0.20646	50.57788658055\\
62.75	0.21012	53.0697087994042\\
62.75	0.21378	55.6279209554213\\
62.75	0.21744	58.2525230486013\\
62.75	0.2211	60.9435150789441\\
62.75	0.22476	63.7008970464498\\
62.75	0.22842	66.5246689511184\\
62.75	0.23208	69.4148307929499\\
62.75	0.23574	72.3713825719443\\
62.75	0.2394	75.3943242881015\\
62.75	0.24306	78.4836559414216\\
62.75	0.24672	81.6393775319045\\
62.75	0.25038	84.8614890595504\\
62.75	0.25404	88.1499905243591\\
62.75	0.2577	91.5048819263307\\
62.75	0.26136	94.9261632654652\\
62.75	0.26502	98.4138345417625\\
62.75	0.26868	101.967895755223\\
62.75	0.27234	105.588346905846\\
62.75	0.276	109.275187993632\\
63.125	0.093	6.324952202926\\
63.125	0.09666	6.75910351635778\\
63.125	0.10032	7.25964476695242\\
63.125	0.10398	7.82657595470992\\
63.125	0.10764	8.45989707963033\\
63.125	0.1113	9.15960814171358\\
63.125	0.11496	9.92570914095972\\
63.125	0.11862	10.7582000773687\\
63.125	0.12228	11.6570809509406\\
63.125	0.12594	12.6223517616754\\
63.125	0.1296	13.654012509573\\
63.125	0.13326	14.7520631946335\\
63.125	0.13692	15.9165038168569\\
63.125	0.14058	17.1473343762432\\
63.125	0.14424	18.4445548727923\\
63.125	0.1479	19.8081653065043\\
63.125	0.15156	21.2381656773792\\
63.125	0.15522	22.734555985417\\
63.125	0.15888	24.2973362306176\\
63.125	0.16254	25.9265064129811\\
63.125	0.1662	27.6220665325075\\
63.125	0.16986	29.3840165891967\\
63.125	0.17352	31.2123565830489\\
63.125	0.17718	33.1070865140639\\
63.125	0.18084	35.0682063822417\\
63.125	0.1845	37.0957161875825\\
63.125	0.18816	39.1896159300861\\
63.125	0.19182	41.3499056097526\\
63.125	0.19548	43.576585226582\\
63.125	0.19914	45.8696547805743\\
63.125	0.2028	48.2291142717294\\
63.125	0.20646	50.6549637000474\\
63.125	0.21012	53.1472030655283\\
63.125	0.21378	55.705832368172\\
63.125	0.21744	58.3308516079787\\
63.125	0.2211	61.0222607849482\\
63.125	0.22476	63.7800598990805\\
63.125	0.22842	66.6042489503758\\
63.125	0.23208	69.4948279388339\\
63.125	0.23574	72.4517968644549\\
63.125	0.2394	75.4751557272388\\
63.125	0.24306	78.5649045271855\\
63.125	0.24672	81.7210432642951\\
63.125	0.25038	84.9435719385677\\
63.125	0.25404	88.232490550003\\
63.125	0.2577	91.5877990986013\\
63.125	0.26136	95.0094975843624\\
63.125	0.26502	98.4975860072864\\
63.125	0.26868	102.052064367373\\
63.125	0.27234	105.672932664623\\
63.125	0.276	109.360190899036\\
63.5	0.093	6.39284014618478\\
63.5	0.09666	6.82740860624321\\
63.5	0.10032	7.32836700346451\\
63.5	0.10398	7.89571533784866\\
63.5	0.10764	8.52945360939573\\
63.5	0.1113	9.22958181810564\\
63.5	0.11496	9.99609996397843\\
63.5	0.11862	10.8290080470141\\
63.5	0.12228	11.7283060672126\\
63.5	0.12594	12.6939940245741\\
63.5	0.1296	13.7260719190983\\
63.5	0.13326	14.8245397507855\\
63.5	0.13692	15.9893975196355\\
63.5	0.14058	17.2206452256485\\
63.5	0.14424	18.5182828688243\\
63.5	0.1479	19.8823104491629\\
63.5	0.15156	21.3127279666645\\
63.5	0.15522	22.8095354213289\\
63.5	0.15888	24.3727328131562\\
63.5	0.16254	26.0023201421463\\
63.5	0.1662	27.6982974082994\\
63.5	0.16986	29.4606646116153\\
63.5	0.17352	31.2894217520941\\
63.5	0.17718	33.1845688297357\\
63.5	0.18084	35.1461058445402\\
63.5	0.1845	37.1740327965077\\
63.5	0.18816	39.2683496856379\\
63.5	0.19182	41.4290565119311\\
63.5	0.19548	43.6561532753871\\
63.5	0.19914	45.949639976006\\
63.5	0.2028	48.3095166137878\\
63.5	0.20646	50.7357831887325\\
63.5	0.21012	53.22843970084\\
63.5	0.21378	55.7874861501104\\
63.5	0.21744	58.4129225365437\\
63.5	0.2211	61.1047488601398\\
63.5	0.22476	63.8629651208989\\
63.5	0.22842	66.6875713188207\\
63.5	0.23208	69.5785674539055\\
63.5	0.23574	72.5359535261532\\
63.5	0.2394	75.5597295355637\\
63.5	0.24306	78.6498954821371\\
63.5	0.24672	81.8064513658734\\
63.5	0.25038	85.0293971867726\\
63.5	0.25404	88.3187329448346\\
63.5	0.2577	91.6744586400595\\
63.5	0.26136	95.0965742724473\\
63.5	0.26502	98.5850798419979\\
63.5	0.26868	102.139975348711\\
63.5	0.27234	105.761260792588\\
63.5	0.276	109.448936173627\\
63.875	0.093	6.46447045863123\\
63.875	0.09666	6.8994560653163\\
63.875	0.10032	7.40083160916426\\
63.875	0.10398	7.96859709017506\\
63.875	0.10764	8.60275250834878\\
63.875	0.1113	9.30329786368535\\
63.875	0.11496	10.0702331561848\\
63.875	0.11862	10.9035583858471\\
63.875	0.12228	11.8032735526723\\
63.875	0.12594	12.7693786566604\\
63.875	0.1296	13.8018736978113\\
63.875	0.13326	14.9007586761252\\
63.875	0.13692	16.0660335916018\\
63.875	0.14058	17.2976984442414\\
63.875	0.14424	18.5957532340439\\
63.875	0.1479	19.9601979610092\\
63.875	0.15156	21.3910326251374\\
63.875	0.15522	22.8882572264285\\
63.875	0.15888	24.4518717648824\\
63.875	0.16254	26.0818762404992\\
63.875	0.1662	27.7782706532789\\
63.875	0.16986	29.5410550032215\\
63.875	0.17352	31.3702292903269\\
63.875	0.17718	33.2657935145952\\
63.875	0.18084	35.2277476760264\\
63.875	0.1845	37.2560917746205\\
63.875	0.18816	39.3508258103774\\
63.875	0.19182	41.5119497832972\\
63.875	0.19548	43.7394636933799\\
63.875	0.19914	46.0333675406255\\
63.875	0.2028	48.3936613250339\\
63.875	0.20646	50.8203450466052\\
63.875	0.21012	53.3134187053394\\
63.875	0.21378	55.8728823012365\\
63.875	0.21744	58.4987358342964\\
63.875	0.2211	61.1909793045192\\
63.875	0.22476	63.9496127119049\\
63.875	0.22842	66.7746360564534\\
63.875	0.23208	69.6660493381648\\
63.875	0.23574	72.6238525570392\\
63.875	0.2394	75.6480457130764\\
63.875	0.24306	78.7386288062764\\
63.875	0.24672	81.8956018366393\\
63.875	0.25038	85.1189648041652\\
63.875	0.25404	88.4087177088538\\
63.875	0.2577	91.7648605507054\\
63.875	0.26136	95.1873933297198\\
63.875	0.26502	98.6763160458971\\
63.875	0.26868	102.231628699237\\
63.875	0.27234	105.85333128974\\
63.875	0.276	109.541423817406\\
64.25	0.093	6.53984314026532\\
64.25	0.09666	6.97524589357706\\
64.25	0.10032	7.47703858405167\\
64.25	0.10398	8.04522121168913\\
64.25	0.10764	8.6797937764895\\
64.25	0.1113	9.38075627845273\\
64.25	0.11496	10.1481087175788\\
64.25	0.11862	10.9818510938678\\
64.25	0.12228	11.8819834073196\\
64.25	0.12594	12.8485056579344\\
64.25	0.1296	13.881417845712\\
64.25	0.13326	14.9807199706525\\
64.25	0.13692	16.1464120327558\\
64.25	0.14058	17.378494032022\\
64.25	0.14424	18.6769659684511\\
64.25	0.1479	20.0418278420431\\
64.25	0.15156	21.4730796527979\\
64.25	0.15522	22.9707214007157\\
64.25	0.15888	24.5347530857963\\
64.25	0.16254	26.1651747080397\\
64.25	0.1662	27.8619862674461\\
64.25	0.16986	29.6251877640153\\
64.25	0.17352	31.4547791977474\\
64.25	0.17718	33.3507605686424\\
64.25	0.18084	35.3131318767002\\
64.25	0.1845	37.3418931219209\\
64.25	0.18816	39.4370443043045\\
64.25	0.19182	41.598585423851\\
64.25	0.19548	43.8265164805603\\
64.25	0.19914	46.1208374744325\\
64.25	0.2028	48.4815484054677\\
64.25	0.20646	50.9086492736656\\
64.25	0.21012	53.4021400790264\\
64.25	0.21378	55.9620208215501\\
64.25	0.21744	58.5882915012367\\
64.25	0.2211	61.2809521180862\\
64.25	0.22476	64.0400026720985\\
64.25	0.22842	66.8654431632737\\
64.25	0.23208	69.7572735916118\\
64.25	0.23574	72.7154939571128\\
64.25	0.2394	75.7401042597766\\
64.25	0.24306	78.8311044996034\\
64.25	0.24672	81.9884946765929\\
64.25	0.25038	85.2122747907454\\
64.25	0.25404	88.5024448420607\\
64.25	0.2577	91.859004830539\\
64.25	0.26136	95.2819547561801\\
64.25	0.26502	98.771294618984\\
64.25	0.26868	102.327024418951\\
64.25	0.27234	105.949144156081\\
64.25	0.276	109.637653830373\\
64.625	0.093	6.61895819108708\\
64.625	0.09666	7.05477809102547\\
64.625	0.10032	7.55698792812673\\
64.625	0.10398	8.12558770239087\\
64.625	0.10764	8.76057741381787\\
64.625	0.1113	9.46195706240776\\
64.625	0.11496	10.2297266481605\\
64.625	0.11862	11.0638861710761\\
64.625	0.12228	11.9644356311547\\
64.625	0.12594	12.931375028396\\
64.625	0.1296	13.9647043628003\\
64.625	0.13326	15.0644236343674\\
64.625	0.13692	16.2305328430974\\
64.625	0.14058	17.4630319889903\\
64.625	0.14424	18.7619210720461\\
64.625	0.1479	20.1272000922647\\
64.625	0.15156	21.5588690496462\\
64.625	0.15522	23.0569279441906\\
64.625	0.15888	24.6213767758978\\
64.625	0.16254	26.2522155447679\\
64.625	0.1662	27.9494442508009\\
64.625	0.16986	29.7130628939968\\
64.625	0.17352	31.5430714743556\\
64.625	0.17718	33.4394699918772\\
64.625	0.18084	35.4022584465617\\
64.625	0.1845	37.4314368384091\\
64.625	0.18816	39.5270051674193\\
64.625	0.19182	41.6889634335924\\
64.625	0.19548	43.9173116369284\\
64.625	0.19914	46.2120497774273\\
64.625	0.2028	48.573177855089\\
64.625	0.20646	51.0006958699137\\
64.625	0.21012	53.4946038219012\\
64.625	0.21378	56.0549017110515\\
64.625	0.21744	58.6815895373648\\
64.625	0.2211	61.3746673008409\\
64.625	0.22476	64.1341350014799\\
64.625	0.22842	66.9599926392817\\
64.625	0.23208	69.8522402142465\\
64.625	0.23574	72.8108777263741\\
64.625	0.2394	75.8359051756646\\
64.625	0.24306	78.927322562118\\
64.625	0.24672	82.0851298857342\\
64.625	0.25038	85.3093271465133\\
64.625	0.25404	88.5999143444553\\
64.625	0.2577	91.9568914795602\\
64.625	0.26136	95.3802585518279\\
64.625	0.26502	98.8700155612585\\
64.625	0.26868	102.426162507852\\
64.625	0.27234	106.048699391608\\
64.625	0.276	109.737626212528\\
65	0.093	6.70181561109652\\
65	0.09666	7.13805265766156\\
65	0.10032	7.64067964138947\\
65	0.10398	8.20969656228026\\
65	0.10764	8.84510342033392\\
65	0.1113	9.54690021555046\\
65	0.11496	10.3150869479299\\
65	0.11862	11.1496636174721\\
65	0.12228	12.0506302241773\\
65	0.12594	13.0179867680454\\
65	0.1296	14.0517332490763\\
65	0.13326	15.15186966727\\
65	0.13692	16.3183960226267\\
65	0.14058	17.5513123151463\\
65	0.14424	18.8506185448286\\
65	0.1479	20.2163147116739\\
65	0.15156	21.6484008156821\\
65	0.15522	23.1468768568531\\
65	0.15888	24.711742835187\\
65	0.16254	26.3429987506838\\
65	0.1662	28.0406446033435\\
65	0.16986	29.804680393166\\
65	0.17352	31.6351061201514\\
65	0.17718	33.5319217842997\\
65	0.18084	35.4951273856108\\
65	0.1845	37.5247229240848\\
65	0.18816	39.6207083997218\\
65	0.19182	41.7830838125215\\
65	0.19548	44.0118491624842\\
65	0.19914	46.3070044496097\\
65	0.2028	48.6685496738981\\
65	0.20646	51.0964848353494\\
65	0.21012	53.5908099339635\\
65	0.21378	56.1515249697406\\
65	0.21744	58.7786299426804\\
65	0.2211	61.4721248527832\\
65	0.22476	64.2320097000488\\
65	0.22842	67.0582844844774\\
65	0.23208	69.9509492060687\\
65	0.23574	72.9100038648231\\
65	0.2394	75.9354484607402\\
65	0.24306	79.0272829938203\\
65	0.24672	82.1855074640631\\
65	0.25038	85.4101218714689\\
65	0.25404	88.7011262160375\\
65	0.2577	92.0585204977691\\
65	0.26136	95.4823047166635\\
65	0.26502	98.9724788727207\\
65	0.26868	102.529042965941\\
65	0.27234	106.151996996324\\
65	0.276	109.84134096387\\
65.375	0.093	6.78841540029362\\
65.375	0.09666	7.22506959348531\\
65.375	0.10032	7.72811372383989\\
65.375	0.10398	8.29754779135734\\
65.375	0.10764	8.93337179603764\\
65.375	0.1113	9.63558573788084\\
65.375	0.11496	10.4041896168869\\
65.375	0.11862	11.2391834330558\\
65.375	0.12228	12.1405671863877\\
65.375	0.12594	13.1083408768824\\
65.375	0.1296	14.1425045045399\\
65.375	0.13326	15.2430580693603\\
65.375	0.13692	16.4100015713437\\
65.375	0.14058	17.6433350104899\\
65.375	0.14424	18.9430583867989\\
65.375	0.1479	20.3091717002709\\
65.375	0.15156	21.7416749509057\\
65.375	0.15522	23.2405681387034\\
65.375	0.15888	24.8058512636639\\
65.375	0.16254	26.4375243257874\\
65.375	0.1662	28.1355873250737\\
65.375	0.16986	29.9000402615229\\
65.375	0.17352	31.7308831351349\\
65.375	0.17718	33.6281159459098\\
65.375	0.18084	35.5917386938476\\
65.375	0.1845	37.6217513789483\\
65.375	0.18816	39.7181540012119\\
65.375	0.19182	41.8809465606383\\
65.375	0.19548	44.1101290572276\\
65.375	0.19914	46.4057014909798\\
65.375	0.2028	48.7676638618949\\
65.375	0.20646	51.1960161699728\\
65.375	0.21012	53.6907584152136\\
65.375	0.21378	56.2518905976173\\
65.375	0.21744	58.8794127171838\\
65.375	0.2211	61.5733247739132\\
65.375	0.22476	64.3336267678055\\
65.375	0.22842	67.1603186988607\\
65.375	0.23208	70.0534005670787\\
65.375	0.23574	73.0128723724597\\
65.375	0.2394	76.0387341150035\\
65.375	0.24306	79.1309857947102\\
65.375	0.24672	82.2896274115797\\
65.375	0.25038	85.5146589656122\\
65.375	0.25404	88.8060804568075\\
65.375	0.2577	92.1638918851656\\
65.375	0.26136	95.5880932506867\\
65.375	0.26502	99.0786845533706\\
65.375	0.26868	102.635665793217\\
65.375	0.27234	106.259036970227\\
65.375	0.276	109.9487980844\\
65.75	0.093	6.87875755867838\\
65.75	0.09666	7.31582889849674\\
65.75	0.10032	7.81929017547796\\
65.75	0.10398	8.38914138962205\\
65.75	0.10764	9.02538254092902\\
65.75	0.1113	9.72801362939887\\
65.75	0.11496	10.4970346550316\\
65.75	0.11862	11.3324456178272\\
65.75	0.12228	12.2342465177857\\
65.75	0.12594	13.202437354907\\
65.75	0.1296	14.2370181291912\\
65.75	0.13326	15.3379888406383\\
65.75	0.13692	16.5053494892483\\
65.75	0.14058	17.7391000750211\\
65.75	0.14424	19.0392405979568\\
65.75	0.1479	20.4057710580554\\
65.75	0.15156	21.8386914553169\\
65.75	0.15522	23.3380017897412\\
65.75	0.15888	24.9037020613285\\
65.75	0.16254	26.5357922700786\\
65.75	0.1662	28.2342724159915\\
65.75	0.16986	29.9991424990674\\
65.75	0.17352	31.8304025193061\\
65.75	0.17718	33.7280524767076\\
65.75	0.18084	35.6920923712721\\
65.75	0.1845	37.7225222029994\\
65.75	0.18816	39.8193419718897\\
65.75	0.19182	41.9825516779427\\
65.75	0.19548	44.2121513211587\\
65.75	0.19914	46.5081409015375\\
65.75	0.2028	48.8705204190793\\
65.75	0.20646	51.2992898737838\\
65.75	0.21012	53.7944492656513\\
65.75	0.21378	56.3559985946816\\
65.75	0.21744	58.9839378608748\\
65.75	0.2211	61.6782670642309\\
65.75	0.22476	64.4389862047499\\
65.75	0.22842	67.2660952824317\\
65.75	0.23208	70.1595942972764\\
65.75	0.23574	73.119483249284\\
65.75	0.2394	76.1457621384544\\
65.75	0.24306	79.2384309647878\\
65.75	0.24672	82.397489728284\\
65.75	0.25038	85.6229384289431\\
65.75	0.25404	88.914777066765\\
65.75	0.2577	92.2730056417498\\
65.75	0.26136	95.6976241538976\\
65.75	0.26502	99.1886326032081\\
65.75	0.26868	102.746030989682\\
65.75	0.27234	106.369819313318\\
65.75	0.276	110.059997574117\\
66.125	0.093	6.9728420862508\\
66.125	0.09666	7.4103305726958\\
66.125	0.10032	7.9142089963037\\
66.125	0.10398	8.48447735707443\\
66.125	0.10764	9.12113565500808\\
66.125	0.1113	9.82418389010456\\
66.125	0.11496	10.5936220623639\\
66.125	0.11862	11.4294501717862\\
66.125	0.12228	12.3316682183713\\
66.125	0.12594	13.3002762021193\\
66.125	0.1296	14.3352741230302\\
66.125	0.13326	15.4366619811039\\
66.125	0.13692	16.6044397763405\\
66.125	0.14058	17.8386075087401\\
66.125	0.14424	19.1391651783024\\
66.125	0.1479	20.5061127850277\\
66.125	0.15156	21.9394503289158\\
66.125	0.15522	23.4391778099668\\
66.125	0.15888	25.0052952281807\\
66.125	0.16254	26.6378025835574\\
66.125	0.1662	28.336699876097\\
66.125	0.16986	30.1019871057995\\
66.125	0.17352	31.9336642726649\\
66.125	0.17718	33.8317313766931\\
66.125	0.18084	35.7961884178842\\
66.125	0.1845	37.8270353962382\\
66.125	0.18816	39.9242723117551\\
66.125	0.19182	42.0878991644348\\
66.125	0.19548	44.3179159542774\\
66.125	0.19914	46.6143226812829\\
66.125	0.2028	48.9771193454513\\
66.125	0.20646	51.4063059467826\\
66.125	0.21012	53.9018824852767\\
66.125	0.21378	56.4638489609336\\
66.125	0.21744	59.0922053737535\\
66.125	0.2211	61.7869517237362\\
66.125	0.22476	64.5480880108818\\
66.125	0.22842	67.3756142351903\\
66.125	0.23208	70.2695303966617\\
66.125	0.23574	73.2298364952959\\
66.125	0.2394	76.2565325310931\\
66.125	0.24306	79.349618504053\\
66.125	0.24672	82.5090944141758\\
66.125	0.25038	85.7349602614616\\
66.125	0.25404	89.0272160459102\\
66.125	0.2577	92.3858617675217\\
66.125	0.26136	95.8108974262961\\
66.125	0.26502	99.3023230222333\\
66.125	0.26868	102.860138555333\\
66.125	0.27234	106.484344025596\\
66.125	0.276	110.174939433022\\
66.5	0.093	7.07066898301089\\
66.5	0.09666	7.50857461608255\\
66.5	0.10032	8.01287018631711\\
66.5	0.10398	8.58355569371449\\
66.5	0.10764	9.22063113827479\\
66.5	0.1113	9.92409651999794\\
66.5	0.11496	10.693951838884\\
66.5	0.11862	11.5301970949329\\
66.5	0.12228	12.4328322881446\\
66.5	0.12594	13.4018574185193\\
66.5	0.1296	14.4372724860568\\
66.5	0.13326	15.5390774907572\\
66.5	0.13692	16.7072724326205\\
66.5	0.14058	17.9418573116467\\
66.5	0.14424	19.2428321278357\\
66.5	0.1479	20.6101968811876\\
66.5	0.15156	22.0439515717024\\
66.5	0.15522	23.54409619938\\
66.5	0.15888	25.1106307642206\\
66.5	0.16254	26.7435552662239\\
66.5	0.1662	28.4428697053902\\
66.5	0.16986	30.2085740817194\\
66.5	0.17352	32.0406683952114\\
66.5	0.17718	33.9391526458663\\
66.5	0.18084	35.904026833684\\
66.5	0.1845	37.9352909586647\\
66.5	0.18816	40.0329450208082\\
66.5	0.19182	42.1969890201146\\
66.5	0.19548	44.4274229565839\\
66.5	0.19914	46.724246830216\\
66.5	0.2028	49.0874606410111\\
66.5	0.20646	51.517064388969\\
66.5	0.21012	54.0130580740897\\
66.5	0.21378	56.5754416963733\\
66.5	0.21744	59.2042152558199\\
66.5	0.2211	61.8993787524293\\
66.5	0.22476	64.6609321862015\\
66.5	0.22842	67.4888755571366\\
66.5	0.23208	70.3832088652346\\
66.5	0.23574	73.3439321104956\\
66.5	0.2394	76.3710452929193\\
66.5	0.24306	79.464548412506\\
66.5	0.24672	82.6244414692555\\
66.5	0.25038	85.8507244631679\\
66.5	0.25404	89.1433973942431\\
66.5	0.2577	92.5024602624813\\
66.5	0.26136	95.9279130678823\\
66.5	0.26502	99.4197558104462\\
66.5	0.26868	102.977988490173\\
66.5	0.27234	106.602611107063\\
66.5	0.276	110.293623661115\\
66.875	0.093	7.17223824895865\\
66.875	0.09666	7.61056102865697\\
66.875	0.10032	8.11527374551817\\
66.875	0.10398	8.6863763995422\\
66.875	0.10764	9.32386899072916\\
66.875	0.1113	10.027751519079\\
66.875	0.11496	10.7980239845916\\
66.875	0.11862	11.6346863872672\\
66.875	0.12228	12.5377387271056\\
66.875	0.12594	13.507181004107\\
66.875	0.1296	14.5430132182711\\
66.875	0.13326	15.6452353695982\\
66.875	0.13692	16.8138474580881\\
66.875	0.14058	18.0488494837409\\
66.875	0.14424	19.3502414465566\\
66.875	0.1479	20.7180233465352\\
66.875	0.15156	22.1521951836766\\
66.875	0.15522	23.6527569579809\\
66.875	0.15888	25.2197086694481\\
66.875	0.16254	26.8530503180781\\
66.875	0.1662	28.5527819038711\\
66.875	0.16986	30.3189034268269\\
66.875	0.17352	32.1514148869456\\
66.875	0.17718	34.0503162842271\\
66.875	0.18084	36.0156076186715\\
66.875	0.1845	38.0472888902788\\
66.875	0.18816	40.145360099049\\
66.875	0.19182	42.309821244982\\
66.875	0.19548	44.540672328078\\
66.875	0.19914	46.8379133483368\\
66.875	0.2028	49.2015443057585\\
66.875	0.20646	51.631565200343\\
66.875	0.21012	54.1279760320904\\
66.875	0.21378	56.6907768010007\\
66.875	0.21744	59.3199675070739\\
66.875	0.2211	62.0155481503099\\
66.875	0.22476	64.7775187307088\\
66.875	0.22842	67.6058792482706\\
66.875	0.23208	70.5006297029953\\
66.875	0.23574	73.4617700948828\\
66.875	0.2394	76.4893004239333\\
66.875	0.24306	79.5832206901466\\
66.875	0.24672	82.7435308935227\\
66.875	0.25038	85.9702310340618\\
66.875	0.25404	89.2633211117637\\
66.875	0.2577	92.6228011266285\\
66.875	0.26136	96.0486710786562\\
66.875	0.26502	99.5409309678467\\
66.875	0.26868	103.0995807942\\
66.875	0.27234	106.724620557716\\
66.875	0.276	110.416050258396\\
67.25	0.093	7.27754988409409\\
67.25	0.09666	7.71628981041905\\
67.25	0.10032	8.22141967390691\\
67.25	0.10398	8.79293947455761\\
67.25	0.10764	9.43084921237121\\
67.25	0.1113	10.1351488873477\\
67.25	0.11496	10.905838499487\\
67.25	0.11862	11.7429180487892\\
67.25	0.12228	12.6463875352543\\
67.25	0.12594	13.6162469588823\\
67.25	0.1296	14.6524963196731\\
67.25	0.13326	15.7551356176268\\
67.25	0.13692	16.9241648527434\\
67.25	0.14058	18.1595840250229\\
67.25	0.14424	19.4613931344652\\
67.25	0.1479	20.8295921810704\\
67.25	0.15156	22.2641811648385\\
67.25	0.15522	23.7651600857695\\
67.25	0.15888	25.3325289438633\\
67.25	0.16254	26.96628773912\\
67.25	0.1662	28.6664364715396\\
67.25	0.16986	30.432975141122\\
67.25	0.17352	32.2659037478674\\
67.25	0.17718	34.1652222917756\\
67.25	0.18084	36.1309307728466\\
67.25	0.1845	38.1630291910806\\
67.25	0.18816	40.2615175464774\\
67.25	0.19182	42.4263958390371\\
67.25	0.19548	44.6576640687597\\
67.25	0.19914	46.9553222356452\\
67.25	0.2028	49.3193703396935\\
67.25	0.20646	51.7498083809047\\
67.25	0.21012	54.2466363592788\\
67.25	0.21378	56.8098542748157\\
67.25	0.21744	59.4394621275156\\
67.25	0.2211	62.1354599173783\\
67.25	0.22476	64.8978476444038\\
67.25	0.22842	67.7266253085923\\
67.25	0.23208	70.6217929099436\\
67.25	0.23574	73.5833504484578\\
67.25	0.2394	76.6112979241349\\
67.25	0.24306	79.7056353369748\\
67.25	0.24672	82.8663626869777\\
67.25	0.25038	86.0934799741434\\
67.25	0.25404	89.3869871984719\\
67.25	0.2577	92.7468843599634\\
67.25	0.26136	96.1731714586177\\
67.25	0.26502	99.6658484944349\\
67.25	0.26868	103.224915467415\\
67.25	0.27234	106.850372377558\\
67.25	0.276	110.542219224864\\
67.625	0.093	7.38660388841716\\
67.625	0.09666	7.82576096136878\\
67.625	0.10032	8.33130797148327\\
67.625	0.10398	8.90324491876065\\
67.625	0.10764	9.54157180320088\\
67.625	0.1113	10.246288624804\\
67.625	0.11496	11.01739538357\\
67.625	0.11862	11.8548920794989\\
67.625	0.12228	12.7587787125906\\
67.625	0.12594	13.7290552828452\\
67.625	0.1296	14.7657217902627\\
67.625	0.13326	15.8687782348431\\
67.625	0.13692	17.0382246165863\\
67.625	0.14058	18.2740609354925\\
67.625	0.14424	19.5762871915614\\
67.625	0.1479	20.9449033847933\\
67.625	0.15156	22.379909515188\\
67.625	0.15522	23.8813055827457\\
67.625	0.15888	25.4490915874662\\
67.625	0.16254	27.0832675293495\\
67.625	0.1662	28.7838334083958\\
67.625	0.16986	30.5507892246049\\
67.625	0.17352	32.3841349779769\\
67.625	0.17718	34.2838706685117\\
67.625	0.18084	36.2499962962094\\
67.625	0.1845	38.28251186107\\
67.625	0.18816	40.3814173630935\\
67.625	0.19182	42.5467128022799\\
67.625	0.19548	44.7783981786291\\
67.625	0.19914	47.0764734921412\\
67.625	0.2028	49.4409387428162\\
67.625	0.20646	51.8717939306541\\
67.625	0.21012	54.3690390556548\\
67.625	0.21378	56.9326741178184\\
67.625	0.21744	59.5626991171449\\
67.625	0.2211	62.2591140536342\\
67.625	0.22476	65.0219189272865\\
67.625	0.22842	67.8511137381016\\
67.625	0.23208	70.7466984860795\\
67.625	0.23574	73.7086731712204\\
67.625	0.2394	76.7370377935241\\
67.625	0.24306	79.8317923529908\\
67.625	0.24672	82.9929368496202\\
67.625	0.25038	86.2204712834126\\
67.625	0.25404	89.5143956543678\\
67.625	0.2577	92.8747099624859\\
67.625	0.26136	96.3014142077669\\
67.625	0.26502	99.7945083902107\\
67.625	0.26868	103.353992509817\\
67.625	0.27234	106.979866566587\\
67.625	0.276	110.672130560519\\
68	0.093	7.49940026192791\\
68	0.09666	7.9389744815062\\
68	0.10032	8.44493863824733\\
68	0.10398	9.01729273215138\\
68	0.10764	9.65603676321827\\
68	0.1113	10.361170731448\\
68	0.11496	11.1326946368407\\
68	0.11862	11.9706084793962\\
68	0.12228	12.8749122591146\\
68	0.12594	13.8456059759959\\
68	0.1296	14.88268963004\\
68	0.13326	15.986163221247\\
68	0.13692	17.1560267496169\\
68	0.14058	18.3922802151497\\
68	0.14424	19.6949236178454\\
68	0.1479	21.0639569577039\\
68	0.15156	22.4993802347253\\
68	0.15522	24.0011934489096\\
68	0.15888	25.5693966002567\\
68	0.16254	27.2039896887667\\
68	0.1662	28.9049727144396\\
68	0.16986	30.6723456772754\\
68	0.17352	32.506108577274\\
68	0.17718	34.4062614144355\\
68	0.18084	36.3728041887599\\
68	0.1845	38.4057369002472\\
68	0.18816	40.5050595488973\\
68	0.19182	42.6707721347103\\
68	0.19548	44.9028746576862\\
68	0.19914	47.201367117825\\
68	0.2028	49.5662495151266\\
68	0.20646	51.9975218495911\\
68	0.21012	54.4951841212185\\
68	0.21378	57.0592363300088\\
68	0.21744	59.6896784759619\\
68	0.2211	62.3865105590779\\
68	0.22476	65.1497325793568\\
68	0.22842	67.9793445367985\\
68	0.23208	70.8753464314032\\
68	0.23574	73.8377382631707\\
68	0.2394	76.8665200321011\\
68	0.24306	79.9616917381943\\
68	0.24672	83.1232533814505\\
68	0.25038	86.3512049618695\\
68	0.25404	89.6455464794514\\
68	0.2577	93.0062779341961\\
68	0.26136	96.4333993261037\\
68	0.26502	99.9269106551743\\
68	0.26868	103.486811921408\\
68	0.27234	107.113103124804\\
68	0.276	110.805784265363\\
68.375	0.093	7.61593900462633\\
68.375	0.09666	8.05593037083127\\
68.375	0.10032	8.56231167419906\\
68.375	0.10398	9.13508291472976\\
68.375	0.10764	9.77424409242331\\
68.375	0.1113	10.4797952072797\\
68.375	0.11496	11.251736259299\\
68.375	0.11862	12.0900672484812\\
68.375	0.12228	12.9947881748263\\
68.375	0.12594	13.9658990383342\\
68.375	0.1296	15.003399839005\\
68.375	0.13326	16.1072905768387\\
68.375	0.13692	17.2775712518352\\
68.375	0.14058	18.5142418639947\\
68.375	0.14424	19.8173024133169\\
68.375	0.1479	21.1867528998021\\
68.375	0.15156	22.6225933234502\\
68.375	0.15522	24.1248236842611\\
68.375	0.15888	25.6934439822349\\
68.375	0.16254	27.3284542173716\\
68.375	0.1662	29.0298543896711\\
68.375	0.16986	30.7976444991335\\
68.375	0.17352	32.6318245457588\\
68.375	0.17718	34.532394529547\\
68.375	0.18084	36.499354450498\\
68.375	0.1845	38.532704308612\\
68.375	0.18816	40.6324441038887\\
68.375	0.19182	42.7985738363284\\
68.375	0.19548	45.031093505931\\
68.375	0.19914	47.3300031126964\\
68.375	0.2028	49.6953026566247\\
68.375	0.20646	52.1269921377159\\
68.375	0.21012	54.6250715559699\\
68.375	0.21378	57.1895409113868\\
68.375	0.21744	59.8204002039666\\
68.375	0.2211	62.5176494337093\\
68.375	0.22476	65.2812886006148\\
68.375	0.22842	68.1113177046832\\
68.375	0.23208	71.0077367459144\\
68.375	0.23574	73.9705457243086\\
68.375	0.2394	76.9997446398657\\
68.375	0.24306	80.0953334925856\\
68.375	0.24672	83.2573122824684\\
68.375	0.25038	86.4856810095141\\
68.375	0.25404	89.7804396737226\\
68.375	0.2577	93.141588275094\\
68.375	0.26136	96.5691268136283\\
68.375	0.26502	100.063055289325\\
68.375	0.26868	103.623373702185\\
68.375	0.27234	107.250082052208\\
68.375	0.276	110.943180339394\\
68.75	0.093	7.73622011651242\\
68.75	0.09666	8.17662862934401\\
68.75	0.10032	8.68342707933845\\
68.75	0.10398	9.2566154664958\\
68.75	0.10764	9.896193790816\\
68.75	0.1113	10.6021620522991\\
68.75	0.11496	11.3745202509451\\
68.75	0.11862	12.2132683867539\\
68.75	0.12228	13.1184064597256\\
68.75	0.12594	14.0899344698602\\
68.75	0.1296	15.1278524171576\\
68.75	0.13326	16.2321603016179\\
68.75	0.13692	17.4028581232412\\
68.75	0.14058	18.6399458820272\\
68.75	0.14424	19.9434235779762\\
68.75	0.1479	21.313291211088\\
68.75	0.15156	22.7495487813627\\
68.75	0.15522	24.2521962888003\\
68.75	0.15888	25.8212337334008\\
68.75	0.16254	27.4566611151641\\
68.75	0.1662	29.1584784340903\\
68.75	0.16986	30.9266856901794\\
68.75	0.17352	32.7612828834313\\
68.75	0.17718	34.6622700138461\\
68.75	0.18084	36.6296470814238\\
68.75	0.1845	38.6634140861644\\
68.75	0.18816	40.7635710280679\\
68.75	0.19182	42.9301179071342\\
68.75	0.19548	45.1630547233634\\
68.75	0.19914	47.4623814767554\\
68.75	0.2028	49.8280981673104\\
68.75	0.20646	52.2602047950282\\
68.75	0.21012	54.7587013599089\\
68.75	0.21378	57.3235878619525\\
68.75	0.21744	59.9548643011589\\
68.75	0.2211	62.6525306775282\\
68.75	0.22476	65.4165869910604\\
68.75	0.22842	68.2470332417555\\
68.75	0.23208	71.1438694296134\\
68.75	0.23574	74.1070955546343\\
68.75	0.2394	77.1367116168179\\
68.75	0.24306	80.2327176161645\\
68.75	0.24672	83.3951135526739\\
68.75	0.25038	86.6238994263463\\
68.75	0.25404	89.9190752371815\\
68.75	0.2577	93.2806409851795\\
68.75	0.26136	96.7085966703405\\
68.75	0.26502	100.202942292664\\
68.75	0.26868	103.763677852151\\
68.75	0.27234	107.390803348801\\
68.75	0.276	111.084318782613\\
69.125	0.093	7.86024359758616\\
69.125	0.09666	8.30106925704441\\
69.125	0.10032	8.80828485366552\\
69.125	0.10398	9.38189038744952\\
69.125	0.10764	10.0218858583964\\
69.125	0.1113	10.7282712665061\\
69.125	0.11496	11.5010466117787\\
69.125	0.11862	12.3402118942142\\
69.125	0.12228	13.2457671138126\\
69.125	0.12594	14.2177122705738\\
69.125	0.1296	15.2560473644979\\
69.125	0.13326	16.3607723955849\\
69.125	0.13692	17.5318873638348\\
69.125	0.14058	18.7693922692475\\
69.125	0.14424	20.0732871118231\\
69.125	0.1479	21.4435718915616\\
69.125	0.15156	22.880246608463\\
69.125	0.15522	24.3833112625272\\
69.125	0.15888	25.9527658537543\\
69.125	0.16254	27.5886103821443\\
69.125	0.1662	29.2908448476971\\
69.125	0.16986	31.0594692504129\\
69.125	0.17352	32.8944835902915\\
69.125	0.17718	34.7958878673329\\
69.125	0.18084	36.7636820815373\\
69.125	0.1845	38.7978662329045\\
69.125	0.18816	40.8984403214346\\
69.125	0.19182	43.0654043471276\\
69.125	0.19548	45.2987583099834\\
69.125	0.19914	47.5985022100022\\
69.125	0.2028	49.9646360471838\\
69.125	0.20646	52.3971598215283\\
69.125	0.21012	54.8960735330356\\
69.125	0.21378	57.4613771817058\\
69.125	0.21744	60.0930707675389\\
69.125	0.2211	62.7911542905349\\
69.125	0.22476	65.5556277506937\\
69.125	0.22842	68.3864911480154\\
69.125	0.23208	71.2837444825\\
69.125	0.23574	74.2473877541475\\
69.125	0.2394	77.2774209629579\\
69.125	0.24306	80.3738441089311\\
69.125	0.24672	83.5366571920672\\
69.125	0.25038	86.7658602123662\\
69.125	0.25404	90.061453169828\\
69.125	0.2577	93.4234360644527\\
69.125	0.26136	96.8518088962404\\
69.125	0.26502	100.346571665191\\
69.125	0.26868	103.907724371304\\
69.125	0.27234	107.53526701458\\
69.125	0.276	111.229199595019\\
69.5	0.093	7.98800944784757\\
69.5	0.09666	8.42925225393246\\
69.5	0.10032	8.93688499718024\\
69.5	0.10398	9.51090767759086\\
69.5	0.10764	10.1513202951644\\
69.5	0.1113	10.8581228499008\\
69.5	0.11496	11.6313153418001\\
69.5	0.11862	12.4708977708622\\
69.5	0.12228	13.3768701370872\\
69.5	0.12594	14.3492324404751\\
69.5	0.1296	15.3879846810259\\
69.5	0.13326	16.4931268587395\\
69.5	0.13692	17.664658973616\\
69.5	0.14058	18.9025810256554\\
69.5	0.14424	20.2068930148577\\
69.5	0.1479	21.5775949412228\\
69.5	0.15156	23.0146868047508\\
69.5	0.15522	24.5181686054417\\
69.5	0.15888	26.0880403432955\\
69.5	0.16254	27.7243020183121\\
69.5	0.1662	29.4269536304916\\
69.5	0.16986	31.195995179834\\
69.5	0.17352	33.0314266663393\\
69.5	0.17718	34.9332480900074\\
69.5	0.18084	36.9014594508384\\
69.5	0.1845	38.9360607488323\\
69.5	0.18816	41.037051983989\\
69.5	0.19182	43.2044331563087\\
69.5	0.19548	45.4382042657912\\
69.5	0.19914	47.7383653124366\\
69.5	0.2028	50.1049162962448\\
69.5	0.20646	52.537857217216\\
69.5	0.21012	55.03718807535\\
69.5	0.21378	57.6029088706468\\
69.5	0.21744	60.2350196031066\\
69.5	0.2211	62.9335202727292\\
69.5	0.22476	65.6984108795147\\
69.5	0.22842	68.5296914234631\\
69.5	0.23208	71.4273619045743\\
69.5	0.23574	74.3914223228485\\
69.5	0.2394	77.4218726782855\\
69.5	0.24306	80.5187129708853\\
69.5	0.24672	83.6819432006481\\
69.5	0.25038	86.9115633675737\\
69.5	0.25404	90.2075734716622\\
69.5	0.2577	93.5699735129136\\
69.5	0.26136	96.9987634913279\\
69.5	0.26502	100.493943406905\\
69.5	0.26868	104.055513259645\\
69.5	0.27234	107.683473049548\\
69.5	0.276	111.377822776614\\
69.875	0.093	8.11951766729666\\
69.875	0.09666	8.5611776200082\\
69.875	0.10032	9.06922750988263\\
69.875	0.10398	9.64366733691992\\
69.875	0.10764	10.2844971011201\\
69.875	0.1113	10.9917168024832\\
69.875	0.11496	11.7653264410091\\
69.875	0.11862	12.6053260166979\\
69.875	0.12228	13.5117155295495\\
69.875	0.12594	14.4844949795641\\
69.875	0.1296	15.5236643667415\\
69.875	0.13326	16.6292236910818\\
69.875	0.13692	17.801172952585\\
69.875	0.14058	19.039512151251\\
69.875	0.14424	20.3442412870799\\
69.875	0.1479	21.7153603600717\\
69.875	0.15156	23.1528693702264\\
69.875	0.15522	24.6567683175439\\
69.875	0.15888	26.2270572020244\\
69.875	0.16254	27.8637360236676\\
69.875	0.1662	29.5668047824738\\
69.875	0.16986	31.3362634784428\\
69.875	0.17352	33.1721121115748\\
69.875	0.17718	35.0743506818695\\
69.875	0.18084	37.0429791893272\\
69.875	0.1845	39.0779976339477\\
69.875	0.18816	41.1794060157311\\
69.875	0.19182	43.3472043346774\\
69.875	0.19548	45.5813925907866\\
69.875	0.19914	47.8819707840586\\
69.875	0.2028	50.2489389144936\\
69.875	0.20646	52.6822969820913\\
69.875	0.21012	55.182044986852\\
69.875	0.21378	57.7481829287755\\
69.875	0.21744	60.3807108078619\\
69.875	0.2211	63.0796286241112\\
69.875	0.22476	65.8449363775234\\
69.875	0.22842	68.6766340680984\\
69.875	0.23208	71.5747216958363\\
69.875	0.23574	74.5391992607371\\
69.875	0.2394	77.5700667628007\\
69.875	0.24306	80.6673242020273\\
69.875	0.24672	83.8309715784167\\
69.875	0.25038	87.061008891969\\
69.875	0.25404	90.3574361426841\\
69.875	0.2577	93.7202533305622\\
69.875	0.26136	97.149460455603\\
69.875	0.26502	100.645057517807\\
69.875	0.26868	104.207044517173\\
69.875	0.27234	107.835421453703\\
69.875	0.276	111.530188327395\\
70.25	0.093	8.25476825593339\\
70.25	0.09666	8.69684535527161\\
70.25	0.10032	9.20531239177269\\
70.25	0.10398	9.78016936543663\\
70.25	0.10764	10.4214162762635\\
70.25	0.1113	11.1290531242532\\
70.25	0.11496	11.9030799094057\\
70.25	0.11862	12.7434966317212\\
70.25	0.12228	13.6503032911995\\
70.25	0.12594	14.6234998878407\\
70.25	0.1296	15.6630864216448\\
70.25	0.13326	16.7690628926117\\
70.25	0.13692	17.9414293007416\\
70.25	0.14058	19.1801856460343\\
70.25	0.14424	20.4853319284898\\
70.25	0.1479	21.8568681481083\\
70.25	0.15156	23.2947943048896\\
70.25	0.15522	24.7991103988338\\
70.25	0.15888	26.3698164299409\\
70.25	0.16254	28.0069123982108\\
70.25	0.1662	29.7103983036436\\
70.25	0.16986	31.4802741462393\\
70.25	0.17352	33.3165399259979\\
70.25	0.17718	35.2191956429193\\
70.25	0.18084	37.1882412970036\\
70.25	0.1845	39.2236768882508\\
70.25	0.18816	41.3255024166609\\
70.25	0.19182	43.4937178822339\\
70.25	0.19548	45.7283232849697\\
70.25	0.19914	48.0293186248684\\
70.25	0.2028	50.3967039019299\\
70.25	0.20646	52.8304791161544\\
70.25	0.21012	55.3306442675417\\
70.25	0.21378	57.8971993560919\\
70.25	0.21744	60.5301443818049\\
70.25	0.2211	63.2294793446809\\
70.25	0.22476	65.9952042447197\\
70.25	0.22842	68.8273190819213\\
70.25	0.23208	71.7258238562859\\
70.25	0.23574	74.6907185678133\\
70.25	0.2394	77.7220032165037\\
70.25	0.24306	80.8196778023569\\
70.25	0.24672	83.9837423253729\\
70.25	0.25038	87.2141967855518\\
70.25	0.25404	90.5110411828936\\
70.25	0.2577	93.8742755173984\\
70.25	0.26136	97.3038997890659\\
70.25	0.26502	100.799913997896\\
70.25	0.26868	104.36231814389\\
70.25	0.27234	107.991112227046\\
70.25	0.276	111.686296247365\\
70.625	0.093	8.39376121375781\\
70.625	0.09666	8.83625545972267\\
70.625	0.10032	9.34513964285041\\
70.625	0.10398	9.920413763141\\
70.625	0.10764	10.5620778205945\\
70.625	0.1113	11.2701318152108\\
70.625	0.11496	12.0445757469901\\
70.625	0.11862	12.8854096159322\\
70.625	0.12228	13.7926334220372\\
70.625	0.12594	14.766247165305\\
70.625	0.1296	15.8062508457357\\
70.625	0.13326	16.9126444633294\\
70.625	0.13692	18.0854280180858\\
70.625	0.14058	19.3246015100052\\
70.625	0.14424	20.6301649390874\\
70.625	0.1479	22.0021183053325\\
70.625	0.15156	23.4404616087405\\
70.625	0.15522	24.9451948493113\\
70.625	0.15888	26.5163180270451\\
70.625	0.16254	28.1538311419417\\
70.625	0.1662	29.8577341940011\\
70.625	0.16986	31.6280271832235\\
70.625	0.17352	33.4647101096087\\
70.625	0.17718	35.3677829731568\\
70.625	0.18084	37.3372457738678\\
70.625	0.1845	39.3730985117416\\
70.625	0.18816	41.4753411867784\\
70.625	0.19182	43.6439737989779\\
70.625	0.19548	45.8789963483404\\
70.625	0.19914	48.1804088348658\\
70.625	0.2028	50.548211258554\\
70.625	0.20646	52.9824036194051\\
70.625	0.21012	55.482985917419\\
70.625	0.21378	58.0499581525959\\
70.625	0.21744	60.6833203249356\\
70.625	0.2211	63.3830724344382\\
70.625	0.22476	66.1492144811036\\
70.625	0.22842	68.981746464932\\
70.625	0.23208	71.8806683859232\\
70.625	0.23574	74.8459802440773\\
70.625	0.2394	77.8776820393942\\
70.625	0.24306	80.9757737718741\\
70.625	0.24672	84.1402554415168\\
70.625	0.25038	87.3711270483224\\
70.625	0.25404	90.6683885922909\\
70.625	0.2577	94.0320400734222\\
70.625	0.26136	97.4620814917164\\
70.625	0.26502	100.958512847174\\
70.625	0.26868	104.521334139793\\
70.625	0.27234	108.150545369576\\
70.625	0.276	111.846146536522\\
71	0.093	8.53649654076987\\
71	0.09666	8.97940793336138\\
71	0.10032	9.48870926311576\\
71	0.10398	10.064400530033\\
71	0.10764	10.7064817341132\\
71	0.1113	11.4149528753562\\
71	0.11496	12.1898139537621\\
71	0.11862	13.0310649693308\\
71	0.12228	13.9387059220625\\
71	0.12594	14.912736811957\\
71	0.1296	15.9531576390144\\
71	0.13326	17.0599684032346\\
71	0.13692	18.2331691046177\\
71	0.14058	19.4727597431638\\
71	0.14424	20.7787403188726\\
71	0.1479	22.1511108317444\\
71	0.15156	23.589871281779\\
71	0.15522	25.0950216689765\\
71	0.15888	26.6665619933369\\
71	0.16254	28.3044922548602\\
71	0.1662	30.0088124535463\\
71	0.16986	31.7795225893953\\
71	0.17352	33.6166226624072\\
71	0.17718	35.5201126725819\\
71	0.18084	37.4899926199195\\
71	0.1845	39.52626250442\\
71	0.18816	41.6289223260834\\
71	0.19182	43.7979720849097\\
71	0.19548	46.0334117808988\\
71	0.19914	48.3352414140508\\
71	0.2028	50.7034609843657\\
71	0.20646	53.1380704918434\\
71	0.21012	55.639069936484\\
71	0.21378	58.2064593182875\\
71	0.21744	60.8402386372539\\
71	0.2211	63.5404078933832\\
71	0.22476	66.3069670866753\\
71	0.22842	69.1399162171302\\
71	0.23208	72.0392552847481\\
71	0.23574	75.0049842895289\\
71	0.2394	78.0371032314725\\
71	0.24306	81.135612110579\\
71	0.24672	84.3005109268484\\
71	0.25038	87.5317996802806\\
71	0.25404	90.8294783708757\\
71	0.2577	94.1935469986337\\
71	0.26136	97.6240055635546\\
71	0.26502	101.120854065638\\
71	0.26868	104.684092504885\\
71	0.27234	108.313720881294\\
71	0.276	112.009739194867\\
71.375	0.093	8.68297423696959\\
71.375	0.09666	9.12630277618777\\
71.375	0.10032	9.63602125256881\\
71.375	0.10398	10.2121296661127\\
71.375	0.10764	10.8546280168195\\
71.375	0.1113	11.5635163046892\\
71.375	0.11496	12.3387945297217\\
71.375	0.11862	13.1804626919171\\
71.375	0.12228	14.0885207912754\\
71.375	0.12594	15.0629688277966\\
71.375	0.1296	16.1038068014806\\
71.375	0.13326	17.2110347123275\\
71.375	0.13692	18.3846525603373\\
71.375	0.14058	19.62466034551\\
71.375	0.14424	20.9310580678455\\
71.375	0.1479	22.3038457273439\\
71.375	0.15156	23.7430233240052\\
71.375	0.15522	25.2485908578294\\
71.375	0.15888	26.8205483288164\\
71.375	0.16254	28.4588957369663\\
71.375	0.1662	30.1636330822791\\
71.375	0.16986	31.9347603647548\\
71.375	0.17352	33.7722775843933\\
71.375	0.17718	35.6761847411947\\
71.375	0.18084	37.646481835159\\
71.375	0.1845	39.6831688662862\\
71.375	0.18816	41.7862458345762\\
71.375	0.19182	43.9557127400291\\
71.375	0.19548	46.1915695826449\\
71.375	0.19914	48.4938163624235\\
71.375	0.2028	50.8624530793651\\
71.375	0.20646	53.2974797334695\\
71.375	0.21012	55.7988963247367\\
71.375	0.21378	58.3667028531669\\
71.375	0.21744	61.0008993187599\\
71.375	0.2211	63.7014857215158\\
71.375	0.22476	66.4684620614346\\
71.375	0.22842	69.3018283385162\\
71.375	0.23208	72.2015845527607\\
71.375	0.23574	75.1677307041682\\
71.375	0.2394	78.2002667927384\\
71.375	0.24306	81.2991928184716\\
71.375	0.24672	84.4645087813676\\
71.375	0.25038	87.6962146814265\\
71.375	0.25404	90.9943105186483\\
71.375	0.2577	94.3587962930329\\
71.375	0.26136	97.7896720045804\\
71.375	0.26502	101.286937653291\\
71.375	0.26868	104.850593239164\\
71.375	0.27234	108.4806387622\\
71.375	0.276	112.177074222399\\
71.75	0.093	8.83319430235701\\
71.75	0.09666	9.27693998820184\\
71.75	0.10032	9.78707561120953\\
71.75	0.10398	10.3636011713801\\
71.75	0.10764	11.0065166687135\\
71.75	0.1113	11.7158221032099\\
71.75	0.11496	12.4915174748691\\
71.75	0.11862	13.3336027836911\\
71.75	0.12228	14.2420780296761\\
71.75	0.12594	15.2169432128239\\
71.75	0.1296	16.2581983331346\\
71.75	0.13326	17.3658433906082\\
71.75	0.13692	18.5398783852446\\
71.75	0.14058	19.7803033170439\\
71.75	0.14424	21.0871181860061\\
71.75	0.1479	22.4603229921312\\
71.75	0.15156	23.8999177354191\\
71.75	0.15522	25.4059024158699\\
71.75	0.15888	26.9782770334836\\
71.75	0.16254	28.6170415882602\\
71.75	0.1662	30.3221960801996\\
71.75	0.16986	32.0937405093019\\
71.75	0.17352	33.9316748755671\\
71.75	0.17718	35.8359991789952\\
71.75	0.18084	37.8067134195861\\
71.75	0.1845	39.8438175973399\\
71.75	0.18816	41.9473117122566\\
71.75	0.19182	44.1171957643362\\
71.75	0.19548	46.3534697535786\\
71.75	0.19914	48.6561336799839\\
71.75	0.2028	51.0251875435521\\
71.75	0.20646	53.4606313442832\\
71.75	0.21012	55.9624650821771\\
71.75	0.21378	58.5306887572339\\
71.75	0.21744	61.1653023694536\\
71.75	0.2211	63.8663059188361\\
71.75	0.22476	66.6336994053816\\
71.75	0.22842	69.4674828290898\\
71.75	0.23208	72.367656189961\\
71.75	0.23574	75.3342194879951\\
71.75	0.2394	78.367172723192\\
71.75	0.24306	81.4665158955518\\
71.75	0.24672	84.6322490050745\\
71.75	0.25038	87.8643720517601\\
71.75	0.25404	91.1628850356085\\
71.75	0.2577	94.5277879566198\\
71.75	0.26136	97.959080814794\\
71.75	0.26502	101.456763610131\\
71.75	0.26868	105.020836342631\\
71.75	0.27234	108.651299012294\\
71.75	0.276	112.348151619119\\
72.125	0.093	8.98715673693209\\
72.125	0.09666	9.43131956940356\\
72.125	0.10032	9.94187233903791\\
72.125	0.10398	10.5188150458352\\
72.125	0.10764	11.1621476897952\\
72.125	0.1113	11.8718702709182\\
72.125	0.11496	12.6479827892041\\
72.125	0.11862	13.4904852446528\\
72.125	0.12228	14.3993776372644\\
72.125	0.12594	15.3746599670389\\
72.125	0.1296	16.4163322339762\\
72.125	0.13326	17.5243944380764\\
72.125	0.13692	18.6988465793395\\
72.125	0.14058	19.9396886577655\\
72.125	0.14424	21.2469206733543\\
72.125	0.1479	22.6205426261061\\
72.125	0.15156	24.0605545160207\\
72.125	0.15522	25.5669563430981\\
72.125	0.15888	27.1397481073385\\
72.125	0.16254	28.7789298087417\\
72.125	0.1662	30.4845014473078\\
72.125	0.16986	32.2564630230368\\
72.125	0.17352	34.0948145359286\\
72.125	0.17718	35.9995559859833\\
72.125	0.18084	37.9706873732009\\
72.125	0.1845	40.0082086975814\\
72.125	0.18816	42.1121199591247\\
72.125	0.19182	44.2824211578309\\
72.125	0.19548	46.5191122937\\
72.125	0.19914	48.822193366732\\
72.125	0.2028	51.1916643769268\\
72.125	0.20646	53.6275253242845\\
72.125	0.21012	56.1297762088051\\
72.125	0.21378	58.6984170304886\\
72.125	0.21744	61.3334477893349\\
72.125	0.2211	64.0348684853441\\
72.125	0.22476	66.8026791185162\\
72.125	0.22842	69.6368796888511\\
72.125	0.23208	72.537470196349\\
72.125	0.23574	75.5044506410097\\
72.125	0.2394	78.5378210228333\\
72.125	0.24306	81.6375813418197\\
72.125	0.24672	84.8037315979691\\
72.125	0.25038	88.0362717912813\\
72.125	0.25404	91.3352019217564\\
72.125	0.2577	94.7005219893943\\
72.125	0.26136	98.1322319941951\\
72.125	0.26502	101.630331936159\\
72.125	0.26868	105.194821815285\\
72.125	0.27234	108.825701631575\\
72.125	0.276	112.522971385027\\
72.5	0.093	9.14486154069483\\
72.5	0.09666	9.58944151979296\\
72.5	0.10032	10.100411436054\\
72.5	0.10398	10.6777712894778\\
72.5	0.10764	11.3215210800646\\
72.5	0.1113	12.0316608078142\\
72.5	0.11496	12.8081904727267\\
72.5	0.11862	13.6511100748021\\
72.5	0.12228	14.5604196140404\\
72.5	0.12594	15.5361190904415\\
72.5	0.1296	16.5782085040055\\
72.5	0.13326	17.6866878547324\\
72.5	0.13692	18.8615571426221\\
72.5	0.14058	20.1028163676748\\
72.5	0.14424	21.4104655298903\\
72.5	0.1479	22.7845046292686\\
72.5	0.15156	24.2249336658099\\
72.5	0.15522	25.731752639514\\
72.5	0.15888	27.304961550381\\
72.5	0.16254	28.9445603984109\\
72.5	0.1662	30.6505491836036\\
72.5	0.16986	32.4229279059592\\
72.5	0.17352	34.2616965654777\\
72.5	0.17718	36.1668551621591\\
72.5	0.18084	38.1384036960033\\
72.5	0.1845	40.1763421670105\\
72.5	0.18816	42.2806705751805\\
72.5	0.19182	44.4513889205133\\
72.5	0.19548	46.6884972030091\\
72.5	0.19914	48.9919954226677\\
72.5	0.2028	51.3618835794892\\
72.5	0.20646	53.7981616734736\\
72.5	0.21012	56.3008297046208\\
72.5	0.21378	58.8698876729309\\
72.5	0.21744	61.5053355784039\\
72.5	0.2211	64.2071734210398\\
72.5	0.22476	66.9754012008385\\
72.5	0.22842	69.8100189178001\\
72.5	0.23208	72.7110265719245\\
72.5	0.23574	75.678424163212\\
72.5	0.2394	78.7122116916622\\
72.5	0.24306	81.8123891572753\\
72.5	0.24672	84.9789565600513\\
72.5	0.25038	88.2119138999902\\
72.5	0.25404	91.5112611770919\\
72.5	0.2577	94.8769983913565\\
72.5	0.26136	98.309125542784\\
72.5	0.26502	101.807642631374\\
72.5	0.26868	105.372549657128\\
72.5	0.27234	109.003846620044\\
72.5	0.276	112.701533520123\\
72.875	0.093	9.3063087136452\\
72.875	0.09666	9.75130583936999\\
72.875	0.10032	10.2626929022577\\
72.875	0.10398	10.8404699023082\\
72.875	0.10764	11.4846368395216\\
72.875	0.1113	12.1951937138979\\
72.875	0.11496	12.972140525437\\
72.875	0.11862	13.8154772741391\\
72.875	0.12228	14.725203960004\\
72.875	0.12594	15.7013205830318\\
72.875	0.1296	16.7438271432224\\
72.875	0.13326	17.852723640576\\
72.875	0.13692	19.0280100750924\\
72.875	0.14058	20.2696864467716\\
72.875	0.14424	21.5777527556138\\
72.875	0.1479	22.9522090016188\\
72.875	0.15156	24.3930551847867\\
72.875	0.15522	25.9002913051175\\
72.875	0.15888	27.4739173626112\\
72.875	0.16254	29.1139333572677\\
72.875	0.1662	30.8203392890871\\
72.875	0.16986	32.5931351580694\\
72.875	0.17352	34.4323209642145\\
72.875	0.17718	36.3378967075225\\
72.875	0.18084	38.3098623879934\\
72.875	0.1845	40.3482180056272\\
72.875	0.18816	42.4529635604239\\
72.875	0.19182	44.6240990523834\\
72.875	0.19548	46.8616244815058\\
72.875	0.19914	49.1655398477911\\
72.875	0.2028	51.5358451512392\\
72.875	0.20646	53.9725403918502\\
72.875	0.21012	56.4756255696241\\
72.875	0.21378	59.0451006845609\\
72.875	0.21744	61.6809657366605\\
72.875	0.2211	64.3832207259231\\
72.875	0.22476	67.1518656523484\\
72.875	0.22842	69.9869005159367\\
72.875	0.23208	72.8883253166878\\
72.875	0.23574	75.8561400546019\\
72.875	0.2394	78.8903447296788\\
72.875	0.24306	81.9909393419185\\
72.875	0.24672	85.1579238913211\\
72.875	0.25038	88.3912983778867\\
72.875	0.25404	91.6910628016151\\
72.875	0.2577	95.0572171625064\\
72.875	0.26136	98.4897614605605\\
72.875	0.26502	101.988695695777\\
72.875	0.26868	105.554019868157\\
72.875	0.27234	109.1857339777\\
72.875	0.276	112.883838024406\\
73.25	0.093	9.47149825578327\\
73.25	0.09666	9.91691252813471\\
73.25	0.10032	10.428716737649\\
73.25	0.10398	11.0069108843262\\
73.25	0.10764	11.6514949681663\\
73.25	0.1113	12.3624689891692\\
73.25	0.11496	13.139832947335\\
73.25	0.11862	13.9835868426637\\
73.25	0.12228	14.8937306751553\\
73.25	0.12594	15.8702644448097\\
73.25	0.1296	16.913188151627\\
73.25	0.13326	18.0225017956072\\
73.25	0.13692	19.1982053767503\\
73.25	0.14058	20.4402988950562\\
73.25	0.14424	21.748782350525\\
73.25	0.1479	23.1236557431567\\
73.25	0.15156	24.5649190729513\\
73.25	0.15522	26.0725723399087\\
73.25	0.15888	27.646615544029\\
73.25	0.16254	29.2870486853122\\
73.25	0.1662	30.9938717637582\\
73.25	0.16986	32.7670847793672\\
73.25	0.17352	34.606687732139\\
73.25	0.17718	36.5126806220736\\
73.25	0.18084	38.4850634491712\\
73.25	0.1845	40.5238362134316\\
73.25	0.18816	42.6289989148549\\
73.25	0.19182	44.8005515534411\\
73.25	0.19548	47.0384941291902\\
73.25	0.19914	49.3428266421021\\
73.25	0.2028	51.7135490921769\\
73.25	0.20646	54.1506614794146\\
73.25	0.21012	56.6541638038152\\
73.25	0.21378	59.2240560653785\\
73.25	0.21744	61.8603382641048\\
73.25	0.2211	64.563010399994\\
73.25	0.22476	67.3320724730461\\
73.25	0.22842	70.167524483261\\
73.25	0.23208	73.0693664306388\\
73.25	0.23574	76.0375983151795\\
73.25	0.2394	79.072220136883\\
73.25	0.24306	82.1732318957495\\
73.25	0.24672	85.3406335917787\\
73.25	0.25038	88.5744252249709\\
73.25	0.25404	91.8746067953259\\
73.25	0.2577	95.2411783028439\\
73.25	0.26136	98.6741397475247\\
73.25	0.26502	102.173491129368\\
73.25	0.26868	105.739232448375\\
73.25	0.27234	109.371363704544\\
73.25	0.276	113.069884897877\\
73.625	0.093	9.640430167109\\
73.625	0.09666	10.0862615860871\\
73.625	0.10032	10.5984829422281\\
73.625	0.10398	11.1770942355319\\
73.625	0.10764	11.8220954659986\\
73.625	0.1113	12.5334866336282\\
73.625	0.11496	13.3112677384207\\
73.625	0.11862	14.155438780376\\
73.625	0.12228	15.0659997594942\\
73.625	0.12594	16.0429506757754\\
73.625	0.1296	17.0862915292193\\
73.625	0.13326	18.1960223198261\\
73.625	0.13692	19.3721430475958\\
73.625	0.14058	20.6146537125284\\
73.625	0.14424	21.9235543146239\\
73.625	0.1479	23.2988448538823\\
73.625	0.15156	24.7405253303035\\
73.625	0.15522	26.2485957438876\\
73.625	0.15888	27.8230560946345\\
73.625	0.16254	29.4639063825444\\
73.625	0.1662	31.1711466076171\\
73.625	0.16986	32.9447767698526\\
73.625	0.17352	34.7847968692511\\
73.625	0.17718	36.6912069058124\\
73.625	0.18084	38.6640068795366\\
73.625	0.1845	40.7031967904237\\
73.625	0.18816	42.8087766384737\\
73.625	0.19182	44.9807464236865\\
73.625	0.19548	47.2191061460622\\
73.625	0.19914	49.5238558056008\\
73.625	0.2028	51.8949954023023\\
73.625	0.20646	54.3325249361666\\
73.625	0.21012	56.8364444071938\\
73.625	0.21378	59.4067538153839\\
73.625	0.21744	62.0434531607368\\
73.625	0.2211	64.7465424432527\\
73.625	0.22476	67.5160216629314\\
73.625	0.22842	70.3518908197729\\
73.625	0.23208	73.2541499137774\\
73.625	0.23574	76.2227989449447\\
73.625	0.2394	79.2578379132749\\
73.625	0.24306	82.359266818768\\
73.625	0.24672	85.5270856614239\\
73.625	0.25038	88.7612944412428\\
73.625	0.25404	92.0618931582245\\
73.625	0.2577	95.4288818123691\\
73.625	0.26136	98.8622604036765\\
73.625	0.26502	102.362028932147\\
73.625	0.26868	105.92818739778\\
73.625	0.27234	109.560735800576\\
73.625	0.276	113.259674140535\\
74	0.093	9.81310444762239\\
74	0.09666	10.2593530132271\\
74	0.10032	10.7719915159948\\
74	0.10398	11.3510199559253\\
74	0.10764	11.9964383330186\\
74	0.1113	12.7082466472749\\
74	0.11496	13.486444898694\\
74	0.11862	14.331033087276\\
74	0.12228	15.2420112130209\\
74	0.12594	16.2193792759286\\
74	0.1296	17.2631372759992\\
74	0.13326	18.3732852132328\\
74	0.13692	19.5498230876291\\
74	0.14058	20.7927508991883\\
74	0.14424	22.1020686479105\\
74	0.1479	23.4777763337955\\
74	0.15156	24.9198739568433\\
74	0.15522	26.4283615170541\\
74	0.15888	28.0032390144277\\
74	0.16254	29.6445064489642\\
74	0.1662	31.3521638206636\\
74	0.16986	33.1262111295258\\
74	0.17352	34.9666483755509\\
74	0.17718	36.8734755587389\\
74	0.18084	38.8466926790898\\
74	0.1845	40.8862997366035\\
74	0.18816	42.9922967312801\\
74	0.19182	45.1646836631196\\
74	0.19548	47.403460532122\\
74	0.19914	49.7086273382872\\
74	0.2028	52.0801840816153\\
74	0.20646	54.5181307621063\\
74	0.21012	57.0224673797602\\
74	0.21378	59.5931939345769\\
74	0.21744	62.2303104265565\\
74	0.2211	64.933816855699\\
74	0.22476	67.7037132220043\\
74	0.22842	70.5399995254726\\
74	0.23208	73.4426757661036\\
74	0.23574	76.4117419438977\\
74	0.2394	79.4471980588545\\
74	0.24306	82.5490441109742\\
74	0.24672	85.7172801002568\\
74	0.25038	88.9519060267024\\
74	0.25404	92.2529218903107\\
74	0.2577	95.6203276910819\\
74	0.26136	99.054123429016\\
74	0.26502	102.554309104113\\
74	0.26868	106.120884716373\\
74	0.27234	109.753850265796\\
74	0.276	113.453205752381\\
};
\end{axis}

\begin{axis}[%
width=4.527496cm,
height=3.870968cm,
at={(0cm,5.376344cm)},
scale only axis,
xmin=56,
xmax=74,
tick align=outside,
xlabel={$L_{cut}$},
xmajorgrids,
ymin=0.093,
ymax=0.276,
ylabel={$D_{rlx}$},
ymajorgrids,
zmin=-3.90013578541794,
zmax=52.2197059680429,
zlabel={$x_2$},
zmajorgrids,
view={-140}{50},
legend style={at={(1.03,1)},anchor=north west,legend cell align=left,align=left,draw=white!15!black}
]
\addplot3[only marks,mark=*,mark options={},mark size=1.5000pt,color=mycolor1] plot table[row sep=crcr,]{%
74	0.123	0.706917161850943\\
72	0.113	-0.725188933625476\\
61	0.095	-2.23969084196957\\
56	0.093	-0.663995623424443\\
};
\addplot3[only marks,mark=*,mark options={},mark size=1.5000pt,color=mycolor2] plot table[row sep=crcr,]{%
67	0.276	45.9418866067325\\
66	0.255	36.5734993512285\\
62	0.209	18.3955084875498\\
57	0.193	12.6775369279536\\
};
\addplot3[only marks,mark=*,mark options={},mark size=1.5000pt,color=black] plot table[row sep=crcr,]{%
69	0.104	-2.29954839560715\\
};
\addplot3[only marks,mark=*,mark options={},mark size=1.5000pt,color=black] plot table[row sep=crcr,]{%
64	0.23	25.9519553252518\\
};

\addplot3[%
surf,
opacity=0.7,
shader=interp,
colormap={mymap}{[1pt] rgb(0pt)=(0.0901961,0.239216,0.0745098); rgb(1pt)=(0.0945149,0.242058,0.0739522); rgb(2pt)=(0.0988592,0.244894,0.0733566); rgb(3pt)=(0.103229,0.247724,0.0727241); rgb(4pt)=(0.107623,0.250549,0.0720557); rgb(5pt)=(0.112043,0.253367,0.0713525); rgb(6pt)=(0.116487,0.25618,0.0706154); rgb(7pt)=(0.120956,0.258986,0.0698456); rgb(8pt)=(0.125449,0.261787,0.0690441); rgb(9pt)=(0.129967,0.264581,0.0682118); rgb(10pt)=(0.134508,0.26737,0.06735); rgb(11pt)=(0.139074,0.270152,0.0664596); rgb(12pt)=(0.143663,0.272929,0.0655416); rgb(13pt)=(0.148275,0.275699,0.0645971); rgb(14pt)=(0.152911,0.278463,0.0636271); rgb(15pt)=(0.15757,0.281221,0.0626328); rgb(16pt)=(0.162252,0.283973,0.0616151); rgb(17pt)=(0.166957,0.286719,0.060575); rgb(18pt)=(0.171685,0.289458,0.0595136); rgb(19pt)=(0.176434,0.292191,0.0584321); rgb(20pt)=(0.181207,0.294918,0.0573313); rgb(21pt)=(0.186001,0.297639,0.0562123); rgb(22pt)=(0.190817,0.300353,0.0550763); rgb(23pt)=(0.195655,0.303061,0.0539242); rgb(24pt)=(0.200514,0.305763,0.052757); rgb(25pt)=(0.205395,0.308459,0.0515759); rgb(26pt)=(0.210296,0.311149,0.0503624); rgb(27pt)=(0.215212,0.313846,0.0490067); rgb(28pt)=(0.220142,0.316548,0.0475043); rgb(29pt)=(0.22509,0.319254,0.0458704); rgb(30pt)=(0.230056,0.321962,0.0441205); rgb(31pt)=(0.235042,0.324671,0.04227); rgb(32pt)=(0.240048,0.327379,0.0403343); rgb(33pt)=(0.245078,0.330085,0.0383287); rgb(34pt)=(0.250131,0.332786,0.0362688); rgb(35pt)=(0.25521,0.335482,0.0341698); rgb(36pt)=(0.260317,0.33817,0.0320472); rgb(37pt)=(0.265451,0.340849,0.0299163); rgb(38pt)=(0.270616,0.343517,0.0277927); rgb(39pt)=(0.275813,0.346172,0.0256916); rgb(40pt)=(0.281043,0.348814,0.0236284); rgb(41pt)=(0.286307,0.35144,0.0216186); rgb(42pt)=(0.291607,0.354048,0.0196776); rgb(43pt)=(0.296945,0.356637,0.0178207); rgb(44pt)=(0.302322,0.359206,0.0160634); rgb(45pt)=(0.307739,0.361753,0.0144211); rgb(46pt)=(0.313198,0.364275,0.0129091); rgb(47pt)=(0.318701,0.366772,0.0115428); rgb(48pt)=(0.324249,0.369242,0.0103377); rgb(49pt)=(0.329843,0.371682,0.00930909); rgb(50pt)=(0.335485,0.374093,0.00847245); rgb(51pt)=(0.341176,0.376471,0.00784314); rgb(52pt)=(0.346925,0.378826,0.00732741); rgb(53pt)=(0.352735,0.381168,0.00682184); rgb(54pt)=(0.358605,0.383497,0.00632729); rgb(55pt)=(0.364532,0.385812,0.00584464); rgb(56pt)=(0.370516,0.388113,0.00537476); rgb(57pt)=(0.376552,0.390399,0.00491852); rgb(58pt)=(0.38264,0.39267,0.00447681); rgb(59pt)=(0.388777,0.394925,0.00405048); rgb(60pt)=(0.394962,0.397164,0.00364042); rgb(61pt)=(0.401191,0.399386,0.00324749); rgb(62pt)=(0.407464,0.401592,0.00287258); rgb(63pt)=(0.413777,0.40378,0.00251655); rgb(64pt)=(0.420129,0.40595,0.00218028); rgb(65pt)=(0.426518,0.408102,0.00186463); rgb(66pt)=(0.432942,0.410234,0.00157049); rgb(67pt)=(0.439399,0.412348,0.00129873); rgb(68pt)=(0.445885,0.414441,0.00105022); rgb(69pt)=(0.452401,0.416515,0.000825833); rgb(70pt)=(0.458942,0.418567,0.000626441); rgb(71pt)=(0.465508,0.420599,0.00045292); rgb(72pt)=(0.472096,0.422609,0.000306141); rgb(73pt)=(0.478704,0.424596,0.000186979); rgb(74pt)=(0.485331,0.426562,9.63073e-05); rgb(75pt)=(0.491973,0.428504,3.49981e-05); rgb(76pt)=(0.498628,0.430422,3.92506e-06); rgb(77pt)=(0.505323,0.432315,0); rgb(78pt)=(0.512206,0.434168,0); rgb(79pt)=(0.519282,0.435983,0); rgb(80pt)=(0.526529,0.437764,0); rgb(81pt)=(0.533922,0.439512,0); rgb(82pt)=(0.54144,0.441232,0); rgb(83pt)=(0.549059,0.442927,0); rgb(84pt)=(0.556756,0.444599,0); rgb(85pt)=(0.564508,0.446252,0); rgb(86pt)=(0.572292,0.447889,0); rgb(87pt)=(0.580084,0.449514,0); rgb(88pt)=(0.587863,0.451129,0); rgb(89pt)=(0.595604,0.452737,0); rgb(90pt)=(0.603284,0.454343,0); rgb(91pt)=(0.610882,0.455948,0); rgb(92pt)=(0.618373,0.457556,0); rgb(93pt)=(0.625734,0.459171,0); rgb(94pt)=(0.632943,0.460795,0); rgb(95pt)=(0.639976,0.462432,0); rgb(96pt)=(0.64681,0.464084,0); rgb(97pt)=(0.653423,0.465756,0); rgb(98pt)=(0.659791,0.46745,0); rgb(99pt)=(0.665891,0.469169,0); rgb(100pt)=(0.6717,0.470916,0); rgb(101pt)=(0.677195,0.472696,0); rgb(102pt)=(0.682353,0.47451,0); rgb(103pt)=(0.687242,0.476355,0); rgb(104pt)=(0.691952,0.478225,0); rgb(105pt)=(0.696497,0.480118,0); rgb(106pt)=(0.700887,0.482033,0); rgb(107pt)=(0.705134,0.483968,0); rgb(108pt)=(0.709251,0.485921,0); rgb(109pt)=(0.713249,0.487891,0); rgb(110pt)=(0.71714,0.489876,0); rgb(111pt)=(0.720936,0.491875,0); rgb(112pt)=(0.724649,0.493887,0); rgb(113pt)=(0.72829,0.495909,0); rgb(114pt)=(0.731872,0.49794,0); rgb(115pt)=(0.735406,0.499979,0); rgb(116pt)=(0.738904,0.502025,0); rgb(117pt)=(0.742378,0.504075,0); rgb(118pt)=(0.74584,0.506128,0); rgb(119pt)=(0.749302,0.508182,0); rgb(120pt)=(0.752775,0.510237,0); rgb(121pt)=(0.756272,0.51229,0); rgb(122pt)=(0.759804,0.514339,0); rgb(123pt)=(0.763384,0.516385,0); rgb(124pt)=(0.767022,0.518424,0); rgb(125pt)=(0.770731,0.520455,0); rgb(126pt)=(0.774523,0.522478,0); rgb(127pt)=(0.77841,0.524489,0); rgb(128pt)=(0.782391,0.526491,0); rgb(129pt)=(0.786402,0.528496,0); rgb(130pt)=(0.790431,0.530506,0); rgb(131pt)=(0.794478,0.532521,0); rgb(132pt)=(0.798541,0.534539,0); rgb(133pt)=(0.802619,0.53656,0); rgb(134pt)=(0.806712,0.538584,0); rgb(135pt)=(0.81082,0.540609,0); rgb(136pt)=(0.81494,0.542635,0); rgb(137pt)=(0.819074,0.54466,0); rgb(138pt)=(0.823219,0.546686,0); rgb(139pt)=(0.827374,0.548709,0); rgb(140pt)=(0.831541,0.55073,0); rgb(141pt)=(0.835716,0.552749,0); rgb(142pt)=(0.8399,0.554763,0); rgb(143pt)=(0.844092,0.556774,0); rgb(144pt)=(0.848292,0.558779,0); rgb(145pt)=(0.852497,0.560778,0); rgb(146pt)=(0.856708,0.562771,0); rgb(147pt)=(0.860924,0.564756,0); rgb(148pt)=(0.865143,0.566733,0); rgb(149pt)=(0.869366,0.568701,0); rgb(150pt)=(0.873592,0.57066,0); rgb(151pt)=(0.877819,0.572608,0); rgb(152pt)=(0.882047,0.574545,0); rgb(153pt)=(0.886275,0.576471,0); rgb(154pt)=(0.890659,0.578362,0); rgb(155pt)=(0.895333,0.580203,0); rgb(156pt)=(0.900258,0.581999,0); rgb(157pt)=(0.905397,0.583755,0); rgb(158pt)=(0.910711,0.585479,0); rgb(159pt)=(0.916164,0.587176,0); rgb(160pt)=(0.921717,0.588852,0); rgb(161pt)=(0.927333,0.590513,0); rgb(162pt)=(0.932974,0.592166,0); rgb(163pt)=(0.938602,0.593815,0); rgb(164pt)=(0.94418,0.595468,0); rgb(165pt)=(0.949669,0.59713,0); rgb(166pt)=(0.955033,0.598808,0); rgb(167pt)=(0.960233,0.600507,0); rgb(168pt)=(0.965232,0.602233,0); rgb(169pt)=(0.969992,0.603992,0); rgb(170pt)=(0.974475,0.605791,0); rgb(171pt)=(0.978643,0.607636,0); rgb(172pt)=(0.98246,0.609532,0); rgb(173pt)=(0.985886,0.611486,0); rgb(174pt)=(0.988885,0.613503,0); rgb(175pt)=(0.991419,0.61559,0); rgb(176pt)=(0.99345,0.617753,0); rgb(177pt)=(0.99494,0.619997,0); rgb(178pt)=(0.995851,0.622329,0); rgb(179pt)=(0.996226,0.624763,0); rgb(180pt)=(0.996512,0.627352,0); rgb(181pt)=(0.996788,0.630095,0); rgb(182pt)=(0.997053,0.632982,0); rgb(183pt)=(0.997308,0.636004,0); rgb(184pt)=(0.997552,0.639152,0); rgb(185pt)=(0.997785,0.642416,0); rgb(186pt)=(0.998006,0.645786,0); rgb(187pt)=(0.998217,0.649253,0); rgb(188pt)=(0.998416,0.652807,0); rgb(189pt)=(0.998605,0.656439,0); rgb(190pt)=(0.998781,0.660138,0); rgb(191pt)=(0.998946,0.663897,0); rgb(192pt)=(0.9991,0.667704,0); rgb(193pt)=(0.999242,0.67155,0); rgb(194pt)=(0.999372,0.675427,0); rgb(195pt)=(0.99949,0.679323,0); rgb(196pt)=(0.999596,0.68323,0); rgb(197pt)=(0.99969,0.687139,0); rgb(198pt)=(0.999771,0.691039,0); rgb(199pt)=(0.999841,0.694921,0); rgb(200pt)=(0.999898,0.698775,0); rgb(201pt)=(0.999942,0.702592,0); rgb(202pt)=(0.999974,0.706363,0); rgb(203pt)=(0.999994,0.710077,0); rgb(204pt)=(1,0.713725,0); rgb(205pt)=(1,0.717341,0); rgb(206pt)=(1,0.720963,0); rgb(207pt)=(1,0.724591,0); rgb(208pt)=(1,0.728226,0); rgb(209pt)=(1,0.731867,0); rgb(210pt)=(1,0.735514,0); rgb(211pt)=(1,0.739167,0); rgb(212pt)=(1,0.742827,0); rgb(213pt)=(1,0.746493,0); rgb(214pt)=(1,0.750165,0); rgb(215pt)=(1,0.753843,0); rgb(216pt)=(1,0.757527,0); rgb(217pt)=(1,0.761217,0); rgb(218pt)=(1,0.764913,0); rgb(219pt)=(1,0.768615,0); rgb(220pt)=(1,0.772324,0); rgb(221pt)=(1,0.776038,0); rgb(222pt)=(1,0.779758,0); rgb(223pt)=(1,0.783484,0); rgb(224pt)=(1,0.787215,0); rgb(225pt)=(1,0.790953,0); rgb(226pt)=(1,0.794696,0); rgb(227pt)=(1,0.798445,0); rgb(228pt)=(1,0.8022,0); rgb(229pt)=(1,0.805961,0); rgb(230pt)=(1,0.809727,0); rgb(231pt)=(1,0.8135,0); rgb(232pt)=(1,0.817278,0); rgb(233pt)=(1,0.821063,0); rgb(234pt)=(1,0.824854,0); rgb(235pt)=(1,0.828652,0); rgb(236pt)=(1,0.832455,0); rgb(237pt)=(1,0.836265,0); rgb(238pt)=(1,0.840081,0); rgb(239pt)=(1,0.843903,0); rgb(240pt)=(1,0.847732,0); rgb(241pt)=(1,0.851566,0); rgb(242pt)=(1,0.855406,0); rgb(243pt)=(1,0.859253,0); rgb(244pt)=(1,0.863106,0); rgb(245pt)=(1,0.866964,0); rgb(246pt)=(1,0.870829,0); rgb(247pt)=(1,0.8747,0); rgb(248pt)=(1,0.878577,0); rgb(249pt)=(1,0.88246,0); rgb(250pt)=(1,0.886349,0); rgb(251pt)=(1,0.890243,0); rgb(252pt)=(1,0.894144,0); rgb(253pt)=(1,0.898051,0); rgb(254pt)=(1,0.901964,0); rgb(255pt)=(1,0.905882,0)},
mesh/rows=49]
table[row sep=crcr,header=false] {%
%
56	0.093	-0.774334427159586\\
56	0.09666	-0.624731683335815\\
56	0.10032	-0.449584953080116\\
56	0.10398	-0.248894236392514\\
56	0.10764	-0.0226595332730053\\
56	0.1113	0.229119156278419\\
56	0.11496	0.506441832261782\\
56	0.11862	0.80930849467703\\
56	0.12228	1.1377191435242\\
56	0.12594	1.49167377880329\\
56	0.1296	1.87117240051427\\
56	0.13326	2.27621500865718\\
56	0.13692	2.706801603232\\
56	0.14058	3.16293218423874\\
56	0.14424	3.64460675167738\\
56	0.1479	4.15182530554794\\
56	0.15156	4.68458784585042\\
56	0.15522	5.24289437258479\\
56	0.15888	5.82674488575109\\
56	0.16254	6.43613938534931\\
56	0.1662	7.07107787137943\\
56	0.16986	7.73156034384146\\
56	0.17352	8.41758680273542\\
56	0.17718	9.12915724806128\\
56	0.18084	9.86627167981906\\
56	0.1845	10.6289300980087\\
56	0.18816	11.4171325026303\\
56	0.19182	12.2308788936839\\
56	0.19548	13.0701692711693\\
56	0.19914	13.9350036350866\\
56	0.2028	14.8253819854359\\
56	0.20646	15.7413043222171\\
56	0.21012	16.6827706454301\\
56	0.21378	17.6497809550751\\
56	0.21744	18.642335251152\\
56	0.2211	19.6604335336609\\
56	0.22476	20.7040758026016\\
56	0.22842	21.7732620579742\\
56	0.23208	22.8679922997788\\
56	0.23574	23.9882665280153\\
56	0.2394	25.1340847426837\\
56	0.24306	26.3054469437839\\
56	0.24672	27.5023531313161\\
56	0.25038	28.7248033052803\\
56	0.25404	29.9727974656763\\
56	0.2577	31.2463356125043\\
56	0.26136	32.5454177457641\\
56	0.26502	33.8700438654559\\
56	0.26868	35.2202139715796\\
56	0.27234	36.5959280641352\\
56	0.276	37.9971861431227\\
56.375	0.093	-0.89511921239191\\
56.375	0.09666	-0.738288001408483\\
56.375	0.10032	-0.555912803993127\\
56.375	0.10398	-0.347993620145862\\
56.375	0.10764	-0.114530449866697\\
56.375	0.1113	0.144476706844383\\
56.375	0.11496	0.429027849987403\\
56.375	0.11862	0.739122979562307\\
56.375	0.12228	1.07476209556914\\
56.375	0.12594	1.43594519800788\\
56.375	0.1296	1.82267228687852\\
56.375	0.13326	2.23494336218108\\
56.375	0.13692	2.67275842391557\\
56.375	0.14058	3.13611747208196\\
56.375	0.14424	3.62502050668026\\
56.375	0.1479	4.13946752771048\\
56.375	0.15156	4.67945853517261\\
56.375	0.15522	5.24499352906665\\
56.375	0.15888	5.83607250939261\\
56.375	0.16254	6.45269547615048\\
56.375	0.1662	7.09486242934026\\
56.375	0.16986	7.76257336896195\\
56.375	0.17352	8.45582829501556\\
56.375	0.17718	9.17462720750109\\
56.375	0.18084	9.91897010641852\\
56.375	0.1845	10.6888569917679\\
56.375	0.18816	11.4842878635491\\
56.375	0.19182	12.3052627217623\\
56.375	0.19548	13.1517815664074\\
56.375	0.19914	14.0238443974844\\
56.375	0.2028	14.9214512149933\\
56.375	0.20646	15.8446020189341\\
56.375	0.21012	16.7932968093069\\
56.375	0.21378	17.7675355861115\\
56.375	0.21744	18.7673183493481\\
56.375	0.2211	19.7926450990166\\
56.375	0.22476	20.8435158351169\\
56.375	0.22842	21.9199305576492\\
56.375	0.23208	23.0218892666134\\
56.375	0.23574	24.1493919620096\\
56.375	0.2394	25.3024386438376\\
56.375	0.24306	26.4810293120976\\
56.375	0.24672	27.6851639667895\\
56.375	0.25038	28.9148426079132\\
56.375	0.25404	30.1700652354689\\
56.375	0.2577	31.4508318494565\\
56.375	0.26136	32.757142449876\\
56.375	0.26502	34.0889970367275\\
56.375	0.26868	35.4463956100108\\
56.375	0.27234	36.8293381697261\\
56.375	0.276	38.2378247158733\\
56.75	0.093	-1.01353532002414\\
56.75	0.09666	-0.849475641881041\\
56.75	0.10032	-0.659871977306043\\
56.75	0.10398	-0.444724326299129\\
56.75	0.10764	-0.204032688860279\\
56.75	0.1113	0.0622029350104576\\
56.75	0.11496	0.353982545313105\\
56.75	0.11862	0.67130614204768\\
56.75	0.12228	1.01417372521417\\
56.75	0.12594	1.38258529481255\\
56.75	0.1296	1.77654085084288\\
56.75	0.13326	2.1960403933051\\
56.75	0.13692	2.64108392219922\\
56.75	0.14058	3.11167143752527\\
56.75	0.14424	3.60780293928326\\
56.75	0.1479	4.12947842747312\\
56.75	0.15156	4.67669790209491\\
56.75	0.15522	5.2494613631486\\
56.75	0.15888	5.84776881063422\\
56.75	0.16254	6.47162024455175\\
56.75	0.1662	7.12101566490119\\
56.75	0.16986	7.79595507168253\\
56.75	0.17352	8.4964384648958\\
56.75	0.17718	9.22246584454098\\
56.75	0.18084	9.97403721061808\\
56.75	0.1845	10.7511525631271\\
56.75	0.18816	11.553811902068\\
56.75	0.19182	12.3820152274408\\
56.75	0.19548	13.2357625392456\\
56.75	0.19914	14.1150538374822\\
56.75	0.2028	15.0198891221508\\
56.75	0.20646	15.9502683932513\\
56.75	0.21012	16.9061916507837\\
56.75	0.21378	17.887658894748\\
56.75	0.21744	18.8946701251442\\
56.75	0.2211	19.9272253419723\\
56.75	0.22476	20.9853245452324\\
56.75	0.22842	22.0689677349243\\
56.75	0.23208	23.1781549110482\\
56.75	0.23574	24.312886073604\\
56.75	0.2394	25.4731612225917\\
56.75	0.24306	26.6589803580113\\
56.75	0.24672	27.8703434798628\\
56.75	0.25038	29.1072505881463\\
56.75	0.25404	30.3697016828617\\
56.75	0.2577	31.6576967640089\\
56.75	0.26136	32.9712358315881\\
56.75	0.26502	34.3103188855992\\
56.75	0.26868	35.6749459260421\\
56.75	0.27234	37.0651169529171\\
56.75	0.276	38.4808319662239\\
57.125	0.093	-1.12958275005631\\
57.125	0.09666	-0.958294604753545\\
57.125	0.10032	-0.761462473018884\\
57.125	0.10398	-0.539086354852321\\
57.125	0.10764	-0.291166250253815\\
57.125	0.1113	-0.0177021592234148\\
57.125	0.11496	0.281305918238889\\
57.125	0.11862	0.605857982133127\\
57.125	0.12228	0.955954032459271\\
57.125	0.12594	1.33159406921731\\
57.125	0.1296	1.73277809240729\\
57.125	0.13326	2.15950610202917\\
57.125	0.13692	2.61177809808295\\
57.125	0.14058	3.08959408056866\\
57.125	0.14424	3.59295404948631\\
57.125	0.1479	4.12185800483582\\
57.125	0.15156	4.67630594661727\\
57.125	0.15522	5.25629787483062\\
57.125	0.15888	5.8618337894759\\
57.125	0.16254	6.49291369055308\\
57.125	0.1662	7.14953757806218\\
57.125	0.16986	7.83170545200318\\
57.125	0.17352	8.53941731237611\\
57.125	0.17718	9.27267315918095\\
57.125	0.18084	10.0314729924177\\
57.125	0.1845	10.8158168120864\\
57.125	0.18816	11.6257046181869\\
57.125	0.19182	12.4611364107194\\
57.125	0.19548	13.3221121896838\\
57.125	0.19914	14.2086319550801\\
57.125	0.2028	15.1206957069084\\
57.125	0.20646	16.0583034451685\\
57.125	0.21012	17.0214551698606\\
57.125	0.21378	18.0101508809845\\
57.125	0.21744	19.0243905785404\\
57.125	0.2211	20.0641742625282\\
57.125	0.22476	21.1295019329479\\
57.125	0.22842	22.2203735897995\\
57.125	0.23208	23.336789233083\\
57.125	0.23574	24.4787488627985\\
57.125	0.2394	25.6462524789459\\
57.125	0.24306	26.8393000815251\\
57.125	0.24672	28.0578916705363\\
57.125	0.25038	29.3020272459794\\
57.125	0.25404	30.5717068078544\\
57.125	0.2577	31.8669303561614\\
57.125	0.26136	33.1876978909002\\
57.125	0.26502	34.5340094120709\\
57.125	0.26868	35.9058649196736\\
57.125	0.27234	37.3032644137081\\
57.125	0.276	38.7262078941746\\
57.5	0.093	-1.24326150248837\\
57.5	0.09666	-1.06474489002598\\
57.5	0.10032	-0.860684291131644\\
57.5	0.10398	-0.631079705805403\\
57.5	0.10764	-0.375931134047269\\
57.5	0.1113	-0.0952385758572127\\
57.5	0.11496	0.210997968764783\\
57.5	0.11862	0.542778499818656\\
57.5	0.12228	0.900103017304456\\
57.5	0.12594	1.28297152122218\\
57.5	0.1296	1.69138401157178\\
57.5	0.13326	2.12534048835333\\
57.5	0.13692	2.58484095156678\\
57.5	0.14058	3.06988540121215\\
57.5	0.14424	3.58047383728942\\
57.5	0.1479	4.11660625979862\\
57.5	0.15156	4.67828266873972\\
57.5	0.15522	5.26550306411272\\
57.5	0.15888	5.87826744591766\\
57.5	0.16254	6.5165758141545\\
57.5	0.1662	7.18042816882327\\
57.5	0.16986	7.86982450992392\\
57.5	0.17352	8.58476483745651\\
57.5	0.17718	9.325249151421\\
57.5	0.18084	10.0912774518174\\
57.5	0.1845	10.8828497386457\\
57.5	0.18816	11.699966011906\\
57.5	0.19182	12.5426262715981\\
57.5	0.19548	13.4108305177222\\
57.5	0.19914	14.3045787502781\\
57.5	0.2028	15.223870969266\\
57.5	0.20646	16.1687071746858\\
57.5	0.21012	17.1390873665375\\
57.5	0.21378	18.1350115448212\\
57.5	0.21744	19.1564797095367\\
57.5	0.2211	20.2034918606841\\
57.5	0.22476	21.2760479982635\\
57.5	0.22842	22.3741481222748\\
57.5	0.23208	23.497792232718\\
57.5	0.23574	24.6469803295931\\
57.5	0.2394	25.8217124129001\\
57.5	0.24306	27.021988482639\\
57.5	0.24672	28.2478085388099\\
57.5	0.25038	29.4991725814126\\
57.5	0.25404	30.7760806104473\\
57.5	0.2577	32.0785326259139\\
57.5	0.26136	33.4065286278123\\
57.5	0.26502	34.7600686161428\\
57.5	0.26868	36.1391525909051\\
57.5	0.27234	37.5437805520993\\
57.5	0.276	38.9739524997255\\
57.875	0.093	-1.35457157732037\\
57.875	0.09666	-1.16882649769831\\
57.875	0.10032	-0.957537431644319\\
57.875	0.10398	-0.720704379158429\\
57.875	0.10764	-0.458327340240631\\
57.875	0.1113	-0.170406314890919\\
57.875	0.11496	0.143058696890733\\
57.875	0.11862	0.482067695104263\\
57.875	0.12228	0.846620679749726\\
57.875	0.12594	1.2367176508271\\
57.875	0.1296	1.65235860833637\\
57.875	0.13326	2.09354355227757\\
57.875	0.13692	2.56027248265068\\
57.875	0.14058	3.05254539945571\\
57.875	0.14424	3.57036230269264\\
57.875	0.1479	4.11372319236148\\
57.875	0.15156	4.68262806846225\\
57.875	0.15522	5.27707693099492\\
57.875	0.15888	5.89706977995951\\
57.875	0.16254	6.54260661535601\\
57.875	0.1662	7.21368743718443\\
57.875	0.16986	7.91031224544474\\
57.875	0.17352	8.63248104013698\\
57.875	0.17718	9.38019382126114\\
57.875	0.18084	10.1534505888172\\
57.875	0.1845	10.9522513428052\\
57.875	0.18816	11.7765960832251\\
57.875	0.19182	12.6264848100769\\
57.875	0.19548	13.5019175233606\\
57.875	0.19914	14.4028942230762\\
57.875	0.2028	15.3294149092238\\
57.875	0.20646	16.2814795818032\\
57.875	0.21012	17.2590882408146\\
57.875	0.21378	18.2622408862579\\
57.875	0.21744	19.2909375181331\\
57.875	0.2211	20.3451781364402\\
57.875	0.22476	21.4249627411792\\
57.875	0.22842	22.5302913323501\\
57.875	0.23208	23.661163909953\\
57.875	0.23574	24.8175804739877\\
57.875	0.2394	25.9995410244544\\
57.875	0.24306	27.207045561353\\
57.875	0.24672	28.4400940846835\\
57.875	0.25038	29.6986865944459\\
57.875	0.25404	30.9828230906402\\
57.875	0.2577	32.2925035732665\\
57.875	0.26136	33.6277280423246\\
57.875	0.26502	34.9884964978147\\
57.875	0.26868	36.3748089397366\\
57.875	0.27234	37.7866653680905\\
57.875	0.276	39.2240657828764\\
58.25	0.093	-1.46351297455227\\
58.25	0.09666	-1.27053942777054\\
58.25	0.10032	-1.0520218945569\\
58.25	0.10398	-0.807960374911373\\
58.25	0.10764	-0.538354868833883\\
58.25	0.1113	-0.243205376324514\\
58.25	0.11496	0.0774881026167655\\
58.25	0.11862	0.423725567989973\\
58.25	0.12228	0.795507019795085\\
58.25	0.12594	1.1928324580321\\
58.25	0.1296	1.61570188270106\\
58.25	0.13326	2.0641152938019\\
58.25	0.13692	2.53807269133466\\
58.25	0.14058	3.03757407529935\\
58.25	0.14424	3.56261944569596\\
58.25	0.1479	4.11320880252445\\
58.25	0.15156	4.68934214578487\\
58.25	0.15522	5.2910194754772\\
58.25	0.15888	5.91824079160144\\
58.25	0.16254	6.57100609415761\\
58.25	0.1662	7.24931538314568\\
58.25	0.16986	7.95316865856566\\
58.25	0.17352	8.68256592041755\\
58.25	0.17718	9.43750716870136\\
58.25	0.18084	10.2179924034171\\
58.25	0.1845	11.0240216245647\\
58.25	0.18816	11.8555948321443\\
58.25	0.19182	12.7127120261557\\
58.25	0.19548	13.5953732065991\\
58.25	0.19914	14.5035783734744\\
58.25	0.2028	15.4373275267816\\
58.25	0.20646	16.3966206665207\\
58.25	0.21012	17.3814577926918\\
58.25	0.21378	18.3918389052947\\
58.25	0.21744	19.4277640043295\\
58.25	0.2211	20.4892330897963\\
58.25	0.22476	21.576246161695\\
58.25	0.22842	22.6888032200256\\
58.25	0.23208	23.8269042647881\\
58.25	0.23574	24.9905492959825\\
58.25	0.2394	26.1797383136088\\
58.25	0.24306	27.394471317667\\
58.25	0.24672	28.6347483081572\\
58.25	0.25038	29.9005692850793\\
58.25	0.25404	31.1919342484333\\
58.25	0.2577	32.5088431982192\\
58.25	0.26136	33.851296134437\\
58.25	0.26502	35.2192930570867\\
58.25	0.26868	36.6128339661683\\
58.25	0.27234	38.0319188616819\\
58.25	0.276	39.4765477436273\\
58.625	0.093	-1.57008569418409\\
58.625	0.09666	-1.36988368024271\\
58.625	0.10032	-1.14413767986942\\
58.625	0.10398	-0.892847693064224\\
58.625	0.10764	-0.616013719827086\\
58.625	0.1113	-0.313635760158054\\
58.625	0.11496	0.0142861859428827\\
58.625	0.11862	0.367752118475746\\
58.625	0.12228	0.746762037440522\\
58.625	0.12594	1.1513159428372\\
58.625	0.1296	1.58141383466581\\
58.625	0.13326	2.03705571292631\\
58.625	0.13692	2.51824157761872\\
58.625	0.14058	3.02497142874306\\
58.625	0.14424	3.55724526629934\\
58.625	0.1479	4.11506309028749\\
58.625	0.15156	4.69842490070756\\
58.625	0.15522	5.30733069755955\\
58.625	0.15888	5.94178048084346\\
58.625	0.16254	6.60177425055928\\
58.625	0.1662	7.28731200670701\\
58.625	0.16986	7.99839374928664\\
58.625	0.17352	8.7350194782982\\
58.625	0.17718	9.49718919374167\\
58.625	0.18084	10.284902895617\\
58.625	0.1845	11.0981605839243\\
58.625	0.18816	11.9369622586635\\
58.625	0.19182	12.8013079198347\\
58.625	0.19548	13.6911975674377\\
58.625	0.19914	14.6066312014726\\
58.625	0.2028	15.5476088219395\\
58.625	0.20646	16.5141304288383\\
58.625	0.21012	17.506196022169\\
58.625	0.21378	18.5238056019316\\
58.625	0.21744	19.5669591681261\\
58.625	0.2211	20.6356567207525\\
58.625	0.22476	21.7298982598108\\
58.625	0.22842	22.8496837853011\\
58.625	0.23208	23.9950132972232\\
58.625	0.23574	25.1658867955773\\
58.625	0.2394	26.3623042803633\\
58.625	0.24306	27.5842657515812\\
58.625	0.24672	28.831771209231\\
58.625	0.25038	30.1048206533127\\
58.625	0.25404	31.4034140838264\\
58.625	0.2577	32.7275515007719\\
58.625	0.26136	34.0772329041494\\
58.625	0.26502	35.4524582939588\\
58.625	0.26868	36.8532276702\\
58.625	0.27234	38.2795410328733\\
58.625	0.276	39.7313983819784\\
59	0.093	-1.67428973621583\\
59	0.09666	-1.46685925511481\\
59	0.10032	-1.23388478758184\\
59	0.10398	-0.97536633361697\\
59	0.10764	-0.691303893220203\\
59	0.1113	-0.381697466391515\\
59	0.11496	-0.0465470531308938\\
59	0.11862	0.314147346561612\\
59	0.12228	0.700385732686044\\
59	0.12594	1.1121681052424\\
59	0.1296	1.54949446423064\\
59	0.13326	2.01236480965081\\
59	0.13692	2.50077914150289\\
59	0.14058	3.0147374597869\\
59	0.14424	3.5542397645028\\
59	0.1479	4.11928605565062\\
59	0.15156	4.70987633323036\\
59	0.15522	5.326010597242\\
59	0.15888	5.96768884768556\\
59	0.16254	6.63491108456103\\
59	0.1662	7.32767730786843\\
59	0.16986	8.04598751760771\\
59	0.17352	8.78984171377893\\
59	0.17718	9.55923989638205\\
59	0.18084	10.3541820654171\\
59	0.1845	11.174668220884\\
59	0.18816	12.0206983627829\\
59	0.19182	12.8922724911137\\
59	0.19548	13.7893906058764\\
59	0.19914	14.712052707071\\
59	0.2028	15.6602587946975\\
59	0.20646	16.6340088687559\\
59	0.21012	17.6333029292463\\
59	0.21378	18.6581409761685\\
59	0.21744	19.7085230095227\\
59	0.2211	20.7844490293088\\
59	0.22476	21.8859190355268\\
59	0.22842	23.0129330281767\\
59	0.23208	24.1654910072585\\
59	0.23574	25.3435929727722\\
59	0.2394	26.5472389247179\\
59	0.24306	27.7764288630954\\
59	0.24672	29.0311627879049\\
59	0.25038	30.3114406991463\\
59	0.25404	31.6172625968196\\
59	0.2577	32.9486284809248\\
59	0.26136	34.3055383514619\\
59	0.26502	35.687992208431\\
59	0.26868	37.0959900518319\\
59	0.27234	38.5295318816648\\
59	0.276	39.9886176979295\\
59.375	0.093	-1.7761251006475\\
59.375	0.09666	-1.56146615238682\\
59.375	0.10032	-1.3212632176942\\
59.375	0.10398	-1.05551629656967\\
59.375	0.10764	-0.764225389013236\\
59.375	0.1113	-0.447390495024891\\
59.375	0.11496	-0.105011614604614\\
59.375	0.11862	0.262911252247555\\
59.375	0.12228	0.656378105531644\\
59.375	0.12594	1.07538894524765\\
59.375	0.1296	1.51994377139555\\
59.375	0.13326	1.99004258397538\\
59.375	0.13692	2.48568538298712\\
59.375	0.14058	3.00687216843078\\
59.375	0.14424	3.55360294030634\\
59.375	0.1479	4.12587769861382\\
59.375	0.15156	4.72369644335322\\
59.375	0.15522	5.34705917452451\\
59.375	0.15888	5.99596589212774\\
59.375	0.16254	6.67041659616287\\
59.375	0.1662	7.37041128662992\\
59.375	0.16986	8.09594996352886\\
59.375	0.17352	8.84703262685974\\
59.375	0.17718	9.62365927662252\\
59.375	0.18084	10.4258299128172\\
59.375	0.1845	11.2535445354438\\
59.375	0.18816	12.1068031445023\\
59.375	0.19182	12.9856057399928\\
59.375	0.19548	13.8899523219151\\
59.375	0.19914	14.8198428902694\\
59.375	0.2028	15.7752774450556\\
59.375	0.20646	16.7562559862737\\
59.375	0.21012	17.7627785139237\\
59.375	0.21378	18.7948450280056\\
59.375	0.21744	19.8524555285194\\
59.375	0.2211	20.9356100154651\\
59.375	0.22476	22.0443084888428\\
59.375	0.22842	23.1785509486523\\
59.375	0.23208	24.3383373948938\\
59.375	0.23574	25.5236678275672\\
59.375	0.2394	26.7345422466725\\
59.375	0.24306	27.9709606522097\\
59.375	0.24672	29.2329230441789\\
59.375	0.25038	30.5204294225799\\
59.375	0.25404	31.8334797874129\\
59.375	0.2577	33.1720741386777\\
59.375	0.26136	34.5362124763745\\
59.375	0.26502	35.9258948005032\\
59.375	0.26868	37.3411211110638\\
59.375	0.27234	38.7818914080563\\
59.375	0.276	40.2482056914808\\
59.75	0.093	-1.87559178747907\\
59.75	0.09666	-1.65370437205871\\
59.75	0.10032	-1.40627297020644\\
59.75	0.10398	-1.13329758192227\\
59.75	0.10764	-0.834778207206165\\
59.75	0.1113	-0.510714846058157\\
59.75	0.11496	-0.161107498478252\\
59.75	0.11862	0.214043835533587\\
59.75	0.12228	0.614739155977333\\
59.75	0.12594	1.04097846285298\\
59.75	0.1296	1.49276175616057\\
59.75	0.13326	1.97008903590004\\
59.75	0.13692	2.47296030207143\\
59.75	0.14058	3.00137555467475\\
59.75	0.14424	3.55533479371\\
59.75	0.1479	4.13483801917711\\
59.75	0.15156	4.73988523107617\\
59.75	0.15522	5.37047642940712\\
59.75	0.15888	6.02661161417\\
59.75	0.16254	6.7082907853648\\
59.75	0.1662	7.4155139429915\\
59.75	0.16986	8.1482810870501\\
59.75	0.17352	8.90659221754063\\
59.75	0.17718	9.69044733446308\\
59.75	0.18084	10.4998464378174\\
59.75	0.1845	11.3347895276037\\
59.75	0.18816	12.1952766038219\\
59.75	0.19182	13.081307666472\\
59.75	0.19548	13.992882715554\\
59.75	0.19914	14.9300017510679\\
59.75	0.2028	15.8926647730137\\
59.75	0.20646	16.8808717813915\\
59.75	0.21012	17.8946227762011\\
59.75	0.21378	18.9339177574427\\
59.75	0.21744	19.9987567251162\\
59.75	0.2211	21.0891396792216\\
59.75	0.22476	22.2050666197589\\
59.75	0.22842	23.3465375467281\\
59.75	0.23208	24.5135524601292\\
59.75	0.23574	25.7061113599623\\
59.75	0.2394	26.9242142462273\\
59.75	0.24306	28.1678611189241\\
59.75	0.24672	29.4370519780529\\
59.75	0.25038	30.7317868236136\\
59.75	0.25404	32.0520656556063\\
59.75	0.2577	33.3978884740308\\
59.75	0.26136	34.7692552788872\\
59.75	0.26502	36.1661660701756\\
59.75	0.26868	37.5886208478958\\
59.75	0.27234	39.036619612048\\
59.75	0.276	40.5101623626321\\
60.125	0.093	-1.97268979671057\\
60.125	0.09666	-1.74357391413055\\
60.125	0.10032	-1.48891404511862\\
60.125	0.10398	-1.2087101896748\\
60.125	0.10764	-0.902962347799031\\
60.125	0.1113	-0.571670519491374\\
60.125	0.11496	-0.214834704751805\\
60.125	0.11862	0.167545096419691\\
60.125	0.12228	0.575468884023099\\
60.125	0.12594	1.00893665805841\\
60.125	0.1296	1.46794841852565\\
60.125	0.13326	1.95250416542478\\
60.125	0.13692	2.46260389875582\\
60.125	0.14058	2.9982476185188\\
60.125	0.14424	3.5594353247137\\
60.125	0.1479	4.14616701734048\\
60.125	0.15156	4.75844269639919\\
60.125	0.15522	5.39626236188981\\
60.125	0.15888	6.05962601381234\\
60.125	0.16254	6.7485336521668\\
60.125	0.1662	7.46298527695316\\
60.125	0.16986	8.20298088817141\\
60.125	0.17352	8.9685204858216\\
60.125	0.17718	9.75960406990371\\
60.125	0.18084	10.5762316404177\\
60.125	0.1845	11.4184031973636\\
60.125	0.18816	12.2861187407415\\
60.125	0.19182	13.1793782705512\\
60.125	0.19548	14.0981817867929\\
60.125	0.19914	15.0425292894665\\
60.125	0.2028	16.012420778572\\
60.125	0.20646	17.0078562541094\\
60.125	0.21012	18.0288357160787\\
60.125	0.21378	19.0753591644799\\
60.125	0.21744	20.147426599313\\
60.125	0.2211	21.2450380205781\\
60.125	0.22476	22.3681934282751\\
60.125	0.22842	23.516892822404\\
60.125	0.23208	24.6911362029647\\
60.125	0.23574	25.8909235699575\\
60.125	0.2394	27.1162549233821\\
60.125	0.24306	28.3671302632386\\
60.125	0.24672	29.643549589527\\
60.125	0.25038	30.9455129022474\\
60.125	0.25404	32.2730202013997\\
60.125	0.2577	33.6260714869839\\
60.125	0.26136	35.0046667589999\\
60.125	0.26502	36.4088060174479\\
60.125	0.26868	37.8384892623279\\
60.125	0.27234	39.2937164936397\\
60.125	0.276	40.7744877113835\\
60.5	0.093	-2.06741912834199\\
60.5	0.09666	-1.83107477860231\\
60.5	0.10032	-1.56918644243074\\
60.5	0.10398	-1.28175411982725\\
60.5	0.10764	-0.968777810791819\\
60.5	0.1113	-0.630257515324505\\
60.5	0.11496	-0.26619323342528\\
60.5	0.11862	0.123415034905872\\
60.5	0.12228	0.538567289668936\\
60.5	0.12594	0.979263530863907\\
60.5	0.1296	1.44550375849081\\
60.5	0.13326	1.9372879725496\\
60.5	0.13692	2.4546161730403\\
60.5	0.14058	2.99748835996293\\
60.5	0.14424	3.56590453331749\\
60.5	0.1479	4.15986469310393\\
60.5	0.15156	4.7793688393223\\
60.5	0.15522	5.42441697197257\\
60.5	0.15888	6.09500909105476\\
60.5	0.16254	6.79114519656887\\
60.5	0.1662	7.51282528851489\\
60.5	0.16986	8.26004936689281\\
60.5	0.17352	9.03281743170266\\
60.5	0.17718	9.83112948294442\\
60.5	0.18084	10.6549855206181\\
60.5	0.1845	11.5043855447237\\
60.5	0.18816	12.3793295552612\\
60.5	0.19182	13.2798175522306\\
60.5	0.19548	14.2058495356319\\
60.5	0.19914	15.1574255054651\\
60.5	0.2028	16.1345454617303\\
60.5	0.20646	17.1372094044273\\
60.5	0.21012	18.1654173335563\\
60.5	0.21378	19.2191692491172\\
60.5	0.21744	20.29846515111\\
60.5	0.2211	21.4033050395347\\
60.5	0.22476	22.5336889143913\\
60.5	0.22842	23.6896167756798\\
60.5	0.23208	24.8710886234003\\
60.5	0.23574	26.0781044575527\\
60.5	0.2394	27.310664278137\\
60.5	0.24306	28.5687680851532\\
60.5	0.24672	29.8524158786013\\
60.5	0.25038	31.1616076584813\\
60.5	0.25404	32.4963434247932\\
60.5	0.2577	33.856623177537\\
60.5	0.26136	35.2424469167128\\
60.5	0.26502	36.6538146423205\\
60.5	0.26868	38.09072635436\\
60.5	0.27234	39.5531820528315\\
60.5	0.276	41.041181737735\\
60.875	0.093	-2.15977978237331\\
60.875	0.09666	-1.91620696547399\\
60.875	0.10032	-1.64709016214274\\
60.875	0.10398	-1.35242937237958\\
60.875	0.10764	-1.03222459618453\\
60.875	0.1113	-0.686475833557552\\
60.875	0.11496	-0.315183084498642\\
60.875	0.11862	0.081653650992159\\
60.875	0.12228	0.50403437291488\\
60.875	0.12594	0.951959081269514\\
60.875	0.1296	1.42542777605605\\
60.875	0.13326	1.92444045727451\\
60.875	0.13692	2.44899712492488\\
60.875	0.14058	2.99909777900718\\
60.875	0.14424	3.57474241952136\\
60.875	0.1479	4.17593104646747\\
60.875	0.15156	4.8026636598455\\
60.875	0.15522	5.45494025965543\\
60.875	0.15888	6.13276084589728\\
60.875	0.16254	6.83612541857104\\
60.875	0.1662	7.56503397767673\\
60.875	0.16986	8.3194865232143\\
60.875	0.17352	9.0994830551838\\
60.875	0.17718	9.90502357358522\\
60.875	0.18084	10.7361080784186\\
60.875	0.1845	11.5927365696838\\
60.875	0.18816	12.4749090473809\\
60.875	0.19182	13.38262551151\\
60.875	0.19548	14.315885962071\\
60.875	0.19914	15.2746903990639\\
60.875	0.2028	16.2590388224887\\
60.875	0.20646	17.2689312323454\\
60.875	0.21012	18.304367628634\\
60.875	0.21378	19.3653480113546\\
60.875	0.21744	20.451872380507\\
60.875	0.2211	21.5639407360914\\
60.875	0.22476	22.7015530781077\\
60.875	0.22842	23.8647094065559\\
60.875	0.23208	25.053409721436\\
60.875	0.23574	26.267654022748\\
60.875	0.2394	27.507442310492\\
60.875	0.24306	28.7727745846678\\
60.875	0.24672	30.0636508452755\\
60.875	0.25038	31.3800710923152\\
60.875	0.25404	32.7220353257868\\
60.875	0.2577	34.0895435456903\\
60.875	0.26136	35.4825957520257\\
60.875	0.26502	36.9011919447931\\
60.875	0.26868	38.3453321239923\\
60.875	0.27234	39.8150162896234\\
60.875	0.276	41.3102444416865\\
61.25	0.093	-2.24977175880457\\
61.25	0.09666	-1.9989704747456\\
61.25	0.10032	-1.72262520425469\\
61.25	0.10398	-1.42073594733187\\
61.25	0.10764	-1.09330270397716\\
61.25	0.1113	-0.740325474190524\\
61.25	0.11496	-0.361804257971951\\
61.25	0.11862	0.0422609446784996\\
61.25	0.12228	0.471870133760884\\
61.25	0.12594	0.927023309275175\\
61.25	0.1296	1.40772047122136\\
61.25	0.13326	1.91396161959948\\
61.25	0.13692	2.44574675440951\\
61.25	0.14058	3.00307587565146\\
61.25	0.14424	3.58594898332531\\
61.25	0.1479	4.19436607743108\\
61.25	0.15156	4.82832715796876\\
61.25	0.15522	5.48783222493835\\
61.25	0.15888	6.17288127833985\\
61.25	0.16254	6.88347431817328\\
61.25	0.1662	7.61961134443861\\
61.25	0.16986	8.38129235713585\\
61.25	0.17352	9.16851735626501\\
61.25	0.17718	9.98128634182608\\
61.25	0.18084	10.8195993138191\\
61.25	0.1845	11.683456272244\\
61.25	0.18816	12.5728572171008\\
61.25	0.19182	13.4878021483895\\
61.25	0.19548	14.4282910661101\\
61.25	0.19914	15.3943239702627\\
61.25	0.2028	16.3859008608471\\
61.25	0.20646	17.4030217378635\\
61.25	0.21012	18.4456866013118\\
61.25	0.21378	19.513895451192\\
61.25	0.21744	20.6076482875041\\
61.25	0.2211	21.7269451102482\\
61.25	0.22476	22.8717859194241\\
61.25	0.22842	24.0421707150319\\
61.25	0.23208	25.2380994970717\\
61.25	0.23574	26.4595722655434\\
61.25	0.2394	27.706589020447\\
61.25	0.24306	28.9791497617825\\
61.25	0.24672	30.2772544895499\\
61.25	0.25038	31.6009032037492\\
61.25	0.25404	32.9500959043805\\
61.25	0.2577	34.3248325914437\\
61.25	0.26136	35.7251132649387\\
61.25	0.26502	37.1509379248657\\
61.25	0.26868	38.6023065712246\\
61.25	0.27234	40.0792192040154\\
61.25	0.276	41.5816758232381\\
61.625	0.093	-2.33739505763572\\
61.625	0.09666	-2.07936530641708\\
61.625	0.10032	-1.79579156876652\\
61.625	0.10398	-1.48667384468407\\
61.625	0.10764	-1.15201213416966\\
61.625	0.1113	-0.791806437223375\\
61.625	0.11496	-0.406056753845174\\
61.625	0.11862	0.00523691596494658\\
61.625	0.12228	0.442074572206987\\
61.625	0.12594	0.904456214880927\\
61.625	0.1296	1.3923818439868\\
61.625	0.13326	1.90585145952456\\
61.625	0.13692	2.44486506149424\\
61.625	0.14058	3.00942264989584\\
61.625	0.14424	3.59952422472938\\
61.625	0.1479	4.21516978599479\\
61.625	0.15156	4.85635933369213\\
61.625	0.15522	5.52309286782137\\
61.625	0.15888	6.21537038838254\\
61.625	0.16254	6.93319189537563\\
61.625	0.1662	7.67655738880061\\
61.625	0.16986	8.44546686865751\\
61.625	0.17352	9.23992033494632\\
61.625	0.17718	10.0599177876671\\
61.625	0.18084	10.9054592268197\\
61.625	0.1845	11.7765446524043\\
61.625	0.18816	12.6731740644207\\
61.625	0.19182	13.5953474628691\\
61.625	0.19548	14.5430648477494\\
61.625	0.19914	15.5163262190616\\
61.625	0.2028	16.5151315768057\\
61.625	0.20646	17.5394809209818\\
61.625	0.21012	18.5893742515897\\
61.625	0.21378	19.6648115686296\\
61.625	0.21744	20.7657928721013\\
61.625	0.2211	21.892318162005\\
61.625	0.22476	23.0443874383406\\
61.625	0.22842	24.2220007011081\\
61.625	0.23208	25.4251579503075\\
61.625	0.23574	26.6538591859389\\
61.625	0.2394	27.9081044080021\\
61.625	0.24306	29.1878936164973\\
61.625	0.24672	30.4932268114244\\
61.625	0.25038	31.8241039927834\\
61.625	0.25404	33.1805251605743\\
61.625	0.2577	34.5624903147971\\
61.625	0.26136	35.9699994554518\\
61.625	0.26502	37.4030525825385\\
61.625	0.26868	38.861649696057\\
61.625	0.27234	40.3457907960075\\
61.625	0.276	41.8554758823899\\
62	0.093	-2.42264967886682\\
62	0.09666	-2.15739146048851\\
62	0.10032	-1.8665892556783\\
62	0.10398	-1.55024306443618\\
62	0.10764	-1.20835288676212\\
62	0.1113	-0.840918722656177\\
62	0.11496	-0.44794057211832\\
62	0.11862	-0.0294184351485356\\
62	0.12228	0.414647688253162\\
62	0.12594	0.88425779808675\\
62	0.1296	1.37941189435228\\
62	0.13326	1.90010997704971\\
62	0.13692	2.44635204617904\\
62	0.14058	3.0181381017403\\
62	0.14424	3.61546814373349\\
62	0.1479	4.23834217215856\\
62	0.15156	4.88676018701556\\
62	0.15522	5.56072218830446\\
62	0.15888	6.26022817602529\\
62	0.16254	6.98527815017803\\
62	0.1662	7.73587211076268\\
62	0.16986	8.51201005777923\\
62	0.17352	9.31369199122771\\
62	0.17718	10.1409179111081\\
62	0.18084	10.9936878174204\\
62	0.1845	11.8720017101646\\
62	0.18816	12.7758595893407\\
62	0.19182	13.7052614549488\\
62	0.19548	14.6602073069887\\
62	0.19914	15.6406971454606\\
62	0.2028	16.6467309703644\\
62	0.20646	17.6783087817001\\
62	0.21012	18.7354305794677\\
62	0.21378	19.8180963636672\\
62	0.21744	20.9263061342986\\
62	0.2211	22.060059891362\\
62	0.22476	23.2193576348572\\
62	0.22842	24.4041993647844\\
62	0.23208	25.6145850811434\\
62	0.23574	26.8505147839345\\
62	0.2394	28.1119884731574\\
62	0.24306	29.3990061488122\\
62	0.24672	30.7115678108989\\
62	0.25038	32.0496734594176\\
62	0.25404	33.4133230943681\\
62	0.2577	34.8025167157506\\
62	0.26136	36.217254323565\\
62	0.26502	37.6575359178113\\
62	0.26868	39.1233614984895\\
62	0.27234	40.6147310655996\\
62	0.276	42.1316446191416\\
62.375	0.093	-2.5055356224978\\
62.375	0.09666	-2.23304893695986\\
62.375	0.10032	-1.93501826498998\\
62.375	0.10398	-1.61144360658819\\
62.375	0.10764	-1.2623249617545\\
62.375	0.1113	-0.887662330488894\\
62.375	0.11496	-0.487455712791352\\
62.375	0.11862	-0.0617051086619256\\
62.375	0.12228	0.389589481899428\\
62.375	0.12594	0.866428058892694\\
62.375	0.1296	1.36881062231786\\
62.375	0.13326	1.89673717217495\\
62.375	0.13692	2.45020770846395\\
62.375	0.14058	3.02922223118488\\
62.375	0.14424	3.6337807403377\\
62.375	0.1479	4.26388323592244\\
62.375	0.15156	4.9195297179391\\
62.375	0.15522	5.60072018638765\\
62.375	0.15888	6.30745464126814\\
62.375	0.16254	7.03973308258053\\
62.375	0.1662	7.79755551032484\\
62.375	0.16986	8.58092192450105\\
62.375	0.17352	9.38983232510918\\
62.375	0.17718	10.2242867121492\\
62.375	0.18084	11.0842850856212\\
62.375	0.1845	11.9698274455251\\
62.375	0.18816	12.8809137918608\\
62.375	0.19182	13.8175441246285\\
62.375	0.19548	14.7797184438282\\
62.375	0.19914	15.7674367494597\\
62.375	0.2028	16.7806990415231\\
62.375	0.20646	17.8195053200185\\
62.375	0.21012	18.8838555849457\\
62.375	0.21378	19.9737498363049\\
62.375	0.21744	21.089188074096\\
62.375	0.2211	22.230170298319\\
62.375	0.22476	23.3966965089739\\
62.375	0.22842	24.5887667060607\\
62.375	0.23208	25.8063808895794\\
62.375	0.23574	27.0495390595301\\
62.375	0.2394	28.3182412159127\\
62.375	0.24306	29.6124873587272\\
62.375	0.24672	30.9322774879736\\
62.375	0.25038	32.2776116036519\\
62.375	0.25404	33.6484897057621\\
62.375	0.2577	35.0449117943042\\
62.375	0.26136	36.4668778692782\\
62.375	0.26502	37.9143879306842\\
62.375	0.26868	39.3874419785221\\
62.375	0.27234	40.8860400127918\\
62.375	0.276	42.4101820334935\\
62.75	0.093	-2.58605288852874\\
62.75	0.09666	-2.30633773583113\\
62.75	0.10032	-2.00107859670159\\
62.75	0.10398	-1.67027547114014\\
62.75	0.10764	-1.31392835914679\\
62.75	0.1113	-0.932037260721529\\
62.75	0.11496	-0.524602175864331\\
62.75	0.11862	-0.0916231045752482\\
62.75	0.12228	0.366899953145769\\
62.75	0.12594	0.850966997298691\\
62.75	0.1296	1.36057802788351\\
62.75	0.13326	1.89573304490026\\
62.75	0.13692	2.45643204834893\\
62.75	0.14058	3.0426750382295\\
62.75	0.14424	3.65446201454198\\
62.75	0.1479	4.29179297728638\\
62.75	0.15156	4.95466792646269\\
62.75	0.15522	5.64308686207091\\
62.75	0.15888	6.35704978411106\\
62.75	0.16254	7.09655669258311\\
62.75	0.1662	7.86160758748707\\
62.75	0.16986	8.65220246882294\\
62.75	0.17352	9.46834133659073\\
62.75	0.17718	10.3100241907904\\
62.75	0.18084	11.1772510314221\\
62.75	0.1845	12.0700218584856\\
62.75	0.18816	12.988336671981\\
62.75	0.19182	13.9321954719084\\
62.75	0.19548	14.9015982582676\\
62.75	0.19914	15.8965450310588\\
62.75	0.2028	16.9170357902819\\
62.75	0.20646	17.9630705359369\\
62.75	0.21012	19.0346492680238\\
62.75	0.21378	20.1317719865427\\
62.75	0.21744	21.2544386914934\\
62.75	0.2211	22.4026493828761\\
62.75	0.22476	23.5764040606907\\
62.75	0.22842	24.7757027249371\\
62.75	0.23208	26.0005453756155\\
62.75	0.23574	27.2509320127258\\
62.75	0.2394	28.5268626362681\\
62.75	0.24306	29.8283372462422\\
62.75	0.24672	31.1553558426482\\
62.75	0.25038	32.5079184254862\\
62.75	0.25404	33.8860249947561\\
62.75	0.2577	35.2896755504579\\
62.75	0.26136	36.7188700925916\\
62.75	0.26502	38.1736086211572\\
62.75	0.26868	39.6538911361547\\
62.75	0.27234	41.1597176375842\\
62.75	0.276	42.6910881254455\\
63.125	0.093	-2.66420147695956\\
63.125	0.09666	-2.37725785710228\\
63.125	0.10032	-2.0647702508131\\
63.125	0.10398	-1.72673865809201\\
63.125	0.10764	-1.36316307893898\\
63.125	0.1113	-0.974043513354054\\
63.125	0.11496	-0.559379961337228\\
63.125	0.11862	-0.119172422888468\\
63.125	0.12228	0.346579101992198\\
63.125	0.12594	0.83787461330477\\
63.125	0.1296	1.35471411104927\\
63.125	0.13326	1.89709759522566\\
63.125	0.13692	2.46502506583397\\
63.125	0.14058	3.05849652287421\\
63.125	0.14424	3.67751196634637\\
63.125	0.1479	4.32207139625041\\
63.125	0.15156	4.99217481258639\\
63.125	0.15522	5.68782221535426\\
63.125	0.15888	6.40901360455406\\
63.125	0.16254	7.15574898018578\\
63.125	0.1662	7.9280283422494\\
63.125	0.16986	8.72585169074492\\
63.125	0.17352	9.54921902567237\\
63.125	0.17718	10.3981303470317\\
63.125	0.18084	11.272585654823\\
63.125	0.1845	12.1725849490462\\
63.125	0.18816	13.0981282297013\\
63.125	0.19182	14.0492154967883\\
63.125	0.19548	15.0258467503072\\
63.125	0.19914	16.0280219902581\\
63.125	0.2028	17.0557412166408\\
63.125	0.20646	18.1090044294555\\
63.125	0.21012	19.1878116287021\\
63.125	0.21378	20.2921628143806\\
63.125	0.21744	21.422057986491\\
63.125	0.2211	22.5774971450333\\
63.125	0.22476	23.7584802900075\\
63.125	0.22842	24.9650074214137\\
63.125	0.23208	26.1970785392517\\
63.125	0.23574	27.4546936435217\\
63.125	0.2394	28.7378527342236\\
63.125	0.24306	30.0465558113573\\
63.125	0.24672	31.380802874923\\
63.125	0.25038	32.7405939249207\\
63.125	0.25404	34.1259289613502\\
63.125	0.2577	35.5368079842117\\
63.125	0.26136	36.973230993505\\
63.125	0.26502	38.4351979892303\\
63.125	0.26868	39.9227089713875\\
63.125	0.27234	41.4357639399766\\
63.125	0.276	42.9743628949976\\
63.5	0.093	-2.73998138779032\\
63.5	0.09666	-2.44580930077338\\
63.5	0.10032	-2.12609322732454\\
63.5	0.10398	-1.78083316744379\\
63.5	0.10764	-1.4100291211311\\
63.5	0.1113	-1.01368108838652\\
63.5	0.11496	-0.591789069210037\\
63.5	0.11862	-0.144353063601621\\
63.5	0.12228	0.328626928438709\\
63.5	0.12594	0.82715090691093\\
63.5	0.1296	1.3512188718151\\
63.5	0.13326	1.90083082315115\\
63.5	0.13692	2.47598676091911\\
63.5	0.14058	3.076686685119\\
63.5	0.14424	3.70293059575083\\
63.5	0.1479	4.35471849281452\\
63.5	0.15156	5.03205037631016\\
63.5	0.15522	5.73492624623769\\
63.5	0.15888	6.46334610259715\\
63.5	0.16254	7.21730994538851\\
63.5	0.1662	7.9968177746118\\
63.5	0.16986	8.80186959026698\\
63.5	0.17352	9.63246539235409\\
63.5	0.17718	10.4886051808731\\
63.5	0.18084	11.370288955824\\
63.5	0.1845	12.2775167172069\\
63.5	0.18816	13.2102884650216\\
63.5	0.19182	14.1686041992683\\
63.5	0.19548	15.1524639199469\\
63.5	0.19914	16.1618676270574\\
63.5	0.2028	17.1968153205998\\
63.5	0.20646	18.2573070005741\\
63.5	0.21012	19.3433426669804\\
63.5	0.21378	20.4549223198185\\
63.5	0.21744	21.5920459590885\\
63.5	0.2211	22.7547135847905\\
63.5	0.22476	23.9429251969244\\
63.5	0.22842	25.1566807954902\\
63.5	0.23208	26.3959803804879\\
63.5	0.23574	27.6608239519176\\
63.5	0.2394	28.9512115097791\\
63.5	0.24306	30.2671430540726\\
63.5	0.24672	31.6086185847979\\
63.5	0.25038	32.9756381019552\\
63.5	0.25404	34.3682016055444\\
63.5	0.2577	35.7863090955655\\
63.5	0.26136	37.2299605720185\\
63.5	0.26502	38.6991560349035\\
63.5	0.26868	40.1938954842202\\
63.5	0.27234	41.714178919969\\
63.5	0.276	43.2600063421497\\
63.875	0.093	-2.81339262102098\\
63.875	0.09666	-2.5119920668444\\
63.875	0.10032	-2.18504752623589\\
63.875	0.10398	-1.83255899919547\\
63.875	0.10764	-1.45452648572315\\
63.875	0.1113	-1.05094998581891\\
63.875	0.11496	-0.621829499482743\\
63.875	0.11862	-0.167165026714684\\
63.875	0.12228	0.313043432485301\\
63.875	0.12594	0.8187958781172\\
63.875	0.1296	1.35009231018099\\
63.875	0.13326	1.90693272867672\\
63.875	0.13692	2.48931713360436\\
63.875	0.14058	3.09724552496391\\
63.875	0.14424	3.73071790275536\\
63.875	0.1479	4.38973426697872\\
63.875	0.15156	5.07429461763401\\
63.875	0.15522	5.7843989547212\\
63.875	0.15888	6.52004727824032\\
63.875	0.16254	7.28123958819135\\
63.875	0.1662	8.06797588457429\\
63.875	0.16986	8.88025616738913\\
63.875	0.17352	9.71808043663589\\
63.875	0.17718	10.5814486923146\\
63.875	0.18084	11.4703609344252\\
63.875	0.1845	12.3848171629677\\
63.875	0.18816	13.3248173779421\\
63.875	0.19182	14.2903615793484\\
63.875	0.19548	15.2814497671867\\
63.875	0.19914	16.2980819414568\\
63.875	0.2028	17.3402581021589\\
63.875	0.20646	18.4079782492929\\
63.875	0.21012	19.5012423828587\\
63.875	0.21378	20.6200505028565\\
63.875	0.21744	21.7644026092863\\
63.875	0.2211	22.9342987021479\\
63.875	0.22476	24.1297387814414\\
63.875	0.22842	25.3507228471669\\
63.875	0.23208	26.5972508993243\\
63.875	0.23574	27.8693229379136\\
63.875	0.2394	29.1669389629348\\
63.875	0.24306	30.4900989743878\\
63.875	0.24672	31.8388029722729\\
63.875	0.25038	33.2130509565898\\
63.875	0.25404	34.6128429273387\\
63.875	0.2577	36.0381788845194\\
63.875	0.26136	37.4890588281321\\
63.875	0.26502	38.9654827581767\\
63.875	0.26868	40.4674506746532\\
63.875	0.27234	41.9949625775616\\
63.875	0.276	43.5480184669019\\
64.25	0.093	-2.88443517665157\\
64.25	0.09666	-2.57580615531534\\
64.25	0.10032	-2.24163314754716\\
64.25	0.10398	-1.88191615334709\\
64.25	0.10764	-1.49665517271511\\
64.25	0.1113	-1.08585020565122\\
64.25	0.11496	-0.649501252155389\\
64.25	0.11862	-0.187608312227667\\
64.25	0.12228	0.299828614131975\\
64.25	0.12594	0.81280952692353\\
64.25	0.1296	1.35133442614698\\
64.25	0.13326	1.91540331180236\\
64.25	0.13692	2.50501618388965\\
64.25	0.14058	3.12017304240886\\
64.25	0.14424	3.76087388735997\\
64.25	0.1479	4.427118718743\\
64.25	0.15156	5.11890753655794\\
64.25	0.15522	5.83624034080479\\
64.25	0.15888	6.57911713148357\\
64.25	0.16254	7.34753790859425\\
64.25	0.1662	8.14150267213685\\
64.25	0.16986	8.96101142211135\\
64.25	0.17352	9.80606415851777\\
64.25	0.17718	10.6766608813561\\
64.25	0.18084	11.5728015906264\\
64.25	0.1845	12.4944862863285\\
64.25	0.18816	13.4417149684626\\
64.25	0.19182	14.4144876370286\\
64.25	0.19548	15.4128042920265\\
64.25	0.19914	16.4366649334563\\
64.25	0.2028	17.486069561318\\
64.25	0.20646	18.5610181756117\\
64.25	0.21012	19.6615107763372\\
64.25	0.21378	20.7875473634947\\
64.25	0.21744	21.939127937084\\
64.25	0.2211	23.1162524971053\\
64.25	0.22476	24.3189210435585\\
64.25	0.22842	25.5471335764436\\
64.25	0.23208	26.8008900957606\\
64.25	0.23574	28.0801906015096\\
64.25	0.2394	29.3850350936905\\
64.25	0.24306	30.7154235723032\\
64.25	0.24672	32.0713560373479\\
64.25	0.25038	33.4528324888245\\
64.25	0.25404	34.859852926733\\
64.25	0.2577	36.2924173510734\\
64.25	0.26136	37.7505257618458\\
64.25	0.26502	39.23417815905\\
64.25	0.26868	40.7433745426862\\
64.25	0.27234	42.2781149127542\\
64.25	0.276	43.8383992692542\\
64.625	0.093	-2.95310905468209\\
64.625	0.09666	-2.6372515661862\\
64.625	0.10032	-2.29585009125837\\
64.625	0.10398	-1.92890462989863\\
64.625	0.10764	-1.53641518210699\\
64.625	0.1113	-1.11838174788345\\
64.625	0.11496	-0.674804327227953\\
64.625	0.11862	-0.205682920140582\\
64.625	0.12228	0.288982473378717\\
64.625	0.12594	0.809191853329928\\
64.625	0.1296	1.35494521971304\\
64.625	0.13326	1.92624257252808\\
64.625	0.13692	2.52308391177503\\
64.625	0.14058	3.1454692374539\\
64.625	0.14424	3.79339854956466\\
64.625	0.1479	4.46687184810735\\
64.625	0.15156	5.16588913308195\\
64.625	0.15522	5.89045040448846\\
64.625	0.15888	6.64055566232689\\
64.625	0.16254	7.41620490659723\\
64.625	0.1662	8.21739813729949\\
64.625	0.16986	9.04413535443364\\
64.625	0.17352	9.89641655799972\\
64.625	0.17718	10.7742417479977\\
64.625	0.18084	11.6776109244276\\
64.625	0.1845	12.6065240872894\\
64.625	0.18816	13.5609812365832\\
64.625	0.19182	14.5409823723088\\
64.625	0.19548	15.5465274944664\\
64.625	0.19914	16.5776166030558\\
64.625	0.2028	17.6342496980772\\
64.625	0.20646	18.7164267795305\\
64.625	0.21012	19.8241478474157\\
64.625	0.21378	20.9574129017328\\
64.625	0.21744	22.1162219424819\\
64.625	0.2211	23.3005749696628\\
64.625	0.22476	24.5104719832757\\
64.625	0.22842	25.7459129833205\\
64.625	0.23208	27.0068979697972\\
64.625	0.23574	28.2934269427057\\
64.625	0.2394	29.6054999020463\\
64.625	0.24306	30.9431168478187\\
64.625	0.24672	32.306277780023\\
64.625	0.25038	33.6949826986593\\
64.625	0.25404	35.1092316037274\\
64.625	0.2577	36.5490244952275\\
64.625	0.26136	38.0143613731595\\
64.625	0.26502	39.5052422375234\\
64.625	0.26868	41.0216670883192\\
64.625	0.27234	42.5636359255469\\
64.625	0.276	44.1311487492066\\
65	0.093	-3.0194142551125\\
65	0.09666	-2.69632829945694\\
65	0.10032	-2.34769835736947\\
65	0.10398	-1.97352442885009\\
65	0.10764	-1.57380651389877\\
65	0.1113	-1.14854461251556\\
65	0.11496	-0.697738724700436\\
65	0.11862	-0.221388850453394\\
65	0.12228	0.280505010225568\\
65	0.12594	0.807942857336421\\
65	0.1296	1.36092469087922\\
65	0.13326	1.9394505108539\\
65	0.13692	2.54352031726049\\
65	0.14058	3.17313411009902\\
65	0.14424	3.82829188936947\\
65	0.1479	4.5089936550718\\
65	0.15156	5.21523940720606\\
65	0.15522	5.94702914577223\\
65	0.15888	6.70436287077032\\
65	0.16254	7.48724058220031\\
65	0.1662	8.29566228006223\\
65	0.16986	9.12962796435604\\
65	0.17352	9.98913763508178\\
65	0.17718	10.8741912922394\\
65	0.18084	11.784788935829\\
65	0.1845	12.7209305658505\\
65	0.18816	13.6826161823039\\
65	0.19182	14.6698457851891\\
65	0.19548	15.6826193745064\\
65	0.19914	16.7209369502555\\
65	0.2028	17.7847985124365\\
65	0.20646	18.8742040610495\\
65	0.21012	19.9891535960944\\
65	0.21378	21.1296471175711\\
65	0.21744	22.2956846254798\\
65	0.2211	23.4872661198204\\
65	0.22476	24.7043916005929\\
65	0.22842	25.9470610677974\\
65	0.23208	27.2152745214337\\
65	0.23574	28.509031961502\\
65	0.2394	29.8283333880022\\
65	0.24306	31.1731788009342\\
65	0.24672	32.5435682002982\\
65	0.25038	33.9395015860941\\
65	0.25404	35.360978958322\\
65	0.2577	36.8080003169817\\
65	0.26136	38.2805656620733\\
65	0.26502	39.7786749935969\\
65	0.26868	41.3023283115524\\
65	0.27234	42.8515256159397\\
65	0.276	44.426266906759\\
65.375	0.093	-3.08335077794285\\
65.375	0.09666	-2.75303635512763\\
65.375	0.10032	-2.39717794588049\\
65.375	0.10398	-2.01577555020146\\
65.375	0.10764	-1.60882916809048\\
65.375	0.1113	-1.17633879954762\\
65.375	0.11496	-0.718304444572837\\
65.375	0.11862	-0.234726103166132\\
65.375	0.12228	0.274396224672479\\
65.375	0.12594	0.809062538942996\\
65.375	0.1296	1.36927283964544\\
65.375	0.13326	1.95502712677979\\
65.375	0.13692	2.56632540034604\\
65.375	0.14058	3.20316766034422\\
65.375	0.14424	3.86555390677433\\
65.375	0.1479	4.55348413963632\\
65.375	0.15156	5.26695835893024\\
65.375	0.15522	6.00597656465606\\
65.375	0.15888	6.77053875681381\\
65.375	0.16254	7.56064493540347\\
65.375	0.1662	8.37629510042504\\
65.375	0.16986	9.2174892518785\\
65.375	0.17352	10.0842273897639\\
65.375	0.17718	10.9765095140812\\
65.375	0.18084	11.8943356248304\\
65.375	0.1845	12.8377057220116\\
65.375	0.18816	13.8066198056246\\
65.375	0.19182	14.8010778756696\\
65.375	0.19548	15.8210799321464\\
65.375	0.19914	16.8666259750552\\
65.375	0.2028	17.9377160043959\\
65.375	0.20646	19.0343500201685\\
65.375	0.21012	20.1565280223731\\
65.375	0.21378	21.3042500110095\\
65.375	0.21744	22.4775159860778\\
65.375	0.2211	23.6763259475781\\
65.375	0.22476	24.9006798955103\\
65.375	0.22842	26.1505778298744\\
65.375	0.23208	27.4260197506704\\
65.375	0.23574	28.7270056578983\\
65.375	0.2394	30.0535355515581\\
65.375	0.24306	31.4056094316499\\
65.375	0.24672	32.7832272981735\\
65.375	0.25038	34.1863891511291\\
65.375	0.25404	35.6150949905166\\
65.375	0.2577	37.0693448163359\\
65.375	0.26136	38.5491386285872\\
65.375	0.26502	40.0544764272705\\
65.375	0.26868	41.5853582123856\\
65.375	0.27234	43.1417839839326\\
65.375	0.276	44.7237537419116\\
65.75	0.093	-3.1449186231731\\
65.75	0.09666	-2.80737573319824\\
65.75	0.10032	-2.44428885679143\\
65.75	0.10398	-2.05565799395272\\
65.75	0.10764	-1.64148314468211\\
65.75	0.1113	-1.20176430897959\\
65.75	0.11496	-0.736501486845128\\
65.75	0.11862	-0.245694678278781\\
65.75	0.12228	0.270656116719493\\
65.75	0.12594	0.812550898149674\\
65.75	0.1296	1.37998966601176\\
65.75	0.13326	1.97297242030577\\
65.75	0.13692	2.5914991610317\\
65.75	0.14058	3.23556988818953\\
65.75	0.14424	3.90518460177928\\
65.75	0.1479	4.60034330180094\\
65.75	0.15156	5.32104598825451\\
65.75	0.15522	6.06729266113999\\
65.75	0.15888	6.8390833204574\\
65.75	0.16254	7.63641796620671\\
65.75	0.1662	8.45929659838794\\
65.75	0.16986	9.30771921700106\\
65.75	0.17352	10.1816858220461\\
65.75	0.17718	11.0811964135231\\
65.75	0.18084	12.006250991432\\
65.75	0.1845	12.9568495557728\\
65.75	0.18816	13.9329921065455\\
65.75	0.19182	14.9346786437501\\
65.75	0.19548	15.9619091673866\\
65.75	0.19914	17.014683677455\\
65.75	0.2028	18.0930021739554\\
65.75	0.20646	19.1968646568877\\
65.75	0.21012	20.3262711262519\\
65.75	0.21378	21.4812215820479\\
65.75	0.21744	22.661716024276\\
65.75	0.2211	23.8677544529359\\
65.75	0.22476	25.0993368680277\\
65.75	0.22842	26.3564632695515\\
65.75	0.23208	27.6391336575071\\
65.75	0.23574	28.9473480318947\\
65.75	0.2394	30.2811063927142\\
65.75	0.24306	31.6404087399656\\
65.75	0.24672	33.0252550736489\\
65.75	0.25038	34.4356453937641\\
65.75	0.25404	35.8715797003113\\
65.75	0.2577	37.3330579932903\\
65.75	0.26136	38.8200802727013\\
65.75	0.26502	40.3326465385441\\
65.75	0.26868	41.8707567908189\\
65.75	0.27234	43.4344110295256\\
65.75	0.276	45.0236092546642\\
66.125	0.093	-3.20411779080329\\
66.125	0.09666	-2.85934643366876\\
66.125	0.10032	-2.4890310901023\\
66.125	0.10398	-2.09317176010394\\
66.125	0.10764	-1.67176844367367\\
66.125	0.1113	-1.22482114081149\\
66.125	0.11496	-0.752329851517366\\
66.125	0.11862	-0.254294575791363\\
66.125	0.12228	0.269284686366568\\
66.125	0.12594	0.818407934956412\\
66.125	0.1296	1.39307516997815\\
66.125	0.13326	1.99328639143182\\
66.125	0.13692	2.6190415993174\\
66.125	0.14058	3.2703407936349\\
66.125	0.14424	3.9471839743843\\
66.125	0.1479	4.64957114156562\\
66.125	0.15156	5.37750229517885\\
66.125	0.15522	6.13097743522399\\
66.125	0.15888	6.90999656170105\\
66.125	0.16254	7.71455967461002\\
66.125	0.1662	8.54466677395091\\
66.125	0.16986	9.4003178597237\\
66.125	0.17352	10.2815129319284\\
66.125	0.17718	11.188251990565\\
66.125	0.18084	12.1205350356336\\
66.125	0.1845	13.078362067134\\
66.125	0.18816	14.0617330850664\\
66.125	0.19182	15.0706480894307\\
66.125	0.19548	16.1051070802268\\
66.125	0.19914	17.1651100574549\\
66.125	0.2028	18.250657021115\\
66.125	0.20646	19.3617479712069\\
66.125	0.21012	20.4983829077307\\
66.125	0.21378	21.6605618306865\\
66.125	0.21744	22.8482847400741\\
66.125	0.2211	24.0615516358937\\
66.125	0.22476	25.3003625181452\\
66.125	0.22842	26.5647173868286\\
66.125	0.23208	27.8546162419439\\
66.125	0.23574	29.1700590834912\\
66.125	0.2394	30.5110459114703\\
66.125	0.24306	31.8775767258813\\
66.125	0.24672	33.2696515267243\\
66.125	0.25038	34.6872703139992\\
66.125	0.25404	36.130433087706\\
66.125	0.2577	37.5991398478447\\
66.125	0.26136	39.0933905944153\\
66.125	0.26502	40.6131853274179\\
66.125	0.26868	42.1585240468523\\
66.125	0.27234	43.7294067527186\\
66.125	0.276	45.3258334450169\\
66.5	0.093	-3.26094828083337\\
66.5	0.09666	-2.90894845653917\\
66.5	0.10032	-2.53140464581307\\
66.5	0.10398	-2.12831684865506\\
66.5	0.10764	-1.69968506506511\\
66.5	0.1113	-1.24550929504326\\
66.5	0.11496	-0.765789538589516\\
66.5	0.11862	-0.260525795703842\\
66.5	0.12228	0.270281933613752\\
66.5	0.12594	0.826633649363238\\
66.5	0.1296	1.40852935154466\\
66.5	0.13326	2.01596904015797\\
66.5	0.13692	2.6489527152032\\
66.5	0.14058	3.30748037668036\\
66.5	0.14424	3.99155202458945\\
66.5	0.1479	4.70116765893041\\
66.5	0.15156	5.4363272797033\\
66.5	0.15522	6.19703088690809\\
66.5	0.15888	6.98327848054481\\
66.5	0.16254	7.79507006061344\\
66.5	0.1662	8.63240562711398\\
66.5	0.16986	9.49528518004643\\
66.5	0.17352	10.3837087194108\\
66.5	0.17718	11.2976762452071\\
66.5	0.18084	12.2371877574353\\
66.5	0.1845	13.2022432560954\\
66.5	0.18816	14.1928427411874\\
66.5	0.19182	15.2089862127113\\
66.5	0.19548	16.2506736706672\\
66.5	0.19914	17.3179051150549\\
66.5	0.2028	18.4106805458746\\
66.5	0.20646	19.5289999631262\\
66.5	0.21012	20.6728633668097\\
66.5	0.21378	21.8422707569251\\
66.5	0.21744	23.0372221334724\\
66.5	0.2211	24.2577174964517\\
66.5	0.22476	25.5037568458628\\
66.5	0.22842	26.7753401817058\\
66.5	0.23208	28.0724675039808\\
66.5	0.23574	29.3951388126877\\
66.5	0.2394	30.7433541078265\\
66.5	0.24306	32.1171133893972\\
66.5	0.24672	33.5164166573999\\
66.5	0.25038	34.9412639118344\\
66.5	0.25404	36.3916551527009\\
66.5	0.2577	37.8675903799992\\
66.5	0.26136	39.3690695937295\\
66.5	0.26502	40.8960927938917\\
66.5	0.26868	42.4486599804858\\
66.5	0.27234	44.0267711535118\\
66.5	0.276	45.6304263129698\\
66.875	0.093	-3.31541009326339\\
66.875	0.09666	-2.95618180180953\\
66.875	0.10032	-2.57140952392377\\
66.875	0.10398	-2.16109325960611\\
66.875	0.10764	-1.72523300885649\\
66.875	0.1113	-1.26382877167499\\
66.875	0.11496	-0.776880548061591\\
66.875	0.11862	-0.264388338016253\\
66.875	0.12228	0.27364785846099\\
66.875	0.12594	0.837228041370139\\
66.875	0.1296	1.42635221071122\\
66.875	0.13326	2.04102036648419\\
66.875	0.13692	2.68123250868907\\
66.875	0.14058	3.34698863732589\\
66.875	0.14424	4.03828875239463\\
66.875	0.1479	4.75513285389525\\
66.875	0.15156	5.4975209418278\\
66.875	0.15522	6.26545301619225\\
66.875	0.15888	7.05892907698863\\
66.875	0.16254	7.87794912421692\\
66.875	0.1662	8.72251315787712\\
66.875	0.16986	9.59262117796922\\
66.875	0.17352	10.4882731844932\\
66.875	0.17718	11.4094691774492\\
66.875	0.18084	12.356209156837\\
66.875	0.1845	13.3284931226568\\
66.875	0.18816	14.3263210749085\\
66.875	0.19182	15.3496930135921\\
66.875	0.19548	16.3986089387076\\
66.875	0.19914	17.473068850255\\
66.875	0.2028	18.5730727482343\\
66.875	0.20646	19.6986206326456\\
66.875	0.21012	20.8497125034887\\
66.875	0.21378	22.0263483607638\\
66.875	0.21744	23.2285282044708\\
66.875	0.2211	24.4562520346096\\
66.875	0.22476	25.7095198511804\\
66.875	0.22842	26.9883316541832\\
66.875	0.23208	28.2926874436178\\
66.875	0.23574	29.6225872194844\\
66.875	0.2394	30.9780309817828\\
66.875	0.24306	32.3590187305132\\
66.875	0.24672	33.7655504656755\\
66.875	0.25038	35.1976261872697\\
66.875	0.25404	36.6552458952958\\
66.875	0.2577	38.1384095897538\\
66.875	0.26136	39.6471172706437\\
66.875	0.26502	41.1813689379656\\
66.875	0.26868	42.7411645917193\\
66.875	0.27234	44.326504231905\\
66.875	0.276	45.9373878585226\\
67.25	0.093	-3.36750322809331\\
67.25	0.09666	-3.0010464694798\\
67.25	0.10032	-2.60904572443437\\
67.25	0.10398	-2.19150099295703\\
67.25	0.10764	-1.7484122750478\\
67.25	0.1113	-1.27977957070664\\
67.25	0.11496	-0.785602879933549\\
67.25	0.11862	-0.265882202728569\\
67.25	0.12228	0.279382460908337\\
67.25	0.12594	0.85019111097715\\
67.25	0.1296	1.44654374747786\\
67.25	0.13326	2.06844037041051\\
67.25	0.13692	2.71588097977506\\
67.25	0.14058	3.38886557557153\\
67.25	0.14424	4.08739415779991\\
67.25	0.1479	4.8114667264602\\
67.25	0.15156	5.5610832815524\\
67.25	0.15522	6.33624382307652\\
67.25	0.15888	7.13694835103255\\
67.25	0.16254	7.9631968654205\\
67.25	0.1662	8.81498936624035\\
67.25	0.16986	9.69232585349211\\
67.25	0.17352	10.5952063271758\\
67.25	0.17718	11.5236307872914\\
67.25	0.18084	12.4775992338389\\
67.25	0.1845	13.4571116668183\\
67.25	0.18816	14.4621680862297\\
67.25	0.19182	15.4927684920729\\
67.25	0.19548	16.5489128843481\\
67.25	0.19914	17.6306012630551\\
67.25	0.2028	18.7378336281941\\
67.25	0.20646	19.870609979765\\
67.25	0.21012	21.0289303177678\\
67.25	0.21378	22.2127946422026\\
67.25	0.21744	23.4222029530692\\
67.25	0.2211	24.6571552503677\\
67.25	0.22476	25.9176515340982\\
67.25	0.22842	27.2036918042606\\
67.25	0.23208	28.5152760608549\\
67.25	0.23574	29.8524043038811\\
67.25	0.2394	31.2150765333392\\
67.25	0.24306	32.6032927492292\\
67.25	0.24672	34.0170529515512\\
67.25	0.25038	35.456357140305\\
67.25	0.25404	36.9212053154908\\
67.25	0.2577	38.4115974771085\\
67.25	0.26136	39.9275336251581\\
67.25	0.26502	41.4690137596396\\
67.25	0.26868	43.036037880553\\
67.25	0.27234	44.6286059878983\\
67.25	0.276	46.2467180816756\\
67.625	0.093	-3.41722768532316\\
67.625	0.09666	-3.04354245955\\
67.625	0.10032	-2.64431324734491\\
67.625	0.10398	-2.21954004870791\\
67.625	0.10764	-1.76922286363902\\
67.625	0.1113	-1.2933616921382\\
67.625	0.11496	-0.791956534205454\\
67.625	0.11862	-0.265007389840818\\
67.625	0.12228	0.287485740955745\\
67.625	0.12594	0.865522858184221\\
67.625	0.1296	1.46910396184459\\
67.625	0.13326	2.09822905193689\\
67.625	0.13692	2.75289812846111\\
67.625	0.14058	3.43311119141724\\
67.625	0.14424	4.13886824080527\\
67.625	0.1479	4.87016927662522\\
67.625	0.15156	5.62701429887708\\
67.625	0.15522	6.40940330756084\\
67.625	0.15888	7.21733630267654\\
67.625	0.16254	8.05081328422414\\
67.625	0.1662	8.90983425220366\\
67.625	0.16986	9.79439920661508\\
67.625	0.17352	10.7045081474584\\
67.625	0.17718	11.6401610747337\\
67.625	0.18084	12.6013579884408\\
67.625	0.1845	13.5880988885799\\
67.625	0.18816	14.6003837751509\\
67.625	0.19182	15.6382126481538\\
67.625	0.19548	16.7015855075886\\
67.625	0.19914	17.7905023534554\\
67.625	0.2028	18.904963185754\\
67.625	0.20646	20.0449680044846\\
67.625	0.21012	21.2105168096471\\
67.625	0.21378	22.4016096012414\\
67.625	0.21744	23.6182463792677\\
67.625	0.2211	24.8604271437259\\
67.625	0.22476	26.128151894616\\
67.625	0.22842	27.4214206319381\\
67.625	0.23208	28.740233355692\\
67.625	0.23574	30.0845900658779\\
67.625	0.2394	31.4544907624957\\
67.625	0.24306	32.8499354455454\\
67.625	0.24672	34.2709241150269\\
67.625	0.25038	35.7174567709405\\
67.625	0.25404	37.1895334132859\\
67.625	0.2577	38.6871540420632\\
67.625	0.26136	40.2103186572725\\
67.625	0.26502	41.7590272589136\\
67.625	0.26868	43.3332798469867\\
67.625	0.27234	44.9330764214917\\
67.625	0.276	46.5584169824286\\
68	0.093	-3.46458346495291\\
68	0.09666	-3.08366977202008\\
68	0.10032	-2.67721209265535\\
68	0.10398	-2.24521042685871\\
68	0.10764	-1.78766477463012\\
68	0.1113	-1.30457513596965\\
68	0.11496	-0.79594151087727\\
68	0.11862	-0.261763899352964\\
68	0.12228	0.297957698603255\\
68	0.12594	0.883223282991374\\
68	0.1296	1.49403285381143\\
68	0.13326	2.13038641106337\\
68	0.13692	2.79228395474723\\
68	0.14058	3.47972548486301\\
68	0.14424	4.19271100141074\\
68	0.1479	4.93124050439033\\
68	0.15156	5.69531399380185\\
68	0.15522	6.48493146964528\\
68	0.15888	7.30009293192062\\
68	0.16254	8.14079838062788\\
68	0.1662	9.00704781576706\\
68	0.16986	9.89884123733813\\
68	0.17352	10.8161786453411\\
68	0.17718	11.759060039776\\
68	0.18084	12.7274854206429\\
68	0.1845	13.7214547879416\\
68	0.18816	14.7409681416723\\
68	0.19182	15.7860254818348\\
68	0.19548	16.8566268084293\\
68	0.19914	17.9527721214557\\
68	0.2028	19.074461420914\\
68	0.20646	20.2216947068042\\
68	0.21012	21.3944719791263\\
68	0.21378	22.5927932378804\\
68	0.21744	23.8166584830663\\
68	0.2211	25.0660677146842\\
68	0.22476	26.341020932734\\
68	0.22842	27.6415181372157\\
68	0.23208	28.9675593281293\\
68	0.23574	30.3191445054748\\
68	0.2394	31.6962736692522\\
68	0.24306	33.0989468194616\\
68	0.24672	34.5271639561028\\
68	0.25038	35.980925079176\\
68	0.25404	37.4602301886811\\
68	0.2577	38.9650792846181\\
68	0.26136	40.495472366987\\
68	0.26502	42.0514094357878\\
68	0.26868	43.6328904910205\\
68	0.27234	45.2399155326852\\
68	0.276	46.8724845607817\\
68.375	0.093	-3.5095705669826\\
68.375	0.09666	-3.12142840689012\\
68.375	0.10032	-2.70774226036573\\
68.375	0.10398	-2.26851212740943\\
68.375	0.10764	-1.80373800802119\\
68.375	0.1113	-1.31341990220105\\
68.375	0.11496	-0.797557809949019\\
68.375	0.11862	-0.256151731265057\\
68.375	0.12228	0.310798333850819\\
68.375	0.12594	0.903292385398593\\
68.375	0.1296	1.52133042337831\\
68.375	0.13326	2.16491244778991\\
68.375	0.13692	2.83403845863343\\
68.375	0.14058	3.52870845590887\\
68.375	0.14424	4.24892243961624\\
68.375	0.1479	4.99468040975549\\
68.375	0.15156	5.76598236632668\\
68.375	0.15522	6.56282830932976\\
68.375	0.15888	7.38521823876476\\
68.375	0.16254	8.23315215463169\\
68.375	0.1662	9.10663005693052\\
68.375	0.16986	10.0056519456612\\
68.375	0.17352	10.9302178208239\\
68.375	0.17718	11.8803276824185\\
68.375	0.18084	12.855981530445\\
68.375	0.1845	13.8571793649034\\
68.375	0.18816	14.8839211857937\\
68.375	0.19182	15.9362069931159\\
68.375	0.19548	17.01403678687\\
68.375	0.19914	18.1174105670561\\
68.375	0.2028	19.246328333674\\
68.375	0.20646	20.4007900867239\\
68.375	0.21012	21.5807958262057\\
68.375	0.21378	22.7863455521194\\
68.375	0.21744	24.017439264465\\
68.375	0.2211	25.2740769632425\\
68.375	0.22476	26.556258648452\\
68.375	0.22842	27.8639843200933\\
68.375	0.23208	29.1972539781666\\
68.375	0.23574	30.5560676226718\\
68.375	0.2394	31.9404252536088\\
68.375	0.24306	33.3503268709778\\
68.375	0.24672	34.7857724747787\\
68.375	0.25038	36.2467620650116\\
68.375	0.25404	37.7332956416763\\
68.375	0.2577	39.245373204773\\
68.375	0.26136	40.7829947543015\\
68.375	0.26502	42.346160290262\\
68.375	0.26868	43.9348698126544\\
68.375	0.27234	45.5491233214787\\
68.375	0.276	47.188920816735\\
68.75	0.093	-3.55218899141221\\
68.75	0.09666	-3.15681836416007\\
68.75	0.10032	-2.73590375047602\\
68.75	0.10398	-2.28944515036006\\
68.75	0.10764	-1.81744256381217\\
68.75	0.1113	-1.31989599083238\\
68.75	0.11496	-0.79680543142068\\
68.75	0.11862	-0.248170885577061\\
68.75	0.12228	0.326007646698478\\
68.75	0.12594	0.925730165405909\\
68.75	0.1296	1.55099667054528\\
68.75	0.13326	2.20180716211654\\
68.75	0.13692	2.87816164011972\\
68.75	0.14058	3.58006010455481\\
68.75	0.14424	4.30750255542185\\
68.75	0.1479	5.06048899272075\\
68.75	0.15156	5.8390194164516\\
68.75	0.15522	6.64309382661433\\
68.75	0.15888	7.472712223209\\
68.75	0.16254	8.32787460623558\\
68.75	0.1662	9.20858097569407\\
68.75	0.16986	10.1148313315845\\
68.75	0.17352	11.0466256739068\\
68.75	0.17718	12.003964002661\\
68.75	0.18084	12.9868463178471\\
68.75	0.1845	13.9952726194652\\
68.75	0.18816	15.0292429075152\\
68.75	0.19182	16.088757181997\\
68.75	0.19548	17.1738154429108\\
68.75	0.19914	18.2844176902565\\
68.75	0.2028	19.4205639240342\\
68.75	0.20646	20.5822541442437\\
68.75	0.21012	21.7694883508851\\
68.75	0.21378	22.9822665439585\\
68.75	0.21744	24.2205887234638\\
68.75	0.2211	25.4844548894009\\
68.75	0.22476	26.77386504177\\
68.75	0.22842	28.0888191805711\\
68.75	0.23208	29.4293173058039\\
68.75	0.23574	30.7953594174688\\
68.75	0.2394	32.1869455155656\\
68.75	0.24306	33.6040756000942\\
68.75	0.24672	35.0467496710548\\
68.75	0.25038	36.5149677284473\\
68.75	0.25404	38.0087297722716\\
68.75	0.2577	39.528035802528\\
68.75	0.26136	41.0728858192162\\
68.75	0.26502	42.6432798223363\\
68.75	0.26868	44.2392178118884\\
68.75	0.27234	45.8606997878723\\
68.75	0.276	47.5077257502882\\
69.125	0.093	-3.59243873824172\\
69.125	0.09666	-3.18983964382994\\
69.125	0.10032	-2.76169656298621\\
69.125	0.10398	-2.30800949571059\\
69.125	0.10764	-1.82877844200306\\
69.125	0.1113	-1.32400340186361\\
69.125	0.11496	-0.79368437529223\\
69.125	0.11862	-0.237821362288962\\
69.125	0.12228	0.343585637146226\\
69.125	0.12594	0.950536623013335\\
69.125	0.1296	1.58303159531234\\
69.125	0.13326	2.24107055404327\\
69.125	0.13692	2.92465349920611\\
69.125	0.14058	3.63378043080087\\
69.125	0.14424	4.36845134882753\\
69.125	0.1479	5.12866625328611\\
69.125	0.15156	5.91442514417661\\
69.125	0.15522	6.72572802149901\\
69.125	0.15888	7.56257488525333\\
69.125	0.16254	8.42496573543956\\
69.125	0.1662	9.31290057205771\\
69.125	0.16986	10.2263793951078\\
69.125	0.17352	11.1654022045897\\
69.125	0.17718	12.1299690005036\\
69.125	0.18084	13.1200797828494\\
69.125	0.1845	14.1357345516271\\
69.125	0.18816	15.1769333068367\\
69.125	0.19182	16.2436760484783\\
69.125	0.19548	17.3359627765518\\
69.125	0.19914	18.4537934910571\\
69.125	0.2028	19.5971681919944\\
69.125	0.20646	20.7660868793636\\
69.125	0.21012	21.9605495531647\\
69.125	0.21378	23.1805562133977\\
69.125	0.21744	24.4261068600626\\
69.125	0.2211	25.6972014931594\\
69.125	0.22476	26.9938401126882\\
69.125	0.22842	28.3160227186489\\
69.125	0.23208	29.6637493110414\\
69.125	0.23574	31.037019889866\\
69.125	0.2394	32.4358344551223\\
69.125	0.24306	33.8601930068106\\
69.125	0.24672	35.3100955449309\\
69.125	0.25038	36.785542069483\\
69.125	0.25404	38.2865325804671\\
69.125	0.2577	39.8130670778831\\
69.125	0.26136	41.3651455617309\\
69.125	0.26502	42.9427680320107\\
69.125	0.26868	44.5459344887224\\
69.125	0.27234	46.174644931866\\
69.125	0.276	47.8288993614416\\
69.5	0.093	-3.63031980747115\\
69.5	0.09666	-3.22049224589971\\
69.5	0.10032	-2.78512069789633\\
69.5	0.10398	-2.32420516346105\\
69.5	0.10764	-1.83774564259386\\
69.5	0.1113	-1.32574213529476\\
69.5	0.11496	-0.78819464156372\\
69.5	0.11862	-0.225103161400796\\
69.5	0.12228	0.363532305194056\\
69.5	0.12594	0.977711758220821\\
69.5	0.1296	1.61743519767947\\
69.5	0.13326	2.28270262357006\\
69.5	0.13692	2.97351403589257\\
69.5	0.14058	3.68986943464698\\
69.5	0.14424	4.4317688198333\\
69.5	0.1479	5.19921219145154\\
69.5	0.15156	5.9921995495017\\
69.5	0.15522	6.81073089398375\\
69.5	0.15888	7.65480622489773\\
69.5	0.16254	8.52442554224363\\
69.5	0.1662	9.41958884602143\\
69.5	0.16986	10.3402961362311\\
69.5	0.17352	11.2865474128728\\
69.5	0.17718	12.2583426759463\\
69.5	0.18084	13.2556819254518\\
69.5	0.1845	14.2785651613891\\
69.5	0.18816	15.3269923837584\\
69.5	0.19182	16.4009635925596\\
69.5	0.19548	17.5004787877927\\
69.5	0.19914	18.6255379694577\\
69.5	0.2028	19.7761411375547\\
69.5	0.20646	20.9522882920835\\
69.5	0.21012	22.1539794330443\\
69.5	0.21378	23.3812145604369\\
69.5	0.21744	24.6339936742615\\
69.5	0.2211	25.912316774518\\
69.5	0.22476	27.2161838612064\\
69.5	0.22842	28.5455949343268\\
69.5	0.23208	29.900549993879\\
69.5	0.23574	31.2810490398631\\
69.5	0.2394	32.6870920722792\\
69.5	0.24306	34.1186790911272\\
69.5	0.24672	35.5758100964071\\
69.5	0.25038	37.0584850881189\\
69.5	0.25404	38.5667040662626\\
69.5	0.2577	40.1004670308382\\
69.5	0.26136	41.6597739818458\\
69.5	0.26502	43.2446249192852\\
69.5	0.26868	44.8550198431566\\
69.5	0.27234	46.4909587534599\\
69.5	0.276	48.1524416501951\\
69.875	0.093	-3.66583219910051\\
69.875	0.09666	-3.24877617036939\\
69.875	0.10032	-2.80617615520637\\
69.875	0.10398	-2.33803215361144\\
69.875	0.10764	-1.84434416558457\\
69.875	0.1113	-1.3251121911258\\
69.875	0.11496	-0.780336230235136\\
69.875	0.11862	-0.210016282912541\\
69.875	0.12228	0.385847650841967\\
69.875	0.12594	1.00725557102837\\
69.875	0.1296	1.65420747764672\\
69.875	0.13326	2.32670337069695\\
69.875	0.13692	3.0247432501791\\
69.875	0.14058	3.74832711609318\\
69.875	0.14424	4.49745496843918\\
69.875	0.1479	5.27212680721706\\
69.875	0.15156	6.07234263242687\\
69.875	0.15522	6.89810244406858\\
69.875	0.15888	7.74940624214223\\
69.875	0.16254	8.62625402664778\\
69.875	0.1662	9.52864579758524\\
69.875	0.16986	10.4565815549546\\
69.875	0.17352	11.4100612987559\\
69.875	0.17718	12.3890850289891\\
69.875	0.18084	13.3936527456542\\
69.875	0.1845	14.4237644487512\\
69.875	0.18816	15.4794201382802\\
69.875	0.19182	16.560619814241\\
69.875	0.19548	17.6673634766338\\
69.875	0.19914	18.7996511254585\\
69.875	0.2028	19.9574827607151\\
69.875	0.20646	21.1408583824036\\
69.875	0.21012	22.349777990524\\
69.875	0.21378	23.5842415850763\\
69.875	0.21744	24.8442491660606\\
69.875	0.2211	26.1298007334767\\
69.875	0.22476	27.4408962873248\\
69.875	0.22842	28.7775358276048\\
69.875	0.23208	30.1397193543166\\
69.875	0.23574	31.5274468674605\\
69.875	0.2394	32.9407183670362\\
69.875	0.24306	34.3795338530438\\
69.875	0.24672	35.8438933254834\\
69.875	0.25038	37.3337967843548\\
69.875	0.25404	38.8492442296582\\
69.875	0.2577	40.3902356613935\\
69.875	0.26136	41.9567710795607\\
69.875	0.26502	43.5488504841598\\
69.875	0.26868	45.1664738751908\\
69.875	0.27234	46.8096412526537\\
69.875	0.276	48.4783526165486\\
70.25	0.093	-3.69897591312978\\
70.25	0.09666	-3.274691417239\\
70.25	0.10032	-2.82486293491632\\
70.25	0.10398	-2.34949046616173\\
70.25	0.10764	-1.84857401097521\\
70.25	0.1113	-1.32211356935679\\
70.25	0.11496	-0.770109141306463\\
70.25	0.11862	-0.192560726824205\\
70.25	0.12228	0.41053167408996\\
70.25	0.12594	1.03916806143602\\
70.25	0.1296	1.69334843521402\\
70.25	0.13326	2.37307279542392\\
70.25	0.13692	3.07834114206572\\
70.25	0.14058	3.80915347513946\\
70.25	0.14424	4.56550979464512\\
70.25	0.1479	5.34741010058265\\
70.25	0.15156	6.15485439295212\\
70.25	0.15522	6.9878426717535\\
70.25	0.15888	7.84637493698679\\
70.25	0.16254	8.730451188652\\
70.25	0.1662	9.64007142674912\\
70.25	0.16986	10.5752356512781\\
70.25	0.17352	11.5359438622391\\
70.25	0.17718	12.5221960596319\\
70.25	0.18084	13.5339922434567\\
70.25	0.1845	14.5713324137134\\
70.25	0.18816	15.634216570402\\
70.25	0.19182	16.7226447135225\\
70.25	0.19548	17.8366168430749\\
70.25	0.19914	18.9761329590593\\
70.25	0.2028	20.1411930614755\\
70.25	0.20646	21.3317971503237\\
70.25	0.21012	22.5479452256038\\
70.25	0.21378	23.7896372873157\\
70.25	0.21744	25.0568733354596\\
70.25	0.2211	26.3496533700354\\
70.25	0.22476	27.6679773910432\\
70.25	0.22842	29.0118453984828\\
70.25	0.23208	30.3812573923544\\
70.25	0.23574	31.7762133726578\\
70.25	0.2394	33.1967133393932\\
70.25	0.24306	34.6427572925605\\
70.25	0.24672	36.1143452321597\\
70.25	0.25038	37.6114771581908\\
70.25	0.25404	39.1341530706539\\
70.25	0.2577	40.6823729695488\\
70.25	0.26136	42.2561368548756\\
70.25	0.26502	43.8554447266344\\
70.25	0.26868	45.4802965848251\\
70.25	0.27234	47.1306924294477\\
70.25	0.276	48.8066322605022\\
70.625	0.093	-3.72975094955896\\
70.625	0.09666	-3.29823798650853\\
70.625	0.10032	-2.84118103702618\\
70.625	0.10398	-2.35858010111193\\
70.625	0.10764	-1.85043517876577\\
70.625	0.1113	-1.31674626998769\\
70.625	0.11496	-0.75751337477768\\
70.625	0.11862	-0.17273649313578\\
70.625	0.12228	0.437584374938041\\
70.625	0.12594	1.07344922944378\\
70.625	0.1296	1.73485807038141\\
70.625	0.13326	2.42181089775098\\
70.625	0.13692	3.13430771155245\\
70.625	0.14058	3.87234851178584\\
70.625	0.14424	4.63593329845114\\
70.625	0.1479	5.42506207154834\\
70.625	0.15156	6.23973483107747\\
70.625	0.15522	7.0799515770385\\
70.625	0.15888	7.94571230943146\\
70.625	0.16254	8.83701702825632\\
70.625	0.1662	9.7538657335131\\
70.625	0.16986	10.6962584252018\\
70.625	0.17352	11.6641951033224\\
70.625	0.17718	12.6576757678749\\
70.625	0.18084	13.6767004188593\\
70.625	0.1845	14.7212690562757\\
70.625	0.18816	15.7913816801239\\
70.625	0.19182	16.8870382904041\\
70.625	0.19548	18.0082388871162\\
70.625	0.19914	19.1549834702602\\
70.625	0.2028	20.3272720398361\\
70.625	0.20646	21.5251045958439\\
70.625	0.21012	22.7484811382837\\
70.625	0.21378	23.9974016671552\\
70.625	0.21744	25.2718661824588\\
70.625	0.2211	26.5718746841943\\
70.625	0.22476	27.8974271723617\\
70.625	0.22842	29.248523646961\\
70.625	0.23208	30.6251641079922\\
70.625	0.23574	32.0273485554553\\
70.625	0.2394	33.4550769893504\\
70.625	0.24306	34.9083494096773\\
70.625	0.24672	36.3871658164361\\
70.625	0.25038	37.891526209627\\
70.625	0.25404	39.4214305892496\\
70.625	0.2577	40.9768789553042\\
70.625	0.26136	42.5578713077907\\
70.625	0.26502	44.1644076467092\\
70.625	0.26868	45.7964879720595\\
70.625	0.27234	47.4541122838417\\
70.625	0.276	49.1372805820559\\
71	0.093	-3.75815730838807\\
71	0.09666	-3.31941587817799\\
71	0.10032	-2.85513046153598\\
71	0.10398	-2.36530105846206\\
71	0.10764	-1.84992766895625\\
71	0.1113	-1.30901029301851\\
71	0.11496	-0.742548930648844\\
71	0.11862	-0.150543581847288\\
71	0.12228	0.467005753386196\\
71	0.12594	1.11009907505159\\
71	0.1296	1.77873638314888\\
71	0.13326	2.4729176776781\\
71	0.13692	3.19264295863923\\
71	0.14058	3.93791222603229\\
71	0.14424	4.70872547985723\\
71	0.1479	5.5050827201141\\
71	0.15156	6.32698394680289\\
71	0.15522	7.17442915992357\\
71	0.15888	8.04741835947618\\
71	0.16254	8.94595154546071\\
71	0.1662	9.87002871787715\\
71	0.16986	10.8196498767255\\
71	0.17352	11.7948150220057\\
71	0.17718	12.7955241537179\\
71	0.18084	13.821777271862\\
71	0.1845	14.873574376438\\
71	0.18816	15.9509154674459\\
71	0.19182	17.0538005448857\\
71	0.19548	18.1822296087575\\
71	0.19914	19.3362026590611\\
71	0.2028	20.5157196957967\\
71	0.20646	21.7207807189642\\
71	0.21012	22.9513857285636\\
71	0.21378	24.2075347245949\\
71	0.21744	25.4892277070581\\
71	0.2211	26.7964646759532\\
71	0.22476	28.1292456312803\\
71	0.22842	29.4875705730392\\
71	0.23208	30.87143950123\\
71	0.23574	32.2808524158529\\
71	0.2394	33.7158093169076\\
71	0.24306	35.1763102043941\\
71	0.24672	36.6623550783127\\
71	0.25038	38.1739439386631\\
71	0.25404	39.7110767854455\\
71	0.2577	41.2737536186597\\
71	0.26136	42.8619744383059\\
71	0.26502	44.475739244384\\
71	0.26868	46.1150480368939\\
71	0.27234	47.7799008158358\\
71	0.276	49.4702975812097\\
71.375	0.093	-3.78419498961708\\
71.375	0.09666	-3.33822509224733\\
71.375	0.10032	-2.86671120844568\\
71.375	0.10398	-2.36965333821212\\
71.375	0.10764	-1.84705148154662\\
71.375	0.1113	-1.29890563844922\\
71.375	0.11496	-0.72521580891992\\
71.375	0.11862	-0.1259819929587\\
71.375	0.12228	0.498795809434441\\
71.375	0.12594	1.14911759825948\\
71.375	0.1296	1.82498337351646\\
71.375	0.13326	2.52639313520532\\
71.375	0.13692	3.2533468833261\\
71.375	0.14058	4.0058446178788\\
71.375	0.14424	4.78388633886344\\
71.375	0.1479	5.58747204627995\\
71.375	0.15156	6.4166017401284\\
71.375	0.15522	7.27127542040874\\
71.375	0.15888	8.15149308712101\\
71.375	0.16254	9.05725474026519\\
71.375	0.1662	9.98856037984129\\
71.375	0.16986	10.9454100058493\\
71.375	0.17352	11.9278036182892\\
71.375	0.17718	12.935741217161\\
71.375	0.18084	13.9692228024648\\
71.375	0.1845	15.0282483742004\\
71.375	0.18816	16.112817932368\\
71.375	0.19182	17.2229314769675\\
71.375	0.19548	18.3585890079989\\
71.375	0.19914	19.5197905254622\\
71.375	0.2028	20.7065360293574\\
71.375	0.20646	21.9188255196845\\
71.375	0.21012	23.1566589964436\\
71.375	0.21378	24.4200364596346\\
71.375	0.21744	25.7089579092574\\
71.375	0.2211	27.0234233453122\\
71.375	0.22476	28.3634327677989\\
71.375	0.22842	29.7289861767175\\
71.375	0.23208	31.120083572068\\
71.375	0.23574	32.5367249538505\\
71.375	0.2394	33.9789103220648\\
71.375	0.24306	35.4466396767111\\
71.375	0.24672	36.9399130177893\\
71.375	0.25038	38.4587303452994\\
71.375	0.25404	40.0030916592414\\
71.375	0.2577	41.5729969596153\\
71.375	0.26136	43.1684462464211\\
71.375	0.26502	44.7894395196589\\
71.375	0.26868	46.4359767793285\\
71.375	0.27234	48.1080580254301\\
71.375	0.276	49.8056832579636\\
71.75	0.093	-3.80786399324603\\
71.75	0.09666	-3.35466562871662\\
71.75	0.10032	-2.87592327775531\\
71.75	0.10398	-2.37163694036209\\
71.75	0.10764	-1.84180661653693\\
71.75	0.1113	-1.28643230627988\\
71.75	0.11496	-0.705514009590921\\
71.75	0.11862	-0.0990517264700372\\
71.75	0.12228	0.532954543082759\\
71.75	0.12594	1.19050479906745\\
71.75	0.1296	1.87359904148409\\
71.75	0.13326	2.58223727033261\\
71.75	0.13692	3.31641948561304\\
71.75	0.14058	4.07614568732541\\
71.75	0.14424	4.8614158754697\\
71.75	0.1479	5.67223005004587\\
71.75	0.15156	6.50858821105397\\
71.75	0.15522	7.37049035849397\\
71.75	0.15888	8.25793649236591\\
71.75	0.16254	9.17092661266975\\
71.75	0.1662	10.1094607194055\\
71.75	0.16986	11.0735388125731\\
71.75	0.17352	12.0631608921727\\
71.75	0.17718	13.0783269582042\\
71.75	0.18084	14.1190370106676\\
71.75	0.1845	15.1852910495629\\
71.75	0.18816	16.2770890748902\\
71.75	0.19182	17.3944310866493\\
71.75	0.19548	18.5373170848404\\
71.75	0.19914	19.7057470694633\\
71.75	0.2028	20.8997210405182\\
71.75	0.20646	22.119238998005\\
71.75	0.21012	23.3643009419237\\
71.75	0.21378	24.6349068722743\\
71.75	0.21744	25.9310567890569\\
71.75	0.2211	27.2527506922713\\
71.75	0.22476	28.5999885819176\\
71.75	0.22842	29.9727704579959\\
71.75	0.23208	31.3710963205061\\
71.75	0.23574	32.7949661694482\\
71.75	0.2394	34.2443800048222\\
71.75	0.24306	35.7193378266281\\
71.75	0.24672	37.219839634866\\
71.75	0.25038	38.7458854295357\\
71.75	0.25404	40.2974752106374\\
71.75	0.2577	41.874608978171\\
71.75	0.26136	43.4772867321364\\
71.75	0.26502	45.1055084725338\\
71.75	0.26868	46.7592741993631\\
71.75	0.27234	48.4385839126244\\
71.75	0.276	50.1434376123175\\
72.125	0.093	-3.82916431927489\\
72.125	0.09666	-3.36873748758583\\
72.125	0.10032	-2.88276666946486\\
72.125	0.10398	-2.37125186491198\\
72.125	0.10764	-1.83419307392716\\
72.125	0.1113	-1.27159029651045\\
72.125	0.11496	-0.68344353266184\\
72.125	0.11862	-0.0697527823813004\\
72.125	0.12228	0.569481954331152\\
72.125	0.12594	1.23426067747551\\
72.125	0.1296	1.9245833870518\\
72.125	0.13326	2.64045008305998\\
72.125	0.13692	3.38186076550007\\
72.125	0.14058	4.14881543437209\\
72.125	0.14424	4.94131408967604\\
72.125	0.1479	5.75935673141187\\
72.125	0.15156	6.60294335957963\\
72.125	0.15522	7.47207397417929\\
72.125	0.15888	8.36674857521088\\
72.125	0.16254	9.28696716267437\\
72.125	0.1662	10.2327297365698\\
72.125	0.16986	11.2040362968971\\
72.125	0.17352	12.2008868436563\\
72.125	0.17718	13.2232813768475\\
72.125	0.18084	14.2712198964705\\
72.125	0.1845	15.3447024025255\\
72.125	0.18816	16.4437288950124\\
72.125	0.19182	17.5682993739312\\
72.125	0.19548	18.7184138392819\\
72.125	0.19914	19.8940722910645\\
72.125	0.2028	21.0952747292791\\
72.125	0.20646	22.3220211539255\\
72.125	0.21012	23.5743115650039\\
72.125	0.21378	24.8521459625142\\
72.125	0.21744	26.1555243464563\\
72.125	0.2211	27.4844467168304\\
72.125	0.22476	28.8389130736365\\
72.125	0.22842	30.2189234168744\\
72.125	0.23208	31.6244777465442\\
72.125	0.23574	33.055576062646\\
72.125	0.2394	34.5122183651796\\
72.125	0.24306	35.9944046541452\\
72.125	0.24672	37.5021349295427\\
72.125	0.25038	39.0354091913721\\
72.125	0.25404	40.5942274396335\\
72.125	0.2577	42.1785896743267\\
72.125	0.26136	43.7884958954518\\
72.125	0.26502	45.4239461030089\\
72.125	0.26868	47.0849402969978\\
72.125	0.27234	48.7714784774187\\
72.125	0.276	50.4835606442715\\
72.5	0.093	-3.84809596770366\\
72.5	0.09666	-3.38044066885495\\
72.5	0.10032	-2.88724138357431\\
72.5	0.10398	-2.36849811186176\\
72.5	0.10764	-1.82421085371732\\
72.5	0.1113	-1.25437960914095\\
72.5	0.11496	-0.65900437813265\\
72.5	0.11862	-0.0380851606924679\\
72.5	0.12228	0.608378043179648\\
72.5	0.12594	1.28038523348367\\
72.5	0.1296	1.97793641021959\\
72.5	0.13326	2.70103157338744\\
72.5	0.13692	3.44967072298721\\
72.5	0.14058	4.22385385901889\\
72.5	0.14424	5.02358098148247\\
72.5	0.1479	5.84885209037797\\
72.5	0.15156	6.69966718570538\\
72.5	0.15522	7.5760262674647\\
72.5	0.15888	8.47792933565594\\
72.5	0.16254	9.4053763902791\\
72.5	0.1662	10.3583674313342\\
72.5	0.16986	11.3369024588211\\
72.5	0.17352	12.34098147274\\
72.5	0.17718	13.3706044730908\\
72.5	0.18084	14.4257714598736\\
72.5	0.1845	15.5064824330882\\
72.5	0.18816	16.6127373927347\\
72.5	0.19182	17.7445363388132\\
72.5	0.19548	18.9018792713235\\
72.5	0.19914	20.0847661902658\\
72.5	0.2028	21.29319709564\\
72.5	0.20646	22.5271719874461\\
72.5	0.21012	23.7866908656842\\
72.5	0.21378	25.0717537303541\\
72.5	0.21744	26.3823605814559\\
72.5	0.2211	27.7185114189897\\
72.5	0.22476	29.0802062429554\\
72.5	0.22842	30.4674450533529\\
72.5	0.23208	31.8802278501825\\
72.5	0.23574	33.3185546334439\\
72.5	0.2394	34.7824254031372\\
72.5	0.24306	36.2718401592624\\
72.5	0.24672	37.7867989018196\\
72.5	0.25038	39.3273016308087\\
72.5	0.25404	40.8933483462296\\
72.5	0.2577	42.4849390480825\\
72.5	0.26136	44.1020737363673\\
72.5	0.26502	45.744752411084\\
72.5	0.26868	47.4129750722326\\
72.5	0.27234	49.1067417198132\\
72.5	0.276	50.8260523538256\\
72.875	0.093	-3.86465893853234\\
72.875	0.09666	-3.38977517252396\\
72.875	0.10032	-2.88934742008367\\
72.875	0.10398	-2.36337568121149\\
72.875	0.10764	-1.81185995590735\\
72.875	0.1113	-1.23480024417133\\
72.875	0.11496	-0.632196546003399\\
72.875	0.11862	-0.00404886140353966\\
72.875	0.12228	0.649642809628226\\
72.875	0.12594	1.3288784670919\\
72.875	0.1296	2.03365811098751\\
72.875	0.13326	2.763981741315\\
72.875	0.13692	3.51984935807441\\
72.875	0.14058	4.30126096126574\\
72.875	0.14424	5.10821655088901\\
72.875	0.1479	5.94071612694415\\
72.875	0.15156	6.79875968943123\\
72.875	0.15522	7.6823472383502\\
72.875	0.15888	8.59147877370111\\
72.875	0.16254	9.52615429548392\\
72.875	0.1662	10.4863738036986\\
72.875	0.16986	11.4721372983453\\
72.875	0.17352	12.4834447794238\\
72.875	0.17718	13.5202962469343\\
72.875	0.18084	14.5826917008767\\
72.875	0.1845	15.6706311412509\\
72.875	0.18816	16.7841145680571\\
72.875	0.19182	17.9231419812952\\
72.875	0.19548	19.0877133809653\\
72.875	0.19914	20.2778287670672\\
72.875	0.2028	21.4934881396011\\
72.875	0.20646	22.7346914985669\\
72.875	0.21012	24.0014388439645\\
72.875	0.21378	25.2937301757941\\
72.875	0.21744	26.6115654940556\\
72.875	0.2211	27.954944798749\\
72.875	0.22476	29.3238680898743\\
72.875	0.22842	30.7183353674316\\
72.875	0.23208	32.1383466314208\\
72.875	0.23574	33.5839018818418\\
72.875	0.2394	35.0550011186948\\
72.875	0.24306	36.5516443419797\\
72.875	0.24672	38.0738315516965\\
72.875	0.25038	39.6215627478452\\
72.875	0.25404	41.1948379304259\\
72.875	0.2577	42.7936570994384\\
72.875	0.26136	44.4180202548829\\
72.875	0.26502	46.0679273967592\\
72.875	0.26868	47.7433785250675\\
72.875	0.27234	49.4443736398077\\
72.875	0.276	51.1709127409798\\
73.25	0.093	-3.87885323176096\\
73.25	0.09666	-3.39674099859292\\
73.25	0.10032	-2.88908477899297\\
73.25	0.10398	-2.35588457296113\\
73.25	0.10764	-1.79714038049733\\
73.25	0.1113	-1.21285220160166\\
73.25	0.11496	-0.603020036274067\\
73.25	0.11862	0.0323561154854488\\
73.25	0.12228	0.693276253676878\\
73.25	0.12594	1.37974037830021\\
73.25	0.1296	2.09174848935547\\
73.25	0.13326	2.82930058684262\\
73.25	0.13692	3.59239667076168\\
73.25	0.14058	4.38103674111268\\
73.25	0.14424	5.19522079789561\\
73.25	0.1479	6.0349488411104\\
73.25	0.15156	6.90022087075713\\
73.25	0.15522	7.79103688683577\\
73.25	0.15888	8.70739688934633\\
73.25	0.16254	9.6493008782888\\
73.25	0.1662	10.6167488536632\\
73.25	0.16986	11.6097408154695\\
73.25	0.17352	12.6282767637077\\
73.25	0.17718	13.6723566983778\\
73.25	0.18084	14.7419806194798\\
73.25	0.1845	15.8371485270138\\
73.25	0.18816	16.9578604209796\\
73.25	0.19182	18.1041163013774\\
73.25	0.19548	19.2759161682071\\
73.25	0.19914	20.4732600214687\\
73.25	0.2028	21.6961478611622\\
73.25	0.20646	22.9445796872876\\
73.25	0.21012	24.218555499845\\
73.25	0.21378	25.5180752988342\\
73.25	0.21744	26.8431390842554\\
73.25	0.2211	28.1937468561084\\
73.25	0.22476	29.5698986143934\\
73.25	0.22842	30.9715943591103\\
73.25	0.23208	32.3988340902591\\
73.25	0.23574	33.8516178078399\\
73.25	0.2394	35.3299455118525\\
73.25	0.24306	36.8338172022971\\
73.25	0.24672	38.3632328791735\\
73.25	0.25038	39.9181925424819\\
73.25	0.25404	41.4986961922222\\
73.25	0.2577	43.1047438283944\\
73.25	0.26136	44.7363354509985\\
73.25	0.26502	46.3934710600346\\
73.25	0.26868	48.0761506555025\\
73.25	0.27234	49.7843742374023\\
73.25	0.276	51.5181418057341\\
73.625	0.093	-3.89067884738949\\
73.625	0.09666	-3.40133814706179\\
73.625	0.10032	-2.88645346030219\\
73.625	0.10398	-2.34602478711069\\
73.625	0.10764	-1.78005212748724\\
73.625	0.1113	-1.1885354814319\\
73.625	0.11496	-0.571474848944654\\
73.625	0.11862	0.0711297699745188\\
73.625	0.12228	0.739278375325604\\
73.625	0.12594	1.43297096710859\\
73.625	0.1296	2.15220754532352\\
73.625	0.13326	2.89698810997033\\
73.625	0.13692	3.66731266104905\\
73.625	0.14058	4.4631811985597\\
73.625	0.14424	5.28459372250228\\
73.625	0.1479	6.13155023287674\\
73.625	0.15156	7.00405072968313\\
73.625	0.15522	7.90209521292142\\
73.625	0.15888	8.82568368259164\\
73.625	0.16254	9.77481613869377\\
73.625	0.1662	10.7494925812278\\
73.625	0.16986	11.7497130101937\\
73.625	0.17352	12.7754774255916\\
73.625	0.17718	13.8267858274214\\
73.625	0.18084	14.9036382156831\\
73.625	0.1845	16.0060345903767\\
73.625	0.18816	17.1339749515022\\
73.625	0.19182	18.2874592990596\\
73.625	0.19548	19.466487633049\\
73.625	0.19914	20.6710599534702\\
73.625	0.2028	21.9011762603234\\
73.625	0.20646	23.1568365536085\\
73.625	0.21012	24.4380408333255\\
73.625	0.21378	25.7447890994744\\
73.625	0.21744	27.0770813520552\\
73.625	0.2211	28.4349175910679\\
73.625	0.22476	29.8182978165126\\
73.625	0.22842	31.2272220283891\\
73.625	0.23208	32.6616902266976\\
73.625	0.23574	34.121702411438\\
73.625	0.2394	35.6072585826103\\
73.625	0.24306	37.1183587402145\\
73.625	0.24672	38.6550028842506\\
73.625	0.25038	40.2171910147187\\
73.625	0.25404	41.8049231316186\\
73.625	0.2577	43.4181992349505\\
73.625	0.26136	45.0570193247142\\
73.625	0.26502	46.7213834009099\\
73.625	0.26868	48.4112914635375\\
73.625	0.27234	50.126743512597\\
73.625	0.276	51.8677395480885\\
74	0.093	-3.90013578541794\\
74	0.09666	-3.40356661793058\\
74	0.10032	-2.88145346401132\\
74	0.10398	-2.33379632366016\\
74	0.10764	-1.76059519687706\\
74	0.1113	-1.16185008366206\\
74	0.11496	-0.537560984015158\\
74	0.11862	0.112272102063677\\
74	0.12228	0.787649174574419\\
74	0.12594	1.48857023351706\\
74	0.1296	2.21503527889164\\
74	0.13326	2.96704431069811\\
74	0.13692	3.74459732893649\\
74	0.14058	4.5476943336068\\
74	0.14424	5.37633532470904\\
74	0.1479	6.23052030224315\\
74	0.15156	7.11024926620921\\
74	0.15522	8.01552221660715\\
74	0.15888	8.94633915343703\\
74	0.16254	9.90270007669881\\
74	0.1662	10.8846049863925\\
74	0.16986	11.8920538825181\\
74	0.17352	12.9250467650756\\
74	0.17718	13.983583634065\\
74	0.18084	15.0676644894864\\
74	0.1845	16.1772893313397\\
74	0.18816	17.3124581596249\\
74	0.19182	18.473170974342\\
74	0.19548	19.6594277754909\\
74	0.19914	20.8712285630718\\
74	0.2028	22.1085733370847\\
74	0.20646	23.3714620975294\\
74	0.21012	24.6598948444061\\
74	0.21378	25.9738715777146\\
74	0.21744	27.3133922974551\\
74	0.2211	28.6784570036275\\
74	0.22476	30.0690656962318\\
74	0.22842	31.485218375268\\
74	0.23208	32.9269150407361\\
74	0.23574	34.3941556926362\\
74	0.2394	35.8869403309681\\
74	0.24306	37.405268955732\\
74	0.24672	38.9491415669278\\
74	0.25038	40.5185581645555\\
74	0.25404	42.1135187486151\\
74	0.2577	43.7340233191066\\
74	0.26136	45.3800718760301\\
74	0.26502	47.0516644193854\\
74	0.26868	48.7488009491726\\
74	0.27234	50.4714814653918\\
74	0.276	52.2197059680429\\
};
\end{axis}

\begin{axis}[%
width=4.527496cm,
height=3.870968cm,
at={(6.483547cm,16.129032cm)},
scale only axis,
xmin=56,
xmax=74,
tick align=outside,
xlabel={$L_{cut}$},
xmajorgrids,
ymin=0.093,
ymax=0.276,
ylabel={$D_{rlx}$},
ymajorgrids,
zmin=0,
zmax=517.903625289077,
zlabel={$x_3$},
zmajorgrids,
view={-140}{50},
legend style={at={(1.03,1)},anchor=north west,legend cell align=left,align=left,draw=white!15!black}
]
\addplot3[only marks,mark=*,mark options={},mark size=1.5000pt,color=mycolor1] plot table[row sep=crcr,]{%
74	0.123	75.4245651900459\\
72	0.113	59.5782863482935\\
61	0.095	33.0003406266977\\
56	0.093	26.0569470415911\\
};
\addplot3[only marks,mark=*,mark options={},mark size=1.5000pt,color=mycolor2] plot table[row sep=crcr,]{%
67	0.276	508.93731875208\\
66	0.255	419.53506953587\\
62	0.209	242.35439857544\\
57	0.193	195.15231965962\\
};
\addplot3[only marks,mark=*,mark options={},mark size=1.5000pt,color=black] plot table[row sep=crcr,]{%
69	0.104	48.1087944305374\\
};
\addplot3[only marks,mark=*,mark options={},mark size=1.5000pt,color=black] plot table[row sep=crcr,]{%
64	0.23	317.238786352579\\
};

\addplot3[%
surf,
opacity=0.7,
shader=interp,
colormap={mymap}{[1pt] rgb(0pt)=(0.0901961,0.239216,0.0745098); rgb(1pt)=(0.0945149,0.242058,0.0739522); rgb(2pt)=(0.0988592,0.244894,0.0733566); rgb(3pt)=(0.103229,0.247724,0.0727241); rgb(4pt)=(0.107623,0.250549,0.0720557); rgb(5pt)=(0.112043,0.253367,0.0713525); rgb(6pt)=(0.116487,0.25618,0.0706154); rgb(7pt)=(0.120956,0.258986,0.0698456); rgb(8pt)=(0.125449,0.261787,0.0690441); rgb(9pt)=(0.129967,0.264581,0.0682118); rgb(10pt)=(0.134508,0.26737,0.06735); rgb(11pt)=(0.139074,0.270152,0.0664596); rgb(12pt)=(0.143663,0.272929,0.0655416); rgb(13pt)=(0.148275,0.275699,0.0645971); rgb(14pt)=(0.152911,0.278463,0.0636271); rgb(15pt)=(0.15757,0.281221,0.0626328); rgb(16pt)=(0.162252,0.283973,0.0616151); rgb(17pt)=(0.166957,0.286719,0.060575); rgb(18pt)=(0.171685,0.289458,0.0595136); rgb(19pt)=(0.176434,0.292191,0.0584321); rgb(20pt)=(0.181207,0.294918,0.0573313); rgb(21pt)=(0.186001,0.297639,0.0562123); rgb(22pt)=(0.190817,0.300353,0.0550763); rgb(23pt)=(0.195655,0.303061,0.0539242); rgb(24pt)=(0.200514,0.305763,0.052757); rgb(25pt)=(0.205395,0.308459,0.0515759); rgb(26pt)=(0.210296,0.311149,0.0503624); rgb(27pt)=(0.215212,0.313846,0.0490067); rgb(28pt)=(0.220142,0.316548,0.0475043); rgb(29pt)=(0.22509,0.319254,0.0458704); rgb(30pt)=(0.230056,0.321962,0.0441205); rgb(31pt)=(0.235042,0.324671,0.04227); rgb(32pt)=(0.240048,0.327379,0.0403343); rgb(33pt)=(0.245078,0.330085,0.0383287); rgb(34pt)=(0.250131,0.332786,0.0362688); rgb(35pt)=(0.25521,0.335482,0.0341698); rgb(36pt)=(0.260317,0.33817,0.0320472); rgb(37pt)=(0.265451,0.340849,0.0299163); rgb(38pt)=(0.270616,0.343517,0.0277927); rgb(39pt)=(0.275813,0.346172,0.0256916); rgb(40pt)=(0.281043,0.348814,0.0236284); rgb(41pt)=(0.286307,0.35144,0.0216186); rgb(42pt)=(0.291607,0.354048,0.0196776); rgb(43pt)=(0.296945,0.356637,0.0178207); rgb(44pt)=(0.302322,0.359206,0.0160634); rgb(45pt)=(0.307739,0.361753,0.0144211); rgb(46pt)=(0.313198,0.364275,0.0129091); rgb(47pt)=(0.318701,0.366772,0.0115428); rgb(48pt)=(0.324249,0.369242,0.0103377); rgb(49pt)=(0.329843,0.371682,0.00930909); rgb(50pt)=(0.335485,0.374093,0.00847245); rgb(51pt)=(0.341176,0.376471,0.00784314); rgb(52pt)=(0.346925,0.378826,0.00732741); rgb(53pt)=(0.352735,0.381168,0.00682184); rgb(54pt)=(0.358605,0.383497,0.00632729); rgb(55pt)=(0.364532,0.385812,0.00584464); rgb(56pt)=(0.370516,0.388113,0.00537476); rgb(57pt)=(0.376552,0.390399,0.00491852); rgb(58pt)=(0.38264,0.39267,0.00447681); rgb(59pt)=(0.388777,0.394925,0.00405048); rgb(60pt)=(0.394962,0.397164,0.00364042); rgb(61pt)=(0.401191,0.399386,0.00324749); rgb(62pt)=(0.407464,0.401592,0.00287258); rgb(63pt)=(0.413777,0.40378,0.00251655); rgb(64pt)=(0.420129,0.40595,0.00218028); rgb(65pt)=(0.426518,0.408102,0.00186463); rgb(66pt)=(0.432942,0.410234,0.00157049); rgb(67pt)=(0.439399,0.412348,0.00129873); rgb(68pt)=(0.445885,0.414441,0.00105022); rgb(69pt)=(0.452401,0.416515,0.000825833); rgb(70pt)=(0.458942,0.418567,0.000626441); rgb(71pt)=(0.465508,0.420599,0.00045292); rgb(72pt)=(0.472096,0.422609,0.000306141); rgb(73pt)=(0.478704,0.424596,0.000186979); rgb(74pt)=(0.485331,0.426562,9.63073e-05); rgb(75pt)=(0.491973,0.428504,3.49981e-05); rgb(76pt)=(0.498628,0.430422,3.92506e-06); rgb(77pt)=(0.505323,0.432315,0); rgb(78pt)=(0.512206,0.434168,0); rgb(79pt)=(0.519282,0.435983,0); rgb(80pt)=(0.526529,0.437764,0); rgb(81pt)=(0.533922,0.439512,0); rgb(82pt)=(0.54144,0.441232,0); rgb(83pt)=(0.549059,0.442927,0); rgb(84pt)=(0.556756,0.444599,0); rgb(85pt)=(0.564508,0.446252,0); rgb(86pt)=(0.572292,0.447889,0); rgb(87pt)=(0.580084,0.449514,0); rgb(88pt)=(0.587863,0.451129,0); rgb(89pt)=(0.595604,0.452737,0); rgb(90pt)=(0.603284,0.454343,0); rgb(91pt)=(0.610882,0.455948,0); rgb(92pt)=(0.618373,0.457556,0); rgb(93pt)=(0.625734,0.459171,0); rgb(94pt)=(0.632943,0.460795,0); rgb(95pt)=(0.639976,0.462432,0); rgb(96pt)=(0.64681,0.464084,0); rgb(97pt)=(0.653423,0.465756,0); rgb(98pt)=(0.659791,0.46745,0); rgb(99pt)=(0.665891,0.469169,0); rgb(100pt)=(0.6717,0.470916,0); rgb(101pt)=(0.677195,0.472696,0); rgb(102pt)=(0.682353,0.47451,0); rgb(103pt)=(0.687242,0.476355,0); rgb(104pt)=(0.691952,0.478225,0); rgb(105pt)=(0.696497,0.480118,0); rgb(106pt)=(0.700887,0.482033,0); rgb(107pt)=(0.705134,0.483968,0); rgb(108pt)=(0.709251,0.485921,0); rgb(109pt)=(0.713249,0.487891,0); rgb(110pt)=(0.71714,0.489876,0); rgb(111pt)=(0.720936,0.491875,0); rgb(112pt)=(0.724649,0.493887,0); rgb(113pt)=(0.72829,0.495909,0); rgb(114pt)=(0.731872,0.49794,0); rgb(115pt)=(0.735406,0.499979,0); rgb(116pt)=(0.738904,0.502025,0); rgb(117pt)=(0.742378,0.504075,0); rgb(118pt)=(0.74584,0.506128,0); rgb(119pt)=(0.749302,0.508182,0); rgb(120pt)=(0.752775,0.510237,0); rgb(121pt)=(0.756272,0.51229,0); rgb(122pt)=(0.759804,0.514339,0); rgb(123pt)=(0.763384,0.516385,0); rgb(124pt)=(0.767022,0.518424,0); rgb(125pt)=(0.770731,0.520455,0); rgb(126pt)=(0.774523,0.522478,0); rgb(127pt)=(0.77841,0.524489,0); rgb(128pt)=(0.782391,0.526491,0); rgb(129pt)=(0.786402,0.528496,0); rgb(130pt)=(0.790431,0.530506,0); rgb(131pt)=(0.794478,0.532521,0); rgb(132pt)=(0.798541,0.534539,0); rgb(133pt)=(0.802619,0.53656,0); rgb(134pt)=(0.806712,0.538584,0); rgb(135pt)=(0.81082,0.540609,0); rgb(136pt)=(0.81494,0.542635,0); rgb(137pt)=(0.819074,0.54466,0); rgb(138pt)=(0.823219,0.546686,0); rgb(139pt)=(0.827374,0.548709,0); rgb(140pt)=(0.831541,0.55073,0); rgb(141pt)=(0.835716,0.552749,0); rgb(142pt)=(0.8399,0.554763,0); rgb(143pt)=(0.844092,0.556774,0); rgb(144pt)=(0.848292,0.558779,0); rgb(145pt)=(0.852497,0.560778,0); rgb(146pt)=(0.856708,0.562771,0); rgb(147pt)=(0.860924,0.564756,0); rgb(148pt)=(0.865143,0.566733,0); rgb(149pt)=(0.869366,0.568701,0); rgb(150pt)=(0.873592,0.57066,0); rgb(151pt)=(0.877819,0.572608,0); rgb(152pt)=(0.882047,0.574545,0); rgb(153pt)=(0.886275,0.576471,0); rgb(154pt)=(0.890659,0.578362,0); rgb(155pt)=(0.895333,0.580203,0); rgb(156pt)=(0.900258,0.581999,0); rgb(157pt)=(0.905397,0.583755,0); rgb(158pt)=(0.910711,0.585479,0); rgb(159pt)=(0.916164,0.587176,0); rgb(160pt)=(0.921717,0.588852,0); rgb(161pt)=(0.927333,0.590513,0); rgb(162pt)=(0.932974,0.592166,0); rgb(163pt)=(0.938602,0.593815,0); rgb(164pt)=(0.94418,0.595468,0); rgb(165pt)=(0.949669,0.59713,0); rgb(166pt)=(0.955033,0.598808,0); rgb(167pt)=(0.960233,0.600507,0); rgb(168pt)=(0.965232,0.602233,0); rgb(169pt)=(0.969992,0.603992,0); rgb(170pt)=(0.974475,0.605791,0); rgb(171pt)=(0.978643,0.607636,0); rgb(172pt)=(0.98246,0.609532,0); rgb(173pt)=(0.985886,0.611486,0); rgb(174pt)=(0.988885,0.613503,0); rgb(175pt)=(0.991419,0.61559,0); rgb(176pt)=(0.99345,0.617753,0); rgb(177pt)=(0.99494,0.619997,0); rgb(178pt)=(0.995851,0.622329,0); rgb(179pt)=(0.996226,0.624763,0); rgb(180pt)=(0.996512,0.627352,0); rgb(181pt)=(0.996788,0.630095,0); rgb(182pt)=(0.997053,0.632982,0); rgb(183pt)=(0.997308,0.636004,0); rgb(184pt)=(0.997552,0.639152,0); rgb(185pt)=(0.997785,0.642416,0); rgb(186pt)=(0.998006,0.645786,0); rgb(187pt)=(0.998217,0.649253,0); rgb(188pt)=(0.998416,0.652807,0); rgb(189pt)=(0.998605,0.656439,0); rgb(190pt)=(0.998781,0.660138,0); rgb(191pt)=(0.998946,0.663897,0); rgb(192pt)=(0.9991,0.667704,0); rgb(193pt)=(0.999242,0.67155,0); rgb(194pt)=(0.999372,0.675427,0); rgb(195pt)=(0.99949,0.679323,0); rgb(196pt)=(0.999596,0.68323,0); rgb(197pt)=(0.99969,0.687139,0); rgb(198pt)=(0.999771,0.691039,0); rgb(199pt)=(0.999841,0.694921,0); rgb(200pt)=(0.999898,0.698775,0); rgb(201pt)=(0.999942,0.702592,0); rgb(202pt)=(0.999974,0.706363,0); rgb(203pt)=(0.999994,0.710077,0); rgb(204pt)=(1,0.713725,0); rgb(205pt)=(1,0.717341,0); rgb(206pt)=(1,0.720963,0); rgb(207pt)=(1,0.724591,0); rgb(208pt)=(1,0.728226,0); rgb(209pt)=(1,0.731867,0); rgb(210pt)=(1,0.735514,0); rgb(211pt)=(1,0.739167,0); rgb(212pt)=(1,0.742827,0); rgb(213pt)=(1,0.746493,0); rgb(214pt)=(1,0.750165,0); rgb(215pt)=(1,0.753843,0); rgb(216pt)=(1,0.757527,0); rgb(217pt)=(1,0.761217,0); rgb(218pt)=(1,0.764913,0); rgb(219pt)=(1,0.768615,0); rgb(220pt)=(1,0.772324,0); rgb(221pt)=(1,0.776038,0); rgb(222pt)=(1,0.779758,0); rgb(223pt)=(1,0.783484,0); rgb(224pt)=(1,0.787215,0); rgb(225pt)=(1,0.790953,0); rgb(226pt)=(1,0.794696,0); rgb(227pt)=(1,0.798445,0); rgb(228pt)=(1,0.8022,0); rgb(229pt)=(1,0.805961,0); rgb(230pt)=(1,0.809727,0); rgb(231pt)=(1,0.8135,0); rgb(232pt)=(1,0.817278,0); rgb(233pt)=(1,0.821063,0); rgb(234pt)=(1,0.824854,0); rgb(235pt)=(1,0.828652,0); rgb(236pt)=(1,0.832455,0); rgb(237pt)=(1,0.836265,0); rgb(238pt)=(1,0.840081,0); rgb(239pt)=(1,0.843903,0); rgb(240pt)=(1,0.847732,0); rgb(241pt)=(1,0.851566,0); rgb(242pt)=(1,0.855406,0); rgb(243pt)=(1,0.859253,0); rgb(244pt)=(1,0.863106,0); rgb(245pt)=(1,0.866964,0); rgb(246pt)=(1,0.870829,0); rgb(247pt)=(1,0.8747,0); rgb(248pt)=(1,0.878577,0); rgb(249pt)=(1,0.88246,0); rgb(250pt)=(1,0.886349,0); rgb(251pt)=(1,0.890243,0); rgb(252pt)=(1,0.894144,0); rgb(253pt)=(1,0.898051,0); rgb(254pt)=(1,0.901964,0); rgb(255pt)=(1,0.905882,0)},
mesh/rows=49]
table[row sep=crcr,header=false] {%
%
56	0.093	26.7123045686334\\
56	0.09666	28.8008271328639\\
56	0.10032	31.1960133736808\\
56	0.10398	33.897863291084\\
56	0.10764	36.9063768850736\\
56	0.1113	40.2215541556496\\
56	0.11496	43.8433951028119\\
56	0.11862	47.7718997265605\\
56	0.12228	52.0070680268955\\
56	0.12594	56.5489000038169\\
56	0.1296	61.3973956573246\\
56	0.13326	66.5525549874186\\
56	0.13692	72.014377994099\\
56	0.14058	77.7828646773658\\
56	0.14424	83.8580150372189\\
56	0.1479	90.2398290736583\\
56	0.15156	96.9283067866842\\
56	0.15522	103.923448176296\\
56	0.15888	111.225253242495\\
56	0.16254	118.83372198528\\
56	0.1662	126.748854404651\\
56	0.16986	134.970650500609\\
56	0.17352	143.499110273152\\
56	0.17718	152.334233722283\\
56	0.18084	161.476020847999\\
56	0.1845	170.924471650302\\
56	0.18816	180.679586129192\\
56	0.19182	190.741364284667\\
56	0.19548	201.109806116729\\
56	0.19914	211.784911625378\\
56	0.2028	222.766680810613\\
56	0.20646	234.055113672434\\
56	0.21012	245.650210210841\\
56	0.21378	257.551970425835\\
56	0.21744	269.760394317415\\
56	0.2211	282.275481885581\\
56	0.22476	295.097233130334\\
56	0.22842	308.225648051674\\
56	0.23208	321.660726649599\\
56	0.23574	335.402468924111\\
56	0.2394	349.450874875209\\
56	0.24306	363.805944502894\\
56	0.24672	378.467677807165\\
56	0.25038	393.436074788022\\
56	0.25404	408.711135445466\\
56	0.2577	424.292859779496\\
56	0.26136	440.181247790112\\
56	0.26502	456.376299477315\\
56	0.26868	472.878014841104\\
56	0.27234	489.686393881479\\
56	0.276	506.801436598441\\
56.375	0.093	26.9534015004779\\
56.375	0.09666	29.0365780221604\\
56.375	0.10032	31.4264182204294\\
56.375	0.10398	34.1229220952847\\
56.375	0.10764	37.1260896467263\\
56.375	0.1113	40.4359208747543\\
56.375	0.11496	44.0524157793686\\
56.375	0.11862	47.9755743605693\\
56.375	0.12228	52.2053966183563\\
56.375	0.12594	56.7418825527297\\
56.375	0.1296	61.5850321636895\\
56.375	0.13326	66.7348454512355\\
56.375	0.13692	72.191322415368\\
56.375	0.14058	77.9544630560869\\
56.375	0.14424	84.024267373392\\
56.375	0.1479	90.4007353672835\\
56.375	0.15156	97.0838670377614\\
56.375	0.15522	104.073662384826\\
56.375	0.15888	111.370121408476\\
56.375	0.16254	118.973244108713\\
56.375	0.1662	126.883030485536\\
56.375	0.16986	135.099480538946\\
56.375	0.17352	143.622594268942\\
56.375	0.17718	152.452371675524\\
56.375	0.18084	161.588812758693\\
56.375	0.1845	171.031917518448\\
56.375	0.18816	180.781685954789\\
56.375	0.19182	190.838118067717\\
56.375	0.19548	201.201213857231\\
56.375	0.19914	211.870973323332\\
56.375	0.2028	222.847396466018\\
56.375	0.20646	234.130483285291\\
56.375	0.21012	245.720233781151\\
56.375	0.21378	257.616647953597\\
56.375	0.21744	269.819725802629\\
56.375	0.2211	282.329467328247\\
56.375	0.22476	295.145872530452\\
56.375	0.22842	308.268941409244\\
56.375	0.23208	321.698673964621\\
56.375	0.23574	335.435070196585\\
56.375	0.2394	349.478130105136\\
56.375	0.24306	363.827853690272\\
56.375	0.24672	378.484240951995\\
56.375	0.25038	393.447291890305\\
56.375	0.25404	408.7170065052\\
56.375	0.2577	424.293384796682\\
56.375	0.26136	440.176426764751\\
56.375	0.26502	456.366132409405\\
56.375	0.26868	472.862501730646\\
56.375	0.27234	489.665534728474\\
56.375	0.276	506.775231402888\\
56.75	0.093	27.2054559128872\\
56.75	0.09666	29.2832863920218\\
56.75	0.10032	31.6677805477428\\
56.75	0.10398	34.3589383800501\\
56.75	0.10764	37.3567598889438\\
56.75	0.1113	40.6612450744238\\
56.75	0.11496	44.2723939364902\\
56.75	0.11862	48.1902064751429\\
56.75	0.12228	52.414682690382\\
56.75	0.12594	56.9458225822074\\
56.75	0.1296	61.7836261506192\\
56.75	0.13326	66.9280933956174\\
56.75	0.13692	72.3792243172018\\
56.75	0.14058	78.1370189153727\\
56.75	0.14424	84.20147719013\\
56.75	0.1479	90.5725991414735\\
56.75	0.15156	97.2503847694034\\
56.75	0.15522	104.23483407392\\
56.75	0.15888	111.525947055022\\
56.75	0.16254	119.123723712711\\
56.75	0.1662	127.028164046987\\
56.75	0.16986	135.239268057848\\
56.75	0.17352	143.757035745296\\
56.75	0.17718	152.581467109331\\
56.75	0.18084	161.712562149951\\
56.75	0.1845	171.150320867158\\
56.75	0.18816	180.894743260952\\
56.75	0.19182	190.945829331331\\
56.75	0.19548	201.303579078298\\
56.75	0.19914	211.96799250185\\
56.75	0.2028	222.939069601989\\
56.75	0.20646	234.216810378714\\
56.75	0.21012	245.801214832026\\
56.75	0.21378	257.692282961923\\
56.75	0.21744	269.890014768408\\
56.75	0.2211	282.394410251478\\
56.75	0.22476	295.205469411135\\
56.75	0.22842	308.323192247379\\
56.75	0.23208	321.747578760208\\
56.75	0.23574	335.478628949624\\
56.75	0.2394	349.516342815627\\
56.75	0.24306	363.860720358215\\
56.75	0.24672	378.51176157739\\
56.75	0.25038	393.469466473152\\
56.75	0.25404	408.733835045499\\
56.75	0.2577	424.304867294434\\
56.75	0.26136	440.182563219954\\
56.75	0.26502	456.366922822061\\
56.75	0.26868	472.857946100754\\
56.75	0.27234	489.655633056033\\
56.75	0.276	506.759983687899\\
57.125	0.093	27.4684678058615\\
57.125	0.09666	29.5409522424481\\
57.125	0.10032	31.9201003556211\\
57.125	0.10398	34.6059121453805\\
57.125	0.10764	37.5983876117262\\
57.125	0.1113	40.8975267546583\\
57.125	0.11496	44.5033295741767\\
57.125	0.11862	48.4157960702815\\
57.125	0.12228	52.6349262429726\\
57.125	0.12594	57.1607200922501\\
57.125	0.1296	61.9931776181139\\
57.125	0.13326	67.1322988205641\\
57.125	0.13692	72.5780836996007\\
57.125	0.14058	78.3305322552235\\
57.125	0.14424	84.3896444874328\\
57.125	0.1479	90.7554203962284\\
57.125	0.15156	97.4278599816103\\
57.125	0.15522	104.406963243579\\
57.125	0.15888	111.692730182133\\
57.125	0.16254	119.285160797274\\
57.125	0.1662	127.184255089002\\
57.125	0.16986	135.390013057315\\
57.125	0.17352	143.902434702215\\
57.125	0.17718	152.721520023702\\
57.125	0.18084	161.847269021774\\
57.125	0.1845	171.279681696434\\
57.125	0.18816	181.018758047679\\
57.125	0.19182	191.064498075511\\
57.125	0.19548	201.416901779929\\
57.125	0.19914	212.075969160934\\
57.125	0.2028	223.041700218524\\
57.125	0.20646	234.314094952702\\
57.125	0.21012	245.893153363465\\
57.125	0.21378	257.778875450815\\
57.125	0.21744	269.971261214751\\
57.125	0.2211	282.470310655274\\
57.125	0.22476	295.276023772383\\
57.125	0.22842	308.388400566078\\
57.125	0.23208	321.80744103636\\
57.125	0.23574	335.533145183228\\
57.125	0.2394	349.565513006683\\
57.125	0.24306	363.904544506723\\
57.125	0.24672	378.55023968335\\
57.125	0.25038	393.502598536564\\
57.125	0.25404	408.761621066364\\
57.125	0.2577	424.32730727275\\
57.125	0.26136	440.199657155722\\
57.125	0.26502	456.378670715281\\
57.125	0.26868	472.864347951426\\
57.125	0.27234	489.656688864158\\
57.125	0.276	506.755693453476\\
57.5	0.093	27.7424371794006\\
57.5	0.09666	29.8095755734393\\
57.5	0.10032	32.1833776440644\\
57.5	0.10398	34.8638433912757\\
57.5	0.10764	37.8509728150735\\
57.5	0.1113	41.1447659154576\\
57.5	0.11496	44.7452226924281\\
57.5	0.11862	48.6523431459849\\
57.5	0.12228	52.8661272761281\\
57.5	0.12594	57.3865750828576\\
57.5	0.1296	62.2136865661734\\
57.5	0.13326	67.3474617260757\\
57.5	0.13692	72.7879005625643\\
57.5	0.14058	78.5350030756393\\
57.5	0.14424	84.5887692653005\\
57.5	0.1479	90.9491991315482\\
57.5	0.15156	97.6162926743822\\
57.5	0.15522	104.590049893802\\
57.5	0.15888	111.870470789809\\
57.5	0.16254	119.457555362402\\
57.5	0.1662	127.351303611582\\
57.5	0.16986	135.551715537347\\
57.5	0.17352	144.0587911397\\
57.5	0.17718	152.872530418638\\
57.5	0.18084	161.992933374163\\
57.5	0.1845	171.420000006274\\
57.5	0.18816	181.153730314971\\
57.5	0.19182	191.194124300255\\
57.5	0.19548	201.541181962125\\
57.5	0.19914	212.194903300582\\
57.5	0.2028	223.155288315625\\
57.5	0.20646	234.422337007254\\
57.5	0.21012	245.99604937547\\
57.5	0.21378	257.876425420272\\
57.5	0.21744	270.06346514166\\
57.5	0.2211	282.557168539635\\
57.5	0.22476	295.357535614196\\
57.5	0.22842	308.464566365343\\
57.5	0.23208	321.878260793077\\
57.5	0.23574	335.598618897397\\
57.5	0.2394	349.625640678303\\
57.5	0.24306	363.959326135796\\
57.5	0.24672	378.599675269875\\
57.5	0.25038	393.546688080541\\
57.5	0.25404	408.800364567793\\
57.5	0.2577	424.360704731631\\
57.5	0.26136	440.227708572055\\
57.5	0.26502	456.401376089066\\
57.5	0.26868	472.881707282663\\
57.5	0.27234	489.668702152847\\
57.5	0.276	506.762360699617\\
57.875	0.093	28.0273640335046\\
57.875	0.09666	30.0891563849953\\
57.875	0.10032	32.4576124130725\\
57.875	0.10398	35.1327321177359\\
57.875	0.10764	38.1145154989857\\
57.875	0.1113	41.4029625568218\\
57.875	0.11496	44.9980732912444\\
57.875	0.11862	48.8998477022533\\
57.875	0.12228	53.1082857898485\\
57.875	0.12594	57.62338755403\\
57.875	0.1296	62.445152994798\\
57.875	0.13326	67.5735821121522\\
57.875	0.13692	73.0086749060928\\
57.875	0.14058	78.7504313766198\\
57.875	0.14424	84.7988515237332\\
57.875	0.1479	91.1539353474328\\
57.875	0.15156	97.8156828477188\\
57.875	0.15522	104.784094024591\\
57.875	0.15888	112.05916887805\\
57.875	0.16254	119.640907408095\\
57.875	0.1662	127.529309614727\\
57.875	0.16986	135.724375497944\\
57.875	0.17352	144.226105057748\\
57.875	0.17718	153.034498294139\\
57.875	0.18084	162.149555207116\\
57.875	0.1845	171.571275796679\\
57.875	0.18816	181.299660062829\\
57.875	0.19182	191.334708005564\\
57.875	0.19548	201.676419624887\\
57.875	0.19914	212.324794920795\\
57.875	0.2028	223.27983389329\\
57.875	0.20646	234.541536542372\\
57.875	0.21012	246.109902868039\\
57.875	0.21378	257.984932870293\\
57.875	0.21744	270.166626549133\\
57.875	0.2211	282.65498390456\\
57.875	0.22476	295.450004936573\\
57.875	0.22842	308.551689645173\\
57.875	0.23208	321.960038030358\\
57.875	0.23574	335.675050092131\\
57.875	0.2394	349.696725830489\\
57.875	0.24306	364.025065245434\\
57.875	0.24672	378.660068336965\\
57.875	0.25038	393.601735105083\\
57.875	0.25404	408.850065549787\\
57.875	0.2577	424.405059671077\\
57.875	0.26136	440.266717468953\\
57.875	0.26502	456.435038943416\\
57.875	0.26868	472.910024094466\\
57.875	0.27234	489.691672922101\\
57.875	0.276	506.779985426323\\
58.25	0.093	28.3232483681735\\
58.25	0.09666	30.3796946771163\\
58.25	0.10032	32.7428046626454\\
58.25	0.10398	35.4125783247609\\
58.25	0.10764	38.3890156634628\\
58.25	0.1113	41.672116678751\\
58.25	0.11496	45.2618813706256\\
58.25	0.11862	49.1583097390864\\
58.25	0.12228	53.3614017841337\\
58.25	0.12594	57.8711575057673\\
58.25	0.1296	62.6875769039872\\
58.25	0.13326	67.8106599787936\\
58.25	0.13692	73.2404067301862\\
58.25	0.14058	78.9768171581653\\
58.25	0.14424	85.0198912627307\\
58.25	0.1479	91.3696290438824\\
58.25	0.15156	98.0260305016205\\
58.25	0.15522	104.989095635945\\
58.25	0.15888	112.258824446856\\
58.25	0.16254	119.835216934353\\
58.25	0.1662	127.718273098436\\
58.25	0.16986	135.907992939106\\
58.25	0.17352	144.404376456362\\
58.25	0.17718	153.207423650205\\
58.25	0.18084	162.317134520634\\
58.25	0.1845	171.733509067649\\
58.25	0.18816	181.456547291251\\
58.25	0.19182	191.486249191438\\
58.25	0.19548	201.822614768213\\
58.25	0.19914	212.465644021573\\
58.25	0.2028	223.41533695152\\
58.25	0.20646	234.671693558054\\
58.25	0.21012	246.234713841174\\
58.25	0.21378	258.104397800879\\
58.25	0.21744	270.280745437172\\
58.25	0.2211	282.763756750051\\
58.25	0.22476	295.553431739516\\
58.25	0.22842	308.649770405567\\
58.25	0.23208	322.052772748205\\
58.25	0.23574	335.762438767429\\
58.25	0.2394	349.77876846324\\
58.25	0.24306	364.101761835637\\
58.25	0.24672	378.73141888462\\
58.25	0.25038	393.66773961019\\
58.25	0.25404	408.910724012345\\
58.25	0.2577	424.460372091088\\
58.25	0.26136	440.316683846416\\
58.25	0.26502	456.479659278331\\
58.25	0.26868	472.949298386833\\
58.25	0.27234	489.72560117192\\
58.25	0.276	506.808567633594\\
58.625	0.093	28.6300901834073\\
58.625	0.09666	30.6811904498022\\
58.625	0.10032	33.0389543927833\\
58.625	0.10398	35.7033820123509\\
58.625	0.10764	38.6744733085048\\
58.625	0.1113	41.952228281245\\
58.625	0.11496	45.5366469305716\\
58.625	0.11862	49.4277292564846\\
58.625	0.12228	53.6254752589839\\
58.625	0.12594	58.1298849380695\\
58.625	0.1296	62.9409582937415\\
58.625	0.13326	68.0586953259999\\
58.625	0.13692	73.4830960348446\\
58.625	0.14058	79.2141604202757\\
58.625	0.14424	85.2518884822931\\
58.625	0.1479	91.5962802208968\\
58.625	0.15156	98.2473356360869\\
58.625	0.15522	105.205054727863\\
58.625	0.15888	112.469437496226\\
58.625	0.16254	120.040483941175\\
58.625	0.1662	127.918194062711\\
58.625	0.16986	136.102567860833\\
58.625	0.17352	144.593605335541\\
58.625	0.17718	153.391306486836\\
58.625	0.18084	162.495671314717\\
58.625	0.1845	171.906699819184\\
58.625	0.18816	181.624392000237\\
58.625	0.19182	191.648747857877\\
58.625	0.19548	201.979767392104\\
58.625	0.19914	212.617450602917\\
58.625	0.2028	223.561797490316\\
58.625	0.20646	234.812808054301\\
58.625	0.21012	246.370482294873\\
58.625	0.21378	258.234820212031\\
58.625	0.21744	270.405821805775\\
58.625	0.2211	282.883487076106\\
58.625	0.22476	295.667816023023\\
58.625	0.22842	308.758808646527\\
58.625	0.23208	322.156464946617\\
58.625	0.23574	335.860784923293\\
58.625	0.2394	349.871768576555\\
58.625	0.24306	364.189415906404\\
58.625	0.24672	378.813726912839\\
58.625	0.25038	393.744701595861\\
58.625	0.25404	408.982339955469\\
58.625	0.2577	424.526641991664\\
58.625	0.26136	440.377607704444\\
58.625	0.26502	456.535237093811\\
58.625	0.26868	472.999530159764\\
58.625	0.27234	489.770486902304\\
58.625	0.276	506.84810732143\\
59	0.093	28.947889479206\\
59	0.09666	30.9936437030529\\
59	0.10032	33.3460616034861\\
59	0.10398	36.0051431805057\\
59	0.10764	38.9708884341117\\
59	0.1113	42.2432973643039\\
59	0.11496	45.8223699710826\\
59	0.11862	49.7081062544476\\
59	0.12228	53.9005062143989\\
59	0.12594	58.3995698509366\\
59	0.1296	63.2052971640606\\
59	0.13326	68.3176881537711\\
59	0.13692	73.7367428200678\\
59	0.14058	79.462461162951\\
59	0.14424	85.4948431824204\\
59	0.1479	91.8338888784762\\
59	0.15156	98.4795982511184\\
59	0.15522	105.431971300347\\
59	0.15888	112.691008026162\\
59	0.16254	120.256708428563\\
59	0.1662	128.129072507551\\
59	0.16986	136.308100263124\\
59	0.17352	144.793791695285\\
59	0.17718	153.586146804031\\
59	0.18084	162.685165589364\\
59	0.1845	172.090848051284\\
59	0.18816	181.803194189789\\
59	0.19182	191.822204004881\\
59	0.19548	202.14787749656\\
59	0.19914	212.780214664824\\
59	0.2028	223.719215509676\\
59	0.20646	234.964880031113\\
59	0.21012	246.517208229137\\
59	0.21378	258.376200103747\\
59	0.21744	270.541855654943\\
59	0.2211	283.014174882726\\
59	0.22476	295.793157787095\\
59	0.22842	308.878804368051\\
59	0.23208	322.271114625593\\
59	0.23574	335.970088559721\\
59	0.2394	349.975726170436\\
59	0.24306	364.288027457737\\
59	0.24672	378.906992421624\\
59	0.25038	393.832621062098\\
59	0.25404	409.064913379158\\
59	0.2577	424.603869372804\\
59	0.26136	440.449489043037\\
59	0.26502	456.601772389856\\
59	0.26868	473.060719413261\\
59	0.27234	489.826330113253\\
59	0.276	506.898604489831\\
59.375	0.093	29.2766462555696\\
59.375	0.09666	31.3170544368685\\
59.375	0.10032	33.6641262947538\\
59.375	0.10398	36.3178618292254\\
59.375	0.10764	39.2782610402834\\
59.375	0.1113	42.5453239279277\\
59.375	0.11496	46.1190504921584\\
59.375	0.11862	49.9994407329754\\
59.375	0.12228	54.1864946503788\\
59.375	0.12594	58.6802122443686\\
59.375	0.1296	63.4805935149446\\
59.375	0.13326	68.5876384621071\\
59.375	0.13692	74.0013470858559\\
59.375	0.14058	79.7217193861911\\
59.375	0.14424	85.7487553631126\\
59.375	0.1479	92.0824550166204\\
59.375	0.15156	98.7228183467146\\
59.375	0.15522	105.669845353395\\
59.375	0.15888	112.923536036662\\
59.375	0.16254	120.483890396515\\
59.375	0.1662	128.350908432955\\
59.375	0.16986	136.524590145981\\
59.375	0.17352	145.004935535593\\
59.375	0.17718	153.791944601792\\
59.375	0.18084	162.885617344577\\
59.375	0.1845	172.285953763948\\
59.375	0.18816	181.992953859906\\
59.375	0.19182	192.00661763245\\
59.375	0.19548	202.326945081581\\
59.375	0.19914	212.953936207297\\
59.375	0.2028	223.887591009601\\
59.375	0.20646	235.12790948849\\
59.375	0.21012	246.674891643966\\
59.375	0.21378	258.528537476028\\
59.375	0.21744	270.688846984677\\
59.375	0.2211	283.155820169911\\
59.375	0.22476	295.929457031733\\
59.375	0.22842	309.00975757014\\
59.375	0.23208	322.396721785134\\
59.375	0.23574	336.090349676715\\
59.375	0.2394	350.090641244881\\
59.375	0.24306	364.397596489634\\
59.375	0.24672	379.011215410973\\
59.375	0.25038	393.931498008899\\
59.375	0.25404	409.158444283411\\
59.375	0.2577	424.69205423451\\
59.375	0.26136	440.532327862195\\
59.375	0.26502	456.679265166465\\
59.375	0.26868	473.132866147323\\
59.375	0.27234	489.893130804767\\
59.375	0.276	506.960059138797\\
59.75	0.093	29.6163605124981\\
59.75	0.09666	31.651422651249\\
59.75	0.10032	33.9931484665864\\
59.75	0.10398	36.6415379585101\\
59.75	0.10764	39.5965911270201\\
59.75	0.1113	42.8583079721164\\
59.75	0.11496	46.4266884937992\\
59.75	0.11862	50.3017326920683\\
59.75	0.12228	54.4834405669237\\
59.75	0.12594	58.9718121183655\\
59.75	0.1296	63.7668473463936\\
59.75	0.13326	68.8685462510081\\
59.75	0.13692	74.2769088322089\\
59.75	0.14058	79.9919350899962\\
59.75	0.14424	86.0136250243697\\
59.75	0.1479	92.3419786353296\\
59.75	0.15156	98.9769959228759\\
59.75	0.15522	105.918676887008\\
59.75	0.15888	113.167021527728\\
59.75	0.16254	120.722029845033\\
59.75	0.1662	128.583701838924\\
59.75	0.16986	136.752037509402\\
59.75	0.17352	145.227036856467\\
59.75	0.17718	154.008699880117\\
59.75	0.18084	163.097026580354\\
59.75	0.1845	172.492016957178\\
59.75	0.18816	182.193671010588\\
59.75	0.19182	192.201988740584\\
59.75	0.19548	202.516970147166\\
59.75	0.19914	213.138615230335\\
59.75	0.2028	224.06692399009\\
59.75	0.20646	235.301896426432\\
59.75	0.21012	246.84353253936\\
59.75	0.21378	258.691832328874\\
59.75	0.21744	270.846795794974\\
59.75	0.2211	283.308422937661\\
59.75	0.22476	296.076713756935\\
59.75	0.22842	309.151668252794\\
59.75	0.23208	322.53328642524\\
59.75	0.23574	336.221568274273\\
59.75	0.2394	350.216513799891\\
59.75	0.24306	364.518123002096\\
59.75	0.24672	379.126395880888\\
59.75	0.25038	394.041332436266\\
59.75	0.25404	409.26293266823\\
59.75	0.2577	424.79119657678\\
59.75	0.26136	440.626124161917\\
59.75	0.26502	456.76771542364\\
59.75	0.26868	473.215970361949\\
59.75	0.27234	489.970888976845\\
59.75	0.276	507.032471268328\\
60.125	0.093	29.9670322499914\\
60.125	0.09666	31.9967483461945\\
60.125	0.10032	34.3331281189838\\
60.125	0.10398	36.9761715683595\\
60.125	0.10764	39.9258786943216\\
60.125	0.1113	43.18224949687\\
60.125	0.11496	46.7452839760048\\
60.125	0.11862	50.6149821317259\\
60.125	0.12228	54.7913439640334\\
60.125	0.12594	59.2743694729272\\
60.125	0.1296	64.0640586584074\\
60.125	0.13326	69.1604115204739\\
60.125	0.13692	74.5634280591268\\
60.125	0.14058	80.2731082743661\\
60.125	0.14424	86.2894521661917\\
60.125	0.1479	92.6124597346036\\
60.125	0.15156	99.2421309796019\\
60.125	0.15522	106.178465901187\\
60.125	0.15888	113.421464499358\\
60.125	0.16254	120.971126774115\\
60.125	0.1662	128.827452725459\\
60.125	0.16986	136.990442353389\\
60.125	0.17352	145.460095657905\\
60.125	0.17718	154.236412639008\\
60.125	0.18084	163.319393296697\\
60.125	0.1845	172.709037630972\\
60.125	0.18816	182.405345641834\\
60.125	0.19182	192.408317329282\\
60.125	0.19548	202.717952693317\\
60.125	0.19914	213.334251733938\\
60.125	0.2028	224.257214451145\\
60.125	0.20646	235.486840844939\\
60.125	0.21012	247.023130915318\\
60.125	0.21378	258.866084662285\\
60.125	0.21744	271.015702085837\\
60.125	0.2211	283.471983185976\\
60.125	0.22476	296.234927962702\\
60.125	0.22842	309.304536416013\\
60.125	0.23208	322.680808545911\\
60.125	0.23574	336.363744352396\\
60.125	0.2394	350.353343835466\\
60.125	0.24306	364.649606995124\\
60.125	0.24672	379.252533831367\\
60.125	0.25038	394.162124344197\\
60.125	0.25404	409.378378533613\\
60.125	0.2577	424.901296399616\\
60.125	0.26136	440.730877942204\\
60.125	0.26502	456.867123161379\\
60.125	0.26868	473.310032057141\\
60.125	0.27234	490.059604629489\\
60.125	0.276	507.115840878423\\
60.5	0.093	30.3286614680497\\
60.5	0.09666	32.3530315217048\\
60.5	0.10032	34.6840652519462\\
60.5	0.10398	37.3217626587739\\
60.5	0.10764	40.2661237421881\\
60.5	0.1113	43.5171485021885\\
60.5	0.11496	47.0748369387753\\
60.5	0.11862	50.9391890519485\\
60.5	0.12228	55.1102048417081\\
60.5	0.12594	59.5878843080539\\
60.5	0.1296	64.3722274509861\\
60.5	0.13326	69.4632342705046\\
60.5	0.13692	74.8609047666096\\
60.5	0.14058	80.5652389393009\\
60.5	0.14424	86.5762367885786\\
60.5	0.1479	92.8938983144425\\
60.5	0.15156	99.5182235168928\\
60.5	0.15522	106.44921239593\\
60.5	0.15888	113.686864951553\\
60.5	0.16254	121.231181183762\\
60.5	0.1662	129.082161092558\\
60.5	0.16986	137.23980467794\\
60.5	0.17352	145.704111939908\\
60.5	0.17718	154.475082878463\\
60.5	0.18084	163.552717493604\\
60.5	0.1845	172.937015785332\\
60.5	0.18816	182.627977753646\\
60.5	0.19182	192.625603398546\\
60.5	0.19548	202.929892720032\\
60.5	0.19914	213.540845718105\\
60.5	0.2028	224.458462392765\\
60.5	0.20646	235.68274274401\\
60.5	0.21012	247.213686771842\\
60.5	0.21378	259.05129447626\\
60.5	0.21744	271.195565857265\\
60.5	0.2211	283.646500914856\\
60.5	0.22476	296.404099649033\\
60.5	0.22842	309.468362059797\\
60.5	0.23208	322.839288147147\\
60.5	0.23574	336.516877911084\\
60.5	0.2394	350.501131351606\\
60.5	0.24306	364.792048468716\\
60.5	0.24672	379.389629262411\\
60.5	0.25038	394.293873732693\\
60.5	0.25404	409.504781879561\\
60.5	0.2577	425.022353703016\\
60.5	0.26136	440.846589203057\\
60.5	0.26502	456.977488379684\\
60.5	0.26868	473.415051232897\\
60.5	0.27234	490.159277762697\\
60.5	0.276	507.210167969083\\
60.875	0.093	30.7012481666729\\
60.875	0.09666	32.72027217778\\
60.875	0.10032	35.0459598654734\\
60.875	0.10398	37.6783112297532\\
60.875	0.10764	40.6173262706194\\
60.875	0.1113	43.8630049880719\\
60.875	0.11496	47.4153473821108\\
60.875	0.11862	51.274353452736\\
60.875	0.12228	55.4400231999475\\
60.875	0.12594	59.9123566237454\\
60.875	0.1296	64.6913537241297\\
60.875	0.13326	69.7770145011003\\
60.875	0.13692	75.1693389546573\\
60.875	0.14058	80.8683270848006\\
60.875	0.14424	86.8739788915304\\
60.875	0.1479	93.1862943748463\\
60.875	0.15156	99.8052735347487\\
60.875	0.15522	106.730916371237\\
60.875	0.15888	113.963222884313\\
60.875	0.16254	121.502193073974\\
60.875	0.1662	129.347826940222\\
60.875	0.16986	137.500124483056\\
60.875	0.17352	145.959085702476\\
60.875	0.17718	154.724710598483\\
60.875	0.18084	163.796999171076\\
60.875	0.1845	173.175951420256\\
60.875	0.18816	182.861567346022\\
60.875	0.19182	192.853846948374\\
60.875	0.19548	203.152790227313\\
60.875	0.19914	213.758397182838\\
60.875	0.2028	224.670667814949\\
60.875	0.20646	235.889602123647\\
60.875	0.21012	247.415200108931\\
60.875	0.21378	259.247461770801\\
60.875	0.21744	271.386387109258\\
60.875	0.2211	283.831976124301\\
60.875	0.22476	296.58422881593\\
60.875	0.22842	309.643145184146\\
60.875	0.23208	323.008725228948\\
60.875	0.23574	336.680968950337\\
60.875	0.2394	350.659876348311\\
60.875	0.24306	364.945447422872\\
60.875	0.24672	379.53768217402\\
60.875	0.25038	394.436580601754\\
60.875	0.25404	409.642142706074\\
60.875	0.2577	425.154368486981\\
60.875	0.26136	440.973257944474\\
60.875	0.26502	457.098811078553\\
60.875	0.26868	473.531027889219\\
60.875	0.27234	490.269908376471\\
60.875	0.276	507.315452540309\\
61.25	0.093	31.0847923458609\\
61.25	0.09666	33.0984703144201\\
61.25	0.10032	35.4188119595656\\
61.25	0.10398	38.0458172812974\\
61.25	0.10764	40.9794862796156\\
61.25	0.1113	44.2198189545201\\
61.25	0.11496	47.7668153060111\\
61.25	0.11862	51.6204753340883\\
61.25	0.12228	55.7807990387519\\
61.25	0.12594	60.2477864200019\\
61.25	0.1296	65.0214374778382\\
61.25	0.13326	70.1017522122608\\
61.25	0.13692	75.4887306232698\\
61.25	0.14058	81.1823727108653\\
61.25	0.14424	87.182678475047\\
61.25	0.1479	93.489647915815\\
61.25	0.15156	100.103281033169\\
61.25	0.15522	107.02357782711\\
61.25	0.15888	114.250538297637\\
61.25	0.16254	121.784162444751\\
61.25	0.1662	129.624450268451\\
61.25	0.16986	137.771401768737\\
61.25	0.17352	146.225016945609\\
61.25	0.17718	154.985295799068\\
61.25	0.18084	164.052238329113\\
61.25	0.1845	173.425844535745\\
61.25	0.18816	183.106114418963\\
61.25	0.19182	193.093047978767\\
61.25	0.19548	203.386645215158\\
61.25	0.19914	213.986906128135\\
61.25	0.2028	224.893830717698\\
61.25	0.20646	236.107418983848\\
61.25	0.21012	247.627670926584\\
61.25	0.21378	259.454586545906\\
61.25	0.21744	271.588165841815\\
61.25	0.2211	284.02840881431\\
61.25	0.22476	296.775315463392\\
61.25	0.22842	309.82888578906\\
61.25	0.23208	323.189119791314\\
61.25	0.23574	336.856017470154\\
61.25	0.2394	350.829578825581\\
61.25	0.24306	365.109803857594\\
61.25	0.24672	379.696692566194\\
61.25	0.25038	394.59024495138\\
61.25	0.25404	409.790461013152\\
61.25	0.2577	425.297340751511\\
61.25	0.26136	441.110884166456\\
61.25	0.26502	457.231091257987\\
61.25	0.26868	473.657962026105\\
61.25	0.27234	490.391496470809\\
61.25	0.276	507.431694592099\\
61.625	0.093	31.4792940056139\\
61.625	0.09666	33.487625931625\\
61.625	0.10032	35.8026215342226\\
61.625	0.10398	38.4242808134065\\
61.625	0.10764	41.3526037691767\\
61.625	0.1113	44.5875904015333\\
61.625	0.11496	48.1292407104763\\
61.625	0.11862	51.9775546960055\\
61.625	0.12228	56.1325323581212\\
61.625	0.12594	60.5941736968232\\
61.625	0.1296	65.3624787121115\\
61.625	0.13326	70.4374474039862\\
61.625	0.13692	75.8190797724473\\
61.625	0.14058	81.5073758174948\\
61.625	0.14424	87.5023355391285\\
61.625	0.1479	93.8039589373486\\
61.625	0.15156	100.412246012155\\
61.625	0.15522	107.327196763548\\
61.625	0.15888	114.548811191527\\
61.625	0.16254	122.077089296093\\
61.625	0.1662	129.912031077245\\
61.625	0.16986	138.053636534983\\
61.625	0.17352	146.501905669307\\
61.625	0.17718	155.256838480218\\
61.625	0.18084	164.318434967715\\
61.625	0.1845	173.686695131799\\
61.625	0.18816	183.361618972469\\
61.625	0.19182	193.343206489725\\
61.625	0.19548	203.631457683568\\
61.625	0.19914	214.226372553997\\
61.625	0.2028	225.127951101013\\
61.625	0.20646	236.336193324614\\
61.625	0.21012	247.851099224802\\
61.625	0.21378	259.672668801577\\
61.625	0.21744	271.800902054938\\
61.625	0.2211	284.235798984885\\
61.625	0.22476	296.977359591418\\
61.625	0.22842	310.025583874538\\
61.625	0.23208	323.380471834244\\
61.625	0.23574	337.042023470537\\
61.625	0.2394	351.010238783416\\
61.625	0.24306	365.285117772881\\
61.625	0.24672	379.866660438933\\
61.625	0.25038	394.754866781571\\
61.625	0.25404	409.949736800795\\
61.625	0.2577	425.451270496606\\
61.625	0.26136	441.259467869003\\
61.625	0.26502	457.374328917986\\
61.625	0.26868	473.795853643556\\
61.625	0.27234	490.524042045712\\
61.625	0.276	507.558894124454\\
62	0.093	31.8847531459317\\
62	0.09666	33.8877390293949\\
62	0.10032	36.1973885894445\\
62	0.10398	38.8137018260804\\
62	0.10764	41.7366787393027\\
62	0.1113	44.9663193291113\\
62	0.11496	48.5026235955064\\
62	0.11862	52.3455915384877\\
62	0.12228	56.4952231580554\\
62	0.12594	60.9515184542095\\
62	0.1296	65.7144774269498\\
62	0.13326	70.7841000762766\\
62	0.13692	76.1603864021897\\
62	0.14058	81.8433364046892\\
62	0.14424	87.832950083775\\
62	0.1479	94.1292274394471\\
62	0.15156	100.732168471706\\
62	0.15522	107.641773180551\\
62	0.15888	114.858041565982\\
62	0.16254	122.380973627999\\
62	0.1662	130.210569366603\\
62	0.16986	138.346828781793\\
62	0.17352	146.78975187357\\
62	0.17718	155.539338641933\\
62	0.18084	164.595589086882\\
62	0.1845	173.958503208418\\
62	0.18816	183.62808100654\\
62	0.19182	193.604322481248\\
62	0.19548	203.887227632543\\
62	0.19914	214.476796460424\\
62	0.2028	225.373028964892\\
62	0.20646	236.575925145946\\
62	0.21012	248.085485003586\\
62	0.21378	259.901708537812\\
62	0.21744	272.024595748625\\
62	0.2211	284.454146636024\\
62	0.22476	297.19036120001\\
62	0.22842	310.233239440582\\
62	0.23208	323.58278135774\\
62	0.23574	337.238986951485\\
62	0.2394	351.201856221815\\
62	0.24306	365.471389168733\\
62	0.24672	380.047585792236\\
62	0.25038	394.930446092326\\
62	0.25404	410.119970069003\\
62	0.2577	425.616157722266\\
62	0.26136	441.419009052115\\
62	0.26502	457.52852405855\\
62	0.26868	473.944702741572\\
62	0.27234	490.66754510118\\
62	0.276	507.697051137374\\
62.375	0.093	32.3011697668144\\
62.375	0.09666	34.2988096077297\\
62.375	0.10032	36.6031131252313\\
62.375	0.10398	39.2140803193193\\
62.375	0.10764	42.1317111899936\\
62.375	0.1113	45.3560057372543\\
62.375	0.11496	48.8869639611013\\
62.375	0.11862	52.7245858615347\\
62.375	0.12228	56.8688714385544\\
62.375	0.12594	61.3198206921606\\
62.375	0.1296	66.077433622353\\
62.375	0.13326	71.1417102291317\\
62.375	0.13692	76.5126505124969\\
62.375	0.14058	82.1902544724485\\
62.375	0.14424	88.1745221089863\\
62.375	0.1479	94.4654534221105\\
62.375	0.15156	101.063048411821\\
62.375	0.15522	107.967307078118\\
62.375	0.15888	115.178229421001\\
62.375	0.16254	122.695815440471\\
62.375	0.1662	130.520065136527\\
62.375	0.16986	138.650978509169\\
62.375	0.17352	147.088555558398\\
62.375	0.17718	155.832796284213\\
62.375	0.18084	164.883700686614\\
62.375	0.1845	174.241268765602\\
62.375	0.18816	183.905500521176\\
62.375	0.19182	193.876395953336\\
62.375	0.19548	204.153955062083\\
62.375	0.19914	214.738177847416\\
62.375	0.2028	225.629064309336\\
62.375	0.20646	236.826614447842\\
62.375	0.21012	248.330828262934\\
62.375	0.21378	260.141705754612\\
62.375	0.21744	272.259246922877\\
62.375	0.2211	284.683451767728\\
62.375	0.22476	297.414320289166\\
62.375	0.22842	310.45185248719\\
62.375	0.23208	323.7960483618\\
62.375	0.23574	337.446907912997\\
62.375	0.2394	351.40443114078\\
62.375	0.24306	365.668618045149\\
62.375	0.24672	380.239468626105\\
62.375	0.25038	395.116982883647\\
62.375	0.25404	410.301160817775\\
62.375	0.2577	425.79200242849\\
62.375	0.26136	441.589507715791\\
62.375	0.26502	457.693676679679\\
62.375	0.26868	474.104509320152\\
62.375	0.27234	490.822005637213\\
62.375	0.276	507.846165630859\\
62.75	0.093	32.728543868262\\
62.75	0.09666	34.7208376666293\\
62.75	0.10032	37.019795141583\\
62.75	0.10398	39.625416293123\\
62.75	0.10764	42.5377011212494\\
62.75	0.1113	45.7566496259621\\
62.75	0.11496	49.2822618072612\\
62.75	0.11862	53.1145376651466\\
62.75	0.12228	57.2534771996184\\
62.75	0.12594	61.6990804106765\\
62.75	0.1296	66.451347298321\\
62.75	0.13326	71.5102778625518\\
62.75	0.13692	76.875872103369\\
62.75	0.14058	82.5481300207726\\
62.75	0.14424	88.5270516147625\\
62.75	0.1479	94.8126368853387\\
62.75	0.15156	101.404885832501\\
62.75	0.15522	108.30379845625\\
62.75	0.15888	115.509374756586\\
62.75	0.16254	123.021614733507\\
62.75	0.1662	130.840518387015\\
62.75	0.16986	138.96608571711\\
62.75	0.17352	147.39831672379\\
62.75	0.17718	156.137211407057\\
62.75	0.18084	165.182769766911\\
62.75	0.1845	174.534991803351\\
62.75	0.18816	184.193877516377\\
62.75	0.19182	194.159426905989\\
62.75	0.19548	204.431639972188\\
62.75	0.19914	215.010516714973\\
62.75	0.2028	225.896057134345\\
62.75	0.20646	237.088261230303\\
62.75	0.21012	248.587129002847\\
62.75	0.21378	260.392660451977\\
62.75	0.21744	272.504855577694\\
62.75	0.2211	284.923714379997\\
62.75	0.22476	297.649236858887\\
62.75	0.22842	310.681423014363\\
62.75	0.23208	324.020272846425\\
62.75	0.23574	337.665786355074\\
62.75	0.2394	351.617963540309\\
62.75	0.24306	365.876804402131\\
62.75	0.24672	380.442308940538\\
62.75	0.25038	395.314477155533\\
62.75	0.25404	410.493309047113\\
62.75	0.2577	425.97880461528\\
62.75	0.26136	441.770963860033\\
62.75	0.26502	457.869786781372\\
62.75	0.26868	474.275273379298\\
62.75	0.27234	490.98742365381\\
62.75	0.276	508.006237604909\\
63.125	0.093	33.1668754502745\\
63.125	0.09666	35.1538232060939\\
63.125	0.10032	37.4474346384996\\
63.125	0.10398	40.0477097474917\\
63.125	0.10764	42.9546485330701\\
63.125	0.1113	46.1682509952349\\
63.125	0.11496	49.688517133986\\
63.125	0.11862	53.5154469493235\\
63.125	0.12228	57.6490404412473\\
63.125	0.12594	62.0892976097575\\
63.125	0.1296	66.836218454854\\
63.125	0.13326	71.8898029765369\\
63.125	0.13692	77.2500511748061\\
63.125	0.14058	82.9169630496617\\
63.125	0.14424	88.8905386011037\\
63.125	0.1479	95.1707778291319\\
63.125	0.15156	101.757680733747\\
63.125	0.15522	108.651247314948\\
63.125	0.15888	115.851477572735\\
63.125	0.16254	123.358371507109\\
63.125	0.1662	131.171929118069\\
63.125	0.16986	139.292150405615\\
63.125	0.17352	147.719035369748\\
63.125	0.17718	156.452584010467\\
63.125	0.18084	165.492796327772\\
63.125	0.1845	174.839672321664\\
63.125	0.18816	184.493211992142\\
63.125	0.19182	194.453415339207\\
63.125	0.19548	204.720282362858\\
63.125	0.19914	215.293813063095\\
63.125	0.2028	226.174007439918\\
63.125	0.20646	237.360865493328\\
63.125	0.21012	248.854387223325\\
63.125	0.21378	260.654572629907\\
63.125	0.21744	272.761421713076\\
63.125	0.2211	285.174934472831\\
63.125	0.22476	297.895110909173\\
63.125	0.22842	310.921951022101\\
63.125	0.23208	324.255454811616\\
63.125	0.23574	337.895622277716\\
63.125	0.2394	351.842453420403\\
63.125	0.24306	366.095948239677\\
63.125	0.24672	380.656106735537\\
63.125	0.25038	395.522928907983\\
63.125	0.25404	410.696414757015\\
63.125	0.2577	426.176564282634\\
63.125	0.26136	441.963377484839\\
63.125	0.26502	458.056854363631\\
63.125	0.26868	474.456994919009\\
63.125	0.27234	491.163799150973\\
63.125	0.276	508.177267059524\\
63.5	0.093	33.616164512852\\
63.5	0.09666	35.5977662261234\\
63.5	0.10032	37.8860316159811\\
63.5	0.10398	40.4809606824252\\
63.5	0.10764	43.3825534254557\\
63.5	0.1113	46.5908098450725\\
63.5	0.11496	50.1057299412757\\
63.5	0.11862	53.9273137140651\\
63.5	0.12228	58.0555611634411\\
63.5	0.12594	62.4904722894033\\
63.5	0.1296	67.2320470919518\\
63.5	0.13326	72.2802855710867\\
63.5	0.13692	77.635187726808\\
63.5	0.14058	83.2967535591157\\
63.5	0.14424	89.2649830680097\\
63.5	0.1479	95.53987625349\\
63.5	0.15156	102.121433115557\\
63.5	0.15522	109.00965365421\\
63.5	0.15888	116.204537869449\\
63.5	0.16254	123.706085761275\\
63.5	0.1662	131.514297329687\\
63.5	0.16986	139.629172574685\\
63.5	0.17352	148.05071149627\\
63.5	0.17718	156.778914094441\\
63.5	0.18084	165.813780369199\\
63.5	0.1845	175.155310320543\\
63.5	0.18816	184.803503948473\\
63.5	0.19182	194.758361252989\\
63.5	0.19548	205.019882234092\\
63.5	0.19914	215.588066891782\\
63.5	0.2028	226.462915226057\\
63.5	0.20646	237.644427236919\\
63.5	0.21012	249.132602924367\\
63.5	0.21378	260.927442288402\\
63.5	0.21744	273.028945329023\\
63.5	0.2211	285.43711204623\\
63.5	0.22476	298.151942440024\\
63.5	0.22842	311.173436510404\\
63.5	0.23208	324.501594257371\\
63.5	0.23574	338.136415680924\\
63.5	0.2394	352.077900781063\\
63.5	0.24306	366.326049557788\\
63.5	0.24672	380.8808620111\\
63.5	0.25038	395.742338140998\\
63.5	0.25404	410.910477947483\\
63.5	0.2577	426.385281430554\\
63.5	0.26136	442.166748590211\\
63.5	0.26502	458.254879426454\\
63.5	0.26868	474.649673939284\\
63.5	0.27234	491.351132128701\\
63.5	0.276	508.359253994703\\
63.875	0.093	34.0764110559942\\
63.875	0.09666	36.0526667267177\\
63.875	0.10032	38.3355860740275\\
63.875	0.10398	40.9251690979236\\
63.875	0.10764	43.8214157984061\\
63.875	0.1113	47.024326175475\\
63.875	0.11496	50.5339002291302\\
63.875	0.11862	54.3501379593718\\
63.875	0.12228	58.4730393661997\\
63.875	0.12594	62.9026044496139\\
63.875	0.1296	67.6388332096145\\
63.875	0.13326	72.6817256462014\\
63.875	0.13692	78.0312817593748\\
63.875	0.14058	83.6875015491345\\
63.875	0.14424	89.6503850154806\\
63.875	0.1479	95.9199321584129\\
63.875	0.15156	102.496142977932\\
63.875	0.15522	109.379017474037\\
63.875	0.15888	116.568555646728\\
63.875	0.16254	124.064757496006\\
63.875	0.1662	131.86762302187\\
63.875	0.16986	139.977152224321\\
63.875	0.17352	148.393345103357\\
63.875	0.17718	157.116201658981\\
63.875	0.18084	166.14572189119\\
63.875	0.1845	175.481905799986\\
63.875	0.18816	185.124753385368\\
63.875	0.19182	195.074264647337\\
63.875	0.19548	205.330439585892\\
63.875	0.19914	215.893278201033\\
63.875	0.2028	226.762780492761\\
63.875	0.20646	237.938946461075\\
63.875	0.21012	249.421776105975\\
63.875	0.21378	261.211269427462\\
63.875	0.21744	273.307426425535\\
63.875	0.2211	285.710247100194\\
63.875	0.22476	298.41973145144\\
63.875	0.22842	311.435879479272\\
63.875	0.23208	324.758691183691\\
63.875	0.23574	338.388166564696\\
63.875	0.2394	352.324305622287\\
63.875	0.24306	366.567108356464\\
63.875	0.24672	381.116574767228\\
63.875	0.25038	395.972704854578\\
63.875	0.25404	411.135498618515\\
63.875	0.2577	426.604956059038\\
63.875	0.26136	442.381077176147\\
63.875	0.26502	458.463861969843\\
63.875	0.26868	474.853310440125\\
63.875	0.27234	491.549422586993\\
63.875	0.276	508.552198410448\\
64.25	0.093	34.5476150797014\\
64.25	0.09666	36.5185247078769\\
64.25	0.10032	38.7960980126388\\
64.25	0.10398	41.3803349939869\\
64.25	0.10764	44.2712356519215\\
64.25	0.1113	47.4687999864424\\
64.25	0.11496	50.9730279975497\\
64.25	0.11862	54.7839196852433\\
64.25	0.12228	58.9014750495232\\
64.25	0.12594	63.3256940903896\\
64.25	0.1296	68.0565768078422\\
64.25	0.13326	73.0941232018812\\
64.25	0.13692	78.4383332725066\\
64.25	0.14058	84.0892070197183\\
64.25	0.14424	90.0467444435164\\
64.25	0.1479	96.3109455439007\\
64.25	0.15156	102.881810320872\\
64.25	0.15522	109.759338774429\\
64.25	0.15888	116.943530904572\\
64.25	0.16254	124.434386711302\\
64.25	0.1662	132.231906194618\\
64.25	0.16986	140.336089354521\\
64.25	0.17352	148.74693619101\\
64.25	0.17718	157.464446704085\\
64.25	0.18084	166.488620893746\\
64.25	0.1845	175.819458759994\\
64.25	0.18816	185.456960302829\\
64.25	0.19182	195.401125522249\\
64.25	0.19548	205.651954418256\\
64.25	0.19914	216.20944699085\\
64.25	0.2028	227.073603240029\\
64.25	0.20646	238.244423165795\\
64.25	0.21012	249.721906768148\\
64.25	0.21378	261.506054047086\\
64.25	0.21744	273.596865002612\\
64.25	0.2211	285.994339634723\\
64.25	0.22476	298.698477943421\\
64.25	0.22842	311.709279928705\\
64.25	0.23208	325.026745590576\\
64.25	0.23574	338.650874929033\\
64.25	0.2394	352.581667944076\\
64.25	0.24306	366.819124635705\\
64.25	0.24672	381.363245003921\\
64.25	0.25038	396.214029048723\\
64.25	0.25404	411.371476770112\\
64.25	0.2577	426.835588168087\\
64.25	0.26136	442.606363242648\\
64.25	0.26502	458.683801993796\\
64.25	0.26868	475.06790442153\\
64.25	0.27234	491.75867052585\\
64.25	0.276	508.756100306757\\
64.625	0.093	35.0297765839735\\
64.625	0.09666	36.995340169601\\
64.625	0.10032	39.2675674318149\\
64.625	0.10398	41.8464583706151\\
64.625	0.10764	44.7320129860017\\
64.625	0.1113	47.9242312779747\\
64.625	0.11496	51.423113246534\\
64.625	0.11862	55.2286588916796\\
64.625	0.12228	59.3408682134116\\
64.625	0.12594	63.75974121173\\
64.625	0.1296	68.4852778866346\\
64.625	0.13326	73.5174782381257\\
64.625	0.13692	78.8563422662031\\
64.625	0.14058	84.501869970867\\
64.625	0.14424	90.454061352117\\
64.625	0.1479	96.7129164099535\\
64.625	0.15156	103.278435144376\\
64.625	0.15522	110.150617555386\\
64.625	0.15888	117.329463642981\\
64.625	0.16254	124.814973407163\\
64.625	0.1662	132.607146847931\\
64.625	0.16986	140.705983965286\\
64.625	0.17352	149.111484759227\\
64.625	0.17718	157.823649229754\\
64.625	0.18084	166.842477376867\\
64.625	0.1845	176.167969200567\\
64.625	0.18816	185.800124700854\\
64.625	0.19182	195.738943877726\\
64.625	0.19548	205.984426731186\\
64.625	0.19914	216.536573261231\\
64.625	0.2028	227.395383467863\\
64.625	0.20646	238.560857351081\\
64.625	0.21012	250.032994910885\\
64.625	0.21378	261.811796147276\\
64.625	0.21744	273.897261060253\\
64.625	0.2211	286.289389649817\\
64.625	0.22476	298.988181915966\\
64.625	0.22842	311.993637858703\\
64.625	0.23208	325.305757478025\\
64.625	0.23574	338.924540773934\\
64.625	0.2394	352.849987746429\\
64.625	0.24306	367.082098395511\\
64.625	0.24672	381.620872721179\\
64.625	0.25038	396.466310723433\\
64.625	0.25404	411.618412402274\\
64.625	0.2577	427.077177757701\\
64.625	0.26136	442.842606789714\\
64.625	0.26502	458.914699498314\\
64.625	0.26868	475.2934558835\\
64.625	0.27234	491.978875945273\\
64.625	0.276	508.970959683631\\
65	0.093	35.5228955688105\\
65	0.09666	37.48311311189\\
65	0.10032	39.749994331556\\
65	0.10398	42.3235392278083\\
65	0.10764	45.2037478006469\\
65	0.1113	48.3906200500719\\
65	0.11496	51.8841559760832\\
65	0.11862	55.6843555786809\\
65	0.12228	59.7912188578649\\
65	0.12594	64.2047458136353\\
65	0.1296	68.9249364459921\\
65	0.13326	73.9517907549352\\
65	0.13692	79.2853087404646\\
65	0.14058	84.9254904025805\\
65	0.14424	90.8723357412826\\
65	0.1479	97.1258447565712\\
65	0.15156	103.686017448446\\
65	0.15522	110.552853816907\\
65	0.15888	117.726353861955\\
65	0.16254	125.206517583589\\
65	0.1662	132.993344981809\\
65	0.16986	141.086836056616\\
65	0.17352	149.486990808009\\
65	0.17718	158.193809235988\\
65	0.18084	167.207291340553\\
65	0.1845	176.527437121706\\
65	0.18816	186.154246579444\\
65	0.19182	196.087719713769\\
65	0.19548	206.32785652468\\
65	0.19914	216.874657012177\\
65	0.2028	227.728121176261\\
65	0.20646	238.888249016931\\
65	0.21012	250.355040534188\\
65	0.21378	262.12849572803\\
65	0.21744	274.20861459846\\
65	0.2211	286.595397145475\\
65	0.22476	299.288843369077\\
65	0.22842	312.288953269265\\
65	0.23208	325.59572684604\\
65	0.23574	339.209164099401\\
65	0.2394	353.129265029348\\
65	0.24306	367.356029635882\\
65	0.24672	381.889457919002\\
65	0.25038	396.729549878708\\
65	0.25404	411.876305515001\\
65	0.2577	427.32972482788\\
65	0.26136	443.089807817345\\
65	0.26502	459.156554483397\\
65	0.26868	475.529964826035\\
65	0.27234	492.21003884526\\
65	0.276	509.19677654107\\
65.375	0.093	36.0269720342123\\
65.375	0.09666	37.9818435347439\\
65.375	0.10032	40.2433787118619\\
65.375	0.10398	42.8115775655662\\
65.375	0.10764	45.6864400958569\\
65.375	0.1113	48.8679663027339\\
65.375	0.11496	52.3561561861974\\
65.375	0.11862	56.1510097462471\\
65.375	0.12228	60.2525269828832\\
65.375	0.12594	64.6607078961056\\
65.375	0.1296	69.3755524859143\\
65.375	0.13326	74.3970607523095\\
65.375	0.13692	79.725232695291\\
65.375	0.14058	85.3600683148589\\
65.375	0.14424	91.301567611013\\
65.375	0.1479	97.5497305837536\\
65.375	0.15156	104.104557233081\\
65.375	0.15522	110.966047558994\\
65.375	0.15888	118.134201561493\\
65.375	0.16254	125.609019240579\\
65.375	0.1662	133.390500596252\\
65.375	0.16986	141.47864562851\\
65.375	0.17352	149.873454337355\\
65.375	0.17718	158.574926722787\\
65.375	0.18084	167.583062784804\\
65.375	0.1845	176.897862523408\\
65.375	0.18816	186.519325938599\\
65.375	0.19182	196.447453030376\\
65.375	0.19548	206.682243798739\\
65.375	0.19914	217.223698243688\\
65.375	0.2028	228.071816365224\\
65.375	0.20646	239.226598163346\\
65.375	0.21012	250.688043638055\\
65.375	0.21378	262.45615278935\\
65.375	0.21744	274.530925617231\\
65.375	0.2211	286.912362121698\\
65.375	0.22476	299.600462302752\\
65.375	0.22842	312.595226160393\\
65.375	0.23208	325.896653694619\\
65.375	0.23574	339.504744905432\\
65.375	0.2394	353.419499792832\\
65.375	0.24306	367.640918356817\\
65.375	0.24672	382.169000597389\\
65.375	0.25038	397.003746514548\\
65.375	0.25404	412.145156108293\\
65.375	0.2577	427.593229378624\\
65.375	0.26136	443.347966325541\\
65.375	0.26502	459.409366949045\\
65.375	0.26868	475.777431249135\\
65.375	0.27234	492.452159225812\\
65.375	0.276	509.433550879074\\
65.75	0.093	36.5420059801791\\
65.75	0.09666	38.4915314381627\\
65.75	0.10032	40.7477205727327\\
65.75	0.10398	43.3105733838891\\
65.75	0.10764	46.1800898716318\\
65.75	0.1113	49.3562700359609\\
65.75	0.11496	52.8391138768764\\
65.75	0.11862	56.6286213943781\\
65.75	0.12228	60.7247925884662\\
65.75	0.12594	65.1276274591407\\
65.75	0.1296	69.8371260064016\\
65.75	0.13326	74.8532882302487\\
65.75	0.13692	80.1761141306822\\
65.75	0.14058	85.8056037077021\\
65.75	0.14424	91.7417569613085\\
65.75	0.1479	97.9845738915011\\
65.75	0.15156	104.53405449828\\
65.75	0.15522	111.390198781645\\
65.75	0.15888	118.553006741597\\
65.75	0.16254	126.022478378135\\
65.75	0.1662	133.798613691259\\
65.75	0.16986	141.88141268097\\
65.75	0.17352	150.270875347267\\
65.75	0.17718	158.96700169015\\
65.75	0.18084	167.96979170962\\
65.75	0.1845	177.279245405676\\
65.75	0.18816	186.895362778319\\
65.75	0.19182	196.818143827548\\
65.75	0.19548	207.047588553363\\
65.75	0.19914	217.583696955764\\
65.75	0.2028	228.426469034752\\
65.75	0.20646	239.575904790326\\
65.75	0.21012	251.032004222487\\
65.75	0.21378	262.794767331234\\
65.75	0.21744	274.864194116567\\
65.75	0.2211	287.240284578487\\
65.75	0.22476	299.923038716993\\
65.75	0.22842	312.912456532085\\
65.75	0.23208	326.208538023764\\
65.75	0.23574	339.811283192029\\
65.75	0.2394	353.72069203688\\
65.75	0.24306	367.936764558318\\
65.75	0.24672	382.459500756342\\
65.75	0.25038	397.288900630953\\
65.75	0.25404	412.424964182149\\
65.75	0.2577	427.867691409932\\
65.75	0.26136	443.617082314302\\
65.75	0.26502	459.673136895258\\
65.75	0.26868	476.0358551528\\
65.75	0.27234	492.705237086928\\
65.75	0.276	509.681282697643\\
66.125	0.093	37.0679974067107\\
66.125	0.09666	39.0121768221464\\
66.125	0.10032	41.2630199141685\\
66.125	0.10398	43.8205266827769\\
66.125	0.10764	46.6846971279717\\
66.125	0.1113	49.8555312497528\\
66.125	0.11496	53.3330290481203\\
66.125	0.11862	57.1171905230741\\
66.125	0.12228	61.2080156746142\\
66.125	0.12594	65.6055045027408\\
66.125	0.1296	70.3096570074536\\
66.125	0.13326	75.3204731887528\\
66.125	0.13692	80.6379530466385\\
66.125	0.14058	86.2620965811104\\
66.125	0.14424	92.1929037921688\\
66.125	0.1479	98.4303746798133\\
66.125	0.15156	104.974509244044\\
66.125	0.15522	111.825307484862\\
66.125	0.15888	118.982769402265\\
66.125	0.16254	126.446894996255\\
66.125	0.1662	134.217684266832\\
66.125	0.16986	142.295137213995\\
66.125	0.17352	150.679253837744\\
66.125	0.17718	159.370034138079\\
66.125	0.18084	168.367478115001\\
66.125	0.1845	177.671585768509\\
66.125	0.18816	187.282357098604\\
66.125	0.19182	197.199792105284\\
66.125	0.19548	207.423890788552\\
66.125	0.19914	217.954653148405\\
66.125	0.2028	228.792079184845\\
66.125	0.20646	239.936168897871\\
66.125	0.21012	251.386922287484\\
66.125	0.21378	263.144339353683\\
66.125	0.21744	275.208420096468\\
66.125	0.2211	287.57916451584\\
66.125	0.22476	300.256572611798\\
66.125	0.22842	313.240644384342\\
66.125	0.23208	326.531379833473\\
66.125	0.23574	340.12877895919\\
66.125	0.2394	354.032841761494\\
66.125	0.24306	368.243568240383\\
66.125	0.24672	382.760958395859\\
66.125	0.25038	397.585012227922\\
66.125	0.25404	412.715729736571\\
66.125	0.2577	428.153110921806\\
66.125	0.26136	443.897155783628\\
66.125	0.26502	459.947864322036\\
66.125	0.26868	476.30523653703\\
66.125	0.27234	492.96927242861\\
66.125	0.276	509.939971996777\\
66.5	0.093	37.6049463138072\\
66.5	0.09666	39.543779686695\\
66.5	0.10032	41.7892767361691\\
66.5	0.10398	44.3414374622295\\
66.5	0.10764	47.2002618648764\\
66.5	0.1113	50.3657499441095\\
66.5	0.11496	53.8379016999291\\
66.5	0.11862	57.6167171323349\\
66.5	0.12228	61.7021962413271\\
66.5	0.12594	66.0943390269057\\
66.5	0.1296	70.7931454890706\\
66.5	0.13326	75.7986156278218\\
66.5	0.13692	81.1107494431595\\
66.5	0.14058	86.7295469350835\\
66.5	0.14424	92.6550081035939\\
66.5	0.1479	98.8871329486905\\
66.5	0.15156	105.425921470373\\
66.5	0.15522	112.271373668643\\
66.5	0.15888	119.423489543499\\
66.5	0.16254	126.882269094941\\
66.5	0.1662	134.647712322969\\
66.5	0.16986	142.719819227584\\
66.5	0.17352	151.098589808785\\
66.5	0.17718	159.784024066573\\
66.5	0.18084	168.776122000946\\
66.5	0.1845	178.074883611907\\
66.5	0.18816	187.680308899453\\
66.5	0.19182	197.592397863586\\
66.5	0.19548	207.811150504305\\
66.5	0.19914	218.336566821611\\
66.5	0.2028	229.168646815503\\
66.5	0.20646	240.307390485981\\
66.5	0.21012	251.752797833046\\
66.5	0.21378	263.504868856697\\
66.5	0.21744	275.563603556934\\
66.5	0.2211	287.929001933758\\
66.5	0.22476	300.601063987168\\
66.5	0.22842	313.579789717165\\
66.5	0.23208	326.865179123747\\
66.5	0.23574	340.457232206917\\
66.5	0.2394	354.355948966672\\
66.5	0.24306	368.561329403014\\
66.5	0.24672	383.073373515942\\
66.5	0.25038	397.892081305456\\
66.5	0.25404	413.017452771557\\
66.5	0.2577	428.449487914245\\
66.5	0.26136	444.188186733518\\
66.5	0.26502	460.233549229378\\
66.5	0.26868	476.585575401824\\
66.5	0.27234	493.244265250857\\
66.5	0.276	510.209618776476\\
66.875	0.093	38.1528527014687\\
66.875	0.09666	40.0863400318085\\
66.875	0.10032	42.3264910387347\\
66.875	0.10398	44.8733057222471\\
66.875	0.10764	47.726784082346\\
66.875	0.1113	50.8869261190312\\
66.875	0.11496	54.3537318323028\\
66.875	0.11862	58.1272012221607\\
66.875	0.12228	62.2073342886049\\
66.875	0.12594	66.5941310316355\\
66.875	0.1296	71.2875914512525\\
66.875	0.13326	76.2877155474558\\
66.875	0.13692	81.5945033202455\\
66.875	0.14058	87.2079547696216\\
66.875	0.14424	93.1280698955839\\
66.875	0.1479	99.3548486981326\\
66.875	0.15156	105.888291177268\\
66.875	0.15522	112.728397332989\\
66.875	0.15888	119.875167165297\\
66.875	0.16254	127.328600674191\\
66.875	0.1662	135.088697859672\\
66.875	0.16986	143.155458721738\\
66.875	0.17352	151.528883260392\\
66.875	0.17718	160.208971475631\\
66.875	0.18084	169.195723367457\\
66.875	0.1845	178.489138935869\\
66.875	0.18816	188.089218180868\\
66.875	0.19182	197.995961102453\\
66.875	0.19548	208.209367700624\\
66.875	0.19914	218.729437975382\\
66.875	0.2028	229.556171926726\\
66.875	0.20646	240.689569554656\\
66.875	0.21012	252.129630859173\\
66.875	0.21378	263.876355840276\\
66.875	0.21744	275.929744497965\\
66.875	0.2211	288.289796832241\\
66.875	0.22476	300.956512843103\\
66.875	0.22842	313.929892530552\\
66.875	0.23208	327.209935894586\\
66.875	0.23574	340.796642935208\\
66.875	0.2394	354.690013652415\\
66.875	0.24306	368.890048046209\\
66.875	0.24672	383.396746116589\\
66.875	0.25038	398.210107863556\\
66.875	0.25404	413.330133287109\\
66.875	0.2577	428.756822387248\\
66.875	0.26136	444.490175163974\\
66.875	0.26502	460.530191617286\\
66.875	0.26868	476.876871747184\\
66.875	0.27234	493.530215553669\\
66.875	0.276	510.49022303674\\
67.25	0.093	38.711716569695\\
67.25	0.09666	40.6398578574868\\
67.25	0.10032	42.874662821865\\
67.25	0.10398	45.4161314628295\\
67.25	0.10764	48.2642637803804\\
67.25	0.1113	51.4190597745177\\
67.25	0.11496	54.8805194452413\\
67.25	0.11862	58.6486427925513\\
67.25	0.12228	62.7234298164476\\
67.25	0.12594	67.1048805169302\\
67.25	0.1296	71.7929948939992\\
67.25	0.13326	76.7877729476546\\
67.25	0.13692	82.0892146778963\\
67.25	0.14058	87.6973200847244\\
67.25	0.14424	93.6120891681388\\
67.25	0.1479	99.8335219281395\\
67.25	0.15156	106.361618364727\\
67.25	0.15522	113.1963784779\\
67.25	0.15888	120.33780226766\\
67.25	0.16254	127.785889734006\\
67.25	0.1662	135.540640876939\\
67.25	0.16986	143.602055696458\\
67.25	0.17352	151.970134192563\\
67.25	0.17718	160.644876365254\\
67.25	0.18084	169.626282214532\\
67.25	0.1845	178.914351740397\\
67.25	0.18816	188.509084942847\\
67.25	0.19182	198.410481821884\\
67.25	0.19548	208.618542377508\\
67.25	0.19914	219.133266609717\\
67.25	0.2028	229.954654518513\\
67.25	0.20646	241.082706103896\\
67.25	0.21012	252.517421365864\\
67.25	0.21378	264.258800304419\\
67.25	0.21744	276.306842919561\\
67.25	0.2211	288.661549211289\\
67.25	0.22476	301.322919179603\\
67.25	0.22842	314.290952824503\\
67.25	0.23208	327.56565014599\\
67.25	0.23574	341.147011144064\\
67.25	0.2394	355.035035818723\\
67.25	0.24306	369.229724169969\\
67.25	0.24672	383.731076197801\\
67.25	0.25038	398.53909190222\\
67.25	0.25404	413.653771283225\\
67.25	0.2577	429.075114340816\\
67.25	0.26136	444.803121074994\\
67.25	0.26502	460.837791485758\\
67.25	0.26868	477.179125573108\\
67.25	0.27234	493.827123337045\\
67.25	0.276	510.781784777568\\
67.625	0.093	39.2815379184862\\
67.625	0.09666	41.2043331637301\\
67.625	0.10032	43.4337920855603\\
67.625	0.10398	45.9699146839769\\
67.625	0.10764	48.8127009589799\\
67.625	0.1113	51.9621509105691\\
67.625	0.11496	55.4182645387448\\
67.625	0.11862	59.1810418435068\\
67.625	0.12228	63.2504828248551\\
67.625	0.12594	67.6265874827898\\
67.625	0.1296	72.3093558173109\\
67.625	0.13326	77.2987878284183\\
67.625	0.13692	82.594883516112\\
67.625	0.14058	88.1976428803922\\
67.625	0.14424	94.1070659212586\\
67.625	0.1479	100.323152638711\\
67.625	0.15156	106.845903032751\\
67.625	0.15522	113.675317103376\\
67.625	0.15888	120.811394850588\\
67.625	0.16254	128.254136274386\\
67.625	0.1662	136.003541374771\\
67.625	0.16986	144.059610151742\\
67.625	0.17352	152.422342605299\\
67.625	0.17718	161.091738735443\\
67.625	0.18084	170.067798542172\\
67.625	0.1845	179.350522025489\\
67.625	0.18816	188.939909185392\\
67.625	0.19182	198.835960021881\\
67.625	0.19548	209.038674534956\\
67.625	0.19914	219.548052724618\\
67.625	0.2028	230.364094590866\\
67.625	0.20646	241.4868001337\\
67.625	0.21012	252.916169353121\\
67.625	0.21378	264.652202249128\\
67.625	0.21744	276.694898821722\\
67.625	0.2211	289.044259070901\\
67.625	0.22476	301.700282996668\\
67.625	0.22842	314.66297059902\\
67.625	0.23208	327.932321877959\\
67.625	0.23574	341.508336833484\\
67.625	0.2394	355.391015465596\\
67.625	0.24306	369.580357774294\\
67.625	0.24672	384.076363759578\\
67.625	0.25038	398.879033421449\\
67.625	0.25404	413.988366759906\\
67.625	0.2577	429.40436377495\\
67.625	0.26136	445.127024466579\\
67.625	0.26502	461.156348834795\\
67.625	0.26868	477.492336879597\\
67.625	0.27234	494.134988600986\\
67.625	0.276	511.084303998961\\
68	0.093	39.8623167478423\\
68	0.09666	41.7797659505383\\
68	0.10032	44.0038788298206\\
68	0.10398	46.5346553856892\\
68	0.10764	49.3720956181442\\
68	0.1113	52.5161995271854\\
68	0.11496	55.9669671128132\\
68	0.11862	59.7243983750272\\
68	0.12228	63.7884933138276\\
68	0.12594	68.1592519292143\\
68	0.1296	72.8366742211875\\
68	0.13326	77.8207601897468\\
68	0.13692	83.1115098348926\\
68	0.14058	88.7089231566248\\
68	0.14424	94.6130001549434\\
68	0.1479	100.823740829848\\
68	0.15156	107.341145181339\\
68	0.15522	114.165213209417\\
68	0.15888	121.295944914081\\
68	0.16254	128.733340295331\\
68	0.1662	136.477399353168\\
68	0.16986	144.528122087591\\
68	0.17352	152.8855084986\\
68	0.17718	161.549558586196\\
68	0.18084	170.520272350378\\
68	0.1845	179.797649791146\\
68	0.18816	189.381690908501\\
68	0.19182	199.272395702442\\
68	0.19548	209.469764172969\\
68	0.19914	219.973796320083\\
68	0.2028	230.784492143783\\
68	0.20646	241.90185164407\\
68	0.21012	253.325874820943\\
68	0.21378	265.056561674402\\
68	0.21744	277.093912204447\\
68	0.2211	289.437926411079\\
68	0.22476	302.088604294297\\
68	0.22842	315.045945854102\\
68	0.23208	328.309951090493\\
68	0.23574	341.88062000347\\
68	0.2394	355.757952593034\\
68	0.24306	369.941948859184\\
68	0.24672	384.43260880192\\
68	0.25038	399.229932421243\\
68	0.25404	414.333919717152\\
68	0.2577	429.744570689648\\
68	0.26136	445.461885338729\\
68	0.26502	461.485863664397\\
68	0.26868	477.816505666652\\
68	0.27234	494.453811345493\\
68	0.276	511.39778070092\\
68.375	0.093	40.4540530577633\\
68.375	0.09666	42.3661562179113\\
68.375	0.10032	44.5849230546456\\
68.375	0.10398	47.1103535679663\\
68.375	0.10764	49.9424477578733\\
68.375	0.1113	53.0812056243667\\
68.375	0.11496	56.5266271674464\\
68.375	0.11862	60.2787123871125\\
68.375	0.12228	64.3374612833649\\
68.375	0.12594	68.7028738562037\\
68.375	0.1296	73.3749501056288\\
68.375	0.13326	78.3536900316403\\
68.375	0.13692	83.6390936342381\\
68.375	0.14058	89.2311609134224\\
68.375	0.14424	95.129891869193\\
68.375	0.1479	101.33528650155\\
68.375	0.15156	107.847344810493\\
68.375	0.15522	114.666066796023\\
68.375	0.15888	121.791452458139\\
68.375	0.16254	129.223501796841\\
68.375	0.1662	136.96221481213\\
68.375	0.16986	145.007591504005\\
68.375	0.17352	153.359631872466\\
68.375	0.17718	162.018335917514\\
68.375	0.18084	170.983703639148\\
68.375	0.1845	180.255735037368\\
68.375	0.18816	189.834430112175\\
68.375	0.19182	199.719788863568\\
68.375	0.19548	209.911811291548\\
68.375	0.19914	220.410497396113\\
68.375	0.2028	231.215847177266\\
68.375	0.20646	242.327860635004\\
68.375	0.21012	253.746537769329\\
68.375	0.21378	265.47187858024\\
68.375	0.21744	277.503883067738\\
68.375	0.2211	289.842551231822\\
68.375	0.22476	302.487883072492\\
68.375	0.22842	315.439878589749\\
68.375	0.23208	328.698537783592\\
68.375	0.23574	342.263860654021\\
68.375	0.2394	356.135847201037\\
68.375	0.24306	370.314497424639\\
68.375	0.24672	384.799811324827\\
68.375	0.25038	399.591788901602\\
68.375	0.25404	414.690430154963\\
68.375	0.2577	430.09573508491\\
68.375	0.26136	445.807703691444\\
68.375	0.26502	461.826335974564\\
68.375	0.26868	478.151631934271\\
68.375	0.27234	494.783591570564\\
68.375	0.276	511.722214883443\\
68.75	0.093	41.0567468482492\\
68.75	0.09666	42.9635039658492\\
68.75	0.10032	45.1769247600356\\
68.75	0.10398	47.6970092308083\\
68.75	0.10764	50.5237573781674\\
68.75	0.1113	53.6571692021128\\
68.75	0.11496	57.0972447026446\\
68.75	0.11862	60.8439838797627\\
68.75	0.12228	64.8973867334672\\
68.75	0.12594	69.2574532637581\\
68.75	0.1296	73.9241834706352\\
68.75	0.13326	78.8975773540987\\
68.75	0.13692	84.1776349141486\\
68.75	0.14058	89.7643561507849\\
68.75	0.14424	95.6577410640075\\
68.75	0.1479	101.857789653816\\
68.75	0.15156	108.364501920212\\
68.75	0.15522	115.177877863193\\
68.75	0.15888	122.297917482761\\
68.75	0.16254	129.724620778916\\
68.75	0.1662	137.457987751656\\
68.75	0.16986	145.498018400983\\
68.75	0.17352	153.844712726897\\
68.75	0.17718	162.498070729397\\
68.75	0.18084	171.458092408483\\
68.75	0.1845	180.724777764155\\
68.75	0.18816	190.298126796414\\
68.75	0.19182	200.178139505259\\
68.75	0.19548	210.364815890691\\
68.75	0.19914	220.858155952709\\
68.75	0.2028	231.658159691313\\
68.75	0.20646	242.764827106503\\
68.75	0.21012	254.17815819828\\
68.75	0.21378	265.898152966643\\
68.75	0.21744	277.924811411593\\
68.75	0.2211	290.258133533129\\
68.75	0.22476	302.898119331251\\
68.75	0.22842	315.84476880596\\
68.75	0.23208	329.098081957255\\
68.75	0.23574	342.658058785137\\
68.75	0.2394	356.524699289604\\
68.75	0.24306	370.698003470658\\
68.75	0.24672	385.177971328299\\
68.75	0.25038	399.964602862526\\
68.75	0.25404	415.057898073339\\
68.75	0.2577	430.457856960738\\
68.75	0.26136	446.164479524724\\
68.75	0.26502	462.177765765296\\
68.75	0.26868	478.497715682455\\
68.75	0.27234	495.1243292762\\
68.75	0.276	512.057606546531\\
69.125	0.093	41.6703981193\\
69.125	0.09666	43.571809194352\\
69.125	0.10032	45.7798839459904\\
69.125	0.10398	48.2946223742152\\
69.125	0.10764	51.1160244790263\\
69.125	0.1113	54.2440902604238\\
69.125	0.11496	57.6788197184076\\
69.125	0.11862	61.4202128529778\\
69.125	0.12228	65.4682696641343\\
69.125	0.12594	69.8229901518772\\
69.125	0.1296	74.4843743162064\\
69.125	0.13326	79.4524221571219\\
69.125	0.13692	84.7271336746239\\
69.125	0.14058	90.3085088687122\\
69.125	0.14424	96.1965477393869\\
69.125	0.1479	102.391250286648\\
69.125	0.15156	108.892616510495\\
69.125	0.15522	115.700646410929\\
69.125	0.15888	122.815339987949\\
69.125	0.16254	130.236697241555\\
69.125	0.1662	137.964718171748\\
69.125	0.16986	145.999402778527\\
69.125	0.17352	154.340751061893\\
69.125	0.17718	162.988763021844\\
69.125	0.18084	171.943438658382\\
69.125	0.1845	181.204777971507\\
69.125	0.18816	190.772780961218\\
69.125	0.19182	200.647447627515\\
69.125	0.19548	210.828777970399\\
69.125	0.19914	221.316771989869\\
69.125	0.2028	232.111429685925\\
69.125	0.20646	243.212751058567\\
69.125	0.21012	254.620736107796\\
69.125	0.21378	266.335384833612\\
69.125	0.21744	278.356697236013\\
69.125	0.2211	290.684673315001\\
69.125	0.22476	303.319313070576\\
69.125	0.22842	316.260616502737\\
69.125	0.23208	329.508583611484\\
69.125	0.23574	343.063214396817\\
69.125	0.2394	356.924508858737\\
69.125	0.24306	371.092466997243\\
69.125	0.24672	385.567088812335\\
69.125	0.25038	400.348374304014\\
69.125	0.25404	415.436323472279\\
69.125	0.2577	430.830936317131\\
69.125	0.26136	446.532212838569\\
69.125	0.26502	462.540153036593\\
69.125	0.26868	478.854756911204\\
69.125	0.27234	495.476024462401\\
69.125	0.276	512.403955690184\\
69.5	0.093	42.2950068709157\\
69.5	0.09666	44.1910719034198\\
69.5	0.10032	46.3938006125102\\
69.5	0.10398	48.903192998187\\
69.5	0.10764	51.7192490604502\\
69.5	0.1113	54.8419687992997\\
69.5	0.11496	58.2713522147356\\
69.5	0.11862	62.0073993067578\\
69.5	0.12228	66.0501100753664\\
69.5	0.12594	70.3994845205613\\
69.5	0.1296	75.0555226423425\\
69.5	0.13326	80.0182244407101\\
69.5	0.13692	85.2875899156641\\
69.5	0.14058	90.8636190672045\\
69.5	0.14424	96.7463118953312\\
69.5	0.1479	102.935668400044\\
69.5	0.15156	109.431688581344\\
69.5	0.15522	116.234372439229\\
69.5	0.15888	123.343719973701\\
69.5	0.16254	130.75973118476\\
69.5	0.1662	138.482406072405\\
69.5	0.16986	146.511744636636\\
69.5	0.17352	154.847746877453\\
69.5	0.17718	163.490412794857\\
69.5	0.18084	172.439742388847\\
69.5	0.1845	181.695735659424\\
69.5	0.18816	191.258392606587\\
69.5	0.19182	201.127713230336\\
69.5	0.19548	211.303697530672\\
69.5	0.19914	221.786345507594\\
69.5	0.2028	232.575657161102\\
69.5	0.20646	243.671632491197\\
69.5	0.21012	255.074271497878\\
69.5	0.21378	266.783574181145\\
69.5	0.21744	278.799540540999\\
69.5	0.2211	291.122170577439\\
69.5	0.22476	303.751464290465\\
69.5	0.22842	316.687421680078\\
69.5	0.23208	329.930042746277\\
69.5	0.23574	343.479327489062\\
69.5	0.2394	357.335275908434\\
69.5	0.24306	371.497888004392\\
69.5	0.24672	385.967163776937\\
69.5	0.25038	400.743103226068\\
69.5	0.25404	415.825706351785\\
69.5	0.2577	431.214973154089\\
69.5	0.26136	446.910903632979\\
69.5	0.26502	462.913497788455\\
69.5	0.26868	479.222755620517\\
69.5	0.27234	495.838677129166\\
69.5	0.276	512.761262314402\\
69.875	0.093	42.9305731030962\\
69.875	0.09666	44.8212920930523\\
69.875	0.10032	47.0186747595949\\
69.875	0.10398	49.5227211027237\\
69.875	0.10764	52.3334311224389\\
69.875	0.1113	55.4508048187404\\
69.875	0.11496	58.8748421916284\\
69.875	0.11862	62.6055432411026\\
69.875	0.12228	66.6429079671632\\
69.875	0.12594	70.9869363698102\\
69.875	0.1296	75.6376284490434\\
69.875	0.13326	80.5949842048631\\
69.875	0.13692	85.8590036372692\\
69.875	0.14058	91.4296867462616\\
69.875	0.14424	97.3070335318403\\
69.875	0.1479	103.491043994005\\
69.875	0.15156	109.981718132757\\
69.875	0.15522	116.779055948095\\
69.875	0.15888	123.883057440019\\
69.875	0.16254	131.293722608529\\
69.875	0.1662	139.011051453626\\
69.875	0.16986	147.035043975309\\
69.875	0.17352	155.365700173579\\
69.875	0.17718	164.003020048435\\
69.875	0.18084	172.947003599877\\
69.875	0.1845	182.197650827905\\
69.875	0.18816	191.75496173252\\
69.875	0.19182	201.618936313722\\
69.875	0.19548	211.789574571509\\
69.875	0.19914	222.266876505883\\
69.875	0.2028	233.050842116844\\
69.875	0.20646	244.14147140439\\
69.875	0.21012	255.538764368523\\
69.875	0.21378	267.242721009243\\
69.875	0.21744	279.253341326549\\
69.875	0.2211	291.570625320441\\
69.875	0.22476	304.194572990919\\
69.875	0.22842	317.125184337984\\
69.875	0.23208	330.362459361635\\
69.875	0.23574	343.906398061873\\
69.875	0.2394	357.757000438696\\
69.875	0.24306	371.914266492107\\
69.875	0.24672	386.378196222103\\
69.875	0.25038	401.148789628686\\
69.875	0.25404	416.226046711855\\
69.875	0.2577	431.609967471611\\
69.875	0.26136	447.300551907953\\
69.875	0.26502	463.297800020881\\
69.875	0.26868	479.601711810396\\
69.875	0.27234	496.212287276497\\
69.875	0.276	513.129526419184\\
70.25	0.093	43.5770968158417\\
70.25	0.09666	45.4624697632499\\
70.25	0.10032	47.6545063872444\\
70.25	0.10398	50.1532066878253\\
70.25	0.10764	52.9585706649925\\
70.25	0.1113	56.0705983187461\\
70.25	0.11496	59.4892896490861\\
70.25	0.11862	63.2146446560124\\
70.25	0.12228	67.246663339525\\
70.25	0.12594	71.585345699624\\
70.25	0.1296	76.2306917363094\\
70.25	0.13326	81.1827014495811\\
70.25	0.13692	86.4413748394391\\
70.25	0.14058	92.0067119058836\\
70.25	0.14424	97.8787126489143\\
70.25	0.1479	104.057377068531\\
70.25	0.15156	110.542705164735\\
70.25	0.15522	117.334696937525\\
70.25	0.15888	124.433352386901\\
70.25	0.16254	131.838671512863\\
70.25	0.1662	139.550654315412\\
70.25	0.16986	147.569300794548\\
70.25	0.17352	155.894610950269\\
70.25	0.17718	164.526584782577\\
70.25	0.18084	173.465222291471\\
70.25	0.1845	182.710523476952\\
70.25	0.18816	192.262488339019\\
70.25	0.19182	202.121116877672\\
70.25	0.19548	212.286409092912\\
70.25	0.19914	222.758364984738\\
70.25	0.2028	233.53698455315\\
70.25	0.20646	244.622267798149\\
70.25	0.21012	256.014214719734\\
70.25	0.21378	267.712825317906\\
70.25	0.21744	279.718099592663\\
70.25	0.2211	292.030037544008\\
70.25	0.22476	304.648639171938\\
70.25	0.22842	317.573904476455\\
70.25	0.23208	330.805833457558\\
70.25	0.23574	344.344426115248\\
70.25	0.2394	358.189682449524\\
70.25	0.24306	372.341602460386\\
70.25	0.24672	386.800186147835\\
70.25	0.25038	401.56543351187\\
70.25	0.25404	416.637344552491\\
70.25	0.2577	432.015919269699\\
70.25	0.26136	447.701157663493\\
70.25	0.26502	463.693059733873\\
70.25	0.26868	479.99162548084\\
70.25	0.27234	496.596854904393\\
70.25	0.276	513.508748004532\\
70.625	0.093	44.234578009152\\
70.625	0.09666	46.1146049140123\\
70.625	0.10032	48.3012954954589\\
70.625	0.10398	50.7946497534918\\
70.625	0.10764	53.5946676881111\\
70.625	0.1113	56.7013492993167\\
70.625	0.11496	60.1146945871087\\
70.625	0.11862	63.8347035514871\\
70.625	0.12228	67.8613761924518\\
70.625	0.12594	72.1947125100028\\
70.625	0.1296	76.8347125041402\\
70.625	0.13326	81.7813761748639\\
70.625	0.13692	87.034703522174\\
70.625	0.14058	92.5946945460705\\
70.625	0.14424	98.4613492465534\\
70.625	0.1479	104.634667623623\\
70.625	0.15156	111.114649677278\\
70.625	0.15522	117.90129540752\\
70.625	0.15888	124.994604814348\\
70.625	0.16254	132.394577897763\\
70.625	0.1662	140.101214657764\\
70.625	0.16986	148.114515094351\\
70.625	0.17352	156.434479207524\\
70.625	0.17718	165.061106997284\\
70.625	0.18084	173.994398463631\\
70.625	0.1845	183.234353606563\\
70.625	0.18816	192.780972426082\\
70.625	0.19182	202.634254922188\\
70.625	0.19548	212.79420109488\\
70.625	0.19914	223.260810944158\\
70.625	0.2028	234.034084470022\\
70.625	0.20646	245.114021672473\\
70.625	0.21012	256.50062255151\\
70.625	0.21378	268.193887107133\\
70.625	0.21744	280.193815339343\\
70.625	0.2211	292.500407248139\\
70.625	0.22476	305.113662833522\\
70.625	0.22842	318.033582095491\\
70.625	0.23208	331.260165034046\\
70.625	0.23574	344.793411649188\\
70.625	0.2394	358.633321940916\\
70.625	0.24306	372.77989590923\\
70.625	0.24672	387.233133554131\\
70.625	0.25038	401.993034875618\\
70.625	0.25404	417.059599873691\\
70.625	0.2577	432.432828548351\\
70.625	0.26136	448.112720899597\\
70.625	0.26502	464.099276927429\\
70.625	0.26868	480.392496631848\\
70.625	0.27234	496.992380012853\\
70.625	0.276	513.898927070445\\
71	0.093	44.9030166830273\\
71	0.09666	46.7776975453395\\
71	0.10032	48.9590420842382\\
71	0.10398	51.4470502997231\\
71	0.10764	54.2417221917945\\
71	0.1113	57.3430577604521\\
71	0.11496	60.7510570056962\\
71	0.11862	64.4657199275266\\
71	0.12228	68.4870465259433\\
71	0.12594	72.8150368009464\\
71	0.1296	77.4496907525358\\
71	0.13326	82.3910083807116\\
71	0.13692	87.6389896854737\\
71	0.14058	93.1936346668223\\
71	0.14424	99.0549433247572\\
71	0.1479	105.222915659278\\
71	0.15156	111.697551670386\\
71	0.15522	118.47885135808\\
71	0.15888	125.56681472236\\
71	0.16254	132.961441763227\\
71	0.1662	140.66273248068\\
71	0.16986	148.670686874719\\
71	0.17352	156.985304945345\\
71	0.17718	165.606586692557\\
71	0.18084	174.534532116355\\
71	0.1845	183.76914121674\\
71	0.18816	193.310413993711\\
71	0.19182	203.158350447268\\
71	0.19548	213.312950577412\\
71	0.19914	223.774214384142\\
71	0.2028	234.542141867459\\
71	0.20646	245.616733027361\\
71	0.21012	256.997987863851\\
71	0.21378	268.685906376926\\
71	0.21744	280.680488566588\\
71	0.2211	292.981734432836\\
71	0.22476	305.589643975671\\
71	0.22842	318.504217195092\\
71	0.23208	331.725454091099\\
71	0.23574	345.253354663693\\
71	0.2394	359.087918912873\\
71	0.24306	373.229146838639\\
71	0.24672	387.677038440992\\
71	0.25038	402.431593719931\\
71	0.25404	417.492812675456\\
71	0.2577	432.860695307568\\
71	0.26136	448.535241616266\\
71	0.26502	464.516451601551\\
71	0.26868	480.804325263421\\
71	0.27234	497.398862601878\\
71	0.276	514.300063616922\\
71.375	0.093	45.5824128374674\\
71.375	0.09666	47.4517476572317\\
71.375	0.10032	49.6277461535824\\
71.375	0.10398	52.1104083265194\\
71.375	0.10764	54.8997341760428\\
71.375	0.1113	57.9957237021525\\
71.375	0.11496	61.3983769048486\\
71.375	0.11862	65.107693784131\\
71.375	0.12228	69.1236743399998\\
71.375	0.12594	73.4463185724549\\
71.375	0.1296	78.0756264814964\\
71.375	0.13326	83.0115980671242\\
71.375	0.13692	88.2542333293385\\
71.375	0.14058	93.803532268139\\
71.375	0.14424	99.659494883526\\
71.375	0.1479	105.822121175499\\
71.375	0.15156	112.291411144059\\
71.375	0.15522	119.067364789205\\
71.375	0.15888	126.149982110937\\
71.375	0.16254	133.539263109256\\
71.375	0.1662	141.235207784161\\
71.375	0.16986	149.237816135652\\
71.375	0.17352	157.54708816373\\
71.375	0.17718	166.163023868394\\
71.375	0.18084	175.085623249644\\
71.375	0.1845	184.314886307481\\
71.375	0.18816	193.850813041904\\
71.375	0.19182	203.693403452914\\
71.375	0.19548	213.842657540509\\
71.375	0.19914	224.298575304692\\
71.375	0.2028	235.06115674546\\
71.375	0.20646	246.130401862815\\
71.375	0.21012	257.506310656756\\
71.375	0.21378	269.188883127284\\
71.375	0.21744	281.178119274398\\
71.375	0.2211	293.474019098098\\
71.375	0.22476	306.076582598385\\
71.375	0.22842	318.985809775258\\
71.375	0.23208	332.201700628717\\
71.375	0.23574	345.724255158763\\
71.375	0.2394	359.553473365395\\
71.375	0.24306	373.689355248613\\
71.375	0.24672	388.131900808418\\
71.375	0.25038	402.881110044809\\
71.375	0.25404	417.936982957786\\
71.375	0.2577	433.29951954735\\
71.375	0.26136	448.9687198135\\
71.375	0.26502	464.944583756237\\
71.375	0.26868	481.22711137556\\
71.375	0.27234	497.816302671469\\
71.375	0.276	514.712157643964\\
71.75	0.093	46.2727664724724\\
71.75	0.09666	48.1367552496888\\
71.75	0.10032	50.3074077034915\\
71.75	0.10398	52.7847238338806\\
71.75	0.10764	55.568703640856\\
71.75	0.1113	58.6593471244177\\
71.75	0.11496	62.0566542845659\\
71.75	0.11862	65.7606251213003\\
71.75	0.12228	69.7712596346212\\
71.75	0.12594	74.0885578245284\\
71.75	0.1296	78.7125196910218\\
71.75	0.13326	83.6431452341017\\
71.75	0.13692	88.880434453768\\
71.75	0.14058	94.4243873500206\\
71.75	0.14424	100.27500392286\\
71.75	0.1479	106.432284172285\\
71.75	0.15156	112.896228098296\\
71.75	0.15522	119.666835700894\\
71.75	0.15888	126.744106980079\\
71.75	0.16254	134.128041935849\\
71.75	0.1662	141.818640568207\\
71.75	0.16986	149.81590287715\\
71.75	0.17352	158.11982886268\\
71.75	0.17718	166.730418524796\\
71.75	0.18084	175.647671863498\\
71.75	0.1845	184.871588878787\\
71.75	0.18816	194.402169570662\\
71.75	0.19182	204.239413939124\\
71.75	0.19548	214.383321984172\\
71.75	0.19914	224.833893705806\\
71.75	0.2028	235.591129104026\\
71.75	0.20646	246.655028178833\\
71.75	0.21012	258.025590930227\\
71.75	0.21378	269.702817358206\\
71.75	0.21744	281.686707462772\\
71.75	0.2211	293.977261243924\\
71.75	0.22476	306.574478701663\\
71.75	0.22842	319.478359835988\\
71.75	0.23208	332.6889046469\\
71.75	0.23574	346.206113134397\\
71.75	0.2394	360.029985298481\\
71.75	0.24306	374.160521139152\\
71.75	0.24672	388.597720656409\\
71.75	0.25038	403.341583850252\\
71.75	0.25404	418.392110720681\\
71.75	0.2577	433.749301267697\\
71.75	0.26136	449.413155491299\\
71.75	0.26502	465.383673391488\\
71.75	0.26868	481.660854968263\\
71.75	0.27234	498.244700221624\\
71.75	0.276	515.135209151571\\
72.125	0.093	46.9740775880423\\
72.125	0.09666	48.8327203227107\\
72.125	0.10032	50.9980267339655\\
72.125	0.10398	53.4699968218066\\
72.125	0.10764	56.2486305862341\\
72.125	0.1113	59.3339280272479\\
72.125	0.11496	62.7258891448481\\
72.125	0.11862	66.4245139390346\\
72.125	0.12228	70.4298024098075\\
72.125	0.12594	74.7417545571667\\
72.125	0.1296	79.3603703811122\\
72.125	0.13326	84.2856498816441\\
72.125	0.13692	89.5175930587625\\
72.125	0.14058	95.0561999124671\\
72.125	0.14424	100.901470442758\\
72.125	0.1479	107.053404649635\\
72.125	0.15156	113.512002533099\\
72.125	0.15522	120.277264093149\\
72.125	0.15888	127.349189329786\\
72.125	0.16254	134.727778243008\\
72.125	0.1662	142.413030832817\\
72.125	0.16986	150.404947099213\\
72.125	0.17352	158.703527042195\\
72.125	0.17718	167.308770661763\\
72.125	0.18084	176.220677957917\\
72.125	0.1845	185.439248930658\\
72.125	0.18816	194.964483579985\\
72.125	0.19182	204.796381905899\\
72.125	0.19548	214.934943908399\\
72.125	0.19914	225.380169587485\\
72.125	0.2028	236.132058943158\\
72.125	0.20646	247.190611975417\\
72.125	0.21012	258.555828684262\\
72.125	0.21378	270.227709069693\\
72.125	0.21744	282.206253131712\\
72.125	0.2211	294.491460870316\\
72.125	0.22476	307.083332285507\\
72.125	0.22842	319.981867377284\\
72.125	0.23208	333.187066145647\\
72.125	0.23574	346.698928590597\\
72.125	0.2394	360.517454712133\\
72.125	0.24306	374.642644510255\\
72.125	0.24672	389.074497984964\\
72.125	0.25038	403.81301513626\\
72.125	0.25404	418.858195964141\\
72.125	0.2577	434.210040468609\\
72.125	0.26136	449.868548649663\\
72.125	0.26502	465.833720507304\\
72.125	0.26868	482.105556041531\\
72.125	0.27234	498.684055252344\\
72.125	0.276	515.569218139744\\
72.5	0.093	47.6863461841771\\
72.5	0.09666	49.5396428762976\\
72.5	0.10032	51.6996032450044\\
72.5	0.10398	54.1662272902975\\
72.5	0.10764	56.9395150121771\\
72.5	0.1113	60.0194664106429\\
72.5	0.11496	63.4060814856951\\
72.5	0.11862	67.0993602373337\\
72.5	0.12228	71.0993026655586\\
72.5	0.12594	75.4059087703698\\
72.5	0.1296	80.0191785517674\\
72.5	0.13326	84.9391120097514\\
72.5	0.13692	90.1657091443217\\
72.5	0.14058	95.6989699554785\\
72.5	0.14424	101.538894443221\\
72.5	0.1479	107.685482607551\\
72.5	0.15156	114.138734448467\\
72.5	0.15522	120.898649965969\\
72.5	0.15888	127.965229160057\\
72.5	0.16254	135.338472030732\\
72.5	0.1662	143.018378577993\\
72.5	0.16986	151.00494880184\\
72.5	0.17352	159.298182702274\\
72.5	0.17718	167.898080279295\\
72.5	0.18084	176.804641532901\\
72.5	0.1845	186.017866463094\\
72.5	0.18816	195.537755069873\\
72.5	0.19182	205.364307353239\\
72.5	0.19548	215.497523313191\\
72.5	0.19914	225.937402949729\\
72.5	0.2028	236.683946262854\\
72.5	0.20646	247.737153252565\\
72.5	0.21012	259.097023918862\\
72.5	0.21378	270.763558261746\\
72.5	0.21744	282.736756281216\\
72.5	0.2211	295.016617977272\\
72.5	0.22476	307.603143349915\\
72.5	0.22842	320.496332399144\\
72.5	0.23208	333.69618512496\\
72.5	0.23574	347.202701527362\\
72.5	0.2394	361.01588160635\\
72.5	0.24306	375.135725361924\\
72.5	0.24672	389.562232794085\\
72.5	0.25038	404.295403902832\\
72.5	0.25404	419.335238688166\\
72.5	0.2577	434.681737150086\\
72.5	0.26136	450.334899288592\\
72.5	0.26502	466.294725103685\\
72.5	0.26868	482.561214595364\\
72.5	0.27234	499.134367763629\\
72.5	0.276	516.014184608481\\
72.875	0.093	48.4095722608768\\
72.875	0.09666	50.2575229104493\\
72.875	0.10032	52.4121372366082\\
72.875	0.10398	54.8734152393533\\
72.875	0.10764	57.6413569186849\\
72.875	0.1113	60.7159622746028\\
72.875	0.11496	64.0972313071071\\
72.875	0.11862	67.7851640161977\\
72.875	0.12228	71.7797604018747\\
72.875	0.12594	76.0810204641379\\
72.875	0.1296	80.6889442029876\\
72.875	0.13326	85.6035316184236\\
72.875	0.13692	90.824782710446\\
72.875	0.14058	96.3526974790547\\
72.875	0.14424	102.18727592425\\
72.875	0.1479	108.328518046031\\
72.875	0.15156	114.776423844399\\
72.875	0.15522	121.530993319353\\
72.875	0.15888	128.592226470894\\
72.875	0.16254	135.96012329902\\
72.875	0.1662	143.634683803734\\
72.875	0.16986	151.615907985033\\
72.875	0.17352	159.903795842919\\
72.875	0.17718	168.498347377391\\
72.875	0.18084	177.39956258845\\
72.875	0.1845	186.607441476095\\
72.875	0.18816	196.121984040326\\
72.875	0.19182	205.943190281144\\
72.875	0.19548	216.071060198548\\
72.875	0.19914	226.505593792538\\
72.875	0.2028	237.246791063115\\
72.875	0.20646	248.294652010278\\
72.875	0.21012	259.649176634027\\
72.875	0.21378	271.310364934363\\
72.875	0.21744	283.278216911285\\
72.875	0.2211	295.552732564793\\
72.875	0.22476	308.133911894888\\
72.875	0.22842	321.021754901569\\
72.875	0.23208	334.216261584837\\
72.875	0.23574	347.717431944691\\
72.875	0.2394	361.525265981131\\
72.875	0.24306	375.639763694158\\
72.875	0.24672	390.06092508377\\
72.875	0.25038	404.78875014997\\
72.875	0.25404	419.823238892755\\
72.875	0.2577	435.164391312127\\
72.875	0.26136	450.812207408086\\
72.875	0.26502	466.76668718063\\
72.875	0.26868	483.027830629761\\
72.875	0.27234	499.595637755479\\
72.875	0.276	516.470108557782\\
73.25	0.093	49.1437558181414\\
73.25	0.09666	50.986360425166\\
73.25	0.10032	53.1356287087768\\
73.25	0.10398	55.5915606689741\\
73.25	0.10764	58.3541563057577\\
73.25	0.1113	61.4234156191276\\
73.25	0.11496	64.799338609084\\
73.25	0.11862	68.4819252756266\\
73.25	0.12228	72.4711756187556\\
73.25	0.12594	76.767089638471\\
73.25	0.1296	81.3696673347727\\
73.25	0.13326	86.2789087076607\\
73.25	0.13692	91.4948137571351\\
73.25	0.14058	97.0173824831959\\
73.25	0.14424	102.846614885843\\
73.25	0.1479	108.982510965076\\
73.25	0.15156	115.425070720896\\
73.25	0.15522	122.174294153302\\
73.25	0.15888	129.230181262295\\
73.25	0.16254	136.592732047874\\
73.25	0.1662	144.261946510039\\
73.25	0.16986	152.237824648791\\
73.25	0.17352	160.520366464129\\
73.25	0.17718	169.109571956053\\
73.25	0.18084	178.005441124563\\
73.25	0.1845	187.20797396966\\
73.25	0.18816	196.717170491344\\
73.25	0.19182	206.533030689613\\
73.25	0.19548	216.655554564469\\
73.25	0.19914	227.084742115912\\
73.25	0.2028	237.820593343941\\
73.25	0.20646	248.863108248556\\
73.25	0.21012	260.212286829757\\
73.25	0.21378	271.868129087545\\
73.25	0.21744	283.830635021919\\
73.25	0.2211	296.099804632879\\
73.25	0.22476	308.675637920427\\
73.25	0.22842	321.55813488456\\
73.25	0.23208	334.747295525279\\
73.25	0.23574	348.243119842585\\
73.25	0.2394	362.045607836477\\
73.25	0.24306	376.154759506956\\
73.25	0.24672	390.570574854021\\
73.25	0.25038	405.293053877672\\
73.25	0.25404	420.32219657791\\
73.25	0.2577	435.658002954734\\
73.25	0.26136	451.300473008144\\
73.25	0.26502	467.249606738141\\
73.25	0.26868	483.505404144724\\
73.25	0.27234	500.067865227893\\
73.25	0.276	516.936989987649\\
73.625	0.093	49.8888968559709\\
73.625	0.09666	51.7261554204475\\
73.625	0.10032	53.8700776615104\\
73.625	0.10398	56.3206635791597\\
73.625	0.10764	59.0779131733954\\
73.625	0.1113	62.1418264442173\\
73.625	0.11496	65.5124033916257\\
73.625	0.11862	69.1896440156204\\
73.625	0.12228	73.1735483162014\\
73.625	0.12594	77.4641162933688\\
73.625	0.1296	82.0613479471226\\
73.625	0.13326	86.9652432774626\\
73.625	0.13692	92.1758022843891\\
73.625	0.14058	97.6930249679019\\
73.625	0.14424	103.516911328001\\
73.625	0.1479	109.647461364687\\
73.625	0.15156	116.084675077958\\
73.625	0.15522	122.828552467817\\
73.625	0.15888	129.879093534261\\
73.625	0.16254	137.236298277292\\
73.625	0.1662	144.900166696909\\
73.625	0.16986	152.870698793113\\
73.625	0.17352	161.147894565903\\
73.625	0.17718	169.731754015279\\
73.625	0.18084	178.622277141242\\
73.625	0.1845	187.819463943791\\
73.625	0.18816	197.323314422926\\
73.625	0.19182	207.133828578648\\
73.625	0.19548	217.251006410956\\
73.625	0.19914	227.674847919851\\
73.625	0.2028	238.405353105331\\
73.625	0.20646	249.442521967399\\
73.625	0.21012	260.786354506052\\
73.625	0.21378	272.436850721292\\
73.625	0.21744	284.394010613118\\
73.625	0.2211	296.65783418153\\
73.625	0.22476	309.22832142653\\
73.625	0.22842	322.105472348115\\
73.625	0.23208	335.289286946286\\
73.625	0.23574	348.779765221044\\
73.625	0.2394	362.576907172389\\
73.625	0.24306	376.680712800319\\
73.625	0.24672	391.091182104836\\
73.625	0.25038	405.80831508594\\
73.625	0.25404	420.832111743629\\
73.625	0.2577	436.162572077905\\
73.625	0.26136	451.799696088768\\
73.625	0.26502	467.743483776217\\
73.625	0.26868	483.993935140252\\
73.625	0.27234	500.551050180873\\
73.625	0.276	517.414828898081\\
74	0.093	50.6449953743653\\
74	0.09666	52.4769078962939\\
74	0.10032	54.6154840948089\\
74	0.10398	57.0607239699102\\
74	0.10764	59.8126275215979\\
74	0.1113	62.871194749872\\
74	0.11496	66.2364256547323\\
74	0.11862	69.908320236179\\
74	0.12228	73.8868784942122\\
74	0.12594	78.1721004288316\\
74	0.1296	82.7639860400373\\
74	0.13326	87.6625353278295\\
74	0.13692	92.867748292208\\
74	0.14058	98.3796249331729\\
74	0.14424	104.198165250724\\
74	0.1479	110.323369244862\\
74	0.15156	116.755236915585\\
74	0.15522	123.493768262896\\
74	0.15888	130.538963286792\\
74	0.16254	137.890821987275\\
74	0.1662	145.549344364345\\
74	0.16986	153.514530418\\
74	0.17352	161.786380148242\\
74	0.17718	170.364893555071\\
74	0.18084	179.250070638485\\
74	0.1845	188.441911398486\\
74	0.18816	197.940415835074\\
74	0.19182	207.745583948248\\
74	0.19548	217.857415738008\\
74	0.19914	228.275911204354\\
74	0.2028	239.001070347287\\
74	0.20646	250.032893166806\\
74	0.21012	261.371379662912\\
74	0.21378	273.016529835604\\
74	0.21744	284.968343684882\\
74	0.2211	297.226821210746\\
74	0.22476	309.791962413197\\
74	0.22842	322.663767292235\\
74	0.23208	335.842235847858\\
74	0.23574	349.327368080068\\
74	0.2394	363.119163988865\\
74	0.24306	377.217623574247\\
74	0.24672	391.622746836216\\
74	0.25038	406.334533774772\\
74	0.25404	421.352984389914\\
74	0.2577	436.678098681642\\
74	0.26136	452.309876649956\\
74	0.26502	468.248318294857\\
74	0.26868	484.493423616344\\
74	0.27234	501.045192614418\\
74	0.276	517.903625289077\\
};
\end{axis}

\begin{axis}[%
width=4.527496cm,
height=3.870968cm,
at={(0cm,10.752688cm)},
scale only axis,
xmin=56,
xmax=74,
tick align=outside,
xlabel={$L_{cut}$},
xmajorgrids,
ymin=0.093,
ymax=0.276,
ylabel={$D_{rlx}$},
ymajorgrids,
zmin=-0.468526216891017,
zmax=12.0724461325189,
zlabel={$x_1,x_4$},
zmajorgrids,
view={-140}{50},
legend style={at={(1.03,1)},anchor=north west,legend cell align=left,align=left,draw=white!15!black}
]
\addplot3[only marks,mark=*,mark options={},mark size=1.5000pt,color=mycolor1] plot table[row sep=crcr,]{%
74	0.123	0.616147082677301\\
72	0.113	0.756244217181634\\
61	0.095	0.315136730386088\\
56	0.093	0.479225561787365\\
};
\addplot3[only marks,mark=*,mark options={},mark size=1.5000pt,color=mycolor2] plot table[row sep=crcr,]{%
67	0.276	8.54895561081077\\
66	0.255	7.8265872893112\\
62	0.209	2.65659919911432\\
57	0.193	1.15263735041756\\
};
\addplot3[only marks,mark=*,mark options={},mark size=1.5000pt,color=black] plot table[row sep=crcr,]{%
69	0.104	0.308894496859802\\
};
\addplot3[only marks,mark=*,mark options={},mark size=1.5000pt,color=black] plot table[row sep=crcr,]{%
64	0.23	4.75084671442933\\
};

\addplot3[%
surf,
opacity=0.7,
shader=interp,
colormap={mymap}{[1pt] rgb(0pt)=(0.0901961,0.239216,0.0745098); rgb(1pt)=(0.0945149,0.242058,0.0739522); rgb(2pt)=(0.0988592,0.244894,0.0733566); rgb(3pt)=(0.103229,0.247724,0.0727241); rgb(4pt)=(0.107623,0.250549,0.0720557); rgb(5pt)=(0.112043,0.253367,0.0713525); rgb(6pt)=(0.116487,0.25618,0.0706154); rgb(7pt)=(0.120956,0.258986,0.0698456); rgb(8pt)=(0.125449,0.261787,0.0690441); rgb(9pt)=(0.129967,0.264581,0.0682118); rgb(10pt)=(0.134508,0.26737,0.06735); rgb(11pt)=(0.139074,0.270152,0.0664596); rgb(12pt)=(0.143663,0.272929,0.0655416); rgb(13pt)=(0.148275,0.275699,0.0645971); rgb(14pt)=(0.152911,0.278463,0.0636271); rgb(15pt)=(0.15757,0.281221,0.0626328); rgb(16pt)=(0.162252,0.283973,0.0616151); rgb(17pt)=(0.166957,0.286719,0.060575); rgb(18pt)=(0.171685,0.289458,0.0595136); rgb(19pt)=(0.176434,0.292191,0.0584321); rgb(20pt)=(0.181207,0.294918,0.0573313); rgb(21pt)=(0.186001,0.297639,0.0562123); rgb(22pt)=(0.190817,0.300353,0.0550763); rgb(23pt)=(0.195655,0.303061,0.0539242); rgb(24pt)=(0.200514,0.305763,0.052757); rgb(25pt)=(0.205395,0.308459,0.0515759); rgb(26pt)=(0.210296,0.311149,0.0503624); rgb(27pt)=(0.215212,0.313846,0.0490067); rgb(28pt)=(0.220142,0.316548,0.0475043); rgb(29pt)=(0.22509,0.319254,0.0458704); rgb(30pt)=(0.230056,0.321962,0.0441205); rgb(31pt)=(0.235042,0.324671,0.04227); rgb(32pt)=(0.240048,0.327379,0.0403343); rgb(33pt)=(0.245078,0.330085,0.0383287); rgb(34pt)=(0.250131,0.332786,0.0362688); rgb(35pt)=(0.25521,0.335482,0.0341698); rgb(36pt)=(0.260317,0.33817,0.0320472); rgb(37pt)=(0.265451,0.340849,0.0299163); rgb(38pt)=(0.270616,0.343517,0.0277927); rgb(39pt)=(0.275813,0.346172,0.0256916); rgb(40pt)=(0.281043,0.348814,0.0236284); rgb(41pt)=(0.286307,0.35144,0.0216186); rgb(42pt)=(0.291607,0.354048,0.0196776); rgb(43pt)=(0.296945,0.356637,0.0178207); rgb(44pt)=(0.302322,0.359206,0.0160634); rgb(45pt)=(0.307739,0.361753,0.0144211); rgb(46pt)=(0.313198,0.364275,0.0129091); rgb(47pt)=(0.318701,0.366772,0.0115428); rgb(48pt)=(0.324249,0.369242,0.0103377); rgb(49pt)=(0.329843,0.371682,0.00930909); rgb(50pt)=(0.335485,0.374093,0.00847245); rgb(51pt)=(0.341176,0.376471,0.00784314); rgb(52pt)=(0.346925,0.378826,0.00732741); rgb(53pt)=(0.352735,0.381168,0.00682184); rgb(54pt)=(0.358605,0.383497,0.00632729); rgb(55pt)=(0.364532,0.385812,0.00584464); rgb(56pt)=(0.370516,0.388113,0.00537476); rgb(57pt)=(0.376552,0.390399,0.00491852); rgb(58pt)=(0.38264,0.39267,0.00447681); rgb(59pt)=(0.388777,0.394925,0.00405048); rgb(60pt)=(0.394962,0.397164,0.00364042); rgb(61pt)=(0.401191,0.399386,0.00324749); rgb(62pt)=(0.407464,0.401592,0.00287258); rgb(63pt)=(0.413777,0.40378,0.00251655); rgb(64pt)=(0.420129,0.40595,0.00218028); rgb(65pt)=(0.426518,0.408102,0.00186463); rgb(66pt)=(0.432942,0.410234,0.00157049); rgb(67pt)=(0.439399,0.412348,0.00129873); rgb(68pt)=(0.445885,0.414441,0.00105022); rgb(69pt)=(0.452401,0.416515,0.000825833); rgb(70pt)=(0.458942,0.418567,0.000626441); rgb(71pt)=(0.465508,0.420599,0.00045292); rgb(72pt)=(0.472096,0.422609,0.000306141); rgb(73pt)=(0.478704,0.424596,0.000186979); rgb(74pt)=(0.485331,0.426562,9.63073e-05); rgb(75pt)=(0.491973,0.428504,3.49981e-05); rgb(76pt)=(0.498628,0.430422,3.92506e-06); rgb(77pt)=(0.505323,0.432315,0); rgb(78pt)=(0.512206,0.434168,0); rgb(79pt)=(0.519282,0.435983,0); rgb(80pt)=(0.526529,0.437764,0); rgb(81pt)=(0.533922,0.439512,0); rgb(82pt)=(0.54144,0.441232,0); rgb(83pt)=(0.549059,0.442927,0); rgb(84pt)=(0.556756,0.444599,0); rgb(85pt)=(0.564508,0.446252,0); rgb(86pt)=(0.572292,0.447889,0); rgb(87pt)=(0.580084,0.449514,0); rgb(88pt)=(0.587863,0.451129,0); rgb(89pt)=(0.595604,0.452737,0); rgb(90pt)=(0.603284,0.454343,0); rgb(91pt)=(0.610882,0.455948,0); rgb(92pt)=(0.618373,0.457556,0); rgb(93pt)=(0.625734,0.459171,0); rgb(94pt)=(0.632943,0.460795,0); rgb(95pt)=(0.639976,0.462432,0); rgb(96pt)=(0.64681,0.464084,0); rgb(97pt)=(0.653423,0.465756,0); rgb(98pt)=(0.659791,0.46745,0); rgb(99pt)=(0.665891,0.469169,0); rgb(100pt)=(0.6717,0.470916,0); rgb(101pt)=(0.677195,0.472696,0); rgb(102pt)=(0.682353,0.47451,0); rgb(103pt)=(0.687242,0.476355,0); rgb(104pt)=(0.691952,0.478225,0); rgb(105pt)=(0.696497,0.480118,0); rgb(106pt)=(0.700887,0.482033,0); rgb(107pt)=(0.705134,0.483968,0); rgb(108pt)=(0.709251,0.485921,0); rgb(109pt)=(0.713249,0.487891,0); rgb(110pt)=(0.71714,0.489876,0); rgb(111pt)=(0.720936,0.491875,0); rgb(112pt)=(0.724649,0.493887,0); rgb(113pt)=(0.72829,0.495909,0); rgb(114pt)=(0.731872,0.49794,0); rgb(115pt)=(0.735406,0.499979,0); rgb(116pt)=(0.738904,0.502025,0); rgb(117pt)=(0.742378,0.504075,0); rgb(118pt)=(0.74584,0.506128,0); rgb(119pt)=(0.749302,0.508182,0); rgb(120pt)=(0.752775,0.510237,0); rgb(121pt)=(0.756272,0.51229,0); rgb(122pt)=(0.759804,0.514339,0); rgb(123pt)=(0.763384,0.516385,0); rgb(124pt)=(0.767022,0.518424,0); rgb(125pt)=(0.770731,0.520455,0); rgb(126pt)=(0.774523,0.522478,0); rgb(127pt)=(0.77841,0.524489,0); rgb(128pt)=(0.782391,0.526491,0); rgb(129pt)=(0.786402,0.528496,0); rgb(130pt)=(0.790431,0.530506,0); rgb(131pt)=(0.794478,0.532521,0); rgb(132pt)=(0.798541,0.534539,0); rgb(133pt)=(0.802619,0.53656,0); rgb(134pt)=(0.806712,0.538584,0); rgb(135pt)=(0.81082,0.540609,0); rgb(136pt)=(0.81494,0.542635,0); rgb(137pt)=(0.819074,0.54466,0); rgb(138pt)=(0.823219,0.546686,0); rgb(139pt)=(0.827374,0.548709,0); rgb(140pt)=(0.831541,0.55073,0); rgb(141pt)=(0.835716,0.552749,0); rgb(142pt)=(0.8399,0.554763,0); rgb(143pt)=(0.844092,0.556774,0); rgb(144pt)=(0.848292,0.558779,0); rgb(145pt)=(0.852497,0.560778,0); rgb(146pt)=(0.856708,0.562771,0); rgb(147pt)=(0.860924,0.564756,0); rgb(148pt)=(0.865143,0.566733,0); rgb(149pt)=(0.869366,0.568701,0); rgb(150pt)=(0.873592,0.57066,0); rgb(151pt)=(0.877819,0.572608,0); rgb(152pt)=(0.882047,0.574545,0); rgb(153pt)=(0.886275,0.576471,0); rgb(154pt)=(0.890659,0.578362,0); rgb(155pt)=(0.895333,0.580203,0); rgb(156pt)=(0.900258,0.581999,0); rgb(157pt)=(0.905397,0.583755,0); rgb(158pt)=(0.910711,0.585479,0); rgb(159pt)=(0.916164,0.587176,0); rgb(160pt)=(0.921717,0.588852,0); rgb(161pt)=(0.927333,0.590513,0); rgb(162pt)=(0.932974,0.592166,0); rgb(163pt)=(0.938602,0.593815,0); rgb(164pt)=(0.94418,0.595468,0); rgb(165pt)=(0.949669,0.59713,0); rgb(166pt)=(0.955033,0.598808,0); rgb(167pt)=(0.960233,0.600507,0); rgb(168pt)=(0.965232,0.602233,0); rgb(169pt)=(0.969992,0.603992,0); rgb(170pt)=(0.974475,0.605791,0); rgb(171pt)=(0.978643,0.607636,0); rgb(172pt)=(0.98246,0.609532,0); rgb(173pt)=(0.985886,0.611486,0); rgb(174pt)=(0.988885,0.613503,0); rgb(175pt)=(0.991419,0.61559,0); rgb(176pt)=(0.99345,0.617753,0); rgb(177pt)=(0.99494,0.619997,0); rgb(178pt)=(0.995851,0.622329,0); rgb(179pt)=(0.996226,0.624763,0); rgb(180pt)=(0.996512,0.627352,0); rgb(181pt)=(0.996788,0.630095,0); rgb(182pt)=(0.997053,0.632982,0); rgb(183pt)=(0.997308,0.636004,0); rgb(184pt)=(0.997552,0.639152,0); rgb(185pt)=(0.997785,0.642416,0); rgb(186pt)=(0.998006,0.645786,0); rgb(187pt)=(0.998217,0.649253,0); rgb(188pt)=(0.998416,0.652807,0); rgb(189pt)=(0.998605,0.656439,0); rgb(190pt)=(0.998781,0.660138,0); rgb(191pt)=(0.998946,0.663897,0); rgb(192pt)=(0.9991,0.667704,0); rgb(193pt)=(0.999242,0.67155,0); rgb(194pt)=(0.999372,0.675427,0); rgb(195pt)=(0.99949,0.679323,0); rgb(196pt)=(0.999596,0.68323,0); rgb(197pt)=(0.99969,0.687139,0); rgb(198pt)=(0.999771,0.691039,0); rgb(199pt)=(0.999841,0.694921,0); rgb(200pt)=(0.999898,0.698775,0); rgb(201pt)=(0.999942,0.702592,0); rgb(202pt)=(0.999974,0.706363,0); rgb(203pt)=(0.999994,0.710077,0); rgb(204pt)=(1,0.713725,0); rgb(205pt)=(1,0.717341,0); rgb(206pt)=(1,0.720963,0); rgb(207pt)=(1,0.724591,0); rgb(208pt)=(1,0.728226,0); rgb(209pt)=(1,0.731867,0); rgb(210pt)=(1,0.735514,0); rgb(211pt)=(1,0.739167,0); rgb(212pt)=(1,0.742827,0); rgb(213pt)=(1,0.746493,0); rgb(214pt)=(1,0.750165,0); rgb(215pt)=(1,0.753843,0); rgb(216pt)=(1,0.757527,0); rgb(217pt)=(1,0.761217,0); rgb(218pt)=(1,0.764913,0); rgb(219pt)=(1,0.768615,0); rgb(220pt)=(1,0.772324,0); rgb(221pt)=(1,0.776038,0); rgb(222pt)=(1,0.779758,0); rgb(223pt)=(1,0.783484,0); rgb(224pt)=(1,0.787215,0); rgb(225pt)=(1,0.790953,0); rgb(226pt)=(1,0.794696,0); rgb(227pt)=(1,0.798445,0); rgb(228pt)=(1,0.8022,0); rgb(229pt)=(1,0.805961,0); rgb(230pt)=(1,0.809727,0); rgb(231pt)=(1,0.8135,0); rgb(232pt)=(1,0.817278,0); rgb(233pt)=(1,0.821063,0); rgb(234pt)=(1,0.824854,0); rgb(235pt)=(1,0.828652,0); rgb(236pt)=(1,0.832455,0); rgb(237pt)=(1,0.836265,0); rgb(238pt)=(1,0.840081,0); rgb(239pt)=(1,0.843903,0); rgb(240pt)=(1,0.847732,0); rgb(241pt)=(1,0.851566,0); rgb(242pt)=(1,0.855406,0); rgb(243pt)=(1,0.859253,0); rgb(244pt)=(1,0.863106,0); rgb(245pt)=(1,0.866964,0); rgb(246pt)=(1,0.870829,0); rgb(247pt)=(1,0.8747,0); rgb(248pt)=(1,0.878577,0); rgb(249pt)=(1,0.88246,0); rgb(250pt)=(1,0.886349,0); rgb(251pt)=(1,0.890243,0); rgb(252pt)=(1,0.894144,0); rgb(253pt)=(1,0.898051,0); rgb(254pt)=(1,0.901964,0); rgb(255pt)=(1,0.905882,0)},
mesh/rows=49]
table[row sep=crcr,header=false] {%
%
56	0.093	0.414666213324685\\
56	0.09666	0.37013123446904\\
56	0.10032	0.32991630850448\\
56	0.10398	0.294021435431006\\
56	0.10764	0.262446615248612\\
56	0.1113	0.235191847957301\\
56	0.11496	0.212257133557077\\
56	0.11862	0.193642472047932\\
56	0.12228	0.179347863429871\\
56	0.12594	0.169373307702895\\
56	0.1296	0.163718804867\\
56	0.13326	0.162384354922187\\
56	0.13692	0.165369957868462\\
56	0.14058	0.172675613705816\\
56	0.14424	0.184301322434254\\
56	0.1479	0.200247084053774\\
56	0.15156	0.22051289856438\\
56	0.15522	0.245098765966067\\
56	0.15888	0.274004686258834\\
56	0.16254	0.307230659442694\\
56	0.1662	0.34477668551763\\
56	0.16986	0.386642764483645\\
56	0.17352	0.432828896340753\\
56	0.17718	0.483335081088939\\
56	0.18084	0.538161318728203\\
56	0.1845	0.597307609258562\\
56	0.18816	0.660773952679993\\
56	0.19182	0.728560348992515\\
56	0.19548	0.800666798196115\\
56	0.19914	0.877093300290799\\
56	0.2028	0.957839855276571\\
56	0.20646	1.04290646315342\\
56	0.21012	1.13229312392135\\
56	0.21378	1.22599983758037\\
56	0.21744	1.32402660413047\\
56	0.2211	1.42637342357165\\
56	0.22476	1.53304029590393\\
56	0.22842	1.64402722112727\\
56	0.23208	1.7593341992417\\
56	0.23574	1.87896123024722\\
56	0.2394	2.00290831414382\\
56	0.24306	2.1311754509315\\
56	0.24672	2.26376264061027\\
56	0.25038	2.40066988318012\\
56	0.25404	2.54189717864105\\
56	0.2577	2.68744452699306\\
56	0.26136	2.83731192823616\\
56	0.26502	2.99149938237034\\
56	0.26868	3.1500068893956\\
56	0.27234	3.31283444931195\\
56	0.276	3.47998206211939\\
56.375	0.093	0.420828723367943\\
56.375	0.09666	0.380241934720889\\
56.375	0.10032	0.343975198964918\\
56.375	0.10398	0.312028516100034\\
56.375	0.10764	0.284401886126226\\
56.375	0.1113	0.261095309043508\\
56.375	0.11496	0.24210878485187\\
56.375	0.11862	0.227442313551318\\
56.375	0.12228	0.217095895141846\\
56.375	0.12594	0.211069529623457\\
56.375	0.1296	0.209363216996155\\
56.375	0.13326	0.211976957259932\\
56.375	0.13692	0.218910750414792\\
56.375	0.14058	0.230164596460737\\
56.375	0.14424	0.245738495397767\\
56.375	0.1479	0.265632447225877\\
56.375	0.15156	0.289846451945071\\
56.375	0.15522	0.318380509555348\\
56.375	0.15888	0.351234620056704\\
56.375	0.16254	0.388408783449154\\
56.375	0.1662	0.429902999732676\\
56.375	0.16986	0.475717268907283\\
56.375	0.17352	0.525851590972982\\
56.375	0.17718	0.580305965929753\\
56.375	0.18084	0.63908039377761\\
56.375	0.1845	0.702174874516559\\
56.375	0.18816	0.769589408146577\\
56.375	0.19182	0.841323994667691\\
56.375	0.19548	0.917378634079881\\
56.375	0.19914	0.997753326383155\\
56.375	0.2028	1.08244807157752\\
56.375	0.20646	1.17146286966295\\
56.375	0.21012	1.26479772063948\\
56.375	0.21378	1.36245262450709\\
56.375	0.21744	1.46442758126577\\
56.375	0.2211	1.57072259091555\\
56.375	0.22476	1.68133765345641\\
56.375	0.22842	1.79627276888834\\
56.375	0.23208	1.91552793721137\\
56.375	0.23574	2.03910315842547\\
56.375	0.2394	2.16699843253066\\
56.375	0.24306	2.29921375952694\\
56.375	0.24672	2.43574913941429\\
56.375	0.25038	2.57660457219273\\
56.375	0.25404	2.72178005786225\\
56.375	0.2577	2.87127559642285\\
56.375	0.26136	3.02509118787454\\
56.375	0.26502	3.18322683221731\\
56.375	0.26868	3.34568252945116\\
56.375	0.27234	3.5124582795761\\
56.375	0.276	3.68355408259212\\
56.75	0.093	0.425946026928677\\
56.75	0.09666	0.389307428490212\\
56.75	0.10032	0.356988882942833\\
56.75	0.10398	0.328990390286535\\
56.75	0.10764	0.30531195052132\\
56.75	0.1113	0.285953563647189\\
56.75	0.11496	0.270915229664144\\
56.75	0.11862	0.260196948572181\\
56.75	0.12228	0.253798720371296\\
56.75	0.12594	0.2517205450615\\
56.75	0.1296	0.253962422642787\\
56.75	0.13326	0.26052435311515\\
56.75	0.13692	0.271406336478603\\
56.75	0.14058	0.286608372733137\\
56.75	0.14424	0.306130461878757\\
56.75	0.1479	0.329972603915457\\
56.75	0.15156	0.358134798843239\\
56.75	0.15522	0.390617046662109\\
56.75	0.15888	0.427419347372055\\
56.75	0.16254	0.468541700973094\\
56.75	0.1662	0.513984107465205\\
56.75	0.16986	0.563746566848403\\
56.75	0.17352	0.617829079122691\\
56.75	0.17718	0.676231644288052\\
56.75	0.18084	0.738954262344499\\
56.75	0.1845	0.805996933292037\\
56.75	0.18816	0.877359657130648\\
56.75	0.19182	0.953042433860348\\
56.75	0.19548	1.03304526348113\\
56.75	0.19914	1.11736814599299\\
56.75	0.2028	1.20601108139594\\
56.75	0.20646	1.29897406968997\\
56.75	0.21012	1.39625711087509\\
56.75	0.21378	1.49786020495128\\
56.75	0.21744	1.60378335191856\\
56.75	0.2211	1.71402655177693\\
56.75	0.22476	1.82858980452637\\
56.75	0.22842	1.9474731101669\\
56.75	0.23208	2.07067646869852\\
56.75	0.23574	2.1981998801212\\
56.75	0.2394	2.33004334443498\\
56.75	0.24306	2.46620686163985\\
56.75	0.24672	2.60669043173579\\
56.75	0.25038	2.75149405472283\\
56.75	0.25404	2.90061773060094\\
56.75	0.2577	3.05406145937013\\
56.75	0.26136	3.2118252410304\\
56.75	0.26502	3.37390907558176\\
56.75	0.26868	3.5403129630242\\
56.75	0.27234	3.71103690335774\\
56.75	0.276	3.88608089658235\\
57.125	0.093	0.430018124006879\\
57.125	0.09666	0.397327715777004\\
57.125	0.10032	0.368957360438213\\
57.125	0.10398	0.344907057990508\\
57.125	0.10764	0.325176808433883\\
57.125	0.1113	0.309766611768341\\
57.125	0.11496	0.298676467993885\\
57.125	0.11862	0.291906377110509\\
57.125	0.12228	0.289456339118217\\
57.125	0.12594	0.29132635401701\\
57.125	0.1296	0.297516421806883\\
57.125	0.13326	0.308026542487839\\
57.125	0.13692	0.322856716059883\\
57.125	0.14058	0.342006942523006\\
57.125	0.14424	0.365477221877216\\
57.125	0.1479	0.393267554122501\\
57.125	0.15156	0.425377939258875\\
57.125	0.15522	0.461808377286338\\
57.125	0.15888	0.502558868204874\\
57.125	0.16254	0.547629412014499\\
57.125	0.1662	0.597020008715203\\
57.125	0.16986	0.650730658306991\\
57.125	0.17352	0.708761360789865\\
57.125	0.17718	0.771112116163819\\
57.125	0.18084	0.837782924428855\\
57.125	0.1845	0.908773785584979\\
57.125	0.18816	0.98408469963218\\
57.125	0.19182	1.06371566657047\\
57.125	0.19548	1.14766668639984\\
57.125	0.19914	1.23593775912029\\
57.125	0.2028	1.32852888473183\\
57.125	0.20646	1.42544006323446\\
57.125	0.21012	1.52667129462816\\
57.125	0.21378	1.63222257891294\\
57.125	0.21744	1.74209391608881\\
57.125	0.2211	1.85628530615577\\
57.125	0.22476	1.9747967491138\\
57.125	0.22842	2.09762824496292\\
57.125	0.23208	2.22477979370312\\
57.125	0.23574	2.35625139533441\\
57.125	0.2394	2.49204304985677\\
57.125	0.24306	2.63215475727023\\
57.125	0.24672	2.77658651757476\\
57.125	0.25038	2.92533833077038\\
57.125	0.25404	3.07841019685708\\
57.125	0.2577	3.23580211583486\\
57.125	0.26136	3.39751408770373\\
57.125	0.26502	3.56354611246368\\
57.125	0.26868	3.73389819011471\\
57.125	0.27234	3.90857032065683\\
57.125	0.276	4.08756250409002\\
57.5	0.093	0.433045014602555\\
57.5	0.09666	0.40430279658127\\
57.5	0.10032	0.37988063145107\\
57.5	0.10398	0.35977851921195\\
57.5	0.10764	0.343996459863915\\
57.5	0.1113	0.332534453406966\\
57.5	0.11496	0.325392499841097\\
57.5	0.11862	0.322570599166314\\
57.5	0.12228	0.324068751382611\\
57.5	0.12594	0.32988695648999\\
57.5	0.1296	0.340025214488457\\
57.5	0.13326	0.354483525378003\\
57.5	0.13692	0.373261889158635\\
57.5	0.14058	0.396360305830349\\
57.5	0.14424	0.423778775393144\\
57.5	0.1479	0.455517297847023\\
57.5	0.15156	0.491575873191988\\
57.5	0.15522	0.531954501428037\\
57.5	0.15888	0.576653182555162\\
57.5	0.16254	0.625671916573377\\
57.5	0.1662	0.679010703482671\\
57.5	0.16986	0.736669543283048\\
57.5	0.17352	0.798648435974512\\
57.5	0.17718	0.864947381557055\\
57.5	0.18084	0.935566380030681\\
57.5	0.1845	1.0105054313954\\
57.5	0.18816	1.08976453565119\\
57.5	0.19182	1.17334369279807\\
57.5	0.19548	1.26124290283603\\
57.5	0.19914	1.35346216576507\\
57.5	0.2028	1.4500014815852\\
57.5	0.20646	1.55086085029641\\
57.5	0.21012	1.65604027189871\\
57.5	0.21378	1.76553974639208\\
57.5	0.21744	1.87935927377653\\
57.5	0.2211	1.99749885405208\\
57.5	0.22476	2.1199584872187\\
57.5	0.22842	2.24673817327641\\
57.5	0.23208	2.37783791222521\\
57.5	0.23574	2.51325770406508\\
57.5	0.2394	2.65299754879603\\
57.5	0.24306	2.79705744641808\\
57.5	0.24672	2.9454373969312\\
57.5	0.25038	3.09813740033541\\
57.5	0.25404	3.2551574566307\\
57.5	0.2577	3.41649756581707\\
57.5	0.26136	3.58215772789453\\
57.5	0.26502	3.75213794286307\\
57.5	0.26868	3.92643821072269\\
57.5	0.27234	4.1050585314734\\
57.5	0.276	4.28799890511518\\
57.875	0.093	0.435026698715704\\
57.875	0.09666	0.410232670903006\\
57.875	0.10032	0.389758695981394\\
57.875	0.10398	0.373604773950868\\
57.875	0.10764	0.361770904811422\\
57.875	0.1113	0.35425708856306\\
57.875	0.11496	0.351063325205784\\
57.875	0.11862	0.35218961473959\\
57.875	0.12228	0.357635957164473\\
57.875	0.12594	0.367402352480446\\
57.875	0.1296	0.381488800687502\\
57.875	0.13326	0.399895301785637\\
57.875	0.13692	0.422621855774856\\
57.875	0.14058	0.449668462655159\\
57.875	0.14424	0.481035122426548\\
57.875	0.1479	0.516721835089016\\
57.875	0.15156	0.556728600642571\\
57.875	0.15522	0.601055419087206\\
57.875	0.15888	0.649702290422921\\
57.875	0.16254	0.702669214649728\\
57.875	0.1662	0.759956191767612\\
57.875	0.16986	0.821563221776575\\
57.875	0.17352	0.887490304676632\\
57.875	0.17718	0.957737440467765\\
57.875	0.18084	1.03230462914998\\
57.875	0.1845	1.11119187072328\\
57.875	0.18816	1.19439916518766\\
57.875	0.19182	1.28192651254314\\
57.875	0.19548	1.37377391278968\\
57.875	0.19914	1.46994136592732\\
57.875	0.2028	1.57042887195604\\
57.875	0.20646	1.67523643087583\\
57.875	0.21012	1.78436404268672\\
57.875	0.21378	1.89781170738868\\
57.875	0.21744	2.01557942498173\\
57.875	0.2211	2.13766719546587\\
57.875	0.22476	2.26407501884108\\
57.875	0.22842	2.39480289510737\\
57.875	0.23208	2.52985082426477\\
57.875	0.23574	2.66921880631322\\
57.875	0.2394	2.81290684125276\\
57.875	0.24306	2.9609149290834\\
57.875	0.24672	3.11324306980512\\
57.875	0.25038	3.26989126341792\\
57.875	0.25404	3.43085950992179\\
57.875	0.2577	3.59614780931676\\
57.875	0.26136	3.7657561616028\\
57.875	0.26502	3.93968456677992\\
57.875	0.26868	4.11793302484814\\
57.875	0.27234	4.30050153580744\\
57.875	0.276	4.48739009965782\\
58.25	0.093	0.435963176346323\\
58.25	0.09666	0.415117338742217\\
58.25	0.10032	0.398591554029194\\
58.25	0.10398	0.386385822207258\\
58.25	0.10764	0.378500143276398\\
58.25	0.1113	0.374934517236629\\
58.25	0.11496	0.375688944087942\\
58.25	0.11862	0.380763423830334\\
58.25	0.12228	0.390157956463811\\
58.25	0.12594	0.403872541988373\\
58.25	0.1296	0.421907180404018\\
58.25	0.13326	0.44426187171074\\
58.25	0.13692	0.470936615908552\\
58.25	0.14058	0.501931412997445\\
58.25	0.14424	0.537246262977423\\
58.25	0.1479	0.576881165848481\\
58.25	0.15156	0.620836121610622\\
58.25	0.15522	0.669111130263851\\
58.25	0.15888	0.721706191808156\\
58.25	0.16254	0.778621306243553\\
58.25	0.1662	0.839856473570027\\
58.25	0.16986	0.905411693787583\\
58.25	0.17352	0.975286966896226\\
58.25	0.17718	1.04948229289595\\
58.25	0.18084	1.12799767178675\\
58.25	0.1845	1.21083310356865\\
58.25	0.18816	1.29798858824162\\
58.25	0.19182	1.38946412580568\\
58.25	0.19548	1.48525971626082\\
58.25	0.19914	1.58537535960703\\
58.25	0.2028	1.68981105584434\\
58.25	0.20646	1.79856680497274\\
58.25	0.21012	1.91164260699221\\
58.25	0.21378	2.02903846190277\\
58.25	0.21744	2.1507543697044\\
58.25	0.2211	2.27679033039712\\
58.25	0.22476	2.40714634398093\\
58.25	0.22842	2.54182241045581\\
58.25	0.23208	2.68081852982179\\
58.25	0.23574	2.82413470207885\\
58.25	0.2394	2.97177092722698\\
58.25	0.24306	3.1237272052662\\
58.25	0.24672	3.2800035361965\\
58.25	0.25038	3.4405999200179\\
58.25	0.25404	3.60551635673036\\
58.25	0.2577	3.77475284633391\\
58.25	0.26136	3.94830938882855\\
58.25	0.26502	4.12618598421427\\
58.25	0.26868	4.30838263249107\\
58.25	0.27234	4.49489933365896\\
58.25	0.276	4.68573608771792\\
58.625	0.093	0.435854447494409\\
58.625	0.09666	0.418956800098891\\
58.625	0.10032	0.406379205594462\\
58.625	0.10398	0.398121663981112\\
58.625	0.10764	0.394184175258845\\
58.625	0.1113	0.394566739427665\\
58.625	0.11496	0.399269356487565\\
58.625	0.11862	0.40829202643855\\
58.625	0.12228	0.421634749280616\\
58.625	0.12594	0.439297525013768\\
58.625	0.1296	0.461280353638\\
58.625	0.13326	0.487583235153314\\
58.625	0.13692	0.518206169559716\\
58.625	0.14058	0.553149156857198\\
58.625	0.14424	0.592412197045766\\
58.625	0.1479	0.63599529012541\\
58.625	0.15156	0.683898436096144\\
58.625	0.15522	0.736121634957962\\
58.625	0.15888	0.792664886710856\\
58.625	0.16254	0.853528191354843\\
58.625	0.1662	0.918711548889906\\
58.625	0.16986	0.988214959316048\\
58.625	0.17352	1.06203842263328\\
58.625	0.17718	1.1401819388416\\
58.625	0.18084	1.22264550794099\\
58.625	0.1845	1.30942912993147\\
58.625	0.18816	1.40053280481303\\
58.625	0.19182	1.49595653258569\\
58.625	0.19548	1.59570031324941\\
58.625	0.19914	1.69976414680422\\
58.625	0.2028	1.80814803325012\\
58.625	0.20646	1.9208519725871\\
58.625	0.21012	2.03787596481516\\
58.625	0.21378	2.1592200099343\\
58.625	0.21744	2.28488410794453\\
58.625	0.2211	2.41486825884585\\
58.625	0.22476	2.54917246263824\\
58.625	0.22842	2.68779671932172\\
58.625	0.23208	2.83074102889628\\
58.625	0.23574	2.97800539136192\\
58.625	0.2394	3.12958980671865\\
58.625	0.24306	3.28549427496646\\
58.625	0.24672	3.44571879610536\\
58.625	0.25038	3.61026337013534\\
58.625	0.25404	3.77912799705639\\
58.625	0.2577	3.95231267686853\\
58.625	0.26136	4.12981740957176\\
58.625	0.26502	4.31164219516606\\
58.625	0.26868	4.49778703365145\\
58.625	0.27234	4.68825192502793\\
58.625	0.276	4.88303686929549\\
59	0.093	0.434700512159973\\
59	0.09666	0.421751054973047\\
59	0.10032	0.413121650677204\\
59	0.10398	0.408812299272446\\
59	0.10764	0.408823000758769\\
59	0.1113	0.413153755136175\\
59	0.11496	0.421804562404668\\
59	0.11862	0.434775422564243\\
59	0.12228	0.452066335614895\\
59	0.12594	0.473677301556637\\
59	0.1296	0.499608320389461\\
59	0.13326	0.529859392113366\\
59	0.13692	0.564430516728357\\
59	0.14058	0.603321694234429\\
59	0.14424	0.646532924631587\\
59	0.1479	0.69406420791982\\
59	0.15156	0.745915544099144\\
59	0.15522	0.802086933169549\\
59	0.15888	0.862578375131033\\
59	0.16254	0.927389869983609\\
59	0.1662	0.996521417727262\\
59	0.16986	1.06997301836199\\
59	0.17352	1.14774467188782\\
59	0.17718	1.22983637830472\\
59	0.18084	1.31624813761271\\
59	0.1845	1.40697994981178\\
59	0.18816	1.50203181490193\\
59	0.19182	1.60140373288317\\
59	0.19548	1.70509570375549\\
59	0.19914	1.81310772751889\\
59	0.2028	1.92543980417338\\
59	0.20646	2.04209193371894\\
59	0.21012	2.1630641161556\\
59	0.21378	2.28835635148333\\
59	0.21744	2.41796863970215\\
59	0.2211	2.55190098081205\\
59	0.22476	2.69015337481303\\
59	0.22842	2.8327258217051\\
59	0.23208	2.97961832148825\\
59	0.23574	3.13083087416248\\
59	0.2394	3.2863634797278\\
59	0.24306	3.4462161381842\\
59	0.24672	3.61038884953168\\
59	0.25038	3.77888161377025\\
59	0.25404	3.9516944308999\\
59	0.2577	4.12882730092063\\
59	0.26136	4.31028022383245\\
59	0.26502	4.49605319963534\\
59	0.26868	4.68614622832932\\
59	0.27234	4.88055930991439\\
59	0.276	5.07929244439054\\
59.375	0.093	0.432501370343004\\
59.375	0.09666	0.423500103364665\\
59.375	0.10032	0.418818889277415\\
59.375	0.10398	0.418457728081247\\
59.375	0.10764	0.422416619776156\\
59.375	0.1113	0.430695564362156\\
59.375	0.11496	0.443294561839238\\
59.375	0.11862	0.460213612207399\\
59.375	0.12228	0.481452715466644\\
59.375	0.12594	0.507011871616975\\
59.375	0.1296	0.53689108065839\\
59.375	0.13326	0.571090342590884\\
59.375	0.13692	0.609609657414461\\
59.375	0.14058	0.652449025129122\\
59.375	0.14424	0.699608445734869\\
59.375	0.1479	0.751087919231696\\
59.375	0.15156	0.806887445619609\\
59.375	0.15522	0.867007024898606\\
59.375	0.15888	0.93144665706868\\
59.375	0.16254	1.00020634212984\\
59.375	0.1662	1.07328608008208\\
59.375	0.16986	1.15068587092541\\
59.375	0.17352	1.23240571465982\\
59.375	0.17718	1.31844561128532\\
59.375	0.18084	1.40880556080189\\
59.375	0.1845	1.50348556320955\\
59.375	0.18816	1.60248561850829\\
59.375	0.19182	1.70580572669812\\
59.375	0.19548	1.81344588777903\\
59.375	0.19914	1.92540610175102\\
59.375	0.2028	2.0416863686141\\
59.375	0.20646	2.16228668836825\\
59.375	0.21012	2.2872070610135\\
59.375	0.21378	2.41644748654982\\
59.375	0.21744	2.55000796497722\\
59.375	0.2211	2.68788849629572\\
59.375	0.22476	2.83008908050529\\
59.375	0.22842	2.97660971760595\\
59.375	0.23208	3.12745040759769\\
59.375	0.23574	3.28261115048051\\
59.375	0.2394	3.44209194625442\\
59.375	0.24306	3.60589279491941\\
59.375	0.24672	3.77401369647547\\
59.375	0.25038	3.94645465092263\\
59.375	0.25404	4.12321565826087\\
59.375	0.2577	4.30429671849019\\
59.375	0.26136	4.48969783161061\\
59.375	0.26502	4.67941899762209\\
59.375	0.26868	4.87346021652466\\
59.375	0.27234	5.07182148831831\\
59.375	0.276	5.27450281300305\\
59.75	0.093	0.429257022043504\\
59.75	0.09666	0.424203945273756\\
59.75	0.10032	0.423470921395092\\
59.75	0.10398	0.427057950407514\\
59.75	0.10764	0.434965032311016\\
59.75	0.1113	0.447192167105605\\
59.75	0.11496	0.463739354791274\\
59.75	0.11862	0.484606595368028\\
59.75	0.12228	0.509793888835863\\
59.75	0.12594	0.539301235194784\\
59.75	0.1296	0.573128634444784\\
59.75	0.13326	0.611276086585868\\
59.75	0.13692	0.653743591618038\\
59.75	0.14058	0.700531149541289\\
59.75	0.14424	0.751638760355626\\
59.75	0.1479	0.807066424061042\\
59.75	0.15156	0.866814140657545\\
59.75	0.15522	0.930881910145129\\
59.75	0.15888	0.999269732523793\\
59.75	0.16254	1.07197760779355\\
59.75	0.1662	1.14900553595438\\
59.75	0.16986	1.23035351700629\\
59.75	0.17352	1.3160215509493\\
59.75	0.17718	1.40600963778338\\
59.75	0.18084	1.50031777750854\\
59.75	0.1845	1.5989459701248\\
59.75	0.18816	1.70189421563213\\
59.75	0.19182	1.80916251403054\\
59.75	0.19548	1.92075086532004\\
59.75	0.19914	2.03665926950062\\
59.75	0.2028	2.15688772657229\\
59.75	0.20646	2.28143623653503\\
59.75	0.21012	2.41030479938887\\
59.75	0.21378	2.54349341513378\\
59.75	0.21744	2.68100208376977\\
59.75	0.2211	2.82283080529686\\
59.75	0.22476	2.96897957971502\\
59.75	0.22842	3.11944840702427\\
59.75	0.23208	3.2742372872246\\
59.75	0.23574	3.43334622031601\\
59.75	0.2394	3.5967752062985\\
59.75	0.24306	3.76452424517209\\
59.75	0.24672	3.93659333693674\\
59.75	0.25038	4.11298248159249\\
59.75	0.25404	4.29369167913932\\
59.75	0.2577	4.47872092957723\\
59.75	0.26136	4.66807023290622\\
59.75	0.26502	4.8617395891263\\
59.75	0.26868	5.05972899823746\\
59.75	0.27234	5.26203846023971\\
59.75	0.276	5.46866797513303\\
60.125	0.093	0.424967467261482\\
60.125	0.09666	0.423862580700321\\
60.125	0.10032	0.42707774703025\\
60.125	0.10398	0.434612966251262\\
60.125	0.10764	0.446468238363354\\
60.125	0.1113	0.462643563366529\\
60.125	0.11496	0.48313894126079\\
60.125	0.11862	0.507954372046134\\
60.125	0.12228	0.537089855722559\\
60.125	0.12594	0.570545392290065\\
60.125	0.1296	0.608320981748659\\
60.125	0.13326	0.650416624098332\\
60.125	0.13692	0.696832319339092\\
60.125	0.14058	0.747568067470933\\
60.125	0.14424	0.802623868493859\\
60.125	0.1479	0.861999722407865\\
60.125	0.15156	0.925695629212954\\
60.125	0.15522	0.99371158890913\\
60.125	0.15888	1.06604760149638\\
60.125	0.16254	1.14270366697473\\
60.125	0.1662	1.22367978534415\\
60.125	0.16986	1.30897595660465\\
60.125	0.17352	1.39859218075625\\
60.125	0.17718	1.49252845779892\\
60.125	0.18084	1.59078478773267\\
60.125	0.1845	1.69336117055751\\
60.125	0.18816	1.80025760627343\\
60.125	0.19182	1.91147409488044\\
60.125	0.19548	2.02701063637853\\
60.125	0.19914	2.14686723076769\\
60.125	0.2028	2.27104387804795\\
60.125	0.20646	2.39954057821929\\
60.125	0.21012	2.53235733128171\\
60.125	0.21378	2.66949413723522\\
60.125	0.21744	2.8109509960798\\
60.125	0.2211	2.95672790781547\\
60.125	0.22476	3.10682487244223\\
60.125	0.22842	3.26124188996006\\
60.125	0.23208	3.41997896036898\\
60.125	0.23574	3.58303608366898\\
60.125	0.2394	3.75041325986006\\
60.125	0.24306	3.92211048894224\\
60.125	0.24672	4.09812777091549\\
60.125	0.25038	4.27846510577983\\
60.125	0.25404	4.46312249353524\\
60.125	0.2577	4.65209993418174\\
60.125	0.26136	4.84539742771933\\
60.125	0.26502	5.043014974148\\
60.125	0.26868	5.24495257346774\\
60.125	0.27234	5.45121022567858\\
60.125	0.276	5.66178793078049\\
60.5	0.093	0.419632705996928\\
60.5	0.09666	0.422476009644359\\
60.5	0.10032	0.429639366182878\\
60.5	0.10398	0.441122775612476\\
60.5	0.10764	0.456926237933157\\
60.5	0.1113	0.477049753144925\\
60.5	0.11496	0.501493321247776\\
60.5	0.11862	0.530256942241706\\
60.5	0.12228	0.56334061612672\\
60.5	0.12594	0.600744342902821\\
60.5	0.1296	0.642468122570004\\
60.5	0.13326	0.688511955128266\\
60.5	0.13692	0.738875840577616\\
60.5	0.14058	0.793559778918046\\
60.5	0.14424	0.852563770149559\\
60.5	0.1479	0.915887814272154\\
60.5	0.15156	0.983531911285836\\
60.5	0.15522	1.0554960611906\\
60.5	0.15888	1.13178026398644\\
60.5	0.16254	1.21238451967338\\
60.5	0.1662	1.29730882825139\\
60.5	0.16986	1.38655318972048\\
60.5	0.17352	1.48011760408066\\
60.5	0.17718	1.57800207133192\\
60.5	0.18084	1.68020659147427\\
60.5	0.1845	1.7867311645077\\
60.5	0.18816	1.89757579043221\\
60.5	0.19182	2.01274046924781\\
60.5	0.19548	2.13222520095449\\
60.5	0.19914	2.25602998555224\\
60.5	0.2028	2.38415482304109\\
60.5	0.20646	2.51659971342102\\
60.5	0.21012	2.65336465669203\\
60.5	0.21378	2.79444965285412\\
60.5	0.21744	2.93985470190729\\
60.5	0.2211	3.08957980385156\\
60.5	0.22476	3.2436249586869\\
60.5	0.22842	3.40199016641332\\
60.5	0.23208	3.56467542703084\\
60.5	0.23574	3.73168074053943\\
60.5	0.2394	3.9030061069391\\
60.5	0.24306	4.07865152622985\\
60.5	0.24672	4.2586169984117\\
60.5	0.25038	4.44290252348462\\
60.5	0.25404	4.63150810144863\\
60.5	0.2577	4.82443373230372\\
60.5	0.26136	5.0216794160499\\
60.5	0.26502	5.22324515268716\\
60.5	0.26868	5.42913094221549\\
60.5	0.27234	5.63933678463491\\
60.5	0.276	5.85386267994542\\
60.875	0.093	0.413252738249848\\
60.875	0.09666	0.420044232105869\\
60.875	0.10032	0.431155778852974\\
60.875	0.10398	0.446587378491165\\
60.875	0.10764	0.466339031020436\\
60.875	0.1113	0.49041073644079\\
60.875	0.11496	0.518802494752231\\
60.875	0.11862	0.551514305954754\\
60.875	0.12228	0.588546170048358\\
60.875	0.12594	0.629898087033047\\
60.875	0.1296	0.67557005690882\\
60.875	0.13326	0.725562079675669\\
60.875	0.13692	0.779874155333608\\
60.875	0.14058	0.838506283882631\\
60.875	0.14424	0.901458465322734\\
60.875	0.1479	0.968730699653919\\
60.875	0.15156	1.04032298687619\\
60.875	0.15522	1.11623532698955\\
60.875	0.15888	1.19646771999398\\
60.875	0.16254	1.2810201658895\\
60.875	0.1662	1.3698926646761\\
60.875	0.16986	1.46308521635379\\
60.875	0.17352	1.56059782092256\\
60.875	0.17718	1.66243047838241\\
60.875	0.18084	1.76858318873334\\
60.875	0.1845	1.87905595197537\\
60.875	0.18816	1.99384876810846\\
60.875	0.19182	2.11296163713265\\
60.875	0.19548	2.23639455904791\\
60.875	0.19914	2.36414753385426\\
60.875	0.2028	2.4962205615517\\
60.875	0.20646	2.63261364214021\\
60.875	0.21012	2.77332677561982\\
60.875	0.21378	2.9183599619905\\
60.875	0.21744	3.06771320125227\\
60.875	0.2211	3.22138649340512\\
60.875	0.22476	3.37937983844905\\
60.875	0.22842	3.54169323638406\\
60.875	0.23208	3.70832668721016\\
60.875	0.23574	3.87928019092734\\
60.875	0.2394	4.05455374753561\\
60.875	0.24306	4.23414735703496\\
60.875	0.24672	4.41806101942539\\
60.875	0.25038	4.60629473470691\\
60.875	0.25404	4.7988485028795\\
60.875	0.2577	4.99572232394318\\
60.875	0.26136	5.19691619789795\\
60.875	0.26502	5.40243012474379\\
60.875	0.26868	5.61226410448072\\
60.875	0.27234	5.82641813710873\\
60.875	0.276	6.04489222262783\\
61.25	0.093	0.405827564020234\\
61.25	0.09666	0.416567248084845\\
61.25	0.10032	0.431626985040543\\
61.25	0.10398	0.451006774887324\\
61.25	0.10764	0.474706617625184\\
61.25	0.1113	0.502726513254128\\
61.25	0.11496	0.535066461774159\\
61.25	0.11862	0.571726463185271\\
61.25	0.12228	0.612706517487465\\
61.25	0.12594	0.65800662468074\\
61.25	0.1296	0.707626784765103\\
61.25	0.13326	0.761566997740545\\
61.25	0.13692	0.819827263607074\\
61.25	0.14058	0.882407582364683\\
61.25	0.14424	0.949307954013378\\
61.25	0.1479	1.02052837855315\\
61.25	0.15156	1.09606885598401\\
61.25	0.15522	1.17592938630596\\
61.25	0.15888	1.26010996951898\\
61.25	0.16254	1.34861060562309\\
61.25	0.1662	1.44143129461829\\
61.25	0.16986	1.53857203650455\\
61.25	0.17352	1.64003283128192\\
61.25	0.17718	1.74581367895036\\
61.25	0.18084	1.85591457950988\\
61.25	0.1845	1.97033553296049\\
61.25	0.18816	2.08907653930218\\
61.25	0.19182	2.21213759853496\\
61.25	0.19548	2.33951871065881\\
61.25	0.19914	2.47121987567375\\
61.25	0.2028	2.60724109357978\\
61.25	0.20646	2.74758236437688\\
61.25	0.21012	2.89224368806508\\
61.25	0.21378	3.04122506464434\\
61.25	0.21744	3.1945264941147\\
61.25	0.2211	3.35214797647614\\
61.25	0.22476	3.51408951172866\\
61.25	0.22842	3.68035109987227\\
61.25	0.23208	3.85093274090696\\
61.25	0.23574	4.02583443483272\\
61.25	0.2394	4.20505618164958\\
61.25	0.24306	4.38859798135752\\
61.25	0.24672	4.57645983395654\\
61.25	0.25038	4.76864173944665\\
61.25	0.25404	4.96514369782783\\
61.25	0.2577	5.1659657091001\\
61.25	0.26136	5.37110777326346\\
61.25	0.26502	5.58056989031789\\
61.25	0.26868	5.79435206026341\\
61.25	0.27234	6.01245428310001\\
61.25	0.276	6.2348765588277\\
61.625	0.093	0.397357183308098\\
61.625	0.09666	0.412045057581299\\
61.625	0.10032	0.431052984745583\\
61.625	0.10398	0.454380964800953\\
61.625	0.10764	0.482028997747403\\
61.625	0.1113	0.51399708358494\\
61.625	0.11496	0.55028522231356\\
61.625	0.11862	0.590893413933259\\
61.625	0.12228	0.635821658444041\\
61.625	0.12594	0.68506995584591\\
61.625	0.1296	0.738638306138862\\
61.625	0.13326	0.796526709322894\\
61.625	0.13692	0.858735165398012\\
61.625	0.14058	0.925263674364211\\
61.625	0.14424	0.996112236221496\\
61.625	0.1479	1.07128085096986\\
61.625	0.15156	1.15076951860931\\
61.625	0.15522	1.23457823913984\\
61.625	0.15888	1.32270701256146\\
61.625	0.16254	1.41515583887416\\
61.625	0.1662	1.51192471807794\\
61.625	0.16986	1.6130136501728\\
61.625	0.17352	1.71842263515875\\
61.625	0.17718	1.82815167303578\\
61.625	0.18084	1.9422007638039\\
61.625	0.1845	2.06056990746309\\
61.625	0.18816	2.18325910401337\\
61.625	0.19182	2.31026835345474\\
61.625	0.19548	2.44159765578718\\
61.625	0.19914	2.57724701101071\\
61.625	0.2028	2.71721641912533\\
61.625	0.20646	2.86150588013102\\
61.625	0.21012	3.0101153940278\\
61.625	0.21378	3.16304496081567\\
61.625	0.21744	3.32029458049461\\
61.625	0.2211	3.48186425306464\\
61.625	0.22476	3.64775397852575\\
61.625	0.22842	3.81796375687795\\
61.625	0.23208	3.99249358812123\\
61.625	0.23574	4.17134347225558\\
61.625	0.2394	4.35451340928103\\
61.625	0.24306	4.54200339919756\\
61.625	0.24672	4.73381344200517\\
61.625	0.25038	4.92994353770387\\
61.625	0.25404	5.13039368629364\\
61.625	0.2577	5.3351638877745\\
61.625	0.26136	5.54425414214644\\
61.625	0.26502	5.75766444940946\\
61.625	0.26868	5.97539480956357\\
61.625	0.27234	6.19744522260877\\
61.625	0.276	6.42381568854504\\
62	0.093	0.38784159611343\\
62	0.09666	0.406477660595216\\
62	0.10032	0.429433777968093\\
62	0.10398	0.456709948232053\\
62	0.10764	0.488306171387093\\
62	0.1113	0.524222447433216\\
62	0.11496	0.564458776370425\\
62	0.11862	0.609015158198718\\
62	0.12228	0.65789159291809\\
62	0.12594	0.711088080528548\\
62	0.1296	0.76860462103009\\
62	0.13326	0.830441214422711\\
62	0.13692	0.896597860706416\\
62	0.14058	0.967074559881208\\
62	0.14424	1.04187131194708\\
62	0.1479	1.12098811690403\\
62	0.15156	1.20442497475207\\
62	0.15522	1.2921818854912\\
62	0.15888	1.3842588491214\\
62	0.16254	1.48065586564269\\
62	0.1662	1.58137293505506\\
62	0.16986	1.68641005735851\\
62	0.17352	1.79576723255306\\
62	0.17718	1.90944446063867\\
62	0.18084	2.02744174161537\\
62	0.1845	2.14975907548317\\
62	0.18816	2.27639646224203\\
62	0.19182	2.40735390189199\\
62	0.19548	2.54263139443303\\
62	0.19914	2.68222893986514\\
62	0.2028	2.82614653818835\\
62	0.20646	2.97438418940263\\
62	0.21012	3.126941893508\\
62	0.21378	3.28381965050446\\
62	0.21744	3.44501746039199\\
62	0.2211	3.61053532317061\\
62	0.22476	3.78037323884031\\
62	0.22842	3.95453120740109\\
62	0.23208	4.13300922885296\\
62	0.23574	4.31580730319591\\
62	0.2394	4.50292543042994\\
62	0.24306	4.69436361055507\\
62	0.24672	4.89012184357126\\
62	0.25038	5.09020012947856\\
62	0.25404	5.29459846827692\\
62	0.2577	5.50331685996636\\
62	0.26136	5.71635530454689\\
62	0.26502	5.9337138020185\\
62	0.26868	6.1553923523812\\
62	0.27234	6.38139095563499\\
62	0.276	6.61170961177986\\
62.375	0.093	0.377280802436231\\
62.375	0.09666	0.399865057126611\\
62.375	0.10032	0.426769364708078\\
62.375	0.10398	0.457993725180627\\
62.375	0.10764	0.493538138544253\\
62.375	0.1113	0.533402604798969\\
62.375	0.11496	0.577587123944768\\
62.375	0.11862	0.62609169598165\\
62.375	0.12228	0.678916320909612\\
62.375	0.12594	0.73606099872866\\
62.375	0.1296	0.797525729438787\\
62.375	0.13326	0.863310513039998\\
62.375	0.13692	0.933415349532296\\
62.375	0.14058	1.00784023891567\\
62.375	0.14424	1.08658518119014\\
62.375	0.1479	1.16965017635568\\
62.375	0.15156	1.25703522441231\\
62.375	0.15522	1.34874032536003\\
62.375	0.15888	1.44476547919882\\
62.375	0.16254	1.5451106859287\\
62.375	0.1662	1.64977594554966\\
62.375	0.16986	1.7587612580617\\
62.375	0.17352	1.87206662346483\\
62.375	0.17718	1.98969204175904\\
62.375	0.18084	2.11163751294433\\
62.375	0.1845	2.23790303702072\\
62.375	0.18816	2.36848861398817\\
62.375	0.19182	2.50339424384672\\
62.375	0.19548	2.64261992659634\\
62.375	0.19914	2.78616566223705\\
62.375	0.2028	2.93403145076884\\
62.375	0.20646	3.08621729219172\\
62.375	0.21012	3.24272318650568\\
62.375	0.21378	3.40354913371072\\
62.375	0.21744	3.56869513380684\\
62.375	0.2211	3.73816118679405\\
62.375	0.22476	3.91194729267234\\
62.375	0.22842	4.09005345144171\\
62.375	0.23208	4.27247966310218\\
62.375	0.23574	4.45922592765371\\
62.375	0.2394	4.65029224509633\\
62.375	0.24306	4.84567861543005\\
62.375	0.24672	5.04538503865483\\
62.375	0.25038	5.24941151477071\\
62.375	0.25404	5.45775804377766\\
62.375	0.2577	5.6704246256757\\
62.375	0.26136	5.88741126046483\\
62.375	0.26502	6.10871794814503\\
62.375	0.26868	6.33434468871632\\
62.375	0.27234	6.56429148217869\\
62.375	0.276	6.79855832853214\\
62.75	0.093	0.365674802276511\\
62.75	0.09666	0.392207247175481\\
62.75	0.10032	0.423059744965534\\
62.75	0.10398	0.458232295646672\\
62.75	0.10764	0.497724899218891\\
62.75	0.1113	0.541537555682197\\
62.75	0.11496	0.589670265036586\\
62.75	0.11862	0.642123027282054\\
62.75	0.12228	0.698895842418605\\
62.75	0.12594	0.759988710446243\\
62.75	0.1296	0.825401631364964\\
62.75	0.13326	0.895134605174764\\
62.75	0.13692	0.969187631875652\\
62.75	0.14058	1.04756071146762\\
62.75	0.14424	1.13025384395067\\
62.75	0.1479	1.21726702932481\\
62.75	0.15156	1.30860026759003\\
62.75	0.15522	1.40425355874633\\
62.75	0.15888	1.50422690279371\\
62.75	0.16254	1.60852029973218\\
62.75	0.1662	1.71713374956173\\
62.75	0.16986	1.83006725228236\\
62.75	0.17352	1.94732080789408\\
62.75	0.17718	2.06889441639688\\
62.75	0.18084	2.19478807779076\\
62.75	0.1845	2.32500179207573\\
62.75	0.18816	2.45953555925177\\
62.75	0.19182	2.59838937931892\\
62.75	0.19548	2.74156325227712\\
62.75	0.19914	2.88905717812642\\
62.75	0.2028	3.04087115686681\\
62.75	0.20646	3.19700518849827\\
62.75	0.21012	3.35745927302083\\
62.75	0.21378	3.52223341043445\\
62.75	0.21744	3.69132760073917\\
62.75	0.2211	3.86474184393497\\
62.75	0.22476	4.04247614002185\\
62.75	0.22842	4.22453048899981\\
62.75	0.23208	4.41090489086886\\
62.75	0.23574	4.60159934562899\\
62.75	0.2394	4.7966138532802\\
62.75	0.24306	4.9959484138225\\
62.75	0.24672	5.19960302725587\\
62.75	0.25038	5.40757769358034\\
62.75	0.25404	5.61987241279589\\
62.75	0.2577	5.83648718490251\\
62.75	0.26136	6.05742200990023\\
62.75	0.26502	6.28267688778902\\
62.75	0.26868	6.5122518185689\\
62.75	0.27234	6.74614680223986\\
62.75	0.276	6.9843618388019\\
63.125	0.093	0.353023595634254\\
63.125	0.09666	0.383504230741813\\
63.125	0.10032	0.418304918740459\\
63.125	0.10398	0.457425659630188\\
63.125	0.10764	0.500866453410996\\
63.125	0.1113	0.548627300082888\\
63.125	0.11496	0.600708199645867\\
63.125	0.11862	0.657109152099927\\
63.125	0.12228	0.717830157445069\\
63.125	0.12594	0.782871215681296\\
63.125	0.1296	0.852232326808606\\
63.125	0.13326	0.925913490826996\\
63.125	0.13692	1.00391470773647\\
63.125	0.14058	1.08623597753703\\
63.125	0.14424	1.17287730022867\\
63.125	0.1479	1.2638386758114\\
63.125	0.15156	1.35912010428521\\
63.125	0.15522	1.4587215856501\\
63.125	0.15888	1.56264311990607\\
63.125	0.16254	1.67088470705313\\
63.125	0.1662	1.78344634709127\\
63.125	0.16986	1.90032804002049\\
63.125	0.17352	2.0215297858408\\
63.125	0.17718	2.14705158455219\\
63.125	0.18084	2.27689343615466\\
63.125	0.1845	2.41105534064822\\
63.125	0.18816	2.54953729803285\\
63.125	0.19182	2.69233930830858\\
63.125	0.19548	2.83946137147538\\
63.125	0.19914	2.99090348753327\\
63.125	0.2028	3.14666565648225\\
63.125	0.20646	3.3067478783223\\
63.125	0.21012	3.47115015305344\\
63.125	0.21378	3.63987248067566\\
63.125	0.21744	3.81291486118896\\
63.125	0.2211	3.99027729459335\\
63.125	0.22476	4.17195978088882\\
63.125	0.22842	4.35796232007537\\
63.125	0.23208	4.54828491215301\\
63.125	0.23574	4.74292755712173\\
63.125	0.2394	4.94189025498152\\
63.125	0.24306	5.14517300573243\\
63.125	0.24672	5.35277580937439\\
63.125	0.25038	5.56469866590744\\
63.125	0.25404	5.78094157533158\\
63.125	0.2577	6.00150453764679\\
63.125	0.26136	6.2263875528531\\
63.125	0.26502	6.45559062095047\\
63.125	0.26868	6.68911374193894\\
63.125	0.27234	6.9269569158185\\
63.125	0.276	7.16912014258913\\
63.5	0.093	0.339327182509474\\
63.5	0.09666	0.373756007825623\\
63.5	0.10032	0.412504886032855\\
63.5	0.10398	0.455573817131173\\
63.5	0.10764	0.502962801120571\\
63.5	0.1113	0.554671838001056\\
63.5	0.11496	0.610700927772624\\
63.5	0.11862	0.671050070435275\\
63.5	0.12228	0.735719265989005\\
63.5	0.12594	0.804708514433822\\
63.5	0.1296	0.878017815769722\\
63.5	0.13326	0.955647169996702\\
63.5	0.13692	1.03759657711477\\
63.5	0.14058	1.12386603712392\\
63.5	0.14424	1.21445555002415\\
63.5	0.1479	1.30936511581546\\
63.5	0.15156	1.40859473449786\\
63.5	0.15522	1.51214440607134\\
63.5	0.15888	1.6200141305359\\
63.5	0.16254	1.73220390789155\\
63.5	0.1662	1.84871373813828\\
63.5	0.16986	1.96954362127609\\
63.5	0.17352	2.09469355730499\\
63.5	0.17718	2.22416354622497\\
63.5	0.18084	2.35795358803603\\
63.5	0.1845	2.49606368273818\\
63.5	0.18816	2.63849383033141\\
63.5	0.19182	2.78524403081572\\
63.5	0.19548	2.93631428419112\\
63.5	0.19914	3.09170459045759\\
63.5	0.2028	3.25141494961516\\
63.5	0.20646	3.4154453616638\\
63.5	0.21012	3.58379582660353\\
63.5	0.21378	3.75646634443434\\
63.5	0.21744	3.93345691515623\\
63.5	0.2211	4.11476753876921\\
63.5	0.22476	4.30039821527327\\
63.5	0.22842	4.49034894466841\\
63.5	0.23208	4.68461972695464\\
63.5	0.23574	4.88321056213194\\
63.5	0.2394	5.08612145020034\\
63.5	0.24306	5.29335239115981\\
63.5	0.24672	5.50490338501037\\
63.5	0.25038	5.72077443175202\\
63.5	0.25404	5.94096553138474\\
63.5	0.2577	6.16547668390855\\
63.5	0.26136	6.39430788932344\\
63.5	0.26502	6.62745914762941\\
63.5	0.26868	6.86493045882647\\
63.5	0.27234	7.10672182291461\\
63.5	0.276	7.35283323989383\\
63.875	0.093	0.32458556290216\\
63.875	0.09666	0.362962578426899\\
63.875	0.10032	0.405659646842724\\
63.875	0.10398	0.452676768149632\\
63.875	0.10764	0.504013942347619\\
63.875	0.1113	0.559671169436694\\
63.875	0.11496	0.619648449416851\\
63.875	0.11862	0.683945782288088\\
63.875	0.12228	0.752563168050409\\
63.875	0.12594	0.825500606703815\\
63.875	0.1296	0.902758098248305\\
63.875	0.13326	0.984335642683874\\
63.875	0.13692	1.07023324001053\\
63.875	0.14058	1.16045089022827\\
63.875	0.14424	1.25498859333709\\
63.875	0.1479	1.35384634933699\\
63.875	0.15156	1.45702415822798\\
63.875	0.15522	1.56452202001005\\
63.875	0.15888	1.6763399346832\\
63.875	0.16254	1.79247790224745\\
63.875	0.1662	1.91293592270276\\
63.875	0.16986	2.03771399604916\\
63.875	0.17352	2.16681212228665\\
63.875	0.17718	2.30023030141522\\
63.875	0.18084	2.43796853343487\\
63.875	0.1845	2.58002681834561\\
63.875	0.18816	2.72640515614742\\
63.875	0.19182	2.87710354684033\\
63.875	0.19548	3.03212199042431\\
63.875	0.19914	3.19146048689938\\
63.875	0.2028	3.35511903626553\\
63.875	0.20646	3.52309763852277\\
63.875	0.21012	3.69539629367108\\
63.875	0.21378	3.87201500171048\\
63.875	0.21744	4.05295376264096\\
63.875	0.2211	4.23821257646254\\
63.875	0.22476	4.42779144317518\\
63.875	0.22842	4.62169036277891\\
63.875	0.23208	4.81990933527373\\
63.875	0.23574	5.02244836065963\\
63.875	0.2394	5.22930743893661\\
63.875	0.24306	5.44048657010468\\
63.875	0.24672	5.65598575416383\\
63.875	0.25038	5.87580499111406\\
63.875	0.25404	6.09994428095537\\
63.875	0.2577	6.32840362368777\\
63.875	0.26136	6.56118301931125\\
63.875	0.26502	6.79828246782581\\
63.875	0.26868	7.03970196923146\\
63.875	0.27234	7.28544152352819\\
63.875	0.276	7.535501130716\\
64.25	0.093	0.308798736812324\\
64.25	0.09666	0.351123942545651\\
64.25	0.10032	0.397769201170066\\
64.25	0.10398	0.44873451268556\\
64.25	0.10764	0.504019877092137\\
64.25	0.1113	0.563625294389802\\
64.25	0.11496	0.627550764578549\\
64.25	0.11862	0.695796287658379\\
64.25	0.12228	0.768361863629289\\
64.25	0.12594	0.845247492491285\\
64.25	0.1296	0.926453174244364\\
64.25	0.13326	1.01197890888852\\
64.25	0.13692	1.10182469642377\\
64.25	0.14058	1.19599053685009\\
64.25	0.14424	1.29447643016751\\
64.25	0.1479	1.397282376376\\
64.25	0.15156	1.50440837547558\\
64.25	0.15522	1.61585442746624\\
64.25	0.15888	1.73162053234798\\
64.25	0.16254	1.85170669012081\\
64.25	0.1662	1.97611290078472\\
64.25	0.16986	2.10483916433971\\
64.25	0.17352	2.23788548078579\\
64.25	0.17718	2.37525185012294\\
64.25	0.18084	2.51693827235118\\
64.25	0.1845	2.66294474747051\\
64.25	0.18816	2.81327127548091\\
64.25	0.19182	2.96791785638241\\
64.25	0.19548	3.12688449017498\\
64.25	0.19914	3.29017117685864\\
64.25	0.2028	3.45777791643338\\
64.25	0.20646	3.62970470889921\\
64.25	0.21012	3.80595155425612\\
64.25	0.21378	3.98651845250411\\
64.25	0.21744	4.17140540364317\\
64.25	0.2211	4.36061240767333\\
64.25	0.22476	4.55413946459457\\
64.25	0.22842	4.75198657440689\\
64.25	0.23208	4.9541537371103\\
64.25	0.23574	5.16064095270479\\
64.25	0.2394	5.37144822119036\\
64.25	0.24306	5.58657554256702\\
64.25	0.24672	5.80602291683476\\
64.25	0.25038	6.02979034399358\\
64.25	0.25404	6.25787782404348\\
64.25	0.2577	6.49028535698447\\
64.25	0.26136	6.72701294281654\\
64.25	0.26502	6.96806058153969\\
64.25	0.26868	7.21342827315392\\
64.25	0.27234	7.46311601765925\\
64.25	0.276	7.71712381505565\\
64.625	0.093	0.291966704239957\\
64.625	0.09666	0.338240100181874\\
64.625	0.10032	0.388833549014875\\
64.625	0.10398	0.443747050738962\\
64.625	0.10764	0.502980605354129\\
64.625	0.1113	0.566534212860383\\
64.625	0.11496	0.634407873257719\\
64.625	0.11862	0.706601586546139\\
64.625	0.12228	0.783115352725638\\
64.625	0.12594	0.863949171796224\\
64.625	0.1296	0.949103043757893\\
64.625	0.13326	1.03857696861064\\
64.625	0.13692	1.13237094635448\\
64.625	0.14058	1.23048497698939\\
64.625	0.14424	1.33291906051539\\
64.625	0.1479	1.43967319693248\\
64.625	0.15156	1.55074738624064\\
64.625	0.15522	1.66614162843989\\
64.625	0.15888	1.78585592353022\\
64.625	0.16254	1.90989027151164\\
64.625	0.1662	2.03824467238414\\
64.625	0.16986	2.17091912614772\\
64.625	0.17352	2.30791363280239\\
64.625	0.17718	2.44922819234814\\
64.625	0.18084	2.59486280478497\\
64.625	0.1845	2.74481747011288\\
64.625	0.18816	2.89909218833188\\
64.625	0.19182	3.05768695944197\\
64.625	0.19548	3.22060178344312\\
64.625	0.19914	3.38783666033537\\
64.625	0.2028	3.55939159011871\\
64.625	0.20646	3.73526657279311\\
64.625	0.21012	3.91546160835862\\
64.625	0.21378	4.09997669681519\\
64.625	0.21744	4.28881183816285\\
64.625	0.2211	4.4819670324016\\
64.625	0.22476	4.67944227953143\\
64.625	0.22842	4.88123757955234\\
64.625	0.23208	5.08735293246434\\
64.625	0.23574	5.29778833826742\\
64.625	0.2394	5.51254379696158\\
64.625	0.24306	5.73161930854682\\
64.625	0.24672	5.95501487302314\\
64.625	0.25038	6.18273049039056\\
64.625	0.25404	6.41476616064905\\
64.625	0.2577	6.65112188379863\\
64.625	0.26136	6.8917976598393\\
64.625	0.26502	7.13679348877104\\
64.625	0.26868	7.38610937059386\\
64.625	0.27234	7.63974530530777\\
64.625	0.276	7.89770129291276\\
65	0.093	0.274089465185058\\
65	0.09666	0.324311051335565\\
65	0.10032	0.378852690377159\\
65	0.10398	0.437714382309835\\
65	0.10764	0.500896127133592\\
65	0.1113	0.568397924848435\\
65	0.11496	0.640219775454358\\
65	0.11862	0.716361678951367\\
65	0.12228	0.796823635339456\\
65	0.12594	0.881605644618632\\
65	0.1296	0.97070770678889\\
65	0.13326	1.06412982185023\\
65	0.13692	1.16187198980265\\
65	0.14058	1.26393421064616\\
65	0.14424	1.37031648438075\\
65	0.1479	1.48101881100642\\
65	0.15156	1.59604119052318\\
65	0.15522	1.71538362293102\\
65	0.15888	1.83904610822994\\
65	0.16254	1.96702864641995\\
65	0.1662	2.09933123750104\\
65	0.16986	2.23595388147321\\
65	0.17352	2.37689657833646\\
65	0.17718	2.5221593280908\\
65	0.18084	2.67174213073622\\
65	0.1845	2.82564498627273\\
65	0.18816	2.98386789470031\\
65	0.19182	3.14641085601899\\
65	0.19548	3.31327387022874\\
65	0.19914	3.48445693732958\\
65	0.2028	3.6599600573215\\
65	0.20646	3.8397832302045\\
65	0.21012	4.02392645597859\\
65	0.21378	4.21238973464375\\
65	0.21744	4.40517306620001\\
65	0.2211	4.60227645064734\\
65	0.22476	4.80369988798576\\
65	0.22842	5.00944337821526\\
65	0.23208	5.21950692133585\\
65	0.23574	5.43389051734752\\
65	0.2394	5.65259416625026\\
65	0.24306	5.8756178680441\\
65	0.24672	6.10296162272901\\
65	0.25038	6.33462543030502\\
65	0.25404	6.5706092907721\\
65	0.2577	6.81091320413027\\
65	0.26136	7.05553717037952\\
65	0.26502	7.30448118951985\\
65	0.26868	7.55774526155126\\
65	0.27234	7.81532938647376\\
65	0.276	8.07723356428735\\
65.375	0.093	0.255167019647636\\
65.375	0.09666	0.309336796006732\\
65.375	0.10032	0.367826625256912\\
65.375	0.10398	0.430636507398179\\
65.375	0.10764	0.497766442430525\\
65.375	0.1113	0.569216430353958\\
65.375	0.11496	0.644986471168474\\
65.375	0.11862	0.725076564874072\\
65.375	0.12228	0.809486711470751\\
65.375	0.12594	0.898216910958516\\
65.375	0.1296	0.991267163337364\\
65.375	0.13326	1.08863746860729\\
65.375	0.13692	1.19032782676831\\
65.375	0.14058	1.29633823782041\\
65.375	0.14424	1.40666870176359\\
65.375	0.1479	1.52131921859785\\
65.375	0.15156	1.64028978832319\\
65.375	0.15522	1.76358041093962\\
65.375	0.15888	1.89119108644713\\
65.375	0.16254	2.02312181484573\\
65.375	0.1662	2.15937259613541\\
65.375	0.16986	2.29994343031617\\
65.375	0.17352	2.44483431738801\\
65.375	0.17718	2.59404525735094\\
65.375	0.18084	2.74757625020495\\
65.375	0.1845	2.90542729595005\\
65.375	0.18816	3.06759839458622\\
65.375	0.19182	3.23408954611348\\
65.375	0.19548	3.40490075053183\\
65.375	0.19914	3.58003200784125\\
65.375	0.2028	3.75948331804176\\
65.375	0.20646	3.94325468113336\\
65.375	0.21012	4.13134609711603\\
65.375	0.21378	4.32375756598979\\
65.375	0.21744	4.52048908775463\\
65.375	0.2211	4.72154066241056\\
65.375	0.22476	4.92691228995756\\
65.375	0.22842	5.13660397039565\\
65.375	0.23208	5.35061570372483\\
65.375	0.23574	5.56894748994509\\
65.375	0.2394	5.79159932905643\\
65.375	0.24306	6.01857122105886\\
65.375	0.24672	6.24986316595236\\
65.375	0.25038	6.48547516373695\\
65.375	0.25404	6.72540721441263\\
65.375	0.2577	6.96965931797938\\
65.375	0.26136	7.21823147443722\\
65.375	0.26502	7.47112368378614\\
65.375	0.26868	7.72833594602615\\
65.375	0.27234	7.98986826115724\\
65.375	0.276	8.25572062917941\\
65.75	0.093	0.23519936762768\\
65.75	0.09666	0.293317334195366\\
65.75	0.10032	0.35575535365414\\
65.75	0.10398	0.422513426003995\\
65.75	0.10764	0.493591551244931\\
65.75	0.1113	0.568989729376954\\
65.75	0.11496	0.648707960400059\\
65.75	0.11862	0.732746244314248\\
65.75	0.12228	0.821104581119516\\
65.75	0.12594	0.913782970815871\\
65.75	0.1296	1.01078141340331\\
65.75	0.13326	1.11209990888183\\
65.75	0.13692	1.21773845725143\\
65.75	0.14058	1.32769705851211\\
65.75	0.14424	1.44197571266389\\
65.75	0.1479	1.56057441970674\\
65.75	0.15156	1.68349317964067\\
65.75	0.15522	1.81073199246569\\
65.75	0.15888	1.94229085818179\\
65.75	0.16254	2.07816977678898\\
65.75	0.1662	2.21836874828725\\
65.75	0.16986	2.36288777267659\\
65.75	0.17352	2.51172684995703\\
65.75	0.17718	2.66488598012855\\
65.75	0.18084	2.82236516319115\\
65.75	0.1845	2.98416439914483\\
65.75	0.18816	3.1502836879896\\
65.75	0.19182	3.32072302972545\\
65.75	0.19548	3.49548242435239\\
65.75	0.19914	3.6745618718704\\
65.75	0.2028	3.8579613722795\\
65.75	0.20646	4.04568092557968\\
65.75	0.21012	4.23772053177095\\
65.75	0.21378	4.4340801908533\\
65.75	0.21744	4.63475990282672\\
65.75	0.2211	4.83975966769124\\
65.75	0.22476	5.04907948544684\\
65.75	0.22842	5.26271935609352\\
65.75	0.23208	5.48067927963128\\
65.75	0.23574	5.70295925606013\\
65.75	0.2394	5.92955928538006\\
65.75	0.24306	6.16047936759108\\
65.75	0.24672	6.39571950269317\\
65.75	0.25038	6.63527969068635\\
65.75	0.25404	6.87915993157061\\
65.75	0.2577	7.12736022534595\\
65.75	0.26136	7.3798805720124\\
65.75	0.26502	7.6367209715699\\
65.75	0.26868	7.8978814240185\\
65.75	0.27234	8.16336192935817\\
65.75	0.276	8.43316248758893\\
66.125	0.093	0.214186509125203\\
66.125	0.09666	0.276252665901479\\
66.125	0.10032	0.342638875568842\\
66.125	0.10398	0.413345138127287\\
66.125	0.10764	0.488371453576813\\
66.125	0.1113	0.567717821917421\\
66.125	0.11496	0.651384243149117\\
66.125	0.11862	0.739370717271894\\
66.125	0.12228	0.831677244285753\\
66.125	0.12594	0.9283038241907\\
66.125	0.1296	1.02925045698673\\
66.125	0.13326	1.13451714267383\\
66.125	0.13692	1.24410388125203\\
66.125	0.14058	1.3580106727213\\
66.125	0.14424	1.47623751708166\\
66.125	0.1479	1.5987844143331\\
66.125	0.15156	1.72565136447563\\
66.125	0.15522	1.85683836750924\\
66.125	0.15888	1.99234542343393\\
66.125	0.16254	2.13217253224971\\
66.125	0.1662	2.27631969395656\\
66.125	0.16986	2.4247869085545\\
66.125	0.17352	2.57757417604352\\
66.125	0.17718	2.73468149642363\\
66.125	0.18084	2.89610886969482\\
66.125	0.1845	3.0618562958571\\
66.125	0.18816	3.23192377491045\\
66.125	0.19182	3.4063113068549\\
66.125	0.19548	3.58501889169041\\
66.125	0.19914	3.76804652941702\\
66.125	0.2028	3.95539422003471\\
66.125	0.20646	4.14706196354349\\
66.125	0.21012	4.34304975994334\\
66.125	0.21378	4.54335760923428\\
66.125	0.21744	4.74798551141629\\
66.125	0.2211	4.95693346648941\\
66.125	0.22476	5.17020147445359\\
66.125	0.22842	5.38778953530886\\
66.125	0.23208	5.60969764905522\\
66.125	0.23574	5.83592581569265\\
66.125	0.2394	6.06647403522117\\
66.125	0.24306	6.30134230764077\\
66.125	0.24672	6.54053063295146\\
66.125	0.25038	6.78403901115323\\
66.125	0.25404	7.03186744224609\\
66.125	0.2577	7.28401592623002\\
66.125	0.26136	7.54048446310504\\
66.125	0.26502	7.80127305287114\\
66.125	0.26868	8.06638169552832\\
66.125	0.27234	8.33581039107659\\
66.125	0.276	8.60955913951594\\
66.5	0.093	0.192128444140185\\
66.5	0.09666	0.258142791125051\\
66.5	0.10032	0.328477191001003\\
66.5	0.10398	0.403131643768038\\
66.5	0.10764	0.482106149426153\\
66.5	0.1113	0.565400707975355\\
66.5	0.11496	0.65301531941564\\
66.5	0.11862	0.744949983747007\\
66.5	0.12228	0.841204700969455\\
66.5	0.12594	0.941779471082989\\
66.5	0.1296	1.04667429408761\\
66.5	0.13326	1.1558891699833\\
66.5	0.13692	1.26942409877009\\
66.5	0.14058	1.38727908044795\\
66.5	0.14424	1.5094541150169\\
66.5	0.1479	1.63594920247693\\
66.5	0.15156	1.76676434282805\\
66.5	0.15522	1.90189953607025\\
66.5	0.15888	2.04135478220352\\
66.5	0.16254	2.18513008122789\\
66.5	0.1662	2.33322543314334\\
66.5	0.16986	2.48564083794987\\
66.5	0.17352	2.64237629564748\\
66.5	0.17718	2.80343180623618\\
66.5	0.18084	2.96880736971595\\
66.5	0.1845	3.13850298608682\\
66.5	0.18816	3.31251865534877\\
66.5	0.19182	3.4908543775018\\
66.5	0.19548	3.67351015254591\\
66.5	0.19914	3.8604859804811\\
66.5	0.2028	4.05178186130739\\
66.5	0.20646	4.24739779502474\\
66.5	0.21012	4.44733378163319\\
66.5	0.21378	4.65158982113272\\
66.5	0.21744	4.86016591352333\\
66.5	0.2211	5.07306205880502\\
66.5	0.22476	5.2902782569778\\
66.5	0.22842	5.51181450804166\\
66.5	0.23208	5.73767081199661\\
66.5	0.23574	5.96784716884263\\
66.5	0.2394	6.20234357857974\\
66.5	0.24306	6.44116004120793\\
66.5	0.24672	6.68429655672721\\
66.5	0.25038	6.93175312513757\\
66.5	0.25404	7.18352974643901\\
66.5	0.2577	7.43962642063153\\
66.5	0.26136	7.70004314771515\\
66.5	0.26502	7.96477992768983\\
66.5	0.26868	8.23383676055561\\
66.5	0.27234	8.50721364631247\\
66.5	0.276	8.78491058496041\\
66.875	0.093	0.169025172672648\\
66.875	0.09666	0.238987709866103\\
66.875	0.10032	0.313270299950645\\
66.875	0.10398	0.39187294292627\\
66.875	0.10764	0.474795638792975\\
66.875	0.1113	0.562038387550766\\
66.875	0.11496	0.65360118919964\\
66.875	0.11862	0.749484043739598\\
66.875	0.12228	0.849686951170635\\
66.875	0.12594	0.954209911492758\\
66.875	0.1296	1.06305292470596\\
66.875	0.13326	1.17621599081025\\
66.875	0.13692	1.29369910980562\\
66.875	0.14058	1.41550228169208\\
66.875	0.14424	1.54162550646962\\
66.875	0.1479	1.67206878413824\\
66.875	0.15156	1.80683211469794\\
66.875	0.15522	1.94591549814873\\
66.875	0.15888	2.0893189344906\\
66.875	0.16254	2.23704242372356\\
66.875	0.1662	2.3890859658476\\
66.875	0.16986	2.54544956086271\\
66.875	0.17352	2.70613320876892\\
66.875	0.17718	2.8711369095662\\
66.875	0.18084	3.04046066325457\\
66.875	0.1845	3.21410446983402\\
66.875	0.18816	3.39206832930456\\
66.875	0.19182	3.57435224166618\\
66.875	0.19548	3.76095620691888\\
66.875	0.19914	3.95188022506266\\
66.875	0.2028	4.14712429609754\\
66.875	0.20646	4.34668842002349\\
66.875	0.21012	4.55057259684053\\
66.875	0.21378	4.75877682654864\\
66.875	0.21744	4.97130110914784\\
66.875	0.2211	5.18814544463812\\
66.875	0.22476	5.40930983301949\\
66.875	0.22842	5.63479427429194\\
66.875	0.23208	5.86459876845548\\
66.875	0.23574	6.09872331551009\\
66.875	0.2394	6.33716791545579\\
66.875	0.24306	6.57993256829257\\
66.875	0.24672	6.82701727402044\\
66.875	0.25038	7.07842203263939\\
66.875	0.25404	7.33414684414942\\
66.875	0.2577	7.59419170855053\\
66.875	0.26136	7.85855662584274\\
66.875	0.26502	8.12724159602601\\
66.875	0.26868	8.40024661910038\\
66.875	0.27234	8.67757169506583\\
66.875	0.276	8.95921682392236\\
67.25	0.093	0.144876694722584\\
67.25	0.09666	0.218787422124629\\
67.25	0.10032	0.297018202417757\\
67.25	0.10398	0.379569035601971\\
67.25	0.10764	0.466439921677269\\
67.25	0.1113	0.557630860643647\\
67.25	0.11496	0.653141852501111\\
67.25	0.11862	0.752972897249657\\
67.25	0.12228	0.857123994889288\\
67.25	0.12594	0.965595145420001\\
67.25	0.1296	1.0783863488418\\
67.25	0.13326	1.19549760515467\\
67.25	0.13692	1.31692891435864\\
67.25	0.14058	1.44268027645368\\
67.25	0.14424	1.57275169143981\\
67.25	0.1479	1.70714315931702\\
67.25	0.15156	1.84585468008531\\
67.25	0.15522	1.98888625374469\\
67.25	0.15888	2.13623788029515\\
67.25	0.16254	2.2879095597367\\
67.25	0.1662	2.44390129206932\\
67.25	0.16986	2.60421307729303\\
67.25	0.17352	2.76884491540783\\
67.25	0.17718	2.9377968064137\\
67.25	0.18084	3.11106875031066\\
67.25	0.1845	3.2886607470987\\
67.25	0.18816	3.47057279677783\\
67.25	0.19182	3.65680489934804\\
67.25	0.19548	3.84735705480933\\
67.25	0.19914	4.0422292631617\\
67.25	0.2028	4.24142152440516\\
67.25	0.20646	4.4449338385397\\
67.25	0.21012	4.65276620556533\\
67.25	0.21378	4.86491862548203\\
67.25	0.21744	5.08139109828982\\
67.25	0.2211	5.3021836239887\\
67.25	0.22476	5.52729620257866\\
67.25	0.22842	5.75672883405969\\
67.25	0.23208	5.99048151843182\\
67.25	0.23574	6.22855425569502\\
67.25	0.2394	6.47094704584931\\
67.25	0.24306	6.71765988889469\\
67.25	0.24672	6.96869278483114\\
67.25	0.25038	7.22404573365868\\
67.25	0.25404	7.4837187353773\\
67.25	0.2577	7.747711789987\\
67.25	0.26136	8.01602489748779\\
67.25	0.26502	8.28865805787966\\
67.25	0.26868	8.56561127116261\\
67.25	0.27234	8.84688453733666\\
67.25	0.276	9.13247785640178\\
67.625	0.093	0.119683010289987\\
67.625	0.09666	0.197541927900621\\
67.625	0.10032	0.279720898402342\\
67.625	0.10398	0.366219921795146\\
67.625	0.10764	0.45703899807903\\
67.625	0.1113	0.552178127254001\\
67.625	0.11496	0.651637309320054\\
67.625	0.11862	0.755416544277191\\
67.625	0.12228	0.863515832125407\\
67.625	0.12594	0.97593517286471\\
67.625	0.1296	1.0926745664951\\
67.625	0.13326	1.21373401301656\\
67.625	0.13692	1.33911351242912\\
67.625	0.14058	1.46881306473275\\
67.625	0.14424	1.60283266992747\\
67.625	0.1479	1.74117232801327\\
67.625	0.15156	1.88383203899015\\
67.625	0.15522	2.03081180285812\\
67.625	0.15888	2.18211161961717\\
67.625	0.16254	2.3377314892673\\
67.625	0.1662	2.49767141180852\\
67.625	0.16986	2.66193138724082\\
67.625	0.17352	2.8305114155642\\
67.625	0.17718	3.00341149677867\\
67.625	0.18084	3.18063163088421\\
67.625	0.1845	3.36217181788085\\
67.625	0.18816	3.54803205776856\\
67.625	0.19182	3.73821235054737\\
67.625	0.19548	3.93271269621724\\
67.625	0.19914	4.13153309477821\\
67.625	0.2028	4.33467354623026\\
67.625	0.20646	4.54213405057339\\
67.625	0.21012	4.7539146078076\\
67.625	0.21378	4.9700152179329\\
67.625	0.21744	5.19043588094927\\
67.625	0.2211	5.41517659685674\\
67.625	0.22476	5.64423736565528\\
67.625	0.22842	5.87761818734491\\
67.625	0.23208	6.11531906192563\\
67.625	0.23574	6.35733998939743\\
67.625	0.2394	6.6036809697603\\
67.625	0.24306	6.85434200301427\\
67.625	0.24672	7.10932308915931\\
67.625	0.25038	7.36862422819544\\
67.625	0.25404	7.63224542012265\\
67.625	0.2577	7.90018666494094\\
67.625	0.26136	8.17244796265032\\
67.625	0.26502	8.44902931325078\\
67.625	0.26868	8.72993071674232\\
67.625	0.27234	9.01515217312495\\
67.625	0.276	9.30469368239866\\
68	0.093	0.0934441193748627\\
68	0.09666	0.175251227194087\\
68	0.10032	0.261378387904398\\
68	0.10398	0.351825601505791\\
68	0.10764	0.446592867998264\\
68	0.1113	0.545680187381825\\
68	0.11496	0.649087559656468\\
68	0.11862	0.756814984822194\\
68	0.12228	0.868862462879\\
68	0.12594	0.985229993826892\\
68	0.1296	1.10591757766587\\
68	0.13326	1.23092521439592\\
68	0.13692	1.36025290401706\\
68	0.14058	1.49390064652929\\
68	0.14424	1.6318684419326\\
68	0.1479	1.77415629022699\\
68	0.15156	1.92076419141246\\
68	0.15522	2.07169214548902\\
68	0.15888	2.22694015245666\\
68	0.16254	2.38650821231538\\
68	0.1662	2.55039632506519\\
68	0.16986	2.71860449070608\\
68	0.17352	2.89113270923805\\
68	0.17718	3.0679809806611\\
68	0.18084	3.24914930497524\\
68	0.1845	3.43463768218047\\
68	0.18816	3.62444611227677\\
68	0.19182	3.81857459526416\\
68	0.19548	4.01702313114263\\
68	0.19914	4.21979171991218\\
68	0.2028	4.42688036157282\\
68	0.20646	4.63828905612454\\
68	0.21012	4.85401780356734\\
68	0.21378	5.07406660390123\\
68	0.21744	5.2984354571262\\
68	0.2211	5.52712436324226\\
68	0.22476	5.76013332224939\\
68	0.22842	5.9974623341476\\
68	0.23208	6.23911139893691\\
68	0.23574	6.4850805166173\\
68	0.2394	6.73536968718876\\
68	0.24306	6.98997891065132\\
68	0.24672	7.24890818700494\\
68	0.25038	7.51215751624967\\
68	0.25404	7.77972689838547\\
68	0.2577	8.05161633341235\\
68	0.26136	8.32782582133032\\
68	0.26502	8.60835536213937\\
68	0.26868	8.8932049558395\\
68	0.27234	9.18237460243072\\
68	0.276	9.47586430191301\\
68.375	0.093	0.0661600219772049\\
68.375	0.09666	0.151915320005019\\
68.375	0.10032	0.241990670923919\\
68.375	0.10398	0.336386074733902\\
68.375	0.10764	0.435101531434965\\
68.375	0.1113	0.538137041027115\\
68.375	0.11496	0.645492603510348\\
68.375	0.11862	0.757168218884663\\
68.375	0.12228	0.873163887150063\\
68.375	0.12594	0.993479608306544\\
68.375	0.1296	1.11811538235411\\
68.375	0.13326	1.24707120929275\\
68.375	0.13692	1.38034708912249\\
68.375	0.14058	1.5179430218433\\
68.375	0.14424	1.6598590074552\\
68.375	0.1479	1.80609504595818\\
68.375	0.15156	1.95665113735224\\
68.375	0.15522	2.11152728163739\\
68.375	0.15888	2.27072347881361\\
68.375	0.16254	2.43423972888093\\
68.375	0.1662	2.60207603183932\\
68.375	0.16986	2.7742323876888\\
68.375	0.17352	2.95070879642937\\
68.375	0.17718	3.13150525806101\\
68.375	0.18084	3.31662177258373\\
68.375	0.1845	3.50605833999755\\
68.375	0.18816	3.69981496030244\\
68.375	0.19182	3.89789163349842\\
68.375	0.19548	4.10028835958548\\
68.375	0.19914	4.30700513856362\\
68.375	0.2028	4.51804197043285\\
68.375	0.20646	4.73339885519316\\
68.375	0.21012	4.95307579284456\\
68.375	0.21378	5.17707278338704\\
68.375	0.21744	5.40538982682059\\
68.375	0.2211	5.63802692314524\\
68.375	0.22476	5.87498407236096\\
68.375	0.22842	6.11626127446777\\
68.375	0.23208	6.36185852946566\\
68.375	0.23574	6.61177583735464\\
68.375	0.2394	6.86601319813468\\
68.375	0.24306	7.12457061180584\\
68.375	0.24672	7.38744807836806\\
68.375	0.25038	7.65464559782137\\
68.375	0.25404	7.92616317016576\\
68.375	0.2577	8.20200079540123\\
68.375	0.26136	8.48215847352778\\
68.375	0.26502	8.76663620454542\\
68.375	0.26868	9.05543398845414\\
68.375	0.27234	9.34855182525396\\
68.375	0.276	9.64598971494484\\
68.75	0.093	0.0378307180970312\\
68.75	0.09666	0.127534206333434\\
68.75	0.10032	0.221557747460924\\
68.75	0.10398	0.319901341479497\\
68.75	0.10764	0.42256498838915\\
68.75	0.1113	0.529548688189889\\
68.75	0.11496	0.640852440881712\\
68.75	0.11862	0.756476246464617\\
68.75	0.12228	0.876420104938602\\
68.75	0.12594	1.00068401630367\\
68.75	0.1296	1.12926798055983\\
68.75	0.13326	1.26217199770707\\
68.75	0.13692	1.39939606774539\\
68.75	0.14058	1.54094019067479\\
68.75	0.14424	1.68680436649528\\
68.75	0.1479	1.83698859520685\\
68.75	0.15156	1.9914928768095\\
68.75	0.15522	2.15031721130324\\
68.75	0.15888	2.31346159868805\\
68.75	0.16254	2.48092603896396\\
68.75	0.1662	2.65271053213094\\
68.75	0.16986	2.82881507818901\\
68.75	0.17352	3.00923967713817\\
68.75	0.17718	3.1939843289784\\
68.75	0.18084	3.38304903370971\\
68.75	0.1845	3.57643379133212\\
68.75	0.18816	3.7741386018456\\
68.75	0.19182	3.97616346525017\\
68.75	0.19548	4.18250838154582\\
68.75	0.19914	4.39317335073255\\
68.75	0.2028	4.60815837281037\\
68.75	0.20646	4.82746344777927\\
68.75	0.21012	5.05108857563926\\
68.75	0.21378	5.27903375639031\\
68.75	0.21744	5.51129899003246\\
68.75	0.2211	5.7478842765657\\
68.75	0.22476	5.98878961599001\\
68.75	0.22842	6.2340150083054\\
68.75	0.23208	6.48356045351189\\
68.75	0.23574	6.73742595160945\\
68.75	0.2394	6.9956115025981\\
68.75	0.24306	7.25811710647784\\
68.75	0.24672	7.52494276324865\\
68.75	0.25038	7.79608847291055\\
68.75	0.25404	8.07155423546353\\
68.75	0.2577	8.35134005090759\\
68.75	0.26136	8.63544591924274\\
68.75	0.26502	8.92387184046897\\
68.75	0.26868	9.21661781458627\\
68.75	0.27234	9.51368384159468\\
68.75	0.276	9.81506992149415\\
69.125	0.093	0.0084562077343131\\
69.125	0.09666	0.102107886179309\\
69.125	0.10032	0.200079617515386\\
69.125	0.10398	0.302371401742551\\
69.125	0.10764	0.408983238860794\\
69.125	0.1113	0.519915128870123\\
69.125	0.11496	0.635167071770535\\
69.125	0.11862	0.75473906756203\\
69.125	0.12228	0.878631116244605\\
69.125	0.12594	1.00684321781827\\
69.125	0.1296	1.13937537228301\\
69.125	0.13326	1.27622757963883\\
69.125	0.13692	1.41739983988574\\
69.125	0.14058	1.56289215302374\\
69.125	0.14424	1.71270451905282\\
69.125	0.1479	1.86683693797297\\
69.125	0.15156	2.02528940978422\\
69.125	0.15522	2.18806193448655\\
69.125	0.15888	2.35515451207995\\
69.125	0.16254	2.52656714256445\\
69.125	0.1662	2.70229982594002\\
69.125	0.16986	2.88235256220668\\
69.125	0.17352	3.06672535136442\\
69.125	0.17718	3.25541819341324\\
69.125	0.18084	3.44843108835315\\
69.125	0.1845	3.64576403618414\\
69.125	0.18816	3.84741703690621\\
69.125	0.19182	4.05339009051937\\
69.125	0.19548	4.26368319702361\\
69.125	0.19914	4.47829635641893\\
69.125	0.2028	4.69722956870534\\
69.125	0.20646	4.92048283388283\\
69.125	0.21012	5.14805615195141\\
69.125	0.21378	5.37994952291106\\
69.125	0.21744	5.6161629467618\\
69.125	0.2211	5.85669642350362\\
69.125	0.22476	6.10154995313652\\
69.125	0.22842	6.3507235356605\\
69.125	0.23208	6.60421717107558\\
69.125	0.23574	6.86203085938173\\
69.125	0.2394	7.12416460057897\\
69.125	0.24306	7.39061839466729\\
69.125	0.24672	7.6613922416467\\
69.125	0.25038	7.93648614151719\\
69.125	0.25404	8.21590009427876\\
69.125	0.2577	8.4996340999314\\
69.125	0.26136	8.78768815847515\\
69.125	0.26502	9.08006226990996\\
69.125	0.26868	9.37675643423586\\
69.125	0.27234	9.67777065145285\\
69.125	0.276	9.98310492156092\\
69.5	0.093	-0.0219635091109245\\
69.5	0.09666	0.0756363595426579\\
69.5	0.10032	0.177556281087327\\
69.5	0.10398	0.283796255523079\\
69.5	0.10764	0.394356282849911\\
69.5	0.1113	0.50923636306783\\
69.5	0.11496	0.628436496176831\\
69.5	0.11862	0.751956682176916\\
69.5	0.12228	0.879796921068084\\
69.5	0.12594	1.01195721285033\\
69.5	0.1296	1.14843755752367\\
69.5	0.13326	1.28923795508808\\
69.5	0.13692	1.43435840554358\\
69.5	0.14058	1.58379890889017\\
69.5	0.14424	1.73755946512783\\
69.5	0.1479	1.89564007425658\\
69.5	0.15156	2.05804073627641\\
69.5	0.15522	2.22476145118733\\
69.5	0.15888	2.39580221898932\\
69.5	0.16254	2.57116303968241\\
69.5	0.1662	2.75084391326657\\
69.5	0.16986	2.93484483974182\\
69.5	0.17352	3.12316581910815\\
69.5	0.17718	3.31580685136557\\
69.5	0.18084	3.51276793651406\\
69.5	0.1845	3.71404907455365\\
69.5	0.18816	3.9196502654843\\
69.5	0.19182	4.12957150930606\\
69.5	0.19548	4.34381280601888\\
69.5	0.19914	4.56237415562279\\
69.5	0.2028	4.78525555811779\\
69.5	0.20646	5.01245701350387\\
69.5	0.21012	5.24397852178103\\
69.5	0.21378	5.47982008294928\\
69.5	0.21744	5.7199816970086\\
69.5	0.2211	5.96446336395902\\
69.5	0.22476	6.21326508380052\\
69.5	0.22842	6.46638685653309\\
69.5	0.23208	6.72382868215676\\
69.5	0.23574	6.9855905606715\\
69.5	0.2394	7.25167249207732\\
69.5	0.24306	7.52207447637423\\
69.5	0.24672	7.79679651356223\\
69.5	0.25038	8.07583860364131\\
69.5	0.25404	8.35920074661146\\
69.5	0.2577	8.6468829424727\\
69.5	0.26136	8.93888519122503\\
69.5	0.26502	9.23520749286843\\
69.5	0.26868	9.53584984740293\\
69.5	0.27234	9.84081225482851\\
69.5	0.276	10.1500947151452\\
69.875	0.093	-0.0534284324386887\\
69.875	0.09666	0.0481196264234833\\
69.875	0.10032	0.153987738176742\\
69.875	0.10398	0.264175902821084\\
69.875	0.10764	0.378684120356505\\
69.875	0.1113	0.497512390783014\\
69.875	0.11496	0.620660714100605\\
69.875	0.11862	0.748129090309279\\
69.875	0.12228	0.879917519409033\\
69.875	0.12594	1.01602600139987\\
69.875	0.1296	1.1564545362818\\
69.875	0.13326	1.3012031240548\\
69.875	0.13692	1.45027176471889\\
69.875	0.14058	1.60366045827407\\
69.875	0.14424	1.76136920472032\\
69.875	0.1479	1.92339800405766\\
69.875	0.15156	2.08974685628608\\
69.875	0.15522	2.26041576140559\\
69.875	0.15888	2.43540471941617\\
69.875	0.16254	2.61471373031785\\
69.875	0.1662	2.7983427941106\\
69.875	0.16986	2.98629191079443\\
69.875	0.17352	3.17856108036936\\
69.875	0.17718	3.37515030283536\\
69.875	0.18084	3.57605957819244\\
69.875	0.1845	3.78128890644062\\
69.875	0.18816	3.99083828757987\\
69.875	0.19182	4.20470772161021\\
69.875	0.19548	4.42289720853163\\
69.875	0.19914	4.64540674834413\\
69.875	0.2028	4.87223634104772\\
69.875	0.20646	5.10338598664239\\
69.875	0.21012	5.33885568512814\\
69.875	0.21378	5.57864543650497\\
69.875	0.21744	5.82275524077288\\
69.875	0.2211	6.07118509793189\\
69.875	0.22476	6.32393500798197\\
69.875	0.22842	6.58100497092314\\
69.875	0.23208	6.8423949867554\\
69.875	0.23574	7.10810505547873\\
69.875	0.2394	7.37813517709314\\
69.875	0.24306	7.65248535159864\\
69.875	0.24672	7.93115557899522\\
69.875	0.25038	8.21414585928289\\
69.875	0.25404	8.50145619246164\\
69.875	0.2577	8.79308657853147\\
69.875	0.26136	9.08903701749239\\
69.875	0.26502	9.38930750934439\\
69.875	0.26868	9.69389805408747\\
69.875	0.27234	10.0028086517216\\
69.875	0.276	10.3160393022469\\
70.25	0.093	-0.085938562248983\\
70.25	0.09666	0.0195576868217786\\
70.25	0.10032	0.129373988783627\\
70.25	0.10398	0.243510343636558\\
70.25	0.10764	0.361966751380569\\
70.25	0.1113	0.484743212015667\\
70.25	0.11496	0.611839725541852\\
70.25	0.11862	0.743256291959115\\
70.25	0.12228	0.878992911267459\\
70.25	0.12594	1.01904958346689\\
70.25	0.1296	1.1634263085574\\
70.25	0.13326	1.312123086539\\
70.25	0.13692	1.46513991741168\\
70.25	0.14058	1.62247680117544\\
70.25	0.14424	1.78413373783029\\
70.25	0.1479	1.95011072737621\\
70.25	0.15156	2.12040776981322\\
70.25	0.15522	2.29502486514132\\
70.25	0.15888	2.4739620133605\\
70.25	0.16254	2.65721921447076\\
70.25	0.1662	2.8447964684721\\
70.25	0.16986	3.03669377536453\\
70.25	0.17352	3.23291113514804\\
70.25	0.17718	3.43344854782263\\
70.25	0.18084	3.63830601338831\\
70.25	0.1845	3.84748353184507\\
70.25	0.18816	4.06098110319291\\
70.25	0.19182	4.27879872743184\\
70.25	0.19548	4.50093640456185\\
70.25	0.19914	4.72739413458293\\
70.25	0.2028	4.95817191749512\\
70.25	0.20646	5.19326975329837\\
70.25	0.21012	5.43268764199272\\
70.25	0.21378	5.67642558357814\\
70.25	0.21744	5.92448357805464\\
70.25	0.2211	6.17686162542224\\
70.25	0.22476	6.43355972568091\\
70.25	0.22842	6.69457787883067\\
70.25	0.23208	6.95991608487151\\
70.25	0.23574	7.22957434380343\\
70.25	0.2394	7.50355265562643\\
70.25	0.24306	7.78185102034053\\
70.25	0.24672	8.0644694379457\\
70.25	0.25038	8.35140790844196\\
70.25	0.25404	8.6426664318293\\
70.25	0.2577	8.93824500810771\\
70.25	0.26136	9.23814363727723\\
70.25	0.26502	9.54236231933781\\
70.25	0.26868	9.85090105428948\\
70.25	0.27234	10.1637598421322\\
70.25	0.276	10.4809386828661\\
70.625	0.093	-0.119493898541811\\
70.625	0.09666	-0.0100494592624598\\
70.625	0.10032	0.103715032907978\\
70.625	0.10398	0.221799577969499\\
70.625	0.10764	0.3442041759221\\
70.625	0.1113	0.470928826765787\\
70.625	0.11496	0.601973530500558\\
70.625	0.11862	0.737338287126411\\
70.625	0.12228	0.877023096643348\\
70.625	0.12594	1.02102795905137\\
70.625	0.1296	1.16935287435047\\
70.625	0.13326	1.32199784254065\\
70.625	0.13692	1.47896286362192\\
70.625	0.14058	1.64024793759427\\
70.625	0.14424	1.80585306445771\\
70.625	0.1479	1.97577824421223\\
70.625	0.15156	2.15002347685783\\
70.625	0.15522	2.32858876239451\\
70.625	0.15888	2.51147410082228\\
70.625	0.16254	2.69867949214113\\
70.625	0.1662	2.89020493635106\\
70.625	0.16986	3.08605043345208\\
70.625	0.17352	3.28621598344418\\
70.625	0.17718	3.49070158632736\\
70.625	0.18084	3.69950724210162\\
70.625	0.1845	3.91263295076698\\
70.625	0.18816	4.13007871232341\\
70.625	0.19182	4.35184452677093\\
70.625	0.19548	4.57793039410953\\
70.625	0.19914	4.8083363143392\\
70.625	0.2028	5.04306228745998\\
70.625	0.20646	5.28210831347182\\
70.625	0.21012	5.52547439237476\\
70.625	0.21378	5.77316052416877\\
70.625	0.21744	6.02516670885386\\
70.625	0.2211	6.28149294643004\\
70.625	0.22476	6.54213923689731\\
70.625	0.22842	6.80710558025565\\
70.625	0.23208	7.07639197650508\\
70.625	0.23574	7.3499984256456\\
70.625	0.2394	7.62792492767719\\
70.625	0.24306	7.91017148259987\\
70.625	0.24672	8.19673809041363\\
70.625	0.25038	8.48762475111849\\
70.625	0.25404	8.78283146471441\\
70.625	0.2577	9.08235823120142\\
70.625	0.26136	9.38620505057952\\
70.625	0.26502	9.69437192284869\\
70.625	0.26868	10.006858848009\\
70.625	0.27234	10.3236658260603\\
70.625	0.276	10.6447928570027\\
71	0.093	-0.154094441317162\\
71	0.09666	-0.0407018118292213\\
71	0.10032	0.0770108705498065\\
71	0.10398	0.199043605819917\\
71	0.10764	0.325396393981111\\
71	0.1113	0.456069235033388\\
71	0.11496	0.591062128976748\\
71	0.11862	0.730375075811191\\
71	0.12228	0.874008075536714\\
71	0.12594	1.02196112815333\\
71	0.1296	1.17423423366102\\
71	0.13326	1.33082739205979\\
71	0.13692	1.49174060334965\\
71	0.14058	1.65697386753059\\
71	0.14424	1.82652718460262\\
71	0.1479	2.00040055456572\\
71	0.15156	2.17859397741992\\
71	0.15522	2.36110745316519\\
71	0.15888	2.54794098180154\\
71	0.16254	2.73909456332899\\
71	0.1662	2.93456819774751\\
71	0.16986	3.13436188505711\\
71	0.17352	3.33847562525781\\
71	0.17718	3.54690941834957\\
71	0.18084	3.75966326433243\\
71	0.1845	3.97673716320638\\
71	0.18816	4.19813111497139\\
71	0.19182	4.42384511962751\\
71	0.19548	4.65387917717469\\
71	0.19914	4.88823328761296\\
71	0.2028	5.12690745094232\\
71	0.20646	5.36990166716275\\
71	0.21012	5.61721593627427\\
71	0.21378	5.86885025827688\\
71	0.21744	6.12480463317057\\
71	0.2211	6.38507906095534\\
71	0.22476	6.64967354163119\\
71	0.22842	6.91858807519812\\
71	0.23208	7.19182266165615\\
71	0.23574	7.46937730100524\\
71	0.2394	7.75125199324543\\
71	0.24306	8.0374467383767\\
71	0.24672	8.32796153639905\\
71	0.25038	8.62279638731249\\
71	0.25404	8.921951291117\\
71	0.2577	9.2254262478126\\
71	0.26136	9.5332212573993\\
71	0.26502	9.84533631987706\\
71	0.26868	10.1617714352459\\
71	0.27234	10.4825266035058\\
71	0.276	10.8076018246569\\
71.375	0.093	-0.189740190575043\\
71.375	0.09666	-0.0723993708785129\\
71.375	0.10032	0.0492615017091045\\
71.375	0.10398	0.175242427187804\\
71.375	0.10764	0.305543405557584\\
71.375	0.1113	0.440164436818451\\
71.375	0.11496	0.579105520970404\\
71.375	0.11862	0.722366658013437\\
71.375	0.12228	0.869947847947549\\
71.375	0.12594	1.02184909077275\\
71.375	0.1296	1.17807038648903\\
71.375	0.13326	1.33861173509639\\
71.375	0.13692	1.50347313659484\\
71.375	0.14058	1.67265459098438\\
71.375	0.14424	1.84615609826499\\
71.375	0.1479	2.02397765843668\\
71.375	0.15156	2.20611927149946\\
71.375	0.15522	2.39258093745333\\
71.375	0.15888	2.58336265629827\\
71.375	0.16254	2.77846442803431\\
71.375	0.1662	2.97788625266142\\
71.375	0.16986	3.18162813017961\\
71.375	0.17352	3.3896900605889\\
71.375	0.17718	3.60207204388925\\
71.375	0.18084	3.8187740800807\\
71.375	0.1845	4.03979616916323\\
71.375	0.18816	4.26513831113684\\
71.375	0.19182	4.49480050600154\\
71.375	0.19548	4.72878275375732\\
71.375	0.19914	4.96708505440418\\
71.375	0.2028	5.20970740794212\\
71.375	0.20646	5.45664981437115\\
71.375	0.21012	5.70791227369126\\
71.375	0.21378	5.96349478590245\\
71.375	0.21744	6.22339735100473\\
71.375	0.2211	6.48761996899809\\
71.375	0.22476	6.75616263988253\\
71.375	0.22842	7.02902536365805\\
71.375	0.23208	7.30620814032467\\
71.375	0.23574	7.58771096988236\\
71.375	0.2394	7.87353385233113\\
71.375	0.24306	8.163676787671\\
71.375	0.24672	8.45813977590193\\
71.375	0.25038	8.75692281702397\\
71.375	0.25404	9.06002591103707\\
71.375	0.2577	9.36744905794126\\
71.375	0.26136	9.67919225773653\\
71.375	0.26502	9.99525551042289\\
71.375	0.26868	10.3156388160003\\
71.375	0.27234	10.6403421744689\\
71.375	0.276	10.9693655858285\\
71.75	0.093	-0.226431146315458\\
71.75	0.09666	-0.105142136410335\\
71.75	0.10032	0.0204669263858723\\
71.75	0.10398	0.150396042073162\\
71.75	0.10764	0.284645210651531\\
71.75	0.1113	0.423214432120988\\
71.75	0.11496	0.566103706481527\\
71.75	0.11862	0.713313033733153\\
71.75	0.12228	0.864842413875855\\
71.75	0.12594	1.02069184690964\\
71.75	0.1296	1.18086133283451\\
71.75	0.13326	1.34535087165047\\
71.75	0.13692	1.51416046335751\\
71.75	0.14058	1.68729010795563\\
71.75	0.14424	1.86473980544483\\
71.75	0.1479	2.04650955582512\\
71.75	0.15156	2.23259935909649\\
71.75	0.15522	2.42300921525894\\
71.75	0.15888	2.61773912431248\\
71.75	0.16254	2.8167890862571\\
71.75	0.1662	3.0201591010928\\
71.75	0.16986	3.22784916881958\\
71.75	0.17352	3.43985928943745\\
71.75	0.17718	3.6561894629464\\
71.75	0.18084	3.87683968934644\\
71.75	0.1845	4.10180996863756\\
71.75	0.18816	4.33110030081976\\
71.75	0.19182	4.56471068589305\\
71.75	0.19548	4.80264112385741\\
71.75	0.19914	5.04489161471286\\
71.75	0.2028	5.2914621584594\\
71.75	0.20646	5.54235275509702\\
71.75	0.21012	5.79756340462572\\
71.75	0.21378	6.0570941070455\\
71.75	0.21744	6.32094486235636\\
71.75	0.2211	6.58911567055832\\
71.75	0.22476	6.86160653165135\\
71.75	0.22842	7.13841744563546\\
71.75	0.23208	7.41954841251066\\
71.75	0.23574	7.70499943227694\\
71.75	0.2394	7.9947705049343\\
71.75	0.24306	8.28886163048275\\
71.75	0.24672	8.58727280892228\\
71.75	0.25038	8.8900040402529\\
71.75	0.25404	9.1970553244746\\
71.75	0.2577	9.50842666158738\\
71.75	0.26136	9.82411805159124\\
71.75	0.26502	10.1441294944862\\
71.75	0.26868	10.4684609902722\\
71.75	0.27234	10.7971125389493\\
71.75	0.276	11.1300841405175\\
72.125	0.093	-0.264167308538396\\
72.125	0.09666	-0.138930108424687\\
72.125	0.10032	-0.00937285541988997\\
72.125	0.10398	0.124504450475989\\
72.125	0.10764	0.262701809262952\\
72.125	0.1113	0.405219220940998\\
72.125	0.11496	0.552056685510127\\
72.125	0.11862	0.703214202970338\\
72.125	0.12228	0.85869177332163\\
72.125	0.12594	1.01848939656401\\
72.125	0.1296	1.18260707269747\\
72.125	0.13326	1.35104480172201\\
72.125	0.13692	1.52380258363764\\
72.125	0.14058	1.70088041844435\\
72.125	0.14424	1.88227830614215\\
72.125	0.1479	2.06799624673102\\
72.125	0.15156	2.25803424021098\\
72.125	0.15522	2.45239228658203\\
72.125	0.15888	2.65107038584415\\
72.125	0.16254	2.85406853799736\\
72.125	0.1662	3.06138674304166\\
72.125	0.16986	3.27302500097702\\
72.125	0.17352	3.48898331180349\\
72.125	0.17718	3.70926167552103\\
72.125	0.18084	3.93386009212965\\
72.125	0.1845	4.16277856162936\\
72.125	0.18816	4.39601708402015\\
72.125	0.19182	4.63357565930203\\
72.125	0.19548	4.87545428747499\\
72.125	0.19914	5.12165296853902\\
72.125	0.2028	5.37217170249415\\
72.125	0.20646	5.62701048934035\\
72.125	0.21012	5.88616932907765\\
72.125	0.21378	6.14964822170602\\
72.125	0.21744	6.41744716722547\\
72.125	0.2211	6.68956616563602\\
72.125	0.22476	6.96600521693763\\
72.125	0.22842	7.24676432113034\\
72.125	0.23208	7.53184347821413\\
72.125	0.23574	7.821242688189\\
72.125	0.2394	8.11496195105495\\
72.125	0.24306	8.41300126681199\\
72.125	0.24672	8.71536063546011\\
72.125	0.25038	9.02204005699932\\
72.125	0.25404	9.33303953142961\\
72.125	0.2577	9.64835905875098\\
72.125	0.26136	9.96799863896343\\
72.125	0.26502	10.291958272067\\
72.125	0.26868	10.6202379580616\\
72.125	0.27234	10.9528376969473\\
72.125	0.276	11.2897574887241\\
72.5	0.093	-0.302948677243864\\
72.5	0.09666	-0.173763286921565\\
72.5	0.10032	-0.0402578437081789\\
72.5	0.10398	0.0975676523962898\\
72.5	0.10764	0.239713201391839\\
72.5	0.1113	0.386178803278474\\
72.5	0.11496	0.536964458056196\\
72.5	0.11862	0.692070165724997\\
72.5	0.12228	0.851495926284879\\
72.5	0.12594	1.01524173973585\\
72.5	0.1296	1.1833076060779\\
72.5	0.13326	1.35569352531103\\
72.5	0.13692	1.53239949743525\\
72.5	0.14058	1.71342552245055\\
72.5	0.14424	1.89877160035693\\
72.5	0.1479	2.0884377311544\\
72.5	0.15156	2.28242391484295\\
72.5	0.15522	2.48073015142258\\
72.5	0.15888	2.68335644089329\\
72.5	0.16254	2.8903027832551\\
72.5	0.1662	3.10156917850798\\
72.5	0.16986	3.31715562665194\\
72.5	0.17352	3.53706212768699\\
72.5	0.17718	3.76128868161312\\
72.5	0.18084	3.98983528843033\\
72.5	0.1845	4.22270194813864\\
72.5	0.18816	4.45988866073801\\
72.5	0.19182	4.70139542622848\\
72.5	0.19548	4.94722224461002\\
72.5	0.19914	5.19736911588265\\
72.5	0.2028	5.45183604004637\\
72.5	0.20646	5.71062301710116\\
72.5	0.21012	5.97373004704705\\
72.5	0.21378	6.241157129884\\
72.5	0.21744	6.51290426561205\\
72.5	0.2211	6.78897145423119\\
72.5	0.22476	7.06935869574139\\
72.5	0.22842	7.35406599014268\\
72.5	0.23208	7.64309333743506\\
72.5	0.23574	7.93644073761853\\
72.5	0.2394	8.23410819069307\\
72.5	0.24306	8.5360956966587\\
72.5	0.24672	8.84240325551541\\
72.5	0.25038	9.15303086726321\\
72.5	0.25404	9.46797853190208\\
72.5	0.2577	9.78724624943204\\
72.5	0.26136	10.1108340198531\\
72.5	0.26502	10.4387418431652\\
72.5	0.26868	10.7709697193684\\
72.5	0.27234	11.1075176484627\\
72.5	0.276	11.4483856304481\\
72.875	0.093	-0.342775252431855\\
72.875	0.09666	-0.209641671900967\\
72.875	0.10032	-0.0721880384789908\\
72.875	0.10398	0.0695856478340711\\
72.875	0.10764	0.215679387038209\\
72.875	0.1113	0.366093179133435\\
72.875	0.11496	0.520827024119743\\
72.875	0.11862	0.679880921997134\\
72.875	0.12228	0.843254872765608\\
72.875	0.12594	1.01094887642517\\
72.875	0.1296	1.18296293297581\\
72.875	0.13326	1.35929704241753\\
72.875	0.13692	1.53995120475034\\
72.875	0.14058	1.72492541997423\\
72.875	0.14424	1.9142196880892\\
72.875	0.1479	2.10783400909525\\
72.875	0.15156	2.30576838299239\\
72.875	0.15522	2.50802280978061\\
72.875	0.15888	2.71459728945992\\
72.875	0.16254	2.92549182203032\\
72.875	0.1662	3.14070640749178\\
72.875	0.16986	3.36024104584433\\
72.875	0.17352	3.58409573708797\\
72.875	0.17718	3.81227048122269\\
72.875	0.18084	4.0447652782485\\
72.875	0.1845	4.28158012816538\\
72.875	0.18816	4.52271503097336\\
72.875	0.19182	4.76816998667241\\
72.875	0.19548	5.01794499526255\\
72.875	0.19914	5.27204005674377\\
72.875	0.2028	5.53045517111607\\
72.875	0.20646	5.79319033837946\\
72.875	0.21012	6.06024555853392\\
72.875	0.21378	6.33162083157948\\
72.875	0.21744	6.60731615751611\\
72.875	0.2211	6.88733153634383\\
72.875	0.22476	7.17166696806263\\
72.875	0.22842	7.46032245267252\\
72.875	0.23208	7.75329799017348\\
72.875	0.23574	8.05059358056554\\
72.875	0.2394	8.35220922384866\\
72.875	0.24306	8.65814492002289\\
72.875	0.24672	8.96840066908819\\
72.875	0.25038	9.28297647104457\\
72.875	0.25404	9.60187232589204\\
72.875	0.2577	9.92508823363058\\
72.875	0.26136	10.2526241942602\\
72.875	0.26502	10.5844802077809\\
72.875	0.26868	10.9206562741927\\
72.875	0.27234	11.2611523934956\\
72.875	0.276	11.6059685656896\\
73.25	0.093	-0.383647034102384\\
73.25	0.09666	-0.246565263362905\\
73.25	0.10032	-0.10516343973234\\
73.25	0.10398	0.0405584367893079\\
73.25	0.10764	0.190600366202039\\
73.25	0.1113	0.344962348505854\\
73.25	0.11496	0.503644383700752\\
73.25	0.11862	0.666646471786732\\
73.25	0.12228	0.833968612763793\\
73.25	0.12594	1.00561080663194\\
73.25	0.1296	1.18157305339117\\
73.25	0.13326	1.36185535304148\\
73.25	0.13692	1.54645770558288\\
73.25	0.14058	1.73538011101536\\
73.25	0.14424	1.92862256933893\\
73.25	0.1479	2.12618508055357\\
73.25	0.15156	2.3280676446593\\
73.25	0.15522	2.53427026165611\\
73.25	0.15888	2.744792931544\\
73.25	0.16254	2.95963565432298\\
73.25	0.1662	3.17879842999304\\
73.25	0.16986	3.40228125855419\\
73.25	0.17352	3.63008414000642\\
73.25	0.17718	3.86220707434973\\
73.25	0.18084	4.09865006158412\\
73.25	0.1845	4.3394131017096\\
73.25	0.18816	4.58449619472616\\
73.25	0.19182	4.8338993406338\\
73.25	0.19548	5.08762253943253\\
73.25	0.19914	5.34566579112234\\
73.25	0.2028	5.60802909570323\\
73.25	0.20646	5.87471245317521\\
73.25	0.21012	6.14571586353826\\
73.25	0.21378	6.42103932679241\\
73.25	0.21744	6.70068284293763\\
73.25	0.2211	6.98464641197394\\
73.25	0.22476	7.27293003390133\\
73.25	0.22842	7.5655337087198\\
73.25	0.23208	7.86245743642936\\
73.25	0.23574	8.16370121703\\
73.25	0.2394	8.46926505052172\\
73.25	0.24306	8.77914893690453\\
73.25	0.24672	9.09335287617842\\
73.25	0.25038	9.4118768683434\\
73.25	0.25404	9.73472091339945\\
73.25	0.2577	10.0618850113466\\
73.25	0.26136	10.3933691621848\\
73.25	0.26502	10.7291733659141\\
73.25	0.26868	11.0692976225345\\
73.25	0.27234	11.413741932046\\
73.25	0.276	11.7625062944485\\
73.625	0.093	-0.425564022255435\\
73.625	0.09666	-0.284534061307367\\
73.625	0.10032	-0.139184047468212\\
73.625	0.10398	0.0104860192620289\\
73.625	0.10764	0.164476138883346\\
73.625	0.1113	0.322786311395751\\
73.625	0.11496	0.485416536799242\\
73.625	0.11862	0.652366815093812\\
73.625	0.12228	0.823637146279462\\
73.625	0.12594	0.999227530356198\\
73.625	0.1296	1.17913796732402\\
73.625	0.13326	1.36336845718292\\
73.625	0.13692	1.55191899993291\\
73.625	0.14058	1.74478959557398\\
73.625	0.14424	1.94198024410613\\
73.625	0.1479	2.14349094552936\\
73.625	0.15156	2.34932169984368\\
73.625	0.15522	2.55947250704909\\
73.625	0.15888	2.77394336714557\\
73.625	0.16254	2.99273428013314\\
73.625	0.1662	3.21584524601179\\
73.625	0.16986	3.44327626478152\\
73.625	0.17352	3.67502733644234\\
73.625	0.17718	3.91109846099424\\
73.625	0.18084	4.15148963843722\\
73.625	0.1845	4.39620086877129\\
73.625	0.18816	4.64523215199644\\
73.625	0.19182	4.89858348811268\\
73.625	0.19548	5.15625487711999\\
73.625	0.19914	5.41824631901838\\
73.625	0.2028	5.68455781380787\\
73.625	0.20646	5.95518936148843\\
73.625	0.21012	6.23014096206009\\
73.625	0.21378	6.50941261552282\\
73.625	0.21744	6.79300432187663\\
73.625	0.2211	7.08091608112153\\
73.625	0.22476	7.37314789325751\\
73.625	0.22842	7.66969975828457\\
73.625	0.23208	7.97057167620272\\
73.625	0.23574	8.27576364701195\\
73.625	0.2394	8.58527567071226\\
73.625	0.24306	8.89910774730366\\
73.625	0.24672	9.21725987678614\\
73.625	0.25038	9.53973205915971\\
73.625	0.25404	9.86652429442436\\
73.625	0.2577	10.1976365825801\\
73.625	0.26136	10.5330689236269\\
73.625	0.26502	10.8728213175648\\
73.625	0.26868	11.2168937643938\\
73.625	0.27234	11.5652862641138\\
73.625	0.276	11.917998816725\\
74	0.093	-0.468526216891017\\
74	0.09666	-0.323548065734359\\
74	0.10032	-0.174249861686615\\
74	0.10398	-0.0206316047477875\\
74	0.10764	0.137306705082123\\
74	0.1113	0.299565067803117\\
74	0.11496	0.466143483415194\\
74	0.11862	0.637041951918357\\
74	0.12228	0.812260473312597\\
74	0.12594	0.991799047597923\\
74	0.1296	1.17565767477433\\
74	0.13326	1.36383635484182\\
74	0.13692	1.5563350878004\\
74	0.14058	1.75315387365006\\
74	0.14424	1.9542927123908\\
74	0.1479	2.15975160402263\\
74	0.15156	2.36953054854553\\
74	0.15522	2.58362954595953\\
74	0.15888	2.8020485962646\\
74	0.16254	3.02478769946076\\
74	0.1662	3.251846855548\\
74	0.16986	3.48322606452632\\
74	0.17352	3.71892532639573\\
74	0.17718	3.95894464115622\\
74	0.18084	4.20328400880779\\
74	0.1845	4.45194342935045\\
74	0.18816	4.70492290278418\\
74	0.19182	4.96222242910901\\
74	0.19548	5.22384200832491\\
74	0.19914	5.4897816404319\\
74	0.2028	5.76004132542998\\
74	0.20646	6.03462106331913\\
74	0.21012	6.31352085409937\\
74	0.21378	6.59674069777069\\
74	0.21744	6.88428059433309\\
74	0.2211	7.17614054378658\\
74	0.22476	7.47232054613116\\
74	0.22842	7.7728206013668\\
74	0.23208	8.07764070949355\\
74	0.23574	8.38678087051136\\
74	0.2394	8.70024108442026\\
74	0.24306	9.01802135122026\\
74	0.24672	9.34012167091132\\
74	0.25038	9.66654204349348\\
74	0.25404	9.99728246896671\\
74	0.2577	10.332342947331\\
74	0.26136	10.6717234785864\\
74	0.26502	11.0154240627329\\
74	0.26868	11.3634446997705\\
74	0.27234	11.7157853896991\\
74	0.276	12.0724461325189\\
};
\end{axis}

\begin{axis}[%
width=4.527496cm,
height=3.870968cm,
at={(0cm,16.129032cm)},
scale only axis,
xmin=56,
xmax=74,
tick align=outside,
xlabel={$L_{cut}$},
xmajorgrids,
ymin=0.093,
ymax=0.276,
ylabel={$D_{rlx}$},
ymajorgrids,
zmin=-200.569447084128,
zmax=0,
zlabel={$x_4,x_4$},
zmajorgrids,
view={-140}{50},
legend style={at={(1.03,1)},anchor=north west,legend cell align=left,align=left,draw=white!15!black}
]
\addplot3[only marks,mark=*,mark options={},mark size=1.5000pt,color=mycolor1] plot table[row sep=crcr,]{%
74	0.123	-26.0353957891804\\
72	0.113	-20.9879322169279\\
61	0.095	-10.6920630070547\\
56	0.093	-10.9569379219045\\
};
\addplot3[only marks,mark=*,mark options={},mark size=1.5000pt,color=mycolor2] plot table[row sep=crcr,]{%
67	0.276	-191.779551108501\\
66	0.255	-157.643535964989\\
62	0.209	-87.4196213968237\\
57	0.193	-69.8013569503948\\
};
\addplot3[only marks,mark=*,mark options={},mark size=1.5000pt,color=black] plot table[row sep=crcr,]{%
69	0.104	-15.5770582620907\\
};
\addplot3[only marks,mark=*,mark options={},mark size=1.5000pt,color=black] plot table[row sep=crcr,]{%
64	0.23	-116.694090391038\\
};

\addplot3[%
surf,
opacity=0.7,
shader=interp,
colormap={mymap}{[1pt] rgb(0pt)=(0.0901961,0.239216,0.0745098); rgb(1pt)=(0.0945149,0.242058,0.0739522); rgb(2pt)=(0.0988592,0.244894,0.0733566); rgb(3pt)=(0.103229,0.247724,0.0727241); rgb(4pt)=(0.107623,0.250549,0.0720557); rgb(5pt)=(0.112043,0.253367,0.0713525); rgb(6pt)=(0.116487,0.25618,0.0706154); rgb(7pt)=(0.120956,0.258986,0.0698456); rgb(8pt)=(0.125449,0.261787,0.0690441); rgb(9pt)=(0.129967,0.264581,0.0682118); rgb(10pt)=(0.134508,0.26737,0.06735); rgb(11pt)=(0.139074,0.270152,0.0664596); rgb(12pt)=(0.143663,0.272929,0.0655416); rgb(13pt)=(0.148275,0.275699,0.0645971); rgb(14pt)=(0.152911,0.278463,0.0636271); rgb(15pt)=(0.15757,0.281221,0.0626328); rgb(16pt)=(0.162252,0.283973,0.0616151); rgb(17pt)=(0.166957,0.286719,0.060575); rgb(18pt)=(0.171685,0.289458,0.0595136); rgb(19pt)=(0.176434,0.292191,0.0584321); rgb(20pt)=(0.181207,0.294918,0.0573313); rgb(21pt)=(0.186001,0.297639,0.0562123); rgb(22pt)=(0.190817,0.300353,0.0550763); rgb(23pt)=(0.195655,0.303061,0.0539242); rgb(24pt)=(0.200514,0.305763,0.052757); rgb(25pt)=(0.205395,0.308459,0.0515759); rgb(26pt)=(0.210296,0.311149,0.0503624); rgb(27pt)=(0.215212,0.313846,0.0490067); rgb(28pt)=(0.220142,0.316548,0.0475043); rgb(29pt)=(0.22509,0.319254,0.0458704); rgb(30pt)=(0.230056,0.321962,0.0441205); rgb(31pt)=(0.235042,0.324671,0.04227); rgb(32pt)=(0.240048,0.327379,0.0403343); rgb(33pt)=(0.245078,0.330085,0.0383287); rgb(34pt)=(0.250131,0.332786,0.0362688); rgb(35pt)=(0.25521,0.335482,0.0341698); rgb(36pt)=(0.260317,0.33817,0.0320472); rgb(37pt)=(0.265451,0.340849,0.0299163); rgb(38pt)=(0.270616,0.343517,0.0277927); rgb(39pt)=(0.275813,0.346172,0.0256916); rgb(40pt)=(0.281043,0.348814,0.0236284); rgb(41pt)=(0.286307,0.35144,0.0216186); rgb(42pt)=(0.291607,0.354048,0.0196776); rgb(43pt)=(0.296945,0.356637,0.0178207); rgb(44pt)=(0.302322,0.359206,0.0160634); rgb(45pt)=(0.307739,0.361753,0.0144211); rgb(46pt)=(0.313198,0.364275,0.0129091); rgb(47pt)=(0.318701,0.366772,0.0115428); rgb(48pt)=(0.324249,0.369242,0.0103377); rgb(49pt)=(0.329843,0.371682,0.00930909); rgb(50pt)=(0.335485,0.374093,0.00847245); rgb(51pt)=(0.341176,0.376471,0.00784314); rgb(52pt)=(0.346925,0.378826,0.00732741); rgb(53pt)=(0.352735,0.381168,0.00682184); rgb(54pt)=(0.358605,0.383497,0.00632729); rgb(55pt)=(0.364532,0.385812,0.00584464); rgb(56pt)=(0.370516,0.388113,0.00537476); rgb(57pt)=(0.376552,0.390399,0.00491852); rgb(58pt)=(0.38264,0.39267,0.00447681); rgb(59pt)=(0.388777,0.394925,0.00405048); rgb(60pt)=(0.394962,0.397164,0.00364042); rgb(61pt)=(0.401191,0.399386,0.00324749); rgb(62pt)=(0.407464,0.401592,0.00287258); rgb(63pt)=(0.413777,0.40378,0.00251655); rgb(64pt)=(0.420129,0.40595,0.00218028); rgb(65pt)=(0.426518,0.408102,0.00186463); rgb(66pt)=(0.432942,0.410234,0.00157049); rgb(67pt)=(0.439399,0.412348,0.00129873); rgb(68pt)=(0.445885,0.414441,0.00105022); rgb(69pt)=(0.452401,0.416515,0.000825833); rgb(70pt)=(0.458942,0.418567,0.000626441); rgb(71pt)=(0.465508,0.420599,0.00045292); rgb(72pt)=(0.472096,0.422609,0.000306141); rgb(73pt)=(0.478704,0.424596,0.000186979); rgb(74pt)=(0.485331,0.426562,9.63073e-05); rgb(75pt)=(0.491973,0.428504,3.49981e-05); rgb(76pt)=(0.498628,0.430422,3.92506e-06); rgb(77pt)=(0.505323,0.432315,0); rgb(78pt)=(0.512206,0.434168,0); rgb(79pt)=(0.519282,0.435983,0); rgb(80pt)=(0.526529,0.437764,0); rgb(81pt)=(0.533922,0.439512,0); rgb(82pt)=(0.54144,0.441232,0); rgb(83pt)=(0.549059,0.442927,0); rgb(84pt)=(0.556756,0.444599,0); rgb(85pt)=(0.564508,0.446252,0); rgb(86pt)=(0.572292,0.447889,0); rgb(87pt)=(0.580084,0.449514,0); rgb(88pt)=(0.587863,0.451129,0); rgb(89pt)=(0.595604,0.452737,0); rgb(90pt)=(0.603284,0.454343,0); rgb(91pt)=(0.610882,0.455948,0); rgb(92pt)=(0.618373,0.457556,0); rgb(93pt)=(0.625734,0.459171,0); rgb(94pt)=(0.632943,0.460795,0); rgb(95pt)=(0.639976,0.462432,0); rgb(96pt)=(0.64681,0.464084,0); rgb(97pt)=(0.653423,0.465756,0); rgb(98pt)=(0.659791,0.46745,0); rgb(99pt)=(0.665891,0.469169,0); rgb(100pt)=(0.6717,0.470916,0); rgb(101pt)=(0.677195,0.472696,0); rgb(102pt)=(0.682353,0.47451,0); rgb(103pt)=(0.687242,0.476355,0); rgb(104pt)=(0.691952,0.478225,0); rgb(105pt)=(0.696497,0.480118,0); rgb(106pt)=(0.700887,0.482033,0); rgb(107pt)=(0.705134,0.483968,0); rgb(108pt)=(0.709251,0.485921,0); rgb(109pt)=(0.713249,0.487891,0); rgb(110pt)=(0.71714,0.489876,0); rgb(111pt)=(0.720936,0.491875,0); rgb(112pt)=(0.724649,0.493887,0); rgb(113pt)=(0.72829,0.495909,0); rgb(114pt)=(0.731872,0.49794,0); rgb(115pt)=(0.735406,0.499979,0); rgb(116pt)=(0.738904,0.502025,0); rgb(117pt)=(0.742378,0.504075,0); rgb(118pt)=(0.74584,0.506128,0); rgb(119pt)=(0.749302,0.508182,0); rgb(120pt)=(0.752775,0.510237,0); rgb(121pt)=(0.756272,0.51229,0); rgb(122pt)=(0.759804,0.514339,0); rgb(123pt)=(0.763384,0.516385,0); rgb(124pt)=(0.767022,0.518424,0); rgb(125pt)=(0.770731,0.520455,0); rgb(126pt)=(0.774523,0.522478,0); rgb(127pt)=(0.77841,0.524489,0); rgb(128pt)=(0.782391,0.526491,0); rgb(129pt)=(0.786402,0.528496,0); rgb(130pt)=(0.790431,0.530506,0); rgb(131pt)=(0.794478,0.532521,0); rgb(132pt)=(0.798541,0.534539,0); rgb(133pt)=(0.802619,0.53656,0); rgb(134pt)=(0.806712,0.538584,0); rgb(135pt)=(0.81082,0.540609,0); rgb(136pt)=(0.81494,0.542635,0); rgb(137pt)=(0.819074,0.54466,0); rgb(138pt)=(0.823219,0.546686,0); rgb(139pt)=(0.827374,0.548709,0); rgb(140pt)=(0.831541,0.55073,0); rgb(141pt)=(0.835716,0.552749,0); rgb(142pt)=(0.8399,0.554763,0); rgb(143pt)=(0.844092,0.556774,0); rgb(144pt)=(0.848292,0.558779,0); rgb(145pt)=(0.852497,0.560778,0); rgb(146pt)=(0.856708,0.562771,0); rgb(147pt)=(0.860924,0.564756,0); rgb(148pt)=(0.865143,0.566733,0); rgb(149pt)=(0.869366,0.568701,0); rgb(150pt)=(0.873592,0.57066,0); rgb(151pt)=(0.877819,0.572608,0); rgb(152pt)=(0.882047,0.574545,0); rgb(153pt)=(0.886275,0.576471,0); rgb(154pt)=(0.890659,0.578362,0); rgb(155pt)=(0.895333,0.580203,0); rgb(156pt)=(0.900258,0.581999,0); rgb(157pt)=(0.905397,0.583755,0); rgb(158pt)=(0.910711,0.585479,0); rgb(159pt)=(0.916164,0.587176,0); rgb(160pt)=(0.921717,0.588852,0); rgb(161pt)=(0.927333,0.590513,0); rgb(162pt)=(0.932974,0.592166,0); rgb(163pt)=(0.938602,0.593815,0); rgb(164pt)=(0.94418,0.595468,0); rgb(165pt)=(0.949669,0.59713,0); rgb(166pt)=(0.955033,0.598808,0); rgb(167pt)=(0.960233,0.600507,0); rgb(168pt)=(0.965232,0.602233,0); rgb(169pt)=(0.969992,0.603992,0); rgb(170pt)=(0.974475,0.605791,0); rgb(171pt)=(0.978643,0.607636,0); rgb(172pt)=(0.98246,0.609532,0); rgb(173pt)=(0.985886,0.611486,0); rgb(174pt)=(0.988885,0.613503,0); rgb(175pt)=(0.991419,0.61559,0); rgb(176pt)=(0.99345,0.617753,0); rgb(177pt)=(0.99494,0.619997,0); rgb(178pt)=(0.995851,0.622329,0); rgb(179pt)=(0.996226,0.624763,0); rgb(180pt)=(0.996512,0.627352,0); rgb(181pt)=(0.996788,0.630095,0); rgb(182pt)=(0.997053,0.632982,0); rgb(183pt)=(0.997308,0.636004,0); rgb(184pt)=(0.997552,0.639152,0); rgb(185pt)=(0.997785,0.642416,0); rgb(186pt)=(0.998006,0.645786,0); rgb(187pt)=(0.998217,0.649253,0); rgb(188pt)=(0.998416,0.652807,0); rgb(189pt)=(0.998605,0.656439,0); rgb(190pt)=(0.998781,0.660138,0); rgb(191pt)=(0.998946,0.663897,0); rgb(192pt)=(0.9991,0.667704,0); rgb(193pt)=(0.999242,0.67155,0); rgb(194pt)=(0.999372,0.675427,0); rgb(195pt)=(0.99949,0.679323,0); rgb(196pt)=(0.999596,0.68323,0); rgb(197pt)=(0.99969,0.687139,0); rgb(198pt)=(0.999771,0.691039,0); rgb(199pt)=(0.999841,0.694921,0); rgb(200pt)=(0.999898,0.698775,0); rgb(201pt)=(0.999942,0.702592,0); rgb(202pt)=(0.999974,0.706363,0); rgb(203pt)=(0.999994,0.710077,0); rgb(204pt)=(1,0.713725,0); rgb(205pt)=(1,0.717341,0); rgb(206pt)=(1,0.720963,0); rgb(207pt)=(1,0.724591,0); rgb(208pt)=(1,0.728226,0); rgb(209pt)=(1,0.731867,0); rgb(210pt)=(1,0.735514,0); rgb(211pt)=(1,0.739167,0); rgb(212pt)=(1,0.742827,0); rgb(213pt)=(1,0.746493,0); rgb(214pt)=(1,0.750165,0); rgb(215pt)=(1,0.753843,0); rgb(216pt)=(1,0.757527,0); rgb(217pt)=(1,0.761217,0); rgb(218pt)=(1,0.764913,0); rgb(219pt)=(1,0.768615,0); rgb(220pt)=(1,0.772324,0); rgb(221pt)=(1,0.776038,0); rgb(222pt)=(1,0.779758,0); rgb(223pt)=(1,0.783484,0); rgb(224pt)=(1,0.787215,0); rgb(225pt)=(1,0.790953,0); rgb(226pt)=(1,0.794696,0); rgb(227pt)=(1,0.798445,0); rgb(228pt)=(1,0.8022,0); rgb(229pt)=(1,0.805961,0); rgb(230pt)=(1,0.809727,0); rgb(231pt)=(1,0.8135,0); rgb(232pt)=(1,0.817278,0); rgb(233pt)=(1,0.821063,0); rgb(234pt)=(1,0.824854,0); rgb(235pt)=(1,0.828652,0); rgb(236pt)=(1,0.832455,0); rgb(237pt)=(1,0.836265,0); rgb(238pt)=(1,0.840081,0); rgb(239pt)=(1,0.843903,0); rgb(240pt)=(1,0.847732,0); rgb(241pt)=(1,0.851566,0); rgb(242pt)=(1,0.855406,0); rgb(243pt)=(1,0.859253,0); rgb(244pt)=(1,0.863106,0); rgb(245pt)=(1,0.866964,0); rgb(246pt)=(1,0.870829,0); rgb(247pt)=(1,0.8747,0); rgb(248pt)=(1,0.878577,0); rgb(249pt)=(1,0.88246,0); rgb(250pt)=(1,0.886349,0); rgb(251pt)=(1,0.890243,0); rgb(252pt)=(1,0.894144,0); rgb(253pt)=(1,0.898051,0); rgb(254pt)=(1,0.901964,0); rgb(255pt)=(1,0.905882,0)},
mesh/rows=49]
table[row sep=crcr,header=false] {%
%
56	0.093	-10.9049618708685\\
56	0.09666	-11.4423373147976\\
56	0.10032	-12.1019444097205\\
56	0.10398	-12.8837831556373\\
56	0.10764	-13.7878535525481\\
56	0.1113	-14.8141556004528\\
56	0.11496	-15.9626892993514\\
56	0.11862	-17.2334546492439\\
56	0.12228	-18.6264516501303\\
56	0.12594	-20.1416803020107\\
56	0.1296	-21.779140604885\\
56	0.13326	-23.5388325587532\\
56	0.13692	-25.4207561636153\\
56	0.14058	-27.4249114194713\\
56	0.14424	-29.5512983263213\\
56	0.1479	-31.7999168841651\\
56	0.15156	-34.1707670930029\\
56	0.15522	-36.6638489528346\\
56	0.15888	-39.2791624636602\\
56	0.16254	-42.0167076254798\\
56	0.1662	-44.8764844382932\\
56	0.16986	-47.8584929021006\\
56	0.17352	-50.9627330169019\\
56	0.17718	-54.189204782697\\
56	0.18084	-57.5379081994862\\
56	0.1845	-61.0088432672693\\
56	0.18816	-64.6020099860462\\
56	0.19182	-68.3174083558171\\
56	0.19548	-72.1550383765819\\
56	0.19914	-76.1149000483406\\
56	0.2028	-80.1969933710933\\
56	0.20646	-84.4013183448399\\
56	0.21012	-88.7278749695803\\
56	0.21378	-93.1766632453147\\
56	0.21744	-97.747683172043\\
56	0.2211	-102.440934749765\\
56	0.22476	-107.256417978481\\
56	0.22842	-112.194132858191\\
56	0.23208	-117.254079388895\\
56	0.23574	-122.436257570593\\
56	0.2394	-127.740667403285\\
56	0.24306	-133.167308886971\\
56	0.24672	-138.716182021651\\
56	0.25038	-144.387286807324\\
56	0.25404	-150.180623243992\\
56	0.2577	-156.096191331653\\
56	0.26136	-162.133991070308\\
56	0.26502	-168.294022459957\\
56	0.26868	-174.576285500601\\
56	0.27234	-180.980780192238\\
56	0.276	-187.507506534869\\
56.375	0.093	-10.8126291856249\\
56.375	0.09666	-11.3527834706413\\
56.375	0.10032	-12.0151694066517\\
56.375	0.10398	-12.7997869936559\\
56.375	0.10764	-13.7066362316541\\
56.375	0.1113	-14.7357171206462\\
56.375	0.11496	-15.8870296606322\\
56.375	0.11862	-17.1605738516121\\
56.375	0.12228	-18.556349693586\\
56.375	0.12594	-20.0743571865538\\
56.375	0.1296	-21.7145963305155\\
56.375	0.13326	-23.4770671254711\\
56.375	0.13692	-25.3617695714206\\
56.375	0.14058	-27.3687036683641\\
56.375	0.14424	-29.4978694163015\\
56.375	0.1479	-31.7492668152327\\
56.375	0.15156	-34.1228958651579\\
56.375	0.15522	-36.6187565660771\\
56.375	0.15888	-39.2368489179901\\
56.375	0.16254	-41.9771729208971\\
56.375	0.1662	-44.839728574798\\
56.375	0.16986	-47.8245158796927\\
56.375	0.17352	-50.9315348355815\\
56.375	0.17718	-54.1607854424641\\
56.375	0.18084	-57.5122677003406\\
56.375	0.1845	-60.9859816092111\\
56.375	0.18816	-64.5819271690755\\
56.375	0.19182	-68.3001043799338\\
56.375	0.19548	-72.140513241786\\
56.375	0.19914	-76.1031537546321\\
56.375	0.2028	-80.1880259184722\\
56.375	0.20646	-84.3951297333062\\
56.375	0.21012	-88.7244651991341\\
56.375	0.21378	-93.1760323159559\\
56.375	0.21744	-97.7498310837716\\
56.375	0.2211	-102.445861502581\\
56.375	0.22476	-107.264123572385\\
56.375	0.22842	-112.204617293182\\
56.375	0.23208	-117.267342664974\\
56.375	0.23574	-122.452299687759\\
56.375	0.2394	-127.759488361538\\
56.375	0.24306	-133.188908686311\\
56.375	0.24672	-138.740560662079\\
56.375	0.25038	-144.41444428884\\
56.375	0.25404	-150.210559566594\\
56.375	0.2577	-156.128906495343\\
56.375	0.26136	-162.169485075086\\
56.375	0.26502	-168.332295305823\\
56.375	0.26868	-174.617337187553\\
56.375	0.27234	-181.024610720278\\
56.375	0.276	-187.554115903996\\
56.75	0.093	-10.729892857501\\
56.75	0.09666	-11.2728259836048\\
56.75	0.10032	-11.9379907607026\\
56.75	0.10398	-12.7253871887943\\
56.75	0.10764	-13.6350152678799\\
56.75	0.1113	-14.6668749979594\\
56.75	0.11496	-15.8209663790329\\
56.75	0.11862	-17.0972894111002\\
56.75	0.12228	-18.4958440941615\\
56.75	0.12594	-20.0166304282167\\
56.75	0.1296	-21.6596484132658\\
56.75	0.13326	-23.4248980493089\\
56.75	0.13692	-25.3123793363458\\
56.75	0.14058	-27.3220922743767\\
56.75	0.14424	-29.4540368634015\\
56.75	0.1479	-31.7082131034202\\
56.75	0.15156	-34.0846209944328\\
56.75	0.15522	-36.5832605364393\\
56.75	0.15888	-39.2041317294398\\
56.75	0.16254	-41.9472345734342\\
56.75	0.1662	-44.8125690684225\\
56.75	0.16986	-47.8001352144047\\
56.75	0.17352	-50.9099330113808\\
56.75	0.17718	-54.1419624593509\\
56.75	0.18084	-57.4962235583148\\
56.75	0.1845	-60.9727163082728\\
56.75	0.18816	-64.5714407092246\\
56.75	0.19182	-68.2923967611703\\
56.75	0.19548	-72.1355844641099\\
56.75	0.19914	-76.1010038180435\\
56.75	0.2028	-80.188654822971\\
56.75	0.20646	-84.3985374788924\\
56.75	0.21012	-88.7306517858077\\
56.75	0.21378	-93.1849977437169\\
56.75	0.21744	-97.76157535262\\
56.75	0.2211	-102.460384612517\\
56.75	0.22476	-107.281425523408\\
56.75	0.22842	-112.224698085293\\
56.75	0.23208	-117.290202298172\\
56.75	0.23574	-122.477938162045\\
56.75	0.2394	-127.787905676911\\
56.75	0.24306	-133.220104842772\\
56.75	0.24672	-138.774535659626\\
56.75	0.25038	-144.451198127475\\
56.75	0.25404	-150.250092246317\\
56.75	0.2577	-156.171218016153\\
56.75	0.26136	-162.214575436984\\
56.75	0.26502	-168.380164508808\\
56.75	0.26868	-174.667985231626\\
56.75	0.27234	-181.078037605437\\
56.75	0.276	-187.610321630243\\
57.125	0.093	-10.6567528864969\\
57.125	0.09666	-11.2024648536881\\
57.125	0.10032	-11.8704084718733\\
57.125	0.10398	-12.6605837410525\\
57.125	0.10764	-13.5729906612255\\
57.125	0.1113	-14.6076292323924\\
57.125	0.11496	-15.7644994545533\\
57.125	0.11862	-17.0436013277081\\
57.125	0.12228	-18.4449348518568\\
57.125	0.12594	-19.9685000269994\\
57.125	0.1296	-21.614296853136\\
57.125	0.13326	-23.3823253302664\\
57.125	0.13692	-25.2725854583908\\
57.125	0.14058	-27.2850772375091\\
57.125	0.14424	-29.4198006676213\\
57.125	0.1479	-31.6767557487274\\
57.125	0.15156	-34.0559424808275\\
57.125	0.15522	-36.5573608639214\\
57.125	0.15888	-39.1810108980094\\
57.125	0.16254	-41.9268925830912\\
57.125	0.1662	-44.7950059191669\\
57.125	0.16986	-47.7853509062365\\
57.125	0.17352	-50.8979275443\\
57.125	0.17718	-54.1327358333575\\
57.125	0.18084	-57.4897757734089\\
57.125	0.1845	-60.9690473644542\\
57.125	0.18816	-64.5705506064935\\
57.125	0.19182	-68.2942854995266\\
57.125	0.19548	-72.1402520435537\\
57.125	0.19914	-76.1084502385746\\
57.125	0.2028	-80.1988800845895\\
57.125	0.20646	-84.4115415815984\\
57.125	0.21012	-88.7464347296011\\
57.125	0.21378	-93.2035595285978\\
57.125	0.21744	-97.7829159785883\\
57.125	0.2211	-102.484504079573\\
57.125	0.22476	-107.308323831551\\
57.125	0.22842	-112.254375234524\\
57.125	0.23208	-117.32265828849\\
57.125	0.23574	-122.51317299345\\
57.125	0.2394	-127.825919349404\\
57.125	0.24306	-133.260897356352\\
57.125	0.24672	-138.818107014294\\
57.125	0.25038	-144.49754832323\\
57.125	0.25404	-150.29922128316\\
57.125	0.2577	-156.223125894083\\
57.125	0.26136	-162.269262156001\\
57.125	0.26502	-168.437630068912\\
57.125	0.26868	-174.728229632818\\
57.125	0.27234	-181.141060847717\\
57.125	0.276	-187.67612371361\\
57.5	0.093	-10.5932092726126\\
57.5	0.09666	-11.1417000808913\\
57.5	0.10032	-11.8124225401639\\
57.5	0.10398	-12.6053766504304\\
57.5	0.10764	-13.5205624116909\\
57.5	0.1113	-14.5579798239453\\
57.5	0.11496	-15.7176288871936\\
57.5	0.11862	-16.9995096014358\\
57.5	0.12228	-18.4036219666719\\
57.5	0.12594	-19.929965982902\\
57.5	0.1296	-21.5785416501259\\
57.5	0.13326	-23.3493489683438\\
57.5	0.13692	-25.2423879375556\\
57.5	0.14058	-27.2576585577613\\
57.5	0.14424	-29.3951608289609\\
57.5	0.1479	-31.6548947511545\\
57.5	0.15156	-34.036860324342\\
57.5	0.15522	-36.5410575485233\\
57.5	0.15888	-39.1674864236987\\
57.5	0.16254	-41.9161469498679\\
57.5	0.1662	-44.787039127031\\
57.5	0.16986	-47.7801629551881\\
57.5	0.17352	-50.8955184343391\\
57.5	0.17718	-54.1331055644839\\
57.5	0.18084	-57.4929243456227\\
57.5	0.1845	-60.9749747777555\\
57.5	0.18816	-64.5792568608822\\
57.5	0.19182	-68.3057705950027\\
57.5	0.19548	-72.1545159801172\\
57.5	0.19914	-76.1254930162256\\
57.5	0.2028	-80.2187017033279\\
57.5	0.20646	-84.4341420414242\\
57.5	0.21012	-88.7718140305143\\
57.5	0.21378	-93.2317176705984\\
57.5	0.21744	-97.8138529616764\\
57.5	0.2211	-102.518219903748\\
57.5	0.22476	-107.344818496814\\
57.5	0.22842	-112.293648740874\\
57.5	0.23208	-117.364710635927\\
57.5	0.23574	-122.558004181975\\
57.5	0.2394	-127.873529379017\\
57.5	0.24306	-133.311286227052\\
57.5	0.24672	-138.871274726081\\
57.5	0.25038	-144.553494876105\\
57.5	0.25404	-150.357946677122\\
57.5	0.2577	-156.284630129133\\
57.5	0.26136	-162.333545232138\\
57.5	0.26502	-168.504691986137\\
57.5	0.26868	-174.79807039113\\
57.5	0.27234	-181.213680447116\\
57.5	0.276	-187.751522154097\\
57.875	0.093	-10.5392620158482\\
57.875	0.09666	-11.0905316652143\\
57.875	0.10032	-11.7640329655743\\
57.875	0.10398	-12.5597659169282\\
57.875	0.10764	-13.4777305192761\\
57.875	0.1113	-14.5179267726179\\
57.875	0.11496	-15.6803546769536\\
57.875	0.11862	-16.9650142322833\\
57.875	0.12228	-18.3719054386068\\
57.875	0.12594	-19.9010282959243\\
57.875	0.1296	-21.5523828042356\\
57.875	0.13326	-23.325968963541\\
57.875	0.13692	-25.2217867738402\\
57.875	0.14058	-27.2398362351333\\
57.875	0.14424	-29.3801173474204\\
57.875	0.1479	-31.6426301107013\\
57.875	0.15156	-34.0273745249762\\
57.875	0.15522	-36.5343505902451\\
57.875	0.15888	-39.1635583065078\\
57.875	0.16254	-41.9149976737644\\
57.875	0.1662	-44.788668692015\\
57.875	0.16986	-47.7845713612595\\
57.875	0.17352	-50.9027056814979\\
57.875	0.17718	-54.1430716527301\\
57.875	0.18084	-57.5056692749564\\
57.875	0.1845	-60.9904985481766\\
57.875	0.18816	-64.5975594723906\\
57.875	0.19182	-68.3268520475986\\
57.875	0.19548	-72.1783762738005\\
57.875	0.19914	-76.1521321509963\\
57.875	0.2028	-80.2481196791861\\
57.875	0.20646	-84.4663388583698\\
57.875	0.21012	-88.8067896885474\\
57.875	0.21378	-93.2694721697188\\
57.875	0.21744	-97.8543863018843\\
57.875	0.2211	-102.561532085044\\
57.875	0.22476	-107.390909519197\\
57.875	0.22842	-112.342518604344\\
57.875	0.23208	-117.416359340485\\
57.875	0.23574	-122.61243172762\\
57.875	0.2394	-127.930735765749\\
57.875	0.24306	-133.371271454872\\
57.875	0.24672	-138.934038794989\\
57.875	0.25038	-144.619037786099\\
57.875	0.25404	-150.426268428204\\
57.875	0.2577	-156.355730721302\\
57.875	0.26136	-162.407424665395\\
57.875	0.26502	-168.581350260481\\
57.875	0.26868	-174.877507506561\\
57.875	0.27234	-181.295896403636\\
57.875	0.276	-187.836516951704\\
58.25	0.093	-10.4949111162035\\
58.25	0.09666	-11.048959606657\\
58.25	0.10032	-11.7252397481045\\
58.25	0.10398	-12.5237515405458\\
58.25	0.10764	-13.4444949839811\\
58.25	0.1113	-14.4874700784104\\
58.25	0.11496	-15.6526768238335\\
58.25	0.11862	-16.9401152202505\\
58.25	0.12228	-18.3497852676615\\
58.25	0.12594	-19.8816869660664\\
58.25	0.1296	-21.5358203154652\\
58.25	0.13326	-23.3121853158579\\
58.25	0.13692	-25.2107819672446\\
58.25	0.14058	-27.2316102696251\\
58.25	0.14424	-29.3746702229996\\
58.25	0.1479	-31.639961827368\\
58.25	0.15156	-34.0274850827303\\
58.25	0.15522	-36.5372399890866\\
58.25	0.15888	-39.1692265464367\\
58.25	0.16254	-41.9234447547808\\
58.25	0.1662	-44.7998946141188\\
58.25	0.16986	-47.7985761244506\\
58.25	0.17352	-50.9194892857765\\
58.25	0.17718	-54.1626340980962\\
58.25	0.18084	-57.5280105614098\\
58.25	0.1845	-61.0156186757175\\
58.25	0.18816	-64.625458441019\\
58.25	0.19182	-68.3575298573143\\
58.25	0.19548	-72.2118329246037\\
58.25	0.19914	-76.1883676428869\\
58.25	0.2028	-80.2871340121641\\
58.25	0.20646	-84.5081320324352\\
58.25	0.21012	-88.8513617037002\\
58.25	0.21378	-93.3168230259591\\
58.25	0.21744	-97.904515999212\\
58.25	0.2211	-102.614440623459\\
58.25	0.22476	-107.446596898699\\
58.25	0.22842	-112.400984824934\\
58.25	0.23208	-117.477604402162\\
58.25	0.23574	-122.676455630385\\
58.25	0.2394	-127.997538509601\\
58.25	0.24306	-133.440853039812\\
58.25	0.24672	-139.006399221016\\
58.25	0.25038	-144.694177053214\\
58.25	0.25404	-150.504186536406\\
58.25	0.2577	-156.436427670592\\
58.25	0.26136	-162.490900455772\\
58.25	0.26502	-168.667604891945\\
58.25	0.26868	-174.966540979113\\
58.25	0.27234	-181.387708717275\\
58.25	0.276	-187.93110810643\\
58.625	0.093	-10.4601565736786\\
58.625	0.09666	-11.0169839052196\\
58.625	0.10032	-11.6960428877545\\
58.625	0.10398	-12.4973335212833\\
58.625	0.10764	-13.420855805806\\
58.625	0.1113	-14.4666097413226\\
58.625	0.11496	-15.6345953278332\\
58.625	0.11862	-16.9248125653376\\
58.625	0.12228	-18.337261453836\\
58.625	0.12594	-19.8719419933284\\
58.625	0.1296	-21.5288541838146\\
58.625	0.13326	-23.3079980252947\\
58.625	0.13692	-25.2093735177688\\
58.625	0.14058	-27.2329806612368\\
58.625	0.14424	-29.3788194556987\\
58.625	0.1479	-31.6468899011545\\
58.625	0.15156	-34.0371919976042\\
58.625	0.15522	-36.5497257450479\\
58.625	0.15888	-39.1844911434855\\
58.625	0.16254	-41.941488192917\\
58.625	0.1662	-44.8207168933424\\
58.625	0.16986	-47.8221772447617\\
58.625	0.17352	-50.9458692471749\\
58.625	0.17718	-54.1917929005821\\
58.625	0.18084	-57.5599482049831\\
58.625	0.1845	-61.0503351603781\\
58.625	0.18816	-64.6629537667671\\
58.625	0.19182	-68.3978040241499\\
58.625	0.19548	-72.2548859325266\\
58.625	0.19914	-76.2341994918973\\
58.625	0.2028	-80.3357447022619\\
58.625	0.20646	-84.5595215636204\\
58.625	0.21012	-88.9055300759729\\
58.625	0.21378	-93.3737702393192\\
58.625	0.21744	-97.9642420536594\\
58.625	0.2211	-102.676945518994\\
58.625	0.22476	-107.511880635322\\
58.625	0.22842	-112.469047402644\\
58.625	0.23208	-117.54844582096\\
58.625	0.23574	-122.75007589027\\
58.625	0.2394	-128.073937610573\\
58.625	0.24306	-133.520030981871\\
58.625	0.24672	-139.088356004163\\
58.625	0.25038	-144.778912677448\\
58.625	0.25404	-150.591701001728\\
58.625	0.2577	-156.526720977001\\
58.625	0.26136	-162.583972603268\\
58.625	0.26502	-168.763455880529\\
58.625	0.26868	-175.065170808784\\
58.625	0.27234	-181.489117388033\\
58.625	0.276	-188.035295618276\\
59	0.093	-10.4349983882736\\
59	0.09666	-10.994604560902\\
59	0.10032	-11.6764423845243\\
59	0.10398	-12.4805118591405\\
59	0.10764	-13.4068129847506\\
59	0.1113	-14.4553457613547\\
59	0.11496	-15.6261101889527\\
59	0.11862	-16.9191062675446\\
59	0.12228	-18.3343339971304\\
59	0.12594	-19.8717933777101\\
59	0.1296	-21.5314844092838\\
59	0.13326	-23.3134070918513\\
59	0.13692	-25.2175614254128\\
59	0.14058	-27.2439474099682\\
59	0.14424	-29.3925650455176\\
59	0.1479	-31.6634143320608\\
59	0.15156	-34.0564952695979\\
59	0.15522	-36.571807858129\\
59	0.15888	-39.209352097654\\
59	0.16254	-41.9691279881729\\
59	0.1662	-44.8511355296857\\
59	0.16986	-47.8553747221925\\
59	0.17352	-50.9818455656931\\
59	0.17718	-54.2305480601877\\
59	0.18084	-57.6014822056762\\
59	0.1845	-61.0946480021586\\
59	0.18816	-64.710045449635\\
59	0.19182	-68.4476745481052\\
59	0.19548	-72.3075352975694\\
59	0.19914	-76.2896276980275\\
59	0.2028	-80.3939517494795\\
59	0.20646	-84.6205074519255\\
59	0.21012	-88.9692948053654\\
59	0.21378	-93.4403138097991\\
59	0.21744	-98.0335644652267\\
59	0.2211	-102.749046771648\\
59	0.22476	-107.586760729064\\
59	0.22842	-112.546706337473\\
59	0.23208	-117.628883596877\\
59	0.23574	-122.833292507274\\
59	0.2394	-128.159933068665\\
59	0.24306	-133.60880528105\\
59	0.24672	-139.179909144429\\
59	0.25038	-144.873244658802\\
59	0.25404	-150.688811824169\\
59	0.2577	-156.62661064053\\
59	0.26136	-162.686641107885\\
59	0.26502	-168.868903226233\\
59	0.26868	-175.173396995576\\
59	0.27234	-181.600122415912\\
59	0.276	-188.149079487242\\
59.375	0.093	-10.4194365599884\\
59.375	0.09666	-10.9818215737042\\
59.375	0.10032	-11.6664382384139\\
59.375	0.10398	-12.4732865541175\\
59.375	0.10764	-13.4023665208151\\
59.375	0.1113	-14.4536781385066\\
59.375	0.11496	-15.627221407192\\
59.375	0.11862	-16.9229963268713\\
59.375	0.12228	-18.3410028975445\\
59.375	0.12594	-19.8812411192117\\
59.375	0.1296	-21.5437109918728\\
59.375	0.13326	-23.3284125155277\\
59.375	0.13692	-25.2353456901767\\
59.375	0.14058	-27.2645105158195\\
59.375	0.14424	-29.4159069924562\\
59.375	0.1479	-31.6895351200869\\
59.375	0.15156	-34.0853948987114\\
59.375	0.15522	-36.60348632833\\
59.375	0.15888	-39.2438094089424\\
59.375	0.16254	-42.0063641405487\\
59.375	0.1662	-44.891150523149\\
59.375	0.16986	-47.8981685567431\\
59.375	0.17352	-51.0274182413312\\
59.375	0.17718	-54.2788995769132\\
59.375	0.18084	-57.6526125634891\\
59.375	0.1845	-61.148557201059\\
59.375	0.18816	-64.7667334896228\\
59.375	0.19182	-68.5071414291804\\
59.375	0.19548	-72.369781019732\\
59.375	0.19914	-76.3546522612775\\
59.375	0.2028	-80.461755153817\\
59.375	0.20646	-84.6910896973503\\
59.375	0.21012	-89.0426558918776\\
59.375	0.21378	-93.5164537373988\\
59.375	0.21744	-98.1124832339139\\
59.375	0.2211	-102.830744381423\\
59.375	0.22476	-107.671237179926\\
59.375	0.22842	-112.633961629423\\
59.375	0.23208	-117.718917729913\\
59.375	0.23574	-122.926105481398\\
59.375	0.2394	-128.255524883877\\
59.375	0.24306	-133.707175937349\\
59.375	0.24672	-139.281058641816\\
59.375	0.25038	-144.977172997276\\
59.375	0.25404	-150.79551900373\\
59.375	0.2577	-156.736096661179\\
59.375	0.26136	-162.798905969621\\
59.375	0.26502	-168.983946929057\\
59.375	0.26868	-175.291219539487\\
59.375	0.27234	-181.720723800911\\
59.375	0.276	-188.272459713328\\
59.75	0.093	-10.413471088823\\
59.75	0.09666	-10.9786349436262\\
59.75	0.10032	-11.6660304494233\\
59.75	0.10398	-12.4756576062144\\
59.75	0.10764	-13.4075164139994\\
59.75	0.1113	-14.4616068727783\\
59.75	0.11496	-15.6379289825511\\
59.75	0.11862	-16.9364827433178\\
59.75	0.12228	-18.3572681550785\\
59.75	0.12594	-19.9002852178331\\
59.75	0.1296	-21.5655339315816\\
59.75	0.13326	-23.353014296324\\
59.75	0.13692	-25.2627263120603\\
59.75	0.14058	-27.2946699787905\\
59.75	0.14424	-29.4488452965147\\
59.75	0.1479	-31.7252522652328\\
59.75	0.15156	-34.1238908849448\\
59.75	0.15522	-36.6447611556507\\
59.75	0.15888	-39.2878630773506\\
59.75	0.16254	-42.0531966500443\\
59.75	0.1662	-44.940761873732\\
59.75	0.16986	-47.9505587484136\\
59.75	0.17352	-51.0825872740891\\
59.75	0.17718	-54.3368474507585\\
59.75	0.18084	-57.7133392784218\\
59.75	0.1845	-61.2120627570791\\
59.75	0.18816	-64.8330178867303\\
59.75	0.19182	-68.5762046673754\\
59.75	0.19548	-72.4416230990144\\
59.75	0.19914	-76.4292731816473\\
59.75	0.2028	-80.5391549152742\\
59.75	0.20646	-84.771268299895\\
59.75	0.21012	-89.1256133355097\\
59.75	0.21378	-93.6021900221183\\
59.75	0.21744	-98.2009983597208\\
59.75	0.2211	-102.922038348317\\
59.75	0.22476	-107.765309987908\\
59.75	0.22842	-112.730813278492\\
59.75	0.23208	-117.81854822007\\
59.75	0.23574	-123.028514812642\\
59.75	0.2394	-128.360713056208\\
59.75	0.24306	-133.815142950768\\
59.75	0.24672	-139.391804496322\\
59.75	0.25038	-145.09069769287\\
59.75	0.25404	-150.911822540412\\
59.75	0.2577	-156.855179038947\\
59.75	0.26136	-162.920767188477\\
59.75	0.26502	-169.108586989\\
59.75	0.26868	-175.418638440517\\
59.75	0.27234	-181.850921543029\\
59.75	0.276	-188.405436296534\\
60.125	0.093	-10.4171019747773\\
60.125	0.09666	-10.985044670668\\
60.125	0.10032	-11.6752190175526\\
60.125	0.10398	-12.487625015431\\
60.125	0.10764	-13.4222626643034\\
60.125	0.1113	-14.4791319641698\\
60.125	0.11496	-15.65823291503\\
60.125	0.11862	-16.9595655168842\\
60.125	0.12228	-18.3831297697322\\
60.125	0.12594	-19.9289256735743\\
60.125	0.1296	-21.5969532284102\\
60.125	0.13326	-23.38721243424\\
60.125	0.13692	-25.2997032910637\\
60.125	0.14058	-27.3344257988814\\
60.125	0.14424	-29.491379957693\\
60.125	0.1479	-31.7705657674985\\
60.125	0.15156	-34.1719832282979\\
60.125	0.15522	-36.6956323400913\\
60.125	0.15888	-39.3415131028785\\
60.125	0.16254	-42.1096255166597\\
60.125	0.1662	-44.9999695814348\\
60.125	0.16986	-48.0125452972038\\
60.125	0.17352	-51.1473526639667\\
60.125	0.17718	-54.4043916817236\\
60.125	0.18084	-57.7836623504743\\
60.125	0.1845	-61.285164670219\\
60.125	0.18816	-64.9088986409576\\
60.125	0.19182	-68.6548642626901\\
60.125	0.19548	-72.5230615354166\\
60.125	0.19914	-76.5134904591369\\
60.125	0.2028	-80.6261510338512\\
60.125	0.20646	-84.8610432595595\\
60.125	0.21012	-89.2181671362616\\
60.125	0.21378	-93.6975226639576\\
60.125	0.21744	-98.2991098426475\\
60.125	0.2211	-103.022928672331\\
60.125	0.22476	-107.868979153009\\
60.125	0.22842	-112.837261284681\\
60.125	0.23208	-117.927775067346\\
60.125	0.23574	-123.140520501006\\
60.125	0.2394	-128.475497585659\\
60.125	0.24306	-133.932706321307\\
60.125	0.24672	-139.512146707948\\
60.125	0.25038	-145.213818745583\\
60.125	0.25404	-151.037722434212\\
60.125	0.2577	-156.983857773836\\
60.125	0.26136	-163.052224764453\\
60.125	0.26502	-169.242823406063\\
60.125	0.26868	-175.555653698668\\
60.125	0.27234	-181.990715642267\\
60.125	0.276	-188.548009236859\\
60.5	0.093	-10.4303292178515\\
60.5	0.09666	-11.0010507548296\\
60.5	0.10032	-11.6940039428016\\
60.5	0.10398	-12.5091887817675\\
60.5	0.10764	-13.4466052717273\\
60.5	0.1113	-14.5062534126811\\
60.5	0.11496	-15.6881332046288\\
60.5	0.11862	-16.9922446475703\\
60.5	0.12228	-18.4185877415058\\
60.5	0.12594	-19.9671624864353\\
60.5	0.1296	-21.6379688823586\\
60.5	0.13326	-23.4310069292758\\
60.5	0.13692	-25.346276627187\\
60.5	0.14058	-27.3837779760921\\
60.5	0.14424	-29.5435109759911\\
60.5	0.1479	-31.825475626884\\
60.5	0.15156	-34.2296719287709\\
60.5	0.15522	-36.7560998816516\\
60.5	0.15888	-39.4047594855263\\
60.5	0.16254	-42.1756507403949\\
60.5	0.1662	-45.0687736462574\\
60.5	0.16986	-48.0841282031138\\
60.5	0.17352	-51.2217144109642\\
60.5	0.17718	-54.4815322698084\\
60.5	0.18084	-57.8635817796466\\
60.5	0.1845	-61.3678629404788\\
60.5	0.18816	-64.9943757523048\\
60.5	0.19182	-68.7431202151247\\
60.5	0.19548	-72.6140963289386\\
60.5	0.19914	-76.6073040937464\\
60.5	0.2028	-80.7227435095481\\
60.5	0.20646	-84.9604145763437\\
60.5	0.21012	-89.3203172941333\\
60.5	0.21378	-93.8024516629167\\
60.5	0.21744	-98.406817682694\\
60.5	0.2211	-103.133415353465\\
60.5	0.22476	-107.982244675231\\
60.5	0.22842	-112.95330564799\\
60.5	0.23208	-118.046598271743\\
60.5	0.23574	-123.26212254649\\
60.5	0.2394	-128.599878472231\\
60.5	0.24306	-134.059866048965\\
60.5	0.24672	-139.642085276694\\
60.5	0.25038	-145.346536155417\\
60.5	0.25404	-151.173218685133\\
60.5	0.2577	-157.122132865844\\
60.5	0.26136	-163.193278697548\\
60.5	0.26502	-169.386656180246\\
60.5	0.26868	-175.702265313939\\
60.5	0.27234	-182.140106098625\\
60.5	0.276	-188.700178534305\\
60.875	0.093	-10.4531528180456\\
60.875	0.09666	-11.026653196111\\
60.875	0.10032	-11.7223852251704\\
60.875	0.10398	-12.5403489052238\\
60.875	0.10764	-13.480544236271\\
60.875	0.1113	-14.5429712183122\\
60.875	0.11496	-15.7276298513473\\
60.875	0.11862	-17.0345201353763\\
60.875	0.12228	-18.4636420703992\\
60.875	0.12594	-20.014995656416\\
60.875	0.1296	-21.6885808934268\\
60.875	0.13326	-23.4843977814315\\
60.875	0.13692	-25.4024463204301\\
60.875	0.14058	-27.4427265104226\\
60.875	0.14424	-29.605238351409\\
60.875	0.1479	-31.8899818433894\\
60.875	0.15156	-34.2969569863636\\
60.875	0.15522	-36.8261637803318\\
60.875	0.15888	-39.4776022252939\\
60.875	0.16254	-42.2512723212499\\
60.875	0.1662	-45.1471740681999\\
60.875	0.16986	-48.1653074661437\\
60.875	0.17352	-51.3056725150815\\
60.875	0.17718	-54.5682692150131\\
60.875	0.18084	-57.9530975659387\\
60.875	0.1845	-61.4601575678583\\
60.875	0.18816	-65.0894492207718\\
60.875	0.19182	-68.8409725246791\\
60.875	0.19548	-72.7147274795804\\
60.875	0.19914	-76.7107140854756\\
60.875	0.2028	-80.8289323423648\\
60.875	0.20646	-85.0693822502478\\
60.875	0.21012	-89.4320638091248\\
60.875	0.21378	-93.9169770189956\\
60.875	0.21744	-98.5241218798604\\
60.875	0.2211	-103.253498391719\\
60.875	0.22476	-108.105106554572\\
60.875	0.22842	-113.078946368418\\
60.875	0.23208	-118.175017833259\\
60.875	0.23574	-123.393320949093\\
60.875	0.2394	-128.733855715921\\
60.875	0.24306	-134.196622133744\\
60.875	0.24672	-139.78162020256\\
60.875	0.25038	-145.48884992237\\
60.875	0.25404	-151.318311293174\\
60.875	0.2577	-157.270004314972\\
60.875	0.26136	-163.343928987764\\
60.875	0.26502	-169.540085311549\\
60.875	0.26868	-175.858473286329\\
60.875	0.27234	-182.299092912102\\
60.875	0.276	-188.86194418887\\
61.25	0.093	-10.4855727753594\\
61.25	0.09666	-11.0618519945123\\
61.25	0.10032	-11.7603628646591\\
61.25	0.10398	-12.5811053857998\\
61.25	0.10764	-13.5240795579345\\
61.25	0.1113	-14.5892853810631\\
61.25	0.11496	-15.7767228551856\\
61.25	0.11862	-17.086391980302\\
61.25	0.12228	-18.5182927564124\\
61.25	0.12594	-20.0724251835166\\
61.25	0.1296	-21.7487892616148\\
61.25	0.13326	-23.5473849907069\\
61.25	0.13692	-25.4682123707929\\
61.25	0.14058	-27.5112714018729\\
61.25	0.14424	-29.6765620839467\\
61.25	0.1479	-31.9640844170145\\
61.25	0.15156	-34.3738384010762\\
61.25	0.15522	-36.9058240361318\\
61.25	0.15888	-39.5600413221813\\
61.25	0.16254	-42.3364902592248\\
61.25	0.1662	-45.2351708472621\\
61.25	0.16986	-48.2560830862934\\
61.25	0.17352	-51.3992269763186\\
61.25	0.17718	-54.6646025173377\\
61.25	0.18084	-58.0522097093507\\
61.25	0.1845	-61.5620485523577\\
61.25	0.18816	-65.1941190463585\\
61.25	0.19182	-68.9484211913533\\
61.25	0.19548	-72.824954987342\\
61.25	0.19914	-76.8237204343247\\
61.25	0.2028	-80.9447175323012\\
61.25	0.20646	-85.1879462812717\\
61.25	0.21012	-89.5534066812361\\
61.25	0.21378	-94.0410987321943\\
61.25	0.21744	-98.6510224341465\\
61.25	0.2211	-103.383177787093\\
61.25	0.22476	-108.237564791033\\
61.25	0.22842	-113.214183445967\\
61.25	0.23208	-118.313033751895\\
61.25	0.23574	-123.534115708816\\
61.25	0.2394	-128.877429316732\\
61.25	0.24306	-134.342974575642\\
61.25	0.24672	-139.930751485545\\
61.25	0.25038	-145.640760046443\\
61.25	0.25404	-151.473000258334\\
61.25	0.2577	-157.427472121219\\
61.25	0.26136	-163.504175635099\\
61.25	0.26502	-169.703110799972\\
61.25	0.26868	-176.024277615839\\
61.25	0.27234	-182.4676760827\\
61.25	0.276	-189.033306200555\\
61.625	0.093	-10.527589089793\\
61.625	0.09666	-11.1066471500333\\
61.625	0.10032	-11.8079368612676\\
61.625	0.10398	-12.6314582234957\\
61.625	0.10764	-13.5772112367179\\
61.625	0.1113	-14.6451959009339\\
61.625	0.11496	-15.8354122161438\\
61.625	0.11862	-17.1478601823476\\
61.625	0.12228	-18.5825397995454\\
61.625	0.12594	-20.1394510677371\\
61.625	0.1296	-21.8185939869227\\
61.625	0.13326	-23.6199685571022\\
61.625	0.13692	-25.5435747782756\\
61.625	0.14058	-27.589412650443\\
61.625	0.14424	-29.7574821736043\\
61.625	0.1479	-32.0477833477595\\
61.625	0.15156	-34.4603161729086\\
61.625	0.15522	-36.9950806490516\\
61.625	0.15888	-39.6520767761886\\
61.625	0.16254	-42.4313045543194\\
61.625	0.1662	-45.3327639834442\\
61.625	0.16986	-48.3564550635629\\
61.625	0.17352	-51.5023777946755\\
61.625	0.17718	-54.770532176782\\
61.625	0.18084	-58.1609182098824\\
61.625	0.1845	-61.6735358939769\\
61.625	0.18816	-65.3083852290652\\
61.625	0.19182	-69.0654662151473\\
61.625	0.19548	-72.9447788522235\\
61.625	0.19914	-76.9463231402935\\
61.625	0.2028	-81.0700990793575\\
61.625	0.20646	-85.3161066694154\\
61.625	0.21012	-89.6843459104672\\
61.625	0.21378	-94.1748168025129\\
61.625	0.21744	-98.7875193455525\\
61.625	0.2211	-103.522453539586\\
61.625	0.22476	-108.379619384614\\
61.625	0.22842	-113.359016880635\\
61.625	0.23208	-118.46064602765\\
61.625	0.23574	-123.684506825659\\
61.625	0.2394	-129.030599274663\\
61.625	0.24306	-134.49892337466\\
61.625	0.24672	-140.089479125651\\
61.625	0.25038	-145.802266527636\\
61.625	0.25404	-151.637285580614\\
61.625	0.2577	-157.594536284587\\
61.625	0.26136	-163.674018639554\\
61.625	0.26502	-169.875732645514\\
61.625	0.26868	-176.199678302469\\
61.625	0.27234	-182.645855610417\\
61.625	0.276	-189.214264569359\\
62	0.093	-10.5792017613464\\
62	0.09666	-11.1610386626742\\
62	0.10032	-11.8651072149959\\
62	0.10398	-12.6914074183114\\
62	0.10764	-13.6399392726209\\
62	0.1113	-14.7107027779244\\
62	0.11496	-15.9036979342217\\
62	0.11862	-17.218924741513\\
62	0.12228	-18.6563831997982\\
62	0.12594	-20.2160733090773\\
62	0.1296	-21.8979950693503\\
62	0.13326	-23.7021484806173\\
62	0.13692	-25.6285335428781\\
62	0.14058	-27.6771502561329\\
62	0.14424	-29.8479986203816\\
62	0.1479	-32.1410786356242\\
62	0.15156	-34.5563903018607\\
62	0.15522	-37.0939336190912\\
62	0.15888	-39.7537085873156\\
62	0.16254	-42.5357152065338\\
62	0.1662	-45.439953476746\\
62	0.16986	-48.4664233979522\\
62	0.17352	-51.6151249701522\\
62	0.17718	-54.8860581933461\\
62	0.18084	-58.279223067534\\
62	0.1845	-61.7946195927158\\
62	0.18816	-65.4322477688915\\
62	0.19182	-69.1921075960611\\
62	0.19548	-73.0741990742247\\
62	0.19914	-77.0785222033822\\
62	0.2028	-81.2050769835336\\
62	0.20646	-85.4538634146789\\
62	0.21012	-89.8248814968181\\
62	0.21378	-94.3181312299512\\
62	0.21744	-98.9336126140783\\
62	0.2211	-103.671325649199\\
62	0.22476	-108.531270335314\\
62	0.22842	-113.513446672423\\
62	0.23208	-118.617854660526\\
62	0.23574	-123.844494299622\\
62	0.2394	-129.193365589713\\
62	0.24306	-134.664468530797\\
62	0.24672	-140.257803122876\\
62	0.25038	-145.973369365948\\
62	0.25404	-151.811167260014\\
62	0.2577	-157.771196805075\\
62	0.26136	-163.853458001129\\
62	0.26502	-170.057950848176\\
62	0.26868	-176.384675346218\\
62	0.27234	-182.833631495254\\
62	0.276	-189.404819295284\\
62.375	0.093	-10.6404107900197\\
62.375	0.09666	-11.2250265324349\\
62.375	0.10032	-11.931873925844\\
62.375	0.10398	-12.760952970247\\
62.375	0.10764	-13.7122636656439\\
62.375	0.1113	-14.7858060120348\\
62.375	0.11496	-15.9815800094195\\
62.375	0.11862	-17.2995856577982\\
62.375	0.12228	-18.7398229571708\\
62.375	0.12594	-20.3022919075374\\
62.375	0.1296	-21.9869925088978\\
62.375	0.13326	-23.7939247612521\\
62.375	0.13692	-25.7230886646005\\
62.375	0.14058	-27.7744842189426\\
62.375	0.14424	-29.9481114242788\\
62.375	0.1479	-32.2439702806088\\
62.375	0.15156	-34.6620607879327\\
62.375	0.15522	-37.2023829462506\\
62.375	0.15888	-39.8649367555624\\
62.375	0.16254	-42.6497222158681\\
62.375	0.1662	-45.5567393271677\\
62.375	0.16986	-48.5859880894613\\
62.375	0.17352	-51.7374685027487\\
62.375	0.17718	-55.0111805670301\\
62.375	0.18084	-58.4071242823054\\
62.375	0.1845	-61.9252996485746\\
62.375	0.18816	-65.5657066658378\\
62.375	0.19182	-69.3283453340948\\
62.375	0.19548	-73.2132156533458\\
62.375	0.19914	-77.2203176235907\\
62.375	0.2028	-81.3496512448295\\
62.375	0.20646	-85.6012165170623\\
62.375	0.21012	-89.9750134402889\\
62.375	0.21378	-94.4710420145094\\
62.375	0.21744	-99.0893022397238\\
62.375	0.2211	-103.829794115932\\
62.375	0.22476	-108.692517643135\\
62.375	0.22842	-113.677472821331\\
62.375	0.23208	-118.784659650521\\
62.375	0.23574	-124.014078130705\\
62.375	0.2394	-129.365728261883\\
62.375	0.24306	-134.839610044055\\
62.375	0.24672	-140.435723477221\\
62.375	0.25038	-146.154068561381\\
62.375	0.25404	-151.994645296534\\
62.375	0.2577	-157.957453682682\\
62.375	0.26136	-164.042493719823\\
62.375	0.26502	-170.249765407959\\
62.375	0.26868	-176.579268747088\\
62.375	0.27234	-183.031003737211\\
62.375	0.276	-189.604970378328\\
62.75	0.093	-10.7112161758128\\
62.75	0.09666	-11.2986107593154\\
62.75	0.10032	-12.0082369938119\\
62.75	0.10398	-12.8400948793023\\
62.75	0.10764	-13.7941844157867\\
62.75	0.1113	-14.870505603265\\
62.75	0.11496	-16.0690584417372\\
62.75	0.11862	-17.3898429312033\\
62.75	0.12228	-18.8328590716633\\
62.75	0.12594	-20.3981068631172\\
62.75	0.1296	-22.0855863055651\\
62.75	0.13326	-23.8952973990069\\
62.75	0.13692	-25.8272401434426\\
62.75	0.14058	-27.8814145388722\\
62.75	0.14424	-30.0578205852958\\
62.75	0.1479	-32.3564582827132\\
62.75	0.15156	-34.7773276311246\\
62.75	0.15522	-37.3204286305299\\
62.75	0.15888	-39.9857612809291\\
62.75	0.16254	-42.7733255823222\\
62.75	0.1662	-45.6831215347093\\
62.75	0.16986	-48.7151491380902\\
62.75	0.17352	-51.8694083924651\\
62.75	0.17718	-55.1458992978339\\
62.75	0.18084	-58.5446218541966\\
62.75	0.1845	-62.0655760615533\\
62.75	0.18816	-65.7087619199038\\
62.75	0.19182	-69.4741794292482\\
62.75	0.19548	-73.3618285895866\\
62.75	0.19914	-77.3717094009189\\
62.75	0.2028	-81.5038218632452\\
62.75	0.20646	-85.7581659765654\\
62.75	0.21012	-90.1347417408794\\
62.75	0.21378	-94.6335491561874\\
62.75	0.21744	-99.2545882224892\\
62.75	0.2211	-103.997858939785\\
62.75	0.22476	-108.863361308075\\
62.75	0.22842	-113.851095327359\\
62.75	0.23208	-118.961060997636\\
62.75	0.23574	-124.193258318908\\
62.75	0.2394	-129.547687291173\\
62.75	0.24306	-135.024347914432\\
62.75	0.24672	-140.623240188686\\
62.75	0.25038	-146.344364113933\\
62.75	0.25404	-152.187719690174\\
62.75	0.2577	-158.153306917409\\
62.75	0.26136	-164.241125795638\\
62.75	0.26502	-170.451176324861\\
62.75	0.26868	-176.783458505077\\
62.75	0.27234	-183.237972336288\\
62.75	0.276	-189.814717818493\\
63.125	0.093	-10.7916179187256\\
63.125	0.09666	-11.3817913433157\\
63.125	0.10032	-12.0941964188996\\
63.125	0.10398	-12.9288331454774\\
63.125	0.10764	-13.8857015230492\\
63.125	0.1113	-14.9648015516149\\
63.125	0.11496	-16.1661332311746\\
63.125	0.11862	-17.4896965617281\\
63.125	0.12228	-18.9354915432755\\
63.125	0.12594	-20.5035181758169\\
63.125	0.1296	-22.1937764593522\\
63.125	0.13326	-24.0062663938814\\
63.125	0.13692	-25.9409879794045\\
63.125	0.14058	-27.9979412159216\\
63.125	0.14424	-30.1771261034325\\
63.125	0.1479	-32.4785426419374\\
63.125	0.15156	-34.9021908314362\\
63.125	0.15522	-37.4480706719289\\
63.125	0.15888	-40.1161821634156\\
63.125	0.16254	-42.9065253058961\\
63.125	0.1662	-45.8191000993706\\
63.125	0.16986	-48.8539065438389\\
63.125	0.17352	-52.0109446393012\\
63.125	0.17718	-55.2902143857575\\
63.125	0.18084	-58.6917157832076\\
63.125	0.1845	-62.2154488316517\\
63.125	0.18816	-65.8614135310897\\
63.125	0.19182	-69.6296098815215\\
63.125	0.19548	-73.5200378829473\\
63.125	0.19914	-77.532697535367\\
63.125	0.2028	-81.6675888387807\\
63.125	0.20646	-85.9247117931883\\
63.125	0.21012	-90.3040663985898\\
63.125	0.21378	-94.8056526549852\\
63.125	0.21744	-99.4294705623745\\
63.125	0.2211	-104.175520120758\\
63.125	0.22476	-109.043801330135\\
63.125	0.22842	-114.034314190506\\
63.125	0.23208	-119.147058701871\\
63.125	0.23574	-124.38203486423\\
63.125	0.2394	-129.739242677583\\
63.125	0.24306	-135.218682141929\\
63.125	0.24672	-140.82035325727\\
63.125	0.25038	-146.544256023605\\
63.125	0.25404	-152.390390440933\\
63.125	0.2577	-158.358756509256\\
63.125	0.26136	-164.449354228572\\
63.125	0.26502	-170.662183598882\\
63.125	0.26868	-176.997244620186\\
63.125	0.27234	-183.454537292484\\
63.125	0.276	-190.034061615776\\
63.5	0.093	-10.8816160187583\\
63.5	0.09666	-11.4745682844357\\
63.5	0.10032	-12.1897522011071\\
63.5	0.10398	-13.0271677687724\\
63.5	0.10764	-13.9868149874316\\
63.5	0.1113	-15.0686938570847\\
63.5	0.11496	-16.2728043777317\\
63.5	0.11862	-17.5991465493727\\
63.5	0.12228	-19.0477203720076\\
63.5	0.12594	-20.6185258456364\\
63.5	0.1296	-22.3115629702591\\
63.5	0.13326	-24.1268317458757\\
63.5	0.13692	-26.0643321724862\\
63.5	0.14058	-28.1240642500907\\
63.5	0.14424	-30.3060279786891\\
63.5	0.1479	-32.6102233582814\\
63.5	0.15156	-35.0366503888676\\
63.5	0.15522	-37.5853090704477\\
63.5	0.15888	-40.2561994030218\\
63.5	0.16254	-43.0493213865898\\
63.5	0.1662	-45.9646750211517\\
63.5	0.16986	-49.0022603067075\\
63.5	0.17352	-52.1620772432572\\
63.5	0.17718	-55.4441258308008\\
63.5	0.18084	-58.8484060693384\\
63.5	0.1845	-62.3749179588699\\
63.5	0.18816	-66.0236614993953\\
63.5	0.19182	-69.7946366909145\\
63.5	0.19548	-73.6878435334278\\
63.5	0.19914	-77.7032820269349\\
63.5	0.2028	-81.840952171436\\
63.5	0.20646	-86.100853966931\\
63.5	0.21012	-90.4829874134199\\
63.5	0.21378	-94.9873525109027\\
63.5	0.21744	-99.6139492593795\\
63.5	0.2211	-104.36277765885\\
63.5	0.22476	-109.233837709315\\
63.5	0.22842	-114.227129410773\\
63.5	0.23208	-119.342652763226\\
63.5	0.23574	-124.580407766672\\
63.5	0.2394	-129.940394421112\\
63.5	0.24306	-135.422612726546\\
63.5	0.24672	-141.027062682975\\
63.5	0.25038	-146.753744290396\\
63.5	0.25404	-152.602657548812\\
63.5	0.2577	-158.573802458222\\
63.5	0.26136	-164.667179018626\\
63.5	0.26502	-170.882787230024\\
63.5	0.26868	-177.220627092415\\
63.5	0.27234	-183.680698605801\\
63.5	0.276	-190.26300177018\\
63.875	0.093	-10.9812104759108\\
63.875	0.09666	-11.5769415826757\\
63.875	0.10032	-12.2949043404345\\
63.875	0.10398	-13.1350987491871\\
63.875	0.10764	-14.0975248089338\\
63.875	0.1113	-15.1821825196743\\
63.875	0.11496	-16.3890718814088\\
63.875	0.11862	-17.7181928941371\\
63.875	0.12228	-19.1695455578594\\
63.875	0.12594	-20.7431298725757\\
63.875	0.1296	-22.4389458382858\\
63.875	0.13326	-24.2569934549898\\
63.875	0.13692	-26.1972727226878\\
63.875	0.14058	-28.2597836413797\\
63.875	0.14424	-30.4445262110655\\
63.875	0.1479	-32.7515004317452\\
63.875	0.15156	-35.1807063034188\\
63.875	0.15522	-37.7321438260864\\
63.875	0.15888	-40.4058129997479\\
63.875	0.16254	-43.2017138244033\\
63.875	0.1662	-46.1198463000526\\
63.875	0.16986	-49.1602104266958\\
63.875	0.17352	-52.3228062043329\\
63.875	0.17718	-55.607633632964\\
63.875	0.18084	-59.014692712589\\
63.875	0.1845	-62.5439834432079\\
63.875	0.18816	-66.1955058248207\\
63.875	0.19182	-69.9692598574274\\
63.875	0.19548	-73.8652455410281\\
63.875	0.19914	-77.8834628756227\\
63.875	0.2028	-82.0239118612112\\
63.875	0.20646	-86.2865924977936\\
63.875	0.21012	-90.6715047853699\\
63.875	0.21378	-95.1786487239401\\
63.875	0.21744	-99.8080243135043\\
63.875	0.2211	-104.559631554062\\
63.875	0.22476	-109.433470445614\\
63.875	0.22842	-114.42954098816\\
63.875	0.23208	-119.5478431817\\
63.875	0.23574	-124.788377026234\\
63.875	0.2394	-130.151142521762\\
63.875	0.24306	-135.636139668283\\
63.875	0.24672	-141.243368465799\\
63.875	0.25038	-146.972828914308\\
63.875	0.25404	-152.824521013811\\
63.875	0.2577	-158.798444764309\\
63.875	0.26136	-164.8946001658\\
63.875	0.26502	-171.112987218285\\
63.875	0.26868	-177.453605921764\\
63.875	0.27234	-183.916456276237\\
63.875	0.276	-190.501538281704\\
64.25	0.093	-11.0904012901831\\
64.25	0.09666	-11.6889112380354\\
64.25	0.10032	-12.4096528368816\\
64.25	0.10398	-13.2526260867217\\
64.25	0.10764	-14.2178309875557\\
64.25	0.1113	-15.3052675393837\\
64.25	0.11496	-16.5149357422056\\
64.25	0.11862	-17.8468355960214\\
64.25	0.12228	-19.3009671008311\\
64.25	0.12594	-20.8773302566347\\
64.25	0.1296	-22.5759250634323\\
64.25	0.13326	-24.3967515212237\\
64.25	0.13692	-26.3398096300091\\
64.25	0.14058	-28.4050993897884\\
64.25	0.14424	-30.5926208005617\\
64.25	0.1479	-32.9023738623288\\
64.25	0.15156	-35.3343585750899\\
64.25	0.15522	-37.8885749388449\\
64.25	0.15888	-40.5650229535938\\
64.25	0.16254	-43.3637026193366\\
64.25	0.1662	-46.2846139360733\\
64.25	0.16986	-49.3277569038039\\
64.25	0.17352	-52.4931315225285\\
64.25	0.17718	-55.780737792247\\
64.25	0.18084	-59.1905757129594\\
64.25	0.1845	-62.7226452846657\\
64.25	0.18816	-66.3769465073659\\
64.25	0.19182	-70.1534793810601\\
64.25	0.19548	-74.0522439057482\\
64.25	0.19914	-78.0732400814302\\
64.25	0.2028	-82.2164679081061\\
64.25	0.20646	-86.4819273857759\\
64.25	0.21012	-90.8696185144397\\
64.25	0.21378	-95.3795412940973\\
64.25	0.21744	-100.011695724749\\
64.25	0.2211	-104.766081806394\\
64.25	0.22476	-109.642699539034\\
64.25	0.22842	-114.641548922667\\
64.25	0.23208	-119.762629957294\\
64.25	0.23574	-125.005942642916\\
64.25	0.2394	-130.371486979531\\
64.25	0.24306	-135.85926296714\\
64.25	0.24672	-141.469270605743\\
64.25	0.25038	-147.20150989534\\
64.25	0.25404	-153.05598083593\\
64.25	0.2577	-159.032683427515\\
64.25	0.26136	-165.131617670094\\
64.25	0.26502	-171.352783563666\\
64.25	0.26868	-177.696181108233\\
64.25	0.27234	-184.161810303793\\
64.25	0.276	-190.749671150347\\
64.625	0.093	-11.2091884615752\\
64.625	0.09666	-11.8104772505149\\
64.625	0.10032	-12.5339976904485\\
64.625	0.10398	-13.379749781376\\
64.625	0.10764	-14.3477335232975\\
64.625	0.1113	-15.4379489162129\\
64.625	0.11496	-16.6503959601222\\
64.625	0.11862	-17.9850746550254\\
64.625	0.12228	-19.4419850009225\\
64.625	0.12594	-21.0211269978136\\
64.625	0.1296	-22.7225006456986\\
64.625	0.13326	-24.5461059445775\\
64.625	0.13692	-26.4919428944503\\
64.625	0.14058	-28.560011495317\\
64.625	0.14424	-30.7503117471777\\
64.625	0.1479	-33.0628436500322\\
64.625	0.15156	-35.4976072038807\\
64.625	0.15522	-38.0546024087231\\
64.625	0.15888	-40.7338292645595\\
64.625	0.16254	-43.5352877713897\\
64.625	0.1662	-46.4589779292138\\
64.625	0.16986	-49.5048997380319\\
64.625	0.17352	-52.6730531978439\\
64.625	0.17718	-55.9634383086498\\
64.625	0.18084	-59.3760550704496\\
64.625	0.1845	-62.9109034832434\\
64.625	0.18816	-66.567983547031\\
64.625	0.19182	-70.3472952618125\\
64.625	0.19548	-74.2488386275881\\
64.625	0.19914	-78.2726136443575\\
64.625	0.2028	-82.4186203121209\\
64.625	0.20646	-86.6868586308781\\
64.625	0.21012	-91.0773286006293\\
64.625	0.21378	-95.5900302213743\\
64.625	0.21744	-100.224963493113\\
64.625	0.2211	-104.982128415846\\
64.625	0.22476	-109.861524989573\\
64.625	0.22842	-114.863153214294\\
64.625	0.23208	-119.987013090009\\
64.625	0.23574	-125.233104616717\\
64.625	0.2394	-130.60142779442\\
64.625	0.24306	-136.091982623116\\
64.625	0.24672	-141.704769102806\\
64.625	0.25038	-147.439787233491\\
64.625	0.25404	-153.297037015169\\
64.625	0.2577	-159.276518447841\\
64.625	0.26136	-165.378231531507\\
64.625	0.26502	-171.602176266167\\
64.625	0.26868	-177.948352651821\\
64.625	0.27234	-184.416760688469\\
64.625	0.276	-191.00740037611\\
65	0.093	-11.3375719900871\\
65	0.09666	-11.9416396201142\\
65	0.10032	-12.6679389011353\\
65	0.10398	-13.5164698331502\\
65	0.10764	-14.4872324161591\\
65	0.1113	-15.5802266501619\\
65	0.11496	-16.7954525351586\\
65	0.11862	-18.1329100711493\\
65	0.12228	-19.5925992581338\\
65	0.12594	-21.1745200961123\\
65	0.1296	-22.8786725850847\\
65	0.13326	-24.705056725051\\
65	0.13692	-26.6536725160113\\
65	0.14058	-28.7245199579654\\
65	0.14424	-30.9175990509135\\
65	0.1479	-33.2329097948555\\
65	0.15156	-35.6704521897914\\
65	0.15522	-38.2302262357212\\
65	0.15888	-40.9122319326449\\
65	0.16254	-43.7164692805626\\
65	0.1662	-46.6429382794742\\
65	0.16986	-49.6916389293796\\
65	0.17352	-52.8625712302791\\
65	0.17718	-56.1557351821724\\
65	0.18084	-59.5711307850596\\
65	0.1845	-63.1087580389408\\
65	0.18816	-66.7686169438159\\
65	0.19182	-70.5507074996848\\
65	0.19548	-74.4550297065478\\
65	0.19914	-78.4815835644046\\
65	0.2028	-82.6303690732554\\
65	0.20646	-86.9013862331001\\
65	0.21012	-91.2946350439387\\
65	0.21378	-95.8101155057712\\
65	0.21744	-100.447827618598\\
65	0.2211	-105.207771382418\\
65	0.22476	-110.089946797232\\
65	0.22842	-115.09435386304\\
65	0.23208	-120.220992579842\\
65	0.23574	-125.469862947639\\
65	0.2394	-130.840964966428\\
65	0.24306	-136.334298636212\\
65	0.24672	-141.94986395699\\
65	0.25038	-147.687660928762\\
65	0.25404	-153.547689551527\\
65	0.2577	-159.529949825287\\
65	0.26136	-165.63444175004\\
65	0.26502	-171.861165325788\\
65	0.26868	-178.210120552529\\
65	0.27234	-184.681307430264\\
65	0.276	-191.274725958993\\
65.375	0.093	-11.4755518757188\\
65.375	0.09666	-12.0823983468333\\
65.375	0.10032	-12.8114764689418\\
65.375	0.10398	-13.6627862420442\\
65.375	0.10764	-14.6363276661405\\
65.375	0.1113	-15.7321007412307\\
65.375	0.11496	-16.9501054673149\\
65.375	0.11862	-18.290341844393\\
65.375	0.12228	-19.7528098724649\\
65.375	0.12594	-21.3375095515309\\
65.375	0.1296	-23.0444408815907\\
65.375	0.13326	-24.8736038626444\\
65.375	0.13692	-26.824998494692\\
65.375	0.14058	-28.8986247777336\\
65.375	0.14424	-31.0944827117691\\
65.375	0.1479	-33.4125722967985\\
65.375	0.15156	-35.8528935328218\\
65.375	0.15522	-38.4154464198391\\
65.375	0.15888	-41.1002309578503\\
65.375	0.16254	-43.9072471468554\\
65.375	0.1662	-46.8364949868543\\
65.375	0.16986	-49.8879744778473\\
65.375	0.17352	-53.0616856198341\\
65.375	0.17718	-56.3576284128148\\
65.375	0.18084	-59.7758028567894\\
65.375	0.1845	-63.3162089517581\\
65.375	0.18816	-66.9788466977206\\
65.375	0.19182	-70.763716094677\\
65.375	0.19548	-74.6708171426273\\
65.375	0.19914	-78.7001498415716\\
65.375	0.2028	-82.8517141915098\\
65.375	0.20646	-87.1255101924419\\
65.375	0.21012	-91.5215378443679\\
65.375	0.21378	-96.0397971472878\\
65.375	0.21744	-100.680288101202\\
65.375	0.2211	-105.443010706109\\
65.375	0.22476	-110.327964962011\\
65.375	0.22842	-115.335150868907\\
65.375	0.23208	-120.464568426796\\
65.375	0.23574	-125.71621763568\\
65.375	0.2394	-131.090098495557\\
65.375	0.24306	-136.586211006428\\
65.375	0.24672	-142.204555168294\\
65.375	0.25038	-147.945130981153\\
65.375	0.25404	-153.807938445006\\
65.375	0.2577	-159.792977559853\\
65.375	0.26136	-165.900248325694\\
65.375	0.26502	-172.129750742528\\
65.375	0.26868	-178.481484810357\\
65.375	0.27234	-184.95545052918\\
65.375	0.276	-191.551647898996\\
65.75	0.093	-11.6231281184703\\
65.75	0.09666	-12.2327534306723\\
65.75	0.10032	-12.9646103938682\\
65.75	0.10398	-13.818699008058\\
65.75	0.10764	-14.7950192732417\\
65.75	0.1113	-15.8935711894194\\
65.75	0.11496	-17.114354756591\\
65.75	0.11862	-18.4573699747564\\
65.75	0.12228	-19.9226168439158\\
65.75	0.12594	-21.5100953640692\\
65.75	0.1296	-23.2198055352164\\
65.75	0.13326	-25.0517473573575\\
65.75	0.13692	-27.0059208304926\\
65.75	0.14058	-29.0823259546216\\
65.75	0.14424	-31.2809627297446\\
65.75	0.1479	-33.6018311558614\\
65.75	0.15156	-36.0449312329721\\
65.75	0.15522	-38.6102629610768\\
65.75	0.15888	-41.2978263401754\\
65.75	0.16254	-44.1076213702679\\
65.75	0.1662	-47.0396480513543\\
65.75	0.16986	-50.0939063834346\\
65.75	0.17352	-53.2703963665089\\
65.75	0.17718	-56.569118000577\\
65.75	0.18084	-59.9900712856391\\
65.75	0.1845	-63.5332562216952\\
65.75	0.18816	-67.1986728087451\\
65.75	0.19182	-70.9863210467889\\
65.75	0.19548	-74.8962009358266\\
65.75	0.19914	-78.9283124758583\\
65.75	0.2028	-83.082655666884\\
65.75	0.20646	-87.3592305089035\\
65.75	0.21012	-91.7580370019169\\
65.75	0.21378	-96.2790751459242\\
65.75	0.21744	-100.922344940926\\
65.75	0.2211	-105.687846386921\\
65.75	0.22476	-110.57557948391\\
65.75	0.22842	-115.585544231893\\
65.75	0.23208	-120.71774063087\\
65.75	0.23574	-125.972168680841\\
65.75	0.2394	-131.348828381805\\
65.75	0.24306	-136.847719733764\\
65.75	0.24672	-142.468842736717\\
65.75	0.25038	-148.212197390663\\
65.75	0.25404	-154.077783695604\\
65.75	0.2577	-160.065601651538\\
65.75	0.26136	-166.175651258467\\
65.75	0.26502	-172.407932516389\\
65.75	0.26868	-178.762445425305\\
65.75	0.27234	-185.239189985215\\
65.75	0.276	-191.838166196119\\
66.125	0.093	-11.7803007183416\\
66.125	0.09666	-12.392704871631\\
66.125	0.10032	-13.1273406759143\\
66.125	0.10398	-13.9842081311916\\
66.125	0.10764	-14.9633072374627\\
66.125	0.1113	-16.0646379947278\\
66.125	0.11496	-17.2882004029868\\
66.125	0.11862	-18.6339944622397\\
66.125	0.12228	-20.1020201724865\\
66.125	0.12594	-21.6922775337273\\
66.125	0.1296	-23.4047665459619\\
66.125	0.13326	-25.2394872091905\\
66.125	0.13692	-27.196439523413\\
66.125	0.14058	-29.2756234886294\\
66.125	0.14424	-31.4770391048398\\
66.125	0.1479	-33.800686372044\\
66.125	0.15156	-36.2465652902422\\
66.125	0.15522	-38.8146758594342\\
66.125	0.15888	-41.5050180796203\\
66.125	0.16254	-44.3175919508002\\
66.125	0.1662	-47.252397472974\\
66.125	0.16986	-50.3094346461418\\
66.125	0.17352	-53.4887034703034\\
66.125	0.17718	-56.790203945459\\
66.125	0.18084	-60.2139360716085\\
66.125	0.1845	-63.759899848752\\
66.125	0.18816	-67.4280952768893\\
66.125	0.19182	-71.2185223560206\\
66.125	0.19548	-75.1311810861458\\
66.125	0.19914	-79.1660714672649\\
66.125	0.2028	-83.3231934993779\\
66.125	0.20646	-87.6025471824849\\
66.125	0.21012	-92.0041325165857\\
66.125	0.21378	-96.5279495016805\\
66.125	0.21744	-101.173998137769\\
66.125	0.2211	-105.942278424852\\
66.125	0.22476	-110.832790362928\\
66.125	0.22842	-115.845533951999\\
66.125	0.23208	-120.980509192063\\
66.125	0.23574	-126.237716083121\\
66.125	0.2394	-131.617154625174\\
66.125	0.24306	-137.11882481822\\
66.125	0.24672	-142.74272666226\\
66.125	0.25038	-148.488860157294\\
66.125	0.25404	-154.357225303322\\
66.125	0.2577	-160.347822100343\\
66.125	0.26136	-166.460650548359\\
66.125	0.26502	-172.695710647369\\
66.125	0.26868	-179.053002397372\\
66.125	0.27234	-185.53252579837\\
66.125	0.276	-192.134280850361\\
66.5	0.093	-11.9470696753328\\
66.5	0.09666	-12.5622526697096\\
66.5	0.10032	-13.2996673150804\\
66.5	0.10398	-14.159313611445\\
66.5	0.10764	-15.1411915588036\\
66.5	0.1113	-16.2453011571561\\
66.5	0.11496	-17.4716424065025\\
66.5	0.11862	-18.8202153068428\\
66.5	0.12228	-20.291019858177\\
66.5	0.12594	-21.8840560605052\\
66.5	0.1296	-23.5993239138273\\
66.5	0.13326	-25.4368234181433\\
66.5	0.13692	-27.3965545734532\\
66.5	0.14058	-29.478517379757\\
66.5	0.14424	-31.6827118370548\\
66.5	0.1479	-34.0091379453465\\
66.5	0.15156	-36.4577957046321\\
66.5	0.15522	-39.0286851149116\\
66.5	0.15888	-41.721806176185\\
66.5	0.16254	-44.5371588884524\\
66.5	0.1662	-47.4747432517136\\
66.5	0.16986	-50.5345592659688\\
66.5	0.17352	-53.7166069312179\\
66.5	0.17718	-57.0208862474609\\
66.5	0.18084	-60.4473972146978\\
66.5	0.1845	-63.9961398329287\\
66.5	0.18816	-67.6671141021534\\
66.5	0.19182	-71.4603200223721\\
66.5	0.19548	-75.3757575935848\\
66.5	0.19914	-79.4134268157912\\
66.5	0.2028	-83.5733276889917\\
66.5	0.20646	-87.8554602131861\\
66.5	0.21012	-92.2598243883744\\
66.5	0.21378	-96.7864202145566\\
66.5	0.21744	-101.435247691733\\
66.5	0.2211	-106.206306819903\\
66.5	0.22476	-111.099597599067\\
66.5	0.22842	-116.115120029225\\
66.5	0.23208	-121.252874110376\\
66.5	0.23574	-126.512859842522\\
66.5	0.2394	-131.895077225662\\
66.5	0.24306	-137.399526259795\\
66.5	0.24672	-143.026206944923\\
66.5	0.25038	-148.775119281044\\
66.5	0.25404	-154.646263268159\\
66.5	0.2577	-160.639638906269\\
66.5	0.26136	-166.755246195372\\
66.5	0.26502	-172.993085135469\\
66.5	0.26868	-179.35315572656\\
66.5	0.27234	-185.835457968645\\
66.5	0.276	-192.439991861723\\
66.875	0.093	-12.1234349894437\\
66.875	0.09666	-12.741396824908\\
66.875	0.10032	-13.4815903113661\\
66.875	0.10398	-14.3440154488182\\
66.875	0.10764	-15.3286722372642\\
66.875	0.1113	-16.4355606767041\\
66.875	0.11496	-17.664680767138\\
66.875	0.11862	-19.0160325085657\\
66.875	0.12228	-20.4896159009873\\
66.875	0.12594	-22.085430944403\\
66.875	0.1296	-23.8034776388125\\
66.875	0.13326	-25.6437559842159\\
66.875	0.13692	-27.6062659806132\\
66.875	0.14058	-29.6910076280045\\
66.875	0.14424	-31.8979809263897\\
66.875	0.1479	-34.2271858757688\\
66.875	0.15156	-36.6786224761418\\
66.875	0.15522	-39.2522907275087\\
66.875	0.15888	-41.9481906298696\\
66.875	0.16254	-44.7663221832243\\
66.875	0.1662	-47.706685387573\\
66.875	0.16986	-50.7692802429156\\
66.875	0.17352	-53.9541067492521\\
66.875	0.17718	-57.2611649065825\\
66.875	0.18084	-60.6904547149068\\
66.875	0.1845	-64.2419761742252\\
66.875	0.18816	-67.9157292845374\\
66.875	0.19182	-71.7117140458435\\
66.875	0.19548	-75.6299304581435\\
66.875	0.19914	-79.6703785214374\\
66.875	0.2028	-83.8330582357253\\
66.875	0.20646	-88.1179696010071\\
66.875	0.21012	-92.5251126172828\\
66.875	0.21378	-97.0544872845524\\
66.875	0.21744	-101.706093602816\\
66.875	0.2211	-106.479931572073\\
66.875	0.22476	-111.376001192325\\
66.875	0.22842	-116.39430246357\\
66.875	0.23208	-121.534835385809\\
66.875	0.23574	-126.797599959042\\
66.875	0.2394	-132.182596183269\\
66.875	0.24306	-137.68982405849\\
66.875	0.24672	-143.319283584705\\
66.875	0.25038	-149.070974761914\\
66.875	0.25404	-154.944897590117\\
66.875	0.2577	-160.941052069313\\
66.875	0.26136	-167.059438199504\\
66.875	0.26502	-173.300055980688\\
66.875	0.26868	-179.662905412867\\
66.875	0.27234	-186.147986496039\\
66.875	0.276	-192.755299230205\\
67.25	0.093	-12.3093966606745\\
67.25	0.09666	-12.9301373372262\\
67.25	0.10032	-13.6731096647718\\
67.25	0.10398	-14.5383136433112\\
67.25	0.10764	-15.5257492728446\\
67.25	0.1113	-16.635416553372\\
67.25	0.11496	-17.8673154848933\\
67.25	0.11862	-19.2214460674084\\
67.25	0.12228	-20.6978083009175\\
67.25	0.12594	-22.2964021854205\\
67.25	0.1296	-24.0172277209174\\
67.25	0.13326	-25.8602849074083\\
67.25	0.13692	-27.8255737448931\\
67.25	0.14058	-29.9130942333717\\
67.25	0.14424	-32.1228463728443\\
67.25	0.1479	-34.4548301633108\\
67.25	0.15156	-36.9090456047713\\
67.25	0.15522	-39.4854926972256\\
67.25	0.15888	-42.1841714406739\\
67.25	0.16254	-45.0050818351161\\
67.25	0.1662	-47.9482238805522\\
67.25	0.16986	-51.0135975769822\\
67.25	0.17352	-54.2012029244061\\
67.25	0.17718	-57.511039922824\\
67.25	0.18084	-60.9431085722357\\
67.25	0.1845	-64.4974088726415\\
67.25	0.18816	-68.1739408240411\\
67.25	0.19182	-71.9727044264346\\
67.25	0.19548	-75.8936996798221\\
67.25	0.19914	-79.9369265842034\\
67.25	0.2028	-84.1023851395787\\
67.25	0.20646	-88.390075345948\\
67.25	0.21012	-92.7999972033111\\
67.25	0.21378	-97.3321507116681\\
67.25	0.21744	-101.986535871019\\
67.25	0.2211	-106.763152681364\\
67.25	0.22476	-111.662001142703\\
67.25	0.22842	-116.683081255035\\
67.25	0.23208	-121.826393018362\\
67.25	0.23574	-127.091936432683\\
67.25	0.2394	-132.479711497997\\
67.25	0.24306	-137.989718214306\\
67.25	0.24672	-143.621956581608\\
67.25	0.25038	-149.376426599904\\
67.25	0.25404	-155.253128269194\\
67.25	0.2577	-161.252061589478\\
67.25	0.26136	-167.373226560756\\
67.25	0.26502	-173.616623183028\\
67.25	0.26868	-179.982251456294\\
67.25	0.27234	-186.470111380554\\
67.25	0.276	-193.080202955807\\
67.625	0.093	-12.504954689025\\
67.625	0.09666	-13.1284742066641\\
67.625	0.10032	-13.8742253752971\\
67.625	0.10398	-14.742208194924\\
67.625	0.10764	-15.7324226655449\\
67.625	0.1113	-16.8448687871596\\
67.625	0.11496	-18.0795465597683\\
67.625	0.11862	-19.4364559833709\\
67.625	0.12228	-20.9155970579674\\
67.625	0.12594	-22.5169697835579\\
67.625	0.1296	-24.2405741601422\\
67.625	0.13326	-26.0864101877205\\
67.625	0.13692	-28.0544778662927\\
67.625	0.14058	-30.1447771958587\\
67.625	0.14424	-32.3573081764188\\
67.625	0.1479	-34.6920708079727\\
67.625	0.15156	-37.1490650905206\\
67.625	0.15522	-39.7282910240624\\
67.625	0.15888	-42.4297486085981\\
67.625	0.16254	-45.2534378441277\\
67.625	0.1662	-48.1993587306512\\
67.625	0.16986	-51.2675112681686\\
67.625	0.17352	-54.45789545668\\
67.625	0.17718	-57.7705112961852\\
67.625	0.18084	-61.2053587866844\\
67.625	0.1845	-64.7624379281776\\
67.625	0.18816	-68.4417487206646\\
67.625	0.19182	-72.2432911641455\\
67.625	0.19548	-76.1670652586204\\
67.625	0.19914	-80.2130710040892\\
67.625	0.2028	-84.381308400552\\
67.625	0.20646	-88.6717774480086\\
67.625	0.21012	-93.0844781464591\\
67.625	0.21378	-97.6194104959036\\
67.625	0.21744	-102.276574496342\\
67.625	0.2211	-107.055970147774\\
67.625	0.22476	-111.957597450201\\
67.625	0.22842	-116.981456403621\\
67.625	0.23208	-122.127547008035\\
67.625	0.23574	-127.395869263443\\
67.625	0.2394	-132.786423169845\\
67.625	0.24306	-138.29920872724\\
67.625	0.24672	-143.93422593563\\
67.625	0.25038	-149.691474795014\\
67.625	0.25404	-155.570955305391\\
67.625	0.2577	-161.572667466763\\
67.625	0.26136	-167.696611279128\\
67.625	0.26502	-173.942786742487\\
67.625	0.26868	-180.311193856841\\
67.625	0.27234	-186.801832622188\\
67.625	0.276	-193.414703038529\\
68	0.093	-12.7101090744954\\
68	0.09666	-13.3364074332219\\
68	0.10032	-14.0849374429423\\
68	0.10398	-14.9556991036567\\
68	0.10764	-15.9486924153649\\
68	0.1113	-17.0639173780671\\
68	0.11496	-18.3013739917632\\
68	0.11862	-19.6610622564532\\
68	0.12228	-21.1429821721372\\
68	0.12594	-22.747133738815\\
68	0.1296	-24.4735169564868\\
68	0.13326	-26.3221318251525\\
68	0.13692	-28.2929783448121\\
68	0.14058	-30.3860565154656\\
68	0.14424	-32.6013663371131\\
68	0.1479	-34.9389078097544\\
68	0.15156	-37.3986809333897\\
68	0.15522	-39.9806857080189\\
68	0.15888	-42.684922133642\\
68	0.16254	-45.5113902102591\\
68	0.1662	-48.46008993787\\
68	0.16986	-51.5310213164748\\
68	0.17352	-54.7241843460736\\
68	0.17718	-58.0395790266663\\
68	0.18084	-61.4772053582529\\
68	0.1845	-65.0370633408335\\
68	0.18816	-68.719152974408\\
68	0.19182	-72.5234742589763\\
68	0.19548	-76.4500271945386\\
68	0.19914	-80.4988117810948\\
68	0.2028	-84.669828018645\\
68	0.20646	-88.9630759071891\\
68	0.21012	-93.378555446727\\
68	0.21378	-97.9162666372589\\
68	0.21744	-102.576209478785\\
68	0.2211	-107.358383971304\\
68	0.22476	-112.262790114818\\
68	0.22842	-117.289427909326\\
68	0.23208	-122.438297354827\\
68	0.23574	-127.709398451322\\
68	0.2394	-133.102731198812\\
68	0.24306	-138.618295597295\\
68	0.24672	-144.256091646772\\
68	0.25038	-150.016119347243\\
68	0.25404	-155.898378698708\\
68	0.2577	-161.902869701167\\
68	0.26136	-168.02959235462\\
68	0.26502	-174.278546659067\\
68	0.26868	-180.649732614507\\
68	0.27234	-187.143150220942\\
68	0.276	-193.75879947837\\
68.375	0.093	-12.9248598170856\\
68.375	0.09666	-13.5539370168995\\
68.375	0.10032	-14.3052458677074\\
68.375	0.10398	-15.1787863695091\\
68.375	0.10764	-16.1745585223048\\
68.375	0.1113	-17.2925623260944\\
68.375	0.11496	-18.532797780878\\
68.375	0.11862	-19.8952648866554\\
68.375	0.12228	-21.3799636434267\\
68.375	0.12594	-22.986894051192\\
68.375	0.1296	-24.7160561099512\\
68.375	0.13326	-26.5674498197043\\
68.375	0.13692	-28.5410751804513\\
68.375	0.14058	-30.6369321921923\\
68.375	0.14424	-32.8550208549272\\
68.375	0.1479	-35.1953411686559\\
68.375	0.15156	-37.6578931333786\\
68.375	0.15522	-40.2426767490953\\
68.375	0.15888	-42.9496920158058\\
68.375	0.16254	-45.7789389335102\\
68.375	0.1662	-48.7304175022086\\
68.375	0.16986	-51.8041277219009\\
68.375	0.17352	-55.0000695925871\\
68.375	0.17718	-58.3182431142672\\
68.375	0.18084	-61.7586482869412\\
68.375	0.1845	-65.3212851106092\\
68.375	0.18816	-69.0061535852711\\
68.375	0.19182	-72.8132537109269\\
68.375	0.19548	-76.7425854875766\\
68.375	0.19914	-80.7941489152202\\
68.375	0.2028	-84.9679439938578\\
68.375	0.20646	-89.2639707234893\\
68.375	0.21012	-93.6822291041147\\
68.375	0.21378	-98.2227191357339\\
68.375	0.21744	-102.885440818347\\
68.375	0.2211	-107.670394151954\\
68.375	0.22476	-112.577579136555\\
68.375	0.22842	-117.60699577215\\
68.375	0.23208	-122.758644058739\\
68.375	0.23574	-128.032523996322\\
68.375	0.2394	-133.428635584899\\
68.375	0.24306	-138.94697882447\\
68.375	0.24672	-144.587553715034\\
68.375	0.25038	-150.350360256593\\
68.375	0.25404	-156.235398449145\\
68.375	0.2577	-162.242668292691\\
68.375	0.26136	-168.372169787232\\
68.375	0.26502	-174.623902932766\\
68.375	0.26868	-180.997867729294\\
68.375	0.27234	-187.494064176816\\
68.375	0.276	-194.112492275332\\
68.75	0.093	-13.1492069167956\\
68.75	0.09666	-13.781062957697\\
68.75	0.10032	-14.5351506495922\\
68.75	0.10398	-15.4114699924814\\
68.75	0.10764	-16.4100209863645\\
68.75	0.1113	-17.5308036312415\\
68.75	0.11496	-18.7738179271125\\
68.75	0.11862	-20.1390638739773\\
68.75	0.12228	-21.6265414718361\\
68.75	0.12594	-23.2362507206888\\
68.75	0.1296	-24.9681916205354\\
68.75	0.13326	-26.822364171376\\
68.75	0.13692	-28.7987683732104\\
68.75	0.14058	-30.8974042260388\\
68.75	0.14424	-33.1182717298611\\
68.75	0.1479	-35.4613708846773\\
68.75	0.15156	-37.9267016904874\\
68.75	0.15522	-40.5142641472914\\
68.75	0.15888	-43.2240582550894\\
68.75	0.16254	-46.0560840138812\\
68.75	0.1662	-49.0103414236671\\
68.75	0.16986	-52.0868304844468\\
68.75	0.17352	-55.2855511962204\\
68.75	0.17718	-58.6065035589879\\
68.75	0.18084	-62.0496875727493\\
68.75	0.1845	-65.6151032375048\\
68.75	0.18816	-69.3027505532541\\
68.75	0.19182	-73.1126295199973\\
68.75	0.19548	-77.0447401377344\\
68.75	0.19914	-81.0990824064655\\
68.75	0.2028	-85.2756563261904\\
68.75	0.20646	-89.5744618969094\\
68.75	0.21012	-93.9954991186222\\
68.75	0.21378	-98.5387679913289\\
68.75	0.21744	-103.20426851503\\
68.75	0.2211	-107.992000689724\\
68.75	0.22476	-112.901964515413\\
68.75	0.22842	-117.934159992095\\
68.75	0.23208	-123.088587119771\\
68.75	0.23574	-128.365245898442\\
68.75	0.2394	-133.764136328106\\
68.75	0.24306	-139.285258408764\\
68.75	0.24672	-144.928612140416\\
68.75	0.25038	-150.694197523062\\
68.75	0.25404	-156.582014556701\\
68.75	0.2577	-162.592063241335\\
68.75	0.26136	-168.724343576963\\
68.75	0.26502	-174.978855563584\\
68.75	0.26868	-181.3555992012\\
68.75	0.27234	-187.854574489809\\
68.75	0.276	-194.475781429413\\
69.125	0.093	-13.3831503736254\\
69.125	0.09666	-14.0177852556142\\
69.125	0.10032	-14.7746517885969\\
69.125	0.10398	-15.6537499725735\\
69.125	0.10764	-16.655079807544\\
69.125	0.1113	-17.7786412935085\\
69.125	0.11496	-19.0244344304668\\
69.125	0.11862	-20.3924592184191\\
69.125	0.12228	-21.8827156573653\\
69.125	0.12594	-23.4952037473054\\
69.125	0.1296	-25.2299234882395\\
69.125	0.13326	-27.0868748801674\\
69.125	0.13692	-29.0660579230893\\
69.125	0.14058	-31.167472617005\\
69.125	0.14424	-33.3911189619148\\
69.125	0.1479	-35.7369969578184\\
69.125	0.15156	-38.2051066047159\\
69.125	0.15522	-40.7954479026074\\
69.125	0.15888	-43.5080208514928\\
69.125	0.16254	-46.3428254513721\\
69.125	0.1662	-49.2998617022453\\
69.125	0.16986	-52.3791296041124\\
69.125	0.17352	-55.5806291569735\\
69.125	0.17718	-58.9043603608284\\
69.125	0.18084	-62.3503232156773\\
69.125	0.1845	-65.9185177215201\\
69.125	0.18816	-69.6089438783569\\
69.125	0.19182	-73.4216016861875\\
69.125	0.19548	-77.356491145012\\
69.125	0.19914	-81.4136122548305\\
69.125	0.2028	-85.5929650156429\\
69.125	0.20646	-89.8945494274492\\
69.125	0.21012	-94.3183654902495\\
69.125	0.21378	-98.8644132040436\\
69.125	0.21744	-103.532692568832\\
69.125	0.2211	-108.323203584614\\
69.125	0.22476	-113.23594625139\\
69.125	0.22842	-118.270920569159\\
69.125	0.23208	-123.428126537923\\
69.125	0.23574	-128.707564157681\\
69.125	0.2394	-134.109233428432\\
69.125	0.24306	-139.633134350178\\
69.125	0.24672	-145.279266922917\\
69.125	0.25038	-151.047631146651\\
69.125	0.25404	-156.938227021378\\
69.125	0.2577	-162.951054547099\\
69.125	0.26136	-169.086113723814\\
69.125	0.26502	-175.343404551523\\
69.125	0.26868	-181.722927030226\\
69.125	0.27234	-188.224681159923\\
69.125	0.276	-194.848666940614\\
69.5	0.093	-13.626690187575\\
69.5	0.09666	-14.2641039106512\\
69.5	0.10032	-15.0237492847213\\
69.5	0.10398	-15.9056263097854\\
69.5	0.10764	-16.9097349858433\\
69.5	0.1113	-18.0360753128952\\
69.5	0.11496	-19.284647290941\\
69.5	0.11862	-20.6554509199807\\
69.5	0.12228	-22.1484862000143\\
69.5	0.12594	-23.7637531310418\\
69.5	0.1296	-25.5012517130633\\
69.5	0.13326	-27.3609819460787\\
69.5	0.13692	-29.342943830088\\
69.5	0.14058	-31.4471373650911\\
69.5	0.14424	-33.6735625510883\\
69.5	0.1479	-36.0222193880793\\
69.5	0.15156	-38.4931078760643\\
69.5	0.15522	-41.0862280150432\\
69.5	0.15888	-43.801579805016\\
69.5	0.16254	-46.6391632459827\\
69.5	0.1662	-49.5989783379433\\
69.5	0.16986	-52.6810250808979\\
69.5	0.17352	-55.8853034748464\\
69.5	0.17718	-59.2118135197887\\
69.5	0.18084	-62.660555215725\\
69.5	0.1845	-66.2315285626553\\
69.5	0.18816	-69.9247335605795\\
69.5	0.19182	-73.7401702094975\\
69.5	0.19548	-77.6778385094095\\
69.5	0.19914	-81.7377384603153\\
69.5	0.2028	-85.9198700622152\\
69.5	0.20646	-90.2242333151089\\
69.5	0.21012	-94.6508282189966\\
69.5	0.21378	-99.1996547738782\\
69.5	0.21744	-103.870712979754\\
69.5	0.2211	-108.664002836623\\
69.5	0.22476	-113.579524344486\\
69.5	0.22842	-118.617277503344\\
69.5	0.23208	-123.777262313195\\
69.5	0.23574	-129.05947877404\\
69.5	0.2394	-134.463926885879\\
69.5	0.24306	-139.990606648712\\
69.5	0.24672	-145.639518062539\\
69.5	0.25038	-151.410661127359\\
69.5	0.25404	-157.304035843174\\
69.5	0.2577	-163.319642209983\\
69.5	0.26136	-169.457480227785\\
69.5	0.26502	-175.717549896582\\
69.5	0.26868	-182.099851216372\\
69.5	0.27234	-188.604384187156\\
69.5	0.276	-195.231148808934\\
69.875	0.093	-13.8798263586445\\
69.875	0.09666	-14.520018922808\\
69.875	0.10032	-15.2824431379656\\
69.875	0.10398	-16.167099004117\\
69.875	0.10764	-17.1739865212624\\
69.875	0.1113	-18.3031056894017\\
69.875	0.11496	-19.5544565085349\\
69.875	0.11862	-20.928038978662\\
69.875	0.12228	-22.4238530997831\\
69.875	0.12594	-24.041898871898\\
69.875	0.1296	-25.7821762950069\\
69.875	0.13326	-27.6446853691097\\
69.875	0.13692	-29.6294260942064\\
69.875	0.14058	-31.736398470297\\
69.875	0.14424	-33.9656024973816\\
69.875	0.1479	-36.3170381754601\\
69.875	0.15156	-38.7907055045325\\
69.875	0.15522	-41.3866044845987\\
69.875	0.15888	-44.104735115659\\
69.875	0.16254	-46.9450973977131\\
69.875	0.1662	-49.9076913307612\\
69.875	0.16986	-52.9925169148031\\
69.875	0.17352	-56.199574149839\\
69.875	0.17718	-59.5288630358688\\
69.875	0.18084	-62.9803835728925\\
69.875	0.1845	-66.5541357609102\\
69.875	0.18816	-70.2501195999218\\
69.875	0.19182	-74.0683350899273\\
69.875	0.19548	-78.0087822309267\\
69.875	0.19914	-82.0714610229199\\
69.875	0.2028	-86.2563714659073\\
69.875	0.20646	-90.5635135598884\\
69.875	0.21012	-94.9928873048635\\
69.875	0.21378	-99.5444927008324\\
69.875	0.21744	-104.218329747795\\
69.875	0.2211	-109.014398445752\\
69.875	0.22476	-113.932698794703\\
69.875	0.22842	-118.973230794648\\
69.875	0.23208	-124.135994445586\\
69.875	0.23574	-129.420989747519\\
69.875	0.2394	-134.828216700445\\
69.875	0.24306	-140.357675304366\\
69.875	0.24672	-146.00936555928\\
69.875	0.25038	-151.783287465188\\
69.875	0.25404	-157.67944102209\\
69.875	0.2577	-163.697826229986\\
69.875	0.26136	-169.838443088876\\
69.875	0.26502	-176.10129159876\\
69.875	0.26868	-182.486371759638\\
69.875	0.27234	-188.993683571509\\
69.875	0.276	-195.623227034375\\
70.25	0.093	-14.1425588868337\\
70.25	0.09666	-14.7855302920847\\
70.25	0.10032	-15.5507333483297\\
70.25	0.10398	-16.4381680555685\\
70.25	0.10764	-17.4478344138013\\
70.25	0.1113	-18.579732423028\\
70.25	0.11496	-19.8338620832487\\
70.25	0.11862	-21.2102233944632\\
70.25	0.12228	-22.7088163566717\\
70.25	0.12594	-24.3296409698741\\
70.25	0.1296	-26.0726972340703\\
70.25	0.13326	-27.9379851492606\\
70.25	0.13692	-29.9255047154447\\
70.25	0.14058	-32.0352559326227\\
70.25	0.14424	-34.2672388007947\\
70.25	0.1479	-36.6214533199606\\
70.25	0.15156	-39.0978994901204\\
70.25	0.15522	-41.6965773112741\\
70.25	0.15888	-44.4174867834218\\
70.25	0.16254	-47.2606279065634\\
70.25	0.1662	-50.2260006806988\\
70.25	0.16986	-53.3136051058282\\
70.25	0.17352	-56.5234411819515\\
70.25	0.17718	-59.8555089090687\\
70.25	0.18084	-63.3098082871799\\
70.25	0.1845	-66.886339316285\\
70.25	0.18816	-70.585101996384\\
70.25	0.19182	-74.4060963274769\\
70.25	0.19548	-78.3493223095637\\
70.25	0.19914	-82.4147799426444\\
70.25	0.2028	-86.6024692267191\\
70.25	0.20646	-90.9123901617878\\
70.25	0.21012	-95.3445427478502\\
70.25	0.21378	-99.8989269849066\\
70.25	0.21744	-104.575542872957\\
70.25	0.2211	-109.374390412001\\
70.25	0.22476	-114.295469602039\\
70.25	0.22842	-119.338780443071\\
70.25	0.23208	-124.504322935097\\
70.25	0.23574	-129.792097078117\\
70.25	0.2394	-135.202102872131\\
70.25	0.24306	-140.734340317139\\
70.25	0.24672	-146.388809413141\\
70.25	0.25038	-152.165510160136\\
70.25	0.25404	-158.064442558126\\
70.25	0.2577	-164.085606607109\\
70.25	0.26136	-170.229002307087\\
70.25	0.26502	-176.494629658058\\
70.25	0.26868	-182.882488660023\\
70.25	0.27234	-189.392579312982\\
70.25	0.276	-196.024901616935\\
70.625	0.093	-14.4148877721427\\
70.625	0.09666	-15.0606380184812\\
70.625	0.10032	-15.8286199158135\\
70.625	0.10398	-16.7188334641398\\
70.625	0.10764	-17.73127866346\\
70.625	0.1113	-18.8659555137742\\
70.625	0.11496	-20.1228640150822\\
70.625	0.11862	-21.5020041673842\\
70.625	0.12228	-23.0033759706801\\
70.625	0.12594	-24.6269794249699\\
70.625	0.1296	-26.3728145302536\\
70.625	0.13326	-28.2408812865312\\
70.625	0.13692	-30.2311796938028\\
70.625	0.14058	-32.3437097520683\\
70.625	0.14424	-34.5784714613277\\
70.625	0.1479	-36.935464821581\\
70.625	0.15156	-39.4146898328282\\
70.625	0.15522	-42.0161464950693\\
70.625	0.15888	-44.7398348083044\\
70.625	0.16254	-47.5857547725334\\
70.625	0.1662	-50.5539063877563\\
70.625	0.16986	-53.6442896539731\\
70.625	0.17352	-56.8569045711839\\
70.625	0.17718	-60.1917511393885\\
70.625	0.18084	-63.6488293585871\\
70.625	0.1845	-67.2281392287796\\
70.625	0.18816	-70.929680749966\\
70.625	0.19182	-74.7534539221463\\
70.625	0.19548	-78.6994587453206\\
70.625	0.19914	-82.7676952194887\\
70.625	0.2028	-86.9581633446508\\
70.625	0.20646	-91.2708631208069\\
70.625	0.21012	-95.7057945479567\\
70.625	0.21378	-100.262957626101\\
70.625	0.21744	-104.942352355238\\
70.625	0.2211	-109.74397873537\\
70.625	0.22476	-114.667836766496\\
70.625	0.22842	-119.713926448615\\
70.625	0.23208	-124.882247781728\\
70.625	0.23574	-130.172800765836\\
70.625	0.2394	-135.585585400937\\
70.625	0.24306	-141.120601687032\\
70.625	0.24672	-146.777849624121\\
70.625	0.25038	-152.557329212204\\
70.625	0.25404	-158.459040451281\\
70.625	0.2577	-164.482983341352\\
70.625	0.26136	-170.629157882417\\
70.625	0.26502	-176.897564074476\\
70.625	0.26868	-183.288201917528\\
70.625	0.27234	-189.801071411575\\
70.625	0.276	-196.436172556615\\
71	0.093	-14.6968130145715\\
71	0.09666	-15.3453421019974\\
71	0.10032	-16.1161028404172\\
71	0.10398	-17.0090952298309\\
71	0.10764	-18.0243192702386\\
71	0.1113	-19.1617749616401\\
71	0.11496	-20.4214623040356\\
71	0.11862	-21.803381297425\\
71	0.12228	-23.3075319418083\\
71	0.12594	-24.9339142371855\\
71	0.1296	-26.6825281835567\\
71	0.13326	-28.5533737809217\\
71	0.13692	-30.5464510292807\\
71	0.14058	-32.6617599286336\\
71	0.14424	-34.8993004789804\\
71	0.1479	-37.2590726803212\\
71	0.15156	-39.7410765326558\\
71	0.15522	-42.3453120359843\\
71	0.15888	-45.0717791903069\\
71	0.16254	-47.9204779956233\\
71	0.1662	-50.8914084519336\\
71	0.16986	-53.9845705592378\\
71	0.17352	-57.199964317536\\
71	0.17718	-60.537589726828\\
71	0.18084	-63.997446787114\\
71	0.1845	-67.579535498394\\
71	0.18816	-71.2838558606678\\
71	0.19182	-75.1104078739355\\
71	0.19548	-79.0591915381972\\
71	0.19914	-83.1302068534528\\
71	0.2028	-87.3234538197023\\
71	0.20646	-91.6389324369458\\
71	0.21012	-96.0766427051831\\
71	0.21378	-100.636584624414\\
71	0.21744	-105.318758194639\\
71	0.2211	-110.123163415859\\
71	0.22476	-115.049800288072\\
71	0.22842	-120.098668811279\\
71	0.23208	-125.269768985479\\
71	0.23574	-130.563100810674\\
71	0.2394	-135.978664286863\\
71	0.24306	-141.516459414045\\
71	0.24672	-147.176486192222\\
71	0.25038	-152.958744621392\\
71	0.25404	-158.863234701557\\
71	0.2577	-164.889956432715\\
71	0.26136	-171.038909814867\\
71	0.26502	-177.310094848013\\
71	0.26868	-183.703511532153\\
71	0.27234	-190.219159867287\\
71	0.276	-196.857039853415\\
71.375	0.093	-14.9883346141203\\
71.375	0.09666	-15.6396425426336\\
71.375	0.10032	-16.4131821221408\\
71.375	0.10398	-17.3089533526419\\
71.375	0.10764	-18.326956234137\\
71.375	0.1113	-19.467190766626\\
71.375	0.11496	-20.7296569501089\\
71.375	0.11862	-22.1143547845857\\
71.375	0.12228	-23.6212842700564\\
71.375	0.12594	-25.2504454065211\\
71.375	0.1296	-27.0018381939796\\
71.375	0.13326	-28.8754626324321\\
71.375	0.13692	-30.8713187218785\\
71.375	0.14058	-32.9894064623188\\
71.375	0.14424	-35.2297258537531\\
71.375	0.1479	-37.5922768961812\\
71.375	0.15156	-40.0770595896033\\
71.375	0.15522	-42.6840739340193\\
71.375	0.15888	-45.4133199294292\\
71.375	0.16254	-48.2647975758331\\
71.375	0.1662	-51.2385068732308\\
71.375	0.16986	-54.3344478216225\\
71.375	0.17352	-57.552620421008\\
71.375	0.17718	-60.8930246713875\\
71.375	0.18084	-64.3556605727609\\
71.375	0.1845	-67.9405281251283\\
71.375	0.18816	-71.6476273284895\\
71.375	0.19182	-75.4769581828447\\
71.375	0.19548	-79.4285206881938\\
71.375	0.19914	-83.5023148445368\\
71.375	0.2028	-87.6983406518738\\
71.375	0.20646	-92.0165981102046\\
71.375	0.21012	-96.4570872195294\\
71.375	0.21378	-101.019807979848\\
71.375	0.21744	-105.704760391161\\
71.375	0.2211	-110.511944453467\\
71.375	0.22476	-115.441360166768\\
71.375	0.22842	-120.493007531062\\
71.375	0.23208	-125.66688654635\\
71.375	0.23574	-130.962997212632\\
71.375	0.2394	-136.381339529909\\
71.375	0.24306	-141.921913498178\\
71.375	0.24672	-147.584719117442\\
71.375	0.25038	-153.3697563877\\
71.375	0.25404	-159.277025308952\\
71.375	0.2577	-165.306525881198\\
71.375	0.26136	-171.458258104437\\
71.375	0.26502	-177.732221978671\\
71.375	0.26868	-184.128417503898\\
71.375	0.27234	-190.64684468012\\
71.375	0.276	-197.287503507335\\
71.75	0.093	-15.2894525707887\\
71.75	0.09666	-15.9435393403894\\
71.75	0.10032	-16.7198577609841\\
71.75	0.10398	-17.6184078325726\\
71.75	0.10764	-18.6391895551551\\
71.75	0.1113	-19.7822029287315\\
71.75	0.11496	-21.0474479533019\\
71.75	0.11862	-22.4349246288661\\
71.75	0.12228	-23.9446329554242\\
71.75	0.12594	-25.5765729329763\\
71.75	0.1296	-27.3307445615223\\
71.75	0.13326	-29.2071478410622\\
71.75	0.13692	-31.205782771596\\
71.75	0.14058	-33.3266493531237\\
71.75	0.14424	-35.5697475856454\\
71.75	0.1479	-37.935077469161\\
71.75	0.15156	-40.4226390036705\\
71.75	0.15522	-43.0324321891739\\
71.75	0.15888	-45.7644570256713\\
71.75	0.16254	-48.6187135131625\\
71.75	0.1662	-51.5952016516477\\
71.75	0.16986	-54.6939214411267\\
71.75	0.17352	-57.9148728815998\\
71.75	0.17718	-61.2580559730667\\
71.75	0.18084	-64.7234707155275\\
71.75	0.1845	-68.3111171089823\\
71.75	0.18816	-72.020995153431\\
71.75	0.19182	-75.8531048488735\\
71.75	0.19548	-79.80744619531\\
71.75	0.19914	-83.8840191927405\\
71.75	0.2028	-88.0828238411648\\
71.75	0.20646	-92.4038601405831\\
71.75	0.21012	-96.8471280909953\\
71.75	0.21378	-101.412627692401\\
71.75	0.21744	-106.100358944801\\
71.75	0.2211	-110.910321848195\\
71.75	0.22476	-115.842516402583\\
71.75	0.22842	-120.896942607965\\
71.75	0.23208	-126.073600464341\\
71.75	0.23574	-131.37248997171\\
71.75	0.2394	-136.793611130074\\
71.75	0.24306	-142.336963939431\\
71.75	0.24672	-148.002548399783\\
71.75	0.25038	-153.790364511128\\
71.75	0.25404	-159.700412273467\\
71.75	0.2577	-165.7326916868\\
71.75	0.26136	-171.887202751127\\
71.75	0.26502	-178.163945466448\\
71.75	0.26868	-184.562919832763\\
71.75	0.27234	-191.084125850072\\
71.75	0.276	-197.727563518375\\
72.125	0.093	-15.600166884577\\
72.125	0.09666	-16.2570324952651\\
72.125	0.10032	-17.0361297569472\\
72.125	0.10398	-17.9374586696232\\
72.125	0.10764	-18.9610192332931\\
72.125	0.1113	-20.1068114479569\\
72.125	0.11496	-21.3748353136147\\
72.125	0.11862	-22.7650908302663\\
72.125	0.12228	-24.2775779979119\\
72.125	0.12594	-25.9122968165514\\
72.125	0.1296	-27.6692472861848\\
72.125	0.13326	-29.5484294068121\\
72.125	0.13692	-31.5498431784333\\
72.125	0.14058	-33.6734886010485\\
72.125	0.14424	-35.9193656746576\\
72.125	0.1479	-38.2874743992606\\
72.125	0.15156	-40.7778147748575\\
72.125	0.15522	-43.3903868014484\\
72.125	0.15888	-46.1251904790332\\
72.125	0.16254	-48.9822258076118\\
72.125	0.1662	-51.9614927871844\\
72.125	0.16986	-55.0629914177509\\
72.125	0.17352	-58.2867216993113\\
72.125	0.17718	-61.6326836318657\\
72.125	0.18084	-65.1008772154139\\
72.125	0.1845	-68.6913024499561\\
72.125	0.18816	-72.4039593354922\\
72.125	0.19182	-76.2388478720222\\
72.125	0.19548	-80.1959680595461\\
72.125	0.19914	-84.275319898064\\
72.125	0.2028	-88.4769033875758\\
72.125	0.20646	-92.8007185280815\\
72.125	0.21012	-97.2467653195811\\
72.125	0.21378	-101.815043762075\\
72.125	0.21744	-106.505553855562\\
72.125	0.2211	-111.318295600043\\
72.125	0.22476	-116.253268995519\\
72.125	0.22842	-121.310474041988\\
72.125	0.23208	-126.489910739451\\
72.125	0.23574	-131.791579087908\\
72.125	0.2394	-137.215479087359\\
72.125	0.24306	-142.761610737804\\
72.125	0.24672	-148.429974039243\\
72.125	0.25038	-154.220568991675\\
72.125	0.25404	-160.133395595102\\
72.125	0.2577	-166.168453849523\\
72.125	0.26136	-172.325743754937\\
72.125	0.26502	-178.605265311345\\
72.125	0.26868	-185.007018518748\\
72.125	0.27234	-191.531003377144\\
72.125	0.276	-198.177219886534\\
72.5	0.093	-15.9204775554851\\
72.5	0.09666	-16.5801220072606\\
72.5	0.10032	-17.3619981100301\\
72.5	0.10398	-18.2661058637935\\
72.5	0.10764	-19.2924452685509\\
72.5	0.1113	-20.4410163243021\\
72.5	0.11496	-21.7118190310473\\
72.5	0.11862	-23.1048533887863\\
72.5	0.12228	-24.6201193975193\\
72.5	0.12594	-26.2576170572463\\
72.5	0.1296	-28.0173463679671\\
72.5	0.13326	-29.8993073296818\\
72.5	0.13692	-31.9034999423905\\
72.5	0.14058	-34.0299242060931\\
72.5	0.14424	-36.2785801207896\\
72.5	0.1479	-38.64946768648\\
72.5	0.15156	-41.1425869031644\\
72.5	0.15522	-43.7579377708426\\
72.5	0.15888	-46.4955202895148\\
72.5	0.16254	-49.3553344591809\\
72.5	0.1662	-52.3373802798409\\
72.5	0.16986	-55.4416577514948\\
72.5	0.17352	-58.6681668741427\\
72.5	0.17718	-62.0169076477844\\
72.5	0.18084	-65.4878800724201\\
72.5	0.1845	-69.0810841480497\\
72.5	0.18816	-72.7965198746733\\
72.5	0.19182	-76.6341872522906\\
72.5	0.19548	-80.594086280902\\
72.5	0.19914	-84.6762169605073\\
72.5	0.2028	-88.8805792911065\\
72.5	0.20646	-93.2071732726996\\
72.5	0.21012	-97.6559989052867\\
72.5	0.21378	-102.227056188868\\
72.5	0.21744	-106.920345123442\\
72.5	0.2211	-111.735865709011\\
72.5	0.22476	-116.673617945574\\
72.5	0.22842	-121.733601833131\\
72.5	0.23208	-126.915817371681\\
72.5	0.23574	-132.220264561226\\
72.5	0.2394	-137.646943401764\\
72.5	0.24306	-143.195853893296\\
72.5	0.24672	-148.866996035822\\
72.5	0.25038	-154.660369829343\\
72.5	0.25404	-160.575975273857\\
72.5	0.2577	-166.613812369365\\
72.5	0.26136	-172.773881115866\\
72.5	0.26502	-179.056181513362\\
72.5	0.26868	-185.460713561852\\
72.5	0.27234	-191.987477261336\\
72.5	0.276	-198.636472611813\\
72.875	0.093	-16.250384583513\\
72.875	0.09666	-16.912807876376\\
72.875	0.10032	-17.6974628202329\\
72.875	0.10398	-18.6043494150837\\
72.875	0.10764	-19.6334676609285\\
72.875	0.1113	-20.7848175577671\\
72.875	0.11496	-22.0583991055997\\
72.875	0.11862	-23.4542123044262\\
72.875	0.12228	-24.9722571542466\\
72.875	0.12594	-26.6125336550609\\
72.875	0.1296	-28.3750418068692\\
72.875	0.13326	-30.2597816096714\\
72.875	0.13692	-32.2667530634675\\
72.875	0.14058	-34.3959561682574\\
72.875	0.14424	-36.6473909240414\\
72.875	0.1479	-39.0210573308192\\
72.875	0.15156	-41.516955388591\\
72.875	0.15522	-44.1350850973567\\
72.875	0.15888	-46.8754464571163\\
72.875	0.16254	-49.7380394678698\\
72.875	0.1662	-52.7228641296172\\
72.875	0.16986	-55.8299204423586\\
72.875	0.17352	-59.0592084060939\\
72.875	0.17718	-62.410728020823\\
72.875	0.18084	-65.8844792865461\\
72.875	0.1845	-69.4804622032632\\
72.875	0.18816	-73.1986767709741\\
72.875	0.19182	-77.0391229896789\\
72.875	0.19548	-81.0018008593778\\
72.875	0.19914	-85.0867103800704\\
72.875	0.2028	-89.2938515517571\\
72.875	0.20646	-93.6232243744376\\
72.875	0.21012	-98.0748288481121\\
72.875	0.21378	-102.64866497278\\
72.875	0.21744	-107.344732748443\\
72.875	0.2211	-112.163032175099\\
72.875	0.22476	-117.103563252749\\
72.875	0.22842	-122.166325981393\\
72.875	0.23208	-127.351320361031\\
72.875	0.23574	-132.658546391663\\
72.875	0.2394	-138.088004073289\\
72.875	0.24306	-143.639693405908\\
72.875	0.24672	-149.313614389522\\
72.875	0.25038	-155.10976702413\\
72.875	0.25404	-161.028151309731\\
72.875	0.2577	-167.068767246326\\
72.875	0.26136	-173.231614833916\\
72.875	0.26502	-179.516694072499\\
72.875	0.26868	-185.924004962076\\
72.875	0.27234	-192.453547502647\\
72.875	0.276	-199.105321694212\\
73.25	0.093	-16.5898879686607\\
73.25	0.09666	-17.2550901026111\\
73.25	0.10032	-18.0425238875554\\
73.25	0.10398	-18.9521893234936\\
73.25	0.10764	-19.9840864104258\\
73.25	0.1113	-21.1382151483519\\
73.25	0.11496	-22.4145755372719\\
73.25	0.11862	-23.8131675771858\\
73.25	0.12228	-25.3339912680937\\
73.25	0.12594	-26.9770466099954\\
73.25	0.1296	-28.7423336028911\\
73.25	0.13326	-30.6298522467807\\
73.25	0.13692	-32.6396025416642\\
73.25	0.14058	-34.7715844875416\\
73.25	0.14424	-37.025798084413\\
73.25	0.1479	-39.4022433322783\\
73.25	0.15156	-41.9009202311375\\
73.25	0.15522	-44.5218287809906\\
73.25	0.15888	-47.2649689818376\\
73.25	0.16254	-50.1303408336785\\
73.25	0.1662	-53.1179443365134\\
73.25	0.16986	-56.2277794903421\\
73.25	0.17352	-59.4598462951648\\
73.25	0.17718	-62.8141447509814\\
73.25	0.18084	-66.2906748577919\\
73.25	0.1845	-69.8894366155964\\
73.25	0.18816	-73.6104300243948\\
73.25	0.19182	-77.453655084187\\
73.25	0.19548	-81.4191117949732\\
73.25	0.19914	-85.5068001567533\\
73.25	0.2028	-89.7167201695274\\
73.25	0.20646	-94.0488718332954\\
73.25	0.21012	-98.5032551480573\\
73.25	0.21378	-103.079870113813\\
73.25	0.21744	-107.778716730563\\
73.25	0.2211	-112.599794998306\\
73.25	0.22476	-117.543104917044\\
73.25	0.22842	-122.608646486775\\
73.25	0.23208	-127.796419707501\\
73.25	0.23574	-133.10642457922\\
73.25	0.2394	-138.538661101933\\
73.25	0.24306	-144.09312927564\\
73.25	0.24672	-149.769829100341\\
73.25	0.25038	-155.568760576036\\
73.25	0.25404	-161.489923702725\\
73.25	0.2577	-167.533318480408\\
73.25	0.26136	-173.698944909085\\
73.25	0.26502	-179.986802988755\\
73.25	0.26868	-186.39689271942\\
73.25	0.27234	-192.929214101078\\
73.25	0.276	-199.583767133731\\
73.625	0.093	-16.9389877109282\\
73.625	0.09666	-17.606968685966\\
73.625	0.10032	-18.3971813119978\\
73.625	0.10398	-19.3096255890234\\
73.625	0.10764	-20.344301517043\\
73.625	0.1113	-21.5012090960565\\
73.625	0.11496	-22.7803483260639\\
73.625	0.11862	-24.1817192070653\\
73.625	0.12228	-25.7053217390605\\
73.625	0.12594	-27.3511559220497\\
73.625	0.1296	-29.1192217560328\\
73.625	0.13326	-31.0095192410098\\
73.625	0.13692	-33.0220483769808\\
73.625	0.14058	-35.1568091639456\\
73.625	0.14424	-37.4138016019044\\
73.625	0.1479	-39.7930256908571\\
73.625	0.15156	-42.2944814308037\\
73.625	0.15522	-44.9181688217442\\
73.625	0.15888	-47.6640878636787\\
73.625	0.16254	-50.532238556607\\
73.625	0.1662	-53.5226209005293\\
73.625	0.16986	-56.6352348954455\\
73.625	0.17352	-59.8700805413556\\
73.625	0.17718	-63.2271578382596\\
73.625	0.18084	-66.7064667861576\\
73.625	0.1845	-70.3080073850495\\
73.625	0.18816	-74.0317796349353\\
73.625	0.19182	-77.8777835358149\\
73.625	0.19548	-81.8460190876885\\
73.625	0.19914	-85.9364862905561\\
73.625	0.2028	-90.1491851444176\\
73.625	0.20646	-94.484115649273\\
73.625	0.21012	-98.9412778051223\\
73.625	0.21378	-103.520671611965\\
73.625	0.21744	-108.222297069803\\
73.625	0.2211	-113.046154178634\\
73.625	0.22476	-117.992242938459\\
73.625	0.22842	-123.060563349277\\
73.625	0.23208	-128.25111541109\\
73.625	0.23574	-133.563899123897\\
73.625	0.2394	-138.998914487698\\
73.625	0.24306	-144.556161502492\\
73.625	0.24672	-150.235640168281\\
73.625	0.25038	-156.037350485063\\
73.625	0.25404	-161.961292452839\\
73.625	0.2577	-168.00746607161\\
73.625	0.26136	-174.175871341374\\
73.625	0.26502	-180.466508262132\\
73.625	0.26868	-186.879376833884\\
73.625	0.27234	-193.41447705663\\
73.625	0.276	-200.071808930369\\
74	0.093	-17.2976838103155\\
74	0.09666	-17.9684436264408\\
74	0.10032	-18.7614350935599\\
74	0.10398	-19.676658211673\\
74	0.10764	-20.71411298078\\
74	0.1113	-21.873799400881\\
74	0.11496	-23.1557174719758\\
74	0.11862	-24.5598671940646\\
74	0.12228	-26.0862485671472\\
74	0.12594	-27.7348615912239\\
74	0.1296	-29.5057062662944\\
74	0.13326	-31.3987825923588\\
74	0.13692	-33.4140905694171\\
74	0.14058	-35.5516301974694\\
74	0.14424	-37.8114014765157\\
74	0.1479	-40.1934044065557\\
74	0.15156	-42.6976389875898\\
74	0.15522	-45.3241052196177\\
74	0.15888	-48.0728031026396\\
74	0.16254	-50.9437326366554\\
74	0.1662	-53.9368938216651\\
74	0.16986	-57.0522866576686\\
74	0.17352	-60.2899111446662\\
74	0.17718	-63.6497672826576\\
74	0.18084	-67.131855071643\\
74	0.1845	-70.7361745116223\\
74	0.18816	-74.4627256025956\\
74	0.19182	-78.3115083445626\\
74	0.19548	-82.2825227375237\\
74	0.19914	-86.3757687814787\\
74	0.2028	-90.5912464764275\\
74	0.20646	-94.9289558223704\\
74	0.21012	-99.3888968193071\\
74	0.21378	-103.971069467238\\
74	0.21744	-108.675473766162\\
74	0.2211	-113.502109716081\\
74	0.22476	-118.450977316993\\
74	0.22842	-123.522076568899\\
74	0.23208	-128.7154074718\\
74	0.23574	-134.030970025694\\
74	0.2394	-139.468764230582\\
74	0.24306	-145.028790086464\\
74	0.24672	-150.71104759334\\
74	0.25038	-156.515536751209\\
74	0.25404	-162.442257560073\\
74	0.2577	-168.491210019931\\
74	0.26136	-174.662394130782\\
74	0.26502	-180.955809892628\\
74	0.26868	-187.371457305467\\
74	0.27234	-193.909336369301\\
74	0.276	-200.569447084128\\
};\label{tikz:theta_surf}
\end{axis}
\end{tikzpicture}%
	\caption{Parameter maps reconstructed with LS surface fitting to the internal parameters estimated from the data sample of length 2000.}
\end{figure}
\begin{table}[!h]
	\centering
	\caption{Estimated polynomial coefficients for the sample length 4000.}
	\small
	\input{betas_C_ny_0_nu_4_size_4000.tex}
\end{table}
\begin{figure}[!h]
	\centering
	% This file was created by matlab2tikz.
% Minimal pgfplots version: 1.3
%
\definecolor{mycolor1}{rgb}{0.00000,0.44700,0.74100}%
\definecolor{mycolor2}{rgb}{0.85000,0.32500,0.09800}%
%
\begin{tikzpicture}

\begin{axis}[%
width=4.527496cm,
height=3.050847cm,
at={(6.483547cm,16.949153cm)},
scale only axis,
xmin=56,
xmax=74,
tick align=outside,
xlabel={$L_{cut}$},
xmajorgrids,
ymin=0.093,
ymax=0.276,
ylabel={$D_{rlx}$},
ymajorgrids,
zmin=0,
zmax=500,
zlabel={$x_3$},
zmajorgrids,
view={-140}{50},
legend style={at={(1.03,1)},anchor=north west,legend cell align=left,align=left,draw=white!15!black}
]
\addplot3[only marks,mark=*,mark options={},mark size=1.5000pt,color=mycolor1] plot table[row sep=crcr,]{%
74	0.123	63.1998222381781\\
72	0.113	53.0854272097683\\
61	0.095	30.0912757972533\\
56	0.093	24.5470249636231\\
};
\addplot3[only marks,mark=*,mark options={},mark size=1.5000pt,color=mycolor2] plot table[row sep=crcr,]{%
67	0.276	426.089911235736\\
66	0.255	362.142127417182\\
62	0.209	208.403395630899\\
57	0.193	168.782441587851\\
};
\addplot3[only marks,mark=*,mark options={},mark size=1.5000pt,color=black] plot table[row sep=crcr,]{%
69	0.104	42.006270654595\\
};
\addplot3[only marks,mark=*,mark options={},mark size=1.5000pt,color=black] plot table[row sep=crcr,]{%
64	0.23	273.266329965504\\
};

\addplot3[%
surf,
opacity=0.7,
shader=interp,
colormap={mymap}{[1pt] rgb(0pt)=(0.0901961,0.239216,0.0745098); rgb(1pt)=(0.0945149,0.242058,0.0739522); rgb(2pt)=(0.0988592,0.244894,0.0733566); rgb(3pt)=(0.103229,0.247724,0.0727241); rgb(4pt)=(0.107623,0.250549,0.0720557); rgb(5pt)=(0.112043,0.253367,0.0713525); rgb(6pt)=(0.116487,0.25618,0.0706154); rgb(7pt)=(0.120956,0.258986,0.0698456); rgb(8pt)=(0.125449,0.261787,0.0690441); rgb(9pt)=(0.129967,0.264581,0.0682118); rgb(10pt)=(0.134508,0.26737,0.06735); rgb(11pt)=(0.139074,0.270152,0.0664596); rgb(12pt)=(0.143663,0.272929,0.0655416); rgb(13pt)=(0.148275,0.275699,0.0645971); rgb(14pt)=(0.152911,0.278463,0.0636271); rgb(15pt)=(0.15757,0.281221,0.0626328); rgb(16pt)=(0.162252,0.283973,0.0616151); rgb(17pt)=(0.166957,0.286719,0.060575); rgb(18pt)=(0.171685,0.289458,0.0595136); rgb(19pt)=(0.176434,0.292191,0.0584321); rgb(20pt)=(0.181207,0.294918,0.0573313); rgb(21pt)=(0.186001,0.297639,0.0562123); rgb(22pt)=(0.190817,0.300353,0.0550763); rgb(23pt)=(0.195655,0.303061,0.0539242); rgb(24pt)=(0.200514,0.305763,0.052757); rgb(25pt)=(0.205395,0.308459,0.0515759); rgb(26pt)=(0.210296,0.311149,0.0503624); rgb(27pt)=(0.215212,0.313846,0.0490067); rgb(28pt)=(0.220142,0.316548,0.0475043); rgb(29pt)=(0.22509,0.319254,0.0458704); rgb(30pt)=(0.230056,0.321962,0.0441205); rgb(31pt)=(0.235042,0.324671,0.04227); rgb(32pt)=(0.240048,0.327379,0.0403343); rgb(33pt)=(0.245078,0.330085,0.0383287); rgb(34pt)=(0.250131,0.332786,0.0362688); rgb(35pt)=(0.25521,0.335482,0.0341698); rgb(36pt)=(0.260317,0.33817,0.0320472); rgb(37pt)=(0.265451,0.340849,0.0299163); rgb(38pt)=(0.270616,0.343517,0.0277927); rgb(39pt)=(0.275813,0.346172,0.0256916); rgb(40pt)=(0.281043,0.348814,0.0236284); rgb(41pt)=(0.286307,0.35144,0.0216186); rgb(42pt)=(0.291607,0.354048,0.0196776); rgb(43pt)=(0.296945,0.356637,0.0178207); rgb(44pt)=(0.302322,0.359206,0.0160634); rgb(45pt)=(0.307739,0.361753,0.0144211); rgb(46pt)=(0.313198,0.364275,0.0129091); rgb(47pt)=(0.318701,0.366772,0.0115428); rgb(48pt)=(0.324249,0.369242,0.0103377); rgb(49pt)=(0.329843,0.371682,0.00930909); rgb(50pt)=(0.335485,0.374093,0.00847245); rgb(51pt)=(0.341176,0.376471,0.00784314); rgb(52pt)=(0.346925,0.378826,0.00732741); rgb(53pt)=(0.352735,0.381168,0.00682184); rgb(54pt)=(0.358605,0.383497,0.00632729); rgb(55pt)=(0.364532,0.385812,0.00584464); rgb(56pt)=(0.370516,0.388113,0.00537476); rgb(57pt)=(0.376552,0.390399,0.00491852); rgb(58pt)=(0.38264,0.39267,0.00447681); rgb(59pt)=(0.388777,0.394925,0.00405048); rgb(60pt)=(0.394962,0.397164,0.00364042); rgb(61pt)=(0.401191,0.399386,0.00324749); rgb(62pt)=(0.407464,0.401592,0.00287258); rgb(63pt)=(0.413777,0.40378,0.00251655); rgb(64pt)=(0.420129,0.40595,0.00218028); rgb(65pt)=(0.426518,0.408102,0.00186463); rgb(66pt)=(0.432942,0.410234,0.00157049); rgb(67pt)=(0.439399,0.412348,0.00129873); rgb(68pt)=(0.445885,0.414441,0.00105022); rgb(69pt)=(0.452401,0.416515,0.000825833); rgb(70pt)=(0.458942,0.418567,0.000626441); rgb(71pt)=(0.465508,0.420599,0.00045292); rgb(72pt)=(0.472096,0.422609,0.000306141); rgb(73pt)=(0.478704,0.424596,0.000186979); rgb(74pt)=(0.485331,0.426562,9.63073e-05); rgb(75pt)=(0.491973,0.428504,3.49981e-05); rgb(76pt)=(0.498628,0.430422,3.92506e-06); rgb(77pt)=(0.505323,0.432315,0); rgb(78pt)=(0.512206,0.434168,0); rgb(79pt)=(0.519282,0.435983,0); rgb(80pt)=(0.526529,0.437764,0); rgb(81pt)=(0.533922,0.439512,0); rgb(82pt)=(0.54144,0.441232,0); rgb(83pt)=(0.549059,0.442927,0); rgb(84pt)=(0.556756,0.444599,0); rgb(85pt)=(0.564508,0.446252,0); rgb(86pt)=(0.572292,0.447889,0); rgb(87pt)=(0.580084,0.449514,0); rgb(88pt)=(0.587863,0.451129,0); rgb(89pt)=(0.595604,0.452737,0); rgb(90pt)=(0.603284,0.454343,0); rgb(91pt)=(0.610882,0.455948,0); rgb(92pt)=(0.618373,0.457556,0); rgb(93pt)=(0.625734,0.459171,0); rgb(94pt)=(0.632943,0.460795,0); rgb(95pt)=(0.639976,0.462432,0); rgb(96pt)=(0.64681,0.464084,0); rgb(97pt)=(0.653423,0.465756,0); rgb(98pt)=(0.659791,0.46745,0); rgb(99pt)=(0.665891,0.469169,0); rgb(100pt)=(0.6717,0.470916,0); rgb(101pt)=(0.677195,0.472696,0); rgb(102pt)=(0.682353,0.47451,0); rgb(103pt)=(0.687242,0.476355,0); rgb(104pt)=(0.691952,0.478225,0); rgb(105pt)=(0.696497,0.480118,0); rgb(106pt)=(0.700887,0.482033,0); rgb(107pt)=(0.705134,0.483968,0); rgb(108pt)=(0.709251,0.485921,0); rgb(109pt)=(0.713249,0.487891,0); rgb(110pt)=(0.71714,0.489876,0); rgb(111pt)=(0.720936,0.491875,0); rgb(112pt)=(0.724649,0.493887,0); rgb(113pt)=(0.72829,0.495909,0); rgb(114pt)=(0.731872,0.49794,0); rgb(115pt)=(0.735406,0.499979,0); rgb(116pt)=(0.738904,0.502025,0); rgb(117pt)=(0.742378,0.504075,0); rgb(118pt)=(0.74584,0.506128,0); rgb(119pt)=(0.749302,0.508182,0); rgb(120pt)=(0.752775,0.510237,0); rgb(121pt)=(0.756272,0.51229,0); rgb(122pt)=(0.759804,0.514339,0); rgb(123pt)=(0.763384,0.516385,0); rgb(124pt)=(0.767022,0.518424,0); rgb(125pt)=(0.770731,0.520455,0); rgb(126pt)=(0.774523,0.522478,0); rgb(127pt)=(0.77841,0.524489,0); rgb(128pt)=(0.782391,0.526491,0); rgb(129pt)=(0.786402,0.528496,0); rgb(130pt)=(0.790431,0.530506,0); rgb(131pt)=(0.794478,0.532521,0); rgb(132pt)=(0.798541,0.534539,0); rgb(133pt)=(0.802619,0.53656,0); rgb(134pt)=(0.806712,0.538584,0); rgb(135pt)=(0.81082,0.540609,0); rgb(136pt)=(0.81494,0.542635,0); rgb(137pt)=(0.819074,0.54466,0); rgb(138pt)=(0.823219,0.546686,0); rgb(139pt)=(0.827374,0.548709,0); rgb(140pt)=(0.831541,0.55073,0); rgb(141pt)=(0.835716,0.552749,0); rgb(142pt)=(0.8399,0.554763,0); rgb(143pt)=(0.844092,0.556774,0); rgb(144pt)=(0.848292,0.558779,0); rgb(145pt)=(0.852497,0.560778,0); rgb(146pt)=(0.856708,0.562771,0); rgb(147pt)=(0.860924,0.564756,0); rgb(148pt)=(0.865143,0.566733,0); rgb(149pt)=(0.869366,0.568701,0); rgb(150pt)=(0.873592,0.57066,0); rgb(151pt)=(0.877819,0.572608,0); rgb(152pt)=(0.882047,0.574545,0); rgb(153pt)=(0.886275,0.576471,0); rgb(154pt)=(0.890659,0.578362,0); rgb(155pt)=(0.895333,0.580203,0); rgb(156pt)=(0.900258,0.581999,0); rgb(157pt)=(0.905397,0.583755,0); rgb(158pt)=(0.910711,0.585479,0); rgb(159pt)=(0.916164,0.587176,0); rgb(160pt)=(0.921717,0.588852,0); rgb(161pt)=(0.927333,0.590513,0); rgb(162pt)=(0.932974,0.592166,0); rgb(163pt)=(0.938602,0.593815,0); rgb(164pt)=(0.94418,0.595468,0); rgb(165pt)=(0.949669,0.59713,0); rgb(166pt)=(0.955033,0.598808,0); rgb(167pt)=(0.960233,0.600507,0); rgb(168pt)=(0.965232,0.602233,0); rgb(169pt)=(0.969992,0.603992,0); rgb(170pt)=(0.974475,0.605791,0); rgb(171pt)=(0.978643,0.607636,0); rgb(172pt)=(0.98246,0.609532,0); rgb(173pt)=(0.985886,0.611486,0); rgb(174pt)=(0.988885,0.613503,0); rgb(175pt)=(0.991419,0.61559,0); rgb(176pt)=(0.99345,0.617753,0); rgb(177pt)=(0.99494,0.619997,0); rgb(178pt)=(0.995851,0.622329,0); rgb(179pt)=(0.996226,0.624763,0); rgb(180pt)=(0.996512,0.627352,0); rgb(181pt)=(0.996788,0.630095,0); rgb(182pt)=(0.997053,0.632982,0); rgb(183pt)=(0.997308,0.636004,0); rgb(184pt)=(0.997552,0.639152,0); rgb(185pt)=(0.997785,0.642416,0); rgb(186pt)=(0.998006,0.645786,0); rgb(187pt)=(0.998217,0.649253,0); rgb(188pt)=(0.998416,0.652807,0); rgb(189pt)=(0.998605,0.656439,0); rgb(190pt)=(0.998781,0.660138,0); rgb(191pt)=(0.998946,0.663897,0); rgb(192pt)=(0.9991,0.667704,0); rgb(193pt)=(0.999242,0.67155,0); rgb(194pt)=(0.999372,0.675427,0); rgb(195pt)=(0.99949,0.679323,0); rgb(196pt)=(0.999596,0.68323,0); rgb(197pt)=(0.99969,0.687139,0); rgb(198pt)=(0.999771,0.691039,0); rgb(199pt)=(0.999841,0.694921,0); rgb(200pt)=(0.999898,0.698775,0); rgb(201pt)=(0.999942,0.702592,0); rgb(202pt)=(0.999974,0.706363,0); rgb(203pt)=(0.999994,0.710077,0); rgb(204pt)=(1,0.713725,0); rgb(205pt)=(1,0.717341,0); rgb(206pt)=(1,0.720963,0); rgb(207pt)=(1,0.724591,0); rgb(208pt)=(1,0.728226,0); rgb(209pt)=(1,0.731867,0); rgb(210pt)=(1,0.735514,0); rgb(211pt)=(1,0.739167,0); rgb(212pt)=(1,0.742827,0); rgb(213pt)=(1,0.746493,0); rgb(214pt)=(1,0.750165,0); rgb(215pt)=(1,0.753843,0); rgb(216pt)=(1,0.757527,0); rgb(217pt)=(1,0.761217,0); rgb(218pt)=(1,0.764913,0); rgb(219pt)=(1,0.768615,0); rgb(220pt)=(1,0.772324,0); rgb(221pt)=(1,0.776038,0); rgb(222pt)=(1,0.779758,0); rgb(223pt)=(1,0.783484,0); rgb(224pt)=(1,0.787215,0); rgb(225pt)=(1,0.790953,0); rgb(226pt)=(1,0.794696,0); rgb(227pt)=(1,0.798445,0); rgb(228pt)=(1,0.8022,0); rgb(229pt)=(1,0.805961,0); rgb(230pt)=(1,0.809727,0); rgb(231pt)=(1,0.8135,0); rgb(232pt)=(1,0.817278,0); rgb(233pt)=(1,0.821063,0); rgb(234pt)=(1,0.824854,0); rgb(235pt)=(1,0.828652,0); rgb(236pt)=(1,0.832455,0); rgb(237pt)=(1,0.836265,0); rgb(238pt)=(1,0.840081,0); rgb(239pt)=(1,0.843903,0); rgb(240pt)=(1,0.847732,0); rgb(241pt)=(1,0.851566,0); rgb(242pt)=(1,0.855406,0); rgb(243pt)=(1,0.859253,0); rgb(244pt)=(1,0.863106,0); rgb(245pt)=(1,0.866964,0); rgb(246pt)=(1,0.870829,0); rgb(247pt)=(1,0.8747,0); rgb(248pt)=(1,0.878577,0); rgb(249pt)=(1,0.88246,0); rgb(250pt)=(1,0.886349,0); rgb(251pt)=(1,0.890243,0); rgb(252pt)=(1,0.894144,0); rgb(253pt)=(1,0.898051,0); rgb(254pt)=(1,0.901964,0); rgb(255pt)=(1,0.905882,0)},
mesh/rows=49]
table[row sep=crcr,header=false] {%
%
56	0.093	24.6312372359698\\
56	0.09666	26.8404315298396\\
56	0.10032	29.2751166615813\\
56	0.10398	31.9352926311949\\
56	0.10764	34.8209594386804\\
56	0.1113	37.9321170840378\\
56	0.11496	41.2687655672671\\
56	0.11862	44.8309048883682\\
56	0.12228	48.6185350473413\\
56	0.12594	52.6316560441862\\
56	0.1296	56.8702678789031\\
56	0.13326	61.3343705514918\\
56	0.13692	66.0239640619524\\
56	0.14058	70.939048410285\\
56	0.14424	76.0796235964894\\
56	0.1479	81.4456896205657\\
56	0.15156	87.0372464825139\\
56	0.15522	92.854294182334\\
56	0.15888	98.896832720026\\
56	0.16254	105.16486209559\\
56	0.1662	111.658382309026\\
56	0.16986	118.377393360333\\
56	0.17352	125.321895249513\\
56	0.17718	132.491887976564\\
56	0.18084	139.887371541487\\
56	0.1845	147.508345944283\\
56	0.18816	155.35481118495\\
56	0.19182	163.426767263489\\
56	0.19548	171.7242141799\\
56	0.19914	180.247151934182\\
56	0.2028	188.995580526337\\
56	0.20646	197.969499956364\\
56	0.21012	207.168910224262\\
56	0.21378	216.593811330032\\
56	0.21744	226.244203273675\\
56	0.2211	236.120086055189\\
56	0.22476	246.221459674575\\
56	0.22842	256.548324131833\\
56	0.23208	267.100679426962\\
56	0.23574	277.878525559964\\
56	0.2394	288.881862530838\\
56	0.24306	300.110690339583\\
56	0.24672	311.5650089862\\
56	0.25038	323.24481847069\\
56	0.25404	335.150118793051\\
56	0.2577	347.280909953284\\
56	0.26136	359.637191951389\\
56	0.26502	372.218964787366\\
56	0.26868	385.026228461214\\
56	0.27234	398.058982972935\\
56	0.276	411.317228322527\\
56.375	0.093	24.9524900466747\\
56.375	0.09666	27.169236976769\\
56.375	0.10032	29.6114747447352\\
56.375	0.10398	32.2792033505732\\
56.375	0.10764	35.1724227942832\\
56.375	0.1113	38.291133075865\\
56.375	0.11496	41.6353341953187\\
56.375	0.11862	45.2050261526443\\
56.375	0.12228	49.0002089478418\\
56.375	0.12594	53.0208825809112\\
56.375	0.1296	57.2670470518525\\
56.375	0.13326	61.7387023606656\\
56.375	0.13692	66.4358485073507\\
56.375	0.14058	71.3584854919077\\
56.375	0.14424	76.5066133143365\\
56.375	0.1479	81.8802319746373\\
56.375	0.15156	87.4793414728099\\
56.375	0.15522	93.3039418088544\\
56.375	0.15888	99.3540329827709\\
56.375	0.16254	105.629614994559\\
56.375	0.1662	112.130687844219\\
56.375	0.16986	118.857251531751\\
56.375	0.17352	125.809306057155\\
56.375	0.17718	132.986851420431\\
56.375	0.18084	140.389887621579\\
56.375	0.1845	148.018414660599\\
56.375	0.18816	155.87243253749\\
56.375	0.19182	163.951941252254\\
56.375	0.19548	172.256940804889\\
56.375	0.19914	180.787431195396\\
56.375	0.2028	189.543412423775\\
56.375	0.20646	198.524884490026\\
56.375	0.21012	207.731847394149\\
56.375	0.21378	217.164301136144\\
56.375	0.21744	226.822245716011\\
56.375	0.2211	236.705681133749\\
56.375	0.22476	246.81460738936\\
56.375	0.22842	257.149024482842\\
56.375	0.23208	267.708932414196\\
56.375	0.23574	278.494331183422\\
56.375	0.2394	289.50522079052\\
56.375	0.24306	300.74160123549\\
56.375	0.24672	312.203472518332\\
56.375	0.25038	323.890834639046\\
56.375	0.25404	335.803687597631\\
56.375	0.2577	347.942031394089\\
56.375	0.26136	360.305866028418\\
56.375	0.26502	372.895191500619\\
56.375	0.26868	385.710007810692\\
56.375	0.27234	398.750314958638\\
56.375	0.276	412.016112944454\\
56.75	0.093	25.27054205343\\
56.75	0.09666	27.4948416197487\\
56.75	0.10032	29.9446320239393\\
56.75	0.10398	32.6199132660018\\
56.75	0.10764	35.5206853459362\\
56.75	0.1113	38.6469482637424\\
56.75	0.11496	41.9987020194206\\
56.75	0.11862	45.5759466129706\\
56.75	0.12228	49.3786820443926\\
56.75	0.12594	53.4069083136864\\
56.75	0.1296	57.6606254208521\\
56.75	0.13326	62.1398333658897\\
56.75	0.13692	66.8445321487992\\
56.75	0.14058	71.7747217695807\\
56.75	0.14424	76.930402228234\\
56.75	0.1479	82.3115735247591\\
56.75	0.15156	87.9182356591562\\
56.75	0.15522	93.7503886314252\\
56.75	0.15888	99.8080324415661\\
56.75	0.16254	106.091167089579\\
56.75	0.1662	112.599792575463\\
56.75	0.16986	119.33390889922\\
56.75	0.17352	126.293516060848\\
56.75	0.17718	133.478614060349\\
56.75	0.18084	140.889202897721\\
56.75	0.1845	148.525282572965\\
56.75	0.18816	156.386853086081\\
56.75	0.19182	164.473914437069\\
56.75	0.19548	172.786466625929\\
56.75	0.19914	181.32450965266\\
56.75	0.2028	190.088043517264\\
56.75	0.20646	199.077068219739\\
56.75	0.21012	208.291583760087\\
56.75	0.21378	217.731590138306\\
56.75	0.21744	227.397087354397\\
56.75	0.2211	237.28807540836\\
56.75	0.22476	247.404554300195\\
56.75	0.22842	257.746524029901\\
56.75	0.23208	268.31398459748\\
56.75	0.23574	279.106936002931\\
56.75	0.2394	290.125378246253\\
56.75	0.24306	301.369311327448\\
56.75	0.24672	312.838735246514\\
56.75	0.25038	324.533650003452\\
56.75	0.25404	336.454055598262\\
56.75	0.2577	348.599952030944\\
56.75	0.26136	360.971339301498\\
56.75	0.26502	373.568217409923\\
56.75	0.26868	386.390586356221\\
56.75	0.27234	399.43844614039\\
56.75	0.276	412.711796762432\\
57.125	0.093	25.5853932562355\\
57.125	0.09666	27.8172454587787\\
57.125	0.10032	30.2745884991937\\
57.125	0.10398	32.9574223774806\\
57.125	0.10764	35.8657470936394\\
57.125	0.1113	38.9995626476702\\
57.125	0.11496	42.3588690395728\\
57.125	0.11862	45.9436662693472\\
57.125	0.12228	49.7539543369936\\
57.125	0.12594	53.7897332425119\\
57.125	0.1296	58.0510029859021\\
57.125	0.13326	62.5377635671641\\
57.125	0.13692	67.250014986298\\
57.125	0.14058	72.1877572433039\\
57.125	0.14424	77.3509903381817\\
57.125	0.1479	82.7397142709313\\
57.125	0.15156	88.3539290415528\\
57.125	0.15522	94.1936346500462\\
57.125	0.15888	100.258831096412\\
57.125	0.16254	106.549518380649\\
57.125	0.1662	113.065696502758\\
57.125	0.16986	119.807365462739\\
57.125	0.17352	126.774525260592\\
57.125	0.17718	133.967175896316\\
57.125	0.18084	141.385317369913\\
57.125	0.1845	149.028949681382\\
57.125	0.18816	156.898072830722\\
57.125	0.19182	164.992686817934\\
57.125	0.19548	173.312791643018\\
57.125	0.19914	181.858387305975\\
57.125	0.2028	190.629473806803\\
57.125	0.20646	199.626051145502\\
57.125	0.21012	208.848119322074\\
57.125	0.21378	218.295678336518\\
57.125	0.21744	227.968728188833\\
57.125	0.2211	237.867268879021\\
57.125	0.22476	247.99130040708\\
57.125	0.22842	258.340822773011\\
57.125	0.23208	268.915835976814\\
57.125	0.23574	279.71634001849\\
57.125	0.2394	290.742334898036\\
57.125	0.24306	301.993820615455\\
57.125	0.24672	313.470797170746\\
57.125	0.25038	325.173264563908\\
57.125	0.25404	337.101222794943\\
57.125	0.2577	349.254671863849\\
57.125	0.26136	361.633611770628\\
57.125	0.26502	374.238042515278\\
57.125	0.26868	387.0679640978\\
57.125	0.27234	400.123376518194\\
57.125	0.276	413.404279776459\\
57.5	0.093	25.8970436550913\\
57.5	0.09666	28.1364484938589\\
57.5	0.10032	30.6013441704984\\
57.5	0.10398	33.2917306850097\\
57.5	0.10764	36.207608037393\\
57.5	0.1113	39.3489762276482\\
57.5	0.11496	42.7158352557752\\
57.5	0.11862	46.3081851217741\\
57.5	0.12228	50.126025825645\\
57.5	0.12594	54.1693573673877\\
57.5	0.1296	58.4381797470023\\
57.5	0.13326	62.9324929644888\\
57.5	0.13692	67.6522970198471\\
57.5	0.14058	72.5975919130775\\
57.5	0.14424	77.7683776441797\\
57.5	0.1479	83.1646542131537\\
57.5	0.15156	88.7864216199996\\
57.5	0.15522	94.6336798647175\\
57.5	0.15888	100.706428947307\\
57.5	0.16254	107.004668867769\\
57.5	0.1662	113.528399626102\\
57.5	0.16986	120.277621222308\\
57.5	0.17352	127.252333656385\\
57.5	0.17718	134.452536928334\\
57.5	0.18084	141.878231038155\\
57.5	0.1845	149.529415985848\\
57.5	0.18816	157.406091771413\\
57.5	0.19182	165.50825839485\\
57.5	0.19548	173.835915856159\\
57.5	0.19914	182.389064155339\\
57.5	0.2028	191.167703292392\\
57.5	0.20646	200.171833267316\\
57.5	0.21012	209.401454080112\\
57.5	0.21378	218.85656573078\\
57.5	0.21744	228.53716821932\\
57.5	0.2211	238.443261545732\\
57.5	0.22476	248.574845710016\\
57.5	0.22842	258.931920712172\\
57.5	0.23208	269.514486552199\\
57.5	0.23574	280.322543230099\\
57.5	0.2394	291.35609074587\\
57.5	0.24306	302.615129099513\\
57.5	0.24672	314.099658291028\\
57.5	0.25038	325.809678320415\\
57.5	0.25404	337.745189187674\\
57.5	0.2577	349.906190892805\\
57.5	0.26136	362.292683435808\\
57.5	0.26502	374.904666816682\\
57.5	0.26868	387.742141035429\\
57.5	0.27234	400.805106092047\\
57.5	0.276	414.093561986537\\
57.875	0.093	26.2054932499974\\
57.875	0.09666	28.4524507249894\\
57.875	0.10032	30.9248990378533\\
57.875	0.10398	33.6228381885891\\
57.875	0.10764	36.5462681771969\\
57.875	0.1113	39.6951890036764\\
57.875	0.11496	43.069600668028\\
57.875	0.11862	46.6695031702513\\
57.875	0.12228	50.4948965103466\\
57.875	0.12594	54.5457806883137\\
57.875	0.1296	58.8221557041528\\
57.875	0.13326	63.3240215578637\\
57.875	0.13692	68.0513782494465\\
57.875	0.14058	73.0042257789013\\
57.875	0.14424	78.1825641462279\\
57.875	0.1479	83.5863933514264\\
57.875	0.15156	89.2157133944968\\
57.875	0.15522	95.0705242754391\\
57.875	0.15888	101.150825994253\\
57.875	0.16254	107.456618550939\\
57.875	0.1662	113.987901945497\\
57.875	0.16986	120.744676177927\\
57.875	0.17352	127.726941248229\\
57.875	0.17718	134.934697156403\\
57.875	0.18084	142.367943902448\\
57.875	0.1845	150.026681486365\\
57.875	0.18816	157.910909908155\\
57.875	0.19182	166.020629167816\\
57.875	0.19548	174.355839265349\\
57.875	0.19914	182.916540200754\\
57.875	0.2028	191.702731974031\\
57.875	0.20646	200.71441458518\\
57.875	0.21012	209.9515880342\\
57.875	0.21378	219.414252321093\\
57.875	0.21744	229.102407445857\\
57.875	0.2211	239.016053408494\\
57.875	0.22476	249.155190209002\\
57.875	0.22842	259.519817847382\\
57.875	0.23208	270.109936323634\\
57.875	0.23574	280.925545637758\\
57.875	0.2394	291.966645789754\\
57.875	0.24306	303.233236779621\\
57.875	0.24672	314.725318607361\\
57.875	0.25038	326.442891272972\\
57.875	0.25404	338.385954776456\\
57.875	0.2577	350.554509117811\\
57.875	0.26136	362.948554297038\\
57.875	0.26502	375.568090314137\\
57.875	0.26868	388.413117169108\\
57.875	0.27234	401.483634861951\\
57.875	0.276	414.779643392665\\
58.25	0.093	26.5107420409537\\
58.25	0.09666	28.7652521521702\\
58.25	0.10032	31.2452531012586\\
58.25	0.10398	33.9507448882188\\
58.25	0.10764	36.881727513051\\
58.25	0.1113	40.038200975755\\
58.25	0.11496	43.420165276331\\
58.25	0.11862	47.0276204147788\\
58.25	0.12228	50.8605663910985\\
58.25	0.12594	54.9190032052901\\
58.25	0.1296	59.2029308573536\\
58.25	0.13326	63.7123493472889\\
58.25	0.13692	68.4472586750962\\
58.25	0.14058	73.4076588407754\\
58.25	0.14424	78.5935498443265\\
58.25	0.1479	84.0049316857494\\
58.25	0.15156	89.6418043650442\\
58.25	0.15522	95.504167882211\\
58.25	0.15888	101.59202223725\\
58.25	0.16254	107.90536743016\\
58.25	0.1662	114.444203460943\\
58.25	0.16986	121.208530329597\\
58.25	0.17352	128.198348036123\\
58.25	0.17718	135.413656580521\\
58.25	0.18084	142.854455962791\\
58.25	0.1845	150.520746182933\\
58.25	0.18816	158.412527240947\\
58.25	0.19182	166.529799136832\\
58.25	0.19548	174.87256187059\\
58.25	0.19914	183.440815442219\\
58.25	0.2028	192.234559851721\\
58.25	0.20646	201.253795099094\\
58.25	0.21012	210.498521184339\\
58.25	0.21378	219.968738107456\\
58.25	0.21744	229.664445868445\\
58.25	0.2211	239.585644467305\\
58.25	0.22476	249.732333904038\\
58.25	0.22842	260.104514178643\\
58.25	0.23208	270.702185291119\\
58.25	0.23574	281.525347241467\\
58.25	0.2394	292.574000029688\\
58.25	0.24306	303.84814365578\\
58.25	0.24672	315.347778119744\\
58.25	0.25038	327.07290342158\\
58.25	0.25404	339.023519561287\\
58.25	0.2577	351.199626538867\\
58.25	0.26136	363.601224354319\\
58.25	0.26502	376.228313007642\\
58.25	0.26868	389.080892498837\\
58.25	0.27234	402.158962827905\\
58.25	0.276	415.462523994844\\
58.625	0.093	26.8127900279604\\
58.625	0.09666	29.0748527754013\\
58.625	0.10032	31.5624063607141\\
58.625	0.10398	34.2754507838988\\
58.625	0.10764	37.2139860449554\\
58.625	0.1113	40.3780121438839\\
58.625	0.11496	43.7675290806843\\
58.625	0.11862	47.3825368553565\\
58.625	0.12228	51.2230354679007\\
58.625	0.12594	55.2890249183167\\
58.625	0.1296	59.5805052066046\\
58.625	0.13326	64.0974763327644\\
58.625	0.13692	68.8399382967961\\
58.625	0.14058	73.8078910986998\\
58.625	0.14424	79.0013347384753\\
58.625	0.1479	84.4202692161227\\
58.625	0.15156	90.0646945316419\\
58.625	0.15522	95.9346106850331\\
58.625	0.15888	102.030017676296\\
58.625	0.16254	108.350915505431\\
58.625	0.1662	114.897304172438\\
58.625	0.16986	121.669183677317\\
58.625	0.17352	128.666554020067\\
58.625	0.17718	135.88941520069\\
58.625	0.18084	143.337767219184\\
58.625	0.1845	151.011610075551\\
58.625	0.18816	158.910943769789\\
58.625	0.19182	167.035768301899\\
58.625	0.19548	175.386083671881\\
58.625	0.19914	183.961889879735\\
58.625	0.2028	192.76318692546\\
58.625	0.20646	201.789974809058\\
58.625	0.21012	211.042253530528\\
58.625	0.21378	220.520023089869\\
58.625	0.21744	230.223283487082\\
58.625	0.2211	240.152034722168\\
58.625	0.22476	250.306276795125\\
58.625	0.22842	260.686009705954\\
58.625	0.23208	271.291233454654\\
58.625	0.23574	282.121948041227\\
58.625	0.2394	293.178153465672\\
58.625	0.24306	304.459849727989\\
58.625	0.24672	315.967036828177\\
58.625	0.25038	327.699714766237\\
58.625	0.25404	339.65788354217\\
58.625	0.2577	351.841543155974\\
58.625	0.26136	364.25069360765\\
58.625	0.26502	376.885334897198\\
58.625	0.26868	389.745467024617\\
58.625	0.27234	402.831089989909\\
58.625	0.276	416.142203793072\\
59	0.093	27.1116372110173\\
59	0.09666	29.3812525946827\\
59	0.10032	31.8763588162199\\
59	0.10398	34.596955875629\\
59	0.10764	37.5430437729101\\
59	0.1113	40.714622508063\\
59	0.11496	44.1116920810878\\
59	0.11862	47.7342524919845\\
59	0.12228	51.5823037407531\\
59	0.12594	55.6558458273936\\
59	0.1296	59.954878751906\\
59	0.13326	64.4794025142902\\
59	0.13692	69.2294171145463\\
59	0.14058	74.2049225526744\\
59	0.14424	79.4059188286744\\
59	0.1479	84.8324059425462\\
59	0.15156	90.4843838942899\\
59	0.15522	96.3618526839055\\
59	0.15888	102.464812311393\\
59	0.16254	108.793262776752\\
59	0.1662	115.347204079984\\
59	0.16986	122.126636221087\\
59	0.17352	129.131559200062\\
59	0.17718	136.361973016909\\
59	0.18084	143.817877671628\\
59	0.1845	151.499273164218\\
59	0.18816	159.406159494681\\
59	0.19182	167.538536663016\\
59	0.19548	175.896404669222\\
59	0.19914	184.4797635133\\
59	0.2028	193.288613195251\\
59	0.20646	202.322953715073\\
59	0.21012	211.582785072767\\
59	0.21378	221.068107268332\\
59	0.21744	230.77892030177\\
59	0.2211	240.71522417308\\
59	0.22476	250.877018882261\\
59	0.22842	261.264304429315\\
59	0.23208	271.87708081424\\
59	0.23574	282.715348037037\\
59	0.2394	293.779106097707\\
59	0.24306	305.068354996248\\
59	0.24672	316.58309473266\\
59	0.25038	328.323325306945\\
59	0.25404	340.289046719102\\
59	0.2577	352.48025896913\\
59	0.26136	364.896962057031\\
59	0.26502	377.539155982803\\
59	0.26868	390.406840746447\\
59	0.27234	403.500016347964\\
59	0.276	416.818682787351\\
59.375	0.093	27.4072835901245\\
59.375	0.09666	29.6844516100143\\
59.375	0.10032	32.187110467776\\
59.375	0.10398	34.9152601634096\\
59.375	0.10764	37.8689006969151\\
59.375	0.1113	41.0480320682924\\
59.375	0.11496	44.4526542775417\\
59.375	0.11862	48.0827673246628\\
59.375	0.12228	51.9383712096558\\
59.375	0.12594	56.0194659325208\\
59.375	0.1296	60.3260514932576\\
59.375	0.13326	64.8581278918663\\
59.375	0.13692	69.6156951283468\\
59.375	0.14058	74.5987532026994\\
59.375	0.14424	79.8073021149238\\
59.375	0.1479	85.24134186502\\
59.375	0.15156	90.9008724529882\\
59.375	0.15522	96.7858938788283\\
59.375	0.15888	102.89640614254\\
59.375	0.16254	109.232409244124\\
59.375	0.1662	115.79390318358\\
59.375	0.16986	122.580887960907\\
59.375	0.17352	129.593363576107\\
59.375	0.17718	136.831330029178\\
59.375	0.18084	144.294787320122\\
59.375	0.1845	151.983735448937\\
59.375	0.18816	159.898174415624\\
59.375	0.19182	168.038104220183\\
59.375	0.19548	176.403524862614\\
59.375	0.19914	184.994436342916\\
59.375	0.2028	193.810838661091\\
59.375	0.20646	202.852731817138\\
59.375	0.21012	212.120115811056\\
59.375	0.21378	221.612990642846\\
59.375	0.21744	231.331356312509\\
59.375	0.2211	241.275212820043\\
59.375	0.22476	251.444560165449\\
59.375	0.22842	261.839398348726\\
59.375	0.23208	272.459727369876\\
59.375	0.23574	283.305547228898\\
59.375	0.2394	294.376857925791\\
59.375	0.24306	305.673659460557\\
59.375	0.24672	317.195951833194\\
59.375	0.25038	328.943735043703\\
59.375	0.25404	340.917009092085\\
59.375	0.2577	353.115773978338\\
59.375	0.26136	365.540029702462\\
59.375	0.26502	378.189776264459\\
59.375	0.26868	391.065013664328\\
59.375	0.27234	404.165741902068\\
59.375	0.276	417.491960977681\\
59.75	0.093	27.699729165282\\
59.75	0.09666	29.9844498213963\\
59.75	0.10032	32.4946613153824\\
59.75	0.10398	35.2303636472404\\
59.75	0.10764	38.1915568169703\\
59.75	0.1113	41.3782408245721\\
59.75	0.11496	44.7904156700458\\
59.75	0.11862	48.4280813533914\\
59.75	0.12228	52.2912378746089\\
59.75	0.12594	56.3798852336982\\
59.75	0.1296	60.6940234306595\\
59.75	0.13326	65.2336524654926\\
59.75	0.13692	69.9987723381976\\
59.75	0.14058	74.9893830487746\\
59.75	0.14424	80.2054845972235\\
59.75	0.1479	85.6470769835442\\
59.75	0.15156	91.3141602077367\\
59.75	0.15522	97.2067342698013\\
59.75	0.15888	103.324799169738\\
59.75	0.16254	109.668354907546\\
59.75	0.1662	116.237401483226\\
59.75	0.16986	123.031938896778\\
59.75	0.17352	130.051967148202\\
59.75	0.17718	137.297486237498\\
59.75	0.18084	144.768496164666\\
59.75	0.1845	152.464996929705\\
59.75	0.18816	160.386988532617\\
59.75	0.19182	168.5344709734\\
59.75	0.19548	176.907444252056\\
59.75	0.19914	185.505908368583\\
59.75	0.2028	194.329863322982\\
59.75	0.20646	203.379309115253\\
59.75	0.21012	212.654245745396\\
59.75	0.21378	222.15467321341\\
59.75	0.21744	231.880591519297\\
59.75	0.2211	241.832000663056\\
59.75	0.22476	252.008900644686\\
59.75	0.22842	262.411291464188\\
59.75	0.23208	273.039173121562\\
59.75	0.23574	283.892545616809\\
59.75	0.2394	294.971408949927\\
59.75	0.24306	306.275763120916\\
59.75	0.24672	317.805608129778\\
59.75	0.25038	329.560943976512\\
59.75	0.25404	341.541770661117\\
59.75	0.2577	353.748088183595\\
59.75	0.26136	366.179896543944\\
59.75	0.26502	378.837195742165\\
59.75	0.26868	391.719985778258\\
59.75	0.27234	404.828266652224\\
59.75	0.276	418.16203836406\\
60.125	0.093	27.9889739364898\\
60.125	0.09666	30.2812472288284\\
60.125	0.10032	32.799011359039\\
60.125	0.10398	35.5422663271215\\
60.125	0.10764	38.5110121330758\\
60.125	0.1113	41.7052487769021\\
60.125	0.11496	45.1249762586002\\
60.125	0.11862	48.7701945781702\\
60.125	0.12228	52.6409037356122\\
60.125	0.12594	56.737103730926\\
60.125	0.1296	61.0587945641117\\
60.125	0.13326	65.6059762351693\\
60.125	0.13692	70.3786487440987\\
60.125	0.14058	75.3768120909001\\
60.125	0.14424	80.6004662755734\\
60.125	0.1479	86.0496112981185\\
60.125	0.15156	91.7242471585356\\
60.125	0.15522	97.6243738568245\\
60.125	0.15888	103.749991392985\\
60.125	0.16254	110.101099767018\\
60.125	0.1662	116.677698978923\\
60.125	0.16986	123.479789028699\\
60.125	0.17352	130.507369916348\\
60.125	0.17718	137.760441641868\\
60.125	0.18084	145.23900420526\\
60.125	0.1845	152.943057606524\\
60.125	0.18816	160.87260184566\\
60.125	0.19182	169.027636922668\\
60.125	0.19548	177.408162837548\\
60.125	0.19914	186.014179590299\\
60.125	0.2028	194.845687180923\\
60.125	0.20646	203.902685609418\\
60.125	0.21012	213.185174875786\\
60.125	0.21378	222.693154980025\\
60.125	0.21744	232.426625922136\\
60.125	0.2211	242.385587702119\\
60.125	0.22476	252.570040319974\\
60.125	0.22842	262.9799837757\\
60.125	0.23208	273.615418069299\\
60.125	0.23574	284.47634320077\\
60.125	0.2394	295.562759170112\\
60.125	0.24306	306.874665977326\\
60.125	0.24672	318.412063622412\\
60.125	0.25038	330.174952105371\\
60.125	0.25404	342.163331426201\\
60.125	0.2577	354.377201584903\\
60.125	0.26136	366.816562581476\\
60.125	0.26502	379.481414415922\\
60.125	0.26868	392.371757088239\\
60.125	0.27234	405.487590598429\\
60.125	0.276	418.82891494649\\
60.5	0.093	28.2750179037478\\
60.5	0.09666	30.5748438323109\\
60.5	0.10032	33.100160598746\\
60.5	0.10398	35.8509682030528\\
60.5	0.10764	38.8272666452317\\
60.5	0.1113	42.0290559252823\\
60.5	0.11496	45.4563360432049\\
60.5	0.11862	49.1091069989994\\
60.5	0.12228	52.9873687926657\\
60.5	0.12594	57.091121424204\\
60.5	0.1296	61.4203648936141\\
60.5	0.13326	65.9750992008962\\
60.5	0.13692	70.75532434605\\
60.5	0.14058	75.7610403290759\\
60.5	0.14424	80.9922471499737\\
60.5	0.1479	86.4489448087432\\
60.5	0.15156	92.1311333053847\\
60.5	0.15522	98.0388126398981\\
60.5	0.15888	104.171982812283\\
60.5	0.16254	110.530643822541\\
60.5	0.1662	117.11479567067\\
60.5	0.16986	123.924438356671\\
60.5	0.17352	130.959571880543\\
60.5	0.17718	138.220196242288\\
60.5	0.18084	145.706311441905\\
60.5	0.1845	153.417917479393\\
60.5	0.18816	161.355014354754\\
60.5	0.19182	169.517602067986\\
60.5	0.19548	177.90568061909\\
60.5	0.19914	186.519250008066\\
60.5	0.2028	195.358310234914\\
60.5	0.20646	204.422861299634\\
60.5	0.21012	213.712903202226\\
60.5	0.21378	223.228435942689\\
60.5	0.21744	232.969459521025\\
60.5	0.2211	242.935973937232\\
60.5	0.22476	253.127979191312\\
60.5	0.22842	263.545475283263\\
60.5	0.23208	274.188462213086\\
60.5	0.23574	285.056939980781\\
60.5	0.2394	296.150908586348\\
60.5	0.24306	307.470368029786\\
60.5	0.24672	319.015318311097\\
60.5	0.25038	330.78575943028\\
60.5	0.25404	342.781691387334\\
60.5	0.2577	355.003114182261\\
60.5	0.26136	367.450027815059\\
60.5	0.26502	380.122432285729\\
60.5	0.26868	393.020327594271\\
60.5	0.27234	406.143713740685\\
60.5	0.276	419.49259072497\\
60.875	0.093	28.5578610670561\\
60.875	0.09666	30.8652396318437\\
60.875	0.10032	33.3981090345032\\
60.875	0.10398	36.1564692750345\\
60.875	0.10764	39.1403203534378\\
60.875	0.1113	42.3496622697129\\
60.875	0.11496	45.7844950238599\\
60.875	0.11862	49.4448186158788\\
60.875	0.12228	53.3306330457696\\
60.875	0.12594	57.4419383135323\\
60.875	0.1296	61.7787344191669\\
60.875	0.13326	66.3410213626733\\
60.875	0.13692	71.1287991440517\\
60.875	0.14058	76.142067763302\\
60.875	0.14424	81.3808272204242\\
60.875	0.1479	86.8450775154182\\
60.875	0.15156	92.5348186482841\\
60.875	0.15522	98.4500506190219\\
60.875	0.15888	104.590773427632\\
60.875	0.16254	110.956987074113\\
60.875	0.1662	117.548691558467\\
60.875	0.16986	124.365886880692\\
60.875	0.17352	131.408573040789\\
60.875	0.17718	138.676750038759\\
60.875	0.18084	146.1704178746\\
60.875	0.1845	153.889576548313\\
60.875	0.18816	161.834226059897\\
60.875	0.19182	170.004366409354\\
60.875	0.19548	178.399997596683\\
60.875	0.19914	187.021119621883\\
60.875	0.2028	195.867732484956\\
60.875	0.20646	204.9398361859\\
60.875	0.21012	214.237430724716\\
60.875	0.21378	223.760516101404\\
60.875	0.21744	233.509092315964\\
60.875	0.2211	243.483159368396\\
60.875	0.22476	253.6827172587\\
60.875	0.22842	264.107765986875\\
60.875	0.23208	274.758305552923\\
60.875	0.23574	285.634335956842\\
60.875	0.2394	296.735857198634\\
60.875	0.24306	308.062869278297\\
60.875	0.24672	319.615372195832\\
60.875	0.25038	331.393365951239\\
60.875	0.25404	343.396850544518\\
60.875	0.2577	355.625825975669\\
60.875	0.26136	368.080292244691\\
60.875	0.26502	380.760249351586\\
60.875	0.26868	393.665697296352\\
60.875	0.27234	406.796636078991\\
60.875	0.276	420.153065699501\\
61.25	0.093	28.8375034264148\\
61.25	0.09666	31.1524346274268\\
61.25	0.10032	33.6928566663107\\
61.25	0.10398	36.4587695430664\\
61.25	0.10764	39.4501732576941\\
61.25	0.1113	42.6670678101937\\
61.25	0.11496	46.1094532005652\\
61.25	0.11862	49.7773294288085\\
61.25	0.12228	53.6706964949238\\
61.25	0.12594	57.7895543989109\\
61.25	0.1296	62.1339031407699\\
61.25	0.13326	66.7037427205008\\
61.25	0.13692	71.4990731381036\\
61.25	0.14058	76.5198943935783\\
61.25	0.14424	81.7662064869249\\
61.25	0.1479	87.2380094181434\\
61.25	0.15156	92.9353031872338\\
61.25	0.15522	98.8580877941961\\
61.25	0.15888	105.00636323903\\
61.25	0.16254	111.380129521736\\
61.25	0.1662	117.979386642314\\
61.25	0.16986	124.804134600764\\
61.25	0.17352	131.854373397086\\
61.25	0.17718	139.130103031279\\
61.25	0.18084	146.631323503345\\
61.25	0.1845	154.358034813282\\
61.25	0.18816	162.310236961092\\
61.25	0.19182	170.487929946773\\
61.25	0.19548	178.891113770326\\
61.25	0.19914	187.519788431751\\
61.25	0.2028	196.373953931048\\
61.25	0.20646	205.453610268216\\
61.25	0.21012	214.758757443257\\
61.25	0.21378	224.289395456169\\
61.25	0.21744	234.045524306954\\
61.25	0.2211	244.02714399561\\
61.25	0.22476	254.234254522138\\
61.25	0.22842	264.666855886538\\
61.25	0.23208	275.32494808881\\
61.25	0.23574	286.208531128954\\
61.25	0.2394	297.31760500697\\
61.25	0.24306	308.652169722858\\
61.25	0.24672	320.212225276617\\
61.25	0.25038	331.997771668249\\
61.25	0.25404	344.008808897752\\
61.25	0.2577	356.245336965127\\
61.25	0.26136	368.707355870374\\
61.25	0.26502	381.394865613493\\
61.25	0.26868	394.307866194484\\
61.25	0.27234	407.446357613347\\
61.25	0.276	420.810339870081\\
61.625	0.093	29.1139449818236\\
61.625	0.09666	31.4364288190601\\
61.625	0.10032	33.9844034941684\\
61.625	0.10398	36.7578690071486\\
61.625	0.10764	39.7568253580008\\
61.625	0.1113	42.9812725467248\\
61.625	0.11496	46.4312105733207\\
61.625	0.11862	50.1066394377885\\
61.625	0.12228	54.0075591401282\\
61.625	0.12594	58.1339696803397\\
61.625	0.1296	62.4858710584232\\
61.625	0.13326	67.0632632743786\\
61.625	0.13692	71.8661463282058\\
61.625	0.14058	76.894520219905\\
61.625	0.14424	82.148384949476\\
61.625	0.1479	87.6277405169189\\
61.625	0.15156	93.3325869222337\\
61.625	0.15522	99.2629241654205\\
61.625	0.15888	105.418752246479\\
61.625	0.16254	111.80007116541\\
61.625	0.1662	118.406880922212\\
61.625	0.16986	125.239181516886\\
61.625	0.17352	132.296972949432\\
61.625	0.17718	139.58025521985\\
61.625	0.18084	147.08902832814\\
61.625	0.1845	154.823292274302\\
61.625	0.18816	162.783047058336\\
61.625	0.19182	170.968292680242\\
61.625	0.19548	179.379029140019\\
61.625	0.19914	188.015256437668\\
61.625	0.2028	196.87697457319\\
61.625	0.20646	205.964183546583\\
61.625	0.21012	215.276883357848\\
61.625	0.21378	224.815074006985\\
61.625	0.21744	234.578755493994\\
61.625	0.2211	244.567927818874\\
61.625	0.22476	254.782590981627\\
61.625	0.22842	265.222744982252\\
61.625	0.23208	275.888389820748\\
61.625	0.23574	286.779525497116\\
61.625	0.2394	297.896152011357\\
61.625	0.24306	309.238269363469\\
61.625	0.24672	320.805877553452\\
61.625	0.25038	332.598976581308\\
61.625	0.25404	344.617566447036\\
61.625	0.2577	356.861647150636\\
61.625	0.26136	369.331218692107\\
61.625	0.26502	382.026281071451\\
61.625	0.26868	394.946834288666\\
61.625	0.27234	408.092878343753\\
61.625	0.276	421.464413236712\\
62	0.093	29.3871857332828\\
62	0.09666	31.7172222067437\\
62	0.10032	34.2727495180765\\
62	0.10398	37.0537676672812\\
62	0.10764	40.0602766543577\\
62	0.1113	43.2922764793062\\
62	0.11496	46.7497671421265\\
62	0.11862	50.4327486428188\\
62	0.12228	54.3412209813829\\
62	0.12594	58.4751841578189\\
62	0.1296	62.8346381721268\\
62	0.13326	67.4195830243066\\
62	0.13692	72.2300187143583\\
62	0.14058	77.2659452422819\\
62	0.14424	82.5273626080774\\
62	0.1479	88.0142708117447\\
62	0.15156	93.726669853284\\
62	0.15522	99.6645597326951\\
62	0.15888	105.827940449978\\
62	0.16254	112.216812005133\\
62	0.1662	118.83117439816\\
62	0.16986	125.671027629059\\
62	0.17352	132.736371697829\\
62	0.17718	140.027206604472\\
62	0.18084	147.543532348986\\
62	0.1845	155.285348931372\\
62	0.18816	163.252656351631\\
62	0.19182	171.445454609761\\
62	0.19548	179.863743705763\\
62	0.19914	188.507523639636\\
62	0.2028	197.376794411382\\
62	0.20646	206.471556021\\
62	0.21012	215.791808468489\\
62	0.21378	225.337551753851\\
62	0.21744	235.108785877084\\
62	0.2211	245.105510838189\\
62	0.22476	255.327726637166\\
62	0.22842	265.775433274015\\
62	0.23208	276.448630748736\\
62	0.23574	287.347319061329\\
62	0.2394	298.471498211793\\
62	0.24306	309.82116820013\\
62	0.24672	321.396329026338\\
62	0.25038	333.196980690419\\
62	0.25404	345.223123192371\\
62	0.2577	357.474756532195\\
62	0.26136	369.951880709891\\
62	0.26502	382.654495725459\\
62	0.26868	395.582601578898\\
62	0.27234	408.73619827021\\
62	0.276	422.115285799394\\
62.375	0.093	29.6572256807923\\
62.375	0.09666	31.9948147904776\\
62.375	0.10032	34.5578947380348\\
62.375	0.10398	37.3464655234639\\
62.375	0.10764	40.360527146765\\
62.375	0.1113	43.6000796079378\\
62.375	0.11496	47.0651229069827\\
62.375	0.11862	50.7556570438993\\
62.375	0.12228	54.6716820186879\\
62.375	0.12594	58.8131978313483\\
62.375	0.1296	63.1802044818807\\
62.375	0.13326	67.7727019702849\\
62.375	0.13692	72.590690296561\\
62.375	0.14058	77.6341694607091\\
62.375	0.14424	82.903139462729\\
62.375	0.1479	88.3976003026208\\
62.375	0.15156	94.1175519803845\\
62.375	0.15522	100.06299449602\\
62.375	0.15888	106.233927849528\\
62.375	0.16254	112.630352040907\\
62.375	0.1662	119.252267070158\\
62.375	0.16986	126.099672937281\\
62.375	0.17352	133.172569642276\\
62.375	0.17718	140.470957185143\\
62.375	0.18084	147.994835565882\\
62.375	0.1845	155.744204784493\\
62.375	0.18816	163.719064840976\\
62.375	0.19182	171.91941573533\\
62.375	0.19548	180.345257467556\\
62.375	0.19914	188.996590037655\\
62.375	0.2028	197.873413445625\\
62.375	0.20646	206.975727691467\\
62.375	0.21012	216.303532775181\\
62.375	0.21378	225.856828696767\\
62.375	0.21744	235.635615456224\\
62.375	0.2211	245.639893053554\\
62.375	0.22476	255.869661488756\\
62.375	0.22842	266.324920761829\\
62.375	0.23208	277.005670872774\\
62.375	0.23574	287.911911821591\\
62.375	0.2394	299.04364360828\\
62.375	0.24306	310.400866232842\\
62.375	0.24672	321.983579695274\\
62.375	0.25038	333.791783995579\\
62.375	0.25404	345.825479133756\\
62.375	0.2577	358.084665109804\\
62.375	0.26136	370.569341923725\\
62.375	0.26502	383.279509575517\\
62.375	0.26868	396.215168065181\\
62.375	0.27234	409.376317392717\\
62.375	0.276	422.762957558125\\
62.75	0.093	29.924064824352\\
62.75	0.09666	32.2692065702618\\
62.75	0.10032	34.8398391540435\\
62.75	0.10398	37.635962575697\\
62.75	0.10764	40.6575768352225\\
62.75	0.1113	43.9046819326198\\
62.75	0.11496	47.377277867889\\
62.75	0.11862	51.0753646410301\\
62.75	0.12228	54.9989422520432\\
62.75	0.12594	59.148010700928\\
62.75	0.1296	63.5225699876848\\
62.75	0.13326	68.1226201123135\\
62.75	0.13692	72.948161074814\\
62.75	0.14058	77.9991928751866\\
62.75	0.14424	83.2757155134309\\
62.75	0.1479	88.7777289895472\\
62.75	0.15156	94.5052333035353\\
62.75	0.15522	100.458228455395\\
62.75	0.15888	106.636714445127\\
62.75	0.16254	113.040691272731\\
62.75	0.1662	119.670158938207\\
62.75	0.16986	126.525117441554\\
62.75	0.17352	133.605566782774\\
62.75	0.17718	140.911506961865\\
62.75	0.18084	148.442937978828\\
62.75	0.1845	156.199859833664\\
62.75	0.18816	164.182272526371\\
62.75	0.19182	172.39017605695\\
62.75	0.19548	180.8235704254\\
62.75	0.19914	189.482455631723\\
62.75	0.2028	198.366831675918\\
62.75	0.20646	207.476698557984\\
62.75	0.21012	216.812056277923\\
62.75	0.21378	226.372904835733\\
62.75	0.21744	236.159244231415\\
62.75	0.2211	246.171074464969\\
62.75	0.22476	256.408395536395\\
62.75	0.22842	266.871207445693\\
62.75	0.23208	277.559510192863\\
62.75	0.23574	288.473303777904\\
62.75	0.2394	299.612588200818\\
62.75	0.24306	310.977363461603\\
62.75	0.24672	322.567629560261\\
62.75	0.25038	334.38338649679\\
62.75	0.25404	346.424634271191\\
62.75	0.2577	358.691372883464\\
62.75	0.26136	371.183602333609\\
62.75	0.26502	383.901322621625\\
62.75	0.26868	396.844533747514\\
62.75	0.27234	410.013235711275\\
62.75	0.276	423.407428512907\\
63.125	0.093	30.187703163962\\
63.125	0.09666	32.5403975460963\\
63.125	0.10032	35.1185827661024\\
63.125	0.10398	37.9222588239803\\
63.125	0.10764	40.9514257197302\\
63.125	0.1113	44.206083453352\\
63.125	0.11496	47.6862320248457\\
63.125	0.11862	51.3918714342112\\
63.125	0.12228	55.3230016814487\\
63.125	0.12594	59.479622766558\\
63.125	0.1296	63.8617346895393\\
63.125	0.13326	68.4693374503924\\
63.125	0.13692	73.3024310491174\\
63.125	0.14058	78.3610154857143\\
63.125	0.14424	83.6450907601831\\
63.125	0.1479	89.1546568725238\\
63.125	0.15156	94.8897138227364\\
63.125	0.15522	100.850261610821\\
63.125	0.15888	107.036300236777\\
63.125	0.16254	113.447829700605\\
63.125	0.1662	120.084850002306\\
63.125	0.16986	126.947361141878\\
63.125	0.17352	134.035363119322\\
63.125	0.17718	141.348855934637\\
63.125	0.18084	148.887839587825\\
63.125	0.1845	156.652314078885\\
63.125	0.18816	164.642279407816\\
63.125	0.19182	172.85773557462\\
63.125	0.19548	181.298682579295\\
63.125	0.19914	189.965120421842\\
63.125	0.2028	198.857049102261\\
63.125	0.20646	207.974468620552\\
63.125	0.21012	217.317378976715\\
63.125	0.21378	226.88578017075\\
63.125	0.21744	236.679672202656\\
63.125	0.2211	246.699055072435\\
63.125	0.22476	256.943928780085\\
63.125	0.22842	267.414293325607\\
63.125	0.23208	278.110148709001\\
63.125	0.23574	289.031494930268\\
63.125	0.2394	300.178331989406\\
63.125	0.24306	311.550659886415\\
63.125	0.24672	323.148478621297\\
63.125	0.25038	334.971788194051\\
63.125	0.25404	347.020588604676\\
63.125	0.2577	359.294879853174\\
63.125	0.26136	371.794661939543\\
63.125	0.26502	384.519934863784\\
63.125	0.26868	397.470698625897\\
63.125	0.27234	410.646953225882\\
63.125	0.276	424.048698663739\\
63.5	0.093	30.4481406996223\\
63.5	0.09666	32.808387717981\\
63.5	0.10032	35.3941255742115\\
63.5	0.10398	38.2053542683139\\
63.5	0.10764	41.2420738002883\\
63.5	0.1113	44.5042841701345\\
63.5	0.11496	47.9919853778527\\
63.5	0.11862	51.7051774234426\\
63.5	0.12228	55.6438603069045\\
63.5	0.12594	59.8080340282383\\
63.5	0.1296	64.197698587444\\
63.5	0.13326	68.8128539845215\\
63.5	0.13692	73.6535002194709\\
63.5	0.14058	78.7196372922924\\
63.5	0.14424	84.0112652029856\\
63.5	0.1479	89.5283839515507\\
63.5	0.15156	95.2709935379877\\
63.5	0.15522	101.239093962297\\
63.5	0.15888	107.432685224478\\
63.5	0.16254	113.85176732453\\
63.5	0.1662	120.496340262455\\
63.5	0.16986	127.366404038251\\
63.5	0.17352	134.46195865192\\
63.5	0.17718	141.78300410346\\
63.5	0.18084	149.329540392872\\
63.5	0.1845	157.101567520156\\
63.5	0.18816	165.099085485312\\
63.5	0.19182	173.32209428834\\
63.5	0.19548	181.770593929239\\
63.5	0.19914	190.444584408011\\
63.5	0.2028	199.344065724655\\
63.5	0.20646	208.46903787917\\
63.5	0.21012	217.819500871557\\
63.5	0.21378	227.395454701816\\
63.5	0.21744	237.196899369948\\
63.5	0.2211	247.22383487595\\
63.5	0.22476	257.476261219825\\
63.5	0.22842	267.954178401572\\
63.5	0.23208	278.657586421191\\
63.5	0.23574	289.586485278681\\
63.5	0.2394	300.740874974044\\
63.5	0.24306	312.120755507278\\
63.5	0.24672	323.726126878384\\
63.5	0.25038	335.556989087362\\
63.5	0.25404	347.613342134212\\
63.5	0.2577	359.895186018934\\
63.5	0.26136	372.402520741528\\
63.5	0.26502	385.135346301993\\
63.5	0.26868	398.093662700331\\
63.5	0.27234	411.27746993654\\
63.5	0.276	424.686768010622\\
63.875	0.093	30.7053774313329\\
63.875	0.09666	33.073177085916\\
63.875	0.10032	35.666467578371\\
63.875	0.10398	38.4852489086979\\
63.875	0.10764	41.5295210768967\\
63.875	0.1113	44.7992840829673\\
63.875	0.11496	48.2945379269099\\
63.875	0.11862	52.0152826087243\\
63.875	0.12228	55.9615181284106\\
63.875	0.12594	60.1332444859688\\
63.875	0.1296	64.530461681399\\
63.875	0.13326	69.153169714701\\
63.875	0.13692	74.0013685858748\\
63.875	0.14058	79.0750582949207\\
63.875	0.14424	84.3742388418384\\
63.875	0.1479	89.8989102266279\\
63.875	0.15156	95.6490724492894\\
63.875	0.15522	101.624725509823\\
63.875	0.15888	107.825869408228\\
63.875	0.16254	114.252504144505\\
63.875	0.1662	120.904629718654\\
63.875	0.16986	127.782246130675\\
63.875	0.17352	134.885353380568\\
63.875	0.17718	142.213951468333\\
63.875	0.18084	149.768040393969\\
63.875	0.1845	157.547620157478\\
63.875	0.18816	165.552690758858\\
63.875	0.19182	173.78325219811\\
63.875	0.19548	182.239304475234\\
63.875	0.19914	190.92084759023\\
63.875	0.2028	199.827881543098\\
63.875	0.20646	208.960406333838\\
63.875	0.21012	218.31842196245\\
63.875	0.21378	227.901928428934\\
63.875	0.21744	237.710925733289\\
63.875	0.2211	247.745413875516\\
63.875	0.22476	258.005392855616\\
63.875	0.22842	268.490862673587\\
63.875	0.23208	279.20182332943\\
63.875	0.23574	290.138274823145\\
63.875	0.2394	301.300217154732\\
63.875	0.24306	312.687650324191\\
63.875	0.24672	324.300574331521\\
63.875	0.25038	336.138989176724\\
63.875	0.25404	348.202894859798\\
63.875	0.2577	360.492291380744\\
63.875	0.26136	373.007178739563\\
63.875	0.26502	385.747556936253\\
63.875	0.26868	398.713425970815\\
63.875	0.27234	411.904785843248\\
63.875	0.276	425.321636553554\\
64.25	0.093	30.9594133590938\\
64.25	0.09666	33.3347656499013\\
64.25	0.10032	35.9356087785807\\
64.25	0.10398	38.761942745132\\
64.25	0.10764	41.8137675495553\\
64.25	0.1113	45.0910831918504\\
64.25	0.11496	48.5938896720174\\
64.25	0.11862	52.3221869900562\\
64.25	0.12228	56.275975145967\\
64.25	0.12594	60.4552541397497\\
64.25	0.1296	64.8600239714042\\
64.25	0.13326	69.4902846409307\\
64.25	0.13692	74.346036148329\\
64.25	0.14058	79.4272784935993\\
64.25	0.14424	84.7340116767414\\
64.25	0.1479	90.2662356977554\\
64.25	0.15156	96.0239505566413\\
64.25	0.15522	102.007156253399\\
64.25	0.15888	108.215852788029\\
64.25	0.16254	114.65004016053\\
64.25	0.1662	121.309718370904\\
64.25	0.16986	128.194887419149\\
64.25	0.17352	135.305547305266\\
64.25	0.17718	142.641698029256\\
64.25	0.18084	150.203339591117\\
64.25	0.1845	157.99047199085\\
64.25	0.18816	166.003095228454\\
64.25	0.19182	174.241209303931\\
64.25	0.19548	182.70481421728\\
64.25	0.19914	191.3939099685\\
64.25	0.2028	200.308496557593\\
64.25	0.20646	209.448573984557\\
64.25	0.21012	218.814142249393\\
64.25	0.21378	228.405201352101\\
64.25	0.21744	238.221751292681\\
64.25	0.2211	248.263792071133\\
64.25	0.22476	258.531323687457\\
64.25	0.22842	269.024346141652\\
64.25	0.23208	279.74285943372\\
64.25	0.23574	290.686863563659\\
64.25	0.2394	301.85635853147\\
64.25	0.24306	313.251344337153\\
64.25	0.24672	324.871820980708\\
64.25	0.25038	336.717788462136\\
64.25	0.25404	348.789246781434\\
64.25	0.2577	361.086195938605\\
64.25	0.26136	373.608635933648\\
64.25	0.26502	386.356566766562\\
64.25	0.26868	399.329988437349\\
64.25	0.27234	412.528900946007\\
64.25	0.276	425.953304292537\\
64.625	0.093	31.2102484829049\\
64.625	0.09666	33.5931534099369\\
64.625	0.10032	36.2015491748408\\
64.625	0.10398	39.0354357776165\\
64.625	0.10764	42.0948132182642\\
64.625	0.1113	45.3796814967837\\
64.625	0.11496	48.8900406131752\\
64.625	0.11862	52.6258905674385\\
64.625	0.12228	56.5872313595737\\
64.625	0.12594	60.7740629895808\\
64.625	0.1296	65.1863854574598\\
64.625	0.13326	69.8241987632107\\
64.625	0.13692	74.6875029068334\\
64.625	0.14058	79.7762978883282\\
64.625	0.14424	85.0905837076947\\
64.625	0.1479	90.6303603649332\\
64.625	0.15156	96.3956278600435\\
64.625	0.15522	102.386386193026\\
64.625	0.15888	108.60263536388\\
64.625	0.16254	115.044375372606\\
64.625	0.1662	121.711606219204\\
64.625	0.16986	128.604327903674\\
64.625	0.17352	135.722540426015\\
64.625	0.17718	143.066243786229\\
64.625	0.18084	150.635437984314\\
64.625	0.1845	158.430123020272\\
64.625	0.18816	166.450298894101\\
64.625	0.19182	174.695965605802\\
64.625	0.19548	183.167123155375\\
64.625	0.19914	191.86377154282\\
64.625	0.2028	200.785910768137\\
64.625	0.20646	209.933540831326\\
64.625	0.21012	219.306661732386\\
64.625	0.21378	228.905273471319\\
64.625	0.21744	238.729376048123\\
64.625	0.2211	248.778969462799\\
64.625	0.22476	259.054053715348\\
64.625	0.22842	269.554628805768\\
64.625	0.23208	280.280694734059\\
64.625	0.23574	291.232251500223\\
64.625	0.2394	302.409299104259\\
64.625	0.24306	313.811837546167\\
64.625	0.24672	325.439866825946\\
64.625	0.25038	337.293386943598\\
64.625	0.25404	349.372397899121\\
64.625	0.2577	361.676899692516\\
64.625	0.26136	374.206892323783\\
64.625	0.26502	386.962375792922\\
64.625	0.26868	399.943350099933\\
64.625	0.27234	413.149815244816\\
64.625	0.276	426.58177122757\\
65	0.093	31.4578828027663\\
65	0.09666	33.8483403660228\\
65	0.10032	36.4642887671511\\
65	0.10398	39.3057280061513\\
65	0.10764	42.3726580830234\\
65	0.1113	45.6650789977674\\
65	0.11496	49.1829907503833\\
65	0.11862	52.926393340871\\
65	0.12228	56.8952867692307\\
65	0.12594	61.0896710354622\\
65	0.1296	65.5095461395656\\
65	0.13326	70.154912081541\\
65	0.13692	75.0257688613882\\
65	0.14058	80.1221164791073\\
65	0.14424	85.4439549346984\\
65	0.1479	90.9912842281612\\
65	0.15156	96.764104359496\\
65	0.15522	102.762415328703\\
65	0.15888	108.986217135781\\
65	0.16254	115.435509780732\\
65	0.1662	122.110293263554\\
65	0.16986	129.010567584248\\
65	0.17352	136.136332742814\\
65	0.17718	143.487588739252\\
65	0.18084	151.064335573562\\
65	0.1845	158.866573245744\\
65	0.18816	166.894301755798\\
65	0.19182	175.147521103724\\
65	0.19548	183.626231289521\\
65	0.19914	192.33043231319\\
65	0.2028	201.260124174732\\
65	0.20646	210.415306874145\\
65	0.21012	219.79598041143\\
65	0.21378	229.402144786587\\
65	0.21744	239.233799999616\\
65	0.2211	249.290946050516\\
65	0.22476	259.573582939289\\
65	0.22842	270.081710665933\\
65	0.23208	280.81532923045\\
65	0.23574	291.774438632838\\
65	0.2394	302.959038873098\\
65	0.24306	314.36912995123\\
65	0.24672	326.004711867234\\
65	0.25038	337.86578462111\\
65	0.25404	349.952348212858\\
65	0.2577	362.264402642478\\
65	0.26136	374.801947909969\\
65	0.26502	387.564984015332\\
65	0.26868	400.553510958568\\
65	0.27234	413.767528739675\\
65	0.276	427.207037358654\\
65.375	0.093	31.702316318678\\
65.375	0.09666	34.1003265181589\\
65.375	0.10032	36.7238275555117\\
65.375	0.10398	39.5728194307363\\
65.375	0.10764	42.6473021438328\\
65.375	0.1113	45.9472756948013\\
65.375	0.11496	49.4727400836416\\
65.375	0.11862	53.2236953103538\\
65.375	0.12228	57.2001413749379\\
65.375	0.12594	61.4020782773939\\
65.375	0.1296	65.8295060177218\\
65.375	0.13326	70.4824245959215\\
65.375	0.13692	75.3608340119931\\
65.375	0.14058	80.4647342659368\\
65.375	0.14424	85.7941253577522\\
65.375	0.1479	91.3490072874395\\
65.375	0.15156	97.1293800549988\\
65.375	0.15522	103.13524366043\\
65.375	0.15888	109.366598103733\\
65.375	0.16254	115.823443384908\\
65.375	0.1662	122.505779503955\\
65.375	0.16986	129.413606460873\\
65.375	0.17352	136.546924255664\\
65.375	0.17718	143.905732888326\\
65.375	0.18084	151.490032358861\\
65.375	0.1845	159.299822667267\\
65.375	0.18816	167.335103813545\\
65.375	0.19182	175.595875797695\\
65.375	0.19548	184.082138619717\\
65.375	0.19914	192.793892279611\\
65.375	0.2028	201.731136777377\\
65.375	0.20646	210.893872113014\\
65.375	0.21012	220.282098286524\\
65.375	0.21378	229.895815297905\\
65.375	0.21744	239.735023147158\\
65.375	0.2211	249.799721834283\\
65.375	0.22476	260.08991135928\\
65.375	0.22842	270.605591722149\\
65.375	0.23208	281.34676292289\\
65.375	0.23574	292.313424961503\\
65.375	0.2394	303.505577837988\\
65.375	0.24306	314.923221552344\\
65.375	0.24672	326.566356104572\\
65.375	0.25038	338.434981494673\\
65.375	0.25404	350.529097722645\\
65.375	0.2577	362.848704788489\\
65.375	0.26136	375.393802692205\\
65.375	0.26502	388.164391433793\\
65.375	0.26868	401.160471013252\\
65.375	0.27234	414.382041430584\\
65.375	0.276	427.829102685788\\
65.75	0.093	31.94354903064\\
65.75	0.09666	34.3491118663453\\
65.75	0.10032	36.9801655399225\\
65.75	0.10398	39.8367100513716\\
65.75	0.10764	42.9187454006926\\
65.75	0.1113	46.2262715878855\\
65.75	0.11496	49.7592886129502\\
65.75	0.11862	53.5177964758869\\
65.75	0.12228	57.5017951766954\\
65.75	0.12594	61.7112847153758\\
65.75	0.1296	66.1462650919282\\
65.75	0.13326	70.8067363063524\\
65.75	0.13692	75.6926983586484\\
65.75	0.14058	80.8041512488165\\
65.75	0.14424	86.1410949768564\\
65.75	0.1479	91.7035295427682\\
65.75	0.15156	97.4914549465518\\
65.75	0.15522	103.504871188207\\
65.75	0.15888	109.743778267735\\
65.75	0.16254	116.208176185134\\
65.75	0.1662	122.898064940405\\
65.75	0.16986	129.813444533549\\
65.75	0.17352	136.954314964564\\
65.75	0.17718	144.32067623345\\
65.75	0.18084	151.912528340209\\
65.75	0.1845	159.72987128484\\
65.75	0.18816	167.772705067343\\
65.75	0.19182	176.041029687717\\
65.75	0.19548	184.534845145963\\
65.75	0.19914	193.254151442082\\
65.75	0.2028	202.198948576072\\
65.75	0.20646	211.369236547934\\
65.75	0.21012	220.765015357668\\
65.75	0.21378	230.386285005274\\
65.75	0.21744	240.233045490751\\
65.75	0.2211	250.305296814101\\
65.75	0.22476	260.603038975322\\
65.75	0.22842	271.126271974416\\
65.75	0.23208	281.874995811381\\
65.75	0.23574	292.849210486218\\
65.75	0.2394	304.048915998927\\
65.75	0.24306	315.474112349508\\
65.75	0.24672	327.124799537961\\
65.75	0.25038	339.000977564286\\
65.75	0.25404	351.102646428482\\
65.75	0.2577	363.429806130551\\
65.75	0.26136	375.982456670491\\
65.75	0.26502	388.760598048304\\
65.75	0.26868	401.764230263988\\
65.75	0.27234	414.993353317544\\
65.75	0.276	428.447967208972\\
66.125	0.093	32.1815809386523\\
66.125	0.09666	34.594696410582\\
66.125	0.10032	37.2333027203837\\
66.125	0.10398	40.0973998680572\\
66.125	0.10764	43.1869878536026\\
66.125	0.1113	46.5020666770199\\
66.125	0.11496	50.0426363383092\\
66.125	0.11862	53.8086968374702\\
66.125	0.12228	57.8002481745032\\
66.125	0.12594	62.0172903494081\\
66.125	0.1296	66.4598233621848\\
66.125	0.13326	71.1278472128335\\
66.125	0.13692	76.021361901354\\
66.125	0.14058	81.1403674277465\\
66.125	0.14424	86.4848637920108\\
66.125	0.1479	92.054850994147\\
66.125	0.15156	97.8503290341551\\
66.125	0.15522	103.871297912035\\
66.125	0.15888	110.117757627787\\
66.125	0.16254	116.589708181411\\
66.125	0.1662	123.287149572907\\
66.125	0.16986	130.210081802274\\
66.125	0.17352	137.358504869514\\
66.125	0.17718	144.732418774625\\
66.125	0.18084	152.331823517608\\
66.125	0.1845	160.156719098463\\
66.125	0.18816	168.20710551719\\
66.125	0.19182	176.482982773789\\
66.125	0.19548	184.98435086826\\
66.125	0.19914	193.711209800603\\
66.125	0.2028	202.663559570817\\
66.125	0.20646	211.841400178904\\
66.125	0.21012	221.244731624862\\
66.125	0.21378	230.873553908692\\
66.125	0.21744	240.727867030395\\
66.125	0.2211	250.807670989969\\
66.125	0.22476	261.112965787415\\
66.125	0.22842	271.643751422732\\
66.125	0.23208	282.400027895922\\
66.125	0.23574	293.381795206984\\
66.125	0.2394	304.589053355917\\
66.125	0.24306	316.021802342723\\
66.125	0.24672	327.6800421674\\
66.125	0.25038	339.563772829949\\
66.125	0.25404	351.67299433037\\
66.125	0.2577	364.007706668663\\
66.125	0.26136	376.567909844828\\
66.125	0.26502	389.353603858865\\
66.125	0.26868	402.364788710773\\
66.125	0.27234	415.601464400554\\
66.125	0.276	429.063630928206\\
66.5	0.093	32.4164120427148\\
66.5	0.09666	34.837080150869\\
66.5	0.10032	37.4832390968951\\
66.5	0.10398	40.354888880793\\
66.5	0.10764	43.4520295025629\\
66.5	0.1113	46.7746609622047\\
66.5	0.11496	50.3227832597183\\
66.5	0.11862	54.0963963951039\\
66.5	0.12228	58.0955003683613\\
66.5	0.12594	62.3200951794906\\
66.5	0.1296	66.7701808284918\\
66.5	0.13326	71.4457573153649\\
66.5	0.13692	76.3468246401098\\
66.5	0.14058	81.4733828027268\\
66.5	0.14424	86.8254318032156\\
66.5	0.1479	92.4029716415762\\
66.5	0.15156	98.2060023178088\\
66.5	0.15522	104.234523831913\\
66.5	0.15888	110.48853618389\\
66.5	0.16254	116.968039373738\\
66.5	0.1662	123.673033401458\\
66.5	0.16986	130.60351826705\\
66.5	0.17352	137.759493970514\\
66.5	0.17718	145.14096051185\\
66.5	0.18084	152.747917891057\\
66.5	0.1845	160.580366108137\\
66.5	0.18816	168.638305163088\\
66.5	0.19182	176.921735055912\\
66.5	0.19548	185.430655786607\\
66.5	0.19914	194.165067355174\\
66.5	0.2028	203.124969761613\\
66.5	0.20646	212.310363005924\\
66.5	0.21012	221.721247088107\\
66.5	0.21378	231.357622008161\\
66.5	0.21744	241.219487766088\\
66.5	0.2211	251.306844361887\\
66.5	0.22476	261.619691795557\\
66.5	0.22842	272.158030067099\\
66.5	0.23208	282.921859176513\\
66.5	0.23574	293.911179123799\\
66.5	0.2394	305.125989908957\\
66.5	0.24306	316.566291531987\\
66.5	0.24672	328.232083992889\\
66.5	0.25038	340.123367291663\\
66.5	0.25404	352.240141428308\\
66.5	0.2577	364.582406402826\\
66.5	0.26136	377.150162215215\\
66.5	0.26502	389.943408865476\\
66.5	0.26868	402.962146353609\\
66.5	0.27234	416.206374679614\\
66.5	0.276	429.676093843491\\
66.875	0.093	32.6480423428277\\
66.875	0.09666	35.0762630872063\\
66.875	0.10032	37.7299746694568\\
66.875	0.10398	40.6091770895792\\
66.875	0.10764	43.7138703475735\\
66.875	0.1113	47.0440544434397\\
66.875	0.11496	50.5997293771778\\
66.875	0.11862	54.3808951487878\\
66.875	0.12228	58.3875517582697\\
66.875	0.12594	62.6196992056234\\
66.875	0.1296	67.077337490849\\
66.875	0.13326	71.7604666139466\\
66.875	0.13692	76.669086574916\\
66.875	0.14058	81.8031973737574\\
66.875	0.14424	87.1627990104706\\
66.875	0.1479	92.7478914850557\\
66.875	0.15156	98.5584747975126\\
66.875	0.15522	104.594548947842\\
66.875	0.15888	110.856113936042\\
66.875	0.16254	117.343169762115\\
66.875	0.1662	124.05571642606\\
66.875	0.16986	130.993753927876\\
66.875	0.17352	138.157282267564\\
66.875	0.17718	145.546301445125\\
66.875	0.18084	153.160811460557\\
66.875	0.1845	161.000812313861\\
66.875	0.18816	169.066304005037\\
66.875	0.19182	177.357286534084\\
66.875	0.19548	185.873759901004\\
66.875	0.19914	194.615724105796\\
66.875	0.2028	203.583179148459\\
66.875	0.20646	212.776125028995\\
66.875	0.21012	222.194561747402\\
66.875	0.21378	231.838489303681\\
66.875	0.21744	241.707907697832\\
66.875	0.2211	251.802816929855\\
66.875	0.22476	262.12321699975\\
66.875	0.22842	272.669107907516\\
66.875	0.23208	283.440489653155\\
66.875	0.23574	294.437362236666\\
66.875	0.2394	305.659725658048\\
66.875	0.24306	317.107579917302\\
66.875	0.24672	328.780925014428\\
66.875	0.25038	340.679760949426\\
66.875	0.25404	352.804087722296\\
66.875	0.2577	365.153905333038\\
66.875	0.26136	377.729213781652\\
66.875	0.26502	390.530013068137\\
66.875	0.26868	403.556303192495\\
66.875	0.27234	416.808084154724\\
66.875	0.276	430.285355954825\\
67.25	0.093	32.8764718389908\\
67.25	0.09666	35.3122452195938\\
67.25	0.10032	37.9735094380688\\
67.25	0.10398	40.8602644944156\\
67.25	0.10764	43.9725103886344\\
67.25	0.1113	47.310247120725\\
67.25	0.11496	50.8734746906876\\
67.25	0.11862	54.662193098522\\
67.25	0.12228	58.6764023442283\\
67.25	0.12594	62.9161024278065\\
67.25	0.1296	67.3812933492566\\
67.25	0.13326	72.0719751085786\\
67.25	0.13692	76.9881477057724\\
67.25	0.14058	82.1298111408382\\
67.25	0.14424	87.4969654137759\\
67.25	0.1479	93.0896105245854\\
67.25	0.15156	98.9077464732668\\
67.25	0.15522	104.95137325982\\
67.25	0.15888	111.220490884245\\
67.25	0.16254	117.715099346543\\
67.25	0.1662	124.435198646712\\
67.25	0.16986	131.380788784752\\
67.25	0.17352	138.551869760665\\
67.25	0.17718	145.94844157445\\
67.25	0.18084	153.570504226106\\
67.25	0.1845	161.418057715635\\
67.25	0.18816	169.491102043035\\
67.25	0.19182	177.789637208307\\
67.25	0.19548	186.313663211452\\
67.25	0.19914	195.063180052468\\
67.25	0.2028	204.038187731356\\
67.25	0.20646	213.238686248115\\
67.25	0.21012	222.664675602747\\
67.25	0.21378	232.316155795251\\
67.25	0.21744	242.193126825626\\
67.25	0.2211	252.295588693873\\
67.25	0.22476	262.623541399993\\
67.25	0.22842	273.176984943984\\
67.25	0.23208	283.955919325847\\
67.25	0.23574	294.960344545582\\
67.25	0.2394	306.190260603189\\
67.25	0.24306	317.645667498667\\
67.25	0.24672	329.326565232018\\
67.25	0.25038	341.23295380324\\
67.25	0.25404	353.364833212335\\
67.25	0.2577	365.722203459301\\
67.25	0.26136	378.305064544139\\
67.25	0.26502	391.113416466849\\
67.25	0.26868	404.147259227431\\
67.25	0.27234	417.406592825885\\
67.25	0.276	430.891417262211\\
67.625	0.093	33.1017005312041\\
67.625	0.09666	35.5450265480317\\
67.625	0.10032	38.2138434027311\\
67.625	0.10398	41.1081510953023\\
67.625	0.10764	44.2279496257456\\
67.625	0.1113	47.5732389940606\\
67.625	0.11496	51.1440192002476\\
67.625	0.11862	54.9402902443065\\
67.625	0.12228	58.9620521262372\\
67.625	0.12594	63.2093048460398\\
67.625	0.1296	67.6820484037144\\
67.625	0.13326	72.3802827992608\\
67.625	0.13692	77.3040080326791\\
67.625	0.14058	82.4532241039693\\
67.625	0.14424	87.8279310131314\\
67.625	0.1479	93.4281287601654\\
67.625	0.15156	99.2538173450713\\
67.625	0.15522	105.304996767849\\
67.625	0.15888	111.581667028499\\
67.625	0.16254	118.08382812702\\
67.625	0.1662	124.811480063414\\
67.625	0.16986	131.764622837679\\
67.625	0.17352	138.943256449816\\
67.625	0.17718	146.347380899825\\
67.625	0.18084	153.976996187706\\
67.625	0.1845	161.832102313459\\
67.625	0.18816	169.912699277084\\
67.625	0.19182	178.218787078581\\
67.625	0.19548	186.750365717949\\
67.625	0.19914	195.50743519519\\
67.625	0.2028	204.489995510302\\
67.625	0.20646	213.698046663286\\
67.625	0.21012	223.131588654143\\
67.625	0.21378	232.790621482871\\
67.625	0.21744	242.67514514947\\
67.625	0.2211	252.785159653942\\
67.625	0.22476	263.120664996286\\
67.625	0.22842	273.681661176502\\
67.625	0.23208	284.468148194589\\
67.625	0.23574	295.480126050548\\
67.625	0.2394	306.71759474438\\
67.625	0.24306	318.180554276083\\
67.625	0.24672	329.869004645658\\
67.625	0.25038	341.782945853105\\
67.625	0.25404	353.922377898424\\
67.625	0.2577	366.287300781614\\
67.625	0.26136	378.877714502677\\
67.625	0.26502	391.693619061611\\
67.625	0.26868	404.735014458418\\
67.625	0.27234	418.001900693096\\
67.625	0.276	431.494277765646\\
68	0.093	33.3237284194678\\
68	0.09666	35.7746070725198\\
68	0.10032	38.4509765634436\\
68	0.10398	41.3528368922393\\
68	0.10764	44.480188058907\\
68	0.1113	47.8330300634465\\
68	0.11496	51.4113629058579\\
68	0.11862	55.2151865861412\\
68	0.12228	59.2445011042964\\
68	0.12594	63.4993064603235\\
68	0.1296	67.9796026542225\\
68	0.13326	72.6853896859933\\
68	0.13692	77.616667555636\\
68	0.14058	82.7734362631507\\
68	0.14424	88.1556958085373\\
68	0.1479	93.7634461917957\\
68	0.15156	99.596687412926\\
68	0.15522	105.655419471928\\
68	0.15888	111.939642368802\\
68	0.16254	118.449356103548\\
68	0.1662	125.184560676166\\
68	0.16986	132.145256086656\\
68	0.17352	139.331442335018\\
68	0.17718	146.743119421251\\
68	0.18084	154.380287345357\\
68	0.1845	162.242946107334\\
68	0.18816	170.331095707183\\
68	0.19182	178.644736144904\\
68	0.19548	187.183867420497\\
68	0.19914	195.948489533962\\
68	0.2028	204.938602485299\\
68	0.20646	214.154206274508\\
68	0.21012	223.595300901588\\
68	0.21378	233.261886366541\\
68	0.21744	243.153962669365\\
68	0.2211	253.271529810061\\
68	0.22476	263.61458778863\\
68	0.22842	274.18313660507\\
68	0.23208	284.977176259381\\
68	0.23574	295.996706751565\\
68	0.2394	307.241728081621\\
68	0.24306	318.712240249549\\
68	0.24672	330.408243255348\\
68	0.25038	342.329737099019\\
68	0.25404	354.476721780563\\
68	0.2577	366.849197299978\\
68	0.26136	379.447163657265\\
68	0.26502	392.270620852424\\
68	0.26868	405.319568885455\\
68	0.27234	418.594007756357\\
68	0.276	432.093937465132\\
68.375	0.093	33.5425555037818\\
68.375	0.09666	36.0009867930582\\
68.375	0.10032	38.6849089202065\\
68.375	0.10398	41.5943218852266\\
68.375	0.10764	44.7292256881187\\
68.375	0.1113	48.0896203288827\\
68.375	0.11496	51.6755058075185\\
68.375	0.11862	55.4868821240263\\
68.375	0.12228	59.5237492784059\\
68.375	0.12594	63.7861072706574\\
68.375	0.1296	68.2739561007808\\
68.375	0.13326	72.9872957687761\\
68.375	0.13692	77.9261262746433\\
68.375	0.14058	83.0904476183824\\
68.375	0.14424	88.4802597999934\\
68.375	0.1479	94.0955628194763\\
68.375	0.15156	99.936356676831\\
68.375	0.15522	106.002641372058\\
68.375	0.15888	112.294416905156\\
68.375	0.16254	118.811683276127\\
68.375	0.1662	125.554440484969\\
68.375	0.16986	132.522688531683\\
68.375	0.17352	139.716427416269\\
68.375	0.17718	147.135657138727\\
68.375	0.18084	154.780377699057\\
68.375	0.1845	162.650589097259\\
68.375	0.18816	170.746291333333\\
68.375	0.19182	179.067484407278\\
68.375	0.19548	187.614168319096\\
68.375	0.19914	196.386343068785\\
68.375	0.2028	205.384008656346\\
68.375	0.20646	214.607165081779\\
68.375	0.21012	224.055812345084\\
68.375	0.21378	233.729950446261\\
68.375	0.21744	243.62957938531\\
68.375	0.2211	253.754699162231\\
68.375	0.22476	264.105309777023\\
68.375	0.22842	274.681411229688\\
68.375	0.23208	285.483003520224\\
68.375	0.23574	296.510086648633\\
68.375	0.2394	307.762660614913\\
68.375	0.24306	319.240725419065\\
68.375	0.24672	330.944281061089\\
68.375	0.25038	342.873327540984\\
68.375	0.25404	355.027864858752\\
68.375	0.2577	367.407893014392\\
68.375	0.26136	380.013412007903\\
68.375	0.26502	392.844421839287\\
68.375	0.26868	405.900922508542\\
68.375	0.27234	419.182914015669\\
68.375	0.276	432.690396360668\\
68.75	0.093	33.758181784146\\
68.75	0.09666	36.2241657096469\\
68.75	0.10032	38.9156404730196\\
68.75	0.10398	41.8326060742642\\
68.75	0.10764	44.9750625133807\\
68.75	0.1113	48.3430097903691\\
68.75	0.11496	51.9364479052294\\
68.75	0.11862	55.7553768579616\\
68.75	0.12228	59.7997966485657\\
68.75	0.12594	64.0697072770416\\
68.75	0.1296	68.5651087433895\\
68.75	0.13326	73.2860010476092\\
68.75	0.13692	78.2323841897008\\
68.75	0.14058	83.4042581696644\\
68.75	0.14424	88.8016229874999\\
68.75	0.1479	94.4244786432072\\
68.75	0.15156	100.272825136786\\
68.75	0.15522	106.346662468237\\
68.75	0.15888	112.64599063756\\
68.75	0.16254	119.170809644755\\
68.75	0.1662	125.921119489822\\
68.75	0.16986	132.896920172761\\
68.75	0.17352	140.098211693571\\
68.75	0.17718	147.524994052254\\
68.75	0.18084	155.177267248808\\
68.75	0.1845	163.055031283234\\
68.75	0.18816	171.158286155532\\
68.75	0.19182	179.487031865702\\
68.75	0.19548	188.041268413744\\
68.75	0.19914	196.820995799658\\
68.75	0.2028	205.826214023444\\
68.75	0.20646	215.056923085101\\
68.75	0.21012	224.513122984631\\
68.75	0.21378	234.194813722032\\
68.75	0.21744	244.101995297305\\
68.75	0.2211	254.23466771045\\
68.75	0.22476	264.592830961467\\
68.75	0.22842	275.176485050356\\
68.75	0.23208	285.985629977117\\
68.75	0.23574	297.02026574175\\
68.75	0.2394	308.280392344255\\
68.75	0.24306	319.766009784631\\
68.75	0.24672	331.477118062879\\
68.75	0.25038	343.413717179\\
68.75	0.25404	355.575807132992\\
68.75	0.2577	367.963387924856\\
68.75	0.26136	380.576459554592\\
68.75	0.26502	393.4150220222\\
68.75	0.26868	406.479075327679\\
68.75	0.27234	419.768619471031\\
68.75	0.276	433.283654452254\\
69.125	0.093	33.9706072605605\\
69.125	0.09666	36.4441438222858\\
69.125	0.10032	39.143171221883\\
69.125	0.10398	42.067689459352\\
69.125	0.10764	45.217698534693\\
69.125	0.1113	48.5931984479058\\
69.125	0.11496	52.1941891989906\\
69.125	0.11862	56.0206707879472\\
69.125	0.12228	60.0726432147757\\
69.125	0.12594	64.3501064794761\\
69.125	0.1296	68.8530605820484\\
69.125	0.13326	73.5815055224926\\
69.125	0.13692	78.5354413008086\\
69.125	0.14058	83.7148679169967\\
69.125	0.14424	89.1197853710566\\
69.125	0.1479	94.7501936629883\\
69.125	0.15156	100.606092792792\\
69.125	0.15522	106.687482760467\\
69.125	0.15888	112.994363566015\\
69.125	0.16254	119.526735209434\\
69.125	0.1662	126.284597690725\\
69.125	0.16986	133.267951009889\\
69.125	0.17352	140.476795166924\\
69.125	0.17718	147.91113016183\\
69.125	0.18084	155.570955994609\\
69.125	0.1845	163.45627266526\\
69.125	0.18816	171.567080173782\\
69.125	0.19182	179.903378520177\\
69.125	0.19548	188.465167704443\\
69.125	0.19914	197.252447726581\\
69.125	0.2028	206.265218586592\\
69.125	0.20646	215.503480284474\\
69.125	0.21012	224.967232820227\\
69.125	0.21378	234.656476193853\\
69.125	0.21744	244.571210405351\\
69.125	0.2211	254.71143545472\\
69.125	0.22476	265.077151341962\\
69.125	0.22842	275.668358067075\\
69.125	0.23208	286.48505563006\\
69.125	0.23574	297.527244030918\\
69.125	0.2394	308.794923269647\\
69.125	0.24306	320.288093346248\\
69.125	0.24672	332.00675426072\\
69.125	0.25038	343.950906013065\\
69.125	0.25404	356.120548603282\\
69.125	0.2577	368.51568203137\\
69.125	0.26136	381.136306297331\\
69.125	0.26502	393.982421401163\\
69.125	0.26868	407.054027342867\\
69.125	0.27234	420.351124122443\\
69.125	0.276	433.873711739891\\
69.5	0.093	34.1798319330253\\
69.5	0.09666	36.660921130975\\
69.5	0.10032	39.3675011667967\\
69.5	0.10398	42.2995720404901\\
69.5	0.10764	45.4571337520556\\
69.5	0.1113	48.8401863014928\\
69.5	0.11496	52.448729688802\\
69.5	0.11862	56.2827639139831\\
69.5	0.12228	60.342288977036\\
69.5	0.12594	64.6273048779609\\
69.5	0.1296	69.1378116167576\\
69.5	0.13326	73.8738091934262\\
69.5	0.13692	78.8352976079667\\
69.5	0.14058	84.0222768603792\\
69.5	0.14424	89.4347469506635\\
69.5	0.1479	95.0727078788197\\
69.5	0.15156	100.936159644848\\
69.5	0.15522	107.025102248748\\
69.5	0.15888	113.33953569052\\
69.5	0.16254	119.879459970163\\
69.5	0.1662	126.644875087679\\
69.5	0.16986	133.635781043067\\
69.5	0.17352	140.852177836326\\
69.5	0.17718	148.294065467457\\
69.5	0.18084	155.961443936461\\
69.5	0.1845	163.854313243336\\
69.5	0.18816	171.972673388083\\
69.5	0.19182	180.316524370702\\
69.5	0.19548	188.885866191192\\
69.5	0.19914	197.680698849555\\
69.5	0.2028	206.70102234579\\
69.5	0.20646	215.946836679896\\
69.5	0.21012	225.418141851874\\
69.5	0.21378	235.114937861725\\
69.5	0.21744	245.037224709447\\
69.5	0.2211	255.185002395041\\
69.5	0.22476	265.558270918507\\
69.5	0.22842	276.157030279844\\
69.5	0.23208	286.981280479054\\
69.5	0.23574	298.031021516136\\
69.5	0.2394	309.306253391089\\
69.5	0.24306	320.806976103915\\
69.5	0.24672	332.533189654612\\
69.5	0.25038	344.484894043181\\
69.5	0.25404	356.662089269622\\
69.5	0.2577	369.064775333935\\
69.5	0.26136	381.69295223612\\
69.5	0.26502	394.546619976176\\
69.5	0.26868	407.625778554105\\
69.5	0.27234	420.930427969905\\
69.5	0.276	434.460568223578\\
69.875	0.093	34.3858558015404\\
69.875	0.09666	36.8744976357146\\
69.875	0.10032	39.5886303077606\\
69.875	0.10398	42.5282538176785\\
69.875	0.10764	45.6933681654684\\
69.875	0.1113	49.0839733511301\\
69.875	0.11496	52.7000693746638\\
69.875	0.11862	56.5416562360693\\
69.875	0.12228	60.6087339353466\\
69.875	0.12594	64.9013024724959\\
69.875	0.1296	69.4193618475171\\
69.875	0.13326	74.1629120604102\\
69.875	0.13692	79.1319531111751\\
69.875	0.14058	84.326484999812\\
69.875	0.14424	89.7465077263208\\
69.875	0.1479	95.3920212907014\\
69.875	0.15156	101.263025692954\\
69.875	0.15522	107.359520933078\\
69.875	0.15888	113.681507011075\\
69.875	0.16254	120.228983926943\\
69.875	0.1662	127.001951680683\\
69.875	0.16986	134.000410272295\\
69.875	0.17352	141.224359701779\\
69.875	0.17718	148.673799969135\\
69.875	0.18084	156.348731074362\\
69.875	0.1845	164.249153017462\\
69.875	0.18816	172.375065798433\\
69.875	0.19182	180.726469417277\\
69.875	0.19548	189.303363873992\\
69.875	0.19914	198.105749168579\\
69.875	0.2028	207.133625301038\\
69.875	0.20646	216.386992271369\\
69.875	0.21012	225.865850079572\\
69.875	0.21378	235.570198725646\\
69.875	0.21744	245.500038209593\\
69.875	0.2211	255.655368531411\\
69.875	0.22476	266.036189691102\\
69.875	0.22842	276.642501688664\\
69.875	0.23208	287.474304524098\\
69.875	0.23574	298.531598197404\\
69.875	0.2394	309.814382708582\\
69.875	0.24306	321.322658057632\\
69.875	0.24672	333.056424244553\\
69.875	0.25038	345.015681269347\\
69.875	0.25404	357.200429132013\\
69.875	0.2577	369.61066783255\\
69.875	0.26136	382.246397370959\\
69.875	0.26502	395.10761774724\\
69.875	0.26868	408.194328961393\\
69.875	0.27234	421.506531013418\\
69.875	0.276	435.044223903315\\
70.25	0.093	34.5886788661057\\
70.25	0.09666	37.0848733365044\\
70.25	0.10032	39.8065586447749\\
70.25	0.10398	42.7537347909172\\
70.25	0.10764	45.9264017749315\\
70.25	0.1113	49.3245595968177\\
70.25	0.11496	52.9482082565758\\
70.25	0.11862	56.7973477542057\\
70.25	0.12228	60.8719780897076\\
70.25	0.12594	65.1720992630813\\
70.25	0.1296	69.6977112743269\\
70.25	0.13326	74.4488141234444\\
70.25	0.13692	79.4254078104338\\
70.25	0.14058	84.6274923352951\\
70.25	0.14424	90.0550676980283\\
70.25	0.1479	95.7081338986334\\
70.25	0.15156	101.58669093711\\
70.25	0.15522	107.690738813459\\
70.25	0.15888	114.02027752768\\
70.25	0.16254	120.575307079773\\
70.25	0.1662	127.355827469737\\
70.25	0.16986	134.361838697574\\
70.25	0.17352	141.593340763282\\
70.25	0.17718	149.050333666862\\
70.25	0.18084	156.732817408314\\
70.25	0.1845	164.640791987638\\
70.25	0.18816	172.774257404834\\
70.25	0.19182	181.133213659902\\
70.25	0.19548	189.717660752841\\
70.25	0.19914	198.527598683653\\
70.25	0.2028	207.563027452337\\
70.25	0.20646	216.823947058892\\
70.25	0.21012	226.310357503319\\
70.25	0.21378	236.022258785618\\
70.25	0.21744	245.959650905789\\
70.25	0.2211	256.122533863832\\
70.25	0.22476	266.510907659747\\
70.25	0.22842	277.124772293534\\
70.25	0.23208	287.964127765192\\
70.25	0.23574	299.028974074723\\
70.25	0.2394	310.319311222125\\
70.25	0.24306	321.835139207399\\
70.25	0.24672	333.576458030545\\
70.25	0.25038	345.543267691563\\
70.25	0.25404	357.735568190453\\
70.25	0.2577	370.153359527215\\
70.25	0.26136	382.796641701849\\
70.25	0.26502	395.665414714354\\
70.25	0.26868	408.759678564732\\
70.25	0.27234	422.079433252981\\
70.25	0.276	435.624678779102\\
70.625	0.093	34.7883011267214\\
70.625	0.09666	37.2920482333444\\
70.625	0.10032	40.0212861778394\\
70.625	0.10398	42.9760149602062\\
70.625	0.10764	46.1562345804449\\
70.625	0.1113	49.5619450385555\\
70.625	0.11496	53.1931463345381\\
70.625	0.11862	57.0498384683924\\
70.625	0.12228	61.1320214401187\\
70.625	0.12594	65.4396952497169\\
70.625	0.1296	69.9728598971869\\
70.625	0.13326	74.7315153825289\\
70.625	0.13692	79.7156617057427\\
70.625	0.14058	84.9252988668285\\
70.625	0.14424	90.3604268657861\\
70.625	0.1479	96.0210457026157\\
70.625	0.15156	101.907155377317\\
70.625	0.15522	108.01875588989\\
70.625	0.15888	114.355847240336\\
70.625	0.16254	120.918429428653\\
70.625	0.1662	127.706502454842\\
70.625	0.16986	134.720066318903\\
70.625	0.17352	141.959121020835\\
70.625	0.17718	149.42366656064\\
70.625	0.18084	157.113702938316\\
70.625	0.1845	165.029230153865\\
70.625	0.18816	173.170248207285\\
70.625	0.19182	181.536757098577\\
70.625	0.19548	190.128756827741\\
70.625	0.19914	198.946247394777\\
70.625	0.2028	207.989228799685\\
70.625	0.20646	217.257701042465\\
70.625	0.21012	226.751664123117\\
70.625	0.21378	236.47111804164\\
70.625	0.21744	246.416062798036\\
70.625	0.2211	256.586498392303\\
70.625	0.22476	266.982424824442\\
70.625	0.22842	277.603842094453\\
70.625	0.23208	288.450750202336\\
70.625	0.23574	299.523149148091\\
70.625	0.2394	310.821038931718\\
70.625	0.24306	322.344419553217\\
70.625	0.24672	334.093291012587\\
70.625	0.25038	346.06765330983\\
70.625	0.25404	358.267506444944\\
70.625	0.2577	370.692850417931\\
70.625	0.26136	383.343685228789\\
70.625	0.26502	396.220010877519\\
70.625	0.26868	409.321827364121\\
70.625	0.27234	422.649134688594\\
70.625	0.276	436.20193285094\\
71	0.093	34.9847225833873\\
71	0.09666	37.4960223262348\\
71	0.10032	40.2328129069542\\
71	0.10398	43.1950943255454\\
71	0.10764	46.3828665820086\\
71	0.1113	49.7961296763437\\
71	0.11496	53.4348836085506\\
71	0.11862	57.2991283786294\\
71	0.12228	61.3888639865802\\
71	0.12594	65.7040904324028\\
71	0.1296	70.2448077160973\\
71	0.13326	75.0110158376637\\
71	0.13692	80.0027147971019\\
71	0.14058	85.2199045944122\\
71	0.14424	90.6625852295942\\
71	0.1479	96.3307567026482\\
71	0.15156	102.224419013574\\
71	0.15522	108.343572162372\\
71	0.15888	114.688216149041\\
71	0.16254	121.258350973583\\
71	0.1662	128.053976635996\\
71	0.16986	135.075093136282\\
71	0.17352	142.321700474439\\
71	0.17718	149.793798650468\\
71	0.18084	157.491387664369\\
71	0.1845	165.414467516142\\
71	0.18816	173.563038205787\\
71	0.19182	181.937099733303\\
71	0.19548	190.536652098692\\
71	0.19914	199.361695301952\\
71	0.2028	208.412229343085\\
71	0.20646	217.688254222089\\
71	0.21012	227.189769938965\\
71	0.21378	236.916776493713\\
71	0.21744	246.869273886333\\
71	0.2211	257.047262116825\\
71	0.22476	267.450741185188\\
71	0.22842	278.079711091424\\
71	0.23208	288.934171835531\\
71	0.23574	300.014123417511\\
71	0.2394	311.319565837362\\
71	0.24306	322.850499095085\\
71	0.24672	334.60692319068\\
71	0.25038	346.588838124147\\
71	0.25404	358.796243895486\\
71	0.2577	371.229140504696\\
71	0.26136	383.887527951779\\
71	0.26502	396.771406236733\\
71	0.26868	409.88077535956\\
71	0.27234	423.215635320258\\
71	0.276	436.775986118828\\
71.375	0.093	35.1779432361035\\
71.375	0.09666	37.6967956151754\\
71.375	0.10032	40.4411388321193\\
71.375	0.10398	43.410972886935\\
71.375	0.10764	46.6062977796226\\
71.375	0.1113	50.0271135101821\\
71.375	0.11496	53.6734200786135\\
71.375	0.11862	57.5452174849167\\
71.375	0.12228	61.6425057290919\\
71.375	0.12594	65.965284811139\\
71.375	0.1296	70.5135547310579\\
71.375	0.13326	75.2873154888487\\
71.375	0.13692	80.2865670845114\\
71.375	0.14058	85.5113095180461\\
71.375	0.14424	90.9615427894526\\
71.375	0.1479	96.637266898731\\
71.375	0.15156	102.538481845881\\
71.375	0.15522	108.665187630904\\
71.375	0.15888	115.017384253798\\
71.375	0.16254	121.595071714564\\
71.375	0.1662	128.398250013201\\
71.375	0.16986	135.426919149711\\
71.375	0.17352	142.681079124093\\
71.375	0.17718	150.160729936346\\
71.375	0.18084	157.865871586472\\
71.375	0.1845	165.796504074469\\
71.375	0.18816	173.952627400338\\
71.375	0.19182	182.334241564079\\
71.375	0.19548	190.941346565692\\
71.375	0.19914	199.773942405177\\
71.375	0.2028	208.832029082534\\
71.375	0.20646	218.115606597763\\
71.375	0.21012	227.624674950863\\
71.375	0.21378	237.359234141836\\
71.375	0.21744	247.31928417068\\
71.375	0.2211	257.504825037396\\
71.375	0.22476	267.915856741984\\
71.375	0.22842	278.552379284444\\
71.375	0.23208	289.414392664776\\
71.375	0.23574	300.50189688298\\
71.375	0.2394	311.814891939056\\
71.375	0.24306	323.353377833003\\
71.375	0.24672	335.117354564823\\
71.375	0.25038	347.106822134514\\
71.375	0.25404	359.321780542077\\
71.375	0.2577	371.762229787513\\
71.375	0.26136	384.428169870819\\
71.375	0.26502	397.319600791998\\
71.375	0.26868	410.436522551049\\
71.375	0.27234	423.778935147972\\
71.375	0.276	437.346838582766\\
71.75	0.093	35.36796308487\\
71.75	0.09666	37.8943681001664\\
71.75	0.10032	40.6462639533346\\
71.75	0.10398	43.6236506443748\\
71.75	0.10764	46.8265281732868\\
71.75	0.1113	50.2548965400708\\
71.75	0.11496	53.9087557447266\\
71.75	0.11862	57.7881057872543\\
71.75	0.12228	61.8929466676539\\
71.75	0.12594	66.2232783859254\\
71.75	0.1296	70.7791009420688\\
71.75	0.13326	75.5604143360841\\
71.75	0.13692	80.5672185679712\\
71.75	0.14058	85.7995136377303\\
71.75	0.14424	91.2572995453613\\
71.75	0.1479	96.9405762908641\\
71.75	0.15156	102.849343874239\\
71.75	0.15522	108.983602295485\\
71.75	0.15888	115.343351554604\\
71.75	0.16254	121.928591651594\\
71.75	0.1662	128.739322586457\\
71.75	0.16986	135.775544359191\\
71.75	0.17352	143.037256969797\\
71.75	0.17718	150.524460418275\\
71.75	0.18084	158.237154704625\\
71.75	0.1845	166.175339828847\\
71.75	0.18816	174.33901579094\\
71.75	0.19182	182.728182590906\\
71.75	0.19548	191.342840228743\\
71.75	0.19914	200.182988704452\\
71.75	0.2028	209.248628018034\\
71.75	0.20646	218.539758169487\\
71.75	0.21012	228.056379158812\\
71.75	0.21378	237.798490986009\\
71.75	0.21744	247.766093651077\\
71.75	0.2211	257.959187154018\\
71.75	0.22476	268.377771494831\\
71.75	0.22842	279.021846673515\\
71.75	0.23208	289.891412690071\\
71.75	0.23574	300.9864695445\\
71.75	0.2394	312.3070172368\\
71.75	0.24306	323.853055766972\\
71.75	0.24672	335.624585135016\\
71.75	0.25038	347.621605340932\\
71.75	0.25404	359.844116384719\\
71.75	0.2577	372.292118266379\\
71.75	0.26136	384.96561098591\\
71.75	0.26502	397.864594543314\\
71.75	0.26868	410.989068938589\\
71.75	0.27234	424.339034171736\\
71.75	0.276	437.914490242755\\
72.125	0.093	35.5547821296867\\
72.125	0.09666	38.0887397812076\\
72.125	0.10032	40.8481882706003\\
72.125	0.10398	43.8331275978649\\
72.125	0.10764	47.0435577630014\\
72.125	0.1113	50.4794787660097\\
72.125	0.11496	54.14089060689\\
72.125	0.11862	58.0277932856422\\
72.125	0.12228	62.1401868022662\\
72.125	0.12594	66.4780711567621\\
72.125	0.1296	71.04144634913\\
72.125	0.13326	75.8303123793697\\
72.125	0.13692	80.8446692474813\\
72.125	0.14058	86.0845169534648\\
72.125	0.14424	91.5498554973202\\
72.125	0.1479	97.2406848790475\\
72.125	0.15156	103.157005098647\\
72.125	0.15522	109.298816156118\\
72.125	0.15888	115.666118051461\\
72.125	0.16254	122.258910784676\\
72.125	0.1662	129.077194355762\\
72.125	0.16986	136.120968764721\\
72.125	0.17352	143.390234011551\\
72.125	0.17718	150.884990096254\\
72.125	0.18084	158.605237018828\\
72.125	0.1845	166.550974779274\\
72.125	0.18816	174.722203377593\\
72.125	0.19182	183.118922813782\\
72.125	0.19548	191.741133087844\\
72.125	0.19914	200.588834199778\\
72.125	0.2028	209.662026149584\\
72.125	0.20646	218.960708937261\\
72.125	0.21012	228.484882562811\\
72.125	0.21378	238.234547026232\\
72.125	0.21744	248.209702327525\\
72.125	0.2211	258.41034846669\\
72.125	0.22476	268.836485443727\\
72.125	0.22842	279.488113258636\\
72.125	0.23208	290.365231911417\\
72.125	0.23574	301.46784140207\\
72.125	0.2394	312.795941730594\\
72.125	0.24306	324.349532896991\\
72.125	0.24672	336.128614901259\\
72.125	0.25038	348.133187743399\\
72.125	0.25404	360.363251423411\\
72.125	0.2577	372.818805941296\\
72.125	0.26136	385.499851297051\\
72.125	0.26502	398.406387490679\\
72.125	0.26868	411.538414522179\\
72.125	0.27234	424.89593239155\\
72.125	0.276	438.478941098794\\
72.5	0.093	35.7384003705538\\
72.5	0.09666	38.2799106582991\\
72.5	0.10032	41.0469117839162\\
72.5	0.10398	44.0394037474052\\
72.5	0.10764	47.2573865487662\\
72.5	0.1113	50.700860187999\\
72.5	0.11496	54.3698246651037\\
72.5	0.11862	58.2642799800803\\
72.5	0.12228	62.3842261329288\\
72.5	0.12594	66.7296631236492\\
72.5	0.1296	71.3005909522414\\
72.5	0.13326	76.0970096187056\\
72.5	0.13692	81.1189191230416\\
72.5	0.14058	86.3663194652496\\
72.5	0.14424	91.8392106453295\\
72.5	0.1479	97.5375926632812\\
72.5	0.15156	103.461465519105\\
72.5	0.15522	109.6108292128\\
72.5	0.15888	115.985683744368\\
72.5	0.16254	122.586029113807\\
72.5	0.1662	129.411865321118\\
72.5	0.16986	136.463192366301\\
72.5	0.17352	143.740010249356\\
72.5	0.17718	151.242318970283\\
72.5	0.18084	158.970118529082\\
72.5	0.1845	166.923408925752\\
72.5	0.18816	175.102190160295\\
72.5	0.19182	183.506462232709\\
72.5	0.19548	192.136225142996\\
72.5	0.19914	200.991478891154\\
72.5	0.2028	210.072223477184\\
72.5	0.20646	219.378458901086\\
72.5	0.21012	228.91018516286\\
72.5	0.21378	238.667402262506\\
72.5	0.21744	248.650110200023\\
72.5	0.2211	258.858308975413\\
72.5	0.22476	269.291998588674\\
72.5	0.22842	279.951179039808\\
72.5	0.23208	290.835850328813\\
72.5	0.23574	301.94601245569\\
72.5	0.2394	313.281665420439\\
72.5	0.24306	324.84280922306\\
72.5	0.24672	336.629443863553\\
72.5	0.25038	348.641569341917\\
72.5	0.25404	360.879185658154\\
72.5	0.2577	373.342292812262\\
72.5	0.26136	386.030890804243\\
72.5	0.26502	398.944979634095\\
72.5	0.26868	412.084559301819\\
72.5	0.27234	425.449629807415\\
72.5	0.276	439.040191150883\\
72.875	0.093	35.9188178074711\\
72.875	0.09666	38.4678807314408\\
72.875	0.10032	41.2424344932824\\
72.875	0.10398	44.2424790929959\\
72.875	0.10764	47.4680145305813\\
72.875	0.1113	50.9190408060385\\
72.875	0.11496	54.5955579193677\\
72.875	0.11862	58.4975658705687\\
72.875	0.12228	62.6250646596416\\
72.875	0.12594	66.9780542865865\\
72.875	0.1296	71.5565347514032\\
72.875	0.13326	76.3605060540918\\
72.875	0.13692	81.3899681946522\\
72.875	0.14058	86.6449211730847\\
72.875	0.14424	92.125364989389\\
72.875	0.1479	97.8312996435651\\
72.875	0.15156	103.762725135613\\
72.875	0.15522	109.919641465533\\
72.875	0.15888	116.302048633325\\
72.875	0.16254	122.909946638989\\
72.875	0.1662	129.743335482524\\
72.875	0.16986	136.802215163932\\
72.875	0.17352	144.086585683211\\
72.875	0.17718	151.596447040363\\
72.875	0.18084	159.331799235386\\
72.875	0.1845	167.292642268281\\
72.875	0.18816	175.478976139048\\
72.875	0.19182	183.890800847687\\
72.875	0.19548	192.528116394197\\
72.875	0.19914	201.39092277858\\
72.875	0.2028	210.479220000835\\
72.875	0.20646	219.793008060961\\
72.875	0.21012	229.332286958959\\
72.875	0.21378	239.09705669483\\
72.875	0.21744	249.087317268572\\
72.875	0.2211	259.303068680186\\
72.875	0.22476	269.744310929672\\
72.875	0.22842	280.411044017029\\
72.875	0.23208	291.303267942259\\
72.875	0.23574	302.420982705361\\
72.875	0.2394	313.764188306334\\
72.875	0.24306	325.332884745179\\
72.875	0.24672	337.127072021896\\
72.875	0.25038	349.146750136486\\
72.875	0.25404	361.391919088947\\
72.875	0.2577	373.86257887928\\
72.875	0.26136	386.558729507484\\
72.875	0.26502	399.480370973561\\
72.875	0.26868	412.62750327751\\
72.875	0.27234	426.00012641933\\
72.875	0.276	439.598240399022\\
73.25	0.093	36.0960344404387\\
73.25	0.09666	38.6526500006329\\
73.25	0.10032	41.4347563986989\\
73.25	0.10398	44.4423536346368\\
73.25	0.10764	47.6754417084466\\
73.25	0.1113	51.1340206201283\\
73.25	0.11496	54.8180903696819\\
73.25	0.11862	58.7276509571074\\
73.25	0.12228	62.8627023824048\\
73.25	0.12594	67.223244645574\\
73.25	0.1296	71.8092777466152\\
73.25	0.13326	76.6208016855282\\
73.25	0.13692	81.6578164623131\\
73.25	0.14058	86.92032207697\\
73.25	0.14424	92.4083185294988\\
73.25	0.1479	98.1218058198994\\
73.25	0.15156	104.060783948172\\
73.25	0.15522	110.225252914316\\
73.25	0.15888	116.615212718333\\
73.25	0.16254	123.230663360221\\
73.25	0.1662	130.071604839981\\
73.25	0.16986	137.138037157613\\
73.25	0.17352	144.429960313117\\
73.25	0.17718	151.947374306492\\
73.25	0.18084	159.69027913774\\
73.25	0.1845	167.658674806859\\
73.25	0.18816	175.852561313851\\
73.25	0.19182	184.271938658714\\
73.25	0.19548	192.916806841449\\
73.25	0.19914	201.787165862056\\
73.25	0.2028	210.883015720535\\
73.25	0.20646	220.204356416886\\
73.25	0.21012	229.751187951109\\
73.25	0.21378	239.523510323204\\
73.25	0.21744	249.52132353317\\
73.25	0.2211	259.744627581009\\
73.25	0.22476	270.193422466719\\
73.25	0.22842	280.867708190301\\
73.25	0.23208	291.767484751755\\
73.25	0.23574	302.892752151081\\
73.25	0.2394	314.243510388279\\
73.25	0.24306	325.819759463349\\
73.25	0.24672	337.621499376291\\
73.25	0.25038	349.648730127104\\
73.25	0.25404	361.90145171579\\
73.25	0.2577	374.379664142347\\
73.25	0.26136	387.083367406776\\
73.25	0.26502	400.012561509077\\
73.25	0.26868	413.16724644925\\
73.25	0.27234	426.547422227295\\
73.25	0.276	440.153088843212\\
73.625	0.093	36.2700502694566\\
73.625	0.09666	38.8342184658752\\
73.625	0.10032	41.6238775001657\\
73.625	0.10398	44.639027372328\\
73.625	0.10764	47.8796680823623\\
73.625	0.1113	51.3457996302684\\
73.625	0.11496	55.0374220160465\\
73.625	0.11862	58.9545352396964\\
73.625	0.12228	63.0971393012182\\
73.625	0.12594	67.4652342006119\\
73.625	0.1296	72.0588199378775\\
73.625	0.13326	76.877896513015\\
73.625	0.13692	81.9224639260243\\
73.625	0.14058	87.1925221769057\\
73.625	0.14424	92.6880712656588\\
73.625	0.1479	98.4091111922839\\
73.625	0.15156	104.355641956781\\
73.625	0.15522	110.52766355915\\
73.625	0.15888	116.92517599939\\
73.625	0.16254	123.548179277503\\
73.625	0.1662	130.396673393488\\
73.625	0.16986	137.470658347344\\
73.625	0.17352	144.770134139072\\
73.625	0.17718	152.295100768672\\
73.625	0.18084	160.045558236144\\
73.625	0.1845	168.021506541488\\
73.625	0.18816	176.222945684704\\
73.625	0.19182	184.649875665792\\
73.625	0.19548	193.302296484752\\
73.625	0.19914	202.180208141583\\
73.625	0.2028	211.283610636287\\
73.625	0.20646	220.612503968862\\
73.625	0.21012	230.166888139309\\
73.625	0.21378	239.946763147628\\
73.625	0.21744	249.952128993819\\
73.625	0.2211	260.182985677882\\
73.625	0.22476	270.639333199817\\
73.625	0.22842	281.321171559623\\
73.625	0.23208	292.228500757302\\
73.625	0.23574	303.361320792852\\
73.625	0.2394	314.719631666275\\
73.625	0.24306	326.303433377569\\
73.625	0.24672	338.112725926735\\
73.625	0.25038	350.147509313773\\
73.625	0.25404	362.407783538683\\
73.625	0.2577	374.893548601465\\
73.625	0.26136	387.604804502118\\
73.625	0.26502	400.541551240644\\
73.625	0.26868	413.703788817041\\
73.625	0.27234	427.091517231311\\
73.625	0.276	440.704736483452\\
74	0.093	36.4408652945248\\
74	0.09666	39.0125861271678\\
74	0.10032	41.8097977976827\\
74	0.10398	44.8325003060695\\
74	0.10764	48.0806936523282\\
74	0.1113	51.5543778364588\\
74	0.11496	55.2535528584613\\
74	0.11862	59.1782187183356\\
74	0.12228	63.3283754160819\\
74	0.12594	67.7040229517\\
74	0.1296	72.3051613251901\\
74	0.13326	77.131790536552\\
74	0.13692	82.1839105857858\\
74	0.14058	87.4615214728916\\
74	0.14424	92.9646231978692\\
74	0.1479	98.6932157607187\\
74	0.15156	104.64729916144\\
74	0.15522	110.826873400033\\
74	0.15888	117.231938476499\\
74	0.16254	123.862494390836\\
74	0.1662	130.718541143045\\
74	0.16986	137.800078733125\\
74	0.17352	145.107107161078\\
74	0.17718	152.639626426903\\
74	0.18084	160.397636530599\\
74	0.1845	168.381137472168\\
74	0.18816	176.590129251608\\
74	0.19182	185.02461186892\\
74	0.19548	193.684585324104\\
74	0.19914	202.57004961716\\
74	0.2028	211.681004748088\\
74	0.20646	221.017450716888\\
74	0.21012	230.579387523559\\
74	0.21378	240.366815168103\\
74	0.21744	250.379733650518\\
74	0.2211	260.618142970806\\
74	0.22476	271.082043128965\\
74	0.22842	281.771434124996\\
74	0.23208	292.686315958899\\
74	0.23574	303.826688630674\\
74	0.2394	315.192552140321\\
74	0.24306	326.783906487839\\
74	0.24672	338.60075167323\\
74	0.25038	350.643087696492\\
74	0.25404	362.910914557627\\
74	0.2577	375.404232256633\\
74	0.26136	388.123040793511\\
74	0.26502	401.067340168261\\
74	0.26868	414.237130380883\\
74	0.27234	427.632411431377\\
74	0.276	441.253183319742\\
};
\end{axis}

\begin{axis}[%
width=4.527496cm,
height=3.050847cm,
at={(0cm,0cm)},
scale only axis,
xmin=56,
xmax=74,
tick align=outside,
xlabel={$L_{cut}$},
xmajorgrids,
ymin=0.093,
ymax=0.276,
ylabel={$D_{rlx}$},
ymajorgrids,
zmin=-54.7576562434357,
zmax=0,
zlabel={$x_1,x_3$},
zmajorgrids,
view={-140}{50},
legend style={at={(1.03,1)},anchor=north west,legend cell align=left,align=left,draw=white!15!black}
]
\addplot3[only marks,mark=*,mark options={},mark size=1.5000pt,color=mycolor1] plot table[row sep=crcr,]{%
74	0.123	-4.80311424947976\\
72	0.113	-4.80448601686883\\
61	0.095	-2.63695019186077\\
56	0.093	-4.25841520818981\\
};
\addplot3[only marks,mark=*,mark options={},mark size=1.5000pt,color=mycolor2] plot table[row sep=crcr,]{%
67	0.276	-45.0476491070073\\
66	0.255	-31.9529024161306\\
62	0.209	-19.4725831194201\\
57	0.193	-15.9543749738838\\
};
\addplot3[only marks,mark=*,mark options={},mark size=1.5000pt,color=black] plot table[row sep=crcr,]{%
69	0.104	-3.84172842242625\\
};
\addplot3[only marks,mark=*,mark options={},mark size=1.5000pt,color=black] plot table[row sep=crcr,]{%
64	0.23	-24.4563529419062\\
};

\addplot3[%
surf,
opacity=0.7,
shader=interp,
colormap={mymap}{[1pt] rgb(0pt)=(0.0901961,0.239216,0.0745098); rgb(1pt)=(0.0945149,0.242058,0.0739522); rgb(2pt)=(0.0988592,0.244894,0.0733566); rgb(3pt)=(0.103229,0.247724,0.0727241); rgb(4pt)=(0.107623,0.250549,0.0720557); rgb(5pt)=(0.112043,0.253367,0.0713525); rgb(6pt)=(0.116487,0.25618,0.0706154); rgb(7pt)=(0.120956,0.258986,0.0698456); rgb(8pt)=(0.125449,0.261787,0.0690441); rgb(9pt)=(0.129967,0.264581,0.0682118); rgb(10pt)=(0.134508,0.26737,0.06735); rgb(11pt)=(0.139074,0.270152,0.0664596); rgb(12pt)=(0.143663,0.272929,0.0655416); rgb(13pt)=(0.148275,0.275699,0.0645971); rgb(14pt)=(0.152911,0.278463,0.0636271); rgb(15pt)=(0.15757,0.281221,0.0626328); rgb(16pt)=(0.162252,0.283973,0.0616151); rgb(17pt)=(0.166957,0.286719,0.060575); rgb(18pt)=(0.171685,0.289458,0.0595136); rgb(19pt)=(0.176434,0.292191,0.0584321); rgb(20pt)=(0.181207,0.294918,0.0573313); rgb(21pt)=(0.186001,0.297639,0.0562123); rgb(22pt)=(0.190817,0.300353,0.0550763); rgb(23pt)=(0.195655,0.303061,0.0539242); rgb(24pt)=(0.200514,0.305763,0.052757); rgb(25pt)=(0.205395,0.308459,0.0515759); rgb(26pt)=(0.210296,0.311149,0.0503624); rgb(27pt)=(0.215212,0.313846,0.0490067); rgb(28pt)=(0.220142,0.316548,0.0475043); rgb(29pt)=(0.22509,0.319254,0.0458704); rgb(30pt)=(0.230056,0.321962,0.0441205); rgb(31pt)=(0.235042,0.324671,0.04227); rgb(32pt)=(0.240048,0.327379,0.0403343); rgb(33pt)=(0.245078,0.330085,0.0383287); rgb(34pt)=(0.250131,0.332786,0.0362688); rgb(35pt)=(0.25521,0.335482,0.0341698); rgb(36pt)=(0.260317,0.33817,0.0320472); rgb(37pt)=(0.265451,0.340849,0.0299163); rgb(38pt)=(0.270616,0.343517,0.0277927); rgb(39pt)=(0.275813,0.346172,0.0256916); rgb(40pt)=(0.281043,0.348814,0.0236284); rgb(41pt)=(0.286307,0.35144,0.0216186); rgb(42pt)=(0.291607,0.354048,0.0196776); rgb(43pt)=(0.296945,0.356637,0.0178207); rgb(44pt)=(0.302322,0.359206,0.0160634); rgb(45pt)=(0.307739,0.361753,0.0144211); rgb(46pt)=(0.313198,0.364275,0.0129091); rgb(47pt)=(0.318701,0.366772,0.0115428); rgb(48pt)=(0.324249,0.369242,0.0103377); rgb(49pt)=(0.329843,0.371682,0.00930909); rgb(50pt)=(0.335485,0.374093,0.00847245); rgb(51pt)=(0.341176,0.376471,0.00784314); rgb(52pt)=(0.346925,0.378826,0.00732741); rgb(53pt)=(0.352735,0.381168,0.00682184); rgb(54pt)=(0.358605,0.383497,0.00632729); rgb(55pt)=(0.364532,0.385812,0.00584464); rgb(56pt)=(0.370516,0.388113,0.00537476); rgb(57pt)=(0.376552,0.390399,0.00491852); rgb(58pt)=(0.38264,0.39267,0.00447681); rgb(59pt)=(0.388777,0.394925,0.00405048); rgb(60pt)=(0.394962,0.397164,0.00364042); rgb(61pt)=(0.401191,0.399386,0.00324749); rgb(62pt)=(0.407464,0.401592,0.00287258); rgb(63pt)=(0.413777,0.40378,0.00251655); rgb(64pt)=(0.420129,0.40595,0.00218028); rgb(65pt)=(0.426518,0.408102,0.00186463); rgb(66pt)=(0.432942,0.410234,0.00157049); rgb(67pt)=(0.439399,0.412348,0.00129873); rgb(68pt)=(0.445885,0.414441,0.00105022); rgb(69pt)=(0.452401,0.416515,0.000825833); rgb(70pt)=(0.458942,0.418567,0.000626441); rgb(71pt)=(0.465508,0.420599,0.00045292); rgb(72pt)=(0.472096,0.422609,0.000306141); rgb(73pt)=(0.478704,0.424596,0.000186979); rgb(74pt)=(0.485331,0.426562,9.63073e-05); rgb(75pt)=(0.491973,0.428504,3.49981e-05); rgb(76pt)=(0.498628,0.430422,3.92506e-06); rgb(77pt)=(0.505323,0.432315,0); rgb(78pt)=(0.512206,0.434168,0); rgb(79pt)=(0.519282,0.435983,0); rgb(80pt)=(0.526529,0.437764,0); rgb(81pt)=(0.533922,0.439512,0); rgb(82pt)=(0.54144,0.441232,0); rgb(83pt)=(0.549059,0.442927,0); rgb(84pt)=(0.556756,0.444599,0); rgb(85pt)=(0.564508,0.446252,0); rgb(86pt)=(0.572292,0.447889,0); rgb(87pt)=(0.580084,0.449514,0); rgb(88pt)=(0.587863,0.451129,0); rgb(89pt)=(0.595604,0.452737,0); rgb(90pt)=(0.603284,0.454343,0); rgb(91pt)=(0.610882,0.455948,0); rgb(92pt)=(0.618373,0.457556,0); rgb(93pt)=(0.625734,0.459171,0); rgb(94pt)=(0.632943,0.460795,0); rgb(95pt)=(0.639976,0.462432,0); rgb(96pt)=(0.64681,0.464084,0); rgb(97pt)=(0.653423,0.465756,0); rgb(98pt)=(0.659791,0.46745,0); rgb(99pt)=(0.665891,0.469169,0); rgb(100pt)=(0.6717,0.470916,0); rgb(101pt)=(0.677195,0.472696,0); rgb(102pt)=(0.682353,0.47451,0); rgb(103pt)=(0.687242,0.476355,0); rgb(104pt)=(0.691952,0.478225,0); rgb(105pt)=(0.696497,0.480118,0); rgb(106pt)=(0.700887,0.482033,0); rgb(107pt)=(0.705134,0.483968,0); rgb(108pt)=(0.709251,0.485921,0); rgb(109pt)=(0.713249,0.487891,0); rgb(110pt)=(0.71714,0.489876,0); rgb(111pt)=(0.720936,0.491875,0); rgb(112pt)=(0.724649,0.493887,0); rgb(113pt)=(0.72829,0.495909,0); rgb(114pt)=(0.731872,0.49794,0); rgb(115pt)=(0.735406,0.499979,0); rgb(116pt)=(0.738904,0.502025,0); rgb(117pt)=(0.742378,0.504075,0); rgb(118pt)=(0.74584,0.506128,0); rgb(119pt)=(0.749302,0.508182,0); rgb(120pt)=(0.752775,0.510237,0); rgb(121pt)=(0.756272,0.51229,0); rgb(122pt)=(0.759804,0.514339,0); rgb(123pt)=(0.763384,0.516385,0); rgb(124pt)=(0.767022,0.518424,0); rgb(125pt)=(0.770731,0.520455,0); rgb(126pt)=(0.774523,0.522478,0); rgb(127pt)=(0.77841,0.524489,0); rgb(128pt)=(0.782391,0.526491,0); rgb(129pt)=(0.786402,0.528496,0); rgb(130pt)=(0.790431,0.530506,0); rgb(131pt)=(0.794478,0.532521,0); rgb(132pt)=(0.798541,0.534539,0); rgb(133pt)=(0.802619,0.53656,0); rgb(134pt)=(0.806712,0.538584,0); rgb(135pt)=(0.81082,0.540609,0); rgb(136pt)=(0.81494,0.542635,0); rgb(137pt)=(0.819074,0.54466,0); rgb(138pt)=(0.823219,0.546686,0); rgb(139pt)=(0.827374,0.548709,0); rgb(140pt)=(0.831541,0.55073,0); rgb(141pt)=(0.835716,0.552749,0); rgb(142pt)=(0.8399,0.554763,0); rgb(143pt)=(0.844092,0.556774,0); rgb(144pt)=(0.848292,0.558779,0); rgb(145pt)=(0.852497,0.560778,0); rgb(146pt)=(0.856708,0.562771,0); rgb(147pt)=(0.860924,0.564756,0); rgb(148pt)=(0.865143,0.566733,0); rgb(149pt)=(0.869366,0.568701,0); rgb(150pt)=(0.873592,0.57066,0); rgb(151pt)=(0.877819,0.572608,0); rgb(152pt)=(0.882047,0.574545,0); rgb(153pt)=(0.886275,0.576471,0); rgb(154pt)=(0.890659,0.578362,0); rgb(155pt)=(0.895333,0.580203,0); rgb(156pt)=(0.900258,0.581999,0); rgb(157pt)=(0.905397,0.583755,0); rgb(158pt)=(0.910711,0.585479,0); rgb(159pt)=(0.916164,0.587176,0); rgb(160pt)=(0.921717,0.588852,0); rgb(161pt)=(0.927333,0.590513,0); rgb(162pt)=(0.932974,0.592166,0); rgb(163pt)=(0.938602,0.593815,0); rgb(164pt)=(0.94418,0.595468,0); rgb(165pt)=(0.949669,0.59713,0); rgb(166pt)=(0.955033,0.598808,0); rgb(167pt)=(0.960233,0.600507,0); rgb(168pt)=(0.965232,0.602233,0); rgb(169pt)=(0.969992,0.603992,0); rgb(170pt)=(0.974475,0.605791,0); rgb(171pt)=(0.978643,0.607636,0); rgb(172pt)=(0.98246,0.609532,0); rgb(173pt)=(0.985886,0.611486,0); rgb(174pt)=(0.988885,0.613503,0); rgb(175pt)=(0.991419,0.61559,0); rgb(176pt)=(0.99345,0.617753,0); rgb(177pt)=(0.99494,0.619997,0); rgb(178pt)=(0.995851,0.622329,0); rgb(179pt)=(0.996226,0.624763,0); rgb(180pt)=(0.996512,0.627352,0); rgb(181pt)=(0.996788,0.630095,0); rgb(182pt)=(0.997053,0.632982,0); rgb(183pt)=(0.997308,0.636004,0); rgb(184pt)=(0.997552,0.639152,0); rgb(185pt)=(0.997785,0.642416,0); rgb(186pt)=(0.998006,0.645786,0); rgb(187pt)=(0.998217,0.649253,0); rgb(188pt)=(0.998416,0.652807,0); rgb(189pt)=(0.998605,0.656439,0); rgb(190pt)=(0.998781,0.660138,0); rgb(191pt)=(0.998946,0.663897,0); rgb(192pt)=(0.9991,0.667704,0); rgb(193pt)=(0.999242,0.67155,0); rgb(194pt)=(0.999372,0.675427,0); rgb(195pt)=(0.99949,0.679323,0); rgb(196pt)=(0.999596,0.68323,0); rgb(197pt)=(0.99969,0.687139,0); rgb(198pt)=(0.999771,0.691039,0); rgb(199pt)=(0.999841,0.694921,0); rgb(200pt)=(0.999898,0.698775,0); rgb(201pt)=(0.999942,0.702592,0); rgb(202pt)=(0.999974,0.706363,0); rgb(203pt)=(0.999994,0.710077,0); rgb(204pt)=(1,0.713725,0); rgb(205pt)=(1,0.717341,0); rgb(206pt)=(1,0.720963,0); rgb(207pt)=(1,0.724591,0); rgb(208pt)=(1,0.728226,0); rgb(209pt)=(1,0.731867,0); rgb(210pt)=(1,0.735514,0); rgb(211pt)=(1,0.739167,0); rgb(212pt)=(1,0.742827,0); rgb(213pt)=(1,0.746493,0); rgb(214pt)=(1,0.750165,0); rgb(215pt)=(1,0.753843,0); rgb(216pt)=(1,0.757527,0); rgb(217pt)=(1,0.761217,0); rgb(218pt)=(1,0.764913,0); rgb(219pt)=(1,0.768615,0); rgb(220pt)=(1,0.772324,0); rgb(221pt)=(1,0.776038,0); rgb(222pt)=(1,0.779758,0); rgb(223pt)=(1,0.783484,0); rgb(224pt)=(1,0.787215,0); rgb(225pt)=(1,0.790953,0); rgb(226pt)=(1,0.794696,0); rgb(227pt)=(1,0.798445,0); rgb(228pt)=(1,0.8022,0); rgb(229pt)=(1,0.805961,0); rgb(230pt)=(1,0.809727,0); rgb(231pt)=(1,0.8135,0); rgb(232pt)=(1,0.817278,0); rgb(233pt)=(1,0.821063,0); rgb(234pt)=(1,0.824854,0); rgb(235pt)=(1,0.828652,0); rgb(236pt)=(1,0.832455,0); rgb(237pt)=(1,0.836265,0); rgb(238pt)=(1,0.840081,0); rgb(239pt)=(1,0.843903,0); rgb(240pt)=(1,0.847732,0); rgb(241pt)=(1,0.851566,0); rgb(242pt)=(1,0.855406,0); rgb(243pt)=(1,0.859253,0); rgb(244pt)=(1,0.863106,0); rgb(245pt)=(1,0.866964,0); rgb(246pt)=(1,0.870829,0); rgb(247pt)=(1,0.8747,0); rgb(248pt)=(1,0.878577,0); rgb(249pt)=(1,0.88246,0); rgb(250pt)=(1,0.886349,0); rgb(251pt)=(1,0.890243,0); rgb(252pt)=(1,0.894144,0); rgb(253pt)=(1,0.898051,0); rgb(254pt)=(1,0.901964,0); rgb(255pt)=(1,0.905882,0)},
mesh/rows=49]
table[row sep=crcr,header=false] {%
%
56	0.093	-3.97868526578156\\
56	0.09666	-3.84887229460491\\
56	0.10032	-3.76581003325399\\
56	0.10398	-3.72949848172876\\
56	0.10764	-3.73993764002923\\
56	0.1113	-3.79712750815541\\
56	0.11496	-3.90106808610729\\
56	0.11862	-4.05175937388486\\
56	0.12228	-4.24920137148817\\
56	0.12594	-4.49339407891715\\
56	0.1296	-4.78433749617183\\
56	0.13326	-5.12203162325222\\
56	0.13692	-5.50647646015831\\
56	0.14058	-5.93767200689012\\
56	0.14424	-6.41561826344763\\
56	0.1479	-6.94031522983083\\
56	0.15156	-7.51176290603973\\
56	0.15522	-8.12996129207436\\
56	0.15888	-8.79491038793467\\
56	0.16254	-9.50661019362067\\
56	0.1662	-10.2650607091324\\
56	0.16986	-11.0702619344698\\
56	0.17352	-11.9222138696329\\
56	0.17718	-12.8209165146218\\
56	0.18084	-13.7663698694363\\
56	0.1845	-14.7585739340765\\
56	0.18816	-15.7975287085425\\
56	0.19182	-16.8832341928341\\
56	0.19548	-18.0156903869514\\
56	0.19914	-19.1948972908945\\
56	0.2028	-20.4208549046633\\
56	0.20646	-21.6935632282577\\
56	0.21012	-23.0130222616778\\
56	0.21378	-24.3792320049237\\
56	0.21744	-25.7921924579952\\
56	0.2211	-27.2519036208925\\
56	0.22476	-28.7583654936155\\
56	0.22842	-30.3115780761641\\
56	0.23208	-31.9115413685385\\
56	0.23574	-33.5582553707386\\
56	0.2394	-35.2517200827644\\
56	0.24306	-36.9919355046158\\
56	0.24672	-38.778901636293\\
56	0.25038	-40.6126184777959\\
56	0.25404	-42.4930860291245\\
56	0.2577	-44.4203042902788\\
56	0.26136	-46.3942732612587\\
56	0.26502	-48.4149929420645\\
56	0.26868	-50.4824633326959\\
56	0.27234	-52.596684433153\\
56	0.276	-54.7576562434357\\
56.375	0.093	-3.87731513558214\\
56.375	0.09666	-3.73996686563128\\
56.375	0.10032	-3.64936930550612\\
56.375	0.10398	-3.60552245520664\\
56.375	0.10764	-3.6084263147329\\
56.375	0.1113	-3.65808088408484\\
56.375	0.11496	-3.75448616326248\\
56.375	0.11862	-3.89764215226582\\
56.375	0.12228	-4.08754885109488\\
56.375	0.12594	-4.32420625974964\\
56.375	0.1296	-4.60761437823009\\
56.375	0.13326	-4.93777320653622\\
56.375	0.13692	-5.31468274466809\\
56.375	0.14058	-5.73834299262565\\
56.375	0.14424	-6.20875395040893\\
56.375	0.1479	-6.72591561801789\\
56.375	0.15156	-7.28982799545256\\
56.375	0.15522	-7.90049108271295\\
56.375	0.15888	-8.55790487979902\\
56.375	0.16254	-9.26206938671081\\
56.375	0.1662	-10.0129846034483\\
56.375	0.16986	-10.8106505300115\\
56.375	0.17352	-11.6550671664004\\
56.375	0.17718	-12.546234512615\\
56.375	0.18084	-13.4841525686552\\
56.375	0.1845	-14.4688213345212\\
56.375	0.18816	-15.5002408102129\\
56.375	0.19182	-16.5784109957303\\
56.375	0.19548	-17.7033318910734\\
56.375	0.19914	-18.8750034962423\\
56.375	0.2028	-20.0934258112368\\
56.375	0.20646	-21.358598836057\\
56.375	0.21012	-22.6705225707029\\
56.375	0.21378	-24.0291970151745\\
56.375	0.21744	-25.4346221694718\\
56.375	0.2211	-26.8867980335949\\
56.375	0.22476	-28.3857246075436\\
56.375	0.22842	-29.931401891318\\
56.375	0.23208	-31.5238298849182\\
56.375	0.23574	-33.163008588344\\
56.375	0.2394	-34.8489380015956\\
56.375	0.24306	-36.5816181246728\\
56.375	0.24672	-38.3610489575757\\
56.375	0.25038	-40.1872305003044\\
56.375	0.25404	-42.0601627528587\\
56.375	0.2577	-43.9798457152388\\
56.375	0.26136	-45.9462793874445\\
56.375	0.26502	-47.959463769476\\
56.375	0.26868	-50.0193988613331\\
56.375	0.27234	-52.126084663016\\
56.375	0.276	-54.2795211745246\\
56.75	0.093	-3.78350480483674\\
56.75	0.09666	-3.63862123611164\\
56.75	0.10032	-3.54048837721225\\
56.75	0.10398	-3.48910622813853\\
56.75	0.10764	-3.48447478889054\\
56.75	0.1113	-3.52659405946825\\
56.75	0.11496	-3.61546403987166\\
56.75	0.11862	-3.75108473010077\\
56.75	0.12228	-3.93345613015559\\
56.75	0.12594	-4.16257824003611\\
56.75	0.1296	-4.43845105974232\\
56.75	0.13326	-4.76107458927423\\
56.75	0.13692	-5.13044882863186\\
56.75	0.14058	-5.54657377781519\\
56.75	0.14424	-6.00944943682423\\
56.75	0.1479	-6.51907580565896\\
56.75	0.15156	-7.0754528843194\\
56.75	0.15522	-7.67858067280554\\
56.75	0.15888	-8.32845917111739\\
56.75	0.16254	-9.02508837925494\\
56.75	0.1662	-9.76846829721816\\
56.75	0.16986	-10.5585989250071\\
56.75	0.17352	-11.3954802626218\\
56.75	0.17718	-12.2791123100621\\
56.75	0.18084	-13.2094950673282\\
56.75	0.1845	-14.18662853442\\
56.75	0.18816	-15.2105127113374\\
56.75	0.19182	-16.2811475980806\\
56.75	0.19548	-17.3985331946495\\
56.75	0.19914	-18.562669501044\\
56.75	0.2028	-19.7735565172643\\
56.75	0.20646	-21.0311942433103\\
56.75	0.21012	-22.335582679182\\
56.75	0.21378	-23.6867218248794\\
56.75	0.21744	-25.0846116804024\\
56.75	0.2211	-26.5292522457512\\
56.75	0.22476	-28.0206435209257\\
56.75	0.22842	-29.5587855059259\\
56.75	0.23208	-31.1436782007518\\
56.75	0.23574	-32.7753216054034\\
56.75	0.2394	-34.4537157198807\\
56.75	0.24306	-36.1788605441837\\
56.75	0.24672	-37.9507560783124\\
56.75	0.25038	-39.7694023222668\\
56.75	0.25404	-41.6347992760469\\
56.75	0.2577	-43.5469469396528\\
56.75	0.26136	-45.5058453130843\\
56.75	0.26502	-47.5114943963415\\
56.75	0.26868	-49.5638941894244\\
56.75	0.27234	-51.6630446923331\\
56.75	0.276	-53.8089459050674\\
57.125	0.093	-3.69725427354533\\
57.125	0.09666	-3.544835406046\\
57.125	0.10032	-3.43916724837236\\
57.125	0.10398	-3.38024980052441\\
57.125	0.10764	-3.36808306250219\\
57.125	0.1113	-3.40266703430568\\
57.125	0.11496	-3.48400171593484\\
57.125	0.11862	-3.61208710738971\\
57.125	0.12228	-3.7869232086703\\
57.125	0.12594	-4.00851001977659\\
57.125	0.1296	-4.27684754070856\\
57.125	0.13326	-4.59193577146624\\
57.125	0.13692	-4.95377471204964\\
57.125	0.14058	-5.36236436245873\\
57.125	0.14424	-5.81770472269353\\
57.125	0.1479	-6.31979579275404\\
57.125	0.15156	-6.86863757264023\\
57.125	0.15522	-7.46423006235213\\
57.125	0.15888	-8.10657326188976\\
57.125	0.16254	-8.79566717125306\\
57.125	0.1662	-9.53151179044207\\
57.125	0.16986	-10.3141071194568\\
57.125	0.17352	-11.1434531582972\\
57.125	0.17718	-12.0195499069633\\
57.125	0.18084	-12.9423973654551\\
57.125	0.1845	-13.9119955337727\\
57.125	0.18816	-14.9283444119159\\
57.125	0.19182	-15.9914439998848\\
57.125	0.19548	-17.1012942976795\\
57.125	0.19914	-18.2578953052998\\
57.125	0.2028	-19.4612470227459\\
57.125	0.20646	-20.7113494500176\\
57.125	0.21012	-22.008202587115\\
57.125	0.21378	-23.3518064340382\\
57.125	0.21744	-24.742160990787\\
57.125	0.2211	-26.1792662573616\\
57.125	0.22476	-27.6631222337618\\
57.125	0.22842	-29.1937289199878\\
57.125	0.23208	-30.7710863160394\\
57.125	0.23574	-32.3951944219169\\
57.125	0.2394	-34.0660532376199\\
57.125	0.24306	-35.7836627631487\\
57.125	0.24672	-37.5480229985032\\
57.125	0.25038	-39.3591339436834\\
57.125	0.25404	-41.2169955986892\\
57.125	0.2577	-43.1216079635208\\
57.125	0.26136	-45.0729710381781\\
57.125	0.26502	-47.0710848226611\\
57.125	0.26868	-49.1159493169698\\
57.125	0.27234	-51.2075645211042\\
57.125	0.276	-53.3459304350642\\
57.5	0.093	-3.61856354170788\\
57.5	0.09666	-3.4586093754343\\
57.5	0.10032	-3.34540591898643\\
57.5	0.10398	-3.27895317236426\\
57.5	0.10764	-3.25925113556781\\
57.5	0.1113	-3.28629980859703\\
57.5	0.11496	-3.36009919145197\\
57.5	0.11862	-3.48064928413261\\
57.5	0.12228	-3.64795008663897\\
57.5	0.12594	-3.862001598971\\
57.5	0.1296	-4.12280382112876\\
57.5	0.13326	-4.4303567531122\\
57.5	0.13692	-4.78466039492135\\
57.5	0.14058	-5.18571474655622\\
57.5	0.14424	-5.63351980801677\\
57.5	0.1479	-6.12807557930304\\
57.5	0.15156	-6.66938206041501\\
57.5	0.15522	-7.25743925135268\\
57.5	0.15888	-7.89224715211605\\
57.5	0.16254	-8.57380576270513\\
57.5	0.1662	-9.3021150831199\\
57.5	0.16986	-10.0771751133604\\
57.5	0.17352	-10.8989858534266\\
57.5	0.17718	-11.7675473033185\\
57.5	0.18084	-12.682859463036\\
57.5	0.1845	-13.6449223325793\\
57.5	0.18816	-14.6537359119483\\
57.5	0.19182	-15.709300201143\\
57.5	0.19548	-16.8116152001634\\
57.5	0.19914	-17.9606809090095\\
57.5	0.2028	-19.1564973276814\\
57.5	0.20646	-20.3990644561788\\
57.5	0.21012	-21.688382294502\\
57.5	0.21378	-23.024450842651\\
57.5	0.21744	-24.4072701006256\\
57.5	0.2211	-25.8368400684259\\
57.5	0.22476	-27.3131607460519\\
57.5	0.22842	-28.8362321335036\\
57.5	0.23208	-30.4060542307811\\
57.5	0.23574	-32.0226270378842\\
57.5	0.2394	-33.6859505548131\\
57.5	0.24306	-35.3960247815676\\
57.5	0.24672	-37.1528497181478\\
57.5	0.25038	-38.9564253645538\\
57.5	0.25404	-40.8067517207854\\
57.5	0.2577	-42.7038287868428\\
57.5	0.26136	-44.6476565627258\\
57.5	0.26502	-46.6382350484346\\
57.5	0.26868	-48.675564243969\\
57.5	0.27234	-50.7596441493292\\
57.5	0.276	-52.890474764515\\
57.875	0.093	-3.54743260932444\\
57.875	0.09666	-3.37994314427663\\
57.875	0.10032	-3.25920438905452\\
57.875	0.10398	-3.18521634365812\\
57.875	0.10764	-3.15797900808742\\
57.875	0.1113	-3.17749238234242\\
57.875	0.11496	-3.24375646642313\\
57.875	0.11862	-3.35677126032952\\
57.875	0.12228	-3.51653676406165\\
57.875	0.12594	-3.72305297761946\\
57.875	0.1296	-3.97631990100297\\
57.875	0.13326	-4.27633753421216\\
57.875	0.13692	-4.62310587724708\\
57.875	0.14058	-5.01662493010773\\
57.875	0.14424	-5.45689469279405\\
57.875	0.1479	-5.94391516530608\\
57.875	0.15156	-6.47768634764382\\
57.875	0.15522	-7.05820823980724\\
57.875	0.15888	-7.68548084179638\\
57.875	0.16254	-8.35950415361123\\
57.875	0.1662	-9.08027817525178\\
57.875	0.16986	-9.84780290671802\\
57.875	0.17352	-10.6620783480099\\
57.875	0.17718	-11.5231044991276\\
57.875	0.18084	-12.430881360071\\
57.875	0.1845	-13.38540893084\\
57.875	0.18816	-14.3866872114348\\
57.875	0.19182	-15.4347162018553\\
57.875	0.19548	-16.5294959021014\\
57.875	0.19914	-17.6710263121733\\
57.875	0.2028	-18.8593074320708\\
57.875	0.20646	-20.0943392617941\\
57.875	0.21012	-21.3761218013431\\
57.875	0.21378	-22.7046550507178\\
57.875	0.21744	-24.0799390099182\\
57.875	0.2211	-25.5019736789442\\
57.875	0.22476	-26.970759057796\\
57.875	0.22842	-28.4862951464735\\
57.875	0.23208	-30.0485819449767\\
57.875	0.23574	-31.6576194533056\\
57.875	0.2394	-33.3134076714602\\
57.875	0.24306	-35.0159465994405\\
57.875	0.24672	-36.7652362372465\\
57.875	0.25038	-38.5612765848782\\
57.875	0.25404	-40.4040676423356\\
57.875	0.2577	-42.2936094096188\\
57.875	0.26136	-44.2299018867275\\
57.875	0.26502	-46.2129450736621\\
57.875	0.26868	-48.2427389704223\\
57.875	0.27234	-50.3192835770082\\
57.875	0.276	-52.4425788934198\\
58.25	0.093	-3.48386147639497\\
58.25	0.09666	-3.30883671257292\\
58.25	0.10032	-3.18056265857659\\
58.25	0.10398	-3.09903931440595\\
58.25	0.10764	-3.06426668006101\\
58.25	0.1113	-3.07624475554178\\
58.25	0.11496	-3.13497354084824\\
58.25	0.11862	-3.24045303598041\\
58.25	0.12228	-3.39268324093828\\
58.25	0.12594	-3.59166415572187\\
58.25	0.1296	-3.83739578033115\\
58.25	0.13326	-4.1298781147661\\
58.25	0.13692	-4.4691111590268\\
58.25	0.14058	-4.85509491311318\\
58.25	0.14424	-5.28782937702529\\
58.25	0.1479	-5.76731455076307\\
58.25	0.15156	-6.29355043432658\\
58.25	0.15522	-6.86653702771578\\
58.25	0.15888	-7.48627433093069\\
58.25	0.16254	-8.15276234397129\\
58.25	0.1662	-8.8660010668376\\
58.25	0.16986	-9.62599049952961\\
58.25	0.17352	-10.4327306420473\\
58.25	0.17718	-11.2862214943907\\
58.25	0.18084	-12.1864630565598\\
58.25	0.1845	-13.1334553285547\\
58.25	0.18816	-14.1271983103752\\
58.25	0.19182	-15.1676920020214\\
58.25	0.19548	-16.2549364034933\\
58.25	0.19914	-17.388931514791\\
58.25	0.2028	-18.5696773359143\\
58.25	0.20646	-19.7971738668634\\
58.25	0.21012	-21.0714211076381\\
58.25	0.21378	-22.3924190582386\\
58.25	0.21744	-23.7601677186647\\
58.25	0.2211	-25.1746670889165\\
58.25	0.22476	-26.6359171689941\\
58.25	0.22842	-28.1439179588973\\
58.25	0.23208	-29.6986694586263\\
58.25	0.23574	-31.300171668181\\
58.25	0.2394	-32.9484245875613\\
58.25	0.24306	-34.6434282167674\\
58.25	0.24672	-36.3851825557992\\
58.25	0.25038	-38.1736876046566\\
58.25	0.25404	-40.0089433633398\\
58.25	0.2577	-41.8909498318487\\
58.25	0.26136	-43.8197070101833\\
58.25	0.26502	-45.7952148983435\\
58.25	0.26868	-47.8174734963296\\
58.25	0.27234	-49.8864828041412\\
58.25	0.276	-52.0022428217786\\
58.625	0.093	-3.42785014291946\\
58.625	0.09666	-3.24529008032318\\
58.625	0.10032	-3.10948072755261\\
58.625	0.10398	-3.02042208460773\\
58.625	0.10764	-2.97811415148857\\
58.625	0.1113	-2.98255692819509\\
58.625	0.11496	-3.03375041472733\\
58.625	0.11862	-3.13169461108525\\
58.625	0.12228	-3.27638951726889\\
58.625	0.12594	-3.46783513327824\\
58.625	0.1296	-3.70603145911328\\
58.625	0.13326	-3.99097849477403\\
58.625	0.13692	-4.32267624026046\\
58.625	0.14058	-4.70112469557262\\
58.625	0.14424	-5.12632386071049\\
58.625	0.1479	-5.59827373567404\\
58.625	0.15156	-6.11697432046331\\
58.625	0.15522	-6.68242561507827\\
58.625	0.15888	-7.29462761951896\\
58.625	0.16254	-7.95358033378533\\
58.625	0.1662	-8.65928375787738\\
58.625	0.16986	-9.41173789179516\\
58.625	0.17352	-10.2109427355386\\
58.625	0.17718	-11.0568982891078\\
58.625	0.18084	-11.9496045525027\\
58.625	0.1845	-12.8890615257233\\
58.625	0.18816	-13.8752692087696\\
58.625	0.19182	-14.9082276016416\\
58.625	0.19548	-15.9879367043392\\
58.625	0.19914	-17.1143965168627\\
58.625	0.2028	-18.2876070392118\\
58.625	0.20646	-19.5075682713866\\
58.625	0.21012	-20.7742802133871\\
58.625	0.21378	-22.0877428652133\\
58.625	0.21744	-23.4479562268652\\
58.625	0.2211	-24.8549202983428\\
58.625	0.22476	-26.3086350796461\\
58.625	0.22842	-27.8091005707751\\
58.625	0.23208	-29.3563167717299\\
58.625	0.23574	-30.9502836825103\\
58.625	0.2394	-32.5910013031164\\
58.625	0.24306	-34.2784696335482\\
58.625	0.24672	-36.0126886738058\\
58.625	0.25038	-37.793658423889\\
58.625	0.25404	-39.621378883798\\
58.625	0.2577	-41.4958500535326\\
58.625	0.26136	-43.4170719330929\\
58.625	0.26502	-45.385044522479\\
58.625	0.26868	-47.3997678216907\\
58.625	0.27234	-49.4612418307282\\
58.625	0.276	-51.5694665495913\\
59	0.093	-3.37939860889796\\
59	0.09666	-3.18930324752744\\
59	0.10032	-3.04595859598264\\
59	0.10398	-2.94936465426352\\
59	0.10764	-2.89952142237011\\
59	0.1113	-2.89642890030242\\
59	0.11496	-2.94008708806042\\
59	0.11862	-3.03049598564411\\
59	0.12228	-3.16765559305352\\
59	0.12594	-3.35156591028864\\
59	0.1296	-3.58222693734944\\
59	0.13326	-3.85963867423594\\
59	0.13692	-4.18380112094815\\
59	0.14058	-4.55471427748608\\
59	0.14424	-4.9723781438497\\
59	0.1479	-5.43679272003901\\
59	0.15156	-5.94795800605404\\
59	0.15522	-6.50587400189478\\
59	0.15888	-7.11054070756121\\
59	0.16254	-7.76195812305335\\
59	0.1662	-8.46012624837117\\
59	0.16986	-9.20504508351472\\
59	0.17352	-9.99671462848394\\
59	0.17718	-10.8351348832789\\
59	0.18084	-11.7203058478995\\
59	0.1845	-12.6522275223459\\
59	0.18816	-13.6308999066179\\
59	0.19182	-14.6563230007157\\
59	0.19548	-15.7284968046391\\
59	0.19914	-16.8474213183884\\
59	0.2028	-18.0130965419632\\
59	0.20646	-19.2255224753638\\
59	0.21012	-20.48469911859\\
59	0.21378	-21.790626471642\\
59	0.21744	-23.1433045345197\\
59	0.2211	-24.5427333072231\\
59	0.22476	-25.9889127897521\\
59	0.22842	-27.4818429821069\\
59	0.23208	-29.0215238842874\\
59	0.23574	-30.6079554962936\\
59	0.2394	-32.2411378181255\\
59	0.24306	-33.9210708497831\\
59	0.24672	-35.6477545912664\\
59	0.25038	-37.4211890425754\\
59	0.25404	-39.2413742037101\\
59	0.2577	-41.1083100746705\\
59	0.26136	-43.0219966554566\\
59	0.26502	-44.9824339460684\\
59	0.26868	-46.989621946506\\
59	0.27234	-49.0435606567692\\
59	0.276	-51.1442500768581\\
59.375	0.093	-3.33850687433041\\
59.375	0.09666	-3.14087621418566\\
59.375	0.10032	-2.98999626386662\\
59.375	0.10398	-2.88586702337327\\
59.375	0.10764	-2.82848849270563\\
59.375	0.1113	-2.81786067186368\\
59.375	0.11496	-2.85398356084746\\
59.375	0.11862	-2.93685715965692\\
59.375	0.12228	-3.06648146829209\\
59.375	0.12594	-3.24285648675297\\
59.375	0.1296	-3.46598221503953\\
59.375	0.13326	-3.7358586531518\\
59.375	0.13692	-4.05248580108977\\
59.375	0.14058	-4.41586365885347\\
59.375	0.14424	-4.82599222644285\\
59.375	0.1479	-5.28287150385795\\
59.375	0.15156	-5.78650149109873\\
59.375	0.15522	-6.33688218816523\\
59.375	0.15888	-6.93401359505743\\
59.375	0.16254	-7.57789571177534\\
59.375	0.1662	-8.2685285383189\\
59.375	0.16986	-9.00591207468823\\
59.375	0.17352	-9.79004632088324\\
59.375	0.17718	-10.6209312769039\\
59.375	0.18084	-11.4985669427504\\
59.375	0.1845	-12.4229533184225\\
59.375	0.18816	-13.3940904039203\\
59.375	0.19182	-14.4119781992438\\
59.375	0.19548	-15.476616704393\\
59.375	0.19914	-16.588005919368\\
59.375	0.2028	-17.7461458441686\\
59.375	0.20646	-18.9510364787949\\
59.375	0.21012	-20.202677823247\\
59.375	0.21378	-21.5010698775247\\
59.375	0.21744	-22.8462126416281\\
59.375	0.2211	-24.2381061155573\\
59.375	0.22476	-25.6767502993121\\
59.375	0.22842	-27.1621451928927\\
59.375	0.23208	-28.6942907962989\\
59.375	0.23574	-30.2731871095309\\
59.375	0.2394	-31.8988341325886\\
59.375	0.24306	-33.5712318654719\\
59.375	0.24672	-35.290380308181\\
59.375	0.25038	-37.0562794607158\\
59.375	0.25404	-38.8689293230762\\
59.375	0.2577	-40.7283298952624\\
59.375	0.26136	-42.6344811772743\\
59.375	0.26502	-44.5873831691119\\
59.375	0.26868	-46.5870358707751\\
59.375	0.27234	-48.6334392822641\\
59.375	0.276	-50.7265934035788\\
59.75	0.093	-3.30517493921687\\
59.75	0.09666	-3.10000898029789\\
59.75	0.10032	-2.94159373120461\\
59.75	0.10398	-2.82992919193702\\
59.75	0.10764	-2.76501536249515\\
59.75	0.1113	-2.74685224287898\\
59.75	0.11496	-2.77543983308851\\
59.75	0.11862	-2.85077813312373\\
59.75	0.12228	-2.97286714298466\\
59.75	0.12594	-3.14170686267131\\
59.75	0.1296	-3.35729729218363\\
59.75	0.13326	-3.61963843152167\\
59.75	0.13692	-3.9287302806854\\
59.75	0.14058	-4.28457283967486\\
59.75	0.14424	-4.68716610849003\\
59.75	0.1479	-5.13651008713087\\
59.75	0.15156	-5.63260477559742\\
59.75	0.15522	-6.17545017388967\\
59.75	0.15888	-6.76504628200764\\
59.75	0.16254	-7.40139309995132\\
59.75	0.1662	-8.08449062772068\\
59.75	0.16986	-8.81433886531575\\
59.75	0.17352	-9.59093781273651\\
59.75	0.17718	-10.414287469983\\
59.75	0.18084	-11.2843878370552\\
59.75	0.1845	-12.201238913953\\
59.75	0.18816	-13.1648407006766\\
59.75	0.19182	-14.1751931972259\\
59.75	0.19548	-15.2322964036009\\
59.75	0.19914	-16.3361503198016\\
59.75	0.2028	-17.486754945828\\
59.75	0.20646	-18.6841102816801\\
59.75	0.21012	-19.9282163273579\\
59.75	0.21378	-21.2190730828614\\
59.75	0.21744	-22.5566805481906\\
59.75	0.2211	-23.9410387233455\\
59.75	0.22476	-25.3721476083261\\
59.75	0.22842	-26.8500072031324\\
59.75	0.23208	-28.3746175077644\\
59.75	0.23574	-29.9459785222222\\
59.75	0.2394	-31.5640902465056\\
59.75	0.24306	-33.2289526806147\\
59.75	0.24672	-34.9405658245496\\
59.75	0.25038	-36.6989296783101\\
59.75	0.25404	-38.5040442418963\\
59.75	0.2577	-40.3559095153083\\
59.75	0.26136	-42.2545254985459\\
59.75	0.26502	-44.1998921916093\\
59.75	0.26868	-46.1920095944983\\
59.75	0.27234	-48.2308777072131\\
59.75	0.276	-50.3164965297535\\
60.125	0.093	-3.27940280355732\\
60.125	0.09666	-3.0667015458641\\
60.125	0.10032	-2.90075099799659\\
60.125	0.10398	-2.78155115995476\\
60.125	0.10764	-2.70910203173865\\
60.125	0.1113	-2.68340361334824\\
60.125	0.11496	-2.70445590478354\\
60.125	0.11862	-2.77225890604454\\
60.125	0.12228	-2.88681261713124\\
60.125	0.12594	-3.04811703804365\\
60.125	0.1296	-3.25617216878175\\
60.125	0.13326	-3.51097800934552\\
60.125	0.13692	-3.81253455973504\\
60.125	0.14058	-4.16084181995026\\
60.125	0.14424	-4.55589978999117\\
60.125	0.1479	-4.99770846985778\\
60.125	0.15156	-5.48626785955013\\
60.125	0.15522	-6.02157795906815\\
60.125	0.15888	-6.60363876841189\\
60.125	0.16254	-7.23245028758129\\
60.125	0.1662	-7.90801251657642\\
60.125	0.16986	-8.63032545539726\\
60.125	0.17352	-9.39938910404381\\
60.125	0.17718	-10.215203462516\\
60.125	0.18084	-11.077768530814\\
60.125	0.1845	-11.9870843089376\\
60.125	0.18816	-12.943150796887\\
60.125	0.19182	-13.945967994662\\
60.125	0.19548	-14.9955359022628\\
60.125	0.19914	-16.0918545196892\\
60.125	0.2028	-17.2349238469414\\
60.125	0.20646	-18.4247438840193\\
60.125	0.21012	-19.6613146309228\\
60.125	0.21378	-20.9446360876521\\
60.125	0.21744	-22.2747082542071\\
60.125	0.2211	-23.6515311305877\\
60.125	0.22476	-25.0751047167941\\
60.125	0.22842	-26.5454290128262\\
60.125	0.23208	-28.062504018684\\
60.125	0.23574	-29.6263297343675\\
60.125	0.2394	-31.2369061598767\\
60.125	0.24306	-32.8942332952115\\
60.125	0.24672	-34.5983111403721\\
60.125	0.25038	-36.3491396953584\\
60.125	0.25404	-38.1467189601704\\
60.125	0.2577	-39.9910489348081\\
60.125	0.26136	-41.8821296192716\\
60.125	0.26502	-43.8199610135606\\
60.125	0.26868	-45.8045431176755\\
60.125	0.27234	-47.835875931616\\
60.125	0.276	-49.9139594553822\\
60.5	0.093	-3.26119046735173\\
60.5	0.09666	-3.04095391088426\\
60.5	0.10032	-2.86746806424252\\
60.5	0.10398	-2.74073292742647\\
60.5	0.10764	-2.66074850043613\\
60.5	0.1113	-2.62751478327148\\
60.5	0.11496	-2.64103177593254\\
60.5	0.11862	-2.70129947841929\\
60.5	0.12228	-2.80831789073175\\
60.5	0.12594	-2.96208701286993\\
60.5	0.1296	-3.16260684483379\\
60.5	0.13326	-3.40987738662335\\
60.5	0.13692	-3.70389863823861\\
60.5	0.14058	-4.04467059967961\\
60.5	0.14424	-4.4321932709463\\
60.5	0.1479	-4.86646665203868\\
60.5	0.15156	-5.34749074295677\\
60.5	0.15522	-5.87526554370056\\
60.5	0.15888	-6.44979105427004\\
60.5	0.16254	-7.07106727466524\\
60.5	0.1662	-7.73909420488611\\
60.5	0.16986	-8.45387184493272\\
60.5	0.17352	-9.21540019480502\\
60.5	0.17718	-10.023679254503\\
60.5	0.18084	-10.8787090240267\\
60.5	0.1845	-11.7804895033762\\
60.5	0.18816	-12.7290206925513\\
60.5	0.19182	-13.7243025915521\\
60.5	0.19548	-14.7663352003786\\
60.5	0.19914	-15.8551185190308\\
60.5	0.2028	-16.9906525475087\\
60.5	0.20646	-18.1729372858124\\
60.5	0.21012	-19.4019727339417\\
60.5	0.21378	-20.6777588918967\\
60.5	0.21744	-22.0002957596775\\
60.5	0.2211	-23.3695833372839\\
60.5	0.22476	-24.7856216247161\\
60.5	0.22842	-26.2484106219739\\
60.5	0.23208	-27.7579503290574\\
60.5	0.23574	-29.3142407459667\\
60.5	0.2394	-30.9172818727016\\
60.5	0.24306	-32.5670737092623\\
60.5	0.24672	-34.2636162556487\\
60.5	0.25038	-36.0069095118607\\
60.5	0.25404	-37.7969534778985\\
60.5	0.2577	-39.633748153762\\
60.5	0.26136	-41.5172935394511\\
60.5	0.26502	-43.447589634966\\
60.5	0.26868	-45.4246364403066\\
60.5	0.27234	-47.4484339554729\\
60.5	0.276	-49.5189821804648\\
60.875	0.093	-3.25053793060014\\
60.875	0.09666	-3.02276607535844\\
60.875	0.10032	-2.84174492994245\\
60.875	0.10398	-2.70747449435217\\
60.875	0.10764	-2.61995476858758\\
60.875	0.1113	-2.57918575264871\\
60.875	0.11496	-2.58516744653553\\
60.875	0.11862	-2.63789985024805\\
60.875	0.12228	-2.73738296378628\\
60.875	0.12594	-2.88361678715021\\
60.875	0.1296	-3.07660132033984\\
60.875	0.13326	-3.31633656335517\\
60.875	0.13692	-3.6028225161962\\
60.875	0.14058	-3.93605917886294\\
60.875	0.14424	-4.3160465513554\\
60.875	0.1479	-4.74278463367355\\
60.875	0.15156	-5.21627342581738\\
60.875	0.15522	-5.73651292778695\\
60.875	0.15888	-6.30350313958223\\
60.875	0.16254	-6.91724406120316\\
60.875	0.1662	-7.57773569264981\\
60.875	0.16986	-8.28497803392219\\
60.875	0.17352	-9.03897108502026\\
60.875	0.17718	-9.83971484594401\\
60.875	0.18084	-10.6872093166935\\
60.875	0.1845	-11.5814544972687\\
60.875	0.18816	-12.5224503876695\\
60.875	0.19182	-13.5101969878961\\
60.875	0.19548	-14.5446942979484\\
60.875	0.19914	-15.6259423178264\\
60.875	0.2028	-16.7539410475301\\
60.875	0.20646	-17.9286904870595\\
60.875	0.21012	-19.1501906364146\\
60.875	0.21378	-20.4184414955954\\
60.875	0.21744	-21.7334430646019\\
60.875	0.2211	-23.0951953434341\\
60.875	0.22476	-24.503698332092\\
60.875	0.22842	-25.9589520305756\\
60.875	0.23208	-27.4609564388849\\
60.875	0.23574	-29.0097115570199\\
60.875	0.2394	-30.6052173849806\\
60.875	0.24306	-32.247473922767\\
60.875	0.24672	-33.9364811703792\\
60.875	0.25038	-35.672239127817\\
60.875	0.25404	-37.4547477950805\\
60.875	0.2577	-39.2840071721698\\
60.875	0.26136	-41.1600172590847\\
60.875	0.26502	-43.0827780558253\\
60.875	0.26868	-45.0522895623917\\
60.875	0.27234	-47.0685517787837\\
60.875	0.276	-49.1315647050014\\
61.25	0.093	-3.24744519330253\\
61.25	0.09666	-3.0121380392866\\
61.25	0.10032	-2.82358159509637\\
61.25	0.10398	-2.68177586073186\\
61.25	0.10764	-2.58672083619304\\
61.25	0.1113	-2.53841652147993\\
61.25	0.11496	-2.53686291659252\\
61.25	0.11862	-2.58206002153081\\
61.25	0.12228	-2.6740078362948\\
61.25	0.12594	-2.81270636088449\\
61.25	0.1296	-2.9981555952999\\
61.25	0.13326	-3.23035553954098\\
61.25	0.13692	-3.50930619360777\\
61.25	0.14058	-3.83500755750031\\
61.25	0.14424	-4.20745963121851\\
61.25	0.1479	-4.62666241476241\\
61.25	0.15156	-5.09261590813204\\
61.25	0.15522	-5.60532011132734\\
61.25	0.15888	-6.16477502434837\\
61.25	0.16254	-6.7709806471951\\
61.25	0.1662	-7.42393697986752\\
61.25	0.16986	-8.12364402236564\\
61.25	0.17352	-8.87010177468948\\
61.25	0.17718	-9.663310236839\\
61.25	0.18084	-10.5032694088142\\
61.25	0.1845	-11.3899792906152\\
61.25	0.18816	-12.3234398822418\\
61.25	0.19182	-13.3036511836942\\
61.25	0.19548	-14.3306131949722\\
61.25	0.19914	-15.404325916076\\
61.25	0.2028	-16.5247893470055\\
61.25	0.20646	-17.6920034877606\\
61.25	0.21012	-18.9059683383415\\
61.25	0.21378	-20.166683898748\\
61.25	0.21744	-21.4741501689803\\
61.25	0.2211	-22.8283671490382\\
61.25	0.22476	-24.2293348389219\\
61.25	0.22842	-25.6770532386313\\
61.25	0.23208	-27.1715223481663\\
61.25	0.23574	-28.7127421675272\\
61.25	0.2394	-30.3007126967136\\
61.25	0.24306	-31.9354339357258\\
61.25	0.24672	-33.6169058845637\\
61.25	0.25038	-35.3451285432273\\
61.25	0.25404	-37.1201019117166\\
61.25	0.2577	-38.9418259900316\\
61.25	0.26136	-40.8103007781723\\
61.25	0.26502	-42.7255262761387\\
61.25	0.26868	-44.6875024839308\\
61.25	0.27234	-46.6962294015486\\
61.25	0.276	-48.7517070289921\\
61.625	0.093	-3.25191225545888\\
61.625	0.09666	-3.00906980266871\\
61.625	0.10032	-2.81297805970425\\
61.625	0.10398	-2.6636370265655\\
61.625	0.10764	-2.56104670325246\\
61.625	0.1113	-2.5052070897651\\
61.625	0.11496	-2.49611818610345\\
61.625	0.11862	-2.5337799922675\\
61.625	0.12228	-2.61819250825726\\
61.625	0.12594	-2.74935573407273\\
61.625	0.1296	-2.92726966971388\\
61.625	0.13326	-3.15193431518074\\
61.625	0.13692	-3.4233496704733\\
61.625	0.14058	-3.74151573559158\\
61.625	0.14424	-4.10643251053556\\
61.625	0.1479	-4.51809999530525\\
61.625	0.15156	-4.97651818990062\\
61.625	0.15522	-5.4816870943217\\
61.625	0.15888	-6.03360670856849\\
61.625	0.16254	-6.63227703264097\\
61.625	0.1662	-7.27769806653916\\
61.625	0.16986	-7.96986981026305\\
61.625	0.17352	-8.70879226381263\\
61.625	0.17718	-9.49446542718795\\
61.625	0.18084	-10.3268893003889\\
61.625	0.1845	-11.2060638834157\\
61.625	0.18816	-12.1319891762681\\
61.625	0.19182	-13.1046651789462\\
61.625	0.19548	-14.12409189145\\
61.625	0.19914	-15.1902693137795\\
61.625	0.2028	-16.3031974459347\\
61.625	0.20646	-17.4628762879157\\
61.625	0.21012	-18.6693058397223\\
61.625	0.21378	-19.9224861013546\\
61.625	0.21744	-21.2224170728126\\
61.625	0.2211	-22.5690987540964\\
61.625	0.22476	-23.9625311452058\\
61.625	0.22842	-25.402714246141\\
61.625	0.23208	-26.8896480569018\\
61.625	0.23574	-28.4233325774884\\
61.625	0.2394	-30.0037678079006\\
61.625	0.24306	-31.6309537481385\\
61.625	0.24672	-33.3048903982022\\
61.625	0.25038	-35.0255777580915\\
61.625	0.25404	-36.7930158278066\\
61.625	0.2577	-38.6072046073474\\
61.625	0.26136	-40.4681440967138\\
61.625	0.26502	-42.375834295906\\
61.625	0.26868	-44.3302752049239\\
61.625	0.27234	-46.3314668237674\\
61.625	0.276	-48.3794091524367\\
62	0.093	-3.26393911706923\\
62	0.09666	-3.01356136550483\\
62	0.10032	-2.80993432376613\\
62	0.10398	-2.65305799185314\\
62	0.10764	-2.54293236976586\\
62	0.1113	-2.47955745750427\\
62	0.11496	-2.46293325506839\\
62	0.11862	-2.49305976245822\\
62	0.12228	-2.56993697967374\\
62	0.12594	-2.69356490671495\\
62	0.1296	-2.8639435435819\\
62	0.13326	-3.0810728902745\\
62	0.13692	-3.34495294679283\\
62	0.14058	-3.65558371313688\\
62	0.14424	-4.01296518930663\\
62	0.1479	-4.41709737530206\\
62	0.15156	-4.86798027112321\\
62	0.15522	-5.36561387677006\\
62	0.15888	-5.90999819224262\\
62	0.16254	-6.50113321754087\\
62	0.1662	-7.1390189526648\\
62	0.16986	-7.82365539761449\\
62	0.17352	-8.55504255238981\\
62	0.17718	-9.33318041699091\\
62	0.18084	-10.1580689914177\\
62	0.1845	-11.0297082756701\\
62	0.18816	-11.9480982697483\\
62	0.19182	-12.9132389736522\\
62	0.19548	-13.9251303873817\\
62	0.19914	-14.9837725109371\\
62	0.2028	-16.089165344318\\
62	0.20646	-17.2413088875247\\
62	0.21012	-18.4402031405571\\
62	0.21378	-19.6858481034152\\
62	0.21744	-20.978243776099\\
62	0.2211	-22.3173901586085\\
62	0.22476	-23.7032872509437\\
62	0.22842	-25.1359350531046\\
62	0.23208	-26.6153335650912\\
62	0.23574	-28.1414827869035\\
62	0.2394	-29.7143827185415\\
62	0.24306	-31.3340333600052\\
62	0.24672	-33.0004347112947\\
62	0.25038	-34.7135867724098\\
62	0.25404	-36.4734895433506\\
62	0.2577	-38.2801430241171\\
62	0.26136	-40.1335472147094\\
62	0.26502	-42.0337021151273\\
62	0.26868	-43.9806077253709\\
62	0.27234	-45.9742640454403\\
62	0.276	-48.0146710753353\\
62.375	0.093	-3.28352577813357\\
62.375	0.09666	-3.02561272779493\\
62.375	0.10032	-2.814450387282\\
62.375	0.10398	-2.65003875659477\\
62.375	0.10764	-2.53237783573326\\
62.375	0.1113	-2.46146762469744\\
62.375	0.11496	-2.43730812348732\\
62.375	0.11862	-2.45989933210289\\
62.375	0.12228	-2.52924125054419\\
62.375	0.12594	-2.64533387881118\\
62.375	0.1296	-2.80817721690388\\
62.375	0.13326	-3.01777126482225\\
62.375	0.13692	-3.27411602256635\\
62.375	0.14058	-3.57721149013614\\
62.375	0.14424	-3.92705766753166\\
62.375	0.1479	-4.32365455475286\\
62.375	0.15156	-4.76700215179978\\
62.375	0.15522	-5.2571004586724\\
62.375	0.15888	-5.79394947537073\\
62.375	0.16254	-6.37754920189472\\
62.375	0.1662	-7.00789963824442\\
62.375	0.16986	-7.68500078441986\\
62.375	0.17352	-8.40885264042098\\
62.375	0.17718	-9.17945520624782\\
62.375	0.18084	-9.99680848190035\\
62.375	0.1845	-10.8609124673786\\
62.375	0.18816	-11.7717671626825\\
62.375	0.19182	-12.7293725678122\\
62.375	0.19548	-13.7337286827675\\
62.375	0.19914	-14.7848355075486\\
62.375	0.2028	-15.8826930421553\\
62.375	0.20646	-17.0273012865878\\
62.375	0.21012	-18.2186602408459\\
62.375	0.21378	-19.4567699049298\\
62.375	0.21744	-20.7416302788393\\
62.375	0.2211	-22.0732413625746\\
62.375	0.22476	-23.4516031561355\\
62.375	0.22842	-24.8767156595222\\
62.375	0.23208	-26.3485788727346\\
62.375	0.23574	-27.8671927957727\\
62.375	0.2394	-29.4325574286365\\
62.375	0.24306	-31.0446727713259\\
62.375	0.24672	-32.7035388238411\\
62.375	0.25038	-34.409155586182\\
62.375	0.25404	-36.1615230583486\\
62.375	0.2577	-37.9606412403409\\
62.375	0.26136	-39.8065101321589\\
62.375	0.26502	-41.6991297338026\\
62.375	0.26868	-43.638500045272\\
62.375	0.27234	-45.6246210665671\\
62.375	0.276	-47.6574927976879\\
62.75	0.093	-3.31067223865185\\
62.75	0.09666	-3.045223889539\\
62.75	0.10032	-2.82652625025183\\
62.75	0.10398	-2.65457932079036\\
62.75	0.10764	-2.52938310115462\\
62.75	0.1113	-2.45093759134455\\
62.75	0.11496	-2.41924279136021\\
62.75	0.11862	-2.43429870120156\\
62.75	0.12228	-2.4961053208686\\
62.75	0.12594	-2.60466265036137\\
62.75	0.1296	-2.75997068967981\\
62.75	0.13326	-2.96202943882398\\
62.75	0.13692	-3.21083889779382\\
62.75	0.14058	-3.50639906658939\\
62.75	0.14424	-3.84870994521067\\
62.75	0.1479	-4.23777153365765\\
62.75	0.15156	-4.67358383193031\\
62.75	0.15522	-5.1561468400287\\
62.75	0.15888	-5.68546055795277\\
62.75	0.16254	-6.26152498570256\\
62.75	0.1662	-6.88434012327804\\
62.75	0.16986	-7.55390597067924\\
62.75	0.17352	-8.27022252790611\\
62.75	0.17718	-9.03328979495871\\
62.75	0.18084	-9.84310777183701\\
62.75	0.1845	-10.699676458541\\
62.75	0.18816	-11.6029958550707\\
62.75	0.19182	-12.5530659614261\\
62.75	0.19548	-13.5498867776072\\
62.75	0.19914	-14.5934583036141\\
62.75	0.2028	-15.6837805394466\\
62.75	0.20646	-16.8208534851048\\
62.75	0.21012	-18.0046771405887\\
62.75	0.21378	-19.2352515058983\\
62.75	0.21744	-20.5125765810336\\
62.75	0.2211	-21.8366523659946\\
62.75	0.22476	-23.2074788607814\\
62.75	0.22842	-24.6250560653938\\
62.75	0.23208	-26.0893839798319\\
62.75	0.23574	-27.6004626040958\\
62.75	0.2394	-29.1582919381854\\
62.75	0.24306	-30.7628719821006\\
62.75	0.24672	-32.4142027358415\\
62.75	0.25038	-34.1122841994082\\
62.75	0.25404	-35.8571163728005\\
62.75	0.2577	-37.6486992560186\\
62.75	0.26136	-39.4870328490623\\
62.75	0.26502	-41.3721171519318\\
62.75	0.26868	-43.303952164627\\
62.75	0.27234	-45.2825378871478\\
62.75	0.276	-47.3078743194944\\
63.125	0.093	-3.34537849862418\\
63.125	0.09666	-3.07239485073706\\
63.125	0.10032	-2.84616191267567\\
63.125	0.10398	-2.66667968443997\\
63.125	0.10764	-2.53394816602997\\
63.125	0.1113	-2.44796735744569\\
63.125	0.11496	-2.4087372586871\\
63.125	0.11862	-2.4162578697542\\
63.125	0.12228	-2.47052919064704\\
63.125	0.12594	-2.57155122136555\\
63.125	0.1296	-2.71932396190978\\
63.125	0.13326	-2.91384741227967\\
63.125	0.13692	-3.15512157247531\\
63.125	0.14058	-3.44314644249665\\
63.125	0.14424	-3.77792202234367\\
63.125	0.1479	-4.15944831201642\\
63.125	0.15156	-4.58772531151485\\
63.125	0.15522	-5.06275302083901\\
63.125	0.15888	-5.58453143998886\\
63.125	0.16254	-6.15306056896439\\
63.125	0.1662	-6.76834040776563\\
63.125	0.16986	-7.43037095639261\\
63.125	0.17352	-8.13915221484525\\
63.125	0.17718	-8.89468418312359\\
63.125	0.18084	-9.69696686122766\\
63.125	0.1845	-10.5460002491574\\
63.125	0.18816	-11.4417843469129\\
63.125	0.19182	-12.3843191544941\\
63.125	0.19548	-13.373604671901\\
63.125	0.19914	-14.4096408991335\\
63.125	0.2028	-15.4924278361918\\
63.125	0.20646	-16.6219654830758\\
63.125	0.21012	-17.7982538397855\\
63.125	0.21378	-19.0212929063209\\
63.125	0.21744	-20.2910826826819\\
63.125	0.2211	-21.6076231688687\\
63.125	0.22476	-22.9709143648812\\
63.125	0.22842	-24.3809562707195\\
63.125	0.23208	-25.8377488863833\\
63.125	0.23574	-27.341292211873\\
63.125	0.2394	-28.8915862471882\\
63.125	0.24306	-30.4886309923292\\
63.125	0.24672	-32.132426447296\\
63.125	0.25038	-33.8229726120884\\
63.125	0.25404	-35.5602694867065\\
63.125	0.2577	-37.3443170711503\\
63.125	0.26136	-39.1751153654198\\
63.125	0.26502	-41.0526643695151\\
63.125	0.26868	-42.976964083436\\
63.125	0.27234	-44.9480145071826\\
63.125	0.276	-46.9658156407549\\
63.5	0.093	-3.38764455805042\\
63.5	0.09666	-3.10712561138909\\
63.5	0.10032	-2.87335737455345\\
63.5	0.10398	-2.68633984754353\\
63.5	0.10764	-2.5460730303593\\
63.5	0.1113	-2.45255692300077\\
63.5	0.11496	-2.40579152546794\\
63.5	0.11862	-2.40577683776081\\
63.5	0.12228	-2.45251285987942\\
63.5	0.12594	-2.5459995918237\\
63.5	0.1296	-2.68623703359367\\
63.5	0.13326	-2.87322518518936\\
63.5	0.13692	-3.10696404661074\\
63.5	0.14058	-3.38745361785785\\
63.5	0.14424	-3.71469389893065\\
63.5	0.1479	-4.08868488982917\\
63.5	0.15156	-4.50942659055337\\
63.5	0.15522	-4.97691900110327\\
63.5	0.15888	-5.49116212147889\\
63.5	0.16254	-6.05215595168019\\
63.5	0.1662	-6.65990049170721\\
63.5	0.16986	-7.31439574155992\\
63.5	0.17352	-8.01564170123833\\
63.5	0.17718	-8.76363837074245\\
63.5	0.18084	-9.55838575007226\\
63.5	0.1845	-10.3998838392278\\
63.5	0.18816	-11.288132638209\\
63.5	0.19182	-12.223132147016\\
63.5	0.19548	-13.2048823656486\\
63.5	0.19914	-14.233383294107\\
63.5	0.2028	-15.308634932391\\
63.5	0.20646	-16.4306372805008\\
63.5	0.21012	-17.5993903384362\\
63.5	0.21378	-18.8148941061974\\
63.5	0.21744	-20.0771485837842\\
63.5	0.2211	-21.3861537711968\\
63.5	0.22476	-22.741909668435\\
63.5	0.22842	-24.144416275499\\
63.5	0.23208	-25.5936735923886\\
63.5	0.23574	-27.089681619104\\
63.5	0.2394	-28.6324403556451\\
63.5	0.24306	-30.2219498020118\\
63.5	0.24672	-31.8582099582043\\
63.5	0.25038	-33.5412208242225\\
63.5	0.25404	-35.2709824000664\\
63.5	0.2577	-37.047494685736\\
63.5	0.26136	-38.8707576812313\\
63.5	0.26502	-40.7407713865523\\
63.5	0.26868	-42.657535801699\\
63.5	0.27234	-44.6210509266714\\
63.5	0.276	-46.6313167614694\\
63.875	0.093	-3.4374704169307\\
63.875	0.09666	-3.14941617149513\\
63.875	0.10032	-2.90811263588525\\
63.875	0.10398	-2.71355981010108\\
63.875	0.10764	-2.56575769414262\\
63.875	0.1113	-2.46470628800985\\
63.875	0.11496	-2.41040559170282\\
63.875	0.11862	-2.40285560522143\\
63.875	0.12228	-2.44205632856581\\
63.875	0.12594	-2.52800776173583\\
63.875	0.1296	-2.66070990473161\\
63.875	0.13326	-2.84016275755303\\
63.875	0.13692	-3.06636632020019\\
63.875	0.14058	-3.33932059267307\\
63.875	0.14424	-3.65902557497161\\
63.875	0.1479	-4.0254812670959\\
63.875	0.15156	-4.43868766904584\\
63.875	0.15522	-4.89864478082154\\
63.875	0.15888	-5.4053526024229\\
63.875	0.16254	-5.95881113384998\\
63.875	0.1662	-6.55902037510276\\
63.875	0.16986	-7.20598032618125\\
63.875	0.17352	-7.89969098708543\\
63.875	0.17718	-8.64015235781532\\
63.875	0.18084	-9.4273644383709\\
63.875	0.1845	-10.2613272287522\\
63.875	0.18816	-11.1420407289592\\
63.875	0.19182	-12.0695049389919\\
63.875	0.19548	-13.0437198588503\\
63.875	0.19914	-14.0646854885344\\
63.875	0.2028	-15.1324018280442\\
63.875	0.20646	-16.2468688773797\\
63.875	0.21012	-17.4080866365409\\
63.875	0.21378	-18.6160551055279\\
63.875	0.21744	-19.8707742843404\\
63.875	0.2211	-21.1722441729788\\
63.875	0.22476	-22.5204647714428\\
63.875	0.22842	-23.9154360797326\\
63.875	0.23208	-25.357158097848\\
63.875	0.23574	-26.8456308257891\\
63.875	0.2394	-28.3808542635559\\
63.875	0.24306	-29.9628284111485\\
63.875	0.24672	-31.5915532685667\\
63.875	0.25038	-33.2670288358107\\
63.875	0.25404	-34.9892551128803\\
63.875	0.2577	-36.7582320997757\\
63.875	0.26136	-38.5739597964967\\
63.875	0.26502	-40.4364382030435\\
63.875	0.26868	-42.345667319416\\
63.875	0.27234	-44.3016471456141\\
63.875	0.276	-46.304377681638\\
64.25	0.093	-3.49485607526494\\
64.25	0.09666	-3.19926653105514\\
64.25	0.10032	-2.95042769667105\\
64.25	0.10398	-2.74833957211263\\
64.25	0.10764	-2.59300215737993\\
64.25	0.1113	-2.48441545247294\\
64.25	0.11496	-2.42257945739166\\
64.25	0.11862	-2.40749417213604\\
64.25	0.12228	-2.43915959670616\\
64.25	0.12594	-2.51757573110198\\
64.25	0.1296	-2.6427425753235\\
64.25	0.13326	-2.8146601293707\\
64.25	0.13692	-3.03332839324362\\
64.25	0.14058	-3.29874736694224\\
64.25	0.14424	-3.61091705046658\\
64.25	0.1479	-3.96983744381662\\
64.25	0.15156	-4.37550854699236\\
64.25	0.15522	-4.8279303599938\\
64.25	0.15888	-5.32710288282093\\
64.25	0.16254	-5.87302611547378\\
64.25	0.1662	-6.4657000579523\\
64.25	0.16986	-7.10512471025656\\
64.25	0.17352	-7.79130007238651\\
64.25	0.17718	-8.52422614434217\\
64.25	0.18084	-9.30390292612353\\
64.25	0.1845	-10.1303304177306\\
64.25	0.18816	-11.0035086191633\\
64.25	0.19182	-11.9234375304218\\
64.25	0.19548	-12.890117151506\\
64.25	0.19914	-13.9035474824158\\
64.25	0.2028	-14.9637285231514\\
64.25	0.20646	-16.0706602737127\\
64.25	0.21012	-17.2243427340997\\
64.25	0.21378	-18.4247759043123\\
64.25	0.21744	-19.6719597843507\\
64.25	0.2211	-20.9658943742148\\
64.25	0.22476	-22.3065796739046\\
64.25	0.22842	-23.6940156834201\\
64.25	0.23208	-25.1282024027613\\
64.25	0.23574	-26.6091398319282\\
64.25	0.2394	-28.1368279709208\\
64.25	0.24306	-29.7112668197391\\
64.25	0.24672	-31.3324563783831\\
64.25	0.25038	-33.0003966468528\\
64.25	0.25404	-34.7150876251482\\
64.25	0.2577	-36.4765293132694\\
64.25	0.26136	-38.2847217112162\\
64.25	0.26502	-40.1396648189887\\
64.25	0.26868	-42.0413586365869\\
64.25	0.27234	-43.9898031640108\\
64.25	0.276	-45.9849984012605\\
64.625	0.093	-3.55980153305317\\
64.625	0.09666	-3.25667669006911\\
64.625	0.10032	-3.00030255691079\\
64.625	0.10398	-2.79067913357813\\
64.625	0.10764	-2.62780642007121\\
64.625	0.1113	-2.51168441638996\\
64.625	0.11496	-2.44231312253445\\
64.625	0.11862	-2.41969253850463\\
64.625	0.12228	-2.44382266430049\\
64.625	0.12594	-2.51470349992206\\
64.625	0.1296	-2.63233504536935\\
64.625	0.13326	-2.79671730064232\\
64.625	0.13692	-3.00785026574101\\
64.625	0.14058	-3.26573394066541\\
64.625	0.14424	-3.57036832541552\\
64.625	0.1479	-3.92175341999129\\
64.625	0.15156	-4.3198892243928\\
64.625	0.15522	-4.76477573861999\\
64.625	0.15888	-5.25641296267292\\
64.625	0.16254	-5.79480089655151\\
64.625	0.1662	-6.37993954025583\\
64.625	0.16986	-7.01182889378583\\
64.625	0.17352	-7.69046895714153\\
64.625	0.17718	-8.41585973032296\\
64.625	0.18084	-9.18800121333008\\
64.625	0.1845	-10.0068934061629\\
64.625	0.18816	-10.8725363088214\\
64.625	0.19182	-11.7849299213057\\
64.625	0.19548	-12.7440742436156\\
64.625	0.19914	-13.7499692757512\\
64.625	0.2028	-14.8026150177126\\
64.625	0.20646	-15.9020114694996\\
64.625	0.21012	-17.0481586311124\\
64.625	0.21378	-18.2410565025508\\
64.625	0.21744	-19.4807050838149\\
64.625	0.2211	-20.7671043749048\\
64.625	0.22476	-22.1002543758204\\
64.625	0.22842	-23.4801550865616\\
64.625	0.23208	-24.9068065071286\\
64.625	0.23574	-26.3802086375212\\
64.625	0.2394	-27.9003614777396\\
64.625	0.24306	-29.4672650277837\\
64.625	0.24672	-31.0809192876534\\
64.625	0.25038	-32.7413242573489\\
64.625	0.25404	-34.4484799368701\\
64.625	0.2577	-36.202386326217\\
64.625	0.26136	-38.0030434253895\\
64.625	0.26502	-39.8504512343878\\
64.625	0.26868	-41.7446097532118\\
64.625	0.27234	-43.6855189818615\\
64.625	0.276	-45.6731789203369\\
65	0.093	-3.63230679029536\\
65	0.09666	-3.32164664853709\\
65	0.10032	-3.05773721660453\\
65	0.10398	-2.84057849449764\\
65	0.10764	-2.67017048221649\\
65	0.1113	-2.54651317976102\\
65	0.11496	-2.46960658713125\\
65	0.11862	-2.4394507043272\\
65	0.12228	-2.45604553134883\\
65	0.12594	-2.51939106819617\\
65	0.1296	-2.62948731486923\\
65	0.13326	-2.78633427136794\\
65	0.13692	-2.9899319376924\\
65	0.14058	-3.24028031384257\\
65	0.14424	-3.53737939981842\\
65	0.1479	-3.88122919561999\\
65	0.15156	-4.27182970124725\\
65	0.15522	-4.70918091670021\\
65	0.15888	-5.19328284197891\\
65	0.16254	-5.72413547708327\\
65	0.1662	-6.30173882201333\\
65	0.16986	-6.9260928767691\\
65	0.17352	-7.5971976413506\\
65	0.17718	-8.31505311575777\\
65	0.18084	-9.07965929999067\\
65	0.1845	-9.89101619404926\\
65	0.18816	-10.7491237979335\\
65	0.19182	-11.6539821116435\\
65	0.19548	-12.6055911351792\\
65	0.19914	-13.6039508685406\\
65	0.2028	-14.6490613117277\\
65	0.20646	-15.7409224647405\\
65	0.21012	-16.8795343275791\\
65	0.21378	-18.0648969002432\\
65	0.21744	-19.2970101827332\\
65	0.2211	-20.5758741750488\\
65	0.22476	-21.9014888771901\\
65	0.22842	-23.2738542891571\\
65	0.23208	-24.6929704109499\\
65	0.23574	-26.1588372425683\\
65	0.2394	-27.6714547840124\\
65	0.24306	-29.2308230352822\\
65	0.24672	-30.8369419963778\\
65	0.25038	-32.4898116672991\\
65	0.25404	-34.1894320480459\\
65	0.2577	-35.9358031386186\\
65	0.26136	-37.728924939017\\
65	0.26502	-39.568797449241\\
65	0.26868	-41.4554206692908\\
65	0.27234	-43.3887945991662\\
65	0.276	-45.3689192388674\\
65.375	0.093	-3.71237184699156\\
65.375	0.09666	-3.39417640645906\\
65.375	0.10032	-3.12273167575225\\
65.375	0.10398	-2.89803765487114\\
65.375	0.10764	-2.72009434381573\\
65.375	0.1113	-2.58890174258603\\
65.375	0.11496	-2.50445985118203\\
65.375	0.11862	-2.46676866960375\\
65.375	0.12228	-2.47582819785115\\
65.375	0.12594	-2.53163843592426\\
65.375	0.1296	-2.63419938382306\\
65.375	0.13326	-2.78351104154757\\
65.375	0.13692	-2.97957340909778\\
65.375	0.14058	-3.22238648647372\\
65.375	0.14424	-3.51195027367534\\
65.375	0.1479	-3.84826477070266\\
65.375	0.15156	-4.23132997755568\\
65.375	0.15522	-4.66114589423441\\
65.375	0.15888	-5.13771252073885\\
65.375	0.16254	-5.66102985706898\\
65.375	0.1662	-6.23109790322482\\
65.375	0.16986	-6.84791665920636\\
65.375	0.17352	-7.5114861250136\\
65.375	0.17718	-8.22180630064654\\
65.375	0.18084	-8.97887718610518\\
65.375	0.1845	-9.78269878138957\\
65.375	0.18816	-10.6332710864996\\
65.375	0.19182	-11.5305941014354\\
65.375	0.19548	-12.4746678261968\\
65.375	0.19914	-13.465492260784\\
65.375	0.2028	-14.5030674051969\\
65.375	0.20646	-15.5873932594354\\
65.375	0.21012	-16.7184698234997\\
65.375	0.21378	-17.8962970973897\\
65.375	0.21744	-19.1208750811054\\
65.375	0.2211	-20.3922037746468\\
65.375	0.22476	-21.7102831780138\\
65.375	0.22842	-23.0751132912066\\
65.375	0.23208	-24.4866941142251\\
65.375	0.23574	-25.9450256470693\\
65.375	0.2394	-27.4501078897392\\
65.375	0.24306	-29.0019408422348\\
65.375	0.24672	-30.6005245045561\\
65.375	0.25038	-32.2458588767031\\
65.375	0.25404	-33.9379439586758\\
65.375	0.2577	-35.6767797504742\\
65.375	0.26136	-37.4623662520983\\
65.375	0.26502	-39.2947034635481\\
65.375	0.26868	-41.1737913848237\\
65.375	0.27234	-43.0996300159249\\
65.375	0.276	-45.0722193568518\\
65.75	0.093	-3.79999670314171\\
65.75	0.09666	-3.47426596383496\\
65.75	0.10032	-3.19528593435392\\
65.75	0.10398	-2.96305661469858\\
65.75	0.10764	-2.77757800486894\\
65.75	0.1113	-2.63885010486498\\
65.75	0.11496	-2.54687291468678\\
65.75	0.11862	-2.50164643433422\\
65.75	0.12228	-2.50317066380739\\
65.75	0.12594	-2.55144560310627\\
65.75	0.1296	-2.64647125223084\\
65.75	0.13326	-2.78824761118112\\
65.75	0.13692	-2.9767746799571\\
65.75	0.14058	-3.21205245855878\\
65.75	0.14424	-3.49408094698617\\
65.75	0.1479	-3.82286014523926\\
65.75	0.15156	-4.19839005331806\\
65.75	0.15522	-4.62067067122256\\
65.75	0.15888	-5.08970199895277\\
65.75	0.16254	-5.60548403650867\\
65.75	0.1662	-6.16801678389025\\
65.75	0.16986	-6.77730024109756\\
65.75	0.17352	-7.43333440813057\\
65.75	0.17718	-8.13611928498928\\
65.75	0.18084	-8.88565487167369\\
65.75	0.1845	-9.68194116818383\\
65.75	0.18816	-10.5249781745196\\
65.75	0.19182	-11.4147658906812\\
65.75	0.19548	-12.3513043166684\\
65.75	0.19914	-13.3345934524813\\
65.75	0.2028	-14.36463329812\\
65.75	0.20646	-15.4414238535843\\
65.75	0.21012	-16.5649651188743\\
65.75	0.21378	-17.7352570939901\\
65.75	0.21744	-18.9522997789315\\
65.75	0.2211	-20.2160931736986\\
65.75	0.22476	-21.5266372782915\\
65.75	0.22842	-22.8839320927101\\
65.75	0.23208	-24.2879776169543\\
65.75	0.23574	-25.7387738510243\\
65.75	0.2394	-27.2363207949199\\
65.75	0.24306	-28.7806184486413\\
65.75	0.24672	-30.3716668121884\\
65.75	0.25038	-32.0094658855611\\
65.75	0.25404	-33.6940156687596\\
65.75	0.2577	-35.4253161617838\\
65.75	0.26136	-37.2033673646337\\
65.75	0.26502	-39.0281692773092\\
65.75	0.26868	-40.8997218998105\\
65.75	0.27234	-42.8180252321375\\
65.75	0.276	-44.7830792742902\\
66.125	0.093	-3.89518135874586\\
66.125	0.09666	-3.56191532066489\\
66.125	0.10032	-3.27539999240962\\
66.125	0.10398	-3.03563537398005\\
66.125	0.10764	-2.84262146537618\\
66.125	0.1113	-2.69635826659799\\
66.125	0.11496	-2.59684577764553\\
66.125	0.11862	-2.54408399851874\\
66.125	0.12228	-2.53807292921769\\
66.125	0.12594	-2.57881256974234\\
66.125	0.1296	-2.66630292009268\\
66.125	0.13326	-2.8005439802687\\
66.125	0.13692	-2.98153575027045\\
66.125	0.14058	-3.2092782300979\\
66.125	0.14424	-3.48377141975106\\
66.125	0.1479	-3.80501531922992\\
66.125	0.15156	-4.17300992853449\\
66.125	0.15522	-4.58775524766473\\
66.125	0.15888	-5.04925127662069\\
66.125	0.16254	-5.55749801540236\\
66.125	0.1662	-6.11249546400974\\
66.125	0.16986	-6.71424362244279\\
66.125	0.17352	-7.36274249070157\\
66.125	0.17718	-8.05799206878606\\
66.125	0.18084	-8.79999235669621\\
66.125	0.1845	-9.58874335443211\\
66.125	0.18816	-10.4242450619937\\
66.125	0.19182	-11.306497479381\\
66.125	0.19548	-12.235500606594\\
66.125	0.19914	-13.2112544436327\\
66.125	0.2028	-14.2337589904971\\
66.125	0.20646	-15.3030142471872\\
66.125	0.21012	-16.419020213703\\
66.125	0.21378	-17.5817768900445\\
66.125	0.21744	-18.7912842762117\\
66.125	0.2211	-20.0475423722046\\
66.125	0.22476	-21.3505511780232\\
66.125	0.22842	-22.7003106936675\\
66.125	0.23208	-24.0968209191375\\
66.125	0.23574	-25.5400818544333\\
66.125	0.2394	-27.0300934995547\\
66.125	0.24306	-28.5668558545018\\
66.125	0.24672	-30.1503689192747\\
66.125	0.25038	-31.7806326938732\\
66.125	0.25404	-33.4576471782974\\
66.125	0.2577	-35.1814123725474\\
66.125	0.26136	-36.951928276623\\
66.125	0.26502	-38.7691948905243\\
66.125	0.26868	-40.6332122142514\\
66.125	0.27234	-42.5439802478042\\
66.125	0.276	-44.5014989911826\\
66.5	0.093	-3.99792581380402\\
66.5	0.09666	-3.65712447694882\\
66.5	0.10032	-3.36307384991929\\
66.5	0.10398	-3.11577393271546\\
66.5	0.10764	-2.91522472533737\\
66.5	0.1113	-2.76142622778495\\
66.5	0.11496	-2.65437844005826\\
66.5	0.11862	-2.59408136215724\\
66.5	0.12228	-2.58053499408195\\
66.5	0.12594	-2.61373933583237\\
66.5	0.1296	-2.69369438740846\\
66.5	0.13326	-2.82040014881025\\
66.5	0.13692	-2.99385662003775\\
66.5	0.14058	-3.21406380109099\\
66.5	0.14424	-3.4810216919699\\
66.5	0.1479	-3.79473029267453\\
66.5	0.15156	-4.15518960320484\\
66.5	0.15522	-4.56239962356088\\
66.5	0.15888	-5.01636035374261\\
66.5	0.16254	-5.51707179375005\\
66.5	0.1662	-6.06453394358317\\
66.5	0.16986	-6.658746803242\\
66.5	0.17352	-7.29971037272654\\
66.5	0.17718	-7.9874246520368\\
66.5	0.18084	-8.72188964117272\\
66.5	0.1845	-9.50310534013437\\
66.5	0.18816	-10.3310717489217\\
66.5	0.19182	-11.2057888675348\\
66.5	0.19548	-12.1272566959735\\
66.5	0.19914	-13.095475234238\\
66.5	0.2028	-14.1104444823282\\
66.5	0.20646	-15.172164440244\\
66.5	0.21012	-16.2806351079856\\
66.5	0.21378	-17.4358564855529\\
66.5	0.21744	-18.6378285729458\\
66.5	0.2211	-19.8865513701645\\
66.5	0.22476	-21.1820248772089\\
66.5	0.22842	-22.524249094079\\
66.5	0.23208	-23.9132240207747\\
66.5	0.23574	-25.3489496572962\\
66.5	0.2394	-26.8314260036434\\
66.5	0.24306	-28.3606530598163\\
66.5	0.24672	-29.9366308258149\\
66.5	0.25038	-31.5593593016392\\
66.5	0.25404	-33.2288384872892\\
66.5	0.2577	-34.9450683827649\\
66.5	0.26136	-36.7080489880664\\
66.5	0.26502	-38.5177803031934\\
66.5	0.26868	-40.3742623281462\\
66.5	0.27234	-42.2774950629248\\
66.5	0.276	-44.227478507529\\
66.875	0.093	-4.10823006831611\\
66.875	0.09666	-3.75989343268667\\
66.875	0.10032	-3.45830750688292\\
66.875	0.10398	-3.20347229090486\\
66.875	0.10764	-2.99538778475254\\
66.875	0.1113	-2.83405398842589\\
66.875	0.11496	-2.71947090192494\\
66.875	0.11862	-2.65163852524972\\
66.875	0.12228	-2.63055685840018\\
66.875	0.12594	-2.65622590137634\\
66.875	0.1296	-2.72864565417823\\
66.875	0.13326	-2.84781611680576\\
66.875	0.13692	-3.01373728925905\\
66.875	0.14058	-3.22640917153804\\
66.875	0.14424	-3.48583176364272\\
66.875	0.1479	-3.79200506557312\\
66.875	0.15156	-4.1449290773292\\
66.875	0.15522	-4.54460379891098\\
66.875	0.15888	-4.99102923031848\\
66.875	0.16254	-5.48420537155167\\
66.875	0.1662	-6.02413222261056\\
66.875	0.16986	-6.61080978349516\\
66.875	0.17352	-7.24423805420547\\
66.875	0.17718	-7.92441703474147\\
66.875	0.18084	-8.65134672510317\\
66.875	0.1845	-9.42502712529061\\
66.875	0.18816	-10.2454582353037\\
66.875	0.19182	-11.1126400551425\\
66.875	0.19548	-12.0265725848071\\
66.875	0.19914	-12.9872558242973\\
66.875	0.2028	-13.9946897736132\\
66.875	0.20646	-15.0488744327548\\
66.875	0.21012	-16.1498098017222\\
66.875	0.21378	-17.2974958805152\\
66.875	0.21744	-18.4919326691339\\
66.875	0.2211	-19.7331201675784\\
66.875	0.22476	-21.0210583758485\\
66.875	0.22842	-22.3557472939444\\
66.875	0.23208	-23.7371869218659\\
66.875	0.23574	-25.1653772596132\\
66.875	0.2394	-26.6403183071861\\
66.875	0.24306	-28.1620100645848\\
66.875	0.24672	-29.7304525318091\\
66.875	0.25038	-31.3456457088592\\
66.875	0.25404	-33.007589595735\\
66.875	0.2577	-34.7162841924365\\
66.875	0.26136	-36.4717294989636\\
66.875	0.26502	-38.2739255153165\\
66.875	0.26868	-40.1228722414951\\
66.875	0.27234	-42.0185696774994\\
66.875	0.276	-43.9610178233293\\
67.25	0.093	-4.22609412228223\\
67.25	0.09666	-3.87022218787853\\
67.25	0.10032	-3.56110096330055\\
67.25	0.10398	-3.29873044854826\\
67.25	0.10764	-3.08311064362168\\
67.25	0.1113	-2.9142415485208\\
67.25	0.11496	-2.79212316324563\\
67.25	0.11862	-2.71675548779618\\
67.25	0.12228	-2.68813852217241\\
67.25	0.12594	-2.70627226637434\\
67.25	0.1296	-2.77115672040197\\
67.25	0.13326	-2.8827918842553\\
67.25	0.13692	-3.04117775793434\\
67.25	0.14058	-3.24631434143907\\
67.25	0.14424	-3.49820163476954\\
67.25	0.1479	-3.79683963792569\\
67.25	0.15156	-4.14222835090754\\
67.25	0.15522	-4.53436777371509\\
67.25	0.15888	-4.97325790634836\\
67.25	0.16254	-5.45889874880731\\
67.25	0.1662	-5.99129030109195\\
67.25	0.16986	-6.57043256320235\\
67.25	0.17352	-7.19632553513841\\
67.25	0.17718	-7.86896921690018\\
67.25	0.18084	-8.58836360848764\\
67.25	0.1845	-9.35450870990083\\
67.25	0.18816	-10.1674045211397\\
67.25	0.19182	-11.0270510422043\\
67.25	0.19548	-11.9334482730946\\
67.25	0.19914	-12.8865962138106\\
67.25	0.2028	-13.8864948643523\\
67.25	0.20646	-14.9331442247197\\
67.25	0.21012	-16.0265442949128\\
67.25	0.21378	-17.1666950749316\\
67.25	0.21744	-18.353596564776\\
67.25	0.2211	-19.5872487644462\\
67.25	0.22476	-20.8676516739421\\
67.25	0.22842	-22.1948052932638\\
67.25	0.23208	-23.5687096224111\\
67.25	0.23574	-24.9893646613841\\
67.25	0.2394	-26.4567704101828\\
67.25	0.24306	-27.9709268688072\\
67.25	0.24672	-29.5318340372574\\
67.25	0.25038	-31.1394919155332\\
67.25	0.25404	-32.7939005036347\\
67.25	0.2577	-34.495059801562\\
67.25	0.26136	-36.2429698093149\\
67.25	0.26502	-38.0376305268935\\
67.25	0.26868	-39.8790419542979\\
67.25	0.27234	-41.7672040915279\\
67.25	0.276	-43.7021169385836\\
67.625	0.093	-4.35151797570227\\
67.625	0.09666	-3.98811074252438\\
67.625	0.10032	-3.67145421917214\\
67.625	0.10398	-3.40154840564562\\
67.625	0.10764	-3.17839330194481\\
67.625	0.1113	-3.0019889080697\\
67.625	0.11496	-2.8723352240203\\
67.625	0.11862	-2.78943224979656\\
67.625	0.12228	-2.75327998539859\\
67.625	0.12594	-2.76387843082626\\
67.625	0.1296	-2.82122758607966\\
67.625	0.13326	-2.92532745115877\\
67.625	0.13692	-3.07617802606357\\
67.625	0.14058	-3.27377931079408\\
67.625	0.14424	-3.51813130535029\\
67.625	0.1479	-3.80923400973221\\
67.625	0.15156	-4.14708742393983\\
67.625	0.15522	-4.53169154797315\\
67.625	0.15888	-4.96304638183219\\
67.625	0.16254	-5.44115192551689\\
67.625	0.1662	-5.96600817902733\\
67.625	0.16986	-6.53761514236346\\
67.625	0.17352	-7.15597281552527\\
67.625	0.17718	-7.82108119851284\\
67.625	0.18084	-8.53294029132604\\
67.625	0.1845	-9.29155009396503\\
67.625	0.18816	-10.0969106064297\\
67.625	0.19182	-10.94902182872\\
67.625	0.19548	-11.8478837608361\\
67.625	0.19914	-12.7934964027778\\
67.625	0.2028	-13.7858597545453\\
67.625	0.20646	-14.8249738161384\\
67.625	0.21012	-15.9108385875573\\
67.625	0.21378	-17.0434540688019\\
67.625	0.21744	-18.2228202598721\\
67.625	0.2211	-19.4489371607681\\
67.625	0.22476	-20.7218047714897\\
67.625	0.22842	-22.0414230920371\\
67.625	0.23208	-23.4077921224102\\
67.625	0.23574	-24.820911862609\\
67.625	0.2394	-26.2807823126335\\
67.625	0.24306	-27.7874034724837\\
67.625	0.24672	-29.3407753421596\\
67.625	0.25038	-30.9408979216612\\
67.625	0.25404	-32.5877712109884\\
67.625	0.2577	-34.2813952101415\\
67.625	0.26136	-36.0217699191201\\
67.625	0.26502	-37.8088953379246\\
67.625	0.26868	-39.6427714665547\\
67.625	0.27234	-41.5233983050105\\
67.625	0.276	-43.450775853292\\
68	0.093	-4.48450162857636\\
68	0.09666	-4.11355909662421\\
68	0.10032	-3.78936727449774\\
68	0.10398	-3.51192616219699\\
68	0.10764	-3.28123575972195\\
68	0.1113	-3.09729606707259\\
68	0.11496	-2.96010708424895\\
68	0.11862	-2.86966881125099\\
68	0.12228	-2.82598124807876\\
68	0.12594	-2.82904439473223\\
68	0.1296	-2.8788582512114\\
68	0.13326	-2.97542281751625\\
68	0.13692	-3.1187380936468\\
68	0.14058	-3.3088040796031\\
68	0.14424	-3.54562077538509\\
68	0.1479	-3.82918818099274\\
68	0.15156	-4.15950629642614\\
68	0.15522	-4.53657512168523\\
68	0.15888	-4.96039465677001\\
68	0.16254	-5.43096490168051\\
68	0.1662	-5.94828585641669\\
68	0.16986	-6.5123575209786\\
68	0.17352	-7.12317989536617\\
68	0.17718	-7.78075297957948\\
68	0.18084	-8.48507677361849\\
68	0.1845	-9.23615127748322\\
68	0.18816	-10.0339764911736\\
68	0.19182	-10.8785524146897\\
68	0.19548	-11.7698790480315\\
68	0.19914	-12.7079563911991\\
68	0.2028	-13.6927844441923\\
68	0.20646	-14.7243632070112\\
68	0.21012	-15.8026926796558\\
68	0.21378	-16.9277728621262\\
68	0.21744	-18.0996037544222\\
68	0.2211	-19.3181853565439\\
68	0.22476	-20.5835176684914\\
68	0.22842	-21.8956006902645\\
68	0.23208	-23.2544344218633\\
68	0.23574	-24.6600188632879\\
68	0.2394	-26.1123540145382\\
68	0.24306	-27.6114398756141\\
68	0.24672	-29.1572764465157\\
68	0.25038	-30.7498637272431\\
68	0.25404	-32.3892017177962\\
68	0.2577	-34.0752904181749\\
68	0.26136	-35.8081298283794\\
68	0.26502	-37.5877199484096\\
68	0.26868	-39.4140607782654\\
68	0.27234	-41.287152317947\\
68	0.276	-43.2069945674543\\
68.375	0.093	-4.62504508090443\\
68.375	0.09666	-4.24656725017802\\
68.375	0.10032	-3.91484012927733\\
68.375	0.10398	-3.62986371820235\\
68.375	0.10764	-3.39163801695305\\
68.375	0.1113	-3.20016302552949\\
68.375	0.11496	-3.0554387439316\\
68.375	0.11862	-2.95746517215943\\
68.375	0.12228	-2.90624231021294\\
68.375	0.12594	-2.90177015809219\\
68.375	0.1296	-2.9440487157971\\
68.375	0.13326	-3.03307798332772\\
68.375	0.13692	-3.16885796068406\\
68.375	0.14058	-3.35138864786611\\
68.375	0.14424	-3.58067004487384\\
68.375	0.1479	-3.8567021517073\\
68.375	0.15156	-4.17948496836643\\
68.375	0.15522	-4.5490184948513\\
68.375	0.15888	-4.96530273116185\\
68.375	0.16254	-5.42833767729812\\
68.375	0.1662	-5.93812333326007\\
68.375	0.16986	-6.49465969904772\\
68.375	0.17352	-7.09794677466106\\
68.375	0.17718	-7.74798456010014\\
68.375	0.18084	-8.44477305536492\\
68.375	0.1845	-9.18831226045539\\
68.375	0.18816	-9.97860217537156\\
68.375	0.19182	-10.8156428001134\\
68.375	0.19548	-11.699434134681\\
68.375	0.19914	-12.6299761790743\\
68.375	0.2028	-13.6072689332933\\
68.375	0.20646	-14.631312397338\\
68.375	0.21012	-15.7021065712084\\
68.375	0.21378	-16.8196514549044\\
68.375	0.21744	-17.9839470484262\\
68.375	0.2211	-19.1949933517738\\
68.375	0.22476	-20.4527903649469\\
68.375	0.22842	-21.7573380879459\\
68.375	0.23208	-23.1086365207705\\
68.375	0.23574	-24.5066856634208\\
68.375	0.2394	-25.9514855158968\\
68.375	0.24306	-27.4430360781985\\
68.375	0.24672	-28.9813373503259\\
68.375	0.25038	-30.5663893322791\\
68.375	0.25404	-32.1981920240579\\
68.375	0.2577	-33.8767454256624\\
68.375	0.26136	-35.6020495370927\\
68.375	0.26502	-37.3741043583486\\
68.375	0.26868	-39.1929098894302\\
68.375	0.27234	-41.0584661303376\\
68.375	0.276	-42.9707730810706\\
68.75	0.093	-4.77314833268644\\
68.75	0.09666	-4.3871352031858\\
68.75	0.10032	-4.04787278351087\\
68.75	0.10398	-3.75536107366164\\
68.75	0.10764	-3.50960007363814\\
68.75	0.1113	-3.31058978344032\\
68.75	0.11496	-3.1583302030682\\
68.75	0.11862	-3.0528213325218\\
68.75	0.12228	-2.99406317180108\\
68.75	0.12594	-2.98205572090607\\
68.75	0.1296	-3.01679897983678\\
68.75	0.13326	-3.09829294859314\\
68.75	0.13692	-3.22653762717526\\
68.75	0.14058	-3.40153301558305\\
68.75	0.14424	-3.62327911381658\\
68.75	0.1479	-3.89177592187578\\
68.75	0.15156	-4.20702343976068\\
68.75	0.15522	-4.56902166747132\\
68.75	0.15888	-4.97777060500761\\
68.75	0.16254	-5.43327025236965\\
68.75	0.1662	-5.93552060955734\\
68.75	0.16986	-6.48452167657079\\
68.75	0.17352	-7.08027345340991\\
68.75	0.17718	-7.72277594007473\\
68.75	0.18084	-8.41202913656528\\
68.75	0.1845	-9.14803304288152\\
68.75	0.18816	-9.93078765902346\\
68.75	0.19182	-10.7602929849911\\
68.75	0.19548	-11.6365490207844\\
68.75	0.19914	-12.5595557664035\\
68.75	0.2028	-13.5293132218482\\
68.75	0.20646	-14.5458213871187\\
68.75	0.21012	-15.6090802622148\\
68.75	0.21378	-16.7190898471367\\
68.75	0.21744	-17.8758501418843\\
68.75	0.2211	-19.0793611464575\\
68.75	0.22476	-20.3296228608565\\
68.75	0.22842	-21.6266352850812\\
68.75	0.23208	-22.9703984191315\\
68.75	0.23574	-24.3609122630077\\
68.75	0.2394	-25.7981768167094\\
68.75	0.24306	-27.2821920802369\\
68.75	0.24672	-28.8129580535901\\
68.75	0.25038	-30.390474736769\\
68.75	0.25404	-32.0147421297736\\
68.75	0.2577	-33.6857602326039\\
68.75	0.26136	-35.4035290452599\\
68.75	0.26502	-37.1680485677416\\
68.75	0.26868	-38.9793188000489\\
68.75	0.27234	-40.8373397421821\\
68.75	0.276	-42.7421113941408\\
69.125	0.093	-4.92881138392247\\
69.125	0.09666	-4.5352629556476\\
69.125	0.10032	-4.18846523719845\\
69.125	0.10398	-3.88841822857498\\
69.125	0.10764	-3.63512192977723\\
69.125	0.1113	-3.42857634080518\\
69.125	0.11496	-3.26878146165883\\
69.125	0.11862	-3.15573729233817\\
69.125	0.12228	-3.08944383284323\\
69.125	0.12594	-3.06990108317398\\
69.125	0.1296	-3.09710904333044\\
69.125	0.13326	-3.1710677133126\\
69.125	0.13692	-3.29177709312046\\
69.125	0.14058	-3.45923718275402\\
69.125	0.14424	-3.67344798221329\\
69.125	0.1479	-3.93440949149829\\
69.125	0.15156	-4.24212171060896\\
69.125	0.15522	-4.59658463954534\\
69.125	0.15888	-4.99779827830741\\
69.125	0.16254	-5.44576262689522\\
69.125	0.1662	-5.94047768530868\\
69.125	0.16986	-6.48194345354787\\
69.125	0.17352	-7.07015993161276\\
69.125	0.17718	-7.70512711950335\\
69.125	0.18084	-8.38684501721967\\
69.125	0.1845	-9.11531362476168\\
69.125	0.18816	-9.89053294212936\\
69.125	0.19182	-10.7125029693228\\
69.125	0.19548	-11.5812237063419\\
69.125	0.19914	-12.4966951531867\\
69.125	0.2028	-13.4589173098572\\
69.125	0.20646	-14.4678901763534\\
69.125	0.21012	-15.5236137526753\\
69.125	0.21378	-16.626088038823\\
69.125	0.21744	-17.7753130347963\\
69.125	0.2211	-18.9712887405953\\
69.125	0.22476	-20.2140151562201\\
69.125	0.22842	-21.5034922816705\\
69.125	0.23208	-22.8397201169466\\
69.125	0.23574	-24.2226986620485\\
69.125	0.2394	-25.652427916976\\
69.125	0.24306	-27.1289078817292\\
69.125	0.24672	-28.6521385563082\\
69.125	0.25038	-30.2221199407129\\
69.125	0.25404	-31.8388520349432\\
69.125	0.2577	-33.5023348389993\\
69.125	0.26136	-35.212568352881\\
69.125	0.26502	-36.9695525765885\\
69.125	0.26868	-38.7732875101217\\
69.125	0.27234	-40.6237731534806\\
69.125	0.276	-42.5210095066651\\
69.5	0.093	-5.09203423461245\\
69.5	0.09666	-4.69095050756335\\
69.5	0.10032	-4.33661749033996\\
69.5	0.10398	-4.02903518294224\\
69.5	0.10764	-3.76820358537026\\
69.5	0.1113	-3.55412269762398\\
69.5	0.11496	-3.3867925197034\\
69.5	0.11862	-3.26621305160852\\
69.5	0.12228	-3.19238429333931\\
69.5	0.12594	-3.16530624489584\\
69.5	0.1296	-3.18497890627807\\
69.5	0.13326	-3.25140227748597\\
69.5	0.13692	-3.36457635851963\\
69.5	0.14058	-3.52450114937896\\
69.5	0.14424	-3.731176650064\\
69.5	0.1479	-3.98460286057474\\
69.5	0.15156	-4.28477978091116\\
69.5	0.15522	-4.63170741107334\\
69.5	0.15888	-5.02538575106117\\
69.5	0.16254	-5.46581480087472\\
69.5	0.1662	-5.95299456051396\\
69.5	0.16986	-6.48692502997892\\
69.5	0.17352	-7.06760620926958\\
69.5	0.17718	-7.69503809838594\\
69.5	0.18084	-8.36922069732798\\
69.5	0.1845	-9.09015400609579\\
69.5	0.18816	-9.85783802468924\\
69.5	0.19182	-10.6722727531084\\
69.5	0.19548	-11.5334581913533\\
69.5	0.19914	-12.4413943394239\\
69.5	0.2028	-13.3960811973202\\
69.5	0.20646	-14.3975187650422\\
69.5	0.21012	-15.4457070425898\\
69.5	0.21378	-16.5406460299632\\
69.5	0.21744	-17.6823357271623\\
69.5	0.2211	-18.8707761341871\\
69.5	0.22476	-20.1059672510376\\
69.5	0.22842	-21.3879090777138\\
69.5	0.23208	-22.7166016142157\\
69.5	0.23574	-24.0920448605433\\
69.5	0.2394	-25.5142388166966\\
69.5	0.24306	-26.9831834826756\\
69.5	0.24672	-28.4988788584803\\
69.5	0.25038	-30.0613249441107\\
69.5	0.25404	-31.6705217395669\\
69.5	0.2577	-33.3264692448487\\
69.5	0.26136	-35.0291674599562\\
69.5	0.26502	-36.7786163848894\\
69.5	0.26868	-38.5748160196484\\
69.5	0.27234	-40.417766364233\\
69.5	0.276	-42.3074674186433\\
69.875	0.093	-5.26281688475644\\
69.875	0.09666	-4.85419785893308\\
69.875	0.10032	-4.49232954293544\\
69.875	0.10398	-4.17721193676352\\
69.875	0.10764	-3.9088450404173\\
69.875	0.1113	-3.68722885389679\\
69.875	0.11496	-3.51236337720196\\
69.875	0.11862	-3.38424861033285\\
69.875	0.12228	-3.30288455328941\\
69.875	0.12594	-3.26827120607171\\
69.875	0.1296	-3.28040856867971\\
69.875	0.13326	-3.33929664111338\\
69.875	0.13692	-3.44493542337275\\
69.875	0.14058	-3.59732491545788\\
69.875	0.14424	-3.7964651173687\\
69.875	0.1479	-4.04235602910518\\
69.875	0.15156	-4.3349976506674\\
69.875	0.15522	-4.67438998205529\\
69.875	0.15888	-5.06053302326892\\
69.875	0.16254	-5.49342677430825\\
69.875	0.1662	-5.97307123517322\\
69.875	0.16986	-6.49946640586396\\
69.875	0.17352	-7.07261228638038\\
69.875	0.17718	-7.69250887672252\\
69.875	0.18084	-8.35915617689032\\
69.875	0.1845	-9.07255418688388\\
69.875	0.18816	-9.8327029067031\\
69.875	0.19182	-10.639602336348\\
69.875	0.19548	-11.4932524758187\\
69.875	0.19914	-12.393653325115\\
69.875	0.2028	-13.3408048842371\\
69.875	0.20646	-14.3347071531848\\
69.875	0.21012	-15.3753601319583\\
69.875	0.21378	-16.4627638205574\\
69.875	0.21744	-17.5969182189823\\
69.875	0.2211	-18.7778233272328\\
69.875	0.22476	-20.0054791453091\\
69.875	0.22842	-21.2798856732111\\
69.875	0.23208	-22.6010429109387\\
69.875	0.23574	-23.9689508584921\\
69.875	0.2394	-25.3836095158712\\
69.875	0.24306	-26.845018883076\\
69.875	0.24672	-28.3531789601064\\
69.875	0.25038	-29.9080897469626\\
69.875	0.25404	-31.5097512436445\\
69.875	0.2577	-33.1581634501521\\
69.875	0.26136	-34.8533263664854\\
69.875	0.26502	-36.5952399926444\\
69.875	0.26868	-38.3839043286291\\
69.875	0.27234	-40.2193193744395\\
69.875	0.276	-42.1014851300756\\
70.25	0.093	-5.44115933435439\\
70.25	0.09666	-5.0250050097568\\
70.25	0.10032	-4.65560139498493\\
70.25	0.10398	-4.33294849003878\\
70.25	0.10764	-4.05704629491834\\
70.25	0.1113	-3.82789480962357\\
70.25	0.11496	-3.6454940341545\\
70.25	0.11862	-3.50984396851113\\
70.25	0.12228	-3.4209446126935\\
70.25	0.12594	-3.37879596670157\\
70.25	0.1296	-3.38339803053531\\
70.25	0.13326	-3.43475080419475\\
70.25	0.13692	-3.5328542876799\\
70.25	0.14058	-3.67770848099077\\
70.25	0.14424	-3.86931338412735\\
70.25	0.1479	-4.1076689970896\\
70.25	0.15156	-4.39277531987759\\
70.25	0.15522	-4.72463235249126\\
70.25	0.15888	-5.10324009493063\\
70.25	0.16254	-5.52859854719573\\
70.25	0.1662	-6.00070770928647\\
70.25	0.16986	-6.51956758120298\\
70.25	0.17352	-7.08517816294518\\
70.25	0.17718	-7.69753945451306\\
70.25	0.18084	-8.35665145590663\\
70.25	0.1845	-9.06251416712595\\
70.25	0.18816	-9.81512758817095\\
70.25	0.19182	-10.6144917190417\\
70.25	0.19548	-11.4606065597381\\
70.25	0.19914	-12.3534721102602\\
70.25	0.2028	-13.293088370608\\
70.25	0.20646	-14.2794553407815\\
70.25	0.21012	-15.3125730207807\\
70.25	0.21378	-16.3924414106056\\
70.25	0.21744	-17.5190605102562\\
70.25	0.2211	-18.6924303197326\\
70.25	0.22476	-19.9125508390346\\
70.25	0.22842	-21.1794220681623\\
70.25	0.23208	-22.4930440071158\\
70.25	0.23574	-23.8534166558949\\
70.25	0.2394	-25.2605400144997\\
70.25	0.24306	-26.7144140829303\\
70.25	0.24672	-28.2150388611865\\
70.25	0.25038	-29.7624143492685\\
70.25	0.25404	-31.3565405471761\\
70.25	0.2577	-32.9974174549095\\
70.25	0.26136	-34.6850450724685\\
70.25	0.26502	-36.4194233998533\\
70.25	0.26868	-38.2005524370637\\
70.25	0.27234	-40.0284321840999\\
70.25	0.276	-41.9030626409618\\
70.625	0.093	-5.62706158340632\\
70.625	0.09666	-5.20337196003454\\
70.625	0.10032	-4.82643304648841\\
70.625	0.10398	-4.49624484276803\\
70.625	0.10764	-4.21280734887333\\
70.625	0.1113	-3.97612056480433\\
70.625	0.11496	-3.78618449056103\\
70.625	0.11862	-3.64299912614344\\
70.625	0.12228	-3.54656447155157\\
70.625	0.12594	-3.49688052678539\\
70.625	0.1296	-3.49394729184489\\
70.625	0.13326	-3.53776476673011\\
70.625	0.13692	-3.62833295144102\\
70.625	0.14058	-3.76565184597766\\
70.625	0.14424	-3.94972145033999\\
70.625	0.1479	-4.18054176452804\\
70.625	0.15156	-4.45811278854178\\
70.625	0.15522	-4.78243452238121\\
70.625	0.15888	-5.15350696604636\\
70.625	0.16254	-5.57133011953719\\
70.625	0.1662	-6.03590398285374\\
70.625	0.16986	-6.54722855599599\\
70.625	0.17352	-7.10530383896396\\
70.625	0.17718	-7.71012983175758\\
70.625	0.18084	-8.36170653437695\\
70.625	0.1845	-9.06003394682202\\
70.625	0.18816	-9.80511206909278\\
70.625	0.19182	-10.5969409011892\\
70.625	0.19548	-11.4355204431114\\
70.625	0.19914	-12.3208506948593\\
70.625	0.2028	-13.2529316564329\\
70.625	0.20646	-14.2317633278321\\
70.625	0.21012	-15.2573457090571\\
70.625	0.21378	-16.3296788001078\\
70.625	0.21744	-17.4487626009842\\
70.625	0.2211	-18.6145971116863\\
70.625	0.22476	-19.8271823322141\\
70.625	0.22842	-21.0865182625676\\
70.625	0.23208	-22.3926049027468\\
70.625	0.23574	-23.7454422527517\\
70.625	0.2394	-25.1450303125823\\
70.625	0.24306	-26.5913690822385\\
70.625	0.24672	-28.0844585617206\\
70.625	0.25038	-29.6242987510283\\
70.625	0.25404	-31.2108896501617\\
70.625	0.2577	-32.8442312591209\\
70.625	0.26136	-34.5243235779056\\
70.625	0.26502	-36.2511666065162\\
70.625	0.26868	-38.0247603449524\\
70.625	0.27234	-39.8451047932143\\
70.625	0.276	-41.712199951302\\
71	0.093	-5.82052363191224\\
71	0.09666	-5.3892987097662\\
71	0.10032	-5.00482449744587\\
71	0.10398	-4.66710099495123\\
71	0.10764	-4.3761282022823\\
71	0.1113	-4.13190611943908\\
71	0.11496	-3.93443474642155\\
71	0.11862	-3.78371408322973\\
71	0.12228	-3.6797441298636\\
71	0.12594	-3.62252488632319\\
71	0.1296	-3.61205635260847\\
71	0.13326	-3.64833852871942\\
71	0.13692	-3.73137141465611\\
71	0.14058	-3.86115501041852\\
71	0.14424	-4.03768931600662\\
71	0.1479	-4.26097433142041\\
71	0.15156	-4.53101005665992\\
71	0.15522	-4.84779649172512\\
71	0.15888	-5.21133363661604\\
71	0.16254	-5.62162149133265\\
71	0.1662	-6.07866005587493\\
71	0.16986	-6.58244933024295\\
71	0.17352	-7.13298931443666\\
71	0.17718	-7.73028000845608\\
71	0.18084	-8.3743214123012\\
71	0.1845	-9.06511352597204\\
71	0.18816	-9.80265634946857\\
71	0.19182	-10.5869498827908\\
71	0.19548	-11.4179941259387\\
71	0.19914	-12.2957890789124\\
71	0.2028	-13.2203347417117\\
71	0.20646	-14.1916311143368\\
71	0.21012	-15.2096781967875\\
71	0.21378	-16.274475989064\\
71	0.21744	-17.3860244911661\\
71	0.2211	-18.544323703094\\
71	0.22476	-19.7493736248475\\
71	0.22842	-21.0011742564268\\
71	0.23208	-22.2997255978317\\
71	0.23574	-23.6450276490624\\
71	0.2394	-25.0370804101188\\
71	0.24306	-26.4758838810008\\
71	0.24672	-27.9614380617086\\
71	0.25038	-29.4937429522421\\
71	0.25404	-31.0727985526013\\
71	0.2577	-32.6986048627862\\
71	0.26136	-34.3711618827967\\
71	0.26502	-36.090469612633\\
71	0.26868	-37.856528052295\\
71	0.27234	-39.6693372017827\\
71	0.276	-41.5288970610961\\
71.375	0.093	-6.02154547987212\\
71.375	0.09666	-5.58278525895185\\
71.375	0.10032	-5.19077574785729\\
71.375	0.10398	-4.84551694658839\\
71.375	0.10764	-4.54700885514524\\
71.375	0.1113	-4.29525147352778\\
71.375	0.11496	-4.09024480173603\\
71.375	0.11862	-3.93198883976997\\
71.375	0.12228	-3.82048358762959\\
71.375	0.12594	-3.75572904531495\\
71.375	0.1296	-3.737725212826\\
71.375	0.13326	-3.76647209016273\\
71.375	0.13692	-3.84196967732518\\
71.375	0.14058	-3.96421797431336\\
71.375	0.14424	-4.13321698112721\\
71.375	0.1479	-4.34896669776677\\
71.375	0.15156	-4.61146712423204\\
71.375	0.15522	-4.92071826052302\\
71.375	0.15888	-5.27672010663968\\
71.375	0.16254	-5.67947266258206\\
71.375	0.1662	-6.12897592835012\\
71.375	0.16986	-6.62522990394393\\
71.375	0.17352	-7.16823458936339\\
71.375	0.17718	-7.75798998460858\\
71.375	0.18084	-8.39449608967946\\
71.375	0.1845	-9.07775290457604\\
71.375	0.18816	-9.80776042929835\\
71.375	0.19182	-10.5845186638464\\
71.375	0.19548	-11.40802760822\\
71.375	0.19914	-12.2782872624195\\
71.375	0.2028	-13.1952976264446\\
71.375	0.20646	-14.1590587002954\\
71.375	0.21012	-15.1695704839719\\
71.375	0.21378	-16.2268329774741\\
71.375	0.21744	-17.330846180802\\
71.375	0.2211	-18.4816100939556\\
71.375	0.22476	-19.6791247169349\\
71.375	0.22842	-20.92339004974\\
71.375	0.23208	-22.2144060923707\\
71.375	0.23574	-23.5521728448271\\
71.375	0.2394	-24.9366903071093\\
71.375	0.24306	-26.3679584792171\\
71.375	0.24672	-27.8459773611506\\
71.375	0.25038	-29.3707469529099\\
71.375	0.25404	-30.9422672544948\\
71.375	0.2577	-32.5605382659055\\
71.375	0.26136	-34.2255599871418\\
71.375	0.26502	-35.9373324182039\\
71.375	0.26868	-37.6958555590916\\
71.375	0.27234	-39.5011294098051\\
71.375	0.276	-41.3531539703442\\
71.75	0.093	-6.23012712728602\\
71.75	0.09666	-5.78383160759152\\
71.75	0.10032	-5.3842867977227\\
71.75	0.10398	-5.03149269767957\\
71.75	0.10764	-4.72544930746218\\
71.75	0.1113	-4.4661566270705\\
71.75	0.11496	-4.25361465650449\\
71.75	0.11862	-4.0878233957642\\
71.75	0.12228	-3.9687828448496\\
71.75	0.12594	-3.89649300376072\\
71.75	0.1296	-3.87095387249754\\
71.75	0.13326	-3.89216545106004\\
71.75	0.13692	-3.96012773944824\\
71.75	0.14058	-4.07484073766219\\
71.75	0.14424	-4.2363044457018\\
71.75	0.1479	-4.44451886356714\\
71.75	0.15156	-4.69948399125816\\
71.75	0.15522	-5.0011998287749\\
71.75	0.15888	-5.34966637611733\\
71.75	0.16254	-5.74488363328548\\
71.75	0.1662	-6.18685160027931\\
71.75	0.16986	-6.67557027709887\\
71.75	0.17352	-7.2110396637441\\
71.75	0.17718	-7.79325976021503\\
71.75	0.18084	-8.42223056651169\\
71.75	0.1845	-9.09795208263407\\
71.75	0.18816	-9.82042430858211\\
71.75	0.19182	-10.5896472443559\\
71.75	0.19548	-11.4056208899553\\
71.75	0.19914	-12.2683452453805\\
71.75	0.2028	-13.1778203106314\\
71.75	0.20646	-14.134046085708\\
71.75	0.21012	-15.1370225706102\\
71.75	0.21378	-16.1867497653382\\
71.75	0.21744	-17.2832276698919\\
71.75	0.2211	-18.4264562842713\\
71.75	0.22476	-19.6164356084763\\
71.75	0.22842	-20.8531656425072\\
71.75	0.23208	-22.1366463863636\\
71.75	0.23574	-23.4668778400459\\
71.75	0.2394	-24.8438600035537\\
71.75	0.24306	-26.2675928768873\\
71.75	0.24672	-27.7380764600467\\
71.75	0.25038	-29.2553107530316\\
71.75	0.25404	-30.8192957558423\\
71.75	0.2577	-32.4300314684788\\
71.75	0.26136	-34.0875178909409\\
71.75	0.26502	-35.7917550232287\\
71.75	0.26868	-37.5427428653422\\
71.75	0.27234	-39.3404814172814\\
71.75	0.276	-41.1849706790464\\
72.125	0.093	-6.44626857415387\\
72.125	0.09666	-5.99243775568511\\
72.125	0.10032	-5.58535764704209\\
72.125	0.10398	-5.22502824822473\\
72.125	0.10764	-4.91144955923312\\
72.125	0.1113	-4.64462158006718\\
72.125	0.11496	-4.42454431072694\\
72.125	0.11862	-4.25121775121239\\
72.125	0.12228	-4.12464190152359\\
72.125	0.12594	-4.04481676166048\\
72.125	0.1296	-4.01174233162304\\
72.125	0.13326	-4.02541861141131\\
72.125	0.13692	-4.08584560102528\\
72.125	0.14058	-4.19302330046498\\
72.125	0.14424	-4.34695170973036\\
72.125	0.1479	-4.54763082882147\\
72.125	0.15156	-4.79506065773825\\
72.125	0.15522	-5.08924119648077\\
72.125	0.15888	-5.43017244504897\\
72.125	0.16254	-5.81785440344287\\
72.125	0.1662	-6.25228707166247\\
72.125	0.16986	-6.73347044970777\\
72.125	0.17352	-7.26140453757879\\
72.125	0.17718	-7.8360893352755\\
72.125	0.18084	-8.45752484279792\\
72.125	0.1845	-9.12571106014605\\
72.125	0.18816	-9.84064798731986\\
72.125	0.19182	-10.6023356243194\\
72.125	0.19548	-11.4107739711446\\
72.125	0.19914	-12.2659630277956\\
72.125	0.2028	-13.1679027942722\\
72.125	0.20646	-14.1165932705745\\
72.125	0.21012	-15.1120344567026\\
72.125	0.21378	-16.1542263526563\\
72.125	0.21744	-17.2431689584357\\
72.125	0.2211	-18.3788622740409\\
72.125	0.22476	-19.5613062994717\\
72.125	0.22842	-20.7905010347283\\
72.125	0.23208	-22.0664464798106\\
72.125	0.23574	-23.3891426347185\\
72.125	0.2394	-24.7585894994522\\
72.125	0.24306	-26.1747870740115\\
72.125	0.24672	-27.6377353583966\\
72.125	0.25038	-29.1474343526074\\
72.125	0.25404	-30.7038840566439\\
72.125	0.2577	-32.3070844705061\\
72.125	0.26136	-33.9570355941939\\
72.125	0.26502	-35.6537374277075\\
72.125	0.26868	-37.3971899710468\\
72.125	0.27234	-39.1873932242118\\
72.125	0.276	-41.0243471872025\\
72.5	0.093	-6.66996982047576\\
72.5	0.09666	-6.20860370323278\\
72.5	0.10032	-5.7939882958155\\
72.5	0.10398	-5.42612359822391\\
72.5	0.10764	-5.10500961045804\\
72.5	0.1113	-4.83064633251787\\
72.5	0.11496	-4.60303376440343\\
72.5	0.11862	-4.42217190611466\\
72.5	0.12228	-4.28806075765159\\
72.5	0.12594	-4.20070031901423\\
72.5	0.1296	-4.16009059020256\\
72.5	0.13326	-4.1662315712166\\
72.5	0.13692	-4.21912326205634\\
72.5	0.14058	-4.31876566272181\\
72.5	0.14424	-4.46515877321296\\
72.5	0.1479	-4.65830259352984\\
72.5	0.15156	-4.89819712367237\\
72.5	0.15522	-5.18484236364066\\
72.5	0.15888	-5.5182383134346\\
72.5	0.16254	-5.89838497305429\\
72.5	0.1662	-6.32528234249963\\
72.5	0.16986	-6.79893042177073\\
72.5	0.17352	-7.3193292108675\\
72.5	0.17718	-7.88647870978998\\
72.5	0.18084	-8.50037891853817\\
72.5	0.1845	-9.16102983711204\\
72.5	0.18816	-9.86843146551166\\
72.5	0.19182	-10.6225838037369\\
72.5	0.19548	-11.4234868517879\\
72.5	0.19914	-12.2711406096646\\
72.5	0.2028	-13.165545077367\\
72.5	0.20646	-14.1067002548951\\
72.5	0.21012	-15.0946061422489\\
72.5	0.21378	-16.1292627394284\\
72.5	0.21744	-17.2106700464336\\
72.5	0.2211	-18.3388280632645\\
72.5	0.22476	-19.5137367899212\\
72.5	0.22842	-20.7353962264035\\
72.5	0.23208	-22.0038063727115\\
72.5	0.23574	-23.3189672288453\\
72.5	0.2394	-24.6808787948047\\
72.5	0.24306	-26.0895410705898\\
72.5	0.24672	-27.5449540562006\\
72.5	0.25038	-29.0471177516372\\
72.5	0.25404	-30.5960321568994\\
72.5	0.2577	-32.1916972719874\\
72.5	0.26136	-33.834113096901\\
72.5	0.26502	-35.5232796316403\\
72.5	0.26868	-37.2591968762054\\
72.5	0.27234	-39.0418648305961\\
72.5	0.276	-40.8712834948126\\
72.875	0.093	-6.90123086625156\\
72.875	0.09666	-6.43232945023434\\
72.875	0.10032	-6.01017874404284\\
72.875	0.10398	-5.63477874767702\\
72.875	0.10764	-5.30612946113692\\
72.875	0.1113	-5.02423088442252\\
72.875	0.11496	-4.78908301753382\\
72.875	0.11862	-4.60068586047082\\
72.875	0.12228	-4.45903941323352\\
72.875	0.12594	-4.36414367582193\\
72.875	0.1296	-4.31599864823604\\
72.875	0.13326	-4.31460433047582\\
72.875	0.13692	-4.35996072254133\\
72.875	0.14058	-4.45206782443257\\
72.875	0.14424	-4.59092563614946\\
72.875	0.1479	-4.77653415769208\\
72.875	0.15156	-5.00889338906041\\
72.875	0.15522	-5.28800333025444\\
72.875	0.15888	-5.61386398127419\\
72.875	0.16254	-5.98647534211962\\
72.875	0.1662	-6.40583741279076\\
72.875	0.16986	-6.8719501932876\\
72.875	0.17352	-7.38481368361014\\
72.875	0.17718	-7.94442788375839\\
72.875	0.18084	-8.55079279373233\\
72.875	0.1845	-9.20390841353199\\
72.875	0.18816	-9.90377474315732\\
72.875	0.19182	-10.6503917826084\\
72.875	0.19548	-11.4437595318851\\
72.875	0.19914	-12.2838779909876\\
72.875	0.2028	-13.1707471599158\\
72.875	0.20646	-14.1043670386696\\
72.875	0.21012	-15.0847376272492\\
72.875	0.21378	-16.1118589256545\\
72.875	0.21744	-17.1857309338854\\
72.875	0.2211	-18.3063536519421\\
72.875	0.22476	-19.4737270798245\\
72.875	0.22842	-20.6878512175326\\
72.875	0.23208	-21.9487260650664\\
72.875	0.23574	-23.2563516224259\\
72.875	0.2394	-24.6107278896111\\
72.875	0.24306	-26.0118548666219\\
72.875	0.24672	-27.4597325534586\\
72.875	0.25038	-28.9543609501208\\
72.875	0.25404	-30.4957400566088\\
72.875	0.2577	-32.0838698729226\\
72.875	0.26136	-33.718750399062\\
72.875	0.26502	-35.4003816350271\\
72.875	0.26868	-37.1287635808179\\
72.875	0.27234	-38.9038962364344\\
72.875	0.276	-40.7257796018766\\
73.25	0.093	-7.1400517114814\\
73.25	0.09666	-6.66361499668995\\
73.25	0.10032	-6.23392899172419\\
73.25	0.10398	-5.85099369658414\\
73.25	0.10764	-5.51480911126981\\
73.25	0.1113	-5.22537523578115\\
73.25	0.11496	-4.98269207011823\\
73.25	0.11862	-4.786759614281\\
73.25	0.12228	-4.63757786826944\\
73.25	0.12594	-4.53514683208362\\
73.25	0.1296	-4.4794665057235\\
73.25	0.13326	-4.47053688918905\\
73.25	0.13692	-4.50835798248033\\
73.25	0.14058	-4.59292978559734\\
73.25	0.14424	-4.72425229854001\\
73.25	0.1479	-4.9023255213084\\
73.25	0.15156	-5.1271494539025\\
73.25	0.15522	-5.3987240963223\\
73.25	0.15888	-5.71704944856778\\
73.25	0.16254	-6.08212551063899\\
73.25	0.1662	-6.49395228253587\\
73.25	0.16986	-6.95252976425849\\
73.25	0.17352	-7.45785795580679\\
73.25	0.17718	-8.00993685718078\\
73.25	0.18084	-8.60876646838049\\
73.25	0.1845	-9.25434678940593\\
73.25	0.18816	-9.94667782025703\\
73.25	0.19182	-10.6857595609339\\
73.25	0.19548	-11.4715920114364\\
73.25	0.19914	-12.3041751717646\\
73.25	0.2028	-13.1835090419186\\
73.25	0.20646	-14.1095936218982\\
73.25	0.21012	-15.0824289117035\\
73.25	0.21378	-16.1020149113345\\
73.25	0.21744	-17.1683516207913\\
73.25	0.2211	-18.2814390400737\\
73.25	0.22476	-19.4412771691819\\
73.25	0.22842	-20.6478660081157\\
73.25	0.23208	-21.9012055568753\\
73.25	0.23574	-23.2012958154605\\
73.25	0.2394	-24.5481367838715\\
73.25	0.24306	-25.9417284621081\\
73.25	0.24672	-27.3820708501705\\
73.25	0.25038	-28.8691639480586\\
73.25	0.25404	-30.4030077557723\\
73.25	0.2577	-31.9836022733118\\
73.25	0.26136	-33.610947500677\\
73.25	0.26502	-35.2850434378679\\
73.25	0.26868	-37.0058900848845\\
73.25	0.27234	-38.7734874417267\\
73.25	0.276	-40.5878355083947\\
73.625	0.093	-7.38643235616519\\
73.625	0.09666	-6.90246034259952\\
73.625	0.10032	-6.46523903885953\\
73.625	0.10398	-6.07476844494525\\
73.625	0.10764	-5.73104856085666\\
73.625	0.1113	-5.43407938659378\\
73.625	0.11496	-5.18386092215662\\
73.625	0.11862	-4.98039316754516\\
73.625	0.12228	-4.82367612275938\\
73.625	0.12594	-4.71370978779933\\
73.625	0.1296	-4.65049416266494\\
73.625	0.13326	-4.6340292473563\\
73.625	0.13692	-4.66431504187332\\
73.625	0.14058	-4.74135154621607\\
73.625	0.14424	-4.86513876038454\\
73.625	0.1479	-5.03567668437867\\
73.625	0.15156	-5.25296531819854\\
73.625	0.15522	-5.51700466184408\\
73.625	0.15888	-5.82779471531537\\
73.625	0.16254	-6.18533547861232\\
73.625	0.1662	-6.58962695173497\\
73.625	0.16986	-7.04066913468336\\
73.625	0.17352	-7.53846202745743\\
73.625	0.17718	-8.08300563005719\\
73.625	0.18084	-8.67429994248265\\
73.625	0.1845	-9.31234496473385\\
73.625	0.18816	-9.99714069681072\\
73.625	0.19182	-10.7286871387133\\
73.625	0.19548	-11.5069842904416\\
73.625	0.19914	-12.3320321519956\\
73.625	0.2028	-13.2038307233753\\
73.625	0.20646	-14.1223800045807\\
73.625	0.21012	-15.0876799956118\\
73.625	0.21378	-16.0997306964686\\
73.625	0.21744	-17.1585321071511\\
73.625	0.2211	-18.2640842276593\\
73.625	0.22476	-19.4163870579932\\
73.625	0.22842	-20.6154405981528\\
73.625	0.23208	-21.8612448481381\\
73.625	0.23574	-23.1537998079492\\
73.625	0.2394	-24.4931054775859\\
73.625	0.24306	-25.8791618570483\\
73.625	0.24672	-27.3119689463364\\
73.625	0.25038	-28.7915267454503\\
73.625	0.25404	-30.3178352543898\\
73.625	0.2577	-31.8908944731551\\
73.625	0.26136	-33.510704401746\\
73.625	0.26502	-35.1772650401626\\
73.625	0.26868	-36.890576388405\\
73.625	0.27234	-38.650638446473\\
73.625	0.276	-40.4574512143667\\
74	0.093	-7.64037280030298\\
74	0.09666	-7.14886548796304\\
74	0.10032	-6.70410888544882\\
74	0.10398	-6.30610299276032\\
74	0.10764	-5.9548478098975\\
74	0.1113	-5.65034333686038\\
74	0.11496	-5.39258957364897\\
74	0.11862	-5.18158652026328\\
74	0.12228	-5.01733417670327\\
74	0.12594	-4.89983254296899\\
74	0.1296	-4.82908161906038\\
74	0.13326	-4.80508140497748\\
74	0.13692	-4.82783190072027\\
74	0.14058	-4.89733310628879\\
74	0.14424	-5.013585021683\\
74	0.1479	-5.17658764690293\\
74	0.15156	-5.38634098194854\\
74	0.15522	-5.64284502681988\\
74	0.15888	-5.94609978151691\\
74	0.16254	-6.29610524603963\\
74	0.1662	-6.69286142038806\\
74	0.16986	-7.13636830456218\\
74	0.17352	-7.626625898562\\
74	0.17718	-8.16363420238756\\
74	0.18084	-8.74739321603879\\
74	0.1845	-9.37790293951576\\
74	0.18816	-10.0551633728184\\
74	0.19182	-10.7791745159467\\
74	0.19548	-11.5499363689008\\
74	0.19914	-12.3674489316805\\
74	0.2028	-13.231712204286\\
74	0.20646	-14.1427261867172\\
74	0.21012	-15.100490878974\\
74	0.21378	-16.1050062810566\\
74	0.21744	-17.1562723929649\\
74	0.2211	-18.2542892146988\\
74	0.22476	-19.3990567462585\\
74	0.22842	-20.5905749876439\\
74	0.23208	-21.828843938855\\
74	0.23574	-23.1138635998918\\
74	0.2394	-24.4456339707543\\
74	0.24306	-25.8241550514424\\
74	0.24672	-27.2494268419563\\
74	0.25038	-28.721449342296\\
74	0.25404	-30.2402225524612\\
74	0.2577	-31.8057464724522\\
74	0.26136	-33.4180211022689\\
74	0.26502	-35.0770464419113\\
74	0.26868	-36.7828224913794\\
74	0.27234	-38.5353492506733\\
74	0.276	-40.3346267197928\\
};
\end{axis}

\begin{axis}[%
width=4.527496cm,
height=3.050847cm,
at={(6.483547cm,4.237288cm)},
scale only axis,
xmin=56,
xmax=74,
tick align=outside,
xlabel={$L_{cut}$},
xmajorgrids,
ymin=0.093,
ymax=0.276,
ylabel={$D_{rlx}$},
ymajorgrids,
zmin=0,
zmax=72.6009909699934,
zlabel={$x_3,x_3$},
zmajorgrids,
view={-140}{50},
legend style={at={(1.03,1)},anchor=north west,legend cell align=left,align=left,draw=white!15!black}
]
\addplot3[only marks,mark=*,mark options={},mark size=1.5000pt,color=mycolor1] plot table[row sep=crcr,]{%
74	0.123	9.20647951614942\\
72	0.113	7.92418746201103\\
61	0.095	4.31034862287736\\
56	0.093	4.66548553880804\\
};
\addplot3[only marks,mark=*,mark options={},mark size=1.5000pt,color=mycolor2] plot table[row sep=crcr,]{%
67	0.276	70.0021748564066\\
66	0.255	55.9817320198023\\
62	0.209	32.6510614505353\\
57	0.193	26.3215007456573\\
};
\addplot3[only marks,mark=*,mark options={},mark size=1.5000pt,color=black] plot table[row sep=crcr,]{%
69	0.104	6.17306469439544\\
};
\addplot3[only marks,mark=*,mark options={},mark size=1.5000pt,color=black] plot table[row sep=crcr,]{%
64	0.23	42.3546530293684\\
};

\addplot3[%
surf,
opacity=0.7,
shader=interp,
colormap={mymap}{[1pt] rgb(0pt)=(0.0901961,0.239216,0.0745098); rgb(1pt)=(0.0945149,0.242058,0.0739522); rgb(2pt)=(0.0988592,0.244894,0.0733566); rgb(3pt)=(0.103229,0.247724,0.0727241); rgb(4pt)=(0.107623,0.250549,0.0720557); rgb(5pt)=(0.112043,0.253367,0.0713525); rgb(6pt)=(0.116487,0.25618,0.0706154); rgb(7pt)=(0.120956,0.258986,0.0698456); rgb(8pt)=(0.125449,0.261787,0.0690441); rgb(9pt)=(0.129967,0.264581,0.0682118); rgb(10pt)=(0.134508,0.26737,0.06735); rgb(11pt)=(0.139074,0.270152,0.0664596); rgb(12pt)=(0.143663,0.272929,0.0655416); rgb(13pt)=(0.148275,0.275699,0.0645971); rgb(14pt)=(0.152911,0.278463,0.0636271); rgb(15pt)=(0.15757,0.281221,0.0626328); rgb(16pt)=(0.162252,0.283973,0.0616151); rgb(17pt)=(0.166957,0.286719,0.060575); rgb(18pt)=(0.171685,0.289458,0.0595136); rgb(19pt)=(0.176434,0.292191,0.0584321); rgb(20pt)=(0.181207,0.294918,0.0573313); rgb(21pt)=(0.186001,0.297639,0.0562123); rgb(22pt)=(0.190817,0.300353,0.0550763); rgb(23pt)=(0.195655,0.303061,0.0539242); rgb(24pt)=(0.200514,0.305763,0.052757); rgb(25pt)=(0.205395,0.308459,0.0515759); rgb(26pt)=(0.210296,0.311149,0.0503624); rgb(27pt)=(0.215212,0.313846,0.0490067); rgb(28pt)=(0.220142,0.316548,0.0475043); rgb(29pt)=(0.22509,0.319254,0.0458704); rgb(30pt)=(0.230056,0.321962,0.0441205); rgb(31pt)=(0.235042,0.324671,0.04227); rgb(32pt)=(0.240048,0.327379,0.0403343); rgb(33pt)=(0.245078,0.330085,0.0383287); rgb(34pt)=(0.250131,0.332786,0.0362688); rgb(35pt)=(0.25521,0.335482,0.0341698); rgb(36pt)=(0.260317,0.33817,0.0320472); rgb(37pt)=(0.265451,0.340849,0.0299163); rgb(38pt)=(0.270616,0.343517,0.0277927); rgb(39pt)=(0.275813,0.346172,0.0256916); rgb(40pt)=(0.281043,0.348814,0.0236284); rgb(41pt)=(0.286307,0.35144,0.0216186); rgb(42pt)=(0.291607,0.354048,0.0196776); rgb(43pt)=(0.296945,0.356637,0.0178207); rgb(44pt)=(0.302322,0.359206,0.0160634); rgb(45pt)=(0.307739,0.361753,0.0144211); rgb(46pt)=(0.313198,0.364275,0.0129091); rgb(47pt)=(0.318701,0.366772,0.0115428); rgb(48pt)=(0.324249,0.369242,0.0103377); rgb(49pt)=(0.329843,0.371682,0.00930909); rgb(50pt)=(0.335485,0.374093,0.00847245); rgb(51pt)=(0.341176,0.376471,0.00784314); rgb(52pt)=(0.346925,0.378826,0.00732741); rgb(53pt)=(0.352735,0.381168,0.00682184); rgb(54pt)=(0.358605,0.383497,0.00632729); rgb(55pt)=(0.364532,0.385812,0.00584464); rgb(56pt)=(0.370516,0.388113,0.00537476); rgb(57pt)=(0.376552,0.390399,0.00491852); rgb(58pt)=(0.38264,0.39267,0.00447681); rgb(59pt)=(0.388777,0.394925,0.00405048); rgb(60pt)=(0.394962,0.397164,0.00364042); rgb(61pt)=(0.401191,0.399386,0.00324749); rgb(62pt)=(0.407464,0.401592,0.00287258); rgb(63pt)=(0.413777,0.40378,0.00251655); rgb(64pt)=(0.420129,0.40595,0.00218028); rgb(65pt)=(0.426518,0.408102,0.00186463); rgb(66pt)=(0.432942,0.410234,0.00157049); rgb(67pt)=(0.439399,0.412348,0.00129873); rgb(68pt)=(0.445885,0.414441,0.00105022); rgb(69pt)=(0.452401,0.416515,0.000825833); rgb(70pt)=(0.458942,0.418567,0.000626441); rgb(71pt)=(0.465508,0.420599,0.00045292); rgb(72pt)=(0.472096,0.422609,0.000306141); rgb(73pt)=(0.478704,0.424596,0.000186979); rgb(74pt)=(0.485331,0.426562,9.63073e-05); rgb(75pt)=(0.491973,0.428504,3.49981e-05); rgb(76pt)=(0.498628,0.430422,3.92506e-06); rgb(77pt)=(0.505323,0.432315,0); rgb(78pt)=(0.512206,0.434168,0); rgb(79pt)=(0.519282,0.435983,0); rgb(80pt)=(0.526529,0.437764,0); rgb(81pt)=(0.533922,0.439512,0); rgb(82pt)=(0.54144,0.441232,0); rgb(83pt)=(0.549059,0.442927,0); rgb(84pt)=(0.556756,0.444599,0); rgb(85pt)=(0.564508,0.446252,0); rgb(86pt)=(0.572292,0.447889,0); rgb(87pt)=(0.580084,0.449514,0); rgb(88pt)=(0.587863,0.451129,0); rgb(89pt)=(0.595604,0.452737,0); rgb(90pt)=(0.603284,0.454343,0); rgb(91pt)=(0.610882,0.455948,0); rgb(92pt)=(0.618373,0.457556,0); rgb(93pt)=(0.625734,0.459171,0); rgb(94pt)=(0.632943,0.460795,0); rgb(95pt)=(0.639976,0.462432,0); rgb(96pt)=(0.64681,0.464084,0); rgb(97pt)=(0.653423,0.465756,0); rgb(98pt)=(0.659791,0.46745,0); rgb(99pt)=(0.665891,0.469169,0); rgb(100pt)=(0.6717,0.470916,0); rgb(101pt)=(0.677195,0.472696,0); rgb(102pt)=(0.682353,0.47451,0); rgb(103pt)=(0.687242,0.476355,0); rgb(104pt)=(0.691952,0.478225,0); rgb(105pt)=(0.696497,0.480118,0); rgb(106pt)=(0.700887,0.482033,0); rgb(107pt)=(0.705134,0.483968,0); rgb(108pt)=(0.709251,0.485921,0); rgb(109pt)=(0.713249,0.487891,0); rgb(110pt)=(0.71714,0.489876,0); rgb(111pt)=(0.720936,0.491875,0); rgb(112pt)=(0.724649,0.493887,0); rgb(113pt)=(0.72829,0.495909,0); rgb(114pt)=(0.731872,0.49794,0); rgb(115pt)=(0.735406,0.499979,0); rgb(116pt)=(0.738904,0.502025,0); rgb(117pt)=(0.742378,0.504075,0); rgb(118pt)=(0.74584,0.506128,0); rgb(119pt)=(0.749302,0.508182,0); rgb(120pt)=(0.752775,0.510237,0); rgb(121pt)=(0.756272,0.51229,0); rgb(122pt)=(0.759804,0.514339,0); rgb(123pt)=(0.763384,0.516385,0); rgb(124pt)=(0.767022,0.518424,0); rgb(125pt)=(0.770731,0.520455,0); rgb(126pt)=(0.774523,0.522478,0); rgb(127pt)=(0.77841,0.524489,0); rgb(128pt)=(0.782391,0.526491,0); rgb(129pt)=(0.786402,0.528496,0); rgb(130pt)=(0.790431,0.530506,0); rgb(131pt)=(0.794478,0.532521,0); rgb(132pt)=(0.798541,0.534539,0); rgb(133pt)=(0.802619,0.53656,0); rgb(134pt)=(0.806712,0.538584,0); rgb(135pt)=(0.81082,0.540609,0); rgb(136pt)=(0.81494,0.542635,0); rgb(137pt)=(0.819074,0.54466,0); rgb(138pt)=(0.823219,0.546686,0); rgb(139pt)=(0.827374,0.548709,0); rgb(140pt)=(0.831541,0.55073,0); rgb(141pt)=(0.835716,0.552749,0); rgb(142pt)=(0.8399,0.554763,0); rgb(143pt)=(0.844092,0.556774,0); rgb(144pt)=(0.848292,0.558779,0); rgb(145pt)=(0.852497,0.560778,0); rgb(146pt)=(0.856708,0.562771,0); rgb(147pt)=(0.860924,0.564756,0); rgb(148pt)=(0.865143,0.566733,0); rgb(149pt)=(0.869366,0.568701,0); rgb(150pt)=(0.873592,0.57066,0); rgb(151pt)=(0.877819,0.572608,0); rgb(152pt)=(0.882047,0.574545,0); rgb(153pt)=(0.886275,0.576471,0); rgb(154pt)=(0.890659,0.578362,0); rgb(155pt)=(0.895333,0.580203,0); rgb(156pt)=(0.900258,0.581999,0); rgb(157pt)=(0.905397,0.583755,0); rgb(158pt)=(0.910711,0.585479,0); rgb(159pt)=(0.916164,0.587176,0); rgb(160pt)=(0.921717,0.588852,0); rgb(161pt)=(0.927333,0.590513,0); rgb(162pt)=(0.932974,0.592166,0); rgb(163pt)=(0.938602,0.593815,0); rgb(164pt)=(0.94418,0.595468,0); rgb(165pt)=(0.949669,0.59713,0); rgb(166pt)=(0.955033,0.598808,0); rgb(167pt)=(0.960233,0.600507,0); rgb(168pt)=(0.965232,0.602233,0); rgb(169pt)=(0.969992,0.603992,0); rgb(170pt)=(0.974475,0.605791,0); rgb(171pt)=(0.978643,0.607636,0); rgb(172pt)=(0.98246,0.609532,0); rgb(173pt)=(0.985886,0.611486,0); rgb(174pt)=(0.988885,0.613503,0); rgb(175pt)=(0.991419,0.61559,0); rgb(176pt)=(0.99345,0.617753,0); rgb(177pt)=(0.99494,0.619997,0); rgb(178pt)=(0.995851,0.622329,0); rgb(179pt)=(0.996226,0.624763,0); rgb(180pt)=(0.996512,0.627352,0); rgb(181pt)=(0.996788,0.630095,0); rgb(182pt)=(0.997053,0.632982,0); rgb(183pt)=(0.997308,0.636004,0); rgb(184pt)=(0.997552,0.639152,0); rgb(185pt)=(0.997785,0.642416,0); rgb(186pt)=(0.998006,0.645786,0); rgb(187pt)=(0.998217,0.649253,0); rgb(188pt)=(0.998416,0.652807,0); rgb(189pt)=(0.998605,0.656439,0); rgb(190pt)=(0.998781,0.660138,0); rgb(191pt)=(0.998946,0.663897,0); rgb(192pt)=(0.9991,0.667704,0); rgb(193pt)=(0.999242,0.67155,0); rgb(194pt)=(0.999372,0.675427,0); rgb(195pt)=(0.99949,0.679323,0); rgb(196pt)=(0.999596,0.68323,0); rgb(197pt)=(0.99969,0.687139,0); rgb(198pt)=(0.999771,0.691039,0); rgb(199pt)=(0.999841,0.694921,0); rgb(200pt)=(0.999898,0.698775,0); rgb(201pt)=(0.999942,0.702592,0); rgb(202pt)=(0.999974,0.706363,0); rgb(203pt)=(0.999994,0.710077,0); rgb(204pt)=(1,0.713725,0); rgb(205pt)=(1,0.717341,0); rgb(206pt)=(1,0.720963,0); rgb(207pt)=(1,0.724591,0); rgb(208pt)=(1,0.728226,0); rgb(209pt)=(1,0.731867,0); rgb(210pt)=(1,0.735514,0); rgb(211pt)=(1,0.739167,0); rgb(212pt)=(1,0.742827,0); rgb(213pt)=(1,0.746493,0); rgb(214pt)=(1,0.750165,0); rgb(215pt)=(1,0.753843,0); rgb(216pt)=(1,0.757527,0); rgb(217pt)=(1,0.761217,0); rgb(218pt)=(1,0.764913,0); rgb(219pt)=(1,0.768615,0); rgb(220pt)=(1,0.772324,0); rgb(221pt)=(1,0.776038,0); rgb(222pt)=(1,0.779758,0); rgb(223pt)=(1,0.783484,0); rgb(224pt)=(1,0.787215,0); rgb(225pt)=(1,0.790953,0); rgb(226pt)=(1,0.794696,0); rgb(227pt)=(1,0.798445,0); rgb(228pt)=(1,0.8022,0); rgb(229pt)=(1,0.805961,0); rgb(230pt)=(1,0.809727,0); rgb(231pt)=(1,0.8135,0); rgb(232pt)=(1,0.817278,0); rgb(233pt)=(1,0.821063,0); rgb(234pt)=(1,0.824854,0); rgb(235pt)=(1,0.828652,0); rgb(236pt)=(1,0.832455,0); rgb(237pt)=(1,0.836265,0); rgb(238pt)=(1,0.840081,0); rgb(239pt)=(1,0.843903,0); rgb(240pt)=(1,0.847732,0); rgb(241pt)=(1,0.851566,0); rgb(242pt)=(1,0.855406,0); rgb(243pt)=(1,0.859253,0); rgb(244pt)=(1,0.863106,0); rgb(245pt)=(1,0.866964,0); rgb(246pt)=(1,0.870829,0); rgb(247pt)=(1,0.8747,0); rgb(248pt)=(1,0.878577,0); rgb(249pt)=(1,0.88246,0); rgb(250pt)=(1,0.886349,0); rgb(251pt)=(1,0.890243,0); rgb(252pt)=(1,0.894144,0); rgb(253pt)=(1,0.898051,0); rgb(254pt)=(1,0.901964,0); rgb(255pt)=(1,0.905882,0)},
mesh/rows=49]
table[row sep=crcr,header=false] {%
%
56	0.093	4.5387683424875\\
56	0.09666	4.71056810167383\\
56	0.10032	4.9309166238546\\
56	0.10398	5.19981390902979\\
56	0.10764	5.51725995719944\\
56	0.1113	5.88325476836353\\
56	0.11496	6.29779834252206\\
56	0.11862	6.76089067967504\\
56	0.12228	7.27253177982244\\
56	0.12594	7.83272164296428\\
56	0.1296	8.44146026910057\\
56	0.13326	9.0987476582313\\
56	0.13692	9.80458381035648\\
56	0.14058	10.5589687254761\\
56	0.14424	11.3619024035902\\
56	0.1479	12.2133848446986\\
56	0.15156	13.1134160488016\\
56	0.15522	14.061996015899\\
56	0.15888	15.0591247459908\\
56	0.16254	16.104802239077\\
56	0.1662	17.1990284951577\\
56	0.16986	18.3418035142329\\
56	0.17352	19.5331272963024\\
56	0.17718	20.7729998413664\\
56	0.18084	22.0614211494249\\
56	0.1845	23.3983912204778\\
56	0.18816	24.7839100545251\\
56	0.19182	26.2179776515669\\
56	0.19548	27.7005940116031\\
56	0.19914	29.2317591346338\\
56	0.2028	30.8114730206589\\
56	0.20646	32.4397356696785\\
56	0.21012	34.1165470816924\\
56	0.21378	35.8419072567008\\
56	0.21744	37.6158161947037\\
56	0.2211	39.438273895701\\
56	0.22476	41.3092803596928\\
56	0.22842	43.2288355866789\\
56	0.23208	45.1969395766595\\
56	0.23574	47.2135923296346\\
56	0.2394	49.2787938456041\\
56	0.24306	51.3925441245681\\
56	0.24672	53.5548431665265\\
56	0.25038	55.7656909714793\\
56	0.25404	58.0250875394266\\
56	0.2577	60.3330328703683\\
56	0.26136	62.6895269643044\\
56	0.26502	65.094569821235\\
56	0.26868	67.54816144116\\
56	0.27234	70.0503018240795\\
56	0.276	72.6009909699934\\
56.375	0.093	4.50815893448501\\
56.375	0.09666	4.67756290951721\\
56.375	0.10032	4.89551564754386\\
56.375	0.10398	5.16201714856494\\
56.375	0.10764	5.47706741258045\\
56.375	0.1113	5.84066643959041\\
56.375	0.11496	6.25281422959483\\
56.375	0.11862	6.71351078259368\\
56.375	0.12228	7.22275609858696\\
56.375	0.12594	7.78055017757468\\
56.375	0.1296	8.38689301955685\\
56.375	0.13326	9.04178462453346\\
56.375	0.13692	9.7452249925045\\
56.375	0.14058	10.49721412347\\
56.375	0.14424	11.2977520174299\\
56.375	0.1479	12.1468386743843\\
56.375	0.15156	13.0444740943331\\
56.375	0.15522	13.9906582772764\\
56.375	0.15888	14.9853912232141\\
56.375	0.16254	16.0286729321462\\
56.375	0.1662	17.1205034040728\\
56.375	0.16986	18.2608826389938\\
56.375	0.17352	19.4498106369092\\
56.375	0.17718	20.6872873978191\\
56.375	0.18084	21.9733129217234\\
56.375	0.1845	23.3078872086222\\
56.375	0.18816	24.6910102585154\\
56.375	0.19182	26.1226820714031\\
56.375	0.19548	27.6029026472852\\
56.375	0.19914	29.1316719861617\\
56.375	0.2028	30.7089900880327\\
56.375	0.20646	32.3348569528981\\
56.375	0.21012	34.009272580758\\
56.375	0.21378	35.7322369716123\\
56.375	0.21744	37.503750125461\\
56.375	0.2211	39.3238120423042\\
56.375	0.22476	41.1924227221418\\
56.375	0.22842	43.1095821649739\\
56.375	0.23208	45.0752903708004\\
56.375	0.23574	47.0895473396213\\
56.375	0.2394	49.1523530714367\\
56.375	0.24306	51.2637075662465\\
56.375	0.24672	53.4236108240508\\
56.375	0.25038	55.6320628448495\\
56.375	0.25404	57.8890636286426\\
56.375	0.2577	60.1946131754302\\
56.375	0.26136	62.5487114852122\\
56.375	0.26502	64.9513585579887\\
56.375	0.26868	67.4025543937596\\
56.375	0.27234	69.9022989925249\\
56.375	0.276	72.4505923542847\\
56.75	0.093	4.48146070434747\\
56.75	0.09666	4.64846889522555\\
56.75	0.10032	4.86402584909807\\
56.75	0.10398	5.12813156596502\\
56.75	0.10764	5.44078604582641\\
56.75	0.1113	5.80198928868226\\
56.75	0.11496	6.21174129453255\\
56.75	0.11862	6.67004206337727\\
56.75	0.12228	7.17689159521644\\
56.75	0.12594	7.73228989005002\\
56.75	0.1296	8.33623694787806\\
56.75	0.13326	8.98873276870056\\
56.75	0.13692	9.68977735251747\\
56.75	0.14058	10.4393706993288\\
56.75	0.14424	11.2375128091347\\
56.75	0.1479	12.0842036819349\\
56.75	0.15156	12.9794433177296\\
56.75	0.15522	13.9232317165187\\
56.75	0.15888	14.9155688783023\\
56.75	0.16254	15.9564548030803\\
56.75	0.1662	17.0458894908527\\
56.75	0.16986	18.1838729416196\\
56.75	0.17352	19.3704051553809\\
56.75	0.17718	20.6054861321367\\
56.75	0.18084	21.8891158718869\\
56.75	0.1845	23.2212943746316\\
56.75	0.18816	24.6020216403707\\
56.75	0.19182	26.0312976691042\\
56.75	0.19548	27.5091224608322\\
56.75	0.19914	29.0354960155546\\
56.75	0.2028	30.6104183332714\\
56.75	0.20646	32.2338894139827\\
56.75	0.21012	33.9059092576885\\
56.75	0.21378	35.6264778643886\\
56.75	0.21744	37.3955952340832\\
56.75	0.2211	39.2132613667723\\
56.75	0.22476	41.0794762624558\\
56.75	0.22842	42.9942399211337\\
56.75	0.23208	44.9575523428061\\
56.75	0.23574	46.9694135274729\\
56.75	0.2394	49.0298234751342\\
56.75	0.24306	51.1387821857899\\
56.75	0.24672	53.29628965944\\
56.75	0.25038	55.5023458960846\\
56.75	0.25404	57.7569508957236\\
56.75	0.2577	60.0601046583571\\
56.75	0.26136	62.411807183985\\
56.75	0.26502	64.8120584726073\\
56.75	0.26868	67.2608585242241\\
56.75	0.27234	69.7582073388353\\
56.75	0.276	72.304104916441\\
57.125	0.093	4.45867365207484\\
57.125	0.09666	4.6232860587988\\
57.125	0.10032	4.83644722851719\\
57.125	0.10398	5.09815716123002\\
57.125	0.10764	5.4084158569373\\
57.125	0.1113	5.76722331563901\\
57.125	0.11496	6.17457953733518\\
57.125	0.11862	6.63048452202578\\
57.125	0.12228	7.13493826971081\\
57.125	0.12594	7.6879407803903\\
57.125	0.1296	8.2894920540642\\
57.125	0.13326	8.93959209073257\\
57.125	0.13692	9.63824089039537\\
57.125	0.14058	10.3854384530526\\
57.125	0.14424	11.1811847787043\\
57.125	0.1479	12.0254798673504\\
57.125	0.15156	12.918323718991\\
57.125	0.15522	13.859716333626\\
57.125	0.15888	14.8496577112554\\
57.125	0.16254	15.8881478518793\\
57.125	0.1662	16.9751867554976\\
57.125	0.16986	18.1107744221104\\
57.125	0.17352	19.2949108517176\\
57.125	0.17718	20.5275960443192\\
57.125	0.18084	21.8088299999153\\
57.125	0.1845	23.1386127185059\\
57.125	0.18816	24.5169442000908\\
57.125	0.19182	25.9438244446702\\
57.125	0.19548	27.4192534522441\\
57.125	0.19914	28.9432312228124\\
57.125	0.2028	30.5157577563751\\
57.125	0.20646	32.1368330529323\\
57.125	0.21012	33.8064571124839\\
57.125	0.21378	35.5246299350299\\
57.125	0.21744	37.2913515205704\\
57.125	0.2211	39.1066218691053\\
57.125	0.22476	40.9704409806347\\
57.125	0.22842	42.8828088551585\\
57.125	0.23208	44.8437254926768\\
57.125	0.23574	46.8531908931895\\
57.125	0.2394	48.9112050566966\\
57.125	0.24306	51.0177679831982\\
57.125	0.24672	53.1728796726942\\
57.125	0.25038	55.3765401251847\\
57.125	0.25404	57.6287493406696\\
57.125	0.2577	59.9295073191489\\
57.125	0.26136	62.2788140606227\\
57.125	0.26502	64.6766695650909\\
57.125	0.26868	67.1230738325535\\
57.125	0.27234	69.6180268630106\\
57.125	0.276	72.1615286564622\\
57.5	0.093	4.43979777766713\\
57.5	0.09666	4.60201440023698\\
57.5	0.10032	4.81277978580125\\
57.5	0.10398	5.07209393435996\\
57.5	0.10764	5.37995684591311\\
57.5	0.1113	5.73636852046069\\
57.5	0.11496	6.14132895800273\\
57.5	0.11862	6.59483815853922\\
57.5	0.12228	7.09689612207013\\
57.5	0.12594	7.64750284859548\\
57.5	0.1296	8.24665833811527\\
57.5	0.13326	8.89436259062951\\
57.5	0.13692	9.59061560613819\\
57.5	0.14058	10.3354173846413\\
57.5	0.14424	11.1287679261389\\
57.5	0.1479	11.9706672306309\\
57.5	0.15156	12.8611152981173\\
57.5	0.15522	13.8001121285982\\
57.5	0.15888	14.7876577220735\\
57.5	0.16254	15.8237520785433\\
57.5	0.1662	16.9083951980075\\
57.5	0.16986	18.0415870804661\\
57.5	0.17352	19.2233277259192\\
57.5	0.17718	20.4536171343667\\
57.5	0.18084	21.7324553058086\\
57.5	0.1845	23.0598422402451\\
57.5	0.18816	24.4357779376759\\
57.5	0.19182	25.8602623981012\\
57.5	0.19548	27.3332956215209\\
57.5	0.19914	28.8548776079351\\
57.5	0.2028	30.4250083573437\\
57.5	0.20646	32.0436878697467\\
57.5	0.21012	33.7109161451442\\
57.5	0.21378	35.4266931835362\\
57.5	0.21744	37.1910189849225\\
57.5	0.2211	39.0038935493033\\
57.5	0.22476	40.8653168766786\\
57.5	0.22842	42.7752889670483\\
57.5	0.23208	44.7338098204124\\
57.5	0.23574	46.740879436771\\
57.5	0.2394	48.796497816124\\
57.5	0.24306	50.9006649584714\\
57.5	0.24672	53.0533808638133\\
57.5	0.25038	55.2546455321496\\
57.5	0.25404	57.5044589634804\\
57.5	0.2577	59.8028211578056\\
57.5	0.26136	62.1497321151253\\
57.5	0.26502	64.5451918354394\\
57.5	0.26868	66.9892003187479\\
57.5	0.27234	69.4817575650509\\
57.5	0.276	72.0228635743483\\
57.875	0.093	4.42483308112438\\
57.875	0.09666	4.5846539195401\\
57.875	0.10032	4.79302352095025\\
57.875	0.10398	5.04994188535483\\
57.875	0.10764	5.35540901275385\\
57.875	0.1113	5.70942490314732\\
57.875	0.11496	6.11198955653525\\
57.875	0.11862	6.56310297291758\\
57.875	0.12228	7.06276515229438\\
57.875	0.12594	7.61097609466561\\
57.875	0.1296	8.20773580003127\\
57.875	0.13326	8.85304426839139\\
57.875	0.13692	9.54690149974594\\
57.875	0.14058	10.2893074940949\\
57.875	0.14424	11.0802622514384\\
57.875	0.1479	11.9197657717762\\
57.875	0.15156	12.8078180551085\\
57.875	0.15522	13.7444191014353\\
57.875	0.15888	14.7295689107565\\
57.875	0.16254	15.7632674830721\\
57.875	0.1662	16.8455148183822\\
57.875	0.16986	17.9763109166867\\
57.875	0.17352	19.1556557779857\\
57.875	0.17718	20.3835494022791\\
57.875	0.18084	21.6599917895669\\
57.875	0.1845	22.9849829398492\\
57.875	0.18816	24.3585228531259\\
57.875	0.19182	25.7806115293971\\
57.875	0.19548	27.2512489686627\\
57.875	0.19914	28.7704351709227\\
57.875	0.2028	30.3381701361772\\
57.875	0.20646	31.9544538644262\\
57.875	0.21012	33.6192863556695\\
57.875	0.21378	35.3326676099073\\
57.875	0.21744	37.0945976271395\\
57.875	0.2211	38.9050764073662\\
57.875	0.22476	40.7641039505874\\
57.875	0.22842	42.6716802568029\\
57.875	0.23208	44.6278053260129\\
57.875	0.23574	46.6324791582174\\
57.875	0.2394	48.6857017534163\\
57.875	0.24306	50.7874731116096\\
57.875	0.24672	52.9377932327973\\
57.875	0.25038	55.1366621169796\\
57.875	0.25404	57.3840797641562\\
57.875	0.2577	59.6800461743273\\
57.875	0.26136	62.0245613474928\\
57.875	0.26502	64.4176252836528\\
57.875	0.26868	66.8592379828072\\
57.875	0.27234	69.349399444956\\
57.875	0.276	71.8881096700993\\
58.25	0.093	4.41377956244655\\
58.25	0.09666	4.57120461670815\\
58.25	0.10032	4.77717843396417\\
58.25	0.10398	5.03170101421463\\
58.25	0.10764	5.33477235745953\\
58.25	0.1113	5.68639246369887\\
58.25	0.11496	6.08656133293267\\
58.25	0.11862	6.53527896516089\\
58.25	0.12228	7.03254536038357\\
58.25	0.12594	7.57836051860067\\
58.25	0.1296	8.17272443981221\\
58.25	0.13326	8.81563712401821\\
58.25	0.13692	9.50709857121863\\
58.25	0.14058	10.2471087814135\\
58.25	0.14424	11.0356677546028\\
58.25	0.1479	11.8727754907866\\
58.25	0.15156	12.7584319899647\\
58.25	0.15522	13.6926372521374\\
58.25	0.15888	14.6753912773045\\
58.25	0.16254	15.706694065466\\
58.25	0.1662	16.7865456166219\\
58.25	0.16986	17.9149459307723\\
58.25	0.17352	19.0918950079172\\
58.25	0.17718	20.3173928480564\\
58.25	0.18084	21.5914394511901\\
58.25	0.1845	22.9140348173183\\
58.25	0.18816	24.2851789464409\\
58.25	0.19182	25.7048718385579\\
58.25	0.19548	27.1731134936694\\
58.25	0.19914	28.6899039117753\\
58.25	0.2028	30.2552430928757\\
58.25	0.20646	31.8691310369705\\
58.25	0.21012	33.5315677440597\\
58.25	0.21378	35.2425532141434\\
58.25	0.21744	37.0020874472215\\
58.25	0.2211	38.8101704432941\\
58.25	0.22476	40.6668022023611\\
58.25	0.22842	42.5719827244225\\
58.25	0.23208	44.5257120094784\\
58.25	0.23574	46.5279900575287\\
58.25	0.2394	48.5788168685735\\
58.25	0.24306	50.6781924426127\\
58.25	0.24672	52.8261167796463\\
58.25	0.25038	55.0225898796744\\
58.25	0.25404	57.267611742697\\
58.25	0.2577	59.5611823687139\\
58.25	0.26136	61.9033017577253\\
58.25	0.26502	64.2939699097311\\
58.25	0.26868	66.7331868247314\\
58.25	0.27234	69.2209525027262\\
58.25	0.276	71.7572669437153\\
58.625	0.093	4.40663722163366\\
58.625	0.09666	4.56166649174112\\
58.625	0.10032	4.76524452484302\\
58.625	0.10398	5.01737132093936\\
58.625	0.10764	5.31804688003014\\
58.625	0.1113	5.66727120211536\\
58.625	0.11496	6.06504428719503\\
58.625	0.11862	6.51136613526913\\
58.625	0.12228	7.00623674633767\\
58.625	0.12594	7.54965612040066\\
58.625	0.1296	8.14162425745808\\
58.625	0.13326	8.78214115750995\\
58.625	0.13692	9.47120682055625\\
58.625	0.14058	10.208821246597\\
58.625	0.14424	10.9949844356322\\
58.625	0.1479	11.8296963876618\\
58.625	0.15156	12.7129571026859\\
58.625	0.15522	13.6447665807044\\
58.625	0.15888	14.6251248217173\\
58.625	0.16254	15.6540318257247\\
58.625	0.1662	16.7314875927266\\
58.625	0.16986	17.8574921227228\\
58.625	0.17352	19.0320454157135\\
58.625	0.17718	20.2551474716987\\
58.625	0.18084	21.5267982906783\\
58.625	0.1845	22.8469978726523\\
58.625	0.18816	24.2157462176208\\
58.625	0.19182	25.6330433255837\\
58.625	0.19548	27.098889196541\\
58.625	0.19914	28.6132838304928\\
58.625	0.2028	30.1762272274391\\
58.625	0.20646	31.7877193873798\\
58.625	0.21012	33.4477603103149\\
58.625	0.21378	35.1563499962444\\
58.625	0.21744	36.9134884451684\\
58.625	0.2211	38.7191756570868\\
58.625	0.22476	40.5734116319997\\
58.625	0.22842	42.4761963699071\\
58.625	0.23208	44.4275298708088\\
58.625	0.23574	46.427412134705\\
58.625	0.2394	48.4758431615956\\
58.625	0.24306	50.5728229514807\\
58.625	0.24672	52.7183515043602\\
58.625	0.25038	54.9124288202342\\
58.625	0.25404	57.1550548991026\\
58.625	0.2577	59.4462297409655\\
58.625	0.26136	61.7859533458227\\
58.625	0.26502	64.1742257136744\\
58.625	0.26868	66.6110468445206\\
58.625	0.27234	69.0964167383612\\
58.625	0.276	71.6303353951962\\
59	0.093	4.40340605868571\\
59	0.09666	4.55603954463906\\
59	0.10032	4.75722179358683\\
59	0.10398	5.00695280552905\\
59	0.10764	5.30523258046571\\
59	0.1113	5.6520611183968\\
59	0.11496	6.04743841932235\\
59	0.11862	6.49136448324232\\
59	0.12228	6.98383931015674\\
59	0.12594	7.5248629000656\\
59	0.1296	8.11443525296889\\
59	0.13326	8.75255636886664\\
59	0.13692	9.43922624775881\\
59	0.14058	10.1744448896454\\
59	0.14424	10.9582122945265\\
59	0.1479	11.790528462402\\
59	0.15156	12.6713933932719\\
59	0.15522	13.6008070871363\\
59	0.15888	14.5787695439952\\
59	0.16254	15.6052807638484\\
59	0.1662	16.6803407466961\\
59	0.16986	17.8039494925383\\
59	0.17352	18.9761070013749\\
59	0.17718	20.1968132732059\\
59	0.18084	21.4660683080314\\
59	0.1845	22.7838721058513\\
59	0.18816	24.1502246666656\\
59	0.19182	25.5651259904744\\
59	0.19548	27.0285760772776\\
59	0.19914	28.5405749270753\\
59	0.2028	30.1011225398674\\
59	0.20646	31.710218915654\\
59	0.21012	33.367864054435\\
59	0.21378	35.0740579562104\\
59	0.21744	36.8288006209803\\
59	0.2211	38.6320920487446\\
59	0.22476	40.4839322395033\\
59	0.22842	42.3843211932565\\
59	0.23208	44.3332589100041\\
59	0.23574	46.3307453897462\\
59	0.2394	48.3767806324828\\
59	0.24306	50.4713646382137\\
59	0.24672	52.6144974069391\\
59	0.25038	54.806178938659\\
59	0.25404	57.0464092333732\\
59	0.2577	59.335188291082\\
59	0.26136	61.6725161117851\\
59	0.26502	64.0583926954827\\
59	0.26868	66.4928180421747\\
59	0.27234	68.9757921518612\\
59	0.276	71.5073150245421\\
59.375	0.093	4.40408607360268\\
59.375	0.09666	4.5543237754019\\
59.375	0.10032	4.75311024019555\\
59.375	0.10398	5.00044546798364\\
59.375	0.10764	5.29632945876617\\
59.375	0.1113	5.64076221254315\\
59.375	0.11496	6.03374372931457\\
59.375	0.11862	6.47527400908042\\
59.375	0.12228	6.96535305184072\\
59.375	0.12594	7.50398085759545\\
59.375	0.1296	8.09115742634463\\
59.375	0.13326	8.72688275808824\\
59.375	0.13692	9.41115685282629\\
59.375	0.14058	10.1439797105588\\
59.375	0.14424	10.9253513312857\\
59.375	0.1479	11.7552717150071\\
59.375	0.15156	12.6337408617229\\
59.375	0.15522	13.5607587714332\\
59.375	0.15888	14.5363254441379\\
59.375	0.16254	15.560440879837\\
59.375	0.1662	16.6331050785306\\
59.375	0.16986	17.7543180402186\\
59.375	0.17352	18.9240797649011\\
59.375	0.17718	20.142390252578\\
59.375	0.18084	21.4092495032493\\
59.375	0.1845	22.7246575169151\\
59.375	0.18816	24.0886142935754\\
59.375	0.19182	25.50111983323\\
59.375	0.19548	26.9621741358791\\
59.375	0.19914	28.4717772015227\\
59.375	0.2028	30.0299290301607\\
59.375	0.20646	31.6366296217931\\
59.375	0.21012	33.29187897642\\
59.375	0.21378	34.9956770940413\\
59.375	0.21744	36.748023974657\\
59.375	0.2211	38.5489196182672\\
59.375	0.22476	40.3983640248718\\
59.375	0.22842	42.2963571944709\\
59.375	0.23208	44.2428991270644\\
59.375	0.23574	46.2379898226524\\
59.375	0.2394	48.2816292812348\\
59.375	0.24306	50.3738175028116\\
59.375	0.24672	52.5145544873829\\
59.375	0.25038	54.7038402349486\\
59.375	0.25404	56.9416747455087\\
59.375	0.2577	59.2280580190633\\
59.375	0.26136	61.5629900556124\\
59.375	0.26502	63.9464708551558\\
59.375	0.26868	66.3785004176937\\
59.375	0.27234	68.8590787432261\\
59.375	0.276	71.3882058317529\\
59.75	0.093	4.40867726638459\\
59.75	0.09666	4.55651918402969\\
59.75	0.10032	4.75290986466922\\
59.75	0.10398	4.99784930830318\\
59.75	0.10764	5.29133751493158\\
59.75	0.1113	5.63337448455443\\
59.75	0.11496	6.02396021717173\\
59.75	0.11862	6.46309471278347\\
59.75	0.12228	6.95077797138964\\
59.75	0.12594	7.48700999299025\\
59.75	0.1296	8.07179077758529\\
59.75	0.13326	8.70512032517479\\
59.75	0.13692	9.38699863575872\\
59.75	0.14058	10.1174257093371\\
59.75	0.14424	10.8964015459099\\
59.75	0.1479	11.7239261454772\\
59.75	0.15156	12.5999995080389\\
59.75	0.15522	13.524621633595\\
59.75	0.15888	14.4977925221456\\
59.75	0.16254	15.5195121736906\\
59.75	0.1662	16.5897805882301\\
59.75	0.16986	17.708597765764\\
59.75	0.17352	18.8759637062923\\
59.75	0.17718	20.0918784098151\\
59.75	0.18084	21.3563418763323\\
59.75	0.1845	22.669354105844\\
59.75	0.18816	24.0309150983501\\
59.75	0.19182	25.4410248538506\\
59.75	0.19548	26.8996833723456\\
59.75	0.19914	28.406890653835\\
59.75	0.2028	29.9626466983189\\
59.75	0.20646	31.5669515057972\\
59.75	0.21012	33.2198050762699\\
59.75	0.21378	34.9212074097371\\
59.75	0.21744	36.6711585061987\\
59.75	0.2211	38.4696583656548\\
59.75	0.22476	40.3167069881053\\
59.75	0.22842	42.2123043735502\\
59.75	0.23208	44.1564505219896\\
59.75	0.23574	46.1491454334235\\
59.75	0.2394	48.1903891078517\\
59.75	0.24306	50.2801815452744\\
59.75	0.24672	52.4185227456916\\
59.75	0.25038	54.6054127091032\\
59.75	0.25404	56.8408514355092\\
59.75	0.2577	59.1248389249097\\
59.75	0.26136	61.4573751773046\\
59.75	0.26502	63.8384601926939\\
59.75	0.26868	66.2680939710777\\
59.75	0.27234	68.7462765124559\\
59.75	0.276	71.2730078168286\\
60.125	0.093	4.41717963703143\\
60.125	0.09666	4.5626257705224\\
60.125	0.10032	4.75662066700781\\
60.125	0.10398	4.99916432648764\\
60.125	0.10764	5.29025674896192\\
60.125	0.1113	5.62989793443065\\
60.125	0.11496	6.01808788289383\\
60.125	0.11862	6.45482659435144\\
60.125	0.12228	6.94011406880349\\
60.125	0.12594	7.47395030624997\\
60.125	0.1296	8.05633530669089\\
60.125	0.13326	8.68726907012626\\
60.125	0.13692	9.36675159655608\\
60.125	0.14058	10.0947828859803\\
60.125	0.14424	10.871362938399\\
60.125	0.1479	11.6964917538122\\
60.125	0.15156	12.5701693322197\\
60.125	0.15522	13.4923956736217\\
60.125	0.15888	14.4631707780182\\
60.125	0.16254	15.4824946454091\\
60.125	0.1662	16.5503672757944\\
60.125	0.16986	17.6667886691742\\
60.125	0.17352	18.8317588255484\\
60.125	0.17718	20.0452777449171\\
60.125	0.18084	21.3073454272801\\
60.125	0.1845	22.6179618726377\\
60.125	0.18816	23.9771270809897\\
60.125	0.19182	25.3848410523361\\
60.125	0.19548	26.8411037866769\\
60.125	0.19914	28.3459152840122\\
60.125	0.2028	29.899275544342\\
60.125	0.20646	31.5011845676662\\
60.125	0.21012	33.1516423539848\\
60.125	0.21378	34.8506489032979\\
60.125	0.21744	36.5982042156053\\
60.125	0.2211	38.3943082909073\\
60.125	0.22476	40.2389611292037\\
60.125	0.22842	42.1321627304945\\
60.125	0.23208	44.0739130947798\\
60.125	0.23574	46.0642122220595\\
60.125	0.2394	48.1030601123336\\
60.125	0.24306	50.1904567656022\\
60.125	0.24672	52.3264021818652\\
60.125	0.25038	54.5108963611227\\
60.125	0.25404	56.7439393033746\\
60.125	0.2577	59.025531008621\\
60.125	0.26136	61.3556714768617\\
60.125	0.26502	63.7343607080969\\
60.125	0.26868	66.1615987023266\\
60.125	0.27234	68.6373854595507\\
60.125	0.276	71.1617209797693\\
60.5	0.093	4.4295931855432\\
60.5	0.09666	4.57264353488005\\
60.5	0.10032	4.76424264721132\\
60.5	0.10398	5.00439052253704\\
60.5	0.10764	5.29308716085721\\
60.5	0.1113	5.63033256217181\\
60.5	0.11496	6.01612672648086\\
60.5	0.11862	6.45046965378435\\
60.5	0.12228	6.93336134408226\\
60.5	0.12594	7.46480179737463\\
60.5	0.1296	8.04479101366143\\
60.5	0.13326	8.67332899294269\\
60.5	0.13692	9.35041573521837\\
60.5	0.14058	10.0760512404885\\
60.5	0.14424	10.8502355087531\\
60.5	0.1479	11.6729685400121\\
60.5	0.15156	12.5442503342655\\
60.5	0.15522	13.4640808915134\\
60.5	0.15888	14.4324602117557\\
60.5	0.16254	15.4493882949925\\
60.5	0.1662	16.5148651412237\\
60.5	0.16986	17.6288907504494\\
60.5	0.17352	18.7914651226695\\
60.5	0.17718	20.002588257884\\
60.5	0.18084	21.262260156093\\
60.5	0.1845	22.5704808172964\\
60.5	0.18816	23.9272502414942\\
60.5	0.19182	25.3325684286865\\
60.5	0.19548	26.7864353788733\\
60.5	0.19914	28.2888510920544\\
60.5	0.2028	29.8398155682301\\
60.5	0.20646	31.4393288074001\\
60.5	0.21012	33.0873908095646\\
60.5	0.21378	34.7840015747235\\
60.5	0.21744	36.5291611028769\\
60.5	0.2211	38.3228693940247\\
60.5	0.22476	40.165126448167\\
60.5	0.22842	42.0559322653037\\
60.5	0.23208	43.9952868454348\\
60.5	0.23574	45.9831901885604\\
60.5	0.2394	48.0196422946804\\
60.5	0.24306	50.1046431637949\\
60.5	0.24672	52.2381927959038\\
60.5	0.25038	54.4202911910072\\
60.5	0.25404	56.6509383491049\\
60.5	0.2577	58.9301342701972\\
60.5	0.26136	61.2578789542838\\
60.5	0.26502	63.6341724013649\\
60.5	0.26868	66.0590146114404\\
60.5	0.27234	68.5324055845104\\
60.5	0.276	71.0543453205748\\
60.875	0.093	4.4459179119199\\
60.875	0.09666	4.58657247710262\\
60.875	0.10032	4.77577580527979\\
60.875	0.10398	5.01352789645138\\
60.875	0.10764	5.29982875061742\\
60.875	0.1113	5.63467836777789\\
60.875	0.11496	6.01807674793281\\
60.875	0.11862	6.45002389108218\\
60.875	0.12228	6.93051979722598\\
60.875	0.12594	7.45956446636422\\
60.875	0.1296	8.03715789849689\\
60.875	0.13326	8.66330009362402\\
60.875	0.13692	9.33799105174558\\
60.875	0.14058	10.0612307728616\\
60.875	0.14424	10.833019256972\\
60.875	0.1479	11.6533565040769\\
60.875	0.15156	12.5222425141762\\
60.875	0.15522	13.43967728727\\
60.875	0.15888	14.4056608233582\\
60.875	0.16254	15.4201931224408\\
60.875	0.1662	16.4832741845179\\
60.875	0.16986	17.5949040095895\\
60.875	0.17352	18.7550825976554\\
60.875	0.17718	19.9638099487159\\
60.875	0.18084	21.2210860627707\\
60.875	0.1845	22.52691093982\\
60.875	0.18816	23.8812845798637\\
60.875	0.19182	25.2842069829019\\
60.875	0.19548	26.7356781489345\\
60.875	0.19914	28.2356980779615\\
60.875	0.2028	29.784266769983\\
60.875	0.20646	31.381384224999\\
60.875	0.21012	33.0270504430094\\
60.875	0.21378	34.7212654240142\\
60.875	0.21744	36.4640291680134\\
60.875	0.2211	38.2553416750071\\
60.875	0.22476	40.0952029449952\\
60.875	0.22842	41.9836129779778\\
60.875	0.23208	43.9205717739548\\
60.875	0.23574	45.9060793329263\\
60.875	0.2394	47.9401356548922\\
60.875	0.24306	50.0227407398525\\
60.875	0.24672	52.1538945878073\\
60.875	0.25038	54.3335971987565\\
60.875	0.25404	56.5618485727002\\
60.875	0.2577	58.8386487096383\\
60.875	0.26136	61.1639976095708\\
60.875	0.26502	63.5378952724978\\
60.875	0.26868	65.9603416984192\\
60.875	0.27234	68.4313368873351\\
60.875	0.276	70.9508808392453\\
61.25	0.093	4.46615381616154\\
61.25	0.09666	4.60441259719014\\
61.25	0.10032	4.79122014121318\\
61.25	0.10398	5.02657644823064\\
61.25	0.10764	5.31048151824255\\
61.25	0.1113	5.6429353512489\\
61.25	0.11496	6.02393794724972\\
61.25	0.11862	6.45348930624495\\
61.25	0.12228	6.93158942823463\\
61.25	0.12594	7.45823831321874\\
61.25	0.1296	8.03343596119729\\
61.25	0.13326	8.6571823721703\\
61.25	0.13692	9.32947754613773\\
61.25	0.14058	10.0503214830996\\
61.25	0.14424	10.8197141830559\\
61.25	0.1479	11.6376556460067\\
61.25	0.15156	12.5041458719519\\
61.25	0.15522	13.4191848608915\\
61.25	0.15888	14.3827726128256\\
61.25	0.16254	15.3949091277542\\
61.25	0.1662	16.4555944056771\\
61.25	0.16986	17.5648284465945\\
61.25	0.17352	18.7226112505064\\
61.25	0.17718	19.9289428174126\\
61.25	0.18084	21.1838231473134\\
61.25	0.1845	22.4872522402085\\
61.25	0.18816	23.8392300960981\\
61.25	0.19182	25.2397567149822\\
61.25	0.19548	26.6888320968607\\
61.25	0.19914	28.1864562417336\\
61.25	0.2028	29.732629149601\\
61.25	0.20646	31.3273508204628\\
61.25	0.21012	32.970621254319\\
61.25	0.21378	34.6624404511697\\
61.25	0.21744	36.4028084110148\\
61.25	0.2211	38.1917251338544\\
61.25	0.22476	40.0291906196884\\
61.25	0.22842	41.9152048685169\\
61.25	0.23208	43.8497678803398\\
61.25	0.23574	45.8328796551571\\
61.25	0.2394	47.8645401929689\\
61.25	0.24306	49.9447494937751\\
61.25	0.24672	52.0735075575757\\
61.25	0.25038	54.2508143843709\\
61.25	0.25404	56.4766699741604\\
61.25	0.2577	58.7510743269444\\
61.25	0.26136	61.0740274427228\\
61.25	0.26502	63.4455293214956\\
61.25	0.26868	65.8655799632629\\
61.25	0.27234	68.3341793680246\\
61.25	0.276	70.8513275357808\\
61.625	0.093	4.49030089826813\\
61.625	0.09666	4.62616389514261\\
61.625	0.10032	4.81057565501152\\
61.625	0.10398	5.04353617787486\\
61.625	0.10764	5.32504546373264\\
61.625	0.1113	5.65510351258487\\
61.625	0.11496	6.03371032443156\\
61.625	0.11862	6.46086589927268\\
61.625	0.12228	6.93657023710822\\
61.625	0.12594	7.46082333793821\\
61.625	0.1296	8.03362520176264\\
61.625	0.13326	8.65497582858152\\
61.625	0.13692	9.32487521839484\\
61.625	0.14058	10.0433233712026\\
61.625	0.14424	10.8103202870048\\
61.625	0.1479	11.6258659658014\\
61.625	0.15156	12.4899604075925\\
61.625	0.15522	13.402603612378\\
61.625	0.15888	14.363795580158\\
61.625	0.16254	15.3735363109324\\
61.625	0.1662	16.4318258047012\\
61.625	0.16986	17.5386640614645\\
61.625	0.17352	18.6940510812222\\
61.625	0.17718	19.8979868639744\\
61.625	0.18084	21.150471409721\\
61.625	0.1845	22.451504718462\\
61.625	0.18816	23.8010867901975\\
61.625	0.19182	25.1992176249274\\
61.625	0.19548	26.6458972226518\\
61.625	0.19914	28.1411255833706\\
61.625	0.2028	29.6849027070838\\
61.625	0.20646	31.2772285937915\\
61.625	0.21012	32.9181032434936\\
61.625	0.21378	34.6075266561902\\
61.625	0.21744	36.3454988318812\\
61.625	0.2211	38.1320197705667\\
61.625	0.22476	39.9670894722466\\
61.625	0.22842	41.8507079369209\\
61.625	0.23208	43.7828751645896\\
61.625	0.23574	45.7635911552529\\
61.625	0.2394	47.7928559089105\\
61.625	0.24306	49.8706694255626\\
61.625	0.24672	51.9970317052091\\
61.625	0.25038	54.1719427478501\\
61.625	0.25404	56.3954025534855\\
61.625	0.2577	58.6674111221154\\
61.625	0.26136	60.9879684537397\\
61.625	0.26502	63.3570745483584\\
61.625	0.26868	65.7747294059716\\
61.625	0.27234	68.2409330265792\\
61.625	0.276	70.7556854101812\\
62	0.093	4.51835915823963\\
62	0.09666	4.65182637095998\\
62	0.10032	4.83384234667477\\
62	0.10398	5.06440708538399\\
62	0.10764	5.34352058708765\\
62	0.1113	5.67118285178576\\
62	0.11496	6.04739387947831\\
62	0.11862	6.47215367016531\\
62	0.12228	6.94546222384673\\
62	0.12594	7.4673195405226\\
62	0.1296	8.03772562019291\\
62	0.13326	8.65668046285766\\
62	0.13692	9.32418406851686\\
62	0.14058	10.0402364371705\\
62	0.14424	10.8048375688186\\
62	0.1479	11.6179874634611\\
62	0.15156	12.479686121098\\
62	0.15522	13.3899335417294\\
62	0.15888	14.3487297253553\\
62	0.16254	15.3560746719755\\
62	0.1662	16.4119683815902\\
62	0.16986	17.5164108541994\\
62	0.17352	18.669402089803\\
62	0.17718	19.870942088401\\
62	0.18084	21.1210308499935\\
62	0.1845	22.4196683745804\\
62	0.18816	23.7668546621618\\
62	0.19182	25.1625897127376\\
62	0.19548	26.6068735263078\\
62	0.19914	28.0997061028725\\
62	0.2028	29.6410874424316\\
62	0.20646	31.2310175449852\\
62	0.21012	32.8694964105332\\
62	0.21378	34.5565240390756\\
62	0.21744	36.2921004306125\\
62	0.2211	38.0762255851439\\
62	0.22476	39.9088995026696\\
62	0.22842	41.7901221831898\\
62	0.23208	43.7198936267044\\
62	0.23574	45.6982138332135\\
62	0.2394	47.7250828027171\\
62	0.24306	49.800500535215\\
62	0.24672	51.9244670307074\\
62	0.25038	54.0969822891943\\
62	0.25404	56.3180463106756\\
62	0.2577	58.5876590951513\\
62	0.26136	60.9058206426215\\
62	0.26502	63.2725309530861\\
62	0.26868	65.6877900265451\\
62	0.27234	68.1515978629986\\
62	0.276	70.6639544624465\\
62.375	0.093	4.55032859607607\\
62.375	0.09666	4.6814000246423\\
62.375	0.10032	4.86102021620296\\
62.375	0.10398	5.08918917075805\\
62.375	0.10764	5.3659068883076\\
62.375	0.1113	5.69117336885158\\
62.375	0.11496	6.06498861239001\\
62.375	0.11862	6.48735261892288\\
62.375	0.12228	6.95826538845018\\
62.375	0.12594	7.47772692097192\\
62.375	0.1296	8.04573721648811\\
62.375	0.13326	8.66229627499873\\
62.375	0.13692	9.3274040965038\\
62.375	0.14058	10.0410606810033\\
62.375	0.14424	10.8032660284973\\
62.375	0.1479	11.6140201389857\\
62.375	0.15156	12.4733230124685\\
62.375	0.15522	13.3811746489458\\
62.375	0.15888	14.3375750484175\\
62.375	0.16254	15.3425242108836\\
62.375	0.1662	16.3960221363442\\
62.375	0.16986	17.4980688247992\\
62.375	0.17352	18.6486642762487\\
62.375	0.17718	19.8478084906926\\
62.375	0.18084	21.095501468131\\
62.375	0.1845	22.3917432085638\\
62.375	0.18816	23.736533711991\\
62.375	0.19182	25.1298729784127\\
62.375	0.19548	26.5717610078288\\
62.375	0.19914	28.0621978002394\\
62.375	0.2028	29.6011833556443\\
62.375	0.20646	31.1887176740438\\
62.375	0.21012	32.8248007554377\\
62.375	0.21378	34.509432599826\\
62.375	0.21744	36.2426132072087\\
62.375	0.2211	38.0243425775859\\
62.375	0.22476	39.8546207109576\\
62.375	0.22842	41.7334476073237\\
62.375	0.23208	43.6608232666842\\
62.375	0.23574	45.6367476890391\\
62.375	0.2394	47.6612208743886\\
62.375	0.24306	49.7342428227324\\
62.375	0.24672	51.8558135340707\\
62.375	0.25038	54.0259330084034\\
62.375	0.25404	56.2446012457306\\
62.375	0.2577	58.5118182460522\\
62.375	0.26136	60.8275840093682\\
62.375	0.26502	63.1918985356787\\
62.375	0.26868	65.6047618249836\\
62.375	0.27234	68.066173877283\\
62.375	0.276	70.5761346925768\\
62.75	0.093	4.58620921177743\\
62.75	0.09666	4.71488485618953\\
62.75	0.10032	4.89210926359607\\
62.75	0.10398	5.11788243399705\\
62.75	0.10764	5.39220436739247\\
62.75	0.1113	5.71507506378233\\
62.75	0.11496	6.08649452316663\\
62.75	0.11862	6.50646274554538\\
62.75	0.12228	6.97497973091855\\
62.75	0.12594	7.49204547928617\\
62.75	0.1296	8.05765999064823\\
62.75	0.13326	8.67182326500475\\
62.75	0.13692	9.33453530235568\\
62.75	0.14058	10.0457961027011\\
62.75	0.14424	10.8056056660409\\
62.75	0.1479	11.6139639923752\\
62.75	0.15156	12.4708710817039\\
62.75	0.15522	13.376326934027\\
62.75	0.15888	14.3303315493446\\
62.75	0.16254	15.3328849276566\\
62.75	0.1662	16.3839870689631\\
62.75	0.16986	17.483637973264\\
62.75	0.17352	18.6318376405594\\
62.75	0.17718	19.8285860708492\\
62.75	0.18084	21.0738832641334\\
62.75	0.1845	22.367729220412\\
62.75	0.18816	23.7101239396851\\
62.75	0.19182	25.1010674219527\\
62.75	0.19548	26.5405596672147\\
62.75	0.19914	28.0286006754711\\
62.75	0.2028	29.565190446722\\
62.75	0.20646	31.1503289809673\\
62.75	0.21012	32.7840162782071\\
62.75	0.21378	34.4662523384413\\
62.75	0.21744	36.1970371616699\\
62.75	0.2211	37.976370747893\\
62.75	0.22476	39.8042530971105\\
62.75	0.22842	41.6806842093225\\
62.75	0.23208	43.6056640845288\\
62.75	0.23574	45.5791927227297\\
62.75	0.2394	47.601270123925\\
62.75	0.24306	49.6718962881147\\
62.75	0.24672	51.7910712152988\\
62.75	0.25038	53.9587949054774\\
62.75	0.25404	56.1750673586505\\
62.75	0.2577	58.439888574818\\
62.75	0.26136	60.7532585539799\\
62.75	0.26502	63.1151772961362\\
62.75	0.26868	65.525644801287\\
62.75	0.27234	67.9846610694323\\
62.75	0.276	70.4922261005719\\
63.125	0.093	4.62600100534373\\
63.125	0.09666	4.75228086560172\\
63.125	0.10032	4.92710948885412\\
63.125	0.10398	5.15048687510098\\
63.125	0.10764	5.42241302434227\\
63.125	0.1113	5.74288793657801\\
63.125	0.11496	6.11191161180819\\
63.125	0.11862	6.52948405003282\\
63.125	0.12228	6.99560525125187\\
63.125	0.12594	7.51027521546536\\
63.125	0.1296	8.0734939426733\\
63.125	0.13326	8.68526143287567\\
63.125	0.13692	9.3455776860725\\
63.125	0.14058	10.0544427022638\\
63.125	0.14424	10.8118564814495\\
63.125	0.1479	11.6178190236296\\
63.125	0.15156	12.4723303288042\\
63.125	0.15522	13.3753903969732\\
63.125	0.15888	14.3269992281367\\
63.125	0.16254	15.3271568222946\\
63.125	0.1662	16.3758631794469\\
63.125	0.16986	17.4731182995937\\
63.125	0.17352	18.6189221827349\\
63.125	0.17718	19.8132748288706\\
63.125	0.18084	21.0561762380007\\
63.125	0.1845	22.3476264101253\\
63.125	0.18816	23.6876253452442\\
63.125	0.19182	25.0761730433577\\
63.125	0.19548	26.5132695044655\\
63.125	0.19914	27.9989147285678\\
63.125	0.2028	29.5331087156646\\
63.125	0.20646	31.1158514657558\\
63.125	0.21012	32.7471429788414\\
63.125	0.21378	34.4269832549215\\
63.125	0.21744	36.155372293996\\
63.125	0.2211	37.9323100960649\\
63.125	0.22476	39.7577966611284\\
63.125	0.22842	41.6318319891862\\
63.125	0.23208	43.5544160802384\\
63.125	0.23574	45.5255489342852\\
63.125	0.2394	47.5452305513263\\
63.125	0.24306	49.6134609313619\\
63.125	0.24672	51.7302400743919\\
63.125	0.25038	53.8955679804164\\
63.125	0.25404	56.1094446494353\\
63.125	0.2577	58.3718700814487\\
63.125	0.26136	60.6828442764565\\
63.125	0.26502	63.0423672344587\\
63.125	0.26868	65.4504389554554\\
63.125	0.27234	67.9070594394465\\
63.125	0.276	70.4122286864321\\
63.5	0.093	4.66970397677497\\
63.5	0.09666	4.79358805287882\\
63.5	0.10032	4.96602089197711\\
63.5	0.10398	5.18700249406984\\
63.5	0.10764	5.45653285915701\\
63.5	0.1113	5.77461198723862\\
63.5	0.11496	6.14123987831469\\
63.5	0.11862	6.55641653238518\\
63.5	0.12228	7.02014194945011\\
63.5	0.12594	7.53241612950948\\
63.5	0.1296	8.09323907256329\\
63.5	0.13326	8.70261077861156\\
63.5	0.13692	9.36053124765424\\
63.5	0.14058	10.0670004796914\\
63.5	0.14424	10.822018474723\\
63.5	0.1479	11.625585232749\\
63.5	0.15156	12.4777007537694\\
63.5	0.15522	13.3783650377843\\
63.5	0.15888	14.3275780847937\\
63.5	0.16254	15.3253398947975\\
63.5	0.1662	16.3716504677957\\
63.5	0.16986	17.4665098037884\\
63.5	0.17352	18.6099179027755\\
63.5	0.17718	19.801874764757\\
63.5	0.18084	21.042380389733\\
63.5	0.1845	22.3314347777034\\
63.5	0.18816	23.6690379286682\\
63.5	0.19182	25.0551898426276\\
63.5	0.19548	26.4898905195813\\
63.5	0.19914	27.9731399595295\\
63.5	0.2028	29.5049381624721\\
63.5	0.20646	31.0852851284092\\
63.5	0.21012	32.7141808573407\\
63.5	0.21378	34.3916253492666\\
63.5	0.21744	36.117618604187\\
63.5	0.2211	37.8921606221019\\
63.5	0.22476	39.7152514030111\\
63.5	0.22842	41.5868909469148\\
63.5	0.23208	43.507079253813\\
63.5	0.23574	45.4758163237056\\
63.5	0.2394	47.4931021565926\\
63.5	0.24306	49.5589367524741\\
63.5	0.24672	51.67332011135\\
63.5	0.25038	53.8362522332203\\
63.5	0.25404	56.0477331180851\\
63.5	0.2577	58.3077627659444\\
63.5	0.26136	60.616341176798\\
63.5	0.26502	62.9734683506461\\
63.5	0.26868	65.3791442874887\\
63.5	0.27234	67.8333689873257\\
63.5	0.276	70.3361424501571\\
63.875	0.093	4.71731812607115\\
63.875	0.09666	4.83880641802088\\
63.875	0.10032	5.00884347296504\\
63.875	0.10398	5.22742929090365\\
63.875	0.10764	5.49456387183669\\
63.875	0.1113	5.81024721576419\\
63.875	0.11496	6.17447932268612\\
63.875	0.11862	6.58726019260249\\
63.875	0.12228	7.04858982551331\\
63.875	0.12594	7.55846822141856\\
63.875	0.1296	8.11689538031824\\
63.875	0.13326	8.72387130221239\\
63.875	0.13692	9.37939598710094\\
63.875	0.14058	10.083469434984\\
63.875	0.14424	10.8360916458614\\
63.875	0.1479	11.6372626197333\\
63.875	0.15156	12.4869823565996\\
63.875	0.15522	13.3852508564604\\
63.875	0.15888	14.3320681193156\\
63.875	0.16254	15.3274341451653\\
63.875	0.1662	16.3713489340094\\
63.875	0.16986	17.4638124858479\\
63.875	0.17352	18.6048248006809\\
63.875	0.17718	19.7943858785083\\
63.875	0.18084	21.0324957193302\\
63.875	0.1845	22.3191543231465\\
63.875	0.18816	23.6543616899572\\
63.875	0.19182	25.0381178197624\\
63.875	0.19548	26.470422712562\\
63.875	0.19914	27.9512763683561\\
63.875	0.2028	29.4806787871446\\
63.875	0.20646	31.0586299689275\\
63.875	0.21012	32.6851299137049\\
63.875	0.21378	34.3601786214767\\
63.875	0.21744	36.083776092243\\
63.875	0.2211	37.8559223260037\\
63.875	0.22476	39.6766173227588\\
63.875	0.22842	41.5458610825084\\
63.875	0.23208	43.4636536052524\\
63.875	0.23574	45.4299948909909\\
63.875	0.2394	47.4448849397238\\
63.875	0.24306	49.5083237514512\\
63.875	0.24672	51.620311326173\\
63.875	0.25038	53.7808476638892\\
63.875	0.25404	55.9899327645999\\
63.875	0.2577	58.247566628305\\
63.875	0.26136	60.5537492550045\\
63.875	0.26502	62.9084806446985\\
63.875	0.26868	65.3117607973869\\
63.875	0.27234	67.7635897130698\\
63.875	0.276	70.2639673917471\\
64.25	0.093	4.76884345323225\\
64.25	0.09666	4.88793596102785\\
64.25	0.10032	5.05557723181791\\
64.25	0.10398	5.27176726560239\\
64.25	0.10764	5.53650606238131\\
64.25	0.1113	5.84979362215467\\
64.25	0.11496	6.21162994492247\\
64.25	0.11862	6.62201503068473\\
64.25	0.12228	7.08094887944141\\
64.25	0.12594	7.58843149119253\\
64.25	0.1296	8.14446286593811\\
64.25	0.13326	8.74904300367811\\
64.25	0.13692	9.40217190441256\\
64.25	0.14058	10.1038495681414\\
64.25	0.14424	10.8540759948648\\
64.25	0.1479	11.6528511845826\\
64.25	0.15156	12.5001751372948\\
64.25	0.15522	13.3960478530014\\
64.25	0.15888	14.3404693317025\\
64.25	0.16254	15.3334395733981\\
64.25	0.1662	16.374958578088\\
64.25	0.16986	17.4650263457724\\
64.25	0.17352	18.6036428764513\\
64.25	0.17718	19.7908081701246\\
64.25	0.18084	21.0265222267923\\
64.25	0.1845	22.3107850464545\\
64.25	0.18816	23.6435966291111\\
64.25	0.19182	25.0249569747622\\
64.25	0.19548	26.4548660834077\\
64.25	0.19914	27.9333239550476\\
64.25	0.2028	29.460330589682\\
64.25	0.20646	31.0358859873108\\
64.25	0.21012	32.6599901479341\\
64.25	0.21378	34.3326430715518\\
64.25	0.21744	36.0538447581639\\
64.25	0.2211	37.8235952077705\\
64.25	0.22476	39.6418944203715\\
64.25	0.22842	41.508742395967\\
64.25	0.23208	43.4241391345568\\
64.25	0.23574	45.3880846361412\\
64.25	0.2394	47.40057890072\\
64.25	0.24306	49.4616219282932\\
64.25	0.24672	51.5712137188609\\
64.25	0.25038	53.729354272423\\
64.25	0.25404	55.9360435889795\\
64.25	0.2577	58.1912816685305\\
64.25	0.26136	60.4950685110759\\
64.25	0.26502	62.8474041166158\\
64.25	0.26868	65.2482884851501\\
64.25	0.27234	67.6977216166788\\
64.25	0.276	70.195703511202\\
64.625	0.093	4.82427995825829\\
64.625	0.09666	4.94097668189977\\
64.625	0.10032	5.1062221685357\\
64.625	0.10398	5.32001641816604\\
64.625	0.10764	5.58235943079085\\
64.625	0.1113	5.89325120641009\\
64.625	0.11496	6.25269174502378\\
64.625	0.11862	6.6606810466319\\
64.625	0.12228	7.11721911123447\\
64.625	0.12594	7.62230593883147\\
64.625	0.1296	8.1759415294229\\
64.625	0.13326	8.7781258830088\\
64.625	0.13692	9.42885899958912\\
64.625	0.14058	10.1281408791639\\
64.625	0.14424	10.8759715217331\\
64.625	0.1479	11.6723509272967\\
64.625	0.15156	12.5172790958548\\
64.625	0.15522	13.4107560274073\\
64.625	0.15888	14.3527817219543\\
64.625	0.16254	15.3433561794957\\
64.625	0.1662	16.3824794000316\\
64.625	0.16986	17.4701513835619\\
64.625	0.17352	18.6063721300866\\
64.625	0.17718	19.7911416396058\\
64.625	0.18084	21.0244599121194\\
64.625	0.1845	22.3063269476274\\
64.625	0.18816	23.6367427461299\\
64.625	0.19182	25.0157073076269\\
64.625	0.19548	26.4432206321182\\
64.625	0.19914	27.919282719604\\
64.625	0.2028	29.4438935700843\\
64.625	0.20646	31.017053183559\\
64.625	0.21012	32.6387615600281\\
64.625	0.21378	34.3090186994917\\
64.625	0.21744	36.0278246019497\\
64.625	0.2211	37.7951792674022\\
64.625	0.22476	39.6110826958491\\
64.625	0.22842	41.4755348872904\\
64.625	0.23208	43.3885358417262\\
64.625	0.23574	45.3500855591564\\
64.625	0.2394	47.3601840395811\\
64.625	0.24306	49.4188312830002\\
64.625	0.24672	51.5260272894137\\
64.625	0.25038	53.6817720588217\\
64.625	0.25404	55.8860655912241\\
64.625	0.2577	58.138907886621\\
64.625	0.26136	60.4402989450123\\
64.625	0.26502	62.790238766398\\
64.625	0.26868	65.1887273507782\\
64.625	0.27234	67.6357646981528\\
64.625	0.276	70.1313508085219\\
65	0.093	4.88362764114926\\
65	0.09666	4.99792858063661\\
65	0.10032	5.16077828311841\\
65	0.10398	5.37217674859464\\
65	0.10764	5.63212397706531\\
65	0.1113	5.94061996853043\\
65	0.11496	6.29766472299\\
65	0.11862	6.703258240444\\
65	0.12228	7.15740052089243\\
65	0.12594	7.66009156433531\\
65	0.1296	8.21133137077263\\
65	0.13326	8.81111994020439\\
65	0.13692	9.45945727263059\\
65	0.14058	10.1563433680512\\
65	0.14424	10.9017782264663\\
65	0.1479	11.6957618478758\\
65	0.15156	12.5382942322798\\
65	0.15522	13.4293753796782\\
65	0.15888	14.3690052900711\\
65	0.16254	15.3571839634584\\
65	0.1662	16.3939113998401\\
65	0.16986	17.4791875992162\\
65	0.17352	18.6130125615868\\
65	0.17718	19.7953862869519\\
65	0.18084	21.0263087753114\\
65	0.1845	22.3057800266653\\
65	0.18816	23.6338000410137\\
65	0.19182	25.0103688183565\\
65	0.19548	26.4354863586937\\
65	0.19914	27.9091526620254\\
65	0.2028	29.4313677283516\\
65	0.20646	31.0021315576721\\
65	0.21012	32.6214441499871\\
65	0.21378	34.2893055052966\\
65	0.21744	36.0057156236005\\
65	0.2211	37.7706745048988\\
65	0.22476	39.5841821491916\\
65	0.22842	41.4462385564788\\
65	0.23208	43.3568437267604\\
65	0.23574	45.3159976600366\\
65	0.2394	47.3237003563071\\
65	0.24306	49.3799518155721\\
65	0.24672	51.4847520378315\\
65	0.25038	53.6381010230854\\
65	0.25404	55.8399987713336\\
65	0.2577	58.0904452825764\\
65	0.26136	60.3894405568135\\
65	0.26502	62.7369845940452\\
65	0.26868	65.1330773942712\\
65	0.27234	67.5777189574917\\
65	0.276	70.0709092837066\\
65.375	0.093	4.94688650190516\\
65.375	0.09666	5.05879165723839\\
65.375	0.10032	5.21924557556607\\
65.375	0.10398	5.42824825688817\\
65.375	0.10764	5.68579970120472\\
65.375	0.1113	5.99189990851571\\
65.375	0.11496	6.34654887882116\\
65.375	0.11862	6.74974661212104\\
65.375	0.12228	7.20149310841535\\
65.375	0.12594	7.70178836770411\\
65.375	0.1296	8.2506323899873\\
65.375	0.13326	8.84802517526494\\
65.375	0.13692	9.49396672353701\\
65.375	0.14058	10.1884570348035\\
65.375	0.14424	10.9314961090645\\
65.375	0.1479	11.7230839463199\\
65.375	0.15156	12.5632205465697\\
65.375	0.15522	13.451905909814\\
65.375	0.15888	14.3891400360527\\
65.375	0.16254	15.3749229252859\\
65.375	0.1662	16.4092545775135\\
65.375	0.16986	17.4921349927356\\
65.375	0.17352	18.623564170952\\
65.375	0.17718	19.8035421121629\\
65.375	0.18084	21.0320688163683\\
65.375	0.1845	22.3091442835681\\
65.375	0.18816	23.6347685137624\\
65.375	0.19182	25.008941506951\\
65.375	0.19548	26.4316632631342\\
65.375	0.19914	27.9029337823117\\
65.375	0.2028	29.4227530644837\\
65.375	0.20646	30.9911211096502\\
65.375	0.21012	32.6080379178111\\
65.375	0.21378	34.2735034889664\\
65.375	0.21744	35.9875178231162\\
65.375	0.2211	37.7500809202604\\
65.375	0.22476	39.561192780399\\
65.375	0.22842	41.4208534035321\\
65.375	0.23208	43.3290627896596\\
65.375	0.23574	45.2858209387816\\
65.375	0.2394	47.291127850898\\
65.375	0.24306	49.3449835260089\\
65.375	0.24672	51.4473879641142\\
65.375	0.25038	53.5983411652139\\
65.375	0.25404	55.7978431293081\\
65.375	0.2577	58.0458938563967\\
65.375	0.26136	60.3424933464798\\
65.375	0.26502	62.6876415995572\\
65.375	0.26868	65.0813386156292\\
65.375	0.27234	67.5235843946956\\
65.375	0.276	70.0143789367564\\
65.75	0.093	5.01405654052597\\
65.75	0.09666	5.12356591170509\\
65.75	0.10032	5.28162404587865\\
65.75	0.10398	5.48823094304662\\
65.75	0.10764	5.74338660320906\\
65.75	0.1113	6.04709102636593\\
65.75	0.11496	6.39934421251723\\
65.75	0.11862	6.800146161663\\
65.75	0.12228	7.24949687380319\\
65.75	0.12594	7.74739634893782\\
65.75	0.1296	8.29384458706688\\
65.75	0.13326	8.8888415881904\\
65.75	0.13692	9.53238735230835\\
65.75	0.14058	10.2244818794207\\
65.75	0.14424	10.9651251695276\\
65.75	0.1479	11.7543172226289\\
65.75	0.15156	12.5920580387246\\
65.75	0.15522	13.4783476178147\\
65.75	0.15888	14.4131859598993\\
65.75	0.16254	15.3965730649784\\
65.75	0.1662	16.4285089330519\\
65.75	0.16986	17.5089935641198\\
65.75	0.17352	18.6380269581821\\
65.75	0.17718	19.8156091152389\\
65.75	0.18084	21.0417400352902\\
65.75	0.1845	22.3164197183359\\
65.75	0.18816	23.639648164376\\
65.75	0.19182	25.0114253734105\\
65.75	0.19548	26.4317513454395\\
65.75	0.19914	27.900626080463\\
65.75	0.2028	29.4180495784809\\
65.75	0.20646	30.9840218394932\\
65.75	0.21012	32.5985428634999\\
65.75	0.21378	34.2616126505011\\
65.75	0.21744	35.9732312004968\\
65.75	0.2211	37.7333985134869\\
65.75	0.22476	39.5421145894714\\
65.75	0.22842	41.3993794284504\\
65.75	0.23208	43.3051930304238\\
65.75	0.23574	45.2595553953916\\
65.75	0.2394	47.2624665233539\\
65.75	0.24306	49.3139264143107\\
65.75	0.24672	51.4139350682618\\
65.75	0.25038	53.5624924852074\\
65.75	0.25404	55.7595986651475\\
65.75	0.2577	58.005253608082\\
65.75	0.26136	60.2994573140109\\
65.75	0.26502	62.6422097829343\\
65.75	0.26868	65.0335110148521\\
65.75	0.27234	67.4733610097643\\
65.75	0.276	69.961759767671\\
66.125	0.093	5.08513775701175\\
66.125	0.09666	5.19225134403675\\
66.125	0.10032	5.34791369405618\\
66.125	0.10398	5.55212480707004\\
66.125	0.10764	5.80488468307834\\
66.125	0.1113	6.10619332208109\\
66.125	0.11496	6.45605072407828\\
66.125	0.11862	6.85445688906992\\
66.125	0.12228	7.30141181705598\\
66.125	0.12594	7.79691550803649\\
66.125	0.1296	8.34096796201143\\
66.125	0.13326	8.93356917898082\\
66.125	0.13692	9.57471915894465\\
66.125	0.14058	10.2644179019029\\
66.125	0.14424	11.0026654078556\\
66.125	0.1479	11.7894616768028\\
66.125	0.15156	12.6248067087444\\
66.125	0.15522	13.5087005036804\\
66.125	0.15888	14.4411430616109\\
66.125	0.16254	15.4221343825358\\
66.125	0.1662	16.4516744664552\\
66.125	0.16986	17.529763313369\\
66.125	0.17352	18.6564009232772\\
66.125	0.17718	19.8315872961798\\
66.125	0.18084	21.055322432077\\
66.125	0.1845	22.3276063309685\\
66.125	0.18816	23.6484389928545\\
66.125	0.19182	25.017820417735\\
66.125	0.19548	26.4357506056098\\
66.125	0.19914	27.9022295564792\\
66.125	0.2028	29.4172572703429\\
66.125	0.20646	30.9808337472011\\
66.125	0.21012	32.5929589870538\\
66.125	0.21378	34.2536329899009\\
66.125	0.21744	35.9628557557424\\
66.125	0.2211	37.7206272845783\\
66.125	0.22476	39.5269475764087\\
66.125	0.22842	41.3818166312336\\
66.125	0.23208	43.2852344490529\\
66.125	0.23574	45.2372010298666\\
66.125	0.2394	47.2377163736748\\
66.125	0.24306	49.2867804804774\\
66.125	0.24672	51.3843933502744\\
66.125	0.25038	53.5305549830659\\
66.125	0.25404	55.7252653788518\\
66.125	0.2577	57.9685245376322\\
66.125	0.26136	60.260332459407\\
66.125	0.26502	62.6006891441762\\
66.125	0.26868	64.9895945919399\\
66.125	0.27234	67.427048802698\\
66.125	0.276	69.9130517764506\\
66.5	0.093	5.16013015136247\\
66.5	0.09666	5.26484795423332\\
66.5	0.10032	5.41811452009863\\
66.5	0.10398	5.61992984895837\\
66.5	0.10764	5.87029394081254\\
66.5	0.1113	6.16920679566116\\
66.5	0.11496	6.51666841350425\\
66.5	0.11862	6.91267879434175\\
66.5	0.12228	7.35723793817369\\
66.5	0.12594	7.85034584500007\\
66.5	0.1296	8.39200251482088\\
66.5	0.13326	8.98220794763617\\
66.5	0.13692	9.62096214344586\\
66.5	0.14058	10.30826510225\\
66.5	0.14424	11.0441168240486\\
66.5	0.1479	11.8285173088416\\
66.5	0.15156	12.6614665566291\\
66.5	0.15522	13.542964567411\\
66.5	0.15888	14.4730113411874\\
66.5	0.16254	15.4516068779581\\
66.5	0.1662	16.4787511777234\\
66.5	0.16986	17.5544442404831\\
66.5	0.17352	18.6786860662372\\
66.5	0.17718	19.8514766549857\\
66.5	0.18084	21.0728160067287\\
66.5	0.1845	22.3427041214661\\
66.5	0.18816	23.661140999198\\
66.5	0.19182	25.0281266399243\\
66.5	0.19548	26.4436610436451\\
66.5	0.19914	27.9077442103603\\
66.5	0.2028	29.4203761400699\\
66.5	0.20646	30.981556832774\\
66.5	0.21012	32.5912862884725\\
66.5	0.21378	34.2495645071655\\
66.5	0.21744	35.9563914888529\\
66.5	0.2211	37.7117672335347\\
66.5	0.22476	39.515691741211\\
66.5	0.22842	41.3681650118817\\
66.5	0.23208	43.2691870455468\\
66.5	0.23574	45.2187578422065\\
66.5	0.2394	47.2168774018605\\
66.5	0.24306	49.263545724509\\
66.5	0.24672	51.3587628101519\\
66.5	0.25038	53.5025286587893\\
66.5	0.25404	55.6948432704211\\
66.5	0.2577	57.9357066450473\\
66.5	0.26136	60.225118782668\\
66.5	0.26502	62.5630796832831\\
66.5	0.26868	64.9495893468927\\
66.5	0.27234	67.3846477734967\\
66.5	0.276	69.8682549630951\\
66.875	0.093	5.23903372357809\\
66.875	0.09666	5.34135574229483\\
66.875	0.10032	5.49222652400601\\
66.875	0.10398	5.69164606871162\\
66.875	0.10764	5.93961437641168\\
66.875	0.1113	6.23613144710617\\
66.875	0.11496	6.58119728079513\\
66.875	0.11862	6.97481187747852\\
66.875	0.12228	7.41697523715633\\
66.875	0.12594	7.90768735982859\\
66.875	0.1296	8.44694824549528\\
66.875	0.13326	9.03475789415642\\
66.875	0.13692	9.67111630581201\\
66.875	0.14058	10.356023480462\\
66.875	0.14424	11.0894794181065\\
66.875	0.1479	11.8714841187454\\
66.875	0.15156	12.7020375823787\\
66.875	0.15522	13.5811398090066\\
66.875	0.15888	14.5087907986288\\
66.875	0.16254	15.4849905512454\\
66.875	0.1662	16.5097390668565\\
66.875	0.16986	17.5830363454621\\
66.875	0.17352	18.7048823870621\\
66.875	0.17718	19.8752771916565\\
66.875	0.18084	21.0942207592454\\
66.875	0.1845	22.3617130898287\\
66.875	0.18816	23.6777541834064\\
66.875	0.19182	25.0423440399786\\
66.875	0.19548	26.4554826595452\\
66.875	0.19914	27.9171700421063\\
66.875	0.2028	29.4274061876618\\
66.875	0.20646	30.9861910962118\\
66.875	0.21012	32.5935247677562\\
66.875	0.21378	34.249407202295\\
66.875	0.21744	35.9538383998283\\
66.875	0.2211	37.706818360356\\
66.875	0.22476	39.5083470838782\\
66.875	0.22842	41.3584245703947\\
66.875	0.23208	43.2570508199058\\
66.875	0.23574	45.2042258324113\\
66.875	0.2394	47.1999496079112\\
66.875	0.24306	49.2442221464056\\
66.875	0.24672	51.3370434478943\\
66.875	0.25038	53.4784135123776\\
66.875	0.25404	55.6683323398552\\
66.875	0.2577	57.9067999303274\\
66.875	0.26136	60.1938162837939\\
66.875	0.26502	62.5293814002549\\
66.875	0.26868	64.9134952797104\\
66.875	0.27234	67.3461579221602\\
66.875	0.276	69.8273693276045\\
67.25	0.093	5.32184847365865\\
67.25	0.09666	5.42177470822128\\
67.25	0.10032	5.57024970577833\\
67.25	0.10398	5.76727346632982\\
67.25	0.10764	6.01284598987576\\
67.25	0.1113	6.30696727641612\\
67.25	0.11496	6.64963732595096\\
67.25	0.11862	7.0408561384802\\
67.25	0.12228	7.4806237140039\\
67.25	0.12594	7.96894005252204\\
67.25	0.1296	8.50580515403461\\
67.25	0.13326	9.09121901854164\\
67.25	0.13692	9.72518164604309\\
67.25	0.14058	10.407693036539\\
67.25	0.14424	11.1387531900293\\
67.25	0.1479	11.9183621065141\\
67.25	0.15156	12.7465197859933\\
67.25	0.15522	13.623226228467\\
67.25	0.15888	14.5484814339351\\
67.25	0.16254	15.5222854023976\\
67.25	0.1662	16.5446381338546\\
67.25	0.16986	17.6155396283061\\
67.25	0.17352	18.7349898857519\\
67.25	0.17718	19.9029889061922\\
67.25	0.18084	21.1195366896269\\
67.25	0.1845	22.3846332360562\\
67.25	0.18816	23.6982785454798\\
67.25	0.19182	25.0604726178978\\
67.25	0.19548	26.4712154533103\\
67.25	0.19914	27.9305070517173\\
67.25	0.2028	29.4383474131187\\
67.25	0.20646	30.9947365375145\\
67.25	0.21012	32.5996744249048\\
67.25	0.21378	34.2531610752895\\
67.25	0.21744	35.9551964886686\\
67.25	0.2211	37.7057806650422\\
67.25	0.22476	39.5049136044103\\
67.25	0.22842	41.3525953067727\\
67.25	0.23208	43.2488257721296\\
67.25	0.23574	45.193605000481\\
67.25	0.2394	47.1869329918268\\
67.25	0.24306	49.228809746167\\
67.25	0.24672	51.3192352635017\\
67.25	0.25038	53.4582095438308\\
67.25	0.25404	55.6457325871544\\
67.25	0.2577	57.8818043934724\\
67.25	0.26136	60.1664249627848\\
67.25	0.26502	62.4995942950917\\
67.25	0.26868	64.881312390393\\
67.25	0.27234	67.3115792486888\\
67.25	0.276	69.7903948699789\\
67.625	0.093	5.40857440160415\\
67.625	0.09666	5.50610485201265\\
67.625	0.10032	5.65218406541558\\
67.625	0.10398	5.84681204181295\\
67.625	0.10764	6.08998878120476\\
67.625	0.1113	6.38171428359101\\
67.625	0.11496	6.7219885489717\\
67.625	0.11862	7.11081157734684\\
67.625	0.12228	7.54818336871642\\
67.625	0.12594	8.03410392308043\\
67.625	0.1296	8.56857324043887\\
67.625	0.13326	9.15159132079177\\
67.625	0.13692	9.78315816413911\\
67.625	0.14058	10.4632737704809\\
67.625	0.14424	11.1919381398171\\
67.625	0.1479	11.9691512721478\\
67.625	0.15156	12.7949131674728\\
67.625	0.15522	13.6692238257924\\
67.625	0.15888	14.5920832471064\\
67.625	0.16254	15.5634914314148\\
67.625	0.1662	16.5834483787177\\
67.625	0.16986	17.651954089015\\
67.625	0.17352	18.7690085623067\\
67.625	0.17718	19.9346117985929\\
67.625	0.18084	21.1487637978735\\
67.625	0.1845	22.4114645601486\\
67.625	0.18816	23.7227140854181\\
67.625	0.19182	25.082512373682\\
67.625	0.19548	26.4908594249404\\
67.625	0.19914	27.9477552391932\\
67.625	0.2028	29.4531998164405\\
67.625	0.20646	31.0071931566822\\
67.625	0.21012	32.6097352599183\\
67.625	0.21378	34.2608261261489\\
67.625	0.21744	35.9604657553739\\
67.625	0.2211	37.7086541475934\\
67.625	0.22476	39.5053913028073\\
67.625	0.22842	41.3506772210157\\
67.625	0.23208	43.2445119022184\\
67.625	0.23574	45.1868953464157\\
67.625	0.2394	47.1778275536074\\
67.625	0.24306	49.2173085237935\\
67.625	0.24672	51.305338256974\\
67.625	0.25038	53.441916753149\\
67.625	0.25404	55.6270440123184\\
67.625	0.2577	57.8607200344823\\
67.625	0.26136	60.1429448196406\\
67.625	0.26502	62.4737183677934\\
67.625	0.26868	64.8530406789405\\
67.625	0.27234	67.2809117530822\\
67.625	0.276	69.7573315902182\\
68	0.093	5.49921150741458\\
68	0.09666	5.59434617366896\\
68	0.10032	5.73802960291775\\
68	0.10398	5.93026179516099\\
68	0.10764	6.17104275039868\\
68	0.1113	6.46037246863082\\
68	0.11496	6.7982509498574\\
68	0.11862	7.18467819407839\\
68	0.12228	7.61965420129384\\
68	0.12594	8.10317897150373\\
68	0.1296	8.63525250470806\\
68	0.13326	9.21587480090685\\
68	0.13692	9.84504586010005\\
68	0.14058	10.5227656822877\\
68	0.14424	11.2490342674698\\
68	0.1479	12.0238516156463\\
68	0.15156	12.8472177268173\\
68	0.15522	13.7191326009827\\
68	0.15888	14.6395962381426\\
68	0.16254	15.6086086382969\\
68	0.1662	16.6261698014456\\
68	0.16986	17.6922797275888\\
68	0.17352	18.8069384167264\\
68	0.17718	19.9701458688585\\
68	0.18084	21.1819020839849\\
68	0.1845	22.4422070621059\\
68	0.18816	23.7510608032213\\
68	0.19182	25.1084633073311\\
68	0.19548	26.5144145744353\\
68	0.19914	27.968914604534\\
68	0.2028	29.4719633976272\\
68	0.20646	31.0235609537148\\
68	0.21012	32.6237072727968\\
68	0.21378	34.2724023548733\\
68	0.21744	35.9696461999442\\
68	0.2211	37.7154388080095\\
68	0.22476	39.5097801790693\\
68	0.22842	41.3526703131235\\
68	0.23208	43.2441092101722\\
68	0.23574	45.1840968702153\\
68	0.2394	47.1726332932528\\
68	0.24306	49.2097184792848\\
68	0.24672	51.2953524283112\\
68	0.25038	53.4295351403321\\
68	0.25404	55.6122666153474\\
68	0.2577	57.8435468533572\\
68	0.26136	60.1233758543613\\
68	0.26502	62.45175361836\\
68	0.26868	64.828680145353\\
68	0.27234	67.2541554353405\\
68	0.276	69.7281794883225\\
68.375	0.093	5.59375979108995\\
68.375	0.09666	5.6864986731902\\
68.375	0.10032	5.82778631828488\\
68.375	0.10398	6.01762272637401\\
68.375	0.10764	6.25600789745756\\
68.375	0.1113	6.54294183153558\\
68.375	0.11496	6.87842452860803\\
68.375	0.11862	7.26245598867491\\
68.375	0.12228	7.69503621173624\\
68.375	0.12594	8.176165197792\\
68.375	0.1296	8.70584294684219\\
68.375	0.13326	9.28406945888686\\
68.375	0.13692	9.91084473392593\\
68.375	0.14058	10.5861687719595\\
68.375	0.14424	11.3100415729874\\
68.375	0.1479	12.0824631370098\\
68.375	0.15156	12.9034334640267\\
68.375	0.15522	13.772952554038\\
68.375	0.15888	14.6910204070437\\
68.375	0.16254	15.6576370230439\\
68.375	0.1662	16.6728024020385\\
68.375	0.16986	17.7365165440276\\
68.375	0.17352	18.848779449011\\
68.375	0.17718	20.009591116989\\
68.375	0.18084	21.2189515479613\\
68.375	0.1845	22.4768607419282\\
68.375	0.18816	23.7833186988894\\
68.375	0.19182	25.1383254188451\\
68.375	0.19548	26.5418809017953\\
68.375	0.19914	27.9939851477398\\
68.375	0.2028	29.4946381566788\\
68.375	0.20646	31.0438399286123\\
68.375	0.21012	32.6415904635402\\
68.375	0.21378	34.2878897614625\\
68.375	0.21744	35.9827378223793\\
68.375	0.2211	37.7261346462905\\
68.375	0.22476	39.5180802331962\\
68.375	0.22842	41.3585745830963\\
68.375	0.23208	43.2476176959908\\
68.375	0.23574	45.1852095718798\\
68.375	0.2394	47.1713502107633\\
68.375	0.24306	49.2060396126411\\
68.375	0.24672	51.2892777775134\\
68.375	0.25038	53.4210647053802\\
68.375	0.25404	55.6014003962413\\
68.375	0.2577	57.830284850097\\
68.375	0.26136	60.107718066947\\
68.375	0.26502	62.4337000467915\\
68.375	0.26868	64.8082307896305\\
68.375	0.27234	67.2313102954639\\
68.375	0.276	69.7029385642917\\
68.75	0.093	5.69221925263027\\
68.75	0.09666	5.78256235057639\\
68.75	0.10032	5.92145421151693\\
68.75	0.10398	6.10889483545193\\
68.75	0.10764	6.34488422238137\\
68.75	0.1113	6.62942237230525\\
68.75	0.11496	6.96250928522357\\
68.75	0.11862	7.34414496113634\\
68.75	0.12228	7.77432940004354\\
68.75	0.12594	8.25306260194519\\
68.75	0.1296	8.78034456684126\\
68.75	0.13326	9.35617529473178\\
68.75	0.13692	9.98055478561675\\
68.75	0.14058	10.6534830394962\\
68.75	0.14424	11.37496005637\\
68.75	0.1479	12.1449858362383\\
68.75	0.15156	12.963560379101\\
68.75	0.15522	13.8306836849582\\
68.75	0.15888	14.7463557538098\\
68.75	0.16254	15.7105765856559\\
68.75	0.1662	16.7233461804963\\
68.75	0.16986	17.7846645383313\\
68.75	0.17352	18.8945316591606\\
68.75	0.17718	20.0529475429844\\
68.75	0.18084	21.2599121898027\\
68.75	0.1845	22.5154255996154\\
68.75	0.18816	23.8194877724225\\
68.75	0.19182	25.1720987082241\\
68.75	0.19548	26.5732584070201\\
68.75	0.19914	28.0229668688106\\
68.75	0.2028	29.5212240935954\\
68.75	0.20646	31.0680300813748\\
68.75	0.21012	32.6633848321485\\
68.75	0.21378	34.3072883459168\\
68.75	0.21744	35.9997406226794\\
68.75	0.2211	37.7407416624365\\
68.75	0.22476	39.5302914651881\\
68.75	0.22842	41.368390030934\\
68.75	0.23208	43.2550373596744\\
68.75	0.23574	45.1902334514093\\
68.75	0.2394	47.1739783061386\\
68.75	0.24306	49.2062719238623\\
68.75	0.24672	51.2871143045805\\
68.75	0.25038	53.4165054482932\\
68.75	0.25404	55.5944453550002\\
68.75	0.2577	57.8209340247017\\
68.75	0.26136	60.0959714573976\\
68.75	0.26502	62.419557653088\\
68.75	0.26868	64.7916926117728\\
68.75	0.27234	67.2123763334521\\
68.75	0.276	69.6816088181258\\
69.125	0.093	5.79458989203549\\
69.125	0.09666	5.88253720582749\\
69.125	0.10032	6.01903328261394\\
69.125	0.10398	6.2040781223948\\
69.125	0.10764	6.43767172517012\\
69.125	0.1113	6.71981409093987\\
69.125	0.11496	7.05050521970409\\
69.125	0.11862	7.42974511146272\\
69.125	0.12228	7.8575337662158\\
69.125	0.12594	8.33387118396331\\
69.125	0.1296	8.85875736470527\\
69.125	0.13326	9.43219230844167\\
69.125	0.13692	10.0541760151725\\
69.125	0.14058	10.7247084848978\\
69.125	0.14424	11.4437897176175\\
69.125	0.1479	12.2114197133317\\
69.125	0.15156	13.0275984720403\\
69.125	0.15522	13.8923259937433\\
69.125	0.15888	14.8056022784408\\
69.125	0.16254	15.7674273261327\\
69.125	0.1662	16.7778011368191\\
69.125	0.16986	17.8367237104999\\
69.125	0.17352	18.9441950471751\\
69.125	0.17718	20.1002151468448\\
69.125	0.18084	21.3047840095089\\
69.125	0.1845	22.5579016351675\\
69.125	0.18816	23.8595680238205\\
69.125	0.19182	25.209783175468\\
69.125	0.19548	26.6085470901099\\
69.125	0.19914	28.0558597677462\\
69.125	0.2028	29.551721208377\\
69.125	0.20646	31.0961314120022\\
69.125	0.21012	32.6890903786218\\
69.125	0.21378	34.3305981082359\\
69.125	0.21744	36.0206546008444\\
69.125	0.2211	37.7592598564474\\
69.125	0.22476	39.5464138750448\\
69.125	0.22842	41.3821166566367\\
69.125	0.23208	43.266368201223\\
69.125	0.23574	45.1991685088037\\
69.125	0.2394	47.1805175793789\\
69.125	0.24306	49.2104154129485\\
69.125	0.24672	51.2888620095126\\
69.125	0.25038	53.4158573690711\\
69.125	0.25404	55.591401491624\\
69.125	0.2577	57.8154943771714\\
69.125	0.26136	60.0881360257132\\
69.125	0.26502	62.4093264372494\\
69.125	0.26868	64.7790656117801\\
69.125	0.27234	67.1973535493053\\
69.125	0.276	69.6641902498248\\
69.5	0.093	5.90087170930567\\
69.5	0.09666	5.98642323894353\\
69.5	0.10032	6.12052353157585\\
69.5	0.10398	6.30317258720259\\
69.5	0.10764	6.53437040582378\\
69.5	0.1113	6.81411698743941\\
69.5	0.11496	7.14241233204949\\
69.5	0.11862	7.51925643965401\\
69.5	0.12228	7.94464931025296\\
69.5	0.12594	8.41859094384635\\
69.5	0.1296	8.94108134043418\\
69.5	0.13326	9.51212050001645\\
69.5	0.13692	10.1317084225932\\
69.5	0.14058	10.7998451081643\\
69.5	0.14424	11.5165305567299\\
69.5	0.1479	12.28176476829\\
69.5	0.15156	13.0955477428445\\
69.5	0.15522	13.9578794803934\\
69.5	0.15888	14.8687599809367\\
69.5	0.16254	15.8281892444746\\
69.5	0.1662	16.8361672710068\\
69.5	0.16986	17.8926940605335\\
69.5	0.17352	18.9977696130546\\
69.5	0.17718	20.1513939285701\\
69.5	0.18084	21.3535670070801\\
69.5	0.1845	22.6042888485846\\
69.5	0.18816	23.9035594530835\\
69.5	0.19182	25.2513788205768\\
69.5	0.19548	26.6477469510645\\
69.5	0.19914	28.0926638445468\\
69.5	0.2028	29.5861295010234\\
69.5	0.20646	31.1281439204945\\
69.5	0.21012	32.71870710296\\
69.5	0.21378	34.35781904842\\
69.5	0.21744	36.0454797568744\\
69.5	0.2211	37.7816892283232\\
69.5	0.22476	39.5664474627665\\
69.5	0.22842	41.3997544602043\\
69.5	0.23208	43.2816102206364\\
69.5	0.23574	45.2120147440631\\
69.5	0.2394	47.1909680304841\\
69.5	0.24306	49.2184700798996\\
69.5	0.24672	51.2945208923095\\
69.5	0.25038	53.4191204677139\\
69.5	0.25404	55.5922688061127\\
69.5	0.2577	57.813965907506\\
69.5	0.26136	60.0842117718936\\
69.5	0.26502	62.4030063992758\\
69.5	0.26868	64.7703497896524\\
69.5	0.27234	67.1862419430233\\
69.5	0.276	69.6506828593888\\
69.875	0.093	6.01106470444075\\
69.875	0.09666	6.0942204499245\\
69.875	0.10032	6.22592495840269\\
69.875	0.10398	6.40617822987532\\
69.875	0.10764	6.63498026434237\\
69.875	0.1113	6.91233106180389\\
69.875	0.11496	7.23823062225985\\
69.875	0.11862	7.61267894571024\\
69.875	0.12228	8.03567603215507\\
69.875	0.12594	8.50722188159433\\
69.875	0.1296	9.02731649402804\\
69.875	0.13326	9.59595986945619\\
69.875	0.13692	10.2131520078788\\
69.875	0.14058	10.8788929092958\\
69.875	0.14424	11.5931825737073\\
69.875	0.1479	12.3560210011132\\
69.875	0.15156	13.1674081915136\\
69.875	0.15522	14.0273441449084\\
69.875	0.15888	14.9358288612976\\
69.875	0.16254	15.8928623406813\\
69.875	0.1662	16.8984445830594\\
69.875	0.16986	17.952575588432\\
69.875	0.17352	19.0552553567989\\
69.875	0.17718	20.2064838881604\\
69.875	0.18084	21.4062611825163\\
69.875	0.1845	22.6545872398666\\
69.875	0.18816	23.9514620602113\\
69.875	0.19182	25.2968856435505\\
69.875	0.19548	26.6908579898842\\
69.875	0.19914	28.1333790992123\\
69.875	0.2028	29.6244489715348\\
69.875	0.20646	31.1640676068517\\
69.875	0.21012	32.7522350051632\\
69.875	0.21378	34.388951166469\\
69.875	0.21744	36.0742160907693\\
69.875	0.2211	37.808029778064\\
69.875	0.22476	39.5903922283532\\
69.875	0.22842	41.4213034416368\\
69.875	0.23208	43.3007634179148\\
69.875	0.23574	45.2287721571873\\
69.875	0.2394	47.2053296594542\\
69.875	0.24306	49.2304359247156\\
69.875	0.24672	51.3040909529714\\
69.875	0.25038	53.4262947442217\\
69.875	0.25404	55.5970472984664\\
69.875	0.2577	57.8163486157055\\
69.875	0.26136	60.084198695939\\
69.875	0.26502	62.4005975391671\\
69.875	0.26868	64.7655451453895\\
69.875	0.27234	67.1790415146064\\
69.875	0.276	69.6410866468177\\
70.25	0.093	6.12516887744079\\
70.25	0.09666	6.20592883877041\\
70.25	0.10032	6.33523756309447\\
70.25	0.10398	6.51309505041296\\
70.25	0.10764	6.73950130072591\\
70.25	0.1113	7.0144563140333\\
70.25	0.11496	7.33796009033513\\
70.25	0.11862	7.7100126296314\\
70.25	0.12228	8.1306139319221\\
70.25	0.12594	8.59976399720724\\
70.25	0.1296	9.11746282548684\\
70.25	0.13326	9.68371041676086\\
70.25	0.13692	10.2985067710293\\
70.25	0.14058	10.9618518882922\\
70.25	0.14424	11.6737457685496\\
70.25	0.1479	12.4341884118014\\
70.25	0.15156	13.2431798180476\\
70.25	0.15522	14.1007199872883\\
70.25	0.15888	15.0068089195234\\
70.25	0.16254	15.961446614753\\
70.25	0.1662	16.964633072977\\
70.25	0.16986	18.0163682941954\\
70.25	0.17352	19.1166522784083\\
70.25	0.17718	20.2654850256156\\
70.25	0.18084	21.4628665358173\\
70.25	0.1845	22.7087968090135\\
70.25	0.18816	24.0032758452042\\
70.25	0.19182	25.3463036443892\\
70.25	0.19548	26.7378802065687\\
70.25	0.19914	28.1780055317427\\
70.25	0.2028	29.6666796199111\\
70.25	0.20646	31.203902471074\\
70.25	0.21012	32.7896740852312\\
70.25	0.21378	34.423994462383\\
70.25	0.21744	36.1068636025291\\
70.25	0.2211	37.8382815056697\\
70.25	0.22476	39.6182481718048\\
70.25	0.22842	41.4467636009342\\
70.25	0.23208	43.3238277930581\\
70.25	0.23574	45.2494407481765\\
70.25	0.2394	47.2236024662893\\
70.25	0.24306	49.2463129473966\\
70.25	0.24672	51.3175721914982\\
70.25	0.25038	53.4373801985944\\
70.25	0.25404	55.6057369686849\\
70.25	0.2577	57.82264250177\\
70.25	0.26136	60.0880967978494\\
70.25	0.26502	62.4020998569233\\
70.25	0.26868	64.7646516789916\\
70.25	0.27234	67.1757522640543\\
70.25	0.276	69.6354016121116\\
70.625	0.093	6.24318422830574\\
70.625	0.09666	6.32154840548124\\
70.625	0.10032	6.44846134565119\\
70.625	0.10398	6.62392304881556\\
70.625	0.10764	6.84793351497439\\
70.625	0.1113	7.12049274412764\\
70.625	0.11496	7.44160073627536\\
70.625	0.11862	7.8112574914175\\
70.625	0.12228	8.22946300955408\\
70.625	0.12594	8.69621729068509\\
70.625	0.1296	9.21152033481055\\
70.625	0.13326	9.77537214193048\\
70.625	0.13692	10.3877727120448\\
70.625	0.14058	11.0487220451536\\
70.625	0.14424	11.7582201412568\\
70.625	0.1479	12.5162670003545\\
70.625	0.15156	13.3228626224466\\
70.625	0.15522	14.1780070075332\\
70.625	0.15888	15.0817001556141\\
70.625	0.16254	16.0339420666896\\
70.625	0.1662	17.0347327407595\\
70.625	0.16986	18.0840721778237\\
70.625	0.17352	19.1819603778825\\
70.625	0.17718	20.3283973409357\\
70.625	0.18084	21.5233830669833\\
70.625	0.1845	22.7669175560254\\
70.625	0.18816	24.0590008080619\\
70.625	0.19182	25.3996328230929\\
70.625	0.19548	26.7888136011182\\
70.625	0.19914	28.2265431421381\\
70.625	0.2028	29.7128214461524\\
70.625	0.20646	31.2476485131611\\
70.625	0.21012	32.8310243431642\\
70.625	0.21378	34.4629489361618\\
70.625	0.21744	36.1434222921539\\
70.625	0.2211	37.8724444111403\\
70.625	0.22476	39.6500152931213\\
70.625	0.22842	41.4761349380966\\
70.625	0.23208	43.3508033460664\\
70.625	0.23574	45.2740205170307\\
70.625	0.2394	47.2457864509893\\
70.625	0.24306	49.2661011479425\\
70.625	0.24672	51.33496460789\\
70.625	0.25038	53.452376830832\\
70.625	0.25404	55.6183378167685\\
70.625	0.2577	57.8328475656993\\
70.625	0.26136	60.0959060776246\\
70.625	0.26502	62.4075133525444\\
70.625	0.26868	64.7676693904586\\
70.625	0.27234	67.1763741913672\\
70.625	0.276	69.6336277552703\\
71	0.093	6.36511075703563\\
71	0.09666	6.44107915005702\\
71	0.10032	6.56559630607284\\
71	0.10398	6.73866222508308\\
71	0.10764	6.96027690708779\\
71	0.1113	7.23044035208692\\
71	0.11496	7.54915256008051\\
71	0.11862	7.91641353106854\\
71	0.12228	8.33222326505099\\
71	0.12594	8.79658176202788\\
71	0.1296	9.30948902199922\\
71	0.13326	9.87094504496501\\
71	0.13692	10.4809498309252\\
71	0.14058	11.1395033798799\\
71	0.14424	11.846605691829\\
71	0.1479	12.6022567667725\\
71	0.15156	13.4064566047105\\
71	0.15522	14.2592052056429\\
71	0.15888	15.1605025695698\\
71	0.16254	16.1103486964911\\
71	0.1662	17.1087435864069\\
71	0.16986	18.1556872393171\\
71	0.17352	19.2511796552217\\
71	0.17718	20.3952208341207\\
71	0.18084	21.5878107760142\\
71	0.1845	22.8289494809022\\
71	0.18816	24.1186369487846\\
71	0.19182	25.4568731796614\\
71	0.19548	26.8436581735327\\
71	0.19914	28.2789919303984\\
71	0.2028	29.7628744502585\\
71	0.20646	31.2953057331131\\
71	0.21012	32.8762857789622\\
71	0.21378	34.5058145878056\\
71	0.21744	36.1838921596435\\
71	0.2211	37.9105184944759\\
71	0.22476	39.6856935923027\\
71	0.22842	41.5094174531239\\
71	0.23208	43.3816900769396\\
71	0.23574	45.3025114637497\\
71	0.2394	47.2718816135543\\
71	0.24306	49.2898005263533\\
71	0.24672	51.3562682021467\\
71	0.25038	53.4712846409346\\
71	0.25404	55.6348498427169\\
71	0.2577	57.8469638074937\\
71	0.26136	60.1076265352649\\
71	0.26502	62.4168380260305\\
71	0.26868	64.7745982797906\\
71	0.27234	67.1809072965451\\
71	0.276	69.635765076294\\
71.375	0.093	6.49094846363048\\
71.375	0.09666	6.56452107249773\\
71.375	0.10032	6.68664244435942\\
71.375	0.10398	6.85731257921555\\
71.375	0.10764	7.07653147706613\\
71.375	0.1113	7.34429913791114\\
71.375	0.11496	7.6606155617506\\
71.375	0.11862	8.02548074858451\\
71.375	0.12228	8.43889469841283\\
71.375	0.12594	8.9008574112356\\
71.375	0.1296	9.41136888705281\\
71.375	0.13326	9.97042912586447\\
71.375	0.13692	10.5780381276706\\
71.375	0.14058	11.2341958924711\\
71.375	0.14424	11.9389024202661\\
71.375	0.1479	12.6921577110555\\
71.375	0.15156	13.4939617648394\\
71.375	0.15522	14.3443145816177\\
71.375	0.15888	15.2432161613904\\
71.375	0.16254	16.1906665041576\\
71.375	0.1662	17.1866656099192\\
71.375	0.16986	18.2312134786753\\
71.375	0.17352	19.3243101104258\\
71.375	0.17718	20.4659555051707\\
71.375	0.18084	21.6561496629101\\
71.375	0.1845	22.8948925836439\\
71.375	0.18816	24.1821842673722\\
71.375	0.19182	25.5180247140949\\
71.375	0.19548	26.902413923812\\
71.375	0.19914	28.3353518965236\\
71.375	0.2028	29.8168386322297\\
71.375	0.20646	31.3468741309301\\
71.375	0.21012	32.925458392625\\
71.375	0.21378	34.5525914173144\\
71.375	0.21744	36.2282732049982\\
71.375	0.2211	37.9525037556764\\
71.375	0.22476	39.7252830693491\\
71.375	0.22842	41.5466111460162\\
71.375	0.23208	43.4164879856777\\
71.375	0.23574	45.3349135883337\\
71.375	0.2394	47.3018879539842\\
71.375	0.24306	49.3174110826291\\
71.375	0.24672	51.3814829742684\\
71.375	0.25038	53.4941036289021\\
71.375	0.25404	55.6552730465303\\
71.375	0.2577	57.8649912271529\\
71.375	0.26136	60.12325817077\\
71.375	0.26502	62.4300738773815\\
71.375	0.26868	64.7854383469875\\
71.375	0.27234	67.1893515795878\\
71.375	0.276	69.6418135751827\\
71.75	0.093	6.62069734809022\\
71.75	0.09666	6.69187417280337\\
71.75	0.10032	6.81159976051093\\
71.75	0.10398	6.97987411121294\\
71.75	0.10764	7.19669722490939\\
71.75	0.1113	7.46206910160029\\
71.75	0.11496	7.77598974128562\\
71.75	0.11862	8.1384591439654\\
71.75	0.12228	8.5494773096396\\
71.75	0.12594	9.00904423830825\\
71.75	0.1296	9.51715992997134\\
71.75	0.13326	10.0738243846289\\
71.75	0.13692	10.6790376022809\\
71.75	0.14058	11.3327995829273\\
71.75	0.14424	12.0351103265681\\
71.75	0.1479	12.7859698332034\\
71.75	0.15156	13.5853781028331\\
71.75	0.15522	14.4333351354573\\
71.75	0.15888	15.3298409310759\\
71.75	0.16254	16.274895489689\\
71.75	0.1662	17.2684988112965\\
71.75	0.16986	18.3106508958985\\
71.75	0.17352	19.4013517434948\\
71.75	0.17718	20.5406013540857\\
71.75	0.18084	21.7283997276709\\
71.75	0.1845	22.9647468642506\\
71.75	0.18816	24.2496427638247\\
71.75	0.19182	25.5830874263933\\
71.75	0.19548	26.9650808519563\\
71.75	0.19914	28.3956230405138\\
71.75	0.2028	29.8747139920657\\
71.75	0.20646	31.4023537066121\\
71.75	0.21012	32.9785421841528\\
71.75	0.21378	34.6032794246881\\
71.75	0.21744	36.2765654282177\\
71.75	0.2211	37.9984001947418\\
71.75	0.22476	39.7687837242604\\
71.75	0.22842	41.5877160167734\\
71.75	0.23208	43.4551970722808\\
71.75	0.23574	45.3712268907827\\
71.75	0.2394	47.335805472279\\
71.75	0.24306	49.3489328167697\\
71.75	0.24672	51.4106089242549\\
71.75	0.25038	53.5208337947345\\
71.75	0.25404	55.6796074282086\\
71.75	0.2577	57.8869298246771\\
71.75	0.26136	60.1428009841401\\
71.75	0.26502	62.4472209065975\\
71.75	0.26868	64.8001895920493\\
71.75	0.27234	67.2017070404955\\
71.75	0.276	69.6517732519363\\
72.125	0.093	6.75435741041493\\
72.125	0.09666	6.82313845097393\\
72.125	0.10032	6.94046825452738\\
72.125	0.10398	7.10634682107526\\
72.125	0.10764	7.32077415061758\\
72.125	0.1113	7.58375024315435\\
72.125	0.11496	7.89527509868556\\
72.125	0.11862	8.25534871721123\\
72.125	0.12228	8.66397109873131\\
72.125	0.12594	9.12114224324584\\
72.125	0.1296	9.6268621507548\\
72.125	0.13326	10.1811308212582\\
72.125	0.13692	10.7839482547561\\
72.125	0.14058	11.4353144512484\\
72.125	0.14424	12.1352294107351\\
72.125	0.1479	12.8836931332162\\
72.125	0.15156	13.6807056186919\\
72.125	0.15522	14.5262668671619\\
72.125	0.15888	15.4203768786264\\
72.125	0.16254	16.3630356530854\\
72.125	0.1662	17.3542431905387\\
72.125	0.16986	18.3939994909866\\
72.125	0.17352	19.4823045544288\\
72.125	0.17718	20.6191583808655\\
72.125	0.18084	21.8045609702966\\
72.125	0.1845	23.0385123227222\\
72.125	0.18816	24.3210124381422\\
72.125	0.19182	25.6520613165567\\
72.125	0.19548	27.0316589579656\\
72.125	0.19914	28.4598053623689\\
72.125	0.2028	29.9365005297667\\
72.125	0.20646	31.4617444601589\\
72.125	0.21012	33.0355371535456\\
72.125	0.21378	34.6578786099267\\
72.125	0.21744	36.3287688293022\\
72.125	0.2211	38.0482078116722\\
72.125	0.22476	39.8161955570366\\
72.125	0.22842	41.6327320653955\\
72.125	0.23208	43.4978173367488\\
72.125	0.23574	45.4114513710965\\
72.125	0.2394	47.3736341684387\\
72.125	0.24306	49.3843657287754\\
72.125	0.24672	51.4436460521064\\
72.125	0.25038	53.5514751384319\\
72.125	0.25404	55.7078529877518\\
72.125	0.2577	57.9127796000663\\
72.125	0.26136	60.1662549753751\\
72.125	0.26502	62.4682791136783\\
72.125	0.26868	64.818852014976\\
72.125	0.27234	67.2179736792682\\
72.125	0.276	69.6656441065548\\
72.5	0.093	6.89192865060454\\
72.5	0.09666	6.95831390700944\\
72.5	0.10032	7.07324792640876\\
72.5	0.10398	7.23673070880252\\
72.5	0.10764	7.44876225419073\\
72.5	0.1113	7.70934256257336\\
72.5	0.11496	8.01847163395047\\
72.5	0.11862	8.37614946832198\\
72.5	0.12228	8.78237606568795\\
72.5	0.12594	9.23715142604834\\
72.5	0.1296	9.74047554940318\\
72.5	0.13326	10.2923484357525\\
72.5	0.13692	10.8927700850962\\
72.5	0.14058	11.5417404974344\\
72.5	0.14424	12.239259672767\\
72.5	0.1479	12.985327611094\\
72.5	0.15156	13.7799443124155\\
72.5	0.15522	14.6231097767314\\
72.5	0.15888	15.5148240040418\\
72.5	0.16254	16.4550869943466\\
72.5	0.1662	17.4438987476459\\
72.5	0.16986	18.4812592639396\\
72.5	0.17352	19.5671685432277\\
72.5	0.17718	20.7016265855103\\
72.5	0.18084	21.8846333907873\\
72.5	0.1845	23.1161889590587\\
72.5	0.18816	24.3962932903246\\
72.5	0.19182	25.724946384585\\
72.5	0.19548	27.1021482418397\\
72.5	0.19914	28.5278988620889\\
72.5	0.2028	30.0021982453326\\
72.5	0.20646	31.5250463915707\\
72.5	0.21012	33.0964433008032\\
72.5	0.21378	34.7163889730302\\
72.5	0.21744	36.3848834082516\\
72.5	0.2211	38.1019266064675\\
72.5	0.22476	39.8675185676778\\
72.5	0.22842	41.6816592918825\\
72.5	0.23208	43.5443487790817\\
72.5	0.23574	45.4555870292753\\
72.5	0.2394	47.4153740424634\\
72.5	0.24306	49.4237098186459\\
72.5	0.24672	51.4805943578228\\
72.5	0.25038	53.5860276599942\\
72.5	0.25404	55.74000972516\\
72.5	0.2577	57.9425405533203\\
72.5	0.26136	60.193620144475\\
72.5	0.26502	62.4932484986241\\
72.5	0.26868	64.8414256157677\\
72.5	0.27234	67.2381514959058\\
72.5	0.276	69.6834261390382\\
72.875	0.093	7.03341106865911\\
72.875	0.09666	7.09740054090988\\
72.875	0.10032	7.20993877615507\\
72.875	0.10398	7.3710257743947\\
72.875	0.10764	7.58066153562878\\
72.875	0.1113	7.8388460598573\\
72.875	0.11496	8.14557934708025\\
72.875	0.11862	8.50086139729766\\
72.875	0.12228	8.9046922105095\\
72.875	0.12594	9.35707178671579\\
72.875	0.1296	9.8580001259165\\
72.875	0.13326	10.4074772281117\\
72.875	0.13692	11.0055030933013\\
72.875	0.14058	11.6520777214853\\
72.875	0.14424	12.3472011126638\\
72.875	0.1479	13.0908732668367\\
72.875	0.15156	13.8830941840041\\
72.875	0.15522	14.7238638641659\\
72.875	0.15888	15.6131823073221\\
72.875	0.16254	16.5510495134728\\
72.875	0.1662	17.537465482618\\
72.875	0.16986	18.5724302147575\\
72.875	0.17352	19.6559437098915\\
72.875	0.17718	20.78800596802\\
72.875	0.18084	21.9686169891429\\
72.875	0.1845	23.1977767732602\\
72.875	0.18816	24.475485320372\\
72.875	0.19182	25.8017426304782\\
72.875	0.19548	27.1765487035788\\
72.875	0.19914	28.5999035396739\\
72.875	0.2028	30.0718071387634\\
72.875	0.20646	31.5922595008474\\
72.875	0.21012	33.1612606259258\\
72.875	0.21378	34.7788105139987\\
72.875	0.21744	36.444909165066\\
72.875	0.2211	38.1595565791277\\
72.875	0.22476	39.9227527561839\\
72.875	0.22842	41.7344976962345\\
72.875	0.23208	43.5947913992796\\
72.875	0.23574	45.5036338653191\\
72.875	0.2394	47.461025094353\\
72.875	0.24306	49.4669650863814\\
72.875	0.24672	51.5214538414042\\
72.875	0.25038	53.6244913594215\\
72.875	0.25404	55.7760776404331\\
72.875	0.2577	57.9762126844393\\
72.875	0.26136	60.2248964914399\\
72.875	0.26502	62.5221290614349\\
72.875	0.26868	64.8679103944243\\
72.875	0.27234	67.2622404904082\\
72.875	0.276	69.7051193493865\\
73.25	0.093	7.17880466457862\\
73.25	0.09666	7.24039835267526\\
73.25	0.10032	7.35054080376634\\
73.25	0.10398	7.50923201785184\\
73.25	0.10764	7.71647199493179\\
73.25	0.1113	7.97226073500619\\
73.25	0.11496	8.27659823807502\\
73.25	0.11862	8.62948450413831\\
73.25	0.12228	9.03091953319603\\
73.25	0.12594	9.48090332524819\\
73.25	0.1296	9.97943588029477\\
73.25	0.13326	10.5265171983358\\
73.25	0.13692	11.1221472793713\\
73.25	0.14058	11.7663261234012\\
73.25	0.14424	12.4590537304256\\
73.25	0.1479	13.2003301004444\\
73.25	0.15156	13.9901552334576\\
73.25	0.15522	14.8285291294653\\
73.25	0.15888	15.7154517884674\\
73.25	0.16254	16.650923210464\\
73.25	0.1662	17.634943395455\\
73.25	0.16986	18.6675123434404\\
73.25	0.17352	19.7486300544203\\
73.25	0.17718	20.8782965283946\\
73.25	0.18084	22.0565117653634\\
73.25	0.1845	23.2832757653266\\
73.25	0.18816	24.5585885282843\\
73.25	0.19182	25.8824500542363\\
73.25	0.19548	27.2548603431829\\
73.25	0.19914	28.6758193951239\\
73.25	0.2028	30.1453272100592\\
73.25	0.20646	31.6633837879891\\
73.25	0.21012	33.2299891289134\\
73.25	0.21378	34.8451432328321\\
73.25	0.21744	36.5088460997453\\
73.25	0.2211	38.2210977296529\\
73.25	0.22476	39.981898122555\\
73.25	0.22842	41.7912472784514\\
73.25	0.23208	43.6491451973423\\
73.25	0.23574	45.5555918792278\\
73.25	0.2394	47.5105873241076\\
73.25	0.24306	49.5141315319818\\
73.25	0.24672	51.5662245028505\\
73.25	0.25038	53.6668662367137\\
73.25	0.25404	55.8160567335712\\
73.25	0.2577	58.0137959934232\\
73.25	0.26136	60.2600840162697\\
73.25	0.26502	62.5549208021106\\
73.25	0.26868	64.8983063509459\\
73.25	0.27234	67.2902406627757\\
73.25	0.276	69.7307237375999\\
73.625	0.093	7.32810943836303\\
73.625	0.09666	7.38730734230557\\
73.625	0.10032	7.49505400924251\\
73.625	0.10398	7.65134943917389\\
73.625	0.10764	7.85619363209972\\
73.625	0.1113	8.10958658801999\\
73.625	0.11496	8.41152830693472\\
73.625	0.11862	8.76201878884387\\
73.625	0.12228	9.16105803374746\\
73.625	0.12594	9.6086460416455\\
73.625	0.1296	10.104782812538\\
73.625	0.13326	10.6494683464249\\
73.625	0.13692	11.2427026433062\\
73.625	0.14058	11.884485703182\\
73.625	0.14424	12.5748175260523\\
73.625	0.1479	13.3136981119169\\
73.625	0.15156	14.1011274607761\\
73.625	0.15522	14.9371055726296\\
73.625	0.15888	15.8216324474776\\
73.625	0.16254	16.7547080853201\\
73.625	0.1662	17.736332486157\\
73.625	0.16986	18.7665056499883\\
73.625	0.17352	19.845227576814\\
73.625	0.17718	20.9724982666342\\
73.625	0.18084	22.1483177194488\\
73.625	0.1845	23.3726859352579\\
73.625	0.18816	24.6456029140615\\
73.625	0.19182	25.9670686558594\\
73.625	0.19548	27.3370831606518\\
73.625	0.19914	28.7556464284387\\
73.625	0.2028	30.22275845922\\
73.625	0.20646	31.7384192529957\\
73.625	0.21012	33.3026288097658\\
73.625	0.21378	34.9153871295305\\
73.625	0.21744	36.5766942122895\\
73.625	0.2211	38.286550058043\\
73.625	0.22476	40.0449546667909\\
73.625	0.22842	41.8519080385333\\
73.625	0.23208	43.7074101732701\\
73.625	0.23574	45.6114610710014\\
73.625	0.2394	47.564060731727\\
73.625	0.24306	49.5652091554472\\
73.625	0.24672	51.6149063421617\\
73.625	0.25038	53.7131522918707\\
73.625	0.25404	55.8599470045742\\
73.625	0.2577	58.0552904802721\\
73.625	0.26136	60.2991827189644\\
73.625	0.26502	62.5916237206512\\
73.625	0.26868	64.9326134853324\\
73.625	0.27234	67.3221520130081\\
73.625	0.276	69.7602393036781\\
74	0.093	7.4813253900124\\
74	0.09666	7.53812750980079\\
74	0.10032	7.64347839258362\\
74	0.10398	7.79737803836088\\
74	0.10764	7.99982644713259\\
74	0.1113	8.25082361889873\\
74	0.11496	8.55036955365932\\
74	0.11862	8.89846425141436\\
74	0.12228	9.29510771216383\\
74	0.12594	9.74029993590774\\
74	0.1296	10.2340409226461\\
74	0.13326	10.7763306723789\\
74	0.13692	11.3671691851061\\
74	0.14058	12.0065564608278\\
74	0.14424	12.6944924995439\\
74	0.1479	13.4309773012545\\
74	0.15156	14.2160108659594\\
74	0.15522	15.0495931936589\\
74	0.15888	15.9317242843528\\
74	0.16254	16.8624041380411\\
74	0.1662	17.8416327547238\\
74	0.16986	18.869410134401\\
74	0.17352	19.9457362770727\\
74	0.17718	21.0706111827387\\
74	0.18084	22.2440348513992\\
74	0.1845	23.4660072830542\\
74	0.18816	24.7365284777036\\
74	0.19182	26.0555984353474\\
74	0.19548	27.4232171559857\\
74	0.19914	28.8393846396184\\
74	0.2028	30.3041008862456\\
74	0.20646	31.8173658958672\\
74	0.21012	33.3791796684832\\
74	0.21378	34.9895422040937\\
74	0.21744	36.6484535026987\\
74	0.2211	38.355913564298\\
74	0.22476	40.1119223888918\\
74	0.22842	41.9164799764801\\
74	0.23208	43.7695863270627\\
74	0.23574	45.6712414406399\\
74	0.2394	47.6214453172115\\
74	0.24306	49.6201979567775\\
74	0.24672	51.6674993593379\\
74	0.25038	53.7633495248928\\
74	0.25404	55.9077484534421\\
74	0.2577	58.1006961449859\\
74	0.26136	60.3421925995241\\
74	0.26502	62.6322378170567\\
74	0.26868	64.9708317975838\\
74	0.27234	67.3579745411053\\
74	0.276	69.7936660476213\\
};
\end{axis}

\begin{axis}[%
width=4.527496cm,
height=3.050847cm,
at={(0cm,8.474576cm)},
scale only axis,
xmin=56,
xmax=74,
tick align=outside,
xlabel={$L_{cut}$},
xmajorgrids,
ymin=0.093,
ymax=0.276,
ylabel={$D_{rlx}$},
ymajorgrids,
zmin=-2.89383166911362,
zmax=48.4709141889897,
zlabel={$x_2$},
zmajorgrids,
view={-140}{50},
legend style={at={(1.03,1)},anchor=north west,legend cell align=left,align=left,draw=white!15!black}
]
\addplot3[only marks,mark=*,mark options={},mark size=1.5000pt,color=mycolor1] plot table[row sep=crcr,]{%
74	0.123	2.34053015284914\\
72	0.113	0.0721418562456539\\
61	0.095	-1.90021477490919\\
56	0.093	-2.12993862890179\\
};
\addplot3[only marks,mark=*,mark options={},mark size=1.5000pt,color=mycolor2] plot table[row sep=crcr,]{%
67	0.276	43.4031449012412\\
66	0.255	34.964519373533\\
62	0.209	18.6183965306525\\
57	0.193	12.7146234160201\\
};
\addplot3[only marks,mark=*,mark options={},mark size=1.5000pt,color=black] plot table[row sep=crcr,]{%
69	0.104	-1.14157783533982\\
};
\addplot3[only marks,mark=*,mark options={},mark size=1.5000pt,color=black] plot table[row sep=crcr,]{%
64	0.23	25.6353520232494\\
};

\addplot3[%
surf,
opacity=0.7,
shader=interp,
colormap={mymap}{[1pt] rgb(0pt)=(0.0901961,0.239216,0.0745098); rgb(1pt)=(0.0945149,0.242058,0.0739522); rgb(2pt)=(0.0988592,0.244894,0.0733566); rgb(3pt)=(0.103229,0.247724,0.0727241); rgb(4pt)=(0.107623,0.250549,0.0720557); rgb(5pt)=(0.112043,0.253367,0.0713525); rgb(6pt)=(0.116487,0.25618,0.0706154); rgb(7pt)=(0.120956,0.258986,0.0698456); rgb(8pt)=(0.125449,0.261787,0.0690441); rgb(9pt)=(0.129967,0.264581,0.0682118); rgb(10pt)=(0.134508,0.26737,0.06735); rgb(11pt)=(0.139074,0.270152,0.0664596); rgb(12pt)=(0.143663,0.272929,0.0655416); rgb(13pt)=(0.148275,0.275699,0.0645971); rgb(14pt)=(0.152911,0.278463,0.0636271); rgb(15pt)=(0.15757,0.281221,0.0626328); rgb(16pt)=(0.162252,0.283973,0.0616151); rgb(17pt)=(0.166957,0.286719,0.060575); rgb(18pt)=(0.171685,0.289458,0.0595136); rgb(19pt)=(0.176434,0.292191,0.0584321); rgb(20pt)=(0.181207,0.294918,0.0573313); rgb(21pt)=(0.186001,0.297639,0.0562123); rgb(22pt)=(0.190817,0.300353,0.0550763); rgb(23pt)=(0.195655,0.303061,0.0539242); rgb(24pt)=(0.200514,0.305763,0.052757); rgb(25pt)=(0.205395,0.308459,0.0515759); rgb(26pt)=(0.210296,0.311149,0.0503624); rgb(27pt)=(0.215212,0.313846,0.0490067); rgb(28pt)=(0.220142,0.316548,0.0475043); rgb(29pt)=(0.22509,0.319254,0.0458704); rgb(30pt)=(0.230056,0.321962,0.0441205); rgb(31pt)=(0.235042,0.324671,0.04227); rgb(32pt)=(0.240048,0.327379,0.0403343); rgb(33pt)=(0.245078,0.330085,0.0383287); rgb(34pt)=(0.250131,0.332786,0.0362688); rgb(35pt)=(0.25521,0.335482,0.0341698); rgb(36pt)=(0.260317,0.33817,0.0320472); rgb(37pt)=(0.265451,0.340849,0.0299163); rgb(38pt)=(0.270616,0.343517,0.0277927); rgb(39pt)=(0.275813,0.346172,0.0256916); rgb(40pt)=(0.281043,0.348814,0.0236284); rgb(41pt)=(0.286307,0.35144,0.0216186); rgb(42pt)=(0.291607,0.354048,0.0196776); rgb(43pt)=(0.296945,0.356637,0.0178207); rgb(44pt)=(0.302322,0.359206,0.0160634); rgb(45pt)=(0.307739,0.361753,0.0144211); rgb(46pt)=(0.313198,0.364275,0.0129091); rgb(47pt)=(0.318701,0.366772,0.0115428); rgb(48pt)=(0.324249,0.369242,0.0103377); rgb(49pt)=(0.329843,0.371682,0.00930909); rgb(50pt)=(0.335485,0.374093,0.00847245); rgb(51pt)=(0.341176,0.376471,0.00784314); rgb(52pt)=(0.346925,0.378826,0.00732741); rgb(53pt)=(0.352735,0.381168,0.00682184); rgb(54pt)=(0.358605,0.383497,0.00632729); rgb(55pt)=(0.364532,0.385812,0.00584464); rgb(56pt)=(0.370516,0.388113,0.00537476); rgb(57pt)=(0.376552,0.390399,0.00491852); rgb(58pt)=(0.38264,0.39267,0.00447681); rgb(59pt)=(0.388777,0.394925,0.00405048); rgb(60pt)=(0.394962,0.397164,0.00364042); rgb(61pt)=(0.401191,0.399386,0.00324749); rgb(62pt)=(0.407464,0.401592,0.00287258); rgb(63pt)=(0.413777,0.40378,0.00251655); rgb(64pt)=(0.420129,0.40595,0.00218028); rgb(65pt)=(0.426518,0.408102,0.00186463); rgb(66pt)=(0.432942,0.410234,0.00157049); rgb(67pt)=(0.439399,0.412348,0.00129873); rgb(68pt)=(0.445885,0.414441,0.00105022); rgb(69pt)=(0.452401,0.416515,0.000825833); rgb(70pt)=(0.458942,0.418567,0.000626441); rgb(71pt)=(0.465508,0.420599,0.00045292); rgb(72pt)=(0.472096,0.422609,0.000306141); rgb(73pt)=(0.478704,0.424596,0.000186979); rgb(74pt)=(0.485331,0.426562,9.63073e-05); rgb(75pt)=(0.491973,0.428504,3.49981e-05); rgb(76pt)=(0.498628,0.430422,3.92506e-06); rgb(77pt)=(0.505323,0.432315,0); rgb(78pt)=(0.512206,0.434168,0); rgb(79pt)=(0.519282,0.435983,0); rgb(80pt)=(0.526529,0.437764,0); rgb(81pt)=(0.533922,0.439512,0); rgb(82pt)=(0.54144,0.441232,0); rgb(83pt)=(0.549059,0.442927,0); rgb(84pt)=(0.556756,0.444599,0); rgb(85pt)=(0.564508,0.446252,0); rgb(86pt)=(0.572292,0.447889,0); rgb(87pt)=(0.580084,0.449514,0); rgb(88pt)=(0.587863,0.451129,0); rgb(89pt)=(0.595604,0.452737,0); rgb(90pt)=(0.603284,0.454343,0); rgb(91pt)=(0.610882,0.455948,0); rgb(92pt)=(0.618373,0.457556,0); rgb(93pt)=(0.625734,0.459171,0); rgb(94pt)=(0.632943,0.460795,0); rgb(95pt)=(0.639976,0.462432,0); rgb(96pt)=(0.64681,0.464084,0); rgb(97pt)=(0.653423,0.465756,0); rgb(98pt)=(0.659791,0.46745,0); rgb(99pt)=(0.665891,0.469169,0); rgb(100pt)=(0.6717,0.470916,0); rgb(101pt)=(0.677195,0.472696,0); rgb(102pt)=(0.682353,0.47451,0); rgb(103pt)=(0.687242,0.476355,0); rgb(104pt)=(0.691952,0.478225,0); rgb(105pt)=(0.696497,0.480118,0); rgb(106pt)=(0.700887,0.482033,0); rgb(107pt)=(0.705134,0.483968,0); rgb(108pt)=(0.709251,0.485921,0); rgb(109pt)=(0.713249,0.487891,0); rgb(110pt)=(0.71714,0.489876,0); rgb(111pt)=(0.720936,0.491875,0); rgb(112pt)=(0.724649,0.493887,0); rgb(113pt)=(0.72829,0.495909,0); rgb(114pt)=(0.731872,0.49794,0); rgb(115pt)=(0.735406,0.499979,0); rgb(116pt)=(0.738904,0.502025,0); rgb(117pt)=(0.742378,0.504075,0); rgb(118pt)=(0.74584,0.506128,0); rgb(119pt)=(0.749302,0.508182,0); rgb(120pt)=(0.752775,0.510237,0); rgb(121pt)=(0.756272,0.51229,0); rgb(122pt)=(0.759804,0.514339,0); rgb(123pt)=(0.763384,0.516385,0); rgb(124pt)=(0.767022,0.518424,0); rgb(125pt)=(0.770731,0.520455,0); rgb(126pt)=(0.774523,0.522478,0); rgb(127pt)=(0.77841,0.524489,0); rgb(128pt)=(0.782391,0.526491,0); rgb(129pt)=(0.786402,0.528496,0); rgb(130pt)=(0.790431,0.530506,0); rgb(131pt)=(0.794478,0.532521,0); rgb(132pt)=(0.798541,0.534539,0); rgb(133pt)=(0.802619,0.53656,0); rgb(134pt)=(0.806712,0.538584,0); rgb(135pt)=(0.81082,0.540609,0); rgb(136pt)=(0.81494,0.542635,0); rgb(137pt)=(0.819074,0.54466,0); rgb(138pt)=(0.823219,0.546686,0); rgb(139pt)=(0.827374,0.548709,0); rgb(140pt)=(0.831541,0.55073,0); rgb(141pt)=(0.835716,0.552749,0); rgb(142pt)=(0.8399,0.554763,0); rgb(143pt)=(0.844092,0.556774,0); rgb(144pt)=(0.848292,0.558779,0); rgb(145pt)=(0.852497,0.560778,0); rgb(146pt)=(0.856708,0.562771,0); rgb(147pt)=(0.860924,0.564756,0); rgb(148pt)=(0.865143,0.566733,0); rgb(149pt)=(0.869366,0.568701,0); rgb(150pt)=(0.873592,0.57066,0); rgb(151pt)=(0.877819,0.572608,0); rgb(152pt)=(0.882047,0.574545,0); rgb(153pt)=(0.886275,0.576471,0); rgb(154pt)=(0.890659,0.578362,0); rgb(155pt)=(0.895333,0.580203,0); rgb(156pt)=(0.900258,0.581999,0); rgb(157pt)=(0.905397,0.583755,0); rgb(158pt)=(0.910711,0.585479,0); rgb(159pt)=(0.916164,0.587176,0); rgb(160pt)=(0.921717,0.588852,0); rgb(161pt)=(0.927333,0.590513,0); rgb(162pt)=(0.932974,0.592166,0); rgb(163pt)=(0.938602,0.593815,0); rgb(164pt)=(0.94418,0.595468,0); rgb(165pt)=(0.949669,0.59713,0); rgb(166pt)=(0.955033,0.598808,0); rgb(167pt)=(0.960233,0.600507,0); rgb(168pt)=(0.965232,0.602233,0); rgb(169pt)=(0.969992,0.603992,0); rgb(170pt)=(0.974475,0.605791,0); rgb(171pt)=(0.978643,0.607636,0); rgb(172pt)=(0.98246,0.609532,0); rgb(173pt)=(0.985886,0.611486,0); rgb(174pt)=(0.988885,0.613503,0); rgb(175pt)=(0.991419,0.61559,0); rgb(176pt)=(0.99345,0.617753,0); rgb(177pt)=(0.99494,0.619997,0); rgb(178pt)=(0.995851,0.622329,0); rgb(179pt)=(0.996226,0.624763,0); rgb(180pt)=(0.996512,0.627352,0); rgb(181pt)=(0.996788,0.630095,0); rgb(182pt)=(0.997053,0.632982,0); rgb(183pt)=(0.997308,0.636004,0); rgb(184pt)=(0.997552,0.639152,0); rgb(185pt)=(0.997785,0.642416,0); rgb(186pt)=(0.998006,0.645786,0); rgb(187pt)=(0.998217,0.649253,0); rgb(188pt)=(0.998416,0.652807,0); rgb(189pt)=(0.998605,0.656439,0); rgb(190pt)=(0.998781,0.660138,0); rgb(191pt)=(0.998946,0.663897,0); rgb(192pt)=(0.9991,0.667704,0); rgb(193pt)=(0.999242,0.67155,0); rgb(194pt)=(0.999372,0.675427,0); rgb(195pt)=(0.99949,0.679323,0); rgb(196pt)=(0.999596,0.68323,0); rgb(197pt)=(0.99969,0.687139,0); rgb(198pt)=(0.999771,0.691039,0); rgb(199pt)=(0.999841,0.694921,0); rgb(200pt)=(0.999898,0.698775,0); rgb(201pt)=(0.999942,0.702592,0); rgb(202pt)=(0.999974,0.706363,0); rgb(203pt)=(0.999994,0.710077,0); rgb(204pt)=(1,0.713725,0); rgb(205pt)=(1,0.717341,0); rgb(206pt)=(1,0.720963,0); rgb(207pt)=(1,0.724591,0); rgb(208pt)=(1,0.728226,0); rgb(209pt)=(1,0.731867,0); rgb(210pt)=(1,0.735514,0); rgb(211pt)=(1,0.739167,0); rgb(212pt)=(1,0.742827,0); rgb(213pt)=(1,0.746493,0); rgb(214pt)=(1,0.750165,0); rgb(215pt)=(1,0.753843,0); rgb(216pt)=(1,0.757527,0); rgb(217pt)=(1,0.761217,0); rgb(218pt)=(1,0.764913,0); rgb(219pt)=(1,0.768615,0); rgb(220pt)=(1,0.772324,0); rgb(221pt)=(1,0.776038,0); rgb(222pt)=(1,0.779758,0); rgb(223pt)=(1,0.783484,0); rgb(224pt)=(1,0.787215,0); rgb(225pt)=(1,0.790953,0); rgb(226pt)=(1,0.794696,0); rgb(227pt)=(1,0.798445,0); rgb(228pt)=(1,0.8022,0); rgb(229pt)=(1,0.805961,0); rgb(230pt)=(1,0.809727,0); rgb(231pt)=(1,0.8135,0); rgb(232pt)=(1,0.817278,0); rgb(233pt)=(1,0.821063,0); rgb(234pt)=(1,0.824854,0); rgb(235pt)=(1,0.828652,0); rgb(236pt)=(1,0.832455,0); rgb(237pt)=(1,0.836265,0); rgb(238pt)=(1,0.840081,0); rgb(239pt)=(1,0.843903,0); rgb(240pt)=(1,0.847732,0); rgb(241pt)=(1,0.851566,0); rgb(242pt)=(1,0.855406,0); rgb(243pt)=(1,0.859253,0); rgb(244pt)=(1,0.863106,0); rgb(245pt)=(1,0.866964,0); rgb(246pt)=(1,0.870829,0); rgb(247pt)=(1,0.8747,0); rgb(248pt)=(1,0.878577,0); rgb(249pt)=(1,0.88246,0); rgb(250pt)=(1,0.886349,0); rgb(251pt)=(1,0.890243,0); rgb(252pt)=(1,0.894144,0); rgb(253pt)=(1,0.898051,0); rgb(254pt)=(1,0.901964,0); rgb(255pt)=(1,0.905882,0)},
mesh/rows=49]
table[row sep=crcr,header=false] {%
%
56	0.093	-2.04011112489684\\
56	0.09666	-1.77311452607124\\
56	0.10032	-1.48657576808861\\
56	0.10398	-1.18049485094895\\
56	0.10764	-0.854871774652246\\
56	0.1113	-0.509706539198515\\
56	0.11496	-0.144999144587743\\
56	0.11862	0.239250409180059\\
56	0.12228	0.643042122104895\\
56	0.12594	1.06637599418677\\
56	0.1296	1.50925202542567\\
56	0.13326	1.97167021582161\\
56	0.13692	2.45363056537459\\
56	0.14058	2.95513307408459\\
56	0.14424	3.47617774195164\\
56	0.1479	4.01676456897571\\
56	0.15156	4.57689355515683\\
56	0.15522	5.15656470049497\\
56	0.15888	5.75577800499016\\
56	0.16254	6.37453346864237\\
56	0.1662	7.01283109145163\\
56	0.16986	7.6706708734179\\
56	0.17352	8.34805281454122\\
56	0.17718	9.04497691482157\\
56	0.18084	9.76144317425895\\
56	0.1845	10.4974515928534\\
56	0.18816	11.2530021706048\\
56	0.19182	12.0280949075133\\
56	0.19548	12.8227298035789\\
56	0.19914	13.6369068588014\\
56	0.2028	14.470626073181\\
56	0.20646	15.3238874467176\\
56	0.21012	16.1966909794113\\
56	0.21378	17.089036671262\\
56	0.21744	18.0009245222697\\
56	0.2211	18.9323545324345\\
56	0.22476	19.8833267017563\\
56	0.22842	20.8538410302351\\
56	0.23208	21.843897517871\\
56	0.23574	22.8534961646639\\
56	0.2394	23.8826369706138\\
56	0.24306	24.9313199357208\\
56	0.24672	25.9995450599848\\
56	0.25038	27.0873123434058\\
56	0.25404	28.1946217859839\\
56	0.2577	29.321473387719\\
56	0.26136	30.4678671486112\\
56	0.26502	31.6338030686603\\
56	0.26868	32.8192811478665\\
56	0.27234	34.0243013862298\\
56	0.276	35.2488637837501\\
56.375	0.093	-2.05355454810592\\
56.375	0.09666	-1.78069304471805\\
56.375	0.10032	-1.48828938217314\\
56.375	0.10398	-1.1763435604712\\
56.375	0.10764	-0.844855579612231\\
56.375	0.1113	-0.493825439596227\\
56.375	0.11496	-0.123253140423179\\
56.375	0.11862	0.266861317906896\\
56.375	0.12228	0.676517935394005\\
56.375	0.12594	1.10571671203815\\
56.375	0.1296	1.55445764783933\\
56.375	0.13326	2.02274074279753\\
56.375	0.13692	2.51056599691279\\
56.375	0.14058	3.01793341018507\\
56.375	0.14424	3.54484298261439\\
56.375	0.1479	4.09129471420074\\
56.375	0.15156	4.65728860494412\\
56.375	0.15522	5.24282465484454\\
56.375	0.15888	5.847902863902\\
56.375	0.16254	6.47252323211649\\
56.375	0.1662	7.11668575948802\\
56.375	0.16986	7.78039044601657\\
56.375	0.17352	8.46363729170216\\
56.375	0.17718	9.16642629654478\\
56.375	0.18084	9.88875746054444\\
56.375	0.1845	10.6306307837011\\
56.375	0.18816	11.3920462660149\\
56.375	0.19182	12.1730039074856\\
56.375	0.19548	12.9735037081134\\
56.375	0.19914	13.7935456678983\\
56.375	0.2028	14.6331297868401\\
56.375	0.20646	15.492256064939\\
56.375	0.21012	16.370924502195\\
56.375	0.21378	17.2691350986079\\
56.375	0.21744	18.1868878541779\\
56.375	0.2211	19.124182768905\\
56.375	0.22476	20.081019842789\\
56.375	0.22842	21.0573990758302\\
56.375	0.23208	22.0533204680283\\
56.375	0.23574	23.0687840193835\\
56.375	0.2394	24.1037897298957\\
56.375	0.24306	25.1583375995649\\
56.375	0.24672	26.2324276283912\\
56.375	0.25038	27.3260598163745\\
56.375	0.25404	28.4392341635148\\
56.375	0.2577	29.5719506698122\\
56.375	0.26136	30.7242093352667\\
56.375	0.26502	31.8960101598781\\
56.375	0.26868	33.0873531436466\\
56.375	0.27234	34.2982382865721\\
56.375	0.276	35.5286655886547\\
56.75	0.093	-2.067182755207\\
56.75	0.09666	-1.78845634725686\\
56.75	0.10032	-1.49018778014968\\
56.75	0.10398	-1.17237705388547\\
56.75	0.10764	-0.835024168464219\\
56.75	0.1113	-0.478129123885942\\
56.75	0.11496	-0.101691920150621\\
56.75	0.11862	0.294287442741727\\
56.75	0.12228	0.709808964791108\\
56.75	0.12594	1.14487264599753\\
56.75	0.1296	1.59947848636098\\
56.75	0.13326	2.07362648588146\\
56.75	0.13692	2.56731664455899\\
56.75	0.14058	3.08054896239354\\
56.75	0.14424	3.61332343938514\\
56.75	0.1479	4.16564007553376\\
56.75	0.15156	4.73749887083941\\
56.75	0.15522	5.32889982530211\\
56.75	0.15888	5.93984293892184\\
56.75	0.16254	6.5703282116986\\
56.75	0.1662	7.2203556436324\\
56.75	0.16986	7.88992523472323\\
56.75	0.17352	8.5790369849711\\
56.75	0.17718	9.28769089437599\\
56.75	0.18084	10.0158869629379\\
56.75	0.1845	10.7636251906569\\
56.75	0.18816	11.5309055775329\\
56.75	0.19182	12.3177281235659\\
56.75	0.19548	13.124092828756\\
56.75	0.19914	13.9499996931031\\
56.75	0.2028	14.7954487166073\\
56.75	0.20646	15.6604398992684\\
56.75	0.21012	16.5449732410866\\
56.75	0.21378	17.4490487420619\\
56.75	0.21744	18.3726664021942\\
56.75	0.2211	19.3158262214835\\
56.75	0.22476	20.2785281999298\\
56.75	0.22842	21.2607723375332\\
56.75	0.23208	22.2625586342936\\
56.75	0.23574	23.2838870902111\\
56.75	0.2394	24.3247577052855\\
56.75	0.24306	25.3851704795171\\
56.75	0.24672	26.4651254129056\\
56.75	0.25038	27.5646225054512\\
56.75	0.25404	28.6836617571538\\
56.75	0.2577	29.8222431680135\\
56.75	0.26136	30.9803667380302\\
56.75	0.26502	32.1580324672039\\
56.75	0.26868	33.3552403555346\\
56.75	0.27234	34.5719904030224\\
56.75	0.276	35.8082826096673\\
57.125	0.093	-2.08099574620009\\
57.125	0.09666	-1.79640443368767\\
57.125	0.10032	-1.49227096201822\\
57.125	0.10398	-1.16859533119173\\
57.125	0.10764	-0.82537754120821\\
57.125	0.1113	-0.462617592067661\\
57.125	0.11496	-0.0803154837700664\\
57.125	0.11862	0.321528783684554\\
57.125	0.12228	0.742915210296209\\
57.125	0.12594	1.1838437960649\\
57.125	0.1296	1.64431454099063\\
57.125	0.13326	2.12432744507338\\
57.125	0.13692	2.62388250831318\\
57.125	0.14058	3.14297973071001\\
57.125	0.14424	3.68161911226388\\
57.125	0.1479	4.23980065297477\\
57.125	0.15156	4.81752435284271\\
57.125	0.15522	5.41479021186767\\
57.125	0.15888	6.03159823004968\\
57.125	0.16254	6.66794840738871\\
57.125	0.1662	7.32384074388479\\
57.125	0.16986	7.99927523953789\\
57.125	0.17352	8.69425189434802\\
57.125	0.17718	9.40877070831519\\
57.125	0.18084	10.1428316814394\\
57.125	0.1845	10.8964348137206\\
57.125	0.18816	11.6695801051589\\
57.125	0.19182	12.4622675557542\\
57.125	0.19548	13.2744971655066\\
57.125	0.19914	14.106268934416\\
57.125	0.2028	14.9575828624824\\
57.125	0.20646	15.8284389497058\\
57.125	0.21012	16.7188371960863\\
57.125	0.21378	17.6287776016238\\
57.125	0.21744	18.5582601663184\\
57.125	0.2211	19.50728489017\\
57.125	0.22476	20.4758517731786\\
57.125	0.22842	21.4639608153442\\
57.125	0.23208	22.4716120166669\\
57.125	0.23574	23.4988053771467\\
57.125	0.2394	24.5455408967834\\
57.125	0.24306	25.6118185755772\\
57.125	0.24672	26.697638413528\\
57.125	0.25038	27.8030004106359\\
57.125	0.25404	28.9279045669007\\
57.125	0.2577	30.0723508823227\\
57.125	0.26136	31.2363393569017\\
57.125	0.26502	32.4198699906377\\
57.125	0.26868	33.6229427835307\\
57.125	0.27234	34.8455577355807\\
57.125	0.276	36.0877148467878\\
57.5	0.093	-2.09499352108517\\
57.5	0.09666	-1.80453730401049\\
57.5	0.10032	-1.49453892777876\\
57.5	0.10398	-1.16499839239\\
57.5	0.10764	-0.815915697844206\\
57.5	0.1113	-0.447290844141383\\
57.5	0.11496	-0.0591238312815126\\
57.5	0.11862	0.348585340735381\\
57.5	0.12228	0.775836671909312\\
57.5	0.12594	1.22263016224028\\
57.5	0.1296	1.68896581172827\\
57.5	0.13326	2.1748436203733\\
57.5	0.13692	2.68026358817538\\
57.5	0.14058	3.20522571513449\\
57.5	0.14424	3.74973000125062\\
57.5	0.1479	4.31377644652379\\
57.5	0.15156	4.89736505095399\\
57.5	0.15522	5.50049581454123\\
57.5	0.15888	6.12316873728552\\
57.5	0.16254	6.76538381918682\\
57.5	0.1662	7.42714106024517\\
57.5	0.16986	8.10844046046054\\
57.5	0.17352	8.80928201983295\\
57.5	0.17718	9.5296657383624\\
57.5	0.18084	10.2695916160489\\
57.5	0.1845	11.0290596528924\\
57.5	0.18816	11.8080698488929\\
57.5	0.19182	12.6066222040505\\
57.5	0.19548	13.4247167183651\\
57.5	0.19914	14.2623533918368\\
57.5	0.2028	15.1195322244655\\
57.5	0.20646	15.9962532162512\\
57.5	0.21012	16.892516367194\\
57.5	0.21378	17.8083216772938\\
57.5	0.21744	18.7436691465506\\
57.5	0.2211	19.6985587749644\\
57.5	0.22476	20.6729905625353\\
57.5	0.22842	21.6669645092633\\
57.5	0.23208	22.6804806151482\\
57.5	0.23574	23.7135388801902\\
57.5	0.2394	24.7661393043893\\
57.5	0.24306	25.8382818877453\\
57.5	0.24672	26.9299666302584\\
57.5	0.25038	28.0411935319285\\
57.5	0.25404	29.1719625927557\\
57.5	0.2577	30.3222738127399\\
57.5	0.26136	31.4921271918812\\
57.5	0.26502	32.6815227301794\\
57.5	0.26868	33.8904604276347\\
57.5	0.27234	35.1189402842471\\
57.5	0.276	36.3669623000164\\
57.875	0.093	-2.10917607986227\\
57.875	0.09666	-1.81285495822531\\
57.875	0.10032	-1.49699167743131\\
57.875	0.10398	-1.16158623748028\\
57.875	0.10764	-0.806638638372206\\
57.875	0.1113	-0.432148880107111\\
57.875	0.11496	-0.0381169626849669\\
57.875	0.11862	0.3754571138942\\
57.875	0.12228	0.808573349630404\\
57.875	0.12594	1.26123174452364\\
57.875	0.1296	1.73343229857391\\
57.875	0.13326	2.22517501178122\\
57.875	0.13692	2.73645988414556\\
57.875	0.14058	3.26728691566695\\
57.875	0.14424	3.81765610634536\\
57.875	0.1479	4.3875674561808\\
57.875	0.15156	4.97702096517327\\
57.875	0.15522	5.58601663332279\\
57.875	0.15888	6.21455446062934\\
57.875	0.16254	6.86263444709292\\
57.875	0.1662	7.53025659271354\\
57.875	0.16986	8.21742089749119\\
57.875	0.17352	8.92412736142587\\
57.875	0.17718	9.65037598451759\\
57.875	0.18084	10.3961667667663\\
57.875	0.1845	11.1614997081721\\
57.875	0.18816	11.946374808735\\
57.875	0.19182	12.7507920684548\\
57.875	0.19548	13.5747514873317\\
57.875	0.19914	14.4182530653656\\
57.875	0.2028	15.2812968025566\\
57.875	0.20646	16.1638826989046\\
57.875	0.21012	17.0660107544096\\
57.875	0.21378	17.9876809690717\\
57.875	0.21744	18.9288933428908\\
57.875	0.2211	19.8896478758669\\
57.875	0.22476	20.8699445680001\\
57.875	0.22842	21.8697834192903\\
57.875	0.23208	22.8891644297375\\
57.875	0.23574	23.9280875993418\\
57.875	0.2394	24.9865529281031\\
57.875	0.24306	26.0645604160214\\
57.875	0.24672	27.1621100630968\\
57.875	0.25038	28.2792018693292\\
57.875	0.25404	29.4158358347186\\
57.875	0.2577	30.5720119592651\\
57.875	0.26136	31.7477302429686\\
57.875	0.26502	32.9429906858292\\
57.875	0.26868	34.1577932878468\\
57.875	0.27234	35.3921380490214\\
57.875	0.276	36.646024969353\\
58.25	0.093	-2.12354342253137\\
58.25	0.09666	-1.82135739633214\\
58.25	0.10032	-1.49962921097586\\
58.25	0.10398	-1.15835886646255\\
58.25	0.10764	-0.797546362792215\\
58.25	0.1113	-0.417191699964842\\
58.25	0.11496	-0.0172948779804258\\
58.25	0.11862	0.402144103161014\\
58.25	0.12228	0.841125243459491\\
58.25	0.12594	1.299648542915\\
58.25	0.1296	1.77771400152755\\
58.25	0.13326	2.27532161929713\\
58.25	0.13692	2.79247139622374\\
58.25	0.14058	3.3291633323074\\
58.25	0.14424	3.88539742754808\\
58.25	0.1479	4.4611736819458\\
58.25	0.15156	5.05649209550055\\
58.25	0.15522	5.67135266821234\\
58.25	0.15888	6.30575540008116\\
58.25	0.16254	6.95970029110702\\
58.25	0.1662	7.63318734128991\\
58.25	0.16986	8.32621655062983\\
58.25	0.17352	9.03878791912679\\
58.25	0.17718	9.77090144678078\\
58.25	0.18084	10.5225571335918\\
58.25	0.1845	11.2937549795599\\
58.25	0.18816	12.084494984685\\
58.25	0.19182	12.8947771489671\\
58.25	0.19548	13.7246014724063\\
58.25	0.19914	14.5739679550025\\
58.25	0.2028	15.4428765967557\\
58.25	0.20646	16.331327397666\\
58.25	0.21012	17.2393203577333\\
58.25	0.21378	18.1668554769576\\
58.25	0.21744	19.113932755339\\
58.25	0.2211	20.0805521928774\\
58.25	0.22476	21.0667137895728\\
58.25	0.22842	22.0724175454253\\
58.25	0.23208	23.0976634604348\\
58.25	0.23574	24.1424515346014\\
58.25	0.2394	25.2067817679249\\
58.25	0.24306	26.2906541604055\\
58.25	0.24672	27.3940687120432\\
58.25	0.25038	28.5170254228379\\
58.25	0.25404	29.6595242927896\\
58.25	0.2577	30.8215653218983\\
58.25	0.26136	32.0031485101641\\
58.25	0.26502	33.2042738575869\\
58.25	0.26868	34.4249413641668\\
58.25	0.27234	35.6651510299037\\
58.25	0.276	36.9249028547976\\
58.625	0.093	-2.13809554909247\\
58.625	0.09666	-1.83004461833096\\
58.625	0.10032	-1.50245152841242\\
58.625	0.10398	-1.15531627933684\\
58.625	0.10764	-0.78863887110422\\
58.625	0.1113	-0.402419303714575\\
58.625	0.11496	0.00334242283211061\\
58.625	0.11862	0.428646308535827\\
58.625	0.12228	0.873492353396577\\
58.625	0.12594	1.33788055741436\\
58.625	0.1296	1.82181092058918\\
58.625	0.13326	2.32528344292103\\
58.625	0.13692	2.84829812440993\\
58.625	0.14058	3.39085496505585\\
58.625	0.14424	3.95295396485881\\
58.625	0.1479	4.5345951238188\\
58.625	0.15156	5.13577844193582\\
58.625	0.15522	5.75650391920989\\
58.625	0.15888	6.39677155564098\\
58.625	0.16254	7.05658135122911\\
58.625	0.1662	7.73593330597428\\
58.625	0.16986	8.43482741987647\\
58.625	0.17352	9.15326369293571\\
58.625	0.17718	9.89124212515197\\
58.625	0.18084	10.6487627165253\\
58.625	0.1845	11.4258254670556\\
58.625	0.18816	12.222430376743\\
58.625	0.19182	13.0385774455874\\
58.625	0.19548	13.8742666735888\\
58.625	0.19914	14.7294980607473\\
58.625	0.2028	15.6042716070628\\
58.625	0.20646	16.4985873125353\\
58.625	0.21012	17.4124451771649\\
58.625	0.21378	18.3458452009515\\
58.625	0.21744	19.2987873838952\\
58.625	0.2211	20.2712717259959\\
58.625	0.22476	21.2632982272536\\
58.625	0.22842	22.2748668876683\\
58.625	0.23208	23.3059777072401\\
58.625	0.23574	24.3566306859689\\
58.625	0.2394	25.4268258238548\\
58.625	0.24306	26.5165631208977\\
58.625	0.24672	27.6258425770976\\
58.625	0.25038	28.7546641924545\\
58.625	0.25404	29.9030279669685\\
58.625	0.2577	31.0709339006395\\
58.625	0.26136	32.2583819934676\\
58.625	0.26502	33.4653722454527\\
58.625	0.26868	34.6919046565948\\
58.625	0.27234	35.937979226894\\
58.625	0.276	37.2035959563502\\
59	0.093	-2.15283245954558\\
59	0.09666	-1.8389166242218\\
59	0.10032	-1.50545862974098\\
59	0.10398	-1.15245847610313\\
59	0.10764	-0.779916163308238\\
59	0.1113	-0.38783169135632\\
59	0.11496	0.0237949397526425\\
59	0.11862	0.454963730018632\\
59	0.12228	0.905674679441654\\
59	0.12594	1.37592778802171\\
59	0.1296	1.86572305575881\\
59	0.13326	2.37506048265293\\
59	0.13692	2.9039400687041\\
59	0.14058	3.4523618139123\\
59	0.14424	4.02032571827753\\
59	0.1479	4.60783178179979\\
59	0.15156	5.21488000447909\\
59	0.15522	5.84147038631542\\
59	0.15888	6.48760292730879\\
59	0.16254	7.1532776274592\\
59	0.1662	7.83849448676664\\
59	0.16986	8.54325350523111\\
59	0.17352	9.26755468285261\\
59	0.17718	10.0113980196311\\
59	0.18084	10.7747835155667\\
59	0.1845	11.5577111706593\\
59	0.18816	12.360180984909\\
59	0.19182	13.1821929583157\\
59	0.19548	14.0237470908794\\
59	0.19914	14.8848433826001\\
59	0.2028	15.7654818334779\\
59	0.20646	16.6656624435127\\
59	0.21012	17.5853852127046\\
59	0.21378	18.5246501410535\\
59	0.21744	19.4834572285594\\
59	0.2211	20.4618064752223\\
59	0.22476	21.4596978810423\\
59	0.22842	22.4771314460193\\
59	0.23208	23.5141071701534\\
59	0.23574	24.5706250534445\\
59	0.2394	25.6466850958926\\
59	0.24306	26.7422872974978\\
59	0.24672	27.8574316582599\\
59	0.25038	28.9921181781792\\
59	0.25404	30.1463468572554\\
59	0.2577	31.3201176954887\\
59	0.26136	32.5134306928791\\
59	0.26502	33.7262858494264\\
59	0.26868	34.9586831651308\\
59	0.27234	36.2106226399923\\
59	0.276	37.4821042740107\\
59.375	0.093	-2.1677541538907\\
59.375	0.09666	-1.84797341400464\\
59.375	0.10032	-1.50865051496154\\
59.375	0.10398	-1.14978545676142\\
59.375	0.10764	-0.771378239404257\\
59.375	0.1113	-0.373428862890066\\
59.375	0.11496	0.0440626727811697\\
59.375	0.11862	0.481096367609432\\
59.375	0.12228	0.937672221594728\\
59.375	0.12594	1.41379023473706\\
59.375	0.1296	1.90945040703642\\
59.375	0.13326	2.42465273849283\\
59.375	0.13692	2.95939722910626\\
59.375	0.14058	3.51368387887674\\
59.375	0.14424	4.08751268780425\\
59.375	0.1479	4.68088365588878\\
59.375	0.15156	5.29379678313035\\
59.375	0.15522	5.92625206952896\\
59.375	0.15888	6.57824951508461\\
59.375	0.16254	7.24978911979728\\
59.375	0.1662	7.940870883667\\
59.375	0.16986	8.65149480669373\\
59.375	0.17352	9.38166088887752\\
59.375	0.17718	10.1313691302183\\
59.375	0.18084	10.9006195307162\\
59.375	0.1845	11.6894120903711\\
59.375	0.18816	12.497746809183\\
59.375	0.19182	13.3256236871519\\
59.375	0.19548	14.1730427242779\\
59.375	0.19914	15.0400039205609\\
59.375	0.2028	15.926507276001\\
59.375	0.20646	16.8325527905981\\
59.375	0.21012	17.7581404643522\\
59.375	0.21378	18.7032702972634\\
59.375	0.21744	19.6679422893316\\
59.375	0.2211	20.6521564405568\\
59.375	0.22476	21.655912750939\\
59.375	0.22842	22.6792112204783\\
59.375	0.23208	23.7220518491747\\
59.375	0.23574	24.784434637028\\
59.375	0.2394	25.8663595840384\\
59.375	0.24306	26.9678266902059\\
59.375	0.24672	28.0888359555303\\
59.375	0.25038	29.2293873800118\\
59.375	0.25404	30.3894809636503\\
59.375	0.2577	31.5691167064459\\
59.375	0.26136	32.7682946083985\\
59.375	0.26502	33.9870146695082\\
59.375	0.26868	35.2252768897748\\
59.375	0.27234	36.4830812691985\\
59.375	0.276	37.7604278077793\\
59.75	0.093	-2.18286063212781\\
59.75	0.09666	-1.85721498767948\\
59.75	0.10032	-1.51202718407411\\
59.75	0.10398	-1.14729722131171\\
59.75	0.10764	-0.763025099392273\\
59.75	0.1113	-0.359210818315812\\
59.75	0.11496	0.064145621917703\\
59.75	0.11862	0.507044221308238\\
59.75	0.12228	0.96948497985581\\
59.75	0.12594	1.45146789756041\\
59.75	0.1296	1.95299297442206\\
59.75	0.13326	2.47406021044072\\
59.75	0.13692	3.01466960561643\\
59.75	0.14058	3.57482115994918\\
59.75	0.14424	4.15451487343896\\
59.75	0.1479	4.75375074608578\\
59.75	0.15156	5.37252877788962\\
59.75	0.15522	6.0108489688505\\
59.75	0.15888	6.66871131896842\\
59.75	0.16254	7.34611582824336\\
59.75	0.1662	8.04306249667536\\
59.75	0.16986	8.75955132426438\\
59.75	0.17352	9.49558231101042\\
59.75	0.17718	10.2511554569135\\
59.75	0.18084	11.0262707619736\\
59.75	0.1845	11.8209282261908\\
59.75	0.18816	12.635127849565\\
59.75	0.19182	13.4688696320962\\
59.75	0.19548	14.3221535737845\\
59.75	0.19914	15.1949796746298\\
59.75	0.2028	16.0873479346321\\
59.75	0.20646	16.9992583537914\\
59.75	0.21012	17.9307109321079\\
59.75	0.21378	18.8817056695813\\
59.75	0.21744	19.8522425662117\\
59.75	0.2211	20.8423216219992\\
59.75	0.22476	21.8519428369438\\
59.75	0.22842	22.8811062110453\\
59.75	0.23208	23.9298117443039\\
59.75	0.23574	24.9980594367196\\
59.75	0.2394	26.0858492882923\\
59.75	0.24306	27.193181299022\\
59.75	0.24672	28.3200554689087\\
59.75	0.25038	29.4664717979525\\
59.75	0.25404	30.6324302861533\\
59.75	0.2577	31.8179309335111\\
59.75	0.26136	33.022973740026\\
59.75	0.26502	34.2475587056979\\
59.75	0.26868	35.4916858305269\\
59.75	0.27234	36.7553551145128\\
59.75	0.276	38.0385665576559\\
60.125	0.093	-2.19815189425693\\
60.125	0.09666	-1.86664134524633\\
60.125	0.10032	-1.51558863707869\\
60.125	0.10398	-1.14499376975401\\
60.125	0.10764	-0.754856743272301\\
60.125	0.1113	-0.345177557633571\\
60.125	0.11496	0.0840437871622211\\
60.125	0.11862	0.532807291115029\\
60.125	0.12228	1.00111295422487\\
60.125	0.12594	1.48896077649175\\
60.125	0.1296	1.99635075791566\\
60.125	0.13326	2.52328289849661\\
60.125	0.13692	3.06975719823459\\
60.125	0.14058	3.63577365712962\\
60.125	0.14424	4.22133227518167\\
60.125	0.1479	4.82643305239075\\
60.125	0.15156	5.45107598875687\\
60.125	0.15522	6.09526108428002\\
60.125	0.15888	6.75898833896021\\
60.125	0.16254	7.44225775279744\\
60.125	0.1662	8.1450693257917\\
60.125	0.16986	8.86742305794299\\
60.125	0.17352	9.60931894925132\\
60.125	0.17718	10.3707569997167\\
60.125	0.18084	11.1517372093391\\
60.125	0.1845	11.9522595781185\\
60.125	0.18816	12.772324106055\\
60.125	0.19182	13.6119307931485\\
60.125	0.19548	14.471079639399\\
60.125	0.19914	15.3497706448066\\
60.125	0.2028	16.2480038093712\\
60.125	0.20646	17.1657791330928\\
60.125	0.21012	18.1030966159715\\
60.125	0.21378	19.0599562580072\\
60.125	0.21744	20.0363580591999\\
60.125	0.2211	21.0323020195497\\
60.125	0.22476	22.0477881390565\\
60.125	0.22842	23.0828164177203\\
60.125	0.23208	24.1373868555412\\
60.125	0.23574	25.2114994525191\\
60.125	0.2394	26.3051542086541\\
60.125	0.24306	27.418351123946\\
60.125	0.24672	28.5510901983951\\
60.125	0.25038	29.7033714320011\\
60.125	0.25404	30.8751948247642\\
60.125	0.2577	32.0665603766843\\
60.125	0.26136	33.2774680877615\\
60.125	0.26502	34.5079179579956\\
60.125	0.26868	35.7579099873869\\
60.125	0.27234	37.0274441759351\\
60.125	0.276	38.3165205236404\\
60.5	0.093	-2.21362794027805\\
60.5	0.09666	-1.87625248670518\\
60.5	0.10032	-1.51933487397526\\
60.5	0.10398	-1.14287510208831\\
60.5	0.10764	-0.746873171044331\\
60.5	0.1113	-0.331329080843325\\
60.5	0.11496	0.10375716851474\\
60.5	0.11862	0.558385577029821\\
60.5	0.12228	1.03255614470194\\
60.5	0.12594	1.52626887153109\\
60.5	0.1296	2.03952375751728\\
60.5	0.13326	2.57232080266049\\
60.5	0.13692	3.12466000696075\\
60.5	0.14058	3.69654137041805\\
60.5	0.14424	4.28796489303238\\
60.5	0.1479	4.89893057480374\\
60.5	0.15156	5.52943841573212\\
60.5	0.15522	6.17948841581755\\
60.5	0.15888	6.84908057506001\\
60.5	0.16254	7.53821489345951\\
60.5	0.1662	8.24689137101605\\
60.5	0.16986	8.97511000772961\\
60.5	0.17352	9.72287080360021\\
60.5	0.17718	10.4901737586278\\
60.5	0.18084	11.2770188728125\\
60.5	0.1845	12.0834061461542\\
60.5	0.18816	12.9093355786529\\
60.5	0.19182	13.7548071703087\\
60.5	0.19548	14.6198209211215\\
60.5	0.19914	15.5043768310914\\
60.5	0.2028	16.4084749002182\\
60.5	0.20646	17.3321151285022\\
60.5	0.21012	18.2752975159431\\
60.5	0.21378	19.2380220625411\\
60.5	0.21744	20.2202887682961\\
60.5	0.2211	21.2220976332081\\
60.5	0.22476	22.2434486572772\\
60.5	0.22842	23.2843418405033\\
60.5	0.23208	24.3447771828865\\
60.5	0.23574	25.4247546844267\\
60.5	0.2394	26.5242743451239\\
60.5	0.24306	27.6433361649781\\
60.5	0.24672	28.7819401439894\\
60.5	0.25038	29.9400862821578\\
60.5	0.25404	31.1177745794831\\
60.5	0.2577	32.3150050359655\\
60.5	0.26136	33.5317776516049\\
60.5	0.26502	34.7680924264014\\
60.5	0.26868	36.0239493603549\\
60.5	0.27234	37.2993484534654\\
60.5	0.276	38.594289705733\\
60.875	0.093	-2.22928877019119\\
60.875	0.09666	-1.88604841205604\\
60.875	0.10032	-1.52326589476385\\
60.875	0.10398	-1.14094121831463\\
60.875	0.10764	-0.739074382708372\\
60.875	0.1113	-0.317665387945093\\
60.875	0.11496	0.123285765975245\\
60.875	0.11862	0.583779079052599\\
60.875	0.12228	1.06381455128699\\
60.875	0.12594	1.56339218267842\\
60.875	0.1296	2.08251197322688\\
60.875	0.13326	2.62117392293237\\
60.875	0.13692	3.17937803179489\\
60.875	0.14058	3.75712429981447\\
60.875	0.14424	4.35441272699107\\
60.875	0.1479	4.9712433133247\\
60.875	0.15156	5.60761605881537\\
60.875	0.15522	6.26353096346307\\
60.875	0.15888	6.9389880272678\\
60.875	0.16254	7.63398725022956\\
60.875	0.1662	8.34852863234838\\
60.875	0.16986	9.08261217362422\\
60.875	0.17352	9.83623787405709\\
60.875	0.17718	10.609405733647\\
60.875	0.18084	11.4021157523939\\
60.875	0.1845	12.2143679302979\\
60.875	0.18816	13.0461622673589\\
60.875	0.19182	13.897498763577\\
60.875	0.19548	14.7683774189521\\
60.875	0.19914	15.6587982334842\\
60.875	0.2028	16.5687612071733\\
60.875	0.20646	17.4982663400195\\
60.875	0.21012	18.4473136320227\\
60.875	0.21378	19.415903083183\\
60.875	0.21744	20.4040346935003\\
60.875	0.2211	21.4117084629746\\
60.875	0.22476	22.4389243916059\\
60.875	0.22842	23.4856824793943\\
60.875	0.23208	24.5519827263397\\
60.875	0.23574	25.6378251324422\\
60.875	0.2394	26.7432096977017\\
60.875	0.24306	27.8681364221182\\
60.875	0.24672	29.0126053056918\\
60.875	0.25038	30.1766163484224\\
60.875	0.25404	31.36016955031\\
60.875	0.2577	32.5632649113547\\
60.875	0.26136	33.7859024315564\\
60.875	0.26502	35.0280821109151\\
60.875	0.26868	36.2898039494308\\
60.875	0.27234	37.5710679471037\\
60.875	0.276	38.8718741039335\\
61.25	0.093	-2.24513438399632\\
61.25	0.09666	-1.8960291212989\\
61.25	0.10032	-1.52738169944444\\
61.25	0.10398	-1.13919211843294\\
61.25	0.10764	-0.73146037826441\\
61.25	0.1113	-0.304186478938858\\
61.25	0.11496	0.142629579543749\\
61.25	0.11862	0.60898779718338\\
61.25	0.12228	1.09488817398004\\
61.25	0.12594	1.60033070993374\\
61.25	0.1296	2.12531540504448\\
61.25	0.13326	2.66984225931224\\
61.25	0.13692	3.23391127273704\\
61.25	0.14058	3.81752244531889\\
61.25	0.14424	4.42067577705777\\
61.25	0.1479	5.04337126795367\\
61.25	0.15156	5.68560891800661\\
61.25	0.15522	6.34738872721658\\
61.25	0.15888	7.02871069558359\\
61.25	0.16254	7.72957482310763\\
61.25	0.1662	8.44998110978872\\
61.25	0.16986	9.18992955562683\\
61.25	0.17352	9.94942016062197\\
61.25	0.17718	10.7284529247742\\
61.25	0.18084	11.5270278480834\\
61.25	0.1845	12.3451449305496\\
61.25	0.18816	13.1828041721729\\
61.25	0.19182	14.0400055729532\\
61.25	0.19548	14.9167491328906\\
61.25	0.19914	15.813034851985\\
61.25	0.2028	16.7288627302364\\
61.25	0.20646	17.6642327676448\\
61.25	0.21012	18.6191449642103\\
61.25	0.21378	19.5935993199329\\
61.25	0.21744	20.5875958348124\\
61.25	0.2211	21.601134508849\\
61.25	0.22476	22.6342153420426\\
61.25	0.22842	23.6868383343933\\
61.25	0.23208	24.759003485901\\
61.25	0.23574	25.8507107965657\\
61.25	0.2394	26.9619602663875\\
61.25	0.24306	28.0927518953663\\
61.25	0.24672	29.2430856835021\\
61.25	0.25038	30.412961630795\\
61.25	0.25404	31.6023797372449\\
61.25	0.2577	32.8113400028518\\
61.25	0.26136	34.0398424276158\\
61.25	0.26502	35.2878870115368\\
61.25	0.26868	36.5554737546148\\
61.25	0.27234	37.8426026568499\\
61.25	0.276	39.149273718242\\
61.625	0.093	-2.26116478169346\\
61.625	0.09666	-1.90619461443376\\
61.625	0.10032	-1.53168228801703\\
61.625	0.10398	-1.13762780244326\\
61.625	0.10764	-0.724031157712457\\
61.625	0.1113	-0.290892353824631\\
61.625	0.11496	0.161788609220253\\
61.625	0.11862	0.634011731422156\\
61.625	0.12228	1.12577701278109\\
61.625	0.12594	1.63708445329707\\
61.625	0.1296	2.16793405297007\\
61.625	0.13326	2.71832581180011\\
61.625	0.13692	3.28825972978719\\
61.625	0.14058	3.87773580693131\\
61.625	0.14424	4.48675404323245\\
61.625	0.1479	5.11531443869063\\
61.625	0.15156	5.76341699330584\\
61.625	0.15522	6.43106170707809\\
61.625	0.15888	7.11824858000737\\
61.625	0.16254	7.82497761209369\\
61.625	0.1662	8.55124880333706\\
61.625	0.16986	9.29706215373744\\
61.625	0.17352	10.0624176632949\\
61.625	0.17718	10.8473153320093\\
61.625	0.18084	11.6517551598808\\
61.625	0.1845	12.4757371469093\\
61.625	0.18816	13.3192612930949\\
61.625	0.19182	14.1823275984375\\
61.625	0.19548	15.0649360629371\\
61.625	0.19914	15.9670866865938\\
61.625	0.2028	16.8887794694075\\
61.625	0.20646	17.8300144113782\\
61.625	0.21012	18.790791512506\\
61.625	0.21378	19.7711107727908\\
61.625	0.21744	20.7709721922326\\
61.625	0.2211	21.7903757708314\\
61.625	0.22476	22.8293215085873\\
61.625	0.22842	23.8878094055003\\
61.625	0.23208	24.9658394615702\\
61.625	0.23574	26.0634116767973\\
61.625	0.2394	27.1805260511813\\
61.625	0.24306	28.3171825847224\\
61.625	0.24672	29.4733812774205\\
61.625	0.25038	30.6491221292756\\
61.625	0.25404	31.8444051402878\\
61.625	0.2577	33.059230310457\\
61.625	0.26136	34.2935976397833\\
61.625	0.26502	35.5475071282665\\
61.625	0.26868	36.8209587759068\\
61.625	0.27234	38.1139525827042\\
61.625	0.276	39.4264885486586\\
62	0.093	-2.27737996328261\\
62	0.09666	-1.91654489146064\\
62	0.10032	-1.53616766048163\\
62	0.10398	-1.13624827034558\\
62	0.10764	-0.716786721052509\\
62	0.1113	-0.277783012602411\\
62	0.11496	0.180762855004746\\
62	0.11862	0.658850881768922\\
62	0.12228	1.15648106769013\\
62	0.12594	1.67365341276838\\
62	0.1296	2.21036791700366\\
62	0.13326	2.76662458039598\\
62	0.13692	3.34242340294532\\
62	0.14058	3.93776438465172\\
62	0.14424	4.55264752551514\\
62	0.1479	5.18707282553559\\
62	0.15156	5.84104028471307\\
62	0.15522	6.51454990304759\\
62	0.15888	7.20760168053915\\
62	0.16254	7.92019561718774\\
62	0.1662	8.65233171299337\\
62	0.16986	9.40400996795603\\
62	0.17352	10.1752303820757\\
62	0.17718	10.9659929553524\\
62	0.18084	11.7762976877862\\
62	0.1845	12.606144579377\\
62	0.18816	13.4555336301248\\
62	0.19182	14.3244648400297\\
62	0.19548	15.2129382090916\\
62	0.19914	16.1209537373105\\
62	0.2028	17.0485114246865\\
62	0.20646	17.9956112712195\\
62	0.21012	18.9622532769096\\
62	0.21378	19.9484374417566\\
62	0.21744	20.9541637657607\\
62	0.2211	21.9794322489219\\
62	0.22476	23.02424289124\\
62	0.22842	24.0885956927153\\
62	0.23208	25.1724906533475\\
62	0.23574	26.2759277731368\\
62	0.2394	27.3989070520831\\
62	0.24306	28.5414284901864\\
62	0.24672	29.7034920874468\\
62	0.25038	30.8850978438642\\
62	0.25404	32.0862457594387\\
62	0.2577	33.3069358341702\\
62	0.26136	34.5471680680587\\
62	0.26502	35.8069424611042\\
62	0.26868	37.0862590133068\\
62	0.27234	38.3851177246664\\
62	0.276	39.7035185951831\\
62.375	0.093	-2.29377992876376\\
62.375	0.09666	-1.92707995237951\\
62.375	0.10032	-1.54083781683823\\
62.375	0.10398	-1.13505352213991\\
62.375	0.10764	-0.709727068284565\\
62.375	0.1113	-0.26485845527219\\
62.375	0.11496	0.19955231689724\\
62.375	0.11862	0.683505248223689\\
62.375	0.12228	1.18700033870717\\
62.375	0.12594	1.71003758834769\\
62.375	0.1296	2.25261699714525\\
62.375	0.13326	2.81473856509984\\
62.375	0.13692	3.39640229221146\\
62.375	0.14058	3.99760817848012\\
62.375	0.14424	4.61835622390582\\
62.375	0.1479	5.25864642848854\\
62.375	0.15156	5.9184787922283\\
62.375	0.15522	6.5978533151251\\
62.375	0.15888	7.29676999717892\\
62.375	0.16254	8.01522883838978\\
62.375	0.1662	8.75322983875769\\
62.375	0.16986	9.51077299828263\\
62.375	0.17352	10.2878583169646\\
62.375	0.17718	11.0844857948036\\
62.375	0.18084	11.9006554317996\\
62.375	0.1845	12.7363672279527\\
62.375	0.18816	13.5916211832628\\
62.375	0.19182	14.46641729773\\
62.375	0.19548	15.3607555713541\\
62.375	0.19914	16.2746360041353\\
62.375	0.2028	17.2080585960736\\
62.375	0.20646	18.1610233471688\\
62.375	0.21012	19.1335302574212\\
62.375	0.21378	20.1255793268305\\
62.375	0.21744	21.1371705553969\\
62.375	0.2211	22.1683039431203\\
62.375	0.22476	23.2189794900007\\
62.375	0.22842	24.2891971960382\\
62.375	0.23208	25.3789570612327\\
62.375	0.23574	26.4882590855843\\
62.375	0.2394	27.6171032690929\\
62.375	0.24306	28.7654896117585\\
62.375	0.24672	29.9334181135812\\
62.375	0.25038	31.1208887745608\\
62.375	0.25404	32.3279015946976\\
62.375	0.2577	33.5544565739913\\
62.375	0.26136	34.8005537124421\\
62.375	0.26502	36.0661930100499\\
62.375	0.26868	37.3513744668148\\
62.375	0.27234	38.6560980827367\\
62.375	0.276	39.9803638578156\\
62.75	0.093	-2.31036467813691\\
62.75	0.09666	-1.93779979719039\\
62.75	0.10032	-1.54569275708684\\
62.75	0.10398	-1.13404355782625\\
62.75	0.10764	-0.702852199408627\\
62.75	0.1113	-0.252118681833979\\
62.75	0.11496	0.218156994897724\\
62.75	0.11862	0.707974830786446\\
62.75	0.12228	1.2173348258322\\
62.75	0.12594	1.746236980035\\
62.75	0.1296	2.29468129339483\\
62.75	0.13326	2.86266776591168\\
62.75	0.13692	3.45019639758558\\
62.75	0.14058	4.05726718841652\\
62.75	0.14424	4.68388013840449\\
62.75	0.1479	5.33003524754949\\
62.75	0.15156	5.99573251585152\\
62.75	0.15522	6.68097194331058\\
62.75	0.15888	7.38575352992669\\
62.75	0.16254	8.11007727569982\\
62.75	0.1662	8.85394318063001\\
62.75	0.16986	9.61735124471721\\
62.75	0.17352	10.4003014679615\\
62.75	0.17718	11.2027938503627\\
62.75	0.18084	12.024828391921\\
62.75	0.1845	12.8664050926364\\
62.75	0.18816	13.7275239525087\\
62.75	0.19182	14.6081849715382\\
62.75	0.19548	15.5083881497246\\
62.75	0.19914	16.4281334870681\\
62.75	0.2028	17.3674209835686\\
62.75	0.20646	18.3262506392262\\
62.75	0.21012	19.3046224540408\\
62.75	0.21378	20.3025364280124\\
62.75	0.21744	21.319992561141\\
62.75	0.2211	22.3569908534267\\
62.75	0.22476	23.4135313048694\\
62.75	0.22842	24.4896139154692\\
62.75	0.23208	25.585238685226\\
62.75	0.23574	26.7004056141398\\
62.75	0.2394	27.8351147022107\\
62.75	0.24306	28.9893659494386\\
62.75	0.24672	30.1631593558235\\
62.75	0.25038	31.3564949213654\\
62.75	0.25404	32.5693726460644\\
62.75	0.2577	33.8017925299205\\
62.75	0.26136	35.0537545729335\\
62.75	0.26502	36.3252587751036\\
62.75	0.26868	37.6163051364308\\
62.75	0.27234	38.9268936569149\\
62.75	0.276	40.2570243365561\\
63.125	0.093	-2.32713421140207\\
63.125	0.09666	-1.94870442589327\\
63.125	0.10032	-1.55073248122745\\
63.125	0.10398	-1.13321837740459\\
63.125	0.10764	-0.696162114424691\\
63.125	0.1113	-0.239563692287771\\
63.125	0.11496	0.236576889006209\\
63.125	0.11862	0.732259629457204\\
63.125	0.12228	1.24748452906523\\
63.125	0.12594	1.7822515878303\\
63.125	0.1296	2.3365608057524\\
63.125	0.13326	2.91041218283153\\
63.125	0.13692	3.50380571906771\\
63.125	0.14058	4.11674141446092\\
63.125	0.14424	4.74921926901116\\
63.125	0.1479	5.40123928271843\\
63.125	0.15156	6.07280145558274\\
63.125	0.15522	6.76390578760408\\
63.125	0.15888	7.47455227878246\\
63.125	0.16254	8.20474092911786\\
63.125	0.1662	8.95447173861032\\
63.125	0.16986	9.7237447072598\\
63.125	0.17352	10.5125598350663\\
63.125	0.17718	11.3209171220299\\
63.125	0.18084	12.1488165681504\\
63.125	0.1845	12.9962581734281\\
63.125	0.18816	13.8632419378627\\
63.125	0.19182	14.7497678614544\\
63.125	0.19548	15.6558359442031\\
63.125	0.19914	16.5814461861089\\
63.125	0.2028	17.5265985871717\\
63.125	0.20646	18.4912931473915\\
63.125	0.21012	19.4755298667684\\
63.125	0.21378	20.4793087453022\\
63.125	0.21744	21.5026297829932\\
63.125	0.2211	22.5454929798411\\
63.125	0.22476	23.6078983358461\\
63.125	0.22842	24.6898458510081\\
63.125	0.23208	25.7913355253272\\
63.125	0.23574	26.9123673588033\\
63.125	0.2394	28.0529413514364\\
63.125	0.24306	29.2130575032266\\
63.125	0.24672	30.3927158141738\\
63.125	0.25038	31.591916284278\\
63.125	0.25404	32.8106589135393\\
63.125	0.2577	34.0489437019576\\
63.125	0.26136	35.306770649533\\
63.125	0.26502	36.5841397562653\\
63.125	0.26868	37.8810510221548\\
63.125	0.27234	39.1975044472012\\
63.125	0.276	40.5335000314047\\
63.5	0.093	-2.34408852855923\\
63.5	0.09666	-1.95979383848816\\
63.5	0.10032	-1.55595698926007\\
63.5	0.10398	-1.13257798087493\\
63.5	0.10764	-0.689656813332755\\
63.5	0.1113	-0.227193486633562\\
63.5	0.11496	0.25481199922269\\
63.5	0.11862	0.756359644235959\\
63.5	0.12228	1.27744944840627\\
63.5	0.12594	1.8180814117336\\
63.5	0.1296	2.37825553421798\\
63.5	0.13326	2.95797181585939\\
63.5	0.13692	3.55723025665783\\
63.5	0.14058	4.17603085661332\\
63.5	0.14424	4.81437361572583\\
63.5	0.1479	5.47225853399537\\
63.5	0.15156	6.14968561142196\\
63.5	0.15522	6.84665484800557\\
63.5	0.15888	7.56316624374622\\
63.5	0.16254	8.2992197986439\\
63.5	0.1662	9.05481551269863\\
63.5	0.16986	9.82995338591038\\
63.5	0.17352	10.6246334182792\\
63.5	0.17718	11.438855609805\\
63.5	0.18084	12.2726199604878\\
63.5	0.1845	13.1259264703277\\
63.5	0.18816	13.9987751393247\\
63.5	0.19182	14.8911659674786\\
63.5	0.19548	15.8030989547896\\
63.5	0.19914	16.7345741012576\\
63.5	0.2028	17.6855914068827\\
63.5	0.20646	18.6561508716648\\
63.5	0.21012	19.6462524956039\\
63.5	0.21378	20.6558962787001\\
63.5	0.21744	21.6850822209533\\
63.5	0.2211	22.7338103223635\\
63.5	0.22476	23.8020805829308\\
63.5	0.22842	24.8898930026551\\
63.5	0.23208	25.9972475815364\\
63.5	0.23574	27.1241443195748\\
63.5	0.2394	28.2705832167702\\
63.5	0.24306	29.4365642731227\\
63.5	0.24672	30.6220874886321\\
63.5	0.25038	31.8271528632987\\
63.5	0.25404	33.0517603971222\\
63.5	0.2577	34.2959100901028\\
63.5	0.26136	35.5596019422404\\
63.5	0.26502	36.842835953535\\
63.5	0.26868	38.1456121239867\\
63.5	0.27234	39.4679304535954\\
63.5	0.276	40.8097909423612\\
63.875	0.093	-2.3612276296084\\
63.875	0.09666	-1.97106803497506\\
63.875	0.10032	-1.56136628118469\\
63.875	0.10398	-1.13212236823727\\
63.875	0.10764	-0.683336296132831\\
63.875	0.1113	-0.215008064871364\\
63.875	0.11496	0.272862325547161\\
63.875	0.11862	0.780274875122702\\
63.875	0.12228	1.30722958385528\\
63.875	0.12594	1.85372645174489\\
63.875	0.1296	2.41976547879155\\
63.875	0.13326	3.00534666499522\\
63.875	0.13692	3.61047001035593\\
63.875	0.14058	4.2351355148737\\
63.875	0.14424	4.87934317854848\\
63.875	0.1479	5.54309300138031\\
63.875	0.15156	6.22638498336916\\
63.875	0.15522	6.92921912451505\\
63.875	0.15888	7.65159542481797\\
63.875	0.16254	8.39351388427793\\
63.875	0.1662	9.15497450289493\\
63.875	0.16986	9.93597728066896\\
63.875	0.17352	10.7365222176\\
63.875	0.17718	11.5566093136881\\
63.875	0.18084	12.3962385689332\\
63.875	0.1845	13.2554099833354\\
63.875	0.18816	14.1341235568946\\
63.875	0.19182	15.0323792896108\\
63.875	0.19548	15.9501771814841\\
63.875	0.19914	16.8875172325144\\
63.875	0.2028	17.8443994427017\\
63.875	0.20646	18.8208238120461\\
63.875	0.21012	19.8167903405475\\
63.875	0.21378	20.832299028206\\
63.875	0.21744	21.8673498750214\\
63.875	0.2211	22.9219428809939\\
63.875	0.22476	23.9960780461235\\
63.875	0.22842	25.0897553704101\\
63.875	0.23208	26.2029748538537\\
63.875	0.23574	27.3357364964543\\
63.875	0.2394	28.488040298212\\
63.875	0.24306	29.6598862591267\\
63.875	0.24672	30.8512743791985\\
63.875	0.25038	32.0622046584272\\
63.875	0.25404	33.292677096813\\
63.875	0.2577	34.5426916943559\\
63.875	0.26136	35.8122484510558\\
63.875	0.26502	37.1013473669127\\
63.875	0.26868	38.4099884419267\\
63.875	0.27234	39.7381716760977\\
63.875	0.276	41.0858970694257\\
64.25	0.093	-2.37855151454957\\
64.25	0.09666	-1.98252701535395\\
64.25	0.10032	-1.56696035700131\\
64.25	0.10398	-1.13185153949162\\
64.25	0.10764	-0.677200562824908\\
64.25	0.1113	-0.203007427001168\\
64.25	0.11496	0.29072786797963\\
64.25	0.11862	0.804005322117447\\
64.25	0.12228	1.3368249354123\\
64.25	0.12594	1.88918670786419\\
64.25	0.1296	2.46109063947311\\
64.25	0.13326	3.05253673023906\\
64.25	0.13692	3.66352498016205\\
64.25	0.14058	4.29405538924208\\
64.25	0.14424	4.94412795747915\\
64.25	0.1479	5.61374268487324\\
64.25	0.15156	6.30289957142437\\
64.25	0.15522	7.01159861713253\\
64.25	0.15888	7.73983982199772\\
64.25	0.16254	8.48762318601996\\
64.25	0.1662	9.25494870919923\\
64.25	0.16986	10.0418163915355\\
64.25	0.17352	10.8482262330289\\
64.25	0.17718	11.6741782336792\\
64.25	0.18084	12.5196723934866\\
64.25	0.1845	13.3847087124511\\
64.25	0.18816	14.2692871905725\\
64.25	0.19182	15.1734078278511\\
64.25	0.19548	16.0970706242866\\
64.25	0.19914	17.0402755798792\\
64.25	0.2028	18.0030226946288\\
64.25	0.20646	18.9853119685354\\
64.25	0.21012	19.9871434015991\\
64.25	0.21378	21.0085169938198\\
64.25	0.21744	22.0494327451976\\
64.25	0.2211	23.1098906557323\\
64.25	0.22476	24.1898907254242\\
64.25	0.22842	25.289432954273\\
64.25	0.23208	26.4085173422789\\
64.25	0.23574	27.5471438894418\\
64.25	0.2394	28.7053125957618\\
64.25	0.24306	29.8830234612387\\
64.25	0.24672	31.0802764858728\\
64.25	0.25038	32.2970716696638\\
64.25	0.25404	33.5334090126119\\
64.25	0.2577	34.789288514717\\
64.25	0.26136	36.0647101759792\\
64.25	0.26502	37.3596739963984\\
64.25	0.26868	38.6741799759746\\
64.25	0.27234	40.0082281147079\\
64.25	0.276	41.3618184125982\\
64.625	0.093	-2.39606018338274\\
64.625	0.09666	-1.99417077962486\\
64.625	0.10032	-1.57273921670994\\
64.625	0.10398	-1.13176549463798\\
64.625	0.10764	-0.671249613408987\\
64.625	0.1113	-0.191191573022975\\
64.625	0.11496	0.308408626520096\\
64.625	0.11862	0.827550985220187\\
64.625	0.12228	1.36623550307731\\
64.625	0.12594	1.92446218009147\\
64.625	0.1296	2.50223101626267\\
64.625	0.13326	3.0995420115909\\
64.625	0.13692	3.71639516607616\\
64.625	0.14058	4.35279047971847\\
64.625	0.14424	5.00872795251781\\
64.625	0.1479	5.68420758447417\\
64.625	0.15156	6.37922937558757\\
64.625	0.15522	7.093793325858\\
64.625	0.15888	7.82789943528547\\
64.625	0.16254	8.58154770386998\\
64.625	0.1662	9.35473813161152\\
64.625	0.16986	10.1474707185101\\
64.625	0.17352	10.9597454645657\\
64.625	0.17718	11.7915623697783\\
64.625	0.18084	12.642921434148\\
64.625	0.1845	13.5138226576747\\
64.625	0.18816	14.4042660403585\\
64.625	0.19182	15.3142515821993\\
64.625	0.19548	16.2437792831971\\
64.625	0.19914	17.1928491433519\\
64.625	0.2028	18.1614611626638\\
64.625	0.20646	19.1496153411327\\
64.625	0.21012	20.1573116787587\\
64.625	0.21378	21.1845501755417\\
64.625	0.21744	22.2313308314817\\
64.625	0.2211	23.2976536465787\\
64.625	0.22476	24.3835186208328\\
64.625	0.22842	25.488925754244\\
64.625	0.23208	26.6138750468121\\
64.625	0.23574	27.7583664985373\\
64.625	0.2394	28.9224001094195\\
64.625	0.24306	30.1059758794588\\
64.625	0.24672	31.3090938086551\\
64.625	0.25038	32.5317538970084\\
64.625	0.25404	33.7739561445188\\
64.625	0.2577	35.0357005511862\\
64.625	0.26136	36.3169871170106\\
64.625	0.26502	37.6178158419921\\
64.625	0.26868	38.9381867261306\\
64.625	0.27234	40.2780997694261\\
64.625	0.276	41.6375549718787\\
65	0.093	-2.41375363610793\\
65	0.09666	-2.00599932778777\\
65	0.10032	-1.57870286031057\\
65	0.10398	-1.13186423367634\\
65	0.10764	-0.66548344788508\\
65	0.1113	-0.179560502936784\\
65	0.11496	0.325904601168553\\
65	0.11862	0.850911864430916\\
65	0.12228	1.39546128685032\\
65	0.12594	1.95955286842675\\
65	0.1296	2.54318660916022\\
65	0.13326	3.14636250905072\\
65	0.13692	3.76908056809826\\
65	0.14058	4.41134078630284\\
65	0.14424	5.07314316366445\\
65	0.1479	5.75448770018309\\
65	0.15156	6.45537439585875\\
65	0.15522	7.17580325069147\\
65	0.15888	7.91577426468122\\
65	0.16254	8.675287437828\\
65	0.1662	9.45434277013181\\
65	0.16986	10.2529402615927\\
65	0.17352	11.0710799122105\\
65	0.17718	11.9087617219854\\
65	0.18084	12.7659856909174\\
65	0.1845	13.6427518190064\\
65	0.18816	14.5390601062524\\
65	0.19182	15.4549105526555\\
65	0.19548	16.3903031582155\\
65	0.19914	17.3452379229327\\
65	0.2028	18.3197148468068\\
65	0.20646	19.313733929838\\
65	0.21012	20.3272951720263\\
65	0.21378	21.3603985733715\\
65	0.21744	22.4130441338738\\
65	0.2211	23.4852318535331\\
65	0.22476	24.5769617323495\\
65	0.22842	25.6882337703229\\
65	0.23208	26.8190479674533\\
65	0.23574	27.9694043237408\\
65	0.2394	29.1393028391853\\
65	0.24306	30.3287435137868\\
65	0.24672	31.5377263475454\\
65	0.25038	32.766251340461\\
65	0.25404	34.0143184925336\\
65	0.2577	35.2819278037633\\
65	0.26136	36.56907927415\\
65	0.26502	37.8757729036937\\
65	0.26868	39.2020086923945\\
65	0.27234	40.5477866402523\\
65	0.276	41.9131067472672\\
65.375	0.093	-2.43163187272511\\
65.375	0.09666	-2.01801265984268\\
65.375	0.10032	-1.5848512878032\\
65.375	0.10398	-1.13214775660671\\
65.375	0.10764	-0.659902066253167\\
65.375	0.1113	-0.168114216742602\\
65.375	0.11496	0.343215791925008\\
65.375	0.11862	0.874087959749648\\
65.375	0.12228	1.42450228673132\\
65.375	0.12594	1.99445877287003\\
65.375	0.1296	2.58395741816577\\
65.375	0.13326	3.19299822261854\\
65.375	0.13692	3.82158118622836\\
65.375	0.14058	4.46970630899521\\
65.375	0.14424	5.1373735909191\\
65.375	0.1479	5.824583032\\
65.375	0.15156	6.53133463223795\\
65.375	0.15522	7.25762839163293\\
65.375	0.15888	8.00346431018496\\
65.375	0.16254	8.76884238789401\\
65.375	0.1662	9.5537626247601\\
65.375	0.16986	10.3582250207832\\
65.375	0.17352	11.1822295759634\\
65.375	0.17718	12.0257762903006\\
65.375	0.18084	12.8888651637948\\
65.375	0.1845	13.771496196446\\
65.375	0.18816	14.6736693882543\\
65.375	0.19182	15.5953847392197\\
65.375	0.19548	16.536642249342\\
65.375	0.19914	17.4974419186214\\
65.375	0.2028	18.4777837470579\\
65.375	0.20646	19.4776677346513\\
65.375	0.21012	20.4970938814018\\
65.375	0.21378	21.5360621873093\\
65.375	0.21744	22.5945726523739\\
65.375	0.2211	23.6726252765955\\
65.375	0.22476	24.7702200599742\\
65.375	0.22842	25.8873570025098\\
65.375	0.23208	27.0240361042025\\
65.375	0.23574	28.1802573650523\\
65.375	0.2394	29.356020785059\\
65.375	0.24306	30.5513263642228\\
65.375	0.24672	31.7661741025437\\
65.375	0.25038	33.0005640000216\\
65.375	0.25404	34.2544960566565\\
65.375	0.2577	35.5279702724484\\
65.375	0.26136	36.8209866473974\\
65.375	0.26502	38.1335451815034\\
65.375	0.26868	39.4656458747665\\
65.375	0.27234	40.8172887271865\\
65.375	0.276	42.1884737387637\\
65.75	0.093	-2.4496948932343\\
65.75	0.09666	-2.03021077578959\\
65.75	0.10032	-1.59118449918784\\
65.75	0.10398	-1.13261606342907\\
65.75	0.10764	-0.654505468513262\\
65.75	0.1113	-0.15685271444042\\
65.75	0.11496	0.360342198789462\\
65.75	0.11862	0.897079271176375\\
65.75	0.12228	1.45335850272032\\
65.75	0.12594	2.0291798934213\\
65.75	0.1296	2.62454344327932\\
65.75	0.13326	3.23944915229437\\
65.75	0.13692	3.87389702046645\\
65.75	0.14058	4.52788704779558\\
65.75	0.14424	5.20141923428173\\
65.75	0.1479	5.89449357992492\\
65.75	0.15156	6.60711008472514\\
65.75	0.15522	7.3392687486824\\
65.75	0.15888	8.0909695717967\\
65.75	0.16254	8.86221255406802\\
65.75	0.1662	9.65299769549638\\
65.75	0.16986	10.4633249960818\\
65.75	0.17352	11.2931944558242\\
65.75	0.17718	12.1426060747237\\
65.75	0.18084	13.0115598527802\\
65.75	0.1845	13.9000557899937\\
65.75	0.18816	14.8080938863643\\
65.75	0.19182	15.7356741418919\\
65.75	0.19548	16.6827965565765\\
65.75	0.19914	17.6494611304182\\
65.75	0.2028	18.6356678634169\\
65.75	0.20646	19.6414167555726\\
65.75	0.21012	20.6667078068854\\
65.75	0.21378	21.7115410173552\\
65.75	0.21744	22.775916386982\\
65.75	0.2211	23.8598339157659\\
65.75	0.22476	24.9632936037068\\
65.75	0.22842	26.0862954508048\\
65.75	0.23208	27.2288394570597\\
65.75	0.23574	28.3909256224718\\
65.75	0.2394	29.5725539470408\\
65.75	0.24306	30.7737244307669\\
65.75	0.24672	31.99443707365\\
65.75	0.25038	33.2346918756902\\
65.75	0.25404	34.4944888368873\\
65.75	0.2577	35.7738279572416\\
65.75	0.26136	37.0727092367528\\
65.75	0.26502	38.3911326754211\\
65.75	0.26868	39.7290982732464\\
65.75	0.27234	41.0866060302288\\
65.75	0.276	42.4636559463681\\
66.125	0.093	-2.46794269763549\\
66.125	0.09666	-2.04259367562852\\
66.125	0.10032	-1.5977024944645\\
66.125	0.10398	-1.13326915414345\\
66.125	0.10764	-0.649293654665366\\
66.125	0.1113	-0.145775996030251\\
66.125	0.11496	0.377283821761905\\
66.125	0.11862	0.919885798711091\\
66.125	0.12228	1.48202993481731\\
66.125	0.12594	2.06371623008057\\
66.125	0.1296	2.66494468450085\\
66.125	0.13326	3.28571529807818\\
66.125	0.13692	3.92602807081254\\
66.125	0.14058	4.58588300270394\\
66.125	0.14424	5.26528009375237\\
66.125	0.1479	5.96421934395783\\
66.125	0.15156	6.68270075332031\\
66.125	0.15522	7.42072432183984\\
66.125	0.15888	8.17829004951641\\
66.125	0.16254	8.95539793635001\\
66.125	0.1662	9.75204798234065\\
66.125	0.16986	10.5682401874883\\
66.125	0.17352	11.403974551793\\
66.125	0.17718	12.2592510752548\\
66.125	0.18084	13.1340697578735\\
66.125	0.1845	14.0284305996493\\
66.125	0.18816	14.9423336005822\\
66.125	0.19182	15.8757787606721\\
66.125	0.19548	16.828766079919\\
66.125	0.19914	17.8012955583229\\
66.125	0.2028	18.7933671958839\\
66.125	0.20646	19.8049809926019\\
66.125	0.21012	20.8361369484769\\
66.125	0.21378	21.886835063509\\
66.125	0.21744	22.9570753376981\\
66.125	0.2211	24.0468577710443\\
66.125	0.22476	25.1561823635475\\
66.125	0.22842	26.2850491152077\\
66.125	0.23208	27.4334580260249\\
66.125	0.23574	28.6014090959992\\
66.125	0.2394	29.7889023251305\\
66.125	0.24306	30.9959377134189\\
66.125	0.24672	32.2225152608643\\
66.125	0.25038	33.4686349674667\\
66.125	0.25404	34.7342968332262\\
66.125	0.2577	36.0195008581426\\
66.125	0.26136	37.3242470422162\\
66.125	0.26502	38.6485353854467\\
66.125	0.26868	39.9923658878343\\
66.125	0.27234	41.3557385493789\\
66.125	0.276	42.7386533700806\\
66.5	0.093	-2.48637528592869\\
66.5	0.09666	-2.05516135935944\\
66.5	0.10032	-1.60440527363315\\
66.5	0.10398	-1.13410702874983\\
66.5	0.10764	-0.64426662470947\\
66.5	0.1113	-0.134884061512082\\
66.5	0.11496	0.39404066084235\\
66.5	0.11862	0.942507542353809\\
66.5	0.12228	1.5105165830223\\
66.5	0.12594	2.09806778284783\\
66.5	0.1296	2.70516114183039\\
66.5	0.13326	3.33179665996999\\
66.5	0.13692	3.97797433726662\\
66.5	0.14058	4.64369417372029\\
66.5	0.14424	5.328956169331\\
66.5	0.1479	6.03376032409873\\
66.5	0.15156	6.7581066380235\\
66.5	0.15522	7.5019951111053\\
66.5	0.15888	8.26542574334414\\
66.5	0.16254	9.04839853474001\\
66.5	0.1662	9.85091348529292\\
66.5	0.16986	10.6729705950029\\
66.5	0.17352	11.5145698638698\\
66.5	0.17718	12.3757112918938\\
66.5	0.18084	13.2563948790749\\
66.5	0.1845	14.156620625413\\
66.5	0.18816	15.0763885309081\\
66.5	0.19182	16.0156985955602\\
66.5	0.19548	16.9745508193694\\
66.5	0.19914	17.9529452023356\\
66.5	0.2028	18.9508817444589\\
66.5	0.20646	19.9683604457392\\
66.5	0.21012	21.0053813061765\\
66.5	0.21378	22.0619443257708\\
66.5	0.21744	23.1380495045222\\
66.5	0.2211	24.2336968424306\\
66.5	0.22476	25.3488863394961\\
66.5	0.22842	26.4836179957186\\
66.5	0.23208	27.6378918110981\\
66.5	0.23574	28.8117077856347\\
66.5	0.2394	30.0050659193283\\
66.5	0.24306	31.2179662121789\\
66.5	0.24672	32.4504086641866\\
66.5	0.25038	33.7023932753513\\
66.5	0.25404	34.973920045673\\
66.5	0.2577	36.2649889751518\\
66.5	0.26136	37.5756000637876\\
66.5	0.26502	38.9057533115804\\
66.5	0.26868	40.2554487185303\\
66.5	0.27234	41.6246862846372\\
66.5	0.276	43.0134660099011\\
66.875	0.093	-2.50499265811389\\
66.875	0.09666	-2.06791382698237\\
66.875	0.10032	-1.6112928366938\\
66.875	0.10398	-1.13512968724821\\
66.875	0.10764	-0.639424378645575\\
66.875	0.1113	-0.124176910885915\\
66.875	0.11496	0.410612716030791\\
66.875	0.11862	0.964944502104522\\
66.875	0.12228	1.53881844733529\\
66.875	0.12594	2.13223455172309\\
66.875	0.1296	2.74519281526792\\
66.875	0.13326	3.3776932379698\\
66.875	0.13692	4.02973581982871\\
66.875	0.14058	4.70132056084465\\
66.875	0.14424	5.39244746101763\\
66.875	0.1479	6.10311652034763\\
66.875	0.15156	6.83332773883467\\
66.875	0.15522	7.58308111647875\\
66.875	0.15888	8.35237665327987\\
66.875	0.16254	9.14121434923801\\
66.875	0.1662	9.94959420435319\\
66.875	0.16986	10.7775162186254\\
66.875	0.17352	11.6249803920547\\
66.875	0.17718	12.4919867246409\\
66.875	0.18084	13.3785352163843\\
66.875	0.1845	14.2846258672846\\
66.875	0.18816	15.210258677342\\
66.875	0.19182	16.1554336465564\\
66.875	0.19548	17.1201507749279\\
66.875	0.19914	18.1044100624564\\
66.875	0.2028	19.1082115091419\\
66.875	0.20646	20.1315551149845\\
66.875	0.21012	21.1744408799841\\
66.875	0.21378	22.2368688041407\\
66.875	0.21744	23.3188388874543\\
66.875	0.2211	24.420351129925\\
66.875	0.22476	25.5414055315527\\
66.875	0.22842	26.6820020923375\\
66.875	0.23208	27.8421408122793\\
66.875	0.23574	29.0218216913782\\
66.875	0.2394	30.221044729634\\
66.875	0.24306	31.4398099270469\\
66.875	0.24672	32.6781172836169\\
66.875	0.25038	33.9359667993438\\
66.875	0.25404	35.2133584742278\\
66.875	0.2577	36.5102923082689\\
66.875	0.26136	37.826768301467\\
66.875	0.26502	39.1627864538221\\
66.875	0.26868	40.5183467653342\\
66.875	0.27234	41.8934492360034\\
66.875	0.276	43.2880938658296\\
67.25	0.093	-2.52379481419111\\
67.25	0.09666	-2.0808510784973\\
67.25	0.10032	-1.61836518364647\\
67.25	0.10398	-1.1363371296386\\
67.25	0.10764	-0.634766916473692\\
67.25	0.1113	-0.113654544151759\\
67.25	0.11496	0.426999987327219\\
67.25	0.11862	0.987196677963224\\
67.25	0.12228	1.56693552775626\\
67.25	0.12594	2.16621653670634\\
67.25	0.1296	2.78503970481345\\
67.25	0.13326	3.42340503207759\\
67.25	0.13692	4.08131251849878\\
67.25	0.14058	4.75876216407699\\
67.25	0.14424	5.45575396881224\\
67.25	0.1479	6.17228793270452\\
67.25	0.15156	6.90836405575384\\
67.25	0.15522	7.66398233796019\\
67.25	0.15888	8.43914277932358\\
67.25	0.16254	9.23384537984399\\
67.25	0.1662	10.0480901395215\\
67.25	0.16986	10.8818770583559\\
67.25	0.17352	11.7352061363475\\
67.25	0.17718	12.608077373496\\
67.25	0.18084	13.5004907698016\\
67.25	0.1845	14.4124463252642\\
67.25	0.18816	15.3439440398839\\
67.25	0.19182	16.2949839136606\\
67.25	0.19548	17.2655659465943\\
67.25	0.19914	18.2556901386851\\
67.25	0.2028	19.2653564899329\\
67.25	0.20646	20.2945650003377\\
67.25	0.21012	21.3433156698996\\
67.25	0.21378	22.4116084986185\\
67.25	0.21744	23.4994434864944\\
67.25	0.2211	24.6068206335274\\
67.25	0.22476	25.7337399397174\\
67.25	0.22842	26.8802014050644\\
67.25	0.23208	28.0462050295685\\
67.25	0.23574	29.2317508132296\\
67.25	0.2394	30.4368387560477\\
67.25	0.24306	31.6614688580229\\
67.25	0.24672	32.9056411191551\\
67.25	0.25038	34.1693555394444\\
67.25	0.25404	35.4526121188906\\
67.25	0.2577	36.755410857494\\
67.25	0.26136	38.0777517552543\\
67.25	0.26502	39.4196348121717\\
67.25	0.26868	40.7810600282461\\
67.25	0.27234	42.1620274034775\\
67.25	0.276	43.562536937866\\
67.625	0.093	-2.54278175416032\\
67.625	0.09666	-2.09397311390424\\
67.625	0.10032	-1.62562231449113\\
67.625	0.10398	-1.13772935592099\\
67.625	0.10764	-0.630294238193809\\
67.625	0.1113	-0.103316961309599\\
67.625	0.11496	0.443202474731653\\
67.625	0.11862	1.00926406992993\\
67.625	0.12228	1.59486782428524\\
67.625	0.12594	2.2000137377976\\
67.625	0.1296	2.82470181046698\\
67.625	0.13326	3.46893204229339\\
67.625	0.13692	4.13270443327685\\
67.625	0.14058	4.81601898341734\\
67.625	0.14424	5.51887569271486\\
67.625	0.1479	6.24127456116941\\
67.625	0.15156	6.983215588781\\
67.625	0.15522	7.74469877554963\\
67.625	0.15888	8.52572412147529\\
67.625	0.16254	9.32629162655798\\
67.625	0.1662	10.1464012907977\\
67.625	0.16986	10.9860531141945\\
67.625	0.17352	11.8452470967483\\
67.625	0.17718	12.7239832384591\\
67.625	0.18084	13.622261539327\\
67.625	0.1845	14.5400819993519\\
67.625	0.18816	15.4774446185338\\
67.625	0.19182	16.4343493968728\\
67.625	0.19548	17.4107963343688\\
67.625	0.19914	18.4067854310218\\
67.625	0.2028	19.4223166868319\\
67.625	0.20646	20.457390101799\\
67.625	0.21012	21.5120056759231\\
67.625	0.21378	22.5861634092043\\
67.625	0.21744	23.6798633016425\\
67.625	0.2211	24.7931053532377\\
67.625	0.22476	25.92588956399\\
67.625	0.22842	27.0782159338993\\
67.625	0.23208	28.2500844629657\\
67.625	0.23574	29.4414951511891\\
67.625	0.2394	30.6524479985695\\
67.625	0.24306	31.8829430051069\\
67.625	0.24672	33.1329801708014\\
67.625	0.25038	34.4025594956529\\
67.625	0.25404	35.6916809796615\\
67.625	0.2577	37.0003446228271\\
67.625	0.26136	38.3285504251497\\
67.625	0.26502	39.6762983866293\\
67.625	0.26868	41.043588507266\\
67.625	0.27234	42.4304207870598\\
67.625	0.276	43.8367952260105\\
68	0.093	-2.56195347802153\\
68	0.09666	-2.10727993320318\\
68	0.10032	-1.6330642292278\\
68	0.10398	-1.13930636609538\\
68	0.10764	-0.626006343805928\\
68	0.1113	-0.0931641623594448\\
68	0.11496	0.459220178244079\\
68	0.11862	1.03114667800463\\
68	0.12228	1.62261533692222\\
68	0.12594	2.23362615499684\\
68	0.1296	2.86417913222849\\
68	0.13326	3.51427426861718\\
68	0.13692	4.18391156416292\\
68	0.14058	4.87309101886568\\
68	0.14424	5.58181263272547\\
68	0.1479	6.3100764057423\\
68	0.15156	7.05788233791616\\
68	0.15522	7.82523042924706\\
68	0.15888	8.612120679735\\
68	0.16254	9.41855308937996\\
68	0.1662	10.244527658182\\
68	0.16986	11.090044386141\\
68	0.17352	11.9551032732571\\
68	0.17718	12.8397043195302\\
68	0.18084	13.7438475249603\\
68	0.1845	14.6675328895475\\
68	0.18816	15.6107604132917\\
68	0.19182	16.5735300961929\\
68	0.19548	17.5558419382512\\
68	0.19914	18.5576959394665\\
68	0.2028	19.5790920998389\\
68	0.20646	20.6200304193683\\
68	0.21012	21.6805108980547\\
68	0.21378	22.7605335358981\\
68	0.21744	23.8600983328986\\
68	0.2211	24.9792052890561\\
68	0.22476	26.1178544043707\\
68	0.22842	27.2760456788422\\
68	0.23208	28.4537791124708\\
68	0.23574	29.6510547052565\\
68	0.2394	30.8678724571992\\
68	0.24306	32.1042323682989\\
68	0.24672	33.3601344385557\\
68	0.25038	34.6355786679695\\
68	0.25404	35.9305650565403\\
68	0.2577	37.2450936042681\\
68	0.26136	38.5791643111531\\
68	0.26502	39.932777177195\\
68	0.26868	41.3059322023939\\
68	0.27234	42.6986293867499\\
68	0.276	44.110868730263\\
68.375	0.093	-2.58130998577476\\
68.375	0.09666	-2.12077153639413\\
68.375	0.10032	-1.64069092785647\\
68.375	0.10398	-1.14106816016178\\
68.375	0.10764	-0.621903233310062\\
68.375	0.1113	-0.0831961473013028\\
68.375	0.11496	0.475053097864494\\
68.375	0.11862	1.05284450218732\\
68.375	0.12228	1.65017806566718\\
68.375	0.12594	2.26705378830408\\
68.375	0.1296	2.903471670098\\
68.375	0.13326	3.55943171104897\\
68.375	0.13692	4.23493391115697\\
68.375	0.14058	4.92997827042201\\
68.375	0.14424	5.64456478884408\\
68.375	0.1479	6.37869346642318\\
68.375	0.15156	7.13236430315932\\
68.375	0.15522	7.90557729905248\\
68.375	0.15888	8.69833245410269\\
68.375	0.16254	9.51062976830993\\
68.375	0.1662	10.3424692416742\\
68.375	0.16986	11.1938508741955\\
68.375	0.17352	12.0647746658739\\
68.375	0.17718	12.9552406167092\\
68.375	0.18084	13.8652487267016\\
68.375	0.1845	14.7947989958511\\
68.375	0.18816	15.7438914241576\\
68.375	0.19182	16.7125260116211\\
68.375	0.19548	17.7007027582416\\
68.375	0.19914	18.7084216640192\\
68.375	0.2028	19.7356827289539\\
68.375	0.20646	20.7824859530455\\
68.375	0.21012	21.8488313362942\\
68.375	0.21378	22.9347188786999\\
68.375	0.21744	24.0401485802627\\
68.375	0.2211	25.1651204409824\\
68.375	0.22476	26.3096344608593\\
68.375	0.22842	27.4736906398931\\
68.375	0.23208	28.657288978084\\
68.375	0.23574	29.860429475432\\
68.375	0.2394	31.0831121319369\\
68.375	0.24306	32.3253369475989\\
68.375	0.24672	33.5871039224179\\
68.375	0.25038	34.868413056394\\
68.375	0.25404	36.1692643495271\\
68.375	0.2577	37.4896578018172\\
68.375	0.26136	38.8295934132644\\
68.375	0.26502	40.1890711838686\\
68.375	0.26868	41.5680911136298\\
68.375	0.27234	42.9666532025481\\
68.375	0.276	44.3847574506234\\
68.75	0.093	-2.60085127741998\\
68.75	0.09666	-2.13444792347709\\
68.75	0.10032	-1.64850241037715\\
68.75	0.10398	-1.14301473812019\\
68.75	0.10764	-0.617984906706191\\
68.75	0.1113	-0.0734129161351653\\
68.75	0.11496	0.490701233592912\\
68.75	0.11862	1.07435754247801\\
68.75	0.12228	1.67755601052015\\
68.75	0.12594	2.30029663771931\\
68.75	0.1296	2.94257942407552\\
68.75	0.13326	3.60440436958875\\
68.75	0.13692	4.28577147425903\\
68.75	0.14058	4.98668073808634\\
68.75	0.14424	5.70713216107068\\
68.75	0.1479	6.44712574321206\\
68.75	0.15156	7.20666148451046\\
68.75	0.15522	7.98573938496591\\
68.75	0.15888	8.78435944457839\\
68.75	0.16254	9.60252166334791\\
68.75	0.1662	10.4402260412745\\
68.75	0.16986	11.297472578358\\
68.75	0.17352	12.1742612745987\\
68.75	0.17718	13.0705921299963\\
68.75	0.18084	13.986465144551\\
68.75	0.1845	14.9218803182627\\
68.75	0.18816	15.8768376511315\\
68.75	0.19182	16.8513371431573\\
68.75	0.19548	17.8453787943401\\
68.75	0.19914	18.8589626046799\\
68.75	0.2028	19.8920885741768\\
68.75	0.20646	20.9447567028308\\
68.75	0.21012	22.0169669906417\\
68.75	0.21378	23.1087194376097\\
68.75	0.21744	24.2200140437347\\
68.75	0.2211	25.3508508090168\\
68.75	0.22476	26.5012297334559\\
68.75	0.22842	27.671150817052\\
68.75	0.23208	28.8606140598052\\
68.75	0.23574	30.0696194617154\\
68.75	0.2394	31.2981670227826\\
68.75	0.24306	32.5462567430069\\
68.75	0.24672	33.8138886223882\\
68.75	0.25038	35.1010626609265\\
68.75	0.25404	36.4077788586219\\
68.75	0.2577	37.7340372154743\\
68.75	0.26136	39.0798377314838\\
68.75	0.26502	40.4451804066502\\
68.75	0.26868	41.8300652409737\\
68.75	0.27234	43.2344922344543\\
68.75	0.276	44.6584613870919\\
69.125	0.093	-2.62057735295721\\
69.125	0.09666	-2.14830909445204\\
69.125	0.10032	-1.65649867678983\\
69.125	0.10398	-1.1451460999706\\
69.125	0.10764	-0.614251363994326\\
69.125	0.1113	-0.0638144688610236\\
69.125	0.11496	0.506164585429326\\
69.125	0.11862	1.0956857988767\\
69.125	0.12228	1.70474917148111\\
69.125	0.12594	2.33335470324255\\
69.125	0.1296	2.98150239416102\\
69.125	0.13326	3.64919224423653\\
69.125	0.13692	4.33642425346908\\
69.125	0.14058	5.04319842185867\\
69.125	0.14424	5.76951474940529\\
69.125	0.1479	6.51537323610894\\
69.125	0.15156	7.28077388196961\\
69.125	0.15522	8.06571668698733\\
69.125	0.15888	8.87020165116209\\
69.125	0.16254	9.69422877449388\\
69.125	0.1662	10.5377980569827\\
69.125	0.16986	11.4009094986286\\
69.125	0.17352	12.2835630994314\\
69.125	0.17718	13.1857588593914\\
69.125	0.18084	14.1074967785083\\
69.125	0.1845	15.0487768567823\\
69.125	0.18816	16.0095990942133\\
69.125	0.19182	16.9899634908014\\
69.125	0.19548	17.9898700465465\\
69.125	0.19914	19.0093187614486\\
69.125	0.2028	20.0483096355078\\
69.125	0.20646	21.106842668724\\
69.125	0.21012	22.1849178610972\\
69.125	0.21378	23.2825352126275\\
69.125	0.21744	24.3996947233148\\
69.125	0.2211	25.5363963931592\\
69.125	0.22476	26.6926402221605\\
69.125	0.22842	27.8684262103189\\
69.125	0.23208	29.0637543576344\\
69.125	0.23574	30.2786246641068\\
69.125	0.2394	31.5130371297364\\
69.125	0.24306	32.7669917545229\\
69.125	0.24672	34.0404885384665\\
69.125	0.25038	35.3335274815671\\
69.125	0.25404	36.6461085838247\\
69.125	0.2577	37.9782318452394\\
69.125	0.26136	39.3298972658111\\
69.125	0.26502	40.7011048455399\\
69.125	0.26868	42.0918545844256\\
69.125	0.27234	43.5021464824685\\
69.125	0.276	44.9319805396683\\
69.5	0.093	-2.64048821238645\\
69.5	0.09666	-2.16235504931901\\
69.5	0.10032	-1.66467972709452\\
69.5	0.10398	-1.14746224571302\\
69.5	0.10764	-0.61070260517447\\
69.5	0.1113	-0.0544008054788954\\
69.5	0.11496	0.521443153373728\\
69.5	0.11862	1.11682927138337\\
69.5	0.12228	1.73175754855005\\
69.5	0.12594	2.36622798487377\\
69.5	0.1296	3.02024058035452\\
69.5	0.13326	3.6937953349923\\
69.5	0.13692	4.38689224878712\\
69.5	0.14058	5.09953132173898\\
69.5	0.14424	5.83171255384788\\
69.5	0.1479	6.58343594511379\\
69.5	0.15156	7.35470149553675\\
69.5	0.15522	8.14550920511674\\
69.5	0.15888	8.95585907385377\\
69.5	0.16254	9.78575110174783\\
69.5	0.1662	10.6351852887989\\
69.5	0.16986	11.5041616350071\\
69.5	0.17352	12.3926801403722\\
69.5	0.17718	13.3007408048944\\
69.5	0.18084	14.2283436285736\\
69.5	0.1845	15.1754886114099\\
69.5	0.18816	16.1421757534032\\
69.5	0.19182	17.1284050545536\\
69.5	0.19548	18.1341765148609\\
69.5	0.19914	19.1594901343253\\
69.5	0.2028	20.2043459129468\\
69.5	0.20646	21.2687438507253\\
69.5	0.21012	22.3526839476608\\
69.5	0.21378	23.4561662037533\\
69.5	0.21744	24.5791906190029\\
69.5	0.2211	25.7217571934095\\
69.5	0.22476	26.8838659269731\\
69.5	0.22842	28.0655168196938\\
69.5	0.23208	29.2667098715715\\
69.5	0.23574	30.4874450826063\\
69.5	0.2394	31.727722452798\\
69.5	0.24306	32.9875419821469\\
69.5	0.24672	34.2669036706527\\
69.5	0.25038	35.5658075183156\\
69.5	0.25404	36.8842535251355\\
69.5	0.2577	38.2222416911125\\
69.5	0.26136	39.5797720162465\\
69.5	0.26502	40.9568445005375\\
69.5	0.26868	42.3534591439855\\
69.5	0.27234	43.7696159465906\\
69.5	0.276	45.2053149083527\\
69.875	0.093	-2.66058385570768\\
69.875	0.09666	-2.17658578807797\\
69.875	0.10032	-1.67304556129121\\
69.875	0.10398	-1.14996317534743\\
69.875	0.10764	-0.607338630246611\\
69.875	0.1113	-0.0451719259887629\\
69.875	0.11496	0.536536937426133\\
69.875	0.11862	1.13778795999805\\
69.875	0.12228	1.758581141727\\
69.875	0.12594	2.398916482613\\
69.875	0.1296	3.05879398265601\\
69.875	0.13326	3.73821364185607\\
69.875	0.13692	4.43717546021317\\
69.875	0.14058	5.1556794377273\\
69.875	0.14424	5.89372557439847\\
69.875	0.1479	6.65131387022666\\
69.875	0.15156	7.42844432521189\\
69.875	0.15522	8.22511693935415\\
69.875	0.15888	9.04133171265346\\
69.875	0.16254	9.8770886451098\\
69.875	0.1662	10.7323877367232\\
69.875	0.16986	11.6072289874936\\
69.875	0.17352	12.501612397421\\
69.875	0.17718	13.4155379665055\\
69.875	0.18084	14.349005694747\\
69.875	0.1845	15.3020155821455\\
69.875	0.18816	16.2745676287011\\
69.875	0.19182	17.2666618344137\\
69.875	0.19548	18.2782981992833\\
69.875	0.19914	19.30947672331\\
69.875	0.2028	20.3601974064937\\
69.875	0.20646	21.4304602488345\\
69.875	0.21012	22.5202652503323\\
69.875	0.21378	23.6296124109871\\
69.875	0.21744	24.7585017307989\\
69.875	0.2211	25.9069332097678\\
69.875	0.22476	27.0749068478937\\
69.875	0.22842	28.2624226451767\\
69.875	0.23208	29.4694806016167\\
69.875	0.23574	30.6960807172137\\
69.875	0.2394	31.9422229919678\\
69.875	0.24306	33.2079074258788\\
69.875	0.24672	34.493134018947\\
69.875	0.25038	35.7979027711721\\
69.875	0.25404	37.1222136825543\\
69.875	0.2577	38.4660667530935\\
69.875	0.26136	39.8294619827898\\
69.875	0.26502	41.2123993716431\\
69.875	0.26868	42.6148789196534\\
69.875	0.27234	44.0369006268208\\
69.875	0.276	45.4784644931452\\
70.25	0.093	-2.68086428292092\\
70.25	0.09666	-2.19100131072894\\
70.25	0.10032	-1.68159617937991\\
70.25	0.10398	-1.15264888887386\\
70.25	0.10764	-0.604159439210765\\
70.25	0.1113	-0.0361278303906367\\
70.25	0.11496	0.551445937586525\\
70.25	0.11862	1.15856186472072\\
70.25	0.12228	1.78521995101195\\
70.25	0.12594	2.43142019646021\\
70.25	0.1296	3.09716260106551\\
70.25	0.13326	3.78244716482784\\
70.25	0.13692	4.48727388774721\\
70.25	0.14058	5.21164276982362\\
70.25	0.14424	5.95555381105706\\
70.25	0.1479	6.71900701144752\\
70.25	0.15156	7.50200237099502\\
70.25	0.15522	8.30453988969956\\
70.25	0.15888	9.12661956756114\\
70.25	0.16254	9.96824140457975\\
70.25	0.1662	10.8294054007554\\
70.25	0.16986	11.7101115560881\\
70.25	0.17352	12.6103598705778\\
70.25	0.17718	13.5301503442245\\
70.25	0.18084	14.4694829770283\\
70.25	0.1845	15.4283577689891\\
70.25	0.18816	16.406774720107\\
70.25	0.19182	17.4047338303819\\
70.25	0.19548	18.4222350998138\\
70.25	0.19914	19.4592785284027\\
70.25	0.2028	20.5158641161487\\
70.25	0.20646	21.5919918630517\\
70.25	0.21012	22.6876617691118\\
70.25	0.21378	23.8028738343289\\
70.25	0.21744	24.937628058703\\
70.25	0.2211	26.0919244422341\\
70.25	0.22476	27.2657629849223\\
70.25	0.22842	28.4591436867676\\
70.25	0.23208	29.6720665477698\\
70.25	0.23574	30.9045315679291\\
70.25	0.2394	32.1565387472455\\
70.25	0.24306	33.4280880857188\\
70.25	0.24672	34.7191795833492\\
70.25	0.25038	36.0298132401366\\
70.25	0.25404	37.3599890560811\\
70.25	0.2577	38.7097070311826\\
70.25	0.26136	40.0789671654411\\
70.25	0.26502	41.4677694588567\\
70.25	0.26868	42.8761139114293\\
70.25	0.27234	44.3040005231589\\
70.25	0.276	45.7514292940456\\
70.625	0.093	-2.70132949402618\\
70.625	0.09666	-2.20560161727192\\
70.625	0.10032	-1.69033158136062\\
70.625	0.10398	-1.15551938629229\\
70.625	0.10764	-0.601165032066922\\
70.625	0.1113	-0.0272685186845134\\
70.625	0.11496	0.566170153854914\\
70.625	0.11862	1.17915098555138\\
70.625	0.12228	1.81167397640488\\
70.625	0.12594	2.46373912641542\\
70.625	0.1296	3.13534643558299\\
70.625	0.13326	3.8264959039076\\
70.625	0.13692	4.53718753138924\\
70.625	0.14058	5.26742131802792\\
70.625	0.14424	6.01719726382363\\
70.625	0.1479	6.78651536877637\\
70.625	0.15156	7.57537563288615\\
70.625	0.15522	8.38377805615296\\
70.625	0.15888	9.21172263857682\\
70.625	0.16254	10.0592093801577\\
70.625	0.1662	10.9262382808956\\
70.625	0.16986	11.8128093407906\\
70.625	0.17352	12.7189225598425\\
70.625	0.17718	13.6445779380516\\
70.625	0.18084	14.5897754754176\\
70.625	0.1845	15.5545151719407\\
70.625	0.18816	16.5387970276208\\
70.625	0.19182	17.542621042458\\
70.625	0.19548	18.5659872164522\\
70.625	0.19914	19.6088955496034\\
70.625	0.2028	20.6713460419117\\
70.625	0.20646	21.753338693377\\
70.625	0.21012	22.8548735039993\\
70.625	0.21378	23.9759504737786\\
70.625	0.21744	25.116569602715\\
70.625	0.2211	26.2767308908085\\
70.625	0.22476	27.4564343380589\\
70.625	0.22842	28.6556799444664\\
70.625	0.23208	29.8744677100309\\
70.625	0.23574	31.1127976347525\\
70.625	0.2394	32.3706697186311\\
70.625	0.24306	33.6480839616668\\
70.625	0.24672	34.9450403638594\\
70.625	0.25038	36.2615389252091\\
70.625	0.25404	37.5975796457159\\
70.625	0.2577	38.9531625253797\\
70.625	0.26136	40.3282875642005\\
70.625	0.26502	41.7229547621783\\
70.625	0.26868	43.1371641193132\\
70.625	0.27234	44.5709156356051\\
70.625	0.276	46.024209311054\\
71	0.093	-2.72197948902343\\
71	0.09666	-2.22038670770689\\
71	0.10032	-1.69925176723332\\
71	0.10398	-1.15857466760272\\
71	0.10764	-0.598355408815076\\
71	0.1113	-0.0185939908704018\\
71	0.11496	0.580709586231306\\
71	0.11862	1.19955532249005\\
71	0.12228	1.83794321790582\\
71	0.12594	2.49587327247863\\
71	0.1296	3.17334548620848\\
71	0.13326	3.87035985909536\\
71	0.13692	4.58691639113928\\
71	0.14058	5.32301508234023\\
71	0.14424	6.07865593269821\\
71	0.1479	6.85383894221323\\
71	0.15156	7.64856411088528\\
71	0.15522	8.46283143871436\\
71	0.15888	9.2966409257005\\
71	0.16254	10.1499925718436\\
71	0.1662	11.0228863771438\\
71	0.16986	11.915322341601\\
71	0.17352	12.8273004652153\\
71	0.17718	13.7588207479866\\
71	0.18084	14.7098831899149\\
71	0.1845	15.6804877910003\\
71	0.18816	16.6706345512427\\
71	0.19182	17.6803234706421\\
71	0.19548	18.7095545491986\\
71	0.19914	19.7583277869121\\
71	0.2028	20.8266431837826\\
71	0.20646	21.9145007398102\\
71	0.21012	23.0219004549948\\
71	0.21378	24.1488423293364\\
71	0.21744	25.2953263628351\\
71	0.2211	26.4613525554908\\
71	0.22476	27.6469209073035\\
71	0.22842	28.8520314182733\\
71	0.23208	30.0766840884001\\
71	0.23574	31.320878917684\\
71	0.2394	32.5846159061248\\
71	0.24306	33.8678950537227\\
71	0.24672	35.1707163604777\\
71	0.25038	36.4930798263897\\
71	0.25404	37.8349854514587\\
71	0.2577	39.1964332356847\\
71	0.26136	40.5774231790678\\
71	0.26502	41.9779552816079\\
71	0.26868	43.3980295433051\\
71	0.27234	44.8376459641592\\
71	0.276	46.2968045441705\\
71.375	0.093	-2.74281426791269\\
71.375	0.09666	-2.23535658203388\\
71.375	0.10032	-1.70835673699803\\
71.375	0.10398	-1.16181473280515\\
71.375	0.10764	-0.595730569455242\\
71.375	0.1113	-0.0101042469482948\\
71.375	0.11496	0.595064234715686\\
71.375	0.11862	1.2197748755367\\
71.375	0.12228	1.86402767551475\\
71.375	0.12594	2.52782263464983\\
71.375	0.1296	3.21115975294195\\
71.375	0.13326	3.9140390303911\\
71.375	0.13692	4.6364604669973\\
71.375	0.14058	5.37842406276052\\
71.375	0.14424	6.13992981768078\\
71.375	0.1479	6.92097773175806\\
71.375	0.15156	7.72156780499239\\
71.375	0.15522	8.54170003738374\\
71.375	0.15888	9.38137442893215\\
71.375	0.16254	10.2405909796376\\
71.375	0.1662	11.1193496895\\
71.375	0.16986	12.0176505585195\\
71.375	0.17352	12.9354935866961\\
71.375	0.17718	13.8728787740296\\
71.375	0.18084	14.8298061205202\\
71.375	0.1845	15.8062756261679\\
71.375	0.18816	16.8022872909725\\
71.375	0.19182	17.8178411149342\\
71.375	0.19548	18.852937098053\\
71.375	0.19914	19.9075752403287\\
71.375	0.2028	20.9817555417616\\
71.375	0.20646	22.0754780023514\\
71.375	0.21012	23.1887426220983\\
71.375	0.21378	24.3215494010022\\
71.375	0.21744	25.4738983390631\\
71.375	0.2211	26.6457894362811\\
71.375	0.22476	27.8372226926561\\
71.375	0.22842	29.0481981081881\\
71.375	0.23208	30.2787156828772\\
71.375	0.23574	31.5287754167234\\
71.375	0.2394	32.7983773097265\\
71.375	0.24306	34.0875213618867\\
71.375	0.24672	35.3962075732039\\
71.375	0.25038	36.7244359436781\\
71.375	0.25404	38.0722064733094\\
71.375	0.2577	39.4395191620978\\
71.375	0.26136	40.8263740100431\\
71.375	0.26502	42.2327710171455\\
71.375	0.26868	43.6587101834049\\
71.375	0.27234	45.1041915088214\\
71.375	0.276	46.5692149933949\\
71.75	0.093	-2.76383383069395\\
71.75	0.09666	-2.25051124025287\\
71.75	0.10032	-1.71764649065475\\
71.75	0.10398	-1.1652395818996\\
71.75	0.10764	-0.593290513987407\\
71.75	0.1113	-0.00179928691818709\\
71.75	0.11496	0.609234099308066\\
71.75	0.11862	1.23980964469135\\
71.75	0.12228	1.88992734923167\\
71.75	0.12594	2.55958721292903\\
71.75	0.1296	3.24878923578342\\
71.75	0.13326	3.95753341779485\\
71.75	0.13692	4.68581975896332\\
71.75	0.14058	5.43364825928882\\
71.75	0.14424	6.20101891877135\\
71.75	0.1479	6.98793173741091\\
71.75	0.15156	7.7943867152075\\
71.75	0.15522	8.62038385216113\\
71.75	0.15888	9.46592314827181\\
71.75	0.16254	10.3310046035395\\
71.75	0.1662	11.2156282179642\\
71.75	0.16986	12.119793991546\\
71.75	0.17352	13.0435019242848\\
71.75	0.17718	13.9867520161807\\
71.75	0.18084	14.9495442672335\\
71.75	0.1845	15.9318786774434\\
71.75	0.18816	16.9337552468104\\
71.75	0.19182	17.9551739753344\\
71.75	0.19548	18.9961348630154\\
71.75	0.19914	20.0566379098534\\
71.75	0.2028	21.1366831158485\\
71.75	0.20646	22.2362704810006\\
71.75	0.21012	23.3554000053098\\
71.75	0.21378	24.4940716887759\\
71.75	0.21744	25.6522855313992\\
71.75	0.2211	26.8300415331794\\
71.75	0.22476	28.0273396941167\\
71.75	0.22842	29.244180014211\\
71.75	0.23208	30.4805624934624\\
71.75	0.23574	31.7364871318708\\
71.75	0.2394	33.0119539294362\\
71.75	0.24306	34.3069628861586\\
71.75	0.24672	35.6215140020381\\
71.75	0.25038	36.9556072770746\\
71.75	0.25404	38.3092427112682\\
71.75	0.2577	39.6824203046188\\
71.75	0.26136	41.0751400571264\\
71.75	0.26502	42.4874019687911\\
71.75	0.26868	43.9192060396128\\
71.75	0.27234	45.3705522695915\\
71.75	0.276	46.8414406587273\\
72.125	0.093	-2.78503817736722\\
72.125	0.09666	-2.26585068236386\\
72.125	0.10032	-1.72712102820346\\
72.125	0.10398	-1.16884921488604\\
72.125	0.10764	-0.591035242411582\\
72.125	0.1113	0.0063208892199178\\
72.125	0.11496	0.623219180008444\\
72.125	0.11862	1.259659629954\\
72.125	0.12228	1.9156422390566\\
72.125	0.12594	2.59116700731624\\
72.125	0.1296	3.2862339347329\\
72.125	0.13326	4.0008430213066\\
72.125	0.13692	4.73499426703734\\
72.125	0.14058	5.4886876719251\\
72.125	0.14424	6.26192323596991\\
72.125	0.1479	7.05470095917174\\
72.125	0.15156	7.86702084153062\\
72.125	0.15522	8.69888288304652\\
72.125	0.15888	9.55028708371947\\
72.125	0.16254	10.4212334435494\\
72.125	0.1662	11.3117219625365\\
72.125	0.16986	12.2217526406805\\
72.125	0.17352	13.1513254779816\\
72.125	0.17718	14.1004404744397\\
72.125	0.18084	15.0690976300548\\
72.125	0.1845	16.057296944827\\
72.125	0.18816	17.0650384187562\\
72.125	0.19182	18.0923220518425\\
72.125	0.19548	19.1391478440858\\
72.125	0.19914	20.2055157954861\\
72.125	0.2028	21.2914259060434\\
72.125	0.20646	22.3968781757578\\
72.125	0.21012	23.5218726046293\\
72.125	0.21378	24.6664091926577\\
72.125	0.21744	25.8304879398432\\
72.125	0.2211	27.0141088461857\\
72.125	0.22476	28.2172719116853\\
72.125	0.22842	29.4399771363419\\
72.125	0.23208	30.6822245201555\\
72.125	0.23574	31.9440140631262\\
72.125	0.2394	33.2253457652539\\
72.125	0.24306	34.5262196265386\\
72.125	0.24672	35.8466356469803\\
72.125	0.25038	37.1865938265791\\
72.125	0.25404	38.546094165335\\
72.125	0.2577	39.9251366632479\\
72.125	0.26136	41.3237213203178\\
72.125	0.26502	42.7418481365447\\
72.125	0.26868	44.1795171119287\\
72.125	0.27234	45.6367282464697\\
72.125	0.276	47.1134815401677\\
72.5	0.093	-2.80642730793249\\
72.5	0.09666	-2.28137490836686\\
72.5	0.10032	-1.73678034964419\\
72.5	0.10398	-1.17264363176449\\
72.5	0.10764	-0.588964754727762\\
72.5	0.1113	0.014256281466011\\
72.5	0.11496	0.63701947681681\\
72.5	0.11862	1.27932483132464\\
72.5	0.12228	1.94117234498952\\
72.5	0.12594	2.62256201781142\\
72.5	0.1296	3.32349384979036\\
72.5	0.13326	4.04396784092633\\
72.5	0.13692	4.78398399121934\\
72.5	0.14058	5.54354230066939\\
72.5	0.14424	6.32264276927646\\
72.5	0.1479	7.12128539704057\\
72.5	0.15156	7.93947018396171\\
72.5	0.15522	8.77719713003989\\
72.5	0.15888	9.63446623527512\\
72.5	0.16254	10.5112774996674\\
72.5	0.1662	11.4076309232166\\
72.5	0.16986	12.323526505923\\
72.5	0.17352	13.2589642477863\\
72.5	0.17718	14.2139441488067\\
72.5	0.18084	15.1884662089841\\
72.5	0.1845	16.1825304283186\\
72.5	0.18816	17.1961368068101\\
72.5	0.19182	18.2292853444586\\
72.5	0.19548	19.2819760412641\\
72.5	0.19914	20.3542088972267\\
72.5	0.2028	21.4459839123464\\
72.5	0.20646	22.557301086623\\
72.5	0.21012	23.6881604200567\\
72.5	0.21378	24.8385619126474\\
72.5	0.21744	26.0085055643952\\
72.5	0.2211	27.1979913753\\
72.5	0.22476	28.4070193453618\\
72.5	0.22842	29.6355894745807\\
72.5	0.23208	30.8837017629566\\
72.5	0.23574	32.1513562104896\\
72.5	0.2394	33.4385528171795\\
72.5	0.24306	34.7452915830265\\
72.5	0.24672	36.0715725080306\\
72.5	0.25038	37.4173955921916\\
72.5	0.25404	38.7827608355097\\
72.5	0.2577	40.1676682379849\\
72.5	0.26136	41.5721177996171\\
72.5	0.26502	42.9961095204063\\
72.5	0.26868	44.4396434003525\\
72.5	0.27234	45.9027194394558\\
72.5	0.276	47.3853376377161\\
72.875	0.093	-2.82800122238976\\
72.875	0.09666	-2.29708391826186\\
72.875	0.10032	-1.74662445497692\\
72.875	0.10398	-1.17662283253494\\
72.875	0.10764	-0.587079050935939\\
72.875	0.1113	0.0220068898201067\\
72.875	0.11496	0.650634989733179\\
72.875	0.11862	1.29880524880329\\
72.875	0.12228	1.96651766703043\\
72.875	0.12594	2.65377224441461\\
72.875	0.1296	3.36056898095582\\
72.875	0.13326	4.08690787665406\\
72.875	0.13692	4.83278893150936\\
72.875	0.14058	5.59821214552167\\
72.875	0.14424	6.38317751869102\\
72.875	0.1479	7.1876850510174\\
72.875	0.15156	8.01173474250082\\
72.875	0.15522	8.85532659314127\\
72.875	0.15888	9.71846060293877\\
72.875	0.16254	10.6011367718933\\
72.875	0.1662	11.5033551000048\\
72.875	0.16986	12.4251155872734\\
72.875	0.17352	13.3664182336991\\
72.875	0.17718	14.3272630392817\\
72.875	0.18084	15.3076500040214\\
72.875	0.1845	16.3075791279181\\
72.875	0.18816	17.3270504109719\\
72.875	0.19182	18.3660638531827\\
72.875	0.19548	19.4246194545505\\
72.875	0.19914	20.5027172150754\\
72.875	0.2028	21.6003571347573\\
72.875	0.20646	22.7175392135962\\
72.875	0.21012	23.8542634515922\\
72.875	0.21378	25.0105298487452\\
72.875	0.21744	26.1863384050552\\
72.875	0.2211	27.3816891205223\\
72.875	0.22476	28.5965819951464\\
72.875	0.22842	29.8310170289276\\
72.875	0.23208	31.0849942218657\\
72.875	0.23574	32.3585135739609\\
72.875	0.2394	33.6515750852132\\
72.875	0.24306	34.9641787556225\\
72.875	0.24672	36.2963245851888\\
72.875	0.25038	37.6480125739121\\
72.875	0.25404	39.0192427217925\\
72.875	0.2577	40.4100150288299\\
72.875	0.26136	41.8203294950244\\
72.875	0.26502	43.2501861203759\\
72.875	0.26868	44.6995849048844\\
72.875	0.27234	46.1685258485499\\
72.875	0.276	47.6570089513725\\
73.25	0.093	-2.84975992073904\\
73.25	0.09666	-2.31297771204886\\
73.25	0.10032	-1.75665334420165\\
73.25	0.10398	-1.18078681719741\\
73.25	0.10764	-0.585378131036121\\
73.25	0.1113	0.0295727142821978\\
73.25	0.11496	0.664065718757543\\
73.25	0.11862	1.31810088238992\\
73.25	0.12228	1.99167820517934\\
73.25	0.12594	2.68479768712579\\
73.25	0.1296	3.39745932822927\\
73.25	0.13326	4.1296631284898\\
73.25	0.13692	4.88140908790736\\
73.25	0.14058	5.65269720648195\\
73.25	0.14424	6.44352748421357\\
73.25	0.1479	7.25389992110222\\
73.25	0.15156	8.08381451714792\\
73.25	0.15522	8.93327127235064\\
73.25	0.15888	9.80227018671042\\
73.25	0.16254	10.6908112602272\\
73.25	0.1662	11.598894492901\\
73.25	0.16986	12.5265198847319\\
73.25	0.17352	13.4736874357198\\
73.25	0.17718	14.4403971458647\\
73.25	0.18084	15.4266490151667\\
73.25	0.1845	16.4324430436257\\
73.25	0.18816	17.4577792312417\\
73.25	0.19182	18.5026575780148\\
73.25	0.19548	19.5670780839449\\
73.25	0.19914	20.651040749032\\
73.25	0.2028	21.7545455732762\\
73.25	0.20646	22.8775925566774\\
73.25	0.21012	24.0201816992357\\
73.25	0.21378	25.182313000951\\
73.25	0.21744	26.3639864618233\\
73.25	0.2211	27.5652020818526\\
73.25	0.22476	28.785959861039\\
73.25	0.22842	30.0262597993824\\
73.25	0.23208	31.2861018968828\\
73.25	0.23574	32.5654861535403\\
73.25	0.2394	33.8644125693549\\
73.25	0.24306	35.1828811443264\\
73.25	0.24672	36.520891878455\\
73.25	0.25038	37.8784447717406\\
73.25	0.25404	39.2555398241833\\
73.25	0.2577	40.6521770357829\\
73.25	0.26136	42.0683564065397\\
73.25	0.26502	43.5040779364534\\
73.25	0.26868	44.9593416255242\\
73.25	0.27234	46.434147473752\\
73.25	0.276	47.9284954811369\\
73.625	0.093	-2.87170340298033\\
73.625	0.09666	-2.32905628972788\\
73.625	0.10032	-1.76686701731839\\
73.625	0.10398	-1.18513558575187\\
73.625	0.10764	-0.583861995028315\\
73.625	0.1113	0.0369537548522771\\
73.625	0.11496	0.677311663889895\\
73.625	0.11862	1.33721173208455\\
73.625	0.12228	2.01665395943624\\
73.625	0.12594	2.71563834594496\\
73.625	0.1296	3.43416489161072\\
73.625	0.13326	4.17223359643351\\
73.625	0.13692	4.92984446041335\\
73.625	0.14058	5.70699748355021\\
73.625	0.14424	6.50369266584411\\
73.625	0.1479	7.31993000729504\\
73.625	0.15156	8.155709507903\\
73.625	0.15522	9.011031167668\\
73.625	0.15888	9.88589498659005\\
73.625	0.16254	10.7803009646691\\
73.625	0.1662	11.6942491019052\\
73.625	0.16986	12.6277393982983\\
73.625	0.17352	13.5807718538485\\
73.625	0.17718	14.5533464685557\\
73.625	0.18084	15.54546324242\\
73.625	0.1845	16.5571221754412\\
73.625	0.18816	17.5883232676196\\
73.625	0.19182	18.6390665189549\\
73.625	0.19548	19.7093519294473\\
73.625	0.19914	20.7991794990967\\
73.625	0.2028	21.9085492279032\\
73.625	0.20646	23.0374611158666\\
73.625	0.21012	24.1859151629871\\
73.625	0.21378	25.3539113692647\\
73.625	0.21744	26.5414497346993\\
73.625	0.2211	27.7485302592909\\
73.625	0.22476	28.9751529430395\\
73.625	0.22842	30.2213177859452\\
73.625	0.23208	31.4870247880079\\
73.625	0.23574	32.7722739492277\\
73.625	0.2394	34.0770652696045\\
73.625	0.24306	35.4013987491383\\
73.625	0.24672	36.7452743878292\\
73.625	0.25038	38.1086921856771\\
73.625	0.25404	39.491652142682\\
73.625	0.2577	40.894154258844\\
73.625	0.26136	42.316198534163\\
73.625	0.26502	43.757784968639\\
73.625	0.26868	45.218913562272\\
73.625	0.27234	46.6995843150622\\
73.625	0.276	48.1997972270093\\
74	0.093	-2.89383166911362\\
74	0.09666	-2.34531965129889\\
74	0.10032	-1.77726547432713\\
74	0.10398	-1.18966913819834\\
74	0.10764	-0.582530642912513\\
74	0.1113	0.0441500115303519\\
74	0.11496	0.69037282513025\\
74	0.11862	1.35613779788718\\
74	0.12228	2.04144492980114\\
74	0.12594	2.74629422087214\\
74	0.1296	3.47068567110017\\
74	0.13326	4.21461928048523\\
74	0.13692	4.97809504902735\\
74	0.14058	5.76111297672649\\
74	0.14424	6.56367306358265\\
74	0.1479	7.38577530959585\\
74	0.15156	8.22741971476609\\
74	0.15522	9.08860627909336\\
74	0.15888	9.96933500257769\\
74	0.16254	10.869605885219\\
74	0.1662	11.7894189270174\\
74	0.16986	12.7287741279728\\
74	0.17352	13.6876714880853\\
74	0.17718	14.6661110073547\\
74	0.18084	15.6640926857812\\
74	0.1845	16.6816165233648\\
74	0.18816	17.7186825201054\\
74	0.19182	18.775290676003\\
74	0.19548	19.8514409910576\\
74	0.19914	20.9471334652693\\
74	0.2028	22.0623680986381\\
74	0.20646	23.1971448911638\\
74	0.21012	24.3514638428466\\
74	0.21378	25.5253249536864\\
74	0.21744	26.7187282236833\\
74	0.2211	27.9316736528372\\
74	0.22476	29.1641612411481\\
74	0.22842	30.4161909886161\\
74	0.23208	31.6877628952411\\
74	0.23574	32.9788769610231\\
74	0.2394	34.2895331859622\\
74	0.24306	35.6197315700583\\
74	0.24672	36.9694721133114\\
74	0.25038	38.3387548157216\\
74	0.25404	39.7275796772887\\
74	0.2577	41.135946698013\\
74	0.26136	42.5638558778943\\
74	0.26502	44.0113072169326\\
74	0.26868	45.4783007151279\\
74	0.27234	46.9648363724803\\
74	0.276	48.4709141889897\\
};
\end{axis}

\begin{axis}[%
width=4.527496cm,
height=3.050847cm,
at={(6.483547cm,12.711864cm)},
scale only axis,
xmin=56,
xmax=74,
tick align=outside,
xlabel={$L_{cut}$},
xmajorgrids,
ymin=0.093,
ymax=0.276,
ylabel={$D_{rlx}$},
ymajorgrids,
zmin=-0.937692499510501,
zmax=20,
zlabel={$x_1,x_1$},
zmajorgrids,
view={-140}{50},
legend style={at={(1.03,1)},anchor=north west,legend cell align=left,align=left,draw=white!15!black}
]
\addplot3[only marks,mark=*,mark options={},mark size=1.5000pt,color=mycolor1] plot table[row sep=crcr,]{%
74	0.123	0.604502879629624\\
72	0.113	0.869265355514796\\
61	0.095	0.311502215638135\\
56	0.093	0.559357748070687\\
};
\addplot3[only marks,mark=*,mark options={},mark size=1.5000pt,color=mycolor2] plot table[row sep=crcr,]{%
67	0.276	9.23649010843017\\
66	0.255	5.53976715037861\\
62	0.209	4.14525842496664\\
57	0.193	4.10498078977397\\
};
\addplot3[only marks,mark=*,mark options={},mark size=1.5000pt,color=black] plot table[row sep=crcr,]{%
69	0.104	0.865871395480625\\
};
\addplot3[only marks,mark=*,mark options={},mark size=1.5000pt,color=black] plot table[row sep=crcr,]{%
64	0.23	4.65987267620337\\
};

\addplot3[%
surf,
opacity=0.7,
shader=interp,
colormap={mymap}{[1pt] rgb(0pt)=(0.0901961,0.239216,0.0745098); rgb(1pt)=(0.0945149,0.242058,0.0739522); rgb(2pt)=(0.0988592,0.244894,0.0733566); rgb(3pt)=(0.103229,0.247724,0.0727241); rgb(4pt)=(0.107623,0.250549,0.0720557); rgb(5pt)=(0.112043,0.253367,0.0713525); rgb(6pt)=(0.116487,0.25618,0.0706154); rgb(7pt)=(0.120956,0.258986,0.0698456); rgb(8pt)=(0.125449,0.261787,0.0690441); rgb(9pt)=(0.129967,0.264581,0.0682118); rgb(10pt)=(0.134508,0.26737,0.06735); rgb(11pt)=(0.139074,0.270152,0.0664596); rgb(12pt)=(0.143663,0.272929,0.0655416); rgb(13pt)=(0.148275,0.275699,0.0645971); rgb(14pt)=(0.152911,0.278463,0.0636271); rgb(15pt)=(0.15757,0.281221,0.0626328); rgb(16pt)=(0.162252,0.283973,0.0616151); rgb(17pt)=(0.166957,0.286719,0.060575); rgb(18pt)=(0.171685,0.289458,0.0595136); rgb(19pt)=(0.176434,0.292191,0.0584321); rgb(20pt)=(0.181207,0.294918,0.0573313); rgb(21pt)=(0.186001,0.297639,0.0562123); rgb(22pt)=(0.190817,0.300353,0.0550763); rgb(23pt)=(0.195655,0.303061,0.0539242); rgb(24pt)=(0.200514,0.305763,0.052757); rgb(25pt)=(0.205395,0.308459,0.0515759); rgb(26pt)=(0.210296,0.311149,0.0503624); rgb(27pt)=(0.215212,0.313846,0.0490067); rgb(28pt)=(0.220142,0.316548,0.0475043); rgb(29pt)=(0.22509,0.319254,0.0458704); rgb(30pt)=(0.230056,0.321962,0.0441205); rgb(31pt)=(0.235042,0.324671,0.04227); rgb(32pt)=(0.240048,0.327379,0.0403343); rgb(33pt)=(0.245078,0.330085,0.0383287); rgb(34pt)=(0.250131,0.332786,0.0362688); rgb(35pt)=(0.25521,0.335482,0.0341698); rgb(36pt)=(0.260317,0.33817,0.0320472); rgb(37pt)=(0.265451,0.340849,0.0299163); rgb(38pt)=(0.270616,0.343517,0.0277927); rgb(39pt)=(0.275813,0.346172,0.0256916); rgb(40pt)=(0.281043,0.348814,0.0236284); rgb(41pt)=(0.286307,0.35144,0.0216186); rgb(42pt)=(0.291607,0.354048,0.0196776); rgb(43pt)=(0.296945,0.356637,0.0178207); rgb(44pt)=(0.302322,0.359206,0.0160634); rgb(45pt)=(0.307739,0.361753,0.0144211); rgb(46pt)=(0.313198,0.364275,0.0129091); rgb(47pt)=(0.318701,0.366772,0.0115428); rgb(48pt)=(0.324249,0.369242,0.0103377); rgb(49pt)=(0.329843,0.371682,0.00930909); rgb(50pt)=(0.335485,0.374093,0.00847245); rgb(51pt)=(0.341176,0.376471,0.00784314); rgb(52pt)=(0.346925,0.378826,0.00732741); rgb(53pt)=(0.352735,0.381168,0.00682184); rgb(54pt)=(0.358605,0.383497,0.00632729); rgb(55pt)=(0.364532,0.385812,0.00584464); rgb(56pt)=(0.370516,0.388113,0.00537476); rgb(57pt)=(0.376552,0.390399,0.00491852); rgb(58pt)=(0.38264,0.39267,0.00447681); rgb(59pt)=(0.388777,0.394925,0.00405048); rgb(60pt)=(0.394962,0.397164,0.00364042); rgb(61pt)=(0.401191,0.399386,0.00324749); rgb(62pt)=(0.407464,0.401592,0.00287258); rgb(63pt)=(0.413777,0.40378,0.00251655); rgb(64pt)=(0.420129,0.40595,0.00218028); rgb(65pt)=(0.426518,0.408102,0.00186463); rgb(66pt)=(0.432942,0.410234,0.00157049); rgb(67pt)=(0.439399,0.412348,0.00129873); rgb(68pt)=(0.445885,0.414441,0.00105022); rgb(69pt)=(0.452401,0.416515,0.000825833); rgb(70pt)=(0.458942,0.418567,0.000626441); rgb(71pt)=(0.465508,0.420599,0.00045292); rgb(72pt)=(0.472096,0.422609,0.000306141); rgb(73pt)=(0.478704,0.424596,0.000186979); rgb(74pt)=(0.485331,0.426562,9.63073e-05); rgb(75pt)=(0.491973,0.428504,3.49981e-05); rgb(76pt)=(0.498628,0.430422,3.92506e-06); rgb(77pt)=(0.505323,0.432315,0); rgb(78pt)=(0.512206,0.434168,0); rgb(79pt)=(0.519282,0.435983,0); rgb(80pt)=(0.526529,0.437764,0); rgb(81pt)=(0.533922,0.439512,0); rgb(82pt)=(0.54144,0.441232,0); rgb(83pt)=(0.549059,0.442927,0); rgb(84pt)=(0.556756,0.444599,0); rgb(85pt)=(0.564508,0.446252,0); rgb(86pt)=(0.572292,0.447889,0); rgb(87pt)=(0.580084,0.449514,0); rgb(88pt)=(0.587863,0.451129,0); rgb(89pt)=(0.595604,0.452737,0); rgb(90pt)=(0.603284,0.454343,0); rgb(91pt)=(0.610882,0.455948,0); rgb(92pt)=(0.618373,0.457556,0); rgb(93pt)=(0.625734,0.459171,0); rgb(94pt)=(0.632943,0.460795,0); rgb(95pt)=(0.639976,0.462432,0); rgb(96pt)=(0.64681,0.464084,0); rgb(97pt)=(0.653423,0.465756,0); rgb(98pt)=(0.659791,0.46745,0); rgb(99pt)=(0.665891,0.469169,0); rgb(100pt)=(0.6717,0.470916,0); rgb(101pt)=(0.677195,0.472696,0); rgb(102pt)=(0.682353,0.47451,0); rgb(103pt)=(0.687242,0.476355,0); rgb(104pt)=(0.691952,0.478225,0); rgb(105pt)=(0.696497,0.480118,0); rgb(106pt)=(0.700887,0.482033,0); rgb(107pt)=(0.705134,0.483968,0); rgb(108pt)=(0.709251,0.485921,0); rgb(109pt)=(0.713249,0.487891,0); rgb(110pt)=(0.71714,0.489876,0); rgb(111pt)=(0.720936,0.491875,0); rgb(112pt)=(0.724649,0.493887,0); rgb(113pt)=(0.72829,0.495909,0); rgb(114pt)=(0.731872,0.49794,0); rgb(115pt)=(0.735406,0.499979,0); rgb(116pt)=(0.738904,0.502025,0); rgb(117pt)=(0.742378,0.504075,0); rgb(118pt)=(0.74584,0.506128,0); rgb(119pt)=(0.749302,0.508182,0); rgb(120pt)=(0.752775,0.510237,0); rgb(121pt)=(0.756272,0.51229,0); rgb(122pt)=(0.759804,0.514339,0); rgb(123pt)=(0.763384,0.516385,0); rgb(124pt)=(0.767022,0.518424,0); rgb(125pt)=(0.770731,0.520455,0); rgb(126pt)=(0.774523,0.522478,0); rgb(127pt)=(0.77841,0.524489,0); rgb(128pt)=(0.782391,0.526491,0); rgb(129pt)=(0.786402,0.528496,0); rgb(130pt)=(0.790431,0.530506,0); rgb(131pt)=(0.794478,0.532521,0); rgb(132pt)=(0.798541,0.534539,0); rgb(133pt)=(0.802619,0.53656,0); rgb(134pt)=(0.806712,0.538584,0); rgb(135pt)=(0.81082,0.540609,0); rgb(136pt)=(0.81494,0.542635,0); rgb(137pt)=(0.819074,0.54466,0); rgb(138pt)=(0.823219,0.546686,0); rgb(139pt)=(0.827374,0.548709,0); rgb(140pt)=(0.831541,0.55073,0); rgb(141pt)=(0.835716,0.552749,0); rgb(142pt)=(0.8399,0.554763,0); rgb(143pt)=(0.844092,0.556774,0); rgb(144pt)=(0.848292,0.558779,0); rgb(145pt)=(0.852497,0.560778,0); rgb(146pt)=(0.856708,0.562771,0); rgb(147pt)=(0.860924,0.564756,0); rgb(148pt)=(0.865143,0.566733,0); rgb(149pt)=(0.869366,0.568701,0); rgb(150pt)=(0.873592,0.57066,0); rgb(151pt)=(0.877819,0.572608,0); rgb(152pt)=(0.882047,0.574545,0); rgb(153pt)=(0.886275,0.576471,0); rgb(154pt)=(0.890659,0.578362,0); rgb(155pt)=(0.895333,0.580203,0); rgb(156pt)=(0.900258,0.581999,0); rgb(157pt)=(0.905397,0.583755,0); rgb(158pt)=(0.910711,0.585479,0); rgb(159pt)=(0.916164,0.587176,0); rgb(160pt)=(0.921717,0.588852,0); rgb(161pt)=(0.927333,0.590513,0); rgb(162pt)=(0.932974,0.592166,0); rgb(163pt)=(0.938602,0.593815,0); rgb(164pt)=(0.94418,0.595468,0); rgb(165pt)=(0.949669,0.59713,0); rgb(166pt)=(0.955033,0.598808,0); rgb(167pt)=(0.960233,0.600507,0); rgb(168pt)=(0.965232,0.602233,0); rgb(169pt)=(0.969992,0.603992,0); rgb(170pt)=(0.974475,0.605791,0); rgb(171pt)=(0.978643,0.607636,0); rgb(172pt)=(0.98246,0.609532,0); rgb(173pt)=(0.985886,0.611486,0); rgb(174pt)=(0.988885,0.613503,0); rgb(175pt)=(0.991419,0.61559,0); rgb(176pt)=(0.99345,0.617753,0); rgb(177pt)=(0.99494,0.619997,0); rgb(178pt)=(0.995851,0.622329,0); rgb(179pt)=(0.996226,0.624763,0); rgb(180pt)=(0.996512,0.627352,0); rgb(181pt)=(0.996788,0.630095,0); rgb(182pt)=(0.997053,0.632982,0); rgb(183pt)=(0.997308,0.636004,0); rgb(184pt)=(0.997552,0.639152,0); rgb(185pt)=(0.997785,0.642416,0); rgb(186pt)=(0.998006,0.645786,0); rgb(187pt)=(0.998217,0.649253,0); rgb(188pt)=(0.998416,0.652807,0); rgb(189pt)=(0.998605,0.656439,0); rgb(190pt)=(0.998781,0.660138,0); rgb(191pt)=(0.998946,0.663897,0); rgb(192pt)=(0.9991,0.667704,0); rgb(193pt)=(0.999242,0.67155,0); rgb(194pt)=(0.999372,0.675427,0); rgb(195pt)=(0.99949,0.679323,0); rgb(196pt)=(0.999596,0.68323,0); rgb(197pt)=(0.99969,0.687139,0); rgb(198pt)=(0.999771,0.691039,0); rgb(199pt)=(0.999841,0.694921,0); rgb(200pt)=(0.999898,0.698775,0); rgb(201pt)=(0.999942,0.702592,0); rgb(202pt)=(0.999974,0.706363,0); rgb(203pt)=(0.999994,0.710077,0); rgb(204pt)=(1,0.713725,0); rgb(205pt)=(1,0.717341,0); rgb(206pt)=(1,0.720963,0); rgb(207pt)=(1,0.724591,0); rgb(208pt)=(1,0.728226,0); rgb(209pt)=(1,0.731867,0); rgb(210pt)=(1,0.735514,0); rgb(211pt)=(1,0.739167,0); rgb(212pt)=(1,0.742827,0); rgb(213pt)=(1,0.746493,0); rgb(214pt)=(1,0.750165,0); rgb(215pt)=(1,0.753843,0); rgb(216pt)=(1,0.757527,0); rgb(217pt)=(1,0.761217,0); rgb(218pt)=(1,0.764913,0); rgb(219pt)=(1,0.768615,0); rgb(220pt)=(1,0.772324,0); rgb(221pt)=(1,0.776038,0); rgb(222pt)=(1,0.779758,0); rgb(223pt)=(1,0.783484,0); rgb(224pt)=(1,0.787215,0); rgb(225pt)=(1,0.790953,0); rgb(226pt)=(1,0.794696,0); rgb(227pt)=(1,0.798445,0); rgb(228pt)=(1,0.8022,0); rgb(229pt)=(1,0.805961,0); rgb(230pt)=(1,0.809727,0); rgb(231pt)=(1,0.8135,0); rgb(232pt)=(1,0.817278,0); rgb(233pt)=(1,0.821063,0); rgb(234pt)=(1,0.824854,0); rgb(235pt)=(1,0.828652,0); rgb(236pt)=(1,0.832455,0); rgb(237pt)=(1,0.836265,0); rgb(238pt)=(1,0.840081,0); rgb(239pt)=(1,0.843903,0); rgb(240pt)=(1,0.847732,0); rgb(241pt)=(1,0.851566,0); rgb(242pt)=(1,0.855406,0); rgb(243pt)=(1,0.859253,0); rgb(244pt)=(1,0.863106,0); rgb(245pt)=(1,0.866964,0); rgb(246pt)=(1,0.870829,0); rgb(247pt)=(1,0.8747,0); rgb(248pt)=(1,0.878577,0); rgb(249pt)=(1,0.88246,0); rgb(250pt)=(1,0.886349,0); rgb(251pt)=(1,0.890243,0); rgb(252pt)=(1,0.894144,0); rgb(253pt)=(1,0.898051,0); rgb(254pt)=(1,0.901964,0); rgb(255pt)=(1,0.905882,0)},
mesh/rows=49]
table[row sep=crcr,header=false] {%
%
56	0.093	0.483089470777274\\
56	0.09666	0.44279855160543\\
56	0.10032	0.417504871984555\\
56	0.10398	0.40720843191465\\
56	0.10764	0.411909231395704\\
56	0.1113	0.431607270427725\\
56	0.11496	0.466302549010711\\
56	0.11862	0.515995067144662\\
56	0.12228	0.580684824829587\\
56	0.12594	0.660371822065471\\
56	0.1296	0.755056058852318\\
56	0.13326	0.864737535190134\\
56	0.13692	0.989416251078922\\
56	0.14058	1.12909220651867\\
56	0.14424	1.28376540150938\\
56	0.1479	1.45343583605106\\
56	0.15156	1.63810351014371\\
56	0.15522	1.83776842378732\\
56	0.15888	2.0524305769819\\
56	0.16254	2.28208996972744\\
56	0.1662	2.52674660202395\\
56	0.16986	2.78640047387142\\
56	0.17352	3.06105158526987\\
56	0.17718	3.35069993621928\\
56	0.18084	3.65534552671964\\
56	0.1845	3.97498835677099\\
56	0.18816	4.30962842637329\\
56	0.19182	4.65926573552656\\
56	0.19548	5.02390028423079\\
56	0.19914	5.40353207248599\\
56	0.2028	5.79816110029217\\
56	0.20646	6.2077873676493\\
56	0.21012	6.6324108745574\\
56	0.21378	7.07203162101646\\
56	0.21744	7.52664960702649\\
56	0.2211	7.99626483258749\\
56	0.22476	8.48087729769945\\
56	0.22842	8.98048700236238\\
56	0.23208	9.49509394657627\\
56	0.23574	10.0246981303411\\
56	0.2394	10.569299553657\\
56	0.24306	11.1288982165238\\
56	0.24672	11.7034941189415\\
56	0.25038	12.2930872609102\\
56	0.25404	12.8976776424299\\
56	0.2577	13.5172652635006\\
56	0.26136	14.1518501241222\\
56	0.26502	14.8014322242948\\
56	0.26868	15.4660115640183\\
56	0.27234	16.1455881432929\\
56	0.276	16.8401619621183\\
56.375	0.093	0.468569165224552\\
56.375	0.09666	0.422239013162634\\
56.375	0.10032	0.390906100651684\\
56.375	0.10398	0.374570427691697\\
56.375	0.10764	0.373231994282669\\
56.375	0.1113	0.386890800424616\\
56.375	0.11496	0.415546846117527\\
56.375	0.11862	0.459200131361404\\
56.375	0.12228	0.517850656156247\\
56.375	0.12594	0.591498420502056\\
56.375	0.1296	0.680143424398821\\
56.375	0.13326	0.783785667846569\\
56.375	0.13692	0.902425150845268\\
56.375	0.14058	1.03606187339494\\
56.375	0.14424	1.18469583549558\\
56.375	0.1479	1.34832703714718\\
56.375	0.15156	1.52695547834975\\
56.375	0.15522	1.72058115910328\\
56.375	0.15888	1.92920407940779\\
56.375	0.16254	2.15282423926325\\
56.375	0.1662	2.39144163866968\\
56.375	0.16986	2.64505627762708\\
56.375	0.17352	2.91366815613544\\
56.375	0.17718	3.19727727419477\\
56.375	0.18084	3.49588363180506\\
56.375	0.1845	3.80948722896633\\
56.375	0.18816	4.13808806567855\\
56.375	0.19182	4.48168614194174\\
56.375	0.19548	4.8402814577559\\
56.375	0.19914	5.21387401312103\\
56.375	0.2028	5.60246380803711\\
56.375	0.20646	6.00605084250417\\
56.375	0.21012	6.42463511652219\\
56.375	0.21378	6.85821663009118\\
56.375	0.21744	7.30679538321113\\
56.375	0.2211	7.77037137588205\\
56.375	0.22476	8.24894460810393\\
56.375	0.22842	8.74251507987679\\
56.375	0.23208	9.2510827912006\\
56.375	0.23574	9.77464774207538\\
56.375	0.2394	10.3132099325011\\
56.375	0.24306	10.8667693624778\\
56.375	0.24672	11.4353260320055\\
56.375	0.25038	12.0188799410842\\
56.375	0.25404	12.6174310897138\\
56.375	0.2577	13.2309794778944\\
56.375	0.26136	13.8595251056259\\
56.375	0.26502	14.5030679729084\\
56.375	0.26868	15.1616080797419\\
56.375	0.27234	15.8351454261263\\
56.375	0.276	16.5236800120617\\
56.75	0.093	0.456775717244033\\
56.75	0.09666	0.40440633229204\\
56.75	0.10032	0.367034186891008\\
56.75	0.10398	0.344659281040939\\
56.75	0.10764	0.337281614741844\\
56.75	0.1113	0.344901187993709\\
56.75	0.11496	0.367518000796538\\
56.75	0.11862	0.40513205315034\\
56.75	0.12228	0.457743345055109\\
56.75	0.12594	0.525351876510836\\
56.75	0.1296	0.607957647517527\\
56.75	0.13326	0.705560658075193\\
56.75	0.13692	0.818160908183817\\
56.75	0.14058	0.945758397843409\\
56.75	0.14424	1.08835312705397\\
56.75	0.1479	1.24594509581549\\
56.75	0.15156	1.41853430412799\\
56.75	0.15522	1.60612075199144\\
56.75	0.15888	1.80870443940587\\
56.75	0.16254	2.02628536637125\\
56.75	0.1662	2.25886353288761\\
56.75	0.16986	2.50643893895493\\
56.75	0.17352	2.76901158457321\\
56.75	0.17718	3.04658146974246\\
56.75	0.18084	3.33914859446267\\
56.75	0.1845	3.64671295873387\\
56.75	0.18816	3.96927456255602\\
56.75	0.19182	4.30683340592913\\
56.75	0.19548	4.65938948885321\\
56.75	0.19914	5.02694281132827\\
56.75	0.2028	5.40949337335428\\
56.75	0.20646	5.80704117493125\\
56.75	0.21012	6.21958621605919\\
56.75	0.21378	6.64712849673811\\
56.75	0.21744	7.08966801696798\\
56.75	0.2211	7.54720477674882\\
56.75	0.22476	8.01973877608063\\
56.75	0.22842	8.5072700149634\\
56.75	0.23208	9.00979849339713\\
56.75	0.23574	9.52732421138185\\
56.75	0.2394	10.0598471689175\\
56.75	0.24306	10.6073673660042\\
56.75	0.24672	11.1698848026418\\
56.75	0.25038	11.7473994788303\\
56.75	0.25404	12.3399113945698\\
56.75	0.2577	12.9474205498604\\
56.75	0.26136	13.5699269447018\\
56.75	0.26502	14.2074305790942\\
56.75	0.26868	14.8599314530376\\
56.75	0.27234	15.527429566532\\
56.75	0.276	16.2099249195773\\
57.125	0.093	0.447709126835712\\
57.125	0.09666	0.389300508993645\\
57.125	0.10032	0.345889130702531\\
57.125	0.10398	0.317474991962388\\
57.125	0.10764	0.304058092773211\\
57.125	0.1113	0.305638433135001\\
57.125	0.11496	0.322216013047756\\
57.125	0.11862	0.353790832511476\\
57.125	0.12228	0.400362891526163\\
57.125	0.12594	0.461932190091815\\
57.125	0.1296	0.538498728208431\\
57.125	0.13326	0.630062505876023\\
57.125	0.13692	0.736623523094572\\
57.125	0.14058	0.858181779864083\\
57.125	0.14424	0.994737276184573\\
57.125	0.1479	1.14629001205601\\
57.125	0.15156	1.31283998747843\\
57.125	0.15522	1.49438720245181\\
57.125	0.15888	1.69093165697615\\
57.125	0.16254	1.90247335105147\\
57.125	0.1662	2.12901228467774\\
57.125	0.16986	2.37054845785498\\
57.125	0.17352	2.62708187058319\\
57.125	0.17718	2.89861252286236\\
57.125	0.18084	3.1851404146925\\
57.125	0.1845	3.48666554607361\\
57.125	0.18816	3.80318791700568\\
57.125	0.19182	4.13470752748872\\
57.125	0.19548	4.48122437752272\\
57.125	0.19914	4.84273846710769\\
57.125	0.2028	5.21924979624363\\
57.125	0.20646	5.61075836493053\\
57.125	0.21012	6.0172641731684\\
57.125	0.21378	6.43876722095722\\
57.125	0.21744	6.87526750829702\\
57.125	0.2211	7.32676503518779\\
57.125	0.22476	7.79325980162951\\
57.125	0.22842	8.27475180762221\\
57.125	0.23208	8.77124105316586\\
57.125	0.23574	9.2827275382605\\
57.125	0.2394	9.8092112629061\\
57.125	0.24306	10.3506922271026\\
57.125	0.24672	10.9071704308502\\
57.125	0.25038	11.4786458741487\\
57.125	0.25404	12.0651185569981\\
57.125	0.2577	12.6665884793985\\
57.125	0.26136	13.2830556413499\\
57.125	0.26502	13.9145200428523\\
57.125	0.26868	14.5609816839056\\
57.125	0.27234	15.2224405645099\\
57.125	0.276	15.8988966846651\\
57.5	0.093	0.441369393999602\\
57.5	0.09666	0.376921543267453\\
57.5	0.10032	0.327470932086271\\
57.5	0.10398	0.293017560456046\\
57.5	0.10764	0.273561428376794\\
57.5	0.1113	0.269102535848503\\
57.5	0.11496	0.279640882871183\\
57.5	0.11862	0.305176469444828\\
57.5	0.12228	0.345709295569433\\
57.5	0.12594	0.401239361245004\\
57.5	0.1296	0.471766666471545\\
57.5	0.13326	0.557291211249055\\
57.5	0.13692	0.65781299557753\\
57.5	0.14058	0.773332019456966\\
57.5	0.14424	0.903848282887367\\
57.5	0.1479	1.04936178586874\\
57.5	0.15156	1.20987252840108\\
57.5	0.15522	1.38538051048438\\
57.5	0.15888	1.57588573211865\\
57.5	0.16254	1.78138819330388\\
57.5	0.1662	2.00188789404008\\
57.5	0.16986	2.23738483432724\\
57.5	0.17352	2.48787901416538\\
57.5	0.17718	2.75337043355447\\
57.5	0.18084	3.03385909249452\\
57.5	0.1845	3.32934499098556\\
57.5	0.18816	3.63982812902756\\
57.5	0.19182	3.96530850662052\\
57.5	0.19548	4.30578612376444\\
57.5	0.19914	4.66126098045932\\
57.5	0.2028	5.03173307670519\\
57.5	0.20646	5.41720241250201\\
57.5	0.21012	5.8176689878498\\
57.5	0.21378	6.23313280274855\\
57.5	0.21744	6.66359385719827\\
57.5	0.2211	7.10905215119896\\
57.5	0.22476	7.56950768475061\\
57.5	0.22842	8.04496045785323\\
57.5	0.23208	8.53541047050681\\
57.5	0.23574	9.04085772271136\\
57.5	0.2394	9.56130221446688\\
57.5	0.24306	10.0967439457734\\
57.5	0.24672	10.6471829166308\\
57.5	0.25038	11.2126191270392\\
57.5	0.25404	11.7930525769986\\
57.5	0.2577	12.3884832665089\\
57.5	0.26136	12.9989111955703\\
57.5	0.26502	13.6243363641825\\
57.5	0.26868	14.2647587723458\\
57.5	0.27234	14.92017842006\\
57.5	0.276	15.5905953073251\\
57.875	0.093	0.437756518735704\\
57.875	0.09666	0.367269435113474\\
57.875	0.10032	0.31177959104221\\
57.875	0.10398	0.271286986521917\\
57.875	0.10764	0.245791621552584\\
57.875	0.1113	0.235293496134218\\
57.875	0.11496	0.239792610266816\\
57.875	0.11862	0.25928896395038\\
57.875	0.12228	0.293782557184917\\
57.875	0.12594	0.343273389970406\\
57.875	0.1296	0.407761462306873\\
57.875	0.13326	0.487246774194308\\
57.875	0.13692	0.581729325632701\\
57.875	0.14058	0.691209116622055\\
57.875	0.14424	0.815686147162381\\
57.875	0.1479	0.955160417253678\\
57.875	0.15156	1.10963192689594\\
57.875	0.15522	1.27910067608916\\
57.875	0.15888	1.46356666483336\\
57.875	0.16254	1.6630298931285\\
57.875	0.1662	1.87749036097463\\
57.875	0.16986	2.10694806837171\\
57.875	0.17352	2.35140301531976\\
57.875	0.17718	2.61085520181879\\
57.875	0.18084	2.88530462786876\\
57.875	0.1845	3.17475129346972\\
57.875	0.18816	3.47919519862164\\
57.875	0.19182	3.79863634332452\\
57.875	0.19548	4.13307472757837\\
57.875	0.19914	4.48251035138318\\
57.875	0.2028	4.84694321473896\\
57.875	0.20646	5.22637331764571\\
57.875	0.21012	5.62080066010342\\
57.875	0.21378	6.0302252421121\\
57.875	0.21744	6.45464706367175\\
57.875	0.2211	6.89406612478234\\
57.875	0.22476	7.34848242544392\\
57.875	0.22842	7.81789596565646\\
57.875	0.23208	8.30230674541997\\
57.875	0.23574	8.80171476473445\\
57.875	0.2394	9.31612002359988\\
57.875	0.24306	9.84552252201628\\
57.875	0.24672	10.3899222599836\\
57.875	0.25038	10.949319237502\\
57.875	0.25404	11.5237134545713\\
57.875	0.2577	12.1131049111916\\
57.875	0.26136	12.7174936073628\\
57.875	0.26502	13.336879543085\\
57.875	0.26868	13.9712627183581\\
57.875	0.27234	14.6206431331823\\
57.875	0.276	15.2850207875574\\
58.25	0.093	0.436870501043999\\
58.25	0.09666	0.360344184531701\\
58.25	0.10032	0.298815107570356\\
58.25	0.10398	0.252283270159981\\
58.25	0.10764	0.220748672300573\\
58.25	0.1113	0.204211313992125\\
58.25	0.11496	0.202671195234648\\
58.25	0.11862	0.216128316028137\\
58.25	0.12228	0.244582676372593\\
58.25	0.12594	0.288034276268014\\
58.25	0.1296	0.346483115714399\\
58.25	0.13326	0.419929194711752\\
58.25	0.13692	0.508372513260063\\
58.25	0.14058	0.61181307135935\\
58.25	0.14424	0.730250869009602\\
58.25	0.1479	0.863685906210817\\
58.25	0.15156	1.012118182963\\
58.25	0.15522	1.17554769926614\\
58.25	0.15888	1.35397445512026\\
58.25	0.16254	1.54739845052534\\
58.25	0.1662	1.75581968548137\\
58.25	0.16986	1.97923815998838\\
58.25	0.17352	2.21765387404636\\
58.25	0.17718	2.4710668276553\\
58.25	0.18084	2.73947702081521\\
58.25	0.1845	3.02288445352608\\
58.25	0.18816	3.32128912578792\\
58.25	0.19182	3.63469103760072\\
58.25	0.19548	3.9630901889645\\
58.25	0.19914	4.30648657987923\\
58.25	0.2028	4.66488021034493\\
58.25	0.20646	5.0382710803616\\
58.25	0.21012	5.42665918992924\\
58.25	0.21378	5.83004453904784\\
58.25	0.21744	6.2484271277174\\
58.25	0.2211	6.68180695593793\\
58.25	0.22476	7.13018402370943\\
58.25	0.22842	7.59355833103189\\
58.25	0.23208	8.07192987790532\\
58.25	0.23574	8.56529866432971\\
58.25	0.2394	9.07366469030507\\
58.25	0.24306	9.5970279558314\\
58.25	0.24672	10.1353884609087\\
58.25	0.25038	10.688746205537\\
58.25	0.25404	11.2571011897162\\
58.25	0.2577	11.8404534134464\\
58.25	0.26136	12.4388028767275\\
58.25	0.26502	13.0521495795596\\
58.25	0.26868	13.6804935219427\\
58.25	0.27234	14.3238347038768\\
58.25	0.276	14.9821731253618\\
58.625	0.093	0.438711340924507\\
58.625	0.09666	0.35614579152212\\
58.625	0.10032	0.2885774816707\\
58.625	0.10398	0.23600641137025\\
58.625	0.10764	0.198432580620761\\
58.625	0.1113	0.175855989422238\\
58.625	0.11496	0.168276637774687\\
58.625	0.11862	0.175694525678094\\
58.625	0.12228	0.198109653132475\\
58.625	0.12594	0.235522020137822\\
58.625	0.1296	0.287931626694125\\
58.625	0.13326	0.355338472801396\\
58.625	0.13692	0.437742558459639\\
58.625	0.14058	0.535143883668844\\
58.625	0.14424	0.647542448429022\\
58.625	0.1479	0.774938252740155\\
58.625	0.15156	0.917331296602255\\
58.625	0.15522	1.07472158001533\\
58.625	0.15888	1.24710910297936\\
58.625	0.16254	1.43449386549435\\
58.625	0.1662	1.63687586756032\\
58.625	0.16986	1.85425510917725\\
58.625	0.17352	2.08663159034515\\
58.625	0.17718	2.33400531106401\\
58.625	0.18084	2.59637627133384\\
58.625	0.1845	2.87374447115464\\
58.625	0.18816	3.1661099105264\\
58.625	0.19182	3.47347258944913\\
58.625	0.19548	3.79583250792282\\
58.625	0.19914	4.13318966594748\\
58.625	0.2028	4.4855440635231\\
58.625	0.20646	4.8528957006497\\
58.625	0.21012	5.23524457732725\\
58.625	0.21378	5.63259069355577\\
58.625	0.21744	6.04493404933526\\
58.625	0.2211	6.47227464466571\\
58.625	0.22476	6.91461247954713\\
58.625	0.22842	7.37194755397952\\
58.625	0.23208	7.84427986796286\\
58.625	0.23574	8.33160942149719\\
58.625	0.2394	8.83393621458247\\
58.625	0.24306	9.35126024721872\\
58.625	0.24672	9.88358151940594\\
58.625	0.25038	10.4309000311441\\
58.625	0.25404	10.9932157824333\\
58.625	0.2577	11.5705287732734\\
58.625	0.26136	12.1628390036644\\
58.625	0.26502	12.7701464736065\\
58.625	0.26868	13.3924511830995\\
58.625	0.27234	14.0297531321435\\
58.625	0.276	14.6820523207384\\
59	0.093	0.443279038377207\\
59	0.09666	0.354674256084745\\
59	0.10032	0.28106671334325\\
59	0.10398	0.222456410152719\\
59	0.10764	0.178843346513155\\
59	0.1113	0.150227522424557\\
59	0.11496	0.136608937886924\\
59	0.11862	0.137987592900257\\
59	0.12228	0.154363487464556\\
59	0.12594	0.185736621579821\\
59	0.1296	0.232106995246049\\
59	0.13326	0.293474608463246\\
59	0.13692	0.369839461231408\\
59	0.14058	0.461201553550531\\
59	0.14424	0.567560885420633\\
59	0.1479	0.688917456841692\\
59	0.15156	0.825271267813717\\
59	0.15522	0.976622318336707\\
59	0.15888	1.14297060841066\\
59	0.16254	1.32431613803559\\
59	0.1662	1.52065890721147\\
59	0.16986	1.73199891593833\\
59	0.17352	1.95833616421615\\
59	0.17718	2.19967065204493\\
59	0.18084	2.45600237942469\\
59	0.1845	2.7273313463554\\
59	0.18816	3.01365755283709\\
59	0.19182	3.31498099886974\\
59	0.19548	3.63130168445335\\
59	0.19914	3.96261960958794\\
59	0.2028	4.30893477427347\\
59	0.20646	4.67024717851\\
59	0.21012	5.04655682229746\\
59	0.21378	5.43786370563592\\
59	0.21744	5.84416782852533\\
59	0.2211	6.26546919096569\\
59	0.22476	6.70176779295704\\
59	0.22842	7.15306363449934\\
59	0.23208	7.61935671559262\\
59	0.23574	8.10064703623687\\
59	0.2394	8.59693459643206\\
59	0.24306	9.10821939617824\\
59	0.24672	9.63450143547537\\
59	0.25038	10.1757807143235\\
59	0.25404	10.7320572327225\\
59	0.2577	11.3033309906726\\
59	0.26136	11.8896019881736\\
59	0.26502	12.4908702252255\\
59	0.26868	13.1071357018285\\
59	0.27234	13.7383984179824\\
59	0.276	14.3846583736872\\
59.375	0.093	0.450573593402127\\
59.375	0.09666	0.35592957821959\\
59.375	0.10032	0.276282802588014\\
59.375	0.10398	0.211633266507408\\
59.375	0.10764	0.161980969977762\\
59.375	0.1113	0.127325912999082\\
59.375	0.11496	0.107668095571375\\
59.375	0.11862	0.103007517694633\\
59.375	0.12228	0.113344179368857\\
59.375	0.12594	0.13867808059404\\
59.375	0.1296	0.179009221370194\\
59.375	0.13326	0.234337601697316\\
59.375	0.13692	0.304663221575396\\
59.375	0.14058	0.389986081004444\\
59.375	0.14424	0.490306179984465\\
59.375	0.1479	0.605623518515449\\
59.375	0.15156	0.735938096597399\\
59.375	0.15522	0.881249914230308\\
59.375	0.15888	1.04155897141419\\
59.375	0.16254	1.21686526814904\\
59.375	0.1662	1.40716880443485\\
59.375	0.16986	1.61246958027162\\
59.375	0.17352	1.83276759565936\\
59.375	0.17718	2.06806285059807\\
59.375	0.18084	2.31835534508775\\
59.375	0.1845	2.58364507912839\\
59.375	0.18816	2.86393205272\\
59.375	0.19182	3.15921626586256\\
59.375	0.19548	3.4694977185561\\
59.375	0.19914	3.79477641080061\\
59.375	0.2028	4.13505234259608\\
59.375	0.20646	4.49032551394252\\
59.375	0.21012	4.86059592483991\\
59.375	0.21378	5.24586357528828\\
59.375	0.21744	5.64612846528761\\
59.375	0.2211	6.06139059483791\\
59.375	0.22476	6.49164996393917\\
59.375	0.22842	6.9369065725914\\
59.375	0.23208	7.3971604207946\\
59.375	0.23574	7.87241150854877\\
59.375	0.2394	8.3626598358539\\
59.375	0.24306	8.86790540270999\\
59.375	0.24672	9.38814820911706\\
59.375	0.25038	9.92338825507507\\
59.375	0.25404	10.473625540584\\
59.375	0.2577	11.038860065644\\
59.375	0.26136	11.6190918302549\\
59.375	0.26502	12.2143208344168\\
59.375	0.26868	12.8245470781297\\
59.375	0.27234	13.4497705613935\\
59.375	0.276	14.0899912842083\\
59.75	0.093	0.460595005999246\\
59.75	0.09666	0.359911757926628\\
59.75	0.10032	0.274225749404977\\
59.75	0.10398	0.203536980434289\\
59.75	0.10764	0.147845451014575\\
59.75	0.1113	0.107151161145821\\
59.75	0.11496	0.081454110828032\\
59.75	0.11862	0.0707543000612079\\
59.75	0.12228	0.0750517288453505\\
59.75	0.12594	0.0943463971804661\\
59.75	0.1296	0.128638305066538\\
59.75	0.13326	0.177927452503578\\
59.75	0.13692	0.242213839491583\\
59.75	0.14058	0.321497466030557\\
59.75	0.14424	0.415778332120503\\
59.75	0.1479	0.525056437761405\\
59.75	0.15156	0.649331782953274\\
59.75	0.15522	0.788604367696108\\
59.75	0.15888	0.942874191989914\\
59.75	0.16254	1.11214125583469\\
59.75	0.1662	1.29640555923042\\
59.75	0.16986	1.49566710217712\\
59.75	0.17352	1.70992588467477\\
59.75	0.17718	1.93918190672341\\
59.75	0.18084	2.18343516832301\\
59.75	0.1845	2.44268566947357\\
59.75	0.18816	2.7169334101751\\
59.75	0.19182	3.00617839042759\\
59.75	0.19548	3.31042061023105\\
59.75	0.19914	3.62966006958548\\
59.75	0.2028	3.96389676849087\\
59.75	0.20646	4.31313070694723\\
59.75	0.21012	4.67736188495455\\
59.75	0.21378	5.05659030251284\\
59.75	0.21744	5.4508159596221\\
59.75	0.2211	5.86003885628231\\
59.75	0.22476	6.28425899249351\\
59.75	0.22842	6.72347636825565\\
59.75	0.23208	7.17769098356877\\
59.75	0.23574	7.64690283843286\\
59.75	0.2394	8.13111193284791\\
59.75	0.24306	8.63031826681393\\
59.75	0.24672	9.1445218403309\\
59.75	0.25038	9.67372265339886\\
59.75	0.25404	10.2179207060178\\
59.75	0.2577	10.7771159981876\\
59.75	0.26136	11.3513085299085\\
59.75	0.26502	11.9404983011803\\
59.75	0.26868	12.5446853120031\\
59.75	0.27234	13.1638695623768\\
59.75	0.276	13.7980510523015\\
60.125	0.093	0.473343276168564\\
60.125	0.09666	0.366620795205871\\
60.125	0.10032	0.274895553794138\\
60.125	0.10398	0.198167551933376\\
60.125	0.10764	0.136436789623581\\
60.125	0.1113	0.0897032668647517\\
60.125	0.11496	0.0579669836568879\\
60.125	0.11862	0.041227939999982\\
60.125	0.12228	0.0394861358940499\\
60.125	0.12594	0.0527415713390909\\
60.125	0.1296	0.0809942463350808\\
60.125	0.13326	0.124244160882039\\
60.125	0.13692	0.182491314979977\\
60.125	0.14058	0.255735708628869\\
60.125	0.14424	0.343977341828726\\
60.125	0.1479	0.44721621457956\\
60.125	0.15156	0.565452326881347\\
60.125	0.15522	0.698685678734114\\
60.125	0.15888	0.846916270137838\\
60.125	0.16254	1.01014410109252\\
60.125	0.1662	1.18836917159819\\
60.125	0.16986	1.3815914816548\\
60.125	0.17352	1.5898110312624\\
60.125	0.17718	1.81302782042095\\
60.125	0.18084	2.05124184913046\\
60.125	0.1845	2.30445311739095\\
60.125	0.18816	2.5726616252024\\
60.125	0.19182	2.85586737256482\\
60.125	0.19548	3.1540703594782\\
60.125	0.19914	3.46727058594254\\
60.125	0.2028	3.79546805195787\\
60.125	0.20646	4.13866275752414\\
60.125	0.21012	4.4968547026414\\
60.125	0.21378	4.8700438873096\\
60.125	0.21744	5.25823031152877\\
60.125	0.2211	5.66141397529892\\
60.125	0.22476	6.07959487862003\\
60.125	0.22842	6.51277302149211\\
60.125	0.23208	6.96094840391514\\
60.125	0.23574	7.42412102588915\\
60.125	0.2394	7.90229088741413\\
60.125	0.24306	8.39545798849007\\
60.125	0.24672	8.90362232911698\\
60.125	0.25038	9.42678390929485\\
60.125	0.25404	9.96494272902367\\
60.125	0.2577	10.5180987883035\\
60.125	0.26136	11.0862520871342\\
60.125	0.26502	11.669402625516\\
60.125	0.26868	12.2675504034487\\
60.125	0.27234	12.8806954209323\\
60.125	0.276	13.508837677967\\
60.5	0.093	0.488818403910096\\
60.5	0.09666	0.376056690057328\\
60.5	0.10032	0.278292215755521\\
60.5	0.10398	0.195524981004676\\
60.5	0.10764	0.127754985804806\\
60.5	0.1113	0.0749822301558956\\
60.5	0.11496	0.03720671405795\\
60.5	0.11862	0.0144284375109764\\
60.5	0.12228	0.0066474005149626\\
60.5	0.12594	0.0138636030699146\\
60.5	0.1296	0.036077045175837\\
60.5	0.13326	0.0732877268327279\\
60.5	0.13692	0.125495648040577\\
60.5	0.14058	0.192700808799394\\
60.5	0.14424	0.274903209109176\\
60.5	0.1479	0.372102848969929\\
60.5	0.15156	0.484299728381648\\
60.5	0.15522	0.611493847344326\\
60.5	0.15888	0.753685205857975\\
60.5	0.16254	0.910873803922584\\
60.5	0.1662	1.08305964153817\\
60.5	0.16986	1.27024271870471\\
60.5	0.17352	1.47242303542222\\
60.5	0.17718	1.6896005916907\\
60.5	0.18084	1.92177538751013\\
60.5	0.1845	2.16894742288055\\
60.5	0.18816	2.43111669780192\\
60.5	0.19182	2.70828321227426\\
60.5	0.19548	3.00044696629756\\
60.5	0.19914	3.30760795987183\\
60.5	0.2028	3.62976619299707\\
60.5	0.20646	3.96692166567328\\
60.5	0.21012	4.31907437790045\\
60.5	0.21378	4.68622432967858\\
60.5	0.21744	5.06837152100767\\
60.5	0.2211	5.46551595188773\\
60.5	0.22476	5.87765762231877\\
60.5	0.22842	6.30479653230077\\
60.5	0.23208	6.74693268183373\\
60.5	0.23574	7.20406607091767\\
60.5	0.2394	7.67619669955256\\
60.5	0.24306	8.16332456773842\\
60.5	0.24672	8.66544967547524\\
60.5	0.25038	9.18257202276303\\
60.5	0.25404	9.71469160960179\\
60.5	0.2577	10.2618084359915\\
60.5	0.26136	10.8239225019322\\
60.5	0.26502	11.4010338074239\\
60.5	0.26868	11.9931423524665\\
60.5	0.27234	12.6002481370601\\
60.5	0.276	13.2223511612046\\
60.875	0.093	0.507020389223812\\
60.875	0.09666	0.38821944248097\\
60.875	0.10032	0.28441573528908\\
60.875	0.10398	0.195609267648162\\
60.875	0.10764	0.12180003955821\\
60.875	0.1113	0.0629880510192242\\
60.875	0.11496	0.0191733020312039\\
60.875	0.11862	-0.00964420740585137\\
60.875	0.12228	-0.0234644772919399\\
60.875	0.12594	-0.0222875076270697\\
60.875	0.1296	-0.00611329841122199\\
60.875	0.13326	0.0250581503555942\\
60.875	0.13692	0.0712268386733612\\
60.875	0.14058	0.132392766542104\\
60.875	0.14424	0.208555933961804\\
60.875	0.1479	0.299716340932482\\
60.875	0.15156	0.405873987454127\\
60.875	0.15522	0.527028873526723\\
60.875	0.15888	0.663180999150297\\
60.875	0.16254	0.814330364324825\\
60.875	0.1662	0.980476969050333\\
60.875	0.16986	1.16162081332679\\
60.875	0.17352	1.35776189715422\\
60.875	0.17718	1.56890022053263\\
60.875	0.18084	1.79503578346198\\
60.875	0.1845	2.03616858594232\\
60.875	0.18816	2.29229862797362\\
60.875	0.19182	2.56342590955587\\
60.875	0.19548	2.8495504306891\\
60.875	0.19914	3.1506721913733\\
60.875	0.2028	3.46679119160846\\
60.875	0.20646	3.79790743139458\\
60.875	0.21012	4.14402091073167\\
60.875	0.21378	4.50513162961973\\
60.875	0.21744	4.88123958805875\\
60.875	0.2211	5.27234478604874\\
60.875	0.22476	5.67844722358969\\
60.875	0.22842	6.09954690068161\\
60.875	0.23208	6.53564381732449\\
60.875	0.23574	6.98673797351836\\
60.875	0.2394	7.45282936926317\\
60.875	0.24306	7.93391800455896\\
60.875	0.24672	8.4300038794057\\
60.875	0.25038	8.94108699380342\\
60.875	0.25404	9.4671673477521\\
60.875	0.2577	10.0082449412517\\
60.875	0.26136	10.5643197743023\\
60.875	0.26502	11.1353918469039\\
60.875	0.26868	11.7214611590565\\
60.875	0.27234	12.32252771076\\
60.875	0.276	12.9385915020145\\
61.25	0.093	0.527949232109749\\
61.25	0.09666	0.403109052476824\\
61.25	0.10032	0.293266112394861\\
61.25	0.10398	0.198420411863867\\
61.25	0.10764	0.11857195088384\\
61.25	0.1113	0.0537207294547732\\
61.25	0.11496	0.00386674757667826\\
61.25	0.11862	-0.0309899947504588\\
61.25	0.12228	-0.050849497526622\\
61.25	0.12594	-0.0557117607518265\\
61.25	0.1296	-0.0455767844260606\\
61.25	0.13326	-0.0204445685493262\\
61.25	0.13692	0.0196848868783661\\
61.25	0.14058	0.0748115818570341\\
61.25	0.14424	0.14493551638666\\
61.25	0.1479	0.230056690467256\\
61.25	0.15156	0.330175104098819\\
61.25	0.15522	0.44529075728134\\
61.25	0.15888	0.575403650014833\\
61.25	0.16254	0.720513782299292\\
61.25	0.1662	0.880621154134719\\
61.25	0.16986	1.0557257655211\\
61.25	0.17352	1.24582761645845\\
61.25	0.17718	1.45092670694678\\
61.25	0.18084	1.67102303698606\\
61.25	0.1845	1.90611660657632\\
61.25	0.18816	2.15620741571754\\
61.25	0.19182	2.42129546440972\\
61.25	0.19548	2.70138075265287\\
61.25	0.19914	2.99646328044699\\
61.25	0.2028	3.30654304779208\\
61.25	0.20646	3.63162005468812\\
61.25	0.21012	3.97169430113513\\
61.25	0.21378	4.32676578713311\\
61.25	0.21744	4.69683451268206\\
61.25	0.2211	5.08190047778195\\
61.25	0.22476	5.48196368243284\\
61.25	0.22842	5.89702412663468\\
61.25	0.23208	6.32708181038749\\
61.25	0.23574	6.77213673369127\\
61.25	0.2394	7.232188896546\\
61.25	0.24306	7.70723829895171\\
61.25	0.24672	8.19728494090838\\
61.25	0.25038	8.70232882241603\\
61.25	0.25404	9.22236994347461\\
61.25	0.2577	9.75740830408419\\
61.25	0.26136	10.3074439042447\\
61.25	0.26502	10.8724767439562\\
61.25	0.26868	11.4525068232187\\
61.25	0.27234	12.0475341420321\\
61.25	0.276	12.6575587003965\\
61.625	0.093	0.551604932567899\\
61.625	0.09666	0.420725520044885\\
61.625	0.10032	0.304843347072854\\
61.625	0.10398	0.203958413651779\\
61.625	0.10764	0.11807071978167\\
61.625	0.1113	0.0471802654625284\\
61.625	0.11496	-0.00871294930564837\\
61.625	0.11862	-0.049608924522853\\
61.625	0.12228	-0.075507660189098\\
61.625	0.12594	-0.0864091563043772\\
61.625	0.1296	-0.0823134128686931\\
61.625	0.13326	-0.0632204298820405\\
61.625	0.13692	-0.0291302073444157\\
61.625	0.14058	0.0199572547441633\\
61.625	0.14424	0.0840419563837216\\
61.625	0.1479	0.163123897574236\\
61.625	0.15156	0.257203078315717\\
61.625	0.15522	0.36627949860817\\
61.625	0.15888	0.490353158451581\\
61.625	0.16254	0.629424057845966\\
61.625	0.1662	0.783492196791311\\
61.625	0.16986	0.952557575287621\\
61.625	0.17352	1.13662019333489\\
61.625	0.17718	1.33568005093314\\
61.625	0.18084	1.54973714808235\\
61.625	0.1845	1.77879148478253\\
61.625	0.18816	2.02284306103366\\
61.625	0.19182	2.28189187683576\\
61.625	0.19548	2.55593793218884\\
61.625	0.19914	2.84498122709289\\
61.625	0.2028	3.14902176154789\\
61.625	0.20646	3.46805953555386\\
61.625	0.21012	3.80209454911079\\
61.625	0.21378	4.15112680221869\\
61.625	0.21744	4.51515629487757\\
61.625	0.2211	4.89418302708739\\
61.625	0.22476	5.28820699884819\\
61.625	0.22842	5.69722821015995\\
61.625	0.23208	6.12124666102268\\
61.625	0.23574	6.56026235143638\\
61.625	0.2394	7.01427528140105\\
61.625	0.24306	7.48328545091668\\
61.625	0.24672	7.96729285998327\\
61.625	0.25038	8.46629750860082\\
61.625	0.25404	8.98029939676935\\
61.625	0.2577	9.50929852448885\\
61.625	0.26136	10.0532948917593\\
61.625	0.26502	10.6122884985807\\
61.625	0.26868	11.1862793449531\\
61.625	0.27234	11.7752674308765\\
61.625	0.276	12.3792527563508\\
62	0.093	0.577987490598233\\
62	0.09666	0.44106884518516\\
62	0.10032	0.319147439323039\\
62	0.10398	0.212223273011889\\
62	0.10764	0.120296346251706\\
62	0.1113	0.0433666590424826\\
62	0.11496	-0.0185657886157689\\
62	0.11862	-0.0655009967230553\\
62	0.12228	-0.0974389652793821\\
62	0.12594	-0.114379694284729\\
62	0.1296	-0.116323183739127\\
62	0.13326	-0.103269433642541\\
62	0.13692	-0.0752184439950057\\
62	0.14058	-0.0321702147965013\\
62	0.14424	0.0258752539529823\\
62	0.1479	0.0989179622534149\\
62	0.15156	0.186957910104814\\
62	0.15522	0.289995097507193\\
62	0.15888	0.408029524460529\\
62	0.16254	0.541061190964825\\
62	0.1662	0.689090097020102\\
62	0.16986	0.85211624262633\\
62	0.17352	1.03013962778354\\
62	0.17718	1.2231602524917\\
62	0.18084	1.43117811675082\\
62	0.1845	1.65419322056093\\
62	0.18816	1.89220556392199\\
62	0.19182	2.14521514683401\\
62	0.19548	2.41322196929702\\
62	0.19914	2.69622603131097\\
62	0.2028	2.99422733287589\\
62	0.20646	3.3072258739918\\
62	0.21012	3.63522165465865\\
62	0.21378	3.97821467487648\\
62	0.21744	4.33620493464526\\
62	0.2211	4.70919243396501\\
62	0.22476	5.09717717283574\\
62	0.22842	5.50015915125742\\
62	0.23208	5.91813836923007\\
62	0.23574	6.3511148267537\\
62	0.2394	6.79908852382828\\
62	0.24306	7.26205946045384\\
62	0.24672	7.74002763663035\\
62	0.25038	8.23299305235785\\
62	0.25404	8.74095570763627\\
62	0.2577	9.26391560246569\\
62	0.26136	9.80187273684608\\
62	0.26502	10.3548271107774\\
62	0.26868	10.9227787242597\\
62	0.27234	11.505727577293\\
62	0.276	12.1036736698772\\
62.375	0.093	0.607096906200788\\
62.375	0.09666	0.464139027897626\\
62.375	0.10032	0.336178389145431\\
62.375	0.10398	0.223214989944206\\
62.375	0.10764	0.125248830293941\\
62.375	0.1113	0.0422799101946429\\
62.375	0.11496	-0.0256917703536832\\
62.375	0.11862	-0.0786662113510515\\
62.375	0.12228	-0.116643412797453\\
62.375	0.12594	-0.139623374692889\\
62.375	0.1296	-0.147606097037354\\
62.375	0.13326	-0.140591579830851\\
62.375	0.13692	-0.118579823073389\\
62.375	0.14058	-0.0815708267649597\\
62.375	0.14424	-0.0295645909055651\\
62.375	0.1479	0.0374388845047999\\
62.375	0.15156	0.119439599466132\\
62.375	0.15522	0.216437553978421\\
62.375	0.15888	0.328432748041683\\
62.375	0.16254	0.455425181655905\\
62.375	0.1662	0.5974148548211\\
62.375	0.16986	0.754401767537253\\
62.375	0.17352	0.926385919804378\\
62.375	0.17718	1.11336731162246\\
62.375	0.18084	1.31534594299151\\
62.375	0.1845	1.53232181391154\\
62.375	0.18816	1.76429492438253\\
62.375	0.19182	2.01126527440448\\
62.375	0.19548	2.27323286397739\\
62.375	0.19914	2.55019769310127\\
62.375	0.2028	2.84215976177612\\
62.375	0.20646	3.14911907000194\\
62.375	0.21012	3.47107561777871\\
62.375	0.21378	3.80802940510646\\
62.375	0.21744	4.15998043198518\\
62.375	0.2211	4.52692869841485\\
62.375	0.22476	4.9088742043955\\
62.375	0.22842	5.3058169499271\\
62.375	0.23208	5.71775693500968\\
62.375	0.23574	6.14469415964323\\
62.375	0.2394	6.58662862382774\\
62.375	0.24306	7.0435603275632\\
62.375	0.24672	7.51548927084964\\
62.375	0.25038	8.00241545368705\\
62.375	0.25404	8.50433887607541\\
62.375	0.2577	9.02125953801476\\
62.375	0.26136	9.55317743950504\\
62.375	0.26502	10.1000925805463\\
62.375	0.26868	10.6620049611385\\
62.375	0.27234	11.2389145812818\\
62.375	0.276	11.8308214409759\\
62.75	0.093	0.638933179375528\\
62.75	0.09666	0.489936068182291\\
62.75	0.10032	0.355936196540021\\
62.75	0.10398	0.236933564448715\\
62.75	0.10764	0.132928171908375\\
62.75	0.1113	0.0439200189189952\\
62.75	0.11496	-0.0300908945194127\\
62.75	0.11862	-0.0891045684068485\\
62.75	0.12228	-0.133121002743332\\
62.75	0.12594	-0.162140197528835\\
62.75	0.1296	-0.176162152763389\\
62.75	0.13326	-0.175186868446961\\
62.75	0.13692	-0.159214344579581\\
62.75	0.14058	-0.128244581161226\\
62.75	0.14424	-0.0822775781918992\\
62.75	0.1479	-0.0213133356716231\\
62.75	0.15156	0.054648146399634\\
62.75	0.15522	0.145606868021842\\
62.75	0.15888	0.251562829195029\\
62.75	0.16254	0.372516029919169\\
62.75	0.1662	0.508466470194289\\
62.75	0.16986	0.659414150020361\\
62.75	0.17352	0.825359069397404\\
62.75	0.17718	1.00630122832542\\
62.75	0.18084	1.20224062680439\\
62.75	0.1845	1.41317726483434\\
62.75	0.18816	1.63911114241525\\
62.75	0.19182	1.88004225954712\\
62.75	0.19548	2.13597061622996\\
62.75	0.19914	2.40689621246376\\
62.75	0.2028	2.69281904824854\\
62.75	0.20646	2.99373912358428\\
62.75	0.21012	3.30965643847098\\
62.75	0.21378	3.64057099290865\\
62.75	0.21744	3.98648278689728\\
62.75	0.2211	4.34739182043688\\
62.75	0.22476	4.72329809352745\\
62.75	0.22842	5.11420160616898\\
62.75	0.23208	5.52010235836147\\
62.75	0.23574	5.94100035010495\\
62.75	0.2394	6.37689558139937\\
62.75	0.24306	6.82778805224476\\
62.75	0.24672	7.29367776264113\\
62.75	0.25038	7.77456471258845\\
62.75	0.25404	8.27044890208675\\
62.75	0.2577	8.78133033113601\\
62.75	0.26136	9.30720899973622\\
62.75	0.26502	9.84808490788741\\
62.75	0.26868	10.4039580555896\\
62.75	0.27234	10.9748284428427\\
62.75	0.276	11.5606960696468\\
63.125	0.093	0.673496310122489\\
63.125	0.09666	0.51845996603917\\
63.125	0.10032	0.378420861506825\\
63.125	0.10398	0.253378996525444\\
63.125	0.10764	0.143334371095023\\
63.125	0.1113	0.0482869852155678\\
63.125	0.11496	-0.0317631611129148\\
63.125	0.11862	-0.0968160678904324\\
63.125	0.12228	-0.14687173511699\\
63.125	0.12594	-0.181930162792582\\
63.125	0.1296	-0.201991350917204\\
63.125	0.13326	-0.207055299490857\\
63.125	0.13692	-0.197122008513546\\
63.125	0.14058	-0.172191477985272\\
63.125	0.14424	-0.132263707906034\\
63.125	0.1479	-0.0773386982758257\\
63.125	0.15156	-0.00741644909465045\\
63.125	0.15522	0.0775030396374827\\
63.125	0.15888	0.177419767920595\\
63.125	0.16254	0.29233373575466\\
63.125	0.1662	0.422244943139699\\
63.125	0.16986	0.567153390075696\\
63.125	0.17352	0.727059076562657\\
63.125	0.17718	0.901962002600595\\
63.125	0.18084	1.09186216818949\\
63.125	0.1845	1.29675957332936\\
63.125	0.18816	1.51665421802019\\
63.125	0.19182	1.75154610226198\\
63.125	0.19548	2.00143522605475\\
63.125	0.19914	2.26632158939848\\
63.125	0.2028	2.54620519229316\\
63.125	0.20646	2.84108603473883\\
63.125	0.21012	3.15096411673546\\
63.125	0.21378	3.47583943828305\\
63.125	0.21744	3.81571199938161\\
63.125	0.2211	4.17058180003112\\
63.125	0.22476	4.54044884023161\\
63.125	0.22842	4.92531311998307\\
63.125	0.23208	5.32517463928548\\
63.125	0.23574	5.74003339813887\\
63.125	0.2394	6.16988939654323\\
63.125	0.24306	6.61474263449855\\
63.125	0.24672	7.07459311200483\\
63.125	0.25038	7.54944082906209\\
63.125	0.25404	8.03928578567029\\
63.125	0.2577	8.54412798182948\\
63.125	0.26136	9.06396741753962\\
63.125	0.26502	9.59880409280074\\
63.125	0.26868	10.1486380076128\\
63.125	0.27234	10.7134691619759\\
63.125	0.276	11.2932975558899\\
63.5	0.093	0.710786298441648\\
63.5	0.09666	0.549710721468261\\
63.5	0.10032	0.403632384045828\\
63.5	0.10398	0.272551286174372\\
63.5	0.10764	0.156467427853876\\
63.5	0.1113	0.0553808090843395\\
63.5	0.11496	-0.0307085701342249\\
63.5	0.11862	-0.101800709801817\\
63.5	0.12228	-0.15789560991845\\
63.5	0.12594	-0.198993270484117\\
63.5	0.1296	-0.22509369149882\\
63.5	0.13326	-0.236196872962555\\
63.5	0.13692	-0.232302814875318\\
63.5	0.14058	-0.21341151723712\\
63.5	0.14424	-0.179522980047956\\
63.5	0.1479	-0.130637203307829\\
63.5	0.15156	-0.0667541870167359\\
63.5	0.15522	0.0121260688253297\\
63.5	0.15888	0.106003564218353\\
63.5	0.16254	0.214878299162351\\
63.5	0.1662	0.338750273657308\\
63.5	0.16986	0.47761948770323\\
63.5	0.17352	0.631485941300109\\
63.5	0.17718	0.800349634447972\\
63.5	0.18084	0.984210567146796\\
63.5	0.1845	1.18306873939659\\
63.5	0.18816	1.39692415119733\\
63.5	0.19182	1.62577680254905\\
63.5	0.19548	1.86962669345174\\
63.5	0.19914	2.1284738239054\\
63.5	0.2028	2.40231819391001\\
63.5	0.20646	2.69115980346559\\
63.5	0.21012	2.99499865257213\\
63.5	0.21378	3.31383474122964\\
63.5	0.21744	3.64766806943812\\
63.5	0.2211	3.99649863719757\\
63.5	0.22476	4.36032644450798\\
63.5	0.22842	4.73915149136936\\
63.5	0.23208	5.1329737777817\\
63.5	0.23574	5.54179330374501\\
63.5	0.2394	5.96561006925929\\
63.5	0.24306	6.40442407432454\\
63.5	0.24672	6.85823531894074\\
63.5	0.25038	7.32704380310791\\
63.5	0.25404	7.81084952682604\\
63.5	0.2577	8.30965249009515\\
63.5	0.26136	8.82345269291521\\
63.5	0.26502	9.35225013528624\\
63.5	0.26868	9.89604481720824\\
63.5	0.27234	10.4548367386812\\
63.5	0.276	11.0286258997051\\
63.875	0.093	0.750803144333013\\
63.875	0.09666	0.583688334469545\\
63.875	0.10032	0.431570764157037\\
63.875	0.10398	0.294450433395506\\
63.875	0.10764	0.172327342184929\\
63.875	0.1113	0.0652014905253244\\
63.875	0.11496	-0.0269271215833218\\
63.875	0.11862	-0.104058494140989\\
63.875	0.12228	-0.166192627147703\\
63.875	0.12594	-0.213329520603445\\
63.875	0.1296	-0.245469174508223\\
63.875	0.13326	-0.262611588862033\\
63.875	0.13692	-0.264756763664884\\
63.875	0.14058	-0.251904698916761\\
63.875	0.14424	-0.224055394617672\\
63.875	0.1479	-0.181208850767627\\
63.875	0.15156	-0.123365067366601\\
63.875	0.15522	-0.0505240444146242\\
63.875	0.15888	0.0373142180883246\\
63.875	0.16254	0.140149720142247\\
63.875	0.1662	0.257982461747122\\
63.875	0.16986	0.390812442902963\\
63.875	0.17352	0.538639663609782\\
63.875	0.17718	0.701464123867556\\
63.875	0.18084	0.879285823676291\\
63.875	0.1845	1.07210476303601\\
63.875	0.18816	1.27992094194668\\
63.875	0.19182	1.50273436040832\\
63.875	0.19548	1.74054501842094\\
63.875	0.19914	1.9933529159845\\
63.875	0.2028	2.26115805309904\\
63.875	0.20646	2.54396042976455\\
63.875	0.21012	2.84176004598102\\
63.875	0.21378	3.15455690174845\\
63.875	0.21744	3.48235099706686\\
63.875	0.2211	3.82514233193622\\
63.875	0.22476	4.18293090635655\\
63.875	0.22842	4.55571672032785\\
63.875	0.23208	4.94349977385012\\
63.875	0.23574	5.34628006692336\\
63.875	0.2394	5.76405759954755\\
63.875	0.24306	6.19683237172271\\
63.875	0.24672	6.64460438344884\\
63.875	0.25038	7.10737363472594\\
63.875	0.25404	7.58514012555398\\
63.875	0.2577	8.07790385593302\\
63.875	0.26136	8.58566482586301\\
63.875	0.26502	9.10842303534398\\
63.875	0.26868	9.64617848437589\\
63.875	0.27234	10.1989311729588\\
63.875	0.276	10.7666811010926\\
64.25	0.093	0.793546847796585\\
64.25	0.09666	0.620392805043035\\
64.25	0.10032	0.462236001840452\\
64.25	0.10398	0.319076438188847\\
64.25	0.10764	0.190914114088194\\
64.25	0.1113	0.0777490295385084\\
64.25	0.11496	-0.0204188154602125\\
64.25	0.11862	-0.103589420907961\\
64.25	0.12228	-0.17176278680475\\
64.25	0.12594	-0.224938913150567\\
64.25	0.1296	-0.263117799945427\\
64.25	0.13326	-0.286299447189318\\
64.25	0.13692	-0.294483854882237\\
64.25	0.14058	-0.287671023024203\\
64.25	0.14424	-0.265860951615196\\
64.25	0.1479	-0.229053640655218\\
64.25	0.15156	-0.177249090144274\\
64.25	0.15522	-0.110447300082372\\
64.25	0.15888	-0.0286482704694979\\
64.25	0.16254	0.0681479986943359\\
64.25	0.1662	0.179941507409143\\
64.25	0.16986	0.306732255674909\\
64.25	0.17352	0.448520243491647\\
64.25	0.17718	0.605305470859346\\
64.25	0.18084	0.777087937778006\\
64.25	0.1845	0.963867644247646\\
64.25	0.18816	1.16564459026824\\
64.25	0.19182	1.3824187758398\\
64.25	0.19548	1.61419020096233\\
64.25	0.19914	1.86095886563583\\
64.25	0.2028	2.12272476986029\\
64.25	0.20646	2.39948791363572\\
64.25	0.21012	2.6912482969621\\
64.25	0.21378	2.99800591983947\\
64.25	0.21744	3.3197607822678\\
64.25	0.2211	3.65651288424708\\
64.25	0.22476	4.00826222577734\\
64.25	0.22842	4.37500880685856\\
64.25	0.23208	4.75675262749074\\
64.25	0.23574	5.1534936876739\\
64.25	0.2394	5.56523198740803\\
64.25	0.24306	5.99196752669311\\
64.25	0.24672	6.43370030552916\\
64.25	0.25038	6.89043032391618\\
64.25	0.25404	7.36215758185415\\
64.25	0.2577	7.84888207934311\\
64.25	0.26136	8.35060381638301\\
64.25	0.26502	8.86732279297389\\
64.25	0.26868	9.39903900911574\\
64.25	0.27234	9.94575246480855\\
64.25	0.276	10.5074631600523\\
64.625	0.093	0.839017408832349\\
64.625	0.09666	0.659824133188724\\
64.625	0.10032	0.495628097096066\\
64.625	0.10398	0.346429300554372\\
64.625	0.10764	0.212227743563645\\
64.625	0.1113	0.0930234261238843\\
64.625	0.11496	-0.0111836517649113\\
64.625	0.11862	-0.100393490102742\\
64.625	0.12228	-0.174606088889606\\
64.625	0.12594	-0.233821448125504\\
64.625	0.1296	-0.278039567810438\\
64.625	0.13326	-0.307260447944405\\
64.625	0.13692	-0.321484088527406\\
64.625	0.14058	-0.320710489559445\\
64.625	0.14424	-0.304939651040506\\
64.625	0.1479	-0.274171572970618\\
64.625	0.15156	-0.228406255349748\\
64.625	0.15522	-0.167643698177928\\
64.625	0.15888	-0.0918839014551285\\
64.625	0.16254	-0.0011268651813765\\
64.625	0.1662	0.104627410643356\\
64.625	0.16986	0.22537892601904\\
64.625	0.17352	0.361127680945696\\
64.625	0.17718	0.51187367542332\\
64.625	0.18084	0.677616909451906\\
64.625	0.1845	0.858357383031471\\
64.625	0.18816	1.05409509616199\\
64.625	0.19182	1.26483004884347\\
64.625	0.19548	1.49056224107592\\
64.625	0.19914	1.73129167285934\\
64.625	0.2028	1.98701834419373\\
64.625	0.20646	2.25774225507908\\
64.625	0.21012	2.54346340551539\\
64.625	0.21378	2.84418179550267\\
64.625	0.21744	3.15989742504092\\
64.625	0.2211	3.49061029413013\\
64.625	0.22476	3.8363204027703\\
64.625	0.22842	4.19702775096144\\
64.625	0.23208	4.57273233870356\\
64.625	0.23574	4.96343416599663\\
64.625	0.2394	5.36913323284069\\
64.625	0.24306	5.78982953923568\\
64.625	0.24672	6.22552308518166\\
64.625	0.25038	6.6762138706786\\
64.625	0.25404	7.1419018957265\\
64.625	0.2577	7.62258716032537\\
64.625	0.26136	8.11826966447521\\
64.625	0.26502	8.62894940817601\\
64.625	0.26868	9.15462639142778\\
64.625	0.27234	9.69530061423053\\
64.625	0.276	10.2509720765842\\
65	0.093	0.887214827440332\\
65	0.09666	0.701982318906633\\
65	0.10032	0.531747049923894\\
65	0.10398	0.376509020492125\\
65	0.10764	0.236268230611316\\
65	0.1113	0.11102468028148\\
65	0.11496	0.000778369502603127\\
65	0.11862	-0.0944707017253021\\
65	0.12228	-0.174722533402248\\
65	0.12594	-0.23997712552822\\
65	0.1296	-0.29023447810323\\
65	0.13326	-0.325494591127271\\
65	0.13692	-0.345757464600354\\
65	0.14058	-0.351023098522468\\
65	0.14424	-0.34129149289361\\
65	0.1479	-0.31656264771379\\
65	0.15156	-0.276836562983002\\
65	0.15522	-0.222113238701256\\
65	0.15888	-0.152392674868539\\
65	0.16254	-0.0676748714848543\\
65	0.1662	0.0320401714497898\\
65	0.16986	0.146752453935399\\
65	0.17352	0.27646197597198\\
65	0.17718	0.42116873755953\\
65	0.18084	0.580872738698041\\
65	0.1845	0.755573979387517\\
65	0.18816	0.945272459627958\\
65	0.19182	1.14996817941937\\
65	0.19548	1.36966113876174\\
65	0.19914	1.60435133765509\\
65	0.2028	1.85403877609939\\
65	0.20646	2.11872345409466\\
65	0.21012	2.3984053716409\\
65	0.21378	2.6930845287381\\
65	0.21744	3.00276092538627\\
65	0.2211	3.3274345615854\\
65	0.22476	3.66710543733551\\
65	0.22842	4.02177355263658\\
65	0.23208	4.3914389074886\\
65	0.23574	4.77610150189159\\
65	0.2394	5.17576133584557\\
65	0.24306	5.59041840935051\\
65	0.24672	6.0200727224064\\
65	0.25038	6.46472427501326\\
65	0.25404	6.92437306717107\\
65	0.2577	7.39901909887987\\
65	0.26136	7.88866237013964\\
65	0.26502	8.39330288095037\\
65	0.26868	8.91294063131205\\
65	0.27234	9.44757562122471\\
65	0.276	9.99720785068833\\
65.375	0.093	0.938139103620522\\
65.375	0.09666	0.746867362196734\\
65.375	0.10032	0.570592860323927\\
65.375	0.10398	0.409315598002077\\
65.375	0.10764	0.263035575231193\\
65.375	0.1113	0.131752792011276\\
65.375	0.11496	0.0154672483423237\\
65.375	0.11862	-0.0858210557756562\\
65.375	0.12228	-0.172112120342684\\
65.375	0.12594	-0.243405945358731\\
65.375	0.1296	-0.299702530823822\\
65.375	0.13326	-0.341001876737945\\
65.375	0.13692	-0.367303983101102\\
65.375	0.14058	-0.378608849913292\\
65.375	0.14424	-0.374916477174509\\
65.375	0.1479	-0.35622686488477\\
65.375	0.15156	-0.322540013044064\\
65.375	0.15522	-0.273855921652393\\
65.375	0.15888	-0.21017459070975\\
65.375	0.16254	-0.131496020216154\\
65.375	0.1662	-0.0378202101715779\\
65.375	0.16986	0.0708528394239565\\
65.375	0.17352	0.194523128570456\\
65.375	0.17718	0.333190657267924\\
65.375	0.18084	0.486855425516353\\
65.375	0.1845	0.655517433315762\\
65.375	0.18816	0.839176680666128\\
65.375	0.19182	1.03783316756746\\
65.375	0.19548	1.25148689401975\\
65.375	0.19914	1.48013786002301\\
65.375	0.2028	1.72378606557725\\
65.375	0.20646	1.98243151068244\\
65.375	0.21012	2.2560741953386\\
65.375	0.21378	2.54471411954573\\
65.375	0.21744	2.84835128330382\\
65.375	0.2211	3.16698568661286\\
65.375	0.22476	3.5006173294729\\
65.375	0.22842	3.84924621188388\\
65.375	0.23208	4.21287233384583\\
65.375	0.23574	4.59149569535877\\
65.375	0.2394	4.98511629642266\\
65.375	0.24306	5.39373413703751\\
65.375	0.24672	5.81734921720331\\
65.375	0.25038	6.25596153692011\\
65.375	0.25404	6.70957109618784\\
65.375	0.2577	7.17817789500657\\
65.375	0.26136	7.66178193337624\\
65.375	0.26502	8.1603832112969\\
65.375	0.26868	8.67398172876851\\
65.375	0.27234	9.2025774857911\\
65.375	0.276	9.74617048236463\\
65.75	0.093	0.991790237372904\\
65.75	0.09666	0.794479263059055\\
65.75	0.10032	0.61216552829616\\
65.75	0.10398	0.444849033084235\\
65.75	0.10764	0.292529777423276\\
65.75	0.1113	0.155207761313277\\
65.75	0.11496	0.0328829847542504\\
65.75	0.11862	-0.0744445522538113\\
65.75	0.12228	-0.166774849710913\\
65.75	0.12594	-0.244107907617035\\
65.75	0.1296	-0.306443725972208\\
65.75	0.13326	-0.353782304776413\\
65.75	0.13692	-0.386123644029638\\
65.75	0.14058	-0.403467743731909\\
65.75	0.14424	-0.405814603883215\\
65.75	0.1479	-0.393164224483543\\
65.75	0.15156	-0.365516605532919\\
65.75	0.15522	-0.322871747031323\\
65.75	0.15888	-0.265229648978755\\
65.75	0.16254	-0.192590311375234\\
65.75	0.1662	-0.104953734220739\\
65.75	0.16986	-0.00231991751527971\\
65.75	0.17352	0.115311138741138\\
65.75	0.17718	0.247939434548531\\
65.75	0.18084	0.395564969906886\\
65.75	0.1845	0.558187744816212\\
65.75	0.18816	0.735807759276497\\
65.75	0.19182	0.92842501328775\\
65.75	0.19548	1.13603950684997\\
65.75	0.19914	1.35865123996316\\
65.75	0.2028	1.59626021262731\\
65.75	0.20646	1.84886642484243\\
65.75	0.21012	2.11646987660851\\
65.75	0.21378	2.39907056792555\\
65.75	0.21744	2.69666849879357\\
65.75	0.2211	3.00926366921255\\
65.75	0.22476	3.33685607918249\\
65.75	0.22842	3.67944572870341\\
65.75	0.23208	4.03703261777528\\
65.75	0.23574	4.40961674639813\\
65.75	0.2394	4.79719811457193\\
65.75	0.24306	5.19977672229671\\
65.75	0.24672	5.61735256957246\\
65.75	0.25038	6.04992565639917\\
65.75	0.25404	6.49749598277683\\
65.75	0.2577	6.96006354870546\\
65.75	0.26136	7.43762835418507\\
65.75	0.26502	7.93019039921565\\
65.75	0.26868	8.43774968379716\\
65.75	0.27234	8.96030620792968\\
65.75	0.276	9.49785997161315\\
66.125	0.093	1.0481682286975\\
66.125	0.09666	0.844818021493562\\
66.125	0.10032	0.656465053840591\\
66.125	0.10398	0.483109325738591\\
66.125	0.10764	0.324750837187551\\
66.125	0.1113	0.181389588187478\\
66.125	0.11496	0.0530255787383691\\
66.125	0.11862	-0.0603411911597673\\
66.125	0.12228	-0.158710721506944\\
66.125	0.12594	-0.242083012303155\\
66.125	0.1296	-0.310458063548396\\
66.125	0.13326	-0.363835875242668\\
66.125	0.13692	-0.402216447385982\\
66.125	0.14058	-0.425599779978327\\
66.125	0.14424	-0.433985873019708\\
66.125	0.1479	-0.427374726510118\\
66.125	0.15156	-0.405766340449562\\
66.125	0.15522	-0.369160714838047\\
66.125	0.15888	-0.317557849675568\\
66.125	0.16254	-0.250957744962115\\
66.125	0.1662	-0.169360400697702\\
66.125	0.16986	-0.072765816882324\\
66.125	0.17352	0.0388260064840189\\
66.125	0.17718	0.165415069401337\\
66.125	0.18084	0.307001371869617\\
66.125	0.1845	0.463584913888862\\
66.125	0.18816	0.635165695459072\\
66.125	0.19182	0.821743716580244\\
66.125	0.19548	1.02331897725239\\
66.125	0.19914	1.2398914774755\\
66.125	0.2028	1.47146121724956\\
66.125	0.20646	1.7180281965746\\
66.125	0.21012	1.97959241545061\\
66.125	0.21378	2.25615387387758\\
66.125	0.21744	2.54771257185552\\
66.125	0.2211	2.85426850938443\\
66.125	0.22476	3.17582168646428\\
66.125	0.22842	3.51237210309513\\
66.125	0.23208	3.86391975927692\\
66.125	0.23574	4.23046465500969\\
66.125	0.2394	4.61200679029342\\
66.125	0.24306	5.00854616512812\\
66.125	0.24672	5.42008277951378\\
66.125	0.25038	5.84661663345041\\
66.125	0.25404	6.28814772693801\\
66.125	0.2577	6.74467605997657\\
66.125	0.26136	7.21620163256609\\
66.125	0.26502	7.70272444470658\\
66.125	0.26868	8.20424449639805\\
66.125	0.27234	8.72076178764047\\
66.125	0.276	9.25227631843386\\
66.5	0.093	1.10727307759429\\
66.5	0.09666	0.897883637500281\\
66.5	0.10032	0.703491436957236\\
66.5	0.10398	0.524096475965155\\
66.5	0.10764	0.35969875452404\\
66.5	0.1113	0.210298272633884\\
66.5	0.11496	0.075895030294701\\
66.5	0.11862	-0.0435109724935172\\
66.5	0.12228	-0.147919735730769\\
66.5	0.12594	-0.237331259417047\\
66.5	0.1296	-0.311745543552377\\
66.5	0.13326	-0.371162588136723\\
66.5	0.13692	-0.415582393170119\\
66.5	0.14058	-0.44500495865254\\
66.5	0.14424	-0.459430284584002\\
66.5	0.1479	-0.458858370964487\\
66.5	0.15156	-0.443289217794019\\
66.5	0.15522	-0.412722825072573\\
66.5	0.15888	-0.367159192800168\\
66.5	0.16254	-0.306598320976796\\
66.5	0.1662	-0.231040209602458\\
66.5	0.16986	-0.140484858677155\\
66.5	0.17352	-0.0349322682008868\\
66.5	0.17718	0.0856175618263499\\
66.5	0.18084	0.221164631404548\\
66.5	0.1845	0.371708940533718\\
66.5	0.18816	0.537250489213847\\
66.5	0.19182	0.71778927744495\\
66.5	0.19548	0.913325305227008\\
66.5	0.19914	1.12385857256005\\
66.5	0.2028	1.34938907944403\\
66.5	0.20646	1.58991682587901\\
66.5	0.21012	1.84544181186492\\
66.5	0.21378	2.11596403740183\\
66.5	0.21744	2.40148350248969\\
66.5	0.2211	2.70200020712851\\
66.5	0.22476	3.0175141513183\\
66.5	0.22842	3.34802533505905\\
66.5	0.23208	3.69353375835076\\
66.5	0.23574	4.05403942119346\\
66.5	0.2394	4.42954232358711\\
66.5	0.24306	4.82004246553175\\
66.5	0.24672	5.22553984702732\\
66.5	0.25038	5.64603446807388\\
66.5	0.25404	6.08152632867139\\
66.5	0.2577	6.53201542881988\\
66.5	0.26136	6.99750176851931\\
66.5	0.26502	7.47798534776974\\
66.5	0.26868	7.97346616657111\\
66.5	0.27234	8.48394422492346\\
66.5	0.276	9.00941952282678\\
66.875	0.093	1.16910478406329\\
66.875	0.09666	0.9536761110792\\
66.875	0.10032	0.75324467764608\\
66.875	0.10398	0.567810483763924\\
66.875	0.10764	0.397373529432727\\
66.875	0.1113	0.241933814652497\\
66.875	0.11496	0.101491339423239\\
66.875	0.11862	-0.0239538962550538\\
66.875	0.12228	-0.134401892382387\\
66.875	0.12594	-0.229852648958754\\
66.875	0.1296	-0.310306165984152\\
66.875	0.13326	-0.37576244345858\\
66.875	0.13692	-0.426221481382051\\
66.875	0.14058	-0.461683279754553\\
66.875	0.14424	-0.482147838576083\\
66.875	0.1479	-0.487615157846649\\
66.875	0.15156	-0.478085237566249\\
66.875	0.15522	-0.453558077734892\\
66.875	0.15888	-0.414033678352562\\
66.875	0.16254	-0.359512039419265\\
66.875	0.1662	-0.289993160935008\\
66.875	0.16986	-0.205477042899787\\
66.875	0.17352	-0.105963685313593\\
66.875	0.17718	0.00854691182356859\\
66.875	0.18084	0.138054748511692\\
66.875	0.1845	0.282559824750781\\
66.875	0.18816	0.442062140540834\\
66.875	0.19182	0.616561695881856\\
66.875	0.19548	0.80605849077384\\
66.875	0.19914	1.01055252521679\\
66.875	0.2028	1.23004379921071\\
66.875	0.20646	1.46453231275559\\
66.875	0.21012	1.71401806585145\\
66.875	0.21378	1.97850105849826\\
66.875	0.21744	2.25798129069604\\
66.875	0.2211	2.55245876244479\\
66.875	0.22476	2.86193347374451\\
66.875	0.22842	3.18640542459519\\
66.875	0.23208	3.52587461499682\\
66.875	0.23574	3.88034104494943\\
66.875	0.2394	4.24980471445301\\
66.875	0.24306	4.63426562350755\\
66.875	0.24672	5.03372377211306\\
66.875	0.25038	5.44817916026955\\
66.875	0.25404	5.87763178797697\\
66.875	0.2577	6.32208165523538\\
66.875	0.26136	6.78152876204475\\
66.875	0.26502	7.25597310840509\\
66.875	0.26868	7.74541469431638\\
66.875	0.27234	8.24985351977865\\
66.875	0.276	8.76928958479189\\
67.25	0.093	1.23366334810449\\
67.25	0.09666	1.01219544223032\\
67.25	0.10032	0.805724775907123\\
67.25	0.10398	0.614251349134892\\
67.25	0.10764	0.437775161913613\\
67.25	0.1113	0.276296214243316\\
67.25	0.11496	0.129814506123969\\
67.25	0.11862	-0.00166996244439854\\
67.25	0.12228	-0.118157191461806\\
67.25	0.12594	-0.219647180928249\\
67.25	0.1296	-0.306139930843727\\
67.25	0.13326	-0.377635441208238\\
67.25	0.13692	-0.434133712021776\\
67.25	0.14058	-0.47563474328436\\
67.25	0.14424	-0.502138534995964\\
67.25	0.1479	-0.513645087156613\\
67.25	0.15156	-0.510154399766295\\
67.25	0.15522	-0.491666472825012\\
67.25	0.15888	-0.458181306332756\\
67.25	0.16254	-0.409698900289534\\
67.25	0.1662	-0.34621925469536\\
67.25	0.16986	-0.267742369550213\\
67.25	0.17352	-0.174268244854101\\
67.25	0.17718	-0.0657968806070208\\
67.25	0.18084	0.0576717231910209\\
67.25	0.1845	0.196137566540042\\
67.25	0.18816	0.349600649440013\\
67.25	0.19182	0.518060971890961\\
67.25	0.19548	0.701518533892877\\
67.25	0.19914	0.899973335445743\\
67.25	0.2028	1.11342537654958\\
67.25	0.20646	1.34187465720439\\
67.25	0.21012	1.58532117741017\\
67.25	0.21378	1.8437649371669\\
67.25	0.21744	2.11720593647461\\
67.25	0.2211	2.40564417533327\\
67.25	0.22476	2.70907965374291\\
67.25	0.22842	3.02751237170351\\
67.25	0.23208	3.36094232921507\\
67.25	0.23574	3.70936952627761\\
67.25	0.2394	4.07279396289111\\
67.25	0.24306	4.45121563905558\\
67.25	0.24672	4.84463455477101\\
67.25	0.25038	5.2530507100374\\
67.25	0.25404	5.67646410485476\\
67.25	0.2577	6.1148747392231\\
67.25	0.26136	6.56828261314238\\
67.25	0.26502	7.03668772661265\\
67.25	0.26868	7.52009007963387\\
67.25	0.27234	8.01848967220606\\
67.25	0.276	8.53188650432921\\
67.625	0.093	1.3009487697179\\
67.625	0.09666	1.07344163095366\\
67.625	0.10032	0.860931731740379\\
67.625	0.10398	0.663419072078059\\
67.625	0.10764	0.48090365196672\\
67.625	0.1113	0.313385471406334\\
67.625	0.11496	0.160864530396919\\
67.625	0.11862	0.0233408289384629\\
67.625	0.12228	-0.0991856329690126\\
67.625	0.12594	-0.206714855325536\\
67.625	0.1296	-0.29924683813109\\
67.625	0.13326	-0.376781581385675\\
67.625	0.13692	-0.439319085089295\\
67.625	0.14058	-0.486859349241954\\
67.625	0.14424	-0.519402373843647\\
67.625	0.1479	-0.536948158894363\\
67.625	0.15156	-0.539496704394127\\
67.625	0.15522	-0.527048010342925\\
67.625	0.15888	-0.499602076740738\\
67.625	0.16254	-0.457158903587604\\
67.625	0.1662	-0.399718490883505\\
67.625	0.16986	-0.327280838628425\\
67.625	0.17352	-0.239845946822395\\
67.625	0.17718	-0.137413815465397\\
67.625	0.18084	-0.0199844445574229\\
67.625	0.1845	0.112442165901509\\
67.625	0.18816	0.259866015911406\\
67.625	0.19182	0.422287105472279\\
67.625	0.19548	0.599705434584106\\
67.625	0.19914	0.792121003246898\\
67.625	0.2028	0.999533811460658\\
67.625	0.20646	1.2219438592254\\
67.625	0.21012	1.4593511465411\\
67.625	0.21378	1.71175567340775\\
67.625	0.21744	1.97915743982538\\
67.625	0.2211	2.26155644579396\\
67.625	0.22476	2.55895269131353\\
67.625	0.22842	2.87134617638406\\
67.625	0.23208	3.19873690100553\\
67.625	0.23574	3.541124865178\\
67.625	0.2394	3.89851006890142\\
67.625	0.24306	4.27089251217581\\
67.625	0.24672	4.65827219500117\\
67.625	0.25038	5.06064911737749\\
67.625	0.25404	5.47802327930475\\
67.625	0.2577	5.91039468078301\\
67.625	0.26136	6.35776332181222\\
67.625	0.26502	6.8201292023924\\
67.625	0.26868	7.29749232252355\\
67.625	0.27234	7.78985268220568\\
67.625	0.276	8.29721028143875\\
68	0.093	1.3709610489035\\
68	0.09666	1.13741467724919\\
68	0.10032	0.918865545145835\\
68	0.10398	0.71531365259344\\
68	0.10764	0.526758999592012\\
68	0.1113	0.353201586141551\\
68	0.11496	0.194641412242062\\
68	0.11862	0.0510784778935305\\
68	0.12228	-0.0774872169040339\\
68	0.12594	-0.191055672150632\\
68	0.1296	-0.289626887846261\\
68	0.13326	-0.373200863990927\\
68	0.13692	-0.441777600584629\\
68	0.14058	-0.495357097627362\\
68	0.14424	-0.533939355119124\\
68	0.1479	-0.557524373059922\\
68	0.15156	-0.56611215144976\\
68	0.15522	-0.559702690288633\\
68	0.15888	-0.538295989576534\\
68	0.16254	-0.501892049313476\\
68	0.1662	-0.45049086949945\\
68	0.16986	-0.38409245013446\\
68	0.17352	-0.302696791218498\\
68	0.17718	-0.206303892751567\\
68	0.18084	-0.0949137547336818\\
68	0.1845	0.0314736228351755\\
68	0.18816	0.172858239954998\\
68	0.19182	0.329240096625796\\
68	0.19548	0.500619192847534\\
68	0.19914	0.686995528620265\\
68	0.2028	0.888369103943937\\
68	0.20646	1.1047399188186\\
68	0.21012	1.33610797324421\\
68	0.21378	1.5824732672208\\
68	0.21744	1.84383580074836\\
68	0.2211	2.12019557382685\\
68	0.22476	2.41155258645635\\
68	0.22842	2.71790683863678\\
68	0.23208	3.0392583303682\\
68	0.23574	3.37560706165058\\
68	0.2394	3.72695303248392\\
68	0.24306	4.09329624286824\\
68	0.24672	4.4746366928035\\
68	0.25038	4.87097438228976\\
68	0.25404	5.28230931132694\\
68	0.2577	5.70864147991513\\
68	0.26136	6.14997088805425\\
68	0.26502	6.60629753574437\\
68	0.26868	7.07762142298543\\
68	0.27234	7.56394254977748\\
68	0.276	8.06526091612048\\
68.375	0.093	1.44370018566134\\
68.375	0.09666	1.20411458111694\\
68.375	0.10032	0.979526216123503\\
68.375	0.10398	0.769935090681027\\
68.375	0.10764	0.575341204789531\\
68.375	0.1113	0.395744558448988\\
68.375	0.11496	0.231145151659417\\
68.375	0.11862	0.0815429844208113\\
68.375	0.12228	-0.0530619432668278\\
68.375	0.12594	-0.172669631403501\\
68.375	0.1296	-0.277280079989211\\
68.375	0.13326	-0.366893289023952\\
68.375	0.13692	-0.441509258507729\\
68.375	0.14058	-0.501127988440537\\
68.375	0.14424	-0.545749478822387\\
68.375	0.1479	-0.57537372965326\\
68.375	0.15156	-0.59000074093318\\
68.375	0.15522	-0.58963051266212\\
68.375	0.15888	-0.57426304484011\\
68.375	0.16254	-0.54389833746712\\
68.375	0.1662	-0.498536390543176\\
68.375	0.16986	-0.438177204068261\\
68.375	0.17352	-0.36282077804238\\
68.375	0.17718	-0.272467112465531\\
68.375	0.18084	-0.16711620733772\\
68.375	0.1845	-0.0467680626589377\\
68.375	0.18816	0.0885773215708099\\
68.375	0.19182	0.238919945351519\\
68.375	0.19548	0.404259808683197\\
68.375	0.19914	0.584596911565839\\
68.375	0.2028	0.77993125399945\\
68.375	0.20646	0.990262835984026\\
68.375	0.21012	1.21559165751956\\
68.375	0.21378	1.45591771860607\\
68.375	0.21744	1.71124101924354\\
68.375	0.2211	1.98156155943196\\
68.375	0.22476	2.26687933917138\\
68.375	0.22842	2.56719435846175\\
68.375	0.23208	2.88250661730307\\
68.375	0.23574	3.21281611569538\\
68.375	0.2394	3.55812285363864\\
68.375	0.24306	3.91842683113288\\
68.375	0.24672	4.29372804817809\\
68.375	0.25038	4.68402650477423\\
68.375	0.25404	5.08932220092137\\
68.375	0.2577	5.50961513661947\\
68.375	0.26136	5.94490531186851\\
68.375	0.26502	6.39519272666854\\
68.375	0.26868	6.86047738101954\\
68.375	0.27234	7.3407592749215\\
68.375	0.276	7.83603840837441\\
68.75	0.093	1.51916617999135\\
68.75	0.09666	1.27354134255687\\
68.75	0.10032	1.04291374467336\\
68.75	0.10398	0.827283386340812\\
68.75	0.10764	0.626650267559235\\
68.75	0.1113	0.441014388328617\\
68.75	0.11496	0.270375748648972\\
68.75	0.11862	0.114734348520284\\
68.75	0.12228	-0.0259098120574297\\
68.75	0.12594	-0.151556733084185\\
68.75	0.1296	-0.262206414559969\\
68.75	0.13326	-0.357858856484786\\
68.75	0.13692	-0.438514058858637\\
68.75	0.14058	-0.504172021681534\\
68.75	0.14424	-0.554832744953451\\
68.75	0.1479	-0.590496228674406\\
68.75	0.15156	-0.6111624728444\\
68.75	0.15522	-0.61683147746343\\
68.75	0.15888	-0.607503242531481\\
68.75	0.16254	-0.583177768048579\\
68.75	0.1662	-0.54385505401471\\
68.75	0.16986	-0.489535100429869\\
68.75	0.17352	-0.42021790729407\\
68.75	0.17718	-0.335903474607289\\
68.75	0.18084	-0.236591802369553\\
68.75	0.1845	-0.122282890580859\\
68.75	0.18816	0.00702326075881388\\
68.75	0.19182	0.151326651649448\\
68.75	0.19548	0.310627282091037\\
68.75	0.19914	0.484925152083605\\
68.75	0.2028	0.674220261627141\\
68.75	0.20646	0.878512610721629\\
68.75	0.21012	1.09780219936709\\
68.75	0.21378	1.33208902756352\\
68.75	0.21744	1.58137309531091\\
68.75	0.2211	1.84565440260927\\
68.75	0.22476	2.1249329494586\\
68.75	0.22842	2.41920873585888\\
68.75	0.23208	2.72848176181013\\
68.75	0.23574	3.05275202731237\\
68.75	0.2394	3.39201953236556\\
68.75	0.24306	3.74628427696972\\
68.75	0.24672	4.11554626112483\\
68.75	0.25038	4.49980548483094\\
68.75	0.25404	4.89906194808797\\
68.75	0.2577	5.31331565089599\\
68.75	0.26136	5.74256659325498\\
68.75	0.26502	6.18681477516492\\
68.75	0.26868	6.64606019662583\\
68.75	0.27234	7.12030285763772\\
68.75	0.276	7.60954275820055\\
69.125	0.093	1.59735903189358\\
69.125	0.09666	1.34569496156903\\
69.125	0.10032	1.10902813079544\\
69.125	0.10398	0.887358539572811\\
69.125	0.10764	0.680686187901159\\
69.125	0.1113	0.48901107578046\\
69.125	0.11496	0.31233320321074\\
69.125	0.11862	0.15065257019197\\
69.125	0.12228	0.00396917672418162\\
69.125	0.12594	-0.127716977192648\\
69.125	0.1296	-0.244405891558515\\
69.125	0.13326	-0.346097566373412\\
69.125	0.13692	-0.432792001637345\\
69.125	0.14058	-0.504489197350317\\
69.125	0.14424	-0.561189153512309\\
69.125	0.1479	-0.602891870123338\\
69.125	0.15156	-0.629597347183408\\
69.125	0.15522	-0.641305584692512\\
69.125	0.15888	-0.638016582650652\\
69.125	0.16254	-0.619730341057817\\
69.125	0.1662	-0.58644685991403\\
69.125	0.16986	-0.538166139219264\\
69.125	0.17352	-0.47488817897354\\
69.125	0.17718	-0.396612979176847\\
69.125	0.18084	-0.303340539829186\\
69.125	0.1845	-0.195070860930567\\
69.125	0.18816	-0.0718039424809831\\
69.125	0.19182	0.0664602155195766\\
69.125	0.19548	0.219721613071105\\
69.125	0.19914	0.387980250173584\\
69.125	0.2028	0.571236126827031\\
69.125	0.20646	0.769489243031458\\
69.125	0.21012	0.982739598786846\\
69.125	0.21378	1.2109871940932\\
69.125	0.21744	1.4542320289505\\
69.125	0.2211	1.71247410335878\\
69.125	0.22476	1.98571341731804\\
69.125	0.22842	2.27394997082826\\
69.125	0.23208	2.57718376388943\\
69.125	0.23574	2.89541479650157\\
69.125	0.2394	3.22864306866469\\
69.125	0.24306	3.57686858037877\\
69.125	0.24672	3.94009133164382\\
69.125	0.25038	4.31831132245981\\
69.125	0.25404	4.71152855282678\\
69.125	0.2577	5.11974302274474\\
69.125	0.26136	5.54295473221364\\
69.125	0.26502	5.9811636812335\\
69.125	0.26868	6.43436986980434\\
69.125	0.27234	6.90257329792615\\
69.125	0.276	7.38577396559892\\
69.5	0.093	1.678278741368\\
69.5	0.09666	1.42057543815337\\
69.5	0.10032	1.1778693744897\\
69.5	0.10398	0.950160550377002\\
69.5	0.10764	0.737448965815268\\
69.5	0.1113	0.539734620804494\\
69.5	0.11496	0.357017515344692\\
69.5	0.11862	0.189297649435855\\
69.5	0.12228	0.0365750230779849\\
69.5	0.12594	-0.101150363728927\\
69.5	0.1296	-0.223878510984868\\
69.5	0.13326	-0.33160941868984\\
69.5	0.13692	-0.424343086843848\\
69.5	0.14058	-0.502079515446894\\
69.5	0.14424	-0.564818704498968\\
69.5	0.1479	-0.612560654000079\\
69.5	0.15156	-0.64530536395023\\
69.5	0.15522	-0.663052834349402\\
69.5	0.15888	-0.665803065197617\\
69.5	0.16254	-0.653556056494871\\
69.5	0.1662	-0.626311808241159\\
69.5	0.16986	-0.584070320436467\\
69.5	0.17352	-0.526831593080825\\
69.5	0.17718	-0.454595626174214\\
69.5	0.18084	-0.367362419716642\\
69.5	0.1845	-0.265131973708083\\
69.5	0.18816	-0.147904288148574\\
69.5	0.19182	-0.0156793630380889\\
69.5	0.19548	0.13154280162335\\
69.5	0.19914	0.293762205835755\\
69.5	0.2028	0.470978849599128\\
69.5	0.20646	0.66319273291348\\
69.5	0.21012	0.870403855778779\\
69.5	0.21378	1.09261221819506\\
69.5	0.21744	1.32981782016229\\
69.5	0.2211	1.58202066168049\\
69.5	0.22476	1.84922074274967\\
69.5	0.22842	2.13141806336982\\
69.5	0.23208	2.4286126235409\\
69.5	0.23574	2.74080442326297\\
69.5	0.2394	3.06799346253601\\
69.5	0.24306	3.41017974136002\\
69.5	0.24672	3.76736325973496\\
69.5	0.25038	4.1395440176609\\
69.5	0.25404	4.52672201513779\\
69.5	0.2577	4.92889725216567\\
69.5	0.26136	5.34606972874449\\
69.5	0.26502	5.77823944487428\\
69.5	0.26868	6.22540640055504\\
69.5	0.27234	6.68757059578677\\
69.5	0.276	7.16473203056947\\
69.875	0.093	1.76192530841464\\
69.875	0.09666	1.49818277230993\\
69.875	0.10032	1.24943747575618\\
69.875	0.10398	1.0156894187534\\
69.875	0.10764	0.796938601301591\\
69.875	0.1113	0.593185023400735\\
69.875	0.11496	0.404428685050858\\
69.875	0.11862	0.230669586251939\\
69.875	0.12228	0.0719077270039943\\
69.875	0.12594	-0.0718568926929919\\
69.875	0.1296	-0.200624272839015\\
69.875	0.13326	-0.314394413434062\\
69.875	0.13692	-0.413167314478159\\
69.875	0.14058	-0.49694297597128\\
69.875	0.14424	-0.565721397913428\\
69.875	0.1479	-0.619502580304614\\
69.875	0.15156	-0.65828652314484\\
69.875	0.15522	-0.682073226434101\\
69.875	0.15888	-0.69086269017239\\
69.875	0.16254	-0.684654914359719\\
69.875	0.1662	-0.663449898996081\\
69.875	0.16986	-0.627247644081478\\
69.875	0.17352	-0.576048149615897\\
69.875	0.17718	-0.509851415599361\\
69.875	0.18084	-0.428657442031863\\
69.875	0.1845	-0.332466228913393\\
69.875	0.18816	-0.221277776243959\\
69.875	0.19182	-0.0950920840235625\\
69.875	0.19548	0.0460908477478021\\
69.875	0.19914	0.202271019070146\\
69.875	0.2028	0.37344842994343\\
69.875	0.20646	0.559623080367707\\
69.875	0.21012	0.760794970342932\\
69.875	0.21378	0.976964099869125\\
69.875	0.21744	1.2081304689463\\
69.875	0.2211	1.45429407757441\\
69.875	0.22476	1.71545492575351\\
69.875	0.22842	1.99161301348355\\
69.875	0.23208	2.28276834076459\\
69.875	0.23574	2.58892090759657\\
69.875	0.2394	2.91007071397954\\
69.875	0.24306	3.24621775991346\\
69.875	0.24672	3.59736204539834\\
69.875	0.25038	3.96350357043421\\
69.875	0.25404	4.34464233502101\\
69.875	0.2577	4.74077833915881\\
69.875	0.26136	5.15191158284754\\
69.875	0.26502	5.57804206608727\\
69.875	0.26868	6.01916978887795\\
69.875	0.27234	6.47529475121959\\
69.875	0.276	6.94641695311222\\
70.25	0.093	1.84829873303349\\
70.25	0.09666	1.5785169640387\\
70.25	0.10032	1.32373243459488\\
70.25	0.10398	1.08394514470202\\
70.25	0.10764	0.859155094360126\\
70.25	0.1113	0.649362283569196\\
70.25	0.11496	0.454566712329244\\
70.25	0.11862	0.274768380640244\\
70.25	0.12228	0.109967288502224\\
70.25	0.12594	-0.0398365640848439\\
70.25	0.1296	-0.174643177120942\\
70.25	0.13326	-0.294452550606071\\
70.25	0.13692	-0.399264684540235\\
70.25	0.14058	-0.48907957892343\\
70.25	0.14424	-0.563897233755668\\
70.25	0.1479	-0.623717649036928\\
70.25	0.15156	-0.668540824767236\\
70.25	0.15522	-0.698366760946564\\
70.25	0.15888	-0.713195457574942\\
70.25	0.16254	-0.713026914652339\\
70.25	0.1662	-0.697861132178783\\
70.25	0.16986	-0.667698110154255\\
70.25	0.17352	-0.622537848578762\\
70.25	0.17718	-0.562380347452301\\
70.25	0.18084	-0.487225606774878\\
70.25	0.1845	-0.397073626546483\\
70.25	0.18816	-0.291924406767123\\
70.25	0.19182	-0.171777947436802\\
70.25	0.19548	-0.0366342485555116\\
70.25	0.19914	0.113506689876743\\
70.25	0.2028	0.278644867859967\\
70.25	0.20646	0.458780285394141\\
70.25	0.21012	0.653912942479291\\
70.25	0.21378	0.864042839115424\\
70.25	0.21744	1.08916997530251\\
70.25	0.2211	1.32929435104054\\
70.25	0.22476	1.58441596632956\\
70.25	0.22842	1.85453482116955\\
70.25	0.23208	2.13965091556048\\
70.25	0.23574	2.43976424950241\\
70.25	0.2394	2.75487482299528\\
70.25	0.24306	3.08498263603913\\
70.25	0.24672	3.43008768863393\\
70.25	0.25038	3.79018998077971\\
70.25	0.25404	4.16528951247644\\
70.25	0.2577	4.55538628372415\\
70.25	0.26136	4.96048029452282\\
70.25	0.26502	5.38057154487247\\
70.25	0.26868	5.81566003477307\\
70.25	0.27234	6.26574576422464\\
70.25	0.276	6.73082873322718\\
70.625	0.093	1.93739901522452\\
70.625	0.09666	1.66157801333965\\
70.625	0.10032	1.40075425100575\\
70.625	0.10398	1.15492772822282\\
70.625	0.10764	0.924098444990854\\
70.625	0.1113	0.708266401309841\\
70.625	0.11496	0.507431597179808\\
70.625	0.11862	0.321594032600733\\
70.625	0.12228	0.150753707572639\\
70.625	0.12594	-0.00508937790450403\\
70.625	0.1296	-0.145935223830683\\
70.625	0.13326	-0.271783830205887\\
70.625	0.13692	-0.382635197030126\\
70.625	0.14058	-0.478489324303411\\
70.625	0.14424	-0.559346212025716\\
70.625	0.1479	-0.625205860197058\\
70.625	0.15156	-0.67606826881744\\
70.625	0.15522	-0.711933437886858\\
70.625	0.15888	-0.732801367405296\\
70.625	0.16254	-0.738672057372781\\
70.625	0.1662	-0.7295455077893\\
70.625	0.16986	-0.705421718654847\\
70.625	0.17352	-0.666300689969429\\
70.625	0.17718	-0.612182421733042\\
70.625	0.18084	-0.543066913945694\\
70.625	0.1845	-0.458954166607388\\
70.625	0.18816	-0.359844179718102\\
70.625	0.19182	-0.245736953277856\\
70.625	0.19548	-0.116632487286655\\
70.625	0.19914	0.0274692182555256\\
70.625	0.2028	0.186568163348674\\
70.625	0.20646	0.360664347992774\\
70.625	0.21012	0.549757772187849\\
70.625	0.21378	0.753848435933893\\
70.625	0.21744	0.972936339230888\\
70.625	0.2211	1.20702148207886\\
70.625	0.22476	1.4561038644778\\
70.625	0.22842	1.7201834864277\\
70.625	0.23208	1.99926034792857\\
70.625	0.23574	2.29333444898041\\
70.625	0.2394	2.60240578958322\\
70.625	0.24306	2.92647436973699\\
70.625	0.24672	3.26554018944171\\
70.625	0.25038	3.6196032486974\\
70.625	0.25404	3.98866354750406\\
70.625	0.2577	4.3727210858617\\
70.625	0.26136	4.77177586377028\\
70.625	0.26502	5.18582788122986\\
70.625	0.26868	5.61487713824039\\
70.625	0.27234	6.05892363480189\\
70.625	0.276	6.51796737091433\\
71	0.093	2.02922615498778\\
71	0.09666	1.74736592021284\\
71	0.10032	1.48050292498886\\
71	0.10398	1.22863716931585\\
71	0.10764	0.991768653193802\\
71	0.1113	0.769897376622715\\
71	0.11496	0.563023339602607\\
71	0.11862	0.371146542133457\\
71	0.12228	0.194266984215274\\
71	0.12594	0.0323846658480562\\
71	0.1296	-0.114500412968198\\
71	0.13326	-0.246388252233476\\
71	0.13692	-0.363278851947804\\
71	0.14058	-0.465172212111149\\
71	0.14424	-0.552068332723543\\
71	0.1479	-0.62396721378496\\
71	0.15156	-0.680868855295417\\
71	0.15522	-0.722773257254909\\
71	0.15888	-0.749680419663436\\
71	0.16254	-0.761590342520989\\
71	0.1662	-0.75850302582759\\
71	0.16986	-0.740418469583211\\
71	0.17352	-0.707336673787882\\
71	0.17718	-0.65925763844157\\
71	0.18084	-0.596181363544297\\
71	0.1845	-0.518107849096065\\
71	0.18816	-0.425037095096869\\
71	0.19182	-0.316969101546697\\
71	0.19548	-0.193903868445556\\
71	0.19914	-0.0558413957934647\\
71	0.2028	0.0972183164095952\\
71	0.20646	0.265275268163634\\
71	0.21012	0.448329459468635\\
71	0.21378	0.64638089032459\\
71	0.21744	0.85942956073151\\
71	0.2211	1.08747547068941\\
71	0.22476	1.33051862019828\\
71	0.22842	1.58855900925811\\
71	0.23208	1.86159663786889\\
71	0.23574	2.14963150603065\\
71	0.2394	2.45266361374338\\
71	0.24306	2.77069296100707\\
71	0.24672	3.1037195478217\\
71	0.25038	3.45174337418734\\
71	0.25404	3.81476444010391\\
71	0.2577	4.19278274557147\\
71	0.26136	4.58579829058998\\
71	0.26502	4.99381107515947\\
71	0.26868	5.41682109927992\\
71	0.27234	5.85482836295134\\
71	0.276	6.30783286617373\\
71.375	0.093	2.12378015232323\\
71.375	0.09666	1.83588068465822\\
71.375	0.10032	1.56297845654416\\
71.375	0.10398	1.30507346798106\\
71.375	0.10764	1.06216571896895\\
71.375	0.1113	0.83425520950778\\
71.375	0.11496	0.621341939597597\\
71.375	0.11862	0.423425909238365\\
71.375	0.12228	0.240507118430108\\
71.375	0.12594	0.0725855671728155\\
71.375	0.1296	-0.0803387445335133\\
71.375	0.13326	-0.218265816688881\\
71.375	0.13692	-0.341195649293269\\
71.375	0.14058	-0.449128242346703\\
71.375	0.14424	-0.542063595849172\\
71.375	0.1479	-0.620001709800677\\
71.375	0.15156	-0.682942584201202\\
71.375	0.15522	-0.730886219050776\\
71.375	0.15888	-0.763832614349377\\
71.375	0.16254	-0.781781770097005\\
71.375	0.1662	-0.784733686293681\\
71.375	0.16986	-0.772688362939391\\
71.375	0.17352	-0.745645800034136\\
71.375	0.17718	-0.703605997577899\\
71.375	0.18084	-0.6465689555707\\
71.375	0.1845	-0.574534674012543\\
71.375	0.18816	-0.487503152903422\\
71.375	0.19182	-0.385474392243339\\
71.375	0.19548	-0.268448392032273\\
71.375	0.19914	-0.136425152270256\\
71.375	0.2028	0.0105953270427293\\
71.375	0.20646	0.172613045906679\\
71.375	0.21012	0.349628004321605\\
71.375	0.21378	0.5416402022875\\
71.375	0.21744	0.748649639804331\\
71.375	0.2211	0.970656316872155\\
71.375	0.22476	1.20766023349095\\
71.375	0.22842	1.45966138966069\\
71.375	0.23208	1.7266597853814\\
71.375	0.23574	2.00865542065308\\
71.375	0.2394	2.30564829547573\\
71.375	0.24306	2.61763840984934\\
71.375	0.24672	2.94462576377393\\
71.375	0.25038	3.28661035724947\\
71.375	0.25404	3.64359219027597\\
71.375	0.2577	4.01557126285344\\
71.375	0.26136	4.40254757498187\\
71.375	0.26502	4.80452112666131\\
71.375	0.26868	5.22149191789165\\
71.375	0.27234	5.653459948673\\
71.375	0.276	6.1004252190053\\
71.75	0.093	2.22106100723087\\
71.75	0.09666	1.92712230667578\\
71.75	0.10032	1.64818084567165\\
71.75	0.10398	1.38423662421848\\
71.75	0.10764	1.13528964231628\\
71.75	0.1113	0.901339899965037\\
71.75	0.11496	0.682387397164772\\
71.75	0.11862	0.478432133915466\\
71.75	0.12228	0.289474110217133\\
71.75	0.12594	0.11551332606976\\
71.75	0.1296	-0.0434502185266439\\
71.75	0.13326	-0.187416523572086\\
71.75	0.13692	-0.316385589066563\\
71.75	0.14058	-0.430357415010072\\
71.75	0.14424	-0.529332001402615\\
71.75	0.1479	-0.613309348244195\\
71.75	0.15156	-0.682289455534802\\
71.75	0.15522	-0.736272323274443\\
71.75	0.15888	-0.77525795146312\\
71.75	0.16254	-0.799246340100837\\
71.75	0.1662	-0.808237489187587\\
71.75	0.16986	-0.802231398723372\\
71.75	0.17352	-0.781228068708192\\
71.75	0.17718	-0.745227499142043\\
71.75	0.18084	-0.694229690024933\\
71.75	0.1845	-0.628234641356837\\
71.75	0.18816	-0.54724235313779\\
71.75	0.19182	-0.451252825367781\\
71.75	0.19548	-0.340266058046804\\
71.75	0.19914	-0.214282051174862\\
71.75	0.2028	-0.0733008047519377\\
71.75	0.20646	0.0826776812219236\\
71.75	0.21012	0.253653406746775\\
71.75	0.21378	0.43962637182258\\
71.75	0.21744	0.640596576449351\\
71.75	0.2211	0.856564020627086\\
71.75	0.22476	1.08752870435579\\
71.75	0.22842	1.33349062763547\\
71.75	0.23208	1.59444979046609\\
71.75	0.23574	1.8704061928477\\
71.75	0.2394	2.16135983478028\\
71.75	0.24306	2.46731071626381\\
71.75	0.24672	2.78825883729829\\
71.75	0.25038	3.12420419788378\\
71.75	0.25404	3.4751467980202\\
71.75	0.2577	3.8410866377076\\
71.75	0.26136	4.22202371694597\\
71.75	0.26502	4.61795803573528\\
71.75	0.26868	5.02888959407559\\
71.75	0.27234	5.45481839196685\\
71.75	0.276	5.89574442940907\\
72.125	0.093	2.32106871971075\\
72.125	0.09666	2.02109078626558\\
72.125	0.10032	1.73611009237136\\
72.125	0.10398	1.46612663802812\\
72.125	0.10764	1.21114042323585\\
72.125	0.1113	0.971151447994528\\
72.125	0.11496	0.746159712304182\\
72.125	0.11862	0.536165216164801\\
72.125	0.12228	0.341167959576387\\
72.125	0.12594	0.161167942538938\\
72.125	0.1296	-0.00383483494754699\\
72.125	0.13326	-0.153840372883064\\
72.125	0.13692	-0.288848671267615\\
72.125	0.14058	-0.408859730101213\\
72.125	0.14424	-0.513873549383824\\
72.125	0.1479	-0.603890129115486\\
72.125	0.15156	-0.678909469296167\\
72.125	0.15522	-0.738931569925898\\
72.125	0.15888	-0.783956431004649\\
72.125	0.16254	-0.81398405253244\\
72.125	0.1662	-0.829014434509265\\
72.125	0.16986	-0.829047576935125\\
72.125	0.17352	-0.814083479810019\\
72.125	0.17718	-0.784122143133946\\
72.125	0.18084	-0.73916356690691\\
72.125	0.1845	-0.679207751128903\\
72.125	0.18816	-0.604254695799945\\
72.125	0.19182	-0.514304400920011\\
72.125	0.19548	-0.409356866489109\\
72.125	0.19914	-0.289412092507241\\
72.125	0.2028	-0.154470078974406\\
72.125	0.20646	-0.00453082589060472\\
72.125	0.21012	0.160405666744158\\
72.125	0.21378	0.340339398929888\\
72.125	0.21744	0.535270370666584\\
72.125	0.2211	0.745198581954245\\
72.125	0.22476	0.970124032792874\\
72.125	0.22842	1.21004672318247\\
72.125	0.23208	1.46496665312302\\
72.125	0.23574	1.73488382261455\\
72.125	0.2394	2.01979823165703\\
72.125	0.24306	2.31970988025051\\
72.125	0.24672	2.63461876839493\\
72.125	0.25038	2.96452489609032\\
72.125	0.25404	3.30942826333666\\
72.125	0.2577	3.66932887013398\\
72.125	0.26136	4.04422671648225\\
72.125	0.26502	4.43412180238153\\
72.125	0.26868	4.83901412783173\\
72.125	0.27234	5.25890369283292\\
72.125	0.276	5.69379049738505\\
72.5	0.093	2.42380328976281\\
72.5	0.09666	2.11778612342755\\
72.5	0.10032	1.82676619664327\\
72.5	0.10398	1.55074350940994\\
72.5	0.10764	1.28971806172759\\
72.5	0.1113	1.0436898535962\\
72.5	0.11496	0.812658885015777\\
72.5	0.11862	0.596625155986314\\
72.5	0.12228	0.395588666507825\\
72.5	0.12594	0.209549416580295\\
72.5	0.1296	0.0385074062037347\\
72.5	0.13326	-0.117537364621864\\
72.5	0.13692	-0.258584895896497\\
72.5	0.14058	-0.384635187620155\\
72.5	0.14424	-0.495688239792855\\
72.5	0.1479	-0.591744052414592\\
72.5	0.15156	-0.672802625485355\\
72.5	0.15522	-0.738863959005153\\
72.5	0.15888	-0.789928052973993\\
72.5	0.16254	-0.825994907391859\\
72.5	0.1662	-0.847064522258759\\
72.5	0.16986	-0.853136897574693\\
72.5	0.17352	-0.844212033339677\\
72.5	0.17718	-0.820289929553677\\
72.5	0.18084	-0.781370586216717\\
72.5	0.1845	-0.727454003328784\\
72.5	0.18816	-0.658540180889901\\
72.5	0.19182	-0.574629118900042\\
72.5	0.19548	-0.475720817359214\\
72.5	0.19914	-0.361815276267421\\
72.5	0.2028	-0.232912495624674\\
72.5	0.20646	-0.0890124754309483\\
72.5	0.21012	0.0698847843137393\\
72.5	0.21378	0.243779283609395\\
72.5	0.21744	0.432671022456017\\
72.5	0.2211	0.636560000853589\\
72.5	0.22476	0.855446218802143\\
72.5	0.22842	1.08932967630166\\
72.5	0.23208	1.33821037335213\\
72.5	0.23574	1.60208830995358\\
72.5	0.2394	1.88096348610599\\
72.5	0.24306	2.17483590180937\\
72.5	0.24672	2.48370555706374\\
72.5	0.25038	2.80757245186904\\
72.5	0.25404	3.1464365862253\\
72.5	0.2577	3.50029796013255\\
72.5	0.26136	3.86915657359076\\
72.5	0.26502	4.25301242659992\\
72.5	0.26868	4.65186551916006\\
72.5	0.27234	5.06571585127117\\
72.5	0.276	5.49456342293323\\
72.875	0.093	2.52926471738709\\
72.875	0.09666	2.21720831816176\\
72.875	0.10032	1.92014915848739\\
72.875	0.10398	1.63808723836399\\
72.875	0.10764	1.37102255779156\\
72.875	0.1113	1.11895511677009\\
72.875	0.11496	0.881884915299585\\
72.875	0.11862	0.659811953380047\\
72.875	0.12228	0.452736231011484\\
72.875	0.12594	0.260657748193879\\
72.875	0.1296	0.0835765049272368\\
72.875	0.13326	-0.0785074987884364\\
72.875	0.13692	-0.225594262953145\\
72.875	0.14058	-0.357683787566891\\
72.875	0.14424	-0.474776072629659\\
72.875	0.1479	-0.576871118141478\\
72.875	0.15156	-0.663968924102315\\
72.875	0.15522	-0.736069490512188\\
72.875	0.15888	-0.793172817371103\\
72.875	0.16254	-0.835278904679058\\
72.875	0.1662	-0.862387752436032\\
72.875	0.16986	-0.874499360642055\\
72.875	0.17352	-0.871613729297099\\
72.875	0.17718	-0.853730858401175\\
72.875	0.18084	-0.820850747954303\\
72.875	0.1845	-0.772973397956445\\
72.875	0.18816	-0.710098808407636\\
72.875	0.19182	-0.632226979307866\\
72.875	0.19548	-0.539357910657113\\
72.875	0.19914	-0.431491602455395\\
72.875	0.2028	-0.308628054702709\\
72.875	0.20646	-0.170767267399086\\
72.875	0.21012	-0.0179092405444727\\
72.875	0.21378	0.149946025861109\\
72.875	0.21744	0.332798531817655\\
72.875	0.2211	0.530648277325152\\
72.875	0.22476	0.743495262383632\\
72.875	0.22842	0.971339486993063\\
72.875	0.23208	1.21418095115346\\
72.875	0.23574	1.47201965486484\\
72.875	0.2394	1.74485559812717\\
72.875	0.24306	2.03268878094048\\
72.875	0.24672	2.33551920330475\\
72.875	0.25038	2.65334686521997\\
72.875	0.25404	2.98617176668616\\
72.875	0.2577	3.33399390770334\\
72.875	0.26136	3.69681328827147\\
72.875	0.26502	4.07462990839058\\
72.875	0.26868	4.46744376806063\\
72.875	0.27234	4.87525486728165\\
72.875	0.276	5.29806320605365\\
73.25	0.093	2.63745300258356\\
73.25	0.09666	2.31935737046815\\
73.25	0.10032	2.01625897790371\\
73.25	0.10398	1.72815782489023\\
73.25	0.10764	1.45505391142772\\
73.25	0.1113	1.19694723751617\\
73.25	0.11496	0.953837803155592\\
73.25	0.11862	0.725725608345972\\
73.25	0.12228	0.512610653087334\\
73.25	0.12594	0.314492937379647\\
73.25	0.1296	0.131372461222931\\
73.25	0.13326	-0.0367507753828171\\
73.25	0.13692	-0.1898767724376\\
73.25	0.14058	-0.328005529941422\\
73.25	0.14424	-0.451137047894278\\
73.25	0.1479	-0.559271326296171\\
73.25	0.15156	-0.652408365147084\\
73.25	0.15522	-0.730548164447045\\
73.25	0.15888	-0.793690724196034\\
73.25	0.16254	-0.84183604439405\\
73.25	0.1662	-0.874984125041113\\
73.25	0.16986	-0.893134966137211\\
73.25	0.17352	-0.896288567682344\\
73.25	0.17718	-0.884444929676494\\
73.25	0.18084	-0.857604052119683\\
73.25	0.1845	-0.815765935011914\\
73.25	0.18816	-0.75893057835318\\
73.25	0.19182	-0.687097982143484\\
73.25	0.19548	-0.600268146382806\\
73.25	0.19914	-0.498441071071191\\
73.25	0.2028	-0.381616756208565\\
73.25	0.20646	-0.249795201795017\\
73.25	0.21012	-0.102976407830479\\
73.25	0.21378	0.0588396256850139\\
73.25	0.21744	0.235652898751471\\
73.25	0.2211	0.427463411368908\\
73.25	0.22476	0.634271163537299\\
73.25	0.22842	0.85607615525667\\
73.25	0.23208	1.09287838652699\\
73.25	0.23574	1.34467785734828\\
73.25	0.2394	1.61147456772053\\
73.25	0.24306	1.89326851764377\\
73.25	0.24672	2.19005970711797\\
73.25	0.25038	2.50184813614312\\
73.25	0.25404	2.82863380471922\\
73.25	0.2577	3.17041671284632\\
73.25	0.26136	3.52719686052436\\
73.25	0.26502	3.89897424775338\\
73.25	0.26868	4.28574887453337\\
73.25	0.27234	4.68752074086431\\
73.25	0.276	5.10428984674624\\
73.625	0.093	2.74836814535223\\
73.625	0.09666	2.42423328034674\\
73.625	0.10032	2.11509565489222\\
73.625	0.10398	1.82095526898866\\
73.625	0.10764	1.54181212263608\\
73.625	0.1113	1.27766621583445\\
73.625	0.11496	1.0285175485838\\
73.625	0.11862	0.794366120884104\\
73.625	0.12228	0.575211932735384\\
73.625	0.12594	0.371054984137615\\
73.625	0.1296	0.181895275090824\\
73.625	0.13326	0.00773280559499412\\
73.625	0.13692	-0.151432424349871\\
73.625	0.14058	-0.295600414743767\\
73.625	0.14424	-0.424771165586684\\
73.625	0.1479	-0.538944676878666\\
73.625	0.15156	-0.638120948619653\\
73.625	0.15522	-0.722299980809689\\
73.625	0.15888	-0.791481773448753\\
73.625	0.16254	-0.845666326536858\\
73.625	0.1662	-0.884853640073995\\
73.625	0.16986	-0.909043714060168\\
73.625	0.17352	-0.918236548495376\\
73.625	0.17718	-0.912432143379615\\
73.625	0.18084	-0.891630498712892\\
73.625	0.1845	-0.855831614495184\\
73.625	0.18816	-0.805035490726524\\
73.625	0.19182	-0.739242127406904\\
73.625	0.19548	-0.658451524536314\\
73.625	0.19914	-0.56266368211476\\
73.625	0.2028	-0.451878600142251\\
73.625	0.20646	-0.326096278618749\\
73.625	0.21012	-0.1853167175443\\
73.625	0.21378	-0.0295399169188677\\
73.625	0.21744	0.141234123257515\\
73.625	0.2211	0.327005402984863\\
73.625	0.22476	0.527773922263179\\
73.625	0.22842	0.743539681092475\\
73.625	0.23208	0.974302679472704\\
73.625	0.23574	1.22006291740393\\
73.625	0.2394	1.4808203948861\\
73.625	0.24306	1.75657511191926\\
73.625	0.24672	2.04732706850336\\
73.625	0.25038	2.35307626463845\\
73.625	0.25404	2.67382270032447\\
73.625	0.2577	3.0095663755615\\
73.625	0.26136	3.36030729034945\\
73.625	0.26502	3.72604544468841\\
73.625	0.26868	4.10678083857831\\
73.625	0.27234	4.5025134720192\\
73.625	0.276	4.91324334501103\\
74	0.093	2.86201014569312\\
74	0.09666	2.53183604779756\\
74	0.10032	2.21665918945295\\
74	0.10398	1.91647957065932\\
74	0.10764	1.63129719141666\\
74	0.1113	1.36111205172495\\
74	0.11496	1.10592415158422\\
74	0.11862	0.865733490994448\\
74	0.12228	0.640540069955646\\
74	0.12594	0.43034388846781\\
74	0.1296	0.235144946530937\\
74	0.13326	0.0549432441450328\\
74	0.13692	-0.110261218689907\\
74	0.14058	-0.260468441973877\\
74	0.14424	-0.395678425706883\\
74	0.1479	-0.515891169888926\\
74	0.15156	-0.621106674520002\\
74	0.15522	-0.711324939600127\\
74	0.15888	-0.786545965129266\\
74	0.16254	-0.846769751107445\\
74	0.1662	-0.891996297534657\\
74	0.16986	-0.922225604410905\\
74	0.17352	-0.937457671736187\\
74	0.17718	-0.937692499510501\\
74	0.18084	-0.922930087733853\\
74	0.1845	-0.893170436406233\\
74	0.18816	-0.848413545527663\\
74	0.19182	-0.788659415098117\\
74	0.19548	-0.713908045117602\\
74	0.19914	-0.624159435586122\\
74	0.2028	-0.519413586503674\\
74	0.20646	-0.399670497870261\\
74	0.21012	-0.264930169685886\\
74	0.21378	-0.115192601950543\\
74	0.21744	0.049542205335765\\
74	0.2211	0.229274252173038\\
74	0.22476	0.42400353856128\\
74	0.22842	0.633730064500487\\
74	0.23208	0.858453829990641\\
74	0.23574	1.09817483503179\\
74	0.2394	1.35289307962389\\
74	0.24306	1.62260856376696\\
74	0.24672	1.90732128746101\\
74	0.25038	2.20703125070599\\
74	0.25404	2.52173845350196\\
74	0.2577	2.8514428958489\\
74	0.26136	3.19614457774678\\
74	0.26502	3.55584349919565\\
74	0.26868	3.93053966019547\\
74	0.27234	4.32023306074628\\
74	0.276	4.72492370084803\\
};
\end{axis}

\begin{axis}[%
width=4.527496cm,
height=3.050847cm,
at={(0cm,16.949153cm)},
scale only axis,
xmin=56,
xmax=74,
tick align=outside,
xlabel={$L_{cut}$},
xmajorgrids,
ymin=0.093,
ymax=0.276,
ylabel={$D_{rlx}$},
ymajorgrids,
zmin=-146.970116734097,
zmax=0,
zlabel={$x_4,x_4$},
zmajorgrids,
view={-140}{50},
legend style={at={(1.03,1)},anchor=north west,legend cell align=left,align=left,draw=white!15!black}
]
\addplot3[only marks,mark=*,mark options={},mark size=1.5000pt,color=mycolor1] plot table[row sep=crcr,]{%
74	0.123	-18.9697907316621\\
72	0.113	-15.3298020423605\\
61	0.095	-8.33540680697621\\
56	0.093	-8.94144024470852\\
};
\addplot3[only marks,mark=*,mark options={},mark size=1.5000pt,color=mycolor2] plot table[row sep=crcr,]{%
67	0.276	-138.209692790868\\
66	0.255	-114.234836770513\\
62	0.209	-63.5813007613629\\
57	0.193	-50.3034815973187\\
};
\addplot3[only marks,mark=*,mark options={},mark size=1.5000pt,color=black] plot table[row sep=crcr,]{%
69	0.104	-11.3267714979703\\
};
\addplot3[only marks,mark=*,mark options={},mark size=1.5000pt,color=black] plot table[row sep=crcr,]{%
64	0.23	-84.7089912392218\\
};

\addplot3[%
surf,
opacity=0.7,
shader=interp,
colormap={mymap}{[1pt] rgb(0pt)=(0.0901961,0.239216,0.0745098); rgb(1pt)=(0.0945149,0.242058,0.0739522); rgb(2pt)=(0.0988592,0.244894,0.0733566); rgb(3pt)=(0.103229,0.247724,0.0727241); rgb(4pt)=(0.107623,0.250549,0.0720557); rgb(5pt)=(0.112043,0.253367,0.0713525); rgb(6pt)=(0.116487,0.25618,0.0706154); rgb(7pt)=(0.120956,0.258986,0.0698456); rgb(8pt)=(0.125449,0.261787,0.0690441); rgb(9pt)=(0.129967,0.264581,0.0682118); rgb(10pt)=(0.134508,0.26737,0.06735); rgb(11pt)=(0.139074,0.270152,0.0664596); rgb(12pt)=(0.143663,0.272929,0.0655416); rgb(13pt)=(0.148275,0.275699,0.0645971); rgb(14pt)=(0.152911,0.278463,0.0636271); rgb(15pt)=(0.15757,0.281221,0.0626328); rgb(16pt)=(0.162252,0.283973,0.0616151); rgb(17pt)=(0.166957,0.286719,0.060575); rgb(18pt)=(0.171685,0.289458,0.0595136); rgb(19pt)=(0.176434,0.292191,0.0584321); rgb(20pt)=(0.181207,0.294918,0.0573313); rgb(21pt)=(0.186001,0.297639,0.0562123); rgb(22pt)=(0.190817,0.300353,0.0550763); rgb(23pt)=(0.195655,0.303061,0.0539242); rgb(24pt)=(0.200514,0.305763,0.052757); rgb(25pt)=(0.205395,0.308459,0.0515759); rgb(26pt)=(0.210296,0.311149,0.0503624); rgb(27pt)=(0.215212,0.313846,0.0490067); rgb(28pt)=(0.220142,0.316548,0.0475043); rgb(29pt)=(0.22509,0.319254,0.0458704); rgb(30pt)=(0.230056,0.321962,0.0441205); rgb(31pt)=(0.235042,0.324671,0.04227); rgb(32pt)=(0.240048,0.327379,0.0403343); rgb(33pt)=(0.245078,0.330085,0.0383287); rgb(34pt)=(0.250131,0.332786,0.0362688); rgb(35pt)=(0.25521,0.335482,0.0341698); rgb(36pt)=(0.260317,0.33817,0.0320472); rgb(37pt)=(0.265451,0.340849,0.0299163); rgb(38pt)=(0.270616,0.343517,0.0277927); rgb(39pt)=(0.275813,0.346172,0.0256916); rgb(40pt)=(0.281043,0.348814,0.0236284); rgb(41pt)=(0.286307,0.35144,0.0216186); rgb(42pt)=(0.291607,0.354048,0.0196776); rgb(43pt)=(0.296945,0.356637,0.0178207); rgb(44pt)=(0.302322,0.359206,0.0160634); rgb(45pt)=(0.307739,0.361753,0.0144211); rgb(46pt)=(0.313198,0.364275,0.0129091); rgb(47pt)=(0.318701,0.366772,0.0115428); rgb(48pt)=(0.324249,0.369242,0.0103377); rgb(49pt)=(0.329843,0.371682,0.00930909); rgb(50pt)=(0.335485,0.374093,0.00847245); rgb(51pt)=(0.341176,0.376471,0.00784314); rgb(52pt)=(0.346925,0.378826,0.00732741); rgb(53pt)=(0.352735,0.381168,0.00682184); rgb(54pt)=(0.358605,0.383497,0.00632729); rgb(55pt)=(0.364532,0.385812,0.00584464); rgb(56pt)=(0.370516,0.388113,0.00537476); rgb(57pt)=(0.376552,0.390399,0.00491852); rgb(58pt)=(0.38264,0.39267,0.00447681); rgb(59pt)=(0.388777,0.394925,0.00405048); rgb(60pt)=(0.394962,0.397164,0.00364042); rgb(61pt)=(0.401191,0.399386,0.00324749); rgb(62pt)=(0.407464,0.401592,0.00287258); rgb(63pt)=(0.413777,0.40378,0.00251655); rgb(64pt)=(0.420129,0.40595,0.00218028); rgb(65pt)=(0.426518,0.408102,0.00186463); rgb(66pt)=(0.432942,0.410234,0.00157049); rgb(67pt)=(0.439399,0.412348,0.00129873); rgb(68pt)=(0.445885,0.414441,0.00105022); rgb(69pt)=(0.452401,0.416515,0.000825833); rgb(70pt)=(0.458942,0.418567,0.000626441); rgb(71pt)=(0.465508,0.420599,0.00045292); rgb(72pt)=(0.472096,0.422609,0.000306141); rgb(73pt)=(0.478704,0.424596,0.000186979); rgb(74pt)=(0.485331,0.426562,9.63073e-05); rgb(75pt)=(0.491973,0.428504,3.49981e-05); rgb(76pt)=(0.498628,0.430422,3.92506e-06); rgb(77pt)=(0.505323,0.432315,0); rgb(78pt)=(0.512206,0.434168,0); rgb(79pt)=(0.519282,0.435983,0); rgb(80pt)=(0.526529,0.437764,0); rgb(81pt)=(0.533922,0.439512,0); rgb(82pt)=(0.54144,0.441232,0); rgb(83pt)=(0.549059,0.442927,0); rgb(84pt)=(0.556756,0.444599,0); rgb(85pt)=(0.564508,0.446252,0); rgb(86pt)=(0.572292,0.447889,0); rgb(87pt)=(0.580084,0.449514,0); rgb(88pt)=(0.587863,0.451129,0); rgb(89pt)=(0.595604,0.452737,0); rgb(90pt)=(0.603284,0.454343,0); rgb(91pt)=(0.610882,0.455948,0); rgb(92pt)=(0.618373,0.457556,0); rgb(93pt)=(0.625734,0.459171,0); rgb(94pt)=(0.632943,0.460795,0); rgb(95pt)=(0.639976,0.462432,0); rgb(96pt)=(0.64681,0.464084,0); rgb(97pt)=(0.653423,0.465756,0); rgb(98pt)=(0.659791,0.46745,0); rgb(99pt)=(0.665891,0.469169,0); rgb(100pt)=(0.6717,0.470916,0); rgb(101pt)=(0.677195,0.472696,0); rgb(102pt)=(0.682353,0.47451,0); rgb(103pt)=(0.687242,0.476355,0); rgb(104pt)=(0.691952,0.478225,0); rgb(105pt)=(0.696497,0.480118,0); rgb(106pt)=(0.700887,0.482033,0); rgb(107pt)=(0.705134,0.483968,0); rgb(108pt)=(0.709251,0.485921,0); rgb(109pt)=(0.713249,0.487891,0); rgb(110pt)=(0.71714,0.489876,0); rgb(111pt)=(0.720936,0.491875,0); rgb(112pt)=(0.724649,0.493887,0); rgb(113pt)=(0.72829,0.495909,0); rgb(114pt)=(0.731872,0.49794,0); rgb(115pt)=(0.735406,0.499979,0); rgb(116pt)=(0.738904,0.502025,0); rgb(117pt)=(0.742378,0.504075,0); rgb(118pt)=(0.74584,0.506128,0); rgb(119pt)=(0.749302,0.508182,0); rgb(120pt)=(0.752775,0.510237,0); rgb(121pt)=(0.756272,0.51229,0); rgb(122pt)=(0.759804,0.514339,0); rgb(123pt)=(0.763384,0.516385,0); rgb(124pt)=(0.767022,0.518424,0); rgb(125pt)=(0.770731,0.520455,0); rgb(126pt)=(0.774523,0.522478,0); rgb(127pt)=(0.77841,0.524489,0); rgb(128pt)=(0.782391,0.526491,0); rgb(129pt)=(0.786402,0.528496,0); rgb(130pt)=(0.790431,0.530506,0); rgb(131pt)=(0.794478,0.532521,0); rgb(132pt)=(0.798541,0.534539,0); rgb(133pt)=(0.802619,0.53656,0); rgb(134pt)=(0.806712,0.538584,0); rgb(135pt)=(0.81082,0.540609,0); rgb(136pt)=(0.81494,0.542635,0); rgb(137pt)=(0.819074,0.54466,0); rgb(138pt)=(0.823219,0.546686,0); rgb(139pt)=(0.827374,0.548709,0); rgb(140pt)=(0.831541,0.55073,0); rgb(141pt)=(0.835716,0.552749,0); rgb(142pt)=(0.8399,0.554763,0); rgb(143pt)=(0.844092,0.556774,0); rgb(144pt)=(0.848292,0.558779,0); rgb(145pt)=(0.852497,0.560778,0); rgb(146pt)=(0.856708,0.562771,0); rgb(147pt)=(0.860924,0.564756,0); rgb(148pt)=(0.865143,0.566733,0); rgb(149pt)=(0.869366,0.568701,0); rgb(150pt)=(0.873592,0.57066,0); rgb(151pt)=(0.877819,0.572608,0); rgb(152pt)=(0.882047,0.574545,0); rgb(153pt)=(0.886275,0.576471,0); rgb(154pt)=(0.890659,0.578362,0); rgb(155pt)=(0.895333,0.580203,0); rgb(156pt)=(0.900258,0.581999,0); rgb(157pt)=(0.905397,0.583755,0); rgb(158pt)=(0.910711,0.585479,0); rgb(159pt)=(0.916164,0.587176,0); rgb(160pt)=(0.921717,0.588852,0); rgb(161pt)=(0.927333,0.590513,0); rgb(162pt)=(0.932974,0.592166,0); rgb(163pt)=(0.938602,0.593815,0); rgb(164pt)=(0.94418,0.595468,0); rgb(165pt)=(0.949669,0.59713,0); rgb(166pt)=(0.955033,0.598808,0); rgb(167pt)=(0.960233,0.600507,0); rgb(168pt)=(0.965232,0.602233,0); rgb(169pt)=(0.969992,0.603992,0); rgb(170pt)=(0.974475,0.605791,0); rgb(171pt)=(0.978643,0.607636,0); rgb(172pt)=(0.98246,0.609532,0); rgb(173pt)=(0.985886,0.611486,0); rgb(174pt)=(0.988885,0.613503,0); rgb(175pt)=(0.991419,0.61559,0); rgb(176pt)=(0.99345,0.617753,0); rgb(177pt)=(0.99494,0.619997,0); rgb(178pt)=(0.995851,0.622329,0); rgb(179pt)=(0.996226,0.624763,0); rgb(180pt)=(0.996512,0.627352,0); rgb(181pt)=(0.996788,0.630095,0); rgb(182pt)=(0.997053,0.632982,0); rgb(183pt)=(0.997308,0.636004,0); rgb(184pt)=(0.997552,0.639152,0); rgb(185pt)=(0.997785,0.642416,0); rgb(186pt)=(0.998006,0.645786,0); rgb(187pt)=(0.998217,0.649253,0); rgb(188pt)=(0.998416,0.652807,0); rgb(189pt)=(0.998605,0.656439,0); rgb(190pt)=(0.998781,0.660138,0); rgb(191pt)=(0.998946,0.663897,0); rgb(192pt)=(0.9991,0.667704,0); rgb(193pt)=(0.999242,0.67155,0); rgb(194pt)=(0.999372,0.675427,0); rgb(195pt)=(0.99949,0.679323,0); rgb(196pt)=(0.999596,0.68323,0); rgb(197pt)=(0.99969,0.687139,0); rgb(198pt)=(0.999771,0.691039,0); rgb(199pt)=(0.999841,0.694921,0); rgb(200pt)=(0.999898,0.698775,0); rgb(201pt)=(0.999942,0.702592,0); rgb(202pt)=(0.999974,0.706363,0); rgb(203pt)=(0.999994,0.710077,0); rgb(204pt)=(1,0.713725,0); rgb(205pt)=(1,0.717341,0); rgb(206pt)=(1,0.720963,0); rgb(207pt)=(1,0.724591,0); rgb(208pt)=(1,0.728226,0); rgb(209pt)=(1,0.731867,0); rgb(210pt)=(1,0.735514,0); rgb(211pt)=(1,0.739167,0); rgb(212pt)=(1,0.742827,0); rgb(213pt)=(1,0.746493,0); rgb(214pt)=(1,0.750165,0); rgb(215pt)=(1,0.753843,0); rgb(216pt)=(1,0.757527,0); rgb(217pt)=(1,0.761217,0); rgb(218pt)=(1,0.764913,0); rgb(219pt)=(1,0.768615,0); rgb(220pt)=(1,0.772324,0); rgb(221pt)=(1,0.776038,0); rgb(222pt)=(1,0.779758,0); rgb(223pt)=(1,0.783484,0); rgb(224pt)=(1,0.787215,0); rgb(225pt)=(1,0.790953,0); rgb(226pt)=(1,0.794696,0); rgb(227pt)=(1,0.798445,0); rgb(228pt)=(1,0.8022,0); rgb(229pt)=(1,0.805961,0); rgb(230pt)=(1,0.809727,0); rgb(231pt)=(1,0.8135,0); rgb(232pt)=(1,0.817278,0); rgb(233pt)=(1,0.821063,0); rgb(234pt)=(1,0.824854,0); rgb(235pt)=(1,0.828652,0); rgb(236pt)=(1,0.832455,0); rgb(237pt)=(1,0.836265,0); rgb(238pt)=(1,0.840081,0); rgb(239pt)=(1,0.843903,0); rgb(240pt)=(1,0.847732,0); rgb(241pt)=(1,0.851566,0); rgb(242pt)=(1,0.855406,0); rgb(243pt)=(1,0.859253,0); rgb(244pt)=(1,0.863106,0); rgb(245pt)=(1,0.866964,0); rgb(246pt)=(1,0.870829,0); rgb(247pt)=(1,0.8747,0); rgb(248pt)=(1,0.878577,0); rgb(249pt)=(1,0.88246,0); rgb(250pt)=(1,0.886349,0); rgb(251pt)=(1,0.890243,0); rgb(252pt)=(1,0.894144,0); rgb(253pt)=(1,0.898051,0); rgb(254pt)=(1,0.901964,0); rgb(255pt)=(1,0.905882,0)},
mesh/rows=49]
table[row sep=crcr,header=false] {%
%
56	0.093	-8.86715325810529\\
56	0.09666	-9.28168239953796\\
56	0.10032	-9.77879391477874\\
56	0.10398	-10.3584878038276\\
56	0.10764	-11.0207640666846\\
56	0.1113	-11.7656227033497\\
56	0.11496	-12.5930637138229\\
56	0.11862	-13.5030870981042\\
56	0.12228	-14.4956928561937\\
56	0.12594	-15.5708809880912\\
56	0.1296	-16.7286514937968\\
56	0.13326	-17.9690043733106\\
56	0.13692	-19.2919396266324\\
56	0.14058	-20.6974572537624\\
56	0.14424	-22.1855572547005\\
56	0.1479	-23.7562396294466\\
56	0.15156	-25.4095043780009\\
56	0.15522	-27.1453515003633\\
56	0.15888	-28.9637809965338\\
56	0.16254	-30.8647928665124\\
56	0.1662	-32.8483871102991\\
56	0.16986	-34.9145637278939\\
56	0.17352	-37.0633227192969\\
56	0.17718	-39.2946640845079\\
56	0.18084	-41.608587823527\\
56	0.1845	-44.0050939363543\\
56	0.18816	-46.4841824229897\\
56	0.19182	-49.0458532834331\\
56	0.19548	-51.6901065176847\\
56	0.19914	-54.4169421257444\\
56	0.2028	-57.2263601076121\\
56	0.20646	-60.1183604632881\\
56	0.21012	-63.092943192772\\
56	0.21378	-66.1501082960641\\
56	0.21744	-69.2898557731643\\
56	0.2211	-72.5121856240726\\
56	0.22476	-75.8170978487891\\
56	0.22842	-79.2045924473137\\
56	0.23208	-82.6746694196462\\
56	0.23574	-86.227328765787\\
56	0.2394	-89.8625704857359\\
56	0.24306	-93.5803945794928\\
56	0.24672	-97.3808010470579\\
56	0.25038	-101.263789888431\\
56	0.25404	-105.229361103612\\
56	0.2577	-109.277514692602\\
56	0.26136	-113.408250655399\\
56	0.26502	-117.621568992005\\
56	0.26868	-121.917469702419\\
56	0.27234	-126.29595278664\\
56	0.276	-130.75701824467\\
56.375	0.093	-8.78055049202328\\
56.375	0.09666	-9.20085678952984\\
56.375	0.10032	-9.70374546084451\\
56.375	0.10398	-10.2892165059673\\
56.375	0.10764	-10.9572699248982\\
56.375	0.1113	-11.7079057176372\\
56.375	0.11496	-12.5411238841843\\
56.375	0.11862	-13.4569244245395\\
56.375	0.12228	-14.4553073387028\\
56.375	0.12594	-15.5362726266743\\
56.375	0.1296	-16.6998202884538\\
56.375	0.13326	-17.9459503240414\\
56.375	0.13692	-19.2746627334372\\
56.375	0.14058	-20.685957516641\\
56.375	0.14424	-22.179834673653\\
56.375	0.1479	-23.7562942044731\\
56.375	0.15156	-25.4153361091012\\
56.375	0.15522	-27.1569603875375\\
56.375	0.15888	-28.9811670397819\\
56.375	0.16254	-30.8879560658344\\
56.375	0.1662	-32.877327465695\\
56.375	0.16986	-34.9492812393637\\
56.375	0.17352	-37.1038173868406\\
56.375	0.17718	-39.3409359081255\\
56.375	0.18084	-41.6606368032185\\
56.375	0.1845	-44.0629200721197\\
56.375	0.18816	-46.547785714829\\
56.375	0.19182	-49.1152337313463\\
56.375	0.19548	-51.7652641216718\\
56.375	0.19914	-54.4978768858053\\
56.375	0.2028	-57.313072023747\\
56.375	0.20646	-60.2108495354968\\
56.375	0.21012	-63.1912094210547\\
56.375	0.21378	-66.2541516804207\\
56.375	0.21744	-69.3996763135948\\
56.375	0.2211	-72.627783320577\\
56.375	0.22476	-75.9384727013673\\
56.375	0.22842	-79.3317444559658\\
56.375	0.23208	-82.8075985843723\\
56.375	0.23574	-86.366035086587\\
56.375	0.2394	-90.0070539626098\\
56.375	0.24306	-93.7306552124406\\
56.375	0.24672	-97.5368388360796\\
56.375	0.25038	-101.425604833527\\
56.375	0.25404	-105.396953204782\\
56.375	0.2577	-109.450883949845\\
56.375	0.26136	-113.587397068717\\
56.375	0.26502	-117.806492561396\\
56.375	0.26868	-122.108170427884\\
56.375	0.27234	-126.492430668179\\
56.375	0.276	-130.959273282283\\
56.75	0.093	-8.69971444285972\\
56.75	0.09666	-9.12579789644018\\
56.75	0.10032	-9.63446372382875\\
56.75	0.10398	-10.2257119250254\\
56.75	0.10764	-10.8995425000302\\
56.75	0.1113	-11.6559554488431\\
56.75	0.11496	-12.4949507714641\\
56.75	0.11862	-13.4165284678932\\
56.75	0.12228	-14.4206885381304\\
56.75	0.12594	-15.5074309821758\\
56.75	0.1296	-16.6767558000292\\
56.75	0.13326	-17.9286629916907\\
56.75	0.13692	-19.2631525571604\\
56.75	0.14058	-20.6802244964381\\
56.75	0.14424	-22.179878809524\\
56.75	0.1479	-23.762115496418\\
56.75	0.15156	-25.42693455712\\
56.75	0.15522	-27.1743359916302\\
56.75	0.15888	-29.0043197999485\\
56.75	0.16254	-30.9168859820749\\
56.75	0.1662	-32.9120345380094\\
56.75	0.16986	-34.989765467752\\
56.75	0.17352	-37.1500787713027\\
56.75	0.17718	-39.3929744486616\\
56.75	0.18084	-41.7184524998285\\
56.75	0.1845	-44.1265129248035\\
56.75	0.18816	-46.6171557235867\\
56.75	0.19182	-49.1903808961779\\
56.75	0.19548	-51.8461884425773\\
56.75	0.19914	-54.5845783627848\\
56.75	0.2028	-57.4055506568003\\
56.75	0.20646	-60.3091053246241\\
56.75	0.21012	-63.2952423662558\\
56.75	0.21378	-66.3639617816957\\
56.75	0.21744	-69.5152635709437\\
56.75	0.2211	-72.7491477339998\\
56.75	0.22476	-76.0656142708641\\
56.75	0.22842	-79.4646631815364\\
56.75	0.23208	-82.9462944660168\\
56.75	0.23574	-86.5105081243054\\
56.75	0.2394	-90.1573041564021\\
56.75	0.24306	-93.8866825623068\\
56.75	0.24672	-97.6986433420197\\
56.75	0.25038	-101.593186495541\\
56.75	0.25404	-105.57031202287\\
56.75	0.2577	-109.630019924007\\
56.75	0.26136	-113.772310198952\\
56.75	0.26502	-117.997182847706\\
56.75	0.26868	-122.304637870267\\
56.75	0.27234	-126.694675266637\\
56.75	0.276	-131.167295036814\\
57.125	0.093	-8.62464511061461\\
57.125	0.09666	-9.05650572026897\\
57.125	0.10032	-9.57094870373144\\
57.125	0.10398	-10.167974061002\\
57.125	0.10764	-10.8475817920807\\
57.125	0.1113	-11.6097718969675\\
57.125	0.11496	-12.4545443756624\\
57.125	0.11862	-13.3818992281654\\
57.125	0.12228	-14.3918364544765\\
57.125	0.12594	-15.4843560545958\\
57.125	0.1296	-16.6594580285231\\
57.125	0.13326	-17.9171423762585\\
57.125	0.13692	-19.257409097802\\
57.125	0.14058	-20.6802581931537\\
57.125	0.14424	-22.1856896623134\\
57.125	0.1479	-23.7737035052813\\
57.125	0.15156	-25.4442997220573\\
57.125	0.15522	-27.1974783126414\\
57.125	0.15888	-29.0332392770336\\
57.125	0.16254	-30.9515826152338\\
57.125	0.1662	-32.9525083272423\\
57.125	0.16986	-35.0360164130587\\
57.125	0.17352	-37.2021068726834\\
57.125	0.17718	-39.4507797061161\\
57.125	0.18084	-41.7820349133569\\
57.125	0.1845	-44.1958724944058\\
57.125	0.18816	-46.6922924492629\\
57.125	0.19182	-49.271294777928\\
57.125	0.19548	-51.9328794804013\\
57.125	0.19914	-54.6770465566827\\
57.125	0.2028	-57.5037960067721\\
57.125	0.20646	-60.4131278306697\\
57.125	0.21012	-63.4050420283754\\
57.125	0.21378	-66.4795385998892\\
57.125	0.21744	-69.6366175452111\\
57.125	0.2211	-72.8762788643411\\
57.125	0.22476	-76.1985225572792\\
57.125	0.22842	-79.6033486240255\\
57.125	0.23208	-83.0907570645798\\
57.125	0.23574	-86.6607478789423\\
57.125	0.2394	-90.3133210671128\\
57.125	0.24306	-94.0484766290914\\
57.125	0.24672	-97.8662145648782\\
57.125	0.25038	-101.766534874473\\
57.125	0.25404	-105.749437557876\\
57.125	0.2577	-109.814922615087\\
57.125	0.26136	-113.962990046106\\
57.125	0.26502	-118.193639850934\\
57.125	0.26868	-122.506872029569\\
57.125	0.27234	-126.902686582013\\
57.125	0.276	-131.381083508264\\
57.5	0.093	-8.55534249528796\\
57.5	0.09666	-8.99298026101621\\
57.5	0.10032	-9.51320040055258\\
57.5	0.10398	-10.116002913897\\
57.5	0.10764	-10.8013878010496\\
57.5	0.1113	-11.5693550620103\\
57.5	0.11496	-12.4199046967791\\
57.5	0.11862	-13.353036705356\\
57.5	0.12228	-14.368751087741\\
57.5	0.12594	-15.4670478439342\\
57.5	0.1296	-16.6479269739354\\
57.5	0.13326	-17.9113884777447\\
57.5	0.13692	-19.2574323553621\\
57.5	0.14058	-20.6860586067877\\
57.5	0.14424	-22.1972672320213\\
57.5	0.1479	-23.7910582310631\\
57.5	0.15156	-25.467431603913\\
57.5	0.15522	-27.2263873505709\\
57.5	0.15888	-29.067925471037\\
57.5	0.16254	-30.9920459653112\\
57.5	0.1662	-32.9987488333935\\
57.5	0.16986	-35.0880340752839\\
57.5	0.17352	-37.2599016909824\\
57.5	0.17718	-39.5143516804891\\
57.5	0.18084	-41.8513840438038\\
57.5	0.1845	-44.2709987809266\\
57.5	0.18816	-46.7731958918576\\
57.5	0.19182	-49.3579753765966\\
57.5	0.19548	-52.0253372351437\\
57.5	0.19914	-54.775281467499\\
57.5	0.2028	-57.6078080736624\\
57.5	0.20646	-60.5229170536339\\
57.5	0.21012	-63.5206084074135\\
57.5	0.21378	-66.6008821350011\\
57.5	0.21744	-69.7637382363969\\
57.5	0.2211	-73.0091767116008\\
57.5	0.22476	-76.3371975606129\\
57.5	0.22842	-79.747800783433\\
57.5	0.23208	-83.2409863800612\\
57.5	0.23574	-86.8167543504976\\
57.5	0.2394	-90.475104694742\\
57.5	0.24306	-94.2160374127945\\
57.5	0.24672	-98.0395525046552\\
57.5	0.25038	-101.945649970324\\
57.5	0.25404	-105.934329809801\\
57.5	0.2577	-110.005592023086\\
57.5	0.26136	-114.159436610179\\
57.5	0.26502	-118.39586357108\\
57.5	0.26868	-122.714872905789\\
57.5	0.27234	-127.116464614307\\
57.5	0.276	-131.600638696632\\
57.875	0.093	-8.49180659687971\\
57.875	0.09666	-8.93522151868187\\
57.875	0.10032	-9.46121881429213\\
57.875	0.10398	-10.0697984837105\\
57.875	0.10764	-10.760960526937\\
57.875	0.1113	-11.5347049439716\\
57.875	0.11496	-12.3910317348143\\
57.875	0.11862	-13.3299408994651\\
57.875	0.12228	-14.351432437924\\
57.875	0.12594	-15.455506350191\\
57.875	0.1296	-16.6421626362661\\
57.875	0.13326	-17.9114012961493\\
57.875	0.13692	-19.2632223298406\\
57.875	0.14058	-20.6976257373401\\
57.875	0.14424	-22.2146115186477\\
57.875	0.1479	-23.8141796737633\\
57.875	0.15156	-25.496330202687\\
57.875	0.15522	-27.2610631054189\\
57.875	0.15888	-29.1083783819589\\
57.875	0.16254	-31.038276032307\\
57.875	0.1662	-33.0507560564632\\
57.875	0.16986	-35.1458184544275\\
57.875	0.17352	-37.3234632261999\\
57.875	0.17718	-39.5836903717804\\
57.875	0.18084	-41.926499891169\\
57.875	0.1845	-44.3518917843658\\
57.875	0.18816	-46.8598660513706\\
57.875	0.19182	-49.4504226921836\\
57.875	0.19548	-52.1235617068047\\
57.875	0.19914	-54.8792830952337\\
57.875	0.2028	-57.7175868574711\\
57.875	0.20646	-60.6384729935165\\
57.875	0.21012	-63.6419415033699\\
57.875	0.21378	-66.7279923870315\\
57.875	0.21744	-69.8966256445012\\
57.875	0.2211	-73.147841275779\\
57.875	0.22476	-76.4816392808649\\
57.875	0.22842	-79.8980196597589\\
57.875	0.23208	-83.396982412461\\
57.875	0.23574	-86.9785275389713\\
57.875	0.2394	-90.6426550392896\\
57.875	0.24306	-94.3893649134161\\
57.875	0.24672	-98.2186571613507\\
57.875	0.25038	-102.130531783093\\
57.875	0.25404	-106.124988778644\\
57.875	0.2577	-110.202028148003\\
57.875	0.26136	-114.36164989117\\
57.875	0.26502	-118.603854008145\\
57.875	0.26868	-122.928640498928\\
57.875	0.27234	-127.33600936352\\
57.875	0.276	-131.825960601919\\
58.25	0.093	-8.43403741538997\\
58.25	0.09666	-8.88322949326601\\
58.25	0.10032	-9.41500394495016\\
58.25	0.10398	-10.0293607704424\\
58.25	0.10764	-10.7262999697428\\
58.25	0.1113	-11.5058215428513\\
58.25	0.11496	-12.3679254897679\\
58.25	0.11862	-13.3126118104926\\
58.25	0.12228	-14.3398805050254\\
58.25	0.12594	-15.4497315733663\\
58.25	0.1296	-16.6421650155153\\
58.25	0.13326	-17.9171808314724\\
58.25	0.13692	-19.2747790212376\\
58.25	0.14058	-20.714959584811\\
58.25	0.14424	-22.2377225221925\\
58.25	0.1479	-23.843067833382\\
58.25	0.15156	-25.5309955183796\\
58.25	0.15522	-27.3015055771854\\
58.25	0.15888	-29.1545980097993\\
58.25	0.16254	-31.0902728162213\\
58.25	0.1662	-33.1085299964514\\
58.25	0.16986	-35.2093695504896\\
58.25	0.17352	-37.3927914783359\\
58.25	0.17718	-39.6587957799903\\
58.25	0.18084	-42.0073824554528\\
58.25	0.1845	-44.4385515047234\\
58.25	0.18816	-46.9523029278022\\
58.25	0.19182	-49.548636724689\\
58.25	0.19548	-52.227552895384\\
58.25	0.19914	-54.989051439887\\
58.25	0.2028	-57.8331323581982\\
58.25	0.20646	-60.7597956503175\\
58.25	0.21012	-63.7690413162448\\
58.25	0.21378	-66.8608693559803\\
58.25	0.21744	-70.0352797695239\\
58.25	0.2211	-73.2922725568756\\
58.25	0.22476	-76.6318477180354\\
58.25	0.22842	-80.0540052530033\\
58.25	0.23208	-83.5587451617794\\
58.25	0.23574	-87.1460674443635\\
58.25	0.2394	-90.8159721007557\\
58.25	0.24306	-94.5684591309561\\
58.25	0.24672	-98.4035285349646\\
58.25	0.25038	-102.321180312781\\
58.25	0.25404	-106.321414464406\\
58.25	0.2577	-110.404230989839\\
58.25	0.26136	-114.569629889079\\
58.25	0.26502	-118.817611162128\\
58.25	0.26868	-123.148174808985\\
58.25	0.27234	-127.561320829651\\
58.25	0.276	-132.057049224124\\
58.625	0.093	-8.38203495081865\\
58.625	0.09666	-8.8370041847686\\
58.625	0.10032	-9.37455579252666\\
58.625	0.10398	-9.99468977409281\\
58.625	0.10764	-10.6974061294671\\
58.625	0.1113	-11.4827048586495\\
58.625	0.11496	-12.35058596164\\
58.625	0.11862	-13.3010494384386\\
58.625	0.12228	-14.3340952890452\\
58.625	0.12594	-15.44972351346\\
58.625	0.1296	-16.647934111683\\
58.625	0.13326	-17.928727083714\\
58.625	0.13692	-19.2921024295531\\
58.625	0.14058	-20.7380601492004\\
58.625	0.14424	-22.2666002426557\\
58.625	0.1479	-23.8777227099191\\
58.625	0.15156	-25.5714275509907\\
58.625	0.15522	-27.3477147658704\\
58.625	0.15888	-29.2065843545581\\
58.625	0.16254	-31.148036317054\\
58.625	0.1662	-33.172070653358\\
58.625	0.16986	-35.2786873634701\\
58.625	0.17352	-37.4678864473903\\
58.625	0.17718	-39.7396679051186\\
58.625	0.18084	-42.094031736655\\
58.625	0.1845	-44.5309779419996\\
58.625	0.18816	-47.0505065211522\\
58.625	0.19182	-49.6526174741129\\
58.625	0.19548	-52.3373108008818\\
58.625	0.19914	-55.1045865014587\\
58.625	0.2028	-57.9544445758438\\
58.625	0.20646	-60.886885024037\\
58.625	0.21012	-63.9019078460382\\
58.625	0.21378	-66.9995130418476\\
58.625	0.21744	-70.1797006114651\\
58.625	0.2211	-73.4424705548907\\
58.625	0.22476	-76.7878228721243\\
58.625	0.22842	-80.2157575631662\\
58.625	0.23208	-83.7262746280161\\
58.625	0.23574	-87.3193740666742\\
58.625	0.2394	-90.9950558791403\\
58.625	0.24306	-94.7533200654145\\
58.625	0.24672	-98.5941666254969\\
58.625	0.25038	-102.517595559387\\
58.625	0.25404	-106.523606867086\\
58.625	0.2577	-110.612200548593\\
58.625	0.26136	-114.783376603907\\
58.625	0.26502	-119.03713503303\\
58.625	0.26868	-123.373475835961\\
58.625	0.27234	-127.7923990127\\
58.625	0.276	-132.293904563248\\
59	0.093	-8.33579920316579\\
59	0.09666	-8.79654559318963\\
59	0.10032	-9.33987435702158\\
59	0.10398	-9.96578549466163\\
59	0.10764	-10.6742790061098\\
59	0.1113	-11.4653548913661\\
59	0.11496	-12.3390131504305\\
59	0.11862	-13.295253783303\\
59	0.12228	-14.3340767899836\\
59	0.12594	-15.4554821704722\\
59	0.1296	-16.6594699247691\\
59	0.13326	-17.946040052874\\
59	0.13692	-19.315192554787\\
59	0.14058	-20.7669274305081\\
59	0.14424	-22.3012446800374\\
59	0.1479	-23.9181443033747\\
59	0.15156	-25.6176263005201\\
59	0.15522	-27.3996906714737\\
59	0.15888	-29.2643374162354\\
59	0.16254	-31.2115665348052\\
59	0.1662	-33.2413780271831\\
59	0.16986	-35.3537718933691\\
59	0.17352	-37.5487481333632\\
59	0.17718	-39.8263067471653\\
59	0.18084	-42.1864477347757\\
59	0.1845	-44.6291710961941\\
59	0.18816	-47.1544768314206\\
59	0.19182	-49.7623649404553\\
59	0.19548	-52.452835423298\\
59	0.19914	-55.2258882799488\\
59	0.2028	-58.0815235104078\\
59	0.20646	-61.0197411146749\\
59	0.21012	-64.04054109275\\
59	0.21378	-67.1439234446333\\
59	0.21744	-70.3298881703247\\
59	0.2211	-73.5984352698242\\
59	0.22476	-76.9495647431318\\
59	0.22842	-80.3832765902475\\
59	0.23208	-83.8995708111713\\
59	0.23574	-87.4984474059033\\
59	0.2394	-91.1799063744433\\
59	0.24306	-94.9439477167914\\
59	0.24672	-98.7905714329477\\
59	0.25038	-102.719777522912\\
59	0.25404	-106.731565986684\\
59	0.2577	-110.825936824265\\
59	0.26136	-115.002890035654\\
59	0.26502	-119.26242562085\\
59	0.26868	-123.604543579855\\
59	0.27234	-128.029243912668\\
59	0.276	-132.536526619289\\
59.375	0.093	-8.29533017243139\\
59.375	0.09666	-8.76185371852914\\
59.375	0.10032	-9.31095963843497\\
59.375	0.10398	-9.94264793214893\\
59.375	0.10764	-10.656918599671\\
59.375	0.1113	-11.4537716410012\\
59.375	0.11496	-12.3332070561395\\
59.375	0.11862	-13.2952248450858\\
59.375	0.12228	-14.3398250078403\\
59.375	0.12594	-15.4670075444029\\
59.375	0.1296	-16.6767724547736\\
59.375	0.13326	-17.9691197389525\\
59.375	0.13692	-19.3440493969394\\
59.375	0.14058	-20.8015614287344\\
59.375	0.14424	-22.3416558343375\\
59.375	0.1479	-23.9643326137488\\
59.375	0.15156	-25.6695917669681\\
59.375	0.15522	-27.4574332939956\\
59.375	0.15888	-29.3278571948311\\
59.375	0.16254	-31.2808634694748\\
59.375	0.1662	-33.3164521179266\\
59.375	0.16986	-35.4346231401865\\
59.375	0.17352	-37.6353765362545\\
59.375	0.17718	-39.9187123061306\\
59.375	0.18084	-42.2846304498148\\
59.375	0.1845	-44.7331309673071\\
59.375	0.18816	-47.2642138586075\\
59.375	0.19182	-49.8778791237161\\
59.375	0.19548	-52.5741267626327\\
59.375	0.19914	-55.3529567753574\\
59.375	0.2028	-58.2143691618903\\
59.375	0.20646	-61.1583639222313\\
59.375	0.21012	-64.1849410563803\\
59.375	0.21378	-67.2941005643375\\
59.375	0.21744	-70.4858424461028\\
59.375	0.2211	-73.7601667016762\\
59.375	0.22476	-77.1170733310577\\
59.375	0.22842	-80.5565623342473\\
59.375	0.23208	-84.078633711245\\
59.375	0.23574	-87.6832874620508\\
59.375	0.2394	-91.3705235866648\\
59.375	0.24306	-95.1403420850868\\
59.375	0.24672	-98.992742957317\\
59.375	0.25038	-102.927726203355\\
59.375	0.25404	-106.945291823202\\
59.375	0.2577	-111.045439816856\\
59.375	0.26136	-115.228170184319\\
59.375	0.26502	-119.493482925589\\
59.375	0.26868	-123.841378040668\\
59.375	0.27234	-128.271855529555\\
59.375	0.276	-132.78491539225\\
59.75	0.093	-8.26062785861542\\
59.75	0.09666	-8.73292856078707\\
59.75	0.10032	-9.28781163676681\\
59.75	0.10398	-9.92527708655466\\
59.75	0.10764	-10.6453249101506\\
59.75	0.1113	-11.4479551075547\\
59.75	0.11496	-12.3331676787669\\
59.75	0.11862	-13.3009626237872\\
59.75	0.12228	-14.3513399426155\\
59.75	0.12594	-15.484299635252\\
59.75	0.1296	-16.6998417016966\\
59.75	0.13326	-17.9979661419493\\
59.75	0.13692	-19.3786729560101\\
59.75	0.14058	-20.8419621438791\\
59.75	0.14424	-22.3878337055561\\
59.75	0.1479	-24.0162876410413\\
59.75	0.15156	-25.7273239503345\\
59.75	0.15522	-27.5209426334359\\
59.75	0.15888	-29.3971436903453\\
59.75	0.16254	-31.3559271210629\\
59.75	0.1662	-33.3972929255886\\
59.75	0.16986	-35.5212411039224\\
59.75	0.17352	-37.7277716560642\\
59.75	0.17718	-40.0168845820142\\
59.75	0.18084	-42.3885798817724\\
59.75	0.1845	-44.8428575553386\\
59.75	0.18816	-47.3797176027129\\
59.75	0.19182	-49.9991600238953\\
59.75	0.19548	-52.7011848188859\\
59.75	0.19914	-55.4857919876845\\
59.75	0.2028	-58.3529815302913\\
59.75	0.20646	-61.3027534467061\\
59.75	0.21012	-64.3351077369291\\
59.75	0.21378	-67.4500444009601\\
59.75	0.21744	-70.6475634387993\\
59.75	0.2211	-73.9276648504466\\
59.75	0.22476	-77.290348635902\\
59.75	0.22842	-80.7356147951655\\
59.75	0.23208	-84.2634633282371\\
59.75	0.23574	-87.8738942351168\\
59.75	0.2394	-91.5669075158047\\
59.75	0.24306	-95.3425031703006\\
59.75	0.24672	-99.2006811986046\\
59.75	0.25038	-103.141441600717\\
59.75	0.25404	-107.164784376637\\
59.75	0.2577	-111.270709526365\\
59.75	0.26136	-115.459217049902\\
59.75	0.26502	-119.730306947246\\
59.75	0.26868	-124.083979218399\\
59.75	0.27234	-128.52023386336\\
59.75	0.276	-133.039070882129\\
60.125	0.093	-8.23169226171792\\
60.125	0.09666	-8.70977011996347\\
60.125	0.10032	-9.27043035201709\\
60.125	0.10398	-9.91367295787884\\
60.125	0.10764	-10.6394979375487\\
60.125	0.1113	-11.4479052910267\\
60.125	0.11496	-12.3388950183127\\
60.125	0.11862	-13.3124671194069\\
60.125	0.12228	-14.3686215943092\\
60.125	0.12594	-15.5073584430196\\
60.125	0.1296	-16.7286776655381\\
60.125	0.13326	-18.0325792618647\\
60.125	0.13692	-19.4190632319994\\
60.125	0.14058	-20.8881295759422\\
60.125	0.14424	-22.4397782936932\\
60.125	0.1479	-24.0740093852522\\
60.125	0.15156	-25.7908228506193\\
60.125	0.15522	-27.5902186897946\\
60.125	0.15888	-29.472196902778\\
60.125	0.16254	-31.4367574895694\\
60.125	0.1662	-33.483900450169\\
60.125	0.16986	-35.6136257845767\\
60.125	0.17352	-37.8259334927925\\
60.125	0.17718	-40.1208235748164\\
60.125	0.18084	-42.4982960306484\\
60.125	0.1845	-44.9583508602885\\
60.125	0.18816	-47.5009880637367\\
60.125	0.19182	-50.126207640993\\
60.125	0.19548	-52.8340095920575\\
60.125	0.19914	-55.62439391693\\
60.125	0.2028	-58.4973606156107\\
60.125	0.20646	-61.4529096880994\\
60.125	0.21012	-64.4910411343962\\
60.125	0.21378	-67.6117549545012\\
60.125	0.21744	-70.8150511484143\\
60.125	0.2211	-74.1009297161355\\
60.125	0.22476	-77.4693906576647\\
60.125	0.22842	-80.9204339730021\\
60.125	0.23208	-84.4540596621477\\
60.125	0.23574	-88.0702677251013\\
60.125	0.2394	-91.769058161863\\
60.125	0.24306	-95.5504309724328\\
60.125	0.24672	-99.4143861568108\\
60.125	0.25038	-103.360923714997\\
60.125	0.25404	-107.390043646991\\
60.125	0.2577	-111.501745952793\\
60.125	0.26136	-115.696030632404\\
60.125	0.26502	-119.972897685822\\
60.125	0.26868	-124.332347113049\\
60.125	0.27234	-128.774378914083\\
60.125	0.276	-133.298993088926\\
60.5	0.093	-8.20852338173887\\
60.5	0.09666	-8.69237839605829\\
60.5	0.10032	-9.25881578418582\\
60.5	0.10398	-9.90783554612146\\
60.5	0.10764	-10.6394376818652\\
60.5	0.1113	-11.4536221914171\\
60.5	0.11496	-12.3503890747771\\
60.5	0.11862	-13.3297383319451\\
60.5	0.12228	-14.3916699629213\\
60.5	0.12594	-15.5361839677056\\
60.5	0.1296	-16.763280346298\\
60.5	0.13326	-18.0729590986985\\
60.5	0.13692	-19.4652202249071\\
60.5	0.14058	-20.9400637249238\\
60.5	0.14424	-22.4974895987487\\
60.5	0.1479	-24.1374978463816\\
60.5	0.15156	-25.8600884678226\\
60.5	0.15522	-27.6652614630718\\
60.5	0.15888	-29.553016832129\\
60.5	0.16254	-31.5233545749944\\
60.5	0.1662	-33.5762746916679\\
60.5	0.16986	-35.7117771821495\\
60.5	0.17352	-37.9298620464391\\
60.5	0.17718	-40.2305292845369\\
60.5	0.18084	-42.6137788964428\\
60.5	0.1845	-45.0796108821569\\
60.5	0.18816	-47.6280252416789\\
60.5	0.19182	-50.2590219750092\\
60.5	0.19548	-52.9726010821475\\
60.5	0.19914	-55.7687625630939\\
60.5	0.2028	-58.6475064178485\\
60.5	0.20646	-61.6088326464112\\
60.5	0.21012	-64.6527412487819\\
60.5	0.21378	-67.7792322249607\\
60.5	0.21744	-70.9883055749477\\
60.5	0.2211	-74.2799612987428\\
60.5	0.22476	-77.6541993963459\\
60.5	0.22842	-81.1110198677573\\
60.5	0.23208	-84.6504227129767\\
60.5	0.23574	-88.2724079320042\\
60.5	0.2394	-91.9769755248398\\
60.5	0.24306	-95.7641254914836\\
60.5	0.24672	-99.6338578319354\\
60.5	0.25038	-103.586172546195\\
60.5	0.25404	-107.621069634263\\
60.5	0.2577	-111.73854909614\\
60.5	0.26136	-115.938610931824\\
60.5	0.26502	-120.221255141316\\
60.5	0.26868	-124.586481724617\\
60.5	0.27234	-129.034290681725\\
60.5	0.276	-133.564682012642\\
60.875	0.093	-8.19112121867826\\
60.875	0.09666	-8.68075338907158\\
60.875	0.10032	-9.25296793327302\\
60.875	0.10398	-9.90776485128255\\
60.875	0.10764	-10.6451441431002\\
60.875	0.1113	-11.465105808726\\
60.875	0.11496	-12.3676498481598\\
60.875	0.11862	-13.3527762614018\\
60.875	0.12228	-14.4204850484519\\
60.875	0.12594	-15.57077620931\\
60.875	0.1296	-16.8036497439764\\
60.875	0.13326	-18.1191056524508\\
60.875	0.13692	-19.5171439347333\\
60.875	0.14058	-20.9977645908239\\
60.875	0.14424	-22.5609676207226\\
60.875	0.1479	-24.2067530244294\\
60.875	0.15156	-25.9351208019444\\
60.875	0.15522	-27.7460709532674\\
60.875	0.15888	-29.6396034783986\\
60.875	0.16254	-31.6157183773378\\
60.875	0.1662	-33.6744156500852\\
60.875	0.16986	-35.8156952966407\\
60.875	0.17352	-38.0395573170043\\
60.875	0.17718	-40.3460017111759\\
60.875	0.18084	-42.7350284791557\\
60.875	0.1845	-45.2066376209437\\
60.875	0.18816	-47.7608291365396\\
60.875	0.19182	-50.3976030259438\\
60.875	0.19548	-53.116959289156\\
60.875	0.19914	-55.9188979261763\\
60.875	0.2028	-58.8034189370048\\
60.875	0.20646	-61.7705223216414\\
60.875	0.21012	-64.8202080800859\\
60.875	0.21378	-67.9524762123387\\
60.875	0.21744	-71.1673267183996\\
60.875	0.2211	-74.4647595982686\\
60.875	0.22476	-77.8447748519456\\
60.875	0.22842	-81.3073724794308\\
60.875	0.23208	-84.8525524807241\\
60.875	0.23574	-88.4803148558255\\
60.875	0.2394	-92.1906596047351\\
60.875	0.24306	-95.9835867274527\\
60.875	0.24672	-99.8590962239785\\
60.875	0.25038	-103.817188094312\\
60.875	0.25404	-107.857862338454\\
60.875	0.2577	-111.981118956404\\
60.875	0.26136	-116.186957948162\\
60.875	0.26502	-120.475379313729\\
60.875	0.26868	-124.846383053103\\
60.875	0.27234	-129.299969166286\\
60.875	0.276	-133.836137653276\\
61.25	0.093	-8.17948577253607\\
61.25	0.09666	-8.67489509900329\\
61.25	0.10032	-9.25288679927863\\
61.25	0.10398	-9.91346087336206\\
61.25	0.10764	-10.6566173212536\\
61.25	0.1113	-11.4823561429533\\
61.25	0.11496	-12.390677338461\\
61.25	0.11862	-13.3815809077769\\
61.25	0.12228	-14.4550668509009\\
61.25	0.12594	-15.611135167833\\
61.25	0.1296	-16.8497858585731\\
61.25	0.13326	-18.1710189231214\\
61.25	0.13692	-19.5748343614778\\
61.25	0.14058	-21.0612321736424\\
61.25	0.14424	-22.630212359615\\
61.25	0.1479	-24.2817749193957\\
61.25	0.15156	-26.0159198529845\\
61.25	0.15522	-27.8326471603815\\
61.25	0.15888	-29.7319568415865\\
61.25	0.16254	-31.7138488965997\\
61.25	0.1662	-33.778323325421\\
61.25	0.16986	-35.9253801280503\\
61.25	0.17352	-38.1550193044878\\
61.25	0.17718	-40.4672408547334\\
61.25	0.18084	-42.8620447787871\\
61.25	0.1845	-45.3394310766489\\
61.25	0.18816	-47.8993997483188\\
61.25	0.19182	-50.5419507937968\\
61.25	0.19548	-53.2670842130829\\
61.25	0.19914	-56.0748000061772\\
61.25	0.2028	-58.9650981730795\\
61.25	0.20646	-61.9379787137899\\
61.25	0.21012	-64.9934416283085\\
61.25	0.21378	-68.1314869166351\\
61.25	0.21744	-71.3521145787698\\
61.25	0.2211	-74.6553246147127\\
61.25	0.22476	-78.0411170244637\\
61.25	0.22842	-81.5094918080228\\
61.25	0.23208	-85.06044896539\\
61.25	0.23574	-88.6939884965654\\
61.25	0.2394	-92.4101104015488\\
61.25	0.24306	-96.2088146803402\\
61.25	0.24672	-100.09010133294\\
61.25	0.25038	-104.053970359348\\
61.25	0.25404	-108.100421759563\\
61.25	0.2577	-112.229455533587\\
61.25	0.26136	-116.441071681419\\
61.25	0.26502	-120.73527020306\\
61.25	0.26868	-125.112051098508\\
61.25	0.27234	-129.571414367764\\
61.25	0.276	-134.113360010829\\
61.625	0.093	-8.17361704331238\\
61.625	0.09666	-8.67480352585349\\
61.625	0.10032	-9.25857238220272\\
61.625	0.10398	-9.92492361236003\\
61.625	0.10764	-10.6738572163255\\
61.625	0.1113	-11.5053731940991\\
61.625	0.11496	-12.4194715456807\\
61.625	0.11862	-13.4161522710705\\
61.625	0.12228	-14.4954153702683\\
61.625	0.12594	-15.6572608432743\\
61.625	0.1296	-16.9016886900884\\
61.625	0.13326	-18.2286989107106\\
61.625	0.13692	-19.6382915051409\\
61.625	0.14058	-21.1304664733793\\
61.625	0.14424	-22.7052238154258\\
61.625	0.1479	-24.3625635312804\\
61.625	0.15156	-26.1024856209432\\
61.625	0.15522	-27.924990084414\\
61.625	0.15888	-29.830076921693\\
61.625	0.16254	-31.81774613278\\
61.625	0.1662	-33.8879977176752\\
61.625	0.16986	-36.0408316763784\\
61.625	0.17352	-38.2762480088898\\
61.625	0.17718	-40.5942467152093\\
61.625	0.18084	-42.9948277953369\\
61.625	0.1845	-45.4779912492726\\
61.625	0.18816	-48.0437370770164\\
61.625	0.19182	-50.6920652785683\\
61.625	0.19548	-53.4229758539283\\
61.625	0.19914	-56.2364688030964\\
61.625	0.2028	-59.1325441260727\\
61.625	0.20646	-62.111201822857\\
61.625	0.21012	-65.1724418934495\\
61.625	0.21378	-68.31626433785\\
61.625	0.21744	-71.5426691560586\\
61.625	0.2211	-74.8516563480754\\
61.625	0.22476	-78.2432259139003\\
61.625	0.22842	-81.7173778535333\\
61.625	0.23208	-85.2741121669744\\
61.625	0.23574	-88.9134288542236\\
61.625	0.2394	-92.6353279152809\\
61.625	0.24306	-96.4398093501463\\
61.625	0.24672	-100.32687315882\\
61.625	0.25038	-104.296519341302\\
61.625	0.25404	-108.348747897591\\
61.625	0.2577	-112.483558827689\\
61.625	0.26136	-116.700952131595\\
61.625	0.26502	-121.000927809309\\
61.625	0.26868	-125.383485860831\\
61.625	0.27234	-129.848626286161\\
61.625	0.276	-134.3963490853\\
62	0.093	-8.17351503100711\\
62	0.09666	-8.68047866962213\\
62	0.10032	-9.27002468204525\\
62	0.10398	-9.94215306827647\\
62	0.10764	-10.6968638283158\\
62	0.1113	-11.5341569621633\\
62	0.11496	-12.4540324698188\\
62	0.11862	-13.4564903512825\\
62	0.12228	-14.5415306065543\\
62	0.12594	-15.7091532356342\\
62	0.1296	-16.9593582385221\\
62	0.13326	-18.2921456152182\\
62	0.13692	-19.7075153657224\\
62	0.14058	-21.2054674900347\\
62	0.14424	-22.7860019881551\\
62	0.1479	-24.4491188600836\\
62	0.15156	-26.1948181058203\\
62	0.15522	-28.023099725365\\
62	0.15888	-29.9339637187179\\
62	0.16254	-31.9274100858788\\
62	0.1662	-34.0034388268478\\
62	0.16986	-36.162049941625\\
62	0.17352	-38.4032434302103\\
62	0.17718	-40.7270192926037\\
62	0.18084	-43.1333775288051\\
62	0.1845	-45.6223181388147\\
62	0.18816	-48.1938411226324\\
62	0.19182	-50.8479464802582\\
62	0.19548	-53.5846342116921\\
62	0.19914	-56.4039043169342\\
62	0.2028	-59.3057567959843\\
62	0.20646	-62.2901916488426\\
62	0.21012	-65.3572088755089\\
62	0.21378	-68.5068084759833\\
62	0.21744	-71.7389904502659\\
62	0.2211	-75.0537547983566\\
62	0.22476	-78.4511015202553\\
62	0.22842	-81.9310306159622\\
62	0.23208	-85.4935420854772\\
62	0.23574	-89.1386359288003\\
62	0.2394	-92.8663121459315\\
62	0.24306	-96.6765707368708\\
62	0.24672	-100.569411701618\\
62	0.25038	-104.544835040174\\
62	0.25404	-108.602840752537\\
62	0.2577	-112.743428838709\\
62	0.26136	-116.966599298689\\
62	0.26502	-121.272352132477\\
62	0.26868	-125.660687340073\\
62	0.27234	-130.131604921477\\
62	0.276	-134.685104876689\\
62.375	0.093	-8.17917973562031\\
62.375	0.09666	-8.69192053030922\\
62.375	0.10032	-9.28724369880623\\
62.375	0.10398	-9.96514924111135\\
62.375	0.10764	-10.7256371572246\\
62.375	0.1113	-11.568707447146\\
62.375	0.11496	-12.4943601108754\\
62.375	0.11862	-13.5025951484129\\
62.375	0.12228	-14.5934125597586\\
62.375	0.12594	-15.7668123449124\\
62.375	0.1296	-17.0227945038743\\
62.375	0.13326	-18.3613590366442\\
62.375	0.13692	-19.7825059432223\\
62.375	0.14058	-21.2862352236086\\
62.375	0.14424	-22.8725468778028\\
62.375	0.1479	-24.5414409058053\\
62.375	0.15156	-26.2929173076158\\
62.375	0.15522	-28.1269760832344\\
62.375	0.15888	-30.0436172326612\\
62.375	0.16254	-32.042840755896\\
62.375	0.1662	-34.124646652939\\
62.375	0.16986	-36.28903492379\\
62.375	0.17352	-38.5360055684492\\
62.375	0.17718	-40.8655585869165\\
62.375	0.18084	-43.2776939791919\\
62.375	0.1845	-45.7724117452753\\
62.375	0.18816	-48.349711885167\\
62.375	0.19182	-51.0095943988666\\
62.375	0.19548	-53.7520592863744\\
62.375	0.19914	-56.5771065476904\\
62.375	0.2028	-59.4847361828144\\
62.375	0.20646	-62.4749481917466\\
62.375	0.21012	-65.5477425744868\\
62.375	0.21378	-68.7031193310351\\
62.375	0.21744	-71.9410784613915\\
62.375	0.2211	-75.2616199655561\\
62.375	0.22476	-78.6647438435288\\
62.375	0.22842	-82.1504500953096\\
62.375	0.23208	-85.7187387208984\\
62.375	0.23574	-89.3696097202955\\
62.375	0.2394	-93.1030630935006\\
62.375	0.24306	-96.9190988405138\\
62.375	0.24672	-100.817716961335\\
62.375	0.25038	-104.798917455965\\
62.375	0.25404	-108.862700324402\\
62.375	0.2577	-113.009065566648\\
62.375	0.26136	-117.238013182701\\
62.375	0.26502	-121.549543172563\\
62.375	0.26868	-125.943655536233\\
62.375	0.27234	-130.420350273711\\
62.375	0.276	-134.979627384997\\
62.75	0.093	-8.19061115715195\\
62.75	0.09666	-8.70912910791476\\
62.75	0.10032	-9.31022943248568\\
62.75	0.10398	-9.99391213086467\\
62.75	0.10764	-10.7601772030518\\
62.75	0.1113	-11.6090246490471\\
62.75	0.11496	-12.5404544688504\\
62.75	0.11862	-13.5544666624619\\
62.75	0.12228	-14.6510612298814\\
62.75	0.12594	-15.8302381711091\\
62.75	0.1296	-17.0919974861449\\
62.75	0.13326	-18.4363391749887\\
62.75	0.13692	-19.8632632376407\\
62.75	0.14058	-21.3727696741009\\
62.75	0.14424	-22.964858484369\\
62.75	0.1479	-24.6395296684454\\
62.75	0.15156	-26.3967832263298\\
62.75	0.15522	-28.2366191580223\\
62.75	0.15888	-30.159037463523\\
62.75	0.16254	-32.1640381428317\\
62.75	0.1662	-34.2516211959486\\
62.75	0.16986	-36.4217866228735\\
62.75	0.17352	-38.6745344236066\\
62.75	0.17718	-41.0098645981477\\
62.75	0.18084	-43.427777146497\\
62.75	0.1845	-45.9282720686544\\
62.75	0.18816	-48.5113493646199\\
62.75	0.19182	-51.1770090343935\\
62.75	0.19548	-53.9252510779752\\
62.75	0.19914	-56.756075495365\\
62.75	0.2028	-59.669482286563\\
62.75	0.20646	-62.665471451569\\
62.75	0.21012	-65.7440429903831\\
62.75	0.21378	-68.9051969030054\\
62.75	0.21744	-72.1489331894357\\
62.75	0.2211	-75.4752518496742\\
62.75	0.22476	-78.8841528837207\\
62.75	0.22842	-82.3756362915754\\
62.75	0.23208	-85.9497020732381\\
62.75	0.23574	-89.6063502287091\\
62.75	0.2394	-93.3455807579881\\
62.75	0.24306	-97.1673936610752\\
62.75	0.24672	-101.07178893797\\
62.75	0.25038	-105.058766588674\\
62.75	0.25404	-109.128326613185\\
62.75	0.2577	-113.280469011505\\
62.75	0.26136	-117.515193783632\\
62.75	0.26502	-121.832500929568\\
62.75	0.26868	-126.232390449312\\
62.75	0.27234	-130.714862342864\\
62.75	0.276	-135.279916610224\\
63.125	0.093	-8.20780929560206\\
63.125	0.09666	-8.73210440243876\\
63.125	0.10032	-9.33898188308357\\
63.125	0.10398	-10.0284417375365\\
63.125	0.10764	-10.8004839657975\\
63.125	0.1113	-11.6551085678667\\
63.125	0.11496	-12.5923155437439\\
63.125	0.11862	-13.6121048934292\\
63.125	0.12228	-14.7144766169227\\
63.125	0.12594	-15.8994307142243\\
63.125	0.1296	-17.1669671853339\\
63.125	0.13326	-18.5170860302517\\
63.125	0.13692	-19.9497872489776\\
63.125	0.14058	-21.4650708415116\\
63.125	0.14424	-23.0629368078537\\
63.125	0.1479	-24.7433851480039\\
63.125	0.15156	-26.5064158619622\\
63.125	0.15522	-28.3520289497286\\
63.125	0.15888	-30.2802244113032\\
63.125	0.16254	-32.2910022466858\\
63.125	0.1662	-34.3843624558766\\
63.125	0.16986	-36.5603050388754\\
63.125	0.17352	-38.8188299956824\\
63.125	0.17718	-41.1599373262975\\
63.125	0.18084	-43.5836270307206\\
63.125	0.1845	-46.0898991089519\\
63.125	0.18816	-48.6787535609913\\
63.125	0.19182	-51.3501903868388\\
63.125	0.19548	-54.1042095864944\\
63.125	0.19914	-56.9408111599581\\
63.125	0.2028	-59.8599951072299\\
63.125	0.20646	-62.8617614283099\\
63.125	0.21012	-65.9461101231979\\
63.125	0.21378	-69.113041191894\\
63.125	0.21744	-72.3625546343983\\
63.125	0.2211	-75.6946504507106\\
63.125	0.22476	-79.1093286408311\\
63.125	0.22842	-82.6065892047597\\
63.125	0.23208	-86.1864321424963\\
63.125	0.23574	-89.8488574540412\\
63.125	0.2394	-93.5938651393941\\
63.125	0.24306	-97.421455198555\\
63.125	0.24672	-101.331627631524\\
63.125	0.25038	-105.324382438301\\
63.125	0.25404	-109.399719618887\\
63.125	0.2577	-113.55763917328\\
63.125	0.26136	-117.798141101482\\
63.125	0.26502	-122.121225403491\\
63.125	0.26868	-126.526892079309\\
63.125	0.27234	-131.015141128935\\
63.125	0.276	-135.585972552369\\
63.5	0.093	-8.23077415097059\\
63.5	0.09666	-8.7608464138812\\
63.5	0.10032	-9.37350105059991\\
63.5	0.10398	-10.0687380611267\\
63.5	0.10764	-10.8465574454616\\
63.5	0.1113	-11.7069592036047\\
63.5	0.11496	-12.6499433355558\\
63.5	0.11862	-13.675509841315\\
63.5	0.12228	-14.7836587208824\\
63.5	0.12594	-15.9743899742579\\
63.5	0.1296	-17.2477036014414\\
63.5	0.13326	-18.6035996024331\\
63.5	0.13692	-20.0420779772329\\
63.5	0.14058	-21.5631387258408\\
63.5	0.14424	-23.1667818482568\\
63.5	0.1479	-24.8530073444809\\
63.5	0.15156	-26.6218152145131\\
63.5	0.15522	-28.4732054583534\\
63.5	0.15888	-30.4071780760019\\
63.5	0.16254	-32.4237330674584\\
63.5	0.1662	-34.5228704327231\\
63.5	0.16986	-36.7045901717958\\
63.5	0.17352	-38.9688922846767\\
63.5	0.17718	-41.3157767713656\\
63.5	0.18084	-43.7452436318627\\
63.5	0.1845	-46.2572928661679\\
63.5	0.18816	-48.8519244742812\\
63.5	0.19182	-51.5291384562025\\
63.5	0.19548	-54.288934811932\\
63.5	0.19914	-57.1313135414697\\
63.5	0.2028	-60.0562746448154\\
63.5	0.20646	-63.0638181219692\\
63.5	0.21012	-66.1539439729312\\
63.5	0.21378	-69.3266521977012\\
63.5	0.21744	-72.5819427962793\\
63.5	0.2211	-75.9198157686656\\
63.5	0.22476	-79.3402711148599\\
63.5	0.22842	-82.8433088348624\\
63.5	0.23208	-86.4289289286729\\
63.5	0.23574	-90.0971313962917\\
63.5	0.2394	-93.8479162377185\\
63.5	0.24306	-97.6812834529533\\
63.5	0.24672	-101.597233041996\\
63.5	0.25038	-105.595765004847\\
63.5	0.25404	-109.676879341507\\
63.5	0.2577	-113.840576051974\\
63.5	0.26136	-118.086855136249\\
63.5	0.26502	-122.415716594333\\
63.5	0.26868	-126.827160426225\\
63.5	0.27234	-131.321186631924\\
63.5	0.276	-135.897795211432\\
63.875	0.093	-8.25950572325759\\
63.875	0.09666	-8.7953551422421\\
63.875	0.10032	-9.4137869350347\\
63.875	0.10398	-10.1148011016354\\
63.875	0.10764	-10.8983976420442\\
63.875	0.1113	-11.7645765562612\\
63.875	0.11496	-12.7133378442862\\
63.875	0.11862	-13.7446815061193\\
63.875	0.12228	-14.8586075417606\\
63.875	0.12594	-16.05511595121\\
63.875	0.1296	-17.3342067344674\\
63.875	0.13326	-18.695879891533\\
63.875	0.13692	-20.1401354224067\\
63.875	0.14058	-21.6669733270885\\
63.875	0.14424	-23.2763936055783\\
63.875	0.1479	-24.9683962578764\\
63.875	0.15156	-26.7429812839825\\
63.875	0.15522	-28.6001486838967\\
63.875	0.15888	-30.539898457619\\
63.875	0.16254	-32.5622306051494\\
63.875	0.1662	-34.667145126488\\
63.875	0.16986	-36.8546420216346\\
63.875	0.17352	-39.1247212905894\\
63.875	0.17718	-41.4773829333522\\
63.875	0.18084	-43.9126269499232\\
63.875	0.1845	-46.4304533403023\\
63.875	0.18816	-49.0308621044895\\
63.875	0.19182	-51.7138532424847\\
63.875	0.19548	-54.4794267542881\\
63.875	0.19914	-57.3275826398997\\
63.875	0.2028	-60.2583208993193\\
63.875	0.20646	-63.271641532547\\
63.875	0.21012	-66.3675445395829\\
63.875	0.21378	-69.5460299204268\\
63.875	0.21744	-72.8070976750788\\
63.875	0.2211	-76.1507478035389\\
63.875	0.22476	-79.5769803058072\\
63.875	0.22842	-83.0857951818836\\
63.875	0.23208	-86.677192431768\\
63.875	0.23574	-90.3511720554607\\
63.875	0.2394	-94.1077340529613\\
63.875	0.24306	-97.9468784242701\\
63.875	0.24672	-101.868605169387\\
63.875	0.25038	-105.872914288312\\
63.875	0.25404	-109.959805781045\\
63.875	0.2577	-114.129279647586\\
63.875	0.26136	-118.381335887936\\
63.875	0.26502	-122.715974502093\\
63.875	0.26868	-127.133195490059\\
63.875	0.27234	-131.632998851832\\
63.875	0.276	-136.215384587414\\
64.25	0.093	-8.29400401246302\\
64.25	0.09666	-8.83563058752142\\
64.25	0.10032	-9.45983953638792\\
64.25	0.10398	-10.1666308590625\\
64.25	0.10764	-10.9560045555453\\
64.25	0.1113	-11.8279606258361\\
64.25	0.11496	-12.782499069935\\
64.25	0.11862	-13.819619887842\\
64.25	0.12228	-14.9393230795572\\
64.25	0.12594	-16.1416086450805\\
64.25	0.1296	-17.4264765844118\\
64.25	0.13326	-18.7939268975513\\
64.25	0.13692	-20.2439595844989\\
64.25	0.14058	-21.7765746452546\\
64.25	0.14424	-23.3917720798183\\
64.25	0.1479	-25.0895518881903\\
64.25	0.15156	-26.8699140703702\\
64.25	0.15522	-28.7328586263584\\
64.25	0.15888	-30.6783855561546\\
64.25	0.16254	-32.7064948597589\\
64.25	0.1662	-34.8171865371714\\
64.25	0.16986	-37.0104605883919\\
64.25	0.17352	-39.2863170134206\\
64.25	0.17718	-41.6447558122573\\
64.25	0.18084	-44.0857769849022\\
64.25	0.1845	-46.6093805313551\\
64.25	0.18816	-49.2155664516162\\
64.25	0.19182	-51.9043347456854\\
64.25	0.19548	-54.6756854135627\\
64.25	0.19914	-57.5296184552481\\
64.25	0.2028	-60.4661338707416\\
64.25	0.20646	-63.4852316600432\\
64.25	0.21012	-66.586911823153\\
64.25	0.21378	-69.7711743600708\\
64.25	0.21744	-73.0380192707967\\
64.25	0.2211	-76.3874465553307\\
64.25	0.22476	-79.8194562136729\\
64.25	0.22842	-83.3340482458232\\
64.25	0.23208	-86.9312226517815\\
64.25	0.23574	-90.610979431548\\
64.25	0.2394	-94.3733185851226\\
64.25	0.24306	-98.2182401125053\\
64.25	0.24672	-102.145744013696\\
64.25	0.25038	-106.155830288695\\
64.25	0.25404	-110.248498937502\\
64.25	0.2577	-114.423749960117\\
64.25	0.26136	-118.68158335654\\
64.25	0.26502	-123.021999126772\\
64.25	0.26868	-127.444997270811\\
64.25	0.27234	-131.950577788659\\
64.25	0.276	-136.538740680314\\
64.625	0.093	-8.33426901858691\\
64.625	0.09666	-8.8816727497192\\
64.625	0.10032	-9.51165885465959\\
64.625	0.10398	-10.2242273334081\\
64.625	0.10764	-11.0193781859647\\
64.625	0.1113	-11.8971114123294\\
64.625	0.11496	-12.8574270125023\\
64.625	0.11862	-13.9003249864832\\
64.625	0.12228	-15.0258053342723\\
64.625	0.12594	-16.2338680558694\\
64.625	0.1296	-17.5245131512747\\
64.625	0.13326	-18.897740620488\\
64.625	0.13692	-20.3535504635095\\
64.625	0.14058	-21.8919426803391\\
64.625	0.14424	-23.5129172709768\\
64.625	0.1479	-25.2164742354226\\
64.625	0.15156	-27.0026135736764\\
64.625	0.15522	-28.8713352857385\\
64.625	0.15888	-30.8226393716086\\
64.625	0.16254	-32.8565258312868\\
64.625	0.1662	-34.9729946647732\\
64.625	0.16986	-37.1720458720676\\
64.625	0.17352	-39.4536794531701\\
64.625	0.17718	-41.8178954080808\\
64.625	0.18084	-44.2646937367996\\
64.625	0.1845	-46.7940744393264\\
64.625	0.18816	-49.4060375156614\\
64.625	0.19182	-52.1005829658045\\
64.625	0.19548	-54.8777107897557\\
64.625	0.19914	-57.737420987515\\
64.625	0.2028	-60.6797135590824\\
64.625	0.20646	-63.7045885044579\\
64.625	0.21012	-66.8120458236415\\
64.625	0.21378	-70.0020855166332\\
64.625	0.21744	-73.2747075834331\\
64.625	0.2211	-76.629912024041\\
64.625	0.22476	-80.067698838457\\
64.625	0.22842	-83.5880680266812\\
64.625	0.23208	-87.1910195887135\\
64.625	0.23574	-90.8765535245539\\
64.625	0.2394	-94.6446698342023\\
64.625	0.24306	-98.4953685176589\\
64.625	0.24672	-102.428649574924\\
64.625	0.25038	-106.444513005996\\
64.625	0.25404	-110.542958810877\\
64.625	0.2577	-114.723986989566\\
64.625	0.26136	-118.987597542063\\
64.625	0.26502	-123.333790468369\\
64.625	0.26868	-127.762565768482\\
64.625	0.27234	-132.273923442403\\
64.625	0.276	-136.867863490133\\
65	0.093	-8.38030074162927\\
65	0.09666	-8.93348162883544\\
65	0.10032	-9.56924488984973\\
65	0.10398	-10.2875905246722\\
65	0.10764	-11.0885185333026\\
65	0.1113	-11.9720289157413\\
65	0.11496	-12.938121671988\\
65	0.11862	-13.9867968020428\\
65	0.12228	-15.1180543059058\\
65	0.12594	-16.3318941835768\\
65	0.1296	-17.628316435056\\
65	0.13326	-19.0073210603432\\
65	0.13692	-20.4689080594386\\
65	0.14058	-22.0130774323421\\
65	0.14424	-23.6398291790537\\
65	0.1479	-25.3491632995734\\
65	0.15156	-27.1410797939012\\
65	0.15522	-29.0155786620371\\
65	0.15888	-30.9726599039811\\
65	0.16254	-33.0123235197332\\
65	0.1662	-35.1345695092935\\
65	0.16986	-37.3393978726618\\
65	0.17352	-39.6268086098382\\
65	0.17718	-41.9968017208228\\
65	0.18084	-44.4493772056154\\
65	0.1845	-46.9845350642162\\
65	0.18816	-49.602275296625\\
65	0.19182	-52.3025979028421\\
65	0.19548	-55.0855028828671\\
65	0.19914	-57.9509902367003\\
65	0.2028	-60.8990599643417\\
65	0.20646	-63.9297120657911\\
65	0.21012	-67.0429465410485\\
65	0.21378	-70.2387633901141\\
65	0.21744	-73.5171626129879\\
65	0.2211	-76.8781442096698\\
65	0.22476	-80.3217081801596\\
65	0.22842	-83.8478545244577\\
65	0.23208	-87.4565832425639\\
65	0.23574	-91.1478943344782\\
65	0.2394	-94.9217878002005\\
65	0.24306	-98.7782636397311\\
65	0.24672	-102.71732185307\\
65	0.25038	-106.738962440216\\
65	0.25404	-110.843185401171\\
65	0.2577	-115.029990735934\\
65	0.26136	-119.299378444505\\
65	0.26502	-123.651348526884\\
65	0.26868	-128.085900983071\\
65	0.27234	-132.603035813067\\
65	0.276	-137.20275301687\\
65.375	0.093	-8.43209918159006\\
65.375	0.09666	-8.99105722487013\\
65.375	0.10032	-9.63259764195833\\
65.375	0.10398	-10.3567204328546\\
65.375	0.10764	-11.163425597559\\
65.375	0.1113	-12.0527131360715\\
65.375	0.11496	-13.0245830483922\\
65.375	0.11862	-14.0790353345209\\
65.375	0.12228	-15.2160699944578\\
65.375	0.12594	-16.4356870282027\\
65.375	0.1296	-17.7378864357557\\
65.375	0.13326	-19.1226682171169\\
65.375	0.13692	-20.5900323722862\\
65.375	0.14058	-22.1399789012635\\
65.375	0.14424	-23.772507804049\\
65.375	0.1479	-25.4876190806426\\
65.375	0.15156	-27.2853127310443\\
65.375	0.15522	-29.1655887552541\\
65.375	0.15888	-31.128447153272\\
65.375	0.16254	-33.173887925098\\
65.375	0.1662	-35.3019110707322\\
65.375	0.16986	-37.5125165901744\\
65.375	0.17352	-39.8057044834247\\
65.375	0.17718	-42.1814747504831\\
65.375	0.18084	-44.6398273913497\\
65.375	0.1845	-47.1807624060244\\
65.375	0.18816	-49.8042797945072\\
65.375	0.19182	-52.5103795567981\\
65.375	0.19548	-55.299061692897\\
65.375	0.19914	-58.1703262028041\\
65.375	0.2028	-61.1241730865193\\
65.375	0.20646	-64.1606023440427\\
65.375	0.21012	-67.279613975374\\
65.375	0.21378	-70.4812079805135\\
65.375	0.21744	-73.7653843594612\\
65.375	0.2211	-77.1321431122169\\
65.375	0.22476	-80.5814842387807\\
65.375	0.22842	-84.1134077391527\\
65.375	0.23208	-87.7279136133328\\
65.375	0.23574	-91.4250018613209\\
65.375	0.2394	-95.2046724831172\\
65.375	0.24306	-99.0669254787216\\
65.375	0.24672	-103.011760848134\\
65.375	0.25038	-107.039178591355\\
65.375	0.25404	-111.149178708383\\
65.375	0.2577	-115.34176119922\\
65.375	0.26136	-119.616926063865\\
65.375	0.26502	-123.974673302318\\
65.375	0.26868	-128.415002914579\\
65.375	0.27234	-132.937914900648\\
65.375	0.276	-137.543409260526\\
65.75	0.093	-8.4896643384693\\
65.75	0.09666	-9.05439953782327\\
65.75	0.10032	-9.70171711098536\\
65.75	0.10398	-10.4316170579556\\
65.75	0.10764	-11.2440993787338\\
65.75	0.1113	-12.1391640733203\\
65.75	0.11496	-13.1168111417148\\
65.75	0.11862	-14.1770405839174\\
65.75	0.12228	-15.3198523999282\\
65.75	0.12594	-16.545246589747\\
65.75	0.1296	-17.8532231533739\\
65.75	0.13326	-19.243782090809\\
65.75	0.13692	-20.7169234020521\\
65.75	0.14058	-22.2726470871034\\
65.75	0.14424	-23.9109531459628\\
65.75	0.1479	-25.6318415786303\\
65.75	0.15156	-27.4353123851059\\
65.75	0.15522	-29.3213655653896\\
65.75	0.15888	-31.2900011194814\\
65.75	0.16254	-33.3412190473813\\
65.75	0.1662	-35.4750193490893\\
65.75	0.16986	-37.6914020246055\\
65.75	0.17352	-39.9903670739297\\
65.75	0.17718	-42.371914497062\\
65.75	0.18084	-44.8360442940025\\
65.75	0.1845	-47.3827564647511\\
65.75	0.18816	-50.0120510093077\\
65.75	0.19182	-52.7239279276725\\
65.75	0.19548	-55.5183872198454\\
65.75	0.19914	-58.3954288858263\\
65.75	0.2028	-61.3550529256155\\
65.75	0.20646	-64.3972593392127\\
65.75	0.21012	-67.5220481266179\\
65.75	0.21378	-70.7294192878314\\
65.75	0.21744	-74.0193728228529\\
65.75	0.2211	-77.3919087316825\\
65.75	0.22476	-80.8470270143202\\
65.75	0.22842	-84.3847276707661\\
65.75	0.23208	-88.0050107010201\\
65.75	0.23574	-91.7078761050821\\
65.75	0.2394	-95.4933238829523\\
65.75	0.24306	-99.3613540346306\\
65.75	0.24672	-103.311966560117\\
65.75	0.25038	-107.345161459411\\
65.75	0.25404	-111.460938732514\\
65.75	0.2577	-115.659298379425\\
65.75	0.26136	-119.940240400144\\
65.75	0.26502	-124.30376479467\\
65.75	0.26868	-128.749871563006\\
65.75	0.27234	-133.278560705149\\
65.75	0.276	-137.8898322211\\
66.125	0.093	-8.552996212267\\
66.125	0.09666	-9.12350856769487\\
66.125	0.10032	-9.77660329693085\\
66.125	0.10398	-10.512280399975\\
66.125	0.10764	-11.3305398768271\\
66.125	0.1113	-12.2313817274874\\
66.125	0.11496	-13.2148059519559\\
66.125	0.11862	-14.2808125502324\\
66.125	0.12228	-15.429401522317\\
66.125	0.12594	-16.6605728682098\\
66.125	0.1296	-17.9743265879106\\
66.125	0.13326	-19.3706626814196\\
66.125	0.13692	-20.8495811487366\\
66.125	0.14058	-22.4110819898618\\
66.125	0.14424	-24.0551652047951\\
66.125	0.1479	-25.7818307935365\\
66.125	0.15156	-27.5910787560859\\
66.125	0.15522	-29.4829090924435\\
66.125	0.15888	-31.4573218026092\\
66.125	0.16254	-33.514316886583\\
66.125	0.1662	-35.653894344365\\
66.125	0.16986	-37.876054175955\\
66.125	0.17352	-40.1807963813531\\
66.125	0.17718	-42.5681209605594\\
66.125	0.18084	-45.0380279135737\\
66.125	0.1845	-47.5905172403962\\
66.125	0.18816	-50.2255889410267\\
66.125	0.19182	-52.9432430154654\\
66.125	0.19548	-55.7434794637122\\
66.125	0.19914	-58.626298285767\\
66.125	0.2028	-61.5916994816301\\
66.125	0.20646	-64.6396830513012\\
66.125	0.21012	-67.7702489947804\\
66.125	0.21378	-70.9833973120677\\
66.125	0.21744	-74.2791280031631\\
66.125	0.2211	-77.6574410680666\\
66.125	0.22476	-81.1183365067782\\
66.125	0.22842	-84.661814319298\\
66.125	0.23208	-88.2878745056258\\
66.125	0.23574	-91.9965170657618\\
66.125	0.2394	-95.7877419997059\\
66.125	0.24306	-99.661549307458\\
66.125	0.24672	-103.617938989018\\
66.125	0.25038	-107.656911044387\\
66.125	0.25404	-111.778465473563\\
66.125	0.2577	-115.982602276548\\
66.125	0.26136	-120.26932145334\\
66.125	0.26502	-124.638623003941\\
66.125	0.26868	-129.09050692835\\
66.125	0.27234	-133.624973226567\\
66.125	0.276	-138.242021898592\\
66.5	0.093	-8.62209480298316\\
66.5	0.09666	-9.19838431448491\\
66.5	0.10032	-9.8572561997948\\
66.5	0.10398	-10.5987104589128\\
66.5	0.10764	-11.4227470918389\\
66.5	0.1113	-12.3293660985731\\
66.5	0.11496	-13.3185674791154\\
66.5	0.11862	-14.3903512334658\\
66.5	0.12228	-15.5447173616244\\
66.5	0.12594	-16.781665863591\\
66.5	0.1296	-18.1011967393657\\
66.5	0.13326	-19.5033099889486\\
66.5	0.13692	-20.9880056123395\\
66.5	0.14058	-22.5552836095386\\
66.5	0.14424	-24.2051439805458\\
66.5	0.1479	-25.937586725361\\
66.5	0.15156	-27.7526118439844\\
66.5	0.15522	-29.6502193364159\\
66.5	0.15888	-31.6304092026555\\
66.5	0.16254	-33.6931814427032\\
66.5	0.1662	-35.8385360565591\\
66.5	0.16986	-38.066473044223\\
66.5	0.17352	-40.376992405695\\
66.5	0.17718	-42.7700941409751\\
66.5	0.18084	-45.2457782500634\\
66.5	0.1845	-47.8040447329597\\
66.5	0.18816	-50.4448935896642\\
66.5	0.19182	-53.1683248201768\\
66.5	0.19548	-55.9743384244974\\
66.5	0.19914	-58.8629344026262\\
66.5	0.2028	-61.8341127545631\\
66.5	0.20646	-64.8878734803081\\
66.5	0.21012	-68.0242165798612\\
66.5	0.21378	-71.2431420532224\\
66.5	0.21744	-74.5446499003917\\
66.5	0.2211	-77.9287401213691\\
66.5	0.22476	-81.3954127161547\\
66.5	0.22842	-84.9446676847483\\
66.5	0.23208	-88.5765050271501\\
66.5	0.23574	-92.2909247433599\\
66.5	0.2394	-96.0879268333779\\
66.5	0.24306	-99.967511297204\\
66.5	0.24672	-103.929678134838\\
66.5	0.25038	-107.97442734628\\
66.5	0.25404	-112.101758931531\\
66.5	0.2577	-116.311672890589\\
66.5	0.26136	-120.604169223456\\
66.5	0.26502	-124.979247930131\\
66.5	0.26868	-129.436909010613\\
66.5	0.27234	-133.977152464904\\
66.5	0.276	-138.599978293003\\
66.875	0.093	-8.69696011061775\\
66.875	0.09666	-9.27902677819341\\
66.875	0.10032	-9.94367581957719\\
66.875	0.10398	-10.6909072347691\\
66.875	0.10764	-11.5207210237691\\
66.875	0.1113	-12.4331171865772\\
66.875	0.11496	-13.4280957231934\\
66.875	0.11862	-14.5056566336177\\
66.875	0.12228	-15.6657999178501\\
66.875	0.12594	-16.9085255758906\\
66.875	0.1296	-18.2338336077393\\
66.875	0.13326	-19.641724013396\\
66.875	0.13692	-21.1321967928609\\
66.875	0.14058	-22.7052519461338\\
66.875	0.14424	-24.3608894732149\\
66.875	0.1479	-26.0991093741041\\
66.875	0.15156	-27.9199116488014\\
66.875	0.15522	-29.8232962973068\\
66.875	0.15888	-31.8092633196203\\
66.875	0.16254	-33.8778127157419\\
66.875	0.1662	-36.0289444856716\\
66.875	0.16986	-38.2626586294094\\
66.875	0.17352	-40.5789551469553\\
66.875	0.17718	-42.9778340383093\\
66.875	0.18084	-45.4592953034715\\
66.875	0.1845	-48.0233389424418\\
66.875	0.18816	-50.6699649552201\\
66.875	0.19182	-53.3991733418065\\
66.875	0.19548	-56.2109641022011\\
66.875	0.19914	-59.1053372364038\\
66.875	0.2028	-62.0822927444146\\
66.875	0.20646	-65.1418306262335\\
66.875	0.21012	-68.2839508818605\\
66.875	0.21378	-71.5086535112956\\
66.875	0.21744	-74.8159385145388\\
66.875	0.2211	-78.2058058915901\\
66.875	0.22476	-81.6782556424495\\
66.875	0.22842	-85.2332877671171\\
66.875	0.23208	-88.8709022655927\\
66.875	0.23574	-92.5910991378765\\
66.875	0.2394	-96.3938783839683\\
66.875	0.24306	-100.279240003868\\
66.875	0.24672	-104.247183997576\\
66.875	0.25038	-108.297710365093\\
66.875	0.25404	-112.430819106417\\
66.875	0.2577	-116.646510221549\\
66.875	0.26136	-120.94478371049\\
66.875	0.26502	-125.325639573238\\
66.875	0.26868	-129.789077809795\\
66.875	0.27234	-134.33509842016\\
66.875	0.276	-138.963701404333\\
67.25	0.093	-8.77759213517081\\
67.25	0.09666	-9.36543595882036\\
67.25	0.10032	-10.035862156278\\
67.25	0.10398	-10.7888707275438\\
67.25	0.10764	-11.6244616726177\\
67.25	0.1113	-12.5426349914997\\
67.25	0.11496	-13.5433906841898\\
67.25	0.11862	-14.626728750688\\
67.25	0.12228	-15.7926491909944\\
67.25	0.12594	-17.0411520051088\\
67.25	0.1296	-18.3722371930313\\
67.25	0.13326	-19.785904754762\\
67.25	0.13692	-21.2821546903007\\
67.25	0.14058	-22.8609869996476\\
67.25	0.14424	-24.5224016828025\\
67.25	0.1479	-26.2663987397656\\
67.25	0.15156	-28.0929781705367\\
67.25	0.15522	-30.0021399751161\\
67.25	0.15888	-31.9938841535034\\
67.25	0.16254	-34.068210705699\\
67.25	0.1662	-36.2251196317026\\
67.25	0.16986	-38.4646109315143\\
67.25	0.17352	-40.7866846051341\\
67.25	0.17718	-43.191340652562\\
67.25	0.18084	-45.6785790737981\\
67.25	0.1845	-48.2483998688422\\
67.25	0.18816	-50.9008030376944\\
67.25	0.19182	-53.6357885803548\\
67.25	0.19548	-56.4533564968233\\
67.25	0.19914	-59.3535067870998\\
67.25	0.2028	-62.3362394511846\\
67.25	0.20646	-65.4015544890774\\
67.25	0.21012	-68.5494519007782\\
67.25	0.21378	-71.7799316862872\\
67.25	0.21744	-75.0929938456043\\
67.25	0.2211	-78.4886383787296\\
67.25	0.22476	-81.9668652856628\\
67.25	0.22842	-85.5276745664043\\
67.25	0.23208	-89.1710662209538\\
67.25	0.23574	-92.8970402493115\\
67.25	0.2394	-96.7055966514773\\
67.25	0.24306	-100.596735427451\\
67.25	0.24672	-104.570456577233\\
67.25	0.25038	-108.626760100823\\
67.25	0.25404	-112.765645998221\\
67.25	0.2577	-116.987114269428\\
67.25	0.26136	-121.291164914442\\
67.25	0.26502	-125.677797933265\\
67.25	0.26868	-130.147013325895\\
67.25	0.27234	-134.698811092334\\
67.25	0.276	-139.333191232581\\
67.625	0.093	-8.8639908766423\\
67.625	0.09666	-9.45761185636576\\
67.625	0.10032	-10.1338152098973\\
67.625	0.10398	-10.892600937237\\
67.625	0.10764	-11.7339690383848\\
67.625	0.1113	-12.6579195133407\\
67.625	0.11496	-13.6644523621047\\
67.625	0.11862	-14.7535675846768\\
67.625	0.12228	-15.925265181057\\
67.625	0.12594	-17.1795451512453\\
67.625	0.1296	-18.5164074952418\\
67.625	0.13326	-19.9358522130463\\
67.625	0.13692	-21.4378793046589\\
67.625	0.14058	-23.0224887700797\\
67.625	0.14424	-24.6896806093086\\
67.625	0.1479	-26.4394548223456\\
67.625	0.15156	-28.2718114091906\\
67.625	0.15522	-30.1867503698438\\
67.625	0.15888	-32.1842717043051\\
67.625	0.16254	-34.2643754125745\\
67.625	0.1662	-36.427061494652\\
67.625	0.16986	-38.6723299505376\\
67.625	0.17352	-41.0001807802313\\
67.625	0.17718	-43.4106139837331\\
67.625	0.18084	-45.9036295610431\\
67.625	0.1845	-48.4792275121611\\
67.625	0.18816	-51.1374078370873\\
67.625	0.19182	-53.8781705358215\\
67.625	0.19548	-56.7015156083639\\
67.625	0.19914	-59.6074430547143\\
67.625	0.2028	-62.595952874873\\
67.625	0.20646	-65.6670450688397\\
67.625	0.21012	-68.8207196366144\\
67.625	0.21378	-72.0569765781973\\
67.625	0.21744	-75.3758158935883\\
67.625	0.2211	-78.7772375827874\\
67.625	0.22476	-82.2612416457946\\
67.625	0.22842	-85.82782808261\\
67.625	0.23208	-89.4769968932334\\
67.625	0.23574	-93.208748077665\\
67.625	0.2394	-97.0230816359046\\
67.625	0.24306	-100.919997567952\\
67.625	0.24672	-104.899495873808\\
67.625	0.25038	-108.961576553472\\
67.625	0.25404	-113.106239606944\\
67.625	0.2577	-117.333485034225\\
67.625	0.26136	-121.643312835313\\
67.625	0.26502	-126.035723010209\\
67.625	0.26868	-130.510715558914\\
67.625	0.27234	-135.068290481426\\
67.625	0.276	-139.708447777747\\
68	0.093	-8.95615633503225\\
68	0.09666	-9.55555447082961\\
68	0.10032	-10.2375349804351\\
68	0.10398	-11.0020978638486\\
68	0.10764	-11.8492431210703\\
68	0.1113	-12.7789707521001\\
68	0.11496	-13.791280756938\\
68	0.11862	-14.886173135584\\
68	0.12228	-16.0636478880381\\
68	0.12594	-17.3237050143004\\
68	0.1296	-18.6663445143707\\
68	0.13326	-20.0915663882491\\
68	0.13692	-21.5993706359356\\
68	0.14058	-23.1897572574303\\
68	0.14424	-24.8627262527331\\
68	0.1479	-26.6182776218439\\
68	0.15156	-28.4564113647629\\
68	0.15522	-30.37712748149\\
68	0.15888	-32.3804259720252\\
68	0.16254	-34.4663068363685\\
68	0.1662	-36.6347700745199\\
68	0.16986	-38.8858156864794\\
68	0.17352	-41.219443672247\\
68	0.17718	-43.6356540318227\\
68	0.18084	-46.1344467652065\\
68	0.1845	-48.7158218723985\\
68	0.18816	-51.3797793533985\\
68	0.19182	-54.1263192082067\\
68	0.19548	-56.955441436823\\
68	0.19914	-59.8671460392473\\
68	0.2028	-62.8614330154798\\
68	0.20646	-65.9383023655204\\
68	0.21012	-69.0977540893691\\
68	0.21378	-72.3397881870258\\
68	0.21744	-75.6644046584908\\
68	0.2211	-79.0716035037638\\
68	0.22476	-82.5613847228448\\
68	0.22842	-86.1337483157341\\
68	0.23208	-89.7886942824314\\
68	0.23574	-93.5262226229369\\
68	0.2394	-97.3463333372504\\
68	0.24306	-101.249026425372\\
68	0.24672	-105.234301887302\\
68	0.25038	-109.30215972304\\
68	0.25404	-113.452599932586\\
68	0.2577	-117.68562251594\\
68	0.26136	-122.001227473102\\
68	0.26502	-126.399414804072\\
68	0.26868	-130.880184508851\\
68	0.27234	-135.443536587437\\
68	0.276	-140.089471039832\\
68.375	0.093	-9.05408851034059\\
68.375	0.09666	-9.65926380221187\\
68.375	0.10032	-10.3470214678912\\
68.375	0.10398	-11.1173615073787\\
68.375	0.10764	-11.9702839206743\\
68.375	0.1113	-12.905788707778\\
68.375	0.11496	-13.9238758686898\\
68.375	0.11862	-15.0245454034096\\
68.375	0.12228	-16.2077973119377\\
68.375	0.12594	-17.4736315942738\\
68.375	0.1296	-18.822048250418\\
68.375	0.13326	-20.2530472803703\\
68.375	0.13692	-21.7666286841308\\
68.375	0.14058	-23.3627924616993\\
68.375	0.14424	-25.041538613076\\
68.375	0.1479	-26.8028671382608\\
68.375	0.15156	-28.6467780372536\\
68.375	0.15522	-30.5732713100546\\
68.375	0.15888	-32.5823469566637\\
68.375	0.16254	-34.6740049770809\\
68.375	0.1662	-36.8482453713062\\
68.375	0.16986	-39.1050681393396\\
68.375	0.17352	-41.4444732811811\\
68.375	0.17718	-43.8664607968307\\
68.375	0.18084	-46.3710306862884\\
68.375	0.1845	-48.9581829495543\\
68.375	0.18816	-51.6279175866282\\
68.375	0.19182	-54.3802345975102\\
68.375	0.19548	-57.2151339822004\\
68.375	0.19914	-60.1326157406987\\
68.375	0.2028	-63.132679873005\\
68.375	0.20646	-66.2153263791195\\
68.375	0.21012	-69.3805552590421\\
68.375	0.21378	-72.6283665127728\\
68.375	0.21744	-75.9587601403116\\
68.375	0.2211	-79.3717361416585\\
68.375	0.22476	-82.8672945168135\\
68.375	0.22842	-86.4454352657767\\
68.375	0.23208	-90.1061583885479\\
68.375	0.23574	-93.8494638851273\\
68.375	0.2394	-97.6753517555147\\
68.375	0.24306	-101.58382199971\\
68.375	0.24672	-105.574874617714\\
68.375	0.25038	-109.648509609526\\
68.375	0.25404	-113.804726975145\\
68.375	0.2577	-118.043526714574\\
68.375	0.26136	-122.36490882781\\
68.375	0.26502	-126.768873314854\\
68.375	0.26868	-131.255420175706\\
68.375	0.27234	-135.824549410366\\
68.375	0.276	-140.476261018835\\
68.75	0.093	-9.15778740256744\\
68.75	0.09666	-9.76873985051262\\
68.75	0.10032	-10.4622746722659\\
68.75	0.10398	-11.2383918678272\\
68.75	0.10764	-12.0970914371967\\
68.75	0.1113	-13.0383733803743\\
68.75	0.11496	-14.06223769736\\
68.75	0.11862	-15.1686843881538\\
68.75	0.12228	-16.3577134527557\\
68.75	0.12594	-17.6293248911657\\
68.75	0.1296	-18.9835187033838\\
68.75	0.13326	-20.42029488941\\
68.75	0.13692	-21.9396534492444\\
68.75	0.14058	-23.5415943828868\\
68.75	0.14424	-25.2261176903374\\
68.75	0.1479	-26.993223371596\\
68.75	0.15156	-28.8429114266628\\
68.75	0.15522	-30.7751818555377\\
68.75	0.15888	-32.7900346582207\\
68.75	0.16254	-34.8874698347118\\
68.75	0.1662	-37.067487385011\\
68.75	0.16986	-39.3300873091182\\
68.75	0.17352	-41.6752696070337\\
68.75	0.17718	-44.1030342787572\\
68.75	0.18084	-46.6133813242888\\
68.75	0.1845	-49.2063107436285\\
68.75	0.18816	-51.8818225367764\\
68.75	0.19182	-54.6399167037323\\
68.75	0.19548	-57.4805932444963\\
68.75	0.19914	-60.4038521590685\\
68.75	0.2028	-63.4096934474488\\
68.75	0.20646	-66.4981171096372\\
68.75	0.21012	-69.6691231456337\\
68.75	0.21378	-72.9227115554382\\
68.75	0.21744	-76.2588823390509\\
68.75	0.2211	-79.6776354964717\\
68.75	0.22476	-83.1789710277006\\
68.75	0.22842	-86.7628889327377\\
68.75	0.23208	-90.4293892115828\\
68.75	0.23574	-94.1784718642361\\
68.75	0.2394	-98.0101368906974\\
68.75	0.24306	-101.924384290967\\
68.75	0.24672	-105.921214065044\\
68.75	0.25038	-110.00062621293\\
68.75	0.25404	-114.162620734624\\
68.75	0.2577	-118.407197630126\\
68.75	0.26136	-122.734356899436\\
68.75	0.26502	-127.144098542554\\
68.75	0.26868	-131.63642255948\\
68.75	0.27234	-136.211328950214\\
68.75	0.276	-140.868817714757\\
69.125	0.093	-9.26725301171274\\
69.125	0.09666	-9.8839826157318\\
69.125	0.10032	-10.583294593559\\
69.125	0.10398	-11.3651889451942\\
69.125	0.10764	-12.2296656706376\\
69.125	0.1113	-13.1767247698891\\
69.125	0.11496	-14.2063662429487\\
69.125	0.11862	-15.3185900898164\\
69.125	0.12228	-16.5133963104922\\
69.125	0.12594	-17.7907849049761\\
69.125	0.1296	-19.1507558732681\\
69.125	0.13326	-20.5933092153682\\
69.125	0.13692	-22.1184449312764\\
69.125	0.14058	-23.7261630209928\\
69.125	0.14424	-25.4164634845172\\
69.125	0.1479	-27.1893463218498\\
69.125	0.15156	-29.0448115329905\\
69.125	0.15522	-30.9828591179392\\
69.125	0.15888	-33.0034890766961\\
69.125	0.16254	-35.1067014092611\\
69.125	0.1662	-37.2924961156342\\
69.125	0.16986	-39.5608731958154\\
69.125	0.17352	-41.9118326498047\\
69.125	0.17718	-44.3453744776021\\
69.125	0.18084	-46.8614986792076\\
69.125	0.1845	-49.4602052546212\\
69.125	0.18816	-52.141494203843\\
69.125	0.19182	-54.9053655268728\\
69.125	0.19548	-57.7518192237107\\
69.125	0.19914	-60.6808552943568\\
69.125	0.2028	-63.692473738811\\
69.125	0.20646	-66.7866745570733\\
69.125	0.21012	-69.9634577491437\\
69.125	0.21378	-73.2228233150221\\
69.125	0.21744	-76.5647712547087\\
69.125	0.2211	-79.9893015682034\\
69.125	0.22476	-83.4964142555062\\
69.125	0.22842	-87.0861093166172\\
69.125	0.23208	-90.7583867515361\\
69.125	0.23574	-94.5132465602633\\
69.125	0.2394	-98.3506887427986\\
69.125	0.24306	-102.270713299142\\
69.125	0.24672	-106.273320229293\\
69.125	0.25038	-110.358509533253\\
69.125	0.25404	-114.526281211021\\
69.125	0.2577	-118.776635262596\\
69.125	0.26136	-123.10957168798\\
69.125	0.26502	-127.525090487172\\
69.125	0.26868	-132.023191660172\\
69.125	0.27234	-136.60387520698\\
69.125	0.276	-141.267141127597\\
69.5	0.093	-9.38248533777649\\
69.5	0.09666	-10.0049920978694\\
69.5	0.10032	-10.7100812317705\\
69.5	0.10398	-11.4977527394797\\
69.5	0.10764	-12.3680066209969\\
69.5	0.1113	-13.3208428763223\\
69.5	0.11496	-14.3562615054558\\
69.5	0.11862	-15.4742625083974\\
69.5	0.12228	-16.6748458851471\\
69.5	0.12594	-17.9580116357049\\
69.5	0.1296	-19.3237597600708\\
69.5	0.13326	-20.7720902582448\\
69.5	0.13692	-22.3030031302269\\
69.5	0.14058	-23.9164983760172\\
69.5	0.14424	-25.6125759956155\\
69.5	0.1479	-27.391235989022\\
69.5	0.15156	-29.2524783562365\\
69.5	0.15522	-31.1963030972592\\
69.5	0.15888	-33.22271021209\\
69.5	0.16254	-35.3316997007289\\
69.5	0.1662	-37.5232715631759\\
69.5	0.16986	-39.7974257994309\\
69.5	0.17352	-42.1541624094941\\
69.5	0.17718	-44.5934813933655\\
69.5	0.18084	-47.1153827510449\\
69.5	0.1845	-49.7198664825324\\
69.5	0.18816	-52.4069325878281\\
69.5	0.19182	-55.1765810669317\\
69.5	0.19548	-58.0288119198436\\
69.5	0.19914	-60.9636251465636\\
69.5	0.2028	-63.9810207470916\\
69.5	0.20646	-67.0809987214278\\
69.5	0.21012	-70.2635590695721\\
69.5	0.21378	-73.5287017915245\\
69.5	0.21744	-76.8764268872849\\
69.5	0.2211	-80.3067343568535\\
69.5	0.22476	-83.8196242002302\\
69.5	0.22842	-87.4150964174151\\
69.5	0.23208	-91.0931510084079\\
69.5	0.23574	-94.8537879732091\\
69.5	0.2394	-98.6970073118182\\
69.5	0.24306	-102.622809024235\\
69.5	0.24672	-106.631193110461\\
69.5	0.25038	-110.722159570494\\
69.5	0.25404	-114.895708404336\\
69.5	0.2577	-119.151839611985\\
69.5	0.26136	-123.490553193443\\
69.5	0.26502	-127.911849148709\\
69.5	0.26868	-132.415727477783\\
69.5	0.27234	-137.002188180665\\
69.5	0.276	-141.671231257355\\
69.875	0.093	-9.50348438075869\\
69.875	0.09666	-10.1317682969255\\
69.875	0.10032	-10.8426345869005\\
69.875	0.10398	-11.6360832506835\\
69.875	0.10764	-12.5121142882747\\
69.875	0.1113	-13.470727699674\\
69.875	0.11496	-14.5119234848814\\
69.875	0.11862	-15.6357016438969\\
69.875	0.12228	-16.8420621767205\\
69.875	0.12594	-18.1310050833522\\
69.875	0.1296	-19.502530363792\\
69.875	0.13326	-20.9566380180399\\
69.875	0.13692	-22.4933280460959\\
69.875	0.14058	-24.1126004479601\\
69.875	0.14424	-25.8144552236323\\
69.875	0.1479	-27.5988923731126\\
69.875	0.15156	-29.4659118964011\\
69.875	0.15522	-31.4155137934976\\
69.875	0.15888	-33.4476980644023\\
69.875	0.16254	-35.5624647091151\\
69.875	0.1662	-37.759813727636\\
69.875	0.16986	-40.039745119965\\
69.875	0.17352	-42.4022588861021\\
69.875	0.17718	-44.8473550260473\\
69.875	0.18084	-47.3750335398006\\
69.875	0.1845	-49.985294427362\\
69.875	0.18816	-52.6781376887316\\
69.875	0.19182	-55.4535633239092\\
69.875	0.19548	-58.3115713328949\\
69.875	0.19914	-61.2521617156888\\
69.875	0.2028	-64.2753344722907\\
69.875	0.20646	-67.3810896027008\\
69.875	0.21012	-70.569427106919\\
69.875	0.21378	-73.8403469849453\\
69.875	0.21744	-77.1938492367796\\
69.875	0.2211	-80.6299338624221\\
69.875	0.22476	-84.1486008618727\\
69.875	0.22842	-87.7498502351315\\
69.875	0.23208	-91.4336819821982\\
69.875	0.23574	-95.2000961030732\\
69.875	0.2394	-99.0490925977563\\
69.875	0.24306	-102.980671466247\\
69.875	0.24672	-106.994832708547\\
69.875	0.25038	-111.091576324654\\
69.875	0.25404	-115.270902314569\\
69.875	0.2577	-119.532810678293\\
69.875	0.26136	-123.877301415825\\
69.875	0.26502	-128.304374527164\\
69.875	0.26868	-132.814030012312\\
69.875	0.27234	-137.406267871268\\
69.875	0.276	-142.081088104032\\
70.25	0.093	-9.63025014065933\\
70.25	0.09666	-10.2643112129001\\
70.25	0.10032	-10.9809546589489\\
70.25	0.10398	-11.7801804788059\\
70.25	0.10764	-12.6619886724709\\
70.25	0.1113	-13.6263792399441\\
70.25	0.11496	-14.6733521812254\\
70.25	0.11862	-15.8029074963148\\
70.25	0.12228	-17.0150451852123\\
70.25	0.12594	-18.3097652479179\\
70.25	0.1296	-19.6870676844316\\
70.25	0.13326	-21.1469524947534\\
70.25	0.13692	-22.6894196788833\\
70.25	0.14058	-24.3144692368214\\
70.25	0.14424	-26.0221011685675\\
70.25	0.1479	-27.8123154741217\\
70.25	0.15156	-29.6851121534841\\
70.25	0.15522	-31.6404912066545\\
70.25	0.15888	-33.6784526336331\\
70.25	0.16254	-35.7989964344198\\
70.25	0.1662	-38.0021226090146\\
70.25	0.16986	-40.2878311574175\\
70.25	0.17352	-42.6561220796284\\
70.25	0.17718	-45.1069953756475\\
70.25	0.18084	-47.6404510454747\\
70.25	0.1845	-50.2564890891101\\
70.25	0.18816	-52.9551095065535\\
70.25	0.19182	-55.736312297805\\
70.25	0.19548	-58.6000974628647\\
70.25	0.19914	-61.5464650017324\\
70.25	0.2028	-64.5754149144083\\
70.25	0.20646	-67.6869472008922\\
70.25	0.21012	-70.8810618611843\\
70.25	0.21378	-74.1577588952845\\
70.25	0.21744	-77.5170383031927\\
70.25	0.2211	-80.9589000849091\\
70.25	0.22476	-84.4833442404336\\
70.25	0.22842	-88.0903707697663\\
70.25	0.23208	-91.7799796729069\\
70.25	0.23574	-95.5521709498558\\
70.25	0.2394	-99.4069446006127\\
70.25	0.24306	-103.344300625178\\
70.25	0.24672	-107.364239023551\\
70.25	0.25038	-111.466759795732\\
70.25	0.25404	-115.651862941721\\
70.25	0.2577	-119.919548461519\\
70.25	0.26136	-124.269816355125\\
70.25	0.26502	-128.702666622538\\
70.25	0.26868	-133.21809926376\\
70.25	0.27234	-137.81611427879\\
70.25	0.276	-142.496711667628\\
70.625	0.093	-9.76278261747844\\
70.625	0.09666	-10.4026208457931\\
70.625	0.10032	-11.1250414479158\\
70.625	0.10398	-11.9300444238467\\
70.625	0.10764	-12.8176297735856\\
70.625	0.1113	-13.7877974971327\\
70.625	0.11496	-14.8405475944879\\
70.625	0.11862	-15.9758800656511\\
70.625	0.12228	-17.1937949106226\\
70.625	0.12594	-18.494292129402\\
70.625	0.1296	-19.8773717219896\\
70.625	0.13326	-21.3430336883853\\
70.625	0.13692	-22.8912780285892\\
70.625	0.14058	-24.5221047426011\\
70.625	0.14424	-26.2355138304211\\
70.625	0.1479	-28.0315052920493\\
70.625	0.15156	-29.9100791274855\\
70.625	0.15522	-31.8712353367299\\
70.625	0.15888	-33.9149739197824\\
70.625	0.16254	-36.0412948766429\\
70.625	0.1662	-38.2501982073116\\
70.625	0.16986	-40.5416839117884\\
70.625	0.17352	-42.9157519900733\\
70.625	0.17718	-45.3724024421663\\
70.625	0.18084	-47.9116352680674\\
70.625	0.1845	-50.5334504677766\\
70.625	0.18816	-53.2378480412939\\
70.625	0.19182	-56.0248279886193\\
70.625	0.19548	-58.8943903097529\\
70.625	0.19914	-61.8465350046945\\
70.625	0.2028	-64.8812620734443\\
70.625	0.20646	-67.9985715160021\\
70.625	0.21012	-71.1984633323681\\
70.625	0.21378	-74.4809375225422\\
70.625	0.21744	-77.8459940865243\\
70.625	0.2211	-81.2936330243146\\
70.625	0.22476	-84.823854335913\\
70.625	0.22842	-88.4366580213195\\
70.625	0.23208	-92.1320440805341\\
70.625	0.23574	-95.9100125135569\\
70.625	0.2394	-99.7705633203877\\
70.625	0.24306	-103.713696501027\\
70.625	0.24672	-107.739412055474\\
70.625	0.25038	-111.847709983729\\
70.625	0.25404	-116.038590285792\\
70.625	0.2577	-120.312052961663\\
70.625	0.26136	-124.668098011343\\
70.625	0.26502	-129.10672543483\\
70.625	0.26868	-133.627935232126\\
70.625	0.27234	-138.23172740323\\
70.625	0.276	-142.918101948142\\
71	0.093	-9.90108181121599\\
71	0.09666	-10.5466971956045\\
71	0.10032	-11.2748949538012\\
71	0.10398	-12.0856750858059\\
71	0.10764	-12.9790375916188\\
71	0.1113	-13.9549824712398\\
71	0.11496	-15.0135097246688\\
71	0.11862	-16.154619351906\\
71	0.12228	-17.3783113529513\\
71	0.12594	-18.6845857278047\\
71	0.1296	-20.0734424764662\\
71	0.13326	-21.5448815989357\\
71	0.13692	-23.0989030952135\\
71	0.14058	-24.7355069652993\\
71	0.14424	-26.4546932091933\\
71	0.1479	-28.2564618268953\\
71	0.15156	-30.1408128184054\\
71	0.15522	-32.1077461837237\\
71	0.15888	-34.1572619228501\\
71	0.16254	-36.2893600357845\\
71	0.1662	-38.5040405225271\\
71	0.16986	-40.8013033830778\\
71	0.17352	-43.1811486174366\\
71	0.17718	-45.6435762256035\\
71	0.18084	-48.1885862075785\\
71	0.1845	-50.8161785633615\\
71	0.18816	-53.5263532929528\\
71	0.19182	-56.3191103963521\\
71	0.19548	-59.1944498735595\\
71	0.19914	-62.1523717245751\\
71	0.2028	-65.1928759493987\\
71	0.20646	-68.3159625480305\\
71	0.21012	-71.5216315204704\\
71	0.21378	-74.8098828667183\\
71	0.21744	-78.1807165867744\\
71	0.2211	-81.6341326806386\\
71	0.22476	-85.1701311483109\\
71	0.22842	-88.7887119897913\\
71	0.23208	-92.4898752050797\\
71	0.23574	-96.2736207941764\\
71	0.2394	-100.139948757081\\
71	0.24306	-104.088859093794\\
71	0.24672	-108.120351804315\\
71	0.25038	-112.234426888644\\
71	0.25404	-116.431084346781\\
71	0.2577	-120.710324178726\\
71	0.26136	-125.07214638448\\
71	0.26502	-129.516550964041\\
71	0.26868	-134.043537917411\\
71	0.27234	-138.653107244588\\
71	0.276	-143.345258945574\\
71.375	0.093	-10.045147721872\\
71.375	0.09666	-10.6965402623344\\
71.375	0.10032	-11.430515176605\\
71.375	0.10398	-12.2470724646836\\
71.375	0.10764	-13.1462121265704\\
71.375	0.1113	-14.1279341622653\\
71.375	0.11496	-15.1922385717682\\
71.375	0.11862	-16.3391253550793\\
71.375	0.12228	-17.5685945121985\\
71.375	0.12594	-18.8806460431258\\
71.375	0.1296	-20.2752799478611\\
71.375	0.13326	-21.7524962264046\\
71.375	0.13692	-23.3122948787562\\
71.375	0.14058	-24.954675904916\\
71.375	0.14424	-26.6796393048838\\
71.375	0.1479	-28.4871850786597\\
71.375	0.15156	-30.3773132262438\\
71.375	0.15522	-32.3500237476359\\
71.375	0.15888	-34.4053166428362\\
71.375	0.16254	-36.5431919118445\\
71.375	0.1662	-38.763649554661\\
71.375	0.16986	-41.0666895712856\\
71.375	0.17352	-43.4523119617183\\
71.375	0.17718	-45.9205167259591\\
71.375	0.18084	-48.471303864008\\
71.375	0.1845	-51.104673375865\\
71.375	0.18816	-53.8206252615301\\
71.375	0.19182	-56.6191595210033\\
71.375	0.19548	-59.5002761542846\\
71.375	0.19914	-62.4639751613741\\
71.375	0.2028	-65.5102565422716\\
71.375	0.20646	-68.6391202969773\\
71.375	0.21012	-71.8505664254911\\
71.375	0.21378	-75.1445949278129\\
71.375	0.21744	-78.5212058039428\\
71.375	0.2211	-81.9803990538809\\
71.375	0.22476	-85.5221746776271\\
71.375	0.22842	-89.1465326751814\\
71.375	0.23208	-92.8534730465438\\
71.375	0.23574	-96.6429957917144\\
71.375	0.2394	-100.515100910693\\
71.375	0.24306	-104.46978840348\\
71.375	0.24672	-108.507058270075\\
71.375	0.25038	-112.626910510477\\
71.375	0.25404	-116.829345124689\\
71.375	0.2577	-121.114362112708\\
71.375	0.26136	-125.481961474535\\
71.375	0.26502	-129.93214321017\\
71.375	0.26868	-134.464907319614\\
71.375	0.27234	-139.080253802865\\
71.375	0.276	-143.778182659925\\
71.75	0.093	-10.1949803494464\\
71.75	0.09666	-10.8521500459828\\
71.75	0.10032	-11.5919021163272\\
71.75	0.10398	-12.4142365604797\\
71.75	0.10764	-13.3191533784404\\
71.75	0.1113	-14.3066525702091\\
71.75	0.11496	-15.376734135786\\
71.75	0.11862	-16.529398075171\\
71.75	0.12228	-17.764644388364\\
71.75	0.12594	-19.0824730753652\\
71.75	0.1296	-20.4828841361745\\
71.75	0.13326	-21.9658775707919\\
71.75	0.13692	-23.5314533792174\\
71.75	0.14058	-25.1796115614511\\
71.75	0.14424	-26.9103521174928\\
71.75	0.1479	-28.7236750473426\\
71.75	0.15156	-30.6195803510005\\
71.75	0.15522	-32.5980680284666\\
71.75	0.15888	-34.6591380797407\\
71.75	0.16254	-36.802790504823\\
71.75	0.1662	-39.0290253037134\\
71.75	0.16986	-41.3378424764119\\
71.75	0.17352	-43.7292420229184\\
71.75	0.17718	-46.2032239432331\\
71.75	0.18084	-48.7597882373559\\
71.75	0.1845	-51.3989349052868\\
71.75	0.18816	-54.1206639470258\\
71.75	0.19182	-56.9249753625729\\
71.75	0.19548	-59.8118691519282\\
71.75	0.19914	-62.7813453150915\\
71.75	0.2028	-65.833403852063\\
71.75	0.20646	-68.9680447628425\\
71.75	0.21012	-72.1852680474301\\
71.75	0.21378	-75.4850737058259\\
71.75	0.21744	-78.8674617380298\\
71.75	0.2211	-82.3324321440417\\
71.75	0.22476	-85.8799849238618\\
71.75	0.22842	-89.51012007749\\
71.75	0.23208	-93.2228376049263\\
71.75	0.23574	-97.0181375061707\\
71.75	0.2394	-100.896019781223\\
71.75	0.24306	-104.856484430084\\
71.75	0.24672	-108.899531452753\\
71.75	0.25038	-113.025160849229\\
71.75	0.25404	-117.233372619514\\
71.75	0.2577	-121.524166763607\\
71.75	0.26136	-125.897543281509\\
71.75	0.26502	-130.353502173218\\
71.75	0.26868	-134.892043438735\\
71.75	0.27234	-139.513167078061\\
71.75	0.276	-144.216873091194\\
72.125	0.093	-10.3505796939394\\
72.125	0.09666	-11.0135265465496\\
72.125	0.10032	-11.7590557729679\\
72.125	0.10398	-12.5871673731943\\
72.125	0.10764	-13.4978613472288\\
72.125	0.1113	-14.4911376950715\\
72.125	0.11496	-15.5669964167223\\
72.125	0.11862	-16.7254375121812\\
72.125	0.12228	-17.9664609814482\\
72.125	0.12594	-19.2900668245232\\
72.125	0.1296	-20.6962550414064\\
72.125	0.13326	-22.1850256320977\\
72.125	0.13692	-23.7563785965971\\
72.125	0.14058	-25.4103139349046\\
72.125	0.14424	-27.1468316470203\\
72.125	0.1479	-28.965931732944\\
72.125	0.15156	-30.8676141926758\\
72.125	0.15522	-32.8518790262157\\
72.125	0.15888	-34.9187262335638\\
72.125	0.16254	-37.0681558147199\\
72.125	0.1662	-39.3001677696842\\
72.125	0.16986	-41.6147620984566\\
72.125	0.17352	-44.011938801037\\
72.125	0.17718	-46.4916978774256\\
72.125	0.18084	-49.0540393276223\\
72.125	0.1845	-51.6989631516271\\
72.125	0.18816	-54.42646934944\\
72.125	0.19182	-57.2365579210611\\
72.125	0.19548	-60.1292288664902\\
72.125	0.19914	-63.1044821857274\\
72.125	0.2028	-66.1623178787728\\
72.125	0.20646	-69.3027359456262\\
72.125	0.21012	-72.5257363862877\\
72.125	0.21378	-75.8313192007574\\
72.125	0.21744	-79.2194843890352\\
72.125	0.2211	-82.690231951121\\
72.125	0.22476	-86.243561887015\\
72.125	0.22842	-89.8794741967171\\
72.125	0.23208	-93.5979688802273\\
72.125	0.23574	-97.3990459375456\\
72.125	0.2394	-101.282705368672\\
72.125	0.24306	-105.248947173607\\
72.125	0.24672	-109.297771352349\\
72.125	0.25038	-113.4291779049\\
72.125	0.25404	-117.643166831259\\
72.125	0.2577	-121.939738131426\\
72.125	0.26136	-126.318891805401\\
72.125	0.26502	-130.780627853184\\
72.125	0.26868	-135.324946274775\\
72.125	0.27234	-139.951847070175\\
72.125	0.276	-144.661330239382\\
72.5	0.093	-10.5119457553507\\
72.5	0.09666	-11.1806697640348\\
72.5	0.10032	-11.931976146527\\
72.5	0.10398	-12.7658649028274\\
72.5	0.10764	-13.6823360329358\\
72.5	0.1113	-14.6813895368523\\
72.5	0.11496	-15.763025414577\\
72.5	0.11862	-16.9272436661098\\
72.5	0.12228	-18.1740442914507\\
72.5	0.12594	-19.5034272905996\\
72.5	0.1296	-20.9153926635567\\
72.5	0.13326	-22.4099404103219\\
72.5	0.13692	-23.9870705308952\\
72.5	0.14058	-25.6467830252766\\
72.5	0.14424	-27.3890778934661\\
72.5	0.1479	-29.2139551354637\\
72.5	0.15156	-31.1214147512695\\
72.5	0.15522	-33.1114567408833\\
72.5	0.15888	-35.1840811043052\\
72.5	0.16254	-37.3392878415353\\
72.5	0.1662	-39.5770769525735\\
72.5	0.16986	-41.8974484374198\\
72.5	0.17352	-44.3004022960741\\
72.5	0.17718	-46.7859385285366\\
72.5	0.18084	-49.3540571348072\\
72.5	0.1845	-52.0047581148859\\
72.5	0.18816	-54.7380414687727\\
72.5	0.19182	-57.5539071964676\\
72.5	0.19548	-60.4523552979707\\
72.5	0.19914	-63.4333857732817\\
72.5	0.2028	-66.496998622401\\
72.5	0.20646	-69.6431938453284\\
72.5	0.21012	-72.8719714420638\\
72.5	0.21378	-76.1833314126073\\
72.5	0.21744	-79.577273756959\\
72.5	0.2211	-83.0537984751188\\
72.5	0.22476	-86.6129055670866\\
72.5	0.22842	-90.2545950328626\\
72.5	0.23208	-93.9788668724467\\
72.5	0.23574	-97.7857210858389\\
72.5	0.2394	-101.675157673039\\
72.5	0.24306	-105.647176634048\\
72.5	0.24672	-109.701777968864\\
72.5	0.25038	-113.838961677489\\
72.5	0.25404	-118.058727759922\\
72.5	0.2577	-122.361076216162\\
72.5	0.26136	-126.746007046211\\
72.5	0.26502	-131.213520250068\\
72.5	0.26868	-135.763615827734\\
72.5	0.27234	-140.396293779207\\
72.5	0.276	-145.111554104488\\
72.875	0.093	-10.6790785336805\\
72.875	0.09666	-11.3535796984385\\
72.875	0.10032	-12.1106632370046\\
72.875	0.10398	-12.9503291493789\\
72.875	0.10764	-13.8725774355612\\
72.875	0.1113	-14.8774080955516\\
72.875	0.11496	-15.9648211293502\\
72.875	0.11862	-17.1348165369568\\
72.875	0.12228	-18.3873943183716\\
72.875	0.12594	-19.7225544735945\\
72.875	0.1296	-21.1402970026255\\
72.875	0.13326	-22.6406219054646\\
72.875	0.13692	-24.2235291821117\\
72.875	0.14058	-25.8890188325671\\
72.875	0.14424	-27.6370908568305\\
72.875	0.1479	-29.467745254902\\
72.875	0.15156	-31.3809820267816\\
72.875	0.15522	-33.3768011724694\\
72.875	0.15888	-35.4552026919652\\
72.875	0.16254	-37.6161865852691\\
72.875	0.1662	-39.8597528523812\\
72.875	0.16986	-42.1859014933014\\
72.875	0.17352	-44.5946325080296\\
72.875	0.17718	-47.085945896566\\
72.875	0.18084	-49.6598416589105\\
72.875	0.1845	-52.3163197950631\\
72.875	0.18816	-55.0553803050238\\
72.875	0.19182	-57.8770231887926\\
72.875	0.19548	-60.7812484463695\\
72.875	0.19914	-63.7680560777545\\
72.875	0.2028	-66.8374460829477\\
72.875	0.20646	-69.989418461949\\
72.875	0.21012	-73.2239732147583\\
72.875	0.21378	-76.5411103413757\\
72.875	0.21744	-79.9408298418013\\
72.875	0.2211	-83.4231317160349\\
72.875	0.22476	-86.9880159640767\\
72.875	0.22842	-90.6354825859266\\
72.875	0.23208	-94.3655315815846\\
72.875	0.23574	-98.1781629510507\\
72.875	0.2394	-102.073376694325\\
72.875	0.24306	-106.051172811407\\
72.875	0.24672	-110.111551302298\\
72.875	0.25038	-114.254512166996\\
72.875	0.25404	-118.480055405503\\
72.875	0.2577	-122.788181017818\\
72.875	0.26136	-127.17888900394\\
72.875	0.26502	-131.652179363871\\
72.875	0.26868	-136.20805209761\\
72.875	0.27234	-140.846507205158\\
72.875	0.276	-145.567544686513\\
73.25	0.093	-10.8519780289287\\
73.25	0.09666	-11.5322563497606\\
73.25	0.10032	-12.2951170444007\\
73.25	0.10398	-13.1405601128488\\
73.25	0.10764	-14.068585555105\\
73.25	0.1113	-15.0791933711694\\
73.25	0.11496	-16.1723835610418\\
73.25	0.11862	-17.3481561247224\\
73.25	0.12228	-18.606511062211\\
73.25	0.12594	-19.9474483735078\\
73.25	0.1296	-21.3709680586127\\
73.25	0.13326	-22.8770701175257\\
73.25	0.13692	-24.4657545502467\\
73.25	0.14058	-26.137021356776\\
73.25	0.14424	-27.8908705371133\\
73.25	0.1479	-29.7273020912587\\
73.25	0.15156	-31.6463160192122\\
73.25	0.15522	-33.6479123209739\\
73.25	0.15888	-35.7320909965436\\
73.25	0.16254	-37.8988520459214\\
73.25	0.1662	-40.1481954691074\\
73.25	0.16986	-42.4801212661015\\
73.25	0.17352	-44.8946294369036\\
73.25	0.17718	-47.3917199815139\\
73.25	0.18084	-49.9713928999323\\
73.25	0.1845	-52.6336481921588\\
73.25	0.18816	-55.3784858581934\\
73.25	0.19182	-58.2059058980361\\
73.25	0.19548	-61.1159083116869\\
73.25	0.19914	-64.1084930991458\\
73.25	0.2028	-67.1836602604129\\
73.25	0.20646	-70.341409795488\\
73.25	0.21012	-73.5817417043712\\
73.25	0.21378	-76.9046559870626\\
73.25	0.21744	-80.310152643562\\
73.25	0.2211	-83.7982316738696\\
73.25	0.22476	-87.3688930779852\\
73.25	0.22842	-91.022136855909\\
73.25	0.23208	-94.7579630076409\\
73.25	0.23574	-98.5763715331809\\
73.25	0.2394	-102.477362432529\\
73.25	0.24306	-106.460935705685\\
73.25	0.24672	-110.52709135265\\
73.25	0.25038	-114.675829373422\\
73.25	0.25404	-118.907149768002\\
73.25	0.2577	-123.221052536391\\
73.25	0.26136	-127.617537678588\\
73.25	0.26502	-132.096605194593\\
73.25	0.26868	-136.658255084406\\
73.25	0.27234	-141.302487348027\\
73.25	0.276	-146.029301985456\\
73.625	0.093	-11.0306442410954\\
73.625	0.09666	-11.7166997180012\\
73.625	0.10032	-12.4853375687152\\
73.625	0.10398	-13.3365577932372\\
73.625	0.10764	-14.2703603915673\\
73.625	0.1113	-15.2867453637055\\
73.625	0.11496	-16.3857127096519\\
73.625	0.11862	-17.5672624294064\\
73.625	0.12228	-18.8313945229689\\
73.625	0.12594	-20.1781089903396\\
73.625	0.1296	-21.6074058315184\\
73.625	0.13326	-23.1192850465052\\
73.625	0.13692	-24.7137466353002\\
73.625	0.14058	-26.3907905979033\\
73.625	0.14424	-28.1504169343145\\
73.625	0.1479	-29.9926256445339\\
73.625	0.15156	-31.9174167285613\\
73.625	0.15522	-33.9247901863968\\
73.625	0.15888	-36.0147460180404\\
73.625	0.16254	-38.1872842234922\\
73.625	0.1662	-40.442404802752\\
73.625	0.16986	-42.78010775582\\
73.625	0.17352	-45.200393082696\\
73.625	0.17718	-47.7032607833802\\
73.625	0.18084	-50.2887108578725\\
73.625	0.1845	-52.9567433061729\\
73.625	0.18816	-55.7073581282814\\
73.625	0.19182	-58.540555324198\\
73.625	0.19548	-61.4563348939227\\
73.625	0.19914	-64.4546968374555\\
73.625	0.2028	-67.5356411547965\\
73.625	0.20646	-70.6991678459455\\
73.625	0.21012	-73.9452769109026\\
73.625	0.21378	-77.2739683496678\\
73.625	0.21744	-80.6852421622412\\
73.625	0.2211	-84.1790983486227\\
73.625	0.22476	-87.7555369088122\\
73.625	0.22842	-91.4145578428099\\
73.625	0.23208	-95.1561611506157\\
73.625	0.23574	-98.9803468322296\\
73.625	0.2394	-102.887114887652\\
73.625	0.24306	-106.876465316882\\
73.625	0.24672	-110.94839811992\\
73.625	0.25038	-115.102913296766\\
73.625	0.25404	-119.340010847421\\
73.625	0.2577	-123.659690771883\\
73.625	0.26136	-128.061953070154\\
73.625	0.26502	-132.546797742233\\
73.625	0.26868	-137.114224788119\\
73.625	0.27234	-141.764234207814\\
73.625	0.276	-146.496826001317\\
74	0.093	-11.2150771701806\\
74	0.09666	-11.9069098031603\\
74	0.10032	-12.6813248099481\\
74	0.10398	-13.538322190544\\
74	0.10764	-14.4779019449481\\
74	0.1113	-15.5000640731602\\
74	0.11496	-16.6048085751805\\
74	0.11862	-17.7921354510088\\
74	0.12228	-19.0620447006452\\
74	0.12594	-20.4145363240898\\
74	0.1296	-21.8496103213425\\
74	0.13326	-23.3672666924033\\
74	0.13692	-24.9675054372722\\
74	0.14058	-26.6503265559491\\
74	0.14424	-28.4157300484343\\
74	0.1479	-30.2637159147275\\
74	0.15156	-32.1942841548288\\
74	0.15522	-34.2074347687382\\
74	0.15888	-36.3031677564557\\
74	0.16254	-38.4814831179814\\
74	0.1662	-40.7423808533151\\
74	0.16986	-43.085860962457\\
74	0.17352	-45.5119234454069\\
74	0.17718	-48.020568302165\\
74	0.18084	-50.6117955327312\\
74	0.1845	-53.2856051371055\\
74	0.18816	-56.0419971152878\\
74	0.19182	-58.8809714672784\\
74	0.19548	-61.802528193077\\
74	0.19914	-64.8066672926836\\
74	0.2028	-67.8933887660985\\
74	0.20646	-71.0626926133214\\
74	0.21012	-74.3145788343524\\
74	0.21378	-77.6490474291916\\
74	0.21744	-81.0660983978388\\
74	0.2211	-84.5657317402942\\
74	0.22476	-88.1479474565576\\
74	0.22842	-91.8127455466292\\
74	0.23208	-95.5601260105089\\
74	0.23574	-99.3900888481967\\
74	0.2394	-103.302634059693\\
74	0.24306	-107.297761644997\\
74	0.24672	-111.375471604109\\
74	0.25038	-115.535763937029\\
74	0.25404	-119.778638643757\\
74	0.2577	-124.104095724294\\
74	0.26136	-128.512135178638\\
74	0.26502	-133.002757006791\\
74	0.26868	-137.575961208752\\
74	0.27234	-142.23174778452\\
74	0.276	-146.970116734097\\
};
\end{axis}

\begin{axis}[%
width=4.527496cm,
height=3.050847cm,
at={(0cm,4.237288cm)},
scale only axis,
xmin=56,
xmax=74,
tick align=outside,
xlabel={$L_{cut}$},
xmajorgrids,
ymin=0.093,
ymax=0.276,
ylabel={$D_{rlx}$},
ymajorgrids,
zmin=-2000,
zmax=0,
zlabel={$c$},
zmajorgrids,
view={-140}{50},
legend style={at={(1.03,1)},anchor=north west,legend cell align=left,align=left,draw=white!15!black}
]
\addplot3[only marks,mark=*,mark options={},mark size=1.5000pt,color=mycolor1] plot table[row sep=crcr,]{%
74	0.123	-226.026821983115\\
72	0.113	-187.694094163243\\
61	0.095	-100.49997137603\\
56	0.093	-116.40678499957\\
};
\addplot3[only marks,mark=*,mark options={},mark size=1.5000pt,color=mycolor2] plot table[row sep=crcr,]{%
67	0.276	-1722.46689656527\\
66	0.255	-1432.39811434881\\
62	0.209	-789.682056866044\\
57	0.193	-632.303972274105\\
};
\addplot3[only marks,mark=*,mark options={},mark size=1.5000pt,color=black] plot table[row sep=crcr,]{%
69	0.104	-134.758100217952\\
};
\addplot3[only marks,mark=*,mark options={},mark size=1.5000pt,color=black] plot table[row sep=crcr,]{%
64	0.23	-1056.40939583942\\
};

\addplot3[%
surf,
opacity=0.7,
shader=interp,
colormap={mymap}{[1pt] rgb(0pt)=(0.0901961,0.239216,0.0745098); rgb(1pt)=(0.0945149,0.242058,0.0739522); rgb(2pt)=(0.0988592,0.244894,0.0733566); rgb(3pt)=(0.103229,0.247724,0.0727241); rgb(4pt)=(0.107623,0.250549,0.0720557); rgb(5pt)=(0.112043,0.253367,0.0713525); rgb(6pt)=(0.116487,0.25618,0.0706154); rgb(7pt)=(0.120956,0.258986,0.0698456); rgb(8pt)=(0.125449,0.261787,0.0690441); rgb(9pt)=(0.129967,0.264581,0.0682118); rgb(10pt)=(0.134508,0.26737,0.06735); rgb(11pt)=(0.139074,0.270152,0.0664596); rgb(12pt)=(0.143663,0.272929,0.0655416); rgb(13pt)=(0.148275,0.275699,0.0645971); rgb(14pt)=(0.152911,0.278463,0.0636271); rgb(15pt)=(0.15757,0.281221,0.0626328); rgb(16pt)=(0.162252,0.283973,0.0616151); rgb(17pt)=(0.166957,0.286719,0.060575); rgb(18pt)=(0.171685,0.289458,0.0595136); rgb(19pt)=(0.176434,0.292191,0.0584321); rgb(20pt)=(0.181207,0.294918,0.0573313); rgb(21pt)=(0.186001,0.297639,0.0562123); rgb(22pt)=(0.190817,0.300353,0.0550763); rgb(23pt)=(0.195655,0.303061,0.0539242); rgb(24pt)=(0.200514,0.305763,0.052757); rgb(25pt)=(0.205395,0.308459,0.0515759); rgb(26pt)=(0.210296,0.311149,0.0503624); rgb(27pt)=(0.215212,0.313846,0.0490067); rgb(28pt)=(0.220142,0.316548,0.0475043); rgb(29pt)=(0.22509,0.319254,0.0458704); rgb(30pt)=(0.230056,0.321962,0.0441205); rgb(31pt)=(0.235042,0.324671,0.04227); rgb(32pt)=(0.240048,0.327379,0.0403343); rgb(33pt)=(0.245078,0.330085,0.0383287); rgb(34pt)=(0.250131,0.332786,0.0362688); rgb(35pt)=(0.25521,0.335482,0.0341698); rgb(36pt)=(0.260317,0.33817,0.0320472); rgb(37pt)=(0.265451,0.340849,0.0299163); rgb(38pt)=(0.270616,0.343517,0.0277927); rgb(39pt)=(0.275813,0.346172,0.0256916); rgb(40pt)=(0.281043,0.348814,0.0236284); rgb(41pt)=(0.286307,0.35144,0.0216186); rgb(42pt)=(0.291607,0.354048,0.0196776); rgb(43pt)=(0.296945,0.356637,0.0178207); rgb(44pt)=(0.302322,0.359206,0.0160634); rgb(45pt)=(0.307739,0.361753,0.0144211); rgb(46pt)=(0.313198,0.364275,0.0129091); rgb(47pt)=(0.318701,0.366772,0.0115428); rgb(48pt)=(0.324249,0.369242,0.0103377); rgb(49pt)=(0.329843,0.371682,0.00930909); rgb(50pt)=(0.335485,0.374093,0.00847245); rgb(51pt)=(0.341176,0.376471,0.00784314); rgb(52pt)=(0.346925,0.378826,0.00732741); rgb(53pt)=(0.352735,0.381168,0.00682184); rgb(54pt)=(0.358605,0.383497,0.00632729); rgb(55pt)=(0.364532,0.385812,0.00584464); rgb(56pt)=(0.370516,0.388113,0.00537476); rgb(57pt)=(0.376552,0.390399,0.00491852); rgb(58pt)=(0.38264,0.39267,0.00447681); rgb(59pt)=(0.388777,0.394925,0.00405048); rgb(60pt)=(0.394962,0.397164,0.00364042); rgb(61pt)=(0.401191,0.399386,0.00324749); rgb(62pt)=(0.407464,0.401592,0.00287258); rgb(63pt)=(0.413777,0.40378,0.00251655); rgb(64pt)=(0.420129,0.40595,0.00218028); rgb(65pt)=(0.426518,0.408102,0.00186463); rgb(66pt)=(0.432942,0.410234,0.00157049); rgb(67pt)=(0.439399,0.412348,0.00129873); rgb(68pt)=(0.445885,0.414441,0.00105022); rgb(69pt)=(0.452401,0.416515,0.000825833); rgb(70pt)=(0.458942,0.418567,0.000626441); rgb(71pt)=(0.465508,0.420599,0.00045292); rgb(72pt)=(0.472096,0.422609,0.000306141); rgb(73pt)=(0.478704,0.424596,0.000186979); rgb(74pt)=(0.485331,0.426562,9.63073e-05); rgb(75pt)=(0.491973,0.428504,3.49981e-05); rgb(76pt)=(0.498628,0.430422,3.92506e-06); rgb(77pt)=(0.505323,0.432315,0); rgb(78pt)=(0.512206,0.434168,0); rgb(79pt)=(0.519282,0.435983,0); rgb(80pt)=(0.526529,0.437764,0); rgb(81pt)=(0.533922,0.439512,0); rgb(82pt)=(0.54144,0.441232,0); rgb(83pt)=(0.549059,0.442927,0); rgb(84pt)=(0.556756,0.444599,0); rgb(85pt)=(0.564508,0.446252,0); rgb(86pt)=(0.572292,0.447889,0); rgb(87pt)=(0.580084,0.449514,0); rgb(88pt)=(0.587863,0.451129,0); rgb(89pt)=(0.595604,0.452737,0); rgb(90pt)=(0.603284,0.454343,0); rgb(91pt)=(0.610882,0.455948,0); rgb(92pt)=(0.618373,0.457556,0); rgb(93pt)=(0.625734,0.459171,0); rgb(94pt)=(0.632943,0.460795,0); rgb(95pt)=(0.639976,0.462432,0); rgb(96pt)=(0.64681,0.464084,0); rgb(97pt)=(0.653423,0.465756,0); rgb(98pt)=(0.659791,0.46745,0); rgb(99pt)=(0.665891,0.469169,0); rgb(100pt)=(0.6717,0.470916,0); rgb(101pt)=(0.677195,0.472696,0); rgb(102pt)=(0.682353,0.47451,0); rgb(103pt)=(0.687242,0.476355,0); rgb(104pt)=(0.691952,0.478225,0); rgb(105pt)=(0.696497,0.480118,0); rgb(106pt)=(0.700887,0.482033,0); rgb(107pt)=(0.705134,0.483968,0); rgb(108pt)=(0.709251,0.485921,0); rgb(109pt)=(0.713249,0.487891,0); rgb(110pt)=(0.71714,0.489876,0); rgb(111pt)=(0.720936,0.491875,0); rgb(112pt)=(0.724649,0.493887,0); rgb(113pt)=(0.72829,0.495909,0); rgb(114pt)=(0.731872,0.49794,0); rgb(115pt)=(0.735406,0.499979,0); rgb(116pt)=(0.738904,0.502025,0); rgb(117pt)=(0.742378,0.504075,0); rgb(118pt)=(0.74584,0.506128,0); rgb(119pt)=(0.749302,0.508182,0); rgb(120pt)=(0.752775,0.510237,0); rgb(121pt)=(0.756272,0.51229,0); rgb(122pt)=(0.759804,0.514339,0); rgb(123pt)=(0.763384,0.516385,0); rgb(124pt)=(0.767022,0.518424,0); rgb(125pt)=(0.770731,0.520455,0); rgb(126pt)=(0.774523,0.522478,0); rgb(127pt)=(0.77841,0.524489,0); rgb(128pt)=(0.782391,0.526491,0); rgb(129pt)=(0.786402,0.528496,0); rgb(130pt)=(0.790431,0.530506,0); rgb(131pt)=(0.794478,0.532521,0); rgb(132pt)=(0.798541,0.534539,0); rgb(133pt)=(0.802619,0.53656,0); rgb(134pt)=(0.806712,0.538584,0); rgb(135pt)=(0.81082,0.540609,0); rgb(136pt)=(0.81494,0.542635,0); rgb(137pt)=(0.819074,0.54466,0); rgb(138pt)=(0.823219,0.546686,0); rgb(139pt)=(0.827374,0.548709,0); rgb(140pt)=(0.831541,0.55073,0); rgb(141pt)=(0.835716,0.552749,0); rgb(142pt)=(0.8399,0.554763,0); rgb(143pt)=(0.844092,0.556774,0); rgb(144pt)=(0.848292,0.558779,0); rgb(145pt)=(0.852497,0.560778,0); rgb(146pt)=(0.856708,0.562771,0); rgb(147pt)=(0.860924,0.564756,0); rgb(148pt)=(0.865143,0.566733,0); rgb(149pt)=(0.869366,0.568701,0); rgb(150pt)=(0.873592,0.57066,0); rgb(151pt)=(0.877819,0.572608,0); rgb(152pt)=(0.882047,0.574545,0); rgb(153pt)=(0.886275,0.576471,0); rgb(154pt)=(0.890659,0.578362,0); rgb(155pt)=(0.895333,0.580203,0); rgb(156pt)=(0.900258,0.581999,0); rgb(157pt)=(0.905397,0.583755,0); rgb(158pt)=(0.910711,0.585479,0); rgb(159pt)=(0.916164,0.587176,0); rgb(160pt)=(0.921717,0.588852,0); rgb(161pt)=(0.927333,0.590513,0); rgb(162pt)=(0.932974,0.592166,0); rgb(163pt)=(0.938602,0.593815,0); rgb(164pt)=(0.94418,0.595468,0); rgb(165pt)=(0.949669,0.59713,0); rgb(166pt)=(0.955033,0.598808,0); rgb(167pt)=(0.960233,0.600507,0); rgb(168pt)=(0.965232,0.602233,0); rgb(169pt)=(0.969992,0.603992,0); rgb(170pt)=(0.974475,0.605791,0); rgb(171pt)=(0.978643,0.607636,0); rgb(172pt)=(0.98246,0.609532,0); rgb(173pt)=(0.985886,0.611486,0); rgb(174pt)=(0.988885,0.613503,0); rgb(175pt)=(0.991419,0.61559,0); rgb(176pt)=(0.99345,0.617753,0); rgb(177pt)=(0.99494,0.619997,0); rgb(178pt)=(0.995851,0.622329,0); rgb(179pt)=(0.996226,0.624763,0); rgb(180pt)=(0.996512,0.627352,0); rgb(181pt)=(0.996788,0.630095,0); rgb(182pt)=(0.997053,0.632982,0); rgb(183pt)=(0.997308,0.636004,0); rgb(184pt)=(0.997552,0.639152,0); rgb(185pt)=(0.997785,0.642416,0); rgb(186pt)=(0.998006,0.645786,0); rgb(187pt)=(0.998217,0.649253,0); rgb(188pt)=(0.998416,0.652807,0); rgb(189pt)=(0.998605,0.656439,0); rgb(190pt)=(0.998781,0.660138,0); rgb(191pt)=(0.998946,0.663897,0); rgb(192pt)=(0.9991,0.667704,0); rgb(193pt)=(0.999242,0.67155,0); rgb(194pt)=(0.999372,0.675427,0); rgb(195pt)=(0.99949,0.679323,0); rgb(196pt)=(0.999596,0.68323,0); rgb(197pt)=(0.99969,0.687139,0); rgb(198pt)=(0.999771,0.691039,0); rgb(199pt)=(0.999841,0.694921,0); rgb(200pt)=(0.999898,0.698775,0); rgb(201pt)=(0.999942,0.702592,0); rgb(202pt)=(0.999974,0.706363,0); rgb(203pt)=(0.999994,0.710077,0); rgb(204pt)=(1,0.713725,0); rgb(205pt)=(1,0.717341,0); rgb(206pt)=(1,0.720963,0); rgb(207pt)=(1,0.724591,0); rgb(208pt)=(1,0.728226,0); rgb(209pt)=(1,0.731867,0); rgb(210pt)=(1,0.735514,0); rgb(211pt)=(1,0.739167,0); rgb(212pt)=(1,0.742827,0); rgb(213pt)=(1,0.746493,0); rgb(214pt)=(1,0.750165,0); rgb(215pt)=(1,0.753843,0); rgb(216pt)=(1,0.757527,0); rgb(217pt)=(1,0.761217,0); rgb(218pt)=(1,0.764913,0); rgb(219pt)=(1,0.768615,0); rgb(220pt)=(1,0.772324,0); rgb(221pt)=(1,0.776038,0); rgb(222pt)=(1,0.779758,0); rgb(223pt)=(1,0.783484,0); rgb(224pt)=(1,0.787215,0); rgb(225pt)=(1,0.790953,0); rgb(226pt)=(1,0.794696,0); rgb(227pt)=(1,0.798445,0); rgb(228pt)=(1,0.8022,0); rgb(229pt)=(1,0.805961,0); rgb(230pt)=(1,0.809727,0); rgb(231pt)=(1,0.8135,0); rgb(232pt)=(1,0.817278,0); rgb(233pt)=(1,0.821063,0); rgb(234pt)=(1,0.824854,0); rgb(235pt)=(1,0.828652,0); rgb(236pt)=(1,0.832455,0); rgb(237pt)=(1,0.836265,0); rgb(238pt)=(1,0.840081,0); rgb(239pt)=(1,0.843903,0); rgb(240pt)=(1,0.847732,0); rgb(241pt)=(1,0.851566,0); rgb(242pt)=(1,0.855406,0); rgb(243pt)=(1,0.859253,0); rgb(244pt)=(1,0.863106,0); rgb(245pt)=(1,0.866964,0); rgb(246pt)=(1,0.870829,0); rgb(247pt)=(1,0.8747,0); rgb(248pt)=(1,0.878577,0); rgb(249pt)=(1,0.88246,0); rgb(250pt)=(1,0.886349,0); rgb(251pt)=(1,0.890243,0); rgb(252pt)=(1,0.894144,0); rgb(253pt)=(1,0.898051,0); rgb(254pt)=(1,0.901964,0); rgb(255pt)=(1,0.905882,0)},
mesh/rows=49]
table[row sep=crcr,header=false] {%
%
56	0.093	-114.326248720101\\
56	0.09666	-119.406198490127\\
56	0.10032	-125.528387221123\\
56	0.10398	-132.692814913087\\
56	0.10764	-140.899481566021\\
56	0.1113	-150.148387179923\\
56	0.11496	-160.439531754795\\
56	0.11862	-171.772915290636\\
56	0.12228	-184.148537787446\\
56	0.12594	-197.566399245225\\
56	0.1296	-212.026499663973\\
56	0.13326	-227.528839043689\\
56	0.13692	-244.073417384376\\
56	0.14058	-261.660234686031\\
56	0.14424	-280.289290948655\\
56	0.1479	-299.960586172249\\
56	0.15156	-320.674120356811\\
56	0.15522	-342.429893502342\\
56	0.15888	-365.227905608843\\
56	0.16254	-389.068156676312\\
56	0.1662	-413.950646704751\\
56	0.16986	-439.875375694159\\
56	0.17352	-466.842343644536\\
56	0.17718	-494.851550555882\\
56	0.18084	-523.902996428196\\
56	0.1845	-553.996681261481\\
56	0.18816	-585.132605055734\\
56	0.19182	-617.310767810956\\
56	0.19548	-650.531169527146\\
56	0.19914	-684.793810204307\\
56	0.2028	-720.098689842437\\
56	0.20646	-756.445808441535\\
56	0.21012	-793.835166001602\\
56	0.21378	-832.266762522639\\
56	0.21744	-871.740598004644\\
56	0.2211	-912.256672447619\\
56	0.22476	-953.814985851563\\
56	0.22842	-996.415538216475\\
56	0.23208	-1040.05832954236\\
56	0.23574	-1084.74335982921\\
56	0.2394	-1130.47062907703\\
56	0.24306	-1177.24013728582\\
56	0.24672	-1225.05188445558\\
56	0.25038	-1273.9058705863\\
56	0.25404	-1323.802095678\\
56	0.2577	-1374.74055973066\\
56	0.26136	-1426.7212627443\\
56	0.26502	-1479.7442047189\\
56	0.26868	-1533.80938565447\\
56	0.27234	-1588.91680555102\\
56	0.276	-1645.06646440853\\
56.375	0.093	-112.784675935863\\
56.375	0.09666	-117.93510815511\\
56.375	0.10032	-124.127779335325\\
56.375	0.10398	-131.362689476509\\
56.375	0.10764	-139.639838578663\\
56.375	0.1113	-148.959226641785\\
56.375	0.11496	-159.320853665877\\
56.375	0.11862	-170.724719650937\\
56.375	0.12228	-183.170824596967\\
56.375	0.12594	-196.659168503965\\
56.375	0.1296	-211.189751371933\\
56.375	0.13326	-226.76257320087\\
56.375	0.13692	-243.377633990776\\
56.375	0.14058	-261.034933741651\\
56.375	0.14424	-279.734472453495\\
56.375	0.1479	-299.476250126308\\
56.375	0.15156	-320.26026676009\\
56.375	0.15522	-342.086522354841\\
56.375	0.15888	-364.955016910562\\
56.375	0.16254	-388.865750427251\\
56.375	0.1662	-413.81872290491\\
56.375	0.16986	-439.813934343537\\
56.375	0.17352	-466.851384743133\\
56.375	0.17718	-494.931074103699\\
56.375	0.18084	-524.053002425233\\
56.375	0.1845	-554.217169707738\\
56.375	0.18816	-585.423575951211\\
56.375	0.19182	-617.672221155653\\
56.375	0.19548	-650.963105321064\\
56.375	0.19914	-685.296228447444\\
56.375	0.2028	-720.671590534793\\
56.375	0.20646	-757.089191583111\\
56.375	0.21012	-794.549031592398\\
56.375	0.21378	-833.051110562654\\
56.375	0.21744	-872.595428493879\\
56.375	0.2211	-913.181985386074\\
56.375	0.22476	-954.810781239237\\
56.375	0.22842	-997.48181605337\\
56.375	0.23208	-1041.19508982847\\
56.375	0.23574	-1085.95060256454\\
56.375	0.2394	-1131.74835426158\\
56.375	0.24306	-1178.58834491959\\
56.375	0.24672	-1226.47057453857\\
56.375	0.25038	-1275.39504311852\\
56.375	0.25404	-1325.36175065943\\
56.375	0.2577	-1376.37069716132\\
56.375	0.26136	-1428.42188262417\\
56.375	0.26502	-1481.515307048\\
56.375	0.26868	-1535.65097043279\\
56.375	0.27234	-1590.82887277855\\
56.375	0.276	-1647.04901408528\\
56.75	0.093	-111.325564769939\\
56.75	0.09666	-116.546479438405\\
56.75	0.10032	-122.80963306784\\
56.75	0.10398	-130.115025658244\\
56.75	0.10764	-138.462657209617\\
56.75	0.1113	-147.85252772196\\
56.75	0.11496	-158.284637195271\\
56.75	0.11862	-169.758985629551\\
56.75	0.12228	-182.275573024801\\
56.75	0.12594	-195.834399381019\\
56.75	0.1296	-210.435464698207\\
56.75	0.13326	-226.078768976363\\
56.75	0.13692	-242.764312215489\\
56.75	0.14058	-260.492094415584\\
56.75	0.14424	-279.262115576648\\
56.75	0.1479	-299.07437569868\\
56.75	0.15156	-319.928874781682\\
56.75	0.15522	-341.825612825653\\
56.75	0.15888	-364.764589830594\\
56.75	0.16254	-388.745805796503\\
56.75	0.1662	-413.769260723381\\
56.75	0.16986	-439.834954611228\\
56.75	0.17352	-466.942887460045\\
56.75	0.17718	-495.09305926983\\
56.75	0.18084	-524.285470040585\\
56.75	0.1845	-554.520119772308\\
56.75	0.18816	-585.797008465001\\
56.75	0.19182	-618.116136118662\\
56.75	0.19548	-651.477502733293\\
56.75	0.19914	-685.881108308893\\
56.75	0.2028	-721.326952845462\\
56.75	0.20646	-757.815036343\\
56.75	0.21012	-795.345358801507\\
56.75	0.21378	-833.917920220983\\
56.75	0.21744	-873.532720601428\\
56.75	0.2211	-914.189759942843\\
56.75	0.22476	-955.889038245226\\
56.75	0.22842	-998.630555508578\\
56.75	0.23208	-1042.4143117329\\
56.75	0.23574	-1087.24030691819\\
56.75	0.2394	-1133.10854106445\\
56.75	0.24306	-1180.01901417168\\
56.75	0.24672	-1227.97172623988\\
56.75	0.25038	-1276.96667726904\\
56.75	0.25404	-1327.00386725918\\
56.75	0.2577	-1378.08329621028\\
56.75	0.26136	-1430.20496412236\\
56.75	0.26502	-1483.3688709954\\
56.75	0.26868	-1537.57501682941\\
56.75	0.27234	-1592.82340162439\\
56.75	0.276	-1649.11402538034\\
57.125	0.093	-109.948915222327\\
57.125	0.09666	-115.240312340013\\
57.125	0.10032	-121.573948418668\\
57.125	0.10398	-128.949823458292\\
57.125	0.10764	-137.367937458885\\
57.125	0.1113	-146.828290420447\\
57.125	0.11496	-157.330882342978\\
57.125	0.11862	-168.875713226478\\
57.125	0.12228	-181.462783070947\\
57.125	0.12594	-195.092091876386\\
57.125	0.1296	-209.763639642793\\
57.125	0.13326	-225.477426370169\\
57.125	0.13692	-242.233452058514\\
57.125	0.14058	-260.031716707829\\
57.125	0.14424	-278.872220318113\\
57.125	0.1479	-298.754962889366\\
57.125	0.15156	-319.679944421587\\
57.125	0.15522	-341.647164914778\\
57.125	0.15888	-364.656624368938\\
57.125	0.16254	-388.708322784067\\
57.125	0.1662	-413.802260160165\\
57.125	0.16986	-439.938436497232\\
57.125	0.17352	-467.116851795268\\
57.125	0.17718	-495.337506054273\\
57.125	0.18084	-524.600399274247\\
57.125	0.1845	-554.905531455191\\
57.125	0.18816	-586.252902597104\\
57.125	0.19182	-618.642512699985\\
57.125	0.19548	-652.074361763836\\
57.125	0.19914	-686.548449788655\\
57.125	0.2028	-722.064776774444\\
57.125	0.20646	-758.623342721202\\
57.125	0.21012	-796.224147628929\\
57.125	0.21378	-834.867191497625\\
57.125	0.21744	-874.552474327289\\
57.125	0.2211	-915.279996117923\\
57.125	0.22476	-957.049756869526\\
57.125	0.22842	-999.861756582099\\
57.125	0.23208	-1043.71599525564\\
57.125	0.23574	-1088.61247289015\\
57.125	0.2394	-1134.55118948563\\
57.125	0.24306	-1181.53214504208\\
57.125	0.24672	-1229.5553395595\\
57.125	0.25038	-1278.62077303788\\
57.125	0.25404	-1328.72844547724\\
57.125	0.2577	-1379.87835687756\\
57.125	0.26136	-1432.07050723886\\
57.125	0.26502	-1485.30489656112\\
57.125	0.26868	-1539.58152484435\\
57.125	0.27234	-1594.90039208855\\
57.125	0.276	-1651.26149829372\\
57.5	0.093	-108.654727293029\\
57.5	0.09666	-114.016606859934\\
57.5	0.10032	-120.420725387809\\
57.5	0.10398	-127.867082876652\\
57.5	0.10764	-136.355679326465\\
57.5	0.1113	-145.886514737247\\
57.5	0.11496	-156.459589108998\\
57.5	0.11862	-168.074902441718\\
57.5	0.12228	-180.732454735407\\
57.5	0.12594	-194.432245990065\\
57.5	0.1296	-209.174276205692\\
57.5	0.13326	-224.958545382288\\
57.5	0.13692	-241.785053519853\\
57.5	0.14058	-259.653800618388\\
57.5	0.14424	-278.564786677891\\
57.5	0.1479	-298.518011698364\\
57.5	0.15156	-319.513475679805\\
57.5	0.15522	-341.551178622216\\
57.5	0.15888	-364.631120525596\\
57.5	0.16254	-388.753301389944\\
57.5	0.1662	-413.917721215262\\
57.5	0.16986	-440.124380001549\\
57.5	0.17352	-467.373277748805\\
57.5	0.17718	-495.66441445703\\
57.5	0.18084	-524.997790126224\\
57.5	0.1845	-555.373404756387\\
57.5	0.18816	-586.79125834752\\
57.5	0.19182	-619.251350899621\\
57.5	0.19548	-652.753682412691\\
57.5	0.19914	-687.29825288673\\
57.5	0.2028	-722.885062321739\\
57.5	0.20646	-759.514110717717\\
57.5	0.21012	-797.185398074663\\
57.5	0.21378	-835.898924392579\\
57.5	0.21744	-875.654689671464\\
57.5	0.2211	-916.452693911318\\
57.5	0.22476	-958.292937112141\\
57.5	0.22842	-1001.17541927393\\
57.5	0.23208	-1045.10014039669\\
57.5	0.23574	-1090.06710048042\\
57.5	0.2394	-1136.07629952512\\
57.5	0.24306	-1183.12773753079\\
57.5	0.24672	-1231.22141449743\\
57.5	0.25038	-1280.35733042503\\
57.5	0.25404	-1330.53548531361\\
57.5	0.2577	-1381.75587916315\\
57.5	0.26136	-1434.01851197367\\
57.5	0.26502	-1487.32338374515\\
57.5	0.26868	-1541.6704944776\\
57.5	0.27234	-1597.05984417102\\
57.5	0.276	-1653.49143282541\\
57.875	0.093	-107.443000982043\\
57.875	0.09666	-112.875362998169\\
57.875	0.10032	-119.349963975263\\
57.875	0.10398	-126.866803913326\\
57.875	0.10764	-135.425882812359\\
57.875	0.1113	-145.027200672361\\
57.875	0.11496	-155.670757493331\\
57.875	0.11862	-167.356553275271\\
57.875	0.12228	-180.08458801818\\
57.875	0.12594	-193.854861722058\\
57.875	0.1296	-208.667374386905\\
57.875	0.13326	-224.52212601272\\
57.875	0.13692	-241.419116599505\\
57.875	0.14058	-259.35834614726\\
57.875	0.14424	-278.339814655983\\
57.875	0.1479	-298.363522125675\\
57.875	0.15156	-319.429468556337\\
57.875	0.15522	-341.537653947967\\
57.875	0.15888	-364.688078300567\\
57.875	0.16254	-388.880741614135\\
57.875	0.1662	-414.115643888673\\
57.875	0.16986	-440.392785124179\\
57.875	0.17352	-467.712165320655\\
57.875	0.17718	-496.0737844781\\
57.875	0.18084	-525.477642596514\\
57.875	0.1845	-555.923739675897\\
57.875	0.18816	-587.412075716249\\
57.875	0.19182	-619.94265071757\\
57.875	0.19548	-653.51546467986\\
57.875	0.19914	-688.130517603119\\
57.875	0.2028	-723.787809487347\\
57.875	0.20646	-760.487340332545\\
57.875	0.21012	-798.229110138711\\
57.875	0.21378	-837.013118905847\\
57.875	0.21744	-876.839366633951\\
57.875	0.2211	-917.707853323025\\
57.875	0.22476	-959.618578973068\\
57.875	0.22842	-1002.57154358408\\
57.875	0.23208	-1046.56674715606\\
57.875	0.23574	-1091.60418968901\\
57.875	0.2394	-1137.68387118293\\
57.875	0.24306	-1184.80579163782\\
57.875	0.24672	-1232.96995105367\\
57.875	0.25038	-1282.1763494305\\
57.875	0.25404	-1332.4249867683\\
57.875	0.2577	-1383.71586306706\\
57.875	0.26136	-1436.04897832679\\
57.875	0.26502	-1489.4243325475\\
57.875	0.26868	-1543.84192572917\\
57.875	0.27234	-1599.30175787181\\
57.875	0.276	-1655.80382897542\\
58.25	0.093	-106.31373628937\\
58.25	0.09666	-111.816580754715\\
58.25	0.10032	-118.36166418103\\
58.25	0.10398	-125.948986568313\\
58.25	0.10764	-134.578547916565\\
58.25	0.1113	-144.250348225787\\
58.25	0.11496	-154.964387495977\\
58.25	0.11862	-166.720665727137\\
58.25	0.12228	-179.519182919265\\
58.25	0.12594	-193.359939072363\\
58.25	0.1296	-208.24293418643\\
58.25	0.13326	-224.168168261465\\
58.25	0.13692	-241.13564129747\\
58.25	0.14058	-259.145353294444\\
58.25	0.14424	-278.197304252387\\
58.25	0.1479	-298.291494171299\\
58.25	0.15156	-319.427923051181\\
58.25	0.15522	-341.606590892031\\
58.25	0.15888	-364.82749769385\\
58.25	0.16254	-389.090643456638\\
58.25	0.1662	-414.396028180396\\
58.25	0.16986	-440.743651865122\\
58.25	0.17352	-468.133514510817\\
58.25	0.17718	-496.565616117482\\
58.25	0.18084	-526.039956685115\\
58.25	0.1845	-556.556536213719\\
58.25	0.18816	-588.11535470329\\
58.25	0.19182	-620.716412153831\\
58.25	0.19548	-654.359708565341\\
58.25	0.19914	-689.04524393782\\
58.25	0.2028	-724.773018271268\\
58.25	0.20646	-761.543031565686\\
58.25	0.21012	-799.355283821072\\
58.25	0.21378	-838.209775037427\\
58.25	0.21744	-878.106505214751\\
58.25	0.2211	-919.045474353044\\
58.25	0.22476	-961.026682452307\\
58.25	0.22842	-1004.05012951254\\
58.25	0.23208	-1048.11581553374\\
58.25	0.23574	-1093.22374051591\\
58.25	0.2394	-1139.37390445905\\
58.25	0.24306	-1186.56630736316\\
58.25	0.24672	-1234.80094922823\\
58.25	0.25038	-1284.07783005428\\
58.25	0.25404	-1334.39694984129\\
58.25	0.2577	-1385.75830858928\\
58.25	0.26136	-1438.16190629823\\
58.25	0.26502	-1491.60774296815\\
58.25	0.26868	-1546.09581859904\\
58.25	0.27234	-1601.6261331909\\
58.25	0.276	-1658.19868674373\\
58.625	0.093	-105.26693321501\\
58.625	0.09666	-110.840260129575\\
58.625	0.10032	-117.45582600511\\
58.625	0.10398	-125.113630841612\\
58.625	0.10764	-133.813674639085\\
58.625	0.1113	-143.555957397526\\
58.625	0.11496	-154.340479116936\\
58.625	0.11862	-166.167239797315\\
58.625	0.12228	-179.036239438664\\
58.625	0.12594	-192.947478040981\\
58.625	0.1296	-207.900955604268\\
58.625	0.13326	-223.896672128523\\
58.625	0.13692	-240.934627613748\\
58.625	0.14058	-259.014822059942\\
58.625	0.14424	-278.137255467104\\
58.625	0.1479	-298.301927835236\\
58.625	0.15156	-319.508839164337\\
58.625	0.15522	-341.757989454407\\
58.625	0.15888	-365.049378705446\\
58.625	0.16254	-389.383006917454\\
58.625	0.1662	-414.758874090432\\
58.625	0.16986	-441.176980224378\\
58.625	0.17352	-468.637325319293\\
58.625	0.17718	-497.139909375178\\
58.625	0.18084	-526.684732392031\\
58.625	0.1845	-557.271794369854\\
58.625	0.18816	-588.901095308645\\
58.625	0.19182	-621.572635208406\\
58.625	0.19548	-655.286414069135\\
58.625	0.19914	-690.042431890834\\
58.625	0.2028	-725.840688673502\\
58.625	0.20646	-762.681184417139\\
58.625	0.21012	-800.563919121745\\
58.625	0.21378	-839.48889278732\\
58.625	0.21744	-879.456105413864\\
58.625	0.2211	-920.465557001378\\
58.625	0.22476	-962.51724754986\\
58.625	0.22842	-1005.61117705931\\
58.625	0.23208	-1049.74734552973\\
58.625	0.23574	-1094.92575296112\\
58.625	0.2394	-1141.14639935348\\
58.625	0.24306	-1188.40928470681\\
58.625	0.24672	-1236.7144090211\\
58.625	0.25038	-1286.06177229637\\
58.625	0.25404	-1336.4513745326\\
58.625	0.2577	-1387.88321572981\\
58.625	0.26136	-1440.35729588798\\
58.625	0.26502	-1493.87361500712\\
58.625	0.26868	-1548.43217308723\\
58.625	0.27234	-1604.03297012831\\
58.625	0.276	-1660.67600613036\\
59	0.093	-104.302591758963\\
59	0.09666	-109.946401122748\\
59	0.10032	-116.632449447502\\
59	0.10398	-124.360736733225\\
59	0.10764	-133.131262979917\\
59	0.1113	-142.944028187577\\
59	0.11496	-153.799032356208\\
59	0.11862	-165.696275485807\\
59	0.12228	-178.635757576375\\
59	0.12594	-192.617478627912\\
59	0.1296	-207.641438640419\\
59	0.13326	-223.707637613894\\
59	0.13692	-240.816075548338\\
59	0.14058	-258.966752443752\\
59	0.14424	-278.159668300135\\
59	0.1479	-298.394823117486\\
59	0.15156	-319.672216895807\\
59	0.15522	-341.991849635097\\
59	0.15888	-365.353721335356\\
59	0.16254	-389.757831996583\\
59	0.1662	-415.204181618781\\
59	0.16986	-441.692770201947\\
59	0.17352	-469.223597746081\\
59	0.17718	-497.796664251186\\
59	0.18084	-527.411969717258\\
59	0.1845	-558.069514144301\\
59	0.18816	-589.769297532313\\
59	0.19182	-622.511319881293\\
59	0.19548	-656.295581191243\\
59	0.19914	-691.122081462161\\
59	0.2028	-726.990820694049\\
59	0.20646	-763.901798886906\\
59	0.21012	-801.855016040731\\
59	0.21378	-840.850472155526\\
59	0.21744	-880.88816723129\\
59	0.2211	-921.968101268023\\
59	0.22476	-964.090274265725\\
59	0.22842	-1007.2546862244\\
59	0.23208	-1051.46133714404\\
59	0.23574	-1096.71022702465\\
59	0.2394	-1143.00135586622\\
59	0.24306	-1190.33472366877\\
59	0.24672	-1238.71033043229\\
59	0.25038	-1288.12817615677\\
59	0.25404	-1338.58826084223\\
59	0.2577	-1390.09058448865\\
59	0.26136	-1442.63514709604\\
59	0.26502	-1496.22194866441\\
59	0.26868	-1550.85098919374\\
59	0.27234	-1606.52226868404\\
59	0.276	-1663.23578713531\\
59.375	0.093	-103.420711921229\\
59.375	0.09666	-109.135003734234\\
59.375	0.10032	-115.891534508208\\
59.375	0.10398	-123.69030424315\\
59.375	0.10764	-132.531312939062\\
59.375	0.1113	-142.414560595942\\
59.375	0.11496	-153.340047213792\\
59.375	0.11862	-165.307772792611\\
59.375	0.12228	-178.317737332399\\
59.375	0.12594	-192.369940833156\\
59.375	0.1296	-207.464383294882\\
59.375	0.13326	-223.601064717577\\
59.375	0.13692	-240.779985101242\\
59.375	0.14058	-259.001144445875\\
59.375	0.14424	-278.264542751478\\
59.375	0.1479	-298.570180018049\\
59.375	0.15156	-319.918056245589\\
59.375	0.15522	-342.308171434099\\
59.375	0.15888	-365.740525583578\\
59.375	0.16254	-390.215118694025\\
59.375	0.1662	-415.731950765442\\
59.375	0.16986	-442.291021797828\\
59.375	0.17352	-469.892331791183\\
59.375	0.17718	-498.535880745507\\
59.375	0.18084	-528.2216686608\\
59.375	0.1845	-558.949695537062\\
59.375	0.18816	-590.719961374293\\
59.375	0.19182	-623.532466172493\\
59.375	0.19548	-657.387209931662\\
59.375	0.19914	-692.284192651801\\
59.375	0.2028	-728.223414332909\\
59.375	0.20646	-765.204874974985\\
59.375	0.21012	-803.228574578031\\
59.375	0.21378	-842.294513142045\\
59.375	0.21744	-882.402690667029\\
59.375	0.2211	-923.553107152982\\
59.375	0.22476	-965.745762599904\\
59.375	0.22842	-1008.98065700779\\
59.375	0.23208	-1053.25779037665\\
59.375	0.23574	-1098.57716270648\\
59.375	0.2394	-1144.93877399728\\
59.375	0.24306	-1192.34262424905\\
59.375	0.24672	-1240.78871346178\\
59.375	0.25038	-1290.27704163549\\
59.375	0.25404	-1340.80760877016\\
59.375	0.2577	-1392.38041486581\\
59.375	0.26136	-1444.99545992242\\
59.375	0.26502	-1498.65274394\\
59.375	0.26868	-1553.35226691855\\
59.375	0.27234	-1609.09402885807\\
59.375	0.276	-1665.87802975856\\
59.75	0.093	-102.621293701809\\
59.75	0.09666	-108.406067964033\\
59.75	0.10032	-115.233081187226\\
59.75	0.10398	-123.102333371389\\
59.75	0.10764	-132.01382451652\\
59.75	0.1113	-141.967554622621\\
59.75	0.11496	-152.96352368969\\
59.75	0.11862	-165.001731717729\\
59.75	0.12228	-178.082178706737\\
59.75	0.12594	-192.204864656714\\
59.75	0.1296	-207.369789567659\\
59.75	0.13326	-223.576953439574\\
59.75	0.13692	-240.826356272458\\
59.75	0.14058	-259.117998066311\\
59.75	0.14424	-278.451878821134\\
59.75	0.1479	-298.827998536925\\
59.75	0.15156	-320.246357213685\\
59.75	0.15522	-342.706954851414\\
59.75	0.15888	-366.209791450113\\
59.75	0.16254	-390.754867009781\\
59.75	0.1662	-416.342181530417\\
59.75	0.16986	-442.971735012023\\
59.75	0.17352	-470.643527454597\\
59.75	0.17718	-499.357558858141\\
59.75	0.18084	-529.113829222654\\
59.75	0.1845	-559.912338548136\\
59.75	0.18816	-591.753086834587\\
59.75	0.19182	-624.636074082007\\
59.75	0.19548	-658.561300290396\\
59.75	0.19914	-693.528765459754\\
59.75	0.2028	-729.538469590082\\
59.75	0.20646	-766.590412681378\\
59.75	0.21012	-804.684594733643\\
59.75	0.21378	-843.821015746877\\
59.75	0.21744	-883.999675721081\\
59.75	0.2211	-925.220574656254\\
59.75	0.22476	-967.483712552395\\
59.75	0.22842	-1010.78908940951\\
59.75	0.23208	-1055.13670522759\\
59.75	0.23574	-1100.52656000664\\
59.75	0.2394	-1146.95865374665\\
59.75	0.24306	-1194.43298644764\\
59.75	0.24672	-1242.9495581096\\
59.75	0.25038	-1292.50836873252\\
59.75	0.25404	-1343.10941831641\\
59.75	0.2577	-1394.75270686128\\
59.75	0.26136	-1447.43823436711\\
59.75	0.26502	-1501.16600083391\\
59.75	0.26868	-1555.93600626168\\
59.75	0.27234	-1611.74825065042\\
59.75	0.276	-1668.60273400013\\
60.125	0.093	-101.9043371007\\
60.125	0.09666	-107.759593812144\\
60.125	0.10032	-114.657089484557\\
60.125	0.10398	-122.59682411794\\
60.125	0.10764	-131.578797712291\\
60.125	0.1113	-141.603010267611\\
60.125	0.11496	-152.669461783901\\
60.125	0.11862	-164.778152261159\\
60.125	0.12228	-177.929081699387\\
60.125	0.12594	-192.122250098583\\
60.125	0.1296	-207.357657458749\\
60.125	0.13326	-223.635303779883\\
60.125	0.13692	-240.955189061987\\
60.125	0.14058	-259.31731330506\\
60.125	0.14424	-278.721676509102\\
60.125	0.1479	-299.168278674113\\
60.125	0.15156	-320.657119800093\\
60.125	0.15522	-343.188199887043\\
60.125	0.15888	-366.761518934961\\
60.125	0.16254	-391.377076943848\\
60.125	0.1662	-417.034873913705\\
60.125	0.16986	-443.73490984453\\
60.125	0.17352	-471.477184736324\\
60.125	0.17718	-500.261698589088\\
60.125	0.18084	-530.08845140282\\
60.125	0.1845	-560.957443177522\\
60.125	0.18816	-592.868673913193\\
60.125	0.19182	-625.822143609833\\
60.125	0.19548	-659.817852267442\\
60.125	0.19914	-694.85579988602\\
60.125	0.2028	-730.935986465567\\
60.125	0.20646	-768.058412006083\\
60.125	0.21012	-806.223076507568\\
60.125	0.21378	-845.429979970022\\
60.125	0.21744	-885.679122393445\\
60.125	0.2211	-926.970503777838\\
60.125	0.22476	-969.304124123199\\
60.125	0.22842	-1012.67998342953\\
60.125	0.23208	-1057.09808169683\\
60.125	0.23574	-1102.5584189251\\
60.125	0.2394	-1149.06099511434\\
60.125	0.24306	-1196.60581026454\\
60.125	0.24672	-1245.19286437572\\
60.125	0.25038	-1294.82215744786\\
60.125	0.25404	-1345.49368948098\\
60.125	0.2577	-1397.20746047506\\
60.125	0.26136	-1449.96347043011\\
60.125	0.26502	-1503.76171934613\\
60.125	0.26868	-1558.60220722312\\
60.125	0.27234	-1614.48493406108\\
60.125	0.276	-1671.40989986001\\
60.5	0.093	-101.269842117905\\
60.5	0.09666	-107.195581278569\\
60.5	0.10032	-114.163559400202\\
60.5	0.10398	-122.173776482804\\
60.5	0.10764	-131.226232526375\\
60.5	0.1113	-141.320927530915\\
60.5	0.11496	-152.457861496424\\
60.5	0.11862	-164.637034422902\\
60.5	0.12228	-177.85844631035\\
60.5	0.12594	-192.122097158766\\
60.5	0.1296	-207.427986968152\\
60.5	0.13326	-223.776115738506\\
60.5	0.13692	-241.16648346983\\
60.5	0.14058	-259.599090162122\\
60.5	0.14424	-279.073935815384\\
60.5	0.1479	-299.591020429615\\
60.5	0.15156	-321.150344004815\\
60.5	0.15522	-343.751906540984\\
60.5	0.15888	-367.395708038122\\
60.5	0.16254	-392.081748496229\\
60.5	0.1662	-417.810027915305\\
60.5	0.16986	-444.580546295351\\
60.5	0.17352	-472.393303636364\\
60.5	0.17718	-501.248299938348\\
60.5	0.18084	-531.1455352013\\
60.5	0.1845	-562.085009425222\\
60.5	0.18816	-594.066722610112\\
60.5	0.19182	-627.090674755972\\
60.5	0.19548	-661.1568658628\\
60.5	0.19914	-696.265295930598\\
60.5	0.2028	-732.415964959365\\
60.5	0.20646	-769.608872949101\\
60.5	0.21012	-807.844019899806\\
60.5	0.21378	-847.12140581148\\
60.5	0.21744	-887.441030684123\\
60.5	0.2211	-928.802894517735\\
60.5	0.22476	-971.206997312317\\
60.5	0.22842	-1014.65333906787\\
60.5	0.23208	-1059.14191978439\\
60.5	0.23574	-1104.67273946187\\
60.5	0.2394	-1151.24579810033\\
60.5	0.24306	-1198.86109569976\\
60.5	0.24672	-1247.51863226015\\
60.5	0.25038	-1297.21840778152\\
60.5	0.25404	-1347.96042226385\\
60.5	0.2577	-1399.74467570716\\
60.5	0.26136	-1452.57116811143\\
60.5	0.26502	-1506.43989947667\\
60.5	0.26868	-1561.35086980288\\
60.5	0.27234	-1617.30407909006\\
60.5	0.276	-1674.29952733821\\
60.875	0.093	-100.717808753422\\
60.875	0.09666	-106.714030363306\\
60.875	0.10032	-113.752490934159\\
60.875	0.10398	-121.83319046598\\
60.875	0.10764	-130.956128958771\\
60.875	0.1113	-141.121306412531\\
60.875	0.11496	-152.32872282726\\
60.875	0.11862	-164.578378202958\\
60.875	0.12228	-177.870272539625\\
60.875	0.12594	-192.204405837262\\
60.875	0.1296	-207.580778095867\\
60.875	0.13326	-223.999389315441\\
60.875	0.13692	-241.460239495984\\
60.875	0.14058	-259.963328637497\\
60.875	0.14424	-279.508656739979\\
60.875	0.1479	-300.096223803429\\
60.875	0.15156	-321.726029827849\\
60.875	0.15522	-344.398074813238\\
60.875	0.15888	-368.112358759596\\
60.875	0.16254	-392.868881666923\\
60.875	0.1662	-418.667643535219\\
60.875	0.16986	-445.508644364483\\
60.875	0.17352	-473.391884154717\\
60.875	0.17718	-502.31736290592\\
60.875	0.18084	-532.285080618092\\
60.875	0.1845	-563.295037291234\\
60.875	0.18816	-595.347232925345\\
60.875	0.19182	-628.441667520424\\
60.875	0.19548	-662.578341076472\\
60.875	0.19914	-697.75725359349\\
60.875	0.2028	-733.978405071477\\
60.875	0.20646	-771.241795510432\\
60.875	0.21012	-809.547424910357\\
60.875	0.21378	-848.895293271251\\
60.875	0.21744	-889.285400593113\\
60.875	0.2211	-930.717746875945\\
60.875	0.22476	-973.192332119746\\
60.875	0.22842	-1016.70915632452\\
60.875	0.23208	-1061.26821949026\\
60.875	0.23574	-1106.86952161696\\
60.875	0.2394	-1153.51306270464\\
60.875	0.24306	-1201.19884275329\\
60.875	0.24672	-1249.9268617629\\
60.875	0.25038	-1299.69711973349\\
60.875	0.25404	-1350.50961666504\\
60.875	0.2577	-1402.36435255756\\
60.875	0.26136	-1455.26132741106\\
60.875	0.26502	-1509.20054122552\\
60.875	0.26868	-1564.18199400095\\
60.875	0.27234	-1620.20568573734\\
60.875	0.276	-1677.27161643471\\
61.25	0.093	-100.248237007253\\
61.25	0.09666	-106.314941066356\\
61.25	0.10032	-113.423884086429\\
61.25	0.10398	-121.57506606747\\
61.25	0.10764	-130.768487009481\\
61.25	0.1113	-141.004146912461\\
61.25	0.11496	-152.28204577641\\
61.25	0.11862	-164.602183601327\\
61.25	0.12228	-177.964560387214\\
61.25	0.12594	-192.36917613407\\
61.25	0.1296	-207.816030841895\\
61.25	0.13326	-224.305124510689\\
61.25	0.13692	-241.836457140452\\
61.25	0.14058	-260.410028731185\\
61.25	0.14424	-280.025839282886\\
61.25	0.1479	-300.683888795557\\
61.25	0.15156	-322.384177269196\\
61.25	0.15522	-345.126704703804\\
61.25	0.15888	-368.911471099382\\
61.25	0.16254	-393.738476455929\\
61.25	0.1662	-419.607720773445\\
61.25	0.16986	-446.51920405193\\
61.25	0.17352	-474.472926291383\\
61.25	0.17718	-503.468887491806\\
61.25	0.18084	-533.507087653198\\
61.25	0.1845	-564.58752677556\\
61.25	0.18816	-596.710204858889\\
61.25	0.19182	-629.875121903189\\
61.25	0.19548	-664.082277908457\\
61.25	0.19914	-699.331672874694\\
61.25	0.2028	-735.623306801901\\
61.25	0.20646	-772.957179690076\\
61.25	0.21012	-811.333291539221\\
61.25	0.21378	-850.751642349334\\
61.25	0.21744	-891.212232120417\\
61.25	0.2211	-932.715060852469\\
61.25	0.22476	-975.26012854549\\
61.25	0.22842	-1018.84743519948\\
61.25	0.23208	-1063.47698081444\\
61.25	0.23574	-1109.14876539037\\
61.25	0.2394	-1155.86278892726\\
61.25	0.24306	-1203.61905142513\\
61.25	0.24672	-1252.41755288396\\
61.25	0.25038	-1302.25829330377\\
61.25	0.25404	-1353.14127268454\\
61.25	0.2577	-1405.06649102629\\
61.25	0.26136	-1458.033948329\\
61.25	0.26502	-1512.04364459268\\
61.25	0.26868	-1567.09557981733\\
61.25	0.27234	-1623.18975400294\\
61.25	0.276	-1680.32616714953\\
61.625	0.093	-99.8611268793961\\
61.625	0.09666	-105.998313387719\\
61.625	0.10032	-113.177738857012\\
61.625	0.10398	-121.399403287273\\
61.625	0.10764	-130.663306678504\\
61.625	0.1113	-140.969449030703\\
61.625	0.11496	-152.317830343872\\
61.625	0.11862	-164.708450618009\\
61.625	0.12228	-178.141309853116\\
61.625	0.12594	-192.616408049192\\
61.625	0.1296	-208.133745206237\\
61.625	0.13326	-224.69332132425\\
61.625	0.13692	-242.295136403233\\
61.625	0.14058	-260.939190443186\\
61.625	0.14424	-280.625483444107\\
61.625	0.1479	-301.354015405997\\
61.625	0.15156	-323.124786328856\\
61.625	0.15522	-345.937796212685\\
61.625	0.15888	-369.793045057482\\
61.625	0.16254	-394.690532863248\\
61.625	0.1662	-420.630259629984\\
61.625	0.16986	-447.612225357689\\
61.625	0.17352	-475.636430046362\\
61.625	0.17718	-504.702873696005\\
61.625	0.18084	-534.811556306616\\
61.625	0.1845	-565.962477878197\\
61.625	0.18816	-598.155638410748\\
61.625	0.19182	-631.391037904267\\
61.625	0.19548	-665.668676358755\\
61.625	0.19914	-700.988553774212\\
61.625	0.2028	-737.350670150638\\
61.625	0.20646	-774.755025488033\\
61.625	0.21012	-813.201619786397\\
61.625	0.21378	-852.690453045731\\
61.625	0.21744	-893.221525266033\\
61.625	0.2211	-934.794836447304\\
61.625	0.22476	-977.410386589545\\
61.625	0.22842	-1021.06817569275\\
61.625	0.23208	-1065.76820375693\\
61.625	0.23574	-1111.51047078208\\
61.625	0.2394	-1158.2949767682\\
61.625	0.24306	-1206.12172171528\\
61.625	0.24672	-1254.99070562334\\
61.625	0.25038	-1304.90192849236\\
61.625	0.25404	-1355.85539032236\\
61.625	0.2577	-1407.85109111332\\
61.625	0.26136	-1460.88903086525\\
61.625	0.26502	-1514.96920957815\\
61.625	0.26868	-1570.09162725202\\
61.625	0.27234	-1626.25628388686\\
61.625	0.276	-1683.46317948267\\
62	0.093	-99.5564783698526\\
62	0.09666	-105.764147327396\\
62	0.10032	-113.014055245908\\
62	0.10398	-121.306202125389\\
62	0.10764	-130.640587965839\\
62	0.1113	-141.017212767259\\
62	0.11496	-152.436076529647\\
62	0.11862	-164.897179253004\\
62	0.12228	-178.400520937331\\
62	0.12594	-192.946101582627\\
62	0.1296	-208.533921188891\\
62	0.13326	-225.163979756125\\
62	0.13692	-242.836277284327\\
62	0.14058	-261.550813773499\\
62	0.14424	-281.30758922364\\
62	0.1479	-302.10660363475\\
62	0.15156	-323.947857006829\\
62	0.15522	-346.831349339878\\
62	0.15888	-370.757080633895\\
62	0.16254	-395.725050888881\\
62	0.1662	-421.735260104836\\
62	0.16986	-448.787708281761\\
62	0.17352	-476.882395419654\\
62	0.17718	-506.019321518516\\
62	0.18084	-536.198486578348\\
62	0.1845	-567.419890599149\\
62	0.18816	-599.683533580919\\
62	0.19182	-632.989415523657\\
62	0.19548	-667.337536427365\\
62	0.19914	-702.727896292042\\
62	0.2028	-739.160495117688\\
62	0.20646	-776.635332904303\\
62	0.21012	-815.152409651887\\
62	0.21378	-854.711725360441\\
62	0.21744	-895.313280029962\\
62	0.2211	-936.957073660454\\
62	0.22476	-979.643106251914\\
62	0.22842	-1023.37137780434\\
62	0.23208	-1068.14188831774\\
62	0.23574	-1113.95463779211\\
62	0.2394	-1160.80962622745\\
62	0.24306	-1208.70685362375\\
62	0.24672	-1257.64631998103\\
62	0.25038	-1307.62802529927\\
62	0.25404	-1358.65196957848\\
62	0.2577	-1410.71815281867\\
62	0.26136	-1463.82657501982\\
62	0.26502	-1517.97723618194\\
62	0.26868	-1573.17013630503\\
62	0.27234	-1629.40527538908\\
62	0.276	-1686.68265343411\\
62.375	0.093	-99.3342914786219\\
62.375	0.09666	-105.612442885385\\
62.375	0.10032	-112.932833253117\\
62.375	0.10398	-121.295462581817\\
62.375	0.10764	-130.700330871488\\
62.375	0.1113	-141.147438122127\\
62.375	0.11496	-152.636784333735\\
62.375	0.11862	-165.168369506312\\
62.375	0.12228	-178.742193639859\\
62.375	0.12594	-193.358256734374\\
62.375	0.1296	-209.016558789858\\
62.375	0.13326	-225.717099806311\\
62.375	0.13692	-243.459879783734\\
62.375	0.14058	-262.244898722126\\
62.375	0.14424	-282.072156621487\\
62.375	0.1479	-302.941653481816\\
62.375	0.15156	-324.853389303115\\
62.375	0.15522	-347.807364085383\\
62.375	0.15888	-371.80357782862\\
62.375	0.16254	-396.842030532826\\
62.375	0.1662	-422.922722198001\\
62.375	0.16986	-450.045652824146\\
62.375	0.17352	-478.210822411259\\
62.375	0.17718	-507.418230959341\\
62.375	0.18084	-537.667878468392\\
62.375	0.1845	-568.959764938413\\
62.375	0.18816	-601.293890369402\\
62.375	0.19182	-634.670254761361\\
62.375	0.19548	-669.088858114288\\
62.375	0.19914	-704.549700428185\\
62.375	0.2028	-741.052781703051\\
62.375	0.20646	-778.598101938886\\
62.375	0.21012	-817.18566113569\\
62.375	0.21378	-856.815459293463\\
62.375	0.21744	-897.487496412205\\
62.375	0.2211	-939.201772491916\\
62.375	0.22476	-981.958287532596\\
62.375	0.22842	-1025.75704153425\\
62.375	0.23208	-1070.59803449686\\
62.375	0.23574	-1116.48126642045\\
62.375	0.2394	-1163.40673730501\\
62.375	0.24306	-1211.37444715053\\
62.375	0.24672	-1260.38439595703\\
62.375	0.25038	-1310.43658372449\\
62.375	0.25404	-1361.53101045292\\
62.375	0.2577	-1413.66767614233\\
62.375	0.26136	-1466.8465807927\\
62.375	0.26502	-1521.06772440404\\
62.375	0.26868	-1576.33110697635\\
62.375	0.27234	-1632.63672850962\\
62.375	0.276	-1689.98458900387\\
62.75	0.093	-99.1945662057039\\
62.75	0.09666	-105.543200061687\\
62.75	0.10032	-112.934072878639\\
62.75	0.10398	-121.367184656559\\
62.75	0.10764	-130.842535395449\\
62.75	0.1113	-141.360125095308\\
62.75	0.11496	-152.919953756136\\
62.75	0.11862	-165.522021377933\\
62.75	0.12228	-179.166327960699\\
62.75	0.12594	-193.852873504434\\
62.75	0.1296	-209.581658009138\\
62.75	0.13326	-226.352681474811\\
62.75	0.13692	-244.165943901453\\
62.75	0.14058	-263.021445289065\\
62.75	0.14424	-282.919185637646\\
62.75	0.1479	-303.859164947195\\
62.75	0.15156	-325.841383217714\\
62.75	0.15522	-348.865840449202\\
62.75	0.15888	-372.932536641659\\
62.75	0.16254	-398.041471795084\\
62.75	0.1662	-424.192645909479\\
62.75	0.16986	-451.386058984843\\
62.75	0.17352	-479.621711021176\\
62.75	0.17718	-508.899602018478\\
62.75	0.18084	-539.219731976749\\
62.75	0.1845	-570.582100895989\\
62.75	0.18816	-602.986708776199\\
62.75	0.19182	-636.433555617378\\
62.75	0.19548	-670.922641419525\\
62.75	0.19914	-706.453966182641\\
62.75	0.2028	-743.027529906727\\
62.75	0.20646	-780.643332591782\\
62.75	0.21012	-819.301374237805\\
62.75	0.21378	-859.001654844798\\
62.75	0.21744	-899.744174412759\\
62.75	0.2211	-941.528932941691\\
62.75	0.22476	-984.355930431591\\
62.75	0.22842	-1028.22516688246\\
62.75	0.23208	-1073.1366422943\\
62.75	0.23574	-1119.09035666711\\
62.75	0.2394	-1166.08631000088\\
62.75	0.24306	-1214.12450229563\\
62.75	0.24672	-1263.20493355134\\
62.75	0.25038	-1313.32760376802\\
62.75	0.25404	-1364.49251294568\\
62.75	0.2577	-1416.6996610843\\
62.75	0.26136	-1469.94904818389\\
62.75	0.26502	-1524.24067424445\\
62.75	0.26868	-1579.57453926598\\
62.75	0.27234	-1635.95064324847\\
62.75	0.276	-1693.36898619194\\
63.125	0.093	-99.1373025510989\\
63.125	0.09666	-105.556418856301\\
63.125	0.10032	-113.017774122473\\
63.125	0.10398	-121.521368349613\\
63.125	0.10764	-131.067201537723\\
63.125	0.1113	-141.655273686802\\
63.125	0.11496	-153.28558479685\\
63.125	0.11862	-165.958134867866\\
63.125	0.12228	-179.672923899852\\
63.125	0.12594	-194.429951892807\\
63.125	0.1296	-210.229218846731\\
63.125	0.13326	-227.070724761624\\
63.125	0.13692	-244.954469637486\\
63.125	0.14058	-263.880453474318\\
63.125	0.14424	-283.848676272118\\
63.125	0.1479	-304.859138030887\\
63.125	0.15156	-326.911838750626\\
63.125	0.15522	-350.006778431333\\
63.125	0.15888	-374.14395707301\\
63.125	0.16254	-399.323374675655\\
63.125	0.1662	-425.54503123927\\
63.125	0.16986	-452.808926763854\\
63.125	0.17352	-481.115061249407\\
63.125	0.17718	-510.463434695929\\
63.125	0.18084	-540.854047103419\\
63.125	0.1845	-572.28689847188\\
63.125	0.18816	-604.761988801309\\
63.125	0.19182	-638.279318091707\\
63.125	0.19548	-672.838886343074\\
63.125	0.19914	-708.44069355541\\
63.125	0.2028	-745.084739728716\\
63.125	0.20646	-782.77102486299\\
63.125	0.21012	-821.499548958234\\
63.125	0.21378	-861.270312014447\\
63.125	0.21744	-902.083314031628\\
63.125	0.2211	-943.938555009779\\
63.125	0.22476	-986.836034948898\\
63.125	0.22842	-1030.77575384899\\
63.125	0.23208	-1075.75771171004\\
63.125	0.23574	-1121.78190853207\\
63.125	0.2394	-1168.84834431507\\
63.125	0.24306	-1216.95701905903\\
63.125	0.24672	-1266.10793276397\\
63.125	0.25038	-1316.30108542987\\
63.125	0.25404	-1367.53647705674\\
63.125	0.2577	-1419.81410764458\\
63.125	0.26136	-1473.13397719339\\
63.125	0.26502	-1527.49608570317\\
63.125	0.26868	-1582.90043317392\\
63.125	0.27234	-1639.34701960564\\
63.125	0.276	-1696.83584499833\\
63.5	0.093	-99.1625005148069\\
63.5	0.09666	-105.652099269229\\
63.5	0.10032	-113.183936984621\\
63.5	0.10398	-121.758013660981\\
63.5	0.10764	-131.37432929831\\
63.5	0.1113	-142.032883896609\\
63.5	0.11496	-153.733677455876\\
63.5	0.11862	-166.476709976113\\
63.5	0.12228	-180.261981457319\\
63.5	0.12594	-195.089491899493\\
63.5	0.1296	-210.959241302637\\
63.5	0.13326	-227.87122966675\\
63.5	0.13692	-245.825456991832\\
63.5	0.14058	-264.821923277883\\
63.5	0.14424	-284.860628524903\\
63.5	0.1479	-305.941572732892\\
63.5	0.15156	-328.064755901851\\
63.5	0.15522	-351.230178031778\\
63.5	0.15888	-375.437839122674\\
63.5	0.16254	-400.68773917454\\
63.5	0.1662	-426.979878187374\\
63.5	0.16986	-454.314256161178\\
63.5	0.17352	-482.69087309595\\
63.5	0.17718	-512.109728991692\\
63.5	0.18084	-542.570823848402\\
63.5	0.1845	-574.074157666082\\
63.5	0.18816	-606.619730444731\\
63.5	0.19182	-640.207542184349\\
63.5	0.19548	-674.837592884936\\
63.5	0.19914	-710.509882546492\\
63.5	0.2028	-747.224411169018\\
63.5	0.20646	-784.981178752512\\
63.5	0.21012	-823.780185296975\\
63.5	0.21378	-863.621430802408\\
63.5	0.21744	-904.504915268808\\
63.5	0.2211	-946.430638696179\\
63.5	0.22476	-989.398601084519\\
63.5	0.22842	-1033.40880243383\\
63.5	0.23208	-1078.4612427441\\
63.5	0.23574	-1124.55592201535\\
63.5	0.2394	-1171.69284024757\\
63.5	0.24306	-1219.87199744075\\
63.5	0.24672	-1269.09339359491\\
63.5	0.25038	-1319.35702871003\\
63.5	0.25404	-1370.66290278612\\
63.5	0.2577	-1423.01101582318\\
63.5	0.26136	-1476.40136782121\\
63.5	0.26502	-1530.83395878021\\
63.5	0.26868	-1586.30878870018\\
63.5	0.27234	-1642.82585758112\\
63.5	0.276	-1700.38516542302\\
63.875	0.093	-99.2701600968282\\
63.875	0.09666	-105.83024130047\\
63.875	0.10032	-113.432561465082\\
63.875	0.10398	-122.077120590661\\
63.875	0.10764	-131.763918677211\\
63.875	0.1113	-142.492955724729\\
63.875	0.11496	-154.264231733216\\
63.875	0.11862	-167.077746702673\\
63.875	0.12228	-180.933500633098\\
63.875	0.12594	-195.831493524493\\
63.875	0.1296	-211.771725376856\\
63.875	0.13326	-228.754196190189\\
63.875	0.13692	-246.77890596449\\
63.875	0.14058	-265.845854699762\\
63.875	0.14424	-285.955042396001\\
63.875	0.1479	-307.10646905321\\
63.875	0.15156	-329.300134671388\\
63.875	0.15522	-352.536039250536\\
63.875	0.15888	-376.814182790652\\
63.875	0.16254	-402.134565291737\\
63.875	0.1662	-428.497186753791\\
63.875	0.16986	-455.902047176815\\
63.875	0.17352	-484.349146560807\\
63.875	0.17718	-513.838484905768\\
63.875	0.18084	-544.370062211698\\
63.875	0.1845	-575.943878478598\\
63.875	0.18816	-608.559933706467\\
63.875	0.19182	-642.218227895305\\
63.875	0.19548	-676.918761045112\\
63.875	0.19914	-712.661533155888\\
63.875	0.2028	-749.446544227633\\
63.875	0.20646	-787.273794260347\\
63.875	0.21012	-826.143283254029\\
63.875	0.21378	-866.055011208682\\
63.875	0.21744	-907.008978124302\\
63.875	0.2211	-949.005184000893\\
63.875	0.22476	-992.043628838452\\
63.875	0.22842	-1036.12431263698\\
63.875	0.23208	-1081.24723539648\\
63.875	0.23574	-1127.41239711695\\
63.875	0.2394	-1174.61979779838\\
63.875	0.24306	-1222.86943744079\\
63.875	0.24672	-1272.16131604416\\
63.875	0.25038	-1322.4954336085\\
63.875	0.25404	-1373.87179013381\\
63.875	0.2577	-1426.29038562009\\
63.875	0.26136	-1479.75122006734\\
63.875	0.26502	-1534.25429347556\\
63.875	0.26868	-1589.79960584475\\
63.875	0.27234	-1646.38715717491\\
63.875	0.276	-1704.01694746603\\
64.25	0.093	-99.4602812971618\\
64.25	0.09666	-106.090844950024\\
64.25	0.10032	-113.763647563855\\
64.25	0.10398	-122.478689138655\\
64.25	0.10764	-132.235969674423\\
64.25	0.1113	-143.035489171162\\
64.25	0.11496	-154.877247628869\\
64.25	0.11862	-167.761245047545\\
64.25	0.12228	-181.68748142719\\
64.25	0.12594	-196.655956767805\\
64.25	0.1296	-212.666671069388\\
64.25	0.13326	-229.71962433194\\
64.25	0.13692	-247.814816555462\\
64.25	0.14058	-266.952247739953\\
64.25	0.14424	-287.131917885412\\
64.25	0.1479	-308.353826991841\\
64.25	0.15156	-330.617975059239\\
64.25	0.15522	-353.924362087605\\
64.25	0.15888	-378.272988076941\\
64.25	0.16254	-403.663853027246\\
64.25	0.1662	-430.096956938521\\
64.25	0.16986	-457.572299810764\\
64.25	0.17352	-486.089881643976\\
64.25	0.17718	-515.649702438157\\
64.25	0.18084	-546.251762193308\\
64.25	0.1845	-577.896060909427\\
64.25	0.18816	-610.582598586515\\
64.25	0.19182	-644.311375224573\\
64.25	0.19548	-679.082390823599\\
64.25	0.19914	-714.895645383595\\
64.25	0.2028	-751.75113890456\\
64.25	0.20646	-789.648871386494\\
64.25	0.21012	-828.588842829397\\
64.25	0.21378	-868.571053233269\\
64.25	0.21744	-909.595502598109\\
64.25	0.2211	-951.66219092392\\
64.25	0.22476	-994.771118210699\\
64.25	0.22842	-1038.92228445845\\
64.25	0.23208	-1084.11568966716\\
64.25	0.23574	-1130.35133383685\\
64.25	0.2394	-1177.62921696751\\
64.25	0.24306	-1225.94933905913\\
64.25	0.24672	-1275.31170011172\\
64.25	0.25038	-1325.71630012529\\
64.25	0.25404	-1377.16313909982\\
64.25	0.2577	-1429.65221703532\\
64.25	0.26136	-1483.18353393179\\
64.25	0.26502	-1537.75708978923\\
64.25	0.26868	-1593.37288460764\\
64.25	0.27234	-1650.03091838701\\
64.25	0.276	-1707.73119112736\\
64.625	0.093	-99.7328641158088\\
64.625	0.09666	-106.43391021789\\
64.625	0.10032	-114.177195280941\\
64.625	0.10398	-122.962719304961\\
64.625	0.10764	-132.79048228995\\
64.625	0.1113	-143.660484235907\\
64.625	0.11496	-155.572725142834\\
64.625	0.11862	-168.52720501073\\
64.625	0.12228	-182.523923839595\\
64.625	0.12594	-197.56288162943\\
64.625	0.1296	-213.644078380233\\
64.625	0.13326	-230.767514092004\\
64.625	0.13692	-248.933188764746\\
64.625	0.14058	-268.141102398456\\
64.625	0.14424	-288.391254993136\\
64.625	0.1479	-309.683646548785\\
64.625	0.15156	-332.018277065402\\
64.625	0.15522	-355.395146542989\\
64.625	0.15888	-379.814254981544\\
64.625	0.16254	-405.275602381069\\
64.625	0.1662	-431.779188741563\\
64.625	0.16986	-459.325014063026\\
64.625	0.17352	-487.913078345458\\
64.625	0.17718	-517.543381588859\\
64.625	0.18084	-548.215923793229\\
64.625	0.1845	-579.930704958568\\
64.625	0.18816	-612.687725084877\\
64.625	0.19182	-646.486984172154\\
64.625	0.19548	-681.328482220401\\
64.625	0.19914	-717.212219229616\\
64.625	0.2028	-754.138195199801\\
64.625	0.20646	-792.106410130954\\
64.625	0.21012	-831.116864023077\\
64.625	0.21378	-871.169556876168\\
64.625	0.21744	-912.264488690229\\
64.625	0.2211	-954.401659465259\\
64.625	0.22476	-997.581069201258\\
64.625	0.22842	-1041.80271789823\\
64.625	0.23208	-1087.06660555616\\
64.625	0.23574	-1133.37273217507\\
64.625	0.2394	-1180.72109775494\\
64.625	0.24306	-1229.11170229579\\
64.625	0.24672	-1278.5445457976\\
64.625	0.25038	-1329.01962826038\\
64.625	0.25404	-1380.53694968414\\
64.625	0.2577	-1433.09651006886\\
64.625	0.26136	-1486.69830941455\\
64.625	0.26502	-1541.3423477212\\
64.625	0.26868	-1597.02862498883\\
64.625	0.27234	-1653.75714121743\\
64.625	0.276	-1711.52789640699\\
65	0.093	-100.087908552768\\
65	0.09666	-106.859437104069\\
65	0.10032	-114.67320461634\\
65	0.10398	-123.529211089579\\
65	0.10764	-133.427456523788\\
65	0.1113	-144.367940918966\\
65	0.11496	-156.350664275112\\
65	0.11862	-169.375626592228\\
65	0.12228	-183.442827870313\\
65	0.12594	-198.552268109367\\
65	0.1296	-214.70394730939\\
65	0.13326	-231.897865470382\\
65	0.13692	-250.134022592343\\
65	0.14058	-269.412418675273\\
65	0.14424	-289.733053719173\\
65	0.1479	-311.095927724041\\
65	0.15156	-333.501040689878\\
65	0.15522	-356.948392616684\\
65	0.15888	-381.43798350446\\
65	0.16254	-406.969813353205\\
65	0.1662	-433.543882162919\\
65	0.16986	-461.160189933601\\
65	0.17352	-489.818736665253\\
65	0.17718	-519.519522357874\\
65	0.18084	-550.262547011464\\
65	0.1845	-582.047810626023\\
65	0.18816	-614.875313201551\\
65	0.19182	-648.745054738048\\
65	0.19548	-683.657035235514\\
65	0.19914	-719.611254693949\\
65	0.2028	-756.607713113354\\
65	0.20646	-794.646410493727\\
65	0.21012	-833.72734683507\\
65	0.21378	-873.850522137381\\
65	0.21744	-915.015936400661\\
65	0.2211	-957.223589624911\\
65	0.22476	-1000.47348181013\\
65	0.22842	-1044.76561295632\\
65	0.23208	-1090.09998306347\\
65	0.23574	-1136.4765921316\\
65	0.2394	-1183.8954401607\\
65	0.24306	-1232.35652715076\\
65	0.24672	-1281.85985310179\\
65	0.25038	-1332.4054180138\\
65	0.25404	-1383.99322188677\\
65	0.2577	-1436.62326472071\\
65	0.26136	-1490.29554651562\\
65	0.26502	-1545.01006727149\\
65	0.26868	-1600.76682698834\\
65	0.27234	-1657.56582566616\\
65	0.276	-1715.40706330494\\
65.375	0.093	-100.525414608041\\
65.375	0.09666	-107.367425608562\\
65.375	0.10032	-115.251675570052\\
65.375	0.10398	-124.178164492511\\
65.375	0.10764	-134.14689237594\\
65.375	0.1113	-145.157859220337\\
65.375	0.11496	-157.211065025704\\
65.375	0.11862	-170.306509792039\\
65.375	0.12228	-184.444193519344\\
65.375	0.12594	-199.624116207618\\
65.375	0.1296	-215.84627785686\\
65.375	0.13326	-233.110678467072\\
65.375	0.13692	-251.417318038253\\
65.375	0.14058	-270.766196570403\\
65.375	0.14424	-291.157314063522\\
65.375	0.1479	-312.59067051761\\
65.375	0.15156	-335.066265932668\\
65.375	0.15522	-358.584100308694\\
65.375	0.15888	-383.144173645689\\
65.375	0.16254	-408.746485943653\\
65.375	0.1662	-435.391037202587\\
65.375	0.16986	-463.077827422489\\
65.375	0.17352	-491.806856603361\\
65.375	0.17718	-521.578124745201\\
65.375	0.18084	-552.391631848011\\
65.375	0.1845	-584.24737791179\\
65.375	0.18816	-617.145362936538\\
65.375	0.19182	-651.085586922255\\
65.375	0.19548	-686.068049868941\\
65.375	0.19914	-722.092751776596\\
65.375	0.2028	-759.15969264522\\
65.375	0.20646	-797.268872474813\\
65.375	0.21012	-836.420291265375\\
65.375	0.21378	-876.613949016907\\
65.375	0.21744	-917.849845729407\\
65.375	0.2211	-960.127981402876\\
65.375	0.22476	-1003.44835603731\\
65.375	0.22842	-1047.81096963272\\
65.375	0.23208	-1093.2158221891\\
65.375	0.23574	-1139.66291370645\\
65.375	0.2394	-1187.15224418476\\
65.375	0.24306	-1235.68381362404\\
65.375	0.24672	-1285.2576220243\\
65.375	0.25038	-1335.87366938552\\
65.375	0.25404	-1387.53195570771\\
65.375	0.2577	-1440.23248099087\\
65.375	0.26136	-1493.975245235\\
65.375	0.26502	-1548.7602484401\\
65.375	0.26868	-1604.58749060616\\
65.375	0.27234	-1661.4569717332\\
65.375	0.276	-1719.36869182121\\
65.75	0.093	-101.045382281626\\
65.75	0.09666	-107.957875731367\\
65.75	0.10032	-115.912608142078\\
65.75	0.10398	-124.909579513757\\
65.75	0.10764	-134.948789846405\\
65.75	0.1113	-146.030239140022\\
65.75	0.11496	-158.153927394608\\
65.75	0.11862	-171.319854610164\\
65.75	0.12228	-185.528020786688\\
65.75	0.12594	-200.778425924182\\
65.75	0.1296	-217.071070022644\\
65.75	0.13326	-234.405953082075\\
65.75	0.13692	-252.783075102476\\
65.75	0.14058	-272.202436083846\\
65.75	0.14424	-292.664036026185\\
65.75	0.1479	-314.167874929493\\
65.75	0.15156	-336.71395279377\\
65.75	0.15522	-360.302269619016\\
65.75	0.15888	-384.932825405231\\
65.75	0.16254	-410.605620152415\\
65.75	0.1662	-437.320653860569\\
65.75	0.16986	-465.077926529691\\
65.75	0.17352	-493.877438159782\\
65.75	0.17718	-523.719188750842\\
65.75	0.18084	-554.603178302872\\
65.75	0.1845	-586.52940681587\\
65.75	0.18816	-619.497874289838\\
65.75	0.19182	-653.508580724775\\
65.75	0.19548	-688.561526120681\\
65.75	0.19914	-724.656710477556\\
65.75	0.2028	-761.7941337954\\
65.75	0.20646	-799.973796074213\\
65.75	0.21012	-839.195697313995\\
65.75	0.21378	-879.459837514746\\
65.75	0.21744	-920.766216676465\\
65.75	0.2211	-963.114834799155\\
65.75	0.22476	-1006.50569188281\\
65.75	0.22842	-1050.93878792744\\
65.75	0.23208	-1096.41412293304\\
65.75	0.23574	-1142.9316968996\\
65.75	0.2394	-1190.49150982714\\
65.75	0.24306	-1239.09356171564\\
65.75	0.24672	-1288.73785256511\\
65.75	0.25038	-1339.42438237556\\
65.75	0.25404	-1391.15315114697\\
65.75	0.2577	-1443.92415887935\\
65.75	0.26136	-1497.7374055727\\
65.75	0.26502	-1552.59289122701\\
65.75	0.26868	-1608.4906158423\\
65.75	0.27234	-1665.43057941856\\
65.75	0.276	-1723.41278195578\\
66.125	0.093	-101.647811573525\\
66.125	0.09666	-108.630787472486\\
66.125	0.10032	-116.656002332416\\
66.125	0.10398	-125.723456153314\\
66.125	0.10764	-135.833148935182\\
66.125	0.1113	-146.985080678019\\
66.125	0.11496	-159.179251381825\\
66.125	0.11862	-172.4156610466\\
66.125	0.12228	-186.694309672345\\
66.125	0.12594	-202.015197259058\\
66.125	0.1296	-218.37832380674\\
66.125	0.13326	-235.783689315392\\
66.125	0.13692	-254.231293785012\\
66.125	0.14058	-273.721137215602\\
66.125	0.14424	-294.253219607161\\
66.125	0.1479	-315.827540959688\\
66.125	0.15156	-338.444101273185\\
66.125	0.15522	-362.10290054765\\
66.125	0.15888	-386.803938783086\\
66.125	0.16254	-412.54721597949\\
66.125	0.1662	-439.332732136863\\
66.125	0.16986	-467.160487255205\\
66.125	0.17352	-496.030481334516\\
66.125	0.17718	-525.942714374796\\
66.125	0.18084	-556.897186376046\\
66.125	0.1845	-588.893897338264\\
66.125	0.18816	-621.932847261451\\
66.125	0.19182	-656.014036145608\\
66.125	0.19548	-691.137463990733\\
66.125	0.19914	-727.303130796827\\
66.125	0.2028	-764.511036563892\\
66.125	0.20646	-802.761181291924\\
66.125	0.21012	-842.053564980926\\
66.125	0.21378	-882.388187630897\\
66.125	0.21744	-923.765049241837\\
66.125	0.2211	-966.184149813746\\
66.125	0.22476	-1009.64548934662\\
66.125	0.22842	-1054.14906784047\\
66.125	0.23208	-1099.69488529529\\
66.125	0.23574	-1146.28294171107\\
66.125	0.2394	-1193.91323708783\\
66.125	0.24306	-1242.58577142555\\
66.125	0.24672	-1292.30054472424\\
66.125	0.25038	-1343.05755698391\\
66.125	0.25404	-1394.85680820454\\
66.125	0.2577	-1447.69829838614\\
66.125	0.26136	-1501.5820275287\\
66.125	0.26502	-1556.50799563224\\
66.125	0.26868	-1612.47620269675\\
66.125	0.27234	-1669.48664872222\\
66.125	0.276	-1727.53933370867\\
66.5	0.093	-102.332702483736\\
66.5	0.09666	-109.386160831917\\
66.5	0.10032	-117.481858141067\\
66.5	0.10398	-126.619794411185\\
66.5	0.10764	-136.799969642273\\
66.5	0.1113	-148.02238383433\\
66.5	0.11496	-160.287036987355\\
66.5	0.11862	-173.59392910135\\
66.5	0.12228	-187.943060176314\\
66.5	0.12594	-203.334430212248\\
66.5	0.1296	-219.76803920915\\
66.5	0.13326	-237.24388716702\\
66.5	0.13692	-255.761974085861\\
66.5	0.14058	-275.32229996567\\
66.5	0.14424	-295.924864806449\\
66.5	0.1479	-317.569668608196\\
66.5	0.15156	-340.256711370913\\
66.5	0.15522	-363.985993094599\\
66.5	0.15888	-388.757513779253\\
66.5	0.16254	-414.571273424877\\
66.5	0.1662	-441.42727203147\\
66.5	0.16986	-469.325509599032\\
66.5	0.17352	-498.265986127562\\
66.5	0.17718	-528.248701617062\\
66.5	0.18084	-559.273656067531\\
66.5	0.1845	-591.34084947897\\
66.5	0.18816	-624.450281851377\\
66.5	0.19182	-658.601953184753\\
66.5	0.19548	-693.795863479099\\
66.5	0.19914	-730.032012734413\\
66.5	0.2028	-767.310400950697\\
66.5	0.20646	-805.631028127949\\
66.5	0.21012	-844.993894266171\\
66.5	0.21378	-885.398999365362\\
66.5	0.21744	-926.846343425521\\
66.5	0.2211	-969.33592644665\\
66.5	0.22476	-1012.86774842875\\
66.5	0.22842	-1057.44180937181\\
66.5	0.23208	-1103.05810927585\\
66.5	0.23574	-1149.71664814086\\
66.5	0.2394	-1197.41742596683\\
66.5	0.24306	-1246.16044275377\\
66.5	0.24672	-1295.94569850169\\
66.5	0.25038	-1346.77319321057\\
66.5	0.25404	-1398.64292688042\\
66.5	0.2577	-1451.55489951124\\
66.5	0.26136	-1505.50911110303\\
66.5	0.26502	-1560.50556165578\\
66.5	0.26868	-1616.54425116951\\
66.5	0.27234	-1673.6251796442\\
66.5	0.276	-1731.74834707987\\
66.875	0.093	-103.10005501226\\
66.875	0.09666	-110.223995809661\\
66.875	0.10032	-118.39017556803\\
66.875	0.10398	-127.598594287368\\
66.875	0.10764	-137.849251967676\\
66.875	0.1113	-149.142148608953\\
66.875	0.11496	-161.477284211198\\
66.875	0.11862	-174.854658774413\\
66.875	0.12228	-189.274272298597\\
66.875	0.12594	-204.73612478375\\
66.875	0.1296	-221.240216229872\\
66.875	0.13326	-238.786546636963\\
66.875	0.13692	-257.375116005023\\
66.875	0.14058	-277.005924334052\\
66.875	0.14424	-297.67897162405\\
66.875	0.1479	-319.394257875017\\
66.875	0.15156	-342.151783086954\\
66.875	0.15522	-365.951547259859\\
66.875	0.15888	-390.793550393734\\
66.875	0.16254	-416.677792488577\\
66.875	0.1662	-443.60427354439\\
66.875	0.16986	-471.572993561171\\
66.875	0.17352	-500.583952538922\\
66.875	0.17718	-530.637150477642\\
66.875	0.18084	-561.732587377331\\
66.875	0.1845	-593.870263237989\\
66.875	0.18816	-627.050178059616\\
66.875	0.19182	-661.272331842212\\
66.875	0.19548	-696.536724585777\\
66.875	0.19914	-732.843356290311\\
66.875	0.2028	-770.192226955814\\
66.875	0.20646	-808.583336582287\\
66.875	0.21012	-848.016685169728\\
66.875	0.21378	-888.492272718139\\
66.875	0.21744	-930.010099227518\\
66.875	0.2211	-972.570164697867\\
66.875	0.22476	-1016.17246912918\\
66.875	0.22842	-1060.81701252147\\
66.875	0.23208	-1106.50379487473\\
66.875	0.23574	-1153.23281618895\\
66.875	0.2394	-1201.00407646415\\
66.875	0.24306	-1249.81757570031\\
66.875	0.24672	-1299.67331389744\\
66.875	0.25038	-1350.57129105554\\
66.875	0.25404	-1402.51150717461\\
66.875	0.2577	-1455.49396225465\\
66.875	0.26136	-1509.51865629566\\
66.875	0.26502	-1564.58558929764\\
66.875	0.26868	-1620.69476126058\\
66.875	0.27234	-1677.8461721845\\
66.875	0.276	-1736.03982206938\\
67.25	0.093	-103.949869159097\\
67.25	0.09666	-111.144292405717\\
67.25	0.10032	-119.380954613307\\
67.25	0.10398	-128.659855781865\\
67.25	0.10764	-138.980995911392\\
67.25	0.1113	-150.344375001888\\
67.25	0.11496	-162.749993053354\\
67.25	0.11862	-176.197850065789\\
67.25	0.12228	-190.687946039192\\
67.25	0.12594	-206.220280973565\\
67.25	0.1296	-222.794854868907\\
67.25	0.13326	-240.411667725217\\
67.25	0.13692	-259.070719542497\\
67.25	0.14058	-278.772010320746\\
67.25	0.14424	-299.515540059964\\
67.25	0.1479	-321.301308760151\\
67.25	0.15156	-344.129316421307\\
67.25	0.15522	-367.999563043433\\
67.25	0.15888	-392.912048626527\\
67.25	0.16254	-418.86677317059\\
67.25	0.1662	-445.863736675623\\
67.25	0.16986	-473.902939141624\\
67.25	0.17352	-502.984380568594\\
67.25	0.17718	-533.108060956534\\
67.25	0.18084	-564.273980305442\\
67.25	0.1845	-596.48213861532\\
67.25	0.18816	-629.732535886167\\
67.25	0.19182	-664.025172117983\\
67.25	0.19548	-699.360047310768\\
67.25	0.19914	-735.737161464522\\
67.25	0.2028	-773.156514579245\\
67.25	0.20646	-811.618106654937\\
67.25	0.21012	-851.121937691599\\
67.25	0.21378	-891.668007689229\\
67.25	0.21744	-933.256316647828\\
67.25	0.2211	-975.886864567396\\
67.25	0.22476	-1019.55965144793\\
67.25	0.22842	-1064.27467728944\\
67.25	0.23208	-1110.03194209192\\
67.25	0.23574	-1156.83144585536\\
67.25	0.2394	-1204.67318857977\\
67.25	0.24306	-1253.55717026516\\
67.25	0.24672	-1303.48339091151\\
67.25	0.25038	-1354.45185051883\\
67.25	0.25404	-1406.46254908712\\
67.25	0.2577	-1459.51548661638\\
67.25	0.26136	-1513.61066310661\\
67.25	0.26502	-1568.74807855781\\
67.25	0.26868	-1624.92773296997\\
67.25	0.27234	-1682.14962634311\\
67.25	0.276	-1740.41375867721\\
67.625	0.093	-104.882144924247\\
67.625	0.09666	-112.147050620087\\
67.625	0.10032	-120.454195276896\\
67.625	0.10398	-129.803578894674\\
67.625	0.10764	-140.195201473421\\
67.625	0.1113	-151.629063013137\\
67.625	0.11496	-164.105163513823\\
67.625	0.11862	-177.623502975477\\
67.625	0.12228	-192.1840813981\\
67.625	0.12594	-207.786898781693\\
67.625	0.1296	-224.431955126254\\
67.625	0.13326	-242.119250431785\\
67.625	0.13692	-260.848784698284\\
67.625	0.14058	-280.620557925753\\
67.625	0.14424	-301.434570114191\\
67.625	0.1479	-323.290821263598\\
67.625	0.15156	-346.189311373974\\
67.625	0.15522	-370.130040445319\\
67.625	0.15888	-395.113008477633\\
67.625	0.16254	-421.138215470916\\
67.625	0.1662	-448.205661425168\\
67.625	0.16986	-476.31534634039\\
67.625	0.17352	-505.46727021658\\
67.625	0.17718	-535.66143305374\\
67.625	0.18084	-566.897834851868\\
67.625	0.1845	-599.176475610965\\
67.625	0.18816	-632.497355331032\\
67.625	0.19182	-666.860474012067\\
67.625	0.19548	-702.265831654072\\
67.625	0.19914	-738.713428257046\\
67.625	0.2028	-776.203263820989\\
67.625	0.20646	-814.735338345901\\
67.625	0.21012	-854.309651831782\\
67.625	0.21378	-894.926204278632\\
67.625	0.21744	-936.584995686451\\
67.625	0.2211	-979.286026055239\\
67.625	0.22476	-1023.029295385\\
67.625	0.22842	-1067.81480367572\\
67.625	0.23208	-1113.64255092742\\
67.625	0.23574	-1160.51253714008\\
67.625	0.2394	-1208.42476231372\\
67.625	0.24306	-1257.37922644832\\
67.625	0.24672	-1307.37592954389\\
67.625	0.25038	-1358.41487160043\\
67.625	0.25404	-1410.49605261794\\
67.625	0.2577	-1463.61947259642\\
67.625	0.26136	-1517.78513153587\\
67.625	0.26502	-1572.99302943628\\
67.625	0.26868	-1629.24316629767\\
67.625	0.27234	-1686.53554212003\\
67.625	0.276	-1744.87015690335\\
68	0.093	-105.89688230771\\
68	0.09666	-113.23227045277\\
68	0.10032	-121.609897558799\\
68	0.10398	-131.029763625797\\
68	0.10764	-141.491868653764\\
68	0.1113	-152.9962126427\\
68	0.11496	-165.542795592605\\
68	0.11862	-179.131617503479\\
68	0.12228	-193.762678375322\\
68	0.12594	-209.435978208134\\
68	0.1296	-226.151517001916\\
68	0.13326	-243.909294756666\\
68	0.13692	-262.709311472386\\
68	0.14058	-282.551567149074\\
68	0.14424	-303.436061786732\\
68	0.1479	-325.362795385358\\
68	0.15156	-348.331767944954\\
68	0.15522	-372.342979465519\\
68	0.15888	-397.396429947053\\
68	0.16254	-423.492119389556\\
68	0.1662	-450.630047793028\\
68	0.16986	-478.810215157469\\
68	0.17352	-508.032621482879\\
68	0.17718	-538.297266769258\\
68	0.18084	-569.604151016606\\
68	0.1845	-601.953274224924\\
68	0.18816	-635.34463639421\\
68	0.19182	-669.778237524465\\
68	0.19548	-705.254077615689\\
68	0.19914	-741.772156667883\\
68	0.2028	-779.332474681046\\
68	0.20646	-817.935031655178\\
68	0.21012	-857.579827590279\\
68	0.21378	-898.266862486348\\
68	0.21744	-939.996136343387\\
68	0.2211	-982.767649161395\\
68	0.22476	-1026.58140094037\\
68	0.22842	-1071.43739168032\\
68	0.23208	-1117.33562138123\\
68	0.23574	-1164.27609004312\\
68	0.2394	-1212.25879766597\\
68	0.24306	-1261.28374424979\\
68	0.24672	-1311.35092979459\\
68	0.25038	-1362.46035430035\\
68	0.25404	-1414.61201776708\\
68	0.2577	-1467.80592019477\\
68	0.26136	-1522.04206158344\\
68	0.26502	-1577.32044193308\\
68	0.26868	-1633.64106124368\\
68	0.27234	-1691.00391951526\\
68	0.276	-1749.4090167478\\
68.375	0.093	-106.994081309486\\
68.375	0.09666	-114.399951903766\\
68.375	0.10032	-122.848061459015\\
68.375	0.10398	-132.338409975232\\
68.375	0.10764	-142.870997452419\\
68.375	0.1113	-154.445823890575\\
68.375	0.11496	-167.062889289699\\
68.375	0.11862	-180.722193649793\\
68.375	0.12228	-195.423736970856\\
68.375	0.12594	-211.167519252888\\
68.375	0.1296	-227.953540495889\\
68.375	0.13326	-245.781800699859\\
68.375	0.13692	-264.652299864798\\
68.375	0.14058	-284.565037990707\\
68.375	0.14424	-305.520015077584\\
68.375	0.1479	-327.517231125431\\
68.375	0.15156	-350.556686134246\\
68.375	0.15522	-374.638380104031\\
68.375	0.15888	-399.762313034785\\
68.375	0.16254	-425.928484926507\\
68.375	0.1662	-453.136895779199\\
68.375	0.16986	-481.38754559286\\
68.375	0.17352	-510.680434367489\\
68.375	0.17718	-541.015562103089\\
68.375	0.18084	-572.392928799656\\
68.375	0.1845	-604.812534457194\\
68.375	0.18816	-638.2743790757\\
68.375	0.19182	-672.778462655175\\
68.375	0.19548	-708.32478519562\\
68.375	0.19914	-744.913346697033\\
68.375	0.2028	-782.544147159416\\
68.375	0.20646	-821.217186582767\\
68.375	0.21012	-860.932464967088\\
68.375	0.21378	-901.689982312377\\
68.375	0.21744	-943.489738618636\\
68.375	0.2211	-986.331733885864\\
68.375	0.22476	-1030.21596811406\\
68.375	0.22842	-1075.14244130323\\
68.375	0.23208	-1121.11115345336\\
68.375	0.23574	-1168.12210456447\\
68.375	0.2394	-1216.17529463654\\
68.375	0.24306	-1265.27072366958\\
68.375	0.24672	-1315.40839166359\\
68.375	0.25038	-1366.58829861857\\
68.375	0.25404	-1418.81044453452\\
68.375	0.2577	-1472.07482941144\\
68.375	0.26136	-1526.38145324933\\
68.375	0.26502	-1581.73031604818\\
68.375	0.26868	-1638.12141780801\\
68.375	0.27234	-1695.5547585288\\
68.375	0.276	-1754.03033821057\\
68.75	0.093	-108.173741929575\\
68.75	0.09666	-115.650094973074\\
68.75	0.10032	-124.168686977543\\
68.75	0.10398	-133.72951794298\\
68.75	0.10764	-144.332587869387\\
68.75	0.1113	-155.977896756762\\
68.75	0.11496	-168.665444605107\\
68.75	0.11862	-182.395231414421\\
68.75	0.12228	-197.167257184703\\
68.75	0.12594	-212.981521915955\\
68.75	0.1296	-229.838025608176\\
68.75	0.13326	-247.736768261366\\
68.75	0.13692	-266.677749875525\\
68.75	0.14058	-286.660970450653\\
68.75	0.14424	-307.68642998675\\
68.75	0.1479	-329.754128483817\\
68.75	0.15156	-352.864065941852\\
68.75	0.15522	-377.016242360856\\
68.75	0.15888	-402.21065774083\\
68.75	0.16254	-428.447312081772\\
68.75	0.1662	-455.726205383684\\
68.75	0.16986	-484.047337646565\\
68.75	0.17352	-513.410708870414\\
68.75	0.17718	-543.816319055233\\
68.75	0.18084	-575.264168201021\\
68.75	0.1845	-607.754256307778\\
68.75	0.18816	-641.286583375504\\
68.75	0.19182	-675.861149404199\\
68.75	0.19548	-711.477954393863\\
68.75	0.19914	-748.136998344495\\
68.75	0.2028	-785.838281256098\\
68.75	0.20646	-824.581803128669\\
68.75	0.21012	-864.36756396221\\
68.75	0.21378	-905.195563756719\\
68.75	0.21744	-947.065802512198\\
68.75	0.2211	-989.978280228646\\
68.75	0.22476	-1033.93299690606\\
68.75	0.22842	-1078.92995254445\\
68.75	0.23208	-1124.9691471438\\
68.75	0.23574	-1172.05058070413\\
68.75	0.2394	-1220.17425322542\\
68.75	0.24306	-1269.34016470768\\
68.75	0.24672	-1319.54831515091\\
68.75	0.25038	-1370.79870455511\\
68.75	0.25404	-1423.09133292028\\
68.75	0.2577	-1476.42620024642\\
68.75	0.26136	-1530.80330653353\\
68.75	0.26502	-1586.2226517816\\
68.75	0.26868	-1642.68423599065\\
68.75	0.27234	-1700.18805916066\\
68.75	0.276	-1758.73412129165\\
69.125	0.093	-109.435864167976\\
69.125	0.09666	-116.982699660696\\
69.125	0.10032	-125.571774114384\\
69.125	0.10398	-135.203087529041\\
69.125	0.10764	-145.876639904667\\
69.125	0.1113	-157.592431241263\\
69.125	0.11496	-170.350461538827\\
69.125	0.11862	-184.150730797361\\
69.125	0.12228	-198.993239016863\\
69.125	0.12594	-214.877986197335\\
69.125	0.1296	-231.804972338775\\
69.125	0.13326	-249.774197441185\\
69.125	0.13692	-268.785661504564\\
69.125	0.14058	-288.839364528912\\
69.125	0.14424	-309.935306514229\\
69.125	0.1479	-332.073487460515\\
69.125	0.15156	-355.25390736777\\
69.125	0.15522	-379.476566235994\\
69.125	0.15888	-404.741464065187\\
69.125	0.16254	-431.04860085535\\
69.125	0.1662	-458.397976606481\\
69.125	0.16986	-486.789591318582\\
69.125	0.17352	-516.22344499165\\
69.125	0.17718	-546.699537625689\\
69.125	0.18084	-578.217869220696\\
69.125	0.1845	-610.778439776674\\
69.125	0.18816	-644.38124929362\\
69.125	0.19182	-679.026297771534\\
69.125	0.19548	-714.713585210418\\
69.125	0.19914	-751.443111610271\\
69.125	0.2028	-789.214876971093\\
69.125	0.20646	-828.028881292885\\
69.125	0.21012	-867.885124575645\\
69.125	0.21378	-908.783606819374\\
69.125	0.21744	-950.724328024072\\
69.125	0.2211	-993.707288189739\\
69.125	0.22476	-1037.73248731638\\
69.125	0.22842	-1082.79992540398\\
69.125	0.23208	-1128.90960245256\\
69.125	0.23574	-1176.0615184621\\
69.125	0.2394	-1224.25567343261\\
69.125	0.24306	-1273.49206736409\\
69.125	0.24672	-1323.77070025654\\
69.125	0.25038	-1375.09157210997\\
69.125	0.25404	-1427.45468292435\\
69.125	0.2577	-1480.86003269971\\
69.125	0.26136	-1535.30762143604\\
69.125	0.26502	-1590.79744913334\\
69.125	0.26868	-1647.3295157916\\
69.125	0.27234	-1704.90382141083\\
69.125	0.276	-1763.52036599104\\
69.5	0.093	-110.780448024691\\
69.5	0.09666	-118.39776596663\\
69.5	0.10032	-127.057322869538\\
69.5	0.10398	-136.759118733415\\
69.5	0.10764	-147.503153558261\\
69.5	0.1113	-159.289427344076\\
69.5	0.11496	-172.11794009086\\
69.5	0.11862	-185.988691798614\\
69.5	0.12228	-200.901682467336\\
69.5	0.12594	-216.856912097028\\
69.5	0.1296	-233.854380687688\\
69.5	0.13326	-251.894088239318\\
69.5	0.13692	-270.976034751916\\
69.5	0.14058	-291.100220225484\\
69.5	0.14424	-312.26664466002\\
69.5	0.1479	-334.475308055526\\
69.5	0.15156	-357.726210412001\\
69.5	0.15522	-382.019351729445\\
69.5	0.15888	-407.354732007858\\
69.5	0.16254	-433.73235124724\\
69.5	0.1662	-461.152209447592\\
69.5	0.16986	-489.614306608912\\
69.5	0.17352	-519.118642731201\\
69.5	0.17718	-549.66521781446\\
69.5	0.18084	-581.254031858687\\
69.5	0.1845	-613.885084863883\\
69.5	0.18816	-647.558376830049\\
69.5	0.19182	-682.273907757183\\
69.5	0.19548	-718.031677645286\\
69.5	0.19914	-754.831686494359\\
69.5	0.2028	-792.673934304402\\
69.5	0.20646	-831.558421075413\\
69.5	0.21012	-871.485146807393\\
69.5	0.21378	-912.454111500341\\
69.5	0.21744	-954.46531515426\\
69.5	0.2211	-997.518757769147\\
69.5	0.22476	-1041.614439345\\
69.5	0.22842	-1086.75235988183\\
69.5	0.23208	-1132.93251937962\\
69.5	0.23574	-1180.15491783839\\
69.5	0.2394	-1228.41955525812\\
69.5	0.24306	-1277.72643163882\\
69.5	0.24672	-1328.07554698049\\
69.5	0.25038	-1379.46690128313\\
69.5	0.25404	-1431.90049454674\\
69.5	0.2577	-1485.37632677132\\
69.5	0.26136	-1539.89439795686\\
69.5	0.26502	-1595.45470810338\\
69.5	0.26868	-1652.05725721086\\
69.5	0.27234	-1709.70204527932\\
69.5	0.276	-1768.38907230874\\
69.875	0.093	-112.207493499719\\
69.875	0.09666	-119.895293890878\\
69.875	0.10032	-128.625333243006\\
69.875	0.10398	-138.397611556102\\
69.875	0.10764	-149.212128830168\\
69.875	0.1113	-161.068885065203\\
69.875	0.11496	-173.967880261207\\
69.875	0.11862	-187.90911441818\\
69.875	0.12228	-202.892587536122\\
69.875	0.12594	-218.918299615034\\
69.875	0.1296	-235.986250654914\\
69.875	0.13326	-254.096440655763\\
69.875	0.13692	-273.248869617581\\
69.875	0.14058	-293.443537540369\\
69.875	0.14424	-314.680444424126\\
69.875	0.1479	-336.959590268851\\
69.875	0.15156	-360.280975074546\\
69.875	0.15522	-384.644598841209\\
69.875	0.15888	-410.050461568842\\
69.875	0.16254	-436.498563257444\\
69.875	0.1662	-463.988903907015\\
69.875	0.16986	-492.521483517555\\
69.875	0.17352	-522.096302089064\\
69.875	0.17718	-552.713359621542\\
69.875	0.18084	-584.372656114989\\
69.875	0.1845	-617.074191569406\\
69.875	0.18816	-650.817965984791\\
69.875	0.19182	-685.603979361146\\
69.875	0.19548	-721.432231698469\\
69.875	0.19914	-758.302722996761\\
69.875	0.2028	-796.215453256023\\
69.875	0.20646	-835.170422476254\\
69.875	0.21012	-875.167630657454\\
69.875	0.21378	-916.207077799623\\
69.875	0.21744	-958.28876390276\\
69.875	0.2211	-1001.41268896687\\
69.875	0.22476	-1045.57885299194\\
69.875	0.22842	-1090.78725597799\\
69.875	0.23208	-1137.037897925\\
69.875	0.23574	-1184.33077883299\\
69.875	0.2394	-1232.66589870194\\
69.875	0.24306	-1282.04325753186\\
69.875	0.24672	-1332.46285532275\\
69.875	0.25038	-1383.92469207461\\
69.875	0.25404	-1436.42876778744\\
69.875	0.2577	-1489.97508246124\\
69.875	0.26136	-1544.563636096\\
69.875	0.26502	-1600.19442869174\\
69.875	0.26868	-1656.86746024844\\
69.875	0.27234	-1714.58273076612\\
69.875	0.276	-1773.34024024476\\
70.25	0.093	-113.717000593059\\
70.25	0.09666	-121.475283433438\\
70.25	0.10032	-130.275805234786\\
70.25	0.10398	-140.118565997102\\
70.25	0.10764	-151.003565720388\\
70.25	0.1113	-162.930804404642\\
70.25	0.11496	-175.900282049866\\
70.25	0.11862	-189.911998656059\\
70.25	0.12228	-204.965954223221\\
70.25	0.12594	-221.062148751352\\
70.25	0.1296	-238.200582240452\\
70.25	0.13326	-256.381254690521\\
70.25	0.13692	-275.604166101559\\
70.25	0.14058	-295.869316473567\\
70.25	0.14424	-317.176705806543\\
70.25	0.1479	-339.526334100488\\
70.25	0.15156	-362.918201355403\\
70.25	0.15522	-387.352307571286\\
70.25	0.15888	-412.828652748139\\
70.25	0.16254	-439.347236885961\\
70.25	0.1662	-466.908059984752\\
70.25	0.16986	-495.511122044511\\
70.25	0.17352	-525.15642306524\\
70.25	0.17718	-555.843963046938\\
70.25	0.18084	-587.573741989605\\
70.25	0.1845	-620.345759893241\\
70.25	0.18816	-654.160016757846\\
70.25	0.19182	-689.016512583421\\
70.25	0.19548	-724.915247369964\\
70.25	0.19914	-761.856221117476\\
70.25	0.2028	-799.839433825958\\
70.25	0.20646	-838.864885495408\\
70.25	0.21012	-878.932576125827\\
70.25	0.21378	-920.042505717216\\
70.25	0.21744	-962.194674269573\\
70.25	0.2211	-1005.3890817829\\
70.25	0.22476	-1049.6257282572\\
70.25	0.22842	-1094.90461369246\\
70.25	0.23208	-1141.22573808869\\
70.25	0.23574	-1188.5891014459\\
70.25	0.2394	-1236.99470376407\\
70.25	0.24306	-1286.44254504321\\
70.25	0.24672	-1336.93262528332\\
70.25	0.25038	-1388.4649444844\\
70.25	0.25404	-1441.03950264645\\
70.25	0.2577	-1494.65629976947\\
70.25	0.26136	-1549.31533585345\\
70.25	0.26502	-1605.01661089841\\
70.25	0.26868	-1661.76012490433\\
70.25	0.27234	-1719.54587787123\\
70.25	0.276	-1778.37386979909\\
70.625	0.093	-115.308969304712\\
70.625	0.09666	-123.137734594311\\
70.625	0.10032	-132.008738844879\\
70.625	0.10398	-141.921982056414\\
70.625	0.10764	-152.87746422892\\
70.625	0.1113	-164.875185362395\\
70.625	0.11496	-177.915145456838\\
70.625	0.11862	-191.997344512251\\
70.625	0.12228	-207.121782528632\\
70.625	0.12594	-223.288459505984\\
70.625	0.1296	-240.497375444303\\
70.625	0.13326	-258.748530343592\\
70.625	0.13692	-278.04192420385\\
70.625	0.14058	-298.377557025077\\
70.625	0.14424	-319.755428807273\\
70.625	0.1479	-342.175539550439\\
70.625	0.15156	-365.637889254573\\
70.625	0.15522	-390.142477919676\\
70.625	0.15888	-415.689305545748\\
70.625	0.16254	-442.27837213279\\
70.625	0.1662	-469.909677680801\\
70.625	0.16986	-498.58322218978\\
70.625	0.17352	-528.299005659729\\
70.625	0.17718	-559.057028090647\\
70.625	0.18084	-590.857289482533\\
70.625	0.1845	-623.699789835389\\
70.625	0.18816	-657.584529149214\\
70.625	0.19182	-692.511507424008\\
70.625	0.19548	-728.480724659771\\
70.625	0.19914	-765.492180856503\\
70.625	0.2028	-803.545876014204\\
70.625	0.20646	-842.641810132875\\
70.625	0.21012	-882.779983212514\\
70.625	0.21378	-923.960395253123\\
70.625	0.21744	-966.1830462547\\
70.625	0.2211	-1009.44793621725\\
70.625	0.22476	-1053.75506514076\\
70.625	0.22842	-1099.10443302525\\
70.625	0.23208	-1145.4960398707\\
70.625	0.23574	-1192.92988567712\\
70.625	0.2394	-1241.40597044452\\
70.625	0.24306	-1290.92429417288\\
70.625	0.24672	-1341.48485686221\\
70.625	0.25038	-1393.08765851251\\
70.625	0.25404	-1445.73269912377\\
70.625	0.2577	-1499.41997869601\\
70.625	0.26136	-1554.14949722922\\
70.625	0.26502	-1609.92125472339\\
70.625	0.26868	-1666.73525117854\\
70.625	0.27234	-1724.59148659465\\
70.625	0.276	-1783.48996097173\\
71	0.093	-116.983399634679\\
71	0.09666	-124.882647373497\\
71	0.10032	-133.824134073284\\
71	0.10398	-143.80785973404\\
71	0.10764	-154.833824355766\\
71	0.1113	-166.90202793846\\
71	0.11496	-180.012470482123\\
71	0.11862	-194.165151986756\\
71	0.12228	-209.360072452357\\
71	0.12594	-225.597231878928\\
71	0.1296	-242.876630266467\\
71	0.13326	-261.198267614975\\
71	0.13692	-280.562143924453\\
71	0.14058	-300.968259194901\\
71	0.14424	-322.416613426317\\
71	0.1479	-344.907206618701\\
71	0.15156	-368.440038772056\\
71	0.15522	-393.015109886379\\
71	0.15888	-418.632419961671\\
71	0.16254	-445.291968997932\\
71	0.1662	-472.993756995163\\
71	0.16986	-501.737783953362\\
71	0.17352	-531.52404987253\\
71	0.17718	-562.352554752667\\
71	0.18084	-594.223298593774\\
71	0.1845	-627.13628139585\\
71	0.18816	-661.091503158895\\
71	0.19182	-696.088963882908\\
71	0.19548	-732.128663567892\\
71	0.19914	-769.210602213843\\
71	0.2028	-807.334779820764\\
71	0.20646	-846.501196388655\\
71	0.21012	-886.709851917514\\
71	0.21378	-927.960746407342\\
71	0.21744	-970.253879858138\\
71	0.2211	-1013.5892522699\\
71	0.22476	-1057.96686364264\\
71	0.22842	-1103.38671397634\\
71	0.23208	-1149.84880327102\\
71	0.23574	-1197.35313152666\\
71	0.2394	-1245.89969874327\\
71	0.24306	-1295.48850492085\\
71	0.24672	-1346.1195500594\\
71	0.25038	-1397.79283415892\\
71	0.25404	-1450.50835721941\\
71	0.2577	-1504.26611924087\\
71	0.26136	-1559.06612022329\\
71	0.26502	-1614.90836016669\\
71	0.26868	-1671.79283907105\\
71	0.27234	-1729.71955693639\\
71	0.276	-1788.68851376269\\
71.375	0.093	-118.740291582958\\
71.375	0.09666	-126.710021770996\\
71.375	0.10032	-135.721990920003\\
71.375	0.10398	-145.776199029978\\
71.375	0.10764	-156.872646100923\\
71.375	0.1113	-169.011332132838\\
71.375	0.11496	-182.192257125721\\
71.375	0.11862	-196.415421079573\\
71.375	0.12228	-211.680823994394\\
71.375	0.12594	-227.988465870185\\
71.375	0.1296	-245.338346706944\\
71.375	0.13326	-263.730466504673\\
71.375	0.13692	-283.16482526337\\
71.375	0.14058	-303.641422983037\\
71.375	0.14424	-325.160259663673\\
71.375	0.1479	-347.721335305277\\
71.375	0.15156	-371.324649907851\\
71.375	0.15522	-395.970203471394\\
71.375	0.15888	-421.657995995906\\
71.375	0.16254	-448.388027481387\\
71.375	0.1662	-476.160297927838\\
71.375	0.16986	-504.974807335257\\
71.375	0.17352	-534.831555703645\\
71.375	0.17718	-565.730543033002\\
71.375	0.18084	-597.671769323329\\
71.375	0.1845	-630.655234574624\\
71.375	0.18816	-664.680938786888\\
71.375	0.19182	-699.748881960122\\
71.375	0.19548	-735.859064094324\\
71.375	0.19914	-773.011485189496\\
71.375	0.2028	-811.206145245637\\
71.375	0.20646	-850.443044262747\\
71.375	0.21012	-890.722182240826\\
71.375	0.21378	-932.043559179874\\
71.375	0.21744	-974.407175079891\\
71.375	0.2211	-1017.81302994088\\
71.375	0.22476	-1062.26112376283\\
71.375	0.22842	-1107.75145654576\\
71.375	0.23208	-1154.28402828965\\
71.375	0.23574	-1201.85883899451\\
71.375	0.2394	-1250.47588866034\\
71.375	0.24306	-1300.13517728714\\
71.375	0.24672	-1350.83670487491\\
71.375	0.25038	-1402.58047142365\\
71.375	0.25404	-1455.36647693336\\
71.375	0.2577	-1509.19472140404\\
71.375	0.26136	-1564.06520483568\\
71.375	0.26502	-1619.9779272283\\
71.375	0.26868	-1676.93288858188\\
71.375	0.27234	-1734.93008889643\\
71.375	0.276	-1793.96952817196\\
71.75	0.093	-120.579645149551\\
71.75	0.09666	-128.619857786808\\
71.75	0.10032	-137.702309385035\\
71.75	0.10398	-147.82699994423\\
71.75	0.10764	-158.993929464395\\
71.75	0.1113	-171.203097945529\\
71.75	0.11496	-184.454505387632\\
71.75	0.11862	-198.748151790704\\
71.75	0.12228	-214.084037154745\\
71.75	0.12594	-230.462161479756\\
71.75	0.1296	-247.882524765735\\
71.75	0.13326	-266.345127012683\\
71.75	0.13692	-285.8499682206\\
71.75	0.14058	-306.397048389487\\
71.75	0.14424	-327.986367519342\\
71.75	0.1479	-350.617925610167\\
71.75	0.15156	-374.291722661961\\
71.75	0.15522	-399.007758674723\\
71.75	0.15888	-424.766033648455\\
71.75	0.16254	-451.566547583156\\
71.75	0.1662	-479.409300478826\\
71.75	0.16986	-508.294292335465\\
71.75	0.17352	-538.221523153073\\
71.75	0.17718	-569.19099293165\\
71.75	0.18084	-601.202701671196\\
71.75	0.1845	-634.256649371712\\
71.75	0.18816	-668.352836033195\\
71.75	0.19182	-703.491261655649\\
71.75	0.19548	-739.671926239071\\
71.75	0.19914	-776.894829783462\\
71.75	0.2028	-815.159972288823\\
71.75	0.20646	-854.467353755153\\
71.75	0.21012	-894.816974182452\\
71.75	0.21378	-936.20883357072\\
71.75	0.21744	-978.642931919956\\
71.75	0.2211	-1022.11926923016\\
71.75	0.22476	-1066.63784550134\\
71.75	0.22842	-1112.19866073348\\
71.75	0.23208	-1158.80171492659\\
71.75	0.23574	-1206.44700808068\\
71.75	0.2394	-1255.13454019573\\
71.75	0.24306	-1304.86431127175\\
71.75	0.24672	-1355.63632130874\\
71.75	0.25038	-1407.4505703067\\
71.75	0.25404	-1460.30705826562\\
71.75	0.2577	-1514.20578518552\\
71.75	0.26136	-1569.14675106639\\
71.75	0.26502	-1625.12995590822\\
71.75	0.26868	-1682.15539971102\\
71.75	0.27234	-1740.2230824748\\
71.75	0.276	-1799.33300419954\\
72.125	0.093	-122.501460334455\\
72.125	0.09666	-130.612155420933\\
72.125	0.10032	-139.765089468379\\
72.125	0.10398	-149.960262476795\\
72.125	0.10764	-161.197674446179\\
72.125	0.1113	-173.477325376533\\
72.125	0.11496	-186.799215267856\\
72.125	0.11862	-201.163344120148\\
72.125	0.12228	-216.569711933408\\
72.125	0.12594	-233.018318707639\\
72.125	0.1296	-250.509164442837\\
72.125	0.13326	-269.042249139005\\
72.125	0.13692	-288.617572796142\\
72.125	0.14058	-309.235135414249\\
72.125	0.14424	-330.894936993324\\
72.125	0.1479	-353.596977533369\\
72.125	0.15156	-377.341257034382\\
72.125	0.15522	-402.127775496365\\
72.125	0.15888	-427.956532919316\\
72.125	0.16254	-454.827529303237\\
72.125	0.1662	-482.740764648127\\
72.125	0.16986	-511.696238953986\\
72.125	0.17352	-541.693952220812\\
72.125	0.17718	-572.73390444861\\
72.125	0.18084	-604.816095637375\\
72.125	0.1845	-637.940525787111\\
72.125	0.18816	-672.107194897815\\
72.125	0.19182	-707.316102969488\\
72.125	0.19548	-743.56725000213\\
72.125	0.19914	-780.860635995742\\
72.125	0.2028	-819.196260950322\\
72.125	0.20646	-858.574124865872\\
72.125	0.21012	-898.99422774239\\
72.125	0.21378	-940.456569579878\\
72.125	0.21744	-982.961150378334\\
72.125	0.2211	-1026.50797013776\\
72.125	0.22476	-1071.09702885815\\
72.125	0.22842	-1116.72832653952\\
72.125	0.23208	-1163.40186318185\\
72.125	0.23574	-1211.11763878515\\
72.125	0.2394	-1259.87565334942\\
72.125	0.24306	-1309.67590687466\\
72.125	0.24672	-1360.51839936087\\
72.125	0.25038	-1412.40313080805\\
72.125	0.25404	-1465.3301012162\\
72.125	0.2577	-1519.29931058532\\
72.125	0.26136	-1574.3107589154\\
72.125	0.26502	-1630.36444620646\\
72.125	0.26868	-1687.46037245848\\
72.125	0.27234	-1745.59853767147\\
72.125	0.276	-1804.77894184543\\
72.5	0.093	-124.505737137673\\
72.5	0.09666	-132.68691467337\\
72.5	0.10032	-141.910331170037\\
72.5	0.10398	-152.175986627672\\
72.5	0.10764	-163.483881046276\\
72.5	0.1113	-175.83401442585\\
72.5	0.11496	-189.226386766392\\
72.5	0.11862	-203.660998067904\\
72.5	0.12228	-219.137848330385\\
72.5	0.12594	-235.656937553835\\
72.5	0.1296	-253.218265738253\\
72.5	0.13326	-271.821832883641\\
72.5	0.13692	-291.467638989998\\
72.5	0.14058	-312.155684057324\\
72.5	0.14424	-333.885968085619\\
72.5	0.1479	-356.658491074883\\
72.5	0.15156	-380.473253025117\\
72.5	0.15522	-405.330253936319\\
72.5	0.15888	-431.22949380849\\
72.5	0.16254	-458.170972641631\\
72.5	0.1662	-486.154690435741\\
72.5	0.16986	-515.180647190819\\
72.5	0.17352	-545.248842906866\\
72.5	0.17718	-576.359277583883\\
72.5	0.18084	-608.511951221869\\
72.5	0.1845	-641.706863820824\\
72.5	0.18816	-675.944015380747\\
72.5	0.19182	-711.223405901641\\
72.5	0.19548	-747.545035383502\\
72.5	0.19914	-784.908903826333\\
72.5	0.2028	-823.315011230134\\
72.5	0.20646	-862.763357594903\\
72.5	0.21012	-903.253942920642\\
72.5	0.21378	-944.786767207349\\
72.5	0.21744	-987.361830455025\\
72.5	0.2211	-1030.97913266367\\
72.5	0.22476	-1075.63867383329\\
72.5	0.22842	-1121.34045396387\\
72.5	0.23208	-1168.08447305542\\
72.5	0.23574	-1215.87073110794\\
72.5	0.2394	-1264.69922812143\\
72.5	0.24306	-1314.56996409589\\
72.5	0.24672	-1365.48293903132\\
72.5	0.25038	-1417.43815292772\\
72.5	0.25404	-1470.43560578509\\
72.5	0.2577	-1524.47529760342\\
72.5	0.26136	-1579.55722838273\\
72.5	0.26502	-1635.681398123\\
72.5	0.26868	-1692.84780682425\\
72.5	0.27234	-1751.05645448646\\
72.5	0.276	-1810.30734110964\\
72.875	0.093	-126.592475559204\\
72.875	0.09666	-134.844135544121\\
72.875	0.10032	-144.138034490007\\
72.875	0.10398	-154.474172396862\\
72.875	0.10764	-165.852549264686\\
72.875	0.1113	-178.27316509348\\
72.875	0.11496	-191.736019883242\\
72.875	0.11862	-206.241113633973\\
72.875	0.12228	-221.788446345674\\
72.875	0.12594	-238.378018018343\\
72.875	0.1296	-256.009828651982\\
72.875	0.13326	-274.683878246589\\
72.875	0.13692	-294.400166802166\\
72.875	0.14058	-315.158694318712\\
72.875	0.14424	-336.959460796227\\
72.875	0.1479	-359.802466234711\\
72.875	0.15156	-383.687710634164\\
72.875	0.15522	-408.615193994586\\
72.875	0.15888	-434.584916315977\\
72.875	0.16254	-461.596877598337\\
72.875	0.1662	-489.651077841667\\
72.875	0.16986	-518.747517045965\\
72.875	0.17352	-548.886195211232\\
72.875	0.17718	-580.067112337468\\
72.875	0.18084	-612.290268424674\\
72.875	0.1845	-645.555663472849\\
72.875	0.18816	-679.863297481993\\
72.875	0.19182	-715.213170452105\\
72.875	0.19548	-751.605282383187\\
72.875	0.19914	-789.039633275238\\
72.875	0.2028	-827.516223128258\\
72.875	0.20646	-867.035051942247\\
72.875	0.21012	-907.596119717206\\
72.875	0.21378	-949.199426453132\\
72.875	0.21744	-991.844972150028\\
72.875	0.2211	-1035.53275680789\\
72.875	0.22476	-1080.26278042673\\
72.875	0.22842	-1126.03504300653\\
72.875	0.23208	-1172.8495445473\\
72.875	0.23574	-1220.70628504905\\
72.875	0.2394	-1269.60526451176\\
72.875	0.24306	-1319.54648293544\\
72.875	0.24672	-1370.52994032008\\
72.875	0.25038	-1422.5556366657\\
72.875	0.25404	-1475.62357197229\\
72.875	0.2577	-1529.73374623985\\
72.875	0.26136	-1584.88615946837\\
72.875	0.26502	-1641.08081165786\\
72.875	0.26868	-1698.31770280833\\
72.875	0.27234	-1756.59683291976\\
72.875	0.276	-1815.91820199216\\
73.25	0.093	-128.761675599048\\
73.25	0.09666	-137.083818033184\\
73.25	0.10032	-146.44819942829\\
73.25	0.10398	-156.854819784365\\
73.25	0.10764	-168.303679101409\\
73.25	0.1113	-180.794777379422\\
73.25	0.11496	-194.328114618404\\
73.25	0.11862	-208.903690818355\\
73.25	0.12228	-224.521505979276\\
73.25	0.12594	-241.181560101165\\
73.25	0.1296	-258.883853184023\\
73.25	0.13326	-277.628385227851\\
73.25	0.13692	-297.415156232647\\
73.25	0.14058	-318.244166198413\\
73.25	0.14424	-340.115415125148\\
73.25	0.1479	-363.028903012851\\
73.25	0.15156	-386.984629861524\\
73.25	0.15522	-411.982595671166\\
73.25	0.15888	-438.022800441777\\
73.25	0.16254	-465.105244173357\\
73.25	0.1662	-493.229926865906\\
73.25	0.16986	-522.396848519425\\
73.25	0.17352	-552.606009133911\\
73.25	0.17718	-583.857408709368\\
73.25	0.18084	-616.151047245793\\
73.25	0.1845	-649.486924743188\\
73.25	0.18816	-683.865041201551\\
73.25	0.19182	-719.285396620883\\
73.25	0.19548	-755.747991001185\\
73.25	0.19914	-793.252824342456\\
73.25	0.2028	-831.799896644696\\
73.25	0.20646	-871.389207907905\\
73.25	0.21012	-912.020758132082\\
73.25	0.21378	-953.694547317229\\
73.25	0.21744	-996.410575463345\\
73.25	0.2211	-1040.16884257043\\
73.25	0.22476	-1084.96934863848\\
73.25	0.22842	-1130.81209366751\\
73.25	0.23208	-1177.6970776575\\
73.25	0.23574	-1225.62430060846\\
73.25	0.2394	-1274.59376252039\\
73.25	0.24306	-1324.60546339329\\
73.25	0.24672	-1375.65940322716\\
73.25	0.25038	-1427.755582022\\
73.25	0.25404	-1480.8939997778\\
73.25	0.2577	-1535.07465649458\\
73.25	0.26136	-1590.29755217232\\
73.25	0.26502	-1646.56268681104\\
73.25	0.26868	-1703.87006041072\\
73.25	0.27234	-1762.21967297137\\
73.25	0.276	-1821.61152449299\\
73.625	0.093	-131.013337257205\\
73.625	0.09666	-139.405962140561\\
73.625	0.10032	-148.840825984887\\
73.625	0.10398	-159.317928790181\\
73.625	0.10764	-170.837270556445\\
73.625	0.1113	-183.398851283678\\
73.625	0.11496	-197.00267097188\\
73.625	0.11862	-211.648729621051\\
73.625	0.12228	-227.337027231191\\
73.625	0.12594	-244.0675638023\\
73.625	0.1296	-261.840339334378\\
73.625	0.13326	-280.655353827426\\
73.625	0.13692	-300.512607281442\\
73.625	0.14058	-321.412099696428\\
73.625	0.14424	-343.353831072382\\
73.625	0.1479	-366.337801409305\\
73.625	0.15156	-390.364010707198\\
73.625	0.15522	-415.432458966059\\
73.625	0.15888	-441.543146185891\\
73.625	0.16254	-468.69607236669\\
73.625	0.1662	-496.891237508459\\
73.625	0.16986	-526.128641611197\\
73.625	0.17352	-556.408284674904\\
73.625	0.17718	-587.73016669958\\
73.625	0.18084	-620.094287685225\\
73.625	0.1845	-653.50064763184\\
73.625	0.18816	-687.949246539423\\
73.625	0.19182	-723.440084407975\\
73.625	0.19548	-759.973161237496\\
73.625	0.19914	-797.548477027987\\
73.625	0.2028	-836.166031779446\\
73.625	0.20646	-875.825825491875\\
73.625	0.21012	-916.527858165273\\
73.625	0.21378	-958.272129799639\\
73.625	0.21744	-1001.05864039497\\
73.625	0.2211	-1044.88738995128\\
73.625	0.22476	-1089.75837846855\\
73.625	0.22842	-1135.6716059468\\
73.625	0.23208	-1182.62707238601\\
73.625	0.23574	-1230.62477778619\\
73.625	0.2394	-1279.66472214734\\
73.625	0.24306	-1329.74690546946\\
73.625	0.24672	-1380.87132775255\\
73.625	0.25038	-1433.03798899661\\
73.625	0.25404	-1486.24688920163\\
73.625	0.2577	-1540.49802836763\\
73.625	0.26136	-1595.79140649459\\
73.625	0.26502	-1652.12702358253\\
73.625	0.26868	-1709.50487963143\\
73.625	0.27234	-1767.9249746413\\
73.625	0.276	-1827.38730861214\\
74	0.093	-133.347460533674\\
74	0.09666	-141.81056786625\\
74	0.10032	-151.315914159796\\
74	0.10398	-161.863499414311\\
74	0.10764	-173.453323629794\\
74	0.1113	-186.085386806247\\
74	0.11496	-199.759688943668\\
74	0.11862	-214.476230042059\\
74	0.12228	-230.235010101419\\
74	0.12594	-247.036029121748\\
74	0.1296	-264.879287103046\\
74	0.13326	-283.764784045312\\
74	0.13692	-303.692519948549\\
74	0.14058	-324.662494812754\\
74	0.14424	-346.674708637928\\
74	0.1479	-369.729161424072\\
74	0.15156	-393.825853171184\\
74	0.15522	-418.964783879265\\
74	0.15888	-445.145953548316\\
74	0.16254	-472.369362178336\\
74	0.1662	-500.635009769325\\
74	0.16986	-529.942896321282\\
74	0.17352	-560.293021834208\\
74	0.17718	-591.685386308104\\
74	0.18084	-624.119989742969\\
74	0.1845	-657.596832138803\\
74	0.18816	-692.115913495607\\
74	0.19182	-727.677233813379\\
74	0.19548	-764.28079309212\\
74	0.19914	-801.92659133183\\
74	0.2028	-840.61462853251\\
74	0.20646	-880.344904694158\\
74	0.21012	-921.117419816776\\
74	0.21378	-962.932173900362\\
74	0.21744	-1005.78916694492\\
74	0.2211	-1049.68839895044\\
74	0.22476	-1094.62986991694\\
74	0.22842	-1140.6135798444\\
74	0.23208	-1187.63952873283\\
74	0.23574	-1235.70771658223\\
74	0.2394	-1284.8181433926\\
74	0.24306	-1334.97080916394\\
74	0.24672	-1386.16571389625\\
74	0.25038	-1438.40285758953\\
74	0.25404	-1491.68224024377\\
74	0.2577	-1546.00386185899\\
74	0.26136	-1601.36772243517\\
74	0.26502	-1657.77382197233\\
74	0.26868	-1715.22216047045\\
74	0.27234	-1773.71273792954\\
74	0.276	-1833.2455543496\\
};
\end{axis}

\begin{axis}[%
width=4.527496cm,
height=3.050847cm,
at={(6.483547cm,8.474576cm)},
scale only axis,
xmin=56,
xmax=74,
tick align=outside,
xlabel={$L_{cut}$},
xmajorgrids,
ymin=0.093,
ymax=0.276,
ylabel={$D_{rlx}$},
ymajorgrids,
zmin=-1218.53407675412,
zmax=0,
zlabel={$x_4$},
zmajorgrids,
view={-140}{50},
legend style={at={(1.03,1)},anchor=north west,legend cell align=left,align=left,draw=white!15!black}
]
\addplot3[only marks,mark=*,mark options={},mark size=1.5000pt,color=mycolor1] plot table[row sep=crcr,]{%
74	0.123	-157.873870615927\\
72	0.113	-128.813913283569\\
61	0.095	-69.1020638416352\\
56	0.093	-69.2923038407594\\
};
\addplot3[only marks,mark=*,mark options={},mark size=1.5000pt,color=mycolor2] plot table[row sep=crcr,]{%
67	0.276	-1153.95060686096\\
66	0.255	-964.723312302629\\
62	0.209	-539.505617749657\\
57	0.193	-431.830704932479\\
};
\addplot3[only marks,mark=*,mark options={},mark size=1.5000pt,color=black] plot table[row sep=crcr,]{%
69	0.104	-95.7148503042501\\
};
\addplot3[only marks,mark=*,mark options={},mark size=1.5000pt,color=black] plot table[row sep=crcr,]{%
64	0.23	-717.293928523056\\
};

\addplot3[%
surf,
opacity=0.7,
shader=interp,
colormap={mymap}{[1pt] rgb(0pt)=(0.0901961,0.239216,0.0745098); rgb(1pt)=(0.0945149,0.242058,0.0739522); rgb(2pt)=(0.0988592,0.244894,0.0733566); rgb(3pt)=(0.103229,0.247724,0.0727241); rgb(4pt)=(0.107623,0.250549,0.0720557); rgb(5pt)=(0.112043,0.253367,0.0713525); rgb(6pt)=(0.116487,0.25618,0.0706154); rgb(7pt)=(0.120956,0.258986,0.0698456); rgb(8pt)=(0.125449,0.261787,0.0690441); rgb(9pt)=(0.129967,0.264581,0.0682118); rgb(10pt)=(0.134508,0.26737,0.06735); rgb(11pt)=(0.139074,0.270152,0.0664596); rgb(12pt)=(0.143663,0.272929,0.0655416); rgb(13pt)=(0.148275,0.275699,0.0645971); rgb(14pt)=(0.152911,0.278463,0.0636271); rgb(15pt)=(0.15757,0.281221,0.0626328); rgb(16pt)=(0.162252,0.283973,0.0616151); rgb(17pt)=(0.166957,0.286719,0.060575); rgb(18pt)=(0.171685,0.289458,0.0595136); rgb(19pt)=(0.176434,0.292191,0.0584321); rgb(20pt)=(0.181207,0.294918,0.0573313); rgb(21pt)=(0.186001,0.297639,0.0562123); rgb(22pt)=(0.190817,0.300353,0.0550763); rgb(23pt)=(0.195655,0.303061,0.0539242); rgb(24pt)=(0.200514,0.305763,0.052757); rgb(25pt)=(0.205395,0.308459,0.0515759); rgb(26pt)=(0.210296,0.311149,0.0503624); rgb(27pt)=(0.215212,0.313846,0.0490067); rgb(28pt)=(0.220142,0.316548,0.0475043); rgb(29pt)=(0.22509,0.319254,0.0458704); rgb(30pt)=(0.230056,0.321962,0.0441205); rgb(31pt)=(0.235042,0.324671,0.04227); rgb(32pt)=(0.240048,0.327379,0.0403343); rgb(33pt)=(0.245078,0.330085,0.0383287); rgb(34pt)=(0.250131,0.332786,0.0362688); rgb(35pt)=(0.25521,0.335482,0.0341698); rgb(36pt)=(0.260317,0.33817,0.0320472); rgb(37pt)=(0.265451,0.340849,0.0299163); rgb(38pt)=(0.270616,0.343517,0.0277927); rgb(39pt)=(0.275813,0.346172,0.0256916); rgb(40pt)=(0.281043,0.348814,0.0236284); rgb(41pt)=(0.286307,0.35144,0.0216186); rgb(42pt)=(0.291607,0.354048,0.0196776); rgb(43pt)=(0.296945,0.356637,0.0178207); rgb(44pt)=(0.302322,0.359206,0.0160634); rgb(45pt)=(0.307739,0.361753,0.0144211); rgb(46pt)=(0.313198,0.364275,0.0129091); rgb(47pt)=(0.318701,0.366772,0.0115428); rgb(48pt)=(0.324249,0.369242,0.0103377); rgb(49pt)=(0.329843,0.371682,0.00930909); rgb(50pt)=(0.335485,0.374093,0.00847245); rgb(51pt)=(0.341176,0.376471,0.00784314); rgb(52pt)=(0.346925,0.378826,0.00732741); rgb(53pt)=(0.352735,0.381168,0.00682184); rgb(54pt)=(0.358605,0.383497,0.00632729); rgb(55pt)=(0.364532,0.385812,0.00584464); rgb(56pt)=(0.370516,0.388113,0.00537476); rgb(57pt)=(0.376552,0.390399,0.00491852); rgb(58pt)=(0.38264,0.39267,0.00447681); rgb(59pt)=(0.388777,0.394925,0.00405048); rgb(60pt)=(0.394962,0.397164,0.00364042); rgb(61pt)=(0.401191,0.399386,0.00324749); rgb(62pt)=(0.407464,0.401592,0.00287258); rgb(63pt)=(0.413777,0.40378,0.00251655); rgb(64pt)=(0.420129,0.40595,0.00218028); rgb(65pt)=(0.426518,0.408102,0.00186463); rgb(66pt)=(0.432942,0.410234,0.00157049); rgb(67pt)=(0.439399,0.412348,0.00129873); rgb(68pt)=(0.445885,0.414441,0.00105022); rgb(69pt)=(0.452401,0.416515,0.000825833); rgb(70pt)=(0.458942,0.418567,0.000626441); rgb(71pt)=(0.465508,0.420599,0.00045292); rgb(72pt)=(0.472096,0.422609,0.000306141); rgb(73pt)=(0.478704,0.424596,0.000186979); rgb(74pt)=(0.485331,0.426562,9.63073e-05); rgb(75pt)=(0.491973,0.428504,3.49981e-05); rgb(76pt)=(0.498628,0.430422,3.92506e-06); rgb(77pt)=(0.505323,0.432315,0); rgb(78pt)=(0.512206,0.434168,0); rgb(79pt)=(0.519282,0.435983,0); rgb(80pt)=(0.526529,0.437764,0); rgb(81pt)=(0.533922,0.439512,0); rgb(82pt)=(0.54144,0.441232,0); rgb(83pt)=(0.549059,0.442927,0); rgb(84pt)=(0.556756,0.444599,0); rgb(85pt)=(0.564508,0.446252,0); rgb(86pt)=(0.572292,0.447889,0); rgb(87pt)=(0.580084,0.449514,0); rgb(88pt)=(0.587863,0.451129,0); rgb(89pt)=(0.595604,0.452737,0); rgb(90pt)=(0.603284,0.454343,0); rgb(91pt)=(0.610882,0.455948,0); rgb(92pt)=(0.618373,0.457556,0); rgb(93pt)=(0.625734,0.459171,0); rgb(94pt)=(0.632943,0.460795,0); rgb(95pt)=(0.639976,0.462432,0); rgb(96pt)=(0.64681,0.464084,0); rgb(97pt)=(0.653423,0.465756,0); rgb(98pt)=(0.659791,0.46745,0); rgb(99pt)=(0.665891,0.469169,0); rgb(100pt)=(0.6717,0.470916,0); rgb(101pt)=(0.677195,0.472696,0); rgb(102pt)=(0.682353,0.47451,0); rgb(103pt)=(0.687242,0.476355,0); rgb(104pt)=(0.691952,0.478225,0); rgb(105pt)=(0.696497,0.480118,0); rgb(106pt)=(0.700887,0.482033,0); rgb(107pt)=(0.705134,0.483968,0); rgb(108pt)=(0.709251,0.485921,0); rgb(109pt)=(0.713249,0.487891,0); rgb(110pt)=(0.71714,0.489876,0); rgb(111pt)=(0.720936,0.491875,0); rgb(112pt)=(0.724649,0.493887,0); rgb(113pt)=(0.72829,0.495909,0); rgb(114pt)=(0.731872,0.49794,0); rgb(115pt)=(0.735406,0.499979,0); rgb(116pt)=(0.738904,0.502025,0); rgb(117pt)=(0.742378,0.504075,0); rgb(118pt)=(0.74584,0.506128,0); rgb(119pt)=(0.749302,0.508182,0); rgb(120pt)=(0.752775,0.510237,0); rgb(121pt)=(0.756272,0.51229,0); rgb(122pt)=(0.759804,0.514339,0); rgb(123pt)=(0.763384,0.516385,0); rgb(124pt)=(0.767022,0.518424,0); rgb(125pt)=(0.770731,0.520455,0); rgb(126pt)=(0.774523,0.522478,0); rgb(127pt)=(0.77841,0.524489,0); rgb(128pt)=(0.782391,0.526491,0); rgb(129pt)=(0.786402,0.528496,0); rgb(130pt)=(0.790431,0.530506,0); rgb(131pt)=(0.794478,0.532521,0); rgb(132pt)=(0.798541,0.534539,0); rgb(133pt)=(0.802619,0.53656,0); rgb(134pt)=(0.806712,0.538584,0); rgb(135pt)=(0.81082,0.540609,0); rgb(136pt)=(0.81494,0.542635,0); rgb(137pt)=(0.819074,0.54466,0); rgb(138pt)=(0.823219,0.546686,0); rgb(139pt)=(0.827374,0.548709,0); rgb(140pt)=(0.831541,0.55073,0); rgb(141pt)=(0.835716,0.552749,0); rgb(142pt)=(0.8399,0.554763,0); rgb(143pt)=(0.844092,0.556774,0); rgb(144pt)=(0.848292,0.558779,0); rgb(145pt)=(0.852497,0.560778,0); rgb(146pt)=(0.856708,0.562771,0); rgb(147pt)=(0.860924,0.564756,0); rgb(148pt)=(0.865143,0.566733,0); rgb(149pt)=(0.869366,0.568701,0); rgb(150pt)=(0.873592,0.57066,0); rgb(151pt)=(0.877819,0.572608,0); rgb(152pt)=(0.882047,0.574545,0); rgb(153pt)=(0.886275,0.576471,0); rgb(154pt)=(0.890659,0.578362,0); rgb(155pt)=(0.895333,0.580203,0); rgb(156pt)=(0.900258,0.581999,0); rgb(157pt)=(0.905397,0.583755,0); rgb(158pt)=(0.910711,0.585479,0); rgb(159pt)=(0.916164,0.587176,0); rgb(160pt)=(0.921717,0.588852,0); rgb(161pt)=(0.927333,0.590513,0); rgb(162pt)=(0.932974,0.592166,0); rgb(163pt)=(0.938602,0.593815,0); rgb(164pt)=(0.94418,0.595468,0); rgb(165pt)=(0.949669,0.59713,0); rgb(166pt)=(0.955033,0.598808,0); rgb(167pt)=(0.960233,0.600507,0); rgb(168pt)=(0.965232,0.602233,0); rgb(169pt)=(0.969992,0.603992,0); rgb(170pt)=(0.974475,0.605791,0); rgb(171pt)=(0.978643,0.607636,0); rgb(172pt)=(0.98246,0.609532,0); rgb(173pt)=(0.985886,0.611486,0); rgb(174pt)=(0.988885,0.613503,0); rgb(175pt)=(0.991419,0.61559,0); rgb(176pt)=(0.99345,0.617753,0); rgb(177pt)=(0.99494,0.619997,0); rgb(178pt)=(0.995851,0.622329,0); rgb(179pt)=(0.996226,0.624763,0); rgb(180pt)=(0.996512,0.627352,0); rgb(181pt)=(0.996788,0.630095,0); rgb(182pt)=(0.997053,0.632982,0); rgb(183pt)=(0.997308,0.636004,0); rgb(184pt)=(0.997552,0.639152,0); rgb(185pt)=(0.997785,0.642416,0); rgb(186pt)=(0.998006,0.645786,0); rgb(187pt)=(0.998217,0.649253,0); rgb(188pt)=(0.998416,0.652807,0); rgb(189pt)=(0.998605,0.656439,0); rgb(190pt)=(0.998781,0.660138,0); rgb(191pt)=(0.998946,0.663897,0); rgb(192pt)=(0.9991,0.667704,0); rgb(193pt)=(0.999242,0.67155,0); rgb(194pt)=(0.999372,0.675427,0); rgb(195pt)=(0.99949,0.679323,0); rgb(196pt)=(0.999596,0.68323,0); rgb(197pt)=(0.99969,0.687139,0); rgb(198pt)=(0.999771,0.691039,0); rgb(199pt)=(0.999841,0.694921,0); rgb(200pt)=(0.999898,0.698775,0); rgb(201pt)=(0.999942,0.702592,0); rgb(202pt)=(0.999974,0.706363,0); rgb(203pt)=(0.999994,0.710077,0); rgb(204pt)=(1,0.713725,0); rgb(205pt)=(1,0.717341,0); rgb(206pt)=(1,0.720963,0); rgb(207pt)=(1,0.724591,0); rgb(208pt)=(1,0.728226,0); rgb(209pt)=(1,0.731867,0); rgb(210pt)=(1,0.735514,0); rgb(211pt)=(1,0.739167,0); rgb(212pt)=(1,0.742827,0); rgb(213pt)=(1,0.746493,0); rgb(214pt)=(1,0.750165,0); rgb(215pt)=(1,0.753843,0); rgb(216pt)=(1,0.757527,0); rgb(217pt)=(1,0.761217,0); rgb(218pt)=(1,0.764913,0); rgb(219pt)=(1,0.768615,0); rgb(220pt)=(1,0.772324,0); rgb(221pt)=(1,0.776038,0); rgb(222pt)=(1,0.779758,0); rgb(223pt)=(1,0.783484,0); rgb(224pt)=(1,0.787215,0); rgb(225pt)=(1,0.790953,0); rgb(226pt)=(1,0.794696,0); rgb(227pt)=(1,0.798445,0); rgb(228pt)=(1,0.8022,0); rgb(229pt)=(1,0.805961,0); rgb(230pt)=(1,0.809727,0); rgb(231pt)=(1,0.8135,0); rgb(232pt)=(1,0.817278,0); rgb(233pt)=(1,0.821063,0); rgb(234pt)=(1,0.824854,0); rgb(235pt)=(1,0.828652,0); rgb(236pt)=(1,0.832455,0); rgb(237pt)=(1,0.836265,0); rgb(238pt)=(1,0.840081,0); rgb(239pt)=(1,0.843903,0); rgb(240pt)=(1,0.847732,0); rgb(241pt)=(1,0.851566,0); rgb(242pt)=(1,0.855406,0); rgb(243pt)=(1,0.859253,0); rgb(244pt)=(1,0.863106,0); rgb(245pt)=(1,0.866964,0); rgb(246pt)=(1,0.870829,0); rgb(247pt)=(1,0.8747,0); rgb(248pt)=(1,0.878577,0); rgb(249pt)=(1,0.88246,0); rgb(250pt)=(1,0.886349,0); rgb(251pt)=(1,0.890243,0); rgb(252pt)=(1,0.894144,0); rgb(253pt)=(1,0.898051,0); rgb(254pt)=(1,0.901964,0); rgb(255pt)=(1,0.905882,0)},
mesh/rows=49]
table[row sep=crcr,header=false] {%
%
56	0.093	-68.8851680905014\\
56	0.09666	-73.2417407626757\\
56	0.10032	-78.2635614646208\\
56	0.10398	-83.9506301963366\\
56	0.10764	-90.3029469578231\\
56	0.1113	-97.3205117490803\\
56	0.11496	-105.003324570108\\
56	0.11862	-113.351385420907\\
56	0.12228	-122.364694301476\\
56	0.12594	-132.043251211816\\
56	0.1296	-142.387056151927\\
56	0.13326	-153.396109121808\\
56	0.13692	-165.07041012146\\
56	0.14058	-177.409959150883\\
56	0.14424	-190.414756210076\\
56	0.1479	-204.084801299041\\
56	0.15156	-218.420094417776\\
56	0.15522	-233.420635566281\\
56	0.15888	-249.086424744558\\
56	0.16254	-265.417461952605\\
56	0.1662	-282.413747190422\\
56	0.16986	-300.075280458011\\
56	0.17352	-318.40206175537\\
56	0.17718	-337.394091082499\\
56	0.18084	-357.0513684394\\
56	0.1845	-377.373893826072\\
56	0.18816	-398.361667242513\\
56	0.19182	-420.014688688726\\
56	0.19548	-442.332958164709\\
56	0.19914	-465.316475670463\\
56	0.2028	-488.965241205988\\
56	0.20646	-513.279254771284\\
56	0.21012	-538.25851636635\\
56	0.21378	-563.903025991186\\
56	0.21744	-590.212783645794\\
56	0.2211	-617.187789330172\\
56	0.22476	-644.828043044321\\
56	0.22842	-673.133544788241\\
56	0.23208	-702.104294561931\\
56	0.23574	-731.740292365392\\
56	0.2394	-762.041538198624\\
56	0.24306	-793.008032061627\\
56	0.24672	-824.6397739544\\
56	0.25038	-856.936763876944\\
56	0.25404	-889.899001829258\\
56	0.2577	-923.526487811343\\
56	0.26136	-957.8192218232\\
56	0.26502	-992.777203864826\\
56	0.26868	-1028.40043393622\\
56	0.27234	-1064.68891203739\\
56	0.276	-1101.64263816833\\
56.375	0.093	-68.5399545371641\\
56.375	0.09666	-72.9363945906963\\
56.375	0.10032	-77.9980826739995\\
56.375	0.10398	-83.7250187870731\\
56.375	0.10764	-90.1172029299176\\
56.375	0.1113	-97.1746351025326\\
56.375	0.11496	-104.897315304918\\
56.375	0.11862	-113.285243537075\\
56.375	0.12228	-122.338419799002\\
56.375	0.12594	-132.0568440907\\
56.375	0.1296	-142.440516412169\\
56.375	0.13326	-153.489436763408\\
56.375	0.13692	-165.203605144418\\
56.375	0.14058	-177.583021555199\\
56.375	0.14424	-190.62768599575\\
56.375	0.1479	-204.337598466072\\
56.375	0.15156	-218.712758966165\\
56.375	0.15522	-233.753167496029\\
56.375	0.15888	-249.458824055663\\
56.375	0.16254	-265.829728645068\\
56.375	0.1662	-282.865881264243\\
56.375	0.16986	-300.56728191319\\
56.375	0.17352	-318.933930591907\\
56.375	0.17718	-337.965827300394\\
56.375	0.18084	-357.662972038653\\
56.375	0.1845	-378.025364806682\\
56.375	0.18816	-399.053005604482\\
56.375	0.19182	-420.745894432053\\
56.375	0.19548	-443.104031289394\\
56.375	0.19914	-466.127416176506\\
56.375	0.2028	-489.816049093389\\
56.375	0.20646	-514.169930040042\\
56.375	0.21012	-539.189059016466\\
56.375	0.21378	-564.873436022661\\
56.375	0.21744	-591.223061058626\\
56.375	0.2211	-618.237934124362\\
56.375	0.22476	-645.918055219869\\
56.375	0.22842	-674.263424345147\\
56.375	0.23208	-703.274041500195\\
56.375	0.23574	-732.949906685014\\
56.375	0.2394	-763.291019899604\\
56.375	0.24306	-794.297381143964\\
56.375	0.24672	-825.968990418095\\
56.375	0.25038	-858.305847721997\\
56.375	0.25404	-891.307953055669\\
56.375	0.2577	-924.975306419113\\
56.375	0.26136	-959.307907812326\\
56.375	0.26502	-994.305757235311\\
56.375	0.26868	-1029.96885468807\\
56.375	0.27234	-1066.29720017059\\
56.375	0.276	-1103.29079368289\\
56.75	0.093	-68.2282338684784\\
56.75	0.09666	-72.6645413033687\\
56.75	0.10032	-77.7660967680295\\
56.75	0.10398	-83.5329002624611\\
56.75	0.10764	-89.9649517866634\\
56.75	0.1113	-97.0622513406364\\
56.75	0.11496	-104.82479892438\\
56.75	0.11862	-113.252594537895\\
56.75	0.12228	-122.34563818118\\
56.75	0.12594	-132.103929854236\\
56.75	0.1296	-142.527469557062\\
56.75	0.13326	-153.616257289659\\
56.75	0.13692	-165.370293052027\\
56.75	0.14058	-177.789576844166\\
56.75	0.14424	-190.874108666075\\
56.75	0.1479	-204.623888517755\\
56.75	0.15156	-219.038916399206\\
56.75	0.15522	-234.119192310427\\
56.75	0.15888	-249.86471625142\\
56.75	0.16254	-266.275488222182\\
56.75	0.1662	-283.351508222716\\
56.75	0.16986	-301.09277625302\\
56.75	0.17352	-319.499292313095\\
56.75	0.17718	-338.571056402941\\
56.75	0.18084	-358.308068522557\\
56.75	0.1845	-378.710328671944\\
56.75	0.18816	-399.777836851102\\
56.75	0.19182	-421.510593060031\\
56.75	0.19548	-443.90859729873\\
56.75	0.19914	-466.971849567199\\
56.75	0.2028	-490.70034986544\\
56.75	0.20646	-515.094098193452\\
56.75	0.21012	-540.153094551233\\
56.75	0.21378	-565.877338938786\\
56.75	0.21744	-592.266831356109\\
56.75	0.2211	-619.321571803203\\
56.75	0.22476	-647.041560280068\\
56.75	0.22842	-675.426796786704\\
56.75	0.23208	-704.47728132311\\
56.75	0.23574	-734.193013889287\\
56.75	0.2394	-764.573994485234\\
56.75	0.24306	-795.620223110953\\
56.75	0.24672	-827.331699766442\\
56.75	0.25038	-859.708424451702\\
56.75	0.25404	-892.750397166732\\
56.75	0.2577	-926.457617911533\\
56.75	0.26136	-960.830086686105\\
56.75	0.26502	-995.867803490447\\
56.75	0.26868	-1031.57076832456\\
56.75	0.27234	-1067.93898118844\\
56.75	0.276	-1104.9724420821\\
57.125	0.093	-67.9500060844444\\
57.125	0.09666	-72.4261809006924\\
57.125	0.10032	-77.5676037467113\\
57.125	0.10398	-83.3742746225008\\
57.125	0.10764	-89.846193528061\\
57.125	0.1113	-96.983360463392\\
57.125	0.11496	-104.785775428494\\
57.125	0.11862	-113.253438423366\\
57.125	0.12228	-122.386349448009\\
57.125	0.12594	-132.184508502423\\
57.125	0.1296	-142.647915586607\\
57.125	0.13326	-153.776570700562\\
57.125	0.13692	-165.570473844288\\
57.125	0.14058	-178.029625017785\\
57.125	0.14424	-191.154024221052\\
57.125	0.1479	-204.94367145409\\
57.125	0.15156	-219.398566716899\\
57.125	0.15522	-234.518710009478\\
57.125	0.15888	-250.304101331828\\
57.125	0.16254	-266.754740683949\\
57.125	0.1662	-283.870628065841\\
57.125	0.16986	-301.651763477503\\
57.125	0.17352	-320.098146918935\\
57.125	0.17718	-339.209778390139\\
57.125	0.18084	-358.986657891113\\
57.125	0.1845	-379.428785421858\\
57.125	0.18816	-400.536160982374\\
57.125	0.19182	-422.30878457266\\
57.125	0.19548	-444.746656192717\\
57.125	0.19914	-467.849775842545\\
57.125	0.2028	-491.618143522144\\
57.125	0.20646	-516.051759231513\\
57.125	0.21012	-541.150622970653\\
57.125	0.21378	-566.914734739563\\
57.125	0.21744	-593.344094538244\\
57.125	0.2211	-620.438702366697\\
57.125	0.22476	-648.198558224919\\
57.125	0.22842	-676.623662112913\\
57.125	0.23208	-705.714014030676\\
57.125	0.23574	-735.469613978212\\
57.125	0.2394	-765.890461955517\\
57.125	0.24306	-796.976557962593\\
57.125	0.24672	-828.72790199944\\
57.125	0.25038	-861.144494066058\\
57.125	0.25404	-894.226334162446\\
57.125	0.2577	-927.973422288605\\
57.125	0.26136	-962.385758444535\\
57.125	0.26502	-997.463342630236\\
57.125	0.26868	-1033.20617484571\\
57.125	0.27234	-1069.61425509095\\
57.125	0.276	-1106.68758336596\\
57.5	0.093	-67.7052711850617\\
57.5	0.09666	-72.2213133826678\\
57.5	0.10032	-77.4026036100446\\
57.5	0.10398	-83.249141867192\\
57.5	0.10764	-89.7609281541102\\
57.5	0.1113	-96.9379624707989\\
57.5	0.11496	-104.780244817259\\
57.5	0.11862	-113.287775193489\\
57.5	0.12228	-122.46055359949\\
57.5	0.12594	-132.298580035262\\
57.5	0.1296	-142.801854500804\\
57.5	0.13326	-153.970376996117\\
57.5	0.13692	-165.804147521201\\
57.5	0.14058	-178.303166076055\\
57.5	0.14424	-191.46743266068\\
57.5	0.1479	-205.296947275076\\
57.5	0.15156	-219.791709919243\\
57.5	0.15522	-234.95172059318\\
57.5	0.15888	-250.776979296888\\
57.5	0.16254	-267.267486030367\\
57.5	0.1662	-284.423240793616\\
57.5	0.16986	-302.244243586636\\
57.5	0.17352	-320.730494409427\\
57.5	0.17718	-339.881993261988\\
57.5	0.18084	-359.698740144321\\
57.5	0.1845	-380.180735056424\\
57.5	0.18816	-401.327977998297\\
57.5	0.19182	-423.140468969941\\
57.5	0.19548	-445.618207971356\\
57.5	0.19914	-468.761195002542\\
57.5	0.2028	-492.569430063499\\
57.5	0.20646	-517.042913154226\\
57.5	0.21012	-542.181644274724\\
57.5	0.21378	-567.985623424992\\
57.5	0.21744	-594.454850605031\\
57.5	0.2211	-621.589325814841\\
57.5	0.22476	-649.389049054422\\
57.5	0.22842	-677.854020323773\\
57.5	0.23208	-706.984239622895\\
57.5	0.23574	-736.779706951788\\
57.5	0.2394	-767.240422310451\\
57.5	0.24306	-798.366385698885\\
57.5	0.24672	-830.15759711709\\
57.5	0.25038	-862.614056565066\\
57.5	0.25404	-895.735764042812\\
57.5	0.2577	-929.522719550329\\
57.5	0.26136	-963.974923087617\\
57.5	0.26502	-999.092374654675\\
57.5	0.26868	-1034.8750742515\\
57.5	0.27234	-1071.3230218781\\
57.5	0.276	-1108.43621753447\\
57.875	0.093	-67.4940291703307\\
57.875	0.09666	-72.0499387492947\\
57.875	0.10032	-77.2710963580294\\
57.875	0.10398	-83.1575019965348\\
57.875	0.10764	-89.7091556648109\\
57.875	0.1113	-96.9260573628576\\
57.875	0.11496	-104.808207090675\\
57.875	0.11862	-113.355604848263\\
57.875	0.12228	-122.568250635622\\
57.875	0.12594	-132.446144452752\\
57.875	0.1296	-142.989286299652\\
57.875	0.13326	-154.197676176323\\
57.875	0.13692	-166.071314082765\\
57.875	0.14058	-178.610200018977\\
57.875	0.14424	-191.81433398496\\
57.875	0.1479	-205.683715980714\\
57.875	0.15156	-220.218346006238\\
57.875	0.15522	-235.418224061534\\
57.875	0.15888	-251.2833501466\\
57.875	0.16254	-267.813724261436\\
57.875	0.1662	-285.009346406043\\
57.875	0.16986	-302.870216580422\\
57.875	0.17352	-321.39633478457\\
57.875	0.17718	-340.58770101849\\
57.875	0.18084	-360.44431528218\\
57.875	0.1845	-380.96617757564\\
57.875	0.18816	-402.153287898872\\
57.875	0.19182	-424.005646251874\\
57.875	0.19548	-446.523252634647\\
57.875	0.19914	-469.706107047191\\
57.875	0.2028	-493.554209489505\\
57.875	0.20646	-518.06755996159\\
57.875	0.21012	-543.246158463446\\
57.875	0.21378	-569.090004995072\\
57.875	0.21744	-595.59909955647\\
57.875	0.2211	-622.773442147637\\
57.875	0.22476	-650.613032768576\\
57.875	0.22842	-679.117871419285\\
57.875	0.23208	-708.287958099765\\
57.875	0.23574	-738.123292810016\\
57.875	0.2394	-768.623875550037\\
57.875	0.24306	-799.789706319829\\
57.875	0.24672	-831.620785119391\\
57.875	0.25038	-864.117111948725\\
57.875	0.25404	-897.27868680783\\
57.875	0.2577	-931.105509696704\\
57.875	0.26136	-965.59758061535\\
57.875	0.26502	-1000.75489956377\\
57.875	0.26868	-1036.57746654195\\
57.875	0.27234	-1073.06528154991\\
57.875	0.276	-1110.21834458764\\
58.25	0.093	-67.3162800402513\\
58.25	0.09666	-71.9120570005732\\
58.25	0.10032	-77.1730819906658\\
58.25	0.10398	-83.0993550105291\\
58.25	0.10764	-89.6908760601631\\
58.25	0.1113	-96.9476451395677\\
58.25	0.11496	-104.869662248743\\
58.25	0.11862	-113.456927387689\\
58.25	0.12228	-122.709440556406\\
58.25	0.12594	-132.627201754894\\
58.25	0.1296	-143.210210983152\\
58.25	0.13326	-154.458468241181\\
58.25	0.13692	-166.37197352898\\
58.25	0.14058	-178.950726846551\\
58.25	0.14424	-192.194728193891\\
58.25	0.1479	-206.103977571003\\
58.25	0.15156	-220.678474977886\\
58.25	0.15522	-235.918220414539\\
58.25	0.15888	-251.823213880963\\
58.25	0.16254	-268.393455377157\\
58.25	0.1662	-285.628944903122\\
58.25	0.16986	-303.529682458859\\
58.25	0.17352	-322.095668044365\\
58.25	0.17718	-341.326901659642\\
58.25	0.18084	-361.22338330469\\
58.25	0.1845	-381.785112979509\\
58.25	0.18816	-403.012090684098\\
58.25	0.19182	-424.904316418459\\
58.25	0.19548	-447.461790182589\\
58.25	0.19914	-470.684511976491\\
58.25	0.2028	-494.572481800163\\
58.25	0.20646	-519.125699653606\\
58.25	0.21012	-544.34416553682\\
58.25	0.21378	-570.227879449804\\
58.25	0.21744	-596.776841392559\\
58.25	0.2211	-623.991051365085\\
58.25	0.22476	-651.870509367381\\
58.25	0.22842	-680.415215399449\\
58.25	0.23208	-709.625169461286\\
58.25	0.23574	-739.500371552895\\
58.25	0.2394	-770.040821674274\\
58.25	0.24306	-801.246519825424\\
58.25	0.24672	-833.117466006345\\
58.25	0.25038	-865.653660217036\\
58.25	0.25404	-898.855102457498\\
58.25	0.2577	-932.721792727731\\
58.25	0.26136	-967.253731027735\\
58.25	0.26502	-1002.45091735751\\
58.25	0.26868	-1038.31335171705\\
58.25	0.27234	-1074.84103410637\\
58.25	0.276	-1112.03396452546\\
58.625	0.093	-67.1720237948236\\
58.625	0.09666	-71.8076681365032\\
58.625	0.10032	-77.1085605079539\\
58.625	0.10398	-83.0747009091751\\
58.625	0.10764	-89.706089340167\\
58.625	0.1113	-97.0027258009296\\
58.625	0.11496	-104.964610291463\\
58.625	0.11862	-113.591742811767\\
58.625	0.12228	-122.884123361842\\
58.625	0.12594	-132.841751941687\\
58.625	0.1296	-143.464628551303\\
58.625	0.13326	-154.75275319069\\
58.625	0.13692	-166.706125859847\\
58.625	0.14058	-179.324746558776\\
58.625	0.14424	-192.608615287474\\
58.625	0.1479	-206.557732045944\\
58.625	0.15156	-221.172096834185\\
58.625	0.15522	-236.451709652195\\
58.625	0.15888	-252.396570499977\\
58.625	0.16254	-269.00667937753\\
58.625	0.1662	-286.282036284853\\
58.625	0.16986	-304.222641221947\\
58.625	0.17352	-322.828494188811\\
58.625	0.17718	-342.099595185446\\
58.625	0.18084	-362.035944211852\\
58.625	0.1845	-382.637541268029\\
58.625	0.18816	-403.904386353977\\
58.625	0.19182	-425.836479469694\\
58.625	0.19548	-448.433820615183\\
58.625	0.19914	-471.696409790443\\
58.625	0.2028	-495.624246995473\\
58.625	0.20646	-520.217332230274\\
58.625	0.21012	-545.475665494846\\
58.625	0.21378	-571.399246789188\\
58.625	0.21744	-597.9880761133\\
58.625	0.2211	-625.242153467184\\
58.625	0.22476	-653.161478850839\\
58.625	0.22842	-681.746052264264\\
58.625	0.23208	-710.995873707459\\
58.625	0.23574	-740.910943180426\\
58.625	0.2394	-771.491260683163\\
58.625	0.24306	-802.736826215671\\
58.625	0.24672	-834.64763977795\\
58.625	0.25038	-867.223701369999\\
58.625	0.25404	-900.465010991819\\
58.625	0.2577	-934.37156864341\\
58.625	0.26136	-968.943374324771\\
58.625	0.26502	-1004.1804280359\\
58.625	0.26868	-1040.08272977681\\
58.625	0.27234	-1076.65027954748\\
58.625	0.276	-1113.88307734792\\
59	0.093	-67.0612604340474\\
59	0.09666	-71.736772157085\\
59	0.10032	-77.0775319098936\\
59	0.10398	-83.0835396924726\\
59	0.10764	-89.7547955048225\\
59	0.1113	-97.091299346943\\
59	0.11496	-105.093051218834\\
59	0.11862	-113.760051120496\\
59	0.12228	-123.092299051929\\
59	0.12594	-133.089795013132\\
59	0.1296	-143.752539004106\\
59	0.13326	-155.080531024851\\
59	0.13692	-167.073771075366\\
59	0.14058	-179.732259155652\\
59	0.14424	-193.055995265709\\
59	0.1479	-207.044979405537\\
59	0.15156	-221.699211575135\\
59	0.15522	-237.018691774504\\
59	0.15888	-253.003420003644\\
59	0.16254	-269.653396262554\\
59	0.1662	-286.968620551235\\
59	0.16986	-304.949092869687\\
59	0.17352	-323.594813217909\\
59	0.17718	-342.905781595903\\
59	0.18084	-362.881998003666\\
59	0.1845	-383.523462441201\\
59	0.18816	-404.830174908506\\
59	0.19182	-426.802135405582\\
59	0.19548	-449.439343932429\\
59	0.19914	-472.741800489046\\
59	0.2028	-496.709505075434\\
59	0.20646	-521.342457691593\\
59	0.21012	-546.640658337523\\
59	0.21378	-572.604107013223\\
59	0.21744	-599.232803718694\\
59	0.2211	-626.526748453935\\
59	0.22476	-654.485941218947\\
59	0.22842	-683.110382013731\\
59	0.23208	-712.400070838284\\
59	0.23574	-742.355007692609\\
59	0.2394	-772.975192576704\\
59	0.24306	-804.26062549057\\
59	0.24672	-836.211306434206\\
59	0.25038	-868.827235407613\\
59	0.25404	-902.108412410792\\
59	0.2577	-936.05483744374\\
59	0.26136	-970.666510506459\\
59	0.26502	-1005.94343159895\\
59	0.26868	-1041.88560072121\\
59	0.27234	-1078.49301787324\\
59	0.276	-1115.76568305504\\
59.375	0.093	-66.9839899579226\\
59.375	0.09666	-71.6993690623184\\
59.375	0.10032	-77.0799961964846\\
59.375	0.10398	-83.1258713604217\\
59.375	0.10764	-89.8369945541295\\
59.375	0.1113	-97.2133657776078\\
59.375	0.11496	-105.254985030857\\
59.375	0.11862	-113.961852313877\\
59.375	0.12228	-123.333967626668\\
59.375	0.12594	-133.371330969229\\
59.375	0.1296	-144.073942341561\\
59.375	0.13326	-155.441801743663\\
59.375	0.13692	-167.474909175537\\
59.375	0.14058	-180.173264637181\\
59.375	0.14424	-193.536868128595\\
59.375	0.1479	-207.565719649781\\
59.375	0.15156	-222.259819200737\\
59.375	0.15522	-237.619166781464\\
59.375	0.15888	-253.643762391962\\
59.375	0.16254	-270.33360603223\\
59.375	0.1662	-287.688697702269\\
59.375	0.16986	-305.709037402079\\
59.375	0.17352	-324.394625131659\\
59.375	0.17718	-343.74546089101\\
59.375	0.18084	-363.761544680131\\
59.375	0.1845	-384.442876499024\\
59.375	0.18816	-405.789456347687\\
59.375	0.19182	-427.801284226121\\
59.375	0.19548	-450.478360134326\\
59.375	0.19914	-473.820684072301\\
59.375	0.2028	-497.828256040047\\
59.375	0.20646	-522.501076037564\\
59.375	0.21012	-547.839144064851\\
59.375	0.21378	-573.842460121909\\
59.375	0.21744	-600.511024208738\\
59.375	0.2211	-627.844836325338\\
59.375	0.22476	-655.843896471708\\
59.375	0.22842	-684.508204647849\\
59.375	0.23208	-713.83776085376\\
59.375	0.23574	-743.832565089443\\
59.375	0.2394	-774.492617354896\\
59.375	0.24306	-805.817917650119\\
59.375	0.24672	-837.808465975114\\
59.375	0.25038	-870.464262329879\\
59.375	0.25404	-903.785306714415\\
59.375	0.2577	-937.771599128722\\
59.375	0.26136	-972.423139572799\\
59.375	0.26502	-1007.73992804665\\
59.375	0.26868	-1043.72196455027\\
59.375	0.27234	-1080.36924908365\\
59.375	0.276	-1117.68178164681\\
59.75	0.093	-66.9402123664495\\
59.75	0.09666	-71.695458852203\\
59.75	0.10032	-77.1159533677274\\
59.75	0.10398	-83.2016959130223\\
59.75	0.10764	-89.9526864880879\\
59.75	0.1113	-97.3689250929244\\
59.75	0.11496	-105.450411727531\\
59.75	0.11862	-114.197146391909\\
59.75	0.12228	-123.609129086058\\
59.75	0.12594	-133.686359809977\\
59.75	0.1296	-144.428838563667\\
59.75	0.13326	-155.836565347127\\
59.75	0.13692	-167.909540160359\\
59.75	0.14058	-180.64776300336\\
59.75	0.14424	-194.051233876133\\
59.75	0.1479	-208.119952778676\\
59.75	0.15156	-222.853919710991\\
59.75	0.15522	-238.253134673075\\
59.75	0.15888	-254.317597664931\\
59.75	0.16254	-271.047308686557\\
59.75	0.1662	-288.442267737954\\
59.75	0.16986	-306.502474819122\\
59.75	0.17352	-325.22792993006\\
59.75	0.17718	-344.618633070769\\
59.75	0.18084	-364.674584241248\\
59.75	0.1845	-385.395783441499\\
59.75	0.18816	-406.78223067152\\
59.75	0.19182	-428.833925931312\\
59.75	0.19548	-451.550869220874\\
59.75	0.19914	-474.933060540207\\
59.75	0.2028	-498.980499889311\\
59.75	0.20646	-523.693187268186\\
59.75	0.21012	-549.071122676832\\
59.75	0.21378	-575.114306115247\\
59.75	0.21744	-601.822737583434\\
59.75	0.2211	-629.196417081392\\
59.75	0.22476	-657.23534460912\\
59.75	0.22842	-685.939520166619\\
59.75	0.23208	-715.308943753888\\
59.75	0.23574	-745.343615370929\\
59.75	0.2394	-776.043535017739\\
59.75	0.24306	-807.408702694321\\
59.75	0.24672	-839.439118400673\\
59.75	0.25038	-872.134782136797\\
59.75	0.25404	-905.49569390269\\
59.75	0.2577	-939.521853698355\\
59.75	0.26136	-974.21326152379\\
59.75	0.26502	-1009.569917379\\
59.75	0.26868	-1045.59182126397\\
59.75	0.27234	-1082.27897317872\\
59.75	0.276	-1119.63137312324\\
60.125	0.093	-66.929927659628\\
60.125	0.09666	-71.7250415267396\\
60.125	0.10032	-77.1854034236218\\
60.125	0.10398	-83.3110133502746\\
60.125	0.10764	-90.1018713066981\\
60.125	0.1113	-97.5579772928924\\
60.125	0.11496	-105.679331308857\\
60.125	0.11862	-114.465933354593\\
60.125	0.12228	-123.9177834301\\
60.125	0.12594	-134.034881535377\\
60.125	0.1296	-144.817227670424\\
60.125	0.13326	-156.264821835243\\
60.125	0.13692	-168.377664029832\\
60.125	0.14058	-181.155754254192\\
60.125	0.14424	-194.599092508322\\
60.125	0.1479	-208.707678792224\\
60.125	0.15156	-223.481513105896\\
60.125	0.15522	-238.920595449338\\
60.125	0.15888	-255.024925822552\\
60.125	0.16254	-271.794504225536\\
60.125	0.1662	-289.229330658291\\
60.125	0.16986	-307.329405120816\\
60.125	0.17352	-326.094727613112\\
60.125	0.17718	-345.525298135179\\
60.125	0.18084	-365.621116687017\\
60.125	0.1845	-386.382183268625\\
60.125	0.18816	-407.808497880004\\
60.125	0.19182	-429.900060521154\\
60.125	0.19548	-452.656871192075\\
60.125	0.19914	-476.078929892766\\
60.125	0.2028	-500.166236623228\\
60.125	0.20646	-524.91879138346\\
60.125	0.21012	-550.336594173463\\
60.125	0.21378	-576.419644993237\\
60.125	0.21744	-603.167943842782\\
60.125	0.2211	-630.581490722097\\
60.125	0.22476	-658.660285631183\\
60.125	0.22842	-687.40432857004\\
60.125	0.23208	-716.813619538667\\
60.125	0.23574	-746.888158537066\\
60.125	0.2394	-777.627945565234\\
60.125	0.24306	-809.032980623174\\
60.125	0.24672	-841.103263710884\\
60.125	0.25038	-873.838794828365\\
60.125	0.25404	-907.239573975617\\
60.125	0.2577	-941.305601152639\\
60.125	0.26136	-976.036876359433\\
60.125	0.26502	-1011.433399596\\
60.125	0.26868	-1047.49517086233\\
60.125	0.27234	-1084.22219015844\\
60.125	0.276	-1121.61445748431\\
60.5	0.093	-66.9531358374581\\
60.5	0.09666	-71.7881170859273\\
60.5	0.10032	-77.2883463641675\\
60.5	0.10398	-83.4538236721783\\
60.5	0.10764	-90.2845490099598\\
60.5	0.1113	-97.7805223775119\\
60.5	0.11496	-105.941743774835\\
60.5	0.11862	-114.768213201928\\
60.5	0.12228	-124.259930658793\\
60.5	0.12594	-134.416896145428\\
60.5	0.1296	-145.239109661833\\
60.5	0.13326	-156.72657120801\\
60.5	0.13692	-168.879280783957\\
60.5	0.14058	-181.697238389675\\
60.5	0.14424	-195.180444025163\\
60.5	0.1479	-209.328897690422\\
60.5	0.15156	-224.142599385452\\
60.5	0.15522	-239.621549110253\\
60.5	0.15888	-255.765746864825\\
60.5	0.16254	-272.575192649167\\
60.5	0.1662	-290.049886463279\\
60.5	0.16986	-308.189828307163\\
60.5	0.17352	-326.995018180817\\
60.5	0.17718	-346.465456084242\\
60.5	0.18084	-366.601142017437\\
60.5	0.1845	-387.402075980403\\
60.5	0.18816	-408.86825797314\\
60.5	0.19182	-430.999687995648\\
60.5	0.19548	-453.796366047926\\
60.5	0.19914	-477.258292129975\\
60.5	0.2028	-501.385466241795\\
60.5	0.20646	-526.177888383386\\
60.5	0.21012	-551.635558554747\\
60.5	0.21378	-577.758476755879\\
60.5	0.21744	-604.546642986781\\
60.5	0.2211	-632.000057247454\\
60.5	0.22476	-660.118719537898\\
60.5	0.22842	-688.902629858113\\
60.5	0.23208	-718.351788208098\\
60.5	0.23574	-748.466194587855\\
60.5	0.2394	-779.245848997381\\
60.5	0.24306	-810.690751436678\\
60.5	0.24672	-842.800901905747\\
60.5	0.25038	-875.576300404586\\
60.5	0.25404	-909.016946933196\\
60.5	0.2577	-943.122841491576\\
60.5	0.26136	-977.893984079727\\
60.5	0.26502	-1013.33037469765\\
60.5	0.26868	-1049.43201334534\\
60.5	0.27234	-1086.1989000228\\
60.5	0.276	-1123.63103473004\\
60.875	0.093	-67.0098368999398\\
60.875	0.09666	-71.884685529767\\
60.875	0.10032	-77.4247821893651\\
60.875	0.10398	-83.6301268787338\\
60.875	0.10764	-90.5007195978732\\
60.875	0.1113	-98.0365603467833\\
60.875	0.11496	-106.237649125464\\
60.875	0.11862	-115.103985933916\\
60.875	0.12228	-124.635570772138\\
60.875	0.12594	-134.832403640131\\
60.875	0.1296	-145.694484537894\\
60.875	0.13326	-157.221813465429\\
60.875	0.13692	-169.414390422734\\
60.875	0.14058	-182.27221540981\\
60.875	0.14424	-195.795288426656\\
60.875	0.1479	-209.983609473273\\
60.875	0.15156	-224.837178549661\\
60.875	0.15522	-240.355995655819\\
60.875	0.15888	-256.540060791749\\
60.875	0.16254	-273.389373957449\\
60.875	0.1662	-290.90393515292\\
60.875	0.16986	-309.083744378161\\
60.875	0.17352	-327.928801633173\\
60.875	0.17718	-347.439106917955\\
60.875	0.18084	-367.614660232509\\
60.875	0.1845	-388.455461576833\\
60.875	0.18816	-409.961510950928\\
60.875	0.19182	-432.132808354793\\
60.875	0.19548	-454.96935378843\\
60.875	0.19914	-478.471147251836\\
60.875	0.2028	-502.638188745014\\
60.875	0.20646	-527.470478267963\\
60.875	0.21012	-552.968015820682\\
60.875	0.21378	-579.130801403172\\
60.875	0.21744	-605.958835015432\\
60.875	0.2211	-633.452116657463\\
60.875	0.22476	-661.610646329265\\
60.875	0.22842	-690.434424030838\\
60.875	0.23208	-719.923449762181\\
60.875	0.23574	-750.077723523295\\
60.875	0.2394	-780.897245314179\\
60.875	0.24306	-812.382015134835\\
60.875	0.24672	-844.532032985261\\
60.875	0.25038	-877.347298865458\\
60.875	0.25404	-910.827812775425\\
60.875	0.2577	-944.973574715164\\
60.875	0.26136	-979.784584684673\\
60.875	0.26502	-1015.26084268395\\
60.875	0.26868	-1051.402348713\\
60.875	0.27234	-1088.20910277182\\
60.875	0.276	-1125.68110486041\\
61.25	0.093	-67.100030847073\\
61.25	0.09666	-72.0147468582581\\
61.25	0.10032	-77.594710899214\\
61.25	0.10398	-83.8399229699406\\
61.25	0.10764	-90.750383070438\\
61.25	0.1113	-98.3260912007061\\
61.25	0.11496	-106.567047360745\\
61.25	0.11862	-115.473251550554\\
61.25	0.12228	-125.044703770134\\
61.25	0.12594	-135.281404019485\\
61.25	0.1296	-146.183352298607\\
61.25	0.13326	-157.750548607499\\
61.25	0.13692	-169.982992946162\\
61.25	0.14058	-182.880685314595\\
61.25	0.14424	-196.4436257128\\
61.25	0.1479	-210.671814140775\\
61.25	0.15156	-225.565250598521\\
61.25	0.15522	-241.123935086037\\
61.25	0.15888	-257.347867603324\\
61.25	0.16254	-274.237048150382\\
61.25	0.1662	-291.791476727211\\
61.25	0.16986	-310.01115333381\\
61.25	0.17352	-328.89607797018\\
61.25	0.17718	-348.44625063632\\
61.25	0.18084	-368.661671332232\\
61.25	0.1845	-389.542340057914\\
61.25	0.18816	-411.088256813367\\
61.25	0.19182	-433.29942159859\\
61.25	0.19548	-456.175834413584\\
61.25	0.19914	-479.717495258349\\
61.25	0.2028	-503.924404132885\\
61.25	0.20646	-528.796561037192\\
61.25	0.21012	-554.333965971268\\
61.25	0.21378	-580.536618935116\\
61.25	0.21744	-607.404519928734\\
61.25	0.2211	-634.937668952123\\
61.25	0.22476	-663.136066005283\\
61.25	0.22842	-691.999711088214\\
61.25	0.23208	-721.528604200915\\
61.25	0.23574	-751.722745343387\\
61.25	0.2394	-782.582134515629\\
61.25	0.24306	-814.106771717643\\
61.25	0.24672	-846.296656949427\\
61.25	0.25038	-879.151790210982\\
61.25	0.25404	-912.672171502307\\
61.25	0.2577	-946.857800823403\\
61.25	0.26136	-981.70867817427\\
61.25	0.26502	-1017.22480355491\\
61.25	0.26868	-1053.40617696532\\
61.25	0.27234	-1090.25279840549\\
61.25	0.276	-1127.76466787544\\
61.625	0.093	-67.2237176788577\\
61.625	0.09666	-72.1783010714007\\
61.625	0.10032	-77.7981324937147\\
61.625	0.10398	-84.0832119457992\\
61.625	0.10764	-91.0335394276545\\
61.625	0.1113	-98.6491149392804\\
61.625	0.11496	-106.929938480677\\
61.625	0.11862	-115.876010051844\\
61.625	0.12228	-125.487329652783\\
61.625	0.12594	-135.763897283491\\
61.625	0.1296	-146.705712943971\\
61.625	0.13326	-158.312776634221\\
61.625	0.13692	-170.585088354242\\
61.625	0.14058	-183.522648104033\\
61.625	0.14424	-197.125455883595\\
61.625	0.1479	-211.393511692928\\
61.625	0.15156	-226.326815532032\\
61.625	0.15522	-241.925367400907\\
61.625	0.15888	-258.189167299552\\
61.625	0.16254	-275.118215227968\\
61.625	0.1662	-292.712511186154\\
61.625	0.16986	-310.972055174111\\
61.625	0.17352	-329.896847191839\\
61.625	0.17718	-349.486887239338\\
61.625	0.18084	-369.742175316607\\
61.625	0.1845	-390.662711423647\\
61.625	0.18816	-412.248495560457\\
61.625	0.19182	-434.499527727039\\
61.625	0.19548	-457.415807923391\\
61.625	0.19914	-480.997336149514\\
61.625	0.2028	-505.244112405407\\
61.625	0.20646	-530.156136691072\\
61.625	0.21012	-555.733409006506\\
61.625	0.21378	-581.975929351712\\
61.625	0.21744	-608.883697726689\\
61.625	0.2211	-636.456714131435\\
61.625	0.22476	-664.694978565953\\
61.625	0.22842	-693.598491030242\\
61.625	0.23208	-723.167251524301\\
61.625	0.23574	-753.401260048131\\
61.625	0.2394	-784.300516601731\\
61.625	0.24306	-815.865021185102\\
61.625	0.24672	-848.094773798244\\
61.625	0.25038	-880.989774441157\\
61.625	0.25404	-914.55002311384\\
61.625	0.2577	-948.775519816294\\
61.625	0.26136	-983.666264548519\\
61.625	0.26502	-1019.22225731051\\
61.625	0.26868	-1055.44349810228\\
61.625	0.27234	-1092.32998692382\\
61.625	0.276	-1129.88172377512\\
62	0.093	-67.380897395294\\
62	0.09666	-72.375348169195\\
62	0.10032	-78.0350469728669\\
62	0.10398	-84.3599938063093\\
62	0.10764	-91.3501886695224\\
62	0.1113	-99.0056315625062\\
62	0.11496	-107.326322485261\\
62	0.11862	-116.312261437786\\
62	0.12228	-125.963448420082\\
62	0.12594	-136.279883432149\\
62	0.1296	-147.261566473986\\
62	0.13326	-158.908497545594\\
62	0.13692	-171.220676646973\\
62	0.14058	-184.198103778122\\
62	0.14424	-197.840778939043\\
62	0.1479	-212.148702129733\\
62	0.15156	-227.121873350195\\
62	0.15522	-242.760292600428\\
62	0.15888	-259.063959880431\\
62	0.16254	-276.032875190204\\
62	0.1662	-293.667038529749\\
62	0.16986	-311.966449899064\\
62	0.17352	-330.931109298149\\
62	0.17718	-350.561016727006\\
62	0.18084	-370.856172185633\\
62	0.1845	-391.816575674031\\
62	0.18816	-413.4422271922\\
62	0.19182	-435.733126740139\\
62	0.19548	-458.689274317849\\
62	0.19914	-482.31066992533\\
62	0.2028	-506.597313562581\\
62	0.20646	-531.549205229603\\
62	0.21012	-557.166344926396\\
62	0.21378	-583.44873265296\\
62	0.21744	-610.396368409294\\
62	0.2211	-638.009252195399\\
62	0.22476	-666.287384011274\\
62	0.22842	-695.230763856921\\
62	0.23208	-724.839391732338\\
62	0.23574	-755.113267637526\\
62	0.2394	-786.052391572484\\
62	0.24306	-817.656763537213\\
62	0.24672	-849.926383531713\\
62	0.25038	-882.861251555983\\
62	0.25404	-916.461367610025\\
62	0.2577	-950.726731693837\\
62	0.26136	-985.657343807419\\
62	0.26502	-1021.25320395077\\
62	0.26868	-1057.5143121239\\
62	0.27234	-1094.44066832679\\
62	0.276	-1132.03227255946\\
62.375	0.093	-67.571569996382\\
62.375	0.09666	-72.6058881516408\\
62.375	0.10032	-78.3054543366706\\
62.375	0.10398	-84.6702685514711\\
62.375	0.10764	-91.7003307960421\\
62.375	0.1113	-99.3956410703838\\
62.375	0.11496	-107.756199374496\\
62.375	0.11862	-116.78200570838\\
62.375	0.12228	-126.473060072033\\
62.375	0.12594	-136.829362465458\\
62.375	0.1296	-147.850912888653\\
62.375	0.13326	-159.537711341619\\
62.375	0.13692	-171.889757824356\\
62.375	0.14058	-184.907052336863\\
62.375	0.14424	-198.589594879141\\
62.375	0.1479	-212.93738545119\\
62.375	0.15156	-227.95042405301\\
62.375	0.15522	-243.6287106846\\
62.375	0.15888	-259.972245345961\\
62.375	0.16254	-276.981028037093\\
62.375	0.1662	-294.655058757995\\
62.375	0.16986	-312.994337508668\\
62.375	0.17352	-331.998864289112\\
62.375	0.17718	-351.668639099326\\
62.375	0.18084	-372.003661939311\\
62.375	0.1845	-393.003932809067\\
62.375	0.18816	-414.669451708594\\
62.375	0.19182	-437.000218637891\\
62.375	0.19548	-459.996233596959\\
62.375	0.19914	-483.657496585797\\
62.375	0.2028	-507.984007604407\\
62.375	0.20646	-532.975766652787\\
62.375	0.21012	-558.632773730938\\
62.375	0.21378	-584.955028838859\\
62.375	0.21744	-611.942531976551\\
62.375	0.2211	-639.595283144014\\
62.375	0.22476	-667.913282341248\\
62.375	0.22842	-696.896529568252\\
62.375	0.23208	-726.545024825026\\
62.375	0.23574	-756.858768111572\\
62.375	0.2394	-787.837759427888\\
62.375	0.24306	-819.481998773976\\
62.375	0.24672	-851.791486149834\\
62.375	0.25038	-884.766221555462\\
62.375	0.25404	-918.406204990861\\
62.375	0.2577	-952.711436456031\\
62.375	0.26136	-987.681915950972\\
62.375	0.26502	-1023.31764347568\\
62.375	0.26868	-1059.61861903016\\
62.375	0.27234	-1096.58484261442\\
62.375	0.276	-1134.21631422844\\
62.75	0.093	-67.7957354821214\\
62.75	0.09666	-72.8699210187381\\
62.75	0.10032	-78.6093545851258\\
62.75	0.10398	-85.014036181284\\
62.75	0.10764	-92.0839658072131\\
62.75	0.1113	-99.8191434629127\\
62.75	0.11496	-108.219569148383\\
62.75	0.11862	-117.285242863624\\
62.75	0.12228	-127.016164608636\\
62.75	0.12594	-137.412334383419\\
62.75	0.1296	-148.473752187972\\
62.75	0.13326	-160.200418022296\\
62.75	0.13692	-172.59233188639\\
62.75	0.14058	-185.649493780256\\
62.75	0.14424	-199.371903703892\\
62.75	0.1479	-213.759561657298\\
62.75	0.15156	-228.812467640476\\
62.75	0.15522	-244.530621653424\\
62.75	0.15888	-260.914023696143\\
62.75	0.16254	-277.962673768632\\
62.75	0.1662	-295.676571870893\\
62.75	0.16986	-314.055718002924\\
62.75	0.17352	-333.100112164725\\
62.75	0.17718	-352.809754356297\\
62.75	0.18084	-373.18464457764\\
62.75	0.1845	-394.224782828754\\
62.75	0.18816	-415.930169109639\\
62.75	0.19182	-438.300803420294\\
62.75	0.19548	-461.33668576072\\
62.75	0.19914	-485.037816130916\\
62.75	0.2028	-509.404194530883\\
62.75	0.20646	-534.435820960622\\
62.75	0.21012	-560.13269542013\\
62.75	0.21378	-586.494817909409\\
62.75	0.21744	-613.522188428459\\
62.75	0.2211	-641.21480697728\\
62.75	0.22476	-669.572673555872\\
62.75	0.22842	-698.595788164234\\
62.75	0.23208	-728.284150802367\\
62.75	0.23574	-758.63776147027\\
62.75	0.2394	-789.656620167945\\
62.75	0.24306	-821.34072689539\\
62.75	0.24672	-853.690081652605\\
62.75	0.25038	-886.704684439592\\
62.75	0.25404	-920.384535256349\\
62.75	0.2577	-954.729634102877\\
62.75	0.26136	-989.739980979175\\
62.75	0.26502	-1025.41557588524\\
62.75	0.26868	-1061.75641882108\\
62.75	0.27234	-1098.76250978669\\
62.75	0.276	-1136.43384878208\\
63.125	0.093	-68.0533938525124\\
63.125	0.09666	-73.1674467704872\\
63.125	0.10032	-78.9467477182327\\
63.125	0.10398	-85.391296695749\\
63.125	0.10764	-92.501093703036\\
63.125	0.1113	-100.276138740093\\
63.125	0.11496	-108.716431806922\\
63.125	0.11862	-117.821972903521\\
63.125	0.12228	-127.592762029891\\
63.125	0.12594	-138.028799186031\\
63.125	0.1296	-149.130084371942\\
63.125	0.13326	-160.896617587624\\
63.125	0.13692	-173.328398833076\\
63.125	0.14058	-186.4254281083\\
63.125	0.14424	-200.187705413294\\
63.125	0.1479	-214.615230748058\\
63.125	0.15156	-229.708004112594\\
63.125	0.15522	-245.4660255069\\
63.125	0.15888	-261.889294930977\\
63.125	0.16254	-278.977812384824\\
63.125	0.1662	-296.731577868442\\
63.125	0.16986	-315.150591381831\\
63.125	0.17352	-334.23485292499\\
63.125	0.17718	-353.984362497921\\
63.125	0.18084	-374.399120100621\\
63.125	0.1845	-395.479125733093\\
63.125	0.18816	-417.224379395335\\
63.125	0.19182	-439.634881087349\\
63.125	0.19548	-462.710630809133\\
63.125	0.19914	-486.451628560687\\
63.125	0.2028	-510.857874342012\\
63.125	0.20646	-535.929368153108\\
63.125	0.21012	-561.666109993974\\
63.125	0.21378	-588.068099864612\\
63.125	0.21744	-615.13533776502\\
63.125	0.2211	-642.867823695198\\
63.125	0.22476	-671.265557655148\\
63.125	0.22842	-700.328539644868\\
63.125	0.23208	-730.056769664359\\
63.125	0.23574	-760.450247713621\\
63.125	0.2394	-791.508973792653\\
63.125	0.24306	-823.232947901455\\
63.125	0.24672	-855.622170040029\\
63.125	0.25038	-888.676640208373\\
63.125	0.25404	-922.396358406488\\
63.125	0.2577	-956.781324634374\\
63.125	0.26136	-991.83153889203\\
63.125	0.26502	-1027.54700117946\\
63.125	0.26868	-1063.92771149666\\
63.125	0.27234	-1100.97366984362\\
63.125	0.276	-1138.68487622036\\
63.5	0.093	-68.3445451075552\\
63.5	0.09666	-73.4984654068878\\
63.5	0.10032	-79.3176337359913\\
63.5	0.10398	-85.8020500948655\\
63.5	0.10764	-92.9517144835103\\
63.5	0.1113	-100.766626901926\\
63.5	0.11496	-109.246787350112\\
63.5	0.11862	-118.392195828069\\
63.5	0.12228	-128.202852335797\\
63.5	0.12594	-138.678756873295\\
63.5	0.1296	-149.819909440564\\
63.5	0.13326	-161.626310037604\\
63.5	0.13692	-174.097958664414\\
63.5	0.14058	-187.234855320995\\
63.5	0.14424	-201.037000007347\\
63.5	0.1479	-215.50439272347\\
63.5	0.15156	-230.637033469363\\
63.5	0.15522	-246.434922245027\\
63.5	0.15888	-262.898059050462\\
63.5	0.16254	-280.026443885667\\
63.5	0.1662	-297.820076750643\\
63.5	0.16986	-316.27895764539\\
63.5	0.17352	-335.403086569907\\
63.5	0.17718	-355.192463524195\\
63.5	0.18084	-375.647088508254\\
63.5	0.1845	-396.766961522084\\
63.5	0.18816	-418.552082565684\\
63.5	0.19182	-441.002451639055\\
63.5	0.19548	-464.118068742197\\
63.5	0.19914	-487.898933875109\\
63.5	0.2028	-512.345047037792\\
63.5	0.20646	-537.456408230247\\
63.5	0.21012	-563.233017452471\\
63.5	0.21378	-589.674874704466\\
63.5	0.21744	-616.781979986232\\
63.5	0.2211	-644.554333297768\\
63.5	0.22476	-672.991934639076\\
63.5	0.22842	-702.094784010154\\
63.5	0.23208	-731.862881411002\\
63.5	0.23574	-762.296226841622\\
63.5	0.2394	-793.394820302012\\
63.5	0.24306	-825.158661792173\\
63.5	0.24672	-857.587751312104\\
63.5	0.25038	-890.682088861807\\
63.5	0.25404	-924.441674441279\\
63.5	0.2577	-958.866508050523\\
63.5	0.26136	-993.956589689538\\
63.5	0.26502	-1029.71191935832\\
63.5	0.26868	-1066.13249705688\\
63.5	0.27234	-1103.2183227852\\
63.5	0.276	-1140.9693965433\\
63.875	0.093	-68.6691892472494\\
63.875	0.09666	-73.8629769279399\\
63.875	0.10032	-79.7220126384013\\
63.875	0.10398	-86.2462963786334\\
63.875	0.10764	-93.4358281486361\\
63.875	0.1113	-101.29060794841\\
63.875	0.11496	-109.810635777954\\
63.875	0.11862	-118.995911637268\\
63.875	0.12228	-128.846435526354\\
63.875	0.12594	-139.36220744521\\
63.875	0.1296	-150.543227393837\\
63.875	0.13326	-162.389495372235\\
63.875	0.13692	-174.901011380403\\
63.875	0.14058	-188.077775418342\\
63.875	0.14424	-201.919787486052\\
63.875	0.1479	-216.427047583533\\
63.875	0.15156	-231.599555710784\\
63.875	0.15522	-247.437311867806\\
63.875	0.15888	-263.940316054598\\
63.875	0.16254	-281.108568271162\\
63.875	0.1662	-298.942068517496\\
63.875	0.16986	-317.4408167936\\
63.875	0.17352	-336.604813099476\\
63.875	0.17718	-356.434057435122\\
63.875	0.18084	-376.928549800538\\
63.875	0.1845	-398.088290195726\\
63.875	0.18816	-419.913278620684\\
63.875	0.19182	-442.403515075413\\
63.875	0.19548	-465.558999559913\\
63.875	0.19914	-489.379732074183\\
63.875	0.2028	-513.865712618224\\
63.875	0.20646	-539.016941192036\\
63.875	0.21012	-564.833417795618\\
63.875	0.21378	-591.315142428971\\
63.875	0.21744	-618.462115092095\\
63.875	0.2211	-646.27433578499\\
63.875	0.22476	-674.751804507655\\
63.875	0.22842	-703.894521260091\\
63.875	0.23208	-733.702486042297\\
63.875	0.23574	-764.175698854275\\
63.875	0.2394	-795.314159696023\\
63.875	0.24306	-827.117868567542\\
63.875	0.24672	-859.586825468831\\
63.875	0.25038	-892.721030399891\\
63.875	0.25404	-926.520483360722\\
63.875	0.2577	-960.985184351324\\
63.875	0.26136	-996.115133371696\\
63.875	0.26502	-1031.91033042184\\
63.875	0.26868	-1068.37077550175\\
63.875	0.27234	-1105.49646861144\\
63.875	0.276	-1143.28740975089\\
64.25	0.093	-69.0273262715952\\
64.25	0.09666	-74.2609813336437\\
64.25	0.10032	-80.159884425463\\
64.25	0.10398	-86.7240355470529\\
64.25	0.10764	-93.9534346984136\\
64.25	0.1113	-101.848081879545\\
64.25	0.11496	-110.407977090447\\
64.25	0.11862	-119.63312033112\\
64.25	0.12228	-129.523511601563\\
64.25	0.12594	-140.079150901777\\
64.25	0.1296	-151.300038231762\\
64.25	0.13326	-163.186173591518\\
64.25	0.13692	-175.737556981044\\
64.25	0.14058	-188.954188400341\\
64.25	0.14424	-202.836067849409\\
64.25	0.1479	-217.383195328247\\
64.25	0.15156	-232.595570836856\\
64.25	0.15522	-248.473194375236\\
64.25	0.15888	-265.016065943387\\
64.25	0.16254	-282.224185541308\\
64.25	0.1662	-300.097553169\\
64.25	0.16986	-318.636168826463\\
64.25	0.17352	-337.840032513696\\
64.25	0.17718	-357.7091442307\\
64.25	0.18084	-378.243503977474\\
64.25	0.1845	-399.44311175402\\
64.25	0.18816	-421.307967560336\\
64.25	0.19182	-443.838071396423\\
64.25	0.19548	-467.03342326228\\
64.25	0.19914	-490.894023157909\\
64.25	0.2028	-515.419871083308\\
64.25	0.20646	-540.610967038477\\
64.25	0.21012	-566.467311023417\\
64.25	0.21378	-592.988903038129\\
64.25	0.21744	-620.17574308261\\
64.25	0.2211	-648.027831156862\\
64.25	0.22476	-676.545167260886\\
64.25	0.22842	-705.72775139468\\
64.25	0.23208	-735.575583558244\\
64.25	0.23574	-766.08866375158\\
64.25	0.2394	-797.266991974686\\
64.25	0.24306	-829.110568227562\\
64.25	0.24672	-861.619392510209\\
64.25	0.25038	-894.793464822627\\
64.25	0.25404	-928.632785164816\\
64.25	0.2577	-963.137353536776\\
64.25	0.26136	-998.307169938506\\
64.25	0.26502	-1034.14223437001\\
64.25	0.26868	-1070.64254683128\\
64.25	0.27234	-1107.80810732232\\
64.25	0.276	-1145.63891584313\\
64.625	0.093	-69.4189561805924\\
64.625	0.09666	-74.6924786239988\\
64.625	0.10032	-80.6312490971762\\
64.625	0.10398	-87.2352676001239\\
64.625	0.10764	-94.5045341328425\\
64.625	0.1113	-102.439048695332\\
64.625	0.11496	-111.038811287592\\
64.625	0.11862	-120.303821909623\\
64.625	0.12228	-130.234080561424\\
64.625	0.12594	-140.829587242996\\
64.625	0.1296	-152.090341954339\\
64.625	0.13326	-164.016344695452\\
64.625	0.13692	-176.607595466336\\
64.625	0.14058	-189.864094266991\\
64.625	0.14424	-203.785841097417\\
64.625	0.1479	-218.372835957613\\
64.625	0.15156	-233.62507884758\\
64.625	0.15522	-249.542569767318\\
64.625	0.15888	-266.125308716827\\
64.625	0.16254	-283.373295696106\\
64.625	0.1662	-301.286530705155\\
64.625	0.16986	-319.865013743976\\
64.625	0.17352	-339.108744812567\\
64.625	0.17718	-359.017723910929\\
64.625	0.18084	-379.591951039062\\
64.625	0.1845	-400.831426196965\\
64.625	0.18816	-422.736149384639\\
64.625	0.19182	-445.306120602084\\
64.625	0.19548	-468.541339849299\\
64.625	0.19914	-492.441807126285\\
64.625	0.2028	-517.007522433042\\
64.625	0.20646	-542.23848576957\\
64.625	0.21012	-568.134697135868\\
64.625	0.21378	-594.696156531937\\
64.625	0.21744	-621.922863957777\\
64.625	0.2211	-649.814819413387\\
64.625	0.22476	-678.372022898768\\
64.625	0.22842	-707.59447441392\\
64.625	0.23208	-737.482173958842\\
64.625	0.23574	-768.035121533536\\
64.625	0.2394	-799.253317137999\\
64.625	0.24306	-831.136760772234\\
64.625	0.24672	-863.685452436239\\
64.625	0.25038	-896.899392130015\\
64.625	0.25404	-930.778579853562\\
64.625	0.2577	-965.323015606879\\
64.625	0.26136	-1000.53269938997\\
64.625	0.26502	-1036.40763120283\\
64.625	0.26868	-1072.94781104546\\
64.625	0.27234	-1110.15323891786\\
64.625	0.276	-1148.02391482003\\
65	0.093	-69.8440789742414\\
65	0.09666	-75.1574687990057\\
65	0.10032	-81.1361066535409\\
65	0.10398	-87.7799925378467\\
65	0.10764	-95.0891264519232\\
65	0.1113	-103.06350839577\\
65	0.11496	-111.703138369388\\
65	0.11862	-121.008016372777\\
65	0.12228	-130.978142405936\\
65	0.12594	-141.613516468866\\
65	0.1296	-152.914138561567\\
65	0.13326	-164.880008684038\\
65	0.13692	-177.511126836281\\
65	0.14058	-190.807493018293\\
65	0.14424	-204.769107230077\\
65	0.1479	-219.395969471631\\
65	0.15156	-234.688079742956\\
65	0.15522	-250.645438044052\\
65	0.15888	-267.268044374918\\
65	0.16254	-284.555898735555\\
65	0.1662	-302.509001125963\\
65	0.16986	-321.127351546141\\
65	0.17352	-340.41094999609\\
65	0.17718	-360.35979647581\\
65	0.18084	-380.973890985301\\
65	0.1845	-402.253233524562\\
65	0.18816	-424.197824093594\\
65	0.19182	-446.807662692397\\
65	0.19548	-470.08274932097\\
65	0.19914	-494.023083979314\\
65	0.2028	-518.628666667429\\
65	0.20646	-543.899497385315\\
65	0.21012	-569.835576132971\\
65	0.21378	-596.436902910397\\
65	0.21744	-623.703477717595\\
65	0.2211	-651.635300554563\\
65	0.22476	-680.232371421302\\
65	0.22842	-709.494690317812\\
65	0.23208	-739.422257244092\\
65	0.23574	-770.015072200143\\
65	0.2394	-801.273135185965\\
65	0.24306	-833.196446201558\\
65	0.24672	-865.785005246921\\
65	0.25038	-899.038812322055\\
65	0.25404	-932.957867426959\\
65	0.2577	-967.542170561635\\
65	0.26136	-1002.79172172608\\
65	0.26502	-1038.7065209203\\
65	0.26868	-1075.28656814428\\
65	0.27234	-1112.53186339804\\
65	0.276	-1150.44240668157\\
65.375	0.093	-70.3026946525418\\
65.375	0.09666	-75.6559518586641\\
65.375	0.10032	-81.6744570945571\\
65.375	0.10398	-88.3582103602208\\
65.375	0.10764	-95.7072116556553\\
65.375	0.1113	-103.72146098086\\
65.375	0.11496	-112.400958335836\\
65.375	0.11862	-121.745703720583\\
65.375	0.12228	-131.7556971351\\
65.375	0.12594	-142.430938579388\\
65.375	0.1296	-153.771428053446\\
65.375	0.13326	-165.777165557276\\
65.375	0.13692	-178.448151090876\\
65.375	0.14058	-191.784384654246\\
65.375	0.14424	-205.785866247388\\
65.375	0.1479	-220.4525958703\\
65.375	0.15156	-235.784573522983\\
65.375	0.15522	-251.781799205436\\
65.375	0.15888	-268.444272917661\\
65.375	0.16254	-285.771994659656\\
65.375	0.1662	-303.764964431422\\
65.375	0.16986	-322.423182232958\\
65.375	0.17352	-341.746648064265\\
65.375	0.17718	-361.735361925342\\
65.375	0.18084	-382.389323816191\\
65.375	0.1845	-403.70853373681\\
65.375	0.18816	-425.6929916872\\
65.375	0.19182	-448.342697667361\\
65.375	0.19548	-471.657651677292\\
65.375	0.19914	-495.637853716994\\
65.375	0.2028	-520.283303786467\\
65.375	0.20646	-545.59400188571\\
65.375	0.21012	-571.569948014724\\
65.375	0.21378	-598.211142173509\\
65.375	0.21744	-625.517584362064\\
65.375	0.2211	-653.489274580391\\
65.375	0.22476	-682.126212828487\\
65.375	0.22842	-711.428399106355\\
65.375	0.23208	-741.395833413993\\
65.375	0.23574	-772.028515751402\\
65.375	0.2394	-803.326446118582\\
65.375	0.24306	-835.289624515533\\
65.375	0.24672	-867.918050942254\\
65.375	0.25038	-901.211725398746\\
65.375	0.25404	-935.170647885008\\
65.375	0.2577	-969.794818401041\\
65.375	0.26136	-1005.08423694685\\
65.375	0.26502	-1041.03890352242\\
65.375	0.26868	-1077.65881812776\\
65.375	0.27234	-1114.94398076288\\
65.375	0.276	-1152.89439142777\\
65.75	0.093	-70.794803215494\\
65.75	0.09666	-76.1879278029741\\
65.75	0.10032	-82.2463004202251\\
65.75	0.10398	-88.9699210672468\\
65.75	0.10764	-96.3587897440391\\
65.75	0.1113	-104.412906450602\\
65.75	0.11496	-113.132271186936\\
65.75	0.11862	-122.51688395304\\
65.75	0.12228	-132.566744748915\\
65.75	0.12594	-143.281853574561\\
65.75	0.1296	-154.662210429978\\
65.75	0.13326	-166.707815315165\\
65.75	0.13692	-179.418668230123\\
65.75	0.14058	-192.794769174852\\
65.75	0.14424	-206.836118149351\\
65.75	0.1479	-221.542715153621\\
65.75	0.15156	-236.914560187662\\
65.75	0.15522	-252.951653251474\\
65.75	0.15888	-269.653994345056\\
65.75	0.16254	-287.021583468408\\
65.75	0.1662	-305.054420621532\\
65.75	0.16986	-323.752505804426\\
65.75	0.17352	-343.115839017091\\
65.75	0.17718	-363.144420259527\\
65.75	0.18084	-383.838249531733\\
65.75	0.1845	-405.19732683371\\
65.75	0.18816	-427.221652165458\\
65.75	0.19182	-449.911225526977\\
65.75	0.19548	-473.266046918266\\
65.75	0.19914	-497.286116339326\\
65.75	0.2028	-521.971433790157\\
65.75	0.20646	-547.321999270758\\
65.75	0.21012	-573.33781278113\\
65.75	0.21378	-600.018874321273\\
65.75	0.21744	-627.365183891186\\
65.75	0.2211	-655.37674149087\\
65.75	0.22476	-684.053547120324\\
65.75	0.22842	-713.39560077955\\
65.75	0.23208	-743.402902468546\\
65.75	0.23574	-774.075452187314\\
65.75	0.2394	-805.413249935851\\
65.75	0.24306	-837.416295714159\\
65.75	0.24672	-870.084589522238\\
65.75	0.25038	-903.418131360088\\
65.75	0.25404	-937.416921227709\\
65.75	0.2577	-972.0809591251\\
65.75	0.26136	-1007.41024505226\\
65.75	0.26502	-1043.40477900919\\
65.75	0.26868	-1080.0645609959\\
65.75	0.27234	-1117.38959101237\\
65.75	0.276	-1155.37986905862\\
66.125	0.093	-71.3204046630977\\
66.125	0.09666	-76.7533966319358\\
66.125	0.10032	-82.8516366305447\\
66.125	0.10398	-89.6151246589243\\
66.125	0.10764	-97.0438607170744\\
66.125	0.1113	-105.137844804995\\
66.125	0.11496	-113.897076922687\\
66.125	0.11862	-123.321557070149\\
66.125	0.12228	-133.411285247383\\
66.125	0.12594	-144.166261454386\\
66.125	0.1296	-155.586485691161\\
66.125	0.13326	-167.671957957706\\
66.125	0.13692	-180.422678254022\\
66.125	0.14058	-193.838646580108\\
66.125	0.14424	-207.919862935966\\
66.125	0.1479	-222.666327321594\\
66.125	0.15156	-238.078039736992\\
66.125	0.15522	-254.155000182162\\
66.125	0.15888	-270.897208657102\\
66.125	0.16254	-288.304665161813\\
66.125	0.1662	-306.377369696294\\
66.125	0.16986	-325.115322260546\\
66.125	0.17352	-344.518522854569\\
66.125	0.17718	-364.586971478363\\
66.125	0.18084	-385.320668131927\\
66.125	0.1845	-406.719612815262\\
66.125	0.18816	-428.783805528368\\
66.125	0.19182	-451.513246271244\\
66.125	0.19548	-474.907935043891\\
66.125	0.19914	-498.967871846309\\
66.125	0.2028	-523.693056678498\\
66.125	0.20646	-549.083489540457\\
66.125	0.21012	-575.139170432187\\
66.125	0.21378	-601.860099353687\\
66.125	0.21744	-629.246276304959\\
66.125	0.2211	-657.297701286001\\
66.125	0.22476	-686.014374296813\\
66.125	0.22842	-715.396295337397\\
66.125	0.23208	-745.443464407751\\
66.125	0.23574	-776.155881507876\\
66.125	0.2394	-807.533546637771\\
66.125	0.24306	-839.576459797438\\
66.125	0.24672	-872.284620986875\\
66.125	0.25038	-905.658030206082\\
66.125	0.25404	-939.696687455061\\
66.125	0.2577	-974.40059273381\\
66.125	0.26136	-1009.76974604233\\
66.125	0.26502	-1045.80414738062\\
66.125	0.26868	-1082.50379674868\\
66.125	0.27234	-1119.86869414651\\
66.125	0.276	-1157.89883957411\\
66.5	0.093	-71.8794989953528\\
66.5	0.09666	-77.3523583455489\\
66.5	0.10032	-83.4904657255157\\
66.5	0.10398	-90.2938211352531\\
66.5	0.10764	-97.7624245747613\\
66.5	0.1113	-105.89627604404\\
66.5	0.11496	-114.69537554309\\
66.5	0.11862	-124.15972307191\\
66.5	0.12228	-134.289318630501\\
66.5	0.12594	-145.084162218863\\
66.5	0.1296	-156.544253836995\\
66.5	0.13326	-168.669593484898\\
66.5	0.13692	-181.460181162572\\
66.5	0.14058	-194.916016870016\\
66.5	0.14424	-209.037100607232\\
66.5	0.1479	-223.823432374218\\
66.5	0.15156	-239.275012170974\\
66.5	0.15522	-255.391839997501\\
66.5	0.15888	-272.173915853799\\
66.5	0.16254	-289.621239739868\\
66.5	0.1662	-307.733811655708\\
66.5	0.16986	-326.511631601318\\
66.5	0.17352	-345.954699576698\\
66.5	0.17718	-366.06301558185\\
66.5	0.18084	-386.836579616772\\
66.5	0.1845	-408.275391681465\\
66.5	0.18816	-430.379451775929\\
66.5	0.19182	-453.148759900163\\
66.5	0.19548	-476.583316054168\\
66.5	0.19914	-500.683120237944\\
66.5	0.2028	-525.44817245149\\
66.5	0.20646	-550.878472694808\\
66.5	0.21012	-576.974020967895\\
66.5	0.21378	-603.734817270754\\
66.5	0.21744	-631.160861603383\\
66.5	0.2211	-659.252153965783\\
66.5	0.22476	-688.008694357954\\
66.5	0.22842	-717.430482779895\\
66.5	0.23208	-747.517519231607\\
66.5	0.23574	-778.26980371309\\
66.5	0.2394	-809.687336224343\\
66.5	0.24306	-841.770116765368\\
66.5	0.24672	-874.518145336162\\
66.5	0.25038	-907.931421936728\\
66.5	0.25404	-942.009946567064\\
66.5	0.2577	-976.753719227171\\
66.5	0.26136	-1012.16273991705\\
66.5	0.26502	-1048.2370086367\\
66.5	0.26868	-1084.97652538612\\
66.5	0.27234	-1122.38129016531\\
66.5	0.276	-1160.45130297427\\
66.875	0.093	-72.4720862122598\\
66.875	0.09666	-77.9848129438136\\
66.875	0.10032	-84.1627877051385\\
66.875	0.10398	-91.0060104962338\\
66.875	0.10764	-98.5144813170999\\
66.875	0.1113	-106.688200167737\\
66.875	0.11496	-115.527167048144\\
66.875	0.11862	-125.031381958322\\
66.875	0.12228	-135.200844898271\\
66.875	0.12594	-146.035555867991\\
66.875	0.1296	-157.535514867481\\
66.875	0.13326	-169.700721896742\\
66.875	0.13692	-182.531176955774\\
66.875	0.14058	-196.026880044576\\
66.875	0.14424	-210.187831163149\\
66.875	0.1479	-225.014030311493\\
66.875	0.15156	-240.505477489608\\
66.875	0.15522	-256.662172697493\\
66.875	0.15888	-273.484115935149\\
66.875	0.16254	-290.971307202576\\
66.875	0.1662	-309.123746499773\\
66.875	0.16986	-327.941433826741\\
66.875	0.17352	-347.424369183479\\
66.875	0.17718	-367.572552569989\\
66.875	0.18084	-388.385983986269\\
66.875	0.1845	-409.86466343232\\
66.875	0.18816	-432.008590908141\\
66.875	0.19182	-454.817766413734\\
66.875	0.19548	-478.292189949097\\
66.875	0.19914	-502.43186151423\\
66.875	0.2028	-527.236781109135\\
66.875	0.20646	-552.70694873381\\
66.875	0.21012	-578.842364388255\\
66.875	0.21378	-605.643028072472\\
66.875	0.21744	-633.108939786459\\
66.875	0.2211	-661.240099530217\\
66.875	0.22476	-690.036507303745\\
66.875	0.22842	-719.498163107045\\
66.875	0.23208	-749.625066940115\\
66.875	0.23574	-780.417218802956\\
66.875	0.2394	-811.874618695567\\
66.875	0.24306	-843.997266617949\\
66.875	0.24672	-876.785162570101\\
66.875	0.25038	-910.238306552025\\
66.875	0.25404	-944.35669856372\\
66.875	0.2577	-979.140338605185\\
66.875	0.26136	-1014.58922667642\\
66.875	0.26502	-1050.70336277743\\
66.875	0.26868	-1087.4827469082\\
66.875	0.27234	-1124.92737906875\\
66.875	0.276	-1163.03725925907\\
67.25	0.093	-73.0981663138181\\
67.25	0.09666	-78.6507604267298\\
67.25	0.10032	-84.8686025694126\\
67.25	0.10398	-91.7516927418658\\
67.25	0.10764	-99.3000309440898\\
67.25	0.1113	-107.513617176084\\
67.25	0.11496	-116.39245143785\\
67.25	0.11862	-125.936533729386\\
67.25	0.12228	-136.145864050693\\
67.25	0.12594	-147.02044240177\\
67.25	0.1296	-158.560268782619\\
67.25	0.13326	-170.765343193238\\
67.25	0.13692	-183.635665633627\\
67.25	0.14058	-197.171236103788\\
67.25	0.14424	-211.372054603718\\
67.25	0.1479	-226.23812113342\\
67.25	0.15156	-241.769435692893\\
67.25	0.15522	-257.965998282136\\
67.25	0.15888	-274.82780890115\\
67.25	0.16254	-292.354867549934\\
67.25	0.1662	-310.547174228489\\
67.25	0.16986	-329.404728936816\\
67.25	0.17352	-348.927531674912\\
67.25	0.17718	-369.115582442779\\
67.25	0.18084	-389.968881240417\\
67.25	0.1845	-411.487428067826\\
67.25	0.18816	-433.671222925006\\
67.25	0.19182	-456.520265811956\\
67.25	0.19548	-480.034556728677\\
67.25	0.19914	-504.214095675168\\
67.25	0.2028	-529.058882651431\\
67.25	0.20646	-554.568917657464\\
67.25	0.21012	-580.744200693267\\
67.25	0.21378	-607.584731758842\\
67.25	0.21744	-635.090510854187\\
67.25	0.2211	-663.261537979302\\
67.25	0.22476	-692.097813134189\\
67.25	0.22842	-721.599336318846\\
67.25	0.23208	-751.766107533274\\
67.25	0.23574	-782.598126777473\\
67.25	0.2394	-814.095394051442\\
67.25	0.24306	-846.257909355181\\
67.25	0.24672	-879.085672688692\\
67.25	0.25038	-912.578684051974\\
67.25	0.25404	-946.736943445026\\
67.25	0.2577	-981.560450867849\\
67.25	0.26136	-1017.04920632044\\
67.25	0.26502	-1053.20320980281\\
67.25	0.26868	-1090.02246131494\\
67.25	0.27234	-1127.50696085685\\
67.25	0.276	-1165.65670842852\\
67.625	0.093	-73.757739300028\\
67.625	0.09666	-79.3502007942978\\
67.625	0.10032	-85.6079103183383\\
67.625	0.10398	-92.5308678721495\\
67.625	0.10764	-100.119073455731\\
67.625	0.1113	-108.372527069084\\
67.625	0.11496	-117.291228712207\\
67.625	0.11862	-126.875178385101\\
67.625	0.12228	-137.124376087766\\
67.625	0.12594	-148.038821820202\\
67.625	0.1296	-159.618515582408\\
67.625	0.13326	-171.863457374384\\
67.625	0.13692	-184.773647196132\\
67.625	0.14058	-198.34908504765\\
67.625	0.14424	-212.589770928939\\
67.625	0.1479	-227.495704839999\\
67.625	0.15156	-243.066886780829\\
67.625	0.15522	-259.30331675143\\
67.625	0.15888	-276.204994751802\\
67.625	0.16254	-293.771920781945\\
67.625	0.1662	-312.004094841858\\
67.625	0.16986	-330.901516931542\\
67.625	0.17352	-350.464187050996\\
67.625	0.17718	-370.692105200221\\
67.625	0.18084	-391.585271379217\\
67.625	0.1845	-413.143685587984\\
67.625	0.18816	-435.367347826522\\
67.625	0.19182	-458.256258094829\\
67.625	0.19548	-481.810416392908\\
67.625	0.19914	-506.029822720758\\
67.625	0.2028	-530.914477078378\\
67.625	0.20646	-556.464379465769\\
67.625	0.21012	-582.679529882931\\
67.625	0.21378	-609.559928329863\\
67.625	0.21744	-637.105574806566\\
67.625	0.2211	-665.316469313039\\
67.625	0.22476	-694.192611849284\\
67.625	0.22842	-723.734002415299\\
67.625	0.23208	-753.940641011084\\
67.625	0.23574	-784.812527636641\\
67.625	0.2394	-816.349662291968\\
67.625	0.24306	-848.552044977066\\
67.625	0.24672	-881.419675691935\\
67.625	0.25038	-914.952554436575\\
67.625	0.25404	-949.150681210984\\
67.625	0.2577	-984.014056015165\\
67.625	0.26136	-1019.54267884912\\
67.625	0.26502	-1055.73654971284\\
67.625	0.26868	-1092.59566860633\\
67.625	0.27234	-1130.12003552959\\
67.625	0.276	-1168.30965048263\\
68	0.093	-74.4508051708893\\
68	0.09666	-80.083134046517\\
68	0.10032	-86.3807109519156\\
68	0.10398	-93.3435358870846\\
68	0.10764	-100.971608852025\\
68	0.1113	-109.264929846735\\
68	0.11496	-118.223498871216\\
68	0.11862	-127.847315925468\\
68	0.12228	-138.136381009491\\
68	0.12594	-149.090694123284\\
68	0.1296	-160.710255266848\\
68	0.13326	-172.995064440183\\
68	0.13692	-185.945121643289\\
68	0.14058	-199.560426876165\\
68	0.14424	-213.840980138812\\
68	0.1479	-228.786781431229\\
68	0.15156	-244.397830753417\\
68	0.15522	-260.674128105376\\
68	0.15888	-277.615673487106\\
68	0.16254	-295.222466898607\\
68	0.1662	-313.494508339878\\
68	0.16986	-332.431797810919\\
68	0.17352	-352.034335311732\\
68	0.17718	-372.302120842315\\
68	0.18084	-393.235154402669\\
68	0.1845	-414.833435992794\\
68	0.18816	-437.096965612689\\
68	0.19182	-460.025743262355\\
68	0.19548	-483.619768941791\\
68	0.19914	-507.879042650999\\
68	0.2028	-532.803564389977\\
68	0.20646	-558.393334158726\\
68	0.21012	-584.648351957245\\
68	0.21378	-611.568617785535\\
68	0.21744	-639.154131643596\\
68	0.2211	-667.404893531428\\
68	0.22476	-696.32090344903\\
68	0.22842	-725.902161396403\\
68	0.23208	-756.148667373547\\
68	0.23574	-787.060421380462\\
68	0.2394	-818.637423417146\\
68	0.24306	-850.879673483603\\
68	0.24672	-883.787171579829\\
68	0.25038	-917.359917705826\\
68	0.25404	-951.597911861594\\
68	0.2577	-986.501154047133\\
68	0.26136	-1022.06964426244\\
68	0.26502	-1058.30338250752\\
68	0.26868	-1095.20236878237\\
68	0.27234	-1132.76660308699\\
68	0.276	-1170.99608542139\\
68.375	0.093	-75.1773639264026\\
68.375	0.09666	-80.8495601833881\\
68.375	0.10032	-87.1870044701445\\
68.375	0.10398	-94.1896967866716\\
68.375	0.10764	-101.857637132969\\
68.375	0.1113	-110.190825509038\\
68.375	0.11496	-119.189261914877\\
68.375	0.11862	-128.852946350487\\
68.375	0.12228	-139.181878815867\\
68.375	0.12594	-150.176059311019\\
68.375	0.1296	-161.835487835941\\
68.375	0.13326	-174.160164390633\\
68.375	0.13692	-187.150088975097\\
68.375	0.14058	-200.805261589331\\
68.375	0.14424	-215.125682233336\\
68.375	0.1479	-230.111350907111\\
68.375	0.15156	-245.762267610657\\
68.375	0.15522	-262.078432343974\\
68.375	0.15888	-279.059845107062\\
68.375	0.16254	-296.70650589992\\
68.375	0.1662	-315.018414722549\\
68.375	0.16986	-333.995571574949\\
68.375	0.17352	-353.637976457119\\
68.375	0.17718	-373.94562936906\\
68.375	0.18084	-394.918530310772\\
68.375	0.1845	-416.556679282254\\
68.375	0.18816	-438.860076283508\\
68.375	0.19182	-461.828721314532\\
68.375	0.19548	-485.462614375326\\
68.375	0.19914	-509.761755465892\\
68.375	0.2028	-534.726144586228\\
68.375	0.20646	-560.355781736334\\
68.375	0.21012	-586.650666916212\\
68.375	0.21378	-613.61080012586\\
68.375	0.21744	-641.236181365279\\
68.375	0.2211	-669.526810634468\\
68.375	0.22476	-698.482687933428\\
68.375	0.22842	-728.10381326216\\
68.375	0.23208	-758.390186620661\\
68.375	0.23574	-789.341808008934\\
68.375	0.2394	-820.958677426977\\
68.375	0.24306	-853.24079487479\\
68.375	0.24672	-886.188160352374\\
68.375	0.25038	-919.80077385973\\
68.375	0.25404	-954.078635396856\\
68.375	0.2577	-989.021744963753\\
68.375	0.26136	-1024.63010256042\\
68.375	0.26502	-1060.90370818686\\
68.375	0.26868	-1097.84256184307\\
68.375	0.27234	-1135.44666352905\\
68.375	0.276	-1173.7160132448\\
68.75	0.093	-75.9374155665674\\
68.75	0.09666	-81.6494792049108\\
68.75	0.10032	-88.0267908730252\\
68.75	0.10398	-95.0693505709102\\
68.75	0.10764	-102.777158298566\\
68.75	0.1113	-111.150214055992\\
68.75	0.11496	-120.188517843189\\
68.75	0.11862	-129.892069660157\\
68.75	0.12228	-140.260869506896\\
68.75	0.12594	-151.294917383405\\
68.75	0.1296	-162.994213289685\\
68.75	0.13326	-175.358757225735\\
68.75	0.13692	-188.388549191557\\
68.75	0.14058	-202.083589187148\\
68.75	0.14424	-216.443877212511\\
68.75	0.1479	-231.469413267645\\
68.75	0.15156	-247.160197352549\\
68.75	0.15522	-263.516229467223\\
68.75	0.15888	-280.537509611669\\
68.75	0.16254	-298.224037785885\\
68.75	0.1662	-316.575813989872\\
68.75	0.16986	-335.59283822363\\
68.75	0.17352	-355.275110487158\\
68.75	0.17718	-375.622630780457\\
68.75	0.18084	-396.635399103527\\
68.75	0.1845	-418.313415456367\\
68.75	0.18816	-440.656679838979\\
68.75	0.19182	-463.66519225136\\
68.75	0.19548	-487.338952693513\\
68.75	0.19914	-511.677961165436\\
68.75	0.2028	-536.68221766713\\
68.75	0.20646	-562.351722198595\\
68.75	0.21012	-588.68647475983\\
68.75	0.21378	-615.686475350836\\
68.75	0.21744	-643.351723971613\\
68.75	0.2211	-671.68222062216\\
68.75	0.22476	-700.677965302478\\
68.75	0.22842	-730.338958012567\\
68.75	0.23208	-760.665198752426\\
68.75	0.23574	-791.656687522057\\
68.75	0.2394	-823.313424321458\\
68.75	0.24306	-855.63540915063\\
68.75	0.24672	-888.622642009572\\
68.75	0.25038	-922.275122898285\\
68.75	0.25404	-956.592851816769\\
68.75	0.2577	-991.575828765023\\
68.75	0.26136	-1027.22405374305\\
68.75	0.26502	-1063.53752675084\\
68.75	0.26868	-1100.51624778841\\
68.75	0.27234	-1138.16021685575\\
68.75	0.276	-1176.46943395286\\
69.125	0.093	-76.7309600913835\\
69.125	0.09666	-82.4828911110849\\
69.125	0.10032	-88.9000701605572\\
69.125	0.10398	-95.9824972398001\\
69.125	0.10764	-103.730172348814\\
69.125	0.1113	-112.143095487598\\
69.125	0.11496	-121.221266656153\\
69.125	0.11862	-130.964685854479\\
69.125	0.12228	-141.373353082575\\
69.125	0.12594	-152.447268340442\\
69.125	0.1296	-164.18643162808\\
69.125	0.13326	-176.590842945488\\
69.125	0.13692	-189.660502292668\\
69.125	0.14058	-203.395409669617\\
69.125	0.14424	-217.795565076338\\
69.125	0.1479	-232.86096851283\\
69.125	0.15156	-248.591619979092\\
69.125	0.15522	-264.987519475124\\
69.125	0.15888	-282.048667000928\\
69.125	0.16254	-299.775062556502\\
69.125	0.1662	-318.166706141847\\
69.125	0.16986	-337.223597756962\\
69.125	0.17352	-356.945737401848\\
69.125	0.17718	-377.333125076505\\
69.125	0.18084	-398.385760780933\\
69.125	0.1845	-420.103644515132\\
69.125	0.18816	-442.486776279101\\
69.125	0.19182	-465.53515607284\\
69.125	0.19548	-489.24878389635\\
69.125	0.19914	-513.627659749632\\
69.125	0.2028	-538.671783632684\\
69.125	0.20646	-564.381155545507\\
69.125	0.21012	-590.7557754881\\
69.125	0.21378	-617.795643460464\\
69.125	0.21744	-645.500759462598\\
69.125	0.2211	-673.871123494503\\
69.125	0.22476	-702.90673555618\\
69.125	0.22842	-732.607595647626\\
69.125	0.23208	-762.973703768844\\
69.125	0.23574	-794.005059919832\\
69.125	0.2394	-825.701664100591\\
69.125	0.24306	-858.06351631112\\
69.125	0.24672	-891.090616551421\\
69.125	0.25038	-924.782964821492\\
69.125	0.25404	-959.140561121333\\
69.125	0.2577	-994.163405450946\\
69.125	0.26136	-1029.85149781033\\
69.125	0.26502	-1066.20483819948\\
69.125	0.26868	-1103.22342661841\\
69.125	0.27234	-1140.9072630671\\
69.125	0.276	-1179.25634754557\\
69.5	0.093	-77.5579975008515\\
69.5	0.09666	-83.3497959019108\\
69.5	0.10032	-89.806842332741\\
69.5	0.10398	-96.9291367933417\\
69.5	0.10764	-104.716679283713\\
69.5	0.1113	-113.169469803855\\
69.5	0.11496	-122.287508353768\\
69.5	0.11862	-132.070794933452\\
69.5	0.12228	-142.519329542906\\
69.5	0.12594	-153.633112182131\\
69.5	0.1296	-165.412142851127\\
69.5	0.13326	-177.856421549894\\
69.5	0.13692	-190.965948278431\\
69.5	0.14058	-204.740723036739\\
69.5	0.14424	-219.180745824817\\
69.5	0.1479	-234.286016642666\\
69.5	0.15156	-250.056535490286\\
69.5	0.15522	-266.492302367677\\
69.5	0.15888	-283.593317274838\\
69.5	0.16254	-301.35958021177\\
69.5	0.1662	-319.791091178473\\
69.5	0.16986	-338.887850174947\\
69.5	0.17352	-358.64985720119\\
69.5	0.17718	-379.077112257205\\
69.5	0.18084	-400.169615342991\\
69.5	0.1845	-421.927366458547\\
69.5	0.18816	-444.350365603874\\
69.5	0.19182	-467.438612778972\\
69.5	0.19548	-491.19210798384\\
69.5	0.19914	-515.610851218479\\
69.5	0.2028	-540.694842482889\\
69.5	0.20646	-566.44408177707\\
69.5	0.21012	-592.858569101021\\
69.5	0.21378	-619.938304454743\\
69.5	0.21744	-647.683287838235\\
69.5	0.2211	-676.093519251498\\
69.5	0.22476	-705.168998694532\\
69.5	0.22842	-734.909726167337\\
69.5	0.23208	-765.315701669913\\
69.5	0.23574	-796.386925202259\\
69.5	0.2394	-828.123396764376\\
69.5	0.24306	-860.525116356263\\
69.5	0.24672	-893.592083977921\\
69.5	0.25038	-927.32429962935\\
69.5	0.25404	-961.72176331055\\
69.5	0.2577	-996.78447502152\\
69.5	0.26136	-1032.51243476226\\
69.5	0.26502	-1068.90564253277\\
69.5	0.26868	-1105.96409833306\\
69.5	0.27234	-1143.68780216311\\
69.5	0.276	-1182.07675402293\\
69.875	0.093	-78.4185277949706\\
69.875	0.09666	-84.2501935773878\\
69.875	0.10032	-90.747107389576\\
69.875	0.10398	-97.9092692315347\\
69.875	0.10764	-105.736679103264\\
69.875	0.1113	-114.229337004764\\
69.875	0.11496	-123.387242936035\\
69.875	0.11862	-133.210396897077\\
69.875	0.12228	-143.698798887889\\
69.875	0.12594	-154.852448908472\\
69.875	0.1296	-166.671346958825\\
69.875	0.13326	-179.15549303895\\
69.875	0.13692	-192.304887148845\\
69.875	0.14058	-206.119529288511\\
69.875	0.14424	-220.599419457947\\
69.875	0.1479	-235.744557657154\\
69.875	0.15156	-251.554943886132\\
69.875	0.15522	-268.030578144881\\
69.875	0.15888	-285.1714604334\\
69.875	0.16254	-302.97759075169\\
69.875	0.1662	-321.448969099751\\
69.875	0.16986	-340.585595477582\\
69.875	0.17352	-360.387469885184\\
69.875	0.17718	-380.854592322557\\
69.875	0.18084	-401.9869627897\\
69.875	0.1845	-423.784581286614\\
69.875	0.18816	-446.247447813299\\
69.875	0.19182	-469.375562369755\\
69.875	0.19548	-493.168924955981\\
69.875	0.19914	-517.627535571978\\
69.875	0.2028	-542.751394217746\\
69.875	0.20646	-568.540500893284\\
69.875	0.21012	-594.994855598593\\
69.875	0.21378	-622.114458333673\\
69.875	0.21744	-649.899309098524\\
69.875	0.2211	-678.349407893145\\
69.875	0.22476	-707.464754717537\\
69.875	0.22842	-737.2453495717\\
69.875	0.23208	-767.691192455633\\
69.875	0.23574	-798.802283369337\\
69.875	0.2394	-830.578622312812\\
69.875	0.24306	-863.020209286057\\
69.875	0.24672	-896.127044289073\\
69.875	0.25038	-929.89912732186\\
69.875	0.25404	-964.336458384417\\
69.875	0.2577	-999.439037476746\\
69.875	0.26136	-1035.20686459884\\
69.875	0.26502	-1071.63993975071\\
69.875	0.26868	-1108.73826293235\\
69.875	0.27234	-1146.50183414377\\
69.875	0.276	-1184.93065338495\\
70.25	0.093	-79.3125509737417\\
70.25	0.09666	-85.1840841375169\\
70.25	0.10032	-91.7208653310629\\
70.25	0.10398	-98.9228945543795\\
70.25	0.10764	-106.790171807467\\
70.25	0.1113	-115.322697090325\\
70.25	0.11496	-124.520470402954\\
70.25	0.11862	-134.383491745353\\
70.25	0.12228	-144.911761117523\\
70.25	0.12594	-156.105278519464\\
70.25	0.1296	-167.964043951176\\
70.25	0.13326	-180.488057412658\\
70.25	0.13692	-193.677318903911\\
70.25	0.14058	-207.531828424934\\
70.25	0.14424	-222.051585975729\\
70.25	0.1479	-237.236591556294\\
70.25	0.15156	-253.08684516663\\
70.25	0.15522	-269.602346806736\\
70.25	0.15888	-286.783096476614\\
70.25	0.16254	-304.629094176261\\
70.25	0.1662	-323.14033990568\\
70.25	0.16986	-342.316833664869\\
70.25	0.17352	-362.158575453829\\
70.25	0.17718	-382.66556527256\\
70.25	0.18084	-403.837803121061\\
70.25	0.1845	-425.675288999333\\
70.25	0.18816	-448.178022907376\\
70.25	0.19182	-471.34600484519\\
70.25	0.19548	-495.179234812774\\
70.25	0.19914	-519.677712810129\\
70.25	0.2028	-544.841438837254\\
70.25	0.20646	-570.670412894151\\
70.25	0.21012	-597.164634980818\\
70.25	0.21378	-624.324105097255\\
70.25	0.21744	-652.148823243464\\
70.25	0.2211	-680.638789419443\\
70.25	0.22476	-709.794003625193\\
70.25	0.22842	-739.614465860713\\
70.25	0.23208	-770.100176126004\\
70.25	0.23574	-801.251134421067\\
70.25	0.2394	-833.067340745899\\
70.25	0.24306	-865.548795100503\\
70.25	0.24672	-898.695497484876\\
70.25	0.25038	-932.507447899021\\
70.25	0.25404	-966.984646342937\\
70.25	0.2577	-1002.12709281662\\
70.25	0.26136	-1037.93478732008\\
70.25	0.26502	-1074.40772985331\\
70.25	0.26868	-1111.54592041631\\
70.25	0.27234	-1149.34935900907\\
70.25	0.276	-1187.81804563161\\
70.625	0.093	-80.2400670371642\\
70.625	0.09666	-86.1514675822971\\
70.625	0.10032	-92.7281161572011\\
70.625	0.10398	-99.9700127618756\\
70.625	0.10764	-107.877157396321\\
70.625	0.1113	-116.449550060537\\
70.625	0.11496	-125.687190754524\\
70.625	0.11862	-135.590079478281\\
70.625	0.12228	-146.158216231809\\
70.625	0.12594	-157.391601015108\\
70.625	0.1296	-169.290233828177\\
70.625	0.13326	-181.854114671017\\
70.625	0.13692	-195.083243543628\\
70.625	0.14058	-208.97762044601\\
70.625	0.14424	-223.537245378162\\
70.625	0.1479	-238.762118340085\\
70.625	0.15156	-254.652239331779\\
70.625	0.15522	-271.207608353243\\
70.625	0.15888	-288.428225404478\\
70.625	0.16254	-306.314090485484\\
70.625	0.1662	-324.865203596261\\
70.625	0.16986	-344.081564736808\\
70.625	0.17352	-363.963173907126\\
70.625	0.17718	-384.510031107214\\
70.625	0.18084	-405.722136337074\\
70.625	0.1845	-427.599489596704\\
70.625	0.18816	-450.142090886105\\
70.625	0.19182	-473.349940205276\\
70.625	0.19548	-497.223037554218\\
70.625	0.19914	-521.761382932931\\
70.625	0.2028	-546.964976341414\\
70.625	0.20646	-572.833817779669\\
70.625	0.21012	-599.367907247694\\
70.625	0.21378	-626.567244745489\\
70.625	0.21744	-654.431830273055\\
70.625	0.2211	-682.961663830392\\
70.625	0.22476	-712.1567454175\\
70.625	0.22842	-742.017075034379\\
70.625	0.23208	-772.542652681028\\
70.625	0.23574	-803.733478357448\\
70.625	0.2394	-835.589552063638\\
70.625	0.24306	-868.1108737996\\
70.625	0.24672	-901.297443565331\\
70.625	0.25038	-935.149261360834\\
70.625	0.25404	-969.666327186107\\
70.625	0.2577	-1004.84864104115\\
70.625	0.26136	-1040.69620292597\\
70.625	0.26502	-1077.20901284055\\
70.625	0.26868	-1114.38707078491\\
70.625	0.27234	-1152.23037675903\\
70.625	0.276	-1190.73893076293\\
71	0.093	-81.2010759852384\\
71	0.09666	-87.1523439117293\\
71	0.10032	-93.7688598679913\\
71	0.10398	-101.050623854024\\
71	0.10764	-108.997635869827\\
71	0.1113	-117.609895915401\\
71	0.11496	-126.887403990745\\
71	0.11862	-136.830160095861\\
71	0.12228	-147.438164230747\\
71	0.12594	-158.711416395403\\
71	0.1296	-170.649916589831\\
71	0.13326	-183.253664814029\\
71	0.13692	-196.522661067998\\
71	0.14058	-210.456905351737\\
71	0.14424	-225.056397665247\\
71	0.1479	-240.321138008528\\
71	0.15156	-256.25112638158\\
71	0.15522	-272.846362784402\\
71	0.15888	-290.106847216995\\
71	0.16254	-308.032579679359\\
71	0.1662	-326.623560171493\\
71	0.16986	-345.879788693398\\
71	0.17352	-365.801265245074\\
71	0.17718	-386.387989826521\\
71	0.18084	-407.639962437738\\
71	0.1845	-429.557183078726\\
71	0.18816	-452.139651749485\\
71	0.19182	-475.387368450014\\
71	0.19548	-499.300333180314\\
71	0.19914	-523.878545940385\\
71	0.2028	-549.122006730226\\
71	0.20646	-575.030715549838\\
71	0.21012	-601.604672399221\\
71	0.21378	-628.843877278375\\
71	0.21744	-656.748330187299\\
71	0.2211	-685.318031125994\\
71	0.22476	-714.552980094459\\
71	0.22842	-744.453177092696\\
71	0.23208	-775.018622120703\\
71	0.23574	-806.249315178481\\
71	0.2394	-838.145256266029\\
71	0.24306	-870.706445383348\\
71	0.24672	-903.932882530438\\
71	0.25038	-937.824567707299\\
71	0.25404	-972.38150091393\\
71	0.2577	-1007.60368215033\\
71	0.26136	-1043.4911114165\\
71	0.26502	-1080.04378871245\\
71	0.26868	-1117.26171403816\\
71	0.27234	-1155.14488739365\\
71	0.276	-1193.6933087789\\
71.375	0.093	-82.1955778179638\\
71.375	0.09666	-88.1867131258128\\
71.375	0.10032	-94.8430964634325\\
71.375	0.10398	-102.164727830823\\
71.375	0.10764	-110.151607227984\\
71.375	0.1113	-118.803734654916\\
71.375	0.11496	-128.121110111618\\
71.375	0.11862	-138.103733598092\\
71.375	0.12228	-148.751605114336\\
71.375	0.12594	-160.06472466035\\
71.375	0.1296	-172.043092236135\\
71.375	0.13326	-184.686707841691\\
71.375	0.13692	-197.995571477018\\
71.375	0.14058	-211.969683142116\\
71.375	0.14424	-226.609042836984\\
71.375	0.1479	-241.913650561622\\
71.375	0.15156	-257.883506316032\\
71.375	0.15522	-274.518610100212\\
71.375	0.15888	-291.818961914163\\
71.375	0.16254	-309.784561757885\\
71.375	0.1662	-328.415409631377\\
71.375	0.16986	-347.71150553464\\
71.375	0.17352	-367.672849467674\\
71.375	0.17718	-388.299441430478\\
71.375	0.18084	-409.591281423053\\
71.375	0.1845	-431.548369445399\\
71.375	0.18816	-454.170705497516\\
71.375	0.19182	-477.458289579403\\
71.375	0.19548	-501.411121691061\\
71.375	0.19914	-526.02920183249\\
71.375	0.2028	-551.312530003689\\
71.375	0.20646	-577.261106204659\\
71.375	0.21012	-603.8749304354\\
71.375	0.21378	-631.154002695912\\
71.375	0.21744	-659.098322986194\\
71.375	0.2211	-687.707891306246\\
71.375	0.22476	-716.98270765607\\
71.375	0.22842	-746.922772035664\\
71.375	0.23208	-777.528084445029\\
71.375	0.23574	-808.798644884165\\
71.375	0.2394	-840.734453353072\\
71.375	0.24306	-873.335509851748\\
71.375	0.24672	-906.601814380196\\
71.375	0.25038	-940.533366938415\\
71.375	0.25404	-975.130167526404\\
71.375	0.2577	-1010.39221614416\\
71.375	0.26136	-1046.31951279169\\
71.375	0.26502	-1082.912057469\\
71.375	0.26868	-1120.16985017607\\
71.375	0.27234	-1158.09289091291\\
71.375	0.276	-1196.68117967952\\
71.75	0.093	-83.2235725353413\\
71.75	0.09666	-89.2545752245482\\
71.75	0.10032	-95.9508259435258\\
71.75	0.10398	-103.312324692274\\
71.75	0.10764	-111.339071470793\\
71.75	0.1113	-120.031066279083\\
71.75	0.11496	-129.388309117143\\
71.75	0.11862	-139.410799984974\\
71.75	0.12228	-150.098538882576\\
71.75	0.12594	-161.451525809949\\
71.75	0.1296	-173.469760767092\\
71.75	0.13326	-186.153243754006\\
71.75	0.13692	-199.501974770691\\
71.75	0.14058	-213.515953817146\\
71.75	0.14424	-228.195180893372\\
71.75	0.1479	-243.539655999369\\
71.75	0.15156	-259.549379135136\\
71.75	0.15522	-276.224350300674\\
71.75	0.15888	-293.564569495983\\
71.75	0.16254	-311.570036721063\\
71.75	0.1662	-330.240751975913\\
71.75	0.16986	-349.576715260534\\
71.75	0.17352	-369.577926574925\\
71.75	0.17718	-390.244385919088\\
71.75	0.18084	-411.576093293021\\
71.75	0.1845	-433.573048696725\\
71.75	0.18816	-456.235252130199\\
71.75	0.19182	-479.562703593444\\
71.75	0.19548	-503.55540308646\\
71.75	0.19914	-528.213350609247\\
71.75	0.2028	-553.536546161804\\
71.75	0.20646	-579.524989744132\\
71.75	0.21012	-606.178681356231\\
71.75	0.21378	-633.4976209981\\
71.75	0.21744	-661.48180866974\\
71.75	0.2211	-690.131244371151\\
71.75	0.22476	-719.445928102333\\
71.75	0.22842	-749.425859863285\\
71.75	0.23208	-780.071039654007\\
71.75	0.23574	-811.381467474501\\
71.75	0.2394	-843.357143324766\\
71.75	0.24306	-875.9980672048\\
71.75	0.24672	-909.304239114606\\
71.75	0.25038	-943.275659054183\\
71.75	0.25404	-977.91232702353\\
71.75	0.2577	-1013.21424302265\\
71.75	0.26136	-1049.18140705154\\
71.75	0.26502	-1085.8138191102\\
71.75	0.26868	-1123.11147919863\\
71.75	0.27234	-1161.07438731683\\
71.75	0.276	-1199.7025434648\\
72.125	0.093	-84.2850601373702\\
72.125	0.09666	-90.3559302079349\\
72.125	0.10032	-97.0920483082706\\
72.125	0.10398	-104.493414438377\\
72.125	0.10764	-112.560028598254\\
72.125	0.1113	-121.291890787901\\
72.125	0.11496	-130.68900100732\\
72.125	0.11862	-140.751359256509\\
72.125	0.12228	-151.478965535468\\
72.125	0.12594	-162.871819844199\\
72.125	0.1296	-174.9299221827\\
72.125	0.13326	-187.653272550972\\
72.125	0.13692	-201.041870949014\\
72.125	0.14058	-215.095717376828\\
72.125	0.14424	-229.814811834412\\
72.125	0.1479	-245.199154321766\\
72.125	0.15156	-261.248744838892\\
72.125	0.15522	-277.963583385788\\
72.125	0.15888	-295.343669962455\\
72.125	0.16254	-313.389004568892\\
72.125	0.1662	-332.0995872051\\
72.125	0.16986	-351.475417871079\\
72.125	0.17352	-371.516496566829\\
72.125	0.17718	-392.222823292349\\
72.125	0.18084	-413.59439804764\\
72.125	0.1845	-435.631220832701\\
72.125	0.18816	-458.333291647534\\
72.125	0.19182	-481.700610492137\\
72.125	0.19548	-505.733177366511\\
72.125	0.19914	-530.430992270655\\
72.125	0.2028	-555.794055204571\\
72.125	0.20646	-581.822366168257\\
72.125	0.21012	-608.515925161713\\
72.125	0.21378	-635.874732184941\\
72.125	0.21744	-663.898787237939\\
72.125	0.2211	-692.588090320707\\
72.125	0.22476	-721.942641433246\\
72.125	0.22842	-751.962440575557\\
72.125	0.23208	-782.647487747637\\
72.125	0.23574	-813.997782949489\\
72.125	0.2394	-846.013326181111\\
72.125	0.24306	-878.694117442504\\
72.125	0.24672	-912.040156733667\\
72.125	0.25038	-946.051444054602\\
72.125	0.25404	-980.727979405307\\
72.125	0.2577	-1016.06976278578\\
72.125	0.26136	-1052.07679419603\\
72.125	0.26502	-1088.74907363605\\
72.125	0.26868	-1126.08660110583\\
72.125	0.27234	-1164.08937660539\\
72.125	0.276	-1202.75740013472\\
72.5	0.093	-85.3800406240505\\
72.5	0.09666	-91.4907780759731\\
72.5	0.10032	-98.2667635576667\\
72.5	0.10398	-105.707997069131\\
72.5	0.10764	-113.814478610366\\
72.5	0.1113	-122.586208181371\\
72.5	0.11496	-132.023185782147\\
72.5	0.11862	-142.125411412694\\
72.5	0.12228	-152.892885073012\\
72.5	0.12594	-164.3256067631\\
72.5	0.1296	-176.42357648296\\
72.5	0.13326	-189.186794232589\\
72.5	0.13692	-202.61526001199\\
72.5	0.14058	-216.708973821161\\
72.5	0.14424	-231.467935660103\\
72.5	0.1479	-246.892145528815\\
72.5	0.15156	-262.981603427299\\
72.5	0.15522	-279.736309355553\\
72.5	0.15888	-297.156263313578\\
72.5	0.16254	-315.241465301373\\
72.5	0.1662	-333.991915318939\\
72.5	0.16986	-353.407613366276\\
72.5	0.17352	-373.488559443383\\
72.5	0.17718	-394.234753550261\\
72.5	0.18084	-415.64619568691\\
72.5	0.1845	-437.72288585333\\
72.5	0.18816	-460.46482404952\\
72.5	0.19182	-483.872010275481\\
72.5	0.19548	-507.944444531213\\
72.5	0.19914	-532.682126816715\\
72.5	0.2028	-558.085057131988\\
72.5	0.20646	-584.153235477032\\
72.5	0.21012	-610.886661851847\\
72.5	0.21378	-638.285336256432\\
72.5	0.21744	-666.349258690788\\
72.5	0.2211	-695.078429154914\\
72.5	0.22476	-724.472847648811\\
72.5	0.22842	-754.53251417248\\
72.5	0.23208	-785.257428725919\\
72.5	0.23574	-816.647591309128\\
72.5	0.2394	-848.703001922108\\
72.5	0.24306	-881.423660564859\\
72.5	0.24672	-914.80956723738\\
72.5	0.25038	-948.860721939673\\
72.5	0.25404	-983.577124671736\\
72.5	0.2577	-1018.95877543357\\
72.5	0.26136	-1055.00567422517\\
72.5	0.26502	-1091.71782104655\\
72.5	0.26868	-1129.09521589769\\
72.5	0.27234	-1167.13785877861\\
72.5	0.276	-1205.8457496893\\
72.875	0.093	-86.5085139953826\\
72.875	0.09666	-92.6591188286631\\
72.875	0.10032	-99.4749716917147\\
72.875	0.10398	-106.956072584537\\
72.875	0.10764	-115.10242150713\\
72.875	0.1113	-123.914018459493\\
72.875	0.11496	-133.390863441627\\
72.875	0.11862	-143.532956453532\\
72.875	0.12228	-154.340297495208\\
72.875	0.12594	-165.812886566654\\
72.875	0.1296	-177.950723667871\\
72.875	0.13326	-190.753808798858\\
72.875	0.13692	-204.222141959617\\
72.875	0.14058	-218.355723150146\\
72.875	0.14424	-233.154552370446\\
72.875	0.1479	-248.618629620516\\
72.875	0.15156	-264.747954900358\\
72.875	0.15522	-281.542528209969\\
72.875	0.15888	-299.002349549352\\
72.875	0.16254	-317.127418918505\\
72.875	0.1662	-335.91773631743\\
72.875	0.16986	-355.373301746124\\
72.875	0.17352	-375.494115204589\\
72.875	0.17718	-396.280176692825\\
72.875	0.18084	-417.731486210832\\
72.875	0.1845	-439.84804375861\\
72.875	0.18816	-462.629849336158\\
72.875	0.19182	-486.076902943477\\
72.875	0.19548	-510.189204580567\\
72.875	0.19914	-534.966754247427\\
72.875	0.2028	-560.409551944058\\
72.875	0.20646	-586.51759767046\\
72.875	0.21012	-613.290891426632\\
72.875	0.21378	-640.729433212575\\
72.875	0.21744	-668.833223028289\\
72.875	0.2211	-697.602260873774\\
72.875	0.22476	-727.036546749029\\
72.875	0.22842	-757.136080654055\\
72.875	0.23208	-787.900862588851\\
72.875	0.23574	-819.330892553419\\
72.875	0.2394	-851.426170547757\\
72.875	0.24306	-884.186696571866\\
72.875	0.24672	-917.612470625745\\
72.875	0.25038	-951.703492709396\\
72.875	0.25404	-986.459762822816\\
72.875	0.2577	-1021.88128096601\\
72.875	0.26136	-1057.96804713897\\
72.875	0.26502	-1094.7200613417\\
72.875	0.26868	-1132.13732357421\\
72.875	0.27234	-1170.21983383648\\
72.875	0.276	-1208.96759212853\\
73.25	0.093	-87.6704802513659\\
73.25	0.09666	-93.8609524660044\\
73.25	0.10032	-100.716672710414\\
73.25	0.10398	-108.237640984594\\
73.25	0.10764	-116.423857288545\\
73.25	0.1113	-125.275321622266\\
73.25	0.11496	-134.792033985758\\
73.25	0.11862	-144.973994379021\\
73.25	0.12228	-155.821202802054\\
73.25	0.12594	-167.333659254859\\
73.25	0.1296	-179.511363737433\\
73.25	0.13326	-192.354316249779\\
73.25	0.13692	-205.862516791895\\
73.25	0.14058	-220.035965363782\\
73.25	0.14424	-234.87466196544\\
73.25	0.1479	-250.378606596869\\
73.25	0.15156	-266.547799258068\\
73.25	0.15522	-283.382239949037\\
73.25	0.15888	-300.881928669778\\
73.25	0.16254	-319.046865420289\\
73.25	0.1662	-337.877050200571\\
73.25	0.16986	-357.372483010624\\
73.25	0.17352	-377.533163850447\\
73.25	0.17718	-398.359092720041\\
73.25	0.18084	-419.850269619406\\
73.25	0.1845	-442.006694548541\\
73.25	0.18816	-464.828367507447\\
73.25	0.19182	-488.315288496124\\
73.25	0.19548	-512.467457514572\\
73.25	0.19914	-537.28487456279\\
73.25	0.2028	-562.767539640779\\
73.25	0.20646	-588.915452748539\\
73.25	0.21012	-615.728613886069\\
73.25	0.21378	-643.20702305337\\
73.25	0.21744	-671.350680250442\\
73.25	0.2211	-700.159585477284\\
73.25	0.22476	-729.633738733897\\
73.25	0.22842	-759.773140020281\\
73.25	0.23208	-790.577789336436\\
73.25	0.23574	-822.047686682361\\
73.25	0.2394	-854.182832058057\\
73.25	0.24306	-886.983225463524\\
73.25	0.24672	-920.448866898761\\
73.25	0.25038	-954.579756363769\\
73.25	0.25404	-989.375893858548\\
73.25	0.2577	-1024.8372793831\\
73.25	0.26136	-1060.96391293742\\
73.25	0.26502	-1097.75579452151\\
73.25	0.26868	-1135.21292413537\\
73.25	0.27234	-1173.335301779\\
73.25	0.276	-1212.1229274524\\
73.625	0.093	-88.8659393920012\\
73.625	0.09666	-95.0962789879976\\
73.625	0.10032	-101.991866613765\\
73.625	0.10398	-109.552702269303\\
73.625	0.10764	-117.778785954611\\
73.625	0.1113	-126.670117669691\\
73.625	0.11496	-136.226697414541\\
73.625	0.11862	-146.448525189162\\
73.625	0.12228	-157.335600993553\\
73.625	0.12594	-168.887924827715\\
73.625	0.1296	-181.105496691648\\
73.625	0.13326	-193.988316585351\\
73.625	0.13692	-207.536384508825\\
73.625	0.14058	-221.749700462071\\
73.625	0.14424	-236.628264445086\\
73.625	0.1479	-252.172076457872\\
73.625	0.15156	-268.381136500429\\
73.625	0.15522	-285.255444572757\\
73.625	0.15888	-302.795000674856\\
73.625	0.16254	-320.999804806725\\
73.625	0.1662	-339.869856968365\\
73.625	0.16986	-359.405157159775\\
73.625	0.17352	-379.605705380956\\
73.625	0.17718	-400.471501631908\\
73.625	0.18084	-422.002545912631\\
73.625	0.1845	-444.198838223124\\
73.625	0.18816	-467.060378563388\\
73.625	0.19182	-490.587166933423\\
73.625	0.19548	-514.779203333229\\
73.625	0.19914	-539.636487762805\\
73.625	0.2028	-565.159020222152\\
73.625	0.20646	-591.346800711269\\
73.625	0.21012	-618.199829230158\\
73.625	0.21378	-645.718105778817\\
73.625	0.21744	-673.901630357246\\
73.625	0.2211	-702.750402965446\\
73.625	0.22476	-732.264423603417\\
73.625	0.22842	-762.44369227116\\
73.625	0.23208	-793.288208968672\\
73.625	0.23574	-824.797973695956\\
73.625	0.2394	-856.972986453009\\
73.625	0.24306	-889.813247239834\\
73.625	0.24672	-923.318756056429\\
73.625	0.25038	-957.489512902795\\
73.625	0.25404	-992.325517778932\\
73.625	0.2577	-1027.82677068484\\
73.625	0.26136	-1063.99327162052\\
73.625	0.26502	-1100.82502058597\\
73.625	0.26868	-1138.32201758119\\
73.625	0.27234	-1176.48426260618\\
73.625	0.276	-1215.31175566094\\
74	0.093	-90.0948914172878\\
74	0.09666	-96.3650983946422\\
74	0.10032	-103.300553401767\\
74	0.10398	-110.901256438663\\
74	0.10764	-119.16720750533\\
74	0.1113	-128.098406601767\\
74	0.11496	-137.694853727975\\
74	0.11862	-147.956548883954\\
74	0.12228	-158.883492069703\\
74	0.12594	-170.475683285223\\
74	0.1296	-182.733122530514\\
74	0.13326	-195.655809805575\\
74	0.13692	-209.243745110407\\
74	0.14058	-223.49692844501\\
74	0.14424	-238.415359809384\\
74	0.1479	-253.999039203528\\
74	0.15156	-270.247966627443\\
74	0.15522	-287.162142081129\\
74	0.15888	-304.741565564585\\
74	0.16254	-322.986237077812\\
74	0.1662	-341.89615662081\\
74	0.16986	-361.471324193578\\
74	0.17352	-381.711739796117\\
74	0.17718	-402.617403428427\\
74	0.18084	-424.188315090508\\
74	0.1845	-446.424474782359\\
74	0.18816	-469.325882503981\\
74	0.19182	-492.892538255374\\
74	0.19548	-517.124442036537\\
74	0.19914	-542.021593847471\\
74	0.2028	-567.583993688176\\
74	0.20646	-593.811641558651\\
74	0.21012	-620.704537458897\\
74	0.21378	-648.262681388915\\
74	0.21744	-676.486073348702\\
74	0.2211	-705.37471333826\\
74	0.22476	-734.928601357589\\
74	0.22842	-765.147737406689\\
74	0.23208	-796.032121485559\\
74	0.23574	-827.581753594201\\
74	0.2394	-859.796633732612\\
74	0.24306	-892.676761900795\\
74	0.24672	-926.222138098748\\
74	0.25038	-960.432762326472\\
74	0.25404	-995.308634583967\\
74	0.2577	-1030.84975487123\\
74	0.26136	-1067.05612318827\\
74	0.26502	-1103.92773953507\\
74	0.26868	-1141.46460391165\\
74	0.27234	-1179.666716318\\
74	0.276	-1218.53407675412\\
};
\end{axis}

\begin{axis}[%
width=4.527496cm,
height=3.050847cm,
at={(0cm,12.711864cm)},
scale only axis,
xmin=56,
xmax=74,
tick align=outside,
xlabel={$L_{cut}$},
xmajorgrids,
ymin=0.093,
ymax=0.276,
ylabel={$D_{rlx}$},
ymajorgrids,
zmin=0,
zmax=40,
zlabel={$x_1,x_4$},
zmajorgrids,
view={-140}{50},
legend style={at={(1.03,1)},anchor=north west,legend cell align=left,align=left,draw=white!15!black}
]
\addplot3[only marks,mark=*,mark options={},mark size=1.5000pt,color=mycolor1] plot table[row sep=crcr,]{%
74	0.123	4.29401199568982\\
72	0.113	3.45007747285662\\
61	0.095	2.0323093972585\\
56	0.093	3.10665077277973\\
};
\addplot3[only marks,mark=*,mark options={},mark size=1.5000pt,color=mycolor2] plot table[row sep=crcr,]{%
67	0.276	34.3942962701507\\
66	0.255	27.102614273087\\
62	0.209	14.4450769170106\\
57	0.193	10.2033323884491\\
};
\addplot3[only marks,mark=*,mark options={},mark size=1.5000pt,color=black] plot table[row sep=crcr,]{%
69	0.104	2.31762973269431\\
};
\addplot3[only marks,mark=*,mark options={},mark size=1.5000pt,color=black] plot table[row sep=crcr,]{%
64	0.23	19.6472254900045\\
};

\addplot3[%
surf,
opacity=0.7,
shader=interp,
colormap={mymap}{[1pt] rgb(0pt)=(0.0901961,0.239216,0.0745098); rgb(1pt)=(0.0945149,0.242058,0.0739522); rgb(2pt)=(0.0988592,0.244894,0.0733566); rgb(3pt)=(0.103229,0.247724,0.0727241); rgb(4pt)=(0.107623,0.250549,0.0720557); rgb(5pt)=(0.112043,0.253367,0.0713525); rgb(6pt)=(0.116487,0.25618,0.0706154); rgb(7pt)=(0.120956,0.258986,0.0698456); rgb(8pt)=(0.125449,0.261787,0.0690441); rgb(9pt)=(0.129967,0.264581,0.0682118); rgb(10pt)=(0.134508,0.26737,0.06735); rgb(11pt)=(0.139074,0.270152,0.0664596); rgb(12pt)=(0.143663,0.272929,0.0655416); rgb(13pt)=(0.148275,0.275699,0.0645971); rgb(14pt)=(0.152911,0.278463,0.0636271); rgb(15pt)=(0.15757,0.281221,0.0626328); rgb(16pt)=(0.162252,0.283973,0.0616151); rgb(17pt)=(0.166957,0.286719,0.060575); rgb(18pt)=(0.171685,0.289458,0.0595136); rgb(19pt)=(0.176434,0.292191,0.0584321); rgb(20pt)=(0.181207,0.294918,0.0573313); rgb(21pt)=(0.186001,0.297639,0.0562123); rgb(22pt)=(0.190817,0.300353,0.0550763); rgb(23pt)=(0.195655,0.303061,0.0539242); rgb(24pt)=(0.200514,0.305763,0.052757); rgb(25pt)=(0.205395,0.308459,0.0515759); rgb(26pt)=(0.210296,0.311149,0.0503624); rgb(27pt)=(0.215212,0.313846,0.0490067); rgb(28pt)=(0.220142,0.316548,0.0475043); rgb(29pt)=(0.22509,0.319254,0.0458704); rgb(30pt)=(0.230056,0.321962,0.0441205); rgb(31pt)=(0.235042,0.324671,0.04227); rgb(32pt)=(0.240048,0.327379,0.0403343); rgb(33pt)=(0.245078,0.330085,0.0383287); rgb(34pt)=(0.250131,0.332786,0.0362688); rgb(35pt)=(0.25521,0.335482,0.0341698); rgb(36pt)=(0.260317,0.33817,0.0320472); rgb(37pt)=(0.265451,0.340849,0.0299163); rgb(38pt)=(0.270616,0.343517,0.0277927); rgb(39pt)=(0.275813,0.346172,0.0256916); rgb(40pt)=(0.281043,0.348814,0.0236284); rgb(41pt)=(0.286307,0.35144,0.0216186); rgb(42pt)=(0.291607,0.354048,0.0196776); rgb(43pt)=(0.296945,0.356637,0.0178207); rgb(44pt)=(0.302322,0.359206,0.0160634); rgb(45pt)=(0.307739,0.361753,0.0144211); rgb(46pt)=(0.313198,0.364275,0.0129091); rgb(47pt)=(0.318701,0.366772,0.0115428); rgb(48pt)=(0.324249,0.369242,0.0103377); rgb(49pt)=(0.329843,0.371682,0.00930909); rgb(50pt)=(0.335485,0.374093,0.00847245); rgb(51pt)=(0.341176,0.376471,0.00784314); rgb(52pt)=(0.346925,0.378826,0.00732741); rgb(53pt)=(0.352735,0.381168,0.00682184); rgb(54pt)=(0.358605,0.383497,0.00632729); rgb(55pt)=(0.364532,0.385812,0.00584464); rgb(56pt)=(0.370516,0.388113,0.00537476); rgb(57pt)=(0.376552,0.390399,0.00491852); rgb(58pt)=(0.38264,0.39267,0.00447681); rgb(59pt)=(0.388777,0.394925,0.00405048); rgb(60pt)=(0.394962,0.397164,0.00364042); rgb(61pt)=(0.401191,0.399386,0.00324749); rgb(62pt)=(0.407464,0.401592,0.00287258); rgb(63pt)=(0.413777,0.40378,0.00251655); rgb(64pt)=(0.420129,0.40595,0.00218028); rgb(65pt)=(0.426518,0.408102,0.00186463); rgb(66pt)=(0.432942,0.410234,0.00157049); rgb(67pt)=(0.439399,0.412348,0.00129873); rgb(68pt)=(0.445885,0.414441,0.00105022); rgb(69pt)=(0.452401,0.416515,0.000825833); rgb(70pt)=(0.458942,0.418567,0.000626441); rgb(71pt)=(0.465508,0.420599,0.00045292); rgb(72pt)=(0.472096,0.422609,0.000306141); rgb(73pt)=(0.478704,0.424596,0.000186979); rgb(74pt)=(0.485331,0.426562,9.63073e-05); rgb(75pt)=(0.491973,0.428504,3.49981e-05); rgb(76pt)=(0.498628,0.430422,3.92506e-06); rgb(77pt)=(0.505323,0.432315,0); rgb(78pt)=(0.512206,0.434168,0); rgb(79pt)=(0.519282,0.435983,0); rgb(80pt)=(0.526529,0.437764,0); rgb(81pt)=(0.533922,0.439512,0); rgb(82pt)=(0.54144,0.441232,0); rgb(83pt)=(0.549059,0.442927,0); rgb(84pt)=(0.556756,0.444599,0); rgb(85pt)=(0.564508,0.446252,0); rgb(86pt)=(0.572292,0.447889,0); rgb(87pt)=(0.580084,0.449514,0); rgb(88pt)=(0.587863,0.451129,0); rgb(89pt)=(0.595604,0.452737,0); rgb(90pt)=(0.603284,0.454343,0); rgb(91pt)=(0.610882,0.455948,0); rgb(92pt)=(0.618373,0.457556,0); rgb(93pt)=(0.625734,0.459171,0); rgb(94pt)=(0.632943,0.460795,0); rgb(95pt)=(0.639976,0.462432,0); rgb(96pt)=(0.64681,0.464084,0); rgb(97pt)=(0.653423,0.465756,0); rgb(98pt)=(0.659791,0.46745,0); rgb(99pt)=(0.665891,0.469169,0); rgb(100pt)=(0.6717,0.470916,0); rgb(101pt)=(0.677195,0.472696,0); rgb(102pt)=(0.682353,0.47451,0); rgb(103pt)=(0.687242,0.476355,0); rgb(104pt)=(0.691952,0.478225,0); rgb(105pt)=(0.696497,0.480118,0); rgb(106pt)=(0.700887,0.482033,0); rgb(107pt)=(0.705134,0.483968,0); rgb(108pt)=(0.709251,0.485921,0); rgb(109pt)=(0.713249,0.487891,0); rgb(110pt)=(0.71714,0.489876,0); rgb(111pt)=(0.720936,0.491875,0); rgb(112pt)=(0.724649,0.493887,0); rgb(113pt)=(0.72829,0.495909,0); rgb(114pt)=(0.731872,0.49794,0); rgb(115pt)=(0.735406,0.499979,0); rgb(116pt)=(0.738904,0.502025,0); rgb(117pt)=(0.742378,0.504075,0); rgb(118pt)=(0.74584,0.506128,0); rgb(119pt)=(0.749302,0.508182,0); rgb(120pt)=(0.752775,0.510237,0); rgb(121pt)=(0.756272,0.51229,0); rgb(122pt)=(0.759804,0.514339,0); rgb(123pt)=(0.763384,0.516385,0); rgb(124pt)=(0.767022,0.518424,0); rgb(125pt)=(0.770731,0.520455,0); rgb(126pt)=(0.774523,0.522478,0); rgb(127pt)=(0.77841,0.524489,0); rgb(128pt)=(0.782391,0.526491,0); rgb(129pt)=(0.786402,0.528496,0); rgb(130pt)=(0.790431,0.530506,0); rgb(131pt)=(0.794478,0.532521,0); rgb(132pt)=(0.798541,0.534539,0); rgb(133pt)=(0.802619,0.53656,0); rgb(134pt)=(0.806712,0.538584,0); rgb(135pt)=(0.81082,0.540609,0); rgb(136pt)=(0.81494,0.542635,0); rgb(137pt)=(0.819074,0.54466,0); rgb(138pt)=(0.823219,0.546686,0); rgb(139pt)=(0.827374,0.548709,0); rgb(140pt)=(0.831541,0.55073,0); rgb(141pt)=(0.835716,0.552749,0); rgb(142pt)=(0.8399,0.554763,0); rgb(143pt)=(0.844092,0.556774,0); rgb(144pt)=(0.848292,0.558779,0); rgb(145pt)=(0.852497,0.560778,0); rgb(146pt)=(0.856708,0.562771,0); rgb(147pt)=(0.860924,0.564756,0); rgb(148pt)=(0.865143,0.566733,0); rgb(149pt)=(0.869366,0.568701,0); rgb(150pt)=(0.873592,0.57066,0); rgb(151pt)=(0.877819,0.572608,0); rgb(152pt)=(0.882047,0.574545,0); rgb(153pt)=(0.886275,0.576471,0); rgb(154pt)=(0.890659,0.578362,0); rgb(155pt)=(0.895333,0.580203,0); rgb(156pt)=(0.900258,0.581999,0); rgb(157pt)=(0.905397,0.583755,0); rgb(158pt)=(0.910711,0.585479,0); rgb(159pt)=(0.916164,0.587176,0); rgb(160pt)=(0.921717,0.588852,0); rgb(161pt)=(0.927333,0.590513,0); rgb(162pt)=(0.932974,0.592166,0); rgb(163pt)=(0.938602,0.593815,0); rgb(164pt)=(0.94418,0.595468,0); rgb(165pt)=(0.949669,0.59713,0); rgb(166pt)=(0.955033,0.598808,0); rgb(167pt)=(0.960233,0.600507,0); rgb(168pt)=(0.965232,0.602233,0); rgb(169pt)=(0.969992,0.603992,0); rgb(170pt)=(0.974475,0.605791,0); rgb(171pt)=(0.978643,0.607636,0); rgb(172pt)=(0.98246,0.609532,0); rgb(173pt)=(0.985886,0.611486,0); rgb(174pt)=(0.988885,0.613503,0); rgb(175pt)=(0.991419,0.61559,0); rgb(176pt)=(0.99345,0.617753,0); rgb(177pt)=(0.99494,0.619997,0); rgb(178pt)=(0.995851,0.622329,0); rgb(179pt)=(0.996226,0.624763,0); rgb(180pt)=(0.996512,0.627352,0); rgb(181pt)=(0.996788,0.630095,0); rgb(182pt)=(0.997053,0.632982,0); rgb(183pt)=(0.997308,0.636004,0); rgb(184pt)=(0.997552,0.639152,0); rgb(185pt)=(0.997785,0.642416,0); rgb(186pt)=(0.998006,0.645786,0); rgb(187pt)=(0.998217,0.649253,0); rgb(188pt)=(0.998416,0.652807,0); rgb(189pt)=(0.998605,0.656439,0); rgb(190pt)=(0.998781,0.660138,0); rgb(191pt)=(0.998946,0.663897,0); rgb(192pt)=(0.9991,0.667704,0); rgb(193pt)=(0.999242,0.67155,0); rgb(194pt)=(0.999372,0.675427,0); rgb(195pt)=(0.99949,0.679323,0); rgb(196pt)=(0.999596,0.68323,0); rgb(197pt)=(0.99969,0.687139,0); rgb(198pt)=(0.999771,0.691039,0); rgb(199pt)=(0.999841,0.694921,0); rgb(200pt)=(0.999898,0.698775,0); rgb(201pt)=(0.999942,0.702592,0); rgb(202pt)=(0.999974,0.706363,0); rgb(203pt)=(0.999994,0.710077,0); rgb(204pt)=(1,0.713725,0); rgb(205pt)=(1,0.717341,0); rgb(206pt)=(1,0.720963,0); rgb(207pt)=(1,0.724591,0); rgb(208pt)=(1,0.728226,0); rgb(209pt)=(1,0.731867,0); rgb(210pt)=(1,0.735514,0); rgb(211pt)=(1,0.739167,0); rgb(212pt)=(1,0.742827,0); rgb(213pt)=(1,0.746493,0); rgb(214pt)=(1,0.750165,0); rgb(215pt)=(1,0.753843,0); rgb(216pt)=(1,0.757527,0); rgb(217pt)=(1,0.761217,0); rgb(218pt)=(1,0.764913,0); rgb(219pt)=(1,0.768615,0); rgb(220pt)=(1,0.772324,0); rgb(221pt)=(1,0.776038,0); rgb(222pt)=(1,0.779758,0); rgb(223pt)=(1,0.783484,0); rgb(224pt)=(1,0.787215,0); rgb(225pt)=(1,0.790953,0); rgb(226pt)=(1,0.794696,0); rgb(227pt)=(1,0.798445,0); rgb(228pt)=(1,0.8022,0); rgb(229pt)=(1,0.805961,0); rgb(230pt)=(1,0.809727,0); rgb(231pt)=(1,0.8135,0); rgb(232pt)=(1,0.817278,0); rgb(233pt)=(1,0.821063,0); rgb(234pt)=(1,0.824854,0); rgb(235pt)=(1,0.828652,0); rgb(236pt)=(1,0.832455,0); rgb(237pt)=(1,0.836265,0); rgb(238pt)=(1,0.840081,0); rgb(239pt)=(1,0.843903,0); rgb(240pt)=(1,0.847732,0); rgb(241pt)=(1,0.851566,0); rgb(242pt)=(1,0.855406,0); rgb(243pt)=(1,0.859253,0); rgb(244pt)=(1,0.863106,0); rgb(245pt)=(1,0.866964,0); rgb(246pt)=(1,0.870829,0); rgb(247pt)=(1,0.8747,0); rgb(248pt)=(1,0.878577,0); rgb(249pt)=(1,0.88246,0); rgb(250pt)=(1,0.886349,0); rgb(251pt)=(1,0.890243,0); rgb(252pt)=(1,0.894144,0); rgb(253pt)=(1,0.898051,0); rgb(254pt)=(1,0.901964,0); rgb(255pt)=(1,0.905882,0)},
mesh/rows=49]
table[row sep=crcr,header=false] {%
%
56	0.093	2.99683920796662\\
56	0.09666	2.96677134878796\\
56	0.10032	2.9588443141927\\
56	0.10398	2.97305810418084\\
56	0.10764	3.00941271875238\\
56	0.1113	3.06790815790732\\
56	0.11496	3.14854442164567\\
56	0.11862	3.2513215099674\\
56	0.12228	3.37623942287254\\
56	0.12594	3.52329816036108\\
56	0.1296	3.69249772243302\\
56	0.13326	3.88383810908835\\
56	0.13692	4.09731932032709\\
56	0.14058	4.33294135614922\\
56	0.14424	4.59070421655476\\
56	0.1479	4.87060790154369\\
56	0.15156	5.17265241111603\\
56	0.15522	5.49683774527177\\
56	0.15888	5.8431639040109\\
56	0.16254	6.21163088733343\\
56	0.1662	6.60223869523936\\
56	0.16986	7.0149873277287\\
56	0.17352	7.44987678480144\\
56	0.17718	7.90690706645757\\
56	0.18084	8.38607817269708\\
56	0.1845	8.88739010352003\\
56	0.18816	9.41084285892635\\
56	0.19182	9.95643643891609\\
56	0.19548	10.5241708434892\\
56	0.19914	11.1140460726457\\
56	0.2028	11.7260621263857\\
56	0.20646	12.360219004709\\
56	0.21012	13.0165167076157\\
56	0.21378	13.6949552351058\\
56	0.21744	14.3955345871794\\
56	0.2211	15.1182547638363\\
56	0.22476	15.8631157650766\\
56	0.22842	16.6301175909003\\
56	0.23208	17.4192602413075\\
56	0.23574	18.230543716298\\
56	0.2394	19.0639680158719\\
56	0.24306	19.9195331400292\\
56	0.24672	20.7972390887699\\
56	0.25038	21.6970858620941\\
56	0.25404	22.6190734600016\\
56	0.2577	23.5632018824925\\
56	0.26136	24.5294711295668\\
56	0.26502	25.5178812012246\\
56	0.26868	26.5284320974657\\
56	0.27234	27.5611238182902\\
56	0.276	28.6159563636981\\
56.375	0.093	2.92159372196217\\
56.375	0.09666	2.89620927640069\\
56.375	0.10032	2.89296565542262\\
56.375	0.10398	2.91186285902794\\
56.375	0.10764	2.95290088721666\\
56.375	0.1113	3.01607973998878\\
56.375	0.11496	3.10139941734431\\
56.375	0.11862	3.20885991928322\\
56.375	0.12228	3.33846124580554\\
56.375	0.12594	3.49020339691127\\
56.375	0.1296	3.66408637260038\\
56.375	0.13326	3.8601101728729\\
56.375	0.13692	4.07827479772881\\
56.375	0.14058	4.31858024716814\\
56.375	0.14424	4.58102652119085\\
56.375	0.1479	4.86561361979696\\
56.375	0.15156	5.17234154298648\\
56.375	0.15522	5.5012102907594\\
56.375	0.15888	5.85221986311572\\
56.375	0.16254	6.22537026005542\\
56.375	0.1662	6.62066148157854\\
56.375	0.16986	7.03809352768505\\
56.375	0.17352	7.47766639837497\\
56.375	0.17718	7.93938009364828\\
56.375	0.18084	8.42323461350498\\
56.375	0.1845	8.9292299579451\\
56.375	0.18816	9.45736612696862\\
56.375	0.19182	10.0076431205755\\
56.375	0.19548	10.5800609387658\\
56.375	0.19914	11.1746195815395\\
56.375	0.2028	11.7913190488966\\
56.375	0.20646	12.4301593408372\\
56.375	0.21012	13.0911404573611\\
56.375	0.21378	13.7742623984684\\
56.375	0.21744	14.4795251641591\\
56.375	0.2211	15.2069287544332\\
56.375	0.22476	15.9564731692907\\
56.375	0.22842	16.7281584087316\\
56.375	0.23208	17.5219844727559\\
56.375	0.23574	18.3379513613636\\
56.375	0.2394	19.1760590745547\\
56.375	0.24306	20.0363076123292\\
56.375	0.24672	20.9186969746871\\
56.375	0.25038	21.8232271616284\\
56.375	0.25404	22.7498981731531\\
56.375	0.2577	23.6987100092612\\
56.375	0.26136	24.6696626699527\\
56.375	0.26502	25.6627561552276\\
56.375	0.26868	26.6779904650859\\
56.375	0.27234	27.7153655995276\\
56.375	0.276	28.7748815585527\\
56.75	0.093	2.84885867576114\\
56.75	0.09666	2.82815764381684\\
56.75	0.10032	2.82959743645596\\
56.75	0.10398	2.85317805367845\\
56.75	0.10764	2.89889949548435\\
56.75	0.1113	2.96676176187366\\
56.75	0.11496	3.05676485284637\\
56.75	0.11862	3.16890876840246\\
56.75	0.12228	3.30319350854195\\
56.75	0.12594	3.45961907326486\\
56.75	0.1296	3.63818546257115\\
56.75	0.13326	3.83889267646085\\
56.75	0.13692	4.06174071493396\\
56.75	0.14058	4.30672957799045\\
56.75	0.14424	4.57385926563036\\
56.75	0.1479	4.86312977785364\\
56.75	0.15156	5.17454111466034\\
56.75	0.15522	5.50809327605044\\
56.75	0.15888	5.86378626202394\\
56.75	0.16254	6.24162007258082\\
56.75	0.1662	6.64159470772112\\
56.75	0.16986	7.06371016744482\\
56.75	0.17352	7.50796645175191\\
56.75	0.17718	7.97436356064241\\
56.75	0.18084	8.46290149411629\\
56.75	0.1845	8.97358025217359\\
56.75	0.18816	9.50639983481428\\
56.75	0.19182	10.0613602420384\\
56.75	0.19548	10.6384614738459\\
56.75	0.19914	11.2377035302367\\
56.75	0.2028	11.859086411211\\
56.75	0.20646	12.5026101167687\\
56.75	0.21012	13.1682746469098\\
56.75	0.21378	13.8560800016343\\
56.75	0.21744	14.5660261809422\\
56.75	0.2211	15.2981131848335\\
56.75	0.22476	16.0523410133082\\
56.75	0.22842	16.8287096663662\\
56.75	0.23208	17.6272191440077\\
56.75	0.23574	18.4478694462326\\
56.75	0.2394	19.2906605730409\\
56.75	0.24306	20.1555925244326\\
56.75	0.24672	21.0426653004077\\
56.75	0.25038	21.9518789009662\\
56.75	0.25404	22.883233326108\\
56.75	0.2577	23.8367285758333\\
56.75	0.26136	24.812364650142\\
56.75	0.26502	25.8101415490341\\
56.75	0.26868	26.8300592725095\\
56.75	0.27234	27.8721178205684\\
56.75	0.276	28.9363171932107\\
57.125	0.093	2.7786340693635\\
57.125	0.09666	2.76261645103639\\
57.125	0.10032	2.76873965729267\\
57.125	0.10398	2.79700368813235\\
57.125	0.10764	2.84740854355544\\
57.125	0.1113	2.91995422356192\\
57.125	0.11496	3.01464072815181\\
57.125	0.11862	3.13146805732509\\
57.125	0.12228	3.27043621108177\\
57.125	0.12594	3.43154518942185\\
57.125	0.1296	3.61479499234533\\
57.125	0.13326	3.82018561985222\\
57.125	0.13692	4.04771707194249\\
57.125	0.14058	4.29738934861617\\
57.125	0.14424	4.56920244987325\\
57.125	0.1479	4.86315637571374\\
57.125	0.15156	5.1792511261376\\
57.125	0.15522	5.51748670114488\\
57.125	0.15888	5.87786310073555\\
57.125	0.16254	6.26038032490964\\
57.125	0.1662	6.66503837366711\\
57.125	0.16986	7.09183724700798\\
57.125	0.17352	7.54077694493224\\
57.125	0.17718	8.01185746743992\\
57.125	0.18084	8.505078814531\\
57.125	0.1845	9.02044098620547\\
57.125	0.18816	9.55794398246335\\
57.125	0.19182	10.1175878033046\\
57.125	0.19548	10.6993724487293\\
57.125	0.19914	11.3032979187373\\
57.125	0.2028	11.9293642133288\\
57.125	0.20646	12.5775713325037\\
57.125	0.21012	13.247919276262\\
57.125	0.21378	13.9404080446036\\
57.125	0.21744	14.6550376375287\\
57.125	0.2211	15.3918080550372\\
57.125	0.22476	16.150719297129\\
57.125	0.22842	16.9317713638043\\
57.125	0.23208	17.734964255063\\
57.125	0.23574	18.560297970905\\
57.125	0.2394	19.4077725113305\\
57.125	0.24306	20.2773878763394\\
57.125	0.24672	21.1691440659316\\
57.125	0.25038	22.0830410801073\\
57.125	0.25404	23.0190789188663\\
57.125	0.2577	23.9772575822088\\
57.125	0.26136	24.9575770701347\\
57.125	0.26502	25.9600373826439\\
57.125	0.26868	26.9846385197366\\
57.125	0.27234	28.0313804814126\\
57.125	0.276	29.1002632676721\\
57.5	0.093	2.7109199027693\\
57.5	0.09666	2.69958569805936\\
57.5	0.10032	2.71039231793283\\
57.5	0.10398	2.74333976238969\\
57.5	0.10764	2.79842803142995\\
57.5	0.1113	2.87565712505362\\
57.5	0.11496	2.97502704326069\\
57.5	0.11862	3.09653778605114\\
57.5	0.12228	3.24018935342501\\
57.5	0.12594	3.40598174538228\\
57.5	0.1296	3.59391496192294\\
57.5	0.13326	3.80398900304701\\
57.5	0.13692	4.03620386875446\\
57.5	0.14058	4.29055955904531\\
57.5	0.14424	4.56705607391957\\
57.5	0.1479	4.86569341337724\\
57.5	0.15156	5.18647157741828\\
57.5	0.15522	5.52939056604275\\
57.5	0.15888	5.89445037925061\\
57.5	0.16254	6.28165101704187\\
57.5	0.1662	6.69099247941653\\
57.5	0.16986	7.12247476637457\\
57.5	0.17352	7.57609787791602\\
57.5	0.17718	8.05186181404087\\
57.5	0.18084	8.54976657474913\\
57.5	0.1845	9.06981216004079\\
57.5	0.18816	9.61199856991584\\
57.5	0.19182	10.1763258043743\\
57.5	0.19548	10.7627938634161\\
57.5	0.19914	11.3714027470414\\
57.5	0.2028	12.0021524552501\\
57.5	0.20646	12.6550429880421\\
57.5	0.21012	13.3300743454175\\
57.5	0.21378	14.0272465273764\\
57.5	0.21744	14.7465595339186\\
57.5	0.2211	15.4880133650443\\
57.5	0.22476	16.2516080207533\\
57.5	0.22842	17.0373435010458\\
57.5	0.23208	17.8452198059216\\
57.5	0.23574	18.6752369353809\\
57.5	0.2394	19.5273948894235\\
57.5	0.24306	20.4016936680496\\
57.5	0.24672	21.298133271259\\
57.5	0.25038	22.2167136990519\\
57.5	0.25404	23.1574349514281\\
57.5	0.2577	24.1202970283877\\
57.5	0.26136	25.1052999299308\\
57.5	0.26502	26.1124436560572\\
57.5	0.26868	27.1417282067671\\
57.5	0.27234	28.1931535820603\\
57.5	0.276	29.2667197819369\\
57.875	0.093	2.6457161759785\\
57.875	0.09666	2.63906538488575\\
57.875	0.10032	2.6545554183764\\
57.875	0.10398	2.69218627645044\\
57.875	0.10764	2.75195795910789\\
57.875	0.1113	2.83387046634873\\
57.875	0.11496	2.93792379817298\\
57.875	0.11862	3.06411795458062\\
57.875	0.12228	3.21245293557166\\
57.875	0.12594	3.38292874114611\\
57.875	0.1296	3.57554537130395\\
57.875	0.13326	3.79030282604519\\
57.875	0.13692	4.02720110536982\\
57.875	0.14058	4.28624020927786\\
57.875	0.14424	4.56742013776931\\
57.875	0.1479	4.87074089084415\\
57.875	0.15156	5.19620246850238\\
57.875	0.15522	5.54380487074403\\
57.875	0.15888	5.91354809756907\\
57.875	0.16254	6.30543214897751\\
57.875	0.1662	6.71945702496934\\
57.875	0.16986	7.15562272554457\\
57.875	0.17352	7.6139292507032\\
57.875	0.17718	8.09437660044524\\
57.875	0.18084	8.59696477477067\\
57.875	0.1845	9.12169377367951\\
57.875	0.18816	9.66856359717175\\
57.875	0.19182	10.2375742452474\\
57.875	0.19548	10.8287257179064\\
57.875	0.19914	11.4420180151488\\
57.875	0.2028	12.0774511369747\\
57.875	0.20646	12.7350250833839\\
57.875	0.21012	13.4147398543765\\
57.875	0.21378	14.1165954499526\\
57.875	0.21744	14.840591870112\\
57.875	0.2211	15.5867291148548\\
57.875	0.22476	16.3550071841811\\
57.875	0.22842	17.1454260780907\\
57.875	0.23208	17.9579857965837\\
57.875	0.23574	18.7926863396601\\
57.875	0.2394	19.64952770732\\
57.875	0.24306	20.5285098995632\\
57.875	0.24672	21.4296329163898\\
57.875	0.25038	22.3528967577998\\
57.875	0.25404	23.2983014237932\\
57.875	0.2577	24.2658469143701\\
57.875	0.26136	25.2555332295303\\
57.875	0.26502	26.2673603692739\\
57.875	0.26868	27.3013283336009\\
57.875	0.27234	28.3574371225114\\
57.875	0.276	29.4356867360052\\
58.25	0.093	2.58302288899111\\
58.25	0.09666	2.58105551151553\\
58.25	0.10032	2.60122895862337\\
58.25	0.10398	2.64354323031459\\
58.25	0.10764	2.70799832658921\\
58.25	0.1113	2.79459424744724\\
58.25	0.11496	2.90333099288867\\
58.25	0.11862	3.03420856291349\\
58.25	0.12228	3.18722695752171\\
58.25	0.12594	3.36238617671333\\
58.25	0.1296	3.55968622048835\\
58.25	0.13326	3.77912708884678\\
58.25	0.13692	4.0207087817886\\
58.25	0.14058	4.28443129931382\\
58.25	0.14424	4.57029464142245\\
58.25	0.1479	4.87829880811446\\
58.25	0.15156	5.20844379938988\\
58.25	0.15522	5.5607296152487\\
58.25	0.15888	5.93515625569093\\
58.25	0.16254	6.33172372071654\\
58.25	0.1662	6.75043201032555\\
58.25	0.16986	7.19128112451797\\
58.25	0.17352	7.6542710632938\\
58.25	0.17718	8.13940182665301\\
58.25	0.18084	8.64667341459562\\
58.25	0.1845	9.17608582712164\\
58.25	0.18816	9.72763906423106\\
58.25	0.19182	10.3013331259239\\
58.25	0.19548	10.8971680122001\\
58.25	0.19914	11.5151437230597\\
58.25	0.2028	12.1552602585027\\
58.25	0.20646	12.8175176185291\\
58.25	0.21012	13.5019158031389\\
58.25	0.21378	14.2084548123321\\
58.25	0.21744	14.9371346461088\\
58.25	0.2211	15.6879553044687\\
58.25	0.22476	16.4609167874122\\
58.25	0.22842	17.256019094939\\
58.25	0.23208	18.0732622270492\\
58.25	0.23574	18.9126461837428\\
58.25	0.2394	19.7741709650198\\
58.25	0.24306	20.6578365708802\\
58.25	0.24672	21.563643001324\\
58.25	0.25038	22.4915902563512\\
58.25	0.25404	23.4416783359618\\
58.25	0.2577	24.4139072401558\\
58.25	0.26136	25.4082769689332\\
58.25	0.26502	26.424787522294\\
58.25	0.26868	27.4634389002382\\
58.25	0.27234	28.5242311027658\\
58.25	0.276	29.6071641298768\\
58.625	0.093	2.52284004180713\\
58.625	0.09666	2.52555607794874\\
58.625	0.10032	2.55041293867376\\
58.625	0.10398	2.59741062398215\\
58.625	0.10764	2.66654913387396\\
58.625	0.1113	2.75782846834917\\
58.625	0.11496	2.87124862740779\\
58.625	0.11862	3.00680961104978\\
58.625	0.12228	3.16451141927518\\
58.625	0.12594	3.34435405208399\\
58.625	0.1296	3.54633750947619\\
58.625	0.13326	3.77046179145179\\
58.625	0.13692	4.01672689801079\\
58.625	0.14058	4.2851328291532\\
58.625	0.14424	4.575679584879\\
58.625	0.1479	4.8883671651882\\
58.625	0.15156	5.22319557008081\\
58.625	0.15522	5.58016479955681\\
58.625	0.15888	5.95927485361621\\
58.625	0.16254	6.360525732259\\
58.625	0.1662	6.7839174354852\\
58.625	0.16986	7.2294499632948\\
58.625	0.17352	7.6971233156878\\
58.625	0.17718	8.1869374926642\\
58.625	0.18084	8.69889249422398\\
58.625	0.1845	9.23298832036719\\
58.625	0.18816	9.78922497109378\\
58.625	0.19182	10.3676024464038\\
58.625	0.19548	10.9681207462972\\
58.625	0.19914	11.590779870774\\
58.625	0.2028	12.2355798198342\\
58.625	0.20646	12.9025205934778\\
58.625	0.21012	13.5916021917047\\
58.625	0.21378	14.3028246145151\\
58.625	0.21744	15.0361878619089\\
58.625	0.2211	15.7916919338861\\
58.625	0.22476	16.5693368304467\\
58.625	0.22842	17.3691225515907\\
58.625	0.23208	18.1910490973181\\
58.625	0.23574	19.0351164676289\\
58.625	0.2394	19.9013246625231\\
58.625	0.24306	20.7896736820006\\
58.625	0.24672	21.7001635260616\\
58.625	0.25038	22.632794194706\\
58.625	0.25404	23.5875656879338\\
58.625	0.2577	24.564478005745\\
58.625	0.26136	25.5635311481396\\
58.625	0.26502	26.5847251151176\\
58.625	0.26868	27.6280599066789\\
58.625	0.27234	28.6935355228237\\
58.625	0.276	29.7811519635519\\
59	0.093	2.46516763442658\\
59	0.09666	2.47256708418537\\
59	0.10032	2.50210735852756\\
59	0.10398	2.55378845745314\\
59	0.10764	2.62761038096212\\
59	0.1113	2.72357312905451\\
59	0.11496	2.84167670173031\\
59	0.11862	2.98192109898949\\
59	0.12228	3.14430632083208\\
59	0.12594	3.32883236725807\\
59	0.1296	3.53549923826743\\
59	0.13326	3.76430693386022\\
59	0.13692	4.01525545403641\\
59	0.14058	4.28834479879599\\
59	0.14424	4.58357496813898\\
59	0.1479	4.90094596206535\\
59	0.15156	5.24045778057514\\
59	0.15522	5.60211042366833\\
59	0.15888	5.9859038913449\\
59	0.16254	6.39183818360488\\
59	0.1662	6.81991330044825\\
59	0.16986	7.27012924187504\\
59	0.17352	7.74248600788522\\
59	0.17718	8.2369835984788\\
59	0.18084	8.75362201365576\\
59	0.1845	9.29240125341615\\
59	0.18816	9.85332131775992\\
59	0.19182	10.4363822066871\\
59	0.19548	11.0415839201977\\
59	0.19914	11.6689264582916\\
59	0.2028	12.318409820969\\
59	0.20646	12.9900340082298\\
59	0.21012	13.683799020074\\
59	0.21378	14.3997048565015\\
59	0.21744	15.1377515175125\\
59	0.2211	15.8979390031069\\
59	0.22476	16.6802673132847\\
59	0.22842	17.4847364480458\\
59	0.23208	18.3113464073904\\
59	0.23574	19.1600971913184\\
59	0.2394	20.0309887998297\\
59	0.24306	20.9240212329245\\
59	0.24672	21.8391944906027\\
59	0.25038	22.7765085728642\\
59	0.25404	23.7359634797092\\
59	0.2577	24.7175592111376\\
59	0.26136	25.7212957671493\\
59	0.26502	26.7471731477445\\
59	0.26868	27.7951913529231\\
59	0.27234	28.865350382685\\
59	0.276	29.9576502370304\\
59.375	0.093	2.41000566684942\\
59.375	0.09666	2.42208853022539\\
59.375	0.10032	2.45631221818476\\
59.375	0.10398	2.51267673072753\\
59.375	0.10764	2.59118206785369\\
59.375	0.1113	2.69182822956326\\
59.375	0.11496	2.81461521585624\\
59.375	0.11862	2.95954302673259\\
59.375	0.12228	3.12661166219237\\
59.375	0.12594	3.31582112223554\\
59.375	0.1296	3.5271714068621\\
59.375	0.13326	3.76066251607207\\
59.375	0.13692	4.01629444986542\\
59.375	0.14058	4.29406720824219\\
59.375	0.14424	4.59398079120236\\
59.375	0.1479	4.91603519874592\\
59.375	0.15156	5.26023043087287\\
59.375	0.15522	5.62656648758324\\
59.375	0.15888	6.01504336887701\\
59.375	0.16254	6.42566107475416\\
59.375	0.1662	6.85841960521473\\
59.375	0.16986	7.31331896025868\\
59.375	0.17352	7.79035913988603\\
59.375	0.17718	8.28954014409679\\
59.375	0.18084	8.81086197289095\\
59.375	0.1845	9.35432462626852\\
59.375	0.18816	9.91992810422947\\
59.375	0.19182	10.5076724067738\\
59.375	0.19548	11.1175575339016\\
59.375	0.19914	11.7495834856127\\
59.375	0.2028	12.4037502619073\\
59.375	0.20646	13.0800578627852\\
59.375	0.21012	13.7785062882466\\
59.375	0.21378	14.4990955382914\\
59.375	0.21744	15.2418256129195\\
59.375	0.2211	16.006696512131\\
59.375	0.22476	16.793708235926\\
59.375	0.22842	17.6028607843044\\
59.375	0.23208	18.4341541572661\\
59.375	0.23574	19.2875883548113\\
59.375	0.2394	20.1631633769398\\
59.375	0.24306	21.0608792236518\\
59.375	0.24672	21.9807358949471\\
59.375	0.25038	22.9227333908259\\
59.375	0.25404	23.886871711288\\
59.375	0.2577	24.8731508563336\\
59.375	0.26136	25.8815708259625\\
59.375	0.26502	26.9121316201749\\
59.375	0.26868	27.9648332389706\\
59.375	0.27234	29.0396756823497\\
59.375	0.276	30.1366589503123\\
59.75	0.093	2.35735413907569\\
59.75	0.09666	2.37412041606884\\
59.75	0.10032	2.41302751764539\\
59.75	0.10398	2.47407544380533\\
59.75	0.10764	2.55726419454868\\
59.75	0.1113	2.66259376987543\\
59.75	0.11496	2.79006416978559\\
59.75	0.11862	2.93967539427912\\
59.75	0.12228	3.11142744335608\\
59.75	0.12594	3.30532031701642\\
59.75	0.1296	3.52135401526018\\
59.75	0.13326	3.75952853808732\\
59.75	0.13692	4.01984388549786\\
59.75	0.14058	4.30230005749181\\
59.75	0.14424	4.60689705406915\\
59.75	0.1479	4.9336348752299\\
59.75	0.15156	5.28251352097403\\
59.75	0.15522	5.65353299130158\\
59.75	0.15888	6.04669328621252\\
59.75	0.16254	6.46199440570686\\
59.75	0.1662	6.89943634978461\\
59.75	0.16986	7.35901911844574\\
59.75	0.17352	7.84074271169027\\
59.75	0.17718	8.34460712951822\\
59.75	0.18084	8.87061237192956\\
59.75	0.1845	9.4187584389243\\
59.75	0.18816	9.98904533050244\\
59.75	0.19182	10.581473046664\\
59.75	0.19548	11.1960415874089\\
59.75	0.19914	11.8327509527372\\
59.75	0.2028	12.491601142649\\
59.75	0.20646	13.1725921571441\\
59.75	0.21012	13.8757239962226\\
59.75	0.21378	14.6009966598846\\
59.75	0.21744	15.3484101481299\\
59.75	0.2211	16.1179644609586\\
59.75	0.22476	16.9096595983708\\
59.75	0.22842	17.7234955603663\\
59.75	0.23208	18.5594723469452\\
59.75	0.23574	19.4175899581076\\
59.75	0.2394	20.2978483938533\\
59.75	0.24306	21.2002476541824\\
59.75	0.24672	22.124787739095\\
59.75	0.25038	23.0714686485909\\
59.75	0.25404	24.0402903826702\\
59.75	0.2577	25.031252941333\\
59.75	0.26136	26.0443563245791\\
59.75	0.26502	27.0796005324086\\
59.75	0.26868	28.1369855648215\\
59.75	0.27234	29.2165114218178\\
59.75	0.276	30.3181781033976\\
60.125	0.093	2.30721305110536\\
60.125	0.09666	2.3286627417157\\
60.125	0.10032	2.37225325690943\\
60.125	0.10398	2.43798459668655\\
60.125	0.10764	2.52585676104709\\
60.125	0.1113	2.63586974999102\\
60.125	0.11496	2.76802356351835\\
60.125	0.11862	2.92231820162907\\
60.125	0.12228	3.0987536643232\\
60.125	0.12594	3.29732995160073\\
60.125	0.1296	3.51804706346166\\
60.125	0.13326	3.76090499990599\\
60.125	0.13692	4.0259037609337\\
60.125	0.14058	4.31304334654483\\
60.125	0.14424	4.62232375673936\\
60.125	0.1479	4.95374499151729\\
60.125	0.15156	5.3073070508786\\
60.125	0.15522	5.68300993482333\\
60.125	0.15888	6.08085364335145\\
60.125	0.16254	6.50083817646298\\
60.125	0.1662	6.94296353415791\\
60.125	0.16986	7.40722971643622\\
60.125	0.17352	7.89363672329793\\
60.125	0.17718	8.40218455474305\\
60.125	0.18084	8.93287321077157\\
60.125	0.1845	9.4857026913835\\
60.125	0.18816	10.0606729965788\\
60.125	0.19182	10.6577841263575\\
60.125	0.19548	11.2770360807196\\
60.125	0.19914	11.9184288596652\\
60.125	0.2028	12.5819624631941\\
60.125	0.20646	13.2676368913064\\
60.125	0.21012	13.9754521440021\\
60.125	0.21378	14.7054082212812\\
60.125	0.21744	15.4575051231437\\
60.125	0.2211	16.2317428495896\\
60.125	0.22476	17.028121400619\\
60.125	0.22842	17.8466407762317\\
60.125	0.23208	18.6873009764278\\
60.125	0.23574	19.5501020012073\\
60.125	0.2394	20.4350438505702\\
60.125	0.24306	21.3421265245165\\
60.125	0.24672	22.2713500230462\\
60.125	0.25038	23.2227143461594\\
60.125	0.25404	24.1962194938558\\
60.125	0.2577	25.1918654661358\\
60.125	0.26136	26.2096522629991\\
60.125	0.26502	27.2495798844458\\
60.125	0.26868	28.3116483304759\\
60.125	0.27234	29.3958576010894\\
60.125	0.276	30.5022076962863\\
60.5	0.093	2.25958240293845\\
60.5	0.09666	2.28571550716595\\
60.5	0.10032	2.33398943597687\\
60.5	0.10398	2.40440418937118\\
60.5	0.10764	2.49695976734889\\
60.5	0.1113	2.61165616991\\
60.5	0.11496	2.74849339705452\\
60.5	0.11862	2.90747144878242\\
60.5	0.12228	3.08859032509372\\
60.5	0.12594	3.29185002598844\\
60.5	0.1296	3.51725055146653\\
60.5	0.13326	3.76479190152804\\
60.5	0.13692	4.03447407617296\\
60.5	0.14058	4.32629707540126\\
60.5	0.14424	4.64026089921298\\
60.5	0.1479	4.97636554760807\\
60.5	0.15156	5.33461102058657\\
60.5	0.15522	5.71499731814849\\
60.5	0.15888	6.11752444029379\\
60.5	0.16254	6.54219238702248\\
60.5	0.1662	6.98900115833458\\
60.5	0.16986	7.4579507542301\\
60.5	0.17352	7.949041174709\\
60.5	0.17718	8.46227241977131\\
60.5	0.18084	8.99764448941699\\
60.5	0.1845	9.5551573836461\\
60.5	0.18816	10.1348111024586\\
60.5	0.19182	10.7366056458545\\
60.5	0.19548	11.3605410138338\\
60.5	0.19914	12.0066172063965\\
60.5	0.2028	12.6748342235426\\
60.5	0.20646	13.3651920652721\\
60.5	0.21012	14.077690731585\\
60.5	0.21378	14.8123302224813\\
60.5	0.21744	15.569110537961\\
60.5	0.2211	16.348031678024\\
60.5	0.22476	17.1490936426706\\
60.5	0.22842	17.9722964319005\\
60.5	0.23208	18.8176400457138\\
60.5	0.23574	19.6851244841105\\
60.5	0.2394	20.5747497470905\\
60.5	0.24306	21.486515834654\\
60.5	0.24672	22.4204227468009\\
60.5	0.25038	23.3764704835312\\
60.5	0.25404	24.3546590448449\\
60.5	0.2577	25.354988430742\\
60.5	0.26136	26.3774586412225\\
60.5	0.26502	27.4220696762864\\
60.5	0.26868	28.4888215359336\\
60.5	0.27234	29.5777142201643\\
60.5	0.276	30.6887477289784\\
60.875	0.093	2.21446219457494\\
60.875	0.09666	2.24527871241963\\
60.875	0.10032	2.29823605484773\\
60.875	0.10398	2.37333422185922\\
60.875	0.10764	2.47057321345411\\
60.875	0.1113	2.5899530296324\\
60.875	0.11496	2.7314736703941\\
60.875	0.11862	2.89513513573919\\
60.875	0.12228	3.08093742566767\\
60.875	0.12594	3.28888054017956\\
60.875	0.1296	3.51896447927484\\
60.875	0.13326	3.77118924295353\\
60.875	0.13692	4.04555483121563\\
60.875	0.14058	4.34206124406111\\
60.875	0.14424	4.66070848149\\
60.875	0.1479	5.00149654350228\\
60.875	0.15156	5.36442543009797\\
60.875	0.15522	5.74949514127706\\
60.875	0.15888	6.15670567703955\\
60.875	0.16254	6.58605703738542\\
60.875	0.1662	7.03754922231471\\
60.875	0.16986	7.51118223182739\\
60.875	0.17352	8.00695606592348\\
60.875	0.17718	8.52487072460296\\
60.875	0.18084	9.06492620786583\\
60.875	0.1845	9.62712251571212\\
60.875	0.18816	10.2114596481418\\
60.875	0.19182	10.8179376051549\\
60.875	0.19548	11.4465563867514\\
60.875	0.19914	12.0973159929312\\
60.875	0.2028	12.7702164236945\\
60.875	0.20646	13.4652576790412\\
60.875	0.21012	14.1824397589713\\
60.875	0.21378	14.9217626634847\\
60.875	0.21744	15.6832263925816\\
60.875	0.2211	16.4668309462619\\
60.875	0.22476	17.2725763245256\\
60.875	0.22842	18.1004625273726\\
60.875	0.23208	18.9504895548031\\
60.875	0.23574	19.822657406817\\
60.875	0.2394	20.7169660834143\\
60.875	0.24306	21.6334155845949\\
60.875	0.24672	22.572005910359\\
60.875	0.25038	23.5327370607065\\
60.875	0.25404	24.5156090356373\\
60.875	0.2577	25.5206218351516\\
60.875	0.26136	26.5477754592493\\
60.875	0.26502	27.5970699079304\\
60.875	0.26868	28.6685051811948\\
60.875	0.27234	29.7620812790427\\
60.875	0.276	30.877798201474\\
61.25	0.093	2.17185242601486\\
61.25	0.09666	2.20735235747673\\
61.25	0.10032	2.26499311352201\\
61.25	0.10398	2.34477469415068\\
61.25	0.10764	2.44669709936275\\
61.25	0.1113	2.57076032915822\\
61.25	0.11496	2.7169643835371\\
61.25	0.11862	2.88530926249938\\
61.25	0.12228	3.07579496604504\\
61.25	0.12594	3.28842149417411\\
61.25	0.1296	3.52318884688657\\
61.25	0.13326	3.78009702418244\\
61.25	0.13692	4.05914602606171\\
61.25	0.14058	4.36033585252438\\
61.25	0.14424	4.68366650357045\\
61.25	0.1479	5.02913797919991\\
61.25	0.15156	5.39675027941278\\
61.25	0.15522	5.78650340420905\\
61.25	0.15888	6.19839735358872\\
61.25	0.16254	6.63243212755177\\
61.25	0.1662	7.08860772609824\\
61.25	0.16986	7.56692414922811\\
61.25	0.17352	8.06738139694138\\
61.25	0.17718	8.58997946923804\\
61.25	0.18084	9.13471836611809\\
61.25	0.1845	9.70159808758156\\
61.25	0.18816	10.2906186336284\\
61.25	0.19182	10.9017800042587\\
61.25	0.19548	11.5350821994723\\
61.25	0.19914	12.1905252192694\\
61.25	0.2028	12.8681090636499\\
61.25	0.20646	13.5678337326137\\
61.25	0.21012	14.289699226161\\
61.25	0.21378	15.0337055442916\\
61.25	0.21744	15.7998526870057\\
61.25	0.2211	16.5881406543031\\
61.25	0.22476	17.398569446184\\
61.25	0.22842	18.2311390626483\\
61.25	0.23208	19.0858495036959\\
61.25	0.23574	19.962700769327\\
61.25	0.2394	20.8616928595414\\
61.25	0.24306	21.7828257743393\\
61.25	0.24672	22.7260995137205\\
61.25	0.25038	23.6915140776852\\
61.25	0.25404	24.6790694662332\\
61.25	0.2577	25.6887656793647\\
61.25	0.26136	26.7206027170795\\
61.25	0.26502	27.7745805793778\\
61.25	0.26868	28.8506992662594\\
61.25	0.27234	29.9489587777245\\
61.25	0.276	31.0693591137729\\
61.625	0.093	2.1317530972582\\
61.625	0.09666	2.17193644233724\\
61.625	0.10032	2.23426061199971\\
61.625	0.10398	2.31872560624556\\
61.625	0.10764	2.4253314250748\\
61.625	0.1113	2.55407806848746\\
61.625	0.11496	2.70496553648352\\
61.625	0.11862	2.87799382906297\\
61.625	0.12228	3.07316294622582\\
61.625	0.12594	3.29047288797207\\
61.625	0.1296	3.52992365430171\\
61.625	0.13326	3.79151524521476\\
61.625	0.13692	4.07524766071121\\
61.625	0.14058	4.38112090079106\\
61.625	0.14424	4.70913496545432\\
61.625	0.1479	5.05928985470095\\
61.625	0.15156	5.43158556853101\\
61.625	0.15522	5.82602210694445\\
61.625	0.15888	6.2425994699413\\
61.625	0.16254	6.68131765752154\\
61.625	0.1662	7.14217666968519\\
61.625	0.16986	7.62517650643223\\
61.625	0.17352	8.13031716776268\\
61.625	0.17718	8.65759865367653\\
61.625	0.18084	9.20702096417376\\
61.625	0.1845	9.77858409925441\\
61.625	0.18816	10.3722880589185\\
61.625	0.19182	10.9881328431659\\
61.625	0.19548	11.6261184519967\\
61.625	0.19914	12.286244885411\\
61.625	0.2028	12.9685121434086\\
61.625	0.20646	13.6729202259897\\
61.625	0.21012	14.3994691331541\\
61.625	0.21378	15.1481588649019\\
61.625	0.21744	15.9189894212332\\
61.625	0.2211	16.7119608021478\\
61.625	0.22476	17.5270730076458\\
61.625	0.22842	18.3643260377273\\
61.625	0.23208	19.2237198923921\\
61.625	0.23574	20.1052545716404\\
61.625	0.2394	21.008930075472\\
61.625	0.24306	21.934746403887\\
61.625	0.24672	22.8827035568854\\
61.625	0.25038	23.8528015344673\\
61.625	0.25404	24.8450403366325\\
61.625	0.2577	25.8594199633811\\
61.625	0.26136	26.8959404147132\\
61.625	0.26502	27.9546016906286\\
61.625	0.26868	29.0354037911274\\
61.625	0.27234	30.1383467162097\\
61.625	0.276	31.2634304658753\\
62	0.093	2.09416420830492\\
62	0.09666	2.13903096700114\\
62	0.10032	2.20603855028078\\
62	0.10398	2.29518695814381\\
62	0.10764	2.40647619059025\\
62	0.1113	2.53990624762008\\
62	0.11496	2.69547712923332\\
62	0.11862	2.87318883542994\\
62	0.12228	3.07304136620998\\
62	0.12594	3.29503472157342\\
62	0.1296	3.53916890152025\\
62	0.13326	3.80544390605048\\
62	0.13692	4.0938597351641\\
62	0.14058	4.40441638886113\\
62	0.14424	4.73711386714157\\
62	0.1479	5.0919521700054\\
62	0.15156	5.46893129745262\\
62	0.15522	5.86805124948325\\
62	0.15888	6.28931202609728\\
62	0.16254	6.73271362729471\\
62	0.1662	7.19825605307554\\
62	0.16986	7.68593930343975\\
62	0.17352	8.19576337838737\\
62	0.17718	8.72772827791839\\
62	0.18084	9.28183400203282\\
62	0.1845	9.85808055073065\\
62	0.18816	10.4564679240119\\
62	0.19182	11.0769961218765\\
62	0.19548	11.7196651443245\\
62	0.19914	12.3844749913559\\
62	0.2028	13.0714256629708\\
62	0.20646	13.780517159169\\
62	0.21012	14.5117494799506\\
62	0.21378	15.2651226253156\\
62	0.21744	16.040636595264\\
62	0.2211	16.8382913897958\\
62	0.22476	17.6580870089111\\
62	0.22842	18.5000234526097\\
62	0.23208	19.3641007208917\\
62	0.23574	20.2503188137571\\
62	0.2394	21.1586777312059\\
62	0.24306	22.0891774732381\\
62	0.24672	23.0418180398538\\
62	0.25038	24.0165994310528\\
62	0.25404	25.0135216468352\\
62	0.2577	26.032584687201\\
62	0.26136	27.0737885521502\\
62	0.26502	28.1371332416828\\
62	0.26868	29.2226187557988\\
62	0.27234	30.3302450944982\\
62	0.276	31.460012257781\\
62.375	0.093	2.05908575915507\\
62.375	0.09666	2.10863593146849\\
62.375	0.10032	2.18032692836531\\
62.375	0.10398	2.27415874984552\\
62.375	0.10764	2.39013139590914\\
62.375	0.1113	2.52824486655615\\
62.375	0.11496	2.68849916178656\\
62.375	0.11862	2.87089428160036\\
62.375	0.12228	3.07543022599758\\
62.375	0.12594	3.3021069949782\\
62.375	0.1296	3.55092458854222\\
62.375	0.13326	3.82188300668963\\
62.375	0.13692	4.11498224942044\\
62.375	0.14058	4.43022231673465\\
62.375	0.14424	4.76760320863225\\
62.375	0.1479	5.12712492511327\\
62.375	0.15156	5.50878746617767\\
62.375	0.15522	5.91259083182548\\
62.375	0.15888	6.33853502205669\\
62.375	0.16254	6.7866200368713\\
62.375	0.1662	7.25684587626931\\
62.375	0.16986	7.74921254025071\\
62.375	0.17352	8.2637200288155\\
62.375	0.17718	8.80036834196371\\
62.375	0.18084	9.35915747969531\\
62.375	0.1845	9.94008744201033\\
62.375	0.18816	10.5431582289087\\
62.375	0.19182	11.1683698403905\\
62.375	0.19548	11.8157222764557\\
62.375	0.19914	12.4852155371043\\
62.375	0.2028	13.1768496223363\\
62.375	0.20646	13.8906245321517\\
62.375	0.21012	14.6265402665505\\
62.375	0.21378	15.3845968255327\\
62.375	0.21744	16.1647942090983\\
62.375	0.2211	16.9671324172473\\
62.375	0.22476	17.7916114499797\\
62.375	0.22842	18.6382313072955\\
62.375	0.23208	19.5069919891947\\
62.375	0.23574	20.3978934956773\\
62.375	0.2394	21.3109358267433\\
62.375	0.24306	22.2461189823927\\
62.375	0.24672	23.2034429626255\\
62.375	0.25038	24.1829077674417\\
62.375	0.25404	25.1845133968413\\
62.375	0.2577	26.2082598508243\\
62.375	0.26136	27.2541471293907\\
62.375	0.26502	28.3221752325405\\
62.375	0.26868	29.4123441602737\\
62.375	0.27234	30.5246539125902\\
62.375	0.276	31.6591044894902\\
62.75	0.093	2.02651774980864\\
62.75	0.09666	2.08075133573923\\
62.75	0.10032	2.15712574625323\\
62.75	0.10398	2.25564098135062\\
62.75	0.10764	2.37629704103142\\
62.75	0.1113	2.51909392529562\\
62.75	0.11496	2.68403163414322\\
62.75	0.11862	2.87111016757419\\
62.75	0.12228	3.08032952558859\\
62.75	0.12594	3.31168970818639\\
62.75	0.1296	3.56519071536759\\
62.75	0.13326	3.84083254713219\\
62.75	0.13692	4.13861520348016\\
62.75	0.14058	4.45853868441156\\
62.75	0.14424	4.80060298992635\\
62.75	0.1479	5.16480812002455\\
62.75	0.15156	5.55115407470612\\
62.75	0.15522	5.95964085397112\\
62.75	0.15888	6.39026845781951\\
62.75	0.16254	6.8430368862513\\
62.75	0.1662	7.31794613926649\\
62.75	0.16986	7.81499621686507\\
62.75	0.17352	8.33418711904705\\
62.75	0.17718	8.87551884581243\\
62.75	0.18084	9.43899139716122\\
62.75	0.1845	10.0246047730934\\
62.75	0.18816	10.632358973609\\
62.75	0.19182	11.262253998708\\
62.75	0.19548	11.9142898483904\\
62.75	0.19914	12.5884665226561\\
62.75	0.2028	13.2847840215053\\
62.75	0.20646	14.0032423449379\\
62.75	0.21012	14.7438414929539\\
62.75	0.21378	15.5065814655533\\
62.75	0.21744	16.2914622627361\\
62.75	0.2211	17.0984838845022\\
62.75	0.22476	17.9276463308518\\
62.75	0.22842	18.7789496017848\\
62.75	0.23208	19.6523936973012\\
62.75	0.23574	20.547978617401\\
62.75	0.2394	21.4657043620841\\
62.75	0.24306	22.4055709313507\\
62.75	0.24672	23.3675783252007\\
62.75	0.25038	24.3517265436341\\
62.75	0.25404	25.3580155866508\\
62.75	0.2577	26.386445454251\\
62.75	0.26136	27.4370161464346\\
62.75	0.26502	28.5097276632015\\
62.75	0.26868	29.6045800045519\\
62.75	0.27234	30.7215731704857\\
62.75	0.276	31.8607071610029\\
63.125	0.093	1.99646018026559\\
63.125	0.09666	2.05537717981337\\
63.125	0.10032	2.13643500394455\\
63.125	0.10398	2.23963365265912\\
63.125	0.10764	2.36497312595709\\
63.125	0.1113	2.51245342383848\\
63.125	0.11496	2.68207454630326\\
63.125	0.11862	2.87383649335144\\
63.125	0.12228	3.087739264983\\
63.125	0.12594	3.32378286119798\\
63.125	0.1296	3.58196728199634\\
63.125	0.13326	3.86229252737812\\
63.125	0.13692	4.16475859734329\\
63.125	0.14058	4.48936549189187\\
63.125	0.14424	4.83611321102384\\
63.125	0.1479	5.2050017547392\\
63.125	0.15156	5.59603112303798\\
63.125	0.15522	6.00920131592015\\
63.125	0.15888	6.44451233338572\\
63.125	0.16254	6.90196417543468\\
63.125	0.1662	7.38155684206705\\
63.125	0.16986	7.88329033328282\\
63.125	0.17352	8.407164649082\\
63.125	0.17718	8.95317978946457\\
63.125	0.18084	9.52133575443052\\
63.125	0.1845	10.1116325439799\\
63.125	0.18816	10.7240701581127\\
63.125	0.19182	11.3586485968288\\
63.125	0.19548	12.0153678601284\\
63.125	0.19914	12.6942279480113\\
63.125	0.2028	13.3952288604777\\
63.125	0.20646	14.1183705975275\\
63.125	0.21012	14.8636531591606\\
63.125	0.21378	15.6310765453772\\
63.125	0.21744	16.4206407561772\\
63.125	0.2211	17.2323457915605\\
63.125	0.22476	18.0661916515273\\
63.125	0.22842	18.9221783360774\\
63.125	0.23208	19.800305845211\\
63.125	0.23574	20.700574178928\\
63.125	0.2394	21.6229833372283\\
63.125	0.24306	22.5675333201121\\
63.125	0.24672	23.5342241275792\\
63.125	0.25038	24.5230557596298\\
63.125	0.25404	25.5340282162637\\
63.125	0.2577	26.5671414974811\\
63.125	0.26136	27.6223956032818\\
63.125	0.26502	28.699790533666\\
63.125	0.26868	29.7993262886335\\
63.125	0.27234	30.9210028681845\\
63.125	0.276	32.0648202723188\\
63.5	0.093	1.96891305052599\\
63.5	0.09666	2.03251346369095\\
63.5	0.10032	2.11825470143931\\
63.5	0.10398	2.22613676377106\\
63.5	0.10764	2.35615965068621\\
63.5	0.1113	2.50832336218478\\
63.5	0.11496	2.68262789826674\\
63.5	0.11862	2.8790732589321\\
63.5	0.12228	3.09765944418085\\
63.5	0.12594	3.338386454013\\
63.5	0.1296	3.60125428842854\\
63.5	0.13326	3.8862629474275\\
63.5	0.13692	4.19341243100985\\
63.5	0.14058	4.52270273917561\\
63.5	0.14424	4.87413387192476\\
63.5	0.1479	5.24770582925731\\
63.5	0.15156	5.64341861117326\\
63.5	0.15522	6.06127221767262\\
63.5	0.15888	6.50126664875537\\
63.5	0.16254	6.96340190442151\\
63.5	0.1662	7.44767798467106\\
63.5	0.16986	7.95409488950401\\
63.5	0.17352	8.48265261892037\\
63.5	0.17718	9.03335117292012\\
63.5	0.18084	9.60619055150325\\
63.5	0.1845	10.2011707546698\\
63.5	0.18816	10.8182917824198\\
63.5	0.19182	11.4575536347531\\
63.5	0.19548	12.1189563116698\\
63.5	0.19914	12.80249981317\\
63.5	0.2028	13.5081841392535\\
63.5	0.20646	14.2360092899205\\
63.5	0.21012	14.9859752651708\\
63.5	0.21378	15.7580820650046\\
63.5	0.21744	16.5523296894217\\
63.5	0.2211	17.3687181384222\\
63.5	0.22476	18.2072474120062\\
63.5	0.22842	19.0679175101735\\
63.5	0.23208	19.9507284329243\\
63.5	0.23574	20.8556801802584\\
63.5	0.2394	21.7827727521759\\
63.5	0.24306	22.7320061486769\\
63.5	0.24672	23.7033803697612\\
63.5	0.25038	24.696895415429\\
63.5	0.25404	25.7125512856801\\
63.5	0.2577	26.7503479805146\\
63.5	0.26136	27.8102854999326\\
63.5	0.26502	28.8923638439339\\
63.5	0.26868	29.9965830125186\\
63.5	0.27234	31.1229430056867\\
63.5	0.276	32.2714438234383\\
63.875	0.093	1.94387636058979\\
63.875	0.09666	2.01216018737193\\
63.875	0.10032	2.10258483873747\\
63.875	0.10398	2.2151503146864\\
63.875	0.10764	2.34985661521873\\
63.875	0.1113	2.50670374033448\\
63.875	0.11496	2.68569169003362\\
63.875	0.11862	2.88682046431616\\
63.875	0.12228	3.11009006318209\\
63.875	0.12594	3.35550048663143\\
63.875	0.1296	3.62305173466415\\
63.875	0.13326	3.91274380728029\\
63.875	0.13692	4.22457670447982\\
63.875	0.14058	4.55855042626276\\
63.875	0.14424	4.9146649726291\\
63.875	0.1479	5.29292034357882\\
63.875	0.15156	5.69331653911196\\
63.875	0.15522	6.11585355922849\\
63.875	0.15888	6.56053140392842\\
63.875	0.16254	7.02735007321174\\
63.875	0.1662	7.51630956707848\\
63.875	0.16986	8.02740988552861\\
63.875	0.17352	8.56065102856214\\
63.875	0.17718	9.11603299617908\\
63.875	0.18084	9.69355578837938\\
63.875	0.1845	10.2932194051631\\
63.875	0.18816	10.9150238465303\\
63.875	0.19182	11.5589691124808\\
63.875	0.19548	12.2250552030147\\
63.875	0.19914	12.913282118132\\
63.875	0.2028	13.6236498578328\\
63.875	0.20646	14.3561584221169\\
63.875	0.21012	15.1108078109844\\
63.875	0.21378	15.8875980244353\\
63.875	0.21744	16.6865290624696\\
63.875	0.2211	17.5076009250874\\
63.875	0.22476	18.3508136122885\\
63.875	0.22842	19.216167124073\\
63.875	0.23208	20.1036614604409\\
63.875	0.23574	21.0132966213923\\
63.875	0.2394	21.945072606927\\
63.875	0.24306	22.8989894170451\\
63.875	0.24672	23.8750470517466\\
63.875	0.25038	24.8732455110315\\
63.875	0.25404	25.8935847948998\\
63.875	0.2577	26.9360649033516\\
63.875	0.26136	28.0006858363867\\
63.875	0.26502	29.0874475940052\\
63.875	0.26868	30.1963501762071\\
63.875	0.27234	31.3273935829924\\
63.875	0.276	32.4805778143611\\
64.25	0.093	1.92135011045699\\
64.25	0.09666	1.99431735085631\\
64.25	0.10032	2.08942541583903\\
64.25	0.10398	2.20667430540514\\
64.25	0.10764	2.34606401955467\\
64.25	0.1113	2.50759455828758\\
64.25	0.11496	2.69126592160391\\
64.25	0.11862	2.89707810950362\\
64.25	0.12228	3.12503112198673\\
64.25	0.12594	3.37512495905326\\
64.25	0.1296	3.64735962070317\\
64.25	0.13326	3.9417351069365\\
64.25	0.13692	4.25825141775319\\
64.25	0.14058	4.59690855315331\\
64.25	0.14424	4.95770651313683\\
64.25	0.1479	5.34064529770375\\
64.25	0.15156	5.74572490685405\\
64.25	0.15522	6.17294534058777\\
64.25	0.15888	6.62230659890488\\
64.25	0.16254	7.0938086818054\\
64.25	0.1662	7.58745158928931\\
64.25	0.16986	8.10323532135661\\
64.25	0.17352	8.64115987800731\\
64.25	0.17718	9.20122525924142\\
64.25	0.18084	9.78343146505893\\
64.25	0.1845	10.3877784954598\\
64.25	0.18816	11.0142663504442\\
64.25	0.19182	11.6628950300119\\
64.25	0.19548	12.333664534163\\
64.25	0.19914	13.0265748628975\\
64.25	0.2028	13.7416260162154\\
64.25	0.20646	14.4788179941167\\
64.25	0.21012	15.2381507966014\\
64.25	0.21378	16.0196244236695\\
64.25	0.21744	16.823238875321\\
64.25	0.2211	17.6489941515559\\
64.25	0.22476	18.4968902523742\\
64.25	0.22842	19.3669271777759\\
64.25	0.23208	20.259104927761\\
64.25	0.23574	21.1734235023295\\
64.25	0.2394	22.1098829014814\\
64.25	0.24306	23.0684831252167\\
64.25	0.24672	24.0492241735354\\
64.25	0.25038	25.0521060464375\\
64.25	0.25404	26.077128743923\\
64.25	0.2577	27.1242922659919\\
64.25	0.26136	28.1935966126442\\
64.25	0.26502	29.2850417838799\\
64.25	0.26868	30.398627779699\\
64.25	0.27234	31.5343546001015\\
64.25	0.276	32.6922222450874\\
64.625	0.093	1.90133430012762\\
64.625	0.09666	1.97898495414412\\
64.625	0.10032	2.07877643274402\\
64.625	0.10398	2.20070873592732\\
64.625	0.10764	2.34478186369402\\
64.625	0.1113	2.51099581604412\\
64.625	0.11496	2.69935059297762\\
64.625	0.11862	2.90984619449451\\
64.625	0.12228	3.14248262059481\\
64.625	0.12594	3.39725987127851\\
64.625	0.1296	3.67417794654561\\
64.625	0.13326	3.97323684639611\\
64.625	0.13692	4.29443657083\\
64.625	0.14058	4.63777711984729\\
64.625	0.14424	5.00325849344799\\
64.625	0.1479	5.39088069163209\\
64.625	0.15156	5.80064371439957\\
64.625	0.15522	6.23254756175047\\
64.625	0.15888	6.68659223368477\\
64.625	0.16254	7.16277773020246\\
64.625	0.1662	7.66110405130356\\
64.625	0.16986	8.18157119698804\\
64.625	0.17352	8.72417916725592\\
64.625	0.17718	9.28892796210721\\
64.625	0.18084	9.8758175815419\\
64.625	0.1845	10.48484802556\\
64.625	0.18816	11.1160192941615\\
64.625	0.19182	11.7693313873464\\
64.625	0.19548	12.4447843051147\\
64.625	0.19914	13.1423780474663\\
64.625	0.2028	13.8621126144014\\
64.625	0.20646	14.6039880059199\\
64.625	0.21012	15.3680042220218\\
64.625	0.21378	16.1541612627071\\
64.625	0.21744	16.9624591279758\\
64.625	0.2211	17.7928978178278\\
64.625	0.22476	18.6454773322633\\
64.625	0.22842	19.5201976712822\\
64.625	0.23208	20.4170588348845\\
64.625	0.23574	21.3360608230702\\
64.625	0.2394	22.2772036358393\\
64.625	0.24306	23.2404872731917\\
64.625	0.24672	24.2259117351276\\
64.625	0.25038	25.2334770216469\\
64.625	0.25404	26.2631831327496\\
64.625	0.2577	27.3150300684357\\
64.625	0.26136	28.3890178287051\\
64.625	0.26502	29.485146413558\\
64.625	0.26868	30.6034158229943\\
64.625	0.27234	31.743826057014\\
64.625	0.276	32.906377115617\\
65	0.093	1.88382892960166\\
65	0.09666	1.96616299723534\\
65	0.10032	2.07063788945242\\
65	0.10398	2.19725360625291\\
65	0.10764	2.34601014763678\\
65	0.1113	2.51690751360406\\
65	0.11496	2.70994570415475\\
65	0.11862	2.92512471928882\\
65	0.12228	3.16244455900629\\
65	0.12594	3.42190522330718\\
65	0.1296	3.70350671219146\\
65	0.13326	4.00724902565914\\
65	0.13692	4.3331321637102\\
65	0.14058	4.68115612634468\\
65	0.14424	5.05132091356256\\
65	0.1479	5.44362652536384\\
65	0.15156	5.85807296174851\\
65	0.15522	6.29466022271658\\
65	0.15888	6.75338830826806\\
65	0.16254	7.23425721840294\\
65	0.1662	7.73726695312121\\
65	0.16986	8.26241751242288\\
65	0.17352	8.80970889630794\\
65	0.17718	9.37914110477641\\
65	0.18084	9.97071413782828\\
65	0.1845	10.5844279954636\\
65	0.18816	11.2202826776822\\
65	0.19182	11.8782781844843\\
65	0.19548	12.5584145158698\\
65	0.19914	13.2606916718386\\
65	0.2028	13.9851096523909\\
65	0.20646	14.7316684575266\\
65	0.21012	15.5003680872456\\
65	0.21378	16.2912085415481\\
65	0.21744	17.104189820434\\
65	0.2211	17.9393119239032\\
65	0.22476	18.7965748519559\\
65	0.22842	19.6759786045919\\
65	0.23208	20.5775231818114\\
65	0.23574	21.5012085836143\\
65	0.2394	22.4470348100005\\
65	0.24306	23.4150018609702\\
65	0.24672	24.4051097365233\\
65	0.25038	25.4173584366597\\
65	0.25404	26.4517479613796\\
65	0.2577	27.5082783106828\\
65	0.26136	28.5869494845695\\
65	0.26502	29.6877614830396\\
65	0.26868	30.810714306093\\
65	0.27234	31.9558079537299\\
65	0.276	33.1230424259501\\
65.375	0.093	1.86883399887912\\
65.375	0.09666	1.95585148012997\\
65.375	0.10032	2.06500978596423\\
65.375	0.10398	2.19630891638189\\
65.375	0.10764	2.34974887138296\\
65.375	0.1113	2.52532965096742\\
65.375	0.11496	2.72305125513528\\
65.375	0.11862	2.94291368388653\\
65.375	0.12228	3.1849169372212\\
65.375	0.12594	3.44906101513926\\
65.375	0.1296	3.73534591764071\\
65.375	0.13326	4.04377164472558\\
65.375	0.13692	4.37433819639383\\
65.375	0.14058	4.72704557264548\\
65.375	0.14424	5.10189377348055\\
65.375	0.1479	5.498882798899\\
65.375	0.15156	5.91801264890085\\
65.375	0.15522	6.35928332348611\\
65.375	0.15888	6.82269482265477\\
65.375	0.16254	7.30824714640682\\
65.375	0.1662	7.81594029474228\\
65.375	0.16986	8.34577426766113\\
65.375	0.17352	8.89774906516336\\
65.375	0.17718	9.47186468724902\\
65.375	0.18084	10.0681211339181\\
65.375	0.1845	10.6865184051705\\
65.375	0.18816	11.3270565010064\\
65.375	0.19182	11.9897354214256\\
65.375	0.19548	12.6745551664283\\
65.375	0.19914	13.3815157360143\\
65.375	0.2028	14.1106171301838\\
65.375	0.20646	14.8618593489366\\
65.375	0.21012	15.6352423922729\\
65.375	0.21378	16.4307662601925\\
65.375	0.21744	17.2484309526956\\
65.375	0.2211	18.088236469782\\
65.375	0.22476	18.9501828114518\\
65.375	0.22842	19.8342699777051\\
65.375	0.23208	20.7404979685417\\
65.375	0.23574	21.6688667839618\\
65.375	0.2394	22.6193764239652\\
65.375	0.24306	23.5920268885521\\
65.375	0.24672	24.5868181777223\\
65.375	0.25038	25.603750291476\\
65.375	0.25404	26.642823229813\\
65.375	0.2577	27.7040369927334\\
65.375	0.26136	28.7873915802373\\
65.375	0.26502	29.8928869923245\\
65.375	0.26868	31.0205232289951\\
65.375	0.27234	32.1703002902492\\
65.375	0.276	33.3422181760866\\
65.75	0.093	1.85634950795995\\
65.75	0.09666	1.948050402828\\
65.75	0.10032	2.06189212227946\\
65.75	0.10398	2.19787466631428\\
65.75	0.10764	2.35599803493252\\
65.75	0.1113	2.53626222813417\\
65.75	0.11496	2.73866724591922\\
65.75	0.11862	2.96321308828766\\
65.75	0.12228	3.20989975523949\\
65.75	0.12594	3.47872724677474\\
65.75	0.1296	3.76969556289336\\
65.75	0.13326	4.08280470359541\\
65.75	0.13692	4.41805466888084\\
65.75	0.14058	4.77544545874968\\
65.75	0.14424	5.15497707320193\\
65.75	0.1479	5.55664951223756\\
65.75	0.15156	5.98046277585659\\
65.75	0.15522	6.42641686405903\\
65.75	0.15888	6.89451177684487\\
65.75	0.16254	7.38474751421409\\
65.75	0.1662	7.89712407616673\\
65.75	0.16986	8.43164146270277\\
65.75	0.17352	8.98829967382221\\
65.75	0.17718	9.56709870952504\\
65.75	0.18084	10.1680385698113\\
65.75	0.1845	10.7911192546809\\
65.75	0.18816	11.4363407641339\\
65.75	0.19182	12.1037030981704\\
65.75	0.19548	12.7932062567902\\
65.75	0.19914	13.5048502399934\\
65.75	0.2028	14.2386350477801\\
65.75	0.20646	14.9945606801501\\
65.75	0.21012	15.7726271371035\\
65.75	0.21378	16.5728344186403\\
65.75	0.21744	17.3951825247606\\
65.75	0.2211	18.2396714554642\\
65.75	0.22476	19.1063012107512\\
65.75	0.22842	19.9950717906216\\
65.75	0.23208	20.9059831950755\\
65.75	0.23574	21.8390354241127\\
65.75	0.2394	22.7942284777333\\
65.75	0.24306	23.7715623559373\\
65.75	0.24672	24.7710370587247\\
65.75	0.25038	25.7926525860956\\
65.75	0.25404	26.8364089380498\\
65.75	0.2577	27.9023061145874\\
65.75	0.26136	28.9903441157084\\
65.75	0.26502	30.1005229414129\\
65.75	0.26868	31.2328425917007\\
65.75	0.27234	32.3873030665719\\
65.75	0.276	33.5639043660265\\
66.125	0.093	1.84637545684423\\
66.125	0.09666	1.94275976532945\\
66.125	0.10032	2.06128489839809\\
66.125	0.10398	2.2019508560501\\
66.125	0.10764	2.36475763828552\\
66.125	0.1113	2.54970524510435\\
66.125	0.11496	2.75679367650657\\
66.125	0.11862	2.9860229324922\\
66.125	0.12228	3.23739301306121\\
66.125	0.12594	3.51090391821363\\
66.125	0.1296	3.80655564794944\\
66.125	0.13326	4.12434820226866\\
66.125	0.13692	4.46428158117129\\
66.125	0.14058	4.82635578465731\\
66.125	0.14424	5.21057081272673\\
66.125	0.1479	5.61692666537954\\
66.125	0.15156	6.04542334261576\\
66.125	0.15522	6.49606084443538\\
66.125	0.15888	6.96883917083839\\
66.125	0.16254	7.4637583218248\\
66.125	0.1662	7.98081829739462\\
66.125	0.16986	8.52001909754784\\
66.125	0.17352	9.08136072228445\\
66.125	0.17718	9.66484317160447\\
66.125	0.18084	10.2704664455079\\
66.125	0.1845	10.8982305439947\\
66.125	0.18816	11.5481354670649\\
66.125	0.19182	12.2201812147185\\
66.125	0.19548	12.9143677869555\\
66.125	0.19914	13.6306951837759\\
66.125	0.2028	14.3691634051797\\
66.125	0.20646	15.129772451167\\
66.125	0.21012	15.9125223217376\\
66.125	0.21378	16.7174130168916\\
66.125	0.21744	17.544444536629\\
66.125	0.2211	18.3936168809498\\
66.125	0.22476	19.264930049854\\
66.125	0.22842	20.1583840433416\\
66.125	0.23208	21.0739788614126\\
66.125	0.23574	22.011714504067\\
66.125	0.2394	22.9715909713048\\
66.125	0.24306	23.953608263126\\
66.125	0.24672	24.9577663795306\\
66.125	0.25038	25.9840653205186\\
66.125	0.25404	27.03250508609\\
66.125	0.2577	28.1030856762448\\
66.125	0.26136	29.195807090983\\
66.125	0.26502	30.3106693303046\\
66.125	0.26868	31.4476723942096\\
66.125	0.27234	32.606816282698\\
66.125	0.276	33.7881009957698\\
66.5	0.093	1.83891184553192\\
66.5	0.09666	1.93997956763433\\
66.5	0.10032	2.06318811432014\\
66.5	0.10398	2.20853748558934\\
66.5	0.10764	2.37602768144193\\
66.5	0.1113	2.56565870187794\\
66.5	0.11496	2.77743054689735\\
66.5	0.11862	3.01134321650015\\
66.5	0.12228	3.26739671068635\\
66.5	0.12594	3.54559102945595\\
66.5	0.1296	3.84592617280894\\
66.5	0.13326	4.16840214074534\\
66.5	0.13692	4.51301893326515\\
66.5	0.14058	4.87977655036835\\
66.5	0.14424	5.26867499205495\\
66.5	0.1479	5.67971425832494\\
66.5	0.15156	6.11289434917834\\
66.5	0.15522	6.56821526461514\\
66.5	0.15888	7.04567700463534\\
66.5	0.16254	7.54527956923893\\
66.5	0.1662	8.06702295842593\\
66.5	0.16986	8.61090717219633\\
66.5	0.17352	9.17693221055012\\
66.5	0.17718	9.76509807348732\\
66.5	0.18084	10.3754047610079\\
66.5	0.1845	11.0078522731119\\
66.5	0.18816	11.6624406097993\\
66.5	0.19182	12.3391697710701\\
66.5	0.19548	13.0380397569243\\
66.5	0.19914	13.7590505673619\\
66.5	0.2028	14.5022022023829\\
66.5	0.20646	15.2674946619873\\
66.5	0.21012	16.054927946175\\
66.5	0.21378	16.8645020549462\\
66.5	0.21744	17.6962169883008\\
66.5	0.2211	18.5500727462388\\
66.5	0.22476	19.4260693287602\\
66.5	0.22842	20.324206735865\\
66.5	0.23208	21.2444849675532\\
66.5	0.23574	22.1869040238248\\
66.5	0.2394	23.1514639046797\\
66.5	0.24306	24.1381646101181\\
66.5	0.24672	25.1470061401399\\
66.5	0.25038	26.1779884947451\\
66.5	0.25404	27.2311116739337\\
66.5	0.2577	28.3063756777057\\
66.5	0.26136	29.403780506061\\
66.5	0.26502	30.5233261589998\\
66.5	0.26868	31.665012636522\\
66.5	0.27234	32.8288399386275\\
66.5	0.276	34.0148080653165\\
66.875	0.093	1.83395867402301\\
66.875	0.09666	1.93970980974259\\
66.875	0.10032	2.06760177004557\\
66.875	0.10398	2.21763455493196\\
66.875	0.10764	2.38980816440175\\
66.875	0.1113	2.58412259845493\\
66.875	0.11496	2.80057785709152\\
66.875	0.11862	3.03917394031149\\
66.875	0.12228	3.29991084811487\\
66.875	0.12594	3.58278858050166\\
66.875	0.1296	3.88780713747185\\
66.875	0.13326	4.21496651902543\\
66.875	0.13692	4.5642667251624\\
66.875	0.14058	4.93570775588278\\
66.875	0.14424	5.32928961118657\\
66.875	0.1479	5.74501229107375\\
66.875	0.15156	6.18287579554432\\
66.875	0.15522	6.6428801245983\\
66.875	0.15888	7.12502527823568\\
66.875	0.16254	7.62931125645646\\
66.875	0.1662	8.15573805926064\\
66.875	0.16986	8.70430568664821\\
66.875	0.17352	9.27501413861917\\
66.875	0.17718	9.86786341517355\\
66.875	0.18084	10.4828535163113\\
66.875	0.1845	11.1199844420325\\
66.875	0.18816	11.7792561923371\\
66.875	0.19182	12.4606687672251\\
66.875	0.19548	13.1642221666964\\
66.875	0.19914	13.8899163907512\\
66.875	0.2028	14.6377514393894\\
66.875	0.20646	15.4077273126109\\
66.875	0.21012	16.1998440104159\\
66.875	0.21378	17.0141015328043\\
66.875	0.21744	17.850499879776\\
66.875	0.2211	18.7090390513312\\
66.875	0.22476	19.5897190474698\\
66.875	0.22842	20.4925398681917\\
66.875	0.23208	21.4175015134971\\
66.875	0.23574	22.3646039833859\\
66.875	0.2394	23.333847277858\\
66.875	0.24306	24.3252313969136\\
66.875	0.24672	25.3387563405526\\
66.875	0.25038	26.3744221087749\\
66.875	0.25404	27.4322287015807\\
66.875	0.2577	28.5121761189699\\
66.875	0.26136	29.6142643609424\\
66.875	0.26502	30.7384934274984\\
66.875	0.26868	31.8848633186377\\
66.875	0.27234	33.0533740343605\\
66.875	0.276	34.2440255746667\\
67.25	0.093	1.83151594231753\\
67.25	0.09666	1.94195049165429\\
67.25	0.10032	2.07452586557445\\
67.25	0.10398	2.22924206407802\\
67.25	0.10764	2.40609908716499\\
67.25	0.1113	2.60509693483535\\
67.25	0.11496	2.82623560708912\\
67.25	0.11862	3.06951510392627\\
67.25	0.12228	3.33493542534685\\
67.25	0.12594	3.62249657135081\\
67.25	0.1296	3.93219854193817\\
67.25	0.13326	4.26404133710893\\
67.25	0.13692	4.61802495686309\\
67.25	0.14058	4.99414940120065\\
67.25	0.14424	5.39241467012162\\
67.25	0.1479	5.81282076362598\\
67.25	0.15156	6.25536768171373\\
67.25	0.15522	6.72005542438489\\
67.25	0.15888	7.20688399163945\\
67.25	0.16254	7.71585338347742\\
67.25	0.1662	8.24696359989877\\
67.25	0.16986	8.80021464090353\\
67.25	0.17352	9.37560650649167\\
67.25	0.17718	9.97313919666322\\
67.25	0.18084	10.5928127114182\\
67.25	0.1845	11.2346270507565\\
67.25	0.18816	11.8985822146783\\
67.25	0.19182	12.5846782031835\\
67.25	0.19548	13.292915016272\\
67.25	0.19914	14.023292653944\\
67.25	0.2028	14.7758111161993\\
67.25	0.20646	15.5504704030381\\
67.25	0.21012	16.3472705144602\\
67.25	0.21378	17.1662114504658\\
67.25	0.21744	18.0072932110547\\
67.25	0.2211	18.870515796227\\
67.25	0.22476	19.7558792059828\\
67.25	0.22842	20.663383440322\\
67.25	0.23208	21.5930284992445\\
67.25	0.23574	22.5448143827505\\
67.25	0.2394	23.5187410908398\\
67.25	0.24306	24.5148086235125\\
67.25	0.24672	25.5330169807687\\
67.25	0.25038	26.5733661626083\\
67.25	0.25404	27.6358561690312\\
67.25	0.2577	28.7204870000375\\
67.25	0.26136	29.8272586556273\\
67.25	0.26502	30.9561711358004\\
67.25	0.26868	32.1072244405569\\
67.25	0.27234	33.2804185698969\\
67.25	0.276	34.4757535238202\\
67.625	0.093	1.83158365041545\\
67.625	0.09666	1.94670161336939\\
67.625	0.10032	2.08396040090674\\
67.625	0.10398	2.24336001302748\\
67.625	0.10764	2.42490044973164\\
67.625	0.1113	2.62858171101918\\
67.625	0.11496	2.85440379689012\\
67.625	0.11862	3.10236670734446\\
67.625	0.12228	3.37247044238221\\
67.625	0.12594	3.66471500200335\\
67.625	0.1296	3.9791003862079\\
67.625	0.13326	4.31562659499585\\
67.625	0.13692	4.67429362836718\\
67.625	0.14058	5.05510148632192\\
67.625	0.14424	5.45805016886007\\
67.625	0.1479	5.88313967598161\\
67.625	0.15156	6.33037000768654\\
67.625	0.15522	6.79974116397489\\
67.625	0.15888	7.29125314484663\\
67.625	0.16254	7.80490595030177\\
67.625	0.1662	8.34069958034031\\
67.625	0.16986	8.89863403496224\\
67.625	0.17352	9.47870931416756\\
67.625	0.17718	10.0809254179563\\
67.625	0.18084	10.7052823463284\\
67.625	0.1845	11.351780099284\\
67.625	0.18816	12.0204186768229\\
67.625	0.19182	12.7111980789453\\
67.625	0.19548	13.424118305651\\
67.625	0.19914	14.1591793569401\\
67.625	0.2028	14.9163812328127\\
67.625	0.20646	15.6957239332686\\
67.625	0.21012	16.4972074583079\\
67.625	0.21378	17.3208318079307\\
67.625	0.21744	18.1665969821368\\
67.625	0.2211	19.0345029809263\\
67.625	0.22476	19.9245498042992\\
67.625	0.22842	20.8367374522556\\
67.625	0.23208	21.7710659247953\\
67.625	0.23574	22.7275352219184\\
67.625	0.2394	23.706145343625\\
67.625	0.24306	24.7068962899149\\
67.625	0.24672	25.7297880607882\\
67.625	0.25038	26.7748206562449\\
67.625	0.25404	27.841994076285\\
67.625	0.2577	28.9313083209086\\
67.625	0.26136	30.0427633901155\\
67.625	0.26502	31.1763592839058\\
67.625	0.26868	32.3320960022795\\
67.625	0.27234	33.5099735452367\\
67.625	0.276	34.7099919127772\\
68	0.093	1.83416179831676\\
68	0.09666	1.9539631748879\\
68	0.10032	2.09590537604243\\
68	0.10398	2.25998840178035\\
68	0.10764	2.44621225210166\\
68	0.1113	2.65457692700641\\
68	0.11496	2.88508242649453\\
68	0.11862	3.13772875056606\\
68	0.12228	3.41251589922098\\
68	0.12594	3.7094438724593\\
68	0.1296	4.02851267028102\\
68	0.13326	4.36972229268615\\
68	0.13692	4.73307273967467\\
68	0.14058	5.11856401124659\\
68	0.14424	5.52619610740192\\
68	0.1479	5.95596902814064\\
68	0.15156	6.40788277346276\\
68	0.15522	6.88193734336828\\
68	0.15888	7.37813273785721\\
68	0.16254	7.89646895692952\\
68	0.1662	8.43694600058523\\
68	0.16986	8.99956386882436\\
68	0.17352	9.58432256164688\\
68	0.17718	10.1912220790528\\
68	0.18084	10.8202624210421\\
68	0.1845	11.4714435876148\\
68	0.18816	12.1447655787709\\
68	0.19182	12.8402283945105\\
68	0.19548	13.5578320348334\\
68	0.19914	14.2975764997397\\
68	0.2028	15.0594617892294\\
68	0.20646	15.8434879033025\\
68	0.21012	16.649654841959\\
68	0.21378	17.4779626051989\\
68	0.21744	18.3284111930223\\
68	0.2211	19.2010006054289\\
68	0.22476	20.0957308424191\\
68	0.22842	21.0126019039926\\
68	0.23208	21.9516137901495\\
68	0.23574	22.9127665008898\\
68	0.2394	23.8960600362135\\
68	0.24306	24.9014943961206\\
68	0.24672	25.9290695806111\\
68	0.25038	26.978785589685\\
68	0.25404	28.0506424233423\\
68	0.2577	29.1446400815831\\
68	0.26136	30.2607785644071\\
68	0.26502	31.3990578718147\\
68	0.26868	32.5594780038056\\
68	0.27234	33.7420389603799\\
68	0.276	34.9467407415376\\
68.375	0.093	1.83925038602151\\
68.375	0.09666	1.96373517620982\\
68.375	0.10032	2.11036079098154\\
68.375	0.10398	2.27912723033664\\
68.375	0.10764	2.47003449427514\\
68.375	0.1113	2.68308258279705\\
68.375	0.11496	2.91827149590237\\
68.375	0.11862	3.17560123359107\\
68.375	0.12228	3.45507179586317\\
68.375	0.12594	3.75668318271868\\
68.375	0.1296	4.08043539415758\\
68.375	0.13326	4.42632843017988\\
68.375	0.13692	4.79436229078559\\
68.375	0.14058	5.1845369759747\\
68.375	0.14424	5.59685248574721\\
68.375	0.1479	6.0313088201031\\
68.375	0.15156	6.48790597904241\\
68.375	0.15522	6.96664396256511\\
68.375	0.15888	7.46752277067121\\
68.375	0.16254	7.9905424033607\\
68.375	0.1662	8.5357028606336\\
68.375	0.16986	9.10300414248991\\
68.375	0.17352	9.69244624892961\\
68.375	0.17718	10.3040291799527\\
68.375	0.18084	10.9377529355592\\
68.375	0.1845	11.5936175157491\\
68.375	0.18816	12.2716229205224\\
68.375	0.19182	12.9717691498791\\
68.375	0.19548	13.6940562038192\\
68.375	0.19914	14.4384840823427\\
68.375	0.2028	15.2050527854496\\
68.375	0.20646	15.9937623131399\\
68.375	0.21012	16.8046126654136\\
68.375	0.21378	17.6376038422707\\
68.375	0.21744	18.4927358437112\\
68.375	0.2211	19.370008669735\\
68.375	0.22476	20.2694223203423\\
68.375	0.22842	21.190976795533\\
68.375	0.23208	22.1346720953071\\
68.375	0.23574	23.1005082196646\\
68.375	0.2394	24.0884851686055\\
68.375	0.24306	25.0986029421298\\
68.375	0.24672	26.1308615402375\\
68.375	0.25038	27.1852609629286\\
68.375	0.25404	28.261801210203\\
68.375	0.2577	29.3604822820609\\
68.375	0.26136	30.4813041785022\\
68.375	0.26502	31.6242668995269\\
68.375	0.26868	32.789370445135\\
68.375	0.27234	33.9766148153265\\
68.375	0.276	35.1860000101014\\
68.75	0.093	1.84684941352967\\
68.75	0.09666	1.97601761733517\\
68.75	0.10032	2.12732664572407\\
68.75	0.10398	2.30077649869634\\
68.75	0.10764	2.49636717625203\\
68.75	0.1113	2.71409867839113\\
68.75	0.11496	2.95397100511361\\
68.75	0.11862	3.21598415641951\\
68.75	0.12228	3.50013813230878\\
68.75	0.12594	3.80643293278148\\
68.75	0.1296	4.13486855783755\\
68.75	0.13326	4.48544500747704\\
68.75	0.13692	4.85816228169993\\
68.75	0.14058	5.25302038050621\\
68.75	0.14424	5.6700193038959\\
68.75	0.1479	6.10915905186897\\
68.75	0.15156	6.57043962442546\\
68.75	0.15522	7.05386102156534\\
68.75	0.15888	7.55942324328863\\
68.75	0.16254	8.0871262895953\\
68.75	0.1662	8.63697016048538\\
68.75	0.16986	9.20895485595887\\
68.75	0.17352	9.80308037601575\\
68.75	0.17718	10.419346720656\\
68.75	0.18084	11.0577538898797\\
68.75	0.1845	11.7183018836868\\
68.75	0.18816	12.4009907020773\\
68.75	0.19182	13.1058203450511\\
68.75	0.19548	13.8327908126084\\
68.75	0.19914	14.5819021047491\\
68.75	0.2028	15.3531542214732\\
68.75	0.20646	16.1465471627806\\
68.75	0.21012	16.9620809286715\\
68.75	0.21378	17.7997555191458\\
68.75	0.21744	18.6595709342035\\
68.75	0.2211	19.5415271738445\\
68.75	0.22476	20.445624238069\\
68.75	0.22842	21.3718621268769\\
68.75	0.23208	22.3202408402681\\
68.75	0.23574	23.2907603782428\\
68.75	0.2394	24.2834207408009\\
68.75	0.24306	25.2982219279423\\
68.75	0.24672	26.3351639396672\\
68.75	0.25038	27.3942467759755\\
68.75	0.25404	28.4754704368671\\
68.75	0.2577	29.5788349223422\\
68.75	0.26136	30.7043402324007\\
68.75	0.26502	31.8519863670426\\
68.75	0.26868	33.0217733262678\\
68.75	0.27234	34.2137011100765\\
68.75	0.276	35.4277697184686\\
69.125	0.093	1.85695888084125\\
69.125	0.09666	1.99081049826393\\
69.125	0.10032	2.14680294027001\\
69.125	0.10398	2.32493620685947\\
69.125	0.10764	2.52521029803233\\
69.125	0.1113	2.74762521378861\\
69.125	0.11496	2.99218095412828\\
69.125	0.11862	3.25887751905135\\
69.125	0.12228	3.54771490855782\\
69.125	0.12594	3.85869312264769\\
69.125	0.1296	4.19181216132094\\
69.125	0.13326	4.54707202457761\\
69.125	0.13692	4.92447271241767\\
69.125	0.14058	5.32401422484115\\
69.125	0.14424	5.74569656184801\\
69.125	0.1479	6.18951972343827\\
69.125	0.15156	6.65548370961193\\
69.125	0.15522	7.143588520369\\
69.125	0.15888	7.65383415570947\\
69.125	0.16254	8.18622061563332\\
69.125	0.1662	8.74074790014058\\
69.125	0.16986	9.31741600923125\\
69.125	0.17352	9.91622494290531\\
69.125	0.17718	10.5371747011628\\
69.125	0.18084	11.1802652840036\\
69.125	0.1845	11.8454966914279\\
69.125	0.18816	12.5328689234355\\
69.125	0.19182	13.2423819800266\\
69.125	0.19548	13.9740358612011\\
69.125	0.19914	14.7278305669589\\
69.125	0.2028	15.5037660973002\\
69.125	0.20646	16.3018424522248\\
69.125	0.21012	17.1220596317329\\
69.125	0.21378	17.9644176358243\\
69.125	0.21744	18.8289164644992\\
69.125	0.2211	19.7155561177574\\
69.125	0.22476	20.6243365955991\\
69.125	0.22842	21.5552578980242\\
69.125	0.23208	22.5083200250326\\
69.125	0.23574	23.4835229766245\\
69.125	0.2394	24.4808667527997\\
69.125	0.24306	25.5003513535583\\
69.125	0.24672	26.5419767789004\\
69.125	0.25038	27.6057430288259\\
69.125	0.25404	28.6916501033347\\
69.125	0.2577	29.799698002427\\
69.125	0.26136	30.9298867261026\\
69.125	0.26502	32.0822162743617\\
69.125	0.26868	33.2566866472041\\
69.125	0.27234	34.4532978446299\\
69.125	0.276	35.6720498666392\\
69.5	0.093	1.86957878795623\\
69.5	0.09666	2.00811381899607\\
69.5	0.10032	2.16878967461932\\
69.5	0.10398	2.35160635482598\\
69.5	0.10764	2.55656385961604\\
69.5	0.1113	2.78366218898948\\
69.5	0.11496	3.03290134294633\\
69.5	0.11862	3.30428132148657\\
69.5	0.12228	3.59780212461023\\
69.5	0.12594	3.91346375231728\\
69.5	0.1296	4.25126620460772\\
69.5	0.13326	4.61120948148158\\
69.5	0.13692	4.99329358293881\\
69.5	0.14058	5.39751850897946\\
69.5	0.14424	5.82388425960351\\
69.5	0.1479	6.27239083481096\\
69.5	0.15156	6.74303823460179\\
69.5	0.15522	7.23582645897604\\
69.5	0.15888	7.75075550793369\\
69.5	0.16254	8.28782538147473\\
69.5	0.1662	8.84703607959917\\
69.5	0.16986	9.42838760230701\\
69.5	0.17352	10.0318799495982\\
69.5	0.17718	10.6575131214729\\
69.5	0.18084	11.3052871179309\\
69.5	0.1845	11.9752019389724\\
69.5	0.18816	12.6672575845972\\
69.5	0.19182	13.3814540548055\\
69.5	0.19548	14.1177913495971\\
69.5	0.19914	14.8762694689721\\
69.5	0.2028	15.6568884129306\\
69.5	0.20646	16.4596481814724\\
69.5	0.21012	17.2845487745976\\
69.5	0.21378	18.1315901923063\\
69.5	0.21744	19.0007724345983\\
69.5	0.2211	19.8920955014737\\
69.5	0.22476	20.8055593929326\\
69.5	0.22842	21.7411641089748\\
69.5	0.23208	22.6989096496004\\
69.5	0.23574	23.6787960148095\\
69.5	0.2394	24.6808232046019\\
69.5	0.24306	25.7049912189777\\
69.5	0.24672	26.751300057937\\
69.5	0.25038	27.8197497214796\\
69.5	0.25404	28.9103402096056\\
69.5	0.2577	30.0230715223151\\
69.5	0.26136	31.1579436596079\\
69.5	0.26502	32.3149566214841\\
69.5	0.26868	33.4941104079437\\
69.5	0.27234	34.6954050189868\\
69.5	0.276	35.9188404546132\\
69.875	0.093	1.88470913487464\\
69.875	0.09666	2.02792757953166\\
69.875	0.10032	2.19328684877209\\
69.875	0.10398	2.38078694259592\\
69.875	0.10764	2.59042786100316\\
69.875	0.1113	2.82220960399379\\
69.875	0.11496	3.07613217156782\\
69.875	0.11862	3.35219556372524\\
69.875	0.12228	3.65039978046608\\
69.875	0.12594	3.9707448217903\\
69.875	0.1296	4.31323068769794\\
69.875	0.13326	4.67785737818897\\
69.875	0.13692	5.06462489326339\\
69.875	0.14058	5.47353323292122\\
69.875	0.14424	5.90458239716244\\
69.875	0.1479	6.35777238598708\\
69.875	0.15156	6.83310319939509\\
69.875	0.15522	7.33057483738652\\
69.875	0.15888	7.85018729996134\\
69.875	0.16254	8.39194058711957\\
69.875	0.1662	8.9558346988612\\
69.875	0.16986	9.54186963518621\\
69.875	0.17352	10.1500453960946\\
69.875	0.17718	10.7803619815864\\
69.875	0.18084	11.4328193916617\\
69.875	0.1845	12.1074176263203\\
69.875	0.18816	12.8041566855623\\
69.875	0.19182	13.5230365693877\\
69.875	0.19548	14.2640572777965\\
69.875	0.19914	15.0272188107888\\
69.875	0.2028	15.8125211683644\\
69.875	0.20646	16.6199643505234\\
69.875	0.21012	17.4495483572658\\
69.875	0.21378	18.3012731885916\\
69.875	0.21744	19.1751388445009\\
69.875	0.2211	20.0711453249935\\
69.875	0.22476	20.9892926300695\\
69.875	0.22842	21.9295807597289\\
69.875	0.23208	22.8920097139717\\
69.875	0.23574	23.8765794927979\\
69.875	0.2394	24.8832900962075\\
69.875	0.24306	25.9121415242005\\
69.875	0.24672	26.963133776777\\
69.875	0.25038	28.0362668539368\\
69.875	0.25404	29.13154075568\\
69.875	0.2577	30.2489554820066\\
69.875	0.26136	31.3885110329166\\
69.875	0.26502	32.55020740841\\
69.875	0.26868	33.7340446084868\\
69.875	0.27234	34.940022633147\\
69.875	0.276	36.1681414823906\\
70.25	0.093	1.90234992159644\\
70.25	0.09666	2.05025177987065\\
70.25	0.10032	2.22029446272826\\
70.25	0.10398	2.41247797016928\\
70.25	0.10764	2.62680230219369\\
70.25	0.1113	2.8632674588015\\
70.25	0.11496	3.12187343999272\\
70.25	0.11862	3.40262024576731\\
70.25	0.12228	3.70550787612533\\
70.25	0.12594	4.03053633106674\\
70.25	0.1296	4.37770561059155\\
70.25	0.13326	4.74701571469977\\
70.25	0.13692	5.13846664339136\\
70.25	0.14058	5.55205839666637\\
70.25	0.14424	5.98779097452478\\
70.25	0.1479	6.4456643769666\\
70.25	0.15156	6.92567860399179\\
70.25	0.15522	7.4278336556004\\
70.25	0.15888	7.95212953179241\\
70.25	0.16254	8.49856623256782\\
70.25	0.1662	9.06714375792662\\
70.25	0.16986	9.65786210786882\\
70.25	0.17352	10.2707212823944\\
70.25	0.17718	10.9057212815034\\
70.25	0.18084	11.5628621051958\\
70.25	0.1845	12.2421437534716\\
70.25	0.18816	12.9435662263308\\
70.25	0.19182	13.6671295237734\\
70.25	0.19548	14.4128336457994\\
70.25	0.19914	15.1806785924088\\
70.25	0.2028	15.9706643636016\\
70.25	0.20646	16.7827909593778\\
70.25	0.21012	17.6170583797374\\
70.25	0.21378	18.4734666246804\\
70.25	0.21744	19.3520156942068\\
70.25	0.2211	20.2527055883166\\
70.25	0.22476	21.1755363070098\\
70.25	0.22842	22.1205078502864\\
70.25	0.23208	23.0876202181464\\
70.25	0.23574	24.0768734105898\\
70.25	0.2394	25.0882674276166\\
70.25	0.24306	26.1218022692268\\
70.25	0.24672	27.1774779354204\\
70.25	0.25038	28.2552944261974\\
70.25	0.25404	29.3552517415578\\
70.25	0.2577	30.4773498815015\\
70.25	0.26136	31.6215888460288\\
70.25	0.26502	32.7879686351393\\
70.25	0.26868	33.9764892488333\\
70.25	0.27234	35.1871506871107\\
70.25	0.276	36.4199529499715\\
70.625	0.093	1.92250114812165\\
70.625	0.09666	2.07508642001304\\
70.625	0.10032	2.24981251648785\\
70.625	0.10398	2.44667943754604\\
70.625	0.10764	2.66568718318761\\
70.625	0.1113	2.90683575341262\\
70.625	0.11496	3.17012514822101\\
70.625	0.11862	3.45555536761281\\
70.625	0.12228	3.76312641158799\\
70.625	0.12594	4.09283828014659\\
70.625	0.1296	4.44469097328856\\
70.625	0.13326	4.81868449101395\\
70.625	0.13692	5.21481883332275\\
70.625	0.14058	5.63309400021494\\
70.625	0.14424	6.07350999169053\\
70.625	0.1479	6.53606680774951\\
70.625	0.15156	7.0207644483919\\
70.625	0.15522	7.52760291361769\\
70.625	0.15888	8.05658220342688\\
70.625	0.16254	8.60770231781945\\
70.625	0.1662	9.18096325679544\\
70.625	0.16986	9.77636502035483\\
70.625	0.17352	10.3939076084976\\
70.625	0.17718	11.0335910212238\\
70.625	0.18084	11.6954152585334\\
70.625	0.1845	12.3793803204264\\
70.625	0.18816	13.0854862069027\\
70.625	0.19182	13.8137329179625\\
70.625	0.19548	14.5641204536057\\
70.625	0.19914	15.3366488138323\\
70.625	0.2028	16.1313179986423\\
70.625	0.20646	16.9481280080356\\
70.625	0.21012	17.7870788420124\\
70.625	0.21378	18.6481705005726\\
70.625	0.21744	19.5314029837162\\
70.625	0.2211	20.4367762914431\\
70.625	0.22476	21.3642904237535\\
70.625	0.22842	22.3139453806473\\
70.625	0.23208	23.2857411621245\\
70.625	0.23574	24.2796777681851\\
70.625	0.2394	25.295755198829\\
70.625	0.24306	26.3339734540564\\
70.625	0.24672	27.3943325338672\\
70.625	0.25038	28.4768324382614\\
70.625	0.25404	29.5814731672389\\
70.625	0.2577	30.7082547207999\\
70.625	0.26136	31.8571770989443\\
70.625	0.26502	33.028240301672\\
70.625	0.26868	34.2214443289832\\
70.625	0.27234	35.4367891808777\\
70.625	0.276	36.6742748573557\\
71	0.093	1.94516281445028\\
71	0.09666	2.10243149995885\\
71	0.10032	2.28184101005084\\
71	0.10398	2.48339134472621\\
71	0.10764	2.70708250398497\\
71	0.1113	2.95291448782715\\
71	0.11496	3.22088729625272\\
71	0.11862	3.5110009292617\\
71	0.12228	3.82325538685406\\
71	0.12594	4.15765066902984\\
71	0.1296	4.514186775789\\
71	0.13326	4.89286370713157\\
71	0.13692	5.29368146305755\\
71	0.14058	5.71664004356692\\
71	0.14424	6.16173944865969\\
71	0.1479	6.62897967833585\\
71	0.15156	7.11836073259542\\
71	0.15522	7.62988261143839\\
71	0.15888	8.16354531486476\\
71	0.16254	8.71934884287452\\
71	0.1662	9.29729319546768\\
71	0.16986	9.89737837264425\\
71	0.17352	10.5196043744042\\
71	0.17718	11.1639712007476\\
71	0.18084	11.8304788516743\\
71	0.1845	12.5191273271845\\
71	0.18816	13.2299166272781\\
71	0.19182	13.962846751955\\
71	0.19548	14.7179177012154\\
71	0.19914	15.4951294750592\\
71	0.2028	16.2944820734863\\
71	0.20646	17.1159754964969\\
71	0.21012	17.9596097440908\\
71	0.21378	18.8253848162682\\
71	0.21744	19.713300713029\\
71	0.2211	20.6233574343731\\
71	0.22476	21.5555549803006\\
71	0.22842	22.5098933508116\\
71	0.23208	23.486372545906\\
71	0.23574	24.4849925655837\\
71	0.2394	25.5057534098449\\
71	0.24306	26.5486550786894\\
71	0.24672	27.6136975721174\\
71	0.25038	28.7008808901288\\
71	0.25404	29.8102050327235\\
71	0.2577	30.9416699999017\\
71	0.26136	32.0952757916632\\
71	0.26502	33.2710224080082\\
71	0.26868	34.4689098489365\\
71	0.27234	35.6889381144482\\
71	0.276	36.9311072045434\\
71.375	0.093	1.97033492058233\\
71.375	0.09666	2.1322870197081\\
71.375	0.10032	2.31637994341726\\
71.375	0.10398	2.5226136917098\\
71.375	0.10764	2.75098826458575\\
71.375	0.1113	3.00150366204511\\
71.375	0.11496	3.27415988408787\\
71.375	0.11862	3.56895693071402\\
71.375	0.12228	3.88589480192357\\
71.375	0.12594	4.22497349771652\\
71.375	0.1296	4.58619301809286\\
71.375	0.13326	4.96955336305262\\
71.375	0.13692	5.37505453259577\\
71.375	0.14058	5.80269652672233\\
71.375	0.14424	6.25247934543228\\
71.375	0.1479	6.72440298872561\\
71.375	0.15156	7.21846745660237\\
71.375	0.15522	7.73467274906252\\
71.375	0.15888	8.27301886610607\\
71.375	0.16254	8.833505807733\\
71.375	0.1662	9.41613357394336\\
71.375	0.16986	10.0209021647371\\
71.375	0.17352	10.6478115801143\\
71.375	0.17718	11.2968618200748\\
71.375	0.18084	11.9680528846187\\
71.375	0.1845	12.6613847737461\\
71.375	0.18816	13.3768574874568\\
71.375	0.19182	14.114471025751\\
71.375	0.19548	14.8742253886285\\
71.375	0.19914	15.6561205760894\\
71.375	0.2028	16.4601565881338\\
71.375	0.20646	17.2863334247615\\
71.375	0.21012	18.1346510859727\\
71.375	0.21378	19.0051095717672\\
71.375	0.21744	19.8977088821452\\
71.375	0.2211	20.8124490171065\\
71.375	0.22476	21.7493299766512\\
71.375	0.22842	22.7083517607794\\
71.375	0.23208	23.6895143694909\\
71.375	0.23574	24.6928178027859\\
71.375	0.2394	25.7182620606642\\
71.375	0.24306	26.7658471431259\\
71.375	0.24672	27.835573050171\\
71.375	0.25038	28.9274397817996\\
71.375	0.25404	30.0414473380115\\
71.375	0.2577	31.1775957188068\\
71.375	0.26136	32.3358849241856\\
71.375	0.26502	33.5163149541477\\
71.375	0.26868	34.7188858086932\\
71.375	0.27234	35.9435974878222\\
71.375	0.276	37.1904499915345\\
71.75	0.093	1.99801746651779\\
71.75	0.09666	2.16465297926072\\
71.75	0.10032	2.35342931658704\\
71.75	0.10398	2.56434647849679\\
71.75	0.10764	2.79740446498993\\
71.75	0.1113	3.05260327606645\\
71.75	0.11496	3.32994291172639\\
71.75	0.11862	3.62942337196971\\
71.75	0.12228	3.95104465679646\\
71.75	0.12594	4.29480676620659\\
71.75	0.1296	4.66070970020012\\
71.75	0.13326	5.04875345877706\\
71.75	0.13692	5.45893804193738\\
71.75	0.14058	5.89126344968112\\
71.75	0.14424	6.34572968200825\\
71.75	0.1479	6.82233673891879\\
71.75	0.15156	7.3210846204127\\
71.75	0.15522	7.84197332649003\\
71.75	0.15888	8.38500285715077\\
71.75	0.16254	8.95017321239489\\
71.75	0.1662	9.53748439222243\\
71.75	0.16986	10.1469363966333\\
71.75	0.17352	10.7785292256277\\
71.75	0.17718	11.4322628792054\\
71.75	0.18084	12.1081373573665\\
71.75	0.1845	12.806152660111\\
71.75	0.18816	13.526308787439\\
71.75	0.19182	14.2686057393503\\
71.75	0.19548	15.033043515845\\
71.75	0.19914	15.8196221169231\\
71.75	0.2028	16.6283415425847\\
71.75	0.20646	17.4592017928296\\
71.75	0.21012	18.3122028676579\\
71.75	0.21378	19.1873447670696\\
71.75	0.21744	20.0846274910648\\
71.75	0.2211	21.0040510396433\\
71.75	0.22476	21.9456154128052\\
71.75	0.22842	22.9093206105505\\
71.75	0.23208	23.8951666328792\\
71.75	0.23574	24.9031534797913\\
71.75	0.2394	25.9332811512868\\
71.75	0.24306	26.9855496473658\\
71.75	0.24672	28.0599589680281\\
71.75	0.25038	29.1565091132738\\
71.75	0.25404	30.2752000831029\\
71.75	0.2577	31.4160318775154\\
71.75	0.26136	32.5790044965113\\
71.75	0.26502	33.7641179400906\\
71.75	0.26868	34.9713722082533\\
71.75	0.27234	36.2007673009995\\
71.75	0.276	37.452303218329\\
72.125	0.093	2.02821045225666\\
72.125	0.09666	2.19952937861677\\
72.125	0.10032	2.39298912956029\\
72.125	0.10398	2.60858970508721\\
72.125	0.10764	2.84633110519753\\
72.125	0.1113	3.10621332989124\\
72.125	0.11496	3.38823637916836\\
72.125	0.11862	3.69240025302886\\
72.125	0.12228	4.01870495147278\\
72.125	0.12594	4.3671504745001\\
72.125	0.1296	4.73773682211082\\
72.125	0.13326	5.13046399430493\\
72.125	0.13692	5.54533199108243\\
72.125	0.14058	5.98234081244334\\
72.125	0.14424	6.44149045838766\\
72.125	0.1479	6.92278092891538\\
72.125	0.15156	7.42621222402648\\
72.125	0.15522	7.95178434372099\\
72.125	0.15888	8.4994972879989\\
72.125	0.16254	9.06935105686021\\
72.125	0.1662	9.66134565030493\\
72.125	0.16986	10.275481068333\\
72.125	0.17352	10.9117573109445\\
72.125	0.17718	11.5701743781394\\
72.125	0.18084	12.2507322699177\\
72.125	0.1845	12.9534309862795\\
72.125	0.18816	13.6782705272246\\
72.125	0.19182	14.4252508927531\\
72.125	0.19548	15.194372082865\\
72.125	0.19914	15.9856340975603\\
72.125	0.2028	16.799036936839\\
72.125	0.20646	17.6345806007011\\
72.125	0.21012	18.4922650891466\\
72.125	0.21378	19.3720904021755\\
72.125	0.21744	20.2740565397878\\
72.125	0.2211	21.1981635019835\\
72.125	0.22476	22.1444112887626\\
72.125	0.22842	23.1127999001251\\
72.125	0.23208	24.103329336071\\
72.125	0.23574	25.1159995966003\\
72.125	0.2394	26.150810681713\\
72.125	0.24306	27.2077625914091\\
72.125	0.24672	28.2868553256886\\
72.125	0.25038	29.3880888845515\\
72.125	0.25404	30.5114632679977\\
72.125	0.2577	31.6569784760274\\
72.125	0.26136	32.8246345086405\\
72.125	0.26502	34.014431365837\\
72.125	0.26868	35.2263690476169\\
72.125	0.27234	36.4604475539802\\
72.125	0.276	37.7166668849269\\
72.5	0.093	2.06091387779894\\
72.5	0.09666	2.23691621777623\\
72.5	0.10032	2.43505938233693\\
72.5	0.10398	2.65534337148103\\
72.5	0.10764	2.89776818520853\\
72.5	0.1113	3.16233382351942\\
72.5	0.11496	3.44904028641372\\
72.5	0.11862	3.75788757389141\\
72.5	0.12228	4.08887568595251\\
72.5	0.12594	4.442004622597\\
72.5	0.1296	4.8172743838249\\
72.5	0.13326	5.2146849696362\\
72.5	0.13692	5.63423638003088\\
72.5	0.14058	6.07592861500897\\
72.5	0.14424	6.53976167457047\\
72.5	0.1479	7.02573555871537\\
72.5	0.15156	7.53385026744365\\
72.5	0.15522	8.06410580075534\\
72.5	0.15888	8.61650215865043\\
72.5	0.16254	9.19103934112893\\
72.5	0.1662	9.78771734819082\\
72.5	0.16986	10.4065361798361\\
72.5	0.17352	11.0474958360648\\
72.5	0.17718	11.7105963168769\\
72.5	0.18084	12.3958376222724\\
72.5	0.1845	13.1032197522512\\
72.5	0.18816	13.8327427068135\\
72.5	0.19182	14.5844064859592\\
72.5	0.19548	15.3582110896883\\
72.5	0.19914	16.1541565180008\\
72.5	0.2028	16.9722427708967\\
72.5	0.20646	17.812469848376\\
72.5	0.21012	18.6748377504386\\
72.5	0.21378	19.5593464770847\\
72.5	0.21744	20.4659960283142\\
72.5	0.2211	21.3947864041271\\
72.5	0.22476	22.3457176045234\\
72.5	0.22842	23.318789629503\\
72.5	0.23208	24.3140024790661\\
72.5	0.23574	25.3313561532126\\
72.5	0.2394	26.3708506519425\\
72.5	0.24306	27.4324859752558\\
72.5	0.24672	28.5162621231524\\
72.5	0.25038	29.6221790956325\\
72.5	0.25404	30.750236892696\\
72.5	0.2577	31.9004355143429\\
72.5	0.26136	33.0727749605731\\
72.5	0.26502	34.2672552313868\\
72.5	0.26868	35.4838763267839\\
72.5	0.27234	36.7226382467643\\
72.5	0.276	37.9835409913282\\
72.875	0.093	2.09612774314465\\
72.875	0.09666	2.27681349673912\\
72.875	0.10032	2.47964007491699\\
72.875	0.10398	2.70460747767827\\
72.875	0.10764	2.95171570502295\\
72.875	0.1113	3.22096475695103\\
72.875	0.11496	3.51235463346251\\
72.875	0.11862	3.82588533455737\\
72.875	0.12228	4.16155686023565\\
72.875	0.12594	4.51936921049733\\
72.875	0.1296	4.89932238534242\\
72.875	0.13326	5.30141638477089\\
72.875	0.13692	5.72565120878275\\
72.875	0.14058	6.17202685737803\\
72.875	0.14424	6.6405433305567\\
72.875	0.1479	7.13120062831878\\
72.875	0.15156	7.64399875066424\\
72.875	0.15522	8.17893769759312\\
72.875	0.15888	8.73601746910539\\
72.875	0.16254	9.31523806520106\\
72.875	0.1662	9.91659948588014\\
72.875	0.16986	10.5401017311426\\
72.875	0.17352	11.1857448009885\\
72.875	0.17718	11.8535286954177\\
72.875	0.18084	12.5434534144304\\
72.875	0.1845	13.2555189580265\\
72.875	0.18816	13.9897253262059\\
72.875	0.19182	14.7460725189688\\
72.875	0.19548	15.5245605363151\\
72.875	0.19914	16.3251893782447\\
72.875	0.2028	17.1479590447578\\
72.875	0.20646	17.9928695358543\\
72.875	0.21012	18.8599208515341\\
72.875	0.21378	19.7491129917974\\
72.875	0.21744	20.660445956644\\
72.875	0.2211	21.5939197460741\\
72.875	0.22476	22.5495343600876\\
72.875	0.22842	23.5272897986844\\
72.875	0.23208	24.5271860618647\\
72.875	0.23574	25.5492231496284\\
72.875	0.2394	26.5934010619754\\
72.875	0.24306	27.6597197989059\\
72.875	0.24672	28.7481793604197\\
72.875	0.25038	29.858779746517\\
72.875	0.25404	30.9915209571976\\
72.875	0.2577	32.1464029924617\\
72.875	0.26136	33.3234258523091\\
72.875	0.26502	34.52258953674\\
72.875	0.26868	35.7438940457542\\
72.875	0.27234	36.9873393793519\\
72.875	0.276	38.252925537533\\
73.25	0.093	2.13385204829373\\
73.25	0.09666	2.31922121550539\\
73.25	0.10032	2.52673120730046\\
73.25	0.10398	2.75638202367891\\
73.25	0.10764	3.00817366464076\\
73.25	0.1113	3.28210613018603\\
73.25	0.11496	3.57817942031469\\
73.25	0.11862	3.89639353502675\\
73.25	0.12228	4.2367484743222\\
73.25	0.12594	4.59924423820105\\
73.25	0.1296	4.9838808266633\\
73.25	0.13326	5.39065823970896\\
73.25	0.13692	5.81957647733802\\
73.25	0.14058	6.27063553955047\\
73.25	0.14424	6.74383542634633\\
73.25	0.1479	7.23917613772557\\
73.25	0.15156	7.75665767368823\\
73.25	0.15522	8.29628003423428\\
73.25	0.15888	8.85804321936374\\
73.25	0.16254	9.44194722907658\\
73.25	0.1662	10.0479920633728\\
73.25	0.16986	10.6761777222525\\
73.25	0.17352	11.3265042057155\\
73.25	0.17718	11.998971513762\\
73.25	0.18084	12.6935796463918\\
73.25	0.1845	13.4103286036051\\
73.25	0.18816	14.1492183854017\\
73.25	0.19182	14.9102489917818\\
73.25	0.19548	15.6934204227452\\
73.25	0.19914	16.4987326782921\\
73.25	0.2028	17.3261857584223\\
73.25	0.20646	18.175779663136\\
73.25	0.21012	19.047514392433\\
73.25	0.21378	19.9413899463134\\
73.25	0.21744	20.8574063247773\\
73.25	0.2211	21.7955635278245\\
73.25	0.22476	22.7558615554552\\
73.25	0.22842	23.7383004076692\\
73.25	0.23208	24.7428800844667\\
73.25	0.23574	25.7696005858475\\
73.25	0.2394	26.8184619118117\\
73.25	0.24306	27.8894640623594\\
73.25	0.24672	28.9826070374904\\
73.25	0.25038	30.0978908372048\\
73.25	0.25404	31.2353154615027\\
73.25	0.2577	32.3948809103839\\
73.25	0.26136	33.5765871838486\\
73.25	0.26502	34.7804342818966\\
73.25	0.26868	36.006422204528\\
73.25	0.27234	37.2545509517429\\
73.25	0.276	38.5248205235411\\
73.625	0.093	2.17408679324625\\
73.625	0.09666	2.3641393740751\\
73.625	0.10032	2.57633277948735\\
73.625	0.10398	2.81066700948298\\
73.625	0.10764	3.06714206406201\\
73.625	0.1113	3.34575794322445\\
73.625	0.11496	3.6465146469703\\
73.625	0.11862	3.96941217529954\\
73.625	0.12228	4.31445052821217\\
73.625	0.12594	4.68162970570821\\
73.625	0.1296	5.07094970778763\\
73.625	0.13326	5.48241053445047\\
73.625	0.13692	5.91601218569672\\
73.625	0.14058	6.37175466152635\\
73.625	0.14424	6.84963796193939\\
73.625	0.1479	7.34966208693582\\
73.625	0.15156	7.87182703651565\\
73.625	0.15522	8.41613281067889\\
73.625	0.15888	8.98257940942552\\
73.625	0.16254	9.57116683275554\\
73.625	0.1662	10.181895080669\\
73.625	0.16986	10.8147641531658\\
73.625	0.17352	11.469774050246\\
73.625	0.17718	12.1469247719097\\
73.625	0.18084	12.8462163181567\\
73.625	0.1845	13.5676486889871\\
73.625	0.18816	14.311221884401\\
73.625	0.19182	15.0769359043982\\
73.625	0.19548	15.8647907489788\\
73.625	0.19914	16.6747864181428\\
73.625	0.2028	17.5069229118903\\
73.625	0.20646	18.3612002302211\\
73.625	0.21012	19.2376183731353\\
73.625	0.21378	20.1361773406329\\
73.625	0.21744	21.056877132714\\
73.625	0.2211	21.9997177493784\\
73.625	0.22476	22.9646991906262\\
73.625	0.22842	23.9518214564574\\
73.625	0.23208	24.9610845468721\\
73.625	0.23574	25.9924884618701\\
73.625	0.2394	27.0460332014515\\
73.625	0.24306	28.1217187656163\\
73.625	0.24672	29.2195451543645\\
73.625	0.25038	30.3395123676962\\
73.625	0.25404	31.4816204056112\\
73.625	0.2577	32.6458692681096\\
73.625	0.26136	33.8322589551914\\
73.625	0.26502	35.0407894668566\\
73.625	0.26868	36.2714608031052\\
73.625	0.27234	37.5242729639373\\
73.625	0.276	38.7992259493527\\
74	0.093	2.21683197800218\\
74	0.09666	2.4115679724482\\
74	0.10032	2.62844479147764\\
74	0.10398	2.86746243509045\\
74	0.10764	3.12862090328666\\
74	0.1113	3.41192019606629\\
74	0.11496	3.71736031342932\\
74	0.11862	4.04494125537573\\
74	0.12228	4.39466302190555\\
74	0.12594	4.76652561301876\\
74	0.1296	5.16052902871537\\
74	0.13326	5.57667326899539\\
74	0.13692	6.01495833385881\\
74	0.14058	6.47538422330563\\
74	0.14424	6.95795093733584\\
74	0.1479	7.46265847594945\\
74	0.15156	7.98950683914647\\
74	0.15522	8.53849602692689\\
74	0.15888	9.1096260392907\\
74	0.16254	9.70289687623791\\
74	0.1662	10.3183085377685\\
74	0.16986	10.9558610238825\\
74	0.17352	11.61555433458\\
74	0.17718	12.2973884698608\\
74	0.18084	13.001363429725\\
74	0.1845	13.7274792141726\\
74	0.18816	14.4757358232036\\
74	0.19182	15.246133256818\\
74	0.19548	16.0386715150158\\
74	0.19914	16.853350597797\\
74	0.2028	17.6901705051616\\
74	0.20646	18.5491312371096\\
74	0.21012	19.430232793641\\
74	0.21378	20.3334751747558\\
74	0.21744	21.258858380454\\
74	0.2211	22.2063824107356\\
74	0.22476	23.1760472656006\\
74	0.22842	24.167852945049\\
74	0.23208	25.1817994490808\\
74	0.23574	26.2178867776961\\
74	0.2394	27.2761149308946\\
74	0.24306	28.3564839086767\\
74	0.24672	29.4589937110421\\
74	0.25038	30.5836443379909\\
74	0.25404	31.730435789523\\
74	0.2577	32.8993680656387\\
74	0.26136	34.0904411663376\\
74	0.26502	35.3036550916201\\
74	0.26868	36.5390098414858\\
74	0.27234	37.7965054159351\\
74	0.276	39.0761418149676\\
};
\end{axis}
\end{tikzpicture}%
	\caption{Parameter maps reconstructed with LS surface fitting to the internal parameters estimated from the data sample of length 4000.}
\end{figure}
\end{document}