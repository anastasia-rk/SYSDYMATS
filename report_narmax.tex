\documentclass[a4paper,11pt,twoside]{article}
%%%%%%%%%%%%%%%%%%%%%%%%%%%%%%%%%%%%%%%%%%%%%%%%%%%%%%%%%%%%%%%%%%%%%%%%%%%
% Packages
\usepackage[final]{pdfpages}
\usepackage{verbatim}
\usepackage{inputenc}
\usepackage{graphicx} 
\usepackage{amsmath,amssymb,mathrsfs,amsfonts}
\usepackage{mathtools}
\usepackage{amsthm}
\usepackage{mathtools}
\usepackage{calrsfs}
\usepackage{graphicx}
\usepackage{subfig}
\usepackage{eucal}    
\usepackage{amssymb}  
\usepackage{pifont}
\usepackage{color} 
\usepackage{cancel}
\usepackage[toc,page]{appendix}
%\usepackage[usenames,dvipsnames,table]{xcolor}
\usepackage{pgfplots}
\pgfplotsset{every axis/.append style={line width=0.5pt},label style={font=\scriptsize},tick label style={font=\scriptsize},x tick label style={/pgf/number format/.cd,fixed,precision=3, set thousands separator={}},z tick label style={/pgf/number format/.cd,fixed,precision=3, set thousands separator={}}}
\usetikzlibrary{shapes,shadows,arrows,backgrounds,patterns,positioning,automata,calc,decorations.markings,decorations.pathreplacing,bayesnet,arrows.meta}
\usepackage{varwidth}
\usepackage{lscape}
\usepackage{array} 
\usepackage[colorlinks=false,pdfborder={0 0 0}]{hyperref}
\usepackage{tabularx}
\usepackage{textcomp}
\usepackage{multicol} 
\usepackage{booktabs}
\usepackage{multirow}
\usepackage[font=small,labelfont=bf]{caption}                                                           
\usepackage{textcase}
\usepackage{bbm} 
\usepackage{fancyhdr}
\usepackage{enumitem}
\usepackage{soul}
%\usepackage[british]{babel}
\usepackage{wrapfig}
%\usepackage{glossaries}
%%%%%%%%%%%%%%%%%%%%%%%%%%%%%%%%%%%%%%%%%%%%%%%%%%%%%%%%%%%%%%%%%%%%%%%%%%%
\setlength{\parindent}{2em}
\setlength{\parskip}{0.5em}
\renewcommand{\baselinestretch}{1.2}
\usepackage[left=2cm, right=2cm, top=2.5cm, bottom=3cm, headheight=13.6pt]{geometry}
\allowdisplaybreaks 
% Bibliography
%\usepackage[backend=bibtex,style=ieee,sorting=none]{biblatex} 
%\bibliography{Bibliography-thesis}
%\renewcommand*{\bibfont}{\scriptsize}
\makeatletter
\newcommand*{\rom}[1]{\expandafter\@slowromancap\romannumeral #1@}
\newcommand{\ie}{\textit{i.e.} }
\newcommand{\eg}{\textit{e.g.} }
\makeatother
\newcommand\id{\ensuremath{\mathbbm{1}}} 
\DeclareMathOperator{\E}{\mathbb{E}}
\DeclareMathOperator{\eye}{\mathbb{I}}
\DeclareMathOperator{\zeros}{\mathbb{O}}
\DeclareMathOperator{\tr}{\textrm{tr}}
\DeclareMathOperator{\vvec}{\textrm{vec}}
\DeclareMathOperator{\ik}{\mathrm{k}}
\DeclareMathOperator{\ip}{\mathrm{p}}
\DeclareMathOperator{\inn}{\mathrm{n}}
\DeclareMathOperator{\im}{\mathrm{m}}
\DeclareMathOperator{\td}{\mathrm{t}}
\DeclareMathOperator{\kd}{\mathrm{k}}
\DeclareMathOperator{\T}{\mathrm{T}}
\DeclareMathOperator{\K}{\mathrm{K}}
\DeclareSymbolFontAlphabet{\mathcal} {symbols}
\DeclareSymbolFont{symbols}{OMS}{cm}{m}{n}
\DeclareMathAlphabet{\mathbfit}{OML}{cmm}{b}{it}

% Number equations
%\numberwithin{equation}{section}

%%%%%%%%%%%%%%%%%%%%%%%%%%%%%%%%%%%%%%%%%%%%%%%%%%%%%%%%%%%%%%%%%%%%%%%%%%%
%Theorems
\newtheoremstyle{mytheoremstyle} % name
{.5em}                    % Space above
{.8em}                    % Space below
{\itshape}                % Body font
{1em}                           % Indent amount
{\bfseries}                   % Theorem head font
{:}                          % Punctuation after theorem head
{.5em}                       % Space after theorem head
{}  % Theorem head spec (can be left empty, meaning ‘normal’)

\theoremstyle{mytheoremstyle}
\newtheorem{theorem}{Theorem}[section]
\newtheorem{remark}{Remark}[section]
\newtheorem{assumption}{Assumption}[section]
\newtheorem{lemma}{Lemma}[section]
\newtheorem{condition}{Condition}[section]
\newtheorem{definition}{Definition}[section]
\newtheorem{property}{Property}[section]
\newtheorem{corollary}{Corollary}[section]
\renewcommand\qedsymbol{$\blacksquare$}
%%%%%%%%%%%%%%%%%%%%%%%%%%%%%%%%%%%%%%%%%%%%%%%%%%%%%%%%%%%%%%%%%%%%%%%%%%%
% Nomenclature
\usepackage[intoc]{nomencl}
\makenomenclature

%\usepackage[ruled,chapter]{algorithm}
%\usepackage{float}

%\usepackage{algorithmic}
%\algsetup{linenosize=\scriptsize}
%\usepackage{etoolbox}
%\AtBeginEnvironment{algorithmic}{\scriptsize}
%\renewcommand{\thealgorithm}{\thechapter.\arabic{algorithm}} 
%\usepackage{chngcntr}
%\counterwithin{algorithm}{section}

% correct bad hyphenation here
\hyphenation{op-tical net-works semi-conduc-tor}
%%%%%%%%%%%%%%%%%%%%%%%%%%%%%%%%%%%%%%%%%%%%%%%%%%%%%%%%%%%%%%%%%%%%%%%%%%%%%
% Captions
%\newcommand{\xLanguage}{british}          % <-- Added this command
%
%\usepackage[\xLanguage]{babel}             % <-- Implemented here
%
%\expandafter\addto\csname captions\xLanguage\endcsname{% <-- and here
%	\renewcommand{\tableshortname}{Table}%
%	\renewcommand{\figureshortname}{Figure}%
%}
%
%\captionsetup[table]{format=plain,indention=1.15cm,justification=justified}
%\captionsetup[figure]{format=plain,indention=1.25cm,justification=justified}

%%%%%%%%%%%%%%%%%%%%%%%%%%%%%%%%%%%%%%%%%%%%%%%%%%%%%%%%%%%%%%%%%%%%%%%%%%%%%
\usepackage[explicit]{titlesec}
\usepackage{titletoc}
%\titleformat{\section}[block]{\normalfont\Large\rm\filright\bfseries}{\thesection}{1em}{#1}
%\titlecontents{chapter}[1.5em]{}{\scshape\contentslabel{2.3em}}{}{\titlerule*[1pc]{}\contentspage}
%\definecolor{gray75}{gray}{0.5}
%\newcommand{\hsp}{\hspace{20pt}}
%\titleformat{\chapter}[hang]{\huge\scshape\filright\bfseries}{\color{gray75}\thechapter}{20pt}{\begin{tabular}[t]{@{\color{gray75}\vrule width 2pt\hsp}p{0.85\textwidth}}\raggedright#1\end{tabular}}
%\titleformat{name=\chapter,numberless}[display]{}{}{0pt}{\normalfont\huge\bfseries #1} % format for numberless chapters
%\titleformat{\subsection}[block]{\normalfont\large\rm\filright\bfseries}{\thesubsection}{1em}{#1}
%\renewcommand{\sectionmark}[1]{\markright{\thesection ~ \ #1}}
%%\renewcommand{\chaptermark}[1]{\markboth{\chaptername\ \thechapter ~ \ #1}{}} 
%\pagestyle{fancy}
%%\fancyhf{}
%\fancyhead[RO,LE]{\thepage}
%\fancyhead[RE]{\itshape \nouppercase \rightmark}      % chaptertitle left
%\fancyhead[LO]{\itshape \nouppercase \leftmark}       % sectiontitle right
%\renewcommand{\headrulewidth}{0.5pt} 				   % no rule
%\cfoot{}
\interfootnotelinepenalty=10000

\title{Structure and parameter identification of au}


\begin{document}
	\maketitle
%	\section{Experimental setup}
%	
\section{Experimental data}
\begin{figure}[!h]
		\centering
		% This file was created by matlab2tikz.
%
\definecolor{mycolor1}{rgb}{0.00000,0.44700,0.74100}%
%
\begin{tikzpicture}

\begin{axis}[%
width=11.411cm,
height=5cm,
at={(0cm,0cm)},
scale only axis,
xmin=1,
xmax=1500,
xlabel style={font=\color{white!15!black}},
xlabel={Sample index},
ymin=-7,
ymax=-4.999,
ylabel style={font=\color{white!15!black}},
ylabel={$\Delta x$, mm},
axis background/.style={fill=white},
legend style={legend cell align=left, align=left, draw=white!15!black}
]
\addplot [color=mycolor1]
  table[row sep=crcr]{%
1	-5.951\\
3	-5.933\\
5	-5.933\\
7	-5.914\\
9	-5.933\\
11	-5.933\\
13	-5.914\\
15	-5.914\\
17	-5.933\\
19	-5.933\\
21	-5.933\\
23	-5.933\\
25	-5.914\\
27	-5.914\\
29	-5.933\\
31	-5.933\\
33	-5.933\\
35	-5.914\\
37	-5.933\\
39	-5.933\\
41	-5.933\\
43	-5.933\\
45	-5.933\\
47	-5.933\\
49	-5.933\\
51	-5.933\\
53	-5.914\\
55	-5.914\\
57	-5.914\\
59	-5.933\\
61	-5.933\\
63	-5.933\\
65	-5.933\\
67	-5.914\\
69	-5.914\\
71	-5.914\\
73	-5.933\\
75	-5.914\\
77	-5.914\\
79	-5.933\\
81	-5.933\\
83	-5.951\\
85	-5.933\\
87	-5.933\\
89	-5.914\\
91	-5.933\\
93	-5.933\\
95	-5.933\\
97	-5.933\\
99	-5.914\\
101	-5.914\\
103	-5.933\\
105	-5.933\\
107	-5.933\\
109	-5.933\\
111	-5.914\\
113	-5.896\\
115	-5.878\\
117	-5.841\\
119	-5.859\\
121	-6.061\\
123	-6.226\\
125	-6.116\\
127	-6.171\\
129	-6.281\\
131	-6.061\\
133	-5.878\\
135	-6.226\\
137	-6.061\\
139	-6.226\\
141	-5.859\\
143	-5.933\\
145	-6.354\\
147	-6.079\\
149	-5.896\\
151	-5.64\\
153	-5.64\\
155	-5.365\\
157	-5.621\\
159	-5.878\\
161	-5.786\\
163	-5.548\\
165	-5.988\\
167	-6.207\\
169	-5.933\\
171	-6.097\\
173	-5.713\\
175	-5.511\\
177	-5.548\\
179	-5.42\\
181	-5.511\\
183	-5.2\\
185	-4.999\\
187	-5.219\\
189	-5.731\\
191	-5.566\\
193	-5.585\\
195	-5.621\\
197	-5.823\\
199	-6.079\\
201	-6.006\\
203	-5.475\\
205	-5.64\\
207	-5.786\\
209	-5.951\\
211	-5.841\\
213	-6.171\\
215	-6.207\\
217	-6.006\\
219	-5.731\\
221	-5.64\\
223	-5.969\\
225	-5.768\\
227	-5.841\\
229	-5.933\\
231	-6.299\\
233	-5.859\\
235	-5.878\\
237	-5.75\\
239	-5.493\\
241	-5.768\\
243	-5.951\\
245	-5.896\\
247	-5.933\\
249	-6.207\\
251	-5.914\\
253	-6.042\\
255	-5.64\\
257	-5.896\\
259	-5.823\\
261	-5.658\\
263	-5.585\\
265	-6.097\\
267	-5.768\\
269	-5.658\\
271	-5.695\\
273	-5.676\\
275	-5.914\\
277	-6.299\\
279	-6.299\\
281	-5.859\\
283	-5.804\\
285	-5.878\\
287	-6.024\\
289	-6.207\\
291	-5.969\\
293	-6.024\\
295	-6.061\\
297	-6.244\\
299	-6.097\\
301	-5.878\\
303	-5.804\\
305	-5.475\\
307	-5.328\\
309	-5.621\\
311	-5.621\\
313	-5.64\\
315	-5.804\\
317	-5.988\\
319	-6.335\\
321	-6.647\\
323	-6.354\\
325	-5.933\\
327	-6.042\\
329	-5.951\\
331	-5.896\\
333	-6.152\\
335	-6.793\\
337	-6.573\\
339	-6.519\\
341	-6.061\\
343	-5.75\\
345	-5.951\\
347	-5.64\\
349	-5.2\\
351	-5.585\\
353	-6.024\\
355	-6.464\\
357	-6.445\\
359	-6.299\\
361	-6.207\\
363	-6.354\\
365	-6.097\\
367	-6.171\\
369	-5.914\\
371	-5.859\\
373	-5.969\\
375	-6.537\\
377	-6.409\\
379	-6.061\\
381	-5.42\\
383	-5.383\\
385	-5.676\\
387	-6.024\\
389	-6.244\\
391	-6.72\\
393	-6.555\\
395	-6.152\\
397	-6.628\\
399	-6.39\\
401	-6.042\\
403	-5.804\\
405	-5.804\\
407	-5.493\\
409	-5.53\\
411	-5.878\\
413	-5.75\\
415	-6.152\\
417	-6.281\\
419	-6.354\\
421	-6.079\\
423	-5.969\\
425	-5.896\\
427	-5.969\\
429	-5.841\\
431	-5.951\\
433	-5.548\\
435	-6.006\\
437	-6.573\\
439	-6.592\\
441	-6.647\\
443	-6.281\\
445	-6.171\\
447	-5.878\\
449	-5.53\\
451	-5.53\\
453	-5.878\\
455	-6.226\\
457	-6.024\\
459	-6.317\\
461	-6.354\\
463	-6.024\\
465	-5.878\\
467	-5.768\\
469	-6.006\\
471	-5.933\\
473	-5.969\\
475	-6.024\\
477	-6.189\\
479	-6.189\\
481	-6.317\\
483	-6.006\\
485	-5.804\\
487	-5.896\\
489	-6.042\\
491	-6.042\\
493	-5.988\\
495	-5.75\\
497	-6.061\\
499	-5.914\\
501	-5.695\\
503	-5.585\\
505	-5.621\\
507	-5.933\\
509	-6.226\\
511	-6.354\\
513	-6.464\\
515	-6.573\\
517	-6.134\\
519	-5.841\\
521	-6.079\\
523	-5.713\\
525	-5.53\\
527	-5.548\\
529	-5.731\\
531	-5.933\\
533	-6.226\\
535	-6.061\\
537	-6.024\\
539	-6.006\\
541	-5.933\\
543	-6.116\\
545	-6.427\\
547	-6.244\\
549	-6.281\\
551	-5.969\\
553	-6.39\\
555	-5.914\\
557	-6.116\\
559	-6.207\\
561	-5.75\\
563	-5.713\\
565	-5.585\\
567	-5.658\\
569	-5.402\\
571	-5.219\\
573	-5.273\\
575	-5.566\\
577	-5.878\\
579	-6.097\\
581	-5.713\\
583	-5.493\\
585	-5.566\\
587	-5.237\\
589	-5.164\\
591	-5.493\\
593	-5.603\\
595	-5.475\\
597	-5.823\\
599	-6.079\\
601	-6.152\\
603	-6.097\\
605	-5.969\\
607	-5.786\\
609	-5.603\\
611	-5.786\\
613	-5.548\\
615	-5.951\\
617	-6.079\\
619	-6.079\\
621	-6.189\\
623	-6.042\\
625	-5.841\\
627	-5.804\\
629	-6.189\\
631	-6.207\\
633	-6.207\\
635	-5.878\\
637	-6.024\\
639	-6.244\\
641	-6.006\\
643	-5.878\\
645	-5.621\\
647	-5.914\\
649	-5.896\\
651	-5.493\\
653	-5.585\\
655	-5.878\\
657	-5.804\\
659	-5.951\\
661	-5.603\\
663	-5.53\\
665	-5.713\\
667	-5.823\\
669	-6.171\\
671	-6.262\\
673	-6.024\\
675	-6.116\\
677	-6.006\\
679	-6.006\\
681	-5.988\\
683	-6.189\\
685	-6.079\\
687	-6.189\\
689	-6.335\\
691	-6.372\\
693	-6.189\\
695	-6.39\\
697	-6.702\\
699	-6.262\\
701	-6.171\\
703	-6.134\\
705	-6.5\\
707	-6.226\\
709	-6.171\\
711	-6.024\\
713	-6.207\\
715	-6.317\\
717	-6.006\\
719	-5.878\\
721	-5.75\\
723	-5.75\\
725	-6.079\\
727	-6.281\\
729	-6.702\\
731	-6.39\\
733	-6.116\\
735	-5.786\\
737	-5.988\\
739	-6.189\\
741	-5.896\\
743	-6.042\\
745	-6.134\\
747	-6.189\\
749	-5.841\\
751	-5.969\\
753	-5.603\\
755	-5.951\\
757	-5.695\\
759	-5.768\\
761	-5.859\\
763	-6.354\\
765	-6.152\\
767	-6.042\\
769	-6.171\\
771	-6.079\\
773	-5.603\\
775	-5.713\\
777	-5.878\\
779	-5.988\\
781	-6.152\\
783	-5.933\\
785	-6.262\\
787	-6.207\\
789	-6.207\\
791	-6.207\\
793	-6.116\\
795	-5.75\\
797	-6.061\\
799	-6.006\\
801	-5.823\\
803	-5.896\\
805	-5.878\\
807	-5.786\\
809	-5.713\\
811	-5.64\\
813	-5.585\\
815	-5.841\\
817	-5.75\\
819	-6.134\\
821	-5.988\\
823	-6.061\\
825	-5.53\\
827	-5.676\\
829	-6.134\\
831	-6.061\\
833	-6.024\\
835	-6.207\\
837	-5.988\\
839	-5.786\\
841	-5.75\\
843	-5.768\\
845	-5.75\\
847	-6.244\\
849	-5.878\\
851	-5.457\\
853	-5.42\\
855	-5.493\\
857	-5.566\\
859	-6.024\\
861	-6.372\\
863	-6.244\\
865	-6.409\\
867	-6.042\\
869	-5.731\\
871	-5.713\\
873	-5.621\\
875	-5.768\\
877	-5.896\\
879	-5.988\\
881	-6.189\\
883	-6.061\\
885	-6.061\\
887	-6.226\\
889	-6.281\\
891	-6.39\\
893	-6.061\\
895	-5.951\\
897	-5.658\\
899	-5.621\\
901	-5.328\\
903	-5.731\\
905	-5.621\\
907	-5.511\\
909	-5.53\\
911	-5.786\\
913	-5.566\\
915	-6.134\\
917	-6.097\\
919	-6.372\\
921	-6.39\\
923	-6.226\\
925	-6.262\\
927	-5.768\\
929	-5.658\\
931	-5.676\\
933	-5.969\\
935	-5.676\\
937	-5.731\\
939	-5.383\\
941	-5.859\\
943	-5.676\\
945	-5.823\\
947	-5.804\\
949	-6.189\\
951	-6.244\\
953	-6.244\\
955	-6.061\\
957	-6.281\\
959	-6.262\\
961	-6.006\\
963	-5.988\\
965	-5.786\\
967	-5.64\\
969	-5.676\\
971	-5.823\\
973	-5.695\\
975	-5.896\\
977	-5.676\\
979	-5.695\\
981	-5.841\\
983	-5.804\\
985	-6.317\\
987	-5.933\\
989	-6.152\\
991	-6.189\\
993	-6.299\\
995	-6.537\\
997	-6.39\\
999	-6.152\\
1001	-6.244\\
1003	-6.134\\
1005	-6.281\\
1007	-6.244\\
1009	-5.914\\
1011	-5.969\\
1013	-5.585\\
1015	-5.145\\
1017	-5.896\\
1019	-6.061\\
1021	-5.731\\
1023	-5.914\\
1025	-5.933\\
1027	-5.878\\
1029	-6.079\\
1031	-6.079\\
1033	-6.061\\
1035	-5.878\\
1037	-5.969\\
1039	-5.969\\
1041	-6.006\\
1043	-6.207\\
1045	-5.969\\
1047	-5.658\\
1049	-5.676\\
1051	-5.804\\
1053	-5.859\\
1055	-6.042\\
1057	-5.658\\
1059	-5.786\\
1061	-5.493\\
1063	-5.566\\
1065	-5.548\\
1067	-5.841\\
1069	-5.969\\
1071	-6.024\\
1073	-5.951\\
1075	-5.878\\
1077	-6.116\\
1079	-6.519\\
1081	-6.335\\
1083	-6.61\\
1085	-6.555\\
1087	-6.866\\
1089	-6.665\\
1091	-6.244\\
1093	-6.152\\
1095	-5.786\\
1097	-5.695\\
1099	-5.896\\
1101	-6.006\\
1103	-6.226\\
1105	-6.281\\
1107	-6.097\\
1109	-6.042\\
1111	-6.006\\
1113	-5.914\\
1115	-5.804\\
1117	-5.768\\
1119	-5.676\\
1121	-5.969\\
1123	-6.116\\
1125	-5.988\\
1127	-6.226\\
1129	-6.281\\
1131	-5.859\\
1133	-5.969\\
1135	-6.409\\
1137	-6.354\\
1139	-6.281\\
1141	-6.262\\
1143	-5.969\\
1145	-5.841\\
1147	-6.006\\
1149	-6.207\\
1151	-5.951\\
1153	-5.695\\
1155	-5.457\\
1157	-5.64\\
1159	-5.658\\
1161	-5.53\\
1163	-5.219\\
1165	-5.676\\
1167	-6.134\\
1169	-5.804\\
1171	-5.438\\
1173	-5.713\\
1175	-6.024\\
1177	-6.61\\
1179	-6.555\\
1181	-6.335\\
1183	-6.427\\
1185	-6.244\\
1187	-6.171\\
1189	-6.116\\
1191	-6.427\\
1193	-6.079\\
1195	-6.226\\
1197	-6.537\\
1199	-6.683\\
1201	-6.5\\
1203	-6.61\\
1205	-6.152\\
1207	-6.171\\
1209	-5.969\\
1211	-5.969\\
1213	-5.896\\
1215	-5.914\\
1217	-6.134\\
1219	-5.969\\
1221	-6.427\\
1223	-6.152\\
1225	-5.951\\
1227	-5.804\\
1229	-5.859\\
1231	-6.006\\
1233	-6.042\\
1235	-6.207\\
1237	-6.134\\
1239	-6.152\\
1241	-5.731\\
1243	-5.896\\
1245	-5.786\\
1247	-5.933\\
1249	-5.878\\
1251	-5.804\\
1253	-5.676\\
1255	-5.548\\
1257	-5.768\\
1259	-6.024\\
1261	-6.152\\
1263	-5.695\\
1265	-5.75\\
1267	-5.585\\
1269	-5.896\\
1271	-6.134\\
1273	-5.951\\
1275	-5.566\\
1277	-5.328\\
1279	-5.603\\
1281	-5.768\\
1283	-6.097\\
1285	-6.024\\
1287	-6.116\\
1289	-6.226\\
1291	-5.695\\
1293	-5.383\\
1295	-5.42\\
1297	-5.457\\
1299	-5.731\\
1301	-5.786\\
1303	-5.402\\
1305	-5.804\\
1307	-5.914\\
1309	-5.75\\
1311	-6.024\\
1313	-5.988\\
1315	-6.134\\
1317	-6.134\\
1319	-6.061\\
1321	-6.171\\
1323	-6.39\\
1325	-6.006\\
1327	-6.079\\
1329	-6.079\\
1331	-6.171\\
1333	-6.116\\
1335	-6.024\\
1337	-6.299\\
1339	-6.097\\
1341	-5.969\\
1343	-5.511\\
1345	-5.2\\
1347	-5.676\\
1349	-5.768\\
1351	-5.841\\
1353	-5.621\\
1355	-5.511\\
1357	-5.768\\
1359	-5.768\\
1361	-5.548\\
1363	-5.457\\
1365	-6.079\\
1367	-6.189\\
1369	-6.39\\
1371	-6.134\\
1373	-6.061\\
1375	-6.207\\
1377	-6.39\\
1379	-6.628\\
1381	-6.519\\
1383	-6.482\\
1385	-6.482\\
1387	-5.933\\
1389	-5.969\\
1391	-5.75\\
1393	-5.713\\
1395	-5.566\\
1397	-5.53\\
1399	-5.896\\
1401	-6.024\\
1403	-6.354\\
1405	-6.39\\
1407	-6.226\\
1409	-5.969\\
1411	-5.731\\
1413	-5.676\\
1415	-5.365\\
1417	-5.859\\
1419	-6.042\\
1421	-6.097\\
1423	-5.951\\
1425	-5.841\\
1427	-5.914\\
1429	-5.878\\
1431	-6.006\\
1433	-5.804\\
1435	-5.713\\
1437	-5.676\\
1439	-5.878\\
1441	-5.878\\
1443	-5.695\\
1445	-5.896\\
1447	-6.207\\
1449	-6.116\\
1451	-5.951\\
1453	-5.951\\
1455	-5.768\\
1457	-5.914\\
1459	-6.079\\
1461	-6.262\\
1463	-6.592\\
1465	-6.097\\
1467	-5.603\\
1469	-5.621\\
1471	-5.841\\
1473	-5.896\\
1475	-6.079\\
1477	-6.372\\
1479	-6.317\\
1481	-6.061\\
1483	-6.152\\
1485	-6.006\\
1487	-5.75\\
1489	-6.061\\
1491	-5.969\\
1493	-6.189\\
1495	-6.299\\
1497	-6.134\\
1499	-5.823\\
1501	-6.281\\
};
\addlegendentry{data1}

\end{axis}
\end{tikzpicture}%
		\caption{The input for the set C.}
	\end{figure}	
\begin{figure}[!h]
	\centering
	% This file was created by matlab2tikz.
% Minimal pgfplots version: 1.3
%
\definecolor{mycolor1}{rgb}{0.00000,0.44700,0.74100}%
%
\begin{tikzpicture}

\begin{axis}[%
width=5.090835cm,
height=2.240107cm,
at={(0cm,0cm)},
scale only axis,
xmin=1000,
xmax=1500,
xlabel={Sample index},
ymin=-200,
ymax=0,
ylabel={C9 load, kN},
legend style={legend cell align=left,align=left,draw=white!15!black}
]
\addplot [color=mycolor1,solid,forget plot]
  table[row sep=crcr]{%
1000	-68.359\\
1001	-85.449\\
1002	-69.58\\
1003	-65.918\\
1004	-90.332\\
1005	-86.67\\
1006	-103.76\\
1007	-79.346\\
1008	-45.166\\
1009	-57.373\\
1010	-51.27\\
1011	-63.477\\
1012	-48.828\\
1013	-24.414\\
1014	-17.09\\
1015	-17.09\\
1016	-43.945\\
1017	-73.242\\
1018	-76.904\\
1019	-79.346\\
1020	-58.594\\
1021	-36.621\\
1022	-76.904\\
1023	-53.711\\
1024	-43.945\\
1025	-69.58\\
1026	-61.035\\
1027	-48.828\\
1028	-84.229\\
1029	-79.346\\
1030	-59.814\\
1031	-87.891\\
1032	-85.449\\
1033	-67.139\\
1034	-54.932\\
1035	-46.387\\
1036	-56.152\\
1037	-64.697\\
1038	-64.697\\
1039	-67.139\\
1040	-68.359\\
1041	-73.242\\
1042	-100.098\\
1043	-87.891\\
1044	-61.035\\
1045	-56.152\\
1046	-34.18\\
1047	-34.18\\
1048	-48.828\\
1049	-37.842\\
1050	-39.063\\
1051	-56.152\\
1052	-62.256\\
1053	-58.594\\
1054	-90.332\\
1055	-64.697\\
1056	-41.504\\
1057	-34.18\\
1058	-45.166\\
1059	-51.27\\
1060	-28.076\\
1061	-25.635\\
1062	-41.504\\
1063	-34.18\\
1064	-29.297\\
1065	-40.283\\
1066	-43.945\\
1067	-68.359\\
1068	-63.477\\
1069	-79.346\\
1070	-69.58\\
1071	-79.346\\
1072	-63.477\\
1073	-62.256\\
1074	-54.932\\
1075	-53.711\\
1076	-53.711\\
1077	-91.553\\
1078	-126.953\\
1079	-131.836\\
1080	-129.395\\
1081	-83.008\\
1082	-114.746\\
1083	-141.602\\
1084	-147.705\\
1085	-111.084\\
1086	-146.484\\
1087	-185.547\\
1088	-131.836\\
1089	-112.305\\
1090	-75.684\\
1091	-61.035\\
1092	-52.49\\
1093	-64.697\\
1094	-46.387\\
1095	-31.738\\
1096	-31.738\\
1097	-40.283\\
1098	-57.373\\
1099	-52.49\\
1100	-53.711\\
1101	-72.021\\
1102	-72.021\\
1103	-97.656\\
1104	-79.346\\
1105	-101.318\\
1106	-72.021\\
1107	-64.697\\
1108	-73.242\\
1109	-54.932\\
1110	-51.27\\
1111	-59.814\\
1112	-62.256\\
1113	-43.945\\
1114	-47.607\\
1115	-34.18\\
1116	-40.283\\
1117	-42.725\\
1118	-30.518\\
1119	-39.063\\
1120	-47.607\\
1121	-72.021\\
1122	-74.463\\
1123	-81.787\\
1124	-48.828\\
1125	-70.801\\
1126	-101.318\\
1127	-84.229\\
1128	-93.994\\
1129	-91.553\\
1130	-54.932\\
1131	-40.283\\
1132	-56.152\\
1133	-61.035\\
1134	-97.656\\
1135	-119.629\\
1136	-117.188\\
1137	-89.111\\
1138	-92.773\\
1139	-81.787\\
1140	-80.566\\
1141	-79.346\\
1142	-54.932\\
1143	-48.828\\
1144	-45.166\\
1145	-40.283\\
1146	-45.166\\
1147	-64.697\\
1148	-96.436\\
1149	-74.463\\
1150	-52.49\\
1151	-45.166\\
1152	-46.387\\
1153	-29.297\\
1154	-23.193\\
1155	-26.855\\
1156	-46.387\\
1157	-35.4\\
1158	-41.504\\
1159	-40.283\\
1160	-39.063\\
1161	-28.076\\
1162	-21.973\\
1163	-17.09\\
1164	-29.297\\
1165	-58.594\\
1166	-80.566\\
1167	-91.553\\
1168	-59.814\\
1169	-43.945\\
1170	-32.959\\
1171	-23.193\\
1172	-47.607\\
1173	-46.387\\
1174	-67.139\\
1175	-81.787\\
1176	-131.836\\
1177	-147.705\\
1178	-120.85\\
1179	-117.188\\
1180	-78.125\\
1181	-81.787\\
1182	-91.553\\
1183	-98.877\\
1184	-73.242\\
1185	-68.359\\
1186	-69.58\\
1187	-62.256\\
1188	-75.684\\
1189	-53.711\\
1190	-93.994\\
1191	-108.643\\
1192	-80.566\\
1193	-51.27\\
1194	-54.932\\
1195	-92.773\\
1196	-109.863\\
1197	-134.277\\
1198	-140.381\\
1199	-140.381\\
1200	-106.201\\
1201	-96.436\\
1202	-111.084\\
1203	-129.395\\
1204	-78.125\\
1205	-50.049\\
1206	-73.242\\
1207	-57.373\\
1208	-39.063\\
1209	-56.152\\
1210	-61.035\\
1211	-46.387\\
1212	-37.842\\
1213	-50.049\\
1214	-39.063\\
1215	-53.711\\
1216	-76.904\\
1217	-72.021\\
1218	-51.27\\
1219	-51.27\\
1220	-78.125\\
1221	-123.291\\
1222	-86.67\\
1223	-56.152\\
1224	-43.945\\
1225	-48.828\\
1226	-43.945\\
1227	-34.18\\
1228	-37.842\\
1229	-47.607\\
1230	-59.814\\
1231	-64.697\\
1232	-46.387\\
1233	-74.463\\
1234	-85.449\\
1235	-83.008\\
1236	-65.918\\
1237	-73.242\\
1238	-96.436\\
1239	-61.035\\
1240	-29.297\\
1241	-42.725\\
1242	-43.945\\
1243	-58.594\\
1244	-50.049\\
1245	-40.283\\
1246	-58.594\\
1247	-54.932\\
1248	-39.063\\
1249	-48.828\\
1250	-43.945\\
1251	-39.063\\
1252	-50.049\\
1253	-29.297\\
1254	-36.621\\
1255	-26.855\\
1256	-31.738\\
1257	-53.711\\
1258	-61.035\\
1259	-76.904\\
1260	-101.318\\
1261	-67.139\\
1262	-42.725\\
1263	-32.959\\
1264	-31.738\\
1265	-47.607\\
1266	-30.518\\
1267	-36.621\\
1268	-43.945\\
1269	-65.918\\
1270	-62.256\\
1271	-86.67\\
1272	-59.814\\
1273	-56.152\\
1274	-31.738\\
1275	-32.959\\
1276	-20.752\\
1277	-24.414\\
1278	-26.855\\
1279	-47.607\\
1280	-47.607\\
1281	-56.152\\
1282	-59.814\\
1283	-93.994\\
1284	-79.346\\
1285	-62.256\\
1286	-73.242\\
1287	-79.346\\
1288	-101.318\\
1289	-80.566\\
1290	-52.49\\
1291	-30.518\\
1292	-21.973\\
1293	-23.193\\
1294	-29.297\\
1295	-29.297\\
1296	-26.855\\
1297	-36.621\\
1298	-53.711\\
1299	-47.607\\
1300	-51.27\\
1301	-57.373\\
1302	-37.842\\
1303	-20.752\\
1304	-42.725\\
1305	-63.477\\
1306	-61.035\\
1307	-69.58\\
1308	-58.594\\
1309	-45.166\\
1310	-59.814\\
1311	-84.229\\
1312	-85.449\\
1313	-59.814\\
1314	-102.539\\
1315	-76.904\\
1316	-74.463\\
1317	-84.229\\
1318	-83.008\\
1319	-63.477\\
1320	-54.932\\
1321	-90.332\\
1322	-124.512\\
1323	-95.215\\
1324	-57.373\\
1325	-53.711\\
1326	-54.932\\
1327	-72.021\\
1328	-83.008\\
1329	-59.814\\
1330	-64.697\\
1331	-83.008\\
1332	-90.332\\
1333	-61.035\\
1334	-50.049\\
1335	-63.477\\
1336	-104.98\\
1337	-92.773\\
1338	-95.215\\
1339	-58.594\\
1340	-53.711\\
1341	-51.27\\
1342	-34.18\\
1343	-23.193\\
1344	-20.752\\
1345	-17.09\\
1346	-40.283\\
1347	-54.932\\
1348	-62.256\\
1349	-47.607\\
1350	-52.49\\
1351	-57.373\\
1352	-37.842\\
1353	-39.063\\
1354	-39.063\\
1355	-28.076\\
1356	-39.063\\
1357	-57.373\\
1358	-73.242\\
1359	-48.828\\
1360	-35.4\\
1361	-31.738\\
1362	-36.621\\
1363	-26.855\\
1364	-54.932\\
1365	-93.994\\
1366	-72.021\\
1367	-97.656\\
1368	-113.525\\
1369	-114.746\\
1370	-85.449\\
1371	-63.477\\
1372	-63.477\\
1373	-63.477\\
1374	-75.684\\
1375	-89.111\\
1376	-86.67\\
1377	-115.967\\
1378	-125.732\\
1379	-150.146\\
1380	-96.436\\
1381	-109.863\\
1382	-122.07\\
1383	-92.773\\
1384	-107.422\\
1385	-92.773\\
1386	-54.932\\
1387	-39.063\\
1388	-40.283\\
1389	-57.373\\
1390	-42.725\\
1391	-32.959\\
1392	-45.166\\
1393	-34.18\\
1394	-26.855\\
1395	-34.18\\
1396	-37.842\\
1397	-25.635\\
1398	-48.828\\
1399	-64.697\\
1400	-46.387\\
1401	-81.787\\
1402	-115.967\\
1403	-106.201\\
1404	-133.057\\
1405	-92.773\\
1406	-97.656\\
1407	-69.58\\
1408	-43.945\\
1409	-53.711\\
1410	-42.725\\
1411	-34.18\\
1412	-45.166\\
1413	-26.855\\
1414	-20.752\\
1415	-25.635\\
1416	-50.049\\
1417	-61.035\\
1418	-78.125\\
1419	-74.463\\
1420	-69.58\\
1421	-80.566\\
1422	-64.697\\
1423	-53.711\\
1424	-54.932\\
1425	-43.945\\
1426	-69.58\\
1427	-50.049\\
1428	-40.283\\
1429	-58.594\\
1430	-80.566\\
1431	-59.814\\
1432	-46.387\\
1433	-41.504\\
1434	-47.607\\
1435	-32.959\\
1436	-31.738\\
1437	-43.945\\
1438	-53.711\\
1439	-59.814\\
1440	-48.828\\
1441	-61.035\\
1442	-57.373\\
1443	-34.18\\
1444	-37.842\\
1445	-65.918\\
1446	-91.553\\
1447	-90.332\\
1448	-75.684\\
1449	-73.242\\
1450	-57.373\\
1451	-53.711\\
1452	-43.945\\
1453	-61.035\\
1454	-54.932\\
1455	-36.621\\
1456	-52.49\\
1457	-61.035\\
1458	-70.801\\
1459	-76.904\\
1460	-76.904\\
1461	-102.539\\
1462	-124.512\\
1463	-140.381\\
1464	-91.553\\
1465	-52.49\\
1466	-34.18\\
1467	-25.635\\
1468	-36.621\\
1469	-32.959\\
1470	-58.594\\
1471	-53.711\\
1472	-46.387\\
1473	-62.256\\
1474	-72.021\\
1475	-81.787\\
1476	-86.67\\
1477	-122.07\\
1478	-101.318\\
1479	-81.787\\
1480	-58.594\\
1481	-58.594\\
1482	-59.814\\
1483	-75.684\\
1484	-54.932\\
1485	-56.152\\
1486	-53.711\\
1487	-30.518\\
1488	-54.932\\
1489	-70.801\\
1490	-50.049\\
1491	-53.711\\
1492	-100.098\\
1493	-75.684\\
1494	-72.021\\
1495	-102.539\\
1496	-98.877\\
1497	-62.256\\
1498	-40.283\\
1499	-42.725\\
1500	-70.801\\
};
\end{axis}

\begin{axis}[%
width=5.090835cm,
height=2.240107cm,
at={(6.698467cm,15.759893cm)},
scale only axis,
xmin=1000,
xmax=1500,
xlabel={Sample index},
ymin=-50,
ymax=0,
ylabel={C2 load, kN},
legend style={legend cell align=left,align=left,draw=white!15!black}
]
\addplot [color=mycolor1,solid,forget plot]
  table[row sep=crcr]{%
1000	-19.531\\
1001	-24.414\\
1002	-19.531\\
1003	-20.752\\
1004	-25.635\\
1005	-24.414\\
1006	-28.076\\
1007	-23.193\\
1008	-13.428\\
1009	-17.09\\
1010	-17.09\\
1011	-18.311\\
1012	-15.869\\
1013	-8.545\\
1014	-6.104\\
1015	-7.324\\
1016	-9.766\\
1017	-21.973\\
1018	-23.193\\
1019	-23.193\\
1020	-19.531\\
1021	-10.986\\
1022	-17.09\\
1023	-19.531\\
1024	-13.428\\
1025	-18.311\\
1026	-19.531\\
1027	-14.648\\
1028	-24.414\\
1029	-21.973\\
1030	-15.869\\
1031	-23.193\\
1032	-25.635\\
1033	-19.531\\
1034	-17.09\\
1035	-14.648\\
1036	-17.09\\
1037	-20.752\\
1038	-18.311\\
1039	-18.311\\
1040	-19.531\\
1041	-20.752\\
1042	-28.076\\
1043	-25.635\\
1044	-17.09\\
1045	-15.869\\
1046	-12.207\\
1047	-10.986\\
1048	-15.869\\
1049	-12.207\\
1050	-12.207\\
1051	-15.869\\
1052	-18.311\\
1053	-15.869\\
1054	-25.635\\
1055	-21.973\\
1056	-12.207\\
1057	-9.766\\
1058	-13.428\\
1059	-15.869\\
1060	-10.986\\
1061	-10.986\\
1062	-12.207\\
1063	-9.766\\
1064	-8.545\\
1065	-12.207\\
1066	-14.648\\
1067	-17.09\\
1068	-18.311\\
1069	-23.193\\
1070	-21.973\\
1071	-21.973\\
1072	-17.09\\
1073	-17.09\\
1074	-17.09\\
1075	-15.869\\
1076	-17.09\\
1077	-24.414\\
1078	-35.4\\
1079	-36.621\\
1080	-34.18\\
1081	-24.414\\
1082	-29.297\\
1083	-36.621\\
1084	-40.283\\
1085	-31.738\\
1086	-37.842\\
1087	-48.828\\
1088	-39.063\\
1089	-30.518\\
1090	-25.635\\
1091	-19.531\\
1092	-14.648\\
1093	-19.531\\
1094	-15.869\\
1095	-9.766\\
1096	-9.766\\
1097	-12.207\\
1098	-17.09\\
1099	-15.869\\
1100	-15.869\\
1101	-19.531\\
1102	-19.531\\
1103	-28.076\\
1104	-23.193\\
1105	-25.635\\
1106	-23.193\\
1107	-18.311\\
1108	-20.752\\
1109	-17.09\\
1110	-14.648\\
1111	-18.311\\
1112	-17.09\\
1113	-15.869\\
1114	-13.428\\
1115	-10.986\\
1116	-12.207\\
1117	-13.428\\
1118	-10.986\\
1119	-9.766\\
1120	-14.648\\
1121	-19.531\\
1122	-23.193\\
1123	-23.193\\
1124	-17.09\\
1125	-18.311\\
1126	-30.518\\
1127	-23.193\\
1128	-24.414\\
1129	-26.855\\
1130	-17.09\\
1131	-10.986\\
1132	-17.09\\
1133	-18.311\\
1134	-26.855\\
1135	-34.18\\
1136	-31.738\\
1137	-25.635\\
1138	-24.414\\
1139	-23.193\\
1140	-23.193\\
1141	-21.973\\
1142	-18.311\\
1143	-14.648\\
1144	-13.428\\
1145	-12.207\\
1146	-13.428\\
1147	-19.531\\
1148	-25.635\\
1149	-21.973\\
1150	-14.648\\
1151	-13.428\\
1152	-14.648\\
1153	-12.207\\
1154	-8.545\\
1155	-8.545\\
1156	-13.428\\
1157	-12.207\\
1158	-12.207\\
1159	-12.207\\
1160	-12.207\\
1161	-8.545\\
1162	-7.324\\
1163	-4.883\\
1164	-7.324\\
1165	-17.09\\
1166	-24.414\\
1167	-24.414\\
1168	-19.531\\
1169	-12.207\\
1170	-12.207\\
1171	-8.545\\
1172	-10.986\\
1173	-14.648\\
1174	-14.648\\
1175	-24.414\\
1176	-34.18\\
1177	-40.283\\
1178	-32.959\\
1179	-31.738\\
1180	-23.193\\
1181	-25.635\\
1182	-26.855\\
1183	-28.076\\
1184	-21.973\\
1185	-20.752\\
1186	-19.531\\
1187	-19.531\\
1188	-21.973\\
1189	-19.531\\
1190	-23.193\\
1191	-29.297\\
1192	-24.414\\
1193	-14.648\\
1194	-15.869\\
1195	-25.635\\
1196	-32.959\\
1197	-35.4\\
1198	-39.063\\
1199	-37.842\\
1200	-30.518\\
1201	-26.855\\
1202	-30.518\\
1203	-35.4\\
1204	-25.635\\
1205	-14.648\\
1206	-19.531\\
1207	-20.752\\
1208	-12.207\\
1209	-13.428\\
1210	-18.311\\
1211	-14.648\\
1212	-12.207\\
1213	-18.311\\
1214	-10.986\\
1215	-14.648\\
1216	-24.414\\
1217	-21.973\\
1218	-14.648\\
1219	-14.648\\
1220	-20.752\\
1221	-34.18\\
1222	-28.076\\
1223	-15.869\\
1224	-14.648\\
1225	-14.648\\
1226	-12.207\\
1227	-10.986\\
1228	-10.986\\
1229	-13.428\\
1230	-17.09\\
1231	-19.531\\
1232	-14.648\\
1233	-19.531\\
1234	-26.855\\
1235	-23.193\\
1236	-17.09\\
1237	-21.973\\
1238	-25.635\\
1239	-21.973\\
1240	-8.545\\
1241	-12.207\\
1242	-13.428\\
1243	-15.869\\
1244	-15.869\\
1245	-10.986\\
1246	-17.09\\
1247	-15.869\\
1248	-10.986\\
1249	-14.648\\
1250	-14.648\\
1251	-12.207\\
1252	-14.648\\
1253	-10.986\\
1254	-9.766\\
1255	-12.207\\
1256	-8.545\\
1257	-17.09\\
1258	-18.311\\
1259	-20.752\\
1260	-29.297\\
1261	-20.752\\
1262	-10.986\\
1263	-10.986\\
1264	-8.545\\
1265	-13.428\\
1266	-12.207\\
1267	-9.766\\
1268	-15.869\\
1269	-19.531\\
1270	-20.752\\
1271	-23.193\\
1272	-23.193\\
1273	-17.09\\
1274	-15.869\\
1275	-10.986\\
1276	-10.986\\
1277	-7.324\\
1278	-8.545\\
1279	-10.986\\
1280	-17.09\\
1281	-18.311\\
1282	-17.09\\
1283	-24.414\\
1284	-25.635\\
1285	-15.869\\
1286	-19.531\\
1287	-23.193\\
1288	-26.855\\
1289	-25.635\\
1290	-17.09\\
1291	-10.986\\
1292	-7.324\\
1293	-6.104\\
1294	-8.545\\
1295	-9.766\\
1296	-8.545\\
1297	-10.986\\
1298	-17.09\\
1299	-15.869\\
1300	-13.428\\
1301	-15.869\\
1302	-12.207\\
1303	-6.104\\
1304	-9.766\\
1305	-20.752\\
1306	-17.09\\
1307	-18.311\\
1308	-15.869\\
1309	-10.986\\
1310	-17.09\\
1311	-23.193\\
1312	-23.193\\
1313	-17.09\\
1314	-26.855\\
1315	-25.635\\
1316	-18.311\\
1317	-23.193\\
1318	-25.635\\
1319	-19.531\\
1320	-15.869\\
1321	-24.414\\
1322	-35.4\\
1323	-29.297\\
1324	-17.09\\
1325	-17.09\\
1326	-18.311\\
1327	-20.752\\
1328	-23.193\\
1329	-19.531\\
1330	-17.09\\
1331	-24.414\\
1332	-25.635\\
1333	-18.311\\
1334	-15.869\\
1335	-18.311\\
1336	-28.076\\
1337	-26.855\\
1338	-24.414\\
1339	-19.531\\
1340	-17.09\\
1341	-18.311\\
1342	-13.428\\
1343	-6.104\\
1344	-6.104\\
1345	-6.104\\
1346	-9.766\\
1347	-19.531\\
1348	-15.869\\
1349	-14.648\\
1350	-13.428\\
1351	-17.09\\
1352	-13.428\\
1353	-9.766\\
1354	-12.207\\
1355	-9.766\\
1356	-8.545\\
1357	-17.09\\
1358	-20.752\\
1359	-17.09\\
1360	-10.986\\
1361	-10.986\\
1362	-13.428\\
1363	-9.766\\
1364	-10.986\\
1365	-28.076\\
1366	-20.752\\
1367	-25.635\\
1368	-35.4\\
1369	-30.518\\
1370	-24.414\\
1371	-18.311\\
1372	-18.311\\
1373	-19.531\\
1374	-21.973\\
1375	-25.635\\
1376	-25.635\\
1377	-31.738\\
1378	-34.18\\
1379	-41.504\\
1380	-32.959\\
1381	-28.076\\
1382	-35.4\\
1383	-26.855\\
1384	-29.297\\
1385	-28.076\\
1386	-17.09\\
1387	-10.986\\
1388	-12.207\\
1389	-17.09\\
1390	-14.648\\
1391	-9.766\\
1392	-13.428\\
1393	-13.428\\
1394	-7.324\\
1395	-8.545\\
1396	-12.207\\
1397	-7.324\\
1398	-14.648\\
1399	-20.752\\
1400	-13.428\\
1401	-21.973\\
1402	-30.518\\
1403	-29.297\\
1404	-35.4\\
1405	-29.297\\
1406	-28.076\\
1407	-23.193\\
1408	-12.207\\
1409	-14.648\\
1410	-15.869\\
1411	-10.986\\
1412	-12.207\\
1413	-13.428\\
1414	-3.662\\
1415	-8.545\\
1416	-17.09\\
1417	-19.531\\
1418	-20.752\\
1419	-23.193\\
1420	-20.752\\
1421	-23.193\\
1422	-18.311\\
1423	-15.869\\
1424	-15.869\\
1425	-12.207\\
1426	-18.311\\
1427	-17.09\\
1428	-12.207\\
1429	-17.09\\
1430	-23.193\\
1431	-17.09\\
1432	-13.428\\
1433	-12.207\\
1434	-14.648\\
1435	-7.324\\
1436	-8.545\\
1437	-12.207\\
1438	-17.09\\
1439	-17.09\\
1440	-14.648\\
1441	-15.869\\
1442	-18.311\\
1443	-12.207\\
1444	-9.766\\
1445	-18.311\\
1446	-26.855\\
1447	-26.855\\
1448	-21.973\\
1449	-19.531\\
1450	-17.09\\
1451	-14.648\\
1452	-13.428\\
1453	-17.09\\
1454	-17.09\\
1455	-10.986\\
1456	-14.648\\
1457	-18.311\\
1458	-19.531\\
1459	-21.973\\
1460	-21.973\\
1461	-29.297\\
1462	-35.4\\
1463	-36.621\\
1464	-26.855\\
1465	-14.648\\
1466	-10.986\\
1467	-7.324\\
1468	-10.986\\
1469	-10.986\\
1470	-14.648\\
1471	-13.428\\
1472	-12.207\\
1473	-15.869\\
1474	-20.752\\
1475	-20.752\\
1476	-24.414\\
1477	-31.738\\
1478	-29.297\\
1479	-26.855\\
1480	-18.311\\
1481	-17.09\\
1482	-20.752\\
1483	-20.752\\
1484	-17.09\\
1485	-14.648\\
1486	-18.311\\
1487	-12.207\\
1488	-13.428\\
1489	-20.752\\
1490	-17.09\\
1491	-18.311\\
1492	-32.959\\
1493	-25.635\\
1494	-20.752\\
1495	-28.076\\
1496	-28.076\\
1497	-19.531\\
1498	-13.428\\
1499	-13.428\\
1500	-19.531\\
};
\end{axis}

\begin{axis}[%
width=5.090835cm,
height=2.240107cm,
at={(0cm,15.759893cm)},
scale only axis,
xmin=1000,
xmax=1500,
xlabel={Sample index},
ymin=-58.594,
ymax=0,
ylabel={C1 load, kN},
legend style={legend cell align=left,align=left,draw=white!15!black}
]
\addplot [color=mycolor1,solid,forget plot]
  table[row sep=crcr]{%
1000	-23.193\\
1001	-28.076\\
1002	-26.855\\
1003	-20.752\\
1004	-29.297\\
1005	-28.076\\
1006	-32.959\\
1007	-25.635\\
1008	-14.648\\
1009	-20.752\\
1010	-17.09\\
1011	-18.311\\
1012	-20.752\\
1013	-7.324\\
1014	-4.883\\
1015	-4.883\\
1016	-13.428\\
1017	-23.193\\
1018	-24.414\\
1019	-25.635\\
1020	-19.531\\
1021	-12.207\\
1022	-24.414\\
1023	-19.531\\
1024	-14.648\\
1025	-20.752\\
1026	-18.311\\
1027	-17.09\\
1028	-29.297\\
1029	-25.635\\
1030	-18.311\\
1031	-28.076\\
1032	-28.076\\
1033	-21.973\\
1034	-18.311\\
1035	-15.869\\
1036	-15.869\\
1037	-21.973\\
1038	-21.973\\
1039	-21.973\\
1040	-21.973\\
1041	-23.193\\
1042	-30.518\\
1043	-29.297\\
1044	-19.531\\
1045	-18.311\\
1046	-10.986\\
1047	-14.648\\
1048	-15.869\\
1049	-12.207\\
1050	-13.428\\
1051	-18.311\\
1052	-19.531\\
1053	-18.311\\
1054	-29.297\\
1055	-21.973\\
1056	-12.207\\
1057	-12.207\\
1058	-13.428\\
1059	-17.09\\
1060	-9.766\\
1061	-10.986\\
1062	-14.648\\
1063	-9.766\\
1064	-7.324\\
1065	-13.428\\
1066	-13.428\\
1067	-21.973\\
1068	-20.752\\
1069	-25.635\\
1070	-21.973\\
1071	-25.635\\
1072	-20.752\\
1073	-20.752\\
1074	-18.311\\
1075	-18.311\\
1076	-17.09\\
1077	-30.518\\
1078	-41.504\\
1079	-41.504\\
1080	-42.725\\
1081	-28.076\\
1082	-37.842\\
1083	-46.387\\
1084	-46.387\\
1085	-35.4\\
1086	-47.607\\
1087	-58.594\\
1088	-43.945\\
1089	-34.18\\
1090	-24.414\\
1091	-20.752\\
1092	-17.09\\
1093	-21.973\\
1094	-17.09\\
1095	-12.207\\
1096	-9.766\\
1097	-12.207\\
1098	-18.311\\
1099	-18.311\\
1100	-18.311\\
1101	-21.973\\
1102	-23.193\\
1103	-30.518\\
1104	-28.076\\
1105	-30.518\\
1106	-25.635\\
1107	-20.752\\
1108	-24.414\\
1109	-18.311\\
1110	-13.428\\
1111	-19.531\\
1112	-20.752\\
1113	-12.207\\
1114	-15.869\\
1115	-12.207\\
1116	-13.428\\
1117	-13.428\\
1118	-9.766\\
1119	-10.986\\
1120	-17.09\\
1121	-23.193\\
1122	-25.635\\
1123	-26.855\\
1124	-15.869\\
1125	-20.752\\
1126	-32.959\\
1127	-24.414\\
1128	-28.076\\
1129	-29.297\\
1130	-18.311\\
1131	-12.207\\
1132	-17.09\\
1133	-18.311\\
1134	-30.518\\
1135	-36.621\\
1136	-36.621\\
1137	-28.076\\
1138	-28.076\\
1139	-25.635\\
1140	-25.635\\
1141	-25.635\\
1142	-20.752\\
1143	-15.869\\
1144	-14.648\\
1145	-13.428\\
1146	-13.428\\
1147	-20.752\\
1148	-31.738\\
1149	-26.855\\
1150	-17.09\\
1151	-15.869\\
1152	-15.869\\
1153	-9.766\\
1154	-7.324\\
1155	-7.324\\
1156	-15.869\\
1157	-10.986\\
1158	-14.648\\
1159	-14.648\\
1160	-13.428\\
1161	-9.766\\
1162	-6.104\\
1163	-4.883\\
1164	-7.324\\
1165	-18.311\\
1166	-26.855\\
1167	-28.076\\
1168	-20.752\\
1169	-13.428\\
1170	-10.986\\
1171	-6.104\\
1172	-12.207\\
1173	-14.648\\
1174	-18.311\\
1175	-26.855\\
1176	-39.063\\
1177	-47.607\\
1178	-42.725\\
1179	-37.842\\
1180	-28.076\\
1181	-30.518\\
1182	-31.738\\
1183	-34.18\\
1184	-24.414\\
1185	-23.193\\
1186	-23.193\\
1187	-21.973\\
1188	-23.193\\
1189	-18.311\\
1190	-30.518\\
1191	-37.842\\
1192	-28.076\\
1193	-18.311\\
1194	-18.311\\
1195	-30.518\\
1196	-35.4\\
1197	-40.283\\
1198	-45.166\\
1199	-45.166\\
1200	-34.18\\
1201	-32.959\\
1202	-36.621\\
1203	-41.504\\
1204	-29.297\\
1205	-17.09\\
1206	-23.193\\
1207	-20.752\\
1208	-13.428\\
1209	-18.311\\
1210	-19.531\\
1211	-14.648\\
1212	-13.428\\
1213	-15.869\\
1214	-13.428\\
1215	-17.09\\
1216	-25.635\\
1217	-25.635\\
1218	-15.869\\
1219	-14.648\\
1220	-24.414\\
1221	-40.283\\
1222	-29.297\\
1223	-20.752\\
1224	-14.648\\
1225	-18.311\\
1226	-15.869\\
1227	-12.207\\
1228	-13.428\\
1229	-15.869\\
1230	-18.311\\
1231	-20.752\\
1232	-15.869\\
1233	-23.193\\
1234	-29.297\\
1235	-25.635\\
1236	-20.752\\
1237	-23.193\\
1238	-30.518\\
1239	-21.973\\
1240	-8.545\\
1241	-13.428\\
1242	-13.428\\
1243	-17.09\\
1244	-17.09\\
1245	-13.428\\
1246	-17.09\\
1247	-19.531\\
1248	-13.428\\
1249	-15.869\\
1250	-15.869\\
1251	-12.207\\
1252	-15.869\\
1253	-9.766\\
1254	-10.986\\
1255	-8.545\\
1256	-8.545\\
1257	-14.648\\
1258	-20.752\\
1259	-24.414\\
1260	-35.4\\
1261	-24.414\\
1262	-13.428\\
1263	-12.207\\
1264	-10.986\\
1265	-13.428\\
1266	-10.986\\
1267	-12.207\\
1268	-14.648\\
1269	-24.414\\
1270	-21.973\\
1271	-26.855\\
1272	-23.193\\
1273	-17.09\\
1274	-13.428\\
1275	-10.986\\
1276	-8.545\\
1277	-6.104\\
1278	-9.766\\
1279	-14.648\\
1280	-18.311\\
1281	-18.311\\
1282	-19.531\\
1283	-29.297\\
1284	-25.635\\
1285	-18.311\\
1286	-24.414\\
1287	-24.414\\
1288	-31.738\\
1289	-26.855\\
1290	-17.09\\
1291	-9.766\\
1292	-6.104\\
1293	-7.324\\
1294	-9.766\\
1295	-8.545\\
1296	-8.545\\
1297	-12.207\\
1298	-17.09\\
1299	-15.869\\
1300	-15.869\\
1301	-19.531\\
1302	-13.428\\
1303	-4.883\\
1304	-9.766\\
1305	-21.973\\
1306	-19.531\\
1307	-21.973\\
1308	-18.311\\
1309	-14.648\\
1310	-19.531\\
1311	-25.635\\
1312	-25.635\\
1313	-19.531\\
1314	-26.855\\
1315	-26.855\\
1316	-23.193\\
1317	-28.076\\
1318	-26.855\\
1319	-20.752\\
1320	-19.531\\
1321	-29.297\\
1322	-40.283\\
1323	-30.518\\
1324	-18.311\\
1325	-17.09\\
1326	-18.311\\
1327	-21.973\\
1328	-26.855\\
1329	-19.531\\
1330	-19.531\\
1331	-26.855\\
1332	-28.076\\
1333	-20.752\\
1334	-14.648\\
1335	-20.752\\
1336	-34.18\\
1337	-30.518\\
1338	-31.738\\
1339	-21.973\\
1340	-17.09\\
1341	-17.09\\
1342	-10.986\\
1343	-6.104\\
1344	-6.104\\
1345	-4.883\\
1346	-10.986\\
1347	-19.531\\
1348	-18.311\\
1349	-15.869\\
1350	-14.648\\
1351	-20.752\\
1352	-14.648\\
1353	-9.766\\
1354	-13.428\\
1355	-8.545\\
1356	-10.986\\
1357	-18.311\\
1358	-23.193\\
1359	-17.09\\
1360	-9.766\\
1361	-10.986\\
1362	-12.207\\
1363	-9.766\\
1364	-13.428\\
1365	-31.738\\
1366	-25.635\\
1367	-31.738\\
1368	-37.842\\
1369	-36.621\\
1370	-26.855\\
1371	-20.752\\
1372	-18.311\\
1373	-21.973\\
1374	-23.193\\
1375	-29.297\\
1376	-29.297\\
1377	-36.621\\
1378	-42.725\\
1379	-46.387\\
1380	-37.842\\
1381	-35.4\\
1382	-40.283\\
1383	-31.738\\
1384	-34.18\\
1385	-31.738\\
1386	-18.311\\
1387	-12.207\\
1388	-13.428\\
1389	-18.311\\
1390	-17.09\\
1391	-8.545\\
1392	-14.648\\
1393	-13.428\\
1394	-6.104\\
1395	-10.986\\
1396	-12.207\\
1397	-7.324\\
1398	-18.311\\
1399	-20.752\\
1400	-14.648\\
1401	-23.193\\
1402	-37.842\\
1403	-32.959\\
1404	-37.842\\
1405	-32.959\\
1406	-29.297\\
1407	-20.752\\
1408	-14.648\\
1409	-17.09\\
1410	-13.428\\
1411	-10.986\\
1412	-14.648\\
1413	-13.428\\
1414	-3.662\\
1415	-7.324\\
1416	-15.869\\
1417	-20.752\\
1418	-21.973\\
1419	-23.193\\
1420	-21.973\\
1421	-26.855\\
1422	-21.973\\
1423	-17.09\\
1424	-17.09\\
1425	-14.648\\
1426	-20.752\\
1427	-18.311\\
1428	-13.428\\
1429	-18.311\\
1430	-26.855\\
1431	-20.752\\
1432	-15.869\\
1433	-13.428\\
1434	-14.648\\
1435	-12.207\\
1436	-8.545\\
1437	-14.648\\
1438	-19.531\\
1439	-18.311\\
1440	-15.869\\
1441	-19.531\\
1442	-18.311\\
1443	-10.986\\
1444	-10.986\\
1445	-21.973\\
1446	-30.518\\
1447	-29.297\\
1448	-23.193\\
1449	-21.973\\
1450	-19.531\\
1451	-18.311\\
1452	-15.869\\
1453	-19.531\\
1454	-17.09\\
1455	-13.428\\
1456	-17.09\\
1457	-19.531\\
1458	-21.973\\
1459	-25.635\\
1460	-25.635\\
1461	-32.959\\
1462	-41.504\\
1463	-45.166\\
1464	-31.738\\
1465	-17.09\\
1466	-13.428\\
1467	-8.545\\
1468	-10.986\\
1469	-13.428\\
1470	-17.09\\
1471	-18.311\\
1472	-13.428\\
1473	-19.531\\
1474	-24.414\\
1475	-25.635\\
1476	-26.855\\
1477	-39.063\\
1478	-35.4\\
1479	-25.635\\
1480	-20.752\\
1481	-20.752\\
1482	-19.531\\
1483	-23.193\\
1484	-19.531\\
1485	-15.869\\
1486	-17.09\\
1487	-10.986\\
1488	-14.648\\
1489	-28.076\\
1490	-19.531\\
1491	-15.869\\
1492	-34.18\\
1493	-28.076\\
1494	-21.973\\
1495	-34.18\\
1496	-31.738\\
1497	-20.752\\
1498	-15.869\\
1499	-13.428\\
1500	-23.193\\
};
\end{axis}

\begin{axis}[%
width=5.090835cm,
height=2.240107cm,
at={(0cm,3.939973cm)},
scale only axis,
xmin=1000,
xmax=1500,
xlabel={Sample index},
ymin=-400,
ymax=0,
ylabel={C7 load, kN},
legend style={legend cell align=left,align=left,draw=white!15!black}
]
\addplot [color=mycolor1,solid,forget plot]
  table[row sep=crcr]{%
1000	-112.305\\
1001	-144.043\\
1002	-117.188\\
1003	-111.084\\
1004	-156.25\\
1005	-147.705\\
1006	-175.781\\
1007	-134.277\\
1008	-68.359\\
1009	-80.566\\
1010	-80.566\\
1011	-100.098\\
1012	-81.787\\
1013	-34.18\\
1014	-24.414\\
1015	-23.193\\
1016	-73.242\\
1017	-128.174\\
1018	-133.057\\
1019	-137.939\\
1020	-92.773\\
1021	-56.152\\
1022	-125.732\\
1023	-86.67\\
1024	-68.359\\
1025	-114.746\\
1026	-100.098\\
1027	-85.449\\
1028	-142.822\\
1029	-133.057\\
1030	-102.539\\
1031	-150.146\\
1032	-145.264\\
1033	-113.525\\
1034	-92.773\\
1035	-72.021\\
1036	-86.67\\
1037	-112.305\\
1038	-103.76\\
1039	-109.863\\
1040	-112.305\\
1041	-122.07\\
1042	-175.781\\
1043	-155.029\\
1044	-102.539\\
1045	-92.773\\
1046	-53.711\\
1047	-58.594\\
1048	-79.346\\
1049	-59.814\\
1050	-62.256\\
1051	-89.111\\
1052	-102.539\\
1053	-95.215\\
1054	-151.367\\
1055	-111.084\\
1056	-64.697\\
1057	-51.27\\
1058	-69.58\\
1059	-81.787\\
1060	-43.945\\
1061	-45.166\\
1062	-67.139\\
1063	-51.27\\
1064	-42.725\\
1065	-61.035\\
1066	-70.801\\
1067	-109.863\\
1068	-104.98\\
1069	-129.395\\
1070	-120.85\\
1071	-137.939\\
1072	-109.863\\
1073	-107.422\\
1074	-92.773\\
1075	-91.553\\
1076	-87.891\\
1077	-157.471\\
1078	-224.609\\
1079	-231.934\\
1080	-225.83\\
1081	-140.381\\
1082	-205.078\\
1083	-252.686\\
1084	-261.23\\
1085	-200.195\\
1086	-270.996\\
1087	-335.693\\
1088	-235.596\\
1089	-191.65\\
1090	-128.174\\
1091	-98.877\\
1092	-84.229\\
1093	-104.98\\
1094	-74.463\\
1095	-50.049\\
1096	-48.828\\
1097	-63.477\\
1098	-95.215\\
1099	-86.67\\
1100	-89.111\\
1101	-119.629\\
1102	-123.291\\
1103	-167.236\\
1104	-134.277\\
1105	-169.678\\
1106	-122.07\\
1107	-108.643\\
1108	-125.732\\
1109	-89.111\\
1110	-80.566\\
1111	-97.656\\
1112	-98.877\\
1113	-68.359\\
1114	-73.242\\
1115	-54.932\\
1116	-61.035\\
1117	-64.697\\
1118	-45.166\\
1119	-58.594\\
1120	-72.021\\
1121	-117.188\\
1122	-122.07\\
1123	-136.719\\
1124	-76.904\\
1125	-109.863\\
1126	-173.34\\
1127	-131.836\\
1128	-157.471\\
1129	-157.471\\
1130	-85.449\\
1131	-59.814\\
1132	-85.449\\
1133	-98.877\\
1134	-161.133\\
1135	-202.637\\
1136	-198.975\\
1137	-145.264\\
1138	-150.146\\
1139	-135.498\\
1140	-131.836\\
1141	-126.953\\
1142	-85.449\\
1143	-76.904\\
1144	-69.58\\
1145	-62.256\\
1146	-70.801\\
1147	-107.422\\
1148	-164.795\\
1149	-125.732\\
1150	-86.67\\
1151	-73.242\\
1152	-70.801\\
1153	-42.725\\
1154	-31.738\\
1155	-39.063\\
1156	-72.021\\
1157	-57.373\\
1158	-59.814\\
1159	-67.139\\
1160	-63.477\\
1161	-43.945\\
1162	-29.297\\
1163	-25.635\\
1164	-43.945\\
1165	-96.436\\
1166	-140.381\\
1167	-155.029\\
1168	-101.318\\
1169	-69.58\\
1170	-50.049\\
1171	-34.18\\
1172	-83.008\\
1173	-81.787\\
1174	-119.629\\
1175	-145.264\\
1176	-230.713\\
1177	-262.451\\
1178	-211.182\\
1179	-202.637\\
1180	-136.719\\
1181	-151.367\\
1182	-159.912\\
1183	-172.119\\
1184	-129.395\\
1185	-118.408\\
1186	-115.967\\
1187	-103.76\\
1188	-123.291\\
1189	-87.891\\
1190	-153.809\\
1191	-189.209\\
1192	-133.057\\
1193	-78.125\\
1194	-85.449\\
1195	-146.484\\
1196	-181.885\\
1197	-229.492\\
1198	-241.699\\
1199	-240.479\\
1200	-180.664\\
1201	-163.574\\
1202	-187.988\\
1203	-222.168\\
1204	-139.16\\
1205	-81.787\\
1206	-122.07\\
1207	-102.539\\
1208	-65.918\\
1209	-87.891\\
1210	-98.877\\
1211	-76.904\\
1212	-58.594\\
1213	-80.566\\
1214	-65.918\\
1215	-91.553\\
1216	-133.057\\
1217	-122.07\\
1218	-79.346\\
1219	-83.008\\
1220	-126.953\\
1221	-209.961\\
1222	-150.146\\
1223	-92.773\\
1224	-69.58\\
1225	-83.008\\
1226	-74.463\\
1227	-56.152\\
1228	-62.256\\
1229	-74.463\\
1230	-96.436\\
1231	-107.422\\
1232	-72.021\\
1233	-122.07\\
1234	-144.043\\
1235	-137.939\\
1236	-106.201\\
1237	-118.408\\
1238	-158.691\\
1239	-101.318\\
1240	-41.504\\
1241	-58.594\\
1242	-65.918\\
1243	-91.553\\
1244	-81.787\\
1245	-59.814\\
1246	-96.436\\
1247	-91.553\\
1248	-65.918\\
1249	-83.008\\
1250	-76.904\\
1251	-67.139\\
1252	-79.346\\
1253	-46.387\\
1254	-53.711\\
1255	-40.283\\
1256	-43.945\\
1257	-86.67\\
1258	-100.098\\
1259	-137.939\\
1260	-177.002\\
1261	-115.967\\
1262	-68.359\\
1263	-48.828\\
1264	-48.828\\
1265	-73.242\\
1266	-46.387\\
1267	-52.49\\
1268	-70.801\\
1269	-119.629\\
1270	-107.422\\
1271	-150.146\\
1272	-102.539\\
1273	-91.553\\
1274	-47.607\\
1275	-47.607\\
1276	-31.738\\
1277	-34.18\\
1278	-41.504\\
1279	-75.684\\
1280	-83.008\\
1281	-92.773\\
1282	-97.656\\
1283	-152.588\\
1284	-130.615\\
1285	-97.656\\
1286	-119.629\\
1287	-129.395\\
1288	-167.236\\
1289	-133.057\\
1290	-84.229\\
1291	-43.945\\
1292	-31.738\\
1293	-34.18\\
1294	-41.504\\
1295	-43.945\\
1296	-40.283\\
1297	-56.152\\
1298	-87.891\\
1299	-79.346\\
1300	-81.787\\
1301	-96.436\\
1302	-58.594\\
1303	-29.297\\
1304	-64.697\\
1305	-97.656\\
1306	-96.436\\
1307	-109.863\\
1308	-93.994\\
1309	-69.58\\
1310	-97.656\\
1311	-137.939\\
1312	-139.16\\
1313	-97.656\\
1314	-162.354\\
1315	-128.174\\
1316	-118.408\\
1317	-137.939\\
1318	-137.939\\
1319	-103.76\\
1320	-90.332\\
1321	-153.809\\
1322	-214.844\\
1323	-164.795\\
1324	-92.773\\
1325	-92.773\\
1326	-91.553\\
1327	-113.525\\
1328	-136.719\\
1329	-100.098\\
1330	-104.98\\
1331	-140.381\\
1332	-153.809\\
1333	-103.76\\
1334	-79.346\\
1335	-104.98\\
1336	-175.781\\
1337	-159.912\\
1338	-162.354\\
1339	-95.215\\
1340	-84.229\\
1341	-83.008\\
1342	-52.49\\
1343	-34.18\\
1344	-28.076\\
1345	-23.193\\
1346	-59.814\\
1347	-86.67\\
1348	-104.98\\
1349	-76.904\\
1350	-87.891\\
1351	-97.656\\
1352	-59.814\\
1353	-63.477\\
1354	-61.035\\
1355	-45.166\\
1356	-57.373\\
1357	-97.656\\
1358	-124.512\\
1359	-78.125\\
1360	-56.152\\
1361	-46.387\\
1362	-59.814\\
1363	-41.504\\
1364	-91.553\\
1365	-158.691\\
1366	-122.07\\
1367	-167.236\\
1368	-194.092\\
1369	-197.754\\
1370	-140.381\\
1371	-106.201\\
1372	-106.201\\
1373	-104.98\\
1374	-120.85\\
1375	-150.146\\
1376	-147.705\\
1377	-200.195\\
1378	-220.947\\
1379	-263.672\\
1380	-178.223\\
1381	-190.43\\
1382	-212.402\\
1383	-163.574\\
1384	-189.209\\
1385	-163.574\\
1386	-91.553\\
1387	-58.594\\
1388	-62.256\\
1389	-95.215\\
1390	-75.684\\
1391	-51.27\\
1392	-72.021\\
1393	-52.49\\
1394	-40.283\\
1395	-48.828\\
1396	-56.152\\
1397	-40.283\\
1398	-76.904\\
1399	-109.863\\
1400	-78.125\\
1401	-134.277\\
1402	-195.313\\
1403	-178.223\\
1404	-222.168\\
1405	-150.146\\
1406	-161.133\\
1407	-111.084\\
1408	-67.139\\
1409	-85.449\\
1410	-67.139\\
1411	-48.828\\
1412	-72.021\\
1413	-41.504\\
1414	-25.635\\
1415	-36.621\\
1416	-79.346\\
1417	-104.98\\
1418	-128.174\\
1419	-124.512\\
1420	-117.188\\
1421	-136.719\\
1422	-111.084\\
1423	-90.332\\
1424	-90.332\\
1425	-73.242\\
1426	-115.967\\
1427	-78.125\\
1428	-67.139\\
1429	-100.098\\
1430	-137.939\\
1431	-103.76\\
1432	-76.904\\
1433	-64.697\\
1434	-74.463\\
1435	-51.27\\
1436	-50.049\\
1437	-72.021\\
1438	-86.67\\
1439	-98.877\\
1440	-79.346\\
1441	-101.318\\
1442	-92.773\\
1443	-52.49\\
1444	-54.932\\
1445	-112.305\\
1446	-158.691\\
1447	-153.809\\
1448	-129.395\\
1449	-122.07\\
1450	-92.773\\
1451	-89.111\\
1452	-70.801\\
1453	-102.539\\
1454	-90.332\\
1455	-54.932\\
1456	-84.229\\
1457	-101.318\\
1458	-118.408\\
1459	-129.395\\
1460	-129.395\\
1461	-175.781\\
1462	-216.064\\
1463	-244.141\\
1464	-153.809\\
1465	-85.449\\
1466	-54.932\\
1467	-37.842\\
1468	-57.373\\
1469	-53.711\\
1470	-95.215\\
1471	-85.449\\
1472	-75.684\\
1473	-102.539\\
1474	-118.408\\
1475	-141.602\\
1476	-148.926\\
1477	-208.74\\
1478	-181.885\\
1479	-136.719\\
1480	-90.332\\
1481	-92.773\\
1482	-95.215\\
1483	-123.291\\
1484	-87.891\\
1485	-89.111\\
1486	-87.891\\
1487	-47.607\\
1488	-81.787\\
1489	-128.174\\
1490	-90.332\\
1491	-96.436\\
1492	-170.898\\
1493	-129.395\\
1494	-122.07\\
1495	-177.002\\
1496	-166.016\\
1497	-102.539\\
1498	-64.697\\
1499	-65.918\\
1500	-114.746\\
};
\end{axis}

\begin{axis}[%
width=5.090835cm,
height=2.240107cm,
at={(6.698467cm,7.879947cm)},
scale only axis,
xmin=1000,
xmax=1500,
xlabel={Sample index},
ymin=-404.053,
ymax=0,
ylabel={C6 load, kN},
legend style={legend cell align=left,align=left,draw=white!15!black}
]
\addplot [color=mycolor1,solid,forget plot]
  table[row sep=crcr]{%
1000	-140.381\\
1001	-178.223\\
1002	-147.705\\
1003	-139.16\\
1004	-192.871\\
1005	-185.547\\
1006	-219.727\\
1007	-169.678\\
1008	-86.67\\
1009	-107.422\\
1010	-102.539\\
1011	-126.953\\
1012	-103.76\\
1013	-41.504\\
1014	-31.738\\
1015	-28.076\\
1016	-89.111\\
1017	-157.471\\
1018	-167.236\\
1019	-174.561\\
1020	-124.512\\
1021	-74.463\\
1022	-158.691\\
1023	-111.084\\
1024	-83.008\\
1025	-142.822\\
1026	-123.291\\
1027	-103.76\\
1028	-173.34\\
1029	-164.795\\
1030	-124.512\\
1031	-180.664\\
1032	-179.443\\
1033	-142.822\\
1034	-117.188\\
1035	-89.111\\
1036	-108.643\\
1037	-137.939\\
1038	-130.615\\
1039	-136.719\\
1040	-140.381\\
1041	-151.367\\
1042	-213.623\\
1043	-191.65\\
1044	-126.953\\
1045	-115.967\\
1046	-64.697\\
1047	-69.58\\
1048	-96.436\\
1049	-74.463\\
1050	-78.125\\
1051	-111.084\\
1052	-128.174\\
1053	-119.629\\
1054	-190.43\\
1055	-139.16\\
1056	-81.787\\
1057	-63.477\\
1058	-85.449\\
1059	-101.318\\
1060	-51.27\\
1061	-58.594\\
1062	-81.787\\
1063	-65.918\\
1064	-56.152\\
1065	-74.463\\
1066	-86.67\\
1067	-139.16\\
1068	-130.615\\
1069	-163.574\\
1070	-146.484\\
1071	-170.898\\
1072	-139.16\\
1073	-133.057\\
1074	-115.967\\
1075	-111.084\\
1076	-111.084\\
1077	-196.533\\
1078	-280.762\\
1079	-291.748\\
1080	-280.762\\
1081	-175.781\\
1082	-240.479\\
1083	-303.955\\
1084	-317.383\\
1085	-239.258\\
1086	-313.721\\
1087	-404.053\\
1088	-289.307\\
1089	-238.037\\
1090	-158.691\\
1091	-122.07\\
1092	-102.539\\
1093	-129.395\\
1094	-92.773\\
1095	-61.035\\
1096	-59.814\\
1097	-75.684\\
1098	-118.408\\
1099	-107.422\\
1100	-111.084\\
1101	-146.484\\
1102	-152.588\\
1103	-207.52\\
1104	-167.236\\
1105	-208.74\\
1106	-148.926\\
1107	-130.615\\
1108	-151.367\\
1109	-108.643\\
1110	-98.877\\
1111	-122.07\\
1112	-123.291\\
1113	-85.449\\
1114	-95.215\\
1115	-67.139\\
1116	-76.904\\
1117	-81.787\\
1118	-57.373\\
1119	-75.684\\
1120	-93.994\\
1121	-151.367\\
1122	-152.588\\
1123	-170.898\\
1124	-97.656\\
1125	-140.381\\
1126	-211.182\\
1127	-161.133\\
1128	-186.768\\
1129	-191.65\\
1130	-112.305\\
1131	-75.684\\
1132	-107.422\\
1133	-123.291\\
1134	-205.078\\
1135	-252.686\\
1136	-252.686\\
1137	-184.326\\
1138	-189.209\\
1139	-167.236\\
1140	-163.574\\
1141	-162.354\\
1142	-108.643\\
1143	-96.436\\
1144	-86.67\\
1145	-78.125\\
1146	-87.891\\
1147	-133.057\\
1148	-203.857\\
1149	-161.133\\
1150	-109.863\\
1151	-92.773\\
1152	-89.111\\
1153	-52.49\\
1154	-41.504\\
1155	-47.607\\
1156	-90.332\\
1157	-67.139\\
1158	-78.125\\
1159	-85.449\\
1160	-80.566\\
1161	-54.932\\
1162	-37.842\\
1163	-30.518\\
1164	-54.932\\
1165	-119.629\\
1166	-172.119\\
1167	-194.092\\
1168	-136.719\\
1169	-91.553\\
1170	-64.697\\
1171	-43.945\\
1172	-98.877\\
1173	-91.553\\
1174	-137.939\\
1175	-168.457\\
1176	-275.879\\
1177	-316.162\\
1178	-260.01\\
1179	-249.023\\
1180	-159.912\\
1181	-185.547\\
1182	-195.313\\
1183	-212.402\\
1184	-158.691\\
1185	-144.043\\
1186	-142.822\\
1187	-129.395\\
1188	-156.25\\
1189	-109.863\\
1190	-195.313\\
1191	-234.375\\
1192	-175.781\\
1193	-104.98\\
1194	-111.084\\
1195	-192.871\\
1196	-234.375\\
1197	-289.307\\
1198	-303.955\\
1199	-302.734\\
1200	-228.271\\
1201	-205.078\\
1202	-235.596\\
1203	-275.879\\
1204	-170.898\\
1205	-104.98\\
1206	-152.588\\
1207	-122.07\\
1208	-79.346\\
1209	-109.863\\
1210	-122.07\\
1211	-95.215\\
1212	-75.684\\
1213	-101.318\\
1214	-80.566\\
1215	-115.967\\
1216	-162.354\\
1217	-147.705\\
1218	-100.098\\
1219	-101.318\\
1220	-157.471\\
1221	-261.23\\
1222	-185.547\\
1223	-118.408\\
1224	-85.449\\
1225	-101.318\\
1226	-90.332\\
1227	-67.139\\
1228	-76.904\\
1229	-92.773\\
1230	-120.85\\
1231	-134.277\\
1232	-89.111\\
1233	-156.25\\
1234	-179.443\\
1235	-170.898\\
1236	-140.381\\
1237	-150.146\\
1238	-202.637\\
1239	-125.732\\
1240	-53.711\\
1241	-78.125\\
1242	-85.449\\
1243	-119.629\\
1244	-101.318\\
1245	-74.463\\
1246	-118.408\\
1247	-112.305\\
1248	-79.346\\
1249	-102.539\\
1250	-90.332\\
1251	-80.566\\
1252	-95.215\\
1253	-54.932\\
1254	-68.359\\
1255	-47.607\\
1256	-51.27\\
1257	-106.201\\
1258	-122.07\\
1259	-167.236\\
1260	-219.727\\
1261	-142.822\\
1262	-83.008\\
1263	-63.477\\
1264	-62.256\\
1265	-91.553\\
1266	-54.932\\
1267	-70.801\\
1268	-85.449\\
1269	-150.146\\
1270	-133.057\\
1271	-190.43\\
1272	-124.512\\
1273	-117.188\\
1274	-56.152\\
1275	-62.256\\
1276	-34.18\\
1277	-46.387\\
1278	-51.27\\
1279	-95.215\\
1280	-100.098\\
1281	-117.188\\
1282	-123.291\\
1283	-197.754\\
1284	-170.898\\
1285	-128.174\\
1286	-153.809\\
1287	-163.574\\
1288	-213.623\\
1289	-161.133\\
1290	-104.98\\
1291	-53.711\\
1292	-40.283\\
1293	-41.504\\
1294	-52.49\\
1295	-54.932\\
1296	-51.27\\
1297	-70.801\\
1298	-108.643\\
1299	-98.877\\
1300	-103.76\\
1301	-118.408\\
1302	-69.58\\
1303	-36.621\\
1304	-81.787\\
1305	-125.732\\
1306	-125.732\\
1307	-142.822\\
1308	-118.408\\
1309	-87.891\\
1310	-123.291\\
1311	-172.119\\
1312	-172.119\\
1313	-120.85\\
1314	-207.52\\
1315	-155.029\\
1316	-151.367\\
1317	-173.34\\
1318	-172.119\\
1319	-130.615\\
1320	-112.305\\
1321	-192.871\\
1322	-266.113\\
1323	-202.637\\
1324	-124.512\\
1325	-118.408\\
1326	-115.967\\
1327	-150.146\\
1328	-173.34\\
1329	-125.732\\
1330	-134.277\\
1331	-177.002\\
1332	-192.871\\
1333	-120.85\\
1334	-97.656\\
1335	-130.615\\
1336	-218.506\\
1337	-195.313\\
1338	-203.857\\
1339	-115.967\\
1340	-108.643\\
1341	-101.318\\
1342	-63.477\\
1343	-41.504\\
1344	-34.18\\
1345	-29.297\\
1346	-78.125\\
1347	-108.643\\
1348	-131.836\\
1349	-93.994\\
1350	-107.422\\
1351	-119.629\\
1352	-72.021\\
1353	-79.346\\
1354	-73.242\\
1355	-54.932\\
1356	-75.684\\
1357	-120.85\\
1358	-156.25\\
1359	-95.215\\
1360	-73.242\\
1361	-58.594\\
1362	-74.463\\
1363	-45.166\\
1364	-117.188\\
1365	-195.313\\
1366	-163.574\\
1367	-216.064\\
1368	-241.699\\
1369	-249.023\\
1370	-181.885\\
1371	-131.836\\
1372	-130.615\\
1373	-129.395\\
1374	-155.029\\
1375	-184.326\\
1376	-181.885\\
1377	-247.803\\
1378	-270.996\\
1379	-328.369\\
1380	-217.285\\
1381	-238.037\\
1382	-264.893\\
1383	-205.078\\
1384	-239.258\\
1385	-200.195\\
1386	-113.525\\
1387	-74.463\\
1388	-79.346\\
1389	-117.188\\
1390	-97.656\\
1391	-64.697\\
1392	-90.332\\
1393	-63.477\\
1394	-48.828\\
1395	-64.697\\
1396	-72.021\\
1397	-48.828\\
1398	-100.098\\
1399	-135.498\\
1400	-97.656\\
1401	-173.34\\
1402	-241.699\\
1403	-220.947\\
1404	-279.541\\
1405	-187.988\\
1406	-206.299\\
1407	-134.277\\
1408	-85.449\\
1409	-108.643\\
1410	-87.891\\
1411	-63.477\\
1412	-92.773\\
1413	-50.049\\
1414	-32.959\\
1415	-43.945\\
1416	-98.877\\
1417	-125.732\\
1418	-157.471\\
1419	-152.588\\
1420	-146.484\\
1421	-168.457\\
1422	-135.498\\
1423	-108.643\\
1424	-112.305\\
1425	-89.111\\
1426	-146.484\\
1427	-96.436\\
1428	-80.566\\
1429	-120.85\\
1430	-164.795\\
1431	-123.291\\
1432	-93.994\\
1433	-80.566\\
1434	-91.553\\
1435	-61.035\\
1436	-61.035\\
1437	-87.891\\
1438	-109.863\\
1439	-124.512\\
1440	-96.436\\
1441	-128.174\\
1442	-114.746\\
1443	-61.035\\
1444	-69.58\\
1445	-135.498\\
1446	-190.43\\
1447	-186.768\\
1448	-157.471\\
1449	-152.588\\
1450	-114.746\\
1451	-109.863\\
1452	-87.891\\
1453	-125.732\\
1454	-117.188\\
1455	-70.801\\
1456	-107.422\\
1457	-126.953\\
1458	-150.146\\
1459	-164.795\\
1460	-163.574\\
1461	-222.168\\
1462	-267.334\\
1463	-302.734\\
1464	-190.43\\
1465	-106.201\\
1466	-68.359\\
1467	-45.166\\
1468	-70.801\\
1469	-64.697\\
1470	-119.629\\
1471	-102.539\\
1472	-93.994\\
1473	-124.512\\
1474	-145.264\\
1475	-172.119\\
1476	-183.105\\
1477	-258.789\\
1478	-231.934\\
1479	-177.002\\
1480	-120.85\\
1481	-120.85\\
1482	-123.291\\
1483	-157.471\\
1484	-108.643\\
1485	-113.525\\
1486	-107.422\\
1487	-58.594\\
1488	-102.539\\
1489	-152.588\\
1490	-107.422\\
1491	-115.967\\
1492	-216.064\\
1493	-159.912\\
1494	-156.25\\
1495	-219.727\\
1496	-206.299\\
1497	-129.395\\
1498	-85.449\\
1499	-85.449\\
1500	-146.484\\
};
\end{axis}

\begin{axis}[%
width=5.090835cm,
height=2.240107cm,
at={(0cm,7.879947cm)},
scale only axis,
xmin=1000,
xmax=1500,
xlabel={Sample index},
ymin=-28.076,
ymax=0,
ylabel={C5 load, kN},
legend style={legend cell align=left,align=left,draw=white!15!black}
]
\addplot [color=mycolor1,solid,forget plot]
  table[row sep=crcr]{%
1000	-10.986\\
1001	-15.869\\
1002	-10.986\\
1003	-12.207\\
1004	-13.428\\
1005	-15.869\\
1006	-14.648\\
1007	-14.648\\
1008	-7.324\\
1009	-8.545\\
1010	-10.986\\
1011	-9.766\\
1012	-9.766\\
1013	-6.104\\
1014	-2.441\\
1015	-3.662\\
1016	-2.441\\
1017	-12.207\\
1018	-13.428\\
1019	-12.207\\
1020	-10.986\\
1021	-6.104\\
1022	-6.104\\
1023	-1.221\\
1024	-6.104\\
1025	-10.986\\
1026	-12.207\\
1027	-8.545\\
1028	-14.648\\
1029	-14.648\\
1030	-9.766\\
1031	-13.428\\
1032	-12.207\\
1033	-9.766\\
1034	-9.766\\
1035	-8.545\\
1036	-8.545\\
1037	-12.207\\
1038	-10.986\\
1039	-10.986\\
1040	-10.986\\
1041	-10.986\\
1042	-14.648\\
1043	-15.869\\
1044	-9.766\\
1045	-8.545\\
1046	-8.545\\
1047	-6.104\\
1048	-7.324\\
1049	-7.324\\
1050	-4.883\\
1051	-9.766\\
1052	-10.986\\
1053	-8.545\\
1054	-13.428\\
1055	-15.869\\
1056	-8.545\\
1057	-6.104\\
1058	-6.104\\
1059	-9.766\\
1060	-6.104\\
1061	-4.883\\
1062	-7.324\\
1063	-7.324\\
1064	-3.662\\
1065	-6.104\\
1066	-8.545\\
1067	-10.986\\
1068	-10.986\\
1069	-9.766\\
1070	-13.428\\
1071	-10.986\\
1072	-12.207\\
1073	-10.986\\
1074	-10.986\\
1075	-7.324\\
1076	-8.545\\
1077	-14.648\\
1078	-20.752\\
1079	-20.752\\
1080	-19.531\\
1081	-13.428\\
1082	-15.869\\
1083	-24.414\\
1084	-23.193\\
1085	-15.869\\
1086	-21.973\\
1087	-28.076\\
1088	-23.193\\
1089	-17.09\\
1090	-14.648\\
1091	-13.428\\
1092	-9.766\\
1093	-10.986\\
1094	-12.207\\
1095	-4.883\\
1096	-4.883\\
1097	-7.324\\
1098	-10.986\\
1099	-9.766\\
1100	-9.766\\
1101	-10.986\\
1102	-13.428\\
1103	-14.648\\
1104	-15.869\\
1105	-14.648\\
1106	-17.09\\
1107	-12.207\\
1108	-12.207\\
1109	-10.986\\
1110	-8.545\\
1111	-9.766\\
1112	-12.207\\
1113	-8.545\\
1114	-9.766\\
1115	-7.324\\
1116	-6.104\\
1117	-8.545\\
1118	-6.104\\
1119	-4.883\\
1120	-8.545\\
1121	-13.428\\
1122	-12.207\\
1123	-10.986\\
1124	-7.324\\
1125	-8.545\\
1126	-19.531\\
1127	-14.648\\
1128	-10.986\\
1129	-17.09\\
1130	-12.207\\
1131	-6.104\\
1132	-9.766\\
1133	-8.545\\
1134	-14.648\\
1135	-15.869\\
1136	-19.531\\
1137	-14.648\\
1138	-14.648\\
1139	-13.428\\
1140	-13.428\\
1141	-14.648\\
1142	-10.986\\
1143	-7.324\\
1144	-7.324\\
1145	-6.104\\
1146	-8.545\\
1147	-10.986\\
1148	-15.869\\
1149	-14.648\\
1150	-8.545\\
1151	-8.545\\
1152	-9.766\\
1153	-6.104\\
1154	-2.441\\
1155	-4.883\\
1156	-7.324\\
1157	-8.545\\
1158	-6.104\\
1159	-7.324\\
1160	-6.104\\
1161	-4.883\\
1162	-4.883\\
1163	-1.221\\
1164	-3.662\\
1165	-12.207\\
1166	-14.648\\
1167	-15.869\\
1168	-12.207\\
1169	-7.324\\
1170	-6.104\\
1171	-3.662\\
1172	-7.324\\
1173	-8.545\\
1174	-9.766\\
1175	-12.207\\
1176	-18.311\\
1177	-18.311\\
1178	-19.531\\
1179	-18.311\\
1180	-15.869\\
1181	-15.869\\
1182	-17.09\\
1183	-18.311\\
1184	-12.207\\
1185	-10.986\\
1186	-12.207\\
1187	-10.986\\
1188	-12.207\\
1189	-9.766\\
1190	-15.869\\
1191	-18.311\\
1192	-12.207\\
1193	-7.324\\
1194	-9.766\\
1195	-17.09\\
1196	-18.311\\
1197	-18.311\\
1198	-21.973\\
1199	-20.752\\
1200	-17.09\\
1201	-15.869\\
1202	-18.311\\
1203	-20.752\\
1204	-17.09\\
1205	-8.545\\
1206	-14.648\\
1207	-12.207\\
1208	-7.324\\
1209	-10.986\\
1210	-9.766\\
1211	-8.545\\
1212	-7.324\\
1213	-8.545\\
1214	-8.545\\
1215	-9.766\\
1216	-13.428\\
1217	-14.648\\
1218	-7.324\\
1219	-8.545\\
1220	-12.207\\
1221	-17.09\\
1222	-15.869\\
1223	-8.545\\
1224	-8.545\\
1225	-9.766\\
1226	-8.545\\
1227	-7.324\\
1228	-6.104\\
1229	-8.545\\
1230	-9.766\\
1231	-9.766\\
1232	-9.766\\
1233	-12.207\\
1234	-14.648\\
1235	-15.869\\
1236	-10.986\\
1237	-13.428\\
1238	-17.09\\
1239	-12.207\\
1240	-3.662\\
1241	-9.766\\
1242	-8.545\\
1243	-9.766\\
1244	-8.545\\
1245	-8.545\\
1246	-8.545\\
1247	-10.986\\
1248	-7.324\\
1249	-7.324\\
1250	-9.766\\
1251	-6.104\\
1252	-8.545\\
1253	-6.104\\
1254	-3.662\\
1255	-6.104\\
1256	-4.883\\
1257	-10.986\\
1258	-12.207\\
1259	-12.207\\
1260	-18.311\\
1261	-10.986\\
1262	-4.883\\
1263	-6.104\\
1264	-6.104\\
1265	-8.545\\
1266	-6.104\\
1267	-8.545\\
1268	-7.324\\
1269	-13.428\\
1270	-9.766\\
1271	-14.648\\
1272	-10.986\\
1273	-9.766\\
1274	-6.104\\
1275	-3.662\\
1276	-3.662\\
1277	-1.221\\
1278	-2.441\\
1279	-8.545\\
1280	-8.545\\
1281	-7.324\\
1282	-9.766\\
1283	-15.869\\
1284	-10.986\\
1285	-9.766\\
1286	-9.766\\
1287	-13.428\\
1288	-14.648\\
1289	-12.207\\
1290	-12.207\\
1291	-6.104\\
1292	-2.441\\
1293	-4.883\\
1294	-4.883\\
1295	-6.104\\
1296	-3.662\\
1297	-4.883\\
1298	-9.766\\
1299	-7.324\\
1300	-7.324\\
1301	-10.986\\
1302	-7.324\\
1303	-4.883\\
1304	-9.766\\
1305	-12.207\\
1306	-9.766\\
1307	-10.986\\
1308	-7.324\\
1309	-7.324\\
1310	-10.986\\
1311	-14.648\\
1312	-13.428\\
1313	-8.545\\
1314	-14.648\\
1315	-13.428\\
1316	-13.428\\
1317	-13.428\\
1318	-13.428\\
1319	-10.986\\
1320	-10.986\\
1321	-12.207\\
1322	-18.311\\
1323	-17.09\\
1324	-10.986\\
1325	-10.986\\
1326	-10.986\\
1327	-13.428\\
1328	-13.428\\
1329	-9.766\\
1330	-12.207\\
1331	-14.648\\
1332	-14.648\\
1333	-9.766\\
1334	-8.545\\
1335	-10.986\\
1336	-18.311\\
1337	-14.648\\
1338	-13.428\\
1339	-9.766\\
1340	-10.986\\
1341	-9.766\\
1342	-6.104\\
1343	-4.883\\
1344	-3.662\\
1345	-3.662\\
1346	-7.324\\
1347	-10.986\\
1348	-9.766\\
1349	-6.104\\
1350	-7.324\\
1351	-9.766\\
1352	-6.104\\
1353	-6.104\\
1354	-9.766\\
1355	-6.104\\
1356	-7.324\\
1357	-12.207\\
1358	-12.207\\
1359	-8.545\\
1360	-4.883\\
1361	-6.104\\
1362	-7.324\\
1363	-6.104\\
1364	-7.324\\
1365	-14.648\\
1366	-9.766\\
1367	-17.09\\
1368	-20.752\\
1369	-18.311\\
1370	-13.428\\
1371	-10.986\\
1372	-10.986\\
1373	-12.207\\
1374	-13.428\\
1375	-14.648\\
1376	-15.869\\
1377	-20.752\\
1378	-20.752\\
1379	-24.414\\
1380	-15.869\\
1381	-20.752\\
1382	-20.752\\
1383	-14.648\\
1384	-17.09\\
1385	-18.311\\
1386	-9.766\\
1387	-7.324\\
1388	-8.545\\
1389	-9.766\\
1390	-7.324\\
1391	-6.104\\
1392	-7.324\\
1393	-6.104\\
1394	-3.662\\
1395	-4.883\\
1396	-7.324\\
1397	-2.441\\
1398	-10.986\\
1399	-8.545\\
1400	-7.324\\
1401	-15.869\\
1402	-19.531\\
1403	-19.531\\
1404	-19.531\\
1405	-14.648\\
1406	-17.09\\
1407	-12.207\\
1408	-7.324\\
1409	-10.986\\
1410	-7.324\\
1411	-3.662\\
1412	-6.104\\
1413	-4.883\\
1414	-2.441\\
1415	-4.883\\
1416	-10.986\\
1417	-9.766\\
1418	-10.986\\
1419	-10.986\\
1420	-12.207\\
1421	-14.648\\
1422	-10.986\\
1423	-10.986\\
1424	-8.545\\
1425	-7.324\\
1426	-13.428\\
1427	-9.766\\
1428	-6.104\\
1429	-10.986\\
1430	-13.428\\
1431	-10.986\\
1432	-7.324\\
1433	-7.324\\
1434	-7.324\\
1435	-4.883\\
1436	-6.104\\
1437	-7.324\\
1438	-10.986\\
1439	-8.545\\
1440	-7.324\\
1441	-10.986\\
1442	-8.545\\
1443	-6.104\\
1444	-7.324\\
1445	-13.428\\
1446	-14.648\\
1447	-13.428\\
1448	-12.207\\
1449	-12.207\\
1450	-8.545\\
1451	-9.766\\
1452	-8.545\\
1453	-12.207\\
1454	-7.324\\
1455	-4.883\\
1456	-9.766\\
1457	-9.766\\
1458	-10.986\\
1459	-12.207\\
1460	-10.986\\
1461	-17.09\\
1462	-20.752\\
1463	-23.193\\
1464	-15.869\\
1465	-9.766\\
1466	-6.104\\
1467	-6.104\\
1468	-9.766\\
1469	-6.104\\
1470	-9.766\\
1471	-6.104\\
1472	-7.324\\
1473	-8.545\\
1474	-10.986\\
1475	-13.428\\
1476	-14.648\\
1477	-19.531\\
1478	-14.648\\
1479	-13.428\\
1480	-10.986\\
1481	-10.986\\
1482	-10.986\\
1483	-12.207\\
1484	-9.766\\
1485	-10.986\\
1486	-9.766\\
1487	-7.324\\
1488	-12.207\\
1489	-13.428\\
1490	-8.545\\
1491	-9.766\\
1492	-18.311\\
1493	-12.207\\
1494	-12.207\\
1495	-17.09\\
1496	-14.648\\
1497	-10.986\\
1498	-7.324\\
1499	-7.324\\
1500	-12.207\\
};
\end{axis}

\begin{axis}[%
width=5.090835cm,
height=2.240107cm,
at={(0cm,11.81992cm)},
scale only axis,
xmin=1000,
xmax=1500,
xlabel={Sample index},
ymin=-40.283,
ymax=0,
ylabel={C3 load, kN},
legend style={legend cell align=left,align=left,draw=white!15!black}
]
\addplot [color=mycolor1,solid,forget plot]
  table[row sep=crcr]{%
1000	-17.09\\
1001	-19.531\\
1002	-14.648\\
1003	-14.648\\
1004	-19.531\\
1005	-18.311\\
1006	-23.193\\
1007	-18.311\\
1008	-9.766\\
1009	-15.869\\
1010	-12.207\\
1011	-14.648\\
1012	-10.986\\
1013	-3.662\\
1014	-2.441\\
1015	-4.883\\
1016	-6.104\\
1017	-14.648\\
1018	-17.09\\
1019	-17.09\\
1020	-13.428\\
1021	-9.766\\
1022	-18.311\\
1023	-14.648\\
1024	-10.986\\
1025	-13.428\\
1026	-12.207\\
1027	-10.986\\
1028	-20.752\\
1029	-19.531\\
1030	-12.207\\
1031	-20.752\\
1032	-20.752\\
1033	-15.869\\
1034	-13.428\\
1035	-9.766\\
1036	-14.648\\
1037	-14.648\\
1038	-14.648\\
1039	-14.648\\
1040	-14.648\\
1041	-15.869\\
1042	-21.973\\
1043	-20.752\\
1044	-13.428\\
1045	-13.428\\
1046	-8.545\\
1047	-12.207\\
1048	-12.207\\
1049	-13.428\\
1050	-10.986\\
1051	-13.428\\
1052	-14.648\\
1053	-13.428\\
1054	-20.752\\
1055	-13.428\\
1056	-9.766\\
1057	-7.324\\
1058	-8.545\\
1059	-10.986\\
1060	-6.104\\
1061	-9.766\\
1062	-10.986\\
1063	-6.104\\
1064	-6.104\\
1065	-9.766\\
1066	-9.766\\
1067	-15.869\\
1068	-14.648\\
1069	-17.09\\
1070	-15.869\\
1071	-18.311\\
1072	-13.428\\
1073	-14.648\\
1074	-12.207\\
1075	-10.986\\
1076	-12.207\\
1077	-23.193\\
1078	-28.076\\
1079	-30.518\\
1080	-29.297\\
1081	-18.311\\
1082	-28.076\\
1083	-32.959\\
1084	-31.738\\
1085	-20.752\\
1086	-34.18\\
1087	-40.283\\
1088	-29.297\\
1089	-24.414\\
1090	-19.531\\
1091	-15.869\\
1092	-12.207\\
1093	-14.648\\
1094	-9.766\\
1095	-7.324\\
1096	-7.324\\
1097	-10.986\\
1098	-13.428\\
1099	-12.207\\
1100	-12.207\\
1101	-15.869\\
1102	-18.311\\
1103	-19.531\\
1104	-15.869\\
1105	-21.973\\
1106	-15.869\\
1107	-15.869\\
1108	-17.09\\
1109	-12.207\\
1110	-12.207\\
1111	-15.869\\
1112	-14.648\\
1113	-8.545\\
1114	-12.207\\
1115	-8.545\\
1116	-9.766\\
1117	-9.766\\
1118	-7.324\\
1119	-9.766\\
1120	-12.207\\
1121	-17.09\\
1122	-17.09\\
1123	-17.09\\
1124	-10.986\\
1125	-12.207\\
1126	-23.193\\
1127	-15.869\\
1128	-20.752\\
1129	-21.973\\
1130	-12.207\\
1131	-9.766\\
1132	-12.207\\
1133	-13.428\\
1134	-21.973\\
1135	-25.635\\
1136	-26.855\\
1137	-19.531\\
1138	-20.752\\
1139	-18.311\\
1140	-17.09\\
1141	-18.311\\
1142	-13.428\\
1143	-12.207\\
1144	-9.766\\
1145	-9.766\\
1146	-10.986\\
1147	-15.869\\
1148	-23.193\\
1149	-17.09\\
1150	-13.428\\
1151	-10.986\\
1152	-10.986\\
1153	-7.324\\
1154	-6.104\\
1155	-6.104\\
1156	-12.207\\
1157	-7.324\\
1158	-8.545\\
1159	-8.545\\
1160	-8.545\\
1161	-7.324\\
1162	-4.883\\
1163	-3.662\\
1164	-8.545\\
1165	-15.869\\
1166	-18.311\\
1167	-20.752\\
1168	-15.869\\
1169	-10.986\\
1170	-8.545\\
1171	-4.883\\
1172	-9.766\\
1173	-8.545\\
1174	-15.869\\
1175	-17.09\\
1176	-29.297\\
1177	-32.959\\
1178	-25.635\\
1179	-25.635\\
1180	-18.311\\
1181	-23.193\\
1182	-21.973\\
1183	-24.414\\
1184	-15.869\\
1185	-17.09\\
1186	-17.09\\
1187	-15.869\\
1188	-15.869\\
1189	-13.428\\
1190	-23.193\\
1191	-25.635\\
1192	-19.531\\
1193	-13.428\\
1194	-14.648\\
1195	-23.193\\
1196	-24.414\\
1197	-28.076\\
1198	-30.518\\
1199	-29.297\\
1200	-23.193\\
1201	-21.973\\
1202	-25.635\\
1203	-28.076\\
1204	-18.311\\
1205	-12.207\\
1206	-19.531\\
1207	-14.648\\
1208	-9.766\\
1209	-13.428\\
1210	-14.648\\
1211	-10.986\\
1212	-10.986\\
1213	-10.986\\
1214	-8.545\\
1215	-10.986\\
1216	-18.311\\
1217	-17.09\\
1218	-12.207\\
1219	-12.207\\
1220	-18.311\\
1221	-26.855\\
1222	-17.09\\
1223	-14.648\\
1224	-10.986\\
1225	-13.428\\
1226	-10.986\\
1227	-8.545\\
1228	-8.545\\
1229	-10.986\\
1230	-13.428\\
1231	-14.648\\
1232	-12.207\\
1233	-15.869\\
1234	-20.752\\
1235	-17.09\\
1236	-12.207\\
1237	-15.869\\
1238	-20.752\\
1239	-13.428\\
1240	-7.324\\
1241	-13.428\\
1242	-10.986\\
1243	-12.207\\
1244	-9.766\\
1245	-8.545\\
1246	-13.428\\
1247	-13.428\\
1248	-9.766\\
1249	-12.207\\
1250	-9.766\\
1251	-7.324\\
1252	-12.207\\
1253	-8.545\\
1254	-9.766\\
1255	-7.324\\
1256	-4.883\\
1257	-12.207\\
1258	-15.869\\
1259	-17.09\\
1260	-23.193\\
1261	-15.869\\
1262	-9.766\\
1263	-8.545\\
1264	-8.545\\
1265	-10.986\\
1266	-8.545\\
1267	-4.883\\
1268	-10.986\\
1269	-15.869\\
1270	-13.428\\
1271	-20.752\\
1272	-15.869\\
1273	-14.648\\
1274	-8.545\\
1275	-10.986\\
1276	-4.883\\
1277	-3.662\\
1278	-4.883\\
1279	-9.766\\
1280	-10.986\\
1281	-12.207\\
1282	-13.428\\
1283	-20.752\\
1284	-17.09\\
1285	-14.648\\
1286	-17.09\\
1287	-19.531\\
1288	-20.752\\
1289	-17.09\\
1290	-13.428\\
1291	-7.324\\
1292	-6.104\\
1293	-4.883\\
1294	-7.324\\
1295	-7.324\\
1296	-4.883\\
1297	-8.545\\
1298	-12.207\\
1299	-10.986\\
1300	-12.207\\
1301	-12.207\\
1302	-8.545\\
1303	-4.883\\
1304	-9.766\\
1305	-15.869\\
1306	-13.428\\
1307	-15.869\\
1308	-13.428\\
1309	-9.766\\
1310	-14.648\\
1311	-18.311\\
1312	-18.311\\
1313	-13.428\\
1314	-20.752\\
1315	-15.869\\
1316	-17.09\\
1317	-18.311\\
1318	-19.531\\
1319	-13.428\\
1320	-13.428\\
1321	-18.311\\
1322	-26.855\\
1323	-21.973\\
1324	-13.428\\
1325	-13.428\\
1326	-13.428\\
1327	-15.869\\
1328	-17.09\\
1329	-12.207\\
1330	-14.648\\
1331	-18.311\\
1332	-20.752\\
1333	-13.428\\
1334	-12.207\\
1335	-15.869\\
1336	-23.193\\
1337	-20.752\\
1338	-21.973\\
1339	-14.648\\
1340	-14.648\\
1341	-12.207\\
1342	-9.766\\
1343	-4.883\\
1344	-3.662\\
1345	-3.662\\
1346	-7.324\\
1347	-13.428\\
1348	-14.648\\
1349	-9.766\\
1350	-12.207\\
1351	-13.428\\
1352	-9.766\\
1353	-8.545\\
1354	-8.545\\
1355	-6.104\\
1356	-9.766\\
1357	-14.648\\
1358	-15.869\\
1359	-10.986\\
1360	-8.545\\
1361	-8.545\\
1362	-8.545\\
1363	-7.324\\
1364	-10.986\\
1365	-20.752\\
1366	-18.311\\
1367	-18.311\\
1368	-24.414\\
1369	-23.193\\
1370	-18.311\\
1371	-14.648\\
1372	-14.648\\
1373	-14.648\\
1374	-17.09\\
1375	-19.531\\
1376	-19.531\\
1377	-25.635\\
1378	-26.855\\
1379	-31.738\\
1380	-24.414\\
1381	-25.635\\
1382	-26.855\\
1383	-21.973\\
1384	-24.414\\
1385	-20.752\\
1386	-13.428\\
1387	-9.766\\
1388	-9.766\\
1389	-12.207\\
1390	-9.766\\
1391	-7.324\\
1392	-10.986\\
1393	-8.545\\
1394	-6.104\\
1395	-7.324\\
1396	-9.766\\
1397	-6.104\\
1398	-12.207\\
1399	-14.648\\
1400	-10.986\\
1401	-17.09\\
1402	-24.414\\
1403	-24.414\\
1404	-28.076\\
1405	-23.193\\
1406	-21.973\\
1407	-17.09\\
1408	-9.766\\
1409	-10.986\\
1410	-10.986\\
1411	-7.324\\
1412	-10.986\\
1413	-9.766\\
1414	-2.441\\
1415	-6.104\\
1416	-9.766\\
1417	-12.207\\
1418	-14.648\\
1419	-17.09\\
1420	-14.648\\
1421	-17.09\\
1422	-14.648\\
1423	-13.428\\
1424	-12.207\\
1425	-10.986\\
1426	-14.648\\
1427	-13.428\\
1428	-10.986\\
1429	-13.428\\
1430	-18.311\\
1431	-13.428\\
1432	-10.986\\
1433	-10.986\\
1434	-10.986\\
1435	-8.545\\
1436	-7.324\\
1437	-10.986\\
1438	-13.428\\
1439	-13.428\\
1440	-10.986\\
1441	-13.428\\
1442	-13.428\\
1443	-8.545\\
1444	-7.324\\
1445	-15.869\\
1446	-19.531\\
1447	-18.311\\
1448	-18.311\\
1449	-15.869\\
1450	-13.428\\
1451	-10.986\\
1452	-9.766\\
1453	-13.428\\
1454	-13.428\\
1455	-8.545\\
1456	-12.207\\
1457	-14.648\\
1458	-14.648\\
1459	-17.09\\
1460	-17.09\\
1461	-20.752\\
1462	-28.076\\
1463	-30.518\\
1464	-20.752\\
1465	-13.428\\
1466	-8.545\\
1467	-6.104\\
1468	-8.545\\
1469	-7.324\\
1470	-10.986\\
1471	-12.207\\
1472	-12.207\\
1473	-13.428\\
1474	-15.869\\
1475	-19.531\\
1476	-19.531\\
1477	-28.076\\
1478	-23.193\\
1479	-18.311\\
1480	-14.648\\
1481	-14.648\\
1482	-14.648\\
1483	-17.09\\
1484	-14.648\\
1485	-12.207\\
1486	-13.428\\
1487	-8.545\\
1488	-7.324\\
1489	-17.09\\
1490	-14.648\\
1491	-12.207\\
1492	-23.193\\
1493	-18.311\\
1494	-15.869\\
1495	-21.973\\
1496	-23.193\\
1497	-14.648\\
1498	-8.545\\
1499	-9.766\\
1500	-15.869\\
};
\end{axis}

\begin{axis}[%
width=5.090835cm,
height=2.240107cm,
at={(6.698467cm,11.81992cm)},
scale only axis,
xmin=1000,
xmax=1500,
xlabel={Sample index},
ymin=-26.855,
ymax=0,
ylabel={C4 load, kN},
legend style={legend cell align=left,align=left,draw=white!15!black}
]
\addplot [color=mycolor1,solid,forget plot]
  table[row sep=crcr]{%
1000	-10.986\\
1001	-15.869\\
1002	-10.986\\
1003	-10.986\\
1004	-14.648\\
1005	-14.648\\
1006	-17.09\\
1007	-10.986\\
1008	-7.324\\
1009	-14.648\\
1010	-8.545\\
1011	-9.766\\
1012	-6.104\\
1013	-3.662\\
1014	-3.662\\
1015	-3.662\\
1016	-2.441\\
1017	-13.428\\
1018	-13.428\\
1019	-13.428\\
1020	-9.766\\
1021	-7.324\\
1022	-10.986\\
1023	-12.207\\
1024	-6.104\\
1025	-9.766\\
1026	-13.428\\
1027	-7.324\\
1028	-15.869\\
1029	-13.428\\
1030	-10.986\\
1031	-17.09\\
1032	-13.428\\
1033	-10.986\\
1034	-8.545\\
1035	-8.545\\
1036	-10.986\\
1037	-12.207\\
1038	-10.986\\
1039	-9.766\\
1040	-10.986\\
1041	-10.986\\
1042	-15.869\\
1043	-14.648\\
1044	-10.986\\
1045	-10.986\\
1046	-6.104\\
1047	-4.883\\
1048	-8.545\\
1049	-7.324\\
1050	-7.324\\
1051	-10.986\\
1052	-9.766\\
1053	-9.766\\
1054	-15.869\\
1055	-8.545\\
1056	-6.104\\
1057	-7.324\\
1058	-8.545\\
1059	-9.766\\
1060	-4.883\\
1061	-7.324\\
1062	-7.324\\
1063	-7.324\\
1064	-6.104\\
1065	-7.324\\
1066	-8.545\\
1067	-9.766\\
1068	-10.986\\
1069	-10.986\\
1070	-12.207\\
1071	-13.428\\
1072	-10.986\\
1073	-12.207\\
1074	-9.766\\
1075	-9.766\\
1076	-9.766\\
1077	-17.09\\
1078	-20.752\\
1079	-19.531\\
1080	-19.531\\
1081	-15.869\\
1082	-20.752\\
1083	-23.193\\
1084	-23.193\\
1085	-14.648\\
1086	-23.193\\
1087	-26.855\\
1088	-21.973\\
1089	-15.869\\
1090	-14.648\\
1091	-10.986\\
1092	-8.545\\
1093	-10.986\\
1094	-8.545\\
1095	-6.104\\
1096	-6.104\\
1097	-8.545\\
1098	-9.766\\
1099	-10.986\\
1100	-9.766\\
1101	-12.207\\
1102	-10.986\\
1103	-15.869\\
1104	-13.428\\
1105	-18.311\\
1106	-12.207\\
1107	-10.986\\
1108	-12.207\\
1109	-9.766\\
1110	-8.545\\
1111	-10.986\\
1112	-10.986\\
1113	-8.545\\
1114	-8.545\\
1115	-7.324\\
1116	-7.324\\
1117	-7.324\\
1118	-4.883\\
1119	-6.104\\
1120	-8.545\\
1121	-13.428\\
1122	-12.207\\
1123	-13.428\\
1124	-8.545\\
1125	-15.869\\
1126	-17.09\\
1127	-13.428\\
1128	-14.648\\
1129	-14.648\\
1130	-10.986\\
1131	-7.324\\
1132	-8.545\\
1133	-8.545\\
1134	-17.09\\
1135	-20.752\\
1136	-19.531\\
1137	-14.648\\
1138	-15.869\\
1139	-13.428\\
1140	-12.207\\
1141	-12.207\\
1142	-9.766\\
1143	-8.545\\
1144	-8.545\\
1145	-9.766\\
1146	-8.545\\
1147	-13.428\\
1148	-15.869\\
1149	-10.986\\
1150	-8.545\\
1151	-9.766\\
1152	-8.545\\
1153	-6.104\\
1154	-3.662\\
1155	-4.883\\
1156	-8.545\\
1157	-9.766\\
1158	-4.883\\
1159	-7.324\\
1160	-7.324\\
1161	-3.662\\
1162	-3.662\\
1163	-2.441\\
1164	-4.883\\
1165	-10.986\\
1166	-12.207\\
1167	-14.648\\
1168	-13.428\\
1169	-8.545\\
1170	-7.324\\
1171	-4.883\\
1172	-7.324\\
1173	-6.104\\
1174	-10.986\\
1175	-12.207\\
1176	-21.973\\
1177	-23.193\\
1178	-21.973\\
1179	-18.311\\
1180	-13.428\\
1181	-17.09\\
1182	-15.869\\
1183	-19.531\\
1184	-12.207\\
1185	-13.428\\
1186	-12.207\\
1187	-13.428\\
1188	-13.428\\
1189	-10.986\\
1190	-18.311\\
1191	-17.09\\
1192	-13.428\\
1193	-7.324\\
1194	-12.207\\
1195	-15.869\\
1196	-17.09\\
1197	-19.531\\
1198	-21.973\\
1199	-20.752\\
1200	-15.869\\
1201	-14.648\\
1202	-18.311\\
1203	-19.531\\
1204	-13.428\\
1205	-9.766\\
1206	-13.428\\
1207	-9.766\\
1208	-7.324\\
1209	-9.766\\
1210	-10.986\\
1211	-7.324\\
1212	-7.324\\
1213	-9.766\\
1214	-6.104\\
1215	-10.986\\
1216	-13.428\\
1217	-13.428\\
1218	-8.545\\
1219	-9.766\\
1220	-12.207\\
1221	-18.311\\
1222	-15.869\\
1223	-9.766\\
1224	-8.545\\
1225	-9.766\\
1226	-7.324\\
1227	-8.545\\
1228	-7.324\\
1229	-8.545\\
1230	-10.986\\
1231	-10.986\\
1232	-6.104\\
1233	-10.986\\
1234	-13.428\\
1235	-13.428\\
1236	-9.766\\
1237	-12.207\\
1238	-13.428\\
1239	-10.986\\
1240	-4.883\\
1241	-10.986\\
1242	-8.545\\
1243	-8.545\\
1244	-6.104\\
1245	-6.104\\
1246	-10.986\\
1247	-10.986\\
1248	-6.104\\
1249	-8.545\\
1250	-8.545\\
1251	-4.883\\
1252	-8.545\\
1253	-4.883\\
1254	-7.324\\
1255	-4.883\\
1256	-4.883\\
1257	-10.986\\
1258	-12.207\\
1259	-12.207\\
1260	-15.869\\
1261	-10.986\\
1262	-6.104\\
1263	-7.324\\
1264	-4.883\\
1265	-7.324\\
1266	-6.104\\
1267	-3.662\\
1268	-8.545\\
1269	-9.766\\
1270	-9.766\\
1271	-17.09\\
1272	-10.986\\
1273	-13.428\\
1274	-7.324\\
1275	-8.545\\
1276	-3.662\\
1277	-1.221\\
1278	-6.104\\
1279	-9.766\\
1280	-8.545\\
1281	-9.766\\
1282	-9.766\\
1283	-17.09\\
1284	-10.986\\
1285	-10.986\\
1286	-10.986\\
1287	-13.428\\
1288	-15.869\\
1289	-15.869\\
1290	-10.986\\
1291	-6.104\\
1292	-3.662\\
1293	-4.883\\
1294	-6.104\\
1295	-4.883\\
1296	-6.104\\
1297	-6.104\\
1298	-9.766\\
1299	-9.766\\
1300	-8.545\\
1301	-9.766\\
1302	-6.104\\
1303	-3.662\\
1304	-7.324\\
1305	-10.986\\
1306	-8.545\\
1307	-12.207\\
1308	-8.545\\
1309	-7.324\\
1310	-8.545\\
1311	-12.207\\
1312	-14.648\\
1313	-10.986\\
1314	-17.09\\
1315	-9.766\\
1316	-12.207\\
1317	-13.428\\
1318	-13.428\\
1319	-9.766\\
1320	-9.766\\
1321	-12.207\\
1322	-18.311\\
1323	-15.869\\
1324	-8.545\\
1325	-12.207\\
1326	-9.766\\
1327	-12.207\\
1328	-13.428\\
1329	-9.766\\
1330	-9.766\\
1331	-14.648\\
1332	-13.428\\
1333	-12.207\\
1334	-8.545\\
1335	-9.766\\
1336	-15.869\\
1337	-14.648\\
1338	-15.869\\
1339	-12.207\\
1340	-10.986\\
1341	-9.766\\
1342	-8.545\\
1343	-4.883\\
1344	-3.662\\
1345	-3.662\\
1346	-7.324\\
1347	-10.986\\
1348	-9.766\\
1349	-8.545\\
1350	-10.986\\
1351	-9.766\\
1352	-8.545\\
1353	-3.662\\
1354	-6.104\\
1355	-6.104\\
1356	-6.104\\
1357	-8.545\\
1358	-10.986\\
1359	-7.324\\
1360	-6.104\\
1361	-7.324\\
1362	-7.324\\
1363	-6.104\\
1364	-8.545\\
1365	-15.869\\
1366	-14.648\\
1367	-17.09\\
1368	-17.09\\
1369	-17.09\\
1370	-13.428\\
1371	-10.986\\
1372	-9.766\\
1373	-10.986\\
1374	-12.207\\
1375	-14.648\\
1376	-14.648\\
1377	-17.09\\
1378	-19.531\\
1379	-21.973\\
1380	-18.311\\
1381	-21.973\\
1382	-19.531\\
1383	-14.648\\
1384	-17.09\\
1385	-14.648\\
1386	-10.986\\
1387	-8.545\\
1388	-8.545\\
1389	-10.986\\
1390	-7.324\\
1391	-7.324\\
1392	-7.324\\
1393	-7.324\\
1394	-3.662\\
1395	-7.324\\
1396	-7.324\\
1397	-6.104\\
1398	-8.545\\
1399	-12.207\\
1400	-8.545\\
1401	-12.207\\
1402	-18.311\\
1403	-17.09\\
1404	-19.531\\
1405	-17.09\\
1406	-14.648\\
1407	-14.648\\
1408	-7.324\\
1409	-9.766\\
1410	-8.545\\
1411	-6.104\\
1412	-7.324\\
1413	-8.545\\
1414	-2.441\\
1415	-4.883\\
1416	-8.545\\
1417	-9.766\\
1418	-12.207\\
1419	-12.207\\
1420	-10.986\\
1421	-12.207\\
1422	-14.648\\
1423	-10.986\\
1424	-9.766\\
1425	-7.324\\
1426	-10.986\\
1427	-10.986\\
1428	-7.324\\
1429	-10.986\\
1430	-13.428\\
1431	-10.986\\
1432	-8.545\\
1433	-8.545\\
1434	-8.545\\
1435	-6.104\\
1436	-7.324\\
1437	-6.104\\
1438	-9.766\\
1439	-8.545\\
1440	-10.986\\
1441	-10.986\\
1442	-8.545\\
1443	-6.104\\
1444	-6.104\\
1445	-10.986\\
1446	-14.648\\
1447	-14.648\\
1448	-12.207\\
1449	-10.986\\
1450	-9.766\\
1451	-8.545\\
1452	-8.545\\
1453	-9.766\\
1454	-12.207\\
1455	-7.324\\
1456	-7.324\\
1457	-10.986\\
1458	-10.986\\
1459	-13.428\\
1460	-13.428\\
1461	-14.648\\
1462	-20.752\\
1463	-21.973\\
1464	-14.648\\
1465	-9.766\\
1466	-7.324\\
1467	-6.104\\
1468	-7.324\\
1469	-4.883\\
1470	-7.324\\
1471	-7.324\\
1472	-7.324\\
1473	-10.986\\
1474	-12.207\\
1475	-13.428\\
1476	-13.428\\
1477	-18.311\\
1478	-17.09\\
1479	-12.207\\
1480	-12.207\\
1481	-9.766\\
1482	-9.766\\
1483	-13.428\\
1484	-9.766\\
1485	-8.545\\
1486	-10.986\\
1487	-6.104\\
1488	-10.986\\
1489	-14.648\\
1490	-8.545\\
1491	-8.545\\
1492	-15.869\\
1493	-13.428\\
1494	-10.986\\
1495	-17.09\\
1496	-17.09\\
1497	-10.986\\
1498	-8.545\\
1499	-9.766\\
1500	-12.207\\
};
\end{axis}

\begin{axis}[%
width=5.090835cm,
height=2.240107cm,
at={(6.698467cm,3.939973cm)},
scale only axis,
xmin=1000,
xmax=1500,
xlabel={Sample index},
ymin=-279.541,
ymax=0,
ylabel={C8 load, kN},
legend style={legend cell align=left,align=left,draw=white!15!black}
]
\addplot [color=mycolor1,solid,forget plot]
  table[row sep=crcr]{%
1000	-96.436\\
1001	-122.07\\
1002	-100.098\\
1003	-93.994\\
1004	-130.615\\
1005	-125.732\\
1006	-153.809\\
1007	-115.967\\
1008	-61.035\\
1009	-74.463\\
1010	-73.242\\
1011	-86.67\\
1012	-73.242\\
1013	-34.18\\
1014	-20.752\\
1015	-23.193\\
1016	-58.594\\
1017	-100.098\\
1018	-109.863\\
1019	-114.746\\
1020	-83.008\\
1021	-52.49\\
1022	-104.98\\
1023	-81.787\\
1024	-59.814\\
1025	-97.656\\
1026	-83.008\\
1027	-72.021\\
1028	-118.408\\
1029	-107.422\\
1030	-83.008\\
1031	-119.629\\
1032	-123.291\\
1033	-95.215\\
1034	-80.566\\
1035	-62.256\\
1036	-75.684\\
1037	-95.215\\
1038	-87.891\\
1039	-91.553\\
1040	-96.436\\
1041	-102.539\\
1042	-141.602\\
1043	-128.174\\
1044	-85.449\\
1045	-79.346\\
1046	-47.607\\
1047	-48.828\\
1048	-68.359\\
1049	-52.49\\
1050	-57.373\\
1051	-75.684\\
1052	-89.111\\
1053	-81.787\\
1054	-128.174\\
1055	-98.877\\
1056	-57.373\\
1057	-45.166\\
1058	-62.256\\
1059	-72.021\\
1060	-40.283\\
1061	-42.725\\
1062	-56.152\\
1063	-46.387\\
1064	-40.283\\
1065	-52.49\\
1066	-62.256\\
1067	-92.773\\
1068	-90.332\\
1069	-107.422\\
1070	-98.877\\
1071	-115.967\\
1072	-91.553\\
1073	-91.553\\
1074	-79.346\\
1075	-75.684\\
1076	-76.904\\
1077	-133.057\\
1078	-189.209\\
1079	-194.092\\
1080	-189.209\\
1081	-119.629\\
1082	-170.898\\
1083	-213.623\\
1084	-219.727\\
1085	-169.678\\
1086	-219.727\\
1087	-279.541\\
1088	-197.754\\
1089	-161.133\\
1090	-109.863\\
1091	-85.449\\
1092	-73.242\\
1093	-91.553\\
1094	-62.256\\
1095	-46.387\\
1096	-42.725\\
1097	-54.932\\
1098	-79.346\\
1099	-74.463\\
1100	-75.684\\
1101	-100.098\\
1102	-103.76\\
1103	-141.602\\
1104	-114.746\\
1105	-142.822\\
1106	-104.98\\
1107	-90.332\\
1108	-103.76\\
1109	-76.904\\
1110	-69.58\\
1111	-84.229\\
1112	-86.67\\
1113	-59.814\\
1114	-64.697\\
1115	-48.828\\
1116	-53.711\\
1117	-57.373\\
1118	-41.504\\
1119	-52.49\\
1120	-65.918\\
1121	-101.318\\
1122	-104.98\\
1123	-114.746\\
1124	-68.359\\
1125	-93.994\\
1126	-150.146\\
1127	-114.746\\
1128	-125.732\\
1129	-128.174\\
1130	-80.566\\
1131	-53.711\\
1132	-75.684\\
1133	-85.449\\
1134	-137.939\\
1135	-172.119\\
1136	-170.898\\
1137	-123.291\\
1138	-130.615\\
1139	-115.967\\
1140	-113.525\\
1141	-111.084\\
1142	-76.904\\
1143	-68.359\\
1144	-61.035\\
1145	-57.373\\
1146	-62.256\\
1147	-91.553\\
1148	-140.381\\
1149	-111.084\\
1150	-79.346\\
1151	-64.697\\
1152	-62.256\\
1153	-40.283\\
1154	-29.297\\
1155	-36.621\\
1156	-63.477\\
1157	-51.27\\
1158	-52.49\\
1159	-58.594\\
1160	-54.932\\
1161	-37.842\\
1162	-28.076\\
1163	-23.193\\
1164	-39.063\\
1165	-83.008\\
1166	-117.188\\
1167	-130.615\\
1168	-86.67\\
1169	-58.594\\
1170	-45.166\\
1171	-32.959\\
1172	-67.139\\
1173	-65.918\\
1174	-91.553\\
1175	-117.188\\
1176	-186.768\\
1177	-216.064\\
1178	-177.002\\
1179	-168.457\\
1180	-109.863\\
1181	-122.07\\
1182	-131.836\\
1183	-146.484\\
1184	-108.643\\
1185	-100.098\\
1186	-98.877\\
1187	-89.111\\
1188	-106.201\\
1189	-78.125\\
1190	-130.615\\
1191	-159.912\\
1192	-113.525\\
1193	-70.801\\
1194	-73.242\\
1195	-123.291\\
1196	-153.809\\
1197	-189.209\\
1198	-202.637\\
1199	-202.637\\
1200	-153.809\\
1201	-137.939\\
1202	-158.691\\
1203	-187.988\\
1204	-109.863\\
1205	-72.021\\
1206	-101.318\\
1207	-86.67\\
1208	-53.711\\
1209	-74.463\\
1210	-84.229\\
1211	-67.139\\
1212	-51.27\\
1213	-70.801\\
1214	-58.594\\
1215	-79.346\\
1216	-107.422\\
1217	-100.098\\
1218	-69.58\\
1219	-70.801\\
1220	-107.422\\
1221	-177.002\\
1222	-128.174\\
1223	-79.346\\
1224	-61.035\\
1225	-70.801\\
1226	-64.697\\
1227	-48.828\\
1228	-53.711\\
1229	-65.918\\
1230	-83.008\\
1231	-91.553\\
1232	-63.477\\
1233	-104.98\\
1234	-124.512\\
1235	-118.408\\
1236	-91.553\\
1237	-100.098\\
1238	-135.498\\
1239	-87.891\\
1240	-42.725\\
1241	-57.373\\
1242	-62.256\\
1243	-81.787\\
1244	-72.021\\
1245	-52.49\\
1246	-79.346\\
1247	-80.566\\
1248	-57.373\\
1249	-70.801\\
1250	-63.477\\
1251	-56.152\\
1252	-67.139\\
1253	-41.504\\
1254	-48.828\\
1255	-36.621\\
1256	-40.283\\
1257	-75.684\\
1258	-84.229\\
1259	-113.525\\
1260	-146.484\\
1261	-98.877\\
1262	-58.594\\
1263	-46.387\\
1264	-43.945\\
1265	-63.477\\
1266	-42.725\\
1267	-47.607\\
1268	-61.035\\
1269	-102.539\\
1270	-93.994\\
1271	-134.277\\
1272	-89.111\\
1273	-81.787\\
1274	-46.387\\
1275	-45.166\\
1276	-28.076\\
1277	-35.4\\
1278	-37.842\\
1279	-67.139\\
1280	-70.801\\
1281	-79.346\\
1282	-85.449\\
1283	-125.732\\
1284	-112.305\\
1285	-84.229\\
1286	-102.539\\
1287	-109.863\\
1288	-144.043\\
1289	-115.967\\
1290	-75.684\\
1291	-40.283\\
1292	-29.297\\
1293	-29.297\\
1294	-36.621\\
1295	-39.063\\
1296	-37.842\\
1297	-50.049\\
1298	-74.463\\
1299	-68.359\\
1300	-70.801\\
1301	-83.008\\
1302	-51.27\\
1303	-30.518\\
1304	-58.594\\
1305	-90.332\\
1306	-87.891\\
1307	-97.656\\
1308	-83.008\\
1309	-61.035\\
1310	-85.449\\
1311	-118.408\\
1312	-118.408\\
1313	-84.229\\
1314	-136.719\\
1315	-100.098\\
1316	-101.318\\
1317	-117.188\\
1318	-115.967\\
1319	-86.67\\
1320	-76.904\\
1321	-128.174\\
1322	-178.223\\
1323	-139.16\\
1324	-79.346\\
1325	-76.904\\
1326	-79.346\\
1327	-98.877\\
1328	-114.746\\
1329	-85.449\\
1330	-90.332\\
1331	-120.85\\
1332	-131.836\\
1333	-87.891\\
1334	-68.359\\
1335	-91.553\\
1336	-156.25\\
1337	-137.939\\
1338	-140.381\\
1339	-84.229\\
1340	-76.904\\
1341	-72.021\\
1342	-47.607\\
1343	-30.518\\
1344	-26.855\\
1345	-23.193\\
1346	-54.932\\
1347	-70.801\\
1348	-87.891\\
1349	-65.918\\
1350	-73.242\\
1351	-83.008\\
1352	-53.711\\
1353	-56.152\\
1354	-53.711\\
1355	-40.283\\
1356	-51.27\\
1357	-84.229\\
1358	-106.201\\
1359	-63.477\\
1360	-48.828\\
1361	-40.283\\
1362	-50.049\\
1363	-35.4\\
1364	-76.904\\
1365	-130.615\\
1366	-104.98\\
1367	-139.16\\
1368	-162.354\\
1369	-164.795\\
1370	-124.512\\
1371	-90.332\\
1372	-89.111\\
1373	-89.111\\
1374	-106.201\\
1375	-125.732\\
1376	-125.732\\
1377	-168.457\\
1378	-183.105\\
1379	-219.727\\
1380	-147.705\\
1381	-166.016\\
1382	-184.326\\
1383	-140.381\\
1384	-162.354\\
1385	-139.16\\
1386	-80.566\\
1387	-52.49\\
1388	-56.152\\
1389	-81.787\\
1390	-65.918\\
1391	-43.945\\
1392	-62.256\\
1393	-47.607\\
1394	-36.621\\
1395	-46.387\\
1396	-52.49\\
1397	-36.621\\
1398	-70.801\\
1399	-92.773\\
1400	-68.359\\
1401	-114.746\\
1402	-164.795\\
1403	-150.146\\
1404	-196.533\\
1405	-131.836\\
1406	-144.043\\
1407	-96.436\\
1408	-63.477\\
1409	-76.904\\
1410	-59.814\\
1411	-45.166\\
1412	-64.697\\
1413	-36.621\\
1414	-28.076\\
1415	-34.18\\
1416	-70.801\\
1417	-86.67\\
1418	-107.422\\
1419	-103.76\\
1420	-100.098\\
1421	-114.746\\
1422	-93.994\\
1423	-76.904\\
1424	-76.904\\
1425	-62.256\\
1426	-97.656\\
1427	-68.359\\
1428	-56.152\\
1429	-81.787\\
1430	-113.525\\
1431	-86.67\\
1432	-65.918\\
1433	-56.152\\
1434	-65.918\\
1435	-45.166\\
1436	-45.166\\
1437	-59.814\\
1438	-75.684\\
1439	-84.229\\
1440	-65.918\\
1441	-86.67\\
1442	-79.346\\
1443	-47.607\\
1444	-50.049\\
1445	-95.215\\
1446	-131.836\\
1447	-131.836\\
1448	-107.422\\
1449	-103.76\\
1450	-79.346\\
1451	-76.904\\
1452	-61.035\\
1453	-89.111\\
1454	-75.684\\
1455	-50.049\\
1456	-74.463\\
1457	-85.449\\
1458	-101.318\\
1459	-109.863\\
1460	-109.863\\
1461	-148.926\\
1462	-181.885\\
1463	-205.078\\
1464	-129.395\\
1465	-73.242\\
1466	-47.607\\
1467	-34.18\\
1468	-54.932\\
1469	-46.387\\
1470	-80.566\\
1471	-72.021\\
1472	-64.697\\
1473	-85.449\\
1474	-98.877\\
1475	-117.188\\
1476	-123.291\\
1477	-175.781\\
1478	-150.146\\
1479	-117.188\\
1480	-80.566\\
1481	-80.566\\
1482	-83.008\\
1483	-106.201\\
1484	-75.684\\
1485	-79.346\\
1486	-74.463\\
1487	-42.725\\
1488	-74.463\\
1489	-107.422\\
1490	-73.242\\
1491	-80.566\\
1492	-147.705\\
1493	-109.863\\
1494	-106.201\\
1495	-147.705\\
1496	-140.381\\
1497	-87.891\\
1498	-56.152\\
1499	-58.594\\
1500	-97.656\\
};
\end{axis}

\begin{axis}[%
width=5.090835cm,
height=2.240107cm,
at={(6.698467cm,0cm)},
scale only axis,
xmin=1000,
xmax=1500,
xlabel={Sample index},
ymin=-158.691,
ymax=-17.09,
ylabel={C10 load, kN},
legend style={legend cell align=left,align=left,draw=white!15!black}
]
\addplot [color=mycolor1,solid,forget plot]
  table[row sep=crcr]{%
1000	-63.477\\
1001	-75.684\\
1002	-62.256\\
1003	-59.814\\
1004	-80.566\\
1005	-76.904\\
1006	-90.332\\
1007	-68.359\\
1008	-40.283\\
1009	-48.828\\
1010	-46.387\\
1011	-54.932\\
1012	-45.166\\
1013	-21.973\\
1014	-17.09\\
1015	-18.311\\
1016	-40.283\\
1017	-62.256\\
1018	-67.139\\
1019	-69.58\\
1020	-52.49\\
1021	-34.18\\
1022	-64.697\\
1023	-52.49\\
1024	-39.063\\
1025	-61.035\\
1026	-53.711\\
1027	-45.166\\
1028	-73.242\\
1029	-67.139\\
1030	-52.49\\
1031	-75.684\\
1032	-74.463\\
1033	-58.594\\
1034	-50.049\\
1035	-41.504\\
1036	-47.607\\
1037	-58.594\\
1038	-54.932\\
1039	-59.814\\
1040	-59.814\\
1041	-65.918\\
1042	-85.449\\
1043	-76.904\\
1044	-54.932\\
1045	-50.049\\
1046	-35.4\\
1047	-34.18\\
1048	-45.166\\
1049	-34.18\\
1050	-36.621\\
1051	-51.27\\
1052	-54.932\\
1053	-51.27\\
1054	-78.125\\
1055	-62.256\\
1056	-36.621\\
1057	-30.518\\
1058	-40.283\\
1059	-46.387\\
1060	-29.297\\
1061	-29.297\\
1062	-36.621\\
1063	-31.738\\
1064	-26.855\\
1065	-35.4\\
1066	-40.283\\
1067	-58.594\\
1068	-56.152\\
1069	-65.918\\
1070	-61.035\\
1071	-70.801\\
1072	-57.373\\
1073	-58.594\\
1074	-51.27\\
1075	-50.049\\
1076	-48.828\\
1077	-80.566\\
1078	-111.084\\
1079	-114.746\\
1080	-112.305\\
1081	-74.463\\
1082	-100.098\\
1083	-125.732\\
1084	-128.174\\
1085	-96.436\\
1086	-124.512\\
1087	-158.691\\
1088	-115.967\\
1089	-97.656\\
1090	-68.359\\
1091	-53.711\\
1092	-47.607\\
1093	-56.152\\
1094	-45.166\\
1095	-30.518\\
1096	-30.518\\
1097	-36.621\\
1098	-50.049\\
1099	-48.828\\
1100	-47.607\\
1101	-62.256\\
1102	-64.697\\
1103	-85.449\\
1104	-72.021\\
1105	-85.449\\
1106	-63.477\\
1107	-54.932\\
1108	-64.697\\
1109	-48.828\\
1110	-43.945\\
1111	-53.711\\
1112	-54.932\\
1113	-40.283\\
1114	-43.945\\
1115	-32.959\\
1116	-36.621\\
1117	-40.283\\
1118	-30.518\\
1119	-34.18\\
1120	-43.945\\
1121	-63.477\\
1122	-65.918\\
1123	-72.021\\
1124	-45.166\\
1125	-56.152\\
1126	-89.111\\
1127	-70.801\\
1128	-81.787\\
1129	-80.566\\
1130	-50.049\\
1131	-36.621\\
1132	-47.607\\
1133	-52.49\\
1134	-81.787\\
1135	-103.76\\
1136	-102.539\\
1137	-76.904\\
1138	-80.566\\
1139	-72.021\\
1140	-69.58\\
1141	-68.359\\
1142	-51.27\\
1143	-45.166\\
1144	-39.063\\
1145	-37.842\\
1146	-41.504\\
1147	-56.152\\
1148	-84.229\\
1149	-70.801\\
1150	-50.049\\
1151	-42.725\\
1152	-40.283\\
1153	-28.076\\
1154	-23.193\\
1155	-24.414\\
1156	-41.504\\
1157	-32.959\\
1158	-34.18\\
1159	-37.842\\
1160	-35.4\\
1161	-26.855\\
1162	-21.973\\
1163	-18.311\\
1164	-25.635\\
1165	-51.27\\
1166	-73.242\\
1167	-78.125\\
1168	-54.932\\
1169	-39.063\\
1170	-31.738\\
1171	-24.414\\
1172	-41.504\\
1173	-42.725\\
1174	-54.932\\
1175	-73.242\\
1176	-108.643\\
1177	-125.732\\
1178	-103.76\\
1179	-101.318\\
1180	-67.139\\
1181	-75.684\\
1182	-79.346\\
1183	-86.67\\
1184	-69.58\\
1185	-62.256\\
1186	-61.035\\
1187	-56.152\\
1188	-67.139\\
1189	-52.49\\
1190	-76.904\\
1191	-97.656\\
1192	-72.021\\
1193	-45.166\\
1194	-47.607\\
1195	-78.125\\
1196	-93.994\\
1197	-112.305\\
1198	-119.629\\
1199	-119.629\\
1200	-91.553\\
1201	-84.229\\
1202	-96.436\\
1203	-111.084\\
1204	-74.463\\
1205	-46.387\\
1206	-62.256\\
1207	-54.932\\
1208	-34.18\\
1209	-47.607\\
1210	-53.711\\
1211	-42.725\\
1212	-34.18\\
1213	-43.945\\
1214	-40.283\\
1215	-52.49\\
1216	-65.918\\
1217	-62.256\\
1218	-45.166\\
1219	-45.166\\
1220	-67.139\\
1221	-104.98\\
1222	-79.346\\
1223	-51.27\\
1224	-40.283\\
1225	-47.607\\
1226	-40.283\\
1227	-31.738\\
1228	-35.4\\
1229	-42.725\\
1230	-51.27\\
1231	-57.373\\
1232	-42.725\\
1233	-63.477\\
1234	-75.684\\
1235	-70.801\\
1236	-57.373\\
1237	-62.256\\
1238	-81.787\\
1239	-52.49\\
1240	-28.076\\
1241	-36.621\\
1242	-42.725\\
1243	-51.27\\
1244	-46.387\\
1245	-36.621\\
1246	-52.49\\
1247	-52.49\\
1248	-37.842\\
1249	-47.607\\
1250	-41.504\\
1251	-37.842\\
1252	-45.166\\
1253	-29.297\\
1254	-31.738\\
1255	-29.297\\
1256	-26.855\\
1257	-47.607\\
1258	-53.711\\
1259	-69.58\\
1260	-89.111\\
1261	-62.256\\
1262	-36.621\\
1263	-31.738\\
1264	-29.297\\
1265	-40.283\\
1266	-31.738\\
1267	-30.518\\
1268	-39.063\\
1269	-61.035\\
1270	-53.711\\
1271	-79.346\\
1272	-54.932\\
1273	-48.828\\
1274	-32.959\\
1275	-29.297\\
1276	-23.193\\
1277	-23.193\\
1278	-25.635\\
1279	-40.283\\
1280	-46.387\\
1281	-48.828\\
1282	-51.27\\
1283	-79.346\\
1284	-73.242\\
1285	-53.711\\
1286	-63.477\\
1287	-69.58\\
1288	-85.449\\
1289	-72.021\\
1290	-47.607\\
1291	-28.076\\
1292	-21.973\\
1293	-21.973\\
1294	-26.855\\
1295	-26.855\\
1296	-25.635\\
1297	-32.959\\
1298	-47.607\\
1299	-43.945\\
1300	-45.166\\
1301	-52.49\\
1302	-35.4\\
1303	-20.752\\
1304	-35.4\\
1305	-56.152\\
1306	-54.932\\
1307	-59.814\\
1308	-53.711\\
1309	-40.283\\
1310	-51.27\\
1311	-72.021\\
1312	-70.801\\
1313	-51.27\\
1314	-81.787\\
1315	-62.256\\
1316	-63.477\\
1317	-73.242\\
1318	-70.801\\
1319	-54.932\\
1320	-47.607\\
1321	-78.125\\
1322	-108.643\\
1323	-85.449\\
1324	-50.049\\
1325	-51.27\\
1326	-51.27\\
1327	-61.035\\
1328	-72.021\\
1329	-54.932\\
1330	-56.152\\
1331	-73.242\\
1332	-79.346\\
1333	-56.152\\
1334	-45.166\\
1335	-57.373\\
1336	-87.891\\
1337	-78.125\\
1338	-80.566\\
1339	-54.932\\
1340	-45.166\\
1341	-47.607\\
1342	-31.738\\
1343	-20.752\\
1344	-20.752\\
1345	-17.09\\
1346	-31.738\\
1347	-48.828\\
1348	-54.932\\
1349	-40.283\\
1350	-45.166\\
1351	-52.49\\
1352	-35.4\\
1353	-35.4\\
1354	-35.4\\
1355	-29.297\\
1356	-31.738\\
1357	-51.27\\
1358	-64.697\\
1359	-41.504\\
1360	-32.959\\
1361	-28.076\\
1362	-32.959\\
1363	-26.855\\
1364	-45.166\\
1365	-80.566\\
1366	-63.477\\
1367	-81.787\\
1368	-98.877\\
1369	-97.656\\
1370	-76.904\\
1371	-58.594\\
1372	-54.932\\
1373	-57.373\\
1374	-65.918\\
1375	-76.904\\
1376	-76.904\\
1377	-100.098\\
1378	-108.643\\
1379	-130.615\\
1380	-91.553\\
1381	-98.877\\
1382	-108.643\\
1383	-85.449\\
1384	-96.436\\
1385	-85.449\\
1386	-51.27\\
1387	-34.18\\
1388	-36.621\\
1389	-51.27\\
1390	-43.945\\
1391	-31.738\\
1392	-41.504\\
1393	-32.959\\
1394	-23.193\\
1395	-31.738\\
1396	-34.18\\
1397	-25.635\\
1398	-42.725\\
1399	-58.594\\
1400	-45.166\\
1401	-68.359\\
1402	-100.098\\
1403	-87.891\\
1404	-109.863\\
1405	-81.787\\
1406	-81.787\\
1407	-62.256\\
1408	-37.842\\
1409	-46.387\\
1410	-42.725\\
1411	-29.297\\
1412	-40.283\\
1413	-21.973\\
1414	-19.531\\
1415	-24.414\\
1416	-45.166\\
1417	-57.373\\
1418	-65.918\\
1419	-65.918\\
1420	-62.256\\
1421	-69.58\\
1422	-57.373\\
1423	-48.828\\
1424	-50.049\\
1425	-43.945\\
1426	-59.814\\
1427	-46.387\\
1428	-36.621\\
1429	-51.27\\
1430	-69.58\\
1431	-54.932\\
1432	-41.504\\
1433	-37.842\\
1434	-41.504\\
1435	-32.959\\
1436	-30.518\\
1437	-41.504\\
1438	-47.607\\
1439	-52.49\\
1440	-42.725\\
1441	-53.711\\
1442	-51.27\\
1443	-32.959\\
1444	-32.959\\
1445	-58.594\\
1446	-80.566\\
1447	-78.125\\
1448	-67.139\\
1449	-62.256\\
1450	-50.049\\
1451	-48.828\\
1452	-41.504\\
1453	-54.932\\
1454	-50.049\\
1455	-32.959\\
1456	-45.166\\
1457	-52.49\\
1458	-61.035\\
1459	-68.359\\
1460	-68.359\\
1461	-87.891\\
1462	-107.422\\
1463	-119.629\\
1464	-80.566\\
1465	-46.387\\
1466	-32.959\\
1467	-24.414\\
1468	-34.18\\
1469	-29.297\\
1470	-51.27\\
1471	-48.828\\
1472	-40.283\\
1473	-52.49\\
1474	-63.477\\
1475	-70.801\\
1476	-75.684\\
1477	-103.76\\
1478	-92.773\\
1479	-72.021\\
1480	-51.27\\
1481	-52.49\\
1482	-54.932\\
1483	-65.918\\
1484	-51.27\\
1485	-50.049\\
1486	-48.828\\
1487	-30.518\\
1488	-45.166\\
1489	-68.359\\
1490	-48.828\\
1491	-52.49\\
1492	-86.67\\
1493	-63.477\\
1494	-63.477\\
1495	-85.449\\
1496	-81.787\\
1497	-53.711\\
1498	-35.4\\
1499	-36.621\\
1500	-61.035\\
};
\end{axis}
\end{tikzpicture}%
	\caption{Experimental data.}
\end{figure}
\section{Structure identification}
\par The following model structure is assumed. The output of the NARX model $\mathbfit{y}(t)$ is the measured load. The input vector is composed as
\begin{equation}
	\mathbfit{x}(t) = \{x_i(t)\}^{d}_{i=1} = \left[\{y(t - k + 1)\}^{n_y}_{k=1} \quad \{u(t - k + n_y + 1)\}^{n_y + n_u}_{k= n_y + 1} \right]^{\top},
\end{equation}
where $n_u$ is the length of the input lag and $n_y$ is the length of the output lag in discrete time, and where $d = n_u + n_y$. In this case, the identification is performed under the following assumptions:
\begin{itemize}[noitemsep,topsep=0.3pt,parsep=0.3pt,partopsep=0.2pt,labelindent=1cm] 
	\item only the input signal affects the output ($n_y = 0$).
	\item the input signal has a lag of length $n_u = 4$.
\end{itemize}
The unknown model is approximated with a sum of polynomial basis functions up to second degree ($\lambda = 2$), rendering the following structure
\begin{equation}
	\mathbfit{y}(t) = \theta^0 + \sum_{i=1}^{d} \theta_i x_i(t) + \sum_{i=1}^{d} \sum_{j=1}^{d} \theta_{i,j} x_i(t) x_j(t) + e(t).
\end{equation}
The number and order of significant terms are identified within the EFOR-CMSS algorithm based on the data from 8 out of 10 datasets. Figure \ref{fig:aamdl} illustrates the relationship between the number of model terms and the selected criterion of significance, AAMDL.
\begin{figure}[!h]
	\centering
	\subfloat[Sample size 2000.]{% This file was created by matlab2tikz.
% Minimal pgfplots version: 1.3
%
\definecolor{mycolor1}{rgb}{0.00000,0.44700,0.74100}%
\definecolor{mycolor2}{rgb}{0.85000,0.32500,0.09800}%
%
\begin{tikzpicture}

\begin{axis}[%
width=6cm,
height=6cm,
at={(0cm,0cm)},
scale only axis,
xmin=1,
xmax=15,
xlabel={Number of terms},
ymin=-2.21881347579609,
ymax=-1.56472398176224,
ylabel={AAMDL},
legend style={legend cell align=left,align=left,draw=white!15!black}
]
\addplot [color=mycolor1,only marks,mark=o,mark options={solid},forget plot]
  table[row sep=crcr]{%
1	-1.56472398176224\\
2	-1.97701270064064\\
3	-2.12029297016719\\
4	-2.18293755649416\\
5	-2.19697120568313\\
6	-2.19619682060115\\
7	-2.20287253171251\\
8	-2.21881347579609\\
9	-2.21849462617766\\
10	-2.21775809851537\\
11	-2.21624604641005\\
12	-2.21472746636226\\
13	-2.21467778403799\\
14	-2.2138532692636\\
15	-2.21310607093081\\
};
\addplot [color=mycolor2,line width=5.0pt,only marks,mark=asterisk,mark options={solid},forget plot]
  table[row sep=crcr]
	\subfloat[Sample size 4000.]{% This file was created by matlab2tikz.
% Minimal pgfplots version: 1.3
%
\definecolor{mycolor1}{rgb}{0.00000,0.44700,0.74100}%
\definecolor{mycolor2}{rgb}{0.85000,0.32500,0.09800}%
%
\begin{tikzpicture}

\begin{axis}[%
width=6cm,
height=6cm,
at={(0cm,0cm)},
scale only axis,
xmin=1,
xmax=15,
xlabel={Number of terms},
ymin=-2.02477048029684,
ymax=-1.2,
ylabel={AAMDL},
legend style={legend cell align=left,align=left,draw=white!15!black}
]
\addplot [color=mycolor1,only marks,mark=o,mark options={solid},forget plot]
  table[row sep=crcr]{%
1	-1.33883800348504\\
2	-1.77306548432192\\
3	-1.91543316960326\\
4	-1.98526964853583\\
5	-2.00239443319123\\
6	-2.00285528931853\\
7	-2.00918662600606\\
8	-2.00785390808672\\
9	-2.00651290899462\\
10	-2.00799735894325\\
11	-2.02426295946307\\
12	-2.02477048029684\\
13	-2.02323112379129\\
14	-2.0218747341585\\
15	-2.02231189414173\\
};
\addplot [color=mycolor2,line width=5.0pt,only marks,mark=asterisk,mark options={solid},forget plot]
  table[row sep=crcr]
	\caption{AAMDL evolution with the growing number of terms for samples of different size.}\label{fig:aamdl}
\end{figure}	
\section{Parameter estimation}
\par	Results of internal parameter estimation via EFOR-CMSS for different sample sizes are presented in Tables 2 and 3
\begin{table}[!h]
	\centering
	\caption{Estimated parameters for the sample length 2000.}
	\small
	\begin{tabular}{llllllllllll}
Step & Terms & C1 & C2 & C4 & C5 & C6 & C7 & C9 & C10 & AEER & AAMDL \\ 
\hline 
1 & $c$ & -234.54 & -204.62 & -69.32 & -119.47 & -1570.61 & -1369.94 & -810.16 & -578.87 & 0 & -1.565 \\ 
2 & $x_4$ & -177.03 & -144.09 & -68.52 & -76.68 & -1355.31 & -1111.81 & -630.59 & -490.28 & 0 & -1.977 \\ 
3 & $x_3$ & 84.08 & 64.3 & 40.2 & 28.83 & 725.98 & 566.33 & 311.64 & 259.45 & 0 & -2.12 \\ 
4 & $x_4,x_4$ & -23.32 & -17.55 & -8.58 & -11.12 & -150.42 & -144.42 & -83.78 & -56.07 & 0 & -2.183 \\ 
5 & $x_3,x_4$ & 9.41 & 4.86 & 2.53 & 5.82 & 19.58 & 58.54 & 37.45 & 9.85 & 0 & -2.197 \\ 
6 & $x_3,x_3$ & 3.85 & 4.19 & 2.53 & 0.2 & 63.86 & 28.86 & 13.13 & 21.49 & 0 & -2.196 \\ 
7 & $x_2$ & 1.42 & 0.75 & -2.7 & -0.05 & 43.65 & 38.66 & 17.66 & 13.68 & 0 & -2.203 \\ 
8 & $x_1$ & -4.32 & -2.9 & 0.19 & -0.79 & -54.57 & -45.03 & -21.76 & -17.7 & 0 & -2.219 \\ 
\hline 
\end{tabular}
\end{table}
\begin{table}[!h]
	\centering
	\caption{Estimated parameters for the sample length 4000.}
	\small
	\begin{tabular}{llllllllllllll}
Step & Terms & C1 & C2 & C3 & C4 & C5 & C6 & C7 & C8 & C9 & C10 & AEER & AAMDL \\ 
\hline 
1 & $c$ & -165.57 & -140.83 & -106.8 & -57.88 & -93.13 & -1037.61 & -964.13 & -843.17 & -559.51 & -370.04 & 0 & -1.665 \\ 
2 & $x_4$ & -156.14 & -123.14 & -93.42 & -60.44 & -67.23 & -1089.72 & -931.89 & -801 & -530.57 & -398.02 & 0 & -2.209 \\ 
3 & $x_3$ & 77.52 & 56.15 & 39.44 & 33.75 & 22.39 & 466.78 & 421.82 & 365.29 & 244.52 & 171.74 & 0 & -2.389 \\ 
4 & $x_4,x_4$ & -19.62 & -12.42 & -9.15 & -4.16 & -7.3 & -103.98 & -95.74 & -83.16 & -55.36 & -35.81 & 0 & -2.477 \\ 
5 & $x_3,x_4$ & -0.61 & -10.91 & -10.1 & -11.57 & -7.03 & -126.87 & -80.74 & -60.96 & -39.74 & -50.79 & 0 & -2.499 \\ 
6 & $x_3,x_3$ & 16.49 & 22.71 & 16.85 & 13.74 & 12.56 & 259.29 & 184.73 & 152.18 & 99.4 & 95.7 & 0 & -2.499 \\ 
7 & $x_2$ & -19.39 & -14.83 & -7.99 & -13.43 & -5.4 & -54.34 & -71.89 & -59.62 & -35.85 & -9.18 & 0 & -2.507 \\ 
8 & $x_2,x_4$ & 1.31 & 5.3 & 5.13 & 4.35 & 5.21 & 55.56 & 39.19 & 29.86 & 20.47 & 21.91 & 0 & -2.506 \\ 
9 & $x_2,x_3$ & -16.39 & -22.92 & -15.24 & -9.28 & -13.26 & -288.73 & -197.82 & -165.46 & -106.81 & -103.02 & 0 & -2.504 \\ 
10 & $x_2,x_2$ & 5.77 & 7.59 & 4.45 & 1.49 & 3.72 & 108.29 & 70.29 & 60.34 & 38.62 & 38.72 & 0 & -2.506 \\ 
11 & $x_1$ & 25.77 & 21.68 & 15.22 & 13.42 & 10.92 & 220.93 & 169.96 & 139.84 & 85.56 & 72.68 & 0 & -2.526 \\ 
12 & $x_1,x_4$ & 5.03 & 4.03 & 2.82 & 2.28 & 1.83 & 45.48 & 35.21 & 28.83 & 17.78 & 14.85 & 0 & -2.527 \\ 
\hline 
\end{tabular}
\end{table}
Visualised
\begin{figure}[!h]
	\centering
	% This file was created by matlab2tikz.
% Minimal pgfplots version: 1.3
%
\definecolor{mycolor1}{rgb}{0.00000,0.44700,0.74100}%
%
\begin{tikzpicture}

\begin{axis}[%
width=2.306487cm,
height=2.511628cm,
at={(0cm,0cm)},
scale only axis,
xmin=0,
xmax=10,
xlabel={Dataset index},
ymin=-2.23969084196957,
ymax=45.9418866067325,
ylabel={$x_2$},
legend style={legend cell align=left,align=left,draw=white!15!black}
]
\addplot [color=mycolor1,line width=2.0pt,only marks,mark=o,mark options={solid},forget plot]
  table[row sep=crcr]{%
1	0.706917161850943\\
2	-0.725188933625476\\
4	-2.23969084196957\\
5	-0.663995623424443\\
6	45.9418866067325\\
7	36.5734993512285\\
9	18.3955084875498\\
10	12.6775369279536\\
};
\end{axis}

\begin{axis}[%
width=2.306487cm,
height=2.511628cm,
at={(4cm,4cm)},
scale only axis,
xmin=0,
xmax=10,
xlabel={Dataset index},
ymin=0,
ymax=600,
ylabel={$x_3$},
legend style={legend cell align=left,align=left,draw=white!15!black}
]
\addplot [color=mycolor1,line width=2.0pt,only marks,mark=o,mark options={solid},forget plot]
  table[row sep=crcr]{%
1	75.4245651900459\\
2	59.5782863482935\\
4	33.0003406266977\\
5	26.0569470415911\\
6	508.93731875208\\
7	419.53506953587\\
9	242.35439857544\\
10	195.15231965962\\
};
\end{axis}

\begin{axis}[%
width=2.306487cm,
height=2.511628cm,
at={(12cm,4cm)},
scale only axis,
xmin=0,
xmax=10,
xlabel={Dataset index},
ymin=-0.5,
ymax=1,
ylabel={$x_1,x_1$},
legend style={legend cell align=left,align=left,draw=white!15!black}
]
\addplot [color=mycolor1,line width=2.0pt,only marks,mark=o,mark options={solid},forget plot]
  table[row sep=crcr]{%
1	0.00740370360905991\\
2	-0.186626587578763\\
4	-0.150607769785267\\
5	-0.216975707230133\\
6	0.0527524251509314\\
7	-0.475927501616939\\
9	0.435130634728223\\
10	0.76117565090573\\
};
\end{axis}

\begin{axis}[%
width=2.306487cm,
height=2.511628cm,
at={(0cm,4cm)},
scale only axis,
xmin=0,
xmax=10,
xlabel={Dataset index},
ymin=-200,
ymax=0,
ylabel={$x_4,x_4$},
legend style={legend cell align=left,align=left,draw=white!15!black}
]
\addplot [color=mycolor1,line width=2.0pt,only marks,mark=o,mark options={solid},forget plot]
  table[row sep=crcr]{%
1	-26.0353957891804\\
2	-20.9879322169279\\
4	-10.6920630070547\\
5	-10.9569379219045\\
6	-191.779551108501\\
7	-157.643535964989\\
9	-87.4196213968237\\
10	-69.8013569503948\\
};
\end{axis}

\begin{axis}[%
width=2.306487cm,
height=2.511628cm,
at={(8cm,4cm)},
scale only axis,
xmin=0,
xmax=10,
xlabel={Dataset index},
ymin=0,
ymax=10,
ylabel={$x_1,x_4$},
legend style={legend cell align=left,align=left,draw=white!15!black}
]
\addplot [color=mycolor1,line width=2.0pt,only marks,mark=o,mark options={solid},forget plot]
  table[row sep=crcr]{%
1	0.616147082677301\\
2	0.756244217181634\\
4	0.315136730386088\\
5	0.479225561787365\\
6	8.54895561081077\\
7	7.8265872893112\\
9	2.65659919911432\\
10	1.15263735041756\\
};
\end{axis}

\begin{axis}[%
width=2.306487cm,
height=2.511628cm,
at={(8cm,0cm)},
scale only axis,
xmin=0,
xmax=10,
xlabel={Dataset index},
ymin=-2000,
ymax=0,
ylabel={$c$},
legend style={legend cell align=left,align=left,draw=white!15!black}
]
\addplot [color=mycolor1,line width=2.0pt,only marks,mark=o,mark options={solid},forget plot]
  table[row sep=crcr]{%
1	-233.157966898601\\
2	-200.830593420783\\
4	-93.7391747783605\\
5	-119.696332564413\\
6	-1805.89673785913\\
7	-1488.55090215304\\
9	-803.703476355143\\
10	-648.796609601896\\
};
\end{axis}

\begin{axis}[%
width=2.306487cm,
height=2.511628cm,
at={(12cm,0cm)},
scale only axis,
xmin=0,
xmax=10,
xlabel={Dataset index},
ymin=0,
ymax=110.131370256376,
ylabel={$x_3,x_4$},
legend style={legend cell align=left,align=left,draw=white!15!black}
]
\addplot [color=mycolor1,line width=2.0pt,only marks,mark=o,mark options={solid},forget plot]
  table[row sep=crcr]{%
1	15.4680684467597\\
2	12.0981268654473\\
4	6.35969255839677\\
5	5.67589105085895\\
6	110.131370256376\\
7	90.4271696447197\\
9	51.7720891508705\\
10	41.4264697981385\\
};
\end{axis}

\begin{axis}[%
width=2.306487cm,
height=2.511628cm,
at={(4cm,0cm)},
scale only axis,
xmin=0,
xmax=10,
xlabel={Dataset index},
ymin=-1500,
ymax=0,
ylabel={$x_4$},
legend style={legend cell align=left,align=left,draw=white!15!black}
]
\addplot [color=mycolor1,line width=2.0pt,only marks,mark=o,mark options={solid},forget plot]
  table[row sep=crcr]{%
1	-171.244880190644\\
2	-139.217460396701\\
4	-69.6051966848396\\
5	-73.6870961160989\\
6	-1273.01999820883\\
7	-1046.38133574319\\
9	-579.722288788632\\
10	-465.590859275979\\
};
\end{axis}
\end{tikzpicture}%
	\caption{Estimated values of internal parameters.}
\end{figure}
\par In order to link the external and internal parameters an arbitrary polynomial function is selected for two arguments
\begin{equation}
content...
\end{equation}
Curve fitting results are presented in Tablels 4 and 5.
\begin{table}[!h]
	\centering
	\caption{Estimated polynomial coefficients for the sample length 2000.}
	\small
	\input{betas_C_ny_0_nu_4_size_2000.tex}
\end{table}

\begin{figure}[!h]
	\centering
	% This file was created by matlab2tikz.
% Minimal pgfplots version: 1.3
%
\definecolor{mycolor1}{rgb}{0.00000,0.44700,0.74100}%
%
\begin{tikzpicture}

\begin{axis}[%
width=4.927496cm,
height=3.870968cm,
at={(6.483547cm,0cm)},
scale only axis,
xmin=56,
xmax=74,
tick align=outside,
xlabel={$L_{cut}$},
xmajorgrids,
ymin=0.093,
ymax=0.123,
ylabel={$D_{rlx}$},
ymajorgrids,
zmin=5.67589105080756,
zmax=15.4680684467716,
zlabel={$x_3,x_4$},
zmajorgrids,
view={-140}{50},
legend style={at={(1.03,1)},anchor=north west,legend cell align=left,align=left,draw=white!15!black}
]
\addplot3[only marks,mark=*,mark options={},mark size=1.5000pt,color=mycolor1] plot table[row sep=crcr,]{%
74	0.123	15.4680684467597\\
72	0.113	12.0981268654473\\
61	0.095	6.35969255839677\\
56	0.093	5.67589105085895\\
};
\addplot3[only marks,mark=*,mark options={},mark size=1.5000pt,color=black] plot table[row sep=crcr,]{%
69	0.104	9.19562768312139\\
};

\addplot3[%
surf,
opacity=0.7,
shader=interp,
colormap={mymap}{[1pt] rgb(0pt)=(0.0901961,0.239216,0.0745098); rgb(1pt)=(0.0945149,0.242058,0.0739522); rgb(2pt)=(0.0988592,0.244894,0.0733566); rgb(3pt)=(0.103229,0.247724,0.0727241); rgb(4pt)=(0.107623,0.250549,0.0720557); rgb(5pt)=(0.112043,0.253367,0.0713525); rgb(6pt)=(0.116487,0.25618,0.0706154); rgb(7pt)=(0.120956,0.258986,0.0698456); rgb(8pt)=(0.125449,0.261787,0.0690441); rgb(9pt)=(0.129967,0.264581,0.0682118); rgb(10pt)=(0.134508,0.26737,0.06735); rgb(11pt)=(0.139074,0.270152,0.0664596); rgb(12pt)=(0.143663,0.272929,0.0655416); rgb(13pt)=(0.148275,0.275699,0.0645971); rgb(14pt)=(0.152911,0.278463,0.0636271); rgb(15pt)=(0.15757,0.281221,0.0626328); rgb(16pt)=(0.162252,0.283973,0.0616151); rgb(17pt)=(0.166957,0.286719,0.060575); rgb(18pt)=(0.171685,0.289458,0.0595136); rgb(19pt)=(0.176434,0.292191,0.0584321); rgb(20pt)=(0.181207,0.294918,0.0573313); rgb(21pt)=(0.186001,0.297639,0.0562123); rgb(22pt)=(0.190817,0.300353,0.0550763); rgb(23pt)=(0.195655,0.303061,0.0539242); rgb(24pt)=(0.200514,0.305763,0.052757); rgb(25pt)=(0.205395,0.308459,0.0515759); rgb(26pt)=(0.210296,0.311149,0.0503624); rgb(27pt)=(0.215212,0.313846,0.0490067); rgb(28pt)=(0.220142,0.316548,0.0475043); rgb(29pt)=(0.22509,0.319254,0.0458704); rgb(30pt)=(0.230056,0.321962,0.0441205); rgb(31pt)=(0.235042,0.324671,0.04227); rgb(32pt)=(0.240048,0.327379,0.0403343); rgb(33pt)=(0.245078,0.330085,0.0383287); rgb(34pt)=(0.250131,0.332786,0.0362688); rgb(35pt)=(0.25521,0.335482,0.0341698); rgb(36pt)=(0.260317,0.33817,0.0320472); rgb(37pt)=(0.265451,0.340849,0.0299163); rgb(38pt)=(0.270616,0.343517,0.0277927); rgb(39pt)=(0.275813,0.346172,0.0256916); rgb(40pt)=(0.281043,0.348814,0.0236284); rgb(41pt)=(0.286307,0.35144,0.0216186); rgb(42pt)=(0.291607,0.354048,0.0196776); rgb(43pt)=(0.296945,0.356637,0.0178207); rgb(44pt)=(0.302322,0.359206,0.0160634); rgb(45pt)=(0.307739,0.361753,0.0144211); rgb(46pt)=(0.313198,0.364275,0.0129091); rgb(47pt)=(0.318701,0.366772,0.0115428); rgb(48pt)=(0.324249,0.369242,0.0103377); rgb(49pt)=(0.329843,0.371682,0.00930909); rgb(50pt)=(0.335485,0.374093,0.00847245); rgb(51pt)=(0.341176,0.376471,0.00784314); rgb(52pt)=(0.346925,0.378826,0.00732741); rgb(53pt)=(0.352735,0.381168,0.00682184); rgb(54pt)=(0.358605,0.383497,0.00632729); rgb(55pt)=(0.364532,0.385812,0.00584464); rgb(56pt)=(0.370516,0.388113,0.00537476); rgb(57pt)=(0.376552,0.390399,0.00491852); rgb(58pt)=(0.38264,0.39267,0.00447681); rgb(59pt)=(0.388777,0.394925,0.00405048); rgb(60pt)=(0.394962,0.397164,0.00364042); rgb(61pt)=(0.401191,0.399386,0.00324749); rgb(62pt)=(0.407464,0.401592,0.00287258); rgb(63pt)=(0.413777,0.40378,0.00251655); rgb(64pt)=(0.420129,0.40595,0.00218028); rgb(65pt)=(0.426518,0.408102,0.00186463); rgb(66pt)=(0.432942,0.410234,0.00157049); rgb(67pt)=(0.439399,0.412348,0.00129873); rgb(68pt)=(0.445885,0.414441,0.00105022); rgb(69pt)=(0.452401,0.416515,0.000825833); rgb(70pt)=(0.458942,0.418567,0.000626441); rgb(71pt)=(0.465508,0.420599,0.00045292); rgb(72pt)=(0.472096,0.422609,0.000306141); rgb(73pt)=(0.478704,0.424596,0.000186979); rgb(74pt)=(0.485331,0.426562,9.63073e-05); rgb(75pt)=(0.491973,0.428504,3.49981e-05); rgb(76pt)=(0.498628,0.430422,3.92506e-06); rgb(77pt)=(0.505323,0.432315,0); rgb(78pt)=(0.512206,0.434168,0); rgb(79pt)=(0.519282,0.435983,0); rgb(80pt)=(0.526529,0.437764,0); rgb(81pt)=(0.533922,0.439512,0); rgb(82pt)=(0.54144,0.441232,0); rgb(83pt)=(0.549059,0.442927,0); rgb(84pt)=(0.556756,0.444599,0); rgb(85pt)=(0.564508,0.446252,0); rgb(86pt)=(0.572292,0.447889,0); rgb(87pt)=(0.580084,0.449514,0); rgb(88pt)=(0.587863,0.451129,0); rgb(89pt)=(0.595604,0.452737,0); rgb(90pt)=(0.603284,0.454343,0); rgb(91pt)=(0.610882,0.455948,0); rgb(92pt)=(0.618373,0.457556,0); rgb(93pt)=(0.625734,0.459171,0); rgb(94pt)=(0.632943,0.460795,0); rgb(95pt)=(0.639976,0.462432,0); rgb(96pt)=(0.64681,0.464084,0); rgb(97pt)=(0.653423,0.465756,0); rgb(98pt)=(0.659791,0.46745,0); rgb(99pt)=(0.665891,0.469169,0); rgb(100pt)=(0.6717,0.470916,0); rgb(101pt)=(0.677195,0.472696,0); rgb(102pt)=(0.682353,0.47451,0); rgb(103pt)=(0.687242,0.476355,0); rgb(104pt)=(0.691952,0.478225,0); rgb(105pt)=(0.696497,0.480118,0); rgb(106pt)=(0.700887,0.482033,0); rgb(107pt)=(0.705134,0.483968,0); rgb(108pt)=(0.709251,0.485921,0); rgb(109pt)=(0.713249,0.487891,0); rgb(110pt)=(0.71714,0.489876,0); rgb(111pt)=(0.720936,0.491875,0); rgb(112pt)=(0.724649,0.493887,0); rgb(113pt)=(0.72829,0.495909,0); rgb(114pt)=(0.731872,0.49794,0); rgb(115pt)=(0.735406,0.499979,0); rgb(116pt)=(0.738904,0.502025,0); rgb(117pt)=(0.742378,0.504075,0); rgb(118pt)=(0.74584,0.506128,0); rgb(119pt)=(0.749302,0.508182,0); rgb(120pt)=(0.752775,0.510237,0); rgb(121pt)=(0.756272,0.51229,0); rgb(122pt)=(0.759804,0.514339,0); rgb(123pt)=(0.763384,0.516385,0); rgb(124pt)=(0.767022,0.518424,0); rgb(125pt)=(0.770731,0.520455,0); rgb(126pt)=(0.774523,0.522478,0); rgb(127pt)=(0.77841,0.524489,0); rgb(128pt)=(0.782391,0.526491,0); rgb(129pt)=(0.786402,0.528496,0); rgb(130pt)=(0.790431,0.530506,0); rgb(131pt)=(0.794478,0.532521,0); rgb(132pt)=(0.798541,0.534539,0); rgb(133pt)=(0.802619,0.53656,0); rgb(134pt)=(0.806712,0.538584,0); rgb(135pt)=(0.81082,0.540609,0); rgb(136pt)=(0.81494,0.542635,0); rgb(137pt)=(0.819074,0.54466,0); rgb(138pt)=(0.823219,0.546686,0); rgb(139pt)=(0.827374,0.548709,0); rgb(140pt)=(0.831541,0.55073,0); rgb(141pt)=(0.835716,0.552749,0); rgb(142pt)=(0.8399,0.554763,0); rgb(143pt)=(0.844092,0.556774,0); rgb(144pt)=(0.848292,0.558779,0); rgb(145pt)=(0.852497,0.560778,0); rgb(146pt)=(0.856708,0.562771,0); rgb(147pt)=(0.860924,0.564756,0); rgb(148pt)=(0.865143,0.566733,0); rgb(149pt)=(0.869366,0.568701,0); rgb(150pt)=(0.873592,0.57066,0); rgb(151pt)=(0.877819,0.572608,0); rgb(152pt)=(0.882047,0.574545,0); rgb(153pt)=(0.886275,0.576471,0); rgb(154pt)=(0.890659,0.578362,0); rgb(155pt)=(0.895333,0.580203,0); rgb(156pt)=(0.900258,0.581999,0); rgb(157pt)=(0.905397,0.583755,0); rgb(158pt)=(0.910711,0.585479,0); rgb(159pt)=(0.916164,0.587176,0); rgb(160pt)=(0.921717,0.588852,0); rgb(161pt)=(0.927333,0.590513,0); rgb(162pt)=(0.932974,0.592166,0); rgb(163pt)=(0.938602,0.593815,0); rgb(164pt)=(0.94418,0.595468,0); rgb(165pt)=(0.949669,0.59713,0); rgb(166pt)=(0.955033,0.598808,0); rgb(167pt)=(0.960233,0.600507,0); rgb(168pt)=(0.965232,0.602233,0); rgb(169pt)=(0.969992,0.603992,0); rgb(170pt)=(0.974475,0.605791,0); rgb(171pt)=(0.978643,0.607636,0); rgb(172pt)=(0.98246,0.609532,0); rgb(173pt)=(0.985886,0.611486,0); rgb(174pt)=(0.988885,0.613503,0); rgb(175pt)=(0.991419,0.61559,0); rgb(176pt)=(0.99345,0.617753,0); rgb(177pt)=(0.99494,0.619997,0); rgb(178pt)=(0.995851,0.622329,0); rgb(179pt)=(0.996226,0.624763,0); rgb(180pt)=(0.996512,0.627352,0); rgb(181pt)=(0.996788,0.630095,0); rgb(182pt)=(0.997053,0.632982,0); rgb(183pt)=(0.997308,0.636004,0); rgb(184pt)=(0.997552,0.639152,0); rgb(185pt)=(0.997785,0.642416,0); rgb(186pt)=(0.998006,0.645786,0); rgb(187pt)=(0.998217,0.649253,0); rgb(188pt)=(0.998416,0.652807,0); rgb(189pt)=(0.998605,0.656439,0); rgb(190pt)=(0.998781,0.660138,0); rgb(191pt)=(0.998946,0.663897,0); rgb(192pt)=(0.9991,0.667704,0); rgb(193pt)=(0.999242,0.67155,0); rgb(194pt)=(0.999372,0.675427,0); rgb(195pt)=(0.99949,0.679323,0); rgb(196pt)=(0.999596,0.68323,0); rgb(197pt)=(0.99969,0.687139,0); rgb(198pt)=(0.999771,0.691039,0); rgb(199pt)=(0.999841,0.694921,0); rgb(200pt)=(0.999898,0.698775,0); rgb(201pt)=(0.999942,0.702592,0); rgb(202pt)=(0.999974,0.706363,0); rgb(203pt)=(0.999994,0.710077,0); rgb(204pt)=(1,0.713725,0); rgb(205pt)=(1,0.717341,0); rgb(206pt)=(1,0.720963,0); rgb(207pt)=(1,0.724591,0); rgb(208pt)=(1,0.728226,0); rgb(209pt)=(1,0.731867,0); rgb(210pt)=(1,0.735514,0); rgb(211pt)=(1,0.739167,0); rgb(212pt)=(1,0.742827,0); rgb(213pt)=(1,0.746493,0); rgb(214pt)=(1,0.750165,0); rgb(215pt)=(1,0.753843,0); rgb(216pt)=(1,0.757527,0); rgb(217pt)=(1,0.761217,0); rgb(218pt)=(1,0.764913,0); rgb(219pt)=(1,0.768615,0); rgb(220pt)=(1,0.772324,0); rgb(221pt)=(1,0.776038,0); rgb(222pt)=(1,0.779758,0); rgb(223pt)=(1,0.783484,0); rgb(224pt)=(1,0.787215,0); rgb(225pt)=(1,0.790953,0); rgb(226pt)=(1,0.794696,0); rgb(227pt)=(1,0.798445,0); rgb(228pt)=(1,0.8022,0); rgb(229pt)=(1,0.805961,0); rgb(230pt)=(1,0.809727,0); rgb(231pt)=(1,0.8135,0); rgb(232pt)=(1,0.817278,0); rgb(233pt)=(1,0.821063,0); rgb(234pt)=(1,0.824854,0); rgb(235pt)=(1,0.828652,0); rgb(236pt)=(1,0.832455,0); rgb(237pt)=(1,0.836265,0); rgb(238pt)=(1,0.840081,0); rgb(239pt)=(1,0.843903,0); rgb(240pt)=(1,0.847732,0); rgb(241pt)=(1,0.851566,0); rgb(242pt)=(1,0.855406,0); rgb(243pt)=(1,0.859253,0); rgb(244pt)=(1,0.863106,0); rgb(245pt)=(1,0.866964,0); rgb(246pt)=(1,0.870829,0); rgb(247pt)=(1,0.8747,0); rgb(248pt)=(1,0.878577,0); rgb(249pt)=(1,0.88246,0); rgb(250pt)=(1,0.886349,0); rgb(251pt)=(1,0.890243,0); rgb(252pt)=(1,0.894144,0); rgb(253pt)=(1,0.898051,0); rgb(254pt)=(1,0.901964,0); rgb(255pt)=(1,0.905882,0)},
mesh/rows=49]
table[row sep=crcr,header=false] {%
%
56	0.093	5.67589105080756\\
56	0.0936	5.72337040951015\\
56	0.0942	5.77084976821273\\
56	0.0948	5.81832912691532\\
56	0.0954	5.8658084856179\\
56	0.096	5.91328784432049\\
56	0.0966	5.96076720302307\\
56	0.0972	6.00824656172566\\
56	0.0978	6.05572592042825\\
56	0.0984	6.10320527913083\\
56	0.099	6.15068463783342\\
56	0.0996	6.19816399653601\\
56	0.1002	6.24564335523859\\
56	0.1008	6.29312271394118\\
56	0.1014	6.34060207264377\\
56	0.102	6.38808143134635\\
56	0.1026	6.43556079004894\\
56	0.1032	6.48304014875152\\
56	0.1038	6.53051950745411\\
56	0.1044	6.57799886615669\\
56	0.105	6.62547822485927\\
56	0.1056	6.67295758356186\\
56	0.1062	6.72043694226445\\
56	0.1068	6.76791630096703\\
56	0.1074	6.81539565966963\\
56	0.108	6.86287501837221\\
56	0.1086	6.9103543770748\\
56	0.1092	6.95783373577738\\
56	0.1098	7.00531309447996\\
56	0.1104	7.05279245318255\\
56	0.111	7.10027181188514\\
56	0.1116	7.14775117058772\\
56	0.1122	7.19523052929031\\
56	0.1128	7.2427098879929\\
56	0.1134	7.29018924669548\\
56	0.114	7.33766860539807\\
56	0.1146	7.38514796410065\\
56	0.1152	7.43262732280323\\
56	0.1158	7.48010668150582\\
56	0.1164	7.5275860402084\\
56	0.117	7.57506539891099\\
56	0.1176	7.62254475761357\\
56	0.1182	7.67002411631617\\
56	0.1188	7.71750347501876\\
56	0.1194	7.76498283372133\\
56	0.12	7.81246219242392\\
56	0.1206	7.85994155112651\\
56	0.1212	7.9074209098291\\
56	0.1218	7.95490026853167\\
56	0.1224	8.00237962723426\\
56	0.123	8.04985898593685\\
56.375	0.093	5.70708269869846\\
56.375	0.0936	5.75702914505191\\
56.375	0.0942	5.80697559140536\\
56.375	0.0948	5.85692203775881\\
56.375	0.0954	5.90686848411226\\
56.375	0.096	5.95681493046571\\
56.375	0.0966	6.00676137681916\\
56.375	0.0972	6.0567078231726\\
56.375	0.0978	6.10665426952605\\
56.375	0.0984	6.1566007158795\\
56.375	0.099	6.20654716223295\\
56.375	0.0996	6.2564936085864\\
56.375	0.1002	6.30644005493985\\
56.375	0.1008	6.3563865012933\\
56.375	0.1014	6.40633294764675\\
56.375	0.102	6.45627939400019\\
56.375	0.1026	6.50622584035364\\
56.375	0.1032	6.55617228670709\\
56.375	0.1038	6.60611873306054\\
56.375	0.1044	6.65606517941399\\
56.375	0.105	6.70601162576745\\
56.375	0.1056	6.75595807212089\\
56.375	0.1062	6.80590451847434\\
56.375	0.1068	6.85585096482779\\
56.375	0.1074	6.90579741118124\\
56.375	0.108	6.95574385753468\\
56.375	0.1086	7.00569030388814\\
56.375	0.1092	7.05563675024159\\
56.375	0.1098	7.10558319659503\\
56.375	0.1104	7.15552964294849\\
56.375	0.111	7.20547608930193\\
56.375	0.1116	7.25542253565538\\
56.375	0.1122	7.30536898200882\\
56.375	0.1128	7.35531542836228\\
56.375	0.1134	7.40526187471572\\
56.375	0.114	7.45520832106918\\
56.375	0.1146	7.50515476742262\\
56.375	0.1152	7.55510121377607\\
56.375	0.1158	7.60504766012953\\
56.375	0.1164	7.65499410648297\\
56.375	0.117	7.70494055283642\\
56.375	0.1176	7.75488699918986\\
56.375	0.1182	7.80483344554331\\
56.375	0.1188	7.85477989189677\\
56.375	0.1194	7.9047263382502\\
56.375	0.12	7.95467278460367\\
56.375	0.1206	8.0046192309571\\
56.375	0.1212	8.05456567731058\\
56.375	0.1218	8.10451212366401\\
56.375	0.1224	8.15445857001747\\
56.375	0.123	8.2044050163709\\
56.75	0.093	5.73827434658936\\
56.75	0.0936	5.79068788059367\\
56.75	0.0942	5.84310141459799\\
56.75	0.0948	5.8955149486023\\
56.75	0.0954	5.94792848260661\\
56.75	0.096	6.00034201661093\\
56.75	0.0966	6.05275555061524\\
56.75	0.0972	6.10516908461955\\
56.75	0.0978	6.15758261862386\\
56.75	0.0984	6.20999615262817\\
56.75	0.099	6.26240968663249\\
56.75	0.0996	6.3148232206368\\
56.75	0.1002	6.36723675464111\\
56.75	0.1008	6.41965028864542\\
56.75	0.1014	6.47206382264974\\
56.75	0.102	6.52447735665405\\
56.75	0.1026	6.57689089065836\\
56.75	0.1032	6.62930442466266\\
56.75	0.1038	6.68171795866698\\
56.75	0.1044	6.7341314926713\\
56.75	0.105	6.78654502667561\\
56.75	0.1056	6.83895856067991\\
56.75	0.1062	6.89137209468423\\
56.75	0.1068	6.94378562868854\\
56.75	0.1074	6.99619916269286\\
56.75	0.108	7.04861269669717\\
56.75	0.1086	7.10102623070149\\
56.75	0.1092	7.15343976470579\\
56.75	0.1098	7.2058532987101\\
56.75	0.1104	7.25826683271442\\
56.75	0.111	7.31068036671874\\
56.75	0.1116	7.36309390072304\\
56.75	0.1122	7.41550743472735\\
56.75	0.1128	7.46792096873167\\
56.75	0.1134	7.52033450273598\\
56.75	0.114	7.57274803674029\\
56.75	0.1146	7.6251615707446\\
56.75	0.1152	7.67757510474891\\
56.75	0.1158	7.72998863875323\\
56.75	0.1164	7.78240217275753\\
56.75	0.117	7.83481570676184\\
56.75	0.1176	7.88722924076616\\
56.75	0.1182	7.93964277477048\\
56.75	0.1188	7.99205630877478\\
56.75	0.1194	8.04446984277911\\
56.75	0.12	8.0968833767834\\
56.75	0.1206	8.14929691078773\\
56.75	0.1212	8.20171044479203\\
56.75	0.1218	8.25412397879636\\
56.75	0.1224	8.30653751280066\\
56.75	0.123	8.35895104680499\\
57.125	0.093	5.76946599448026\\
57.125	0.0936	5.82434661613544\\
57.125	0.0942	5.87922723779062\\
57.125	0.0948	5.9341078594458\\
57.125	0.0954	5.98898848110096\\
57.125	0.096	6.04386910275614\\
57.125	0.0966	6.09874972441131\\
57.125	0.0972	6.15363034606649\\
57.125	0.0978	6.20851096772166\\
57.125	0.0984	6.26339158937684\\
57.125	0.099	6.31827221103202\\
57.125	0.0996	6.3731528326872\\
57.125	0.1002	6.42803345434236\\
57.125	0.1008	6.48291407599754\\
57.125	0.1014	6.53779469765272\\
57.125	0.102	6.59267531930789\\
57.125	0.1026	6.64755594096307\\
57.125	0.1032	6.70243656261824\\
57.125	0.1038	6.75731718427343\\
57.125	0.1044	6.81219780592859\\
57.125	0.105	6.86707842758377\\
57.125	0.1056	6.92195904923894\\
57.125	0.1062	6.97683967089413\\
57.125	0.1068	7.03172029254929\\
57.125	0.1074	7.08660091420447\\
57.125	0.108	7.14148153585964\\
57.125	0.1086	7.19636215751483\\
57.125	0.1092	7.25124277916999\\
57.125	0.1098	7.30612340082517\\
57.125	0.1104	7.36100402248034\\
57.125	0.111	7.41588464413553\\
57.125	0.1116	7.4707652657907\\
57.125	0.1122	7.52564588744588\\
57.125	0.1128	7.58052650910105\\
57.125	0.1134	7.63540713075623\\
57.125	0.114	7.6902877524114\\
57.125	0.1146	7.74516837406657\\
57.125	0.1152	7.80004899572174\\
57.125	0.1158	7.85492961737694\\
57.125	0.1164	7.9098102390321\\
57.125	0.117	7.96469086068728\\
57.125	0.1176	8.01957148234244\\
57.125	0.1182	8.07445210399763\\
57.125	0.1188	8.1293327256528\\
57.125	0.1194	8.18421334730797\\
57.125	0.12	8.23909396896315\\
57.125	0.1206	8.29397459061833\\
57.125	0.1212	8.34885521227351\\
57.125	0.1218	8.40373583392868\\
57.125	0.1224	8.45861645558385\\
57.125	0.123	8.51349707723904\\
57.5	0.093	5.80065764237116\\
57.5	0.0936	5.85800535167719\\
57.5	0.0942	5.91535306098324\\
57.5	0.0948	5.97270077028927\\
57.5	0.0954	6.03004847959531\\
57.5	0.096	6.08739618890135\\
57.5	0.0966	6.14474389820738\\
57.5	0.0972	6.20209160751342\\
57.5	0.0978	6.25943931681947\\
57.5	0.0984	6.3167870261255\\
57.5	0.099	6.37413473543154\\
57.5	0.0996	6.43148244473758\\
57.5	0.1002	6.48883015404361\\
57.5	0.1008	6.54617786334966\\
57.5	0.1014	6.6035255726557\\
57.5	0.102	6.66087328196173\\
57.5	0.1026	6.71822099126777\\
57.5	0.1032	6.77556870057381\\
57.5	0.1038	6.83291640987985\\
57.5	0.1044	6.89026411918589\\
57.5	0.105	6.94761182849192\\
57.5	0.1056	7.00495953779797\\
57.5	0.1062	7.06230724710401\\
57.5	0.1068	7.11965495641005\\
57.5	0.1074	7.17700266571608\\
57.5	0.108	7.23435037502212\\
57.5	0.1086	7.29169808432816\\
57.5	0.1092	7.3490457936342\\
57.5	0.1098	7.40639350294023\\
57.5	0.1104	7.46374121224628\\
57.5	0.111	7.52108892155231\\
57.5	0.1116	7.57843663085834\\
57.5	0.1122	7.63578434016439\\
57.5	0.1128	7.69313204947042\\
57.5	0.1134	7.75047975877646\\
57.5	0.114	7.80782746808251\\
57.5	0.1146	7.86517517738854\\
57.5	0.1152	7.92252288669458\\
57.5	0.1158	7.97987059600061\\
57.5	0.1164	8.03721830530665\\
57.5	0.117	8.09456601461268\\
57.5	0.1176	8.15191372391874\\
57.5	0.1182	8.20926143322477\\
57.5	0.1188	8.26660914253081\\
57.5	0.1194	8.32395685183684\\
57.5	0.12	8.38130456114288\\
57.5	0.1206	8.43865227044893\\
57.5	0.1212	8.49599997975497\\
57.5	0.1218	8.553347689061\\
57.5	0.1224	8.61069539836704\\
57.5	0.123	8.66804310767309\\
57.875	0.093	5.83184929026206\\
57.875	0.0936	5.89166408721896\\
57.875	0.0942	5.95147888417586\\
57.875	0.0948	6.01129368113276\\
57.875	0.0954	6.07110847808966\\
57.875	0.096	6.13092327504657\\
57.875	0.0966	6.19073807200346\\
57.875	0.0972	6.25055286896037\\
57.875	0.0978	6.31036766591727\\
57.875	0.0984	6.37018246287417\\
57.875	0.099	6.42999725983108\\
57.875	0.0996	6.48981205678798\\
57.875	0.1002	6.54962685374488\\
57.875	0.1008	6.60944165070178\\
57.875	0.1014	6.66925644765868\\
57.875	0.102	6.72907124461558\\
57.875	0.1026	6.78888604157248\\
57.875	0.1032	6.84870083852939\\
57.875	0.1038	6.90851563548629\\
57.875	0.1044	6.9683304324432\\
57.875	0.105	7.02814522940009\\
57.875	0.1056	7.08796002635699\\
57.875	0.1062	7.1477748233139\\
57.875	0.1068	7.2075896202708\\
57.875	0.1074	7.2674044172277\\
57.875	0.108	7.32721921418459\\
57.875	0.1086	7.38703401114149\\
57.875	0.1092	7.4468488080984\\
57.875	0.1098	7.50666360505529\\
57.875	0.1104	7.56647840201221\\
57.875	0.111	7.62629319896911\\
57.875	0.1116	7.68610799592602\\
57.875	0.1122	7.74592279288291\\
57.875	0.1128	7.80573758983982\\
57.875	0.1134	7.86555238679672\\
57.875	0.114	7.92536718375361\\
57.875	0.1146	7.98518198071052\\
57.875	0.1152	8.04499677766741\\
57.875	0.1158	8.10481157462434\\
57.875	0.1164	8.16462637158122\\
57.875	0.117	8.22444116853814\\
57.875	0.1176	8.28425596549502\\
57.875	0.1182	8.34407076245192\\
57.875	0.1188	8.40388555940883\\
57.875	0.1194	8.46370035636572\\
57.875	0.12	8.52351515332263\\
57.875	0.1206	8.58332995027953\\
57.875	0.1212	8.64314474723645\\
57.875	0.1218	8.70295954419333\\
57.875	0.1224	8.76277434115025\\
57.875	0.123	8.82258913810715\\
58.25	0.093	5.86304093815296\\
58.25	0.0936	5.92532282276072\\
58.25	0.0942	5.98760470736849\\
58.25	0.0948	6.04988659197625\\
58.25	0.0954	6.11216847658402\\
58.25	0.096	6.17445036119178\\
58.25	0.0966	6.23673224579954\\
58.25	0.0972	6.29901413040731\\
58.25	0.0978	6.36129601501508\\
58.25	0.0984	6.42357789962283\\
58.25	0.099	6.48585978423061\\
58.25	0.0996	6.54814166883837\\
58.25	0.1002	6.61042355344613\\
58.25	0.1008	6.6727054380539\\
58.25	0.1014	6.73498732266167\\
58.25	0.102	6.79726920726943\\
58.25	0.1026	6.8595510918772\\
58.25	0.1032	6.92183297648495\\
58.25	0.1038	6.98411486109272\\
58.25	0.1044	7.04639674570048\\
58.25	0.105	7.10867863030825\\
58.25	0.1056	7.17096051491602\\
58.25	0.1062	7.23324239952379\\
58.25	0.1068	7.29552428413155\\
58.25	0.1074	7.35780616873932\\
58.25	0.108	7.42008805334708\\
58.25	0.1086	7.48236993795484\\
58.25	0.1092	7.54465182256261\\
58.25	0.1098	7.60693370717036\\
58.25	0.1104	7.66921559177813\\
58.25	0.111	7.7314974763859\\
58.25	0.1116	7.79377936099367\\
58.25	0.1122	7.85606124560143\\
58.25	0.1128	7.91834313020919\\
58.25	0.1134	7.98062501481697\\
58.25	0.114	8.04290689942472\\
58.25	0.1146	8.10518878403248\\
58.25	0.1152	8.16747066864025\\
58.25	0.1158	8.22975255324802\\
58.25	0.1164	8.29203443785579\\
58.25	0.117	8.35431632246355\\
58.25	0.1176	8.41659820707132\\
58.25	0.1182	8.47888009167907\\
58.25	0.1188	8.54116197628686\\
58.25	0.1194	8.6034438608946\\
58.25	0.12	8.66572574550239\\
58.25	0.1206	8.72800763011013\\
58.25	0.1212	8.79028951471791\\
58.25	0.1218	8.85257139932565\\
58.25	0.1224	8.91485328393344\\
58.25	0.123	8.9771351685412\\
58.625	0.093	5.89423258604385\\
58.625	0.0936	5.95898155830248\\
58.625	0.0942	6.0237305305611\\
58.625	0.0948	6.08847950281974\\
58.625	0.0954	6.15322847507836\\
58.625	0.096	6.21797744733699\\
58.625	0.0966	6.28272641959561\\
58.625	0.0972	6.34747539185424\\
58.625	0.0978	6.41222436411287\\
58.625	0.0984	6.4769733363715\\
58.625	0.099	6.54172230863013\\
58.625	0.0996	6.60647128088875\\
58.625	0.1002	6.67122025314737\\
58.625	0.1008	6.73596922540602\\
58.625	0.1014	6.80071819766464\\
58.625	0.102	6.86546716992327\\
58.625	0.1026	6.9302161421819\\
58.625	0.1032	6.99496511444052\\
58.625	0.1038	7.05971408669915\\
58.625	0.1044	7.12446305895778\\
58.625	0.105	7.1892120312164\\
58.625	0.1056	7.25396100347504\\
58.625	0.1062	7.31870997573367\\
58.625	0.1068	7.38345894799229\\
58.625	0.1074	7.44820792025092\\
58.625	0.108	7.51295689250954\\
58.625	0.1086	7.57770586476817\\
58.625	0.1092	7.6424548370268\\
58.625	0.1098	7.70720380928542\\
58.625	0.1104	7.77195278154407\\
58.625	0.111	7.83670175380269\\
58.625	0.1116	7.90145072606131\\
58.625	0.1122	7.96619969831994\\
58.625	0.1128	8.03094867057858\\
58.625	0.1134	8.09569764283719\\
58.625	0.114	8.16044661509584\\
58.625	0.1146	8.22519558735446\\
58.625	0.1152	8.28994455961309\\
58.625	0.1158	8.35469353187172\\
58.625	0.1164	8.41944250413033\\
58.625	0.117	8.48419147638897\\
58.625	0.1176	8.5489404486476\\
58.625	0.1182	8.61368942090623\\
58.625	0.1188	8.67843839316485\\
58.625	0.1194	8.74318736542348\\
58.625	0.12	8.80793633768209\\
58.625	0.1206	8.87268530994075\\
58.625	0.1212	8.93743428219936\\
58.625	0.1218	9.002183254458\\
58.625	0.1224	9.06693222671662\\
58.625	0.123	9.13168119897526\\
59	0.093	5.92542423393476\\
59	0.0936	5.99264029384425\\
59	0.0942	6.05985635375374\\
59	0.0948	6.12707241366324\\
59	0.0954	6.19428847357272\\
59	0.096	6.26150453348222\\
59	0.0966	6.3287205933917\\
59	0.0972	6.39593665330119\\
59	0.0978	6.46315271321069\\
59	0.0984	6.53036877312017\\
59	0.099	6.59758483302967\\
59	0.0996	6.66480089293916\\
59	0.1002	6.73201695284865\\
59	0.1008	6.79923301275814\\
59	0.1014	6.86644907266763\\
59	0.102	6.93366513257712\\
59	0.1026	7.00088119248662\\
59	0.1032	7.06809725239611\\
59	0.1038	7.13531331230561\\
59	0.1044	7.2025293722151\\
59	0.105	7.26974543212457\\
59	0.1056	7.33696149203407\\
59	0.1062	7.40417755194357\\
59	0.1068	7.47139361185306\\
59	0.1074	7.53860967176255\\
59	0.108	7.60582573167204\\
59	0.1086	7.67304179158153\\
59	0.1092	7.74025785149102\\
59	0.1098	7.80747391140051\\
59	0.1104	7.87468997131\\
59	0.111	7.9419060312195\\
59	0.1116	8.00912209112897\\
59	0.1122	8.07633815103847\\
59	0.1128	8.14355421094797\\
59	0.1134	8.21077027085747\\
59	0.114	8.27798633076695\\
59	0.1146	8.34520239067643\\
59	0.1152	8.41241845058593\\
59	0.1158	8.47963451049543\\
59	0.1164	8.54685057040491\\
59	0.117	8.61406663031441\\
59	0.1176	8.6812826902239\\
59	0.1182	8.74849875013339\\
59	0.1188	8.81571481004288\\
59	0.1194	8.88293086995236\\
59	0.12	8.95014692986186\\
59	0.1206	9.01736298977136\\
59	0.1212	9.08457904968085\\
59	0.1218	9.15179510959034\\
59	0.1224	9.21901116949984\\
59	0.123	9.28622722940932\\
59.375	0.093	5.95661588182565\\
59.375	0.0936	6.026299029386\\
59.375	0.0942	6.09598217694636\\
59.375	0.0948	6.16566532450671\\
59.375	0.0954	6.23534847206707\\
59.375	0.096	6.30503161962742\\
59.375	0.0966	6.37471476718778\\
59.375	0.0972	6.44439791474813\\
59.375	0.0978	6.51408106230848\\
59.375	0.0984	6.58376420986883\\
59.375	0.099	6.65344735742919\\
59.375	0.0996	6.72313050498955\\
59.375	0.1002	6.7928136525499\\
59.375	0.1008	6.86249680011025\\
59.375	0.1014	6.93217994767062\\
59.375	0.102	7.00186309523096\\
59.375	0.1026	7.07154624279131\\
59.375	0.1032	7.14122939035168\\
59.375	0.1038	7.21091253791203\\
59.375	0.1044	7.28059568547238\\
59.375	0.105	7.35027883303274\\
59.375	0.1056	7.41996198059309\\
59.375	0.1062	7.48964512815345\\
59.375	0.1068	7.5593282757138\\
59.375	0.1074	7.62901142327416\\
59.375	0.108	7.6986945708345\\
59.375	0.1086	7.76837771839486\\
59.375	0.1092	7.83806086595522\\
59.375	0.1098	7.90774401351557\\
59.375	0.1104	7.97742716107592\\
59.375	0.111	8.04711030863628\\
59.375	0.1116	8.11679345619665\\
59.375	0.1122	8.18647660375699\\
59.375	0.1128	8.25615975131734\\
59.375	0.1134	8.3258428988777\\
59.375	0.114	8.39552604643806\\
59.375	0.1146	8.4652091939984\\
59.375	0.1152	8.53489234155877\\
59.375	0.1158	8.60457548911911\\
59.375	0.1164	8.67425863667947\\
59.375	0.117	8.74394178423982\\
59.375	0.1176	8.81362493180019\\
59.375	0.1182	8.88330807936053\\
59.375	0.1188	8.95299122692089\\
59.375	0.1194	9.02267437448124\\
59.375	0.12	9.0923575220416\\
59.375	0.1206	9.16204066960195\\
59.375	0.1212	9.2317238171623\\
59.375	0.1218	9.30140696472266\\
59.375	0.1224	9.37109011228301\\
59.375	0.123	9.44077325984337\\
59.75	0.093	5.98780752971656\\
59.75	0.0936	6.05995776492777\\
59.75	0.0942	6.13210800013899\\
59.75	0.0948	6.20425823535021\\
59.75	0.0954	6.27640847056143\\
59.75	0.096	6.34855870577265\\
59.75	0.0966	6.42070894098386\\
59.75	0.0972	6.49285917619508\\
59.75	0.0978	6.56500941140629\\
59.75	0.0984	6.63715964661751\\
59.75	0.099	6.70930988182874\\
59.75	0.0996	6.78146011703996\\
59.75	0.1002	6.85361035225117\\
59.75	0.1008	6.92576058746238\\
59.75	0.1014	6.99791082267361\\
59.75	0.102	7.07006105788482\\
59.75	0.1026	7.14221129309605\\
59.75	0.1032	7.21436152830725\\
59.75	0.1038	7.28651176351848\\
59.75	0.1044	7.35866199872969\\
59.75	0.105	7.43081223394091\\
59.75	0.1056	7.50296246915212\\
59.75	0.1062	7.57511270436335\\
59.75	0.1068	7.64726293957457\\
59.75	0.1074	7.71941317478579\\
59.75	0.108	7.79156340999699\\
59.75	0.1086	7.86371364520821\\
59.75	0.1092	7.93586388041943\\
59.75	0.1098	8.00801411563065\\
59.75	0.1104	8.08016435084187\\
59.75	0.111	8.15231458605308\\
59.75	0.1116	8.22446482126431\\
59.75	0.1122	8.29661505647552\\
59.75	0.1128	8.36876529168676\\
59.75	0.1134	8.44091552689795\\
59.75	0.114	8.51306576210918\\
59.75	0.1146	8.5852159973204\\
59.75	0.1152	8.65736623253161\\
59.75	0.1158	8.72951646774283\\
59.75	0.1164	8.80166670295404\\
59.75	0.117	8.87381693816526\\
59.75	0.1176	8.94596717337649\\
59.75	0.1182	9.0181174085877\\
59.75	0.1188	9.09026764379891\\
59.75	0.1194	9.16241787901014\\
59.75	0.12	9.23456811422135\\
59.75	0.1206	9.30671834943257\\
59.75	0.1212	9.37886858464378\\
59.75	0.1218	9.45101881985501\\
59.75	0.1224	9.52316905506622\\
59.75	0.123	9.59531929027746\\
60.125	0.093	6.01899917760745\\
60.125	0.0936	6.09361650046953\\
60.125	0.0942	6.16823382333161\\
60.125	0.0948	6.24285114619369\\
60.125	0.0954	6.31746846905577\\
60.125	0.096	6.39208579191786\\
60.125	0.0966	6.46670311477993\\
60.125	0.0972	6.54132043764201\\
60.125	0.0978	6.6159377605041\\
60.125	0.0984	6.69055508336617\\
60.125	0.099	6.76517240622825\\
60.125	0.0996	6.83978972909034\\
60.125	0.1002	6.91440705195242\\
60.125	0.1008	6.9890243748145\\
60.125	0.1014	7.06364169767658\\
60.125	0.102	7.13825902053866\\
60.125	0.1026	7.21287634340074\\
60.125	0.1032	7.28749366626282\\
60.125	0.1038	7.3621109891249\\
60.125	0.1044	7.43672831198699\\
60.125	0.105	7.51134563484906\\
60.125	0.1056	7.58596295771115\\
60.125	0.1062	7.66058028057323\\
60.125	0.1068	7.7351976034353\\
60.125	0.1074	7.80981492629739\\
60.125	0.108	7.88443224915947\\
60.125	0.1086	7.95904957202155\\
60.125	0.1092	8.03366689488364\\
60.125	0.1098	8.10828421774572\\
60.125	0.1104	8.18290154060779\\
60.125	0.111	8.25751886346988\\
60.125	0.1116	8.33213618633195\\
60.125	0.1122	8.40675350919405\\
60.125	0.1128	8.48137083205611\\
60.125	0.1134	8.5559881549182\\
60.125	0.114	8.63060547778028\\
60.125	0.1146	8.70522280064235\\
60.125	0.1152	8.77984012350444\\
60.125	0.1158	8.85445744636652\\
60.125	0.1164	8.92907476922861\\
60.125	0.117	9.00369209209067\\
60.125	0.1176	9.07830941495277\\
60.125	0.1182	9.15292673781484\\
60.125	0.1188	9.22754406067693\\
60.125	0.1194	9.30216138353899\\
60.125	0.12	9.3767787064011\\
60.125	0.1206	9.45139602926317\\
60.125	0.1212	9.52601335212526\\
60.125	0.1218	9.60063067498731\\
60.125	0.1224	9.67524799784941\\
60.125	0.123	9.74986532071149\\
60.5	0.093	6.05019082549835\\
60.5	0.0936	6.12727523601129\\
60.5	0.0942	6.20435964652424\\
60.5	0.0948	6.28144405703718\\
60.5	0.0954	6.35852846755012\\
60.5	0.096	6.43561287806307\\
60.5	0.0966	6.51269728857601\\
60.5	0.0972	6.58978169908895\\
60.5	0.0978	6.66686610960191\\
60.5	0.0984	6.74395052011484\\
60.5	0.099	6.82103493062779\\
60.5	0.0996	6.89811934114073\\
60.5	0.1002	6.97520375165367\\
60.5	0.1008	7.05228816216662\\
60.5	0.1014	7.12937257267957\\
60.5	0.102	7.20645698319251\\
60.5	0.1026	7.28354139370545\\
60.5	0.1032	7.36062580421839\\
60.5	0.1038	7.43771021473133\\
60.5	0.1044	7.51479462524428\\
60.5	0.105	7.59187903575722\\
60.5	0.1056	7.66896344627018\\
60.5	0.1062	7.74604785678311\\
60.5	0.1068	7.82313226729606\\
60.5	0.1074	7.90021667780901\\
60.5	0.108	7.97730108832194\\
60.5	0.1086	8.05438549883489\\
60.5	0.1092	8.13146990934783\\
60.5	0.1098	8.20855431986078\\
60.5	0.1104	8.28563873037372\\
60.5	0.111	8.36272314088667\\
60.5	0.1116	8.43980755139961\\
60.5	0.1122	8.51689196191255\\
60.5	0.1128	8.5939763724255\\
60.5	0.1134	8.67106078293844\\
60.5	0.114	8.74814519345139\\
60.5	0.1146	8.82522960396433\\
60.5	0.1152	8.90231401447728\\
60.5	0.1158	8.97939842499021\\
60.5	0.1164	9.05648283550316\\
60.5	0.117	9.1335672460161\\
60.5	0.1176	9.21065165652907\\
60.5	0.1182	9.28773606704199\\
60.5	0.1188	9.36482047755494\\
60.5	0.1194	9.44190488806788\\
60.5	0.12	9.51898929858082\\
60.5	0.1206	9.59607370909377\\
60.5	0.1212	9.67315811960673\\
60.5	0.1218	9.75024253011966\\
60.5	0.1224	9.8273269406326\\
60.5	0.123	9.90441135114555\\
60.875	0.093	6.08138247338925\\
60.875	0.0936	6.16093397155306\\
60.875	0.0942	6.24048546971687\\
60.875	0.0948	6.32003696788068\\
60.875	0.0954	6.39958846604448\\
60.875	0.096	6.47913996420829\\
60.875	0.0966	6.5586914623721\\
60.875	0.0972	6.6382429605359\\
60.875	0.0978	6.71779445869971\\
60.875	0.0984	6.79734595686351\\
60.875	0.099	6.87689745502732\\
60.875	0.0996	6.95644895319113\\
60.875	0.1002	7.03600045135494\\
60.875	0.1008	7.11555194951875\\
60.875	0.1014	7.19510344768256\\
60.875	0.102	7.27465494584635\\
60.875	0.1026	7.35420644401016\\
60.875	0.1032	7.43375794217398\\
60.875	0.1038	7.51330944033779\\
60.875	0.1044	7.59286093850159\\
60.875	0.105	7.67241243666539\\
60.875	0.1056	7.7519639348292\\
60.875	0.1062	7.83151543299301\\
60.875	0.1068	7.91106693115681\\
60.875	0.1074	7.99061842932063\\
60.875	0.108	8.07016992748443\\
60.875	0.1086	8.14972142564824\\
60.875	0.1092	8.22927292381205\\
60.875	0.1098	8.30882442197584\\
60.875	0.1104	8.38837592013965\\
60.875	0.111	8.46792741830347\\
60.875	0.1116	8.54747891646727\\
60.875	0.1122	8.62703041463108\\
60.875	0.1128	8.70658191279489\\
60.875	0.1134	8.7861334109587\\
60.875	0.114	8.86568490912249\\
60.875	0.1146	8.9452364072863\\
60.875	0.1152	9.02478790545011\\
60.875	0.1158	9.10433940361393\\
60.875	0.1164	9.18389090177773\\
60.875	0.117	9.26344239994154\\
60.875	0.1176	9.34299389810533\\
60.875	0.1182	9.42254539626916\\
60.875	0.1188	9.50209689443297\\
60.875	0.1194	9.58164839259676\\
60.875	0.12	9.66119989076057\\
60.875	0.1206	9.74075138892438\\
60.875	0.1212	9.82030288708819\\
60.875	0.1218	9.899854385252\\
60.875	0.1224	9.97940588341581\\
60.875	0.123	10.0589573815796\\
61.25	0.093	6.11257412128015\\
61.25	0.0936	6.19459270709482\\
61.25	0.0942	6.27661129290949\\
61.25	0.0948	6.35862987872416\\
61.25	0.0954	6.44064846453882\\
61.25	0.096	6.5226670503535\\
61.25	0.0966	6.60468563616816\\
61.25	0.0972	6.68670422198284\\
61.25	0.0978	6.76872280779752\\
61.25	0.0984	6.85074139361218\\
61.25	0.099	6.93275997942686\\
61.25	0.0996	7.01477856524153\\
61.25	0.1002	7.09679715105619\\
61.25	0.1008	7.17881573687087\\
61.25	0.1014	7.26083432268553\\
61.25	0.102	7.3428529085002\\
61.25	0.1026	7.42487149431488\\
61.25	0.1032	7.50689008012954\\
61.25	0.1038	7.58890866594422\\
61.25	0.1044	7.67092725175889\\
61.25	0.105	7.75294583757355\\
61.25	0.1056	7.83496442338823\\
61.25	0.1062	7.91698300920289\\
61.25	0.1068	7.99900159501757\\
61.25	0.1074	8.08102018083224\\
61.25	0.108	8.16303876664689\\
61.25	0.1086	8.24505735246159\\
61.25	0.1092	8.32707593827625\\
61.25	0.1098	8.40909452409092\\
61.25	0.1104	8.49111310990558\\
61.25	0.111	8.57313169572026\\
61.25	0.1116	8.65515028153493\\
61.25	0.1122	8.73716886734961\\
61.25	0.1128	8.81918745316426\\
61.25	0.1134	8.90120603897894\\
61.25	0.114	8.98322462479361\\
61.25	0.1146	9.06524321060827\\
61.25	0.1152	9.14726179642295\\
61.25	0.1158	9.22928038223762\\
61.25	0.1164	9.3112989680523\\
61.25	0.117	9.39331755386696\\
61.25	0.1176	9.47533613968163\\
61.25	0.1182	9.5573547254963\\
61.25	0.1188	9.63937331131098\\
61.25	0.1194	9.72139189712563\\
61.25	0.12	9.80341048294032\\
61.25	0.1206	9.88542906875497\\
61.25	0.1212	9.96744765456967\\
61.25	0.1218	10.0494662403843\\
61.25	0.1224	10.131484826199\\
61.25	0.123	10.2135034120137\\
61.625	0.093	6.14376576917104\\
61.625	0.0936	6.22825144263657\\
61.625	0.0942	6.31273711610211\\
61.625	0.0948	6.39722278956764\\
61.625	0.0954	6.48170846303317\\
61.625	0.096	6.56619413649871\\
61.625	0.0966	6.65067980996424\\
61.625	0.0972	6.73516548342977\\
61.625	0.0978	6.8196511568953\\
61.625	0.0984	6.90413683036083\\
61.625	0.099	6.98862250382637\\
61.625	0.0996	7.07310817729191\\
61.625	0.1002	7.15759385075744\\
61.625	0.1008	7.24207952422298\\
61.625	0.1014	7.32656519768852\\
61.625	0.102	7.41105087115404\\
61.625	0.1026	7.49553654461958\\
61.625	0.1032	7.58002221808511\\
61.625	0.1038	7.66450789155064\\
61.625	0.1044	7.74899356501617\\
61.625	0.105	7.8334792384817\\
61.625	0.1056	7.91796491194725\\
61.625	0.1062	8.00245058541279\\
61.625	0.1068	8.08693625887832\\
61.625	0.1074	8.17142193234385\\
61.625	0.108	8.25590760580938\\
61.625	0.1086	8.34039327927491\\
61.625	0.1092	8.42487895274044\\
61.625	0.1098	8.50936462620597\\
61.625	0.1104	8.59385029967152\\
61.625	0.111	8.67833597313704\\
61.625	0.1116	8.76282164660257\\
61.625	0.1122	8.84730732006811\\
61.625	0.1128	8.93179299353366\\
61.625	0.1134	9.01627866699918\\
61.625	0.114	9.10076434046472\\
61.625	0.1146	9.18525001393026\\
61.625	0.1152	9.26973568739578\\
61.625	0.1158	9.35422136086132\\
61.625	0.1164	9.43870703432684\\
61.625	0.117	9.52319270779238\\
61.625	0.1176	9.60767838125791\\
61.625	0.1182	9.69216405472345\\
61.625	0.1188	9.77664972818897\\
61.625	0.1194	9.86113540165452\\
61.625	0.12	9.94562107512004\\
61.625	0.1206	10.0301067485856\\
61.625	0.1212	10.1145924220511\\
61.625	0.1218	10.1990780955167\\
61.625	0.1224	10.2835637689822\\
61.625	0.123	10.3680494424477\\
62	0.093	6.17495741706195\\
62	0.0936	6.26191017817835\\
62	0.0942	6.34886293929474\\
62	0.0948	6.43581570041114\\
62	0.0954	6.52276846152754\\
62	0.096	6.60972122264393\\
62	0.0966	6.69667398376033\\
62	0.0972	6.78362674487672\\
62	0.0978	6.87057950599313\\
62	0.0984	6.95753226710951\\
62	0.099	7.04448502822592\\
62	0.0996	7.13143778934232\\
62	0.1002	7.21839055045871\\
62	0.1008	7.30534331157511\\
62	0.1014	7.39229607269151\\
62	0.102	7.4792488338079\\
62	0.1026	7.56620159492429\\
62	0.1032	7.65315435604069\\
62	0.1038	7.74010711715709\\
62	0.1044	7.82705987827349\\
62	0.105	7.91401263938989\\
62	0.1056	8.00096540050628\\
62	0.1062	8.08791816162267\\
62	0.1068	8.17487092273907\\
62	0.1074	8.26182368385547\\
62	0.108	8.34877644497186\\
62	0.1086	8.43572920608827\\
62	0.1092	8.52268196720468\\
62	0.1098	8.60963472832105\\
62	0.1104	8.69658748943745\\
62	0.111	8.78354025055386\\
62	0.1116	8.87049301167023\\
62	0.1122	8.95744577278664\\
62	0.1128	9.04439853390305\\
62	0.1134	9.13135129501944\\
62	0.114	9.21830405613584\\
62	0.1146	9.30525681725223\\
62	0.1152	9.39220957836862\\
62	0.1158	9.47916233948503\\
62	0.1164	9.56611510060142\\
62	0.117	9.65306786171783\\
62	0.1176	9.74002062283421\\
62	0.1182	9.82697338395062\\
62	0.1188	9.91392614506701\\
62	0.1194	10.0008789061834\\
62	0.12	10.0878316672998\\
62	0.1206	10.1747844284162\\
62	0.1212	10.2617371895326\\
62	0.1218	10.348689950649\\
62	0.1224	10.4356427117654\\
62	0.123	10.5225954728818\\
62.375	0.093	6.20614906495284\\
62.375	0.0936	6.2955689137201\\
62.375	0.0942	6.38498876248736\\
62.375	0.0948	6.47440861125462\\
62.375	0.0954	6.56382846002187\\
62.375	0.096	6.65324830878914\\
62.375	0.0966	6.74266815755639\\
62.375	0.0972	6.83208800632365\\
62.375	0.0978	6.92150785509092\\
62.375	0.0984	7.01092770385817\\
62.375	0.099	7.10034755262544\\
62.375	0.0996	7.18976740139271\\
62.375	0.1002	7.27918725015996\\
62.375	0.1008	7.36860709892721\\
62.375	0.1014	7.45802694769448\\
62.375	0.102	7.54744679646174\\
62.375	0.1026	7.63686664522901\\
62.375	0.1032	7.72628649399626\\
62.375	0.1038	7.81570634276353\\
62.375	0.1044	7.90512619153078\\
62.375	0.105	7.99454604029804\\
62.375	0.1056	8.0839658890653\\
62.375	0.1062	8.17338573783258\\
62.375	0.1068	8.26280558659982\\
62.375	0.1074	8.35222543536709\\
62.375	0.108	8.44164528413434\\
62.375	0.1086	8.5310651329016\\
62.375	0.1092	8.62048498166885\\
62.375	0.1098	8.70990483043612\\
62.375	0.1104	8.79932467920338\\
62.375	0.111	8.88874452797064\\
62.375	0.1116	8.97816437673791\\
62.375	0.1122	9.06758422550516\\
62.375	0.1128	9.15700407427242\\
62.375	0.1134	9.24642392303967\\
62.375	0.114	9.33584377180695\\
62.375	0.1146	9.4252636205742\\
62.375	0.1152	9.51468346934146\\
62.375	0.1158	9.60410331810871\\
62.375	0.1164	9.69352316687598\\
62.375	0.117	9.78294301564323\\
62.375	0.1176	9.8723628644105\\
62.375	0.1182	9.96178271317775\\
62.375	0.1188	10.051202561945\\
62.375	0.1194	10.1406224107123\\
62.375	0.12	10.2300422594795\\
62.375	0.1206	10.3194621082468\\
62.375	0.1212	10.4088819570141\\
62.375	0.1218	10.4983018057813\\
62.375	0.1224	10.5877216545486\\
62.375	0.123	10.6771415033158\\
62.75	0.093	6.23734071284373\\
62.75	0.0936	6.32922764926186\\
62.75	0.0942	6.42111458567998\\
62.75	0.0948	6.51300152209811\\
62.75	0.0954	6.60488845851622\\
62.75	0.096	6.69677539493436\\
62.75	0.0966	6.78866233135247\\
62.75	0.0972	6.8805492677706\\
62.75	0.0978	6.97243620418872\\
62.75	0.0984	7.06432314060683\\
62.75	0.099	7.15621007702497\\
62.75	0.0996	7.24809701344309\\
62.75	0.1002	7.33998394986121\\
62.75	0.1008	7.43187088627933\\
62.75	0.1014	7.52375782269746\\
62.75	0.102	7.61564475911558\\
62.75	0.1026	7.7075316955337\\
62.75	0.1032	7.79941863195182\\
62.75	0.1038	7.89130556836995\\
62.75	0.1044	7.98319250478808\\
62.75	0.105	8.07507944120619\\
62.75	0.1056	8.16696637762432\\
62.75	0.1062	8.25885331404244\\
62.75	0.1068	8.35074025046056\\
62.75	0.1074	8.44262718687868\\
62.75	0.108	8.5345141232968\\
62.75	0.1086	8.62640105971494\\
62.75	0.1092	8.71828799613306\\
62.75	0.1098	8.81017493255118\\
62.75	0.1104	8.9020618689693\\
62.75	0.111	8.99394880538743\\
62.75	0.1116	9.08583574180554\\
62.75	0.1122	9.17772267822367\\
62.75	0.1128	9.26960961464179\\
62.75	0.1134	9.36149655105993\\
62.75	0.114	9.45338348747804\\
62.75	0.1146	9.54527042389616\\
62.75	0.1152	9.63715736031428\\
62.75	0.1158	9.72904429673243\\
62.75	0.1164	9.82093123315053\\
62.75	0.117	9.91281816956867\\
62.75	0.1176	10.0047051059868\\
62.75	0.1182	10.0965920424049\\
62.75	0.1188	10.188478978823\\
62.75	0.1194	10.2803659152411\\
62.75	0.12	10.3722528516593\\
62.75	0.1206	10.4641397880774\\
62.75	0.1212	10.5560267244955\\
62.75	0.1218	10.6479136609136\\
62.75	0.1224	10.7398005973318\\
62.75	0.123	10.8316875337499\\
63.125	0.093	6.26853236073465\\
63.125	0.0936	6.36288638480363\\
63.125	0.0942	6.45724040887262\\
63.125	0.0948	6.55159443294161\\
63.125	0.0954	6.64594845701059\\
63.125	0.096	6.74030248107958\\
63.125	0.0966	6.83465650514855\\
63.125	0.0972	6.92901052921755\\
63.125	0.0978	7.02336455328653\\
63.125	0.0984	7.11771857735551\\
63.125	0.099	7.2120726014245\\
63.125	0.0996	7.30642662549349\\
63.125	0.1002	7.40078064956247\\
63.125	0.1008	7.49513467363146\\
63.125	0.1014	7.58948869770045\\
63.125	0.102	7.68384272176943\\
63.125	0.1026	7.77819674583842\\
63.125	0.1032	7.87255076990741\\
63.125	0.1038	7.9669047939764\\
63.125	0.1044	8.06125881804539\\
63.125	0.105	8.15561284211437\\
63.125	0.1056	8.24996686618336\\
63.125	0.1062	8.34432089025235\\
63.125	0.1068	8.43867491432133\\
63.125	0.1074	8.53302893839032\\
63.125	0.108	8.62738296245929\\
63.125	0.1086	8.7217369865283\\
63.125	0.1092	8.81609101059728\\
63.125	0.1098	8.91044503466627\\
63.125	0.1104	9.00479905873524\\
63.125	0.111	9.09915308280424\\
63.125	0.1116	9.19350710687321\\
63.125	0.1122	9.2878611309422\\
63.125	0.1128	9.38221515501118\\
63.125	0.1134	9.47656917908017\\
63.125	0.114	9.57092320314916\\
63.125	0.1146	9.66527722721814\\
63.125	0.1152	9.75963125128713\\
63.125	0.1158	9.85398527535612\\
63.125	0.1164	9.94833929942511\\
63.125	0.117	10.0426933234941\\
63.125	0.1176	10.1370473475631\\
63.125	0.1182	10.2314013716321\\
63.125	0.1188	10.3257553957011\\
63.125	0.1194	10.42010941977\\
63.125	0.12	10.514463443839\\
63.125	0.1206	10.608817467908\\
63.125	0.1212	10.703171491977\\
63.125	0.1218	10.797525516046\\
63.125	0.1224	10.891879540115\\
63.125	0.123	10.986233564184\\
63.5	0.093	6.29972400862554\\
63.5	0.0936	6.39654512034538\\
63.5	0.0942	6.49336623206523\\
63.5	0.0948	6.59018734378508\\
63.5	0.0954	6.68700845550492\\
63.5	0.096	6.78382956722479\\
63.5	0.0966	6.88065067894463\\
63.5	0.0972	6.97747179066448\\
63.5	0.0978	7.07429290238433\\
63.5	0.0984	7.17111401410417\\
63.5	0.099	7.26793512582403\\
63.5	0.0996	7.36475623754387\\
63.5	0.1002	7.46157734926373\\
63.5	0.1008	7.55839846098358\\
63.5	0.1014	7.65521957270342\\
63.5	0.102	7.75204068442327\\
63.5	0.1026	7.84886179614312\\
63.5	0.1032	7.94568290786297\\
63.5	0.1038	8.04250401958282\\
63.5	0.1044	8.13932513130267\\
63.5	0.105	8.23614624302252\\
63.5	0.1056	8.33296735474238\\
63.5	0.1062	8.42978846646223\\
63.5	0.1068	8.52660957818208\\
63.5	0.1074	8.62343068990192\\
63.5	0.108	8.72025180162177\\
63.5	0.1086	8.81707291334162\\
63.5	0.1092	8.91389402506147\\
63.5	0.1098	9.01071513678131\\
63.5	0.1104	9.10753624850118\\
63.5	0.111	9.20435736022101\\
63.5	0.1116	9.30117847194087\\
63.5	0.1122	9.39799958366071\\
63.5	0.1128	9.49482069538058\\
63.5	0.1134	9.59164180710042\\
63.5	0.114	9.68846291882026\\
63.5	0.1146	9.78528403054013\\
63.5	0.1152	9.88210514225996\\
63.5	0.1158	9.97892625397982\\
63.5	0.1164	10.0757473656997\\
63.5	0.117	10.1725684774195\\
63.5	0.1176	10.2693895891394\\
63.5	0.1182	10.3662107008592\\
63.5	0.1188	10.4630318125791\\
63.5	0.1194	10.5598529242989\\
63.5	0.12	10.6566740360188\\
63.5	0.1206	10.7534951477386\\
63.5	0.1212	10.8503162594585\\
63.5	0.1218	10.9471373711783\\
63.5	0.1224	11.0439584828981\\
63.5	0.123	11.140779594618\\
63.875	0.093	6.33091565651644\\
63.875	0.0936	6.43020385588714\\
63.875	0.0942	6.52949205525785\\
63.875	0.0948	6.62878025462857\\
63.875	0.0954	6.72806845399928\\
63.875	0.096	6.82735665337\\
63.875	0.0966	6.92664485274071\\
63.875	0.0972	7.02593305211142\\
63.875	0.0978	7.12522125148214\\
63.875	0.0984	7.22450945085285\\
63.875	0.099	7.32379765022355\\
63.875	0.0996	7.42308584959427\\
63.875	0.1002	7.52237404896498\\
63.875	0.1008	7.6216622483357\\
63.875	0.1014	7.72095044770641\\
63.875	0.102	7.82023864707712\\
63.875	0.1026	7.91952684644784\\
63.875	0.1032	8.01881504581856\\
63.875	0.1038	8.11810324518925\\
63.875	0.1044	8.21739144455998\\
63.875	0.105	8.31667964393067\\
63.875	0.1056	8.4159678433014\\
63.875	0.1062	8.51525604267211\\
63.875	0.1068	8.61454424204283\\
63.875	0.1074	8.71383244141353\\
63.875	0.108	8.81312064078425\\
63.875	0.1086	8.91240884015497\\
63.875	0.1092	9.01169703952569\\
63.875	0.1098	9.11098523889638\\
63.875	0.1104	9.21027343826709\\
63.875	0.111	9.30956163763781\\
63.875	0.1116	9.40884983700852\\
63.875	0.1122	9.50813803637924\\
63.875	0.1128	9.60742623574995\\
63.875	0.1134	9.70671443512067\\
63.875	0.114	9.80600263449136\\
63.875	0.1146	9.90529083386208\\
63.875	0.1152	10.0045790332328\\
63.875	0.1158	10.1038672326035\\
63.875	0.1164	10.2031554319742\\
63.875	0.117	10.3024436313449\\
63.875	0.1176	10.4017318307156\\
63.875	0.1182	10.5010200300864\\
63.875	0.1188	10.6003082294571\\
63.875	0.1194	10.6995964288278\\
63.875	0.12	10.7988846281985\\
63.875	0.1206	10.8981728275692\\
63.875	0.1212	10.9974610269399\\
63.875	0.1218	11.0967492263106\\
63.875	0.1224	11.1960374256814\\
63.875	0.123	11.2953256250521\\
64.25	0.093	6.36210730440733\\
64.25	0.0936	6.46386259142891\\
64.25	0.0942	6.56561787845049\\
64.25	0.0948	6.66737316547206\\
64.25	0.0954	6.76912845249364\\
64.25	0.096	6.87088373951521\\
64.25	0.0966	6.97263902653678\\
64.25	0.0972	7.07439431355836\\
64.25	0.0978	7.17614960057994\\
64.25	0.0984	7.27790488760151\\
64.25	0.099	7.3796601746231\\
64.25	0.0996	7.48141546164467\\
64.25	0.1002	7.58317074866625\\
64.25	0.1008	7.68492603568782\\
64.25	0.1014	7.7866813227094\\
64.25	0.102	7.88843660973097\\
64.25	0.1026	7.99019189675255\\
64.25	0.1032	8.09194718377411\\
64.25	0.1038	8.19370247079571\\
64.25	0.1044	8.29545775781727\\
64.25	0.105	8.39721304483885\\
64.25	0.1056	8.49896833186042\\
64.25	0.1062	8.60072361888201\\
64.25	0.1068	8.70247890590358\\
64.25	0.1074	8.80423419292516\\
64.25	0.108	8.90598947994673\\
64.25	0.1086	9.0077447669683\\
64.25	0.1092	9.10950005398988\\
64.25	0.1098	9.21125534101145\\
64.25	0.1104	9.31301062803304\\
64.25	0.111	9.41476591505462\\
64.25	0.1116	9.51652120207619\\
64.25	0.1122	9.61827648909777\\
64.25	0.1128	9.72003177611934\\
64.25	0.1134	9.8217870631409\\
64.25	0.114	9.92354235016249\\
64.25	0.1146	10.0252976371841\\
64.25	0.1152	10.1270529242056\\
64.25	0.1158	10.2288082112272\\
64.25	0.1164	10.3305634982488\\
64.25	0.117	10.4323187852704\\
64.25	0.1176	10.534074072292\\
64.25	0.1182	10.6358293593135\\
64.25	0.1188	10.7375846463351\\
64.25	0.1194	10.8393399333567\\
64.25	0.12	10.9410952203782\\
64.25	0.1206	11.0428505073998\\
64.25	0.1212	11.1446057944214\\
64.25	0.1218	11.246361081443\\
64.25	0.1224	11.3481163684646\\
64.25	0.123	11.4498716554861\\
64.625	0.093	6.39329895229823\\
64.625	0.0936	6.49752132697067\\
64.625	0.0942	6.60174370164312\\
64.625	0.0948	6.70596607631555\\
64.625	0.0954	6.81018845098799\\
64.625	0.096	6.91441082566043\\
64.625	0.0966	7.01863320033286\\
64.625	0.0972	7.1228555750053\\
64.625	0.0978	7.22707794967775\\
64.625	0.0984	7.33130032435019\\
64.625	0.099	7.43552269902262\\
64.625	0.0996	7.53974507369506\\
64.625	0.1002	7.6439674483675\\
64.625	0.1008	7.74818982303994\\
64.625	0.1014	7.85241219771239\\
64.625	0.102	7.95663457238481\\
64.625	0.1026	8.06085694705726\\
64.625	0.1032	8.16507932172969\\
64.625	0.1038	8.26930169640214\\
64.625	0.1044	8.37352407107457\\
64.625	0.105	8.47774644574702\\
64.625	0.1056	8.58196882041945\\
64.625	0.1062	8.68619119509189\\
64.625	0.1068	8.79041356976434\\
64.625	0.1074	8.89463594443677\\
64.625	0.108	8.99885831910922\\
64.625	0.1086	9.10308069378164\\
64.625	0.1092	9.20730306845408\\
64.625	0.1098	9.31152544312651\\
64.625	0.1104	9.41574781779897\\
64.625	0.111	9.5199701924714\\
64.625	0.1116	9.62419256714384\\
64.625	0.1122	9.72841494181627\\
64.625	0.1128	9.83263731648873\\
64.625	0.1134	9.93685969116116\\
64.625	0.114	10.0410820658336\\
64.625	0.1146	10.145304440506\\
64.625	0.1152	10.2495268151785\\
64.625	0.1158	10.3537491898509\\
64.625	0.1164	10.4579715645233\\
64.625	0.117	10.5621939391958\\
64.625	0.1176	10.6664163138682\\
64.625	0.1182	10.7706386885407\\
64.625	0.1188	10.8748610632131\\
64.625	0.1194	10.9790834378855\\
64.625	0.12	11.083305812558\\
64.625	0.1206	11.1875281872304\\
64.625	0.1212	11.2917505619029\\
64.625	0.1218	11.3959729365753\\
64.625	0.1224	11.5001953112477\\
64.625	0.123	11.6044176859202\\
65	0.093	6.42449060018912\\
65	0.0936	6.53118006251242\\
65	0.0942	6.63786952483574\\
65	0.0948	6.74455898715903\\
65	0.0954	6.85124844948233\\
65	0.096	6.95793791180564\\
65	0.0966	7.06462737412894\\
65	0.0972	7.17131683645223\\
65	0.0978	7.27800629877554\\
65	0.0984	7.38469576109885\\
65	0.099	7.49138522342215\\
65	0.0996	7.59807468574544\\
65	0.1002	7.70476414806875\\
65	0.1008	7.81145361039206\\
65	0.1014	7.91814307271535\\
65	0.102	8.02483253503866\\
65	0.1026	8.13152199736196\\
65	0.1032	8.23821145968526\\
65	0.1038	8.34490092200856\\
65	0.1044	8.45159038433188\\
65	0.105	8.55827984665517\\
65	0.1056	8.66496930897847\\
65	0.1062	8.77165877130177\\
65	0.1068	8.87834823362506\\
65	0.1074	8.98503769594838\\
65	0.108	9.09172715827167\\
65	0.1086	9.19841662059498\\
65	0.1092	9.30510608291829\\
65	0.1098	9.41179554524159\\
65	0.1104	9.51848500756488\\
65	0.111	9.62517446988819\\
65	0.1116	9.7318639322115\\
65	0.1122	9.8385533945348\\
65	0.1128	9.94524285685809\\
65	0.1134	10.0519323191814\\
65	0.114	10.1586217815047\\
65	0.1146	10.265311243828\\
65	0.1152	10.3720007061513\\
65	0.1158	10.4786901684746\\
65	0.1164	10.5853796307979\\
65	0.117	10.6920690931212\\
65	0.1176	10.7987585554445\\
65	0.1182	10.9054480177678\\
65	0.1188	11.0121374800911\\
65	0.1194	11.1188269424144\\
65	0.12	11.2255164047377\\
65	0.1206	11.332205867061\\
65	0.1212	11.4388953293843\\
65	0.1218	11.5455847917076\\
65	0.1224	11.6522742540309\\
65	0.123	11.7589637163542\\
65.375	0.093	6.45568224808004\\
65.375	0.0936	6.5648387980542\\
65.375	0.0942	6.67399534802836\\
65.375	0.0948	6.78315189800253\\
65.375	0.0954	6.8923084479767\\
65.375	0.096	7.00146499795087\\
65.375	0.0966	7.11062154792502\\
65.375	0.0972	7.21977809789919\\
65.375	0.0978	7.32893464787336\\
65.375	0.0984	7.43809119784753\\
65.375	0.099	7.54724774782169\\
65.375	0.0996	7.65640429779585\\
65.375	0.1002	7.76556084777002\\
65.375	0.1008	7.87471739774419\\
65.375	0.1014	7.98387394771835\\
65.375	0.102	8.09303049769251\\
65.375	0.1026	8.20218704766668\\
65.375	0.1032	8.31134359764084\\
65.375	0.1038	8.42050014761502\\
65.375	0.1044	8.52965669758917\\
65.375	0.105	8.63881324756335\\
65.375	0.1056	8.74796979753751\\
65.375	0.1062	8.85712634751168\\
65.375	0.1068	8.96628289748584\\
65.375	0.1074	9.07543944746001\\
65.375	0.108	9.18459599743417\\
65.375	0.1086	9.29375254740833\\
65.375	0.1092	9.40290909738249\\
65.375	0.1098	9.51206564735666\\
65.375	0.1104	9.62122219733084\\
65.375	0.111	9.730378747305\\
65.375	0.1116	9.83953529727917\\
65.375	0.1122	9.94869184725333\\
65.375	0.1128	10.0578483972275\\
65.375	0.1134	10.1670049472016\\
65.375	0.114	10.2761614971758\\
65.375	0.1146	10.38531804715\\
65.375	0.1152	10.4944745971242\\
65.375	0.1158	10.6036311470983\\
65.375	0.1164	10.7127876970725\\
65.375	0.117	10.8219442470466\\
65.375	0.1176	10.9311007970208\\
65.375	0.1182	11.040257346995\\
65.375	0.1188	11.1494138969692\\
65.375	0.1194	11.2585704469433\\
65.375	0.12	11.3677269969175\\
65.375	0.1206	11.4768835468916\\
65.375	0.1212	11.5860400968658\\
65.375	0.1218	11.69519664684\\
65.375	0.1224	11.8043531968141\\
65.375	0.123	11.9135097467883\\
65.75	0.093	6.48687389597093\\
65.75	0.0936	6.59849753359595\\
65.75	0.0942	6.71012117122098\\
65.75	0.0948	6.82174480884601\\
65.75	0.0954	6.93336844647104\\
65.75	0.096	7.04499208409607\\
65.75	0.0966	7.1566157217211\\
65.75	0.0972	7.26823935934612\\
65.75	0.0978	7.37986299697115\\
65.75	0.0984	7.49148663459619\\
65.75	0.099	7.60311027222122\\
65.75	0.0996	7.71473390984623\\
65.75	0.1002	7.82635754747126\\
65.75	0.1008	7.93798118509629\\
65.75	0.1014	8.04960482272134\\
65.75	0.102	8.16122846034635\\
65.75	0.1026	8.27285209797138\\
65.75	0.1032	8.38447573559641\\
65.75	0.1038	8.49609937322143\\
65.75	0.1044	8.60772301084647\\
65.75	0.105	8.71934664847149\\
65.75	0.1056	8.83097028609653\\
65.75	0.1062	8.94259392372155\\
65.75	0.1068	9.05421756134658\\
65.75	0.1074	9.16584119897161\\
65.75	0.108	9.27746483659664\\
65.75	0.1086	9.38908847422167\\
65.75	0.1092	9.50071211184671\\
65.75	0.1098	9.61233574947173\\
65.75	0.1104	9.72395938709676\\
65.75	0.111	9.83558302472179\\
65.75	0.1116	9.9472066623468\\
65.75	0.1122	10.0588302999718\\
65.75	0.1128	10.1704539375969\\
65.75	0.1134	10.2820775752219\\
65.75	0.114	10.3937012128469\\
65.75	0.1146	10.505324850472\\
65.75	0.1152	10.616948488097\\
65.75	0.1158	10.728572125722\\
65.75	0.1164	10.840195763347\\
65.75	0.117	10.9518194009721\\
65.75	0.1176	11.0634430385971\\
65.75	0.1182	11.1750666762221\\
65.75	0.1188	11.2866903138472\\
65.75	0.1194	11.3983139514722\\
65.75	0.12	11.5099375890972\\
65.75	0.1206	11.6215612267222\\
65.75	0.1212	11.7331848643473\\
65.75	0.1218	11.8448085019723\\
65.75	0.1224	11.9564321395973\\
65.75	0.123	12.0680557772224\\
66.125	0.093	6.51806554386183\\
66.125	0.0936	6.63215626913771\\
66.125	0.0942	6.74624699441361\\
66.125	0.0948	6.8603377196895\\
66.125	0.0954	6.97442844496538\\
66.125	0.096	7.08851917024128\\
66.125	0.0966	7.20260989551718\\
66.125	0.0972	7.31670062079306\\
66.125	0.0978	7.43079134606896\\
66.125	0.0984	7.54488207134485\\
66.125	0.099	7.65897279662073\\
66.125	0.0996	7.77306352189663\\
66.125	0.1002	7.88715424717252\\
66.125	0.1008	8.00124497244842\\
66.125	0.1014	8.1153356977243\\
66.125	0.102	8.22942642300021\\
66.125	0.1026	8.34351714827609\\
66.125	0.1032	8.45760787355198\\
66.125	0.1038	8.57169859882788\\
66.125	0.1044	8.68578932410378\\
66.125	0.105	8.79988004937965\\
66.125	0.1056	8.91397077465555\\
66.125	0.1062	9.02806149993144\\
66.125	0.1068	9.14215222520733\\
66.125	0.1074	9.25624295048323\\
66.125	0.108	9.37033367575911\\
66.125	0.1086	9.48442440103501\\
66.125	0.1092	9.5985151263109\\
66.125	0.1098	9.7126058515868\\
66.125	0.1104	9.82669657686267\\
66.125	0.111	9.94078730213859\\
66.125	0.1116	10.0548780274145\\
66.125	0.1122	10.1689687526904\\
66.125	0.1128	10.2830594779662\\
66.125	0.1134	10.3971502032421\\
66.125	0.114	10.511240928518\\
66.125	0.1146	10.6253316537939\\
66.125	0.1152	10.7394223790698\\
66.125	0.1158	10.8535131043457\\
66.125	0.1164	10.9676038296216\\
66.125	0.117	11.0816945548975\\
66.125	0.1176	11.1957852801734\\
66.125	0.1182	11.3098760054493\\
66.125	0.1188	11.4239667307252\\
66.125	0.1194	11.5380574560011\\
66.125	0.12	11.652148181277\\
66.125	0.1206	11.7662389065528\\
66.125	0.1212	11.8803296318287\\
66.125	0.1218	11.9944203571046\\
66.125	0.1224	12.1085110823805\\
66.125	0.123	12.2226018076564\\
66.5	0.093	6.54925719175272\\
66.5	0.0936	6.66581500467947\\
66.5	0.0942	6.78237281760623\\
66.5	0.0948	6.89893063053299\\
66.5	0.0954	7.01548844345974\\
66.5	0.096	7.1320462563865\\
66.5	0.0966	7.24860406931325\\
66.5	0.0972	7.36516188224\\
66.5	0.0978	7.48171969516677\\
66.5	0.0984	7.59827750809352\\
66.5	0.099	7.71483532102029\\
66.5	0.0996	7.83139313394703\\
66.5	0.1002	7.94795094687379\\
66.5	0.1008	8.06450875980053\\
66.5	0.1014	8.1810665727273\\
66.5	0.102	8.29762438565405\\
66.5	0.1026	8.41418219858082\\
66.5	0.1032	8.53074001150756\\
66.5	0.1038	8.64729782443432\\
66.5	0.1044	8.76385563736106\\
66.5	0.105	8.88041345028782\\
66.5	0.1056	8.99697126321458\\
66.5	0.1062	9.11352907614133\\
66.5	0.1068	9.2300868890681\\
66.5	0.1074	9.34664470199485\\
66.5	0.108	9.4632025149216\\
66.5	0.1086	9.57976032784835\\
66.5	0.1092	9.69631814077511\\
66.5	0.1098	9.81287595370185\\
66.5	0.1104	9.92943376662863\\
66.5	0.111	10.0459915795554\\
66.5	0.1116	10.1625493924821\\
66.5	0.1122	10.2791072054089\\
66.5	0.1128	10.3956650183356\\
66.5	0.1134	10.5122228312624\\
66.5	0.114	10.6287806441891\\
66.5	0.1146	10.7453384571159\\
66.5	0.1152	10.8618962700427\\
66.5	0.1158	10.9784540829694\\
66.5	0.1164	11.0950118958962\\
66.5	0.117	11.2115697088229\\
66.5	0.1176	11.3281275217497\\
66.5	0.1182	11.4446853346764\\
66.5	0.1188	11.5612431476032\\
66.5	0.1194	11.67780096053\\
66.5	0.12	11.7943587734567\\
66.5	0.1206	11.9109165863835\\
66.5	0.1212	12.0274743993102\\
66.5	0.1218	12.144032212237\\
66.5	0.1224	12.2605900251637\\
66.5	0.123	12.3771478380905\\
66.875	0.093	6.58044883964362\\
66.875	0.0936	6.69947374022124\\
66.875	0.0942	6.81849864079886\\
66.875	0.0948	6.93752354137648\\
66.875	0.0954	7.0565484419541\\
66.875	0.096	7.17557334253171\\
66.875	0.0966	7.29459824310933\\
66.875	0.0972	7.41362314368696\\
66.875	0.0978	7.53264804426458\\
66.875	0.0984	7.65167294484219\\
66.875	0.099	7.7706978454198\\
66.875	0.0996	7.88972274599742\\
66.875	0.1002	8.00874764657505\\
66.875	0.1008	8.12777254715266\\
66.875	0.1014	8.24679744773029\\
66.875	0.102	8.36582234830789\\
66.875	0.1026	8.48484724888552\\
66.875	0.1032	8.60387214946314\\
66.875	0.1038	8.72289705004074\\
66.875	0.1044	8.84192195061837\\
66.875	0.105	8.96094685119597\\
66.875	0.1056	9.07997175177361\\
66.875	0.1062	9.19899665235123\\
66.875	0.1068	9.31802155292884\\
66.875	0.1074	9.43704645350645\\
66.875	0.108	9.55607135408407\\
66.875	0.1086	9.6750962546617\\
66.875	0.1092	9.79412115523932\\
66.875	0.1098	9.91314605581692\\
66.875	0.1104	10.0321709563946\\
66.875	0.111	10.1511958569722\\
66.875	0.1116	10.2702207575498\\
66.875	0.1122	10.3892456581274\\
66.875	0.1128	10.508270558705\\
66.875	0.1134	10.6272954592826\\
66.875	0.114	10.7463203598602\\
66.875	0.1146	10.8653452604379\\
66.875	0.1152	10.9843701610155\\
66.875	0.1158	11.1033950615931\\
66.875	0.1164	11.2224199621707\\
66.875	0.117	11.3414448627484\\
66.875	0.1176	11.460469763326\\
66.875	0.1182	11.5794946639036\\
66.875	0.1188	11.6985195644812\\
66.875	0.1194	11.8175444650588\\
66.875	0.12	11.9365693656364\\
66.875	0.1206	12.0555942662141\\
66.875	0.1212	12.1746191667917\\
66.875	0.1218	12.2936440673693\\
66.875	0.1224	12.4126689679469\\
66.875	0.123	12.5316938685245\\
67.25	0.093	6.61164048753452\\
67.25	0.0936	6.73313247576299\\
67.25	0.0942	6.85462446399148\\
67.25	0.0948	6.97611645221996\\
67.25	0.0954	7.09760844044843\\
67.25	0.096	7.21910042867692\\
67.25	0.0966	7.34059241690539\\
67.25	0.0972	7.46208440513389\\
67.25	0.0978	7.58357639336236\\
67.25	0.0984	7.70506838159085\\
67.25	0.099	7.82656036981933\\
67.25	0.0996	7.94805235804782\\
67.25	0.1002	8.06954434627629\\
67.25	0.1008	8.19103633450477\\
67.25	0.1014	8.31252832273326\\
67.25	0.102	8.43402031096173\\
67.25	0.1026	8.55551229919021\\
67.25	0.1032	8.67700428741868\\
67.25	0.1038	8.79849627564718\\
67.25	0.1044	8.91998826387567\\
67.25	0.105	9.04148025210415\\
67.25	0.1056	9.16297224033262\\
67.25	0.1062	9.28446422856112\\
67.25	0.1068	9.40595621678959\\
67.25	0.1074	9.52744820501808\\
67.25	0.108	9.64894019324655\\
67.25	0.1086	9.77043218147503\\
67.25	0.1092	9.8919241697035\\
67.25	0.1098	10.013416157932\\
67.25	0.1104	10.1349081461605\\
67.25	0.111	10.256400134389\\
67.25	0.1116	10.3778921226174\\
67.25	0.1122	10.4993841108459\\
67.25	0.1128	10.6208760990744\\
67.25	0.1134	10.7423680873029\\
67.25	0.114	10.8638600755314\\
67.25	0.1146	10.9853520637598\\
67.25	0.1152	11.1068440519883\\
67.25	0.1158	11.2283360402168\\
67.25	0.1164	11.3498280284453\\
67.25	0.117	11.4713200166738\\
67.25	0.1176	11.5928120049023\\
67.25	0.1182	11.7143039931307\\
67.25	0.1188	11.8357959813592\\
67.25	0.1194	11.9572879695877\\
67.25	0.12	12.0787799578162\\
67.25	0.1206	12.2002719460447\\
67.25	0.1212	12.3217639342732\\
67.25	0.1218	12.4432559225016\\
67.25	0.1224	12.5647479107301\\
67.25	0.123	12.6862398989586\\
67.625	0.093	6.64283213542542\\
67.625	0.0936	6.76679121130476\\
67.625	0.0942	6.89075028718411\\
67.625	0.0948	7.01470936306346\\
67.625	0.0954	7.13866843894279\\
67.625	0.096	7.26262751482214\\
67.625	0.0966	7.38658659070148\\
67.625	0.0972	7.51054566658082\\
67.625	0.0978	7.63450474246018\\
67.625	0.0984	7.75846381833952\\
67.625	0.099	7.88242289421886\\
67.625	0.0996	8.0063819700982\\
67.625	0.1002	8.13034104597754\\
67.625	0.1008	8.2543001218569\\
67.625	0.1014	8.37825919773624\\
67.625	0.102	8.50221827361558\\
67.625	0.1026	8.62617734949494\\
67.625	0.1032	8.75013642537428\\
67.625	0.1038	8.87409550125361\\
67.625	0.1044	8.99805457713296\\
67.625	0.105	9.1220136530123\\
67.625	0.1056	9.24597272889166\\
67.625	0.1062	9.369931804771\\
67.625	0.1068	9.49389088065034\\
67.625	0.1074	9.61784995652968\\
67.625	0.108	9.74180903240902\\
67.625	0.1086	9.86576810828838\\
67.625	0.1092	9.98972718416773\\
67.625	0.1098	10.1136862600471\\
67.625	0.1104	10.2376453359264\\
67.625	0.111	10.3616044118058\\
67.625	0.1116	10.4855634876851\\
67.625	0.1122	10.6095225635644\\
67.625	0.1128	10.7334816394438\\
67.625	0.1134	10.8574407153231\\
67.625	0.114	10.9813997912025\\
67.625	0.1146	11.1053588670818\\
67.625	0.1152	11.2293179429612\\
67.625	0.1158	11.3532770188405\\
67.625	0.1164	11.4772360947199\\
67.625	0.117	11.6011951705992\\
67.625	0.1176	11.7251542464785\\
67.625	0.1182	11.8491133223579\\
67.625	0.1188	11.9730723982372\\
67.625	0.1194	12.0970314741166\\
67.625	0.12	12.2209905499959\\
67.625	0.1206	12.3449496258753\\
67.625	0.1212	12.4689087017546\\
67.625	0.1218	12.592867777634\\
67.625	0.1224	12.7168268535133\\
67.625	0.123	12.8407859293927\\
68	0.093	6.67402378331632\\
68	0.0936	6.80044994684653\\
68	0.0942	6.92687611037674\\
68	0.0948	7.05330227390694\\
68	0.0954	7.17972843743715\\
68	0.096	7.30615460096736\\
68	0.0966	7.43258076449756\\
68	0.0972	7.55900692802777\\
68	0.0978	7.68543309155798\\
68	0.0984	7.81185925508819\\
68	0.099	7.93828541861839\\
68	0.0996	8.0647115821486\\
68	0.1002	8.1911377456788\\
68	0.1008	8.31756390920903\\
68	0.1014	8.44399007273923\\
68	0.102	8.57041623626944\\
68	0.1026	8.69684239979964\\
68	0.1032	8.82326856332985\\
68	0.1038	8.94969472686006\\
68	0.1044	9.07612089039027\\
68	0.105	9.20254705392047\\
68	0.1056	9.32897321745068\\
68	0.1062	9.45539938098089\\
68	0.1068	9.58182554451109\\
68	0.1074	9.7082517080413\\
68	0.108	9.8346778715715\\
68	0.1086	9.96110403510173\\
68	0.1092	10.0875301986319\\
68	0.1098	10.2139563621621\\
68	0.1104	10.3403825256923\\
68	0.111	10.4668086892226\\
68	0.1116	10.5932348527528\\
68	0.1122	10.719661016283\\
68	0.1128	10.8460871798132\\
68	0.1134	10.9725133433434\\
68	0.114	11.0989395068736\\
68	0.1146	11.2253656704038\\
68	0.1152	11.351791833934\\
68	0.1158	11.4782179974642\\
68	0.1164	11.6046441609944\\
68	0.117	11.7310703245246\\
68	0.1176	11.8574964880548\\
68	0.1182	11.983922651585\\
68	0.1188	12.1103488151153\\
68	0.1194	12.2367749786454\\
68	0.12	12.3632011421757\\
68	0.1206	12.4896273057059\\
68	0.1212	12.6160534692361\\
68	0.1218	12.7424796327663\\
68	0.1224	12.8689057962965\\
68	0.123	12.9953319598267\\
68.375	0.093	6.70521543120721\\
68.375	0.0936	6.83410868238828\\
68.375	0.0942	6.96300193356934\\
68.375	0.0948	7.09189518475042\\
68.375	0.0954	7.2207884359315\\
68.375	0.096	7.34968168711256\\
68.375	0.0966	7.47857493829363\\
68.375	0.0972	7.6074681894747\\
68.375	0.0978	7.73636144065577\\
68.375	0.0984	7.86525469183684\\
68.375	0.099	7.99414794301792\\
68.375	0.0996	8.12304119419899\\
68.375	0.1002	8.25193444538006\\
68.375	0.1008	8.38082769656113\\
68.375	0.1014	8.50972094774221\\
68.375	0.102	8.63861419892326\\
68.375	0.1026	8.76750745010435\\
68.375	0.1032	8.89640070128542\\
68.375	0.1038	9.02529395246648\\
68.375	0.1044	9.15418720364755\\
68.375	0.105	9.28308045482862\\
68.375	0.1056	9.4119737060097\\
68.375	0.1062	9.54086695719077\\
68.375	0.1068	9.66976020837184\\
68.375	0.1074	9.79865345955291\\
68.375	0.108	9.92754671073398\\
68.375	0.1086	10.056439961915\\
68.375	0.1092	10.1853332130961\\
68.375	0.1098	10.3142264642772\\
68.375	0.1104	10.4431197154583\\
68.375	0.111	10.5720129666393\\
68.375	0.1116	10.7009062178204\\
68.375	0.1122	10.8297994690015\\
68.375	0.1128	10.9586927201826\\
68.375	0.1134	11.0875859713636\\
68.375	0.114	11.2164792225447\\
68.375	0.1146	11.3453724737258\\
68.375	0.1152	11.4742657249068\\
68.375	0.1158	11.6031589760879\\
68.375	0.1164	11.732052227269\\
68.375	0.117	11.86094547845\\
68.375	0.1176	11.9898387296311\\
68.375	0.1182	12.1187319808122\\
68.375	0.1188	12.2476252319933\\
68.375	0.1194	12.3765184831743\\
68.375	0.12	12.5054117343554\\
68.375	0.1206	12.6343049855365\\
68.375	0.1212	12.7631982367175\\
68.375	0.1218	12.8920914878986\\
68.375	0.1224	13.0209847390797\\
68.375	0.123	13.1498779902608\\
68.75	0.093	6.73640707909811\\
68.75	0.0936	6.86776741793004\\
68.75	0.0942	6.99912775676198\\
68.75	0.0948	7.1304880955939\\
68.75	0.0954	7.26184843442584\\
68.75	0.096	7.39320877325778\\
68.75	0.0966	7.52456911208971\\
68.75	0.0972	7.65592945092165\\
68.75	0.0978	7.78728978975359\\
68.75	0.0984	7.91865012858553\\
68.75	0.099	8.05001046741745\\
68.75	0.0996	8.18137080624939\\
68.75	0.1002	8.31273114508132\\
68.75	0.1008	8.44409148391327\\
68.75	0.1014	8.57545182274518\\
68.75	0.102	8.70681216157713\\
68.75	0.1026	8.83817250040904\\
68.75	0.1032	8.96953283924098\\
68.75	0.1038	9.10089317807292\\
68.75	0.1044	9.23225351690486\\
68.75	0.105	9.36361385573679\\
68.75	0.1056	9.49497419456873\\
68.75	0.1062	9.62633453340067\\
68.75	0.1068	9.75769487223259\\
68.75	0.1074	9.88905521106453\\
68.75	0.108	10.0204155498965\\
68.75	0.1086	10.1517758887284\\
68.75	0.1092	10.2831362275603\\
68.75	0.1098	10.4144965663923\\
68.75	0.1104	10.5458569052242\\
68.75	0.111	10.6772172440561\\
68.75	0.1116	10.8085775828881\\
68.75	0.1122	10.93993792172\\
68.75	0.1128	11.0712982605519\\
68.75	0.1134	11.2026585993839\\
68.75	0.114	11.3340189382158\\
68.75	0.1146	11.4653792770477\\
68.75	0.1152	11.5967396158797\\
68.75	0.1158	11.7280999547116\\
68.75	0.1164	11.8594602935435\\
68.75	0.117	11.9908206323755\\
68.75	0.1176	12.1221809712074\\
68.75	0.1182	12.2535413100393\\
68.75	0.1188	12.3849016488713\\
68.75	0.1194	12.5162619877032\\
68.75	0.12	12.6476223265352\\
68.75	0.1206	12.7789826653671\\
68.75	0.1212	12.910343004199\\
68.75	0.1218	13.0417033430309\\
68.75	0.1224	13.1730636818629\\
68.75	0.123	13.3044240206948\\
69.125	0.093	6.76759872698901\\
69.125	0.0936	6.90142615347182\\
69.125	0.0942	7.03525357995461\\
69.125	0.0948	7.16908100643741\\
69.125	0.0954	7.3029084329202\\
69.125	0.096	7.436735859403\\
69.125	0.0966	7.5705632858858\\
69.125	0.0972	7.70439071236859\\
69.125	0.0978	7.83821813885139\\
69.125	0.0984	7.9720455653342\\
69.125	0.099	8.10587299181698\\
69.125	0.0996	8.23970041829979\\
69.125	0.1002	8.37352784478257\\
69.125	0.1008	8.50735527126538\\
69.125	0.1014	8.64118269774818\\
69.125	0.102	8.77501012423097\\
69.125	0.1026	8.90883755071377\\
69.125	0.1032	9.04266497719657\\
69.125	0.1038	9.17649240367936\\
69.125	0.1044	9.31031983016217\\
69.125	0.105	9.44414725664495\\
69.125	0.1056	9.57797468312776\\
69.125	0.1062	9.71180210961056\\
69.125	0.1068	9.84562953609334\\
69.125	0.1074	9.97945696257615\\
69.125	0.108	10.1132843890589\\
69.125	0.1086	10.2471118155418\\
69.125	0.1092	10.3809392420245\\
69.125	0.1098	10.5147666685073\\
69.125	0.1104	10.6485940949901\\
69.125	0.111	10.7824215214729\\
69.125	0.1116	10.9162489479557\\
69.125	0.1122	11.0500763744385\\
69.125	0.1128	11.1839038009213\\
69.125	0.1134	11.3177312274041\\
69.125	0.114	11.4515586538869\\
69.125	0.1146	11.5853860803697\\
69.125	0.1152	11.7192135068525\\
69.125	0.1158	11.8530409333353\\
69.125	0.1164	11.9868683598181\\
69.125	0.117	12.1206957863009\\
69.125	0.1176	12.2545232127837\\
69.125	0.1182	12.3883506392665\\
69.125	0.1188	12.5221780657493\\
69.125	0.1194	12.6560054922321\\
69.125	0.12	12.7898329187149\\
69.125	0.1206	12.9236603451977\\
69.125	0.1212	13.0574877716805\\
69.125	0.1218	13.1913151981633\\
69.125	0.1224	13.3251426246461\\
69.125	0.123	13.4589700511289\\
69.5	0.093	6.7987903748799\\
69.5	0.0936	6.93508488901357\\
69.5	0.0942	7.07137940314723\\
69.5	0.0948	7.2076739172809\\
69.5	0.0954	7.34396843141455\\
69.5	0.096	7.4802629455482\\
69.5	0.0966	7.61655745968187\\
69.5	0.0972	7.75285197381552\\
69.5	0.0978	7.88914648794919\\
69.5	0.0984	8.02544100208284\\
69.5	0.099	8.16173551621652\\
69.5	0.0996	8.29803003035018\\
69.5	0.1002	8.43432454448384\\
69.5	0.1008	8.57061905861748\\
69.5	0.1014	8.70691357275116\\
69.5	0.102	8.8432080868848\\
69.5	0.1026	8.97950260101848\\
69.5	0.1032	9.11579711515213\\
69.5	0.1038	9.2520916292858\\
69.5	0.1044	9.38838614341945\\
69.5	0.105	9.52468065755312\\
69.5	0.1056	9.66097517168677\\
69.5	0.1062	9.79726968582044\\
69.5	0.1068	9.93356419995411\\
69.5	0.1074	10.0698587140878\\
69.5	0.108	10.2061532282214\\
69.5	0.1086	10.3424477423551\\
69.5	0.1092	10.4787422564887\\
69.5	0.1098	10.6150367706224\\
69.5	0.1104	10.7513312847561\\
69.5	0.111	10.8876257988897\\
69.5	0.1116	11.0239203130234\\
69.5	0.1122	11.160214827157\\
69.5	0.1128	11.2965093412907\\
69.5	0.1134	11.4328038554244\\
69.5	0.114	11.569098369558\\
69.5	0.1146	11.7053928836917\\
69.5	0.1152	11.8416873978253\\
69.5	0.1158	11.977981911959\\
69.5	0.1164	12.1142764260927\\
69.5	0.117	12.2505709402263\\
69.5	0.1176	12.38686545436\\
69.5	0.1182	12.5231599684937\\
69.5	0.1188	12.6594544826273\\
69.5	0.1194	12.795748996761\\
69.5	0.12	12.9320435108946\\
69.5	0.1206	13.0683380250283\\
69.5	0.1212	13.2046325391619\\
69.5	0.1218	13.3409270532956\\
69.5	0.1224	13.4772215674293\\
69.5	0.123	13.6135160815629\\
69.875	0.093	6.82998202277082\\
69.875	0.0936	6.96874362455533\\
69.875	0.0942	7.10750522633985\\
69.875	0.0948	7.24626682812438\\
69.875	0.0954	7.3850284299089\\
69.875	0.096	7.52379003169342\\
69.875	0.0966	7.66255163347795\\
69.875	0.0972	7.80131323526247\\
69.875	0.0978	7.94007483704701\\
69.875	0.0984	8.07883643883153\\
69.875	0.099	8.21759804061604\\
69.875	0.0996	8.35635964240058\\
69.875	0.1002	8.49512124418509\\
69.875	0.1008	8.63388284596962\\
69.875	0.1014	8.77264444775413\\
69.875	0.102	8.91140604953867\\
69.875	0.1026	9.05016765132318\\
69.875	0.1032	9.18892925310772\\
69.875	0.1038	9.32769085489224\\
69.875	0.1044	9.46645245667676\\
69.875	0.105	9.60521405846129\\
69.875	0.1056	9.74397566024581\\
69.875	0.1062	9.88273726203033\\
69.875	0.1068	10.0214988638148\\
69.875	0.1074	10.1602604655994\\
69.875	0.108	10.2990220673839\\
69.875	0.1086	10.4377836691684\\
69.875	0.1092	10.5765452709529\\
69.875	0.1098	10.7153068727375\\
69.875	0.1104	10.854068474522\\
69.875	0.111	10.9928300763065\\
69.875	0.1116	11.131591678091\\
69.875	0.1122	11.2703532798756\\
69.875	0.1128	11.4091148816601\\
69.875	0.1134	11.5478764834446\\
69.875	0.114	11.6866380852291\\
69.875	0.1146	11.8253996870137\\
69.875	0.1152	11.9641612887982\\
69.875	0.1158	12.1029228905827\\
69.875	0.1164	12.2416844923672\\
69.875	0.117	12.3804460941518\\
69.875	0.1176	12.5192076959363\\
69.875	0.1182	12.6579692977208\\
69.875	0.1188	12.7967308995053\\
69.875	0.1194	12.9354925012898\\
69.875	0.12	13.0742541030744\\
69.875	0.1206	13.2130157048589\\
69.875	0.1212	13.3517773066434\\
69.875	0.1218	13.4905389084279\\
69.875	0.1224	13.6293005102125\\
69.875	0.123	13.768062111997\\
70.25	0.093	6.8611736706617\\
70.25	0.0936	7.00240236009709\\
70.25	0.0942	7.14363104953247\\
70.25	0.0948	7.28485973896787\\
70.25	0.0954	7.42608842840325\\
70.25	0.096	7.56731711783864\\
70.25	0.0966	7.70854580727402\\
70.25	0.0972	7.8497744967094\\
70.25	0.0978	7.99100318614479\\
70.25	0.0984	8.13223187558017\\
70.25	0.099	8.27346056501558\\
70.25	0.0996	8.41468925445096\\
70.25	0.1002	8.55591794388634\\
70.25	0.1008	8.69714663332172\\
70.25	0.1014	8.83837532275713\\
70.25	0.102	8.9796040121925\\
70.25	0.1026	9.1208327016279\\
70.25	0.1032	9.26206139106327\\
70.25	0.1038	9.40329008049866\\
70.25	0.1044	9.54451876993404\\
70.25	0.105	9.68574745936944\\
70.25	0.1056	9.82697614880483\\
70.25	0.1062	9.96820483824021\\
70.25	0.1068	10.1094335276756\\
70.25	0.1074	10.250662217111\\
70.25	0.108	10.3918909065464\\
70.25	0.1086	10.5331195959817\\
70.25	0.1092	10.6743482854171\\
70.25	0.1098	10.8155769748525\\
70.25	0.1104	10.9568056642879\\
70.25	0.111	11.0980343537233\\
70.25	0.1116	11.2392630431587\\
70.25	0.1122	11.3804917325941\\
70.25	0.1128	11.5217204220295\\
70.25	0.1134	11.6629491114649\\
70.25	0.114	11.8041778009002\\
70.25	0.1146	11.9454064903356\\
70.25	0.1152	12.086635179771\\
70.25	0.1158	12.2278638692064\\
70.25	0.1164	12.3690925586418\\
70.25	0.117	12.5103212480772\\
70.25	0.1176	12.6515499375126\\
70.25	0.1182	12.7927786269479\\
70.25	0.1188	12.9340073163833\\
70.25	0.1194	13.0752360058187\\
70.25	0.12	13.2164646952541\\
70.25	0.1206	13.3576933846895\\
70.25	0.1212	13.4989220741249\\
70.25	0.1218	13.6401507635603\\
70.25	0.1224	13.7813794529957\\
70.25	0.123	13.9226081424311\\
70.625	0.093	6.89236531855261\\
70.625	0.0936	7.03606109563886\\
70.625	0.0942	7.17975687272511\\
70.625	0.0948	7.32345264981137\\
70.625	0.0954	7.46714842689761\\
70.625	0.096	7.61084420398386\\
70.625	0.0966	7.75453998107011\\
70.625	0.0972	7.89823575815636\\
70.625	0.0978	8.04193153524261\\
70.625	0.0984	8.18562731232885\\
70.625	0.099	8.32932308941511\\
70.625	0.0996	8.47301886650136\\
70.625	0.1002	8.61671464358761\\
70.625	0.1008	8.76041042067385\\
70.625	0.1014	8.90410619776011\\
70.625	0.102	9.04780197484635\\
70.625	0.1026	9.19149775193262\\
70.625	0.1032	9.33519352901885\\
70.625	0.1038	9.4788893061051\\
70.625	0.1044	9.62258508319135\\
70.625	0.105	9.7662808602776\\
70.625	0.1056	9.90997663736385\\
70.625	0.1062	10.0536724144501\\
70.625	0.1068	10.1973681915364\\
70.625	0.1074	10.3410639686226\\
70.625	0.108	10.4847597457089\\
70.625	0.1086	10.6284555227951\\
70.625	0.1092	10.7721512998814\\
70.625	0.1098	10.9158470769676\\
70.625	0.1104	11.0595428540539\\
70.625	0.111	11.2032386311401\\
70.625	0.1116	11.3469344082264\\
70.625	0.1122	11.4906301853126\\
70.625	0.1128	11.6343259623989\\
70.625	0.1134	11.7780217394851\\
70.625	0.114	11.9217175165714\\
70.625	0.1146	12.0654132936576\\
70.625	0.1152	12.2091090707438\\
70.625	0.1158	12.3528048478301\\
70.625	0.1164	12.4965006249164\\
70.625	0.117	12.6401964020026\\
70.625	0.1176	12.7838921790889\\
70.625	0.1182	12.9275879561751\\
70.625	0.1188	13.0712837332614\\
70.625	0.1194	13.2149795103476\\
70.625	0.12	13.3586752874339\\
70.625	0.1206	13.5023710645201\\
70.625	0.1212	13.6460668416064\\
70.625	0.1218	13.7897626186926\\
70.625	0.1224	13.9334583957789\\
70.625	0.123	14.0771541728651\\
71	0.093	6.92355696644351\\
71	0.0936	7.06971983118062\\
71	0.0942	7.21588269591774\\
71	0.0948	7.36204556065485\\
71	0.0954	7.50820842539196\\
71	0.096	7.65437129012908\\
71	0.0966	7.80053415486618\\
71	0.0972	7.94669701960331\\
71	0.0978	8.09285988434041\\
71	0.0984	8.23902274907753\\
71	0.099	8.38518561381464\\
71	0.0996	8.53134847855175\\
71	0.1002	8.67751134328886\\
71	0.1008	8.82367420802599\\
71	0.1014	8.9698370727631\\
71	0.102	9.11599993750021\\
71	0.1026	9.26216280223731\\
71	0.1032	9.40832566697443\\
71	0.1038	9.55448853171154\\
71	0.1044	9.70065139644866\\
71	0.105	9.84681426118577\\
71	0.1056	9.99297712592289\\
71	0.1062	10.13913999066\\
71	0.1068	10.2853028553971\\
71	0.1074	10.4314657201342\\
71	0.108	10.5776285848713\\
71	0.1086	10.7237914496085\\
71	0.1092	10.8699543143456\\
71	0.1098	11.0161171790827\\
71	0.1104	11.1622800438198\\
71	0.111	11.3084429085569\\
71	0.1116	11.454605773294\\
71	0.1122	11.6007686380311\\
71	0.1128	11.7469315027682\\
71	0.1134	11.8930943675054\\
71	0.114	12.0392572322425\\
71	0.1146	12.1854200969796\\
71	0.1152	12.3315829617167\\
71	0.1158	12.4777458264538\\
71	0.1164	12.6239086911909\\
71	0.117	12.770071555928\\
71	0.1176	12.9162344206651\\
71	0.1182	13.0623972854023\\
71	0.1188	13.2085601501394\\
71	0.1194	13.3547230148765\\
71	0.12	13.5008858796136\\
71	0.1206	13.6470487443507\\
71	0.1212	13.7932116090878\\
71	0.1218	13.9393744738249\\
71	0.1224	14.0855373385621\\
71	0.123	14.2317002032992\\
71.375	0.093	6.95474861433439\\
71.375	0.0936	7.10337856672237\\
71.375	0.0942	7.25200851911036\\
71.375	0.0948	7.40063847149834\\
71.375	0.0954	7.5492684238863\\
71.375	0.096	7.69789837627428\\
71.375	0.0966	7.84652832866226\\
71.375	0.0972	7.99515828105024\\
71.375	0.0978	8.14378823343822\\
71.375	0.0984	8.29241818582618\\
71.375	0.099	8.44104813821417\\
71.375	0.0996	8.58967809060215\\
71.375	0.1002	8.73830804299013\\
71.375	0.1008	8.88693799537809\\
71.375	0.1014	9.03556794776608\\
71.375	0.102	9.18419790015403\\
71.375	0.1026	9.33282785254202\\
71.375	0.1032	9.48145780493\\
71.375	0.1038	9.63008775731798\\
71.375	0.1044	9.77871770970594\\
71.375	0.105	9.92734766209392\\
71.375	0.1056	10.0759776144819\\
71.375	0.1062	10.2246075668699\\
71.375	0.1068	10.3732375192579\\
71.375	0.1074	10.5218674716458\\
71.375	0.108	10.6704974240338\\
71.375	0.1086	10.8191273764218\\
71.375	0.1092	10.9677573288098\\
71.375	0.1098	11.1163872811977\\
71.375	0.1104	11.2650172335857\\
71.375	0.111	11.4136471859737\\
71.375	0.1116	11.5622771383617\\
71.375	0.1122	11.7109070907496\\
71.375	0.1128	11.8595370431376\\
71.375	0.1134	12.0081669955256\\
71.375	0.114	12.1567969479136\\
71.375	0.1146	12.3054269003016\\
71.375	0.1152	12.4540568526895\\
71.375	0.1158	12.6026868050775\\
71.375	0.1164	12.7513167574655\\
71.375	0.117	12.8999467098535\\
71.375	0.1176	13.0485766622414\\
71.375	0.1182	13.1972066146294\\
71.375	0.1188	13.3458365670174\\
71.375	0.1194	13.4944665194054\\
71.375	0.12	13.6430964717933\\
71.375	0.1206	13.7917264241813\\
71.375	0.1212	13.9403563765693\\
71.375	0.1218	14.0889863289573\\
71.375	0.1224	14.2376162813452\\
71.375	0.123	14.3862462337332\\
71.75	0.093	6.9859402622253\\
71.75	0.0936	7.13703730226413\\
71.75	0.0942	7.28813434230297\\
71.75	0.0948	7.4392313823418\\
71.75	0.0954	7.59032842238065\\
71.75	0.096	7.74142546241949\\
71.75	0.0966	7.89252250245832\\
71.75	0.0972	8.04361954249717\\
71.75	0.0978	8.194716582536\\
71.75	0.0984	8.34581362257485\\
71.75	0.099	8.49691066261369\\
71.75	0.0996	8.64800770265254\\
71.75	0.1002	8.79910474269136\\
71.75	0.1008	8.95020178273022\\
71.75	0.1014	9.10129882276904\\
71.75	0.102	9.25239586280789\\
71.75	0.1026	9.40349290284672\\
71.75	0.1032	9.55458994288556\\
71.75	0.1038	9.70568698292441\\
71.75	0.1044	9.85678402296325\\
71.75	0.105	10.0078810630021\\
71.75	0.1056	10.1589781030409\\
71.75	0.1062	10.3100751430798\\
71.75	0.1068	10.4611721831186\\
71.75	0.1074	10.6122692231574\\
71.75	0.108	10.7633662631963\\
71.75	0.1086	10.9144633032351\\
71.75	0.1092	11.065560343274\\
71.75	0.1098	11.2166573833128\\
71.75	0.1104	11.3677544233516\\
71.75	0.111	11.5188514633905\\
71.75	0.1116	11.6699485034293\\
71.75	0.1122	11.8210455434682\\
71.75	0.1128	11.972142583507\\
71.75	0.1134	12.1232396235458\\
71.75	0.114	12.2743366635847\\
71.75	0.1146	12.4254337036235\\
71.75	0.1152	12.5765307436624\\
71.75	0.1158	12.7276277837012\\
71.75	0.1164	12.87872482374\\
71.75	0.117	13.0298218637789\\
71.75	0.1176	13.1809189038177\\
71.75	0.1182	13.3320159438565\\
71.75	0.1188	13.4831129838954\\
71.75	0.1194	13.6342100239342\\
71.75	0.12	13.7853070639731\\
71.75	0.1206	13.9364041040119\\
71.75	0.1212	14.0875011440508\\
71.75	0.1218	14.2385981840896\\
71.75	0.1224	14.3896952241284\\
71.75	0.123	14.5407922641673\\
72.125	0.093	7.01713191011622\\
72.125	0.0936	7.17069603780591\\
72.125	0.0942	7.32426016549562\\
72.125	0.0948	7.47782429318531\\
72.125	0.0954	7.63138842087501\\
72.125	0.096	7.78495254856472\\
72.125	0.0966	7.93851667625442\\
72.125	0.0972	8.09208080394413\\
72.125	0.0978	8.24564493163383\\
72.125	0.0984	8.39920905932354\\
72.125	0.099	8.55277318701323\\
72.125	0.0996	8.70633731470294\\
72.125	0.1002	8.85990144239263\\
72.125	0.1008	9.01346557008235\\
72.125	0.1014	9.16702969777205\\
72.125	0.102	9.32059382546176\\
72.125	0.1026	9.47415795315145\\
72.125	0.1032	9.62772208084115\\
72.125	0.1038	9.78128620853086\\
72.125	0.1044	9.93485033622056\\
72.125	0.105	10.0884144639103\\
72.125	0.1056	10.2419785916\\
72.125	0.1062	10.3955427192897\\
72.125	0.1068	10.5491068469794\\
72.125	0.1074	10.7026709746691\\
72.125	0.108	10.8562351023588\\
72.125	0.1086	11.0097992300485\\
72.125	0.1092	11.1633633577382\\
72.125	0.1098	11.3169274854279\\
72.125	0.1104	11.4704916131176\\
72.125	0.111	11.6240557408073\\
72.125	0.1116	11.777619868497\\
72.125	0.1122	11.9311839961867\\
72.125	0.1128	12.0847481238764\\
72.125	0.1134	12.2383122515661\\
72.125	0.114	12.3918763792558\\
72.125	0.1146	12.5454405069455\\
72.125	0.1152	12.6990046346352\\
72.125	0.1158	12.8525687623249\\
72.125	0.1164	13.0061328900146\\
72.125	0.117	13.1596970177043\\
72.125	0.1176	13.313261145394\\
72.125	0.1182	13.4668252730837\\
72.125	0.1188	13.6203894007734\\
72.125	0.1194	13.7739535284631\\
72.125	0.12	13.9275176561528\\
72.125	0.1206	14.0810817838425\\
72.125	0.1212	14.2346459115323\\
72.125	0.1218	14.3882100392219\\
72.125	0.1224	14.5417741669117\\
72.125	0.123	14.6953382946013\\
72.5	0.093	7.0483235580071\\
72.5	0.0936	7.20435477334766\\
72.5	0.0942	7.36038598868824\\
72.5	0.0948	7.51641720402881\\
72.5	0.0954	7.67244841936936\\
72.5	0.096	7.82847963470994\\
72.5	0.0966	7.9845108500505\\
72.5	0.0972	8.14054206539106\\
72.5	0.0978	8.29657328073164\\
72.5	0.0984	8.45260449607218\\
72.5	0.099	8.60863571141277\\
72.5	0.0996	8.76466692675332\\
72.5	0.1002	8.9206981420939\\
72.5	0.1008	9.07672935743444\\
72.5	0.1014	9.23276057277504\\
72.5	0.102	9.38879178811558\\
72.5	0.1026	9.54482300345616\\
72.5	0.1032	9.70085421879672\\
72.5	0.1038	9.85688543413728\\
72.5	0.1044	10.0129166494778\\
72.5	0.105	10.1689478648184\\
72.5	0.1056	10.324979080159\\
72.5	0.1062	10.4810102954996\\
72.5	0.1068	10.6370415108401\\
72.5	0.1074	10.7930727261807\\
72.5	0.108	10.9491039415213\\
72.5	0.1086	11.1051351568618\\
72.5	0.1092	11.2611663722024\\
72.5	0.1098	11.4171975875429\\
72.5	0.1104	11.5732288028835\\
72.5	0.111	11.7292600182241\\
72.5	0.1116	11.8852912335646\\
72.5	0.1122	12.0413224489052\\
72.5	0.1128	12.1973536642458\\
72.5	0.1134	12.3533848795863\\
72.5	0.114	12.5094160949269\\
72.5	0.1146	12.6654473102675\\
72.5	0.1152	12.821478525608\\
72.5	0.1158	12.9775097409486\\
72.5	0.1164	13.1335409562892\\
72.5	0.117	13.2895721716297\\
72.5	0.1176	13.4456033869703\\
72.5	0.1182	13.6016346023109\\
72.5	0.1188	13.7576658176514\\
72.5	0.1194	13.913697032992\\
72.5	0.12	14.0697282483326\\
72.5	0.1206	14.2257594636731\\
72.5	0.1212	14.3817906790137\\
72.5	0.1218	14.5378218943543\\
72.5	0.1224	14.6938531096948\\
72.5	0.123	14.8498843250354\\
72.875	0.093	7.07951520589801\\
72.875	0.0936	7.23801350888942\\
72.875	0.0942	7.39651181188086\\
72.875	0.0948	7.55501011487227\\
72.875	0.0954	7.71350841786371\\
72.875	0.096	7.87200672085514\\
72.875	0.0966	8.03050502384656\\
72.875	0.0972	8.18900332683799\\
72.875	0.0978	8.34750162982942\\
72.875	0.0984	8.50599993282086\\
72.875	0.099	8.66449823581227\\
72.875	0.0996	8.82299653880371\\
72.875	0.1002	8.98149484179513\\
72.875	0.1008	9.13999314478657\\
72.875	0.1014	9.29849144777799\\
72.875	0.102	9.45698975076944\\
72.875	0.1026	9.61548805376086\\
72.875	0.1032	9.77398635675229\\
72.875	0.1038	9.93248465974372\\
72.875	0.1044	10.0909829627352\\
72.875	0.105	10.2494812657266\\
72.875	0.1056	10.407979568718\\
72.875	0.1062	10.5664778717094\\
72.875	0.1068	10.7249761747009\\
72.875	0.1074	10.8834744776923\\
72.875	0.108	11.0419727806837\\
72.875	0.1086	11.2004710836752\\
72.875	0.1092	11.3589693866666\\
72.875	0.1098	11.517467689658\\
72.875	0.1104	11.6759659926494\\
72.875	0.111	11.8344642956409\\
72.875	0.1116	11.9929625986323\\
72.875	0.1122	12.1514609016237\\
72.875	0.1128	12.3099592046152\\
72.875	0.1134	12.4684575076066\\
72.875	0.114	12.626955810598\\
72.875	0.1146	12.7854541135894\\
72.875	0.1152	12.9439524165809\\
72.875	0.1158	13.1024507195723\\
72.875	0.1164	13.2609490225637\\
72.875	0.117	13.4194473255552\\
72.875	0.1176	13.5779456285466\\
72.875	0.1182	13.736443931538\\
72.875	0.1188	13.8949422345295\\
72.875	0.1194	14.0534405375209\\
72.875	0.12	14.2119388405123\\
72.875	0.1206	14.3704371435037\\
72.875	0.1212	14.5289354464952\\
72.875	0.1218	14.6874337494866\\
72.875	0.1224	14.845932052478\\
72.875	0.123	15.0044303554695\\
73.25	0.093	7.11070685378888\\
73.25	0.0936	7.27167224443119\\
73.25	0.0942	7.43263763507348\\
73.25	0.0948	7.59360302571578\\
73.25	0.0954	7.75456841635805\\
73.25	0.096	7.91553380700036\\
73.25	0.0966	8.07649919764265\\
73.25	0.0972	8.23746458828494\\
73.25	0.0978	8.39842997892724\\
73.25	0.0984	8.55939536956951\\
73.25	0.099	8.72036076021183\\
73.25	0.0996	8.88132615085411\\
73.25	0.1002	9.04229154149641\\
73.25	0.1008	9.20325693213869\\
73.25	0.1014	9.364222322781\\
73.25	0.102	9.52518771342326\\
73.25	0.1026	9.68615310406558\\
73.25	0.1032	9.84711849470786\\
73.25	0.1038	10.0080838853502\\
73.25	0.1044	10.1690492759924\\
73.25	0.105	10.3300146666347\\
73.25	0.1056	10.490980057277\\
73.25	0.1062	10.6519454479193\\
73.25	0.1068	10.8129108385616\\
73.25	0.1074	10.9738762292039\\
73.25	0.108	11.1348416198462\\
73.25	0.1086	11.2958070104885\\
73.25	0.1092	11.4567724011308\\
73.25	0.1098	11.6177377917731\\
73.25	0.1104	11.7787031824154\\
73.25	0.111	11.9396685730577\\
73.25	0.1116	12.1006339637\\
73.25	0.1122	12.2615993543422\\
73.25	0.1128	12.4225647449846\\
73.25	0.1134	12.5835301356268\\
73.25	0.114	12.7444955262691\\
73.25	0.1146	12.9054609169114\\
73.25	0.1152	13.0664263075537\\
73.25	0.1158	13.227391698196\\
73.25	0.1164	13.3883570888383\\
73.25	0.117	13.5493224794806\\
73.25	0.1176	13.7102878701229\\
73.25	0.1182	13.8712532607652\\
73.25	0.1188	14.0322186514075\\
73.25	0.1194	14.1931840420498\\
73.25	0.12	14.354149432692\\
73.25	0.1206	14.5151148233344\\
73.25	0.1212	14.6760802139766\\
73.25	0.1218	14.8370456046189\\
73.25	0.1224	14.9980109952612\\
73.25	0.123	15.1589763859035\\
73.625	0.093	7.14189850167979\\
73.625	0.0936	7.30533097997294\\
73.625	0.0942	7.4687634582661\\
73.625	0.0948	7.63219593655924\\
73.625	0.0954	7.7956284148524\\
73.625	0.096	7.95906089314556\\
73.625	0.0966	8.12249337143871\\
73.625	0.0972	8.28592584973187\\
73.625	0.0978	8.44935832802503\\
73.625	0.0984	8.61279080631819\\
73.625	0.099	8.77622328461133\\
73.625	0.0996	8.93965576290449\\
73.625	0.1002	9.10308824119764\\
73.625	0.1008	9.26652071949081\\
73.625	0.1014	9.42995319778396\\
73.625	0.102	9.59338567607712\\
73.625	0.1026	9.75681815437027\\
73.625	0.1032	9.92025063266342\\
73.625	0.1038	10.0836831109566\\
73.625	0.1044	10.2471155892497\\
73.625	0.105	10.4105480675429\\
73.625	0.1056	10.5739805458361\\
73.625	0.1062	10.7374130241292\\
73.625	0.1068	10.9008455024224\\
73.625	0.1074	11.0642779807155\\
73.625	0.108	11.2277104590087\\
73.625	0.1086	11.3911429373018\\
73.625	0.1092	11.554575415595\\
73.625	0.1098	11.7180078938881\\
73.625	0.1104	11.8814403721813\\
73.625	0.111	12.0448728504745\\
73.625	0.1116	12.2083053287676\\
73.625	0.1122	12.3717378070608\\
73.625	0.1128	12.5351702853539\\
73.625	0.1134	12.6986027636471\\
73.625	0.114	12.8620352419402\\
73.625	0.1146	13.0254677202334\\
73.625	0.1152	13.1889001985265\\
73.625	0.1158	13.3523326768197\\
73.625	0.1164	13.5157651551129\\
73.625	0.117	13.679197633406\\
73.625	0.1176	13.8426301116992\\
73.625	0.1182	14.0060625899923\\
73.625	0.1188	14.1694950682855\\
73.625	0.1194	14.3329275465786\\
73.625	0.12	14.4963600248718\\
73.625	0.1206	14.6597925031649\\
73.625	0.1212	14.8232249814581\\
73.625	0.1218	14.9866574597512\\
73.625	0.1224	15.1500899380444\\
73.625	0.123	15.3135224163376\\
74	0.093	7.1730901495707\\
74	0.0936	7.33898971551471\\
74	0.0942	7.50488928145873\\
74	0.0948	7.67078884740275\\
74	0.0954	7.83668841334676\\
74	0.096	8.00258797929078\\
74	0.0966	8.1684875452348\\
74	0.0972	8.33438711117881\\
74	0.0978	8.50028667712284\\
74	0.0984	8.66618624306686\\
74	0.099	8.83208580901088\\
74	0.0996	8.99798537495489\\
74	0.1002	9.16388494089891\\
74	0.1008	9.32978450684294\\
74	0.1014	9.49568407278694\\
74	0.102	9.66158363873097\\
74	0.1026	9.82748320467498\\
74	0.1032	9.99338277061901\\
74	0.1038	10.159282336563\\
74	0.1044	10.3251819025071\\
74	0.105	10.4910814684511\\
74	0.1056	10.6569810343951\\
74	0.1062	10.8228806003391\\
74	0.1068	10.9887801662831\\
74	0.1074	11.1546797322271\\
74	0.108	11.3205792981712\\
74	0.1086	11.4864788641152\\
74	0.1092	11.6523784300592\\
74	0.1098	11.8182779960032\\
74	0.1104	11.9841775619472\\
74	0.111	12.1500771278913\\
74	0.1116	12.3159766938353\\
74	0.1122	12.4818762597793\\
74	0.1128	12.6477758257233\\
74	0.1134	12.8136753916673\\
74	0.114	12.9795749576113\\
74	0.1146	13.1454745235554\\
74	0.1152	13.3113740894994\\
74	0.1158	13.4772736554434\\
74	0.1164	13.6431732213874\\
74	0.117	13.8090727873314\\
74	0.1176	13.9749723532755\\
74	0.1182	14.1408719192195\\
74	0.1188	14.3067714851635\\
74	0.1194	14.4726710511075\\
74	0.12	14.6385706170515\\
74	0.1206	14.8044701829955\\
74	0.1212	14.9703697489396\\
74	0.1218	15.1362693148836\\
74	0.1224	15.3021688808276\\
74	0.123	15.4680684467716\\
};
\end{axis}

\begin{axis}[%
width=4.927496cm,
height=3.870968cm,
at={(0cm,0cm)},
scale only axis,
xmin=56,
xmax=74,
tick align=outside,
xlabel={$L_{cut}$},
xmajorgrids,
ymin=0.093,
ymax=0.123,
ylabel={$D_{rlx}$},
ymajorgrids,
zmin=-2532.82546475729,
zmax=436.681088754776,
zlabel={$c$},
zmajorgrids,
view={-140}{50},
legend style={at={(1.03,1)},anchor=north west,legend cell align=left,align=left,draw=white!15!black}
]
\addplot3[only marks,mark=*,mark options={},mark size=1.5000pt,color=mycolor1] plot table[row sep=crcr,]{%
74	0.123	-233.157966898601\\
72	0.113	-200.830593420783\\
61	0.095	-93.7391747783605\\
56	0.093	-119.696332564413\\
};
\addplot3[only marks,mark=*,mark options={},mark size=1.5000pt,color=black] plot table[row sep=crcr,]{%
69	0.104	-141.033171294952\\
};

\addplot3[%
surf,
opacity=0.7,
shader=interp,
colormap={mymap}{[1pt] rgb(0pt)=(0.0901961,0.239216,0.0745098); rgb(1pt)=(0.0945149,0.242058,0.0739522); rgb(2pt)=(0.0988592,0.244894,0.0733566); rgb(3pt)=(0.103229,0.247724,0.0727241); rgb(4pt)=(0.107623,0.250549,0.0720557); rgb(5pt)=(0.112043,0.253367,0.0713525); rgb(6pt)=(0.116487,0.25618,0.0706154); rgb(7pt)=(0.120956,0.258986,0.0698456); rgb(8pt)=(0.125449,0.261787,0.0690441); rgb(9pt)=(0.129967,0.264581,0.0682118); rgb(10pt)=(0.134508,0.26737,0.06735); rgb(11pt)=(0.139074,0.270152,0.0664596); rgb(12pt)=(0.143663,0.272929,0.0655416); rgb(13pt)=(0.148275,0.275699,0.0645971); rgb(14pt)=(0.152911,0.278463,0.0636271); rgb(15pt)=(0.15757,0.281221,0.0626328); rgb(16pt)=(0.162252,0.283973,0.0616151); rgb(17pt)=(0.166957,0.286719,0.060575); rgb(18pt)=(0.171685,0.289458,0.0595136); rgb(19pt)=(0.176434,0.292191,0.0584321); rgb(20pt)=(0.181207,0.294918,0.0573313); rgb(21pt)=(0.186001,0.297639,0.0562123); rgb(22pt)=(0.190817,0.300353,0.0550763); rgb(23pt)=(0.195655,0.303061,0.0539242); rgb(24pt)=(0.200514,0.305763,0.052757); rgb(25pt)=(0.205395,0.308459,0.0515759); rgb(26pt)=(0.210296,0.311149,0.0503624); rgb(27pt)=(0.215212,0.313846,0.0490067); rgb(28pt)=(0.220142,0.316548,0.0475043); rgb(29pt)=(0.22509,0.319254,0.0458704); rgb(30pt)=(0.230056,0.321962,0.0441205); rgb(31pt)=(0.235042,0.324671,0.04227); rgb(32pt)=(0.240048,0.327379,0.0403343); rgb(33pt)=(0.245078,0.330085,0.0383287); rgb(34pt)=(0.250131,0.332786,0.0362688); rgb(35pt)=(0.25521,0.335482,0.0341698); rgb(36pt)=(0.260317,0.33817,0.0320472); rgb(37pt)=(0.265451,0.340849,0.0299163); rgb(38pt)=(0.270616,0.343517,0.0277927); rgb(39pt)=(0.275813,0.346172,0.0256916); rgb(40pt)=(0.281043,0.348814,0.0236284); rgb(41pt)=(0.286307,0.35144,0.0216186); rgb(42pt)=(0.291607,0.354048,0.0196776); rgb(43pt)=(0.296945,0.356637,0.0178207); rgb(44pt)=(0.302322,0.359206,0.0160634); rgb(45pt)=(0.307739,0.361753,0.0144211); rgb(46pt)=(0.313198,0.364275,0.0129091); rgb(47pt)=(0.318701,0.366772,0.0115428); rgb(48pt)=(0.324249,0.369242,0.0103377); rgb(49pt)=(0.329843,0.371682,0.00930909); rgb(50pt)=(0.335485,0.374093,0.00847245); rgb(51pt)=(0.341176,0.376471,0.00784314); rgb(52pt)=(0.346925,0.378826,0.00732741); rgb(53pt)=(0.352735,0.381168,0.00682184); rgb(54pt)=(0.358605,0.383497,0.00632729); rgb(55pt)=(0.364532,0.385812,0.00584464); rgb(56pt)=(0.370516,0.388113,0.00537476); rgb(57pt)=(0.376552,0.390399,0.00491852); rgb(58pt)=(0.38264,0.39267,0.00447681); rgb(59pt)=(0.388777,0.394925,0.00405048); rgb(60pt)=(0.394962,0.397164,0.00364042); rgb(61pt)=(0.401191,0.399386,0.00324749); rgb(62pt)=(0.407464,0.401592,0.00287258); rgb(63pt)=(0.413777,0.40378,0.00251655); rgb(64pt)=(0.420129,0.40595,0.00218028); rgb(65pt)=(0.426518,0.408102,0.00186463); rgb(66pt)=(0.432942,0.410234,0.00157049); rgb(67pt)=(0.439399,0.412348,0.00129873); rgb(68pt)=(0.445885,0.414441,0.00105022); rgb(69pt)=(0.452401,0.416515,0.000825833); rgb(70pt)=(0.458942,0.418567,0.000626441); rgb(71pt)=(0.465508,0.420599,0.00045292); rgb(72pt)=(0.472096,0.422609,0.000306141); rgb(73pt)=(0.478704,0.424596,0.000186979); rgb(74pt)=(0.485331,0.426562,9.63073e-05); rgb(75pt)=(0.491973,0.428504,3.49981e-05); rgb(76pt)=(0.498628,0.430422,3.92506e-06); rgb(77pt)=(0.505323,0.432315,0); rgb(78pt)=(0.512206,0.434168,0); rgb(79pt)=(0.519282,0.435983,0); rgb(80pt)=(0.526529,0.437764,0); rgb(81pt)=(0.533922,0.439512,0); rgb(82pt)=(0.54144,0.441232,0); rgb(83pt)=(0.549059,0.442927,0); rgb(84pt)=(0.556756,0.444599,0); rgb(85pt)=(0.564508,0.446252,0); rgb(86pt)=(0.572292,0.447889,0); rgb(87pt)=(0.580084,0.449514,0); rgb(88pt)=(0.587863,0.451129,0); rgb(89pt)=(0.595604,0.452737,0); rgb(90pt)=(0.603284,0.454343,0); rgb(91pt)=(0.610882,0.455948,0); rgb(92pt)=(0.618373,0.457556,0); rgb(93pt)=(0.625734,0.459171,0); rgb(94pt)=(0.632943,0.460795,0); rgb(95pt)=(0.639976,0.462432,0); rgb(96pt)=(0.64681,0.464084,0); rgb(97pt)=(0.653423,0.465756,0); rgb(98pt)=(0.659791,0.46745,0); rgb(99pt)=(0.665891,0.469169,0); rgb(100pt)=(0.6717,0.470916,0); rgb(101pt)=(0.677195,0.472696,0); rgb(102pt)=(0.682353,0.47451,0); rgb(103pt)=(0.687242,0.476355,0); rgb(104pt)=(0.691952,0.478225,0); rgb(105pt)=(0.696497,0.480118,0); rgb(106pt)=(0.700887,0.482033,0); rgb(107pt)=(0.705134,0.483968,0); rgb(108pt)=(0.709251,0.485921,0); rgb(109pt)=(0.713249,0.487891,0); rgb(110pt)=(0.71714,0.489876,0); rgb(111pt)=(0.720936,0.491875,0); rgb(112pt)=(0.724649,0.493887,0); rgb(113pt)=(0.72829,0.495909,0); rgb(114pt)=(0.731872,0.49794,0); rgb(115pt)=(0.735406,0.499979,0); rgb(116pt)=(0.738904,0.502025,0); rgb(117pt)=(0.742378,0.504075,0); rgb(118pt)=(0.74584,0.506128,0); rgb(119pt)=(0.749302,0.508182,0); rgb(120pt)=(0.752775,0.510237,0); rgb(121pt)=(0.756272,0.51229,0); rgb(122pt)=(0.759804,0.514339,0); rgb(123pt)=(0.763384,0.516385,0); rgb(124pt)=(0.767022,0.518424,0); rgb(125pt)=(0.770731,0.520455,0); rgb(126pt)=(0.774523,0.522478,0); rgb(127pt)=(0.77841,0.524489,0); rgb(128pt)=(0.782391,0.526491,0); rgb(129pt)=(0.786402,0.528496,0); rgb(130pt)=(0.790431,0.530506,0); rgb(131pt)=(0.794478,0.532521,0); rgb(132pt)=(0.798541,0.534539,0); rgb(133pt)=(0.802619,0.53656,0); rgb(134pt)=(0.806712,0.538584,0); rgb(135pt)=(0.81082,0.540609,0); rgb(136pt)=(0.81494,0.542635,0); rgb(137pt)=(0.819074,0.54466,0); rgb(138pt)=(0.823219,0.546686,0); rgb(139pt)=(0.827374,0.548709,0); rgb(140pt)=(0.831541,0.55073,0); rgb(141pt)=(0.835716,0.552749,0); rgb(142pt)=(0.8399,0.554763,0); rgb(143pt)=(0.844092,0.556774,0); rgb(144pt)=(0.848292,0.558779,0); rgb(145pt)=(0.852497,0.560778,0); rgb(146pt)=(0.856708,0.562771,0); rgb(147pt)=(0.860924,0.564756,0); rgb(148pt)=(0.865143,0.566733,0); rgb(149pt)=(0.869366,0.568701,0); rgb(150pt)=(0.873592,0.57066,0); rgb(151pt)=(0.877819,0.572608,0); rgb(152pt)=(0.882047,0.574545,0); rgb(153pt)=(0.886275,0.576471,0); rgb(154pt)=(0.890659,0.578362,0); rgb(155pt)=(0.895333,0.580203,0); rgb(156pt)=(0.900258,0.581999,0); rgb(157pt)=(0.905397,0.583755,0); rgb(158pt)=(0.910711,0.585479,0); rgb(159pt)=(0.916164,0.587176,0); rgb(160pt)=(0.921717,0.588852,0); rgb(161pt)=(0.927333,0.590513,0); rgb(162pt)=(0.932974,0.592166,0); rgb(163pt)=(0.938602,0.593815,0); rgb(164pt)=(0.94418,0.595468,0); rgb(165pt)=(0.949669,0.59713,0); rgb(166pt)=(0.955033,0.598808,0); rgb(167pt)=(0.960233,0.600507,0); rgb(168pt)=(0.965232,0.602233,0); rgb(169pt)=(0.969992,0.603992,0); rgb(170pt)=(0.974475,0.605791,0); rgb(171pt)=(0.978643,0.607636,0); rgb(172pt)=(0.98246,0.609532,0); rgb(173pt)=(0.985886,0.611486,0); rgb(174pt)=(0.988885,0.613503,0); rgb(175pt)=(0.991419,0.61559,0); rgb(176pt)=(0.99345,0.617753,0); rgb(177pt)=(0.99494,0.619997,0); rgb(178pt)=(0.995851,0.622329,0); rgb(179pt)=(0.996226,0.624763,0); rgb(180pt)=(0.996512,0.627352,0); rgb(181pt)=(0.996788,0.630095,0); rgb(182pt)=(0.997053,0.632982,0); rgb(183pt)=(0.997308,0.636004,0); rgb(184pt)=(0.997552,0.639152,0); rgb(185pt)=(0.997785,0.642416,0); rgb(186pt)=(0.998006,0.645786,0); rgb(187pt)=(0.998217,0.649253,0); rgb(188pt)=(0.998416,0.652807,0); rgb(189pt)=(0.998605,0.656439,0); rgb(190pt)=(0.998781,0.660138,0); rgb(191pt)=(0.998946,0.663897,0); rgb(192pt)=(0.9991,0.667704,0); rgb(193pt)=(0.999242,0.67155,0); rgb(194pt)=(0.999372,0.675427,0); rgb(195pt)=(0.99949,0.679323,0); rgb(196pt)=(0.999596,0.68323,0); rgb(197pt)=(0.99969,0.687139,0); rgb(198pt)=(0.999771,0.691039,0); rgb(199pt)=(0.999841,0.694921,0); rgb(200pt)=(0.999898,0.698775,0); rgb(201pt)=(0.999942,0.702592,0); rgb(202pt)=(0.999974,0.706363,0); rgb(203pt)=(0.999994,0.710077,0); rgb(204pt)=(1,0.713725,0); rgb(205pt)=(1,0.717341,0); rgb(206pt)=(1,0.720963,0); rgb(207pt)=(1,0.724591,0); rgb(208pt)=(1,0.728226,0); rgb(209pt)=(1,0.731867,0); rgb(210pt)=(1,0.735514,0); rgb(211pt)=(1,0.739167,0); rgb(212pt)=(1,0.742827,0); rgb(213pt)=(1,0.746493,0); rgb(214pt)=(1,0.750165,0); rgb(215pt)=(1,0.753843,0); rgb(216pt)=(1,0.757527,0); rgb(217pt)=(1,0.761217,0); rgb(218pt)=(1,0.764913,0); rgb(219pt)=(1,0.768615,0); rgb(220pt)=(1,0.772324,0); rgb(221pt)=(1,0.776038,0); rgb(222pt)=(1,0.779758,0); rgb(223pt)=(1,0.783484,0); rgb(224pt)=(1,0.787215,0); rgb(225pt)=(1,0.790953,0); rgb(226pt)=(1,0.794696,0); rgb(227pt)=(1,0.798445,0); rgb(228pt)=(1,0.8022,0); rgb(229pt)=(1,0.805961,0); rgb(230pt)=(1,0.809727,0); rgb(231pt)=(1,0.8135,0); rgb(232pt)=(1,0.817278,0); rgb(233pt)=(1,0.821063,0); rgb(234pt)=(1,0.824854,0); rgb(235pt)=(1,0.828652,0); rgb(236pt)=(1,0.832455,0); rgb(237pt)=(1,0.836265,0); rgb(238pt)=(1,0.840081,0); rgb(239pt)=(1,0.843903,0); rgb(240pt)=(1,0.847732,0); rgb(241pt)=(1,0.851566,0); rgb(242pt)=(1,0.855406,0); rgb(243pt)=(1,0.859253,0); rgb(244pt)=(1,0.863106,0); rgb(245pt)=(1,0.866964,0); rgb(246pt)=(1,0.870829,0); rgb(247pt)=(1,0.8747,0); rgb(248pt)=(1,0.878577,0); rgb(249pt)=(1,0.88246,0); rgb(250pt)=(1,0.886349,0); rgb(251pt)=(1,0.890243,0); rgb(252pt)=(1,0.894144,0); rgb(253pt)=(1,0.898051,0); rgb(254pt)=(1,0.901964,0); rgb(255pt)=(1,0.905882,0)},
mesh/rows=49]
table[row sep=crcr,header=false] {%
%
56	0.093	-119.696332569289\\
56	0.0936	-167.95891521305\\
56	0.0942	-216.221497856808\\
56	0.0948	-264.48408050057\\
56	0.0954	-312.746663144331\\
56	0.096	-361.009245788089\\
56	0.0966	-409.271828431851\\
56	0.0972	-457.534411075609\\
56	0.0978	-505.79699371937\\
56	0.0984	-554.059576363128\\
56	0.099	-602.322159006893\\
56	0.0996	-650.584741650648\\
56	0.1002	-698.847324294409\\
56	0.1008	-747.109906938174\\
56	0.1014	-795.372489581932\\
56	0.102	-843.63507222569\\
56	0.1026	-891.897654869452\\
56	0.1032	-940.160237513213\\
56	0.1038	-988.422820156971\\
56	0.1044	-1036.68540280073\\
56	0.105	-1084.94798544449\\
56	0.1056	-1133.21056808825\\
56	0.1062	-1181.47315073201\\
56	0.1068	-1229.73573337577\\
56	0.1074	-1277.99831601953\\
56	0.108	-1326.26089866329\\
56	0.1086	-1374.52348130705\\
56	0.1092	-1422.78606395081\\
56	0.1098	-1471.04864659457\\
56	0.1104	-1519.31122923833\\
56	0.111	-1567.57381188209\\
56	0.1116	-1615.83639452585\\
56	0.1122	-1664.09897716961\\
56	0.1128	-1712.36155981337\\
56	0.1134	-1760.62414245713\\
56	0.114	-1808.88672510089\\
56	0.1146	-1857.14930774465\\
56	0.1152	-1905.41189038841\\
56	0.1158	-1953.67447303217\\
56	0.1164	-2001.93705567593\\
56	0.117	-2050.19963831969\\
56	0.1176	-2098.46222096345\\
56	0.1182	-2146.72480360721\\
56	0.1188	-2194.98738625097\\
56	0.1194	-2243.24996889473\\
56	0.12	-2291.51255153849\\
56	0.1206	-2339.77513418225\\
56	0.1212	-2388.03771682602\\
56	0.1218	-2436.30029946977\\
56	0.1224	-2484.56288211353\\
56	0.123	-2532.82546475729\\
56.375	0.093	-108.105136291706\\
56.375	0.0936	-155.641348070243\\
56.375	0.0942	-203.177559848777\\
56.375	0.0948	-250.713771627314\\
56.375	0.0954	-298.249983405851\\
56.375	0.096	-345.786195184384\\
56.375	0.0966	-393.322406962921\\
56.375	0.0972	-440.858618741455\\
56.375	0.0978	-488.394830519992\\
56.375	0.0984	-535.931042298529\\
56.375	0.099	-583.467254077066\\
56.375	0.0996	-631.0034658556\\
56.375	0.1002	-678.539677634137\\
56.375	0.1008	-726.075889412674\\
56.375	0.1014	-773.612101191207\\
56.375	0.102	-821.148312969744\\
56.375	0.1026	-868.684524748278\\
56.375	0.1032	-916.220736526815\\
56.375	0.1038	-963.756948305348\\
56.375	0.1044	-1011.29316008389\\
56.375	0.105	-1058.82937186242\\
56.375	0.1056	-1106.36558364096\\
56.375	0.1062	-1153.9017954195\\
56.375	0.1068	-1201.43800719803\\
56.375	0.1074	-1248.97421897657\\
56.375	0.108	-1296.5104307551\\
56.375	0.1086	-1344.04664253364\\
56.375	0.1092	-1391.58285431218\\
56.375	0.1098	-1439.11906609071\\
56.375	0.1104	-1486.65527786925\\
56.375	0.111	-1534.19148964778\\
56.375	0.1116	-1581.72770142632\\
56.375	0.1122	-1629.26391320485\\
56.375	0.1128	-1676.80012498339\\
56.375	0.1134	-1724.33633676192\\
56.375	0.114	-1771.87254854046\\
56.375	0.1146	-1819.408760319\\
56.375	0.1152	-1866.94497209753\\
56.375	0.1158	-1914.48118387607\\
56.375	0.1164	-1962.0173956546\\
56.375	0.117	-2009.55360743314\\
56.375	0.1176	-2057.08981921167\\
56.375	0.1182	-2104.62603099021\\
56.375	0.1188	-2152.16224276875\\
56.375	0.1194	-2199.69845454728\\
56.375	0.12	-2247.23466632582\\
56.375	0.1206	-2294.77087810435\\
56.375	0.1212	-2342.30708988289\\
56.375	0.1218	-2389.84330166142\\
56.375	0.1224	-2437.37951343996\\
56.375	0.123	-2484.91572521849\\
56.75	0.093	-96.5139400141197\\
56.75	0.0936	-143.323780927432\\
56.75	0.0942	-190.133621840741\\
56.75	0.0948	-236.943462754054\\
56.75	0.0954	-283.753303667367\\
56.75	0.096	-330.563144580676\\
56.75	0.0966	-377.372985493992\\
56.75	0.0972	-424.182826407301\\
56.75	0.0978	-470.992667320614\\
56.75	0.0984	-517.802508233923\\
56.75	0.099	-564.612349147235\\
56.75	0.0996	-611.422190060544\\
56.75	0.1002	-658.232030973857\\
56.75	0.1008	-705.04187188717\\
56.75	0.1014	-751.851712800479\\
56.75	0.102	-798.661553713791\\
56.75	0.1026	-845.471394627104\\
56.75	0.1032	-892.281235540417\\
56.75	0.1038	-939.091076453726\\
56.75	0.1044	-985.900917367038\\
56.75	0.105	-1032.71075828035\\
56.75	0.1056	-1079.52059919366\\
56.75	0.1062	-1126.33044010697\\
56.75	0.1068	-1173.14028102028\\
56.75	0.1074	-1219.95012193359\\
56.75	0.108	-1266.75996284691\\
56.75	0.1086	-1313.56980376022\\
56.75	0.1092	-1360.37964467353\\
56.75	0.1098	-1407.18948558684\\
56.75	0.1104	-1453.99932650015\\
56.75	0.111	-1500.80916741346\\
56.75	0.1116	-1547.61900832678\\
56.75	0.1122	-1594.42884924008\\
56.75	0.1128	-1641.2386901534\\
56.75	0.1134	-1688.04853106671\\
56.75	0.114	-1734.85837198002\\
56.75	0.1146	-1781.66821289334\\
56.75	0.1152	-1828.47805380664\\
56.75	0.1158	-1875.28789471996\\
56.75	0.1164	-1922.09773563327\\
56.75	0.117	-1968.90757654658\\
56.75	0.1176	-2015.71741745989\\
56.75	0.1182	-2062.5272583732\\
56.75	0.1188	-2109.33709928652\\
56.75	0.1194	-2156.14694019982\\
56.75	0.12	-2202.95678111314\\
56.75	0.1206	-2249.76662202645\\
56.75	0.1212	-2296.57646293976\\
56.75	0.1218	-2343.38630385307\\
56.75	0.1224	-2390.19614476638\\
56.75	0.123	-2437.00598567969\\
57.125	0.093	-84.9227437365371\\
57.125	0.0936	-131.006213784625\\
57.125	0.0942	-177.08968383271\\
57.125	0.0948	-223.173153880798\\
57.125	0.0954	-269.256623928886\\
57.125	0.096	-315.340093976971\\
57.125	0.0966	-361.423564025063\\
57.125	0.0972	-407.507034073147\\
57.125	0.0978	-453.590504121235\\
57.125	0.0984	-499.67397416932\\
57.125	0.099	-545.757444217408\\
57.125	0.0996	-591.840914265493\\
57.125	0.1002	-637.924384313581\\
57.125	0.1008	-684.007854361673\\
57.125	0.1014	-730.091324409757\\
57.125	0.102	-776.174794457846\\
57.125	0.1026	-822.25826450593\\
57.125	0.1032	-868.341734554018\\
57.125	0.1038	-914.425204602103\\
57.125	0.1044	-960.508674650191\\
57.125	0.105	-1006.59214469828\\
57.125	0.1056	-1052.67561474637\\
57.125	0.1062	-1098.75908479446\\
57.125	0.1068	-1144.84255484254\\
57.125	0.1074	-1190.92602489063\\
57.125	0.108	-1237.00949493871\\
57.125	0.1086	-1283.0929649868\\
57.125	0.1092	-1329.17643503489\\
57.125	0.1098	-1375.25990508298\\
57.125	0.1104	-1421.34337513107\\
57.125	0.111	-1467.42684517915\\
57.125	0.1116	-1513.51031522724\\
57.125	0.1122	-1559.59378527532\\
57.125	0.1128	-1605.67725532342\\
57.125	0.1134	-1651.7607253715\\
57.125	0.114	-1697.84419541959\\
57.125	0.1146	-1743.92766546768\\
57.125	0.1152	-1790.01113551576\\
57.125	0.1158	-1836.09460556385\\
57.125	0.1164	-1882.17807561194\\
57.125	0.117	-1928.26154566002\\
57.125	0.1176	-1974.34501570811\\
57.125	0.1182	-2020.4284857562\\
57.125	0.1188	-2066.51195580429\\
57.125	0.1194	-2112.59542585237\\
57.125	0.12	-2158.67889590046\\
57.125	0.1206	-2204.76236594855\\
57.125	0.1212	-2250.84583599664\\
57.125	0.1218	-2296.92930604472\\
57.125	0.1224	-2343.01277609281\\
57.125	0.123	-2389.09624614089\\
57.5	0.093	-73.3315474589508\\
57.5	0.0936	-118.688646641818\\
57.5	0.0942	-164.045745824678\\
57.5	0.0948	-209.402845007542\\
57.5	0.0954	-254.759944190406\\
57.5	0.096	-300.117043373266\\
57.5	0.0966	-345.474142556133\\
57.5	0.0972	-390.831241738993\\
57.5	0.0978	-436.188340921857\\
57.5	0.0984	-481.545440104717\\
57.5	0.099	-526.902539287581\\
57.5	0.0996	-572.259638470441\\
57.5	0.1002	-617.616737653309\\
57.5	0.1008	-662.973836836172\\
57.5	0.1014	-708.330936019032\\
57.5	0.102	-753.688035201896\\
57.5	0.1026	-799.045134384756\\
57.5	0.1032	-844.402233567624\\
57.5	0.1038	-889.759332750484\\
57.5	0.1044	-935.116431933347\\
57.5	0.105	-980.473531116211\\
57.5	0.1056	-1025.83063029907\\
57.5	0.1062	-1071.18772948194\\
57.5	0.1068	-1116.5448286648\\
57.5	0.1074	-1161.90192784766\\
57.5	0.108	-1207.25902703052\\
57.5	0.1086	-1252.61612621339\\
57.5	0.1092	-1297.97322539625\\
57.5	0.1098	-1343.33032457911\\
57.5	0.1104	-1388.68742376198\\
57.5	0.111	-1434.04452294484\\
57.5	0.1116	-1479.4016221277\\
57.5	0.1122	-1524.75872131056\\
57.5	0.1128	-1570.11582049343\\
57.5	0.1134	-1615.47291967629\\
57.5	0.114	-1660.83001885915\\
57.5	0.1146	-1706.18711804202\\
57.5	0.1152	-1751.54421722488\\
57.5	0.1158	-1796.90131640774\\
57.5	0.1164	-1842.2584155906\\
57.5	0.117	-1887.61551477347\\
57.5	0.1176	-1932.97261395633\\
57.5	0.1182	-1978.32971313919\\
57.5	0.1188	-2023.68681232206\\
57.5	0.1194	-2069.04391150492\\
57.5	0.12	-2114.40101068778\\
57.5	0.1206	-2159.75810987064\\
57.5	0.1212	-2205.11520905351\\
57.5	0.1218	-2250.47230823637\\
57.5	0.1224	-2295.82940741923\\
57.5	0.123	-2341.18650660209\\
57.875	0.093	-61.7403511813682\\
57.875	0.0936	-106.371079499004\\
57.875	0.0942	-151.001807816643\\
57.875	0.0948	-195.632536134282\\
57.875	0.0954	-240.263264451925\\
57.875	0.096	-284.893992769561\\
57.875	0.0966	-329.524721087197\\
57.875	0.0972	-374.15544940484\\
57.875	0.0978	-418.786177722475\\
57.875	0.0984	-463.416906040111\\
57.875	0.099	-508.047634357754\\
57.875	0.0996	-552.678362675386\\
57.875	0.1002	-597.309090993032\\
57.875	0.1008	-641.939819310668\\
57.875	0.1014	-686.570547628304\\
57.875	0.102	-731.201275945943\\
57.875	0.1026	-775.832004263582\\
57.875	0.1032	-820.462732581225\\
57.875	0.1038	-865.093460898861\\
57.875	0.1044	-909.7241892165\\
57.875	0.105	-954.35491753414\\
57.875	0.1056	-998.985645851775\\
57.875	0.1062	-1043.61637416942\\
57.875	0.1068	-1088.24710248706\\
57.875	0.1074	-1132.8778308047\\
57.875	0.108	-1177.50855912233\\
57.875	0.1086	-1222.13928743997\\
57.875	0.1092	-1266.77001575761\\
57.875	0.1098	-1311.40074407525\\
57.875	0.1104	-1356.03147239289\\
57.875	0.111	-1400.66220071053\\
57.875	0.1116	-1445.29292902816\\
57.875	0.1122	-1489.9236573458\\
57.875	0.1128	-1534.55438566344\\
57.875	0.1134	-1579.18511398108\\
57.875	0.114	-1623.81584229872\\
57.875	0.1146	-1668.44657061636\\
57.875	0.1152	-1713.077298934\\
57.875	0.1158	-1757.70802725163\\
57.875	0.1164	-1802.33875556927\\
57.875	0.117	-1846.96948388691\\
57.875	0.1176	-1891.60021220455\\
57.875	0.1182	-1936.23094052219\\
57.875	0.1188	-1980.86166883983\\
57.875	0.1194	-2025.49239715746\\
57.875	0.12	-2070.1231254751\\
57.875	0.1206	-2114.75385379274\\
57.875	0.1212	-2159.38458211038\\
57.875	0.1218	-2204.01531042801\\
57.875	0.1224	-2248.64603874566\\
57.875	0.123	-2293.2767670633\\
58.25	0.093	-50.1491549037819\\
58.25	0.0936	-94.0535123561931\\
58.25	0.0942	-137.957869808608\\
58.25	0.0948	-181.862227261023\\
58.25	0.0954	-225.766584713441\\
58.25	0.096	-269.670942165852\\
58.25	0.0966	-313.575299618267\\
58.25	0.0972	-357.479657070682\\
58.25	0.0978	-401.384014523093\\
58.25	0.0984	-445.288371975505\\
58.25	0.099	-489.192729427927\\
58.25	0.0996	-533.097086880334\\
58.25	0.1002	-577.001444332753\\
58.25	0.1008	-620.905801785168\\
58.25	0.1014	-664.810159237579\\
58.25	0.102	-708.714516689994\\
58.25	0.1026	-752.618874142405\\
58.25	0.1032	-796.523231594827\\
58.25	0.1038	-840.427589047238\\
58.25	0.1044	-884.331946499653\\
58.25	0.105	-928.236303952064\\
58.25	0.1056	-972.140661404479\\
58.25	0.1062	-1016.0450188569\\
58.25	0.1068	-1059.94937630931\\
58.25	0.1074	-1103.85373376172\\
58.25	0.108	-1147.75809121414\\
58.25	0.1086	-1191.66244866655\\
58.25	0.1092	-1235.56680611896\\
58.25	0.1098	-1279.47116357138\\
58.25	0.1104	-1323.3755210238\\
58.25	0.111	-1367.27987847621\\
58.25	0.1116	-1411.18423592862\\
58.25	0.1122	-1455.08859338104\\
58.25	0.1128	-1498.99295083345\\
58.25	0.1134	-1542.89730828586\\
58.25	0.114	-1586.80166573828\\
58.25	0.1146	-1630.70602319069\\
58.25	0.1152	-1674.61038064311\\
58.25	0.1158	-1718.51473809552\\
58.25	0.1164	-1762.41909554794\\
58.25	0.117	-1806.32345300035\\
58.25	0.1176	-1850.22781045276\\
58.25	0.1182	-1894.13216790518\\
58.25	0.1188	-1938.03652535759\\
58.25	0.1194	-1981.94088281001\\
58.25	0.12	-2025.84524026242\\
58.25	0.1206	-2069.74959771484\\
58.25	0.1212	-2113.65395516725\\
58.25	0.1218	-2157.55831261966\\
58.25	0.1224	-2201.46267007208\\
58.25	0.123	-2245.36702752449\\
58.625	0.093	-38.5579586261993\\
58.625	0.0936	-81.735945213386\\
58.625	0.0942	-124.913931800573\\
58.625	0.0948	-168.091918387763\\
58.625	0.0954	-211.269904974961\\
58.625	0.096	-254.447891562148\\
58.625	0.0966	-297.625878149338\\
58.625	0.0972	-340.803864736528\\
58.625	0.0978	-383.981851323715\\
58.625	0.0984	-427.159837910905\\
58.625	0.099	-470.3378244981\\
58.625	0.0996	-513.515811085283\\
58.625	0.1002	-556.693797672477\\
58.625	0.1008	-599.871784259667\\
58.625	0.1014	-643.049770846854\\
58.625	0.102	-686.227757434044\\
58.625	0.1026	-729.405744021235\\
58.625	0.1032	-772.583730608429\\
58.625	0.1038	-815.761717195615\\
58.625	0.1044	-858.939703782809\\
58.625	0.105	-902.117690369996\\
58.625	0.1056	-945.295676957183\\
58.625	0.1062	-988.473663544381\\
58.625	0.1068	-1031.65165013157\\
58.625	0.1074	-1074.82963671876\\
58.625	0.108	-1118.00762330595\\
58.625	0.1086	-1161.18560989313\\
58.625	0.1092	-1204.36359648033\\
58.625	0.1098	-1247.54158306751\\
58.625	0.1104	-1290.71956965471\\
58.625	0.111	-1333.8975562419\\
58.625	0.1116	-1377.07554282909\\
58.625	0.1122	-1420.25352941628\\
58.625	0.1128	-1463.43151600346\\
58.625	0.1134	-1506.60950259065\\
58.625	0.114	-1549.78748917785\\
58.625	0.1146	-1592.96547576504\\
58.625	0.1152	-1636.14346235223\\
58.625	0.1158	-1679.32144893942\\
58.625	0.1164	-1722.4994355266\\
58.625	0.117	-1765.67742211379\\
58.625	0.1176	-1808.85540870098\\
58.625	0.1182	-1852.03339528818\\
58.625	0.1188	-1895.21138187536\\
58.625	0.1194	-1938.38936846256\\
58.625	0.12	-1981.56735504974\\
58.625	0.1206	-2024.74534163693\\
58.625	0.1212	-2067.92332822413\\
58.625	0.1218	-2111.10131481131\\
58.625	0.1224	-2154.27930139851\\
58.625	0.123	-2197.4572879857\\
59	0.093	-26.966762348613\\
59	0.0936	-69.4183780705789\\
59	0.0942	-111.869993792541\\
59	0.0948	-154.321609514507\\
59	0.0954	-196.77322523648\\
59	0.096	-239.224840958443\\
59	0.0966	-281.676456680409\\
59	0.0972	-324.128072402375\\
59	0.0978	-366.579688124337\\
59	0.0984	-409.031303846303\\
59	0.099	-451.482919568272\\
59	0.0996	-493.934535290231\\
59	0.1002	-536.386151012204\\
59	0.1008	-578.837766734167\\
59	0.1014	-621.289382456129\\
59	0.102	-663.740998178098\\
59	0.1026	-706.192613900061\\
59	0.1032	-748.64422962203\\
59	0.1038	-791.095845343996\\
59	0.1044	-833.547461065962\\
59	0.105	-875.999076787924\\
59	0.1056	-918.45069250989\\
59	0.1062	-960.90230823186\\
59	0.1068	-1003.35392395383\\
59	0.1074	-1045.80553967579\\
59	0.108	-1088.25715539775\\
59	0.1086	-1130.70877111972\\
59	0.1092	-1173.16038684169\\
59	0.1098	-1215.61200256365\\
59	0.1104	-1258.06361828562\\
59	0.111	-1300.51523400758\\
59	0.1116	-1342.96684972955\\
59	0.1122	-1385.41846545152\\
59	0.1128	-1427.87008117348\\
59	0.1134	-1470.32169689544\\
59	0.114	-1512.77331261741\\
59	0.1146	-1555.22492833938\\
59	0.1152	-1597.67654406135\\
59	0.1158	-1640.12815978331\\
59	0.1164	-1682.57977550527\\
59	0.117	-1725.03139122724\\
59	0.1176	-1767.4830069492\\
59	0.1182	-1809.93462267118\\
59	0.1188	-1852.38623839314\\
59	0.1194	-1894.8378541151\\
59	0.12	-1937.28946983707\\
59	0.1206	-1979.74108555903\\
59	0.1212	-2022.192701281\\
59	0.1218	-2064.64431700296\\
59	0.1224	-2107.09593272493\\
59	0.123	-2149.5475484469\\
59.375	0.093	-15.3755660710303\\
59.375	0.0936	-57.1008109277682\\
59.375	0.0942	-98.8260557845097\\
59.375	0.0948	-140.551300641251\\
59.375	0.0954	-182.276545498\\
59.375	0.096	-224.001790354738\\
59.375	0.0966	-265.727035211479\\
59.375	0.0972	-307.452280068221\\
59.375	0.0978	-349.177524924959\\
59.375	0.0984	-390.9027697817\\
59.375	0.099	-432.628014638445\\
59.375	0.0996	-474.353259495179\\
59.375	0.1002	-516.078504351928\\
59.375	0.1008	-557.803749208666\\
59.375	0.1014	-599.528994065407\\
59.375	0.102	-641.254238922149\\
59.375	0.1026	-682.979483778887\\
59.375	0.1032	-724.704728635636\\
59.375	0.1038	-766.429973492373\\
59.375	0.1044	-808.155218349119\\
59.375	0.105	-849.880463205856\\
59.375	0.1056	-891.605708062598\\
59.375	0.1062	-933.330952919343\\
59.375	0.1068	-975.056197776081\\
59.375	0.1074	-1016.78144263283\\
59.375	0.108	-1058.50668748956\\
59.375	0.1086	-1100.23193234631\\
59.375	0.1092	-1141.95717720305\\
59.375	0.1098	-1183.68242205978\\
59.375	0.1104	-1225.40766691653\\
59.375	0.111	-1267.13291177327\\
59.375	0.1116	-1308.85815663001\\
59.375	0.1122	-1350.58340148675\\
59.375	0.1128	-1392.30864634349\\
59.375	0.1134	-1434.03389120023\\
59.375	0.114	-1475.75913605698\\
59.375	0.1146	-1517.48438091372\\
59.375	0.1152	-1559.20962577046\\
59.375	0.1158	-1600.9348706272\\
59.375	0.1164	-1642.66011548394\\
59.375	0.117	-1684.38536034068\\
59.375	0.1176	-1726.11060519742\\
59.375	0.1182	-1767.83585005417\\
59.375	0.1188	-1809.56109491091\\
59.375	0.1194	-1851.28633976765\\
59.375	0.12	-1893.01158462439\\
59.375	0.1206	-1934.73682948113\\
59.375	0.1212	-1976.46207433788\\
59.375	0.1218	-2018.18731919461\\
59.375	0.1224	-2059.91256405136\\
59.375	0.123	-2101.6378089081\\
59.75	0.093	-3.78436979344406\\
59.75	0.0936	-44.7832437849574\\
59.75	0.0942	-85.7821177764745\\
59.75	0.0948	-126.780991767992\\
59.75	0.0954	-167.779865759516\\
59.75	0.096	-208.778739751029\\
59.75	0.0966	-249.777613742546\\
59.75	0.0972	-290.776487734063\\
59.75	0.0978	-331.775361725577\\
59.75	0.0984	-372.774235717094\\
59.75	0.099	-413.773109708614\\
59.75	0.0996	-454.771983700128\\
59.75	0.1002	-495.770857691648\\
59.75	0.1008	-536.769731683165\\
59.75	0.1014	-577.768605674679\\
59.75	0.102	-618.767479666196\\
59.75	0.1026	-659.766353657713\\
59.75	0.1032	-700.765227649237\\
59.75	0.1038	-741.764101640751\\
59.75	0.1044	-782.762975632268\\
59.75	0.105	-823.761849623785\\
59.75	0.1056	-864.760723615298\\
59.75	0.1062	-905.759597606822\\
59.75	0.1068	-946.758471598336\\
59.75	0.1074	-987.757345589856\\
59.75	0.108	-1028.75621958137\\
59.75	0.1086	-1069.75509357289\\
59.75	0.1092	-1110.7539675644\\
59.75	0.1098	-1151.75284155592\\
59.75	0.1104	-1192.75171554744\\
59.75	0.111	-1233.75058953895\\
59.75	0.1116	-1274.74946353047\\
59.75	0.1122	-1315.74833752199\\
59.75	0.1128	-1356.74721151351\\
59.75	0.1134	-1397.74608550502\\
59.75	0.114	-1438.74495949654\\
59.75	0.1146	-1479.74383348806\\
59.75	0.1152	-1520.74270747957\\
59.75	0.1158	-1561.74158147109\\
59.75	0.1164	-1602.7404554626\\
59.75	0.117	-1643.73932945413\\
59.75	0.1176	-1684.73820344564\\
59.75	0.1182	-1725.73707743716\\
59.75	0.1188	-1766.73595142868\\
59.75	0.1194	-1807.73482542019\\
59.75	0.12	-1848.73369941171\\
59.75	0.1206	-1889.73257340322\\
59.75	0.1212	-1930.73144739475\\
59.75	0.1218	-1971.73032138626\\
59.75	0.1224	-2012.72919537778\\
59.75	0.123	-2053.7280693693\\
60.125	0.093	7.80682648413858\\
60.125	0.0936	-32.4656766421504\\
60.125	0.0942	-72.7381797684429\\
60.125	0.0948	-113.010682894736\\
60.125	0.0954	-153.283186021035\\
60.125	0.096	-193.555689147324\\
60.125	0.0966	-233.828192273617\\
60.125	0.0972	-274.100695399909\\
60.125	0.0978	-314.373198526202\\
60.125	0.0984	-354.645701652491\\
60.125	0.099	-394.918204778791\\
60.125	0.0996	-435.190707905076\\
60.125	0.1002	-475.463211031372\\
60.125	0.1008	-515.735714157665\\
60.125	0.1014	-556.008217283957\\
60.125	0.102	-596.28072041025\\
60.125	0.1026	-636.553223536539\\
60.125	0.1032	-676.825726662839\\
60.125	0.1038	-717.098229789128\\
60.125	0.1044	-757.370732915424\\
60.125	0.105	-797.643236041713\\
60.125	0.1056	-837.915739168006\\
60.125	0.1062	-878.188242294302\\
60.125	0.1068	-918.460745420594\\
60.125	0.1074	-958.733248546887\\
60.125	0.108	-999.005751673179\\
60.125	0.1086	-1039.27825479947\\
60.125	0.1092	-1079.55075792576\\
60.125	0.1098	-1119.82326105205\\
60.125	0.1104	-1160.09576417835\\
60.125	0.111	-1200.36826730464\\
60.125	0.1116	-1240.64077043093\\
60.125	0.1122	-1280.91327355723\\
60.125	0.1128	-1321.18577668352\\
60.125	0.1134	-1361.45827980981\\
60.125	0.114	-1401.73078293611\\
60.125	0.1146	-1442.0032860624\\
60.125	0.1152	-1482.27578918869\\
60.125	0.1158	-1522.54829231498\\
60.125	0.1164	-1562.82079544128\\
60.125	0.117	-1603.09329856757\\
60.125	0.1176	-1643.36580169386\\
60.125	0.1182	-1683.63830482016\\
60.125	0.1188	-1723.91080794645\\
60.125	0.1194	-1764.18331107274\\
60.125	0.12	-1804.45581419903\\
60.125	0.1206	-1844.72831732532\\
60.125	0.1212	-1885.00082045162\\
60.125	0.1218	-1925.27332357791\\
60.125	0.1224	-1965.5458267042\\
60.125	0.123	-2005.8183298305\\
60.5	0.093	19.3980227617285\\
60.5	0.0936	-20.1481094993433\\
60.5	0.0942	-59.6942417604114\\
60.5	0.0948	-99.2403740214795\\
60.5	0.0954	-138.786506282548\\
60.5	0.096	-178.332638543616\\
60.5	0.0966	-217.878770804688\\
60.5	0.0972	-257.424903065752\\
60.5	0.0978	-296.971035326824\\
60.5	0.0984	-336.517167587888\\
60.5	0.099	-376.06329984896\\
60.5	0.0996	-415.609432110024\\
60.5	0.1002	-455.155564371096\\
60.5	0.1008	-494.701696632164\\
60.5	0.1014	-534.247828893233\\
60.5	0.102	-573.793961154301\\
60.5	0.1026	-613.340093415369\\
60.5	0.1032	-652.886225676437\\
60.5	0.1038	-692.432357937505\\
60.5	0.1044	-731.978490198573\\
60.5	0.105	-771.524622459645\\
60.5	0.1056	-811.070754720709\\
60.5	0.1062	-850.616886981781\\
60.5	0.1068	-890.163019242849\\
60.5	0.1074	-929.709151503917\\
60.5	0.108	-969.255283764986\\
60.5	0.1086	-1008.80141602605\\
60.5	0.1092	-1048.34754828713\\
60.5	0.1098	-1087.89368054819\\
60.5	0.1104	-1127.43981280926\\
60.5	0.111	-1166.98594507033\\
60.5	0.1116	-1206.5320773314\\
60.5	0.1122	-1246.07820959246\\
60.5	0.1128	-1285.62434185353\\
60.5	0.1134	-1325.1704741146\\
60.5	0.114	-1364.71660637567\\
60.5	0.1146	-1404.26273863674\\
60.5	0.1152	-1443.80887089781\\
60.5	0.1158	-1483.35500315888\\
60.5	0.1164	-1522.90113541994\\
60.5	0.117	-1562.44726768101\\
60.5	0.1176	-1601.99339994208\\
60.5	0.1182	-1641.53953220315\\
60.5	0.1188	-1681.08566446422\\
60.5	0.1194	-1720.63179672529\\
60.5	0.12	-1760.17792898636\\
60.5	0.1206	-1799.72406124742\\
60.5	0.1212	-1839.27019350849\\
60.5	0.1218	-1878.81632576956\\
60.5	0.1224	-1918.36245803063\\
60.5	0.123	-1957.9085902917\\
60.875	0.093	30.9892190393148\\
60.875	0.0936	-7.83054235653253\\
60.875	0.0942	-46.6503037523762\\
60.875	0.0948	-85.4700651482199\\
60.875	0.0954	-124.289826544067\\
60.875	0.096	-163.109587939907\\
60.875	0.0966	-201.929349335755\\
60.875	0.0972	-240.749110731595\\
60.875	0.0978	-279.568872127442\\
60.875	0.0984	-318.388633523282\\
60.875	0.099	-357.208394919129\\
60.875	0.0996	-396.028156314969\\
60.875	0.1002	-434.847917710817\\
60.875	0.1008	-473.667679106664\\
60.875	0.1014	-512.487440502504\\
60.875	0.102	-551.307201898351\\
60.875	0.1026	-590.126963294191\\
60.875	0.1032	-628.946724690039\\
60.875	0.1038	-667.766486085882\\
60.875	0.1044	-706.586247481726\\
60.875	0.105	-745.406008877573\\
60.875	0.1056	-784.225770273413\\
60.875	0.1062	-823.045531669261\\
60.875	0.1068	-861.865293065101\\
60.875	0.1074	-900.685054460948\\
60.875	0.108	-939.504815856792\\
60.875	0.1086	-978.324577252635\\
60.875	0.1092	-1017.14433864848\\
60.875	0.1098	-1055.96410004432\\
60.875	0.1104	-1094.78386144017\\
60.875	0.111	-1133.60362283601\\
60.875	0.1116	-1172.42338423186\\
60.875	0.1122	-1211.2431456277\\
60.875	0.1128	-1250.06290702354\\
60.875	0.1134	-1288.88266841939\\
60.875	0.114	-1327.70242981523\\
60.875	0.1146	-1366.52219121108\\
60.875	0.1152	-1405.34195260692\\
60.875	0.1158	-1444.16171400277\\
60.875	0.1164	-1482.98147539861\\
60.875	0.117	-1521.80123679445\\
60.875	0.1176	-1560.6209981903\\
60.875	0.1182	-1599.44075958614\\
60.875	0.1188	-1638.26052098199\\
60.875	0.1194	-1677.08028237783\\
60.875	0.12	-1715.90004377368\\
60.875	0.1206	-1754.71980516952\\
60.875	0.1212	-1793.53956656536\\
60.875	0.1218	-1832.3593279612\\
60.875	0.1224	-1871.17908935705\\
60.875	0.123	-1909.99885075289\\
61.25	0.093	42.5804153168974\\
61.25	0.0936	4.48702478627456\\
61.25	0.0942	-33.606365744341\\
61.25	0.0948	-71.6997562749639\\
61.25	0.0954	-109.793146805583\\
61.25	0.096	-147.886537336202\\
61.25	0.0966	-185.979927866825\\
61.25	0.0972	-224.073318397441\\
61.25	0.0978	-262.166708928064\\
61.25	0.0984	-300.260099458683\\
61.25	0.099	-338.353489989302\\
61.25	0.0996	-376.446880519921\\
61.25	0.1002	-414.54027105054\\
61.25	0.1008	-452.633661581163\\
61.25	0.1014	-490.727052111783\\
61.25	0.102	-528.820442642402\\
61.25	0.1026	-566.913833173021\\
61.25	0.1032	-605.00722370364\\
61.25	0.1038	-643.100614234259\\
61.25	0.1044	-681.194004764879\\
61.25	0.105	-719.287395295501\\
61.25	0.1056	-757.380785826121\\
61.25	0.1062	-795.474176356744\\
61.25	0.1068	-833.567566887359\\
61.25	0.1074	-871.660957417982\\
61.25	0.108	-909.754347948598\\
61.25	0.1086	-947.84773847922\\
61.25	0.1092	-985.94112900984\\
61.25	0.1098	-1024.03451954046\\
61.25	0.1104	-1062.12791007108\\
61.25	0.111	-1100.2213006017\\
61.25	0.1116	-1138.31469113232\\
61.25	0.1122	-1176.40808166294\\
61.25	0.1128	-1214.50147219356\\
61.25	0.1134	-1252.59486272418\\
61.25	0.114	-1290.6882532548\\
61.25	0.1146	-1328.78164378542\\
61.25	0.1152	-1366.87503431604\\
61.25	0.1158	-1404.96842484666\\
61.25	0.1164	-1443.06181537728\\
61.25	0.117	-1481.1552059079\\
61.25	0.1176	-1519.24859643852\\
61.25	0.1182	-1557.34198696914\\
61.25	0.1188	-1595.43537749976\\
61.25	0.1194	-1633.52876803038\\
61.25	0.12	-1671.622158561\\
61.25	0.1206	-1709.71554909162\\
61.25	0.1212	-1747.80893962224\\
61.25	0.1218	-1785.90233015285\\
61.25	0.1224	-1823.99572068348\\
61.25	0.123	-1862.0891112141\\
61.625	0.093	54.1716115944801\\
61.625	0.0936	16.8045919290853\\
61.625	0.0942	-20.5624277363095\\
61.625	0.0948	-57.9294474017079\\
61.625	0.0954	-95.2964670671026\\
61.625	0.096	-132.663486732497\\
61.625	0.0966	-170.030506397896\\
61.625	0.0972	-207.397526063287\\
61.625	0.0978	-244.764545728685\\
61.625	0.0984	-282.13156539408\\
61.625	0.099	-319.498585059475\\
61.625	0.0996	-356.86560472487\\
61.625	0.1002	-394.232624390268\\
61.625	0.1008	-431.599644055663\\
61.625	0.1014	-468.966663721058\\
61.625	0.102	-506.333683386452\\
61.625	0.1026	-543.700703051847\\
61.625	0.1032	-581.067722717245\\
61.625	0.1038	-618.43474238264\\
61.625	0.1044	-655.801762048035\\
61.625	0.105	-693.168781713433\\
61.625	0.1056	-730.535801378825\\
61.625	0.1062	-767.902821044223\\
61.625	0.1068	-805.269840709618\\
61.625	0.1074	-842.636860375012\\
61.625	0.108	-880.003880040407\\
61.625	0.1086	-917.370899705806\\
61.625	0.1092	-954.7379193712\\
61.625	0.1098	-992.104939036595\\
61.625	0.1104	-1029.47195870199\\
61.625	0.111	-1066.83897836738\\
61.625	0.1116	-1104.20599803278\\
61.625	0.1122	-1141.57301769818\\
61.625	0.1128	-1178.94003736357\\
61.625	0.1134	-1216.30705702897\\
61.625	0.114	-1253.67407669437\\
61.625	0.1146	-1291.04109635976\\
61.625	0.1152	-1328.40811602516\\
61.625	0.1158	-1365.77513569055\\
61.625	0.1164	-1403.14215535594\\
61.625	0.117	-1440.50917502134\\
61.625	0.1176	-1477.87619468674\\
61.625	0.1182	-1515.24321435213\\
61.625	0.1188	-1552.61023401753\\
61.625	0.1194	-1589.97725368292\\
61.625	0.12	-1627.34427334832\\
61.625	0.1206	-1664.71129301372\\
61.625	0.1212	-1702.07831267911\\
61.625	0.1218	-1739.44533234451\\
61.625	0.1224	-1776.8123520099\\
61.625	0.123	-1814.17937167529\\
62	0.093	65.7628078720663\\
62	0.0936	29.1221590718924\\
62	0.0942	-7.51848972827793\\
62	0.0948	-44.1591385284482\\
62	0.0954	-80.7997873286222\\
62	0.096	-117.440436128792\\
62	0.0966	-154.081084928966\\
62	0.0972	-190.721733729133\\
62	0.0978	-227.362382529307\\
62	0.0984	-264.003031329477\\
62	0.099	-300.643680129648\\
62	0.0996	-337.284328929818\\
62	0.1002	-373.924977729992\\
62	0.1008	-410.565626530162\\
62	0.1014	-447.206275330333\\
62	0.102	-483.846924130507\\
62	0.1026	-520.487572930677\\
62	0.1032	-557.128221730847\\
62	0.1038	-593.768870531017\\
62	0.1044	-630.409519331191\\
62	0.105	-667.050168131362\\
62	0.1056	-703.690816931532\\
62	0.1062	-740.331465731706\\
62	0.1068	-776.972114531876\\
62	0.1074	-813.612763332047\\
62	0.108	-850.253412132217\\
62	0.1086	-886.894060932391\\
62	0.1092	-923.534709732561\\
62	0.1098	-960.175358532731\\
62	0.1104	-996.816007332905\\
62	0.111	-1033.45665613307\\
62	0.1116	-1070.09730493325\\
62	0.1122	-1106.73795373342\\
62	0.1128	-1143.37860253359\\
62	0.1134	-1180.01925133376\\
62	0.114	-1216.65990013393\\
62	0.1146	-1253.3005489341\\
62	0.1152	-1289.94119773427\\
62	0.1158	-1326.58184653445\\
62	0.1164	-1363.22249533462\\
62	0.117	-1399.86314413479\\
62	0.1176	-1436.50379293496\\
62	0.1182	-1473.14444173513\\
62	0.1188	-1509.7850905353\\
62	0.1194	-1546.42573933547\\
62	0.12	-1583.06638813564\\
62	0.1206	-1619.70703693582\\
62	0.1212	-1656.34768573599\\
62	0.1218	-1692.98833453616\\
62	0.1224	-1729.62898333633\\
62	0.123	-1766.2696321365\\
62.375	0.093	77.3540041496526\\
62.375	0.0936	41.4397262147031\\
62.375	0.0942	5.52544827975726\\
62.375	0.0948	-30.3888296551886\\
62.375	0.0954	-66.3031075901381\\
62.375	0.096	-102.217385525084\\
62.375	0.0966	-138.131663460033\\
62.375	0.0972	-174.045941394976\\
62.375	0.0978	-209.960219329925\\
62.375	0.0984	-245.874497264871\\
62.375	0.099	-281.788775199821\\
62.375	0.0996	-317.703053134763\\
62.375	0.1002	-353.617331069712\\
62.375	0.1008	-389.531609004662\\
62.375	0.1014	-425.445886939608\\
62.375	0.102	-461.360164874553\\
62.375	0.1026	-497.274442809499\\
62.375	0.1032	-533.188720744449\\
62.375	0.1038	-569.102998679395\\
62.375	0.1044	-605.017276614341\\
62.375	0.105	-640.93155454929\\
62.375	0.1056	-676.845832484236\\
62.375	0.1062	-712.760110419185\\
62.375	0.1068	-748.674388354128\\
62.375	0.1074	-784.588666289077\\
62.375	0.108	-820.502944224023\\
62.375	0.1086	-856.417222158972\\
62.375	0.1092	-892.331500093918\\
62.375	0.1098	-928.245778028864\\
62.375	0.1104	-964.160055963814\\
62.375	0.111	-1000.07433389876\\
62.375	0.1116	-1035.98861183371\\
62.375	0.1122	-1071.90288976865\\
62.375	0.1128	-1107.8171677036\\
62.375	0.1134	-1143.73144563854\\
62.375	0.114	-1179.64572357349\\
62.375	0.1146	-1215.56000150844\\
62.375	0.1152	-1251.47427944339\\
62.375	0.1158	-1287.38855737833\\
62.375	0.1164	-1323.30283531328\\
62.375	0.117	-1359.21711324823\\
62.375	0.1176	-1395.13139118317\\
62.375	0.1182	-1431.04566911812\\
62.375	0.1188	-1466.95994705307\\
62.375	0.1194	-1502.87422498802\\
62.375	0.12	-1538.78850292296\\
62.375	0.1206	-1574.70278085791\\
62.375	0.1212	-1610.61705879286\\
62.375	0.1218	-1646.5313367278\\
62.375	0.1224	-1682.44561466275\\
62.375	0.123	-1718.35989259769\\
62.75	0.093	88.9452004272352\\
62.75	0.0936	53.7572933575102\\
62.75	0.0942	18.5693862877888\\
62.75	0.0948	-16.6185207819326\\
62.75	0.0954	-51.8064278516576\\
62.75	0.096	-86.994334921379\\
62.75	0.0966	-122.182241991104\\
62.75	0.0972	-157.370149060822\\
62.75	0.0978	-192.558056130547\\
62.75	0.0984	-227.745963200268\\
62.75	0.099	-262.933870269993\\
62.75	0.0996	-298.121777339715\\
62.75	0.1002	-333.309684409436\\
62.75	0.1008	-368.497591479161\\
62.75	0.1014	-403.685498548883\\
62.75	0.102	-438.873405618608\\
62.75	0.1026	-474.061312688325\\
62.75	0.1032	-509.24921975805\\
62.75	0.1038	-544.437126827772\\
62.75	0.1044	-579.625033897497\\
62.75	0.105	-614.812940967222\\
62.75	0.1056	-650.00084803694\\
62.75	0.1062	-685.188755106665\\
62.75	0.1068	-720.376662176386\\
62.75	0.1074	-755.564569246111\\
62.75	0.108	-790.752476315833\\
62.75	0.1086	-825.940383385554\\
62.75	0.1092	-861.128290455279\\
62.75	0.1098	-896.316197525\\
62.75	0.1104	-931.504104594726\\
62.75	0.111	-966.692011664443\\
62.75	0.1116	-1001.87991873417\\
62.75	0.1122	-1037.06782580389\\
62.75	0.1128	-1072.25573287361\\
62.75	0.1134	-1107.44363994334\\
62.75	0.114	-1142.63154701306\\
62.75	0.1146	-1177.81945408278\\
62.75	0.1152	-1213.0073611525\\
62.75	0.1158	-1248.19526822223\\
62.75	0.1164	-1283.38317529195\\
62.75	0.117	-1318.57108236167\\
62.75	0.1176	-1353.75898943139\\
62.75	0.1182	-1388.94689650112\\
62.75	0.1188	-1424.13480357084\\
62.75	0.1194	-1459.32271064056\\
62.75	0.12	-1494.51061771029\\
62.75	0.1206	-1529.69852478001\\
62.75	0.1212	-1564.88643184973\\
62.75	0.1218	-1600.07433891945\\
62.75	0.1224	-1635.26224598918\\
62.75	0.123	-1670.4501530589\\
63.125	0.093	100.536396704818\\
63.125	0.0936	66.0748605003173\\
63.125	0.0942	31.613324295824\\
63.125	0.0948	-2.8482119086766\\
63.125	0.0954	-37.3097481131772\\
63.125	0.096	-71.7712843176741\\
63.125	0.0966	-106.232820522175\\
63.125	0.0972	-140.694356726668\\
63.125	0.0978	-175.155892931169\\
63.125	0.0984	-209.617429135666\\
63.125	0.099	-244.078965340166\\
63.125	0.0996	-278.540501544663\\
63.125	0.1002	-313.002037749164\\
63.125	0.1008	-347.463573953661\\
63.125	0.1014	-381.925110158158\\
63.125	0.102	-416.386646362658\\
63.125	0.1026	-450.848182567155\\
63.125	0.1032	-485.309718771656\\
63.125	0.1038	-519.771254976153\\
63.125	0.1044	-554.23279118065\\
63.125	0.105	-588.69432738515\\
63.125	0.1056	-623.155863589647\\
63.125	0.1062	-657.617399794148\\
63.125	0.1068	-692.078935998645\\
63.125	0.1074	-726.540472203142\\
63.125	0.108	-761.002008407639\\
63.125	0.1086	-795.463544612139\\
63.125	0.1092	-829.92508081664\\
63.125	0.1098	-864.386617021137\\
63.125	0.1104	-898.848153225637\\
63.125	0.111	-933.309689430131\\
63.125	0.1116	-967.771225634631\\
63.125	0.1122	-1002.23276183913\\
63.125	0.1128	-1036.69429804363\\
63.125	0.1134	-1071.15583424813\\
63.125	0.114	-1105.61737045263\\
63.125	0.1146	-1140.07890665713\\
63.125	0.1152	-1174.54044286162\\
63.125	0.1158	-1209.00197906612\\
63.125	0.1164	-1243.46351527062\\
63.125	0.117	-1277.92505147512\\
63.125	0.1176	-1312.38658767961\\
63.125	0.1182	-1346.84812388411\\
63.125	0.1188	-1381.30966008861\\
63.125	0.1194	-1415.77119629311\\
63.125	0.12	-1450.23273249761\\
63.125	0.1206	-1484.69426870211\\
63.125	0.1212	-1519.1558049066\\
63.125	0.1218	-1553.6173411111\\
63.125	0.1224	-1588.0788773156\\
63.125	0.123	-1622.5404135201\\
63.5	0.093	112.127592982404\\
63.5	0.0936	78.392427643128\\
63.5	0.0942	44.6572623038555\\
63.5	0.0948	10.9220969645794\\
63.5	0.0954	-22.8130683746967\\
63.5	0.096	-56.5482337139692\\
63.5	0.0966	-90.2833990532454\\
63.5	0.0972	-124.018564392514\\
63.5	0.0978	-157.75372973179\\
63.5	0.0984	-191.488895071063\\
63.5	0.099	-225.224060410339\\
63.5	0.0996	-258.959225749612\\
63.5	0.1002	-292.694391088888\\
63.5	0.1008	-326.429556428164\\
63.5	0.1014	-360.164721767436\\
63.5	0.102	-393.899887106709\\
63.5	0.1026	-427.635052445981\\
63.5	0.1032	-461.370217785257\\
63.5	0.1038	-495.10538312453\\
63.5	0.1044	-528.840548463806\\
63.5	0.105	-562.575713803082\\
63.5	0.1056	-596.310879142355\\
63.5	0.1062	-630.046044481631\\
63.5	0.1068	-663.7812098209\\
63.5	0.1074	-697.516375160176\\
63.5	0.108	-731.251540499448\\
63.5	0.1086	-764.986705838724\\
63.5	0.1092	-798.721871178001\\
63.5	0.1098	-832.457036517273\\
63.5	0.1104	-866.192201856549\\
63.5	0.111	-899.927367195818\\
63.5	0.1116	-933.662532535094\\
63.5	0.1122	-967.397697874367\\
63.5	0.1128	-1001.13286321364\\
63.5	0.1134	-1034.86802855292\\
63.5	0.114	-1068.60319389219\\
63.5	0.1146	-1102.33835923147\\
63.5	0.1152	-1136.07352457074\\
63.5	0.1158	-1169.80868991001\\
63.5	0.1164	-1203.54385524929\\
63.5	0.117	-1237.27902058856\\
63.5	0.1176	-1271.01418592783\\
63.5	0.1182	-1304.74935126711\\
63.5	0.1188	-1338.48451660639\\
63.5	0.1194	-1372.21968194565\\
63.5	0.12	-1405.95484728493\\
63.5	0.1206	-1439.6900126242\\
63.5	0.1212	-1473.42517796348\\
63.5	0.1218	-1507.16034330275\\
63.5	0.1224	-1540.89550864203\\
63.5	0.123	-1574.6306739813\\
63.875	0.093	123.71878925999\\
63.875	0.0936	90.7099947859388\\
63.875	0.0942	57.7012003118907\\
63.875	0.0948	24.692405837839\\
63.875	0.0954	-8.31638863621265\\
63.875	0.096	-41.3251831102607\\
63.875	0.0966	-74.3339775843124\\
63.875	0.0972	-107.34277205836\\
63.875	0.0978	-140.351566532412\\
63.875	0.0984	-173.36036100646\\
63.875	0.099	-206.369155480508\\
63.875	0.0996	-239.377949954556\\
63.875	0.1002	-272.386744428608\\
63.875	0.1008	-305.39553890266\\
63.875	0.1014	-338.404333376708\\
63.875	0.102	-371.413127850759\\
63.875	0.1026	-404.421922324807\\
63.875	0.1032	-437.430716798859\\
63.875	0.1038	-470.439511272907\\
63.875	0.1044	-503.448305746955\\
63.875	0.105	-536.457100221007\\
63.875	0.1056	-569.465894695055\\
63.875	0.1062	-602.474689169107\\
63.875	0.1068	-635.483483643155\\
63.875	0.1074	-668.492278117206\\
63.875	0.108	-701.501072591254\\
63.875	0.1086	-734.509867065306\\
63.875	0.1092	-767.518661539358\\
63.875	0.1098	-800.527456013406\\
63.875	0.1104	-833.536250487457\\
63.875	0.111	-866.545044961505\\
63.875	0.1116	-899.553839435557\\
63.875	0.1122	-932.562633909602\\
63.875	0.1128	-965.571428383653\\
63.875	0.1134	-998.580222857701\\
63.875	0.114	-1031.58901733175\\
63.875	0.1146	-1064.5978118058\\
63.875	0.1152	-1097.60660627985\\
63.875	0.1158	-1130.6154007539\\
63.875	0.1164	-1163.62419522795\\
63.875	0.117	-1196.632989702\\
63.875	0.1176	-1229.64178417605\\
63.875	0.1182	-1262.6505786501\\
63.875	0.1188	-1295.65937312415\\
63.875	0.1194	-1328.6681675982\\
63.875	0.12	-1361.67696207225\\
63.875	0.1206	-1394.6857565463\\
63.875	0.1212	-1427.69455102035\\
63.875	0.1218	-1460.7033454944\\
63.875	0.1224	-1493.71213996845\\
63.875	0.123	-1526.7209344425\\
64.25	0.093	135.309985537573\\
64.25	0.0936	103.027561928746\\
64.25	0.0942	70.7451383199223\\
64.25	0.0948	38.462714711095\\
64.25	0.0954	6.1802911022678\\
64.25	0.096	-26.1021325065558\\
64.25	0.0966	-58.384556115383\\
64.25	0.0972	-90.6669797242066\\
64.25	0.0978	-122.949403333034\\
64.25	0.0984	-155.231826941857\\
64.25	0.099	-187.514250550685\\
64.25	0.0996	-219.796674159505\\
64.25	0.1002	-252.079097768332\\
64.25	0.1008	-284.361521377159\\
64.25	0.1014	-316.643944985983\\
64.25	0.102	-348.92636859481\\
64.25	0.1026	-381.208792203633\\
64.25	0.1032	-413.491215812461\\
64.25	0.1038	-445.773639421284\\
64.25	0.1044	-478.056063030112\\
64.25	0.105	-510.338486638939\\
64.25	0.1056	-542.620910247762\\
64.25	0.1062	-574.90333385659\\
64.25	0.1068	-607.185757465413\\
64.25	0.1074	-639.46818107424\\
64.25	0.108	-671.750604683064\\
64.25	0.1086	-704.033028291891\\
64.25	0.1092	-736.315451900718\\
64.25	0.1098	-768.597875509542\\
64.25	0.1104	-800.880299118369\\
64.25	0.111	-833.162722727193\\
64.25	0.1116	-865.44514633602\\
64.25	0.1122	-897.72756994484\\
64.25	0.1128	-930.009993553667\\
64.25	0.1134	-962.292417162491\\
64.25	0.114	-994.574840771318\\
64.25	0.1146	-1026.85726438015\\
64.25	0.1152	-1059.13968798897\\
64.25	0.1158	-1091.4221115978\\
64.25	0.1164	-1123.70453520662\\
64.25	0.117	-1155.98695881545\\
64.25	0.1176	-1188.26938242427\\
64.25	0.1182	-1220.5518060331\\
64.25	0.1188	-1252.83422964193\\
64.25	0.1194	-1285.11665325075\\
64.25	0.12	-1317.39907685958\\
64.25	0.1206	-1349.6815004684\\
64.25	0.1212	-1381.96392407723\\
64.25	0.1218	-1414.24634768605\\
64.25	0.1224	-1446.52877129487\\
64.25	0.123	-1478.8111949037\\
64.625	0.093	146.901181815156\\
64.625	0.0936	115.345129071553\\
64.625	0.0942	83.7890763279538\\
64.625	0.0948	52.233023584351\\
64.625	0.0954	20.6769708407483\\
64.625	0.096	-10.8790819028509\\
64.625	0.0966	-42.4351346464537\\
64.625	0.0972	-73.9911873900528\\
64.625	0.0978	-105.547240133656\\
64.625	0.0984	-137.103292877255\\
64.625	0.099	-168.659345620857\\
64.625	0.0996	-200.215398364457\\
64.625	0.1002	-231.771451108059\\
64.625	0.1008	-263.327503851662\\
64.625	0.1014	-294.883556595261\\
64.625	0.102	-326.439609338864\\
64.625	0.1026	-357.995662082463\\
64.625	0.1032	-389.551714826066\\
64.625	0.1038	-421.107767569665\\
64.625	0.1044	-452.663820313268\\
64.625	0.105	-484.219873056871\\
64.625	0.1056	-515.77592580047\\
64.625	0.1062	-547.331978544073\\
64.625	0.1068	-578.888031287672\\
64.625	0.1074	-610.444084031275\\
64.625	0.108	-642.000136774874\\
64.625	0.1086	-673.556189518476\\
64.625	0.1092	-705.112242262076\\
64.625	0.1098	-736.668295005675\\
64.625	0.1104	-768.224347749281\\
64.625	0.111	-799.78040049288\\
64.625	0.1116	-831.336453236483\\
64.625	0.1122	-862.892505980079\\
64.625	0.1128	-894.448558723681\\
64.625	0.1134	-926.00461146728\\
64.625	0.114	-957.560664210887\\
64.625	0.1146	-989.11671695449\\
64.625	0.1152	-1020.67276969809\\
64.625	0.1158	-1052.22882244169\\
64.625	0.1164	-1083.78487518529\\
64.625	0.117	-1115.34092792889\\
64.625	0.1176	-1146.89698067249\\
64.625	0.1182	-1178.45303341609\\
64.625	0.1188	-1210.00908615969\\
64.625	0.1194	-1241.56513890329\\
64.625	0.12	-1273.1211916469\\
64.625	0.1206	-1304.6772443905\\
64.625	0.1212	-1336.2332971341\\
64.625	0.1218	-1367.7893498777\\
64.625	0.1224	-1399.3454026213\\
64.625	0.123	-1430.9014553649\\
65	0.093	158.492378092746\\
65	0.0936	127.662696214364\\
65	0.0942	96.833014335989\\
65	0.0948	66.0033324576143\\
65	0.0954	35.1736505792323\\
65	0.096	4.34396870085766\\
65	0.0966	-26.4857131775207\\
65	0.0972	-57.3153950558954\\
65	0.0978	-88.1450769342737\\
65	0.0984	-118.974758812648\\
65	0.099	-149.804440691027\\
65	0.0996	-180.634122569401\\
65	0.1002	-211.46380444778\\
65	0.1008	-242.293486326158\\
65	0.1014	-273.123168204533\\
65	0.102	-303.952850082911\\
65	0.1026	-334.782531961286\\
65	0.1032	-365.612213839664\\
65	0.1038	-396.441895718039\\
65	0.1044	-427.271577596417\\
65	0.105	-458.101259474795\\
65	0.1056	-488.93094135317\\
65	0.1062	-519.760623231548\\
65	0.1068	-550.590305109923\\
65	0.1074	-581.419986988301\\
65	0.108	-612.249668866676\\
65	0.1086	-643.079350745054\\
65	0.1092	-673.909032623433\\
65	0.1098	-704.738714501807\\
65	0.1104	-735.568396380186\\
65	0.111	-766.398078258564\\
65	0.1116	-797.227760136942\\
65	0.1122	-828.057442015313\\
65	0.1128	-858.887123893695\\
65	0.1134	-889.71680577207\\
65	0.114	-920.546487650448\\
65	0.1146	-951.376169528827\\
65	0.1152	-982.205851407201\\
65	0.1158	-1013.03553328558\\
65	0.1164	-1043.86521516395\\
65	0.117	-1074.69489704233\\
65	0.1176	-1105.52457892071\\
65	0.1182	-1136.35426079909\\
65	0.1188	-1167.18394267746\\
65	0.1194	-1198.01362455584\\
65	0.12	-1228.84330643422\\
65	0.1206	-1259.67298831259\\
65	0.1212	-1290.50267019097\\
65	0.1218	-1321.33235206934\\
65	0.1224	-1352.16203394772\\
65	0.123	-1382.9917158261\\
65.375	0.093	170.083574370332\\
65.375	0.0936	139.980263357174\\
65.375	0.0942	109.876952344024\\
65.375	0.0948	79.7736413308703\\
65.375	0.0954	49.6703303177164\\
65.375	0.096	19.5670193045662\\
65.375	0.0966	-10.5362917085877\\
65.375	0.0972	-40.6396027217379\\
65.375	0.0978	-70.7429137348918\\
65.375	0.0984	-100.846224748042\\
65.375	0.099	-130.949535761196\\
65.375	0.0996	-161.052846774346\\
65.375	0.1002	-191.1561577875\\
65.375	0.1008	-221.259468800654\\
65.375	0.1014	-251.362779813804\\
65.375	0.102	-281.466090826958\\
65.375	0.1026	-311.569401840112\\
65.375	0.1032	-341.672712853266\\
65.375	0.1038	-371.776023866416\\
65.375	0.1044	-401.87933487957\\
65.375	0.105	-431.982645892724\\
65.375	0.1056	-462.085956905874\\
65.375	0.1062	-492.189267919028\\
65.375	0.1068	-522.292578932178\\
65.375	0.1074	-552.395889945332\\
65.375	0.108	-582.499200958482\\
65.375	0.1086	-612.602511971636\\
65.375	0.1092	-642.70582298479\\
65.375	0.1098	-672.80913399794\\
65.375	0.1104	-702.912445011098\\
65.375	0.111	-733.015756024248\\
65.375	0.1116	-763.119067037402\\
65.375	0.1122	-793.222378050552\\
65.375	0.1128	-823.325689063706\\
65.375	0.1134	-853.429000076856\\
65.375	0.114	-883.53231109001\\
65.375	0.1146	-913.635622103164\\
65.375	0.1152	-943.738933116314\\
65.375	0.1158	-973.842244129468\\
65.375	0.1164	-1003.94555514262\\
65.375	0.117	-1034.04886615577\\
65.375	0.1176	-1064.15217716892\\
65.375	0.1182	-1094.25548818208\\
65.375	0.1188	-1124.35879919523\\
65.375	0.1194	-1154.46211020838\\
65.375	0.12	-1184.56542122154\\
65.375	0.1206	-1214.66873223469\\
65.375	0.1212	-1244.77204324784\\
65.375	0.1218	-1274.87535426099\\
65.375	0.1224	-1304.97866527415\\
65.375	0.123	-1335.0819762873\\
65.75	0.093	181.674770647915\\
65.75	0.0936	152.297830499985\\
65.75	0.0942	122.920890352059\\
65.75	0.0948	93.5439502041299\\
65.75	0.0954	64.1670100561969\\
65.75	0.096	34.7900699082711\\
65.75	0.0966	5.41312976034169\\
65.75	0.0972	-23.9638103875841\\
65.75	0.0978	-53.3407505355135\\
65.75	0.0984	-82.7176906834393\\
65.75	0.099	-112.094630831369\\
65.75	0.0996	-141.471570979294\\
65.75	0.1002	-170.848511127224\\
65.75	0.1008	-200.225451275157\\
65.75	0.1014	-229.602391423083\\
65.75	0.102	-258.979331571012\\
65.75	0.1026	-288.356271718938\\
65.75	0.1032	-317.733211866867\\
65.75	0.1038	-347.110152014793\\
65.75	0.1044	-376.487092162723\\
65.75	0.105	-405.864032310652\\
65.75	0.1056	-435.240972458581\\
65.75	0.1062	-464.617912606511\\
65.75	0.1068	-493.994852754437\\
65.75	0.1074	-523.371792902366\\
65.75	0.108	-552.748733050292\\
65.75	0.1086	-582.125673198221\\
65.75	0.1092	-611.502613346151\\
65.75	0.1098	-640.879553494076\\
65.75	0.1104	-670.256493642006\\
65.75	0.111	-699.633433789935\\
65.75	0.1116	-729.010373937865\\
65.75	0.1122	-758.38731408579\\
65.75	0.1128	-787.76425423372\\
65.75	0.1134	-817.141194381646\\
65.75	0.114	-846.518134529575\\
65.75	0.1146	-875.895074677504\\
65.75	0.1152	-905.27201482543\\
65.75	0.1158	-934.648954973363\\
65.75	0.1164	-964.025895121289\\
65.75	0.117	-993.402835269218\\
65.75	0.1176	-1022.77977541714\\
65.75	0.1182	-1052.15671556507\\
65.75	0.1188	-1081.533655713\\
65.75	0.1194	-1110.91059586093\\
65.75	0.12	-1140.28753600886\\
65.75	0.1206	-1169.66447615678\\
65.75	0.1212	-1199.04141630472\\
65.75	0.1218	-1228.41835645264\\
65.75	0.1224	-1257.79529660057\\
65.75	0.123	-1287.1722367485\\
66.125	0.093	193.265966925497\\
66.125	0.0936	164.615397642792\\
66.125	0.0942	135.964828360091\\
66.125	0.0948	107.314259077386\\
66.125	0.0954	78.663689794681\\
66.125	0.096	50.013120511976\\
66.125	0.0966	21.362551229271\\
66.125	0.0972	-7.28801805343028\\
66.125	0.0978	-35.9385873361352\\
66.125	0.0984	-64.5891566188366\\
66.125	0.099	-93.2397259015415\\
66.125	0.0996	-121.890295184243\\
66.125	0.1002	-150.540864466951\\
66.125	0.1008	-179.191433749656\\
66.125	0.1014	-207.842003032358\\
66.125	0.102	-236.492572315063\\
66.125	0.1026	-265.143141597764\\
66.125	0.1032	-293.793710880473\\
66.125	0.1038	-322.444280163174\\
66.125	0.1044	-351.094849445879\\
66.125	0.105	-379.745418728584\\
66.125	0.1056	-408.395988011285\\
66.125	0.1062	-437.04655729399\\
66.125	0.1068	-465.697126576695\\
66.125	0.1074	-494.3476958594\\
66.125	0.108	-522.998265142101\\
66.125	0.1086	-551.648834424806\\
66.125	0.1092	-580.299403707511\\
66.125	0.1098	-608.949972990213\\
66.125	0.1104	-637.600542272918\\
66.125	0.111	-666.251111555623\\
66.125	0.1116	-694.901680838328\\
66.125	0.1122	-723.552250121029\\
66.125	0.1128	-752.202819403734\\
66.125	0.1134	-780.853388686435\\
66.125	0.114	-809.50395796914\\
66.125	0.1146	-838.154527251849\\
66.125	0.1152	-866.80509653455\\
66.125	0.1158	-895.455665817255\\
66.125	0.1164	-924.106235099956\\
66.125	0.117	-952.756804382661\\
66.125	0.1176	-981.407373665363\\
66.125	0.1182	-1010.05794294807\\
66.125	0.1188	-1038.70851223078\\
66.125	0.1194	-1067.35908151348\\
66.125	0.12	-1096.00965079618\\
66.125	0.1206	-1124.66022007888\\
66.125	0.1212	-1153.31078936159\\
66.125	0.1218	-1181.96135864429\\
66.125	0.1224	-1210.611927927\\
66.125	0.123	-1239.2624972097\\
66.5	0.093	204.857163203083\\
66.5	0.0936	176.932964785599\\
66.5	0.0942	149.008766368122\\
66.5	0.0948	121.084567950642\\
66.5	0.0954	93.1603695331614\\
66.5	0.096	65.2361711156809\\
66.5	0.0966	37.3119726982004\\
66.5	0.0972	9.38777428072353\\
66.5	0.0978	-18.536424136757\\
66.5	0.0984	-46.4606225542339\\
66.5	0.099	-74.384820971718\\
66.5	0.0996	-102.309019389195\\
66.5	0.1002	-130.233217806675\\
66.5	0.1008	-158.157416224156\\
66.5	0.1014	-186.081614641633\\
66.5	0.102	-214.005813059113\\
66.5	0.1026	-241.930011476594\\
66.5	0.1032	-269.854209894074\\
66.5	0.1038	-297.778408311551\\
66.5	0.1044	-325.702606729032\\
66.5	0.105	-353.626805146516\\
66.5	0.1056	-381.551003563993\\
66.5	0.1062	-409.475201981473\\
66.5	0.1068	-437.39940039895\\
66.5	0.1074	-465.323598816431\\
66.5	0.108	-493.247797233911\\
66.5	0.1086	-521.171995651392\\
66.5	0.1092	-549.096194068872\\
66.5	0.1098	-577.020392486349\\
66.5	0.1104	-604.94459090383\\
66.5	0.111	-632.86878932131\\
66.5	0.1116	-660.792987738791\\
66.5	0.1122	-688.717186156267\\
66.5	0.1128	-716.641384573748\\
66.5	0.1134	-744.565582991225\\
66.5	0.114	-772.489781408709\\
66.5	0.1146	-800.413979826189\\
66.5	0.1152	-828.338178243666\\
66.5	0.1158	-856.262376661147\\
66.5	0.1164	-884.186575078624\\
66.5	0.117	-912.110773496108\\
66.5	0.1176	-940.034971913585\\
66.5	0.1182	-967.959170331065\\
66.5	0.1188	-995.883368748546\\
66.5	0.1194	-1023.80756716602\\
66.5	0.12	-1051.73176558351\\
66.5	0.1206	-1079.65596400098\\
66.5	0.1212	-1107.58016241846\\
66.5	0.1218	-1135.50436083594\\
66.5	0.1224	-1163.42855925342\\
66.5	0.123	-1191.3527576709\\
66.875	0.093	216.44835948067\\
66.875	0.0936	189.25053192841\\
66.875	0.0942	162.052704376158\\
66.875	0.0948	134.854876823902\\
66.875	0.0954	107.657049271646\\
66.875	0.096	80.4592217193895\\
66.875	0.0966	53.2613941671334\\
66.875	0.0972	26.063566614881\\
66.875	0.0978	-1.13426093737507\\
66.875	0.0984	-28.3320884896311\\
66.875	0.099	-55.5299160418872\\
66.875	0.0996	-82.7277435941396\\
66.875	0.1002	-109.925571146396\\
66.875	0.1008	-137.123398698655\\
66.875	0.1014	-164.321226250908\\
66.875	0.102	-191.519053803164\\
66.875	0.1026	-218.716881355416\\
66.875	0.1032	-245.914708907676\\
66.875	0.1038	-273.112536459928\\
66.875	0.1044	-300.310364012184\\
66.875	0.105	-327.50819156444\\
66.875	0.1056	-354.706019116697\\
66.875	0.1062	-381.903846668953\\
66.875	0.1068	-409.101674221205\\
66.875	0.1074	-436.299501773461\\
66.875	0.108	-463.497329325714\\
66.875	0.1086	-490.695156877973\\
66.875	0.1092	-517.892984430229\\
66.875	0.1098	-545.090811982482\\
66.875	0.1104	-572.288639534738\\
66.875	0.111	-599.486467086994\\
66.875	0.1116	-626.68429463925\\
66.875	0.1122	-653.882122191502\\
66.875	0.1128	-681.079949743758\\
66.875	0.1134	-708.277777296014\\
66.875	0.114	-735.47560484827\\
66.875	0.1146	-762.673432400527\\
66.875	0.1152	-789.871259952779\\
66.875	0.1158	-817.069087505039\\
66.875	0.1164	-844.266915057291\\
66.875	0.117	-871.464742609547\\
66.875	0.1176	-898.6625701618\\
66.875	0.1182	-925.860397714059\\
66.875	0.1188	-953.058225266315\\
66.875	0.1194	-980.256052818568\\
66.875	0.12	-1007.45388037082\\
66.875	0.1206	-1034.65170792308\\
66.875	0.1212	-1061.84953547534\\
66.875	0.1218	-1089.04736302759\\
66.875	0.1224	-1116.24519057984\\
66.875	0.123	-1143.4430181321\\
67.25	0.093	228.039555758252\\
67.25	0.0936	201.568099071221\\
67.25	0.0942	175.096642384189\\
67.25	0.0948	148.625185697158\\
67.25	0.0954	122.153729010126\\
67.25	0.096	95.6822723230944\\
67.25	0.0966	69.2108156360628\\
67.25	0.0972	42.7393589490348\\
67.25	0.0978	16.2679022620032\\
67.25	0.0984	-10.2035544250284\\
67.25	0.099	-36.67501111206\\
67.25	0.0996	-63.146467799088\\
67.25	0.1002	-89.6179244861232\\
67.25	0.1008	-116.089381173155\\
67.25	0.1014	-142.560837860183\\
67.25	0.102	-169.032294547214\\
67.25	0.1026	-195.503751234246\\
67.25	0.1032	-221.975207921278\\
67.25	0.1038	-248.446664608306\\
67.25	0.1044	-274.918121295337\\
67.25	0.105	-301.389577982372\\
67.25	0.1056	-327.8610346694\\
67.25	0.1062	-354.332491356432\\
67.25	0.1068	-380.803948043464\\
67.25	0.1074	-407.275404730495\\
67.25	0.108	-433.746861417523\\
67.25	0.1086	-460.218318104558\\
67.25	0.1092	-486.68977479159\\
67.25	0.1098	-513.161231478618\\
67.25	0.1104	-539.63268816565\\
67.25	0.111	-566.104144852681\\
67.25	0.1116	-592.575601539713\\
67.25	0.1122	-619.047058226741\\
67.25	0.1128	-645.518514913776\\
67.25	0.1134	-671.989971600804\\
67.25	0.114	-698.461428287836\\
67.25	0.1146	-724.932884974867\\
67.25	0.1152	-751.404341661899\\
67.25	0.1158	-777.87579834893\\
67.25	0.1164	-804.347255035958\\
67.25	0.117	-830.81871172299\\
67.25	0.1176	-857.290168410022\\
67.25	0.1182	-883.761625097053\\
67.25	0.1188	-910.233081784085\\
67.25	0.1194	-936.704538471116\\
67.25	0.12	-963.175995158148\\
67.25	0.1206	-989.647451845176\\
67.25	0.1212	-1016.11890853221\\
67.25	0.1218	-1042.59036521924\\
67.25	0.1224	-1069.06182190627\\
67.25	0.123	-1095.5332785933\\
67.625	0.093	239.630752035835\\
67.625	0.0936	213.885666214028\\
67.625	0.0942	188.140580392221\\
67.625	0.0948	162.395494570414\\
67.625	0.0954	136.650408748606\\
67.625	0.096	110.905322926799\\
67.625	0.0966	85.1602371049921\\
67.625	0.0972	59.4151512831886\\
67.625	0.0978	33.6700654613815\\
67.625	0.0984	7.9249796395743\\
67.625	0.099	-17.8201061822328\\
67.625	0.0996	-43.5651920040364\\
67.625	0.1002	-69.3102778258472\\
67.625	0.1008	-95.0553636476543\\
67.625	0.1014	-120.800449469458\\
67.625	0.102	-146.545535291269\\
67.625	0.1026	-172.290621113072\\
67.625	0.1032	-198.035706934879\\
67.625	0.1038	-223.780792756686\\
67.625	0.1044	-249.525878578494\\
67.625	0.105	-275.270964400301\\
67.625	0.1056	-301.016050222108\\
67.625	0.1062	-326.761136043915\\
67.625	0.1068	-352.506221865719\\
67.625	0.1074	-378.251307687529\\
67.625	0.108	-403.996393509333\\
67.625	0.1086	-429.74147933114\\
67.625	0.1092	-455.486565152951\\
67.625	0.1098	-481.231650974754\\
67.625	0.1104	-506.976736796561\\
67.625	0.111	-532.721822618369\\
67.625	0.1116	-558.466908440176\\
67.625	0.1122	-584.211994261979\\
67.625	0.1128	-609.95708008379\\
67.625	0.1134	-635.702165905594\\
67.625	0.114	-661.447251727401\\
67.625	0.1146	-687.192337549212\\
67.625	0.1152	-712.937423371015\\
67.625	0.1158	-738.682509192822\\
67.625	0.1164	-764.427595014629\\
67.625	0.117	-790.172680836436\\
67.625	0.1176	-815.91776665824\\
67.625	0.1182	-841.662852480051\\
67.625	0.1188	-867.407938301858\\
67.625	0.1194	-893.153024123661\\
67.625	0.12	-918.898109945472\\
67.625	0.1206	-944.643195767276\\
67.625	0.1212	-970.388281589083\\
67.625	0.1218	-996.133367410886\\
67.625	0.1224	-1021.8784532327\\
67.625	0.123	-1047.6235390545\\
68	0.093	251.221948313418\\
68	0.0936	226.203233356835\\
68	0.0942	201.184518400256\\
68	0.0948	176.16580344367\\
68	0.0954	151.147088487087\\
68	0.096	126.128373530508\\
68	0.0966	101.109658573921\\
68	0.0972	76.0909436173424\\
68	0.0978	51.0722286607597\\
68	0.0984	26.053513704177\\
68	0.099	1.03479874759432\\
68	0.0996	-23.9839162089884\\
68	0.1002	-49.0026311655711\\
68	0.1008	-74.0213461221538\\
68	0.1014	-99.0400610787365\\
68	0.102	-124.058776035319\\
68	0.1026	-149.077490991898\\
68	0.1032	-174.096205948485\\
68	0.1038	-199.114920905064\\
68	0.1044	-224.13363586165\\
68	0.105	-249.152350818233\\
68	0.1056	-274.171065774812\\
68	0.1062	-299.189780731398\\
68	0.1068	-324.208495687977\\
68	0.1074	-349.22721064456\\
68	0.108	-374.245925601143\\
68	0.1086	-399.264640557725\\
68	0.1092	-424.283355514308\\
68	0.1098	-449.302070470891\\
68	0.1104	-474.320785427473\\
68	0.111	-499.339500384056\\
68	0.1116	-524.358215340639\\
68	0.1122	-549.376930297218\\
68	0.1128	-574.395645253804\\
68	0.1134	-599.414360210383\\
68	0.114	-624.43307516697\\
68	0.1146	-649.451790123552\\
68	0.1152	-674.470505080131\\
68	0.1158	-699.489220036718\\
68	0.1164	-724.507934993297\\
68	0.117	-749.526649949879\\
68	0.1176	-774.545364906462\\
68	0.1182	-799.564079863045\\
68	0.1188	-824.582794819631\\
68	0.1194	-849.60150977621\\
68	0.12	-874.620224732793\\
68	0.1206	-899.638939689376\\
68	0.1212	-924.657654645958\\
68	0.1218	-949.676369602537\\
68	0.1224	-974.695084559124\\
68	0.123	-999.713799515703\\
68.375	0.093	262.813144591008\\
68.375	0.0936	238.520800499646\\
68.375	0.0942	214.228456408291\\
68.375	0.0948	189.936112316929\\
68.375	0.0954	165.643768225571\\
68.375	0.096	141.351424134213\\
68.375	0.0966	117.059080042854\\
68.375	0.0972	92.7667359514999\\
68.375	0.0978	68.474391860138\\
68.375	0.0984	44.1820477687834\\
68.375	0.099	19.8897036774215\\
68.375	0.0996	-4.40264041393311\\
68.375	0.1002	-28.6949845052914\\
68.375	0.1008	-52.9873285966532\\
68.375	0.1014	-77.2796726880078\\
68.375	0.102	-101.57201677937\\
68.375	0.1026	-125.864360870724\\
68.375	0.1032	-150.156704962083\\
68.375	0.1038	-174.449049053441\\
68.375	0.1044	-198.741393144799\\
68.375	0.105	-223.033737236161\\
68.375	0.1056	-247.326081327516\\
68.375	0.1062	-271.618425418877\\
68.375	0.1068	-295.910769510232\\
68.375	0.1074	-320.20311360159\\
68.375	0.108	-344.495457692949\\
68.375	0.1086	-368.787801784307\\
68.375	0.1092	-393.080145875665\\
68.375	0.1098	-417.372489967023\\
68.375	0.1104	-441.664834058382\\
68.375	0.111	-465.95717814974\\
68.375	0.1116	-490.249522241098\\
68.375	0.1122	-514.541866332456\\
68.375	0.1128	-538.834210423815\\
68.375	0.1134	-563.126554515169\\
68.375	0.114	-587.418898606531\\
68.375	0.1146	-611.711242697889\\
68.375	0.1152	-636.003586789247\\
68.375	0.1158	-660.295930880606\\
68.375	0.1164	-684.588274971964\\
68.375	0.117	-708.880619063322\\
68.375	0.1176	-733.172963154677\\
68.375	0.1182	-757.465307246039\\
68.375	0.1188	-781.757651337397\\
68.375	0.1194	-806.049995428752\\
68.375	0.12	-830.342339520113\\
68.375	0.1206	-854.634683611468\\
68.375	0.1212	-878.92702770283\\
68.375	0.1218	-903.219371794185\\
68.375	0.1224	-927.511715885546\\
68.375	0.123	-951.804059976901\\
68.75	0.093	274.40434086859\\
68.75	0.0936	250.838367642453\\
68.75	0.0942	227.272394416323\\
68.75	0.0948	203.706421190189\\
68.75	0.0954	180.140447964051\\
68.75	0.096	156.574474737921\\
68.75	0.0966	133.008501511784\\
68.75	0.0972	109.442528285654\\
68.75	0.0978	85.8765550595162\\
68.75	0.0984	62.3105818333861\\
68.75	0.099	38.7446086072487\\
68.75	0.0996	15.1786353811185\\
68.75	0.1002	-8.38733784501892\\
68.75	0.1008	-31.9533110711527\\
68.75	0.1014	-55.5192842972865\\
68.75	0.102	-79.0852575234203\\
68.75	0.1026	-102.65123074955\\
68.75	0.1032	-126.217203975688\\
68.75	0.1038	-149.783177201818\\
68.75	0.1044	-173.349150427955\\
68.75	0.105	-196.915123654089\\
68.75	0.1056	-220.481096880223\\
68.75	0.1062	-244.047070106357\\
68.75	0.1068	-267.613043332491\\
68.75	0.1074	-291.179016558624\\
68.75	0.108	-314.744989784755\\
68.75	0.1086	-338.310963010892\\
68.75	0.1092	-361.876936237026\\
68.75	0.1098	-385.44290946316\\
68.75	0.1104	-409.008882689293\\
68.75	0.111	-432.574855915427\\
68.75	0.1116	-456.140829141561\\
68.75	0.1122	-479.706802367695\\
68.75	0.1128	-503.272775593829\\
68.75	0.1134	-526.838748819962\\
68.75	0.114	-550.404722046096\\
68.75	0.1146	-573.970695272234\\
68.75	0.1152	-597.536668498364\\
68.75	0.1158	-621.102641724498\\
68.75	0.1164	-644.668614950631\\
68.75	0.117	-668.234588176765\\
68.75	0.1176	-691.800561402899\\
68.75	0.1182	-715.366534629033\\
68.75	0.1188	-738.93250785517\\
68.75	0.1194	-762.4984810813\\
68.75	0.12	-786.064454307438\\
68.75	0.1206	-809.630427533568\\
68.75	0.1212	-833.196400759702\\
68.75	0.1218	-856.762373985835\\
68.75	0.1224	-880.328347211969\\
68.75	0.123	-903.894320438103\\
69.125	0.093	285.995537146173\\
69.125	0.0936	263.155934785264\\
69.125	0.0942	240.316332424354\\
69.125	0.0948	217.476730063445\\
69.125	0.0954	194.637127702532\\
69.125	0.096	171.797525341626\\
69.125	0.0966	148.957922980713\\
69.125	0.0972	126.118320619807\\
69.125	0.0978	103.278718258895\\
69.125	0.0984	80.4391158979888\\
69.125	0.099	57.5995135370758\\
69.125	0.0996	34.7599111761701\\
69.125	0.1002	11.9203088152572\\
69.125	0.1008	-10.9192935456522\\
69.125	0.1014	-33.7588959065615\\
69.125	0.102	-56.5984982674709\\
69.125	0.1026	-79.4381006283802\\
69.125	0.1032	-102.27770298929\\
69.125	0.1038	-125.117305350199\\
69.125	0.1044	-147.956907711108\\
69.125	0.105	-170.796510072021\\
69.125	0.1056	-193.636112432927\\
69.125	0.1062	-216.47571479384\\
69.125	0.1068	-239.315317154746\\
69.125	0.1074	-262.154919515659\\
69.125	0.108	-284.994521876564\\
69.125	0.1086	-307.834124237477\\
69.125	0.1092	-330.673726598387\\
69.125	0.1098	-353.513328959296\\
69.125	0.1104	-376.352931320205\\
69.125	0.111	-399.192533681115\\
69.125	0.1116	-422.032136042024\\
69.125	0.1122	-444.871738402933\\
69.125	0.1128	-467.711340763843\\
69.125	0.1134	-490.550943124752\\
69.125	0.114	-513.390545485661\\
69.125	0.1146	-536.230147846574\\
69.125	0.1152	-559.06975020748\\
69.125	0.1158	-581.909352568393\\
69.125	0.1164	-604.748954929299\\
69.125	0.117	-627.588557290212\\
69.125	0.1176	-650.428159651117\\
69.125	0.1182	-673.26776201203\\
69.125	0.1188	-696.10736437294\\
69.125	0.1194	-718.946966733849\\
69.125	0.12	-741.786569094758\\
69.125	0.1206	-764.626171455668\\
69.125	0.1212	-787.465773816577\\
69.125	0.1218	-810.305376177486\\
69.125	0.1224	-833.144978538396\\
69.125	0.123	-855.984580899305\\
69.5	0.093	297.586733423756\\
69.5	0.0936	275.473501928071\\
69.5	0.0942	253.360270432386\\
69.5	0.0948	231.247038936701\\
69.5	0.0954	209.133807441012\\
69.5	0.096	187.020575945331\\
69.5	0.0966	164.907344449643\\
69.5	0.0972	142.794112953961\\
69.5	0.0978	120.680881458273\\
69.5	0.0984	98.5676499625915\\
69.5	0.099	76.454418466903\\
69.5	0.0996	54.3411869712218\\
69.5	0.1002	32.2279554755332\\
69.5	0.1008	10.1147239798447\\
69.5	0.1014	-11.9985075158365\\
69.5	0.102	-34.1117390115251\\
69.5	0.1026	-56.2249705072063\\
69.5	0.1032	-78.3382020028948\\
69.5	0.1038	-100.451433498576\\
69.5	0.1044	-122.564664994265\\
69.5	0.105	-144.677896489949\\
69.5	0.1056	-166.791127985634\\
69.5	0.1062	-188.904359481319\\
69.5	0.1068	-211.017590977004\\
69.5	0.1074	-233.130822472689\\
69.5	0.108	-255.244053968374\\
69.5	0.1086	-277.357285464062\\
69.5	0.1092	-299.470516959747\\
69.5	0.1098	-321.583748455432\\
69.5	0.1104	-343.696979951117\\
69.5	0.111	-365.810211446802\\
69.5	0.1116	-387.923442942487\\
69.5	0.1122	-410.036674438172\\
69.5	0.1128	-432.149905933857\\
69.5	0.1134	-454.263137429542\\
69.5	0.114	-476.376368925226\\
69.5	0.1146	-498.489600420915\\
69.5	0.1152	-520.602831916596\\
69.5	0.1158	-542.716063412285\\
69.5	0.1164	-564.82929490797\\
69.5	0.117	-586.942526403654\\
69.5	0.1176	-609.055757899339\\
69.5	0.1182	-631.168989395024\\
69.5	0.1188	-653.282220890713\\
69.5	0.1194	-675.395452386394\\
69.5	0.12	-697.508683882083\\
69.5	0.1206	-719.621915377764\\
69.5	0.1212	-741.735146873452\\
69.5	0.1218	-763.848378369134\\
69.5	0.1224	-785.961609864822\\
69.5	0.123	-808.074841360507\\
69.875	0.093	309.177929701342\\
69.875	0.0936	287.791069070881\\
69.875	0.0942	266.404208440421\\
69.875	0.0948	245.01734780996\\
69.875	0.0954	223.630487179496\\
69.875	0.096	202.24362654904\\
69.875	0.0966	180.856765918576\\
69.875	0.0972	159.469905288115\\
69.875	0.0978	138.083044657655\\
69.875	0.0984	116.696184027194\\
69.875	0.099	95.3093233967338\\
69.875	0.0996	73.9224627662734\\
69.875	0.1002	52.535602135813\\
69.875	0.1008	31.1487415053489\\
69.875	0.1014	9.76188087488845\\
69.875	0.102	-11.624979755572\\
69.875	0.1026	-33.0118403860324\\
69.875	0.1032	-54.3987010164929\\
69.875	0.1038	-75.7855616469533\\
69.875	0.1044	-97.1724222774137\\
69.875	0.105	-118.559282907878\\
69.875	0.1056	-139.946143538338\\
69.875	0.1062	-161.333004168799\\
69.875	0.1068	-182.719864799259\\
69.875	0.1074	-204.10672542972\\
69.875	0.108	-225.49358606018\\
69.875	0.1086	-246.880446690644\\
69.875	0.1092	-268.267307321104\\
69.875	0.1098	-289.654167951565\\
69.875	0.1104	-311.041028582025\\
69.875	0.111	-332.427889212486\\
69.875	0.1116	-353.814749842946\\
69.875	0.1122	-375.201610473407\\
69.875	0.1128	-396.588471103871\\
69.875	0.1134	-417.975331734327\\
69.875	0.114	-439.362192364792\\
69.875	0.1146	-460.749052995252\\
69.875	0.1152	-482.135913625712\\
69.875	0.1158	-503.522774256173\\
69.875	0.1164	-524.909634886633\\
69.875	0.117	-546.296495517097\\
69.875	0.1176	-567.683356147554\\
69.875	0.1182	-589.070216778018\\
69.875	0.1188	-610.457077408479\\
69.875	0.1194	-631.843938038939\\
69.875	0.12	-653.2307986694\\
69.875	0.1206	-674.61765929986\\
69.875	0.1212	-696.004519930324\\
69.875	0.1218	-717.391380560781\\
69.875	0.1224	-738.778241191245\\
69.875	0.123	-760.165101821702\\
70.25	0.093	320.769125978928\\
70.25	0.0936	300.108636213688\\
70.25	0.0942	279.448146448452\\
70.25	0.0948	258.787656683216\\
70.25	0.0954	238.127166917977\\
70.25	0.096	217.466677152745\\
70.25	0.0966	196.806187387505\\
70.25	0.0972	176.145697622269\\
70.25	0.0978	155.485207857033\\
70.25	0.0984	134.824718091797\\
70.25	0.099	114.164228326561\\
70.25	0.0996	93.503738561325\\
70.25	0.1002	72.8432487960854\\
70.25	0.1008	52.1827590308494\\
70.25	0.1014	31.5222692656134\\
70.25	0.102	10.8617795003775\\
70.25	0.1026	-9.79871026485853\\
70.25	0.1032	-30.4592000300981\\
70.25	0.1038	-51.1196897953305\\
70.25	0.1044	-71.7801795605701\\
70.25	0.105	-92.4406693258097\\
70.25	0.1056	-113.101159091042\\
70.25	0.1062	-133.761648856282\\
70.25	0.1068	-154.422138621518\\
70.25	0.1074	-175.082628386754\\
70.25	0.108	-195.74311815199\\
70.25	0.1086	-216.403607917226\\
70.25	0.1092	-237.064097682465\\
70.25	0.1098	-257.724587447701\\
70.25	0.1104	-278.385077212937\\
70.25	0.111	-299.045566978173\\
70.25	0.1116	-319.706056743413\\
70.25	0.1122	-340.366546508645\\
70.25	0.1128	-361.027036273885\\
70.25	0.1134	-381.687526039117\\
70.25	0.114	-402.348015804357\\
70.25	0.1146	-423.008505569596\\
70.25	0.1152	-443.668995334829\\
70.25	0.1158	-464.329485100068\\
70.25	0.1164	-484.989974865301\\
70.25	0.117	-505.65046463054\\
70.25	0.1176	-526.310954395776\\
70.25	0.1182	-546.971444161012\\
70.25	0.1188	-567.631933926252\\
70.25	0.1194	-588.292423691484\\
70.25	0.12	-608.952913456724\\
70.25	0.1206	-629.61340322196\\
70.25	0.1212	-650.273892987196\\
70.25	0.1218	-670.934382752432\\
70.25	0.1224	-691.594872517671\\
70.25	0.123	-712.255362282904\\
70.625	0.093	332.360322256511\\
70.625	0.0936	312.426203356496\\
70.625	0.0942	292.492084456488\\
70.625	0.0948	272.557965556472\\
70.625	0.0954	252.623846656457\\
70.625	0.096	232.689727756449\\
70.625	0.0966	212.755608856434\\
70.625	0.0972	192.821489956423\\
70.625	0.0978	172.887371056411\\
70.625	0.0984	152.9532521564\\
70.625	0.099	133.019133256385\\
70.625	0.0996	113.085014356377\\
70.625	0.1002	93.1508954563615\\
70.625	0.1008	73.2167765563463\\
70.625	0.1014	53.2826576563384\\
70.625	0.102	33.3485387563233\\
70.625	0.1026	13.4144198563117\\
70.625	0.1032	-6.5196990436998\\
70.625	0.1038	-26.4538179437113\\
70.625	0.1044	-46.3879368437229\\
70.625	0.105	-66.322055743738\\
70.625	0.1056	-86.2561746437495\\
70.625	0.1062	-106.190293543765\\
70.625	0.1068	-126.124412443773\\
70.625	0.1074	-146.058531343788\\
70.625	0.108	-165.992650243799\\
70.625	0.1086	-185.926769143811\\
70.625	0.1092	-205.860888043826\\
70.625	0.1098	-225.795006943834\\
70.625	0.1104	-245.729125843849\\
70.625	0.111	-265.663244743861\\
70.625	0.1116	-285.597363643876\\
70.625	0.1122	-305.531482543884\\
70.625	0.1128	-325.465601443899\\
70.625	0.1134	-345.39972034391\\
70.625	0.114	-365.333839243922\\
70.625	0.1146	-385.267958143937\\
70.625	0.1152	-405.202077043945\\
70.625	0.1158	-425.13619594396\\
70.625	0.1164	-445.070314843972\\
70.625	0.117	-465.004433743983\\
70.625	0.1176	-484.938552643995\\
70.625	0.1182	-504.87267154401\\
70.625	0.1188	-524.806790444021\\
70.625	0.1194	-544.740909344033\\
70.625	0.12	-564.675028244048\\
70.625	0.1206	-584.609147144056\\
70.625	0.1212	-604.543266044071\\
70.625	0.1218	-624.477384944083\\
70.625	0.1224	-644.411503844094\\
70.625	0.123	-664.345622744106\\
71	0.093	343.951518534097\\
71	0.0936	324.743770499306\\
71	0.0942	305.536022464523\\
71	0.0948	286.328274429732\\
71	0.0954	267.120526394941\\
71	0.096	247.912778360158\\
71	0.0966	228.705030325367\\
71	0.0972	209.49728229058\\
71	0.0978	190.289534255793\\
71	0.0984	171.081786221006\\
71	0.099	151.874038186215\\
71	0.0996	132.666290151428\\
71	0.1002	113.458542116641\\
71	0.1008	94.2507940818505\\
71	0.1014	75.0430460470634\\
71	0.102	55.8352980122763\\
71	0.1026	36.6275499774893\\
71	0.1032	17.4198019426985\\
71	0.1038	-1.78794609208853\\
71	0.1044	-20.9956941268756\\
71	0.105	-40.2034421616663\\
71	0.1056	-59.4111901964534\\
71	0.1062	-78.6189382312405\\
71	0.1068	-97.8266862660275\\
71	0.1074	-117.034434300818\\
71	0.108	-136.242182335602\\
71	0.1086	-155.449930370392\\
71	0.1092	-174.657678405183\\
71	0.1098	-193.865426439967\\
71	0.1104	-213.073174474757\\
71	0.111	-232.280922509544\\
71	0.1116	-251.488670544335\\
71	0.1122	-270.696418579118\\
71	0.1128	-289.904166613909\\
71	0.1134	-309.111914648696\\
71	0.114	-328.319662683483\\
71	0.1146	-347.527410718274\\
71	0.1152	-366.735158753061\\
71	0.1158	-385.942906787852\\
71	0.1164	-405.150654822635\\
71	0.117	-424.358402857426\\
71	0.1176	-443.566150892213\\
71	0.1182	-462.773898927\\
71	0.1188	-481.981646961791\\
71	0.1194	-501.189394996578\\
71	0.12	-520.397143031365\\
71	0.1206	-539.604891066152\\
71	0.1212	-558.812639100943\\
71	0.1218	-578.02038713573\\
71	0.1224	-597.228135170517\\
71	0.123	-616.435883205304\\
71.375	0.093	355.542714811683\\
71.375	0.0936	337.061337642121\\
71.375	0.0942	318.579960472558\\
71.375	0.0948	300.098583302992\\
71.375	0.0954	281.617206133425\\
71.375	0.096	263.135828963867\\
71.375	0.0966	244.6544517943\\
71.375	0.0972	226.173074624738\\
71.375	0.0978	207.691697455171\\
71.375	0.0984	189.210320285612\\
71.375	0.099	170.728943116046\\
71.375	0.0996	152.247565946484\\
71.375	0.1002	133.766188776921\\
71.375	0.1008	115.284811607355\\
71.375	0.1014	96.803434437792\\
71.375	0.102	78.3220572682258\\
71.375	0.1026	59.8406800986668\\
71.375	0.1032	41.3593029291005\\
71.375	0.1038	22.8779257595379\\
71.375	0.1044	4.39654858997164\\
71.375	0.105	-14.084828579591\\
71.375	0.1056	-32.5662057491536\\
71.375	0.1062	-51.0475829187199\\
71.375	0.1068	-69.5289600882825\\
71.375	0.1074	-88.0103372578451\\
71.375	0.108	-106.491714427408\\
71.375	0.1086	-124.973091596974\\
71.375	0.1092	-143.45446876654\\
71.375	0.1098	-161.935845936099\\
71.375	0.1104	-180.417223105665\\
71.375	0.111	-198.898600275228\\
71.375	0.1116	-217.379977444794\\
71.375	0.1122	-235.861354614353\\
71.375	0.1128	-254.34273178392\\
71.375	0.1134	-272.824108953482\\
71.375	0.114	-291.305486123048\\
71.375	0.1146	-309.786863292611\\
71.375	0.1152	-328.268240462174\\
71.375	0.1158	-346.74961763174\\
71.375	0.1164	-365.230994801303\\
71.375	0.117	-383.712371970865\\
71.375	0.1176	-402.193749140428\\
71.375	0.1182	-420.675126309994\\
71.375	0.1188	-439.15650347956\\
71.375	0.1194	-457.637880649119\\
71.375	0.12	-476.119257818686\\
71.375	0.1206	-494.600634988248\\
71.375	0.1212	-513.082012157814\\
71.375	0.1218	-531.563389327373\\
71.375	0.1224	-550.04476649694\\
71.375	0.123	-568.526143666502\\
71.75	0.093	367.13391108927\\
71.75	0.0936	349.378904784928\\
71.75	0.0942	331.62389848059\\
71.75	0.0948	313.868892176248\\
71.75	0.0954	296.11388587191\\
71.75	0.096	278.358879567571\\
71.75	0.0966	260.60387326323\\
71.75	0.0972	242.848866958891\\
71.75	0.0978	225.09386065455\\
71.75	0.0984	207.338854350215\\
71.75	0.099	189.583848045873\\
71.75	0.0996	171.828841741535\\
71.75	0.1002	154.073835437193\\
71.75	0.1008	136.318829132852\\
71.75	0.1014	118.563822828517\\
71.75	0.102	100.808816524175\\
71.75	0.1026	83.053810219837\\
71.75	0.1032	65.2988039154952\\
71.75	0.1038	47.5437976111607\\
71.75	0.1044	29.7887913068189\\
71.75	0.105	12.0337850024771\\
71.75	0.1056	-5.72122130186108\\
71.75	0.1062	-23.4762276062029\\
71.75	0.1068	-41.2312339105374\\
71.75	0.1074	-58.9862402148792\\
71.75	0.108	-76.7412465192174\\
71.75	0.1086	-94.4962528235592\\
71.75	0.1092	-112.251259127897\\
71.75	0.1098	-130.006265432236\\
71.75	0.1104	-147.761271736577\\
71.75	0.111	-165.516278040915\\
71.75	0.1116	-183.271284345257\\
71.75	0.1122	-201.026290649592\\
71.75	0.1128	-218.781296953934\\
71.75	0.1134	-236.536303258272\\
71.75	0.114	-254.291309562614\\
71.75	0.1146	-272.046315866955\\
71.75	0.1152	-289.80132217129\\
71.75	0.1158	-307.556328475632\\
71.75	0.1164	-325.31133477997\\
71.75	0.117	-343.066341084312\\
71.75	0.1176	-360.82134738865\\
71.75	0.1182	-378.576353692988\\
71.75	0.1188	-396.33135999733\\
71.75	0.1194	-414.086366301668\\
71.75	0.12	-431.84137260601\\
71.75	0.1206	-449.596378910348\\
71.75	0.1212	-467.351385214686\\
71.75	0.1218	-485.106391519024\\
71.75	0.1224	-502.861397823366\\
71.75	0.123	-520.616404127704\\
72.125	0.093	378.725107366852\\
72.125	0.0936	361.696471927735\\
72.125	0.0942	344.667836488621\\
72.125	0.0948	327.639201049507\\
72.125	0.0954	310.61056561039\\
72.125	0.096	293.581930171276\\
72.125	0.0966	276.553294732159\\
72.125	0.0972	259.524659293045\\
72.125	0.0978	242.496023853928\\
72.125	0.0984	225.467388414818\\
72.125	0.099	208.4387529757\\
72.125	0.0996	191.410117536587\\
72.125	0.1002	174.381482097469\\
72.125	0.1008	157.352846658352\\
72.125	0.1014	140.324211219238\\
72.125	0.102	123.295575780125\\
72.125	0.1026	106.266940341011\\
72.125	0.1032	89.2383049018936\\
72.125	0.1038	72.2096694627799\\
72.125	0.1044	55.1810340236625\\
72.125	0.105	38.1523985845452\\
72.125	0.1056	21.1237631454351\\
72.125	0.1062	4.09512770631773\\
72.125	0.1068	-12.933507732796\\
72.125	0.1074	-29.9621431719133\\
72.125	0.108	-46.9907786110271\\
72.125	0.1086	-64.0194140501444\\
72.125	0.1092	-81.0480494892581\\
72.125	0.1098	-98.0766849283718\\
72.125	0.1104	-115.105320367489\\
72.125	0.111	-132.133955806603\\
72.125	0.1116	-149.16259124572\\
72.125	0.1122	-166.191226684834\\
72.125	0.1128	-183.219862123948\\
72.125	0.1134	-200.248497563061\\
72.125	0.114	-217.277133002179\\
72.125	0.1146	-234.305768441296\\
72.125	0.1152	-251.33440388041\\
72.125	0.1158	-268.363039319527\\
72.125	0.1164	-285.391674758637\\
72.125	0.117	-302.420310197755\\
72.125	0.1176	-319.448945636868\\
72.125	0.1182	-336.477581075986\\
72.125	0.1188	-353.506216515103\\
72.125	0.1194	-370.534851954217\\
72.125	0.12	-387.56348739333\\
72.125	0.1206	-404.592122832444\\
72.125	0.1212	-421.620758271561\\
72.125	0.1218	-438.649393710675\\
72.125	0.1224	-455.678029149793\\
72.125	0.123	-472.706664588906\\
72.5	0.093	390.316303644438\\
72.5	0.0936	374.014039070546\\
72.5	0.0942	357.711774496656\\
72.5	0.0948	341.409509922767\\
72.5	0.0954	325.107245348874\\
72.5	0.096	308.804980774985\\
72.5	0.0966	292.502716201092\\
72.5	0.0972	276.200451627203\\
72.5	0.0978	259.89818705331\\
72.5	0.0984	243.595922479421\\
72.5	0.099	227.293657905528\\
72.5	0.0996	210.991393331642\\
72.5	0.1002	194.689128757749\\
72.5	0.1008	178.386864183856\\
72.5	0.1014	162.084599609967\\
72.5	0.102	145.782335036074\\
72.5	0.1026	129.480070462185\\
72.5	0.1032	113.177805888292\\
72.5	0.1038	96.8755413144027\\
72.5	0.1044	80.5732767405134\\
72.5	0.105	64.2710121666205\\
72.5	0.1056	47.9687475927312\\
72.5	0.1062	31.6664830188383\\
72.5	0.1068	15.3642184449491\\
72.5	0.1074	-0.938046128943824\\
72.5	0.108	-17.2403107028331\\
72.5	0.1086	-33.542575276726\\
72.5	0.1092	-49.8448398506152\\
72.5	0.1098	-66.1471044245045\\
72.5	0.1104	-82.4493689983974\\
72.5	0.111	-98.7516335722867\\
72.5	0.1116	-115.05389814618\\
72.5	0.1122	-131.356162720069\\
72.5	0.1128	-147.658427293962\\
72.5	0.1134	-163.960691867851\\
72.5	0.114	-180.26295644174\\
72.5	0.1146	-196.565221015633\\
72.5	0.1152	-212.867485589522\\
72.5	0.1158	-229.169750163415\\
72.5	0.1164	-245.472014737305\\
72.5	0.117	-261.774279311197\\
72.5	0.1176	-278.076543885087\\
72.5	0.1182	-294.378808458976\\
72.5	0.1188	-310.681073032869\\
72.5	0.1194	-326.983337606758\\
72.5	0.12	-343.285602180651\\
72.5	0.1206	-359.58786675454\\
72.5	0.1212	-375.890131328433\\
72.5	0.1218	-392.192395902322\\
72.5	0.1224	-408.494660476215\\
72.5	0.123	-424.796925050101\\
72.875	0.093	401.907499922021\\
72.875	0.0936	386.331606213353\\
72.875	0.0942	370.755712504691\\
72.875	0.0948	355.179818796023\\
72.875	0.0954	339.603925087355\\
72.875	0.096	324.02803137869\\
72.875	0.0966	308.452137670021\\
72.875	0.0972	292.876243961357\\
72.875	0.0978	277.300350252688\\
72.875	0.0984	261.724456544023\\
72.875	0.099	246.148562835355\\
72.875	0.0996	230.57266912669\\
72.875	0.1002	214.996775418022\\
72.875	0.1008	199.420881709357\\
72.875	0.1014	183.844988000692\\
72.875	0.102	168.269094292024\\
72.875	0.1026	152.693200583359\\
72.875	0.1032	137.11730687469\\
72.875	0.1038	121.541413166025\\
72.875	0.1044	105.965519457357\\
72.875	0.105	90.3896257486886\\
72.875	0.1056	74.8137320400238\\
72.875	0.1062	59.2378383313553\\
72.875	0.1068	43.6619446226905\\
72.875	0.1074	28.0860509140221\\
72.875	0.108	12.5101572053609\\
72.875	0.1086	-3.06573650330756\\
72.875	0.1092	-18.641630211976\\
72.875	0.1098	-34.2175239206408\\
72.875	0.1104	-49.7934176293093\\
72.875	0.111	-65.3693113379741\\
72.875	0.1116	-80.9452050466425\\
72.875	0.1122	-96.5210987553073\\
72.875	0.1128	-112.096992463976\\
72.875	0.1134	-127.672886172641\\
72.875	0.114	-143.248779881309\\
72.875	0.1146	-158.824673589977\\
72.875	0.1152	-174.400567298639\\
72.875	0.1158	-189.976461007307\\
72.875	0.1164	-205.552354715972\\
72.875	0.117	-221.12824842464\\
72.875	0.1176	-236.704142133305\\
72.875	0.1182	-252.280035841974\\
72.875	0.1188	-267.855929550642\\
72.875	0.1194	-283.431823259307\\
72.875	0.12	-299.007716967975\\
72.875	0.1206	-314.58361067664\\
72.875	0.1212	-330.159504385309\\
72.875	0.1218	-345.73539809397\\
72.875	0.1224	-361.311291802638\\
72.875	0.123	-376.887185511303\\
73.25	0.093	413.498696199607\\
73.25	0.0936	398.649173356163\\
73.25	0.0942	383.799650512723\\
73.25	0.0948	368.950127669279\\
73.25	0.0954	354.100604825835\\
73.25	0.096	339.251081982395\\
73.25	0.0966	324.401559138951\\
73.25	0.0972	309.55203629551\\
73.25	0.0978	294.702513452066\\
73.25	0.0984	279.852990608626\\
73.25	0.099	265.003467765182\\
73.25	0.0996	250.153944921742\\
73.25	0.1002	235.304422078298\\
73.25	0.1008	220.454899234854\\
73.25	0.1014	205.605376391413\\
73.25	0.102	190.755853547969\\
73.25	0.1026	175.906330704529\\
73.25	0.1032	161.056807861085\\
73.25	0.1038	146.207285017645\\
73.25	0.1044	131.357762174204\\
73.25	0.105	116.50823933076\\
73.25	0.1056	101.65871648732\\
73.25	0.1062	86.8091936438759\\
73.25	0.1068	71.9596708004356\\
73.25	0.1074	57.1101479569916\\
73.25	0.108	42.2606251135512\\
73.25	0.1086	27.4111022701072\\
73.25	0.1092	12.5615794266632\\
73.25	0.1098	-2.28794341677713\\
73.25	0.1104	-17.1374662602211\\
73.25	0.111	-31.9869891036615\\
73.25	0.1116	-46.8365119471055\\
73.25	0.1122	-61.6860347905458\\
73.25	0.1128	-76.5355576339898\\
73.25	0.1134	-91.3850804774302\\
73.25	0.114	-106.234603320874\\
73.25	0.1146	-121.084126164318\\
73.25	0.1152	-135.933649007759\\
73.25	0.1158	-150.783171851202\\
73.25	0.1164	-165.632694694643\\
73.25	0.117	-180.482217538087\\
73.25	0.1176	-195.331740381527\\
73.25	0.1182	-210.181263224968\\
73.25	0.1188	-225.030786068412\\
73.25	0.1194	-239.880308911852\\
73.25	0.12	-254.729831755296\\
73.25	0.1206	-269.579354598736\\
73.25	0.1212	-284.42887744218\\
73.25	0.1218	-299.278400285621\\
73.25	0.1224	-314.127923129065\\
73.25	0.123	-328.977445972505\\
73.625	0.093	425.08989247719\\
73.625	0.0936	410.96674049897\\
73.625	0.0942	396.843588520755\\
73.625	0.0948	382.720436542535\\
73.625	0.0954	368.597284564315\\
73.625	0.096	354.4741325861\\
73.625	0.0966	340.35098060788\\
73.625	0.0972	326.227828629664\\
73.625	0.0978	312.104676651445\\
73.625	0.0984	297.981524673229\\
73.625	0.099	283.858372695009\\
73.625	0.0996	269.735220716793\\
73.625	0.1002	255.612068738574\\
73.625	0.1008	241.488916760354\\
73.625	0.1014	227.365764782138\\
73.625	0.102	213.242612803919\\
73.625	0.1026	199.119460825703\\
73.625	0.1032	184.996308847483\\
73.625	0.1038	170.873156869267\\
73.625	0.1044	156.750004891048\\
73.625	0.105	142.626852912828\\
73.625	0.1056	128.503700934612\\
73.625	0.1062	114.380548956393\\
73.625	0.1068	100.257396978177\\
73.625	0.1074	86.1342449999574\\
73.625	0.108	72.0110930217415\\
73.625	0.1086	57.887941043522\\
73.625	0.1092	43.7647890653025\\
73.625	0.1098	29.6416370870866\\
73.625	0.1104	15.518485108867\\
73.625	0.111	1.39533313065112\\
73.625	0.1116	-12.7278188475684\\
73.625	0.1122	-26.8509708257843\\
73.625	0.1128	-40.9741228040039\\
73.625	0.1134	-55.0972747822198\\
73.625	0.114	-69.2204267604393\\
73.625	0.1146	-83.3435787386588\\
73.625	0.1152	-97.4667307168747\\
73.625	0.1158	-111.589882695094\\
73.625	0.1164	-125.71303467331\\
73.625	0.117	-139.83618665153\\
73.625	0.1176	-153.959338629746\\
73.625	0.1182	-168.082490607965\\
73.625	0.1188	-182.205642586185\\
73.625	0.1194	-196.328794564401\\
73.625	0.12	-210.45194654262\\
73.625	0.1206	-224.575098520836\\
73.625	0.1212	-238.698250499056\\
73.625	0.1218	-252.821402477271\\
73.625	0.1224	-266.944554455491\\
73.625	0.123	-281.067706433707\\
74	0.093	436.681088754776\\
74	0.0936	423.284307641781\\
74	0.0942	409.88752652879\\
74	0.0948	396.490745415795\\
74	0.0954	383.0939643028\\
74	0.096	369.697183189808\\
74	0.0966	356.300402076813\\
74	0.0972	342.903620963822\\
74	0.0978	329.506839850827\\
74	0.0984	316.110058737835\\
74	0.099	302.71327762484\\
74	0.0996	289.316496511849\\
74	0.1002	275.919715398853\\
74	0.1008	262.522934285858\\
74	0.1014	249.126153172863\\
74	0.102	235.729372059872\\
74	0.1026	222.332590946877\\
74	0.1032	208.935809833882\\
74	0.1038	195.53902872089\\
74	0.1044	182.142247607895\\
74	0.105	168.7454664949\\
74	0.1056	155.348685381909\\
74	0.1062	141.951904268913\\
74	0.1068	128.555123155922\\
74	0.1074	115.158342042927\\
74	0.108	101.761560929936\\
74	0.1086	88.3647798169404\\
74	0.1092	74.9679987039453\\
74	0.1098	61.5712175909539\\
74	0.1104	48.1744364779588\\
74	0.111	34.7776553649674\\
74	0.1116	21.3808742519723\\
74	0.1122	7.98409313898082\\
74	0.1128	-5.41268797401426\\
74	0.1134	-18.8094690870057\\
74	0.114	-32.2062502000008\\
74	0.1146	-45.6030313129959\\
74	0.1152	-58.9998124259873\\
74	0.1158	-72.3965935389824\\
74	0.1164	-85.7933746519739\\
74	0.117	-99.1901557649689\\
74	0.1176	-112.586936877964\\
74	0.1182	-125.983717990959\\
74	0.1188	-139.380499103954\\
74	0.1194	-152.777280216946\\
74	0.12	-166.174061329941\\
74	0.1206	-179.570842442932\\
74	0.1212	-192.967623555927\\
74	0.1218	-206.364404668919\\
74	0.1224	-219.761185781914\\
74	0.123	-233.157966894905\\
};
\end{axis}

\begin{axis}[%
width=4.927496cm,
height=3.870968cm,
at={(0cm,10.752688cm)},
scale only axis,
xmin=56,
xmax=74,
tick align=outside,
xlabel={$L_{cut}$},
xmajorgrids,
ymin=0.093,
ymax=0.123,
ylabel={$D_{rlx}$},
ymajorgrids,
zmin=-3.92997043486943,
zmax=20.7603892302203,
zlabel={$x_1,x_4$},
zmajorgrids,
view={-140}{50},
legend style={at={(1.03,1)},anchor=north west,legend cell align=left,align=left,draw=white!15!black}
]
\addplot3[only marks,mark=*,mark options={},mark size=1.5000pt,color=mycolor1] plot table[row sep=crcr,]{%
74	0.123	0.616147082677301\\
72	0.113	0.756244217181634\\
61	0.095	0.315136730386088\\
56	0.093	0.479225561787365\\
};
\addplot3[only marks,mark=*,mark options={},mark size=1.5000pt,color=black] plot table[row sep=crcr,]{%
69	0.104	0.564359576734034\\
};

\addplot3[%
surf,
opacity=0.7,
shader=interp,
colormap={mymap}{[1pt] rgb(0pt)=(0.0901961,0.239216,0.0745098); rgb(1pt)=(0.0945149,0.242058,0.0739522); rgb(2pt)=(0.0988592,0.244894,0.0733566); rgb(3pt)=(0.103229,0.247724,0.0727241); rgb(4pt)=(0.107623,0.250549,0.0720557); rgb(5pt)=(0.112043,0.253367,0.0713525); rgb(6pt)=(0.116487,0.25618,0.0706154); rgb(7pt)=(0.120956,0.258986,0.0698456); rgb(8pt)=(0.125449,0.261787,0.0690441); rgb(9pt)=(0.129967,0.264581,0.0682118); rgb(10pt)=(0.134508,0.26737,0.06735); rgb(11pt)=(0.139074,0.270152,0.0664596); rgb(12pt)=(0.143663,0.272929,0.0655416); rgb(13pt)=(0.148275,0.275699,0.0645971); rgb(14pt)=(0.152911,0.278463,0.0636271); rgb(15pt)=(0.15757,0.281221,0.0626328); rgb(16pt)=(0.162252,0.283973,0.0616151); rgb(17pt)=(0.166957,0.286719,0.060575); rgb(18pt)=(0.171685,0.289458,0.0595136); rgb(19pt)=(0.176434,0.292191,0.0584321); rgb(20pt)=(0.181207,0.294918,0.0573313); rgb(21pt)=(0.186001,0.297639,0.0562123); rgb(22pt)=(0.190817,0.300353,0.0550763); rgb(23pt)=(0.195655,0.303061,0.0539242); rgb(24pt)=(0.200514,0.305763,0.052757); rgb(25pt)=(0.205395,0.308459,0.0515759); rgb(26pt)=(0.210296,0.311149,0.0503624); rgb(27pt)=(0.215212,0.313846,0.0490067); rgb(28pt)=(0.220142,0.316548,0.0475043); rgb(29pt)=(0.22509,0.319254,0.0458704); rgb(30pt)=(0.230056,0.321962,0.0441205); rgb(31pt)=(0.235042,0.324671,0.04227); rgb(32pt)=(0.240048,0.327379,0.0403343); rgb(33pt)=(0.245078,0.330085,0.0383287); rgb(34pt)=(0.250131,0.332786,0.0362688); rgb(35pt)=(0.25521,0.335482,0.0341698); rgb(36pt)=(0.260317,0.33817,0.0320472); rgb(37pt)=(0.265451,0.340849,0.0299163); rgb(38pt)=(0.270616,0.343517,0.0277927); rgb(39pt)=(0.275813,0.346172,0.0256916); rgb(40pt)=(0.281043,0.348814,0.0236284); rgb(41pt)=(0.286307,0.35144,0.0216186); rgb(42pt)=(0.291607,0.354048,0.0196776); rgb(43pt)=(0.296945,0.356637,0.0178207); rgb(44pt)=(0.302322,0.359206,0.0160634); rgb(45pt)=(0.307739,0.361753,0.0144211); rgb(46pt)=(0.313198,0.364275,0.0129091); rgb(47pt)=(0.318701,0.366772,0.0115428); rgb(48pt)=(0.324249,0.369242,0.0103377); rgb(49pt)=(0.329843,0.371682,0.00930909); rgb(50pt)=(0.335485,0.374093,0.00847245); rgb(51pt)=(0.341176,0.376471,0.00784314); rgb(52pt)=(0.346925,0.378826,0.00732741); rgb(53pt)=(0.352735,0.381168,0.00682184); rgb(54pt)=(0.358605,0.383497,0.00632729); rgb(55pt)=(0.364532,0.385812,0.00584464); rgb(56pt)=(0.370516,0.388113,0.00537476); rgb(57pt)=(0.376552,0.390399,0.00491852); rgb(58pt)=(0.38264,0.39267,0.00447681); rgb(59pt)=(0.388777,0.394925,0.00405048); rgb(60pt)=(0.394962,0.397164,0.00364042); rgb(61pt)=(0.401191,0.399386,0.00324749); rgb(62pt)=(0.407464,0.401592,0.00287258); rgb(63pt)=(0.413777,0.40378,0.00251655); rgb(64pt)=(0.420129,0.40595,0.00218028); rgb(65pt)=(0.426518,0.408102,0.00186463); rgb(66pt)=(0.432942,0.410234,0.00157049); rgb(67pt)=(0.439399,0.412348,0.00129873); rgb(68pt)=(0.445885,0.414441,0.00105022); rgb(69pt)=(0.452401,0.416515,0.000825833); rgb(70pt)=(0.458942,0.418567,0.000626441); rgb(71pt)=(0.465508,0.420599,0.00045292); rgb(72pt)=(0.472096,0.422609,0.000306141); rgb(73pt)=(0.478704,0.424596,0.000186979); rgb(74pt)=(0.485331,0.426562,9.63073e-05); rgb(75pt)=(0.491973,0.428504,3.49981e-05); rgb(76pt)=(0.498628,0.430422,3.92506e-06); rgb(77pt)=(0.505323,0.432315,0); rgb(78pt)=(0.512206,0.434168,0); rgb(79pt)=(0.519282,0.435983,0); rgb(80pt)=(0.526529,0.437764,0); rgb(81pt)=(0.533922,0.439512,0); rgb(82pt)=(0.54144,0.441232,0); rgb(83pt)=(0.549059,0.442927,0); rgb(84pt)=(0.556756,0.444599,0); rgb(85pt)=(0.564508,0.446252,0); rgb(86pt)=(0.572292,0.447889,0); rgb(87pt)=(0.580084,0.449514,0); rgb(88pt)=(0.587863,0.451129,0); rgb(89pt)=(0.595604,0.452737,0); rgb(90pt)=(0.603284,0.454343,0); rgb(91pt)=(0.610882,0.455948,0); rgb(92pt)=(0.618373,0.457556,0); rgb(93pt)=(0.625734,0.459171,0); rgb(94pt)=(0.632943,0.460795,0); rgb(95pt)=(0.639976,0.462432,0); rgb(96pt)=(0.64681,0.464084,0); rgb(97pt)=(0.653423,0.465756,0); rgb(98pt)=(0.659791,0.46745,0); rgb(99pt)=(0.665891,0.469169,0); rgb(100pt)=(0.6717,0.470916,0); rgb(101pt)=(0.677195,0.472696,0); rgb(102pt)=(0.682353,0.47451,0); rgb(103pt)=(0.687242,0.476355,0); rgb(104pt)=(0.691952,0.478225,0); rgb(105pt)=(0.696497,0.480118,0); rgb(106pt)=(0.700887,0.482033,0); rgb(107pt)=(0.705134,0.483968,0); rgb(108pt)=(0.709251,0.485921,0); rgb(109pt)=(0.713249,0.487891,0); rgb(110pt)=(0.71714,0.489876,0); rgb(111pt)=(0.720936,0.491875,0); rgb(112pt)=(0.724649,0.493887,0); rgb(113pt)=(0.72829,0.495909,0); rgb(114pt)=(0.731872,0.49794,0); rgb(115pt)=(0.735406,0.499979,0); rgb(116pt)=(0.738904,0.502025,0); rgb(117pt)=(0.742378,0.504075,0); rgb(118pt)=(0.74584,0.506128,0); rgb(119pt)=(0.749302,0.508182,0); rgb(120pt)=(0.752775,0.510237,0); rgb(121pt)=(0.756272,0.51229,0); rgb(122pt)=(0.759804,0.514339,0); rgb(123pt)=(0.763384,0.516385,0); rgb(124pt)=(0.767022,0.518424,0); rgb(125pt)=(0.770731,0.520455,0); rgb(126pt)=(0.774523,0.522478,0); rgb(127pt)=(0.77841,0.524489,0); rgb(128pt)=(0.782391,0.526491,0); rgb(129pt)=(0.786402,0.528496,0); rgb(130pt)=(0.790431,0.530506,0); rgb(131pt)=(0.794478,0.532521,0); rgb(132pt)=(0.798541,0.534539,0); rgb(133pt)=(0.802619,0.53656,0); rgb(134pt)=(0.806712,0.538584,0); rgb(135pt)=(0.81082,0.540609,0); rgb(136pt)=(0.81494,0.542635,0); rgb(137pt)=(0.819074,0.54466,0); rgb(138pt)=(0.823219,0.546686,0); rgb(139pt)=(0.827374,0.548709,0); rgb(140pt)=(0.831541,0.55073,0); rgb(141pt)=(0.835716,0.552749,0); rgb(142pt)=(0.8399,0.554763,0); rgb(143pt)=(0.844092,0.556774,0); rgb(144pt)=(0.848292,0.558779,0); rgb(145pt)=(0.852497,0.560778,0); rgb(146pt)=(0.856708,0.562771,0); rgb(147pt)=(0.860924,0.564756,0); rgb(148pt)=(0.865143,0.566733,0); rgb(149pt)=(0.869366,0.568701,0); rgb(150pt)=(0.873592,0.57066,0); rgb(151pt)=(0.877819,0.572608,0); rgb(152pt)=(0.882047,0.574545,0); rgb(153pt)=(0.886275,0.576471,0); rgb(154pt)=(0.890659,0.578362,0); rgb(155pt)=(0.895333,0.580203,0); rgb(156pt)=(0.900258,0.581999,0); rgb(157pt)=(0.905397,0.583755,0); rgb(158pt)=(0.910711,0.585479,0); rgb(159pt)=(0.916164,0.587176,0); rgb(160pt)=(0.921717,0.588852,0); rgb(161pt)=(0.927333,0.590513,0); rgb(162pt)=(0.932974,0.592166,0); rgb(163pt)=(0.938602,0.593815,0); rgb(164pt)=(0.94418,0.595468,0); rgb(165pt)=(0.949669,0.59713,0); rgb(166pt)=(0.955033,0.598808,0); rgb(167pt)=(0.960233,0.600507,0); rgb(168pt)=(0.965232,0.602233,0); rgb(169pt)=(0.969992,0.603992,0); rgb(170pt)=(0.974475,0.605791,0); rgb(171pt)=(0.978643,0.607636,0); rgb(172pt)=(0.98246,0.609532,0); rgb(173pt)=(0.985886,0.611486,0); rgb(174pt)=(0.988885,0.613503,0); rgb(175pt)=(0.991419,0.61559,0); rgb(176pt)=(0.99345,0.617753,0); rgb(177pt)=(0.99494,0.619997,0); rgb(178pt)=(0.995851,0.622329,0); rgb(179pt)=(0.996226,0.624763,0); rgb(180pt)=(0.996512,0.627352,0); rgb(181pt)=(0.996788,0.630095,0); rgb(182pt)=(0.997053,0.632982,0); rgb(183pt)=(0.997308,0.636004,0); rgb(184pt)=(0.997552,0.639152,0); rgb(185pt)=(0.997785,0.642416,0); rgb(186pt)=(0.998006,0.645786,0); rgb(187pt)=(0.998217,0.649253,0); rgb(188pt)=(0.998416,0.652807,0); rgb(189pt)=(0.998605,0.656439,0); rgb(190pt)=(0.998781,0.660138,0); rgb(191pt)=(0.998946,0.663897,0); rgb(192pt)=(0.9991,0.667704,0); rgb(193pt)=(0.999242,0.67155,0); rgb(194pt)=(0.999372,0.675427,0); rgb(195pt)=(0.99949,0.679323,0); rgb(196pt)=(0.999596,0.68323,0); rgb(197pt)=(0.99969,0.687139,0); rgb(198pt)=(0.999771,0.691039,0); rgb(199pt)=(0.999841,0.694921,0); rgb(200pt)=(0.999898,0.698775,0); rgb(201pt)=(0.999942,0.702592,0); rgb(202pt)=(0.999974,0.706363,0); rgb(203pt)=(0.999994,0.710077,0); rgb(204pt)=(1,0.713725,0); rgb(205pt)=(1,0.717341,0); rgb(206pt)=(1,0.720963,0); rgb(207pt)=(1,0.724591,0); rgb(208pt)=(1,0.728226,0); rgb(209pt)=(1,0.731867,0); rgb(210pt)=(1,0.735514,0); rgb(211pt)=(1,0.739167,0); rgb(212pt)=(1,0.742827,0); rgb(213pt)=(1,0.746493,0); rgb(214pt)=(1,0.750165,0); rgb(215pt)=(1,0.753843,0); rgb(216pt)=(1,0.757527,0); rgb(217pt)=(1,0.761217,0); rgb(218pt)=(1,0.764913,0); rgb(219pt)=(1,0.768615,0); rgb(220pt)=(1,0.772324,0); rgb(221pt)=(1,0.776038,0); rgb(222pt)=(1,0.779758,0); rgb(223pt)=(1,0.783484,0); rgb(224pt)=(1,0.787215,0); rgb(225pt)=(1,0.790953,0); rgb(226pt)=(1,0.794696,0); rgb(227pt)=(1,0.798445,0); rgb(228pt)=(1,0.8022,0); rgb(229pt)=(1,0.805961,0); rgb(230pt)=(1,0.809727,0); rgb(231pt)=(1,0.8135,0); rgb(232pt)=(1,0.817278,0); rgb(233pt)=(1,0.821063,0); rgb(234pt)=(1,0.824854,0); rgb(235pt)=(1,0.828652,0); rgb(236pt)=(1,0.832455,0); rgb(237pt)=(1,0.836265,0); rgb(238pt)=(1,0.840081,0); rgb(239pt)=(1,0.843903,0); rgb(240pt)=(1,0.847732,0); rgb(241pt)=(1,0.851566,0); rgb(242pt)=(1,0.855406,0); rgb(243pt)=(1,0.859253,0); rgb(244pt)=(1,0.863106,0); rgb(245pt)=(1,0.866964,0); rgb(246pt)=(1,0.870829,0); rgb(247pt)=(1,0.8747,0); rgb(248pt)=(1,0.878577,0); rgb(249pt)=(1,0.88246,0); rgb(250pt)=(1,0.886349,0); rgb(251pt)=(1,0.890243,0); rgb(252pt)=(1,0.894144,0); rgb(253pt)=(1,0.898051,0); rgb(254pt)=(1,0.901964,0); rgb(255pt)=(1,0.905882,0)},
mesh/rows=49]
table[row sep=crcr,header=false] {%
%
56	0.093	0.479225561723723\\
56	0.0936	0.884848835093663\\
56	0.0942	1.29047210846358\\
56	0.0948	1.69609538183352\\
56	0.0954	2.10171865520346\\
56	0.096	2.5073419285734\\
56	0.0966	2.91296520194331\\
56	0.0972	3.31858847531325\\
56	0.0978	3.72421174868319\\
56	0.0984	4.12983502205313\\
56	0.099	4.53545829542304\\
56	0.0996	4.94108156879298\\
56	0.1002	5.34670484216292\\
56	0.1008	5.75232811553286\\
56	0.1014	6.15795138890277\\
56	0.102	6.56357466227269\\
56	0.1026	6.96919793564263\\
56	0.1032	7.37482120901257\\
56	0.1038	7.78044448238248\\
56	0.1044	8.18606775575242\\
56	0.105	8.59169102912236\\
56	0.1056	8.9973143024923\\
56	0.1062	9.40293757586221\\
56	0.1068	9.80856084923215\\
56	0.1074	10.2141841226021\\
56	0.108	10.619807395972\\
56	0.1086	11.0254306693419\\
56	0.1092	11.4310539427119\\
56	0.1098	11.8366772160818\\
56	0.1104	12.2423004894517\\
56	0.111	12.6479237628217\\
56	0.1116	13.0535470361916\\
56	0.1122	13.4591703095615\\
56	0.1128	13.8647935829315\\
56	0.1134	14.2704168563014\\
56	0.114	14.6760401296713\\
56	0.1146	15.0816634030413\\
56	0.1152	15.4872866764112\\
56	0.1158	15.8929099497811\\
56	0.1164	16.298533223151\\
56	0.117	16.704156496521\\
56	0.1176	17.1097797698909\\
56	0.1182	17.5154030432608\\
56	0.1188	17.9210263166308\\
56	0.1194	18.3266495900007\\
56	0.12	18.7322728633707\\
56	0.1206	19.1378961367406\\
56	0.1212	19.5435194101105\\
56	0.1218	19.9491426834804\\
56	0.1224	20.3547659568504\\
56	0.123	20.7603892302203\\
56.375	0.093	0.387367311794719\\
56.375	0.0936	0.786434315935082\\
56.375	0.0942	1.18550132007545\\
56.375	0.0948	1.58456832421581\\
56.375	0.0954	1.9836353283562\\
56.375	0.096	2.38270233249656\\
56.375	0.0966	2.78176933663693\\
56.375	0.0972	3.18083634077729\\
56.375	0.0978	3.57990334491768\\
56.375	0.0984	3.97897034905805\\
56.375	0.099	4.37803735319841\\
56.375	0.0996	4.77710435733877\\
56.375	0.1002	5.17617136147916\\
56.375	0.1008	5.57523836561953\\
56.375	0.1014	5.97430536975989\\
56.375	0.102	6.37337237390025\\
56.375	0.1026	6.77243937804064\\
56.375	0.1032	7.17150638218101\\
56.375	0.1038	7.57057338632137\\
56.375	0.1044	7.96964039046173\\
56.375	0.105	8.36870739460213\\
56.375	0.1056	8.76777439874249\\
56.375	0.1062	9.16684140288285\\
56.375	0.1068	9.56590840702322\\
56.375	0.1074	9.96497541116361\\
56.375	0.108	10.3640424153039\\
56.375	0.1086	10.7631094194443\\
56.375	0.1092	11.1621764235847\\
56.375	0.1098	11.5612434277251\\
56.375	0.1104	11.9603104318654\\
56.375	0.111	12.3593774360058\\
56.375	0.1116	12.7584444401462\\
56.375	0.1122	13.1575114442865\\
56.375	0.1128	13.5565784484269\\
56.375	0.1134	13.9556454525673\\
56.375	0.114	14.3547124567077\\
56.375	0.1146	14.7537794608481\\
56.375	0.1152	15.1528464649884\\
56.375	0.1158	15.5519134691288\\
56.375	0.1164	15.9509804732691\\
56.375	0.117	16.3500474774095\\
56.375	0.1176	16.7491144815499\\
56.375	0.1182	17.1481814856903\\
56.375	0.1188	17.5472484898306\\
56.375	0.1194	17.946315493971\\
56.375	0.12	18.3453824981114\\
56.375	0.1206	18.7444495022517\\
56.375	0.1212	19.1435165063921\\
56.375	0.1218	19.5425835105325\\
56.375	0.1224	19.9416505146729\\
56.375	0.123	20.3407175188132\\
56.75	0.093	0.295509061865687\\
56.75	0.0936	0.688019796776501\\
56.75	0.0942	1.08053053168729\\
56.75	0.0948	1.4730412665981\\
56.75	0.0954	1.86555200150892\\
56.75	0.096	2.25806273641973\\
56.75	0.0966	2.65057347133052\\
56.75	0.0972	3.04308420624133\\
56.75	0.0978	3.43559494115215\\
56.75	0.0984	3.82810567606296\\
56.75	0.099	4.22061641097378\\
56.75	0.0996	4.61312714588456\\
56.75	0.1002	5.0056378807954\\
56.75	0.1008	5.39814861570622\\
56.75	0.1014	5.79065935061701\\
56.75	0.102	6.18317008552782\\
56.75	0.1026	6.57568082043863\\
56.75	0.1032	6.96819155534945\\
56.75	0.1038	7.36070229026024\\
56.75	0.1044	7.75321302517105\\
56.75	0.105	8.14572376008186\\
56.75	0.1056	8.53823449499268\\
56.75	0.1062	8.93074522990347\\
56.75	0.1068	9.32325596481428\\
56.75	0.1074	9.71576669972509\\
56.75	0.108	10.1082774346359\\
56.75	0.1086	10.5007881695467\\
56.75	0.1092	10.8932989044575\\
56.75	0.1098	11.2858096393683\\
56.75	0.1104	11.6783203742791\\
56.75	0.111	12.07083110919\\
56.75	0.1116	12.4633418441008\\
56.75	0.1122	12.8558525790116\\
56.75	0.1128	13.2483633139224\\
56.75	0.1134	13.6408740488332\\
56.75	0.114	14.033384783744\\
56.75	0.1146	14.4258955186548\\
56.75	0.1152	14.8184062535656\\
56.75	0.1158	15.2109169884764\\
56.75	0.1164	15.6034277233872\\
56.75	0.117	15.995938458298\\
56.75	0.1176	16.3884491932089\\
56.75	0.1182	16.7809599281196\\
56.75	0.1188	17.1734706630304\\
56.75	0.1194	17.5659813979412\\
56.75	0.12	17.9584921328521\\
56.75	0.1206	18.3510028677629\\
56.75	0.1212	18.7435136026737\\
56.75	0.1218	19.1360243375845\\
56.75	0.1224	19.5285350724953\\
56.75	0.123	19.9210458074061\\
57.125	0.093	0.203650811936626\\
57.125	0.0936	0.589605277617892\\
57.125	0.0942	0.975559743299129\\
57.125	0.0948	1.36151420898037\\
57.125	0.0954	1.74746867466163\\
57.125	0.096	2.13342314034287\\
57.125	0.0966	2.51937760602411\\
57.125	0.0972	2.90533207170535\\
57.125	0.0978	3.29128653738661\\
57.125	0.0984	3.67724100306788\\
57.125	0.099	4.06319546874909\\
57.125	0.0996	4.44914993443035\\
57.125	0.1002	4.83510440011162\\
57.125	0.1008	5.22105886579286\\
57.125	0.1014	5.60701333147409\\
57.125	0.102	5.99296779715533\\
57.125	0.1026	6.3789222628366\\
57.125	0.1032	6.76487672851786\\
57.125	0.1038	7.15083119419907\\
57.125	0.1044	7.53678565988034\\
57.125	0.105	7.92274012556157\\
57.125	0.1056	8.30869459124284\\
57.125	0.1062	8.69464905692408\\
57.125	0.1068	9.08060352260532\\
57.125	0.1074	9.46655798828658\\
57.125	0.108	9.85251245396782\\
57.125	0.1086	10.2384669196491\\
57.125	0.1092	10.6244213853303\\
57.125	0.1098	11.0103758510116\\
57.125	0.1104	11.3963303166928\\
57.125	0.111	11.7822847823741\\
57.125	0.1116	12.1682392480553\\
57.125	0.1122	12.5541937137365\\
57.125	0.1128	12.9401481794178\\
57.125	0.1134	13.326102645099\\
57.125	0.114	13.7120571107803\\
57.125	0.1146	14.0980115764615\\
57.125	0.1152	14.4839660421428\\
57.125	0.1158	14.869920507824\\
57.125	0.1164	15.2558749735053\\
57.125	0.117	15.6418294391865\\
57.125	0.1176	16.0277839048678\\
57.125	0.1182	16.413738370549\\
57.125	0.1188	16.7996928362303\\
57.125	0.1194	17.1856473019115\\
57.125	0.12	17.5716017675928\\
57.125	0.1206	17.957556233274\\
57.125	0.1212	18.3435106989552\\
57.125	0.1218	18.7294651646365\\
57.125	0.1224	19.1154196303178\\
57.125	0.123	19.501374095999\\
57.5	0.093	0.111792562007622\\
57.5	0.0936	0.491190758459311\\
57.5	0.0942	0.870588954910971\\
57.5	0.0948	1.24998715136266\\
57.5	0.0954	1.62938534781435\\
57.5	0.096	2.00878354426604\\
57.5	0.0966	2.38818174071773\\
57.5	0.0972	2.76757993716942\\
57.5	0.0978	3.1469781336211\\
57.5	0.0984	3.52637633007279\\
57.5	0.099	3.90577452652445\\
57.5	0.0996	4.28517272297614\\
57.5	0.1002	4.66457091942786\\
57.5	0.1008	5.04396911587955\\
57.5	0.1014	5.42336731233121\\
57.5	0.102	5.8027655087829\\
57.5	0.1026	6.18216370523459\\
57.5	0.1032	6.56156190168628\\
57.5	0.1038	6.94096009813796\\
57.5	0.1044	7.32035829458965\\
57.5	0.105	7.69975649104134\\
57.5	0.1056	8.07915468749303\\
57.5	0.1062	8.45855288394469\\
57.5	0.1068	8.83795108039641\\
57.5	0.1074	9.2173492768481\\
57.5	0.108	9.59674747329976\\
57.5	0.1086	9.97614566975145\\
57.5	0.1092	10.3555438662031\\
57.5	0.1098	10.7349420626548\\
57.5	0.1104	11.1143402591065\\
57.5	0.111	11.4937384555582\\
57.5	0.1116	11.8731366520099\\
57.5	0.1122	12.2525348484616\\
57.5	0.1128	12.6319330449133\\
57.5	0.1134	13.0113312413649\\
57.5	0.114	13.3907294378167\\
57.5	0.1146	13.7701276342683\\
57.5	0.1152	14.14952583072\\
57.5	0.1158	14.5289240271717\\
57.5	0.1164	14.9083222236233\\
57.5	0.117	15.287720420075\\
57.5	0.1176	15.6671186165268\\
57.5	0.1182	16.0465168129784\\
57.5	0.1188	16.4259150094301\\
57.5	0.1194	16.8053132058818\\
57.5	0.12	17.1847114023335\\
57.5	0.1206	17.5641095987852\\
57.5	0.1212	17.9435077952369\\
57.5	0.1218	18.3229059916885\\
57.5	0.1224	18.7023041881403\\
57.5	0.123	19.0817023845919\\
57.875	0.093	0.0199343120785898\\
57.875	0.0936	0.39277623930073\\
57.875	0.0942	0.765618166522813\\
57.875	0.0948	1.13846009374495\\
57.875	0.0954	1.51130202096709\\
57.875	0.096	1.88414394818923\\
57.875	0.0966	2.25698587541132\\
57.875	0.0972	2.62982780263346\\
57.875	0.0978	3.0026697298556\\
57.875	0.0984	3.37551165707771\\
57.875	0.099	3.74835358429982\\
57.875	0.0996	4.12119551152196\\
57.875	0.1002	4.4940374387441\\
57.875	0.1008	4.86687936596621\\
57.875	0.1014	5.23972129318832\\
57.875	0.102	5.61256322041046\\
57.875	0.1026	5.98540514763258\\
57.875	0.1032	6.35824707485472\\
57.875	0.1038	6.73108900207683\\
57.875	0.1044	7.10393092929897\\
57.875	0.105	7.47677285652108\\
57.875	0.1056	7.84961478374322\\
57.875	0.1062	8.22245671096533\\
57.875	0.1068	8.59529863818747\\
57.875	0.1074	8.96814056540958\\
57.875	0.108	9.3409824926317\\
57.875	0.1086	9.71382441985384\\
57.875	0.1092	10.0866663470759\\
57.875	0.1098	10.4595082742981\\
57.875	0.1104	10.8323502015202\\
57.875	0.111	11.2051921287423\\
57.875	0.1116	11.5780340559645\\
57.875	0.1122	11.9508759831866\\
57.875	0.1128	12.3237179104087\\
57.875	0.1134	12.6965598376308\\
57.875	0.114	13.069401764853\\
57.875	0.1146	13.4422436920751\\
57.875	0.1152	13.8150856192972\\
57.875	0.1158	14.1879275465193\\
57.875	0.1164	14.5607694737414\\
57.875	0.117	14.9336114009636\\
57.875	0.1176	15.3064533281857\\
57.875	0.1182	15.6792952554078\\
57.875	0.1188	16.0521371826299\\
57.875	0.1194	16.4249791098521\\
57.875	0.12	16.7978210370742\\
57.875	0.1206	17.1706629642963\\
57.875	0.1212	17.5435048915184\\
57.875	0.1218	17.9163468187406\\
57.875	0.1224	18.2891887459627\\
57.875	0.123	18.6620306731848\\
58.25	0.093	-0.0719239378504426\\
58.25	0.0936	0.29436172014212\\
58.25	0.0942	0.660647378134684\\
58.25	0.0948	1.02693303612725\\
58.25	0.0954	1.39321869411981\\
58.25	0.096	1.7595043521124\\
58.25	0.0966	2.12579001010494\\
58.25	0.0972	2.4920756680975\\
58.25	0.0978	2.85836132609006\\
58.25	0.0984	3.22464698408265\\
58.25	0.099	3.59093264207519\\
58.25	0.0996	3.95721830006775\\
58.25	0.1002	4.32350395806034\\
58.25	0.1008	4.68978961605291\\
58.25	0.1014	5.05607527404544\\
58.25	0.102	5.422360932038\\
58.25	0.1026	5.7886465900306\\
58.25	0.1032	6.15493224802316\\
58.25	0.1038	6.52121790601569\\
58.25	0.1044	6.88750356400826\\
58.25	0.105	7.25378922200085\\
58.25	0.1056	7.62007487999341\\
58.25	0.1062	7.98636053798595\\
58.25	0.1068	8.35264619597854\\
58.25	0.1074	8.7189318539711\\
58.25	0.108	9.08521751196363\\
58.25	0.1086	9.45150316995623\\
58.25	0.1092	9.81778882794879\\
58.25	0.1098	10.1840744859414\\
58.25	0.1104	10.5503601439339\\
58.25	0.111	10.9166458019265\\
58.25	0.1116	11.2829314599191\\
58.25	0.1122	11.6492171179116\\
58.25	0.1128	12.0155027759042\\
58.25	0.1134	12.3817884338967\\
58.25	0.114	12.7480740918893\\
58.25	0.1146	13.1143597498819\\
58.25	0.1152	13.4806454078744\\
58.25	0.1158	13.846931065867\\
58.25	0.1164	14.2132167238595\\
58.25	0.117	14.5795023818521\\
58.25	0.1176	14.9457880398447\\
58.25	0.1182	15.3120736978372\\
58.25	0.1188	15.6783593558298\\
58.25	0.1194	16.0446450138224\\
58.25	0.12	16.410930671815\\
58.25	0.1206	16.7772163298075\\
58.25	0.1212	17.1435019878001\\
58.25	0.1218	17.5097876457926\\
58.25	0.1224	17.8760733037852\\
58.25	0.123	18.2423589617777\\
58.625	0.093	-0.163782187779475\\
58.625	0.0936	0.195947200983539\\
58.625	0.0942	0.555676589746525\\
58.625	0.0948	0.91540597850954\\
58.625	0.0954	1.27513536727255\\
58.625	0.096	1.63486475603557\\
58.625	0.0966	1.99459414479855\\
58.625	0.0972	2.35432353356154\\
58.625	0.0978	2.71405292232456\\
58.625	0.0984	3.07378231108757\\
58.625	0.099	3.43351169985056\\
58.625	0.0996	3.79324108861357\\
58.625	0.1002	4.15297047737658\\
58.625	0.1008	4.51269986613957\\
58.625	0.1014	4.87242925490256\\
58.625	0.102	5.23215864366557\\
58.625	0.1026	5.59188803242859\\
58.625	0.1032	5.9516174211916\\
58.625	0.1038	6.31134680995459\\
58.625	0.1044	6.67107619871757\\
58.625	0.105	7.03080558748059\\
58.625	0.1056	7.3905349762436\\
58.625	0.1062	7.75026436500659\\
58.625	0.1068	8.1099937537696\\
58.625	0.1074	8.46972314253262\\
58.625	0.108	8.82945253129557\\
58.625	0.1086	9.18918192005859\\
58.625	0.1092	9.5489113088216\\
58.625	0.1098	9.90864069758462\\
58.625	0.1104	10.2683700863476\\
58.625	0.111	10.6280994751106\\
58.625	0.1116	10.9878288638737\\
58.625	0.1122	11.3475582526366\\
58.625	0.1128	11.7072876413996\\
58.625	0.1134	12.0670170301626\\
58.625	0.114	12.4267464189257\\
58.625	0.1146	12.7864758076886\\
58.625	0.1152	13.1462051964517\\
58.625	0.1158	13.5059345852146\\
58.625	0.1164	13.8656639739776\\
58.625	0.117	14.2253933627406\\
58.625	0.1176	14.5851227515037\\
58.625	0.1182	14.9448521402666\\
58.625	0.1188	15.3045815290296\\
58.625	0.1194	15.6643109177926\\
58.625	0.12	16.0240403065557\\
58.625	0.1206	16.3837696953186\\
58.625	0.1212	16.7434990840816\\
58.625	0.1218	17.1032284728446\\
58.625	0.1224	17.4629578616077\\
58.625	0.123	17.8226872503707\\
59	0.093	-0.255640437708479\\
59	0.0936	0.0975326818249584\\
59	0.0942	0.450705801358396\\
59	0.0948	0.803878920891833\\
59	0.0954	1.15705204042527\\
59	0.096	1.51022515995874\\
59	0.0966	1.86339827949215\\
59	0.0972	2.21657139902558\\
59	0.0978	2.56974451855905\\
59	0.0984	2.92291763809249\\
59	0.099	3.27609075762592\\
59	0.0996	3.62926387715936\\
59	0.1002	3.98243699669283\\
59	0.1008	4.33561011622626\\
59	0.1014	4.68878323575967\\
59	0.102	5.04195635529314\\
59	0.1026	5.39512947482658\\
59	0.1032	5.74830259436004\\
59	0.1038	6.10147571389345\\
59	0.1044	6.45464883342689\\
59	0.105	6.80782195296035\\
59	0.1056	7.16099507249379\\
59	0.1062	7.51416819202723\\
59	0.1068	7.86734131156066\\
59	0.1074	8.2205144310941\\
59	0.108	8.57368755062754\\
59	0.1086	8.92686067016098\\
59	0.1092	9.28003378969441\\
59	0.1098	9.63320690922788\\
59	0.1104	9.98638002876129\\
59	0.111	10.3395531482948\\
59	0.1116	10.6927262678282\\
59	0.1122	11.0458993873616\\
59	0.1128	11.3990725068951\\
59	0.1134	11.7522456264285\\
59	0.114	12.105418745962\\
59	0.1146	12.4585918654954\\
59	0.1152	12.8117649850288\\
59	0.1158	13.1649381045623\\
59	0.1164	13.5181112240957\\
59	0.117	13.8712843436292\\
59	0.1176	14.2244574631626\\
59	0.1182	14.577630582696\\
59	0.1188	14.9308037022295\\
59	0.1194	15.2839768217629\\
59	0.12	15.6371499412964\\
59	0.1206	15.9903230608298\\
59	0.1212	16.3434961803632\\
59	0.1218	16.6966692998967\\
59	0.1224	17.0498424194301\\
59	0.123	17.4030155389636\\
59.375	0.093	-0.347498687637511\\
59.375	0.0936	-0.00088183733362257\\
59.375	0.0942	0.345735012970238\\
59.375	0.0948	0.692351863274126\\
59.375	0.0954	1.03896871357802\\
59.375	0.096	1.3855855638819\\
59.375	0.0966	1.73220241418576\\
59.375	0.0972	2.07881926448965\\
59.375	0.0978	2.42543611479354\\
59.375	0.0984	2.77205296509743\\
59.375	0.099	3.11866981540129\\
59.375	0.0996	3.46528666570515\\
59.375	0.1002	3.81190351600904\\
59.375	0.1008	4.15852036631293\\
59.375	0.1014	4.50513721661679\\
59.375	0.102	4.85175406692068\\
59.375	0.1026	5.19837091722457\\
59.375	0.1032	5.54498776752845\\
59.375	0.1038	5.89160461783231\\
59.375	0.1044	6.2382214681362\\
59.375	0.105	6.58483831844009\\
59.375	0.1056	6.93145516874398\\
59.375	0.1062	7.27807201904784\\
59.375	0.1068	7.62468886935173\\
59.375	0.1074	7.97130571965562\\
59.375	0.108	8.31792256995948\\
59.375	0.1086	8.66453942026337\\
59.375	0.1092	9.01115627056726\\
59.375	0.1098	9.35777312087114\\
59.375	0.1104	9.704389971175\\
59.375	0.111	10.0510068214789\\
59.375	0.1116	10.3976236717828\\
59.375	0.1122	10.7442405220866\\
59.375	0.1128	11.0908573723906\\
59.375	0.1134	11.4374742226944\\
59.375	0.114	11.7840910729983\\
59.375	0.1146	12.1307079233022\\
59.375	0.1152	12.4773247736061\\
59.375	0.1158	12.8239416239099\\
59.375	0.1164	13.1705584742138\\
59.375	0.117	13.5171753245177\\
59.375	0.1176	13.8637921748216\\
59.375	0.1182	14.2104090251254\\
59.375	0.1188	14.5570258754293\\
59.375	0.1194	14.9036427257332\\
59.375	0.12	15.2502595760371\\
59.375	0.1206	15.596876426341\\
59.375	0.1212	15.9434932766448\\
59.375	0.1218	16.2901101269487\\
59.375	0.1224	16.6367269772526\\
59.375	0.123	16.9833438275565\\
59.75	0.093	-0.439356937566544\\
59.75	0.0936	-0.0992963564922036\\
59.75	0.0942	0.24076422458208\\
59.75	0.0948	0.58082480565642\\
59.75	0.0954	0.920885386730731\\
59.75	0.096	1.26094596780507\\
59.75	0.0966	1.60100654887935\\
59.75	0.0972	1.94106712995369\\
59.75	0.0978	2.28112771102801\\
59.75	0.0984	2.62118829210235\\
59.75	0.099	2.96124887317666\\
59.75	0.0996	3.30130945425097\\
59.75	0.1002	3.64137003532528\\
59.75	0.1008	3.98143061639962\\
59.75	0.1014	4.32149119747393\\
59.75	0.102	4.66155177854824\\
59.75	0.1026	5.00161235962258\\
59.75	0.1032	5.3416729406969\\
59.75	0.1038	5.68173352177121\\
59.75	0.1044	6.02179410284552\\
59.75	0.105	6.36185468391986\\
59.75	0.1056	6.70191526499417\\
59.75	0.1062	7.04197584606848\\
59.75	0.1068	7.38203642714279\\
59.75	0.1074	7.72209700821713\\
59.75	0.108	8.06215758929142\\
59.75	0.1086	8.40221817036576\\
59.75	0.1092	8.74227875144007\\
59.75	0.1098	9.08233933251441\\
59.75	0.1104	9.42239991358869\\
59.75	0.111	9.76246049466303\\
59.75	0.1116	10.1025210757374\\
59.75	0.1122	10.4425816568117\\
59.75	0.1128	10.782642237886\\
59.75	0.1134	11.1227028189603\\
59.75	0.114	11.4627634000346\\
59.75	0.1146	11.802823981109\\
59.75	0.1152	12.1428845621833\\
59.75	0.1158	12.4829451432576\\
59.75	0.1164	12.8230057243319\\
59.75	0.117	13.1630663054062\\
59.75	0.1176	13.5031268864805\\
59.75	0.1182	13.8431874675549\\
59.75	0.1188	14.1832480486291\\
59.75	0.1194	14.5233086297035\\
59.75	0.12	14.8633692107778\\
59.75	0.1206	15.2034297918521\\
59.75	0.1212	15.5434903729264\\
59.75	0.1218	15.8835509540008\\
59.75	0.1224	16.2236115350751\\
59.75	0.123	16.5636721161494\\
60.125	0.093	-0.531215187495548\\
60.125	0.0936	-0.197710875650756\\
60.125	0.0942	0.135793436193978\\
60.125	0.0948	0.469297748038741\\
60.125	0.0954	0.802802059883504\\
60.125	0.096	1.13630637172827\\
60.125	0.0966	1.469810683573\\
60.125	0.0972	1.80331499541776\\
60.125	0.0978	2.13681930726253\\
60.125	0.0984	2.47032361910729\\
60.125	0.099	2.80382793095202\\
60.125	0.0996	3.13733224279679\\
60.125	0.1002	3.47083655464155\\
60.125	0.1008	3.80434086648631\\
60.125	0.1014	4.13784517833105\\
60.125	0.102	4.47134949017581\\
60.125	0.1026	4.8048538020206\\
60.125	0.1032	5.13835811386537\\
60.125	0.1038	5.4718624257101\\
60.125	0.1044	5.80536673755486\\
60.125	0.105	6.13887104939963\\
60.125	0.1056	6.47237536124439\\
60.125	0.1062	6.80587967308912\\
60.125	0.1068	7.13938398493389\\
60.125	0.1074	7.47288829677865\\
60.125	0.108	7.80639260862338\\
60.125	0.1086	8.13989692046815\\
60.125	0.1092	8.47340123231291\\
60.125	0.1098	8.80690554415767\\
60.125	0.1104	9.14040985600244\\
60.125	0.111	9.4739141678472\\
60.125	0.1116	9.80741847969199\\
60.125	0.1122	10.1409227915367\\
60.125	0.1128	10.4744271033815\\
60.125	0.1134	10.8079314152262\\
60.125	0.114	11.141435727071\\
60.125	0.1146	11.4749400389157\\
60.125	0.1152	11.8084443507605\\
60.125	0.1158	12.1419486626052\\
60.125	0.1164	12.47545297445\\
60.125	0.117	12.8089572862947\\
60.125	0.1176	13.1424615981395\\
60.125	0.1182	13.4759659099843\\
60.125	0.1188	13.809470221829\\
60.125	0.1194	14.1429745336738\\
60.125	0.12	14.4764788455186\\
60.125	0.1206	14.8099831573633\\
60.125	0.1212	15.1434874692081\\
60.125	0.1218	15.4769917810528\\
60.125	0.1224	15.8104960928976\\
60.125	0.123	16.1440004047423\\
60.5	0.093	-0.623073437424551\\
60.5	0.0936	-0.296125394809366\\
60.5	0.0942	0.0308226478058202\\
60.5	0.0948	0.357770690421034\\
60.5	0.0954	0.68471873303622\\
60.5	0.096	1.01166677565143\\
60.5	0.0966	1.33861481826662\\
60.5	0.0972	1.66556286088181\\
60.5	0.0978	1.99251090349702\\
60.5	0.0984	2.31945894611221\\
60.5	0.099	2.64640698872739\\
60.5	0.0996	2.97335503134261\\
60.5	0.1002	3.30030307395779\\
60.5	0.1008	3.62725111657301\\
60.5	0.1014	3.95419915918819\\
60.5	0.102	4.28114720180338\\
60.5	0.1026	4.60809524441859\\
60.5	0.1032	4.93504328703378\\
60.5	0.1038	5.26199132964896\\
60.5	0.1044	5.58893937226418\\
60.5	0.105	5.91588741487936\\
60.5	0.1056	6.24283545749458\\
60.5	0.1062	6.56978350010976\\
60.5	0.1068	6.89673154272495\\
60.5	0.1074	7.22367958534016\\
60.5	0.108	7.55062762795532\\
60.5	0.1086	7.87757567057054\\
60.5	0.1092	8.20452371318575\\
60.5	0.1098	8.53147175580094\\
60.5	0.1104	8.85841979841612\\
60.5	0.111	9.18536784103134\\
60.5	0.1116	9.51231588364655\\
60.5	0.1122	9.83926392626171\\
60.5	0.1128	10.1662119688769\\
60.5	0.1134	10.4931600114921\\
60.5	0.114	10.8201080541074\\
60.5	0.1146	11.1470560967225\\
60.5	0.1152	11.4740041393377\\
60.5	0.1158	11.8009521819529\\
60.5	0.1164	12.1279002245681\\
60.5	0.117	12.4548482671833\\
60.5	0.1176	12.7817963097985\\
60.5	0.1182	13.1087443524137\\
60.5	0.1188	13.4356923950289\\
60.5	0.1194	13.7626404376441\\
60.5	0.12	14.0895884802593\\
60.5	0.1206	14.4165365228745\\
60.5	0.1212	14.7434845654896\\
60.5	0.1218	15.0704326081049\\
60.5	0.1224	15.3973806507201\\
60.5	0.123	15.7243286933353\\
60.875	0.093	-0.714931687353584\\
60.875	0.0936	-0.394539913967947\\
60.875	0.0942	-0.0741481405823379\\
60.875	0.0948	0.246243632803328\\
60.875	0.0954	0.566635406188965\\
60.875	0.096	0.887027179574602\\
60.875	0.0966	1.20741895296021\\
60.875	0.0972	1.52781072634585\\
60.875	0.0978	1.84820249973149\\
60.875	0.0984	2.16859427311715\\
60.875	0.099	2.48898604650276\\
60.875	0.0996	2.8093778198884\\
60.875	0.1002	3.12976959327403\\
60.875	0.1008	3.45016136665967\\
60.875	0.1014	3.77055314004531\\
60.875	0.102	4.09094491343095\\
60.875	0.1026	4.41133668681658\\
60.875	0.1032	4.73172846020222\\
60.875	0.1038	5.05212023358783\\
60.875	0.1044	5.37251200697349\\
60.875	0.105	5.69290378035913\\
60.875	0.1056	6.01329555374477\\
60.875	0.1062	6.33368732713038\\
60.875	0.1068	6.65407910051601\\
60.875	0.1074	6.97447087390165\\
60.875	0.108	7.29486264728729\\
60.875	0.1086	7.61525442067293\\
60.875	0.1092	7.93564619405856\\
60.875	0.1098	8.2560379674442\\
60.875	0.1104	8.57642974082981\\
60.875	0.111	8.89682151421547\\
60.875	0.1116	9.21721328760114\\
60.875	0.1122	9.53760506098672\\
60.875	0.1128	9.85799683437239\\
60.875	0.1134	10.178388607758\\
60.875	0.114	10.4987803811437\\
60.875	0.1146	10.8191721545293\\
60.875	0.1152	11.1395639279149\\
60.875	0.1158	11.4599557013005\\
60.875	0.1164	11.7803474746862\\
60.875	0.117	12.1007392480718\\
60.875	0.1176	12.4211310214575\\
60.875	0.1182	12.7415227948431\\
60.875	0.1188	13.0619145682287\\
60.875	0.1194	13.3823063416143\\
60.875	0.12	13.702698115\\
60.875	0.1206	14.0230898883856\\
60.875	0.1212	14.3434816617712\\
60.875	0.1218	14.6638734351569\\
60.875	0.1224	14.9842652085426\\
60.875	0.123	15.3046569819282\\
61.25	0.093	-0.806789937282616\\
61.25	0.0936	-0.492954433126528\\
61.25	0.0942	-0.179118928970468\\
61.25	0.0948	0.134716575185621\\
61.25	0.0954	0.448552079341681\\
61.25	0.096	0.76238758349777\\
61.25	0.0966	1.07622308765383\\
61.25	0.0972	1.39005859180989\\
61.25	0.0978	1.70389409596598\\
61.25	0.0984	2.01772960012207\\
61.25	0.099	2.33156510427813\\
61.25	0.0996	2.64540060843419\\
61.25	0.1002	2.95923611259028\\
61.25	0.1008	3.27307161674636\\
61.25	0.1014	3.58690712090242\\
61.25	0.102	3.90074262505848\\
61.25	0.1026	4.21457812921457\\
61.25	0.1032	4.52841363337066\\
61.25	0.1038	4.84224913752672\\
61.25	0.1044	5.15608464168278\\
61.25	0.105	5.46992014583887\\
61.25	0.1056	5.78375564999496\\
61.25	0.1062	6.09759115415102\\
61.25	0.1068	6.41142665830708\\
61.25	0.1074	6.72526216246317\\
61.25	0.108	7.03909766661923\\
61.25	0.1086	7.35293317077532\\
61.25	0.1092	7.66676867493138\\
61.25	0.1098	7.98060417908746\\
61.25	0.1104	8.29443968324352\\
61.25	0.111	8.60827518739961\\
61.25	0.1116	8.9221106915557\\
61.25	0.1122	9.23594619571173\\
61.25	0.1128	9.54978169986785\\
61.25	0.1134	9.86361720402391\\
61.25	0.114	10.17745270818\\
61.25	0.1146	10.4912882123361\\
61.25	0.1152	10.8051237164921\\
61.25	0.1158	11.1189592206482\\
61.25	0.1164	11.4327947248042\\
61.25	0.117	11.7466302289603\\
61.25	0.1176	12.0604657331164\\
61.25	0.1182	12.3743012372725\\
61.25	0.1188	12.6881367414285\\
61.25	0.1194	13.0019722455846\\
61.25	0.12	13.3158077497407\\
61.25	0.1206	13.6296432538968\\
61.25	0.1212	13.9434787580528\\
61.25	0.1218	14.2573142622089\\
61.25	0.1224	14.571149766365\\
61.25	0.123	14.8849852705211\\
61.625	0.093	-0.89864818721162\\
61.625	0.0936	-0.591368952285109\\
61.625	0.0942	-0.284089717358626\\
61.625	0.0948	0.0231895175678858\\
61.625	0.0954	0.330468752494426\\
61.625	0.096	0.637747987420937\\
61.625	0.0966	0.94502722234742\\
61.625	0.0972	1.25230645727396\\
61.625	0.0978	1.55958569220047\\
61.625	0.0984	1.86686492712698\\
61.625	0.099	2.17414416205349\\
61.625	0.0996	2.48142339698001\\
61.625	0.1002	2.78870263190652\\
61.625	0.1008	3.09598186683306\\
61.625	0.1014	3.40326110175954\\
61.625	0.102	3.71054033668605\\
61.625	0.1026	4.01781957161256\\
61.625	0.1032	4.3250988065391\\
61.625	0.1038	4.63237804146559\\
61.625	0.1044	4.9396572763921\\
61.625	0.105	5.24693651131864\\
61.625	0.1056	5.55421574624515\\
61.625	0.1062	5.86149498117163\\
61.625	0.1068	6.16877421609817\\
61.625	0.1074	6.47605345102468\\
61.625	0.108	6.78333268595117\\
61.625	0.1086	7.09061192087768\\
61.625	0.1092	7.39789115580422\\
61.625	0.1098	7.70517039073073\\
61.625	0.1104	8.01244962565721\\
61.625	0.111	8.31972886058375\\
61.625	0.1116	8.62700809551029\\
61.625	0.1122	8.93428733043675\\
61.625	0.1128	9.24156656536331\\
61.625	0.1134	9.5488458002898\\
61.625	0.114	9.85612503521634\\
61.625	0.1146	10.1634042701428\\
61.625	0.1152	10.4706835050694\\
61.625	0.1158	10.7779627399958\\
61.625	0.1164	11.0852419749223\\
61.625	0.117	11.3925212098489\\
61.625	0.1176	11.6998004447754\\
61.625	0.1182	12.0070796797019\\
61.625	0.1188	12.3143589146284\\
61.625	0.1194	12.6216381495549\\
61.625	0.12	12.9289173844815\\
61.625	0.1206	13.2361966194079\\
61.625	0.1212	13.5434758543344\\
61.625	0.1218	13.850755089261\\
61.625	0.1224	14.1580343241875\\
61.625	0.123	14.465313559114\\
62	0.093	-0.990506437140652\\
62	0.0936	-0.68978347144369\\
62	0.0942	-0.389060505746755\\
62	0.0948	-0.088337540049821\\
62	0.0954	0.212385425647142\\
62	0.096	0.513108391344105\\
62	0.0966	0.813831357041039\\
62	0.0972	1.114554322738\\
62	0.0978	1.41527728843496\\
62	0.0984	1.71600025413193\\
62	0.099	2.01672321982886\\
62	0.0996	2.3174461855258\\
62	0.1002	2.61816915122276\\
62	0.1008	2.91889211691972\\
62	0.1014	3.21961508261666\\
62	0.102	3.52033804831362\\
62	0.1026	3.82106101401058\\
62	0.1032	4.12178397970754\\
62	0.1038	4.42250694540445\\
62	0.1044	4.72322991110141\\
62	0.105	5.02395287679838\\
62	0.1056	5.32467584249534\\
62	0.1062	5.62539880819227\\
62	0.1068	5.92612177388924\\
62	0.1074	6.22684473958617\\
62	0.108	6.5275677052831\\
62	0.1086	6.82829067098007\\
62	0.1092	7.12901363667703\\
62	0.1098	7.42973660237399\\
62	0.1104	7.73045956807093\\
62	0.111	8.03118253376789\\
62	0.1116	8.33190549946488\\
62	0.1122	8.63262846516176\\
62	0.1128	8.93335143085875\\
62	0.1134	9.23407439655568\\
62	0.114	9.53479736225268\\
62	0.1146	9.83552032794961\\
62	0.1152	10.1362432936466\\
62	0.1158	10.4369662593435\\
62	0.1164	10.7376892250404\\
62	0.117	11.0384121907374\\
62	0.1176	11.3391351564344\\
62	0.1182	11.6398581221313\\
62	0.1188	11.9405810878282\\
62	0.1194	12.2413040535252\\
62	0.12	12.5420270192222\\
62	0.1206	12.8427499849191\\
62	0.1212	13.143472950616\\
62	0.1218	13.444195916313\\
62	0.1224	13.74491888201\\
62	0.123	14.0456418477069\\
62.375	0.093	-1.08236468706968\\
62.375	0.0936	-0.788197990602271\\
62.375	0.0942	-0.494031294134913\\
62.375	0.0948	-0.199864597667528\\
62.375	0.0954	0.0943020987998864\\
62.375	0.096	0.388468795267272\\
62.375	0.0966	0.682635491734658\\
62.375	0.0972	0.976802188202043\\
62.375	0.0978	1.27096888466946\\
62.375	0.0984	1.56513558113684\\
62.375	0.099	1.8593022776042\\
62.375	0.0996	2.15346897407161\\
62.375	0.1002	2.447635670539\\
62.375	0.1008	2.74180236700641\\
62.375	0.1014	3.03596906347377\\
62.375	0.102	3.33013575994116\\
62.375	0.1026	3.62430245640857\\
62.375	0.1032	3.91846915287596\\
62.375	0.1038	4.21263584934334\\
62.375	0.1044	4.50680254581073\\
62.375	0.105	4.80096924227811\\
62.375	0.1056	5.09513593874553\\
62.375	0.1062	5.38930263521289\\
62.375	0.1068	5.6834693316803\\
62.375	0.1074	5.97763602814769\\
62.375	0.108	6.27180272461507\\
62.375	0.1086	6.56596942108246\\
62.375	0.1092	6.86013611754984\\
62.375	0.1098	7.15430281401726\\
62.375	0.1104	7.44846951048461\\
62.375	0.111	7.74263620695203\\
62.375	0.1116	8.03680290341944\\
62.375	0.1122	8.33096959988677\\
62.375	0.1128	8.62513629635421\\
62.375	0.1134	8.91930299282157\\
62.375	0.114	9.21346968928901\\
62.375	0.1146	9.50763638575637\\
62.375	0.1152	9.80180308222378\\
62.375	0.1158	10.0959697786911\\
62.375	0.1164	10.3901364751585\\
62.375	0.117	10.6843031716259\\
62.375	0.1176	10.9784698680933\\
62.375	0.1182	11.2726365645607\\
62.375	0.1188	11.5668032610281\\
62.375	0.1194	11.8609699574955\\
62.375	0.12	12.1551366539629\\
62.375	0.1206	12.4493033504303\\
62.375	0.1212	12.7434700468976\\
62.375	0.1218	13.037636743365\\
62.375	0.1224	13.3318034398324\\
62.375	0.123	13.6259701362998\\
62.75	0.093	-1.17422293699869\\
62.75	0.0936	-0.886612509760852\\
62.75	0.0942	-0.599002082523043\\
62.75	0.0948	-0.311391655285234\\
62.75	0.0954	-0.0237812280473975\\
62.75	0.096	0.26382919919044\\
62.75	0.0966	0.551439626428248\\
62.75	0.0972	0.839050053666085\\
62.75	0.0978	1.12666048090392\\
62.75	0.0984	1.41427090814176\\
62.75	0.099	1.70188133537957\\
62.75	0.0996	1.9894917626174\\
62.75	0.1002	2.27710218985524\\
62.75	0.1008	2.56471261709308\\
62.75	0.1014	2.85232304433089\\
62.75	0.102	3.13993347156872\\
62.75	0.1026	3.42754389880656\\
62.75	0.1032	3.7151543260444\\
62.75	0.1038	4.00276475328221\\
62.75	0.1044	4.29037518052004\\
62.75	0.105	4.57798560775788\\
62.75	0.1056	4.86559603499572\\
62.75	0.1062	5.15320646223353\\
62.75	0.1068	5.44081688947136\\
62.75	0.1074	5.7284273167092\\
62.75	0.108	6.01603774394701\\
62.75	0.1086	6.30364817118485\\
62.75	0.1092	6.59125859842268\\
62.75	0.1098	6.87886902566052\\
62.75	0.1104	7.16647945289833\\
62.75	0.111	7.45408988013617\\
62.75	0.1116	7.74170030737403\\
62.75	0.1122	8.02931073461181\\
62.75	0.1128	8.31692116184968\\
62.75	0.1134	8.60453158908749\\
62.75	0.114	8.89214201632535\\
62.75	0.1146	9.17975244356316\\
62.75	0.1152	9.46736287080097\\
62.75	0.1158	9.75497329803878\\
62.75	0.1164	10.0425837252766\\
62.75	0.117	10.3301941525144\\
62.75	0.1176	10.6178045797523\\
62.75	0.1182	10.9054150069901\\
62.75	0.1188	11.1930254342279\\
62.75	0.1194	11.4806358614657\\
62.75	0.12	11.7682462887036\\
62.75	0.1206	12.0558567159414\\
62.75	0.1212	12.3434671431792\\
62.75	0.1218	12.6310775704171\\
62.75	0.1224	12.9186879976549\\
62.75	0.123	13.2062984248927\\
63.125	0.093	-1.26608118692775\\
63.125	0.0936	-0.98502702891949\\
63.125	0.0942	-0.70397287091123\\
63.125	0.0948	-0.42291871290297\\
63.125	0.0954	-0.141864554894681\\
63.125	0.096	0.139189603113579\\
63.125	0.0966	0.420243761121839\\
63.125	0.0972	0.701297919130099\\
63.125	0.0978	0.982352077138387\\
63.125	0.0984	1.26340623514668\\
63.125	0.099	1.54446039315491\\
63.125	0.0996	1.8255145511632\\
63.125	0.1002	2.10656870917146\\
63.125	0.1008	2.38762286717974\\
63.125	0.1014	2.66867702518798\\
63.125	0.102	2.94973118319626\\
63.125	0.1026	3.23078534120452\\
63.125	0.1032	3.51183949921281\\
63.125	0.1038	3.79289365722104\\
63.125	0.1044	4.07394781522933\\
63.125	0.105	4.35500197323759\\
63.125	0.1056	4.63605613124588\\
63.125	0.1062	4.91711028925411\\
63.125	0.1068	5.1981644472624\\
63.125	0.1074	5.47921860527066\\
63.125	0.108	5.76027276327892\\
63.125	0.1086	6.04132692128721\\
63.125	0.1092	6.32238107929547\\
63.125	0.1098	6.60343523730376\\
63.125	0.1104	6.88448939531199\\
63.125	0.111	7.16554355332028\\
63.125	0.1116	7.44659771132856\\
63.125	0.1122	7.7276518693368\\
63.125	0.1128	8.00870602734508\\
63.125	0.1134	8.28976018535334\\
63.125	0.114	8.57081434336163\\
63.125	0.1146	8.85186850136989\\
63.125	0.1152	9.13292265937815\\
63.125	0.1158	9.41397681738641\\
63.125	0.1164	9.69503097539464\\
63.125	0.117	9.97608513340293\\
63.125	0.1176	10.2571392914112\\
63.125	0.1182	10.5381934494195\\
63.125	0.1188	10.8192476074277\\
63.125	0.1194	11.100301765436\\
63.125	0.12	11.3813559234443\\
63.125	0.1206	11.6624100814525\\
63.125	0.1212	11.9434642394608\\
63.125	0.1218	12.2245183974691\\
63.125	0.1224	12.5055725554774\\
63.125	0.123	12.7866267134856\\
63.5	0.093	-1.35793943685678\\
63.5	0.0936	-1.08344154807807\\
63.5	0.0942	-0.808943659299388\\
63.5	0.0948	-0.534445770520676\\
63.5	0.0954	-0.259947881741965\\
63.5	0.096	0.0145500070367746\\
63.5	0.0966	0.289047895815457\\
63.5	0.0972	0.563545784594169\\
63.5	0.0978	0.83804367337288\\
63.5	0.0984	1.11254156215159\\
63.5	0.099	1.38703945093027\\
63.5	0.0996	1.66153733970899\\
63.5	0.1002	1.9360352284877\\
63.5	0.1008	2.21053311726641\\
63.5	0.1014	2.48503100604509\\
63.5	0.102	2.7595288948238\\
63.5	0.1026	3.03402678360251\\
63.5	0.1032	3.30852467238125\\
63.5	0.1038	3.58302256115994\\
63.5	0.1044	3.85752044993865\\
63.5	0.105	4.13201833871736\\
63.5	0.1056	4.40651622749607\\
63.5	0.1062	4.68101411627475\\
63.5	0.1068	4.95551200505346\\
63.5	0.1074	5.23000989383218\\
63.5	0.108	5.50450778261086\\
63.5	0.1086	5.77900567138957\\
63.5	0.1092	6.05350356016828\\
63.5	0.1098	6.32800144894699\\
63.5	0.1104	6.6024993377257\\
63.5	0.111	6.87699722650441\\
63.5	0.1116	7.15149511528315\\
63.5	0.1122	7.42599300406181\\
63.5	0.1128	7.70049089284055\\
63.5	0.1134	7.97498878161923\\
63.5	0.114	8.24948667039797\\
63.5	0.1146	8.52398455917665\\
63.5	0.1152	8.79848244795537\\
63.5	0.1158	9.07298033673405\\
63.5	0.1164	9.34747822551273\\
63.5	0.117	9.62197611429144\\
63.5	0.1176	9.89647400307021\\
63.5	0.1182	10.1709718918489\\
63.5	0.1188	10.4454697806276\\
63.5	0.1194	10.7199676694063\\
63.5	0.12	10.994465558185\\
63.5	0.1206	11.2689634469637\\
63.5	0.1212	11.5434613357424\\
63.5	0.1218	11.8179592245211\\
63.5	0.1224	12.0924571132998\\
63.5	0.123	12.3669550020785\\
63.875	0.093	-1.44979768678581\\
63.875	0.0936	-1.18185606723665\\
63.875	0.0942	-0.913914447687517\\
63.875	0.0948	-0.645972828138383\\
63.875	0.0954	-0.378031208589221\\
63.875	0.096	-0.110089589040058\\
63.875	0.0966	0.157852030509048\\
63.875	0.0972	0.42579365005821\\
63.875	0.0978	0.693735269607373\\
63.875	0.0984	0.961676889156507\\
63.875	0.099	1.22961850870564\\
63.875	0.0996	1.49756012825478\\
63.875	0.1002	1.76550174780394\\
63.875	0.1008	2.0334433673531\\
63.875	0.1014	2.30138498690221\\
63.875	0.102	2.56932660645137\\
63.875	0.1026	2.83726822600053\\
63.875	0.1032	3.10520984554967\\
63.875	0.1038	3.3731514650988\\
63.875	0.1044	3.64109308464796\\
63.875	0.105	3.9090347041971\\
63.875	0.1056	4.17697632374626\\
63.875	0.1062	4.44491794329539\\
63.875	0.1068	4.71285956284453\\
63.875	0.1074	4.98080118239369\\
63.875	0.108	5.2487428019428\\
63.875	0.1086	5.51668442149196\\
63.875	0.1092	5.78462604104112\\
63.875	0.1098	6.05256766059026\\
63.875	0.1104	6.32050928013939\\
63.875	0.111	6.58845089968855\\
63.875	0.1116	6.85639251923772\\
63.875	0.1122	7.12433413878682\\
63.875	0.1128	7.39227575833601\\
63.875	0.1134	7.66021737788512\\
63.875	0.114	7.92815899743431\\
63.875	0.1146	8.19610061698344\\
63.875	0.1152	8.46404223653258\\
63.875	0.1158	8.73198385608171\\
63.875	0.1164	8.99992547563082\\
63.875	0.117	9.26786709517998\\
63.875	0.1176	9.53580871472917\\
63.875	0.1182	9.80375033427828\\
63.875	0.1188	10.0716919538274\\
63.875	0.1194	10.3396335733766\\
63.875	0.12	10.6075751929257\\
63.875	0.1206	10.8755168124749\\
63.875	0.1212	11.143458432024\\
63.875	0.1218	11.4114000515731\\
63.875	0.1224	11.6793416711223\\
63.875	0.123	11.9472832906715\\
64.25	0.093	-1.54165593671482\\
64.25	0.0936	-1.28027058639523\\
64.25	0.0942	-1.01888523607568\\
64.25	0.0948	-0.75749988575609\\
64.25	0.0954	-0.496114535436476\\
64.25	0.096	-0.23472918511689\\
64.25	0.0966	0.0266561652026667\\
64.25	0.0972	0.288041515522252\\
64.25	0.0978	0.549426865841838\\
64.25	0.0984	0.810812216161452\\
64.25	0.099	1.07219756648101\\
64.25	0.0996	1.33358291680059\\
64.25	0.1002	1.59496826712018\\
64.25	0.1008	1.85635361743977\\
64.25	0.1014	2.11773896775935\\
64.25	0.102	2.37912431807894\\
64.25	0.1026	2.64050966839852\\
64.25	0.1032	2.90189501871811\\
64.25	0.1038	3.16328036903766\\
64.25	0.1044	3.42466571935725\\
64.25	0.105	3.68605106967686\\
64.25	0.1056	3.94743641999645\\
64.25	0.1062	4.20882177031601\\
64.25	0.1068	4.47020712063559\\
64.25	0.1074	4.73159247095518\\
64.25	0.108	4.99297782127476\\
64.25	0.1086	5.25436317159435\\
64.25	0.1092	5.51574852191393\\
64.25	0.1098	5.77713387223352\\
64.25	0.1104	6.03851922255308\\
64.25	0.111	6.29990457287269\\
64.25	0.1116	6.56128992319231\\
64.25	0.1122	6.82267527351183\\
64.25	0.1128	7.08406062383145\\
64.25	0.1134	7.34544597415103\\
64.25	0.114	7.60683132447065\\
64.25	0.1146	7.8682166747902\\
64.25	0.1152	8.12960202510979\\
64.25	0.1158	8.39098737542935\\
64.25	0.1164	8.65237272574893\\
64.25	0.117	8.91375807606852\\
64.25	0.1176	9.17514342638813\\
64.25	0.1182	9.43652877670769\\
64.25	0.1188	9.69791412702725\\
64.25	0.1194	9.95929947734683\\
64.25	0.12	10.2206848276665\\
64.25	0.1206	10.482070177986\\
64.25	0.1212	10.7434555283056\\
64.25	0.1218	11.0048408786252\\
64.25	0.1224	11.2662262289448\\
64.25	0.123	11.5276115792644\\
64.625	0.093	-1.63351418664385\\
64.625	0.0936	-1.37868510555381\\
64.625	0.0942	-1.12385602446381\\
64.625	0.0948	-0.869026943373797\\
64.625	0.0954	-0.61419786228376\\
64.625	0.096	-0.359368781193723\\
64.625	0.0966	-0.104539700103714\\
64.625	0.0972	0.150289380986294\\
64.625	0.0978	0.405118462076331\\
64.625	0.0984	0.659947543166368\\
64.625	0.099	0.914776624256376\\
64.625	0.0996	1.16960570534638\\
64.625	0.1002	1.42443478643642\\
64.625	0.1008	1.67926386752646\\
64.625	0.1014	1.93409294861647\\
64.625	0.102	2.18892202970648\\
64.625	0.1026	2.44375111079651\\
64.625	0.1032	2.69858019188655\\
64.625	0.1038	2.95340927297656\\
64.625	0.1044	3.20823835406657\\
64.625	0.105	3.4630674351566\\
64.625	0.1056	3.71789651624664\\
64.625	0.1062	3.97272559733665\\
64.625	0.1068	4.22755467842666\\
64.625	0.1074	4.48238375951669\\
64.625	0.108	4.7372128406067\\
64.625	0.1086	4.99204192169674\\
64.625	0.1092	5.24687100278675\\
64.625	0.1098	5.50170008387678\\
64.625	0.1104	5.75652916496679\\
64.625	0.111	6.01135824605683\\
64.625	0.1116	6.26618732714687\\
64.625	0.1122	6.52101640823685\\
64.625	0.1128	6.77584548932691\\
64.625	0.1134	7.03067457041692\\
64.625	0.114	7.28550365150699\\
64.625	0.1146	7.54033273259697\\
64.625	0.1152	7.795161813687\\
64.625	0.1158	8.04999089477701\\
64.625	0.1164	8.30481997586702\\
64.625	0.117	8.55964905695703\\
64.625	0.1176	8.81447813804709\\
64.625	0.1182	9.0693072191371\\
64.625	0.1188	9.32413630022711\\
64.625	0.1194	9.57896538131712\\
64.625	0.12	9.83379446240718\\
64.625	0.1206	10.0886235434972\\
64.625	0.1212	10.3434526245872\\
64.625	0.1218	10.5982817056772\\
64.625	0.1224	10.8531107867673\\
64.625	0.123	11.1079398678573\\
65	0.093	-1.72537243657288\\
65	0.0936	-1.47709962471239\\
65	0.0942	-1.22882681285196\\
65	0.0948	-0.980554000991503\\
65	0.0954	-0.732281189131015\\
65	0.096	-0.484008377270555\\
65	0.0966	-0.235735565410124\\
65	0.0972	0.0125372464503641\\
65	0.0978	0.260810058310824\\
65	0.0984	0.509082870171284\\
65	0.099	0.757355682031744\\
65	0.0996	1.0056284938922\\
65	0.1002	1.25390130575266\\
65	0.1008	1.50217411761312\\
65	0.1014	1.75044692947358\\
65	0.102	1.99871974133404\\
65	0.1026	2.2469925531945\\
65	0.1032	2.49526536505499\\
65	0.1038	2.74353817691542\\
65	0.1044	2.99181098877588\\
65	0.105	3.24008380063637\\
65	0.1056	3.48835661249683\\
65	0.1062	3.73662942435726\\
65	0.1068	3.98490223621775\\
65	0.1074	4.23317504807821\\
65	0.108	4.48144785993864\\
65	0.1086	4.7297206717991\\
65	0.1092	4.97799348365959\\
65	0.1098	5.22626629552005\\
65	0.1104	5.47453910738048\\
65	0.111	5.72281191924097\\
65	0.1116	5.97108473110146\\
65	0.1122	6.21935754296186\\
65	0.1128	6.46763035482238\\
65	0.1134	6.71590316668281\\
65	0.114	6.9641759785433\\
65	0.1146	7.21244879040376\\
65	0.1152	7.46072160226421\\
65	0.1158	7.70899441412465\\
65	0.1164	7.95726722598511\\
65	0.117	8.20554003784557\\
65	0.1176	8.45381284970605\\
65	0.1182	8.70208566156649\\
65	0.1188	8.95035847342695\\
65	0.1194	9.19863128528741\\
65	0.12	9.44690409714789\\
65	0.1206	9.69517690900835\\
65	0.1212	9.94344972086878\\
65	0.1218	10.1917225327292\\
65	0.1224	10.4399953445898\\
65	0.123	10.6882681564502\\
65.375	0.093	-1.81723068650189\\
65.375	0.0936	-1.57551414387098\\
65.375	0.0942	-1.33379760124012\\
65.375	0.0948	-1.09208105860921\\
65.375	0.0954	-0.850364515978299\\
65.375	0.096	-0.608647973347388\\
65.375	0.0966	-0.366931430716505\\
65.375	0.0972	-0.125214888085594\\
65.375	0.0978	0.116501654545317\\
65.375	0.0984	0.3582181971762\\
65.375	0.099	0.599934739807082\\
65.375	0.0996	0.841651282437994\\
65.375	0.1002	1.0833678250689\\
65.375	0.1008	1.32508436769982\\
65.375	0.1014	1.5668009103307\\
65.375	0.102	1.80851745296161\\
65.375	0.1026	2.05023399559252\\
65.375	0.1032	2.2919505382234\\
65.375	0.1038	2.53366708085429\\
65.375	0.1044	2.7753836234852\\
65.375	0.105	3.01710016611611\\
65.375	0.1056	3.25881670874702\\
65.375	0.1062	3.5005332513779\\
65.375	0.1068	3.74224979400881\\
65.375	0.1074	3.9839663366397\\
65.375	0.108	4.22568287927058\\
65.375	0.1086	4.46739942190149\\
65.375	0.1092	4.7091159645324\\
65.375	0.1098	4.95083250716331\\
65.375	0.1104	5.1925490497942\\
65.375	0.111	5.43426559242511\\
65.375	0.1116	5.67598213505605\\
65.375	0.1122	5.91769867768687\\
65.375	0.1128	6.15941522031781\\
65.375	0.1134	6.40113176294869\\
65.375	0.114	6.64284830557963\\
65.375	0.1146	6.88456484821052\\
65.375	0.1152	7.12628139084143\\
65.375	0.1158	7.36799793347231\\
65.375	0.1164	7.60971447610319\\
65.375	0.117	7.85143101873408\\
65.375	0.1176	8.09314756136502\\
65.375	0.1182	8.3348641039959\\
65.375	0.1188	8.57658064662678\\
65.375	0.1194	8.81829718925769\\
65.375	0.12	9.06001373188863\\
65.375	0.1206	9.30173027451951\\
65.375	0.1212	9.5434468171504\\
65.375	0.1218	9.78516335978128\\
65.375	0.1224	10.0268799024122\\
65.375	0.123	10.2685964450431\\
65.75	0.093	-1.90908893643092\\
65.75	0.0936	-1.67392866302959\\
65.75	0.0942	-1.43876838962825\\
65.75	0.0948	-1.20360811622692\\
65.75	0.0954	-0.968447842825555\\
65.75	0.096	-0.73328756942422\\
65.75	0.0966	-0.498127296022886\\
65.75	0.0972	-0.262967022621552\\
65.75	0.0978	-0.0278067492202183\\
65.75	0.0984	0.207353524181144\\
65.75	0.099	0.44251379758245\\
65.75	0.0996	0.677674070983812\\
65.75	0.1002	0.912834344385146\\
65.75	0.1008	1.14799461778651\\
65.75	0.1014	1.38315489118781\\
65.75	0.102	1.61831516458915\\
65.75	0.1026	1.85347543799051\\
65.75	0.1032	2.08863571139185\\
65.75	0.1038	2.32379598479318\\
65.75	0.1044	2.55895625819451\\
65.75	0.105	2.79411653159585\\
65.75	0.1056	3.02927680499721\\
65.75	0.1062	3.26443707839852\\
65.75	0.1068	3.49959735179988\\
65.75	0.1074	3.73475762520121\\
65.75	0.108	3.96991789860255\\
65.75	0.1086	4.20507817200388\\
65.75	0.1092	4.44023844540521\\
65.75	0.1098	4.67539871880658\\
65.75	0.1104	4.91055899220788\\
65.75	0.111	5.14571926560924\\
65.75	0.1116	5.38087953901061\\
65.75	0.1122	5.61603981241191\\
65.75	0.1128	5.85120008581328\\
65.75	0.1134	6.08636035921458\\
65.75	0.114	6.32152063261597\\
65.75	0.1146	6.55668090601728\\
65.75	0.1152	6.79184117941864\\
65.75	0.1158	7.02700145281995\\
65.75	0.1164	7.26216172622128\\
65.75	0.117	7.49732199962261\\
65.75	0.1176	7.73248227302398\\
65.75	0.1182	7.96764254642531\\
65.75	0.1188	8.20280281982662\\
65.75	0.1194	8.43796309322798\\
65.75	0.12	8.67312336662934\\
65.75	0.1206	8.90828364003067\\
65.75	0.1212	9.14344391343198\\
65.75	0.1218	9.37860418683331\\
65.75	0.1224	9.61376446023471\\
65.75	0.123	9.84892473363601\\
66.125	0.093	-2.00094718635995\\
66.125	0.0936	-1.77234318218817\\
66.125	0.0942	-1.54373917801641\\
66.125	0.0948	-1.31513517384462\\
66.125	0.0954	-1.08653116967284\\
66.125	0.096	-0.857927165501053\\
66.125	0.0966	-0.629323161329296\\
66.125	0.0972	-0.400719157157511\\
66.125	0.0978	-0.172115152985725\\
66.125	0.0984	0.0564888511860602\\
66.125	0.099	0.285092855357817\\
66.125	0.0996	0.513696859529603\\
66.125	0.1002	0.742300863701388\\
66.125	0.1008	0.970904867873173\\
66.125	0.1014	1.19950887204493\\
66.125	0.102	1.42811287621672\\
66.125	0.1026	1.6567168803885\\
66.125	0.1032	1.88532088456029\\
66.125	0.1038	2.11392488873204\\
66.125	0.1044	2.34252889290383\\
66.125	0.105	2.57113289707561\\
66.125	0.1056	2.7997369012474\\
66.125	0.1062	3.02834090541916\\
66.125	0.1068	3.25694490959094\\
66.125	0.1074	3.48554891376273\\
66.125	0.108	3.71415291793448\\
66.125	0.1086	3.94275692210627\\
66.125	0.1092	4.17136092627806\\
66.125	0.1098	4.39996493044984\\
66.125	0.1104	4.6285689346216\\
66.125	0.111	4.85717293879338\\
66.125	0.1116	5.0857769429652\\
66.125	0.1122	5.31438094713693\\
66.125	0.1128	5.54298495130874\\
66.125	0.1134	5.7715889554805\\
66.125	0.114	6.00019295965231\\
66.125	0.1146	6.22879696382407\\
66.125	0.1152	6.45740096799585\\
66.125	0.1158	6.68600497216761\\
66.125	0.1164	6.91460897633937\\
66.125	0.117	7.14321298051115\\
66.125	0.1176	7.37181698468297\\
66.125	0.1182	7.60042098885472\\
66.125	0.1188	7.82902499302648\\
66.125	0.1194	8.05762899719824\\
66.125	0.12	8.28623300137005\\
66.125	0.1206	8.51483700554181\\
66.125	0.1212	8.74344100971356\\
66.125	0.1218	8.97204501388535\\
66.125	0.1224	9.20064901805716\\
66.125	0.123	9.42925302222892\\
66.5	0.093	-2.09280543628896\\
66.5	0.0936	-1.87075770134675\\
66.5	0.0942	-1.64870996640454\\
66.5	0.0948	-1.42666223146233\\
66.5	0.0954	-1.20461449652009\\
66.5	0.096	-0.982566761577885\\
66.5	0.0966	-0.760519026635677\\
66.5	0.0972	-0.538471291693469\\
66.5	0.0978	-0.316423556751232\\
66.5	0.0984	-0.0943758218090238\\
66.5	0.099	0.127671913133184\\
66.5	0.0996	0.349719648075393\\
66.5	0.1002	0.57176738301763\\
66.5	0.1008	0.793815117959866\\
66.5	0.1014	1.01586285290205\\
66.5	0.102	1.23791058784428\\
66.5	0.1026	1.45995832278649\\
66.5	0.1032	1.68200605772873\\
66.5	0.1038	1.90405379267091\\
66.5	0.1044	2.12610152761314\\
66.5	0.105	2.34814926255535\\
66.5	0.1056	2.57019699749759\\
66.5	0.1062	2.79224473243977\\
66.5	0.1068	3.01429246738201\\
66.5	0.1074	3.23634020232421\\
66.5	0.108	3.45838793726642\\
66.5	0.1086	3.68043567220866\\
66.5	0.1092	3.90248340715087\\
66.5	0.1098	4.1245311420931\\
66.5	0.1104	4.34657887703528\\
66.5	0.111	4.56862661197752\\
66.5	0.1116	4.79067434691976\\
66.5	0.1122	5.01272208186194\\
66.5	0.1128	5.23476981680417\\
66.5	0.1134	5.45681755174638\\
66.5	0.114	5.67886528668862\\
66.5	0.1146	5.90091302163083\\
66.5	0.1152	6.12296075657306\\
66.5	0.1158	6.34500849151524\\
66.5	0.1164	6.56705622645745\\
66.5	0.117	6.78910396139966\\
66.5	0.1176	7.01115169634193\\
66.5	0.1182	7.23319943128411\\
66.5	0.1188	7.45524716622631\\
66.5	0.1194	7.67729490116852\\
66.5	0.12	7.89934263611079\\
66.5	0.1206	8.12139037105297\\
66.5	0.1212	8.34343810599518\\
66.5	0.1218	8.56548584093738\\
66.5	0.1224	8.78753357587965\\
66.5	0.123	9.00958131082186\\
66.875	0.093	-2.18466368621799\\
66.875	0.0936	-1.96917222050533\\
66.875	0.0942	-1.7536807547927\\
66.875	0.0948	-1.53818928908004\\
66.875	0.0954	-1.32269782336738\\
66.875	0.096	-1.10720635765472\\
66.875	0.0966	-0.891714891942087\\
66.875	0.0972	-0.676223426229399\\
66.875	0.0978	-0.460731960516739\\
66.875	0.0984	-0.245240494804079\\
66.875	0.099	-0.0297490290914482\\
66.875	0.0996	0.185742436621211\\
66.875	0.1002	0.401233902333871\\
66.875	0.1008	0.616725368046531\\
66.875	0.1014	0.832216833759162\\
66.875	0.102	1.04770829947182\\
66.875	0.1026	1.26319976518448\\
66.875	0.1032	1.47869123089717\\
66.875	0.1038	1.6941826966098\\
66.875	0.1044	1.90967416232246\\
66.875	0.105	2.12516562803512\\
66.875	0.1056	2.34065709374778\\
66.875	0.1062	2.55614855946041\\
66.875	0.1068	2.77164002517307\\
66.875	0.1074	2.98713149088573\\
66.875	0.108	3.20262295659836\\
66.875	0.1086	3.41811442231102\\
66.875	0.1092	3.63360588802368\\
66.875	0.1098	3.84909735373637\\
66.875	0.1104	4.064588819449\\
66.875	0.111	4.28008028516166\\
66.875	0.1116	4.49557175087435\\
66.875	0.1122	4.71106321658695\\
66.875	0.1128	4.92655468229964\\
66.875	0.1134	5.14204614801227\\
66.875	0.114	5.35753761372496\\
66.875	0.1146	5.57302907943759\\
66.875	0.1152	5.78852054515025\\
66.875	0.1158	6.00401201086291\\
66.875	0.1164	6.21950347657554\\
66.875	0.117	6.4349949422882\\
66.875	0.1176	6.65048640800089\\
66.875	0.1182	6.86597787371352\\
66.875	0.1188	7.08146933942615\\
66.875	0.1194	7.29696080513881\\
66.875	0.12	7.5124522708515\\
66.875	0.1206	7.72794373656413\\
66.875	0.1212	7.94343520227676\\
66.875	0.1218	8.15892666798942\\
66.875	0.1224	8.37441813370214\\
66.875	0.123	8.58990959941477\\
67.25	0.093	-2.27652193614702\\
67.25	0.0936	-2.06758673966391\\
67.25	0.0942	-1.85865154318083\\
67.25	0.0948	-1.64971634669774\\
67.25	0.0954	-1.44078115021463\\
67.25	0.096	-1.23184595373155\\
67.25	0.0966	-1.02291075724847\\
67.25	0.0972	-0.813975560765357\\
67.25	0.0978	-0.605040364282274\\
67.25	0.0984	-0.396105167799163\\
67.25	0.099	-0.187169971316081\\
67.25	0.0996	0.0217652251670017\\
67.25	0.1002	0.230700421650113\\
67.25	0.1008	0.439635618133224\\
67.25	0.1014	0.648570814616278\\
67.25	0.102	0.857506011099389\\
67.25	0.1026	1.0664412075825\\
67.25	0.1032	1.27537640406558\\
67.25	0.1038	1.48431160054866\\
67.25	0.1044	1.69324679703178\\
67.25	0.105	1.90218199351486\\
67.25	0.1056	2.11111718999797\\
67.25	0.1062	2.32005238648105\\
67.25	0.1068	2.52898758296413\\
67.25	0.1074	2.73792277944725\\
67.25	0.108	2.94685797593033\\
67.25	0.1086	3.15579317241341\\
67.25	0.1092	3.36472836889652\\
67.25	0.1098	3.5736635653796\\
67.25	0.1104	3.78259876186269\\
67.25	0.111	3.9915339583458\\
67.25	0.1116	4.20046915482891\\
67.25	0.1122	4.40940435131196\\
67.25	0.1128	4.6183395477951\\
67.25	0.1134	4.82727474427816\\
67.25	0.114	5.0362099407613\\
67.25	0.1146	5.24514513724438\\
67.25	0.1152	5.45408033372746\\
67.25	0.1158	5.66301553021054\\
67.25	0.1164	5.87195072669363\\
67.25	0.117	6.08088592317671\\
67.25	0.1176	6.28982111965985\\
67.25	0.1182	6.49875631614293\\
67.25	0.1188	6.70769151262598\\
67.25	0.1194	6.9166267091091\\
67.25	0.12	7.12556190559224\\
67.25	0.1206	7.33449710207529\\
67.25	0.1212	7.54343229855837\\
67.25	0.1218	7.75236749504145\\
67.25	0.1224	7.96130269152459\\
67.25	0.123	8.17023788800768\\
67.625	0.093	-2.36838018607605\\
67.625	0.0936	-2.16600125882249\\
67.625	0.0942	-1.96362233156898\\
67.625	0.0948	-1.76124340431545\\
67.625	0.0954	-1.55886447706192\\
67.625	0.096	-1.35648554980835\\
67.625	0.0966	-1.15410662255485\\
67.625	0.0972	-0.951727695301315\\
67.625	0.0978	-0.749348768047781\\
67.625	0.0984	-0.546969840794247\\
67.625	0.099	-0.344590913540713\\
67.625	0.0996	-0.14221198628718\\
67.625	0.1002	0.0601669409663543\\
67.625	0.1008	0.262545868219888\\
67.625	0.1014	0.464924795473394\\
67.625	0.102	0.667303722726956\\
67.625	0.1026	0.86968264998049\\
67.625	0.1032	1.07206157723402\\
67.625	0.1038	1.27444050448753\\
67.625	0.1044	1.47681943174106\\
67.625	0.105	1.67919835899463\\
67.625	0.1056	1.88157728624816\\
67.625	0.1062	2.08395621350166\\
67.625	0.1068	2.2863351407552\\
67.625	0.1074	2.48871406800876\\
67.625	0.108	2.69109299526227\\
67.625	0.1086	2.8934719225158\\
67.625	0.1092	3.09585084976933\\
67.625	0.1098	3.29822977702287\\
67.625	0.1104	3.5006087042764\\
67.625	0.111	3.70298763152994\\
67.625	0.1116	3.9053665587835\\
67.625	0.1122	4.10774548603698\\
67.625	0.1128	4.31012441329054\\
67.625	0.1134	4.51250334054407\\
67.625	0.114	4.71488226779763\\
67.625	0.1146	4.91726119505114\\
67.625	0.1152	5.11964012230467\\
67.625	0.1158	5.32201904955818\\
67.625	0.1164	5.52439797681171\\
67.625	0.117	5.72677690406525\\
67.625	0.1176	5.92915583131881\\
67.625	0.1182	6.13153475857231\\
67.625	0.1188	6.33391368582585\\
67.625	0.1194	6.53629261307938\\
67.625	0.12	6.73867154033294\\
67.625	0.1206	6.94105046758645\\
67.625	0.1212	7.14342939483996\\
67.625	0.1218	7.34580832209352\\
67.625	0.1224	7.54818724934708\\
67.625	0.123	7.75056617660059\\
68	0.093	-2.46023843600506\\
68	0.0936	-2.26441577798107\\
68	0.0942	-2.06859311995714\\
68	0.0948	-1.87277046193316\\
68	0.0954	-1.67694780390917\\
68	0.096	-1.48112514588519\\
68	0.0966	-1.28530248786126\\
68	0.0972	-1.08947982983727\\
68	0.0978	-0.893657171813288\\
68	0.0984	-0.697834513789303\\
68	0.099	-0.502011855765375\\
68	0.0996	-0.306189197741389\\
68	0.1002	-0.110366539717404\\
68	0.1008	0.0854561183065812\\
68	0.1014	0.281278776330538\\
68	0.102	0.477101434354495\\
68	0.1026	0.67292409237848\\
68	0.1032	0.868746750402465\\
68	0.1038	1.06456940842642\\
68	0.1044	1.26039206645038\\
68	0.105	1.45621472447436\\
68	0.1056	1.65203738249835\\
68	0.1062	1.84786004052231\\
68	0.1068	2.04368269854626\\
68	0.1074	2.23950535657025\\
68	0.108	2.43532801459421\\
68	0.1086	2.63115067261819\\
68	0.1092	2.82697333064215\\
68	0.1098	3.02279598866613\\
68	0.1104	3.21861864669009\\
68	0.111	3.41444130471407\\
68	0.1116	3.61026396273809\\
68	0.1122	3.80608662076199\\
68	0.1128	4.001909278786\\
68	0.1134	4.19773193680996\\
68	0.114	4.39355459483397\\
68	0.1146	4.5893772528579\\
68	0.1152	4.78519991088189\\
68	0.1158	4.98102256890584\\
68	0.1164	5.1768452269298\\
68	0.117	5.37266788495376\\
68	0.1176	5.56849054297777\\
68	0.1182	5.76431320100173\\
68	0.1188	5.96013585902568\\
68	0.1194	6.15595851704964\\
68	0.12	6.35178117507365\\
68	0.1206	6.54760383309761\\
68	0.1212	6.74342649112157\\
68	0.1218	6.93924914914555\\
68	0.1224	7.13507180716954\\
68	0.123	7.3308944651935\\
68.375	0.093	-2.55209668593406\\
68.375	0.0936	-2.36283029713965\\
68.375	0.0942	-2.17356390834524\\
68.375	0.0948	-1.98429751955084\\
68.375	0.0954	-1.79503113075643\\
68.375	0.096	-1.60576474196202\\
68.375	0.0966	-1.41649835316761\\
68.375	0.0972	-1.2272319643732\\
68.375	0.0978	-1.0379655755788\\
68.375	0.0984	-0.848699186784359\\
68.375	0.099	-0.659432797989979\\
68.375	0.0996	-0.470166409195571\\
68.375	0.1002	-0.280900020401134\\
68.375	0.1008	-0.0916336316067259\\
68.375	0.1014	0.0976327571876539\\
68.375	0.102	0.28689914598209\\
68.375	0.1026	0.476165534776499\\
68.375	0.1032	0.665431923570907\\
68.375	0.1038	0.854698312365315\\
68.375	0.1044	1.04396470115972\\
68.375	0.105	1.23323108995413\\
68.375	0.1056	1.42249747874857\\
68.375	0.1062	1.61176386754295\\
68.375	0.1068	1.80103025633736\\
68.375	0.1074	1.99029664513179\\
68.375	0.108	2.17956303392617\\
68.375	0.1086	2.36882942272058\\
68.375	0.1092	2.55809581151502\\
68.375	0.1098	2.74736220030942\\
68.375	0.1104	2.9366285891038\\
68.375	0.111	3.12589497789824\\
68.375	0.1116	3.31516136669268\\
68.375	0.1122	3.50442775548703\\
68.375	0.1128	3.69369414428149\\
68.375	0.1134	3.88296053307587\\
68.375	0.114	4.07222692187031\\
68.375	0.1146	4.26149331066472\\
68.375	0.1152	4.45075969945913\\
68.375	0.1158	4.64002608825351\\
68.375	0.1164	4.82929247704791\\
68.375	0.117	5.01855886584232\\
68.375	0.1176	5.20782525463676\\
68.375	0.1182	5.39709164343117\\
68.375	0.1188	5.58635803222555\\
68.375	0.1194	5.77562442101996\\
68.375	0.12	5.96489080981442\\
68.375	0.1206	6.1541571986088\\
68.375	0.1212	6.34342358740318\\
68.375	0.1218	6.53268997619762\\
68.375	0.1224	6.72195636499205\\
68.375	0.123	6.91122275378643\\
68.75	0.093	-2.64395493586309\\
68.75	0.0936	-2.46124481629823\\
68.75	0.0942	-2.2785346967334\\
68.75	0.0948	-2.09582457716854\\
68.75	0.0954	-1.91311445760368\\
68.75	0.096	-1.73040433803882\\
68.75	0.0966	-1.54769421847402\\
68.75	0.0972	-1.36498409890916\\
68.75	0.0978	-1.1822739793443\\
68.75	0.0984	-0.999563859779443\\
68.75	0.099	-0.816853740214611\\
68.75	0.0996	-0.634143620649752\\
68.75	0.1002	-0.451433501084892\\
68.75	0.1008	-0.268723381520033\\
68.75	0.1014	-0.0860132619552019\\
68.75	0.102	0.0966968576096292\\
68.75	0.1026	0.279406977174489\\
68.75	0.1032	0.462117096739348\\
68.75	0.1038	0.644827216304179\\
68.75	0.1044	0.827537335869039\\
68.75	0.105	1.0102474554339\\
68.75	0.1056	1.19295757499876\\
68.75	0.1062	1.37566769456359\\
68.75	0.1068	1.55837781412845\\
68.75	0.1074	1.74108793369328\\
68.75	0.108	1.92379805325811\\
68.75	0.1086	2.10650817282297\\
68.75	0.1092	2.28921829238783\\
68.75	0.1098	2.47192841195269\\
68.75	0.1104	2.65463853151752\\
68.75	0.111	2.83734865108238\\
68.75	0.1116	3.02005877064727\\
68.75	0.1122	3.20276889021204\\
68.75	0.1128	3.38547900977693\\
68.75	0.1134	3.56818912934176\\
68.75	0.114	3.75089924890665\\
68.75	0.1146	3.93360936847148\\
68.75	0.1152	4.11631948803634\\
68.75	0.1158	4.29902960760117\\
68.75	0.1164	4.481739727166\\
68.75	0.117	4.66444984673086\\
68.75	0.1176	4.84715996629572\\
68.75	0.1182	5.02987008586055\\
68.75	0.1188	5.21258020542538\\
68.75	0.1194	5.39529032499024\\
68.75	0.12	5.57800044455513\\
68.75	0.1206	5.76071056411996\\
68.75	0.1212	5.94342068368479\\
68.75	0.1218	6.12613080324965\\
68.75	0.1224	6.30884092281451\\
68.75	0.123	6.49155104237934\\
69.125	0.093	-2.73581318579213\\
69.125	0.0936	-2.55965933545681\\
69.125	0.0942	-2.38350548512156\\
69.125	0.0948	-2.20735163478625\\
69.125	0.0954	-2.03119778445097\\
69.125	0.096	-1.85504393411566\\
69.125	0.0966	-1.6788900837804\\
69.125	0.0972	-1.50273623344509\\
69.125	0.0978	-1.32658238310981\\
69.125	0.0984	-1.15042853277453\\
69.125	0.099	-0.974274682439244\\
69.125	0.0996	-0.798120832103962\\
69.125	0.1002	-0.621966981768651\\
69.125	0.1008	-0.445813131433368\\
69.125	0.1014	-0.269659281098086\\
69.125	0.102	-0.0935054307628036\\
69.125	0.1026	0.0826484195725072\\
69.125	0.1032	0.25880226990779\\
69.125	0.1038	0.434956120243044\\
69.125	0.1044	0.611109970578354\\
69.125	0.105	0.787263820913637\\
69.125	0.1056	0.963417671248948\\
69.125	0.1062	1.1395715215842\\
69.125	0.1068	1.31572537191951\\
69.125	0.1074	1.49187922225479\\
69.125	0.108	1.66803307259005\\
69.125	0.1086	1.84418692292536\\
69.125	0.1092	2.02034077326064\\
69.125	0.1098	2.19649462359595\\
69.125	0.1104	2.37264847393121\\
69.125	0.111	2.54880232426652\\
69.125	0.1116	2.72495617460183\\
69.125	0.1122	2.90111002493708\\
69.125	0.1128	3.07726387527239\\
69.125	0.1134	3.25341772560765\\
69.125	0.114	3.42957157594299\\
69.125	0.1146	3.60572542627824\\
69.125	0.1152	3.78187927661355\\
69.125	0.1158	3.95803312694881\\
69.125	0.1164	4.13418697728409\\
69.125	0.117	4.31034082761937\\
69.125	0.1176	4.48649467795471\\
69.125	0.1182	4.66264852828996\\
69.125	0.1188	4.83880237862522\\
69.125	0.1194	5.01495622896053\\
69.125	0.12	5.19111007929584\\
69.125	0.1206	5.36726392963112\\
69.125	0.1212	5.54341777996638\\
69.125	0.1218	5.71957163030169\\
69.125	0.1224	5.895725480637\\
69.125	0.123	6.07187933097228\\
69.5	0.093	-2.82767143572116\\
69.5	0.0936	-2.65807385461542\\
69.5	0.0942	-2.48847627350972\\
69.5	0.0948	-2.31887869240398\\
69.5	0.0954	-2.14928111129825\\
69.5	0.096	-1.97968353019252\\
69.5	0.0966	-1.81008594908681\\
69.5	0.0972	-1.64048836798108\\
69.5	0.0978	-1.47089078687534\\
69.5	0.0984	-1.30129320576961\\
69.5	0.099	-1.13169562466391\\
69.5	0.0996	-0.962098043558171\\
69.5	0.1002	-0.792500462452438\\
69.5	0.1008	-0.622902881346704\\
69.5	0.1014	-0.453305300240999\\
69.5	0.102	-0.283707719135265\\
69.5	0.1026	-0.114110138029531\\
69.5	0.1032	0.0554874430762027\\
69.5	0.1038	0.225085024181908\\
69.5	0.1044	0.394682605287642\\
69.5	0.105	0.564280186393376\\
69.5	0.1056	0.733877767499109\\
69.5	0.1062	0.903475348604815\\
69.5	0.1068	1.07307292971055\\
69.5	0.1074	1.24267051081628\\
69.5	0.108	1.41226809192199\\
69.5	0.1086	1.58186567302772\\
69.5	0.1092	1.75146325413345\\
69.5	0.1098	1.92106083523919\\
69.5	0.1104	2.09065841634489\\
69.5	0.111	2.26025599745063\\
69.5	0.1116	2.42985357855639\\
69.5	0.1122	2.59945115966207\\
69.5	0.1128	2.76904874076783\\
69.5	0.1134	2.93864632187353\\
69.5	0.114	3.1082439029793\\
69.5	0.1146	3.277841484085\\
69.5	0.1152	3.44743906519074\\
69.5	0.1158	3.61703664629644\\
69.5	0.1164	3.78663422740215\\
69.5	0.117	3.95623180850788\\
69.5	0.1176	4.12582938961364\\
69.5	0.1182	4.29542697071935\\
69.5	0.1188	4.46502455182505\\
69.5	0.1194	4.63462213293079\\
69.5	0.12	4.80421971403655\\
69.5	0.1206	4.97381729514225\\
69.5	0.1212	5.14341487624796\\
69.5	0.1218	5.31301245735369\\
69.5	0.1224	5.48261003845946\\
69.5	0.123	5.65220761956516\\
69.875	0.093	-2.91952968565019\\
69.875	0.0936	-2.756488373774\\
69.875	0.0942	-2.59344706189788\\
69.875	0.0948	-2.43040575002169\\
69.875	0.0954	-2.26736443814553\\
69.875	0.096	-2.10432312626935\\
69.875	0.0966	-1.94128181439322\\
69.875	0.0972	-1.77824050251704\\
69.875	0.0978	-1.61519919064088\\
69.875	0.0984	-1.45215787876469\\
69.875	0.099	-1.28911656688854\\
69.875	0.0996	-1.12607525501238\\
69.875	0.1002	-0.963033943136196\\
69.875	0.1008	-0.799992631260039\\
69.875	0.1014	-0.636951319383883\\
69.875	0.102	-0.473910007507726\\
69.875	0.1026	-0.310868695631541\\
69.875	0.1032	-0.147827383755384\\
69.875	0.1038	0.0152139281207724\\
69.875	0.1044	0.178255239996957\\
69.875	0.105	0.341296551873114\\
69.875	0.1056	0.504337863749299\\
69.875	0.1062	0.667379175625427\\
69.875	0.1068	0.830420487501613\\
69.875	0.1074	0.993461799377769\\
69.875	0.108	1.15650311125393\\
69.875	0.1086	1.31954442313008\\
69.875	0.1092	1.48258573500627\\
69.875	0.1098	1.64562704688245\\
69.875	0.1104	1.80866835875858\\
69.875	0.111	1.97170967063477\\
69.875	0.1116	2.13475098251095\\
69.875	0.1122	2.29779229438708\\
69.875	0.1128	2.46083360626326\\
69.875	0.1134	2.62387491813942\\
69.875	0.114	2.78691623001561\\
69.875	0.1146	2.94995754189176\\
69.875	0.1152	3.11299885376795\\
69.875	0.1158	3.27604016564408\\
69.875	0.1164	3.43908147752023\\
69.875	0.117	3.60212278939639\\
69.875	0.1176	3.7651641012726\\
69.875	0.1182	3.92820541314873\\
69.875	0.1188	4.09124672502489\\
69.875	0.1194	4.25428803690104\\
69.875	0.12	4.41732934877726\\
69.875	0.1206	4.58037066065341\\
69.875	0.1212	4.74341197252954\\
69.875	0.1218	4.90645328440573\\
69.875	0.1224	5.06949459628191\\
69.875	0.123	5.23253590815807\\
70.25	0.093	-3.01138793557922\\
70.25	0.0936	-2.85490289293261\\
70.25	0.0942	-2.69841785028601\\
70.25	0.0948	-2.5419328076394\\
70.25	0.0954	-2.38544776499279\\
70.25	0.096	-2.22896272234618\\
70.25	0.0966	-2.0724776796996\\
70.25	0.0972	-1.91599263705299\\
70.25	0.0978	-1.75950759440639\\
70.25	0.0984	-1.60302255175978\\
70.25	0.099	-1.4465375091132\\
70.25	0.0996	-1.29005246646656\\
70.25	0.1002	-1.13356742381995\\
70.25	0.1008	-0.977082381173346\\
70.25	0.1014	-0.820597338526767\\
70.25	0.102	-0.664112295880159\\
70.25	0.1026	-0.507627253233551\\
70.25	0.1032	-0.351142210586943\\
70.25	0.1038	-0.194657167940363\\
70.25	0.1044	-0.0381721252937552\\
70.25	0.105	0.118312917352881\\
70.25	0.1056	0.274797959999489\\
70.25	0.1062	0.431283002646069\\
70.25	0.1068	0.587768045292677\\
70.25	0.1074	0.744253087939285\\
70.25	0.108	0.900738130585864\\
70.25	0.1086	1.05722317323247\\
70.25	0.1092	1.21370821587908\\
70.25	0.1098	1.37019325852569\\
70.25	0.1104	1.5266783011723\\
70.25	0.111	1.6831633438189\\
70.25	0.1116	1.83964838646554\\
70.25	0.1122	1.99613342911209\\
70.25	0.1128	2.15261847175873\\
70.25	0.1134	2.30910351440531\\
70.25	0.114	2.46558855705194\\
70.25	0.1146	2.62207359969852\\
70.25	0.1152	2.77855864234516\\
70.25	0.1158	2.93504368499174\\
70.25	0.1164	3.09152872763832\\
70.25	0.117	3.24801377028493\\
70.25	0.1176	3.40449881293156\\
70.25	0.1182	3.56098385557814\\
70.25	0.1188	3.71746889822472\\
70.25	0.1194	3.87395394087133\\
70.25	0.12	4.03043898351797\\
70.25	0.1206	4.18692402616458\\
70.25	0.1212	4.34340906881116\\
70.25	0.1218	4.49989411145776\\
70.25	0.1224	4.6563791541044\\
70.25	0.123	4.81286419675098\\
70.625	0.093	-3.10324618550825\\
70.625	0.0936	-2.9533174120912\\
70.625	0.0942	-2.80338863867419\\
70.625	0.0948	-2.65345986525713\\
70.625	0.0954	-2.50353109184007\\
70.625	0.096	-2.35360231842304\\
70.625	0.0966	-2.20367354500601\\
70.625	0.0972	-2.05374477158895\\
70.625	0.0978	-1.90381599817192\\
70.625	0.0984	-1.75388722475486\\
70.625	0.099	-1.60395845133783\\
70.625	0.0996	-1.4540296779208\\
70.625	0.1002	-1.30410090450374\\
70.625	0.1008	-1.15417213108668\\
70.625	0.1014	-1.00424335766968\\
70.625	0.102	-0.85431458425262\\
70.625	0.1026	-0.704385810835561\\
70.625	0.1032	-0.55445703741853\\
70.625	0.1038	-0.404528264001499\\
70.625	0.1044	-0.25459949058444\\
70.625	0.105	-0.104670717167409\\
70.625	0.1056	0.0452580562496507\\
70.625	0.1062	0.195186829666682\\
70.625	0.1068	0.345115603083713\\
70.625	0.1074	0.495044376500772\\
70.625	0.108	0.644973149917803\\
70.625	0.1086	0.794901923334834\\
70.625	0.1092	0.944830696751893\\
70.625	0.1098	1.09475947016895\\
70.625	0.1104	1.24468824358595\\
70.625	0.111	1.39461701700301\\
70.625	0.1116	1.5445457904201\\
70.625	0.1122	1.69447456383708\\
70.625	0.1128	1.84440333725416\\
70.625	0.1134	1.99433211067117\\
70.625	0.114	2.14426088408825\\
70.625	0.1146	2.29418965750529\\
70.625	0.1152	2.44411843092232\\
70.625	0.1158	2.59404720433935\\
70.625	0.1164	2.74397597775638\\
70.625	0.117	2.89390475117341\\
70.625	0.1176	3.0438335245905\\
70.625	0.1182	3.19376229800753\\
70.625	0.1188	3.34369107142453\\
70.625	0.1194	3.49361984484159\\
70.625	0.12	3.64354861825868\\
70.625	0.1206	3.79347739167568\\
70.625	0.1212	3.94340616509271\\
70.625	0.1218	4.09333493850977\\
70.625	0.1224	4.24326371192683\\
70.625	0.123	4.39319248534386\\
71	0.093	-3.19510443543729\\
71	0.0936	-3.0517319312498\\
71	0.0942	-2.90835942706232\\
71	0.0948	-2.76498692287484\\
71	0.0954	-2.62161441868736\\
71	0.096	-2.47824191449988\\
71	0.0966	-2.33486941031239\\
71	0.0972	-2.19149690612491\\
71	0.0978	-2.04812440193743\\
71	0.0984	-1.90475189774995\\
71	0.099	-1.76137939356249\\
71	0.0996	-1.61800688937498\\
71	0.1002	-1.4746343851875\\
71	0.1008	-1.33126188100002\\
71	0.1014	-1.18788937681256\\
71	0.102	-1.04451687262505\\
71	0.1026	-0.901144368437571\\
71	0.1032	-0.757771864250088\\
71	0.1038	-0.614399360062635\\
71	0.1044	-0.471026855875152\\
71	0.105	-0.327654351687642\\
71	0.1056	-0.184281847500159\\
71	0.1062	-0.0409093433127055\\
71	0.1068	0.102463160874777\\
71	0.1074	0.245835665062287\\
71	0.108	0.389208169249741\\
71	0.1086	0.532580673437224\\
71	0.1092	0.675953177624706\\
71	0.1098	0.819325681812188\\
71	0.1104	0.96269818599967\\
71	0.111	1.10607069018715\\
71	0.1116	1.24944319437466\\
71	0.1122	1.39281569856209\\
71	0.1128	1.53618820274963\\
71	0.1134	1.67956070693708\\
71	0.114	1.82293321112459\\
71	0.1146	1.96630571531205\\
71	0.1152	2.10967821949953\\
71	0.1158	2.25305072368701\\
71	0.1164	2.39642322787446\\
71	0.117	2.53979573206195\\
71	0.1176	2.68316823624946\\
71	0.1182	2.82654074043691\\
71	0.1188	2.96991324462439\\
71	0.1194	3.11328574881188\\
71	0.12	3.25665825299939\\
71	0.1206	3.40003075718684\\
71	0.1212	3.54340326137432\\
71	0.1218	3.6867757655618\\
71	0.1224	3.83014826974932\\
71	0.123	3.97352077393677\\
71.375	0.093	-3.28696268536629\\
71.375	0.0936	-3.15014645040839\\
71.375	0.0942	-3.01333021545048\\
71.375	0.0948	-2.87651398049255\\
71.375	0.0954	-2.73969774553461\\
71.375	0.096	-2.60288151057671\\
71.375	0.0966	-2.4660652756188\\
71.375	0.0972	-2.32924904066087\\
71.375	0.0978	-2.19243280570294\\
71.375	0.0984	-2.055616570745\\
71.375	0.099	-1.91880033578713\\
71.375	0.0996	-1.78198410082919\\
71.375	0.1002	-1.64516786587126\\
71.375	0.1008	-1.50835163091332\\
71.375	0.1014	-1.37153539595545\\
71.375	0.102	-1.23471916099751\\
71.375	0.1026	-1.09790292603958\\
71.375	0.1032	-0.961086691081647\\
71.375	0.1038	-0.824270456123742\\
71.375	0.1044	-0.687454221165837\\
71.375	0.105	-0.550637986207903\\
71.375	0.1056	-0.413821751249969\\
71.375	0.1062	-0.277005516292064\\
71.375	0.1068	-0.140189281334159\\
71.375	0.1074	-0.00337304637622537\\
71.375	0.108	0.13344318858168\\
71.375	0.1086	0.270259423539613\\
71.375	0.1092	0.407075658497547\\
71.375	0.1098	0.543891893455452\\
71.375	0.1104	0.680708128413357\\
71.375	0.111	0.817524363371291\\
71.375	0.1116	0.954340598329253\\
71.375	0.1122	1.0911568332871\\
71.375	0.1128	1.22797306824506\\
71.375	0.1134	1.36478930320297\\
71.375	0.114	1.50160553816093\\
71.375	0.1146	1.63842177311884\\
71.375	0.1152	1.77523800807674\\
71.375	0.1158	1.91205424303465\\
71.375	0.1164	2.04887047799255\\
71.375	0.117	2.18568671295048\\
71.375	0.1176	2.32250294790842\\
71.375	0.1182	2.45931918286632\\
71.375	0.1188	2.59613541782423\\
71.375	0.1194	2.73295165278216\\
71.375	0.12	2.86976788774012\\
71.375	0.1206	3.006584122698\\
71.375	0.1212	3.14340035765591\\
71.375	0.1218	3.28021659261384\\
71.375	0.1224	3.4170328275718\\
71.375	0.123	3.55384906252968\\
71.75	0.093	-3.37882093529532\\
71.75	0.0936	-3.24856096956697\\
71.75	0.0942	-3.11830100383864\\
71.75	0.0948	-2.98804103811025\\
71.75	0.0954	-2.8577810723819\\
71.75	0.096	-2.72752110665351\\
71.75	0.0966	-2.59726114092518\\
71.75	0.0972	-2.46700117519683\\
71.75	0.0978	-2.33674120946844\\
71.75	0.0984	-2.20648124374009\\
71.75	0.099	-2.07622127801176\\
71.75	0.0996	-1.94596131228337\\
71.75	0.1002	-1.81570134655502\\
71.75	0.1008	-1.68544138082666\\
71.75	0.1014	-1.5551814150983\\
71.75	0.102	-1.42492144936995\\
71.75	0.1026	-1.29466148364159\\
71.75	0.1032	-1.16440151791321\\
71.75	0.1038	-1.03414155218488\\
71.75	0.1044	-0.903881586456521\\
71.75	0.105	-0.773621620728136\\
71.75	0.1056	-0.643361654999779\\
71.75	0.1062	-0.513101689271451\\
71.75	0.1068	-0.382841723543066\\
71.75	0.1074	-0.25258175781471\\
71.75	0.108	-0.122321792086382\\
71.75	0.1086	0.00793817364200322\\
71.75	0.1092	0.13819813937036\\
71.75	0.1098	0.268458105098716\\
71.75	0.1104	0.398718070827073\\
71.75	0.111	0.528978036555429\\
71.75	0.1116	0.659238002283814\\
71.75	0.1122	0.789497968012142\\
71.75	0.1128	0.919757933740527\\
71.75	0.1134	1.05001789946886\\
71.75	0.114	1.18027786519727\\
71.75	0.1146	1.3105378309256\\
71.75	0.1152	1.44079779665395\\
71.75	0.1158	1.57105776238231\\
71.75	0.1164	1.70131772811064\\
71.75	0.117	1.83157769383899\\
71.75	0.1176	1.96183765956741\\
71.75	0.1182	2.09209762529574\\
71.75	0.1188	2.22235759102406\\
71.75	0.1194	2.35261755675245\\
71.75	0.12	2.48287752248083\\
71.75	0.1206	2.61313748820916\\
71.75	0.1212	2.74339745393752\\
71.75	0.1218	2.87365741966588\\
71.75	0.1224	3.00391738539426\\
71.75	0.123	3.13417735112262\\
72.125	0.093	-3.4706791852243\\
72.125	0.0936	-3.34697548872549\\
72.125	0.0942	-3.22327179222674\\
72.125	0.0948	-3.09956809572793\\
72.125	0.0954	-2.97586439922912\\
72.125	0.096	-2.85216070273032\\
72.125	0.0966	-2.72845700623154\\
72.125	0.0972	-2.60475330973273\\
72.125	0.0978	-2.48104961323392\\
72.125	0.0984	-2.35734591673511\\
72.125	0.099	-2.23364222023633\\
72.125	0.0996	-2.10993852373753\\
72.125	0.1002	-1.98623482723875\\
72.125	0.1008	-1.86253113073994\\
72.125	0.1014	-1.73882743424116\\
72.125	0.102	-1.61512373774235\\
72.125	0.1026	-1.49142004124354\\
72.125	0.1032	-1.36771634474474\\
72.125	0.1038	-1.24401264824596\\
72.125	0.1044	-1.12030895174715\\
72.125	0.105	-0.99660525524834\\
72.125	0.1056	-0.872901558749533\\
72.125	0.1062	-0.749197862250782\\
72.125	0.1068	-0.625494165751974\\
72.125	0.1074	-0.501790469253166\\
72.125	0.108	-0.378086772754386\\
72.125	0.1086	-0.254383076255579\\
72.125	0.1092	-0.130679379756771\\
72.125	0.1098	-0.00697568325796283\\
72.125	0.1104	0.116728013240817\\
72.125	0.111	0.240431709739624\\
72.125	0.1116	0.364135406238461\\
72.125	0.1122	0.487839102737183\\
72.125	0.1128	0.61154279923602\\
72.125	0.1134	0.735246495734799\\
72.125	0.114	0.858950192233635\\
72.125	0.1146	0.982653888732415\\
72.125	0.1152	1.10635758523122\\
72.125	0.1158	1.23006128173\\
72.125	0.1164	1.35376497822878\\
72.125	0.117	1.47746867472759\\
72.125	0.1176	1.60117237122643\\
72.125	0.1182	1.72487606772518\\
72.125	0.1188	1.84857976422396\\
72.125	0.1194	1.97228346072276\\
72.125	0.12	2.0959871572216\\
72.125	0.1206	2.21969085372038\\
72.125	0.1212	2.34339455021916\\
72.125	0.1218	2.46709824671797\\
72.125	0.1224	2.59080194321677\\
72.125	0.123	2.71450563971558\\
72.5	0.093	-3.56253743515333\\
72.5	0.0936	-3.4453900078841\\
72.5	0.0942	-3.32824258061487\\
72.5	0.0948	-3.21109515334564\\
72.5	0.0954	-3.09394772607638\\
72.5	0.096	-2.97680029880715\\
72.5	0.0966	-2.85965287153792\\
72.5	0.0972	-2.74250544426869\\
72.5	0.0978	-2.62535801699943\\
72.5	0.0984	-2.5082105897302\\
72.5	0.099	-2.39106316246099\\
72.5	0.0996	-2.27391573519174\\
72.5	0.1002	-2.1567683079225\\
72.5	0.1008	-2.03962088065325\\
72.5	0.1014	-1.92247345338404\\
72.5	0.102	-1.80532602611478\\
72.5	0.1026	-1.68817859884555\\
72.5	0.1032	-1.57103117157629\\
72.5	0.1038	-1.45388374430709\\
72.5	0.1044	-1.33673631703783\\
72.5	0.105	-1.2195888897686\\
72.5	0.1056	-1.10244146249934\\
72.5	0.1062	-0.98529403523014\\
72.5	0.1068	-0.868146607960909\\
72.5	0.1074	-0.75099918069165\\
72.5	0.108	-0.633851753422448\\
72.5	0.1086	-0.516704326153189\\
72.5	0.1092	-0.399556898883958\\
72.5	0.1098	-0.282409471614699\\
72.5	0.1104	-0.165262044345496\\
72.5	0.111	-0.0481146170762372\\
72.5	0.1116	0.069032810193022\\
72.5	0.1122	0.186180237462224\\
72.5	0.1128	0.303327664731484\\
72.5	0.1134	0.420475092000686\\
72.5	0.114	0.537622519269974\\
72.5	0.1146	0.654769946539176\\
72.5	0.1152	0.771917373808435\\
72.5	0.1158	0.889064801077637\\
72.5	0.1164	1.00621222834687\\
72.5	0.117	1.1233596556161\\
72.5	0.1176	1.24050708288539\\
72.5	0.1182	1.35765451015459\\
72.5	0.1188	1.47480193742382\\
72.5	0.1194	1.59194936469305\\
72.5	0.12	1.70909679196234\\
72.5	0.1206	1.82624421923154\\
72.5	0.1212	1.94339164650074\\
72.5	0.1218	2.06053907376997\\
72.5	0.1224	2.17768650103926\\
72.5	0.123	2.29483392830849\\
72.875	0.093	-3.65439568508236\\
72.875	0.0936	-3.54380452704268\\
72.875	0.0942	-3.43321336900303\\
72.875	0.0948	-3.32262221096335\\
72.875	0.0954	-3.21203105292366\\
72.875	0.096	-3.10143989488398\\
72.875	0.0966	-2.99084873684433\\
72.875	0.0972	-2.88025757880465\\
72.875	0.0978	-2.76966642076496\\
72.875	0.0984	-2.65907526272528\\
72.875	0.099	-2.54848410468563\\
72.875	0.0996	-2.43789294664595\\
72.875	0.1002	-2.32730178860626\\
72.875	0.1008	-2.21671063056658\\
72.875	0.1014	-2.10611947252693\\
72.875	0.102	-1.99552831448725\\
72.875	0.1026	-1.88493715644756\\
72.875	0.1032	-1.77434599840788\\
72.875	0.1038	-1.66375484036823\\
72.875	0.1044	-1.55316368232852\\
72.875	0.105	-1.44257252428883\\
72.875	0.1056	-1.33198136624915\\
72.875	0.1062	-1.2213902082095\\
72.875	0.1068	-1.11079905016982\\
72.875	0.1074	-1.00020789213013\\
72.875	0.108	-0.889616734090481\\
72.875	0.1086	-0.779025576050799\\
72.875	0.1092	-0.668434418011117\\
72.875	0.1098	-0.557843259971435\\
72.875	0.1104	-0.447252101931781\\
72.875	0.111	-0.336660943892099\\
72.875	0.1116	-0.226069785852388\\
72.875	0.1122	-0.115478627812763\\
72.875	0.1128	-0.00488746977305254\\
72.875	0.1134	0.105703688266601\\
72.875	0.114	0.216294846306312\\
72.875	0.1146	0.326886004345965\\
72.875	0.1152	0.437477162385647\\
72.875	0.1158	0.548068320425301\\
72.875	0.1164	0.658659478464955\\
72.875	0.117	0.769250636504637\\
72.875	0.1176	0.879841794544348\\
72.875	0.1182	0.990432952584001\\
72.875	0.1188	1.10102411062365\\
72.875	0.1194	1.21161526866334\\
72.875	0.12	1.32220642670305\\
72.875	0.1206	1.43279758474267\\
72.875	0.1212	1.54338874278233\\
72.875	0.1218	1.65397990082204\\
72.875	0.1224	1.76457105886175\\
72.875	0.123	1.8751622169014\\
73.25	0.093	-3.74625393501137\\
73.25	0.0936	-3.64221904620126\\
73.25	0.0942	-3.53818415739116\\
73.25	0.0948	-3.43414926858105\\
73.25	0.0954	-3.33011437977092\\
73.25	0.096	-3.22607949096081\\
73.25	0.0966	-3.12204460215071\\
73.25	0.0972	-3.01800971334058\\
73.25	0.0978	-2.91397482453047\\
73.25	0.0984	-2.80993993572034\\
73.25	0.099	-2.70590504691026\\
73.25	0.0996	-2.60187015810013\\
73.25	0.1002	-2.49783526929002\\
73.25	0.1008	-2.39380038047989\\
73.25	0.1014	-2.28976549166978\\
73.25	0.102	-2.18573060285968\\
73.25	0.1026	-2.08169571404954\\
73.25	0.1032	-1.97766082523944\\
73.25	0.1038	-1.87362593642933\\
73.25	0.1044	-1.76959104761923\\
73.25	0.105	-1.6655561588091\\
73.25	0.1056	-1.56152126999896\\
73.25	0.1062	-1.45748638118889\\
73.25	0.1068	-1.35345149237875\\
73.25	0.1074	-1.24941660356865\\
73.25	0.108	-1.14538171475854\\
73.25	0.1086	-1.04134682594844\\
73.25	0.1092	-0.937311937138304\\
73.25	0.1098	-0.833277048328171\\
73.25	0.1104	-0.729242159518094\\
73.25	0.111	-0.625207270707961\\
73.25	0.1116	-0.521172381897827\\
73.25	0.1122	-0.41713749308775\\
73.25	0.1128	-0.313102604277617\\
73.25	0.1134	-0.209067715467512\\
73.25	0.114	-0.10503282665735\\
73.25	0.1146	-0.000997937847273533\\
73.25	0.1152	0.10303695096286\\
73.25	0.1158	0.207071839772937\\
73.25	0.1164	0.311106728583042\\
73.25	0.117	0.415141617393147\\
73.25	0.1176	0.519176506203308\\
73.25	0.1182	0.623211395013413\\
73.25	0.1188	0.72724628382349\\
73.25	0.1194	0.831281172633624\\
73.25	0.12	0.935316061443757\\
73.25	0.1206	1.03935095025383\\
73.25	0.1212	1.14338583906397\\
73.25	0.1218	1.24742072787404\\
73.25	0.1224	1.35145561668423\\
73.25	0.123	1.45549050549431\\
73.625	0.093	-3.8381121849404\\
73.625	0.0936	-3.74063356535984\\
73.625	0.0942	-3.64315494577932\\
73.625	0.0948	-3.54567632619876\\
73.625	0.0954	-3.4481977066182\\
73.625	0.096	-3.35071908703765\\
73.625	0.0966	-3.25324046745709\\
73.625	0.0972	-3.15576184787653\\
73.625	0.0978	-3.05828322829598\\
73.625	0.0984	-2.96080460871542\\
73.625	0.099	-2.86332598913489\\
73.625	0.0996	-2.76584736955434\\
73.625	0.1002	-2.66836874997378\\
73.625	0.1008	-2.5708901303932\\
73.625	0.1014	-2.47341151081267\\
73.625	0.102	-2.37593289123211\\
73.625	0.1026	-2.27845427165155\\
73.625	0.1032	-2.180975652071\\
73.625	0.1038	-2.08349703249047\\
73.625	0.1044	-1.98601841290991\\
73.625	0.105	-1.88853979332936\\
73.625	0.1056	-1.79106117374877\\
73.625	0.1062	-1.69358255416824\\
73.625	0.1068	-1.59610393458769\\
73.625	0.1074	-1.49862531500713\\
73.625	0.108	-1.4011466954266\\
73.625	0.1086	-1.30366807584605\\
73.625	0.1092	-1.20618945626549\\
73.625	0.1098	-1.10871083668493\\
73.625	0.1104	-1.01123221710438\\
73.625	0.111	-0.913753597523822\\
73.625	0.1116	-0.816274977943237\\
73.625	0.1122	-0.718796358362738\\
73.625	0.1128	-0.621317738782153\\
73.625	0.1134	-0.523839119201625\\
73.625	0.114	-0.42636049962104\\
73.625	0.1146	-0.328881880040512\\
73.625	0.1152	-0.231403260459928\\
73.625	0.1158	-0.1339246408794\\
73.625	0.1164	-0.0364460212988718\\
73.625	0.117	0.0610325982816846\\
73.625	0.1176	0.158511217862269\\
73.625	0.1182	0.255989837442797\\
73.625	0.1188	0.353468457023325\\
73.625	0.1194	0.450947076603882\\
73.625	0.12	0.548425696184495\\
73.625	0.1206	0.645904315764994\\
73.625	0.1212	0.743382935345551\\
73.625	0.1218	0.840861554926107\\
73.625	0.1224	0.93834017450672\\
73.625	0.123	1.03581879408722\\
74	0.093	-3.92997043486943\\
74	0.0936	-3.83904808451842\\
74	0.0942	-3.74812573416747\\
74	0.0948	-3.65720338381647\\
74	0.0954	-3.56628103346546\\
74	0.096	-3.47535868311445\\
74	0.0966	-3.3844363327635\\
74	0.0972	-3.29351398241249\\
74	0.0978	-3.20259163206148\\
74	0.0984	-3.1116692817105\\
74	0.099	-3.02074693135953\\
74	0.0996	-2.92982458100852\\
74	0.1002	-2.83890223065754\\
74	0.1008	-2.74797988030653\\
74	0.1014	-2.65705752995555\\
74	0.102	-2.56613517960457\\
74	0.1026	-2.47521282925356\\
74	0.1032	-2.38429047890256\\
74	0.1038	-2.29336812855161\\
74	0.1044	-2.2024457782006\\
74	0.105	-2.11152342784959\\
74	0.1056	-2.02060107749858\\
74	0.1062	-1.92967872714763\\
74	0.1068	-1.83875637679662\\
74	0.1074	-1.74783402644562\\
74	0.108	-1.65691167609467\\
74	0.1086	-1.56598932574366\\
74	0.1092	-1.47506697539265\\
74	0.1098	-1.38414462504167\\
74	0.1104	-1.29322227469069\\
74	0.111	-1.20229992433968\\
74	0.1116	-1.11137757398868\\
74	0.1122	-1.02045522363773\\
74	0.1128	-0.929532873286689\\
74	0.1134	-0.838610522935738\\
74	0.114	-0.747688172584702\\
74	0.1146	-0.656765822233723\\
74	0.1152	-0.565843471882744\\
74	0.1158	-0.474921121531764\\
74	0.1164	-0.383998771180785\\
74	0.117	-0.293076420829777\\
74	0.1176	-0.20215407047877\\
74	0.1182	-0.11123172012779\\
74	0.1188	-0.0203093697768395\\
74	0.1194	0.0706129805741966\\
74	0.12	0.161535330925176\\
74	0.1206	0.252457681276155\\
74	0.1212	0.343380031627134\\
74	0.1218	0.434302381978171\\
74	0.1224	0.52522473232915\\
74	0.123	0.616147082680129\\
};
\end{axis}

\begin{axis}[%
width=4.927496cm,
height=3.870968cm,
at={(6.483547cm,10.752688cm)},
scale only axis,
xmin=56,
xmax=74,
tick align=outside,
xlabel={$L_{cut}$},
xmajorgrids,
ymin=0.093,
ymax=0.123,
ylabel={$D_{rlx}$},
ymajorgrids,
zmin=-10,
zmax=1.71672619020039,
zlabel={$x_1,x_1$},
zmajorgrids,
view={-140}{50},
legend style={at={(1.03,1)},anchor=north west,legend cell align=left,align=left,draw=white!15!black}
]
\addplot3[only marks,mark=*,mark options={},mark size=1.5000pt,color=mycolor1] plot table[row sep=crcr,]{%
74	0.123	0.00740370360905991\\
72	0.113	-0.186626587578763\\
61	0.095	-0.150607769785267\\
56	0.093	-0.216975707230133\\
};
\addplot3[only marks,mark=*,mark options={},mark size=1.5000pt,color=black] plot table[row sep=crcr,]{%
69	0.104	-0.201969248341598\\
};

\addplot3[%
surf,
opacity=0.7,
shader=interp,
colormap={mymap}{[1pt] rgb(0pt)=(0.0901961,0.239216,0.0745098); rgb(1pt)=(0.0945149,0.242058,0.0739522); rgb(2pt)=(0.0988592,0.244894,0.0733566); rgb(3pt)=(0.103229,0.247724,0.0727241); rgb(4pt)=(0.107623,0.250549,0.0720557); rgb(5pt)=(0.112043,0.253367,0.0713525); rgb(6pt)=(0.116487,0.25618,0.0706154); rgb(7pt)=(0.120956,0.258986,0.0698456); rgb(8pt)=(0.125449,0.261787,0.0690441); rgb(9pt)=(0.129967,0.264581,0.0682118); rgb(10pt)=(0.134508,0.26737,0.06735); rgb(11pt)=(0.139074,0.270152,0.0664596); rgb(12pt)=(0.143663,0.272929,0.0655416); rgb(13pt)=(0.148275,0.275699,0.0645971); rgb(14pt)=(0.152911,0.278463,0.0636271); rgb(15pt)=(0.15757,0.281221,0.0626328); rgb(16pt)=(0.162252,0.283973,0.0616151); rgb(17pt)=(0.166957,0.286719,0.060575); rgb(18pt)=(0.171685,0.289458,0.0595136); rgb(19pt)=(0.176434,0.292191,0.0584321); rgb(20pt)=(0.181207,0.294918,0.0573313); rgb(21pt)=(0.186001,0.297639,0.0562123); rgb(22pt)=(0.190817,0.300353,0.0550763); rgb(23pt)=(0.195655,0.303061,0.0539242); rgb(24pt)=(0.200514,0.305763,0.052757); rgb(25pt)=(0.205395,0.308459,0.0515759); rgb(26pt)=(0.210296,0.311149,0.0503624); rgb(27pt)=(0.215212,0.313846,0.0490067); rgb(28pt)=(0.220142,0.316548,0.0475043); rgb(29pt)=(0.22509,0.319254,0.0458704); rgb(30pt)=(0.230056,0.321962,0.0441205); rgb(31pt)=(0.235042,0.324671,0.04227); rgb(32pt)=(0.240048,0.327379,0.0403343); rgb(33pt)=(0.245078,0.330085,0.0383287); rgb(34pt)=(0.250131,0.332786,0.0362688); rgb(35pt)=(0.25521,0.335482,0.0341698); rgb(36pt)=(0.260317,0.33817,0.0320472); rgb(37pt)=(0.265451,0.340849,0.0299163); rgb(38pt)=(0.270616,0.343517,0.0277927); rgb(39pt)=(0.275813,0.346172,0.0256916); rgb(40pt)=(0.281043,0.348814,0.0236284); rgb(41pt)=(0.286307,0.35144,0.0216186); rgb(42pt)=(0.291607,0.354048,0.0196776); rgb(43pt)=(0.296945,0.356637,0.0178207); rgb(44pt)=(0.302322,0.359206,0.0160634); rgb(45pt)=(0.307739,0.361753,0.0144211); rgb(46pt)=(0.313198,0.364275,0.0129091); rgb(47pt)=(0.318701,0.366772,0.0115428); rgb(48pt)=(0.324249,0.369242,0.0103377); rgb(49pt)=(0.329843,0.371682,0.00930909); rgb(50pt)=(0.335485,0.374093,0.00847245); rgb(51pt)=(0.341176,0.376471,0.00784314); rgb(52pt)=(0.346925,0.378826,0.00732741); rgb(53pt)=(0.352735,0.381168,0.00682184); rgb(54pt)=(0.358605,0.383497,0.00632729); rgb(55pt)=(0.364532,0.385812,0.00584464); rgb(56pt)=(0.370516,0.388113,0.00537476); rgb(57pt)=(0.376552,0.390399,0.00491852); rgb(58pt)=(0.38264,0.39267,0.00447681); rgb(59pt)=(0.388777,0.394925,0.00405048); rgb(60pt)=(0.394962,0.397164,0.00364042); rgb(61pt)=(0.401191,0.399386,0.00324749); rgb(62pt)=(0.407464,0.401592,0.00287258); rgb(63pt)=(0.413777,0.40378,0.00251655); rgb(64pt)=(0.420129,0.40595,0.00218028); rgb(65pt)=(0.426518,0.408102,0.00186463); rgb(66pt)=(0.432942,0.410234,0.00157049); rgb(67pt)=(0.439399,0.412348,0.00129873); rgb(68pt)=(0.445885,0.414441,0.00105022); rgb(69pt)=(0.452401,0.416515,0.000825833); rgb(70pt)=(0.458942,0.418567,0.000626441); rgb(71pt)=(0.465508,0.420599,0.00045292); rgb(72pt)=(0.472096,0.422609,0.000306141); rgb(73pt)=(0.478704,0.424596,0.000186979); rgb(74pt)=(0.485331,0.426562,9.63073e-05); rgb(75pt)=(0.491973,0.428504,3.49981e-05); rgb(76pt)=(0.498628,0.430422,3.92506e-06); rgb(77pt)=(0.505323,0.432315,0); rgb(78pt)=(0.512206,0.434168,0); rgb(79pt)=(0.519282,0.435983,0); rgb(80pt)=(0.526529,0.437764,0); rgb(81pt)=(0.533922,0.439512,0); rgb(82pt)=(0.54144,0.441232,0); rgb(83pt)=(0.549059,0.442927,0); rgb(84pt)=(0.556756,0.444599,0); rgb(85pt)=(0.564508,0.446252,0); rgb(86pt)=(0.572292,0.447889,0); rgb(87pt)=(0.580084,0.449514,0); rgb(88pt)=(0.587863,0.451129,0); rgb(89pt)=(0.595604,0.452737,0); rgb(90pt)=(0.603284,0.454343,0); rgb(91pt)=(0.610882,0.455948,0); rgb(92pt)=(0.618373,0.457556,0); rgb(93pt)=(0.625734,0.459171,0); rgb(94pt)=(0.632943,0.460795,0); rgb(95pt)=(0.639976,0.462432,0); rgb(96pt)=(0.64681,0.464084,0); rgb(97pt)=(0.653423,0.465756,0); rgb(98pt)=(0.659791,0.46745,0); rgb(99pt)=(0.665891,0.469169,0); rgb(100pt)=(0.6717,0.470916,0); rgb(101pt)=(0.677195,0.472696,0); rgb(102pt)=(0.682353,0.47451,0); rgb(103pt)=(0.687242,0.476355,0); rgb(104pt)=(0.691952,0.478225,0); rgb(105pt)=(0.696497,0.480118,0); rgb(106pt)=(0.700887,0.482033,0); rgb(107pt)=(0.705134,0.483968,0); rgb(108pt)=(0.709251,0.485921,0); rgb(109pt)=(0.713249,0.487891,0); rgb(110pt)=(0.71714,0.489876,0); rgb(111pt)=(0.720936,0.491875,0); rgb(112pt)=(0.724649,0.493887,0); rgb(113pt)=(0.72829,0.495909,0); rgb(114pt)=(0.731872,0.49794,0); rgb(115pt)=(0.735406,0.499979,0); rgb(116pt)=(0.738904,0.502025,0); rgb(117pt)=(0.742378,0.504075,0); rgb(118pt)=(0.74584,0.506128,0); rgb(119pt)=(0.749302,0.508182,0); rgb(120pt)=(0.752775,0.510237,0); rgb(121pt)=(0.756272,0.51229,0); rgb(122pt)=(0.759804,0.514339,0); rgb(123pt)=(0.763384,0.516385,0); rgb(124pt)=(0.767022,0.518424,0); rgb(125pt)=(0.770731,0.520455,0); rgb(126pt)=(0.774523,0.522478,0); rgb(127pt)=(0.77841,0.524489,0); rgb(128pt)=(0.782391,0.526491,0); rgb(129pt)=(0.786402,0.528496,0); rgb(130pt)=(0.790431,0.530506,0); rgb(131pt)=(0.794478,0.532521,0); rgb(132pt)=(0.798541,0.534539,0); rgb(133pt)=(0.802619,0.53656,0); rgb(134pt)=(0.806712,0.538584,0); rgb(135pt)=(0.81082,0.540609,0); rgb(136pt)=(0.81494,0.542635,0); rgb(137pt)=(0.819074,0.54466,0); rgb(138pt)=(0.823219,0.546686,0); rgb(139pt)=(0.827374,0.548709,0); rgb(140pt)=(0.831541,0.55073,0); rgb(141pt)=(0.835716,0.552749,0); rgb(142pt)=(0.8399,0.554763,0); rgb(143pt)=(0.844092,0.556774,0); rgb(144pt)=(0.848292,0.558779,0); rgb(145pt)=(0.852497,0.560778,0); rgb(146pt)=(0.856708,0.562771,0); rgb(147pt)=(0.860924,0.564756,0); rgb(148pt)=(0.865143,0.566733,0); rgb(149pt)=(0.869366,0.568701,0); rgb(150pt)=(0.873592,0.57066,0); rgb(151pt)=(0.877819,0.572608,0); rgb(152pt)=(0.882047,0.574545,0); rgb(153pt)=(0.886275,0.576471,0); rgb(154pt)=(0.890659,0.578362,0); rgb(155pt)=(0.895333,0.580203,0); rgb(156pt)=(0.900258,0.581999,0); rgb(157pt)=(0.905397,0.583755,0); rgb(158pt)=(0.910711,0.585479,0); rgb(159pt)=(0.916164,0.587176,0); rgb(160pt)=(0.921717,0.588852,0); rgb(161pt)=(0.927333,0.590513,0); rgb(162pt)=(0.932974,0.592166,0); rgb(163pt)=(0.938602,0.593815,0); rgb(164pt)=(0.94418,0.595468,0); rgb(165pt)=(0.949669,0.59713,0); rgb(166pt)=(0.955033,0.598808,0); rgb(167pt)=(0.960233,0.600507,0); rgb(168pt)=(0.965232,0.602233,0); rgb(169pt)=(0.969992,0.603992,0); rgb(170pt)=(0.974475,0.605791,0); rgb(171pt)=(0.978643,0.607636,0); rgb(172pt)=(0.98246,0.609532,0); rgb(173pt)=(0.985886,0.611486,0); rgb(174pt)=(0.988885,0.613503,0); rgb(175pt)=(0.991419,0.61559,0); rgb(176pt)=(0.99345,0.617753,0); rgb(177pt)=(0.99494,0.619997,0); rgb(178pt)=(0.995851,0.622329,0); rgb(179pt)=(0.996226,0.624763,0); rgb(180pt)=(0.996512,0.627352,0); rgb(181pt)=(0.996788,0.630095,0); rgb(182pt)=(0.997053,0.632982,0); rgb(183pt)=(0.997308,0.636004,0); rgb(184pt)=(0.997552,0.639152,0); rgb(185pt)=(0.997785,0.642416,0); rgb(186pt)=(0.998006,0.645786,0); rgb(187pt)=(0.998217,0.649253,0); rgb(188pt)=(0.998416,0.652807,0); rgb(189pt)=(0.998605,0.656439,0); rgb(190pt)=(0.998781,0.660138,0); rgb(191pt)=(0.998946,0.663897,0); rgb(192pt)=(0.9991,0.667704,0); rgb(193pt)=(0.999242,0.67155,0); rgb(194pt)=(0.999372,0.675427,0); rgb(195pt)=(0.99949,0.679323,0); rgb(196pt)=(0.999596,0.68323,0); rgb(197pt)=(0.99969,0.687139,0); rgb(198pt)=(0.999771,0.691039,0); rgb(199pt)=(0.999841,0.694921,0); rgb(200pt)=(0.999898,0.698775,0); rgb(201pt)=(0.999942,0.702592,0); rgb(202pt)=(0.999974,0.706363,0); rgb(203pt)=(0.999994,0.710077,0); rgb(204pt)=(1,0.713725,0); rgb(205pt)=(1,0.717341,0); rgb(206pt)=(1,0.720963,0); rgb(207pt)=(1,0.724591,0); rgb(208pt)=(1,0.728226,0); rgb(209pt)=(1,0.731867,0); rgb(210pt)=(1,0.735514,0); rgb(211pt)=(1,0.739167,0); rgb(212pt)=(1,0.742827,0); rgb(213pt)=(1,0.746493,0); rgb(214pt)=(1,0.750165,0); rgb(215pt)=(1,0.753843,0); rgb(216pt)=(1,0.757527,0); rgb(217pt)=(1,0.761217,0); rgb(218pt)=(1,0.764913,0); rgb(219pt)=(1,0.768615,0); rgb(220pt)=(1,0.772324,0); rgb(221pt)=(1,0.776038,0); rgb(222pt)=(1,0.779758,0); rgb(223pt)=(1,0.783484,0); rgb(224pt)=(1,0.787215,0); rgb(225pt)=(1,0.790953,0); rgb(226pt)=(1,0.794696,0); rgb(227pt)=(1,0.798445,0); rgb(228pt)=(1,0.8022,0); rgb(229pt)=(1,0.805961,0); rgb(230pt)=(1,0.809727,0); rgb(231pt)=(1,0.8135,0); rgb(232pt)=(1,0.817278,0); rgb(233pt)=(1,0.821063,0); rgb(234pt)=(1,0.824854,0); rgb(235pt)=(1,0.828652,0); rgb(236pt)=(1,0.832455,0); rgb(237pt)=(1,0.836265,0); rgb(238pt)=(1,0.840081,0); rgb(239pt)=(1,0.843903,0); rgb(240pt)=(1,0.847732,0); rgb(241pt)=(1,0.851566,0); rgb(242pt)=(1,0.855406,0); rgb(243pt)=(1,0.859253,0); rgb(244pt)=(1,0.863106,0); rgb(245pt)=(1,0.866964,0); rgb(246pt)=(1,0.870829,0); rgb(247pt)=(1,0.8747,0); rgb(248pt)=(1,0.878577,0); rgb(249pt)=(1,0.88246,0); rgb(250pt)=(1,0.886349,0); rgb(251pt)=(1,0.890243,0); rgb(252pt)=(1,0.894144,0); rgb(253pt)=(1,0.898051,0); rgb(254pt)=(1,0.901964,0); rgb(255pt)=(1,0.905882,0)},
mesh/rows=49]
table[row sep=crcr,header=false] {%
%
56	0.093	-0.216975707099024\\
56	0.0936	-0.399378302235291\\
56	0.0942	-0.581780897371544\\
56	0.0948	-0.764183492507797\\
56	0.0954	-0.94658608764405\\
56	0.096	-1.1289886827803\\
56	0.0966	-1.31139127791656\\
56	0.0972	-1.49379387305281\\
56	0.0978	-1.67619646818908\\
56	0.0984	-1.85859906332533\\
56	0.099	-2.04100165846158\\
56	0.0996	-2.22340425359783\\
56	0.1002	-2.40580684873409\\
56	0.1008	-2.58820944387034\\
56	0.1014	-2.77061203900661\\
56	0.102	-2.95301463414285\\
56	0.1026	-3.13541722927911\\
56	0.1032	-3.31781982441537\\
56	0.1038	-3.50022241955162\\
56	0.1044	-3.68262501468787\\
56	0.105	-3.86502760982412\\
56	0.1056	-4.04743020496039\\
56	0.1062	-4.22983280009663\\
56	0.1068	-4.4122353952329\\
56	0.1074	-4.59463799036915\\
56	0.108	-4.7770405855054\\
56	0.1086	-4.95944318064167\\
56	0.1092	-5.14184577577791\\
56	0.1098	-5.32424837091418\\
56	0.1104	-5.50665096605042\\
56	0.111	-5.68905356118668\\
56	0.1116	-5.87145615632294\\
56	0.1122	-6.05385875145919\\
56	0.1128	-6.23626134659546\\
56	0.1134	-6.41866394173169\\
56	0.114	-6.60106653686796\\
56	0.1146	-6.78346913200421\\
56	0.1152	-6.96587172714047\\
56	0.1158	-7.14827432227672\\
56	0.1164	-7.33067691741297\\
56	0.117	-7.51307951254923\\
56	0.1176	-7.69548210768548\\
56	0.1182	-7.87788470282175\\
56	0.1188	-8.060287297958\\
56	0.1194	-8.24268989309425\\
56	0.12	-8.4250924882305\\
56	0.1206	-8.60749508336674\\
56	0.1212	-8.78989767850302\\
56	0.1218	-8.97230027363925\\
56	0.1224	-9.15470286877553\\
56	0.123	-9.33710546391178\\
56.375	0.093	-0.176690250905281\\
56.375	0.0936	-0.356005009678981\\
56.375	0.0942	-0.535319768452666\\
56.375	0.0948	-0.714634527226366\\
56.375	0.0954	-0.893949286000051\\
56.375	0.096	-1.07326404477375\\
56.375	0.0966	-1.25257880354744\\
56.375	0.0972	-1.43189356232114\\
56.375	0.0978	-1.61120832109484\\
56.375	0.0984	-1.79052307986852\\
56.375	0.099	-1.96983783864222\\
56.375	0.0996	-2.14915259741591\\
56.375	0.1002	-2.32846735618961\\
56.375	0.1008	-2.50778211496329\\
56.375	0.1014	-2.68709687373699\\
56.375	0.102	-2.86641163251068\\
56.375	0.1026	-3.04572639128438\\
56.375	0.1032	-3.22504115005808\\
56.375	0.1038	-3.40435590883176\\
56.375	0.1044	-3.58367066760546\\
56.375	0.105	-3.76298542637915\\
56.375	0.1056	-3.94230018515285\\
56.375	0.1062	-4.12161494392653\\
56.375	0.1068	-4.30092970270023\\
56.375	0.1074	-4.48024446147393\\
56.375	0.108	-4.65955922024762\\
56.375	0.1086	-4.83887397902132\\
56.375	0.1092	-5.018188737795\\
56.375	0.1098	-5.1975034965687\\
56.375	0.1104	-5.37681825534239\\
56.375	0.111	-5.55613301411609\\
56.375	0.1116	-5.73544777288977\\
56.375	0.1122	-5.91476253166347\\
56.375	0.1128	-6.09407729043717\\
56.375	0.1134	-6.27339204921086\\
56.375	0.114	-6.45270680798455\\
56.375	0.1146	-6.63202156675824\\
56.375	0.1152	-6.81133632553194\\
56.375	0.1158	-6.99065108430563\\
56.375	0.1164	-7.16996584307932\\
56.375	0.117	-7.34928060185301\\
56.375	0.1176	-7.52859536062671\\
56.375	0.1182	-7.70791011940041\\
56.375	0.1188	-7.8872248781741\\
56.375	0.1194	-8.06653963694779\\
56.375	0.12	-8.24585439572148\\
56.375	0.1206	-8.42516915449517\\
56.375	0.1212	-8.60448391326888\\
56.375	0.1218	-8.78379867204255\\
56.375	0.1224	-8.96311343081626\\
56.375	0.123	-9.14242818958995\\
56.75	0.093	-0.136404794711552\\
56.75	0.0936	-0.312631717122684\\
56.75	0.0942	-0.488858639533817\\
56.75	0.0948	-0.665085561944949\\
56.75	0.0954	-0.841312484356067\\
56.75	0.096	-1.01753940676721\\
56.75	0.0966	-1.19376632917833\\
56.75	0.0972	-1.36999325158948\\
56.75	0.0978	-1.54622017400061\\
56.75	0.0984	-1.72244709641173\\
56.75	0.099	-1.89867401882287\\
56.75	0.0996	-2.07490094123399\\
56.75	0.1002	-2.25112786364514\\
56.75	0.1008	-2.42735478605626\\
56.75	0.1014	-2.6035817084674\\
56.75	0.102	-2.77980863087852\\
56.75	0.1026	-2.95603555328965\\
56.75	0.1032	-3.1322624757008\\
56.75	0.1038	-3.30848939811192\\
56.75	0.1044	-3.48471632052305\\
56.75	0.105	-3.66094324293418\\
56.75	0.1056	-3.83717016534531\\
56.75	0.1062	-4.01339708775645\\
56.75	0.1068	-4.18962401016758\\
56.75	0.1074	-4.36585093257872\\
56.75	0.108	-4.54207785498984\\
56.75	0.1086	-4.71830477740097\\
56.75	0.1092	-4.89453169981211\\
56.75	0.1098	-5.07075862222324\\
56.75	0.1104	-5.24698554463437\\
56.75	0.111	-5.4232124670455\\
56.75	0.1116	-5.59943938945663\\
56.75	0.1122	-5.77566631186777\\
56.75	0.1128	-5.9518932342789\\
56.75	0.1134	-6.12812015669003\\
56.75	0.114	-6.30434707910116\\
56.75	0.1146	-6.4805740015123\\
56.75	0.1152	-6.65680092392343\\
56.75	0.1158	-6.83302784633455\\
56.75	0.1164	-7.00925476874569\\
56.75	0.117	-7.18548169115681\\
56.75	0.1176	-7.36170861356796\\
56.75	0.1182	-7.53793553597909\\
56.75	0.1188	-7.71416245839021\\
56.75	0.1194	-7.89038938080135\\
56.75	0.12	-8.06661630321247\\
56.75	0.1206	-8.2428432256236\\
56.75	0.1212	-8.41907014803475\\
56.75	0.1218	-8.59529707044585\\
56.75	0.1224	-8.77152399285701\\
56.75	0.123	-8.94775091526813\\
57.125	0.093	-0.096119338517795\\
57.125	0.0936	-0.269258424566374\\
57.125	0.0942	-0.442397510614938\\
57.125	0.0948	-0.615536596663517\\
57.125	0.0954	-0.788675682712082\\
57.125	0.096	-0.961814768760661\\
57.125	0.0966	-1.13495385480921\\
57.125	0.0972	-1.30809294085779\\
57.125	0.0978	-1.48123202690637\\
57.125	0.0984	-1.65437111295493\\
57.125	0.099	-1.82751019900351\\
57.125	0.0996	-2.00064928505206\\
57.125	0.1002	-2.17378837110064\\
57.125	0.1008	-2.34692745714921\\
57.125	0.1014	-2.52006654319779\\
57.125	0.102	-2.69320562924635\\
57.125	0.1026	-2.86634471529493\\
57.125	0.1032	-3.03948380134349\\
57.125	0.1038	-3.21262288739206\\
57.125	0.1044	-3.38576197344064\\
57.125	0.105	-3.5589010594892\\
57.125	0.1056	-3.73204014553778\\
57.125	0.1062	-3.90517923158635\\
57.125	0.1068	-4.07831831763491\\
57.125	0.1074	-4.25145740368349\\
57.125	0.108	-4.42459648973205\\
57.125	0.1086	-4.59773557578063\\
57.125	0.1092	-4.7708746618292\\
57.125	0.1098	-4.94401374787776\\
57.125	0.1104	-5.11715283392633\\
57.125	0.111	-5.2902919199749\\
57.125	0.1116	-5.46343100602347\\
57.125	0.1122	-5.63657009207205\\
57.125	0.1128	-5.80970917812063\\
57.125	0.1134	-5.98284826416918\\
57.125	0.114	-6.15598735021776\\
57.125	0.1146	-6.32912643626632\\
57.125	0.1152	-6.5022655223149\\
57.125	0.1158	-6.67540460836346\\
57.125	0.1164	-6.84854369441204\\
57.125	0.117	-7.02168278046059\\
57.125	0.1176	-7.19482186650917\\
57.125	0.1182	-7.36796095255775\\
57.125	0.1188	-7.54110003860632\\
57.125	0.1194	-7.7142391246549\\
57.125	0.12	-7.88737821070345\\
57.125	0.1206	-8.06051729675201\\
57.125	0.1212	-8.2336563828006\\
57.125	0.1218	-8.40679546884915\\
57.125	0.1224	-8.57993455489775\\
57.125	0.123	-8.75307364094631\\
57.5	0.093	-0.0558338823240661\\
57.5	0.0936	-0.225885132010077\\
57.5	0.0942	-0.395936381696089\\
57.5	0.0948	-0.5659876313821\\
57.5	0.0954	-0.736038881068097\\
57.5	0.096	-0.906090130754109\\
57.5	0.0966	-1.07614138044012\\
57.5	0.0972	-1.24619263012613\\
57.5	0.0978	-1.41624387981214\\
57.5	0.0984	-1.58629512949814\\
57.5	0.099	-1.75634637918417\\
57.5	0.0996	-1.92639762887016\\
57.5	0.1002	-2.09644887855617\\
57.5	0.1008	-2.26650012824217\\
57.5	0.1014	-2.4365513779282\\
57.5	0.102	-2.60660262761419\\
57.5	0.1026	-2.77665387730021\\
57.5	0.1032	-2.94670512698622\\
57.5	0.1038	-3.11675637667221\\
57.5	0.1044	-3.28680762635824\\
57.5	0.105	-3.45685887604424\\
57.5	0.1056	-3.62691012573025\\
57.5	0.1062	-3.79696137541625\\
57.5	0.1068	-3.96701262510227\\
57.5	0.1074	-4.13706387478828\\
57.5	0.108	-4.30711512447428\\
57.5	0.1086	-4.47716637416029\\
57.5	0.1092	-4.64721762384629\\
57.5	0.1098	-4.81726887353231\\
57.5	0.1104	-4.98732012321831\\
57.5	0.111	-5.15737137290432\\
57.5	0.1116	-5.32742262259033\\
57.5	0.1122	-5.49747387227634\\
57.5	0.1128	-5.66752512196236\\
57.5	0.1134	-5.83757637164835\\
57.5	0.114	-6.00762762133436\\
57.5	0.1146	-6.17767887102038\\
57.5	0.1152	-6.34773012070639\\
57.5	0.1158	-6.51778137039238\\
57.5	0.1164	-6.6878326200784\\
57.5	0.117	-6.85788386976439\\
57.5	0.1176	-7.02793511945042\\
57.5	0.1182	-7.19798636913643\\
57.5	0.1188	-7.36803761882243\\
57.5	0.1194	-7.53808886850844\\
57.5	0.12	-7.70814011819445\\
57.5	0.1206	-7.87819136788045\\
57.5	0.1212	-8.04824261756647\\
57.5	0.1218	-8.21829386725246\\
57.5	0.1224	-8.3883451169385\\
57.5	0.123	-8.55839636662449\\
57.875	0.093	-0.0155484261303229\\
57.875	0.0936	-0.182511839453781\\
57.875	0.0942	-0.349475252777211\\
57.875	0.0948	-0.516438666100669\\
57.875	0.0954	-0.683402079424113\\
57.875	0.096	-0.850365492747557\\
57.875	0.0966	-1.017328906071\\
57.875	0.0972	-1.18429231939444\\
57.875	0.0978	-1.3512557327179\\
57.875	0.0984	-1.51821914604135\\
57.875	0.099	-1.68518255936479\\
57.875	0.0996	-1.85214597268823\\
57.875	0.1002	-2.01910938601169\\
57.875	0.1008	-2.18607279933512\\
57.875	0.1014	-2.35303621265858\\
57.875	0.102	-2.51999962598202\\
57.875	0.1026	-2.68696303930547\\
57.875	0.1032	-2.85392645262893\\
57.875	0.1038	-3.02088986595237\\
57.875	0.1044	-3.18785327927581\\
57.875	0.105	-3.35481669259926\\
57.875	0.1056	-3.5217801059227\\
57.875	0.1062	-3.68874351924615\\
57.875	0.1068	-3.8557069325696\\
57.875	0.1074	-4.02267034589305\\
57.875	0.108	-4.18963375921649\\
57.875	0.1086	-4.35659717253995\\
57.875	0.1092	-4.52356058586338\\
57.875	0.1098	-4.69052399918684\\
57.875	0.1104	-4.85748741251028\\
57.875	0.111	-5.02445082583372\\
57.875	0.1116	-5.19141423915717\\
57.875	0.1122	-5.35837765248061\\
57.875	0.1128	-5.52534106580407\\
57.875	0.1134	-5.69230447912751\\
57.875	0.114	-5.85926789245097\\
57.875	0.1146	-6.0262313057744\\
57.875	0.1152	-6.19319471909786\\
57.875	0.1158	-6.36015813242129\\
57.875	0.1164	-6.52712154574475\\
57.875	0.117	-6.69408495906819\\
57.875	0.1176	-6.86104837239164\\
57.875	0.1182	-7.02801178571509\\
57.875	0.1188	-7.19497519903854\\
57.875	0.1194	-7.36193861236198\\
57.875	0.12	-7.52890202568543\\
57.875	0.1206	-7.69586543900887\\
57.875	0.1212	-7.86282885233233\\
57.875	0.1218	-8.02979226565576\\
57.875	0.1224	-8.19675567897922\\
57.875	0.123	-8.36371909230266\\
58.25	0.093	0.024737030063406\\
58.25	0.0936	-0.139138546897485\\
58.25	0.0942	-0.303014123858361\\
58.25	0.0948	-0.466889700819252\\
58.25	0.0954	-0.630765277780128\\
58.25	0.096	-0.794640854741019\\
58.25	0.0966	-0.958516431701895\\
58.25	0.0972	-1.12239200866279\\
58.25	0.0978	-1.28626758562368\\
58.25	0.0984	-1.45014316258455\\
58.25	0.099	-1.61401873954544\\
58.25	0.0996	-1.77789431650632\\
58.25	0.1002	-1.94176989346721\\
58.25	0.1008	-2.10564547042809\\
58.25	0.1014	-2.26952104738899\\
58.25	0.102	-2.43339662434985\\
58.25	0.1026	-2.59727220131076\\
58.25	0.1032	-2.76114777827165\\
58.25	0.1038	-2.92502335523253\\
58.25	0.1044	-3.08889893219342\\
58.25	0.105	-3.25277450915429\\
58.25	0.1056	-3.41665008611518\\
58.25	0.1062	-3.58052566307606\\
58.25	0.1068	-3.74440124003695\\
58.25	0.1074	-3.90827681699784\\
58.25	0.108	-4.07215239395872\\
58.25	0.1086	-4.23602797091961\\
58.25	0.1092	-4.39990354788048\\
58.25	0.1098	-4.56377912484137\\
58.25	0.1104	-4.72765470180225\\
58.25	0.111	-4.89153027876314\\
58.25	0.1116	-5.05540585572402\\
58.25	0.1122	-5.21928143268491\\
58.25	0.1128	-5.3831570096458\\
58.25	0.1134	-5.54703258660669\\
58.25	0.114	-5.71090816356758\\
58.25	0.1146	-5.87478374052846\\
58.25	0.1152	-6.03865931748935\\
58.25	0.1158	-6.20253489445022\\
58.25	0.1164	-6.36641047141111\\
58.25	0.117	-6.53028604837199\\
58.25	0.1176	-6.69416162533288\\
58.25	0.1182	-6.85803720229377\\
58.25	0.1188	-7.02191277925465\\
58.25	0.1194	-7.18578835621554\\
58.25	0.12	-7.34966393317642\\
58.25	0.1206	-7.51353951013729\\
58.25	0.1212	-7.6774150870982\\
58.25	0.1218	-7.84129066405906\\
58.25	0.1224	-8.00516624101996\\
58.25	0.123	-8.16904181798084\\
58.625	0.093	0.0650224862571349\\
58.625	0.0936	-0.0957652543412024\\
58.625	0.0942	-0.256552994939511\\
58.625	0.0948	-0.417340735537849\\
58.625	0.0954	-0.578128476136158\\
58.625	0.096	-0.738916216734495\\
58.625	0.0966	-0.899703957332804\\
58.625	0.0972	-1.06049169793113\\
58.625	0.0978	-1.22127943852946\\
58.625	0.0984	-1.38206717912777\\
58.625	0.099	-1.54285491972611\\
58.625	0.0996	-1.70364266032442\\
58.625	0.1002	-1.86443040092276\\
58.625	0.1008	-2.02521814152107\\
58.625	0.1014	-2.1860058821194\\
58.625	0.102	-2.34679362271771\\
58.625	0.1026	-2.50758136331605\\
58.625	0.1032	-2.66836910391437\\
58.625	0.1038	-2.8291568445127\\
58.625	0.1044	-2.98994458511102\\
58.625	0.105	-3.15073232570934\\
58.625	0.1056	-3.31152006630766\\
58.625	0.1062	-3.47230780690599\\
58.625	0.1068	-3.63309554750431\\
58.625	0.1074	-3.79388328810263\\
58.625	0.108	-3.95467102870096\\
58.625	0.1086	-4.11545876929928\\
58.625	0.1092	-4.2762465098976\\
58.625	0.1098	-4.43703425049593\\
58.625	0.1104	-4.59782199109425\\
58.625	0.111	-4.75860973169257\\
58.625	0.1116	-4.9193974722909\\
58.625	0.1122	-5.08018521288922\\
58.625	0.1128	-5.24097295348756\\
58.625	0.1134	-5.40176069408587\\
58.625	0.114	-5.5625484346842\\
58.625	0.1146	-5.72333617528251\\
58.625	0.1152	-5.88412391588085\\
58.625	0.1158	-6.04491165647916\\
58.625	0.1164	-6.2056993970775\\
58.625	0.117	-6.3664871376758\\
58.625	0.1176	-6.52727487827413\\
58.625	0.1182	-6.68806261887246\\
58.625	0.1188	-6.84885035947077\\
58.625	0.1194	-7.00963810006911\\
58.625	0.12	-7.17042584066742\\
58.625	0.1206	-7.33121358126574\\
58.625	0.1212	-7.49200132186408\\
58.625	0.1218	-7.65278906246239\\
58.625	0.1224	-7.81357680306073\\
58.625	0.123	-7.97436454365905\\
59	0.093	0.105307942450878\\
59	0.0936	-0.0523919617848918\\
59	0.0942	-0.210091866020647\\
59	0.0948	-0.367791770256403\\
59	0.0954	-0.525491674492159\\
59	0.096	-0.683191578727929\\
59	0.0966	-0.840891482963684\\
59	0.0972	-0.998591387199454\\
59	0.0978	-1.15629129143522\\
59	0.0984	-1.31399119567098\\
59	0.099	-1.47169109990675\\
59	0.0996	-1.62939100414251\\
59	0.1002	-1.78709090837826\\
59	0.1008	-1.94479081261402\\
59	0.1014	-2.10249071684979\\
59	0.102	-2.26019062108554\\
59	0.1026	-2.41789052532131\\
59	0.1032	-2.57559042955708\\
59	0.1038	-2.73329033379284\\
59	0.1044	-2.89099023802861\\
59	0.105	-3.04869014226436\\
59	0.1056	-3.20639004650012\\
59	0.1062	-3.36408995073587\\
59	0.1068	-3.52178985497164\\
59	0.1074	-3.67948975920741\\
59	0.108	-3.83718966344317\\
59	0.1086	-3.99488956767894\\
59	0.1092	-4.15258947191469\\
59	0.1098	-4.31028937615046\\
59	0.1104	-4.46798928038621\\
59	0.111	-4.62568918462198\\
59	0.1116	-4.78338908885773\\
59	0.1122	-4.9410889930935\\
59	0.1128	-5.09878889732927\\
59	0.1134	-5.25648880156503\\
59	0.114	-5.4141887058008\\
59	0.1146	-5.57188861003655\\
59	0.1152	-5.72958851427232\\
59	0.1158	-5.88728841850806\\
59	0.1164	-6.04498832274383\\
59	0.117	-6.20268822697959\\
59	0.1176	-6.36038813121536\\
59	0.1182	-6.51808803545113\\
59	0.1188	-6.67578793968688\\
59	0.1194	-6.83348784392265\\
59	0.12	-6.99118774815841\\
59	0.1206	-7.14888765239417\\
59	0.1212	-7.30658755662994\\
59	0.1218	-7.46428746086568\\
59	0.1224	-7.62198736510146\\
59	0.123	-7.77968726933722\\
59.375	0.093	0.145593398644607\\
59.375	0.0936	-0.0090186692285954\\
59.375	0.0942	-0.163630737101784\\
59.375	0.0948	-0.318242804974986\\
59.375	0.0954	-0.472854872848188\\
59.375	0.096	-0.627466940721391\\
59.375	0.0966	-0.782079008594579\\
59.375	0.0972	-0.936691076467795\\
59.375	0.0978	-1.091303144341\\
59.375	0.0984	-1.24591521221419\\
59.375	0.099	-1.4005272800874\\
59.375	0.0996	-1.55513934796059\\
59.375	0.1002	-1.70975141583379\\
59.375	0.1008	-1.86436348370698\\
59.375	0.1014	-2.0189755515802\\
59.375	0.102	-2.17358761945339\\
59.375	0.1026	-2.32819968732659\\
59.375	0.1032	-2.4828117551998\\
59.375	0.1038	-2.63742382307299\\
59.375	0.1044	-2.7920358909462\\
59.375	0.105	-2.94664795881938\\
59.375	0.1056	-3.1012600266926\\
59.375	0.1062	-3.25587209456579\\
59.375	0.1068	-3.41048416243899\\
59.375	0.1074	-3.56509623031221\\
59.375	0.108	-3.7197082981854\\
59.375	0.1086	-3.8743203660586\\
59.375	0.1092	-4.02893243393179\\
59.375	0.1098	-4.183544501805\\
59.375	0.1104	-4.33815656967819\\
59.375	0.111	-4.49276863755139\\
59.375	0.1116	-4.6473807054246\\
59.375	0.1122	-4.8019927732978\\
59.375	0.1128	-4.956604841171\\
59.375	0.1134	-5.1112169090442\\
59.375	0.114	-5.2658289769174\\
59.375	0.1146	-5.42044104479059\\
59.375	0.1152	-5.5750531126638\\
59.375	0.1158	-5.729665180537\\
59.375	0.1164	-5.8842772484102\\
59.375	0.117	-6.03888931628339\\
59.375	0.1176	-6.1935013841566\\
59.375	0.1182	-6.34811345202981\\
59.375	0.1188	-6.502725519903\\
59.375	0.1194	-6.6573375877762\\
59.375	0.12	-6.8119496556494\\
59.375	0.1206	-6.96656172352259\\
59.375	0.1212	-7.12117379139582\\
59.375	0.1218	-7.27578585926899\\
59.375	0.1224	-7.43039792714221\\
59.375	0.123	-7.5850099950154\\
59.75	0.093	0.185878854838364\\
59.75	0.0936	0.0343546233277152\\
59.75	0.0942	-0.11716960818292\\
59.75	0.0948	-0.268693839693555\\
59.75	0.0954	-0.420218071204189\\
59.75	0.096	-0.571742302714839\\
59.75	0.0966	-0.723266534225473\\
59.75	0.0972	-0.874790765736108\\
59.75	0.0978	-1.02631499724676\\
59.75	0.0984	-1.17783922875739\\
59.75	0.099	-1.32936346026803\\
59.75	0.0996	-1.48088769177866\\
59.75	0.1002	-1.63241192328931\\
59.75	0.1008	-1.78393615479995\\
59.75	0.1014	-1.93546038631058\\
59.75	0.102	-2.08698461782122\\
59.75	0.1026	-2.23850884933186\\
59.75	0.1032	-2.3900330808425\\
59.75	0.1038	-2.54155731235313\\
59.75	0.1044	-2.69308154386378\\
59.75	0.105	-2.8446057753744\\
59.75	0.1056	-2.99613000688505\\
59.75	0.1062	-3.14765423839569\\
59.75	0.1068	-3.29917846990634\\
59.75	0.1074	-3.45070270141697\\
59.75	0.108	-3.60222693292761\\
59.75	0.1086	-3.75375116443826\\
59.75	0.1092	-3.90527539594888\\
59.75	0.1098	-4.05679962745953\\
59.75	0.1104	-4.20832385897016\\
59.75	0.111	-4.35984809048081\\
59.75	0.1116	-4.51137232199143\\
59.75	0.1122	-4.66289655350208\\
59.75	0.1128	-4.81442078501273\\
59.75	0.1134	-4.96594501652335\\
59.75	0.114	-5.117469248034\\
59.75	0.1146	-5.26899347954463\\
59.75	0.1152	-5.42051771105527\\
59.75	0.1158	-5.5720419425659\\
59.75	0.1164	-5.72356617407655\\
59.75	0.117	-5.87509040558717\\
59.75	0.1176	-6.02661463709782\\
59.75	0.1182	-6.17813886860847\\
59.75	0.1188	-6.32966310011911\\
59.75	0.1194	-6.48118733162974\\
59.75	0.12	-6.63271156314038\\
59.75	0.1206	-6.78423579465101\\
59.75	0.1212	-6.93576002616167\\
59.75	0.1218	-7.08728425767228\\
59.75	0.1224	-7.23880848918294\\
59.75	0.123	-7.39033272069358\\
60.125	0.093	0.226164311032093\\
60.125	0.0936	0.0777279158840116\\
60.125	0.0942	-0.0707084792640558\\
60.125	0.0948	-0.219144874412137\\
60.125	0.0954	-0.367581269560219\\
60.125	0.096	-0.516017664708301\\
60.125	0.0966	-0.664454059856368\\
60.125	0.0972	-0.81289045500445\\
60.125	0.0978	-0.961326850152531\\
60.125	0.0984	-1.1097632453006\\
60.125	0.099	-1.25819964044868\\
60.125	0.0996	-1.40663603559675\\
60.125	0.1002	-1.55507243074483\\
60.125	0.1008	-1.70350882589291\\
60.125	0.1014	-1.85194522104099\\
60.125	0.102	-2.00038161618906\\
60.125	0.1026	-2.14881801133714\\
60.125	0.1032	-2.29725440648522\\
60.125	0.1038	-2.44569080163329\\
60.125	0.1044	-2.59412719678137\\
60.125	0.105	-2.74256359192944\\
60.125	0.1056	-2.89099998707754\\
60.125	0.1062	-3.0394363822256\\
60.125	0.1068	-3.18787277737368\\
60.125	0.1074	-3.33630917252177\\
60.125	0.108	-3.48474556766983\\
60.125	0.1086	-3.63318196281791\\
60.125	0.1092	-3.78161835796598\\
60.125	0.1098	-3.93005475311406\\
60.125	0.1104	-4.07849114826213\\
60.125	0.111	-4.22692754341023\\
60.125	0.1116	-4.37536393855829\\
60.125	0.1122	-4.52380033370638\\
60.125	0.1128	-4.67223672885446\\
60.125	0.1134	-4.82067312400252\\
60.125	0.114	-4.96910951915061\\
60.125	0.1146	-5.11754591429867\\
60.125	0.1152	-5.26598230944676\\
60.125	0.1158	-5.41441870459484\\
60.125	0.1164	-5.56285509974292\\
60.125	0.117	-5.71129149489099\\
60.125	0.1176	-5.85972789003907\\
60.125	0.1182	-6.00816428518715\\
60.125	0.1188	-6.15660068033522\\
60.125	0.1194	-6.3050370754833\\
60.125	0.12	-6.45347347063137\\
60.125	0.1206	-6.60190986577943\\
60.125	0.1212	-6.75034626092754\\
60.125	0.1218	-6.8987826560756\\
60.125	0.1224	-7.04721905122369\\
60.125	0.123	-7.19565544637176\\
60.5	0.093	0.266449767225836\\
60.5	0.0936	0.121101208440322\\
60.5	0.0942	-0.0242473503451919\\
60.5	0.0948	-0.169595909130706\\
60.5	0.0954	-0.31494446791622\\
60.5	0.096	-0.460293026701748\\
60.5	0.0966	-0.605641585487248\\
60.5	0.0972	-0.750990144272762\\
60.5	0.0978	-0.896338703058291\\
60.5	0.0984	-1.0416872618438\\
60.5	0.099	-1.18703582062932\\
60.5	0.0996	-1.33238437941483\\
60.5	0.1002	-1.47773293820035\\
60.5	0.1008	-1.62308149698586\\
60.5	0.1014	-1.76843005577138\\
60.5	0.102	-1.91377861455689\\
60.5	0.1026	-2.0591271733424\\
60.5	0.1032	-2.20447573212793\\
60.5	0.1038	-2.34982429091343\\
60.5	0.1044	-2.49517284969896\\
60.5	0.105	-2.64052140848446\\
60.5	0.1056	-2.78586996726999\\
60.5	0.1062	-2.93121852605549\\
60.5	0.1068	-3.07656708484102\\
60.5	0.1074	-3.22191564362653\\
60.5	0.108	-3.36726420241204\\
60.5	0.1086	-3.51261276119757\\
60.5	0.1092	-3.65796131998307\\
60.5	0.1098	-3.8033098787686\\
60.5	0.1104	-3.9486584375541\\
60.5	0.111	-4.09400699633963\\
60.5	0.1116	-4.23935555512513\\
60.5	0.1122	-4.38470411391066\\
60.5	0.1128	-4.53005267269617\\
60.5	0.1134	-4.67540123148169\\
60.5	0.114	-4.8207497902672\\
60.5	0.1146	-4.96609834905271\\
60.5	0.1152	-5.11144690783823\\
60.5	0.1158	-5.25679546662374\\
60.5	0.1164	-5.40214402540926\\
60.5	0.117	-5.54749258419477\\
60.5	0.1176	-5.69284114298028\\
60.5	0.1182	-5.83818970176581\\
60.5	0.1188	-5.98353826055133\\
60.5	0.1194	-6.12888681933684\\
60.5	0.12	-6.27423537812236\\
60.5	0.1206	-6.41958393690786\\
60.5	0.1212	-6.5649324956934\\
60.5	0.1218	-6.71028105447888\\
60.5	0.1224	-6.85562961326443\\
60.5	0.123	-7.00097817204993\\
60.875	0.093	0.306735223419565\\
60.875	0.0936	0.164474500996604\\
60.875	0.0942	0.0222137785736578\\
60.875	0.0948	-0.120046943849289\\
60.875	0.0954	-0.262307666272235\\
60.875	0.096	-0.404568388695196\\
60.875	0.0966	-0.546829111118143\\
60.875	0.0972	-0.689089833541104\\
60.875	0.0978	-0.831350555964065\\
60.875	0.0984	-0.973611278387011\\
60.875	0.099	-1.11587200080997\\
60.875	0.0996	-1.25813272323292\\
60.875	0.1002	-1.40039344565588\\
60.875	0.1008	-1.54265416807883\\
60.875	0.1014	-1.68491489050179\\
60.875	0.102	-1.82717561292473\\
60.875	0.1026	-1.96943633534769\\
60.875	0.1032	-2.11169705777066\\
60.875	0.1038	-2.2539577801936\\
60.875	0.1044	-2.39621850261655\\
60.875	0.105	-2.5384792250395\\
60.875	0.1056	-2.68073994746246\\
60.875	0.1062	-2.8230006698854\\
60.875	0.1068	-2.96526139230836\\
60.875	0.1074	-3.10752211473132\\
60.875	0.108	-3.24978283715427\\
60.875	0.1086	-3.39204355957723\\
60.875	0.1092	-3.53430428200018\\
60.875	0.1098	-3.67656500442314\\
60.875	0.1104	-3.81882572684609\\
60.875	0.111	-3.96108644926905\\
60.875	0.1116	-4.10334717169199\\
60.875	0.1122	-4.24560789411495\\
60.875	0.1128	-4.38786861653792\\
60.875	0.1134	-4.53012933896086\\
60.875	0.114	-4.67239006138382\\
60.875	0.1146	-4.81465078380675\\
60.875	0.1152	-4.95691150622972\\
60.875	0.1158	-5.09917222865266\\
60.875	0.1164	-5.24143295107562\\
60.875	0.117	-5.38369367349857\\
60.875	0.1176	-5.52595439592153\\
60.875	0.1182	-5.66821511834449\\
60.875	0.1188	-5.81047584076744\\
60.875	0.1194	-5.9527365631904\\
60.875	0.12	-6.09499728561335\\
60.875	0.1206	-6.23725800803629\\
60.875	0.1212	-6.37951873045927\\
60.875	0.1218	-6.5217794528822\\
60.875	0.1224	-6.66404017530517\\
60.875	0.123	-6.80630089772812\\
61.25	0.093	0.347020679613323\\
61.25	0.0936	0.207847793552915\\
61.25	0.0942	0.0686749074925359\\
61.25	0.0948	-0.0704979785678574\\
61.25	0.0954	-0.209670864628251\\
61.25	0.096	-0.348843750688644\\
61.25	0.0966	-0.488016636749023\\
61.25	0.0972	-0.627189522809431\\
61.25	0.0978	-0.766362408869824\\
61.25	0.0984	-0.905535294930203\\
61.25	0.099	-1.04470818099061\\
61.25	0.0996	-1.18388106705099\\
61.25	0.1002	-1.32305395311138\\
61.25	0.1008	-1.46222683917178\\
61.25	0.1014	-1.60139972523217\\
61.25	0.102	-1.74057261129255\\
61.25	0.1026	-1.87974549735296\\
61.25	0.1032	-2.01891838341335\\
61.25	0.1038	-2.15809126947374\\
61.25	0.1044	-2.29726415553414\\
61.25	0.105	-2.43643704159452\\
61.25	0.1056	-2.57560992765492\\
61.25	0.1062	-2.7147828137153\\
61.25	0.1068	-2.8539556997757\\
61.25	0.1074	-2.9931285858361\\
61.25	0.108	-3.13230147189648\\
61.25	0.1086	-3.27147435795688\\
61.25	0.1092	-3.41064724401727\\
61.25	0.1098	-3.54982013007766\\
61.25	0.1104	-3.68899301613804\\
61.25	0.111	-3.82816590219845\\
61.25	0.1116	-3.96733878825883\\
61.25	0.1122	-4.10651167431922\\
61.25	0.1128	-4.24568456037963\\
61.25	0.1134	-4.38485744644001\\
61.25	0.114	-4.52403033250042\\
61.25	0.1146	-4.6632032185608\\
61.25	0.1152	-4.80237610462119\\
61.25	0.1158	-4.94154899068158\\
61.25	0.1164	-5.08072187674198\\
61.25	0.117	-5.21989476280235\\
61.25	0.1176	-5.35906764886276\\
61.25	0.1182	-5.49824053492316\\
61.25	0.1188	-5.63741342098353\\
61.25	0.1194	-5.77658630704394\\
61.25	0.12	-5.91575919310432\\
61.25	0.1206	-6.0549320791647\\
61.25	0.1212	-6.19410496522512\\
61.25	0.1218	-6.33327785128549\\
61.25	0.1224	-6.47245073734589\\
61.25	0.123	-6.61162362340629\\
61.625	0.093	0.387306135807037\\
61.625	0.0936	0.251221086109197\\
61.625	0.0942	0.115136036411386\\
61.625	0.0948	-0.0209490132864545\\
61.625	0.0954	-0.15703406298428\\
61.625	0.096	-0.29311911268212\\
61.625	0.0966	-0.429204162379932\\
61.625	0.0972	-0.565289212077772\\
61.625	0.0978	-0.701374261775612\\
61.625	0.0984	-0.837459311473438\\
61.625	0.099	-0.973544361171278\\
61.625	0.0996	-1.10962941086909\\
61.625	0.1002	-1.24571446056693\\
61.625	0.1008	-1.38179951026476\\
61.625	0.1014	-1.5178845599626\\
61.625	0.102	-1.65396960966041\\
61.625	0.1026	-1.79005465935825\\
61.625	0.1032	-1.92613970905609\\
61.625	0.1038	-2.06222475875391\\
61.625	0.1044	-2.19830980845174\\
61.625	0.105	-2.33439485814957\\
61.625	0.1056	-2.47047990784741\\
61.625	0.1062	-2.60656495754523\\
61.625	0.1068	-2.74265000724306\\
61.625	0.1074	-2.8787350569409\\
61.625	0.108	-3.01482010663872\\
61.625	0.1086	-3.15090515633656\\
61.625	0.1092	-3.28699020603437\\
61.625	0.1098	-3.42307525573221\\
61.625	0.1104	-3.55916030543004\\
61.625	0.111	-3.69524535512788\\
61.625	0.1116	-3.83133040482571\\
61.625	0.1122	-3.96741545452353\\
61.625	0.1128	-4.10350050422137\\
61.625	0.1134	-4.2395855539192\\
61.625	0.114	-4.37567060361704\\
61.625	0.1146	-4.51175565331485\\
61.625	0.1152	-4.64784070301269\\
61.625	0.1158	-4.78392575271052\\
61.625	0.1164	-4.92001080240836\\
61.625	0.117	-5.05609585210617\\
61.625	0.1176	-5.19218090180401\\
61.625	0.1182	-5.32826595150185\\
61.625	0.1188	-5.46435100119967\\
61.625	0.1194	-5.6004360508975\\
61.625	0.12	-5.73652110059533\\
61.625	0.1206	-5.87260615029315\\
61.625	0.1212	-6.00869119999101\\
61.625	0.1218	-6.1447762496888\\
61.625	0.1224	-6.28086129938666\\
61.625	0.123	-6.41694634908448\\
62	0.093	0.42759159200078\\
62	0.0936	0.294594378665508\\
62	0.0942	0.161597165330249\\
62	0.0948	0.0285999519949769\\
62	0.0954	-0.104397261340281\\
62	0.096	-0.237394474675554\\
62	0.0966	-0.370391688010827\\
62	0.0972	-0.503388901346099\\
62	0.0978	-0.636386114681372\\
62	0.0984	-0.76938332801663\\
62	0.099	-0.902380541351903\\
62	0.0996	-1.03537775468716\\
62	0.1002	-1.16837496802245\\
62	0.1008	-1.30137218135771\\
62	0.1014	-1.43436939469298\\
62	0.102	-1.56736660802824\\
62	0.1026	-1.70036382136351\\
62	0.1032	-1.8333610346988\\
62	0.1038	-1.96635824803406\\
62	0.1044	-2.09935546136933\\
62	0.105	-2.23235267470459\\
62	0.1056	-2.36534988803986\\
62	0.1062	-2.49834710137512\\
62	0.1068	-2.6313443147104\\
62	0.1074	-2.76434152804568\\
62	0.108	-2.89733874138093\\
62	0.1086	-3.03033595471621\\
62	0.1092	-3.16333316805147\\
62	0.1098	-3.29633038138674\\
62	0.1104	-3.42932759472201\\
62	0.111	-3.56232480805728\\
62	0.1116	-3.69532202139254\\
62	0.1122	-3.82831923472781\\
62	0.1128	-3.96131644806309\\
62	0.1134	-4.09431366139836\\
62	0.114	-4.22731087473363\\
62	0.1146	-4.36030808806889\\
62	0.1152	-4.49330530140416\\
62	0.1158	-4.62630251473942\\
62	0.1164	-4.75929972807469\\
62	0.117	-4.89229694140997\\
62	0.1176	-5.02529415474524\\
62	0.1182	-5.15829136808051\\
62	0.1188	-5.29128858141577\\
62	0.1194	-5.42428579475104\\
62	0.12	-5.5572830080863\\
62	0.1206	-5.69028022142157\\
62	0.1212	-5.82327743475686\\
62	0.1218	-5.9562746480921\\
62	0.1224	-6.08927186142739\\
62	0.123	-6.22226907476265\\
62.375	0.093	0.467877048194524\\
62.375	0.0936	0.337967671221804\\
62.375	0.0942	0.208058294249099\\
62.375	0.0948	0.0781489172763941\\
62.375	0.0954	-0.051760459696311\\
62.375	0.096	-0.181669836669016\\
62.375	0.0966	-0.311579213641721\\
62.375	0.0972	-0.441488590614426\\
62.375	0.0978	-0.571397967587146\\
62.375	0.0984	-0.701307344559851\\
62.375	0.099	-0.831216721532556\\
62.375	0.0996	-0.961126098505261\\
62.375	0.1002	-1.09103547547797\\
62.375	0.1008	-1.22094485245067\\
62.375	0.1014	-1.35085422942339\\
62.375	0.102	-1.48076360639608\\
62.375	0.1026	-1.6106729833688\\
62.375	0.1032	-1.74058236034151\\
62.375	0.1038	-1.87049173731421\\
62.375	0.1044	-2.00040111428692\\
62.375	0.105	-2.13031049125962\\
62.375	0.1056	-2.26021986823234\\
62.375	0.1062	-2.39012924520503\\
62.375	0.1068	-2.52003862217775\\
62.375	0.1074	-2.64994799915046\\
62.375	0.108	-2.77985737612316\\
62.375	0.1086	-2.90976675309588\\
62.375	0.1092	-3.03967613006857\\
62.375	0.1098	-3.16958550704129\\
62.375	0.1104	-3.29949488401398\\
62.375	0.111	-3.4294042609867\\
62.375	0.1116	-3.55931363795941\\
62.375	0.1122	-3.68922301493211\\
62.375	0.1128	-3.81913239190483\\
62.375	0.1134	-3.94904176887752\\
62.375	0.114	-4.07895114585024\\
62.375	0.1146	-4.20886052282295\\
62.375	0.1152	-4.33876989979565\\
62.375	0.1158	-4.46867927676836\\
62.375	0.1164	-4.59858865374106\\
62.375	0.117	-4.72849803071377\\
62.375	0.1176	-4.85840740768647\\
62.375	0.1182	-4.98831678465919\\
62.375	0.1188	-5.1182261616319\\
62.375	0.1194	-5.2481355386046\\
62.375	0.12	-5.37804491557731\\
62.375	0.1206	-5.50795429255\\
62.375	0.1212	-5.63786366952273\\
62.375	0.1218	-5.76777304649541\\
62.375	0.1224	-5.89768242346814\\
62.375	0.123	-6.02759180044085\\
62.75	0.093	0.508162504388253\\
62.75	0.0936	0.381340963778101\\
62.75	0.0942	0.254519423167963\\
62.75	0.0948	0.127697882557811\\
62.75	0.0954	0.000876341947673609\\
62.75	0.096	-0.125945198662478\\
62.75	0.0966	-0.252766739272616\\
62.75	0.0972	-0.379588279882768\\
62.75	0.0978	-0.506409820492919\\
62.75	0.0984	-0.633231361103057\\
62.75	0.099	-0.760052901713209\\
62.75	0.0996	-0.886874442323347\\
62.75	0.1002	-1.0136959829335\\
62.75	0.1008	-1.14051752354364\\
62.75	0.1014	-1.26733906415379\\
62.75	0.102	-1.39416060476393\\
62.75	0.1026	-1.52098214537408\\
62.75	0.1032	-1.64780368598423\\
62.75	0.1038	-1.77462522659437\\
62.75	0.1044	-1.90144676720452\\
62.75	0.105	-2.02826830781466\\
62.75	0.1056	-2.15508984842481\\
62.75	0.1062	-2.28191138903495\\
62.75	0.1068	-2.4087329296451\\
62.75	0.1074	-2.53555447025525\\
62.75	0.108	-2.66237601086539\\
62.75	0.1086	-2.78919755147554\\
62.75	0.1092	-2.91601909208568\\
62.75	0.1098	-3.04284063269583\\
62.75	0.1104	-3.16966217330597\\
62.75	0.111	-3.29648371391612\\
62.75	0.1116	-3.42330525452626\\
62.75	0.1122	-3.55012679513641\\
62.75	0.1128	-3.67694833574656\\
62.75	0.1134	-3.8037698763567\\
62.75	0.114	-3.93059141696685\\
62.75	0.1146	-4.05741295757699\\
62.75	0.1152	-4.18423449818714\\
62.75	0.1158	-4.31105603879728\\
62.75	0.1164	-4.43787757940743\\
62.75	0.117	-4.56469912001756\\
62.75	0.1176	-4.69152066062772\\
62.75	0.1182	-4.81834220123787\\
62.75	0.1188	-4.94516374184801\\
62.75	0.1194	-5.07198528245816\\
62.75	0.12	-5.1988068230683\\
62.75	0.1206	-5.32562836367843\\
62.75	0.1212	-5.4524499042886\\
62.75	0.1218	-5.57927144489872\\
62.75	0.1224	-5.70609298550889\\
62.75	0.123	-5.83291452611903\\
63.125	0.093	0.548447960581996\\
63.125	0.0936	0.424714256334411\\
63.125	0.0942	0.300980552086827\\
63.125	0.0948	0.177246847839243\\
63.125	0.0954	0.0535131435916725\\
63.125	0.096	-0.0702205606559261\\
63.125	0.0966	-0.193954264903496\\
63.125	0.0972	-0.317687969151095\\
63.125	0.0978	-0.441421673398679\\
63.125	0.0984	-0.565155377646249\\
63.125	0.099	-0.688889081893848\\
63.125	0.0996	-0.812622786141418\\
63.125	0.1002	-0.936356490389016\\
63.125	0.1008	-1.06009019463659\\
63.125	0.1014	-1.18382389888419\\
63.125	0.102	-1.30755760313176\\
63.125	0.1026	-1.43129130737934\\
63.125	0.1032	-1.55502501162694\\
63.125	0.1038	-1.67875871587451\\
63.125	0.1044	-1.80249242012209\\
63.125	0.105	-1.92622612436968\\
63.125	0.1056	-2.04995982861726\\
63.125	0.1062	-2.17369353286485\\
63.125	0.1068	-2.29742723711243\\
63.125	0.1074	-2.42116094136003\\
63.125	0.108	-2.5448946456076\\
63.125	0.1086	-2.66862834985518\\
63.125	0.1092	-2.79236205410277\\
63.125	0.1098	-2.91609575835035\\
63.125	0.1104	-3.03982946259794\\
63.125	0.111	-3.16356316684552\\
63.125	0.1116	-3.2872968710931\\
63.125	0.1122	-3.41103057534069\\
63.125	0.1128	-3.53476427958827\\
63.125	0.1134	-3.65849798383586\\
63.125	0.114	-3.78223168808344\\
63.125	0.1146	-3.90596539233103\\
63.125	0.1152	-4.02969909657861\\
63.125	0.1158	-4.15343280082618\\
63.125	0.1164	-4.27716650507378\\
63.125	0.117	-4.40090020932135\\
63.125	0.1176	-4.52463391356895\\
63.125	0.1182	-4.64836761781653\\
63.125	0.1188	-4.7721013220641\\
63.125	0.1194	-4.8958350263117\\
63.125	0.12	-5.01956873055927\\
63.125	0.1206	-5.14330243480686\\
63.125	0.1212	-5.26703613905445\\
63.125	0.1218	-5.39076984330201\\
63.125	0.1224	-5.51450354754962\\
63.125	0.123	-5.63823725179719\\
63.5	0.093	0.588733416775739\\
63.5	0.0936	0.468087548890708\\
63.5	0.0942	0.347441681005691\\
63.5	0.0948	0.22679581312066\\
63.5	0.0954	0.106149945235643\\
63.5	0.096	-0.0144959226493881\\
63.5	0.0966	-0.135141790534391\\
63.5	0.0972	-0.255787658419422\\
63.5	0.0978	-0.376433526304453\\
63.5	0.0984	-0.49707939418947\\
63.5	0.099	-0.617725262074501\\
63.5	0.0996	-0.738371129959503\\
63.5	0.1002	-0.859016997844535\\
63.5	0.1008	-0.979662865729551\\
63.5	0.1014	-1.10030873361458\\
63.5	0.102	-1.2209546014996\\
63.5	0.1026	-1.34160046938463\\
63.5	0.1032	-1.46224633726965\\
63.5	0.1038	-1.58289220515466\\
63.5	0.1044	-1.7035380730397\\
63.5	0.105	-1.82418394092471\\
63.5	0.1056	-1.94482980880974\\
63.5	0.1062	-2.06547567669476\\
63.5	0.1068	-2.18612154457978\\
63.5	0.1074	-2.30676741246481\\
63.5	0.108	-2.42741328034982\\
63.5	0.1086	-2.54805914823486\\
63.5	0.1092	-2.66870501611987\\
63.5	0.1098	-2.78935088400489\\
63.5	0.1104	-2.90999675188991\\
63.5	0.111	-3.03064261977494\\
63.5	0.1116	-3.15128848765995\\
63.5	0.1122	-3.27193435554499\\
63.5	0.1128	-3.39258022343002\\
63.5	0.1134	-3.51322609131502\\
63.5	0.114	-3.63387195920005\\
63.5	0.1146	-3.75451782708507\\
63.5	0.1152	-3.8751636949701\\
63.5	0.1158	-3.99580956285511\\
63.5	0.1164	-4.11645543074015\\
63.5	0.117	-4.23710129862515\\
63.5	0.1176	-4.35774716651018\\
63.5	0.1182	-4.47839303439521\\
63.5	0.1188	-4.59903890228023\\
63.5	0.1194	-4.71968477016526\\
63.5	0.12	-4.84033063805026\\
63.5	0.1206	-4.96097650593528\\
63.5	0.1212	-5.08162237382032\\
63.5	0.1218	-5.20226824170533\\
63.5	0.1224	-5.32291410959037\\
63.5	0.123	-5.44355997747539\\
63.875	0.093	0.629018872969482\\
63.875	0.0936	0.511460841447018\\
63.875	0.0942	0.393902809924555\\
63.875	0.0948	0.276344778402091\\
63.875	0.0954	0.158786746879642\\
63.875	0.096	0.0412287153571782\\
63.875	0.0966	-0.0763293161652854\\
63.875	0.0972	-0.193887347687749\\
63.875	0.0978	-0.311445379210213\\
63.875	0.0984	-0.429003410732662\\
63.875	0.099	-0.54656144225514\\
63.875	0.0996	-0.664119473777589\\
63.875	0.1002	-0.781677505300053\\
63.875	0.1008	-0.899235536822502\\
63.875	0.1014	-1.01679356834498\\
63.875	0.102	-1.13435159986743\\
63.875	0.1026	-1.25190963138989\\
63.875	0.1032	-1.36946766291236\\
63.875	0.1038	-1.48702569443481\\
63.875	0.1044	-1.60458372595728\\
63.875	0.105	-1.72214175747973\\
63.875	0.1056	-1.8396997890022\\
63.875	0.1062	-1.95725782052465\\
63.875	0.1068	-2.07481585204712\\
63.875	0.1074	-2.19237388356959\\
63.875	0.108	-2.30993191509204\\
63.875	0.1086	-2.4274899466145\\
63.875	0.1092	-2.54504797813695\\
63.875	0.1098	-2.66260600965943\\
63.875	0.1104	-2.78016404118188\\
63.875	0.111	-2.89772207270434\\
63.875	0.1116	-3.01528010422679\\
63.875	0.1122	-3.13283813574927\\
63.875	0.1128	-3.25039616727173\\
63.875	0.1134	-3.36795419879418\\
63.875	0.114	-3.48551223031664\\
63.875	0.1146	-3.60307026183911\\
63.875	0.1152	-3.72062829336157\\
63.875	0.1158	-3.83818632488402\\
63.875	0.1164	-3.95574435640648\\
63.875	0.117	-4.07330238792893\\
63.875	0.1176	-4.19086041945141\\
63.875	0.1182	-4.30841845097387\\
63.875	0.1188	-4.42597648249632\\
63.875	0.1194	-4.54353451401879\\
63.875	0.12	-4.66109254554125\\
63.875	0.1206	-4.7786505770637\\
63.875	0.1212	-4.89620860858618\\
63.875	0.1218	-5.01376664010861\\
63.875	0.1224	-5.13132467163111\\
63.875	0.123	-5.24888270315355\\
64.25	0.093	0.669304329163211\\
64.25	0.0936	0.554834134003301\\
64.25	0.0942	0.440363938843419\\
64.25	0.0948	0.325893743683508\\
64.25	0.0954	0.211423548523612\\
64.25	0.096	0.0969533533637161\\
64.25	0.0966	-0.01751684179618\\
64.25	0.0972	-0.131987036956076\\
64.25	0.0978	-0.246457232115986\\
64.25	0.0984	-0.360927427275882\\
64.25	0.099	-0.475397622435779\\
64.25	0.0996	-0.589867817595675\\
64.25	0.1002	-0.704338012755585\\
64.25	0.1008	-0.818808207915467\\
64.25	0.1014	-0.933278403075377\\
64.25	0.102	-1.04774859823527\\
64.25	0.1026	-1.16221879339517\\
64.25	0.1032	-1.27668898855508\\
64.25	0.1038	-1.39115918371498\\
64.25	0.1044	-1.50562937887487\\
64.25	0.105	-1.62009957403477\\
64.25	0.1056	-1.73456976919466\\
64.25	0.1062	-1.84903996435456\\
64.25	0.1068	-1.96351015951447\\
64.25	0.1074	-2.07798035467437\\
64.25	0.108	-2.19245054983426\\
64.25	0.1086	-2.30692074499417\\
64.25	0.1092	-2.42139094015405\\
64.25	0.1098	-2.53586113531397\\
64.25	0.1104	-2.65033133047386\\
64.25	0.111	-2.76480152563376\\
64.25	0.1116	-2.87927172079365\\
64.25	0.1122	-2.99374191595356\\
64.25	0.1128	-3.10821211111346\\
64.25	0.1134	-3.22268230627336\\
64.25	0.114	-3.33715250143327\\
64.25	0.1146	-3.45162269659315\\
64.25	0.1152	-3.56609289175306\\
64.25	0.1158	-3.68056308691294\\
64.25	0.1164	-3.79503328207285\\
64.25	0.117	-3.90950347723275\\
64.25	0.1176	-4.02397367239264\\
64.25	0.1182	-4.13844386755255\\
64.25	0.1188	-4.25291406271245\\
64.25	0.1194	-4.36738425787235\\
64.25	0.12	-4.48185445303224\\
64.25	0.1206	-4.59632464819214\\
64.25	0.1212	-4.71079484335205\\
64.25	0.1218	-4.82526503851193\\
64.25	0.1224	-4.93973523367184\\
64.25	0.123	-5.05420542883174\\
64.625	0.093	0.709589785356954\\
64.625	0.0936	0.598207426559597\\
64.625	0.0942	0.486825067762268\\
64.625	0.0948	0.375442708964925\\
64.625	0.0954	0.264060350167597\\
64.625	0.096	0.152677991370254\\
64.625	0.0966	0.0412956325729255\\
64.625	0.0972	-0.0700867262244174\\
64.625	0.0978	-0.18146908502176\\
64.625	0.0984	-0.292851443819089\\
64.625	0.099	-0.404233802616432\\
64.625	0.0996	-0.51561616141376\\
64.625	0.1002	-0.626998520211103\\
64.625	0.1008	-0.738380879008432\\
64.625	0.1014	-0.849763237805774\\
64.625	0.102	-0.961145596603103\\
64.625	0.1026	-1.07252795540045\\
64.625	0.1032	-1.18391031419779\\
64.625	0.1038	-1.29529267299513\\
64.625	0.1044	-1.40667503179247\\
64.625	0.105	-1.5180573905898\\
64.625	0.1056	-1.62943974938715\\
64.625	0.1062	-1.74082210818447\\
64.625	0.1068	-1.85220446698182\\
64.625	0.1074	-1.96358682577916\\
64.625	0.108	-2.07496918457649\\
64.625	0.1086	-2.18635154337383\\
64.625	0.1092	-2.29773390217116\\
64.625	0.1098	-2.4091162609685\\
64.625	0.1104	-2.52049861976583\\
64.625	0.111	-2.63188097856317\\
64.625	0.1116	-2.7432633373605\\
64.625	0.1122	-2.85464569615785\\
64.625	0.1128	-2.96602805495519\\
64.625	0.1134	-3.07741041375252\\
64.625	0.114	-3.18879277254987\\
64.625	0.1146	-3.3001751313472\\
64.625	0.1152	-3.41155749014455\\
64.625	0.1158	-3.52293984894187\\
64.625	0.1164	-3.63432220773922\\
64.625	0.117	-3.74570456653655\\
64.625	0.1176	-3.85708692533389\\
64.625	0.1182	-3.96846928413123\\
64.625	0.1188	-4.07985164292856\\
64.625	0.1194	-4.1912340017259\\
64.625	0.12	-4.30261636052323\\
64.625	0.1206	-4.41399871932056\\
64.625	0.1212	-4.52538107811792\\
64.625	0.1218	-4.63676343691523\\
64.625	0.1224	-4.74814579571259\\
64.625	0.123	-4.85952815450992\\
65	0.093	0.749875241550683\\
65	0.0936	0.641580719115893\\
65	0.0942	0.533286196681132\\
65	0.0948	0.424991674246343\\
65	0.0954	0.316697151811582\\
65	0.096	0.208402629376792\\
65	0.0966	0.100108106942031\\
65	0.0972	-0.00818641549274446\\
65	0.0978	-0.116480937927534\\
65	0.0984	-0.224775460362295\\
65	0.099	-0.333069982797085\\
65	0.0996	-0.441364505231846\\
65	0.1002	-0.549659027666635\\
65	0.1008	-0.657953550101396\\
65	0.1014	-0.766248072536186\\
65	0.102	-0.874542594970947\\
65	0.1026	-0.982837117405737\\
65	0.1032	-1.09113163984051\\
65	0.1038	-1.19942616227529\\
65	0.1044	-1.30772068471006\\
65	0.105	-1.41601520714484\\
65	0.1056	-1.52430972957961\\
65	0.1062	-1.63260425201439\\
65	0.1068	-1.74089877444916\\
65	0.1074	-1.84919329688394\\
65	0.108	-1.95748781931871\\
65	0.1086	-2.06578234175349\\
65	0.1092	-2.17407686418827\\
65	0.1098	-2.28237138662304\\
65	0.1104	-2.39066590905782\\
65	0.111	-2.49896043149259\\
65	0.1116	-2.60725495392737\\
65	0.1122	-2.71554947636214\\
65	0.1128	-2.82384399879693\\
65	0.1134	-2.93213852123169\\
65	0.114	-3.04043304366648\\
65	0.1146	-3.14872756610124\\
65	0.1152	-3.25702208853603\\
65	0.1158	-3.36531661097079\\
65	0.1164	-3.47361113340558\\
65	0.117	-3.58190565584034\\
65	0.1176	-3.69020017827512\\
65	0.1182	-3.79849470070991\\
65	0.1188	-3.90678922314467\\
65	0.1194	-4.01508374557946\\
65	0.12	-4.12337826801422\\
65	0.1206	-4.231672790449\\
65	0.1212	-4.33996731288379\\
65	0.1218	-4.44826183531855\\
65	0.1224	-4.55655635775334\\
65	0.123	-4.66485088018811\\
65.375	0.093	0.790160697744426\\
65.375	0.0936	0.684954011672204\\
65.375	0.0942	0.579747325599996\\
65.375	0.0948	0.474540639527774\\
65.375	0.0954	0.36933395345558\\
65.375	0.096	0.264127267383358\\
65.375	0.0966	0.15892058131115\\
65.375	0.0972	0.0537138952389284\\
65.375	0.0978	-0.0514927908332936\\
65.375	0.0984	-0.156699476905501\\
65.375	0.099	-0.261906162977724\\
65.375	0.0996	-0.367112849049931\\
65.375	0.1002	-0.472319535122139\\
65.375	0.1008	-0.577526221194347\\
65.375	0.1014	-0.682732907266569\\
65.375	0.102	-0.787939593338777\\
65.375	0.1026	-0.893146279410999\\
65.375	0.1032	-0.998352965483221\\
65.375	0.1038	-1.10355965155543\\
65.375	0.1044	-1.20876633762765\\
65.375	0.105	-1.31397302369986\\
65.375	0.1056	-1.41917970977207\\
65.375	0.1062	-1.52438639584427\\
65.375	0.1068	-1.6295930819165\\
65.375	0.1074	-1.73479976798872\\
65.375	0.108	-1.84000645406093\\
65.375	0.1086	-1.94521314013315\\
65.375	0.1092	-2.05041982620536\\
65.375	0.1098	-2.15562651227758\\
65.375	0.1104	-2.26083319834977\\
65.375	0.111	-2.36603988442199\\
65.375	0.1116	-2.4712465704942\\
65.375	0.1122	-2.57645325656642\\
65.375	0.1128	-2.68165994263865\\
65.375	0.1134	-2.78686662871085\\
65.375	0.114	-2.89207331478308\\
65.375	0.1146	-2.99728000085528\\
65.375	0.1152	-3.10248668692751\\
65.375	0.1158	-3.2076933729997\\
65.375	0.1164	-3.31290005907192\\
65.375	0.117	-3.41810674514413\\
65.375	0.1176	-3.52331343121635\\
65.375	0.1182	-3.62852011728857\\
65.375	0.1188	-3.73372680336078\\
65.375	0.1194	-3.838933489433\\
65.375	0.12	-3.94414017550521\\
65.375	0.1206	-4.04934686157742\\
65.375	0.1212	-4.15455354764964\\
65.375	0.1218	-4.25976023372183\\
65.375	0.1224	-4.36496691979407\\
65.375	0.123	-4.47017360586628\\
65.75	0.093	0.830446153938155\\
65.75	0.0936	0.7283273042285\\
65.75	0.0942	0.62620845451886\\
65.75	0.0948	0.524089604809205\\
65.75	0.0954	0.421970755099551\\
65.75	0.096	0.319851905389896\\
65.75	0.0966	0.217733055680256\\
65.75	0.0972	0.115614205970587\\
65.75	0.0978	0.0134953562609326\\
65.75	0.0984	-0.0886234934487078\\
65.75	0.099	-0.190742343158377\\
65.75	0.0996	-0.292861192868017\\
65.75	0.1002	-0.394980042577671\\
65.75	0.1008	-0.497098892287312\\
65.75	0.1014	-0.599217741996981\\
65.75	0.102	-0.701336591706621\\
65.75	0.1026	-0.803455441416276\\
65.75	0.1032	-0.905574291125944\\
65.75	0.1038	-1.00769314083558\\
65.75	0.1044	-1.10981199054524\\
65.75	0.105	-1.21193084025488\\
65.75	0.1056	-1.31404968996455\\
65.75	0.1062	-1.41616853967419\\
65.75	0.1068	-1.51828738938384\\
65.75	0.1074	-1.62040623909351\\
65.75	0.108	-1.72252508880315\\
65.75	0.1086	-1.82464393851281\\
65.75	0.1092	-1.92676278822246\\
65.75	0.1098	-2.02888163793212\\
65.75	0.1104	-2.13100048764176\\
65.75	0.111	-2.23311933735141\\
65.75	0.1116	-2.33523818706107\\
65.75	0.1122	-2.43735703677072\\
65.75	0.1128	-2.53947588648037\\
65.75	0.1134	-2.64159473619003\\
65.75	0.114	-2.74371358589968\\
65.75	0.1146	-2.84583243560932\\
65.75	0.1152	-2.94795128531898\\
65.75	0.1158	-3.05007013502863\\
65.75	0.1164	-3.15218898473829\\
65.75	0.117	-3.25430783444793\\
65.75	0.1176	-3.3564266841576\\
65.75	0.1182	-3.45854553386725\\
65.75	0.1188	-3.56066438357689\\
65.75	0.1194	-3.66278323328656\\
65.75	0.12	-3.7649020829962\\
65.75	0.1206	-3.86702093270584\\
65.75	0.1212	-3.96913978241551\\
65.75	0.1218	-4.07125863212515\\
65.75	0.1224	-4.17337748183482\\
65.75	0.123	-4.27549633154446\\
66.125	0.093	0.870731610131912\\
66.125	0.0936	0.771700596784811\\
66.125	0.0942	0.672669583437724\\
66.125	0.0948	0.573638570090637\\
66.125	0.0954	0.47460755674355\\
66.125	0.096	0.375576543396448\\
66.125	0.0966	0.276545530049361\\
66.125	0.0972	0.177514516702274\\
66.125	0.0978	0.078483503355173\\
66.125	0.0984	-0.0205475099919141\\
66.125	0.099	-0.119578523339001\\
66.125	0.0996	-0.218609536686088\\
66.125	0.1002	-0.31764055003319\\
66.125	0.1008	-0.416671563380277\\
66.125	0.1014	-0.515702576727364\\
66.125	0.102	-0.614733590074451\\
66.125	0.1026	-0.713764603421552\\
66.125	0.1032	-0.812795616768639\\
66.125	0.1038	-0.911826630115726\\
66.125	0.1044	-1.01085764346283\\
66.125	0.105	-1.1098886568099\\
66.125	0.1056	-1.208919670157\\
66.125	0.1062	-1.30795068350409\\
66.125	0.1068	-1.40698169685119\\
66.125	0.1074	-1.50601271019828\\
66.125	0.108	-1.60504372354536\\
66.125	0.1086	-1.70407473689247\\
66.125	0.1092	-1.80310575023954\\
66.125	0.1098	-1.90213676358664\\
66.125	0.1104	-2.00116777693373\\
66.125	0.111	-2.10019879028083\\
66.125	0.1116	-2.1992298036279\\
66.125	0.1122	-2.298260816975\\
66.125	0.1128	-2.3972918303221\\
66.125	0.1134	-2.49632284366918\\
66.125	0.114	-2.59535385701628\\
66.125	0.1146	-2.69438487036336\\
66.125	0.1152	-2.79341588371045\\
66.125	0.1158	-2.89244689705754\\
66.125	0.1164	-2.99147791040464\\
66.125	0.117	-3.09050892375173\\
66.125	0.1176	-3.18953993709881\\
66.125	0.1182	-3.28857095044592\\
66.125	0.1188	-3.387601963793\\
66.125	0.1194	-3.48663297714009\\
66.125	0.12	-3.58566399048718\\
66.125	0.1206	-3.68469500383426\\
66.125	0.1212	-3.78372601718138\\
66.125	0.1218	-3.88275703052844\\
66.125	0.1224	-3.98178804387555\\
66.125	0.123	-4.08081905722264\\
66.5	0.093	0.911017066325641\\
66.5	0.0936	0.815073889341107\\
66.5	0.0942	0.719130712356588\\
66.5	0.0948	0.623187535372054\\
66.5	0.0954	0.52724435838752\\
66.5	0.096	0.431301181402986\\
66.5	0.0966	0.335358004418467\\
66.5	0.0972	0.239414827433933\\
66.5	0.0978	0.143471650449399\\
66.5	0.0984	0.0475284734648795\\
66.5	0.099	-0.0484147035196543\\
66.5	0.0996	-0.144357880504174\\
66.5	0.1002	-0.240301057488708\\
66.5	0.1008	-0.336244234473241\\
66.5	0.1014	-0.432187411457775\\
66.5	0.102	-0.528130588442295\\
66.5	0.1026	-0.624073765426829\\
66.5	0.1032	-0.720016942411362\\
66.5	0.1038	-0.815960119395882\\
66.5	0.1044	-0.911903296380416\\
66.5	0.105	-1.00784647336494\\
66.5	0.1056	-1.10378965034948\\
66.5	0.1062	-1.199732827334\\
66.5	0.1068	-1.29567600431854\\
66.5	0.1074	-1.39161918130307\\
66.5	0.108	-1.48756235828759\\
66.5	0.1086	-1.58350553527212\\
66.5	0.1092	-1.67944871225664\\
66.5	0.1098	-1.77539188924118\\
66.5	0.1104	-1.8713350662257\\
66.5	0.111	-1.96727824321025\\
66.5	0.1116	-2.06322142019476\\
66.5	0.1122	-2.1591645971793\\
66.5	0.1128	-2.25510777416383\\
66.5	0.1134	-2.35105095114835\\
66.5	0.114	-2.44699412813289\\
66.5	0.1146	-2.54293730511741\\
66.5	0.1152	-2.63888048210194\\
66.5	0.1158	-2.73482365908647\\
66.5	0.1164	-2.83076683607101\\
66.5	0.117	-2.92671001305553\\
66.5	0.1176	-3.02265319004006\\
66.5	0.1182	-3.11859636702459\\
66.5	0.1188	-3.21453954400911\\
66.5	0.1194	-3.31048272099365\\
66.5	0.12	-3.40642589797817\\
66.5	0.1206	-3.50236907496269\\
66.5	0.1212	-3.59831225194725\\
66.5	0.1218	-3.69425542893175\\
66.5	0.1224	-3.7901986059163\\
66.5	0.123	-3.88614178290082\\
66.875	0.093	0.951302522519384\\
66.875	0.0936	0.858447181897418\\
66.875	0.0942	0.765591841275452\\
66.875	0.0948	0.672736500653485\\
66.875	0.0954	0.579881160031519\\
66.875	0.096	0.487025819409553\\
66.875	0.0966	0.394170478787586\\
66.875	0.0972	0.30131513816562\\
66.875	0.0978	0.20845979754364\\
66.875	0.0984	0.115604456921673\\
66.875	0.099	0.0227491162997069\\
66.875	0.0996	-0.0701062243222594\\
66.875	0.1002	-0.162961564944226\\
66.875	0.1008	-0.255816905566192\\
66.875	0.1014	-0.348672246188158\\
66.875	0.102	-0.441527586810125\\
66.875	0.1026	-0.534382927432091\\
66.875	0.1032	-0.627238268054072\\
66.875	0.1038	-0.720093608676024\\
66.875	0.1044	-0.812948949298004\\
66.875	0.105	-0.905804289919956\\
66.875	0.1056	-0.998659630541937\\
66.875	0.1062	-1.09151497116389\\
66.875	0.1068	-1.18437031178587\\
66.875	0.1074	-1.27722565240784\\
66.875	0.108	-1.3700809930298\\
66.875	0.1086	-1.46293633365178\\
66.875	0.1092	-1.55579167427373\\
66.875	0.1098	-1.64864701489572\\
66.875	0.1104	-1.74150235551767\\
66.875	0.111	-1.83435769613965\\
66.875	0.1116	-1.9272130367616\\
66.875	0.1122	-2.02006837738358\\
66.875	0.1128	-2.11292371800555\\
66.875	0.1134	-2.20577905862751\\
66.875	0.114	-2.29863439924948\\
66.875	0.1146	-2.39148973987145\\
66.875	0.1152	-2.48434508049341\\
66.875	0.1158	-2.57720042111538\\
66.875	0.1164	-2.67005576173734\\
66.875	0.117	-2.76291110235931\\
66.875	0.1176	-2.85576644298128\\
66.875	0.1182	-2.94862178360326\\
66.875	0.1188	-3.04147712422522\\
66.875	0.1194	-3.13433246484719\\
66.875	0.12	-3.22718780546916\\
66.875	0.1206	-3.32004314609111\\
66.875	0.1212	-3.4128984867131\\
66.875	0.1218	-3.50575382733504\\
66.875	0.1224	-3.59860916795704\\
66.875	0.123	-3.69146450857899\\
67.25	0.093	0.991587978713113\\
67.25	0.0936	0.9018204744537\\
67.25	0.0942	0.812052970194316\\
67.25	0.0948	0.722285465934903\\
67.25	0.0954	0.632517961675504\\
67.25	0.096	0.542750457416091\\
67.25	0.0966	0.452982953156692\\
67.25	0.0972	0.363215448897279\\
67.25	0.0978	0.273447944637866\\
67.25	0.0984	0.183680440378467\\
67.25	0.099	0.0939129361190538\\
67.25	0.0996	0.00414543185965499\\
67.25	0.1002	-0.0856220723997581\\
67.25	0.1008	-0.175389576659157\\
67.25	0.1014	-0.26515708091857\\
67.25	0.102	-0.354924585177969\\
67.25	0.1026	-0.444692089437382\\
67.25	0.1032	-0.534459593696795\\
67.25	0.1038	-0.624227097956194\\
67.25	0.1044	-0.713994602215593\\
67.25	0.105	-0.803762106474991\\
67.25	0.1056	-0.893529610734404\\
67.25	0.1062	-0.983297114993803\\
67.25	0.1068	-1.07306461925322\\
67.25	0.1074	-1.16283212351263\\
67.25	0.108	-1.25259962777203\\
67.25	0.1086	-1.34236713203144\\
67.25	0.1092	-1.43213463629084\\
67.25	0.1098	-1.52190214055025\\
67.25	0.1104	-1.61166964480965\\
67.25	0.111	-1.70143714906906\\
67.25	0.1116	-1.79120465332846\\
67.25	0.1122	-1.88097215758788\\
67.25	0.1128	-1.97073966184729\\
67.25	0.1134	-2.06050716610669\\
67.25	0.114	-2.15027467036609\\
67.25	0.1146	-2.24004217462549\\
67.25	0.1152	-2.3298096788849\\
67.25	0.1158	-2.4195771831443\\
67.25	0.1164	-2.50934468740371\\
67.25	0.117	-2.59911219166311\\
67.25	0.1176	-2.68887969592252\\
67.25	0.1182	-2.77864720018194\\
67.25	0.1188	-2.86841470444134\\
67.25	0.1194	-2.95818220870075\\
67.25	0.12	-3.04794971296015\\
67.25	0.1206	-3.13771721721955\\
67.25	0.1212	-3.22748472147897\\
67.25	0.1218	-3.31725222573836\\
67.25	0.1224	-3.40701972999778\\
67.25	0.123	-3.49678723425718\\
67.625	0.093	1.03187343490684\\
67.625	0.0936	0.945193767009997\\
67.625	0.0942	0.858514099113151\\
67.625	0.0948	0.771834431216305\\
67.625	0.0954	0.685154763319474\\
67.625	0.096	0.598475095422614\\
67.625	0.0966	0.511795427525783\\
67.625	0.0972	0.425115759628923\\
67.625	0.0978	0.338436091732078\\
67.625	0.0984	0.251756423835246\\
67.625	0.099	0.165076755938387\\
67.625	0.0996	0.0783970880415552\\
67.625	0.1002	-0.00828257985529035\\
67.625	0.1008	-0.0949622477521359\\
67.625	0.1014	-0.181641915648981\\
67.625	0.102	-0.268321583545813\\
67.625	0.1026	-0.355001251442673\\
67.625	0.1032	-0.441680919339518\\
67.625	0.1038	-0.528360587236349\\
67.625	0.1044	-0.615040255133209\\
67.625	0.105	-0.701719923030041\\
67.625	0.1056	-0.7883995909269\\
67.625	0.1062	-0.875079258823732\\
67.625	0.1068	-0.961758926720577\\
67.625	0.1074	-1.04843859461744\\
67.625	0.108	-1.13511826251427\\
67.625	0.1086	-1.22179793041111\\
67.625	0.1092	-1.30847759830796\\
67.625	0.1098	-1.39515726620481\\
67.625	0.1104	-1.48183693410164\\
67.625	0.111	-1.5685166019985\\
67.625	0.1116	-1.65519626989533\\
67.625	0.1122	-1.74187593779217\\
67.625	0.1128	-1.82855560568903\\
67.625	0.1134	-1.91523527358586\\
67.625	0.114	-2.00191494148272\\
67.625	0.1146	-2.08859460937956\\
67.625	0.1152	-2.1752742772764\\
67.625	0.1158	-2.26195394517325\\
67.625	0.1164	-2.34863361307009\\
67.625	0.117	-2.43531328096692\\
67.625	0.1176	-2.52199294886378\\
67.625	0.1182	-2.60867261676063\\
67.625	0.1188	-2.69535228465746\\
67.625	0.1194	-2.78203195255432\\
67.625	0.12	-2.86871162045115\\
67.625	0.1206	-2.95539128834798\\
67.625	0.1212	-3.04207095624486\\
67.625	0.1218	-3.12875062414167\\
67.625	0.1224	-3.21543029203853\\
67.625	0.123	-3.30210995993538\\
68	0.093	1.07215889110059\\
68	0.0936	0.988567059566293\\
68	0.0942	0.904975228032029\\
68	0.0948	0.821383396497737\\
68	0.0954	0.737791564963459\\
68	0.096	0.654199733429166\\
68	0.0966	0.570607901894903\\
68	0.0972	0.48701607036061\\
68	0.0978	0.403424238826318\\
68	0.0984	0.31983240729204\\
68	0.099	0.236240575757762\\
68	0.0996	0.152648744223484\\
68	0.1002	0.0690569126891916\\
68	0.1008	-0.0145349188450865\\
68	0.1014	-0.0981267503793788\\
68	0.102	-0.181718581913643\\
68	0.1026	-0.265310413447935\\
68	0.1032	-0.348902244982227\\
68	0.1038	-0.432494076516505\\
68	0.1044	-0.516085908050783\\
68	0.105	-0.599677739585061\\
68	0.1056	-0.683269571119354\\
68	0.1062	-0.766861402653632\\
68	0.1068	-0.85045323418791\\
68	0.1074	-0.934045065722202\\
68	0.108	-1.01763689725648\\
68	0.1086	-1.10122872879077\\
68	0.1092	-1.18482056032504\\
68	0.1098	-1.26841239185933\\
68	0.1104	-1.35200422339361\\
68	0.111	-1.4355960549279\\
68	0.1116	-1.51918788646218\\
68	0.1122	-1.60277971799646\\
68	0.1128	-1.68637154953075\\
68	0.1134	-1.76996338106503\\
68	0.114	-1.85355521259932\\
68	0.1146	-1.93714704413358\\
68	0.1152	-2.02073887566787\\
68	0.1158	-2.10433070720215\\
68	0.1164	-2.18792253873644\\
68	0.117	-2.27151437027071\\
68	0.1176	-2.355106201805\\
68	0.1182	-2.43869803333929\\
68	0.1188	-2.52228986487357\\
68	0.1194	-2.60588169640785\\
68	0.12	-2.68947352794213\\
68	0.1206	-2.7730653594764\\
68	0.1212	-2.85665719101071\\
68	0.1218	-2.94024902254498\\
68	0.1224	-3.02384085407927\\
68	0.123	-3.10743268561355\\
68.375	0.093	1.11244434729431\\
68.375	0.0936	1.03194035212259\\
68.375	0.0942	0.951436356950879\\
68.375	0.0948	0.870932361779154\\
68.375	0.0954	0.790428366607443\\
68.375	0.096	0.709924371435719\\
68.375	0.0966	0.629420376263994\\
68.375	0.0972	0.548916381092269\\
68.375	0.0978	0.468412385920544\\
68.375	0.0984	0.387908390748834\\
68.375	0.099	0.307404395577109\\
68.375	0.0996	0.226900400405398\\
68.375	0.1002	0.146396405233659\\
68.375	0.1008	0.0658924100619487\\
68.375	0.1014	-0.0146115851097761\\
68.375	0.102	-0.0951155802814867\\
68.375	0.1026	-0.175619575453211\\
68.375	0.1032	-0.256123570624951\\
68.375	0.1038	-0.336627565796661\\
68.375	0.1044	-0.417131560968386\\
68.375	0.105	-0.497635556140096\\
68.375	0.1056	-0.578139551311821\\
68.375	0.1062	-0.658643546483532\\
68.375	0.1068	-0.739147541655271\\
68.375	0.1074	-0.819651536826996\\
68.375	0.108	-0.900155531998706\\
68.375	0.1086	-0.980659527170431\\
68.375	0.1092	-1.06116352234214\\
68.375	0.1098	-1.14166751751387\\
68.375	0.1104	-1.22217151268559\\
68.375	0.111	-1.30267550785732\\
68.375	0.1116	-1.38317950302903\\
68.375	0.1122	-1.46368349820075\\
68.375	0.1128	-1.54418749337248\\
68.375	0.1134	-1.6246914885442\\
68.375	0.114	-1.70519548371593\\
68.375	0.1146	-1.78569947888764\\
68.375	0.1152	-1.86620347405936\\
68.375	0.1158	-1.94670746923107\\
68.375	0.1164	-2.0272114644028\\
68.375	0.117	-2.10771545957452\\
68.375	0.1176	-2.18821945474625\\
68.375	0.1182	-2.26872344991797\\
68.375	0.1188	-2.34922744508968\\
68.375	0.1194	-2.42973144026141\\
68.375	0.12	-2.51023543543312\\
68.375	0.1206	-2.59073943060484\\
68.375	0.1212	-2.67124342577658\\
68.375	0.1218	-2.75174742094828\\
68.375	0.1224	-2.83225141612002\\
68.375	0.123	-2.91275541129173\\
68.75	0.093	1.15272980348807\\
68.75	0.0936	1.0753136446789\\
68.75	0.0942	0.997897485869757\\
68.75	0.0948	0.920481327060585\\
68.75	0.0954	0.843065168251428\\
68.75	0.096	0.765649009442271\\
68.75	0.0966	0.688232850633113\\
68.75	0.0972	0.610816691823956\\
68.75	0.0978	0.533400533014785\\
68.75	0.0984	0.455984374205627\\
68.75	0.099	0.37856821539647\\
68.75	0.0996	0.301152056587313\\
68.75	0.1002	0.223735897778155\\
68.75	0.1008	0.146319738968998\\
68.75	0.1014	0.0689035801598266\\
68.75	0.102	-0.00851257864931654\\
68.75	0.1026	-0.0859287374584881\\
68.75	0.1032	-0.163344896267645\\
68.75	0.1038	-0.240761055076803\\
68.75	0.1044	-0.31817721388596\\
68.75	0.105	-0.395593372695117\\
68.75	0.1056	-0.473009531504289\\
68.75	0.1062	-0.550425690313432\\
68.75	0.1068	-0.627841849122603\\
68.75	0.1074	-0.705258007931761\\
68.75	0.108	-0.782674166740918\\
68.75	0.1086	-0.86009032555009\\
68.75	0.1092	-0.937506484359233\\
68.75	0.1098	-1.0149226431684\\
68.75	0.1104	-1.09233880197755\\
68.75	0.111	-1.16975496078672\\
68.75	0.1116	-1.24717111959588\\
68.75	0.1122	-1.32458727840503\\
68.75	0.1128	-1.4020034372142\\
68.75	0.1134	-1.47941959602335\\
68.75	0.114	-1.55683575483252\\
68.75	0.1146	-1.63425191364168\\
68.75	0.1152	-1.71166807245083\\
68.75	0.1158	-1.78908423125999\\
68.75	0.1164	-1.86650039006915\\
68.75	0.117	-1.94391654887831\\
68.75	0.1176	-2.02133270768746\\
68.75	0.1182	-2.09874886649663\\
68.75	0.1188	-2.17616502530579\\
68.75	0.1194	-2.25358118411495\\
68.75	0.12	-2.33099734292411\\
68.75	0.1206	-2.40841350173325\\
68.75	0.1212	-2.48582966054244\\
68.75	0.1218	-2.56324581935156\\
68.75	0.1224	-2.64066197816075\\
68.75	0.123	-2.71807813696991\\
69.125	0.093	1.1930152596818\\
69.125	0.0936	1.1186869372352\\
69.125	0.0942	1.04435861478861\\
69.125	0.0948	0.970030292342003\\
69.125	0.0954	0.895701969895413\\
69.125	0.096	0.821373647448809\\
69.125	0.0966	0.747045325002219\\
69.125	0.0972	0.672717002555615\\
69.125	0.0978	0.598388680109011\\
69.125	0.0984	0.524060357662421\\
69.125	0.099	0.449732035215817\\
69.125	0.0996	0.375403712769227\\
69.125	0.1002	0.301075390322623\\
69.125	0.1008	0.226747067876033\\
69.125	0.1014	0.152418745429429\\
69.125	0.102	0.0780904229828394\\
69.125	0.1026	0.00376210053623538\\
69.125	0.1032	-0.0705662219103687\\
69.125	0.1038	-0.144894544356958\\
69.125	0.1044	-0.219222866803563\\
69.125	0.105	-0.293551189250152\\
69.125	0.1056	-0.367879511696756\\
69.125	0.1062	-0.442207834143346\\
69.125	0.1068	-0.51653615658995\\
69.125	0.1074	-0.590864479036554\\
69.125	0.108	-0.665192801483144\\
69.125	0.1086	-0.739521123929748\\
69.125	0.1092	-0.813849446376338\\
69.125	0.1098	-0.888177768822942\\
69.125	0.1104	-0.962506091269532\\
69.125	0.111	-1.03683441371614\\
69.125	0.1116	-1.11116273616273\\
69.125	0.1122	-1.18549105860933\\
69.125	0.1128	-1.25981938105593\\
69.125	0.1134	-1.33414770350252\\
69.125	0.114	-1.40847602594913\\
69.125	0.1146	-1.48280434839572\\
69.125	0.1152	-1.55713267084232\\
69.125	0.1158	-1.63146099328891\\
69.125	0.1164	-1.70578931573552\\
69.125	0.117	-1.78011763818211\\
69.125	0.1176	-1.85444596062871\\
69.125	0.1182	-1.92877428307531\\
69.125	0.1188	-2.0031026055219\\
69.125	0.1194	-2.07743092796851\\
69.125	0.12	-2.1517592504151\\
69.125	0.1206	-2.22608757286169\\
69.125	0.1212	-2.3004158953083\\
69.125	0.1218	-2.37474421775488\\
69.125	0.1224	-2.4490725402015\\
69.125	0.123	-2.52340086264809\\
69.5	0.093	1.23330071587554\\
69.5	0.0936	1.16206022979151\\
69.5	0.0942	1.09081974370747\\
69.5	0.0948	1.01957925762343\\
69.5	0.0954	0.948338771539412\\
69.5	0.096	0.877098285455361\\
69.5	0.0966	0.805857799371339\\
69.5	0.0972	0.734617313287288\\
69.5	0.0978	0.663376827203251\\
69.5	0.0984	0.592136341119229\\
69.5	0.099	0.520895855035178\\
69.5	0.0996	0.449655368951156\\
69.5	0.1002	0.378414882867105\\
69.5	0.1008	0.307174396783083\\
69.5	0.1014	0.235933910699032\\
69.5	0.102	0.16469342461501\\
69.5	0.1026	0.093452938530973\\
69.5	0.1032	0.0222124524469223\\
69.5	0.1038	-0.0490280336371001\\
69.5	0.1044	-0.120268519721151\\
69.5	0.105	-0.191509005805173\\
69.5	0.1056	-0.26274949188921\\
69.5	0.1062	-0.333989977973246\\
69.5	0.1068	-0.405230464057283\\
69.5	0.1074	-0.476470950141334\\
69.5	0.108	-0.547711436225356\\
69.5	0.1086	-0.618951922309392\\
69.5	0.1092	-0.690192408393429\\
69.5	0.1098	-0.761432894477466\\
69.5	0.1104	-0.832673380561502\\
69.5	0.111	-0.903913866645539\\
69.5	0.1116	-0.975154352729575\\
69.5	0.1122	-1.04639483881361\\
69.5	0.1128	-1.11763532489765\\
69.5	0.1134	-1.18887581098168\\
69.5	0.114	-1.26011629706572\\
69.5	0.1146	-1.33135678314976\\
69.5	0.1152	-1.40259726923379\\
69.5	0.1158	-1.47383775531782\\
69.5	0.1164	-1.54507824140187\\
69.5	0.117	-1.61631872748589\\
69.5	0.1176	-1.68755921356994\\
69.5	0.1182	-1.75879969965398\\
69.5	0.1188	-1.830040185738\\
69.5	0.1194	-1.90128067182205\\
69.5	0.12	-1.97252115790607\\
69.5	0.1206	-2.04376164399011\\
69.5	0.1212	-2.11500213007416\\
69.5	0.1218	-2.18624261615817\\
69.5	0.1224	-2.25748310224223\\
69.5	0.123	-2.32872358832626\\
69.875	0.093	1.27358617206929\\
69.875	0.0936	1.2054335223478\\
69.875	0.0942	1.13728087262633\\
69.875	0.0948	1.06912822290485\\
69.875	0.0954	1.00097557318338\\
69.875	0.096	0.932822923461899\\
69.875	0.0966	0.864670273740444\\
69.875	0.0972	0.796517624018961\\
69.875	0.0978	0.728364974297477\\
69.875	0.0984	0.660212324576008\\
69.875	0.099	0.592059674854525\\
69.875	0.0996	0.52390702513307\\
69.875	0.1002	0.455754375411587\\
69.875	0.1008	0.387601725690118\\
69.875	0.1014	0.319449075968635\\
69.875	0.102	0.251296426247166\\
69.875	0.1026	0.183143776525682\\
69.875	0.1032	0.114991126804213\\
69.875	0.1038	0.0468384770827441\\
69.875	0.1044	-0.0213141726387391\\
69.875	0.105	-0.0894668223602082\\
69.875	0.1056	-0.157619472081691\\
69.875	0.1062	-0.225772121803161\\
69.875	0.1068	-0.29392477152463\\
69.875	0.1074	-0.362077421246113\\
69.875	0.108	-0.430230070967582\\
69.875	0.1086	-0.498382720689065\\
69.875	0.1092	-0.566535370410534\\
69.875	0.1098	-0.634688020132003\\
69.875	0.1104	-0.702840669853472\\
69.875	0.111	-0.770993319574956\\
69.875	0.1116	-0.839145969296425\\
69.875	0.1122	-0.907298619017908\\
69.875	0.1128	-0.975451268739391\\
69.875	0.1134	-1.04360391846085\\
69.875	0.114	-1.11175656818233\\
69.875	0.1146	-1.1799092179038\\
69.875	0.1152	-1.24806186762528\\
69.875	0.1158	-1.31621451734675\\
69.875	0.1164	-1.38436716706823\\
69.875	0.117	-1.45251981678969\\
69.875	0.1176	-1.52067246651117\\
69.875	0.1182	-1.58882511623266\\
69.875	0.1188	-1.65697776595412\\
69.875	0.1194	-1.72513041567561\\
69.875	0.12	-1.79328306539706\\
69.875	0.1206	-1.86143571511853\\
69.875	0.1212	-1.92958836484003\\
69.875	0.1218	-1.99774101456148\\
69.875	0.1224	-2.06589366428298\\
69.875	0.123	-2.13404631400445\\
70.25	0.093	1.31387162826303\\
70.25	0.0936	1.24880681490411\\
70.25	0.0942	1.1837420015452\\
70.25	0.0948	1.11867718818628\\
70.25	0.0954	1.05361237482738\\
70.25	0.096	0.988547561468465\\
70.25	0.0966	0.923482748109549\\
70.25	0.0972	0.858417934750634\\
70.25	0.0978	0.793353121391718\\
70.25	0.0984	0.728288308032816\\
70.25	0.099	0.663223494673886\\
70.25	0.0996	0.598158681314985\\
70.25	0.1002	0.533093867956069\\
70.25	0.1008	0.468029054597167\\
70.25	0.1014	0.402964241238251\\
70.25	0.102	0.337899427879336\\
70.25	0.1026	0.27283461452042\\
70.25	0.1032	0.207769801161504\\
70.25	0.1038	0.142704987802603\\
70.25	0.1044	0.0776401744436725\\
70.25	0.105	0.012575361084771\\
70.25	0.1056	-0.0524894522741448\\
70.25	0.1062	-0.117554265633046\\
70.25	0.1068	-0.182619078991976\\
70.25	0.1074	-0.247683892350892\\
70.25	0.108	-0.312748705709794\\
70.25	0.1086	-0.37781351906871\\
70.25	0.1092	-0.442878332427611\\
70.25	0.1098	-0.507943145786541\\
70.25	0.1104	-0.573007959145443\\
70.25	0.111	-0.638072772504358\\
70.25	0.1116	-0.70313758586326\\
70.25	0.1122	-0.76820239922219\\
70.25	0.1128	-0.833267212581106\\
70.25	0.1134	-0.898332025940007\\
70.25	0.114	-0.963396839298923\\
70.25	0.1146	-1.02846165265784\\
70.25	0.1152	-1.09352646601675\\
70.25	0.1158	-1.15859127937566\\
70.25	0.1164	-1.22365609273457\\
70.25	0.117	-1.28872090609349\\
70.25	0.1176	-1.3537857194524\\
70.25	0.1182	-1.41885053281132\\
70.25	0.1188	-1.48391534617022\\
70.25	0.1194	-1.54898015952914\\
70.25	0.12	-1.61404497288805\\
70.25	0.1206	-1.67910978624695\\
70.25	0.1212	-1.74417459960588\\
70.25	0.1218	-1.80923941296477\\
70.25	0.1224	-1.87430422632372\\
70.25	0.123	-1.93936903968262\\
70.625	0.093	1.35415708445676\\
70.625	0.0936	1.2921801074604\\
70.625	0.0942	1.23020313046406\\
70.625	0.0948	1.1682261534677\\
70.625	0.0954	1.10624917647135\\
70.625	0.096	1.044272199475\\
70.625	0.0966	0.982295222478655\\
70.625	0.0972	0.920318245482306\\
70.625	0.0978	0.858341268485944\\
70.625	0.0984	0.796364291489596\\
70.625	0.099	0.734387314493247\\
70.625	0.0996	0.672410337496899\\
70.625	0.1002	0.610433360500537\\
70.625	0.1008	0.548456383504202\\
70.625	0.1014	0.48647940650784\\
70.625	0.102	0.424502429511492\\
70.625	0.1026	0.362525452515143\\
70.625	0.1032	0.300548475518781\\
70.625	0.1038	0.238571498522433\\
70.625	0.1044	0.176594521526084\\
70.625	0.105	0.114617544529736\\
70.625	0.1056	0.0526405675333876\\
70.625	0.1062	-0.00933640946296066\\
70.625	0.1068	-0.0713133864593232\\
70.625	0.1074	-0.133290363455671\\
70.625	0.108	-0.19526734045202\\
70.625	0.1086	-0.257244317448382\\
70.625	0.1092	-0.319221294444716\\
70.625	0.1098	-0.381198271441079\\
70.625	0.1104	-0.443175248437427\\
70.625	0.111	-0.505152225433775\\
70.625	0.1116	-0.567129202430124\\
70.625	0.1122	-0.629106179426486\\
70.625	0.1128	-0.691083156422835\\
70.625	0.1134	-0.753060133419183\\
70.625	0.114	-0.815037110415545\\
70.625	0.1146	-0.877014087411879\\
70.625	0.1152	-0.938991064408242\\
70.625	0.1158	-1.00096804140458\\
70.625	0.1164	-1.06294501840094\\
70.625	0.117	-1.12492199539729\\
70.625	0.1176	-1.18689897239364\\
70.625	0.1182	-1.24887594939\\
70.625	0.1188	-1.31085292638635\\
70.625	0.1194	-1.37282990338269\\
70.625	0.12	-1.43480688037904\\
70.625	0.1206	-1.49678385737539\\
70.625	0.1212	-1.55876083437175\\
70.625	0.1218	-1.62073781136809\\
70.625	0.1224	-1.68271478836446\\
70.625	0.123	-1.7446917653608\\
71	0.093	1.3944425406505\\
71	0.0936	1.33555340001671\\
71	0.0942	1.27666425938291\\
71	0.0948	1.21777511874912\\
71	0.0954	1.15888597811534\\
71	0.096	1.09999683748154\\
71	0.0966	1.04110769684776\\
71	0.0972	0.982218556213965\\
71	0.0978	0.92332941558017\\
71	0.0984	0.864440274946389\\
71	0.099	0.805551134312594\\
71	0.0996	0.746661993678813\\
71	0.1002	0.687772853045018\\
71	0.1008	0.628883712411238\\
71	0.1014	0.569994571777443\\
71	0.102	0.511105431143662\\
71	0.1026	0.452216290509867\\
71	0.1032	0.393327149876072\\
71	0.1038	0.334438009242277\\
71	0.1044	0.275548868608482\\
71	0.105	0.216659727974701\\
71	0.1056	0.157770587340906\\
71	0.1062	0.0988814467071251\\
71	0.1068	0.03999230607333\\
71	0.1074	-0.018896834560465\\
71	0.108	-0.0777859751942458\\
71	0.1086	-0.136675115828041\\
71	0.1092	-0.195564256461822\\
71	0.1098	-0.254453397095617\\
71	0.1104	-0.313342537729397\\
71	0.111	-0.372231678363192\\
71	0.1116	-0.431120818996973\\
71	0.1122	-0.490009959630768\\
71	0.1128	-0.548899100264563\\
71	0.1134	-0.607788240898344\\
71	0.114	-0.666677381532153\\
71	0.1146	-0.725566522165934\\
71	0.1152	-0.784455662799729\\
71	0.1158	-0.84334480343351\\
71	0.1164	-0.902233944067305\\
71	0.117	-0.961123084701086\\
71	0.1176	-1.02001222533488\\
71	0.1182	-1.07890136596868\\
71	0.1188	-1.13779050660246\\
71	0.1194	-1.19667964723625\\
71	0.12	-1.25556878787003\\
71	0.1206	-1.31445792850381\\
71	0.1212	-1.37334706913762\\
71	0.1218	-1.43223620977139\\
71	0.1224	-1.4911253504052\\
71	0.123	-1.55001449103898\\
71.375	0.093	1.43472799684423\\
71.375	0.0936	1.37892669257299\\
71.375	0.0942	1.32312538830178\\
71.375	0.0948	1.26732408403053\\
71.375	0.0954	1.21152277975932\\
71.375	0.096	1.15572147548808\\
71.375	0.0966	1.09992017121687\\
71.375	0.0972	1.04411886694564\\
71.375	0.0978	0.988317562674396\\
71.375	0.0984	0.932516258403183\\
71.375	0.099	0.876714954131941\\
71.375	0.0996	0.820913649860728\\
71.375	0.1002	0.765112345589486\\
71.375	0.1008	0.709311041318273\\
71.375	0.1014	0.653509737047031\\
71.375	0.102	0.597708432775818\\
71.375	0.1026	0.541907128504576\\
71.375	0.1032	0.486105824233348\\
71.375	0.1038	0.430304519962121\\
71.375	0.1044	0.374503215690893\\
71.375	0.105	0.318701911419666\\
71.375	0.1056	0.262900607148438\\
71.375	0.1062	0.207099302877211\\
71.375	0.1068	0.151297998605983\\
71.375	0.1074	0.0954966943347557\\
71.375	0.108	0.0396953900635282\\
71.375	0.1086	-0.0161059142076994\\
71.375	0.1092	-0.0719072184789269\\
71.375	0.1098	-0.127708522750154\\
71.375	0.1104	-0.183509827021382\\
71.375	0.111	-0.23931113129261\\
71.375	0.1116	-0.295112435563837\\
71.375	0.1122	-0.350913739835065\\
71.375	0.1128	-0.406715044106306\\
71.375	0.1134	-0.46251634837752\\
71.375	0.114	-0.518317652648761\\
71.375	0.1146	-0.574118956919975\\
71.375	0.1152	-0.629920261191216\\
71.375	0.1158	-0.68572156546243\\
71.375	0.1164	-0.741522869733672\\
71.375	0.117	-0.797324174004885\\
71.375	0.1176	-0.853125478276127\\
71.375	0.1182	-0.908926782547354\\
71.375	0.1188	-0.964728086818567\\
71.375	0.1194	-1.02052939108981\\
71.375	0.12	-1.07633069536102\\
71.375	0.1206	-1.13213199963225\\
71.375	0.1212	-1.18793330390349\\
71.375	0.1218	-1.24373460817471\\
71.375	0.1224	-1.29953591244595\\
71.375	0.123	-1.35533721671717\\
71.75	0.093	1.47501345303796\\
71.75	0.0936	1.42229998512929\\
71.75	0.0942	1.36958651722063\\
71.75	0.0948	1.31687304931195\\
71.75	0.0954	1.26415958140331\\
71.75	0.096	1.21144611349463\\
71.75	0.0966	1.15873264558597\\
71.75	0.0972	1.1060191776773\\
71.75	0.0978	1.05330570976862\\
71.75	0.0984	1.00059224185996\\
71.75	0.099	0.947878773951288\\
71.75	0.0996	0.895165306042628\\
71.75	0.1002	0.842451838133968\\
71.75	0.1008	0.789738370225308\\
71.75	0.1014	0.737024902316634\\
71.75	0.102	0.684311434407974\\
71.75	0.1026	0.631597966499299\\
71.75	0.1032	0.578884498590625\\
71.75	0.1038	0.526171030681965\\
71.75	0.1044	0.473457562773291\\
71.75	0.105	0.420744094864631\\
71.75	0.1056	0.368030626955971\\
71.75	0.1062	0.315317159047311\\
71.75	0.1068	0.262603691138636\\
71.75	0.1074	0.209890223229962\\
71.75	0.108	0.157176755321302\\
71.75	0.1086	0.104463287412628\\
71.75	0.1092	0.0517498195039678\\
71.75	0.1098	-0.000963648404706419\\
71.75	0.1104	-0.0536771163133665\\
71.75	0.111	-0.106390584222027\\
71.75	0.1116	-0.159104052130687\\
71.75	0.1122	-0.211817520039361\\
71.75	0.1128	-0.264530987948035\\
71.75	0.1134	-0.317244455856695\\
71.75	0.114	-0.369957923765369\\
71.75	0.1146	-0.422671391674029\\
71.75	0.1152	-0.475384859582704\\
71.75	0.1158	-0.52809832749135\\
71.75	0.1164	-0.580811795400024\\
71.75	0.117	-0.633525263308684\\
71.75	0.1176	-0.686238731217358\\
71.75	0.1182	-0.738952199126032\\
71.75	0.1188	-0.791665667034692\\
71.75	0.1194	-0.844379134943367\\
71.75	0.12	-0.897092602852027\\
71.75	0.1206	-0.949806070760687\\
71.75	0.1212	-1.00251953866936\\
71.75	0.1218	-1.05523300657801\\
71.75	0.1224	-1.1079464744867\\
71.75	0.123	-1.16065994239536\\
72.125	0.093	1.5152989092317\\
72.125	0.0936	1.4656732776856\\
72.125	0.0942	1.4160476461395\\
72.125	0.0948	1.36642201459338\\
72.125	0.0954	1.31679638304729\\
72.125	0.096	1.26717075150118\\
72.125	0.0966	1.21754511995509\\
72.125	0.0972	1.16791948840897\\
72.125	0.0978	1.11829385686286\\
72.125	0.0984	1.06866822531677\\
72.125	0.099	1.01904259377065\\
72.125	0.0996	0.969416962224557\\
72.125	0.1002	0.91979133067845\\
72.125	0.1008	0.870165699132357\\
72.125	0.1014	0.820540067586236\\
72.125	0.102	0.770914436040144\\
72.125	0.1026	0.721288804494037\\
72.125	0.1032	0.671663172947916\\
72.125	0.1038	0.622037541401824\\
72.125	0.1044	0.572411909855717\\
72.125	0.105	0.52278627830961\\
72.125	0.1056	0.473160646763503\\
72.125	0.1062	0.423535015217411\\
72.125	0.1068	0.373909383671304\\
72.125	0.1074	0.324283752125183\\
72.125	0.108	0.27465812057909\\
72.125	0.1086	0.225032489032984\\
72.125	0.1092	0.175406857486877\\
72.125	0.1098	0.12578122594077\\
72.125	0.1104	0.0761555943946775\\
72.125	0.111	0.0265299628485707\\
72.125	0.1116	-0.0230956686975361\\
72.125	0.1122	-0.0727213002436429\\
72.125	0.1128	-0.12234693178975\\
72.125	0.1134	-0.171972563335856\\
72.125	0.114	-0.221598194881963\\
72.125	0.1146	-0.271223826428056\\
72.125	0.1152	-0.320849457974177\\
72.125	0.1158	-0.370475089520269\\
72.125	0.1164	-0.420100721066376\\
72.125	0.117	-0.469726352612469\\
72.125	0.1176	-0.51935198415859\\
72.125	0.1182	-0.568977615704696\\
72.125	0.1188	-0.618603247250789\\
72.125	0.1194	-0.66822887879691\\
72.125	0.12	-0.717854510343003\\
72.125	0.1206	-0.767480141889095\\
72.125	0.1212	-0.817105773435216\\
72.125	0.1218	-0.866731404981309\\
72.125	0.1224	-0.91635703652743\\
72.125	0.123	-0.965982668073522\\
72.5	0.093	1.55558436542545\\
72.5	0.0936	1.50904657024189\\
72.5	0.0942	1.46250877505835\\
72.5	0.0948	1.41597097987481\\
72.5	0.0954	1.36943318469127\\
72.5	0.096	1.32289538950772\\
72.5	0.0966	1.27635759432418\\
72.5	0.0972	1.22981979914064\\
72.5	0.0978	1.18328200395709\\
72.5	0.0984	1.13674420877355\\
72.5	0.099	1.09020641359001\\
72.5	0.0996	1.04366861840647\\
72.5	0.1002	0.997130823222918\\
72.5	0.1008	0.950593028039378\\
72.5	0.1014	0.904055232855839\\
72.5	0.102	0.8575174376723\\
72.5	0.1026	0.810979642488746\\
72.5	0.1032	0.764441847305207\\
72.5	0.1038	0.717904052121668\\
72.5	0.1044	0.671366256938114\\
72.5	0.105	0.624828461754589\\
72.5	0.1056	0.578290666571036\\
72.5	0.1062	0.531752871387496\\
72.5	0.1068	0.485215076203943\\
72.5	0.1074	0.438677281020404\\
72.5	0.108	0.392139485836864\\
72.5	0.1086	0.345601690653311\\
72.5	0.1092	0.299063895469786\\
72.5	0.1098	0.252526100286232\\
72.5	0.1104	0.205988305102693\\
72.5	0.111	0.159450509919139\\
72.5	0.1116	0.112912714735614\\
72.5	0.1122	0.0663749195520609\\
72.5	0.1128	0.0198371243685074\\
72.5	0.1134	-0.0267006708150177\\
72.5	0.114	-0.0732384659985712\\
72.5	0.1146	-0.11977626118211\\
72.5	0.1152	-0.16631405636565\\
72.5	0.1158	-0.212851851549189\\
72.5	0.1164	-0.259389646732743\\
72.5	0.117	-0.305927441916282\\
72.5	0.1176	-0.352465237099821\\
72.5	0.1182	-0.399003032283375\\
72.5	0.1188	-0.445540827466914\\
72.5	0.1194	-0.492078622650453\\
72.5	0.12	-0.538616417833993\\
72.5	0.1206	-0.585154213017532\\
72.5	0.1212	-0.631692008201099\\
72.5	0.1218	-0.67822980338461\\
72.5	0.1224	-0.724767598568178\\
72.5	0.123	-0.771305393751717\\
72.875	0.093	1.59586982161919\\
72.875	0.0936	1.5524198627982\\
72.875	0.0942	1.50896990397723\\
72.875	0.0948	1.46551994515625\\
72.875	0.0954	1.42206998633526\\
72.875	0.096	1.37862002751427\\
72.875	0.0966	1.3351700686933\\
72.875	0.0972	1.29172010987232\\
72.875	0.0978	1.24827015105133\\
72.875	0.0984	1.20482019223036\\
72.875	0.099	1.16137023340937\\
72.875	0.0996	1.1179202745884\\
72.875	0.1002	1.0744703157674\\
72.875	0.1008	1.03102035694643\\
72.875	0.1014	0.987570398125442\\
72.875	0.102	0.94412043930447\\
72.875	0.1026	0.900670480483484\\
72.875	0.1032	0.857220521662498\\
72.875	0.1038	0.813770562841526\\
72.875	0.1044	0.77032060402054\\
72.875	0.105	0.726870645199568\\
72.875	0.1056	0.683420686378568\\
72.875	0.1062	0.639970727557596\\
72.875	0.1068	0.59652076873661\\
72.875	0.1074	0.553070809915624\\
72.875	0.108	0.509620851094652\\
72.875	0.1086	0.466170892273666\\
72.875	0.1092	0.422720933452695\\
72.875	0.1098	0.379270974631709\\
72.875	0.1104	0.335821015810737\\
72.875	0.111	0.292371056989737\\
72.875	0.1116	0.248921098168765\\
72.875	0.1122	0.205471139347779\\
72.875	0.1128	0.162021180526793\\
72.875	0.1134	0.118571221705821\\
72.875	0.114	0.075121262884835\\
72.875	0.1146	0.0316713040638632\\
72.875	0.1152	-0.0117786547571228\\
72.875	0.1158	-0.0552286135781088\\
72.875	0.1164	-0.0986785723990948\\
72.875	0.117	-0.142128531220067\\
72.875	0.1176	-0.185578490041053\\
72.875	0.1182	-0.229028448862039\\
72.875	0.1188	-0.27247840768301\\
72.875	0.1194	-0.315928366503996\\
72.875	0.12	-0.359378325324968\\
72.875	0.1206	-0.40282828414594\\
72.875	0.1212	-0.446278242966955\\
72.875	0.1218	-0.489728201787912\\
72.875	0.1224	-0.533178160608912\\
72.875	0.123	-0.576628119429884\\
73.25	0.093	1.63615527781292\\
73.25	0.0936	1.5957931553545\\
73.25	0.0942	1.55543103289608\\
73.25	0.0948	1.51506891043766\\
73.25	0.0954	1.47470678797924\\
73.25	0.096	1.43434466552083\\
73.25	0.0966	1.39398254306241\\
73.25	0.0972	1.35362042060397\\
73.25	0.0978	1.31325829814556\\
73.25	0.0984	1.27289617568714\\
73.25	0.099	1.23253405322872\\
73.25	0.0996	1.1921719307703\\
73.25	0.1002	1.15180980831188\\
73.25	0.1008	1.11144768585346\\
73.25	0.1014	1.07108556339504\\
73.25	0.102	1.03072344093663\\
73.25	0.1026	0.990361318478207\\
73.25	0.1032	0.949999196019775\\
73.25	0.1038	0.90963707356137\\
73.25	0.1044	0.869274951102938\\
73.25	0.105	0.828912828644533\\
73.25	0.1056	0.788550706186101\\
73.25	0.1062	0.748188583727696\\
73.25	0.1068	0.707826461269264\\
73.25	0.1074	0.667464338810831\\
73.25	0.108	0.627102216352426\\
73.25	0.1086	0.586740093893994\\
73.25	0.1092	0.546377971435589\\
73.25	0.1098	0.506015848977157\\
73.25	0.1104	0.465653726518752\\
73.25	0.111	0.42529160406032\\
73.25	0.1116	0.384929481601915\\
73.25	0.1122	0.344567359143483\\
73.25	0.1128	0.304205236685064\\
73.25	0.1134	0.263843114226646\\
73.25	0.114	0.223480991768227\\
73.25	0.1146	0.183118869309808\\
73.25	0.1152	0.14275674685139\\
73.25	0.1158	0.102394624392971\\
73.25	0.1164	0.0620325019345529\\
73.25	0.117	0.0216703794761344\\
73.25	0.1176	-0.0186917429822842\\
73.25	0.1182	-0.0590538654407169\\
73.25	0.1188	-0.0994159878991354\\
73.25	0.1194	-0.139778110357554\\
73.25	0.12	-0.180140232815972\\
73.25	0.1206	-0.220502355274377\\
73.25	0.1212	-0.260864477732824\\
73.25	0.1218	-0.301226600191214\\
73.25	0.1224	-0.341588722649661\\
73.25	0.123	-0.381950845108065\\
73.625	0.093	1.67644073400666\\
73.625	0.0936	1.6391664479108\\
73.625	0.0942	1.60189216181496\\
73.625	0.0948	1.56461787571909\\
73.625	0.0954	1.52734358962324\\
73.625	0.096	1.49006930352738\\
73.625	0.0966	1.45279501743153\\
73.625	0.0972	1.41552073133566\\
73.625	0.0978	1.3782464452398\\
73.625	0.0984	1.34097215914394\\
73.625	0.099	1.30369787304808\\
73.625	0.0996	1.26642358695223\\
73.625	0.1002	1.22914930085636\\
73.625	0.1008	1.19187501476051\\
73.625	0.1014	1.15460072866465\\
73.625	0.102	1.1173264425688\\
73.625	0.1026	1.08005215647293\\
73.625	0.1032	1.04277787037707\\
73.625	0.1038	1.00550358428123\\
73.625	0.1044	0.968229298185364\\
73.625	0.105	0.930955012089512\\
73.625	0.1056	0.893680725993647\\
73.625	0.1062	0.856406439897796\\
73.625	0.1068	0.819132153801931\\
73.625	0.1074	0.781857867706066\\
73.625	0.108	0.744583581610215\\
73.625	0.1086	0.707309295514349\\
73.625	0.1092	0.670035009418498\\
73.625	0.1098	0.632760723322633\\
73.625	0.1104	0.595486437226782\\
73.625	0.111	0.558212151130917\\
73.625	0.1116	0.520937865035066\\
73.625	0.1122	0.483663578939201\\
73.625	0.1128	0.446389292843335\\
73.625	0.1134	0.409115006747484\\
73.625	0.114	0.371840720651633\\
73.625	0.1146	0.334566434555782\\
73.625	0.1152	0.297292148459917\\
73.625	0.1158	0.260017862364066\\
73.625	0.1164	0.222743576268201\\
73.625	0.117	0.18546929017235\\
73.625	0.1176	0.148195004076484\\
73.625	0.1182	0.110920717980619\\
73.625	0.1188	0.073646431884768\\
73.625	0.1194	0.0363721457889028\\
73.625	0.12	-0.00090214030694824\\
73.625	0.1206	-0.0381764264027993\\
73.625	0.1212	-0.0754507124986787\\
73.625	0.1218	-0.112724998594516\\
73.625	0.1224	-0.149999284690395\\
73.625	0.123	-0.187273570786246\\
74	0.093	1.71672619020039\\
74	0.0936	1.68253974046709\\
74	0.0942	1.64835329073379\\
74	0.0948	1.6141668410005\\
74	0.0954	1.57998039126721\\
74	0.096	1.5457939415339\\
74	0.0966	1.51160749180062\\
74	0.0972	1.47742104206731\\
74	0.0978	1.44323459233401\\
74	0.0984	1.40904814260072\\
74	0.099	1.37486169286741\\
74	0.0996	1.34067524313413\\
74	0.1002	1.30648879340083\\
74	0.1008	1.27230234366753\\
74	0.1014	1.23811589393424\\
74	0.102	1.20392944420095\\
74	0.1026	1.16974299446764\\
74	0.1032	1.13555654473434\\
74	0.1038	1.10137009500106\\
74	0.1044	1.06718364526775\\
74	0.105	1.03299719553446\\
74	0.1056	0.998810745801151\\
74	0.1062	0.964624296067868\\
74	0.1068	0.93043784633457\\
74	0.1074	0.896251396601258\\
74	0.108	0.862064946867974\\
74	0.1086	0.827878497134677\\
74	0.1092	0.793692047401379\\
74	0.1098	0.759505597668081\\
74	0.1104	0.725319147934798\\
74	0.111	0.691132698201486\\
74	0.1116	0.656946248468202\\
74	0.1122	0.622759798734904\\
74	0.1128	0.588573349001592\\
74	0.1134	0.554386899268309\\
74	0.114	0.520200449535011\\
74	0.1146	0.486013999801713\\
74	0.1152	0.451827550068415\\
74	0.1158	0.417641100335118\\
74	0.1164	0.38345465060182\\
74	0.117	0.349268200868536\\
74	0.1176	0.315081751135224\\
74	0.1182	0.280895301401927\\
74	0.1188	0.246708851668643\\
74	0.1194	0.212522401935331\\
74	0.12	0.178335952202048\\
74	0.1206	0.144149502468764\\
74	0.1212	0.109963052735438\\
74	0.1218	0.0757766030021685\\
74	0.1224	0.0415901532688565\\
74	0.123	0.00740370353555875\\
};
\end{axis}

\begin{axis}[%
width=4.927496cm,
height=3.870968cm,
at={(6.483547cm,5.376344cm)},
scale only axis,
xmin=56,
xmax=74,
tick align=outside,
xlabel={$L_{cut}$},
xmajorgrids,
ymin=0.093,
ymax=0.123,
ylabel={$D_{rlx}$},
ymajorgrids,
zmin=-1000,
zmax=89.894634681501,
zlabel={$x_4$},
zmajorgrids,
view={-140}{50},
legend style={at={(1.03,1)},anchor=north west,legend cell align=left,align=left,draw=white!15!black}
]
\addplot3[only marks,mark=*,mark options={},mark size=1.5000pt,color=mycolor1] plot table[row sep=crcr,]{%
74	0.123	-171.244880190644\\
72	0.113	-139.217460396701\\
61	0.095	-69.6051966848396\\
56	0.093	-73.6870961160989\\
};
\addplot3[only marks,mark=*,mark options={},mark size=1.5000pt,color=black] plot table[row sep=crcr,]{%
69	0.104	-101.955479221125\\
};

\addplot3[%
surf,
opacity=0.7,
shader=interp,
colormap={mymap}{[1pt] rgb(0pt)=(0.0901961,0.239216,0.0745098); rgb(1pt)=(0.0945149,0.242058,0.0739522); rgb(2pt)=(0.0988592,0.244894,0.0733566); rgb(3pt)=(0.103229,0.247724,0.0727241); rgb(4pt)=(0.107623,0.250549,0.0720557); rgb(5pt)=(0.112043,0.253367,0.0713525); rgb(6pt)=(0.116487,0.25618,0.0706154); rgb(7pt)=(0.120956,0.258986,0.0698456); rgb(8pt)=(0.125449,0.261787,0.0690441); rgb(9pt)=(0.129967,0.264581,0.0682118); rgb(10pt)=(0.134508,0.26737,0.06735); rgb(11pt)=(0.139074,0.270152,0.0664596); rgb(12pt)=(0.143663,0.272929,0.0655416); rgb(13pt)=(0.148275,0.275699,0.0645971); rgb(14pt)=(0.152911,0.278463,0.0636271); rgb(15pt)=(0.15757,0.281221,0.0626328); rgb(16pt)=(0.162252,0.283973,0.0616151); rgb(17pt)=(0.166957,0.286719,0.060575); rgb(18pt)=(0.171685,0.289458,0.0595136); rgb(19pt)=(0.176434,0.292191,0.0584321); rgb(20pt)=(0.181207,0.294918,0.0573313); rgb(21pt)=(0.186001,0.297639,0.0562123); rgb(22pt)=(0.190817,0.300353,0.0550763); rgb(23pt)=(0.195655,0.303061,0.0539242); rgb(24pt)=(0.200514,0.305763,0.052757); rgb(25pt)=(0.205395,0.308459,0.0515759); rgb(26pt)=(0.210296,0.311149,0.0503624); rgb(27pt)=(0.215212,0.313846,0.0490067); rgb(28pt)=(0.220142,0.316548,0.0475043); rgb(29pt)=(0.22509,0.319254,0.0458704); rgb(30pt)=(0.230056,0.321962,0.0441205); rgb(31pt)=(0.235042,0.324671,0.04227); rgb(32pt)=(0.240048,0.327379,0.0403343); rgb(33pt)=(0.245078,0.330085,0.0383287); rgb(34pt)=(0.250131,0.332786,0.0362688); rgb(35pt)=(0.25521,0.335482,0.0341698); rgb(36pt)=(0.260317,0.33817,0.0320472); rgb(37pt)=(0.265451,0.340849,0.0299163); rgb(38pt)=(0.270616,0.343517,0.0277927); rgb(39pt)=(0.275813,0.346172,0.0256916); rgb(40pt)=(0.281043,0.348814,0.0236284); rgb(41pt)=(0.286307,0.35144,0.0216186); rgb(42pt)=(0.291607,0.354048,0.0196776); rgb(43pt)=(0.296945,0.356637,0.0178207); rgb(44pt)=(0.302322,0.359206,0.0160634); rgb(45pt)=(0.307739,0.361753,0.0144211); rgb(46pt)=(0.313198,0.364275,0.0129091); rgb(47pt)=(0.318701,0.366772,0.0115428); rgb(48pt)=(0.324249,0.369242,0.0103377); rgb(49pt)=(0.329843,0.371682,0.00930909); rgb(50pt)=(0.335485,0.374093,0.00847245); rgb(51pt)=(0.341176,0.376471,0.00784314); rgb(52pt)=(0.346925,0.378826,0.00732741); rgb(53pt)=(0.352735,0.381168,0.00682184); rgb(54pt)=(0.358605,0.383497,0.00632729); rgb(55pt)=(0.364532,0.385812,0.00584464); rgb(56pt)=(0.370516,0.388113,0.00537476); rgb(57pt)=(0.376552,0.390399,0.00491852); rgb(58pt)=(0.38264,0.39267,0.00447681); rgb(59pt)=(0.388777,0.394925,0.00405048); rgb(60pt)=(0.394962,0.397164,0.00364042); rgb(61pt)=(0.401191,0.399386,0.00324749); rgb(62pt)=(0.407464,0.401592,0.00287258); rgb(63pt)=(0.413777,0.40378,0.00251655); rgb(64pt)=(0.420129,0.40595,0.00218028); rgb(65pt)=(0.426518,0.408102,0.00186463); rgb(66pt)=(0.432942,0.410234,0.00157049); rgb(67pt)=(0.439399,0.412348,0.00129873); rgb(68pt)=(0.445885,0.414441,0.00105022); rgb(69pt)=(0.452401,0.416515,0.000825833); rgb(70pt)=(0.458942,0.418567,0.000626441); rgb(71pt)=(0.465508,0.420599,0.00045292); rgb(72pt)=(0.472096,0.422609,0.000306141); rgb(73pt)=(0.478704,0.424596,0.000186979); rgb(74pt)=(0.485331,0.426562,9.63073e-05); rgb(75pt)=(0.491973,0.428504,3.49981e-05); rgb(76pt)=(0.498628,0.430422,3.92506e-06); rgb(77pt)=(0.505323,0.432315,0); rgb(78pt)=(0.512206,0.434168,0); rgb(79pt)=(0.519282,0.435983,0); rgb(80pt)=(0.526529,0.437764,0); rgb(81pt)=(0.533922,0.439512,0); rgb(82pt)=(0.54144,0.441232,0); rgb(83pt)=(0.549059,0.442927,0); rgb(84pt)=(0.556756,0.444599,0); rgb(85pt)=(0.564508,0.446252,0); rgb(86pt)=(0.572292,0.447889,0); rgb(87pt)=(0.580084,0.449514,0); rgb(88pt)=(0.587863,0.451129,0); rgb(89pt)=(0.595604,0.452737,0); rgb(90pt)=(0.603284,0.454343,0); rgb(91pt)=(0.610882,0.455948,0); rgb(92pt)=(0.618373,0.457556,0); rgb(93pt)=(0.625734,0.459171,0); rgb(94pt)=(0.632943,0.460795,0); rgb(95pt)=(0.639976,0.462432,0); rgb(96pt)=(0.64681,0.464084,0); rgb(97pt)=(0.653423,0.465756,0); rgb(98pt)=(0.659791,0.46745,0); rgb(99pt)=(0.665891,0.469169,0); rgb(100pt)=(0.6717,0.470916,0); rgb(101pt)=(0.677195,0.472696,0); rgb(102pt)=(0.682353,0.47451,0); rgb(103pt)=(0.687242,0.476355,0); rgb(104pt)=(0.691952,0.478225,0); rgb(105pt)=(0.696497,0.480118,0); rgb(106pt)=(0.700887,0.482033,0); rgb(107pt)=(0.705134,0.483968,0); rgb(108pt)=(0.709251,0.485921,0); rgb(109pt)=(0.713249,0.487891,0); rgb(110pt)=(0.71714,0.489876,0); rgb(111pt)=(0.720936,0.491875,0); rgb(112pt)=(0.724649,0.493887,0); rgb(113pt)=(0.72829,0.495909,0); rgb(114pt)=(0.731872,0.49794,0); rgb(115pt)=(0.735406,0.499979,0); rgb(116pt)=(0.738904,0.502025,0); rgb(117pt)=(0.742378,0.504075,0); rgb(118pt)=(0.74584,0.506128,0); rgb(119pt)=(0.749302,0.508182,0); rgb(120pt)=(0.752775,0.510237,0); rgb(121pt)=(0.756272,0.51229,0); rgb(122pt)=(0.759804,0.514339,0); rgb(123pt)=(0.763384,0.516385,0); rgb(124pt)=(0.767022,0.518424,0); rgb(125pt)=(0.770731,0.520455,0); rgb(126pt)=(0.774523,0.522478,0); rgb(127pt)=(0.77841,0.524489,0); rgb(128pt)=(0.782391,0.526491,0); rgb(129pt)=(0.786402,0.528496,0); rgb(130pt)=(0.790431,0.530506,0); rgb(131pt)=(0.794478,0.532521,0); rgb(132pt)=(0.798541,0.534539,0); rgb(133pt)=(0.802619,0.53656,0); rgb(134pt)=(0.806712,0.538584,0); rgb(135pt)=(0.81082,0.540609,0); rgb(136pt)=(0.81494,0.542635,0); rgb(137pt)=(0.819074,0.54466,0); rgb(138pt)=(0.823219,0.546686,0); rgb(139pt)=(0.827374,0.548709,0); rgb(140pt)=(0.831541,0.55073,0); rgb(141pt)=(0.835716,0.552749,0); rgb(142pt)=(0.8399,0.554763,0); rgb(143pt)=(0.844092,0.556774,0); rgb(144pt)=(0.848292,0.558779,0); rgb(145pt)=(0.852497,0.560778,0); rgb(146pt)=(0.856708,0.562771,0); rgb(147pt)=(0.860924,0.564756,0); rgb(148pt)=(0.865143,0.566733,0); rgb(149pt)=(0.869366,0.568701,0); rgb(150pt)=(0.873592,0.57066,0); rgb(151pt)=(0.877819,0.572608,0); rgb(152pt)=(0.882047,0.574545,0); rgb(153pt)=(0.886275,0.576471,0); rgb(154pt)=(0.890659,0.578362,0); rgb(155pt)=(0.895333,0.580203,0); rgb(156pt)=(0.900258,0.581999,0); rgb(157pt)=(0.905397,0.583755,0); rgb(158pt)=(0.910711,0.585479,0); rgb(159pt)=(0.916164,0.587176,0); rgb(160pt)=(0.921717,0.588852,0); rgb(161pt)=(0.927333,0.590513,0); rgb(162pt)=(0.932974,0.592166,0); rgb(163pt)=(0.938602,0.593815,0); rgb(164pt)=(0.94418,0.595468,0); rgb(165pt)=(0.949669,0.59713,0); rgb(166pt)=(0.955033,0.598808,0); rgb(167pt)=(0.960233,0.600507,0); rgb(168pt)=(0.965232,0.602233,0); rgb(169pt)=(0.969992,0.603992,0); rgb(170pt)=(0.974475,0.605791,0); rgb(171pt)=(0.978643,0.607636,0); rgb(172pt)=(0.98246,0.609532,0); rgb(173pt)=(0.985886,0.611486,0); rgb(174pt)=(0.988885,0.613503,0); rgb(175pt)=(0.991419,0.61559,0); rgb(176pt)=(0.99345,0.617753,0); rgb(177pt)=(0.99494,0.619997,0); rgb(178pt)=(0.995851,0.622329,0); rgb(179pt)=(0.996226,0.624763,0); rgb(180pt)=(0.996512,0.627352,0); rgb(181pt)=(0.996788,0.630095,0); rgb(182pt)=(0.997053,0.632982,0); rgb(183pt)=(0.997308,0.636004,0); rgb(184pt)=(0.997552,0.639152,0); rgb(185pt)=(0.997785,0.642416,0); rgb(186pt)=(0.998006,0.645786,0); rgb(187pt)=(0.998217,0.649253,0); rgb(188pt)=(0.998416,0.652807,0); rgb(189pt)=(0.998605,0.656439,0); rgb(190pt)=(0.998781,0.660138,0); rgb(191pt)=(0.998946,0.663897,0); rgb(192pt)=(0.9991,0.667704,0); rgb(193pt)=(0.999242,0.67155,0); rgb(194pt)=(0.999372,0.675427,0); rgb(195pt)=(0.99949,0.679323,0); rgb(196pt)=(0.999596,0.68323,0); rgb(197pt)=(0.99969,0.687139,0); rgb(198pt)=(0.999771,0.691039,0); rgb(199pt)=(0.999841,0.694921,0); rgb(200pt)=(0.999898,0.698775,0); rgb(201pt)=(0.999942,0.702592,0); rgb(202pt)=(0.999974,0.706363,0); rgb(203pt)=(0.999994,0.710077,0); rgb(204pt)=(1,0.713725,0); rgb(205pt)=(1,0.717341,0); rgb(206pt)=(1,0.720963,0); rgb(207pt)=(1,0.724591,0); rgb(208pt)=(1,0.728226,0); rgb(209pt)=(1,0.731867,0); rgb(210pt)=(1,0.735514,0); rgb(211pt)=(1,0.739167,0); rgb(212pt)=(1,0.742827,0); rgb(213pt)=(1,0.746493,0); rgb(214pt)=(1,0.750165,0); rgb(215pt)=(1,0.753843,0); rgb(216pt)=(1,0.757527,0); rgb(217pt)=(1,0.761217,0); rgb(218pt)=(1,0.764913,0); rgb(219pt)=(1,0.768615,0); rgb(220pt)=(1,0.772324,0); rgb(221pt)=(1,0.776038,0); rgb(222pt)=(1,0.779758,0); rgb(223pt)=(1,0.783484,0); rgb(224pt)=(1,0.787215,0); rgb(225pt)=(1,0.790953,0); rgb(226pt)=(1,0.794696,0); rgb(227pt)=(1,0.798445,0); rgb(228pt)=(1,0.8022,0); rgb(229pt)=(1,0.805961,0); rgb(230pt)=(1,0.809727,0); rgb(231pt)=(1,0.8135,0); rgb(232pt)=(1,0.817278,0); rgb(233pt)=(1,0.821063,0); rgb(234pt)=(1,0.824854,0); rgb(235pt)=(1,0.828652,0); rgb(236pt)=(1,0.832455,0); rgb(237pt)=(1,0.836265,0); rgb(238pt)=(1,0.840081,0); rgb(239pt)=(1,0.843903,0); rgb(240pt)=(1,0.847732,0); rgb(241pt)=(1,0.851566,0); rgb(242pt)=(1,0.855406,0); rgb(243pt)=(1,0.859253,0); rgb(244pt)=(1,0.863106,0); rgb(245pt)=(1,0.866964,0); rgb(246pt)=(1,0.870829,0); rgb(247pt)=(1,0.8747,0); rgb(248pt)=(1,0.878577,0); rgb(249pt)=(1,0.88246,0); rgb(250pt)=(1,0.886349,0); rgb(251pt)=(1,0.890243,0); rgb(252pt)=(1,0.894144,0); rgb(253pt)=(1,0.898051,0); rgb(254pt)=(1,0.901964,0); rgb(255pt)=(1,0.905882,0)},
mesh/rows=49]
table[row sep=crcr,header=false] {%
%
56	0.093	-73.687096116164\\
56	0.0936	-88.8575874837998\\
56	0.0942	-104.028078851435\\
56	0.0948	-119.19857021907\\
56	0.0954	-134.369061586705\\
56	0.096	-149.53955295434\\
56	0.0966	-164.710044321976\\
56	0.0972	-179.880535689611\\
56	0.0978	-195.051027057246\\
56	0.0984	-210.221518424882\\
56	0.099	-225.392009792517\\
56	0.0996	-240.562501160152\\
56	0.1002	-255.732992527787\\
56	0.1008	-270.903483895423\\
56	0.1014	-286.073975263059\\
56	0.102	-301.244466630694\\
56	0.1026	-316.414957998329\\
56	0.1032	-331.585449365964\\
56	0.1038	-346.755940733599\\
56	0.1044	-361.926432101235\\
56	0.105	-377.096923468871\\
56	0.1056	-392.267414836505\\
56	0.1062	-407.437906204141\\
56	0.1068	-422.608397571777\\
56	0.1074	-437.778888939411\\
56	0.108	-452.949380307047\\
56	0.1086	-468.119871674683\\
56	0.1092	-483.290363042317\\
56	0.1098	-498.460854409953\\
56	0.1104	-513.631345777589\\
56	0.111	-528.801837145224\\
56	0.1116	-543.972328512859\\
56	0.1122	-559.142819880494\\
56	0.1128	-574.31331124813\\
56	0.1134	-589.483802615765\\
56	0.114	-604.654293983401\\
56	0.1146	-619.824785351037\\
56	0.1152	-634.995276718671\\
56	0.1158	-650.165768086306\\
56	0.1164	-665.336259453942\\
56	0.117	-680.506750821576\\
56	0.1176	-695.677242189212\\
56	0.1182	-710.847733556848\\
56	0.1188	-726.018224924484\\
56	0.1194	-741.188716292118\\
56	0.12	-756.359207659753\\
56	0.1206	-771.529699027389\\
56	0.1212	-786.700190395024\\
56	0.1218	-801.870681762659\\
56	0.1224	-817.041173130295\\
56	0.123	-832.21166449793\\
56.375	0.093	-70.2791433912116\\
56.375	0.0936	-85.242390986552\\
56.375	0.0942	-100.205638581891\\
56.375	0.0948	-115.168886177231\\
56.375	0.0954	-130.13213377257\\
56.375	0.096	-145.095381367911\\
56.375	0.0966	-160.058628963251\\
56.375	0.0972	-175.02187655859\\
56.375	0.0978	-189.98512415393\\
56.375	0.0984	-204.948371749269\\
56.375	0.099	-219.91161934461\\
56.375	0.0996	-234.87486693995\\
56.375	0.1002	-249.838114535289\\
56.375	0.1008	-264.801362130629\\
56.375	0.1014	-279.764609725969\\
56.375	0.102	-294.727857321309\\
56.375	0.1026	-309.691104916648\\
56.375	0.1032	-324.654352511988\\
56.375	0.1038	-339.617600107328\\
56.375	0.1044	-354.580847702668\\
56.375	0.105	-369.544095298008\\
56.375	0.1056	-384.507342893347\\
56.375	0.1062	-399.470590488687\\
56.375	0.1068	-414.433838084028\\
56.375	0.1074	-429.397085679367\\
56.375	0.108	-444.360333274707\\
56.375	0.1086	-459.323580870047\\
56.375	0.1092	-474.286828465386\\
56.375	0.1098	-489.250076060726\\
56.375	0.1104	-504.213323656067\\
56.375	0.111	-519.176571251406\\
56.375	0.1116	-534.139818846746\\
56.375	0.1122	-549.103066442085\\
56.375	0.1128	-564.066314037425\\
56.375	0.1134	-579.029561632766\\
56.375	0.114	-593.992809228105\\
56.375	0.1146	-608.956056823446\\
56.375	0.1152	-623.919304418784\\
56.375	0.1158	-638.882552014124\\
56.375	0.1164	-653.845799609465\\
56.375	0.117	-668.809047204803\\
56.375	0.1176	-683.772294800144\\
56.375	0.1182	-698.735542395483\\
56.375	0.1188	-713.698789990824\\
56.375	0.1194	-728.662037586163\\
56.375	0.12	-743.625285181502\\
56.375	0.1206	-758.588532776843\\
56.375	0.1212	-773.551780372183\\
56.375	0.1218	-788.515027967522\\
56.375	0.1224	-803.478275562862\\
56.375	0.123	-818.441523158202\\
56.75	0.093	-66.8711906662602\\
56.75	0.0936	-81.6271944893051\\
56.75	0.0942	-96.3831983123491\\
56.75	0.0948	-111.139202135393\\
56.75	0.0954	-125.895205958437\\
56.75	0.096	-140.651209781482\\
56.75	0.0966	-155.407213604526\\
56.75	0.0972	-170.16321742757\\
56.75	0.0978	-184.919221250614\\
56.75	0.0984	-199.675225073659\\
56.75	0.099	-214.431228896703\\
56.75	0.0996	-229.187232719748\\
56.75	0.1002	-243.943236542791\\
56.75	0.1008	-258.699240365836\\
56.75	0.1014	-273.45524418888\\
56.75	0.102	-288.211248011925\\
56.75	0.1026	-302.967251834968\\
56.75	0.1032	-317.723255658013\\
56.75	0.1038	-332.479259481058\\
56.75	0.1044	-347.235263304102\\
56.75	0.105	-361.991267127147\\
56.75	0.1056	-376.74727095019\\
56.75	0.1062	-391.503274773235\\
56.75	0.1068	-406.25927859628\\
56.75	0.1074	-421.015282419324\\
56.75	0.108	-435.771286242368\\
56.75	0.1086	-450.527290065413\\
56.75	0.1092	-465.283293888456\\
56.75	0.1098	-480.039297711501\\
56.75	0.1104	-494.795301534545\\
56.75	0.111	-509.55130535759\\
56.75	0.1116	-524.307309180634\\
56.75	0.1122	-539.063313003678\\
56.75	0.1128	-553.819316826722\\
56.75	0.1134	-568.575320649767\\
56.75	0.114	-583.331324472811\\
56.75	0.1146	-598.087328295856\\
56.75	0.1152	-612.843332118899\\
56.75	0.1158	-627.599335941944\\
56.75	0.1164	-642.355339764988\\
56.75	0.117	-657.111343588032\\
56.75	0.1176	-671.867347411076\\
56.75	0.1182	-686.62335123412\\
56.75	0.1188	-701.379355057165\\
56.75	0.1194	-716.135358880209\\
56.75	0.12	-730.891362703253\\
56.75	0.1206	-745.647366526297\\
56.75	0.1212	-760.403370349342\\
56.75	0.1218	-775.159374172385\\
56.75	0.1224	-789.91537799543\\
56.75	0.123	-804.671381818474\\
57.125	0.093	-63.4632379413088\\
57.125	0.0936	-78.0119979920582\\
57.125	0.0942	-92.5607580428068\\
57.125	0.0948	-107.109518093554\\
57.125	0.0954	-121.658278144304\\
57.125	0.096	-136.207038195053\\
57.125	0.0966	-150.755798245802\\
57.125	0.0972	-165.30455829655\\
57.125	0.0978	-179.853318347299\\
57.125	0.0984	-194.402078398048\\
57.125	0.099	-208.950838448797\\
57.125	0.0996	-223.499598499546\\
57.125	0.1002	-238.048358550293\\
57.125	0.1008	-252.597118601043\\
57.125	0.1014	-267.145878651791\\
57.125	0.102	-281.694638702541\\
57.125	0.1026	-296.243398753289\\
57.125	0.1032	-310.792158804038\\
57.125	0.1038	-325.340918854787\\
57.125	0.1044	-339.889678905535\\
57.125	0.105	-354.438438956285\\
57.125	0.1056	-368.987199007032\\
57.125	0.1062	-383.535959057782\\
57.125	0.1068	-398.084719108531\\
57.125	0.1074	-412.63347915928\\
57.125	0.108	-427.182239210028\\
57.125	0.1086	-441.730999260778\\
57.125	0.1092	-456.279759311526\\
57.125	0.1098	-470.828519362274\\
57.125	0.1104	-485.377279413024\\
57.125	0.111	-499.926039463772\\
57.125	0.1116	-514.474799514522\\
57.125	0.1122	-529.023559565269\\
57.125	0.1128	-543.572319616019\\
57.125	0.1134	-558.121079666767\\
57.125	0.114	-572.669839717517\\
57.125	0.1146	-587.218599768265\\
57.125	0.1152	-601.767359819013\\
57.125	0.1158	-616.316119869763\\
57.125	0.1164	-630.864879920511\\
57.125	0.117	-645.41363997126\\
57.125	0.1176	-659.962400022008\\
57.125	0.1182	-674.511160072758\\
57.125	0.1188	-689.059920123506\\
57.125	0.1194	-703.608680174254\\
57.125	0.12	-718.157440225003\\
57.125	0.1206	-732.706200275752\\
57.125	0.1212	-747.254960326502\\
57.125	0.1218	-761.803720377249\\
57.125	0.1224	-776.352480427999\\
57.125	0.123	-790.901240478747\\
57.5	0.093	-60.0552852163573\\
57.5	0.0936	-74.3968014948114\\
57.5	0.0942	-88.7383177732645\\
57.5	0.0948	-103.079834051717\\
57.5	0.0954	-117.421350330171\\
57.5	0.096	-131.762866608624\\
57.5	0.0966	-146.104382887078\\
57.5	0.0972	-160.44589916553\\
57.5	0.0978	-174.787415443983\\
57.5	0.0984	-189.128931722437\\
57.5	0.099	-203.47044800089\\
57.5	0.0996	-217.811964279344\\
57.5	0.1002	-232.153480557797\\
57.5	0.1008	-246.49499683625\\
57.5	0.1014	-260.836513114703\\
57.5	0.102	-275.178029393157\\
57.5	0.1026	-289.519545671609\\
57.5	0.1032	-303.861061950062\\
57.5	0.1038	-318.202578228516\\
57.5	0.1044	-332.544094506969\\
57.5	0.105	-346.885610785423\\
57.5	0.1056	-361.227127063876\\
57.5	0.1062	-375.568643342329\\
57.5	0.1068	-389.910159620784\\
57.5	0.1074	-404.251675899236\\
57.5	0.108	-418.593192177689\\
57.5	0.1086	-432.934708456143\\
57.5	0.1092	-447.276224734595\\
57.5	0.1098	-461.617741013049\\
57.5	0.1104	-475.959257291503\\
57.5	0.111	-490.300773569956\\
57.5	0.1116	-504.642289848409\\
57.5	0.1122	-518.983806126862\\
57.5	0.1128	-533.325322405315\\
57.5	0.1134	-547.666838683768\\
57.5	0.114	-562.008354962222\\
57.5	0.1146	-576.349871240675\\
57.5	0.1152	-590.691387519128\\
57.5	0.1158	-605.032903797582\\
57.5	0.1164	-619.374420076035\\
57.5	0.117	-633.715936354487\\
57.5	0.1176	-648.057452632941\\
57.5	0.1182	-662.398968911394\\
57.5	0.1188	-676.740485189848\\
57.5	0.1194	-691.0820014683\\
57.5	0.12	-705.423517746754\\
57.5	0.1206	-719.765034025208\\
57.5	0.1212	-734.106550303661\\
57.5	0.1218	-748.448066582113\\
57.5	0.1224	-762.789582860567\\
57.5	0.123	-777.13109913902\\
57.875	0.093	-56.6473324914059\\
57.875	0.0936	-70.7816049975645\\
57.875	0.0942	-84.9158775037222\\
57.875	0.0948	-99.050150009879\\
57.875	0.0954	-113.184422516037\\
57.875	0.096	-127.318695022195\\
57.875	0.0966	-141.452967528353\\
57.875	0.0972	-155.58724003451\\
57.875	0.0978	-169.721512540668\\
57.875	0.0984	-183.855785046826\\
57.875	0.099	-197.990057552984\\
57.875	0.0996	-212.124330059141\\
57.875	0.1002	-226.258602565299\\
57.875	0.1008	-240.392875071457\\
57.875	0.1014	-254.527147577614\\
57.875	0.102	-268.661420083772\\
57.875	0.1026	-282.79569258993\\
57.875	0.1032	-296.929965096087\\
57.875	0.1038	-311.064237602245\\
57.875	0.1044	-325.198510108404\\
57.875	0.105	-339.332782614561\\
57.875	0.1056	-353.467055120718\\
57.875	0.1062	-367.601327626876\\
57.875	0.1068	-381.735600133035\\
57.875	0.1074	-395.869872639192\\
57.875	0.108	-410.00414514535\\
57.875	0.1086	-424.138417651508\\
57.875	0.1092	-438.272690157665\\
57.875	0.1098	-452.406962663823\\
57.875	0.1104	-466.541235169981\\
57.875	0.111	-480.675507676139\\
57.875	0.1116	-494.809780182297\\
57.875	0.1122	-508.944052688454\\
57.875	0.1128	-523.078325194611\\
57.875	0.1134	-537.21259770077\\
57.875	0.114	-551.346870206928\\
57.875	0.1146	-565.481142713085\\
57.875	0.1152	-579.615415219242\\
57.875	0.1158	-593.749687725401\\
57.875	0.1164	-607.883960231558\\
57.875	0.117	-622.018232737715\\
57.875	0.1176	-636.152505243874\\
57.875	0.1182	-650.286777750031\\
57.875	0.1188	-664.421050256189\\
57.875	0.1194	-678.555322762346\\
57.875	0.12	-692.689595268504\\
57.875	0.1206	-706.823867774662\\
57.875	0.1212	-720.95814028082\\
57.875	0.1218	-735.092412786977\\
57.875	0.1224	-749.226685293135\\
57.875	0.123	-763.360957799293\\
58.25	0.093	-53.2393797664554\\
58.25	0.0936	-67.1664085003176\\
58.25	0.0942	-81.0934372341799\\
58.25	0.0948	-95.0204659680412\\
58.25	0.0954	-108.947494701903\\
58.25	0.096	-122.874523435766\\
58.25	0.0966	-136.801552169629\\
58.25	0.0972	-150.72858090349\\
58.25	0.0978	-164.655609637352\\
58.25	0.0984	-178.582638371215\\
58.25	0.099	-192.509667105077\\
58.25	0.0996	-206.43669583894\\
58.25	0.1002	-220.363724572801\\
58.25	0.1008	-234.290753306664\\
58.25	0.1014	-248.217782040526\\
58.25	0.102	-262.144810774388\\
58.25	0.1026	-276.07183950825\\
58.25	0.1032	-289.998868242113\\
58.25	0.1038	-303.925896975975\\
58.25	0.1044	-317.852925709837\\
58.25	0.105	-331.779954443699\\
58.25	0.1056	-345.706983177562\\
58.25	0.1062	-359.634011911424\\
58.25	0.1068	-373.561040645287\\
58.25	0.1074	-387.488069379148\\
58.25	0.108	-401.415098113011\\
58.25	0.1086	-415.342126846873\\
58.25	0.1092	-429.269155580735\\
58.25	0.1098	-443.196184314597\\
58.25	0.1104	-457.12321304846\\
58.25	0.111	-471.050241782322\\
58.25	0.1116	-484.977270516184\\
58.25	0.1122	-498.904299250046\\
58.25	0.1128	-512.831327983909\\
58.25	0.1134	-526.758356717771\\
58.25	0.114	-540.685385451633\\
58.25	0.1146	-554.612414185495\\
58.25	0.1152	-568.539442919357\\
58.25	0.1158	-582.46647165322\\
58.25	0.1164	-596.393500387082\\
58.25	0.117	-610.320529120943\\
58.25	0.1176	-624.247557854806\\
58.25	0.1182	-638.174586588669\\
58.25	0.1188	-652.101615322531\\
58.25	0.1194	-666.028644056392\\
58.25	0.12	-679.955672790255\\
58.25	0.1206	-693.882701524117\\
58.25	0.1212	-707.809730257979\\
58.25	0.1218	-721.736758991841\\
58.25	0.1224	-735.663787725704\\
58.25	0.123	-749.590816459566\\
58.625	0.093	-49.831427041504\\
58.625	0.0936	-63.5512120030708\\
58.625	0.0942	-77.2709969646376\\
58.625	0.0948	-90.9907819262035\\
58.625	0.0954	-104.71056688777\\
58.625	0.096	-118.430351849337\\
58.625	0.0966	-132.150136810904\\
58.625	0.0972	-145.86992177247\\
58.625	0.0978	-159.589706734037\\
58.625	0.0984	-173.309491695604\\
58.625	0.099	-187.029276657171\\
58.625	0.0996	-200.749061618738\\
58.625	0.1002	-214.468846580304\\
58.625	0.1008	-228.188631541871\\
58.625	0.1014	-241.908416503437\\
58.625	0.102	-255.628201465004\\
58.625	0.1026	-269.34798642657\\
58.625	0.1032	-283.067771388137\\
58.625	0.1038	-296.787556349705\\
58.625	0.1044	-310.507341311271\\
58.625	0.105	-324.227126272838\\
58.625	0.1056	-337.946911234404\\
58.625	0.1062	-351.666696195971\\
58.625	0.1068	-365.386481157539\\
58.625	0.1074	-379.106266119104\\
58.625	0.108	-392.826051080671\\
58.625	0.1086	-406.545836042238\\
58.625	0.1092	-420.265621003804\\
58.625	0.1098	-433.985405965372\\
58.625	0.1104	-447.705190926938\\
58.625	0.111	-461.424975888505\\
58.625	0.1116	-475.144760850072\\
58.625	0.1122	-488.864545811638\\
58.625	0.1128	-502.584330773205\\
58.625	0.1134	-516.304115734772\\
58.625	0.114	-530.023900696338\\
58.625	0.1146	-543.743685657906\\
58.625	0.1152	-557.463470619472\\
58.625	0.1158	-571.183255581039\\
58.625	0.1164	-584.903040542606\\
58.625	0.117	-598.622825504171\\
58.625	0.1176	-612.342610465738\\
58.625	0.1182	-626.062395427305\\
58.625	0.1188	-639.782180388872\\
58.625	0.1194	-653.501965350438\\
58.625	0.12	-667.221750312005\\
58.625	0.1206	-680.941535273572\\
58.625	0.1212	-694.661320235139\\
58.625	0.1218	-708.381105196705\\
58.625	0.1224	-722.100890158272\\
58.625	0.123	-735.820675119839\\
59	0.093	-46.4234743165525\\
59	0.0936	-59.9360155058239\\
59	0.0942	-73.4485566950952\\
59	0.0948	-86.9610978843657\\
59	0.0954	-100.473639073637\\
59	0.096	-113.986180262908\\
59	0.0966	-127.49872145218\\
59	0.0972	-141.01126264145\\
59	0.0978	-154.523803830722\\
59	0.0984	-168.036345019993\\
59	0.099	-181.548886209264\\
59	0.0996	-195.061427398536\\
59	0.1002	-208.573968587806\\
59	0.1008	-222.086509777077\\
59	0.1014	-235.599050966349\\
59	0.102	-249.11159215562\\
59	0.1026	-262.624133344891\\
59	0.1032	-276.136674534162\\
59	0.1038	-289.649215723433\\
59	0.1044	-303.161756912705\\
59	0.105	-316.674298101976\\
59	0.1056	-330.186839291247\\
59	0.1062	-343.699380480518\\
59	0.1068	-357.21192166979\\
59	0.1074	-370.724462859061\\
59	0.108	-384.237004048332\\
59	0.1086	-397.749545237603\\
59	0.1092	-411.262086426874\\
59	0.1098	-424.774627616145\\
59	0.1104	-438.287168805417\\
59	0.111	-451.799709994688\\
59	0.1116	-465.31225118396\\
59	0.1122	-478.82479237323\\
59	0.1128	-492.337333562501\\
59	0.1134	-505.849874751773\\
59	0.114	-519.362415941045\\
59	0.1146	-532.874957130316\\
59	0.1152	-546.387498319586\\
59	0.1158	-559.900039508858\\
59	0.1164	-573.412580698129\\
59	0.117	-586.9251218874\\
59	0.1176	-600.437663076671\\
59	0.1182	-613.950204265942\\
59	0.1188	-627.462745455214\\
59	0.1194	-640.975286644484\\
59	0.12	-654.487827833756\\
59	0.1206	-668.000369023027\\
59	0.1212	-681.512910212298\\
59	0.1218	-695.025451401569\\
59	0.1224	-708.53799259084\\
59	0.123	-722.050533780111\\
59.375	0.093	-43.0155215916011\\
59.375	0.0936	-56.320819008577\\
59.375	0.0942	-69.6261164255529\\
59.375	0.0948	-82.9314138425279\\
59.375	0.0954	-96.2367112595039\\
59.375	0.096	-109.54200867648\\
59.375	0.0966	-122.847306093456\\
59.375	0.0972	-136.15260351043\\
59.375	0.0978	-149.457900927406\\
59.375	0.0984	-162.763198344382\\
59.375	0.099	-176.068495761358\\
59.375	0.0996	-189.373793178333\\
59.375	0.1002	-202.679090595308\\
59.375	0.1008	-215.984388012284\\
59.375	0.1014	-229.28968542926\\
59.375	0.102	-242.594982846236\\
59.375	0.1026	-255.900280263211\\
59.375	0.1032	-269.205577680187\\
59.375	0.1038	-282.510875097163\\
59.375	0.1044	-295.816172514139\\
59.375	0.105	-309.121469931115\\
59.375	0.1056	-322.42676734809\\
59.375	0.1062	-335.732064765066\\
59.375	0.1068	-349.037362182043\\
59.375	0.1074	-362.342659599017\\
59.375	0.108	-375.647957015993\\
59.375	0.1086	-388.953254432969\\
59.375	0.1092	-402.258551849944\\
59.375	0.1098	-415.56384926692\\
59.375	0.1104	-428.869146683895\\
59.375	0.111	-442.174444100871\\
59.375	0.1116	-455.479741517847\\
59.375	0.1122	-468.785038934822\\
59.375	0.1128	-482.090336351798\\
59.375	0.1134	-495.395633768774\\
59.375	0.114	-508.70093118575\\
59.375	0.1146	-522.006228602726\\
59.375	0.1152	-535.311526019701\\
59.375	0.1158	-548.616823436677\\
59.375	0.1164	-561.922120853653\\
59.375	0.117	-575.227418270627\\
59.375	0.1176	-588.532715687603\\
59.375	0.1182	-601.838013104579\\
59.375	0.1188	-615.143310521555\\
59.375	0.1194	-628.44860793853\\
59.375	0.12	-641.753905355506\\
59.375	0.1206	-655.059202772482\\
59.375	0.1212	-668.364500189457\\
59.375	0.1218	-681.669797606432\\
59.375	0.1224	-694.975095023408\\
59.375	0.123	-708.280392440384\\
59.75	0.093	-39.6075688666497\\
59.75	0.0936	-52.7056225113301\\
59.75	0.0942	-65.8036761560106\\
59.75	0.0948	-78.9017298006893\\
59.75	0.0954	-91.9997834453698\\
59.75	0.096	-105.09783709005\\
59.75	0.0966	-118.195890734731\\
59.75	0.0972	-131.29394437941\\
59.75	0.0978	-144.391998024091\\
59.75	0.0984	-157.490051668771\\
59.75	0.099	-170.588105313452\\
59.75	0.0996	-183.686158958131\\
59.75	0.1002	-196.784212602811\\
59.75	0.1008	-209.882266247491\\
59.75	0.1014	-222.980319892172\\
59.75	0.102	-236.078373536852\\
59.75	0.1026	-249.176427181532\\
59.75	0.1032	-262.274480826212\\
59.75	0.1038	-275.372534470892\\
59.75	0.1044	-288.470588115572\\
59.75	0.105	-301.568641760253\\
59.75	0.1056	-314.666695404932\\
59.75	0.1062	-327.764749049613\\
59.75	0.1068	-340.862802694294\\
59.75	0.1074	-353.960856338973\\
59.75	0.108	-367.058909983653\\
59.75	0.1086	-380.156963628334\\
59.75	0.1092	-393.255017273013\\
59.75	0.1098	-406.353070917694\\
59.75	0.1104	-419.451124562374\\
59.75	0.111	-432.549178207055\\
59.75	0.1116	-445.647231851735\\
59.75	0.1122	-458.745285496414\\
59.75	0.1128	-471.843339141094\\
59.75	0.1134	-484.941392785775\\
59.75	0.114	-498.039446430455\\
59.75	0.1146	-511.137500075136\\
59.75	0.1152	-524.235553719815\\
59.75	0.1158	-537.333607364496\\
59.75	0.1164	-550.431661009176\\
59.75	0.117	-563.529714653855\\
59.75	0.1176	-576.627768298536\\
59.75	0.1182	-589.725821943216\\
59.75	0.1188	-602.823875587897\\
59.75	0.1194	-615.921929232576\\
59.75	0.12	-629.019982877256\\
59.75	0.1206	-642.118036521936\\
59.75	0.1212	-655.216090166617\\
59.75	0.1218	-668.314143811296\\
59.75	0.1224	-681.412197455977\\
59.75	0.123	-694.510251100657\\
60.125	0.093	-36.1996161416982\\
60.125	0.0936	-49.0904260140833\\
60.125	0.0942	-61.9812358864683\\
60.125	0.0948	-74.8720457588515\\
60.125	0.0954	-87.7628556312366\\
60.125	0.096	-100.653665503622\\
60.125	0.0966	-113.544475376007\\
60.125	0.0972	-126.43528524839\\
60.125	0.0978	-139.326095120775\\
60.125	0.0984	-152.21690499316\\
60.125	0.099	-165.107714865545\\
60.125	0.0996	-177.998524737929\\
60.125	0.1002	-190.889334610313\\
60.125	0.1008	-203.780144482698\\
60.125	0.1014	-216.670954355083\\
60.125	0.102	-229.561764227468\\
60.125	0.1026	-242.452574099852\\
60.125	0.1032	-255.343383972237\\
60.125	0.1038	-268.234193844622\\
60.125	0.1044	-281.125003717007\\
60.125	0.105	-294.015813589391\\
60.125	0.1056	-306.906623461775\\
60.125	0.1062	-319.79743333416\\
60.125	0.1068	-332.688243206546\\
60.125	0.1074	-345.57905307893\\
60.125	0.108	-358.469862951314\\
60.125	0.1086	-371.360672823699\\
60.125	0.1092	-384.251482696083\\
60.125	0.1098	-397.142292568468\\
60.125	0.1104	-410.033102440852\\
60.125	0.111	-422.923912313237\\
60.125	0.1116	-435.814722185622\\
60.125	0.1122	-448.705532058007\\
60.125	0.1128	-461.596341930392\\
60.125	0.1134	-474.487151802776\\
60.125	0.114	-487.377961675161\\
60.125	0.1146	-500.268771547546\\
60.125	0.1152	-513.15958141993\\
60.125	0.1158	-526.050391292314\\
60.125	0.1164	-538.941201164699\\
60.125	0.117	-551.832011037083\\
60.125	0.1176	-564.722820909468\\
60.125	0.1182	-577.613630781853\\
60.125	0.1188	-590.504440654237\\
60.125	0.1194	-603.395250526622\\
60.125	0.12	-616.286060399007\\
60.125	0.1206	-629.176870271392\\
60.125	0.1212	-642.067680143776\\
60.125	0.1218	-654.95849001616\\
60.125	0.1224	-667.849299888545\\
60.125	0.123	-680.74010976093\\
60.5	0.093	-32.7916634167468\\
60.5	0.0936	-45.4752295168364\\
60.5	0.0942	-58.158795616926\\
60.5	0.0948	-70.8423617170138\\
60.5	0.0954	-83.5259278171034\\
60.5	0.096	-96.209493917193\\
60.5	0.0966	-108.893060017282\\
60.5	0.0972	-121.57662611737\\
60.5	0.0978	-134.26019221746\\
60.5	0.0984	-146.943758317549\\
60.5	0.099	-159.627324417638\\
60.5	0.0996	-172.310890517728\\
60.5	0.1002	-184.994456617816\\
60.5	0.1008	-197.678022717905\\
60.5	0.1014	-210.361588817995\\
60.5	0.102	-223.045154918083\\
60.5	0.1026	-235.728721018172\\
60.5	0.1032	-248.412287118262\\
60.5	0.1038	-261.095853218351\\
60.5	0.1044	-273.77941931844\\
60.5	0.105	-286.46298541853\\
60.5	0.1056	-299.146551518618\\
60.5	0.1062	-311.830117618707\\
60.5	0.1068	-324.513683718797\\
60.5	0.1074	-337.197249818886\\
60.5	0.108	-349.880815918975\\
60.5	0.1086	-362.564382019064\\
60.5	0.1092	-375.247948119153\\
60.5	0.1098	-387.931514219242\\
60.5	0.1104	-400.615080319331\\
60.5	0.111	-413.298646419421\\
60.5	0.1116	-425.982212519511\\
60.5	0.1122	-438.665778619598\\
60.5	0.1128	-451.349344719688\\
60.5	0.1134	-464.032910819777\\
60.5	0.114	-476.716476919866\\
60.5	0.1146	-489.400043019956\\
60.5	0.1152	-502.083609120044\\
60.5	0.1158	-514.767175220133\\
60.5	0.1164	-527.450741320223\\
60.5	0.117	-540.134307420311\\
60.5	0.1176	-552.8178735204\\
60.5	0.1182	-565.50143962049\\
60.5	0.1188	-578.185005720579\\
60.5	0.1194	-590.868571820667\\
60.5	0.12	-603.552137920757\\
60.5	0.1206	-616.235704020846\\
60.5	0.1212	-628.919270120936\\
60.5	0.1218	-641.602836221024\\
60.5	0.1224	-654.286402321113\\
60.5	0.123	-666.969968421203\\
60.875	0.093	-29.3837106917954\\
60.875	0.0936	-41.8600330195895\\
60.875	0.0942	-54.3363553473837\\
60.875	0.0948	-66.812677675176\\
60.875	0.0954	-79.2890000029702\\
60.875	0.096	-91.7653223307634\\
60.875	0.0966	-104.241644658558\\
60.875	0.0972	-116.71796698635\\
60.875	0.0978	-129.194289314144\\
60.875	0.0984	-141.670611641938\\
60.875	0.099	-154.146933969731\\
60.875	0.0996	-166.623256297526\\
60.875	0.1002	-179.099578625319\\
60.875	0.1008	-191.575900953112\\
60.875	0.1014	-204.052223280906\\
60.875	0.102	-216.528545608699\\
60.875	0.1026	-229.004867936493\\
60.875	0.1032	-241.481190264287\\
60.875	0.1038	-253.95751259208\\
60.875	0.1044	-266.433834919874\\
60.875	0.105	-278.910157247667\\
60.875	0.1056	-291.386479575461\\
60.875	0.1062	-303.862801903255\\
60.875	0.1068	-316.339124231049\\
60.875	0.1074	-328.815446558842\\
60.875	0.108	-341.291768886636\\
60.875	0.1086	-353.76809121443\\
60.875	0.1092	-366.244413542223\\
60.875	0.1098	-378.720735870016\\
60.875	0.1104	-391.19705819781\\
60.875	0.111	-403.673380525604\\
60.875	0.1116	-416.149702853398\\
60.875	0.1122	-428.626025181191\\
60.875	0.1128	-441.102347508984\\
60.875	0.1134	-453.578669836778\\
60.875	0.114	-466.054992164572\\
60.875	0.1146	-478.531314492366\\
60.875	0.1152	-491.007636820159\\
60.875	0.1158	-503.483959147952\\
60.875	0.1164	-515.960281475746\\
60.875	0.117	-528.436603803539\\
60.875	0.1176	-540.912926131333\\
60.875	0.1182	-553.389248459127\\
60.875	0.1188	-565.86557078692\\
60.875	0.1194	-578.341893114713\\
60.875	0.12	-590.818215442508\\
60.875	0.1206	-603.294537770301\\
60.875	0.1212	-615.770860098095\\
60.875	0.1218	-628.247182425887\\
60.875	0.1224	-640.723504753681\\
60.875	0.123	-653.199827081476\\
61.25	0.093	-25.975757966844\\
61.25	0.0936	-38.2448365223427\\
61.25	0.0942	-50.5139150778405\\
61.25	0.0948	-62.7829936333383\\
61.25	0.0954	-75.0520721888361\\
61.25	0.096	-87.3211507443348\\
61.25	0.0966	-99.5902292998335\\
61.25	0.0972	-111.85930785533\\
61.25	0.0978	-124.128386410829\\
61.25	0.0984	-136.397464966327\\
61.25	0.099	-148.666543521826\\
61.25	0.0996	-160.935622077323\\
61.25	0.1002	-173.204700632821\\
61.25	0.1008	-185.473779188319\\
61.25	0.1014	-197.742857743818\\
61.25	0.102	-210.011936299315\\
61.25	0.1026	-222.281014854813\\
61.25	0.1032	-234.550093410311\\
61.25	0.1038	-246.81917196581\\
61.25	0.1044	-259.088250521308\\
61.25	0.105	-271.357329076806\\
61.25	0.1056	-283.626407632303\\
61.25	0.1062	-295.895486187802\\
61.25	0.1068	-308.164564743301\\
61.25	0.1074	-320.433643298798\\
61.25	0.108	-332.702721854296\\
61.25	0.1086	-344.971800409795\\
61.25	0.1092	-357.240878965292\\
61.25	0.1098	-369.509957520791\\
61.25	0.1104	-381.779036076289\\
61.25	0.111	-394.048114631787\\
61.25	0.1116	-406.317193187286\\
61.25	0.1122	-418.586271742783\\
61.25	0.1128	-430.855350298281\\
61.25	0.1134	-443.124428853779\\
61.25	0.114	-455.393507409278\\
61.25	0.1146	-467.662585964776\\
61.25	0.1152	-479.931664520273\\
61.25	0.1158	-492.200743075771\\
61.25	0.1164	-504.46982163127\\
61.25	0.117	-516.738900186767\\
61.25	0.1176	-529.007978742266\\
61.25	0.1182	-541.277057297763\\
61.25	0.1188	-553.546135853262\\
61.25	0.1194	-565.815214408759\\
61.25	0.12	-578.084292964258\\
61.25	0.1206	-590.353371519755\\
61.25	0.1212	-602.622450075254\\
61.25	0.1218	-614.891528630751\\
61.25	0.1224	-627.16060718625\\
61.25	0.123	-639.429685741748\\
61.625	0.093	-22.5678052418925\\
61.625	0.0936	-34.6296400250958\\
61.625	0.0942	-46.6914748082982\\
61.625	0.0948	-58.7533095915005\\
61.625	0.0954	-70.8151443747029\\
61.625	0.096	-82.8769791579061\\
61.625	0.0966	-94.9388139411085\\
61.625	0.0972	-107.00064872431\\
61.625	0.0978	-119.062483507513\\
61.625	0.0984	-131.124318290716\\
61.625	0.099	-143.186153073919\\
61.625	0.0996	-155.247987857121\\
61.625	0.1002	-167.309822640324\\
61.625	0.1008	-179.371657423526\\
61.625	0.1014	-191.433492206729\\
61.625	0.102	-203.495326989932\\
61.625	0.1026	-215.557161773133\\
61.625	0.1032	-227.618996556336\\
61.625	0.1038	-239.680831339539\\
61.625	0.1044	-251.742666122742\\
61.625	0.105	-263.804500905944\\
61.625	0.1056	-275.866335689147\\
61.625	0.1062	-287.928170472349\\
61.625	0.1068	-299.990005255553\\
61.625	0.1074	-312.051840038755\\
61.625	0.108	-324.113674821957\\
61.625	0.1086	-336.17550960516\\
61.625	0.1092	-348.237344388362\\
61.625	0.1098	-360.299179171565\\
61.625	0.1104	-372.361013954767\\
61.625	0.111	-384.422848737971\\
61.625	0.1116	-396.484683521173\\
61.625	0.1122	-408.546518304375\\
61.625	0.1128	-420.608353087578\\
61.625	0.1134	-432.67018787078\\
61.625	0.114	-444.732022653983\\
61.625	0.1146	-456.793857437186\\
61.625	0.1152	-468.855692220388\\
61.625	0.1158	-480.91752700359\\
61.625	0.1164	-492.979361786794\\
61.625	0.117	-505.041196569995\\
61.625	0.1176	-517.103031353198\\
61.625	0.1182	-529.164866136401\\
61.625	0.1188	-541.226700919603\\
61.625	0.1194	-553.288535702805\\
61.625	0.12	-565.350370486008\\
61.625	0.1206	-577.412205269211\\
61.625	0.1212	-589.474040052413\\
61.625	0.1218	-601.535874835616\\
61.625	0.1224	-613.597709618818\\
61.625	0.123	-625.659544402021\\
62	0.093	-19.159852516942\\
62	0.0936	-31.0144435278489\\
62	0.0942	-42.8690345387558\\
62	0.0948	-54.7236255496628\\
62	0.0954	-66.5782165605697\\
62	0.096	-78.4328075714766\\
62	0.0966	-90.2873985823844\\
62	0.0972	-102.14198959329\\
62	0.0978	-113.996580604197\\
62	0.0984	-125.851171615105\\
62	0.099	-137.705762626012\\
62	0.0996	-149.560353636919\\
62	0.1002	-161.414944647826\\
62	0.1008	-173.269535658733\\
62	0.1014	-185.124126669641\\
62	0.102	-196.978717680548\\
62	0.1026	-208.833308691454\\
62	0.1032	-220.687899702361\\
62	0.1038	-232.542490713268\\
62	0.1044	-244.397081724175\\
62	0.105	-256.251672735083\\
62	0.1056	-268.106263745989\\
62	0.1062	-279.960854756896\\
62	0.1068	-291.815445767805\\
62	0.1074	-303.670036778711\\
62	0.108	-315.524627789619\\
62	0.1086	-327.379218800525\\
62	0.1092	-339.233809811431\\
62	0.1098	-351.088400822339\\
62	0.1104	-362.942991833246\\
62	0.111	-374.797582844153\\
62	0.1116	-386.652173855061\\
62	0.1122	-398.506764865967\\
62	0.1128	-410.361355876874\\
62	0.1134	-422.215946887782\\
62	0.114	-434.070537898689\\
62	0.1146	-445.925128909596\\
62	0.1152	-457.779719920502\\
62	0.1158	-469.634310931409\\
62	0.1164	-481.488901942317\\
62	0.117	-493.343492953223\\
62	0.1176	-505.19808396413\\
62	0.1182	-517.052674975038\\
62	0.1188	-528.907265985945\\
62	0.1194	-540.761856996851\\
62	0.12	-552.616448007759\\
62	0.1206	-564.471039018666\\
62	0.1212	-576.325630029572\\
62	0.1218	-588.180221040479\\
62	0.1224	-600.034812051386\\
62	0.123	-611.889403062293\\
62.375	0.093	-15.7518997919906\\
62.375	0.0936	-27.3992470306021\\
62.375	0.0942	-39.0465942692135\\
62.375	0.0948	-50.6939415078241\\
62.375	0.0954	-62.3412887464365\\
62.375	0.096	-73.9886359850479\\
62.375	0.0966	-85.6359832236594\\
62.375	0.0972	-97.2833304622709\\
62.375	0.0978	-108.930677700882\\
62.375	0.0984	-120.578024939494\\
62.375	0.099	-132.225372178105\\
62.375	0.0996	-143.872719416718\\
62.375	0.1002	-155.520066655328\\
62.375	0.1008	-167.16741389394\\
62.375	0.1014	-178.814761132552\\
62.375	0.102	-190.462108371164\\
62.375	0.1026	-202.109455609774\\
62.375	0.1032	-213.756802848386\\
62.375	0.1038	-225.404150086998\\
62.375	0.1044	-237.051497325609\\
62.375	0.105	-248.698844564221\\
62.375	0.1056	-260.346191802832\\
62.375	0.1062	-271.993539041444\\
62.375	0.1068	-283.640886280056\\
62.375	0.1074	-295.288233518667\\
62.375	0.108	-306.935580757279\\
62.375	0.1086	-318.582927995891\\
62.375	0.1092	-330.230275234501\\
62.375	0.1098	-341.877622473113\\
62.375	0.1104	-353.524969711725\\
62.375	0.111	-365.172316950337\\
62.375	0.1116	-376.819664188948\\
62.375	0.1122	-388.467011427559\\
62.375	0.1128	-400.114358666171\\
62.375	0.1134	-411.761705904783\\
62.375	0.114	-423.409053143394\\
62.375	0.1146	-435.056400382006\\
62.375	0.1152	-446.703747620617\\
62.375	0.1158	-458.351094859228\\
62.375	0.1164	-469.99844209784\\
62.375	0.117	-481.645789336451\\
62.375	0.1176	-493.293136575063\\
62.375	0.1182	-504.940483813674\\
62.375	0.1188	-516.587831052287\\
62.375	0.1194	-528.235178290897\\
62.375	0.12	-539.882525529509\\
62.375	0.1206	-551.52987276812\\
62.375	0.1212	-563.177220006733\\
62.375	0.1218	-574.824567245343\\
62.375	0.1224	-586.471914483955\\
62.375	0.123	-598.119261722566\\
62.75	0.093	-12.3439470670392\\
62.75	0.0936	-23.7840505333552\\
62.75	0.0942	-35.2241539996712\\
62.75	0.0948	-46.6642574659863\\
62.75	0.0954	-58.1043609323033\\
62.75	0.096	-69.5444643986193\\
62.75	0.0966	-80.9845678649353\\
62.75	0.0972	-92.4246713312505\\
62.75	0.0978	-103.864774797566\\
62.75	0.0984	-115.304878263883\\
62.75	0.099	-126.744981730199\\
62.75	0.0996	-138.185085196516\\
62.75	0.1002	-149.625188662831\\
62.75	0.1008	-161.065292129147\\
62.75	0.1014	-172.505395595463\\
62.75	0.102	-183.94549906178\\
62.75	0.1026	-195.385602528095\\
62.75	0.1032	-206.825705994411\\
62.75	0.1038	-218.265809460727\\
62.75	0.1044	-229.705912927043\\
62.75	0.105	-241.14601639336\\
62.75	0.1056	-252.586119859675\\
62.75	0.1062	-264.026223325991\\
62.75	0.1068	-275.466326792308\\
62.75	0.1074	-286.906430258623\\
62.75	0.108	-298.34653372494\\
62.75	0.1086	-309.786637191256\\
62.75	0.1092	-321.226740657571\\
62.75	0.1098	-332.666844123887\\
62.75	0.1104	-344.106947590203\\
62.75	0.111	-355.54705105652\\
62.75	0.1116	-366.987154522836\\
62.75	0.1122	-378.427257989151\\
62.75	0.1128	-389.867361455467\\
62.75	0.1134	-401.307464921783\\
62.75	0.114	-412.7475683881\\
62.75	0.1146	-424.187671854416\\
62.75	0.1152	-435.627775320731\\
62.75	0.1158	-447.067878787047\\
62.75	0.1164	-458.507982253363\\
62.75	0.117	-469.948085719679\\
62.75	0.1176	-481.388189185996\\
62.75	0.1182	-492.828292652312\\
62.75	0.1188	-504.268396118628\\
62.75	0.1194	-515.708499584943\\
62.75	0.12	-527.148603051259\\
62.75	0.1206	-538.588706517576\\
62.75	0.1212	-550.028809983892\\
62.75	0.1218	-561.468913450207\\
62.75	0.1224	-572.909016916523\\
62.75	0.123	-584.349120382839\\
63.125	0.093	-8.93599434208772\\
63.125	0.0936	-20.1688540361083\\
63.125	0.0942	-31.4017137301289\\
63.125	0.0948	-42.6345734241486\\
63.125	0.0954	-53.8674331181692\\
63.125	0.096	-65.1002928121898\\
63.125	0.0966	-76.3331525062113\\
63.125	0.0972	-87.5660122002309\\
63.125	0.0978	-98.7988718942515\\
63.125	0.0984	-110.031731588272\\
63.125	0.099	-121.264591282293\\
63.125	0.0996	-132.497450976313\\
63.125	0.1002	-143.730310670333\\
63.125	0.1008	-154.963170364354\\
63.125	0.1014	-166.196030058374\\
63.125	0.102	-177.428889752395\\
63.125	0.1026	-188.661749446415\\
63.125	0.1032	-199.894609140436\\
63.125	0.1038	-211.127468834457\\
63.125	0.1044	-222.360328528477\\
63.125	0.105	-233.593188222498\\
63.125	0.1056	-244.826047916517\\
63.125	0.1062	-256.058907610538\\
63.125	0.1068	-267.291767304559\\
63.125	0.1074	-278.524626998579\\
63.125	0.108	-289.757486692601\\
63.125	0.1086	-300.990346386621\\
63.125	0.1092	-312.223206080641\\
63.125	0.1098	-323.456065774662\\
63.125	0.1104	-334.688925468682\\
63.125	0.111	-345.921785162703\\
63.125	0.1116	-357.154644856723\\
63.125	0.1122	-368.387504550743\\
63.125	0.1128	-379.620364244764\\
63.125	0.1134	-390.853223938785\\
63.125	0.114	-402.086083632806\\
63.125	0.1146	-413.318943326826\\
63.125	0.1152	-424.551803020846\\
63.125	0.1158	-435.784662714866\\
63.125	0.1164	-447.017522408887\\
63.125	0.117	-458.250382102907\\
63.125	0.1176	-469.483241796927\\
63.125	0.1182	-480.716101490948\\
63.125	0.1188	-491.948961184969\\
63.125	0.1194	-503.181820878989\\
63.125	0.12	-514.41468057301\\
63.125	0.1206	-525.64754026703\\
63.125	0.1212	-536.880399961051\\
63.125	0.1218	-548.113259655071\\
63.125	0.1224	-559.346119349091\\
63.125	0.123	-570.578979043112\\
63.5	0.093	-5.5280416171363\\
63.5	0.0936	-16.5536575388614\\
63.5	0.0942	-27.5792734605866\\
63.5	0.0948	-38.6048893823108\\
63.5	0.0954	-49.630505304036\\
63.5	0.096	-60.6561212257611\\
63.5	0.0966	-71.6817371474863\\
63.5	0.0972	-82.7073530692105\\
63.5	0.0978	-93.7329689909357\\
63.5	0.0984	-104.758584912661\\
63.5	0.099	-115.784200834386\\
63.5	0.0996	-126.809816756111\\
63.5	0.1002	-137.835432677835\\
63.5	0.1008	-148.86104859956\\
63.5	0.1014	-159.886664521286\\
63.5	0.102	-170.912280443011\\
63.5	0.1026	-181.937896364735\\
63.5	0.1032	-192.96351228646\\
63.5	0.1038	-203.989128208185\\
63.5	0.1044	-215.01474412991\\
63.5	0.105	-226.040360051636\\
63.5	0.1056	-237.065975973361\\
63.5	0.1062	-248.091591895086\\
63.5	0.1068	-259.117207816812\\
63.5	0.1074	-270.142823738536\\
63.5	0.108	-281.168439660261\\
63.5	0.1086	-292.194055581987\\
63.5	0.1092	-303.219671503711\\
63.5	0.1098	-314.245287425436\\
63.5	0.1104	-325.270903347161\\
63.5	0.111	-336.296519268886\\
63.5	0.1116	-347.322135190611\\
63.5	0.1122	-358.347751112336\\
63.5	0.1128	-369.373367034061\\
63.5	0.1134	-380.398982955786\\
63.5	0.114	-391.424598877511\\
63.5	0.1146	-402.450214799236\\
63.5	0.1152	-413.47583072096\\
63.5	0.1158	-424.501446642686\\
63.5	0.1164	-435.527062564411\\
63.5	0.117	-446.552678486135\\
63.5	0.1176	-457.57829440786\\
63.5	0.1182	-468.603910329585\\
63.5	0.1188	-479.62952625131\\
63.5	0.1194	-490.655142173035\\
63.5	0.12	-501.68075809476\\
63.5	0.1206	-512.706374016485\\
63.5	0.1212	-523.73198993821\\
63.5	0.1218	-534.757605859934\\
63.5	0.1224	-545.783221781659\\
63.5	0.123	-556.808837703385\\
63.875	0.093	-2.12008889218487\\
63.875	0.0936	-12.9384610416146\\
63.875	0.0942	-23.7568331910443\\
63.875	0.0948	-34.5752053404731\\
63.875	0.0954	-45.3935774899028\\
63.875	0.096	-56.2119496393325\\
63.875	0.0966	-67.0303217887622\\
63.875	0.0972	-77.848693938191\\
63.875	0.0978	-88.6670660876207\\
63.875	0.0984	-99.4854382370504\\
63.875	0.099	-110.30381038648\\
63.875	0.0996	-121.122182535909\\
63.875	0.1002	-131.940554685338\\
63.875	0.1008	-142.758926834767\\
63.875	0.1014	-153.577298984197\\
63.875	0.102	-164.395671133627\\
63.875	0.1026	-175.214043283056\\
63.875	0.1032	-186.032415432485\\
63.875	0.1038	-196.850787581915\\
63.875	0.1044	-207.669159731345\\
63.875	0.105	-218.487531880774\\
63.875	0.1056	-229.305904030203\\
63.875	0.1062	-240.124276179633\\
63.875	0.1068	-250.942648329064\\
63.875	0.1074	-261.761020478492\\
63.875	0.108	-272.579392627922\\
63.875	0.1086	-283.397764777352\\
63.875	0.1092	-294.21613692678\\
63.875	0.1098	-305.034509076209\\
63.875	0.1104	-315.852881225639\\
63.875	0.111	-326.671253375069\\
63.875	0.1116	-337.489625524498\\
63.875	0.1122	-348.307997673927\\
63.875	0.1128	-359.126369823357\\
63.875	0.1134	-369.944741972787\\
63.875	0.114	-380.763114122216\\
63.875	0.1146	-391.581486271646\\
63.875	0.1152	-402.399858421075\\
63.875	0.1158	-413.218230570505\\
63.875	0.1164	-424.036602719934\\
63.875	0.117	-434.854974869363\\
63.875	0.1176	-445.673347018793\\
63.875	0.1182	-456.491719168223\\
63.875	0.1188	-467.310091317652\\
63.875	0.1194	-478.12846346708\\
63.875	0.12	-488.94683561651\\
63.875	0.1206	-499.76520776594\\
63.875	0.1212	-510.583579915369\\
63.875	0.1218	-521.401952064798\\
63.875	0.1224	-532.220324214228\\
63.875	0.123	-543.038696363657\\
64.25	0.093	1.28786383276656\\
64.25	0.0936	-9.3232645443677\\
64.25	0.0942	-19.934392921502\\
64.25	0.0948	-30.5455212986353\\
64.25	0.0954	-41.1566496757696\\
64.25	0.096	-51.7677780529029\\
64.25	0.0966	-62.3789064300372\\
64.25	0.0972	-72.9900348071706\\
64.25	0.0978	-83.6011631843048\\
64.25	0.0984	-94.2122915614391\\
64.25	0.099	-104.823419938573\\
64.25	0.0996	-115.434548315708\\
64.25	0.1002	-126.04567669284\\
64.25	0.1008	-136.656805069974\\
64.25	0.1014	-147.267933447109\\
64.25	0.102	-157.879061824243\\
64.25	0.1026	-168.490190201376\\
64.25	0.1032	-179.10131857851\\
64.25	0.1038	-189.712446955645\\
64.25	0.1044	-200.323575332778\\
64.25	0.105	-210.934703709912\\
64.25	0.1056	-221.545832087046\\
64.25	0.1062	-232.15696046418\\
64.25	0.1068	-242.768088841315\\
64.25	0.1074	-253.379217218449\\
64.25	0.108	-263.990345595583\\
64.25	0.1086	-274.601473972717\\
64.25	0.1092	-285.212602349849\\
64.25	0.1098	-295.823730726984\\
64.25	0.1104	-306.434859104118\\
64.25	0.111	-317.045987481252\\
64.25	0.1116	-327.657115858387\\
64.25	0.1122	-338.26824423552\\
64.25	0.1128	-348.879372612654\\
64.25	0.1134	-359.490500989788\\
64.25	0.114	-370.101629366922\\
64.25	0.1146	-380.712757744056\\
64.25	0.1152	-391.323886121189\\
64.25	0.1158	-401.935014498324\\
64.25	0.1164	-412.546142875458\\
64.25	0.117	-423.15727125259\\
64.25	0.1176	-433.768399629725\\
64.25	0.1182	-444.379528006859\\
64.25	0.1188	-454.990656383993\\
64.25	0.1194	-465.601784761127\\
64.25	0.12	-476.212913138261\\
64.25	0.1206	-486.824041515395\\
64.25	0.1212	-497.435169892529\\
64.25	0.1218	-508.046298269662\\
64.25	0.1224	-518.657426646796\\
64.25	0.123	-529.26855502393\\
64.625	0.093	4.69581655771799\\
64.625	0.0936	-5.70806804712083\\
64.625	0.0942	-16.1119526519597\\
64.625	0.0948	-26.5158372567976\\
64.625	0.0954	-36.9197218616355\\
64.625	0.096	-47.3236064664743\\
64.625	0.0966	-57.7274910713131\\
64.625	0.0972	-68.131375676151\\
64.625	0.0978	-78.535260280989\\
64.625	0.0984	-88.9391448858278\\
64.625	0.099	-99.3430294906666\\
64.625	0.0996	-109.746914095505\\
64.625	0.1002	-120.150798700343\\
64.625	0.1008	-130.554683305181\\
64.625	0.1014	-140.95856791002\\
64.625	0.102	-151.362452514859\\
64.625	0.1026	-161.766337119697\\
64.625	0.1032	-172.170221724535\\
64.625	0.1038	-182.574106329374\\
64.625	0.1044	-192.977990934212\\
64.625	0.105	-203.381875539051\\
64.625	0.1056	-213.785760143889\\
64.625	0.1062	-224.189644748727\\
64.625	0.1068	-234.593529353567\\
64.625	0.1074	-244.997413958405\\
64.625	0.108	-255.401298563243\\
64.625	0.1086	-265.805183168081\\
64.625	0.1092	-276.209067772919\\
64.625	0.1098	-286.612952377758\\
64.625	0.1104	-297.016836982597\\
64.625	0.111	-307.420721587436\\
64.625	0.1116	-317.824606192274\\
64.625	0.1122	-328.228490797112\\
64.625	0.1128	-338.63237540195\\
64.625	0.1134	-349.036260006789\\
64.625	0.114	-359.440144611627\\
64.625	0.1146	-369.844029216466\\
64.625	0.1152	-380.247913821304\\
64.625	0.1158	-390.651798426143\\
64.625	0.1164	-401.055683030982\\
64.625	0.117	-411.459567635819\\
64.625	0.1176	-421.863452240657\\
64.625	0.1182	-432.267336845496\\
64.625	0.1188	-442.671221450335\\
64.625	0.1194	-453.075106055172\\
64.625	0.12	-463.478990660011\\
64.625	0.1206	-473.88287526485\\
64.625	0.1212	-484.286759869688\\
64.625	0.1218	-494.690644474525\\
64.625	0.1224	-505.094529079364\\
64.625	0.123	-515.498413684203\\
65	0.093	8.10376928266942\\
65	0.0936	-2.09287154987396\\
65	0.0942	-12.2895123824173\\
65	0.0948	-22.4861532149589\\
65	0.0954	-32.6827940475023\\
65	0.096	-42.8794348800457\\
65	0.0966	-53.076075712589\\
65	0.0972	-63.2727165451306\\
65	0.0978	-73.469357377674\\
65	0.0984	-83.6659982102174\\
65	0.099	-93.8626390427598\\
65	0.0996	-104.059279875303\\
65	0.1002	-114.255920707846\\
65	0.1008	-124.452561540388\\
65	0.1014	-134.649202372932\\
65	0.102	-144.845843205475\\
65	0.1026	-155.042484038016\\
65	0.1032	-165.23912487056\\
65	0.1038	-175.435765703103\\
65	0.1044	-185.632406535647\\
65	0.105	-195.829047368189\\
65	0.1056	-206.025688200732\\
65	0.1062	-216.222329033275\\
65	0.1068	-226.418969865818\\
65	0.1074	-236.615610698361\\
65	0.108	-246.812251530904\\
65	0.1086	-257.008892363447\\
65	0.1092	-267.205533195989\\
65	0.1098	-277.402174028532\\
65	0.1104	-287.598814861075\\
65	0.111	-297.795455693618\\
65	0.1116	-307.992096526162\\
65	0.1122	-318.188737358704\\
65	0.1128	-328.385378191247\\
65	0.1134	-338.58201902379\\
65	0.114	-348.778659856333\\
65	0.1146	-358.975300688876\\
65	0.1152	-369.171941521418\\
65	0.1158	-379.368582353962\\
65	0.1164	-389.565223186504\\
65	0.117	-399.761864019047\\
65	0.1176	-409.95850485159\\
65	0.1182	-420.155145684133\\
65	0.1188	-430.351786516676\\
65	0.1194	-440.548427349218\\
65	0.12	-450.745068181762\\
65	0.1206	-460.941709014304\\
65	0.1212	-471.138349846848\\
65	0.1218	-481.33499067939\\
65	0.1224	-491.531631511933\\
65	0.123	-501.728272344476\\
65.375	0.093	11.5117220076208\\
65.375	0.0936	1.52232494737291\\
65.375	0.0942	-8.46707211287503\\
65.375	0.0948	-18.4564691731211\\
65.375	0.0954	-28.4458662333691\\
65.375	0.096	-38.435263293617\\
65.375	0.0966	-48.4246603538641\\
65.375	0.0972	-58.4140574141111\\
65.375	0.0978	-68.4034544743581\\
65.375	0.0984	-78.3928515346061\\
65.375	0.099	-88.382248594854\\
65.375	0.0996	-98.371645655101\\
65.375	0.1002	-108.361042715348\\
65.375	0.1008	-118.350439775595\\
65.375	0.1014	-128.339836835843\\
65.375	0.102	-138.329233896091\\
65.375	0.1026	-148.318630956337\\
65.375	0.1032	-158.308028016585\\
65.375	0.1038	-168.297425076832\\
65.375	0.1044	-178.28682213708\\
65.375	0.105	-188.276219197328\\
65.375	0.1056	-198.265616257574\\
65.375	0.1062	-208.255013317822\\
65.375	0.1068	-218.244410378071\\
65.375	0.1074	-228.233807438317\\
65.375	0.108	-238.223204498565\\
65.375	0.1086	-248.212601558812\\
65.375	0.1092	-258.201998619059\\
65.375	0.1098	-268.191395679307\\
65.375	0.1104	-278.180792739554\\
65.375	0.111	-288.170189799802\\
65.375	0.1116	-298.159586860049\\
65.375	0.1122	-308.148983920296\\
65.375	0.1128	-318.138380980544\\
65.375	0.1134	-328.127778040791\\
65.375	0.114	-338.117175101039\\
65.375	0.1146	-348.106572161286\\
65.375	0.1152	-358.095969221533\\
65.375	0.1158	-368.085366281781\\
65.375	0.1164	-378.074763342028\\
65.375	0.117	-388.064160402275\\
65.375	0.1176	-398.053557462522\\
65.375	0.1182	-408.04295452277\\
65.375	0.1188	-418.032351583018\\
65.375	0.1194	-428.021748643264\\
65.375	0.12	-438.011145703512\\
65.375	0.1206	-448.00054276376\\
65.375	0.1212	-457.989939824007\\
65.375	0.1218	-467.979336884254\\
65.375	0.1224	-477.968733944501\\
65.375	0.123	-487.958131004749\\
65.75	0.093	14.9196747325714\\
65.75	0.0936	5.13752144461978\\
65.75	0.0942	-4.64463184333272\\
65.75	0.0948	-14.4267851312834\\
65.75	0.0954	-24.2089384192359\\
65.75	0.096	-33.9910917071875\\
65.75	0.0966	-43.77324499514\\
65.75	0.0972	-53.5553982830907\\
65.75	0.0978	-63.3375515710432\\
65.75	0.0984	-73.1197048589947\\
65.75	0.099	-82.9018581469472\\
65.75	0.0996	-92.6840114348988\\
65.75	0.1002	-102.46616472285\\
65.75	0.1008	-112.248318010802\\
65.75	0.1014	-122.030471298754\\
65.75	0.102	-131.812624586706\\
65.75	0.1026	-141.594777874658\\
65.75	0.1032	-151.37693116261\\
65.75	0.1038	-161.159084450562\\
65.75	0.1044	-170.941237738514\\
65.75	0.105	-180.723391026466\\
65.75	0.1056	-190.505544314417\\
65.75	0.1062	-200.287697602369\\
65.75	0.1068	-210.069850890322\\
65.75	0.1074	-219.852004178273\\
65.75	0.108	-229.634157466226\\
65.75	0.1086	-239.416310754177\\
65.75	0.1092	-249.198464042129\\
65.75	0.1098	-258.98061733008\\
65.75	0.1104	-268.762770618033\\
65.75	0.111	-278.544923905984\\
65.75	0.1116	-288.327077193937\\
65.75	0.1122	-298.109230481888\\
65.75	0.1128	-307.89138376984\\
65.75	0.1134	-317.673537057792\\
65.75	0.114	-327.455690345744\\
65.75	0.1146	-337.237843633697\\
65.75	0.1152	-347.019996921647\\
65.75	0.1158	-356.8021502096\\
65.75	0.1164	-366.584303497551\\
65.75	0.117	-376.366456785503\\
65.75	0.1176	-386.148610073455\\
65.75	0.1182	-395.930763361407\\
65.75	0.1188	-405.712916649359\\
65.75	0.1194	-415.49506993731\\
65.75	0.12	-425.277223225262\\
65.75	0.1206	-435.059376513214\\
65.75	0.1212	-444.841529801166\\
65.75	0.1218	-454.623683089118\\
65.75	0.1224	-464.405836377069\\
65.75	0.123	-474.187989665022\\
66.125	0.093	18.3276274575228\\
66.125	0.0936	8.75271794186665\\
66.125	0.0942	-0.822191573790406\\
66.125	0.0948	-10.3971010894456\\
66.125	0.0954	-19.9720106051018\\
66.125	0.096	-29.5469201207588\\
66.125	0.0966	-39.121829636415\\
66.125	0.0972	-48.6967391520711\\
66.125	0.0978	-58.2716486677273\\
66.125	0.0984	-67.8465581833843\\
66.125	0.099	-77.4214676990405\\
66.125	0.0996	-86.9963772146975\\
66.125	0.1002	-96.5712867303528\\
66.125	0.1008	-106.146196246009\\
66.125	0.1014	-115.721105761666\\
66.125	0.102	-125.296015277322\\
66.125	0.1026	-134.870924792978\\
66.125	0.1032	-144.445834308634\\
66.125	0.1038	-154.020743824291\\
66.125	0.1044	-163.595653339948\\
66.125	0.105	-173.170562855605\\
66.125	0.1056	-182.74547237126\\
66.125	0.1062	-192.320381886916\\
66.125	0.1068	-201.895291402574\\
66.125	0.1074	-211.470200918229\\
66.125	0.108	-221.045110433886\\
66.125	0.1086	-230.620019949542\\
66.125	0.1092	-240.194929465199\\
66.125	0.1098	-249.769838980855\\
66.125	0.1104	-259.344748496512\\
66.125	0.111	-268.919658012168\\
66.125	0.1116	-278.494567527824\\
66.125	0.1122	-288.06947704348\\
66.125	0.1128	-297.644386559136\\
66.125	0.1134	-307.219296074793\\
66.125	0.114	-316.79420559045\\
66.125	0.1146	-326.369115106107\\
66.125	0.1152	-335.944024621762\\
66.125	0.1158	-345.518934137419\\
66.125	0.1164	-355.093843653075\\
66.125	0.117	-364.66875316873\\
66.125	0.1176	-374.243662684387\\
66.125	0.1182	-383.818572200043\\
66.125	0.1188	-393.393481715701\\
66.125	0.1194	-402.968391231356\\
66.125	0.12	-412.543300747013\\
66.125	0.1206	-422.118210262669\\
66.125	0.1212	-431.693119778326\\
66.125	0.1218	-441.268029293981\\
66.125	0.1224	-450.842938809637\\
66.125	0.123	-460.417848325294\\
66.5	0.093	21.7355801824742\\
66.5	0.0936	12.3679144391135\\
66.5	0.0942	3.00024869575191\\
66.5	0.0948	-6.36741704760789\\
66.5	0.0954	-15.7350827909686\\
66.5	0.096	-25.1027485343302\\
66.5	0.0966	-34.4704142776909\\
66.5	0.0972	-43.8380800210507\\
66.5	0.0978	-53.2057457644123\\
66.5	0.0984	-62.573411507773\\
66.5	0.099	-71.9410772511337\\
66.5	0.0996	-81.3087429944953\\
66.5	0.1002	-90.6764087378551\\
66.5	0.1008	-100.044074481216\\
66.5	0.1014	-109.411740224577\\
66.5	0.102	-118.779405967938\\
66.5	0.1026	-128.147071711299\\
66.5	0.1032	-137.51473745466\\
66.5	0.1038	-146.88240319802\\
66.5	0.1044	-156.250068941382\\
66.5	0.105	-165.617734684743\\
66.5	0.1056	-174.985400428102\\
66.5	0.1062	-184.353066171464\\
66.5	0.1068	-193.720731914826\\
66.5	0.1074	-203.088397658185\\
66.5	0.108	-212.456063401547\\
66.5	0.1086	-221.823729144908\\
66.5	0.1092	-231.191394888267\\
66.5	0.1098	-240.559060631629\\
66.5	0.1104	-249.92672637499\\
66.5	0.111	-259.294392118351\\
66.5	0.1116	-268.662057861712\\
66.5	0.1122	-278.029723605072\\
66.5	0.1128	-287.397389348434\\
66.5	0.1134	-296.765055091794\\
66.5	0.114	-306.132720835155\\
66.5	0.1146	-315.500386578517\\
66.5	0.1152	-324.868052321876\\
66.5	0.1158	-334.235718065237\\
66.5	0.1164	-343.603383808599\\
66.5	0.117	-352.971049551958\\
66.5	0.1176	-362.33871529532\\
66.5	0.1182	-371.706381038681\\
66.5	0.1188	-381.074046782041\\
66.5	0.1194	-390.441712525402\\
66.5	0.12	-399.809378268763\\
66.5	0.1206	-409.177044012124\\
66.5	0.1212	-418.544709755485\\
66.5	0.1218	-427.912375498845\\
66.5	0.1224	-437.280041242206\\
66.5	0.123	-446.647706985567\\
66.875	0.093	25.1435329074257\\
66.875	0.0936	15.9831109363604\\
66.875	0.0942	6.82268896529513\\
66.875	0.0948	-2.33773300577013\\
66.875	0.0954	-11.4981549768354\\
66.875	0.096	-20.6585769479007\\
66.875	0.0966	-29.8189989189668\\
66.875	0.0972	-38.9794208900312\\
66.875	0.0978	-48.1398428610964\\
66.875	0.0984	-57.3002648321617\\
66.875	0.099	-66.4606868032279\\
66.875	0.0996	-75.6211087742931\\
66.875	0.1002	-84.7815307453575\\
66.875	0.1008	-93.9419527164237\\
66.875	0.1014	-103.102374687489\\
66.875	0.102	-112.262796658554\\
66.875	0.1026	-121.423218629619\\
66.875	0.1032	-130.583640600685\\
66.875	0.1038	-139.74406257175\\
66.875	0.1044	-148.904484542815\\
66.875	0.105	-158.06490651388\\
66.875	0.1056	-167.225328484946\\
66.875	0.1062	-176.385750456011\\
66.875	0.1068	-185.546172427077\\
66.875	0.1074	-194.706594398142\\
66.875	0.108	-203.867016369208\\
66.875	0.1086	-213.027438340273\\
66.875	0.1092	-222.187860311337\\
66.875	0.1098	-231.348282282403\\
66.875	0.1104	-240.508704253469\\
66.875	0.111	-249.669126224534\\
66.875	0.1116	-258.8295481956\\
66.875	0.1122	-267.989970166665\\
66.875	0.1128	-277.15039213773\\
66.875	0.1134	-286.310814108795\\
66.875	0.114	-295.471236079861\\
66.875	0.1146	-304.631658050926\\
66.875	0.1152	-313.792080021991\\
66.875	0.1158	-322.952501993056\\
66.875	0.1164	-332.112923964122\\
66.875	0.117	-341.273345935187\\
66.875	0.1176	-350.433767906252\\
66.875	0.1182	-359.594189877318\\
66.875	0.1188	-368.754611848383\\
66.875	0.1194	-377.915033819448\\
66.875	0.12	-387.075455790513\\
66.875	0.1206	-396.235877761579\\
66.875	0.1212	-405.396299732644\\
66.875	0.1218	-414.556721703709\\
66.875	0.1224	-423.717143674774\\
66.875	0.123	-432.87756564584\\
67.25	0.093	28.5514856323771\\
67.25	0.0936	19.5983074336073\\
67.25	0.0942	10.6451292348374\\
67.25	0.0948	1.69195103606762\\
67.25	0.0954	-7.2612271627022\\
67.25	0.096	-16.214405361472\\
67.25	0.0966	-25.1675835602418\\
67.25	0.0972	-34.1207617590107\\
67.25	0.0978	-43.0739399577815\\
67.25	0.0984	-52.0271181565513\\
67.25	0.099	-60.9802963553211\\
67.25	0.0996	-69.9334745540909\\
67.25	0.1002	-78.8866527528598\\
67.25	0.1008	-87.8398309516306\\
67.25	0.1014	-96.7930091504004\\
67.25	0.102	-105.74618734917\\
67.25	0.1026	-114.699365547939\\
67.25	0.1032	-123.652543746709\\
67.25	0.1038	-132.60572194548\\
67.25	0.1044	-141.558900144249\\
67.25	0.105	-150.512078343019\\
67.25	0.1056	-159.465256541788\\
67.25	0.1062	-168.418434740558\\
67.25	0.1068	-177.371612939329\\
67.25	0.1074	-186.324791138099\\
67.25	0.108	-195.277969336868\\
67.25	0.1086	-204.231147535638\\
67.25	0.1092	-213.184325734407\\
67.25	0.1098	-222.137503933177\\
67.25	0.1104	-231.090682131948\\
67.25	0.111	-240.043860330718\\
67.25	0.1116	-248.997038529487\\
67.25	0.1122	-257.950216728256\\
67.25	0.1128	-266.903394927026\\
67.25	0.1134	-275.856573125797\\
67.25	0.114	-284.809751324567\\
67.25	0.1146	-293.762929523336\\
67.25	0.1152	-302.716107722105\\
67.25	0.1158	-311.669285920875\\
67.25	0.1164	-320.622464119646\\
67.25	0.117	-329.575642318415\\
67.25	0.1176	-338.528820517185\\
67.25	0.1182	-347.481998715954\\
67.25	0.1188	-356.435176914724\\
67.25	0.1194	-365.388355113493\\
67.25	0.12	-374.341533312264\\
67.25	0.1206	-383.294711511034\\
67.25	0.1212	-392.247889709804\\
67.25	0.1218	-401.201067908572\\
67.25	0.1224	-410.154246107342\\
67.25	0.123	-419.107424306113\\
67.625	0.093	31.9594383573285\\
67.625	0.0936	23.2135039308541\\
67.625	0.0942	14.4675695043798\\
67.625	0.0948	5.72163507790629\\
67.625	0.0954	-3.024299348569\\
67.625	0.096	-11.7702337750434\\
67.625	0.0966	-20.5161682015178\\
67.625	0.0972	-29.2621026279912\\
67.625	0.0978	-38.0080370544656\\
67.625	0.0984	-46.75397148094\\
67.625	0.099	-55.4999059074144\\
67.625	0.0996	-64.2458403338887\\
67.625	0.1002	-72.9917747603622\\
67.625	0.1008	-81.7377091868375\\
67.625	0.1014	-90.4836436133119\\
67.625	0.102	-99.2295780397862\\
67.625	0.1026	-107.97551246626\\
67.625	0.1032	-116.721446892734\\
67.625	0.1038	-125.467381319208\\
67.625	0.1044	-134.213315745683\\
67.625	0.105	-142.959250172157\\
67.625	0.1056	-151.705184598631\\
67.625	0.1062	-160.451119025106\\
67.625	0.1068	-169.197053451581\\
67.625	0.1074	-177.942987878055\\
67.625	0.108	-186.688922304529\\
67.625	0.1086	-195.434856731003\\
67.625	0.1092	-204.180791157477\\
67.625	0.1098	-212.926725583951\\
67.625	0.1104	-221.672660010426\\
67.625	0.111	-230.4185944369\\
67.625	0.1116	-239.164528863375\\
67.625	0.1122	-247.910463289849\\
67.625	0.1128	-256.656397716323\\
67.625	0.1134	-265.402332142798\\
67.625	0.114	-274.148266569272\\
67.625	0.1146	-282.894200995746\\
67.625	0.1152	-291.64013542222\\
67.625	0.1158	-300.386069848694\\
67.625	0.1164	-309.132004275169\\
67.625	0.117	-317.877938701643\\
67.625	0.1176	-326.623873128117\\
67.625	0.1182	-335.369807554592\\
67.625	0.1188	-344.115741981066\\
67.625	0.1194	-352.86167640754\\
67.625	0.12	-361.607610834014\\
67.625	0.1206	-370.353545260488\\
67.625	0.1212	-379.099479686963\\
67.625	0.1218	-387.845414113436\\
67.625	0.1224	-396.591348539911\\
67.625	0.123	-405.337282966386\\
68	0.093	35.3673910822799\\
68	0.0936	26.828700428101\\
68	0.0942	18.2900097739221\\
68	0.0948	9.75131911974404\\
68	0.0954	1.2126284655651\\
68	0.096	-7.32606218861383\\
68	0.0966	-15.8647528427928\\
68	0.0972	-24.4034434969708\\
68	0.0978	-32.9421341511497\\
68	0.0984	-41.4808248053287\\
68	0.099	-50.0195154595085\\
68	0.0996	-58.5582061136865\\
68	0.1002	-67.0968967678655\\
68	0.1008	-75.6355874220444\\
68	0.1014	-84.1742780762233\\
68	0.102	-92.7129687304023\\
68	0.1026	-101.25165938458\\
68	0.1032	-109.790350038759\\
68	0.1038	-118.329040692938\\
68	0.1044	-126.867731347117\\
68	0.105	-135.406422001296\\
68	0.1056	-143.945112655474\\
68	0.1062	-152.483803309653\\
68	0.1068	-161.022493963833\\
68	0.1074	-169.561184618011\\
68	0.108	-178.09987527219\\
68	0.1086	-186.638565926369\\
68	0.1092	-195.177256580547\\
68	0.1098	-203.715947234726\\
68	0.1104	-212.254637888905\\
68	0.111	-220.793328543084\\
68	0.1116	-229.332019197263\\
68	0.1122	-237.870709851441\\
68	0.1128	-246.409400505619\\
68	0.1134	-254.948091159798\\
68	0.114	-263.486781813977\\
68	0.1146	-272.025472468156\\
68	0.1152	-280.564163122334\\
68	0.1158	-289.102853776513\\
68	0.1164	-297.641544430692\\
68	0.117	-306.18023508487\\
68	0.1176	-314.718925739049\\
68	0.1182	-323.257616393228\\
68	0.1188	-331.796307047407\\
68	0.1194	-340.334997701585\\
68	0.12	-348.873688355764\\
68	0.1206	-357.412379009943\\
68	0.1212	-365.951069664123\\
68	0.1218	-374.4897603183\\
68	0.1224	-383.02845097248\\
68	0.123	-391.567141626659\\
68.375	0.093	38.7753438072314\\
68.375	0.0936	30.4438969253479\\
68.375	0.0942	22.1124500434644\\
68.375	0.0948	13.7810031615818\\
68.375	0.0954	5.4495562796983\\
68.375	0.096	-2.88189060218519\\
68.375	0.0966	-11.2133374840687\\
68.375	0.0972	-19.5447843659513\\
68.375	0.0978	-27.8762312478348\\
68.375	0.0984	-36.2076781297183\\
68.375	0.099	-44.5391250116018\\
68.375	0.0996	-52.8705718934852\\
68.375	0.1002	-61.2020187753678\\
68.375	0.1008	-69.5334656572513\\
68.375	0.1014	-77.8649125391348\\
68.375	0.102	-86.1963594210174\\
68.375	0.1026	-94.5278063029\\
68.375	0.1032	-102.859253184783\\
68.375	0.1038	-111.190700066667\\
68.375	0.1044	-119.52214694855\\
68.375	0.105	-127.853593830434\\
68.375	0.1056	-136.185040712317\\
68.375	0.1062	-144.5164875942\\
68.375	0.1068	-152.847934476084\\
68.375	0.1074	-161.179381357967\\
68.375	0.108	-169.510828239851\\
68.375	0.1086	-177.842275121734\\
68.375	0.1092	-186.173722003617\\
68.375	0.1098	-194.5051688855\\
68.375	0.1104	-202.836615767384\\
68.375	0.111	-211.168062649267\\
68.375	0.1116	-219.499509531151\\
68.375	0.1122	-227.830956413033\\
68.375	0.1128	-236.162403294916\\
68.375	0.1134	-244.493850176799\\
68.375	0.114	-252.825297058683\\
68.375	0.1146	-261.156743940566\\
68.375	0.1152	-269.488190822449\\
68.375	0.1158	-277.819637704332\\
68.375	0.1164	-286.151084586216\\
68.375	0.117	-294.482531468098\\
68.375	0.1176	-302.813978349982\\
68.375	0.1182	-311.145425231865\\
68.375	0.1188	-319.476872113749\\
68.375	0.1194	-327.808318995631\\
68.375	0.12	-336.139765877515\\
68.375	0.1206	-344.471212759398\\
68.375	0.1212	-352.802659641282\\
68.375	0.1218	-361.134106523165\\
68.375	0.1224	-369.465553405048\\
68.375	0.123	-377.797000286932\\
68.75	0.093	42.1832965321828\\
68.75	0.0936	34.0590934225947\\
68.75	0.0942	25.9348903130067\\
68.75	0.0948	17.8106872034195\\
68.75	0.0954	9.6864840938315\\
68.75	0.096	1.56228098424344\\
68.75	0.0966	-6.56192212534461\\
68.75	0.0972	-14.6861252349308\\
68.75	0.0978	-22.8103283445189\\
68.75	0.0984	-30.9345314541069\\
68.75	0.099	-39.058734563695\\
68.75	0.0996	-47.182937673283\\
68.75	0.1002	-55.3071407828702\\
68.75	0.1008	-63.4313438924582\\
68.75	0.1014	-71.5555470020454\\
68.75	0.102	-79.6797501116334\\
68.75	0.1026	-87.8039532212206\\
68.75	0.1032	-95.9281563308086\\
68.75	0.1038	-104.052359440397\\
68.75	0.1044	-112.176562549985\\
68.75	0.105	-120.300765659573\\
68.75	0.1056	-128.424968769159\\
68.75	0.1062	-136.549171878747\\
68.75	0.1068	-144.673374988336\\
68.75	0.1074	-152.797578097923\\
68.75	0.108	-160.921781207511\\
68.75	0.1086	-169.045984317099\\
68.75	0.1092	-177.170187426686\\
68.75	0.1098	-185.294390536274\\
68.75	0.1104	-193.418593645862\\
68.75	0.111	-201.54279675545\\
68.75	0.1116	-209.666999865038\\
68.75	0.1122	-217.791202974625\\
68.75	0.1128	-225.915406084213\\
68.75	0.1134	-234.039609193801\\
68.75	0.114	-242.163812303389\\
68.75	0.1146	-250.288015412976\\
68.75	0.1152	-258.412218522563\\
68.75	0.1158	-266.536421632151\\
68.75	0.1164	-274.660624741739\\
68.75	0.117	-282.784827851327\\
68.75	0.1176	-290.909030960915\\
68.75	0.1182	-299.033234070503\\
68.75	0.1188	-307.15743718009\\
68.75	0.1194	-315.281640289677\\
68.75	0.12	-323.405843399265\\
68.75	0.1206	-331.530046508853\\
68.75	0.1212	-339.654249618441\\
68.75	0.1218	-347.778452728028\\
68.75	0.1224	-355.902655837616\\
68.75	0.123	-364.026858947203\\
69.125	0.093	45.5912492571333\\
69.125	0.0936	37.6742899198416\\
69.125	0.0942	29.757330582549\\
69.125	0.0948	21.8403712452573\\
69.125	0.0954	13.9234119079647\\
69.125	0.096	6.00645257067299\\
69.125	0.0966	-1.91050676661962\\
69.125	0.0972	-9.82746610391132\\
69.125	0.0978	-17.7444254412039\\
69.125	0.0984	-25.6613847784965\\
69.125	0.099	-33.5783441157882\\
69.125	0.0996	-41.4953034530809\\
69.125	0.1002	-49.4122627903726\\
69.125	0.1008	-57.3292221276652\\
69.125	0.1014	-65.2461814649569\\
69.125	0.102	-73.1631408022495\\
69.125	0.1026	-81.0801001395412\\
69.125	0.1032	-88.9970594768338\\
69.125	0.1038	-96.9140188141264\\
69.125	0.1044	-104.830978151418\\
69.125	0.105	-112.747937488711\\
69.125	0.1056	-120.664896826002\\
69.125	0.1062	-128.581856163295\\
69.125	0.1068	-136.498815500588\\
69.125	0.1074	-144.415774837879\\
69.125	0.108	-152.332734175172\\
69.125	0.1086	-160.249693512465\\
69.125	0.1092	-168.166652849756\\
69.125	0.1098	-176.083612187048\\
69.125	0.1104	-184.000571524341\\
69.125	0.111	-191.917530861633\\
69.125	0.1116	-199.834490198926\\
69.125	0.1122	-207.751449536217\\
69.125	0.1128	-215.668408873509\\
69.125	0.1134	-223.585368210802\\
69.125	0.114	-231.502327548094\\
69.125	0.1146	-239.419286885387\\
69.125	0.1152	-247.336246222678\\
69.125	0.1158	-255.25320555997\\
69.125	0.1164	-263.170164897263\\
69.125	0.117	-271.087124234555\\
69.125	0.1176	-279.004083571846\\
69.125	0.1182	-286.921042909139\\
69.125	0.1188	-294.838002246432\\
69.125	0.1194	-302.754961583723\\
69.125	0.12	-310.671920921016\\
69.125	0.1206	-318.588880258308\\
69.125	0.1212	-326.5058395956\\
69.125	0.1218	-334.422798932892\\
69.125	0.1224	-342.339758270185\\
69.125	0.123	-350.256717607476\\
69.5	0.093	48.9992019820847\\
69.5	0.0936	41.2894864170885\\
69.5	0.0942	33.5797708520913\\
69.5	0.0948	25.8700552870951\\
69.5	0.0954	18.1603397220988\\
69.5	0.096	10.4506241571016\\
69.5	0.0966	2.74090859210446\\
69.5	0.0972	-4.96880697289089\\
69.5	0.0978	-12.6785225378881\\
69.5	0.0984	-20.3882381028852\\
69.5	0.099	-28.0979536678824\\
69.5	0.0996	-35.8076692328787\\
69.5	0.1002	-43.5173847978749\\
69.5	0.1008	-51.2271003628721\\
69.5	0.1014	-58.9368159278683\\
69.5	0.102	-66.6465314928655\\
69.5	0.1026	-74.3562470578618\\
69.5	0.1032	-82.065962622858\\
69.5	0.1038	-89.7756781878552\\
69.5	0.1044	-97.4853937528524\\
69.5	0.105	-105.19510931785\\
69.5	0.1056	-112.904824882845\\
69.5	0.1062	-120.614540447842\\
69.5	0.1068	-128.32425601284\\
69.5	0.1074	-136.033971577835\\
69.5	0.108	-143.743687142833\\
69.5	0.1086	-151.45340270783\\
69.5	0.1092	-159.163118272825\\
69.5	0.1098	-166.872833837822\\
69.5	0.1104	-174.582549402819\\
69.5	0.111	-182.292264967816\\
69.5	0.1116	-190.001980532813\\
69.5	0.1122	-197.711696097809\\
69.5	0.1128	-205.421411662806\\
69.5	0.1134	-213.131127227803\\
69.5	0.114	-220.8408427928\\
69.5	0.1146	-228.550558357797\\
69.5	0.1152	-236.260273922792\\
69.5	0.1158	-243.969989487789\\
69.5	0.1164	-251.679705052787\\
69.5	0.117	-259.389420617782\\
69.5	0.1176	-267.099136182779\\
69.5	0.1182	-274.808851747776\\
69.5	0.1188	-282.518567312773\\
69.5	0.1194	-290.228282877769\\
69.5	0.12	-297.937998442766\\
69.5	0.1206	-305.647714007763\\
69.5	0.1212	-313.357429572759\\
69.5	0.1218	-321.067145137756\\
69.5	0.1224	-328.776860702753\\
69.5	0.123	-336.486576267749\\
69.875	0.093	52.4071547070362\\
69.875	0.0936	44.9046829143354\\
69.875	0.0942	37.4022111216336\\
69.875	0.0948	29.8997393289328\\
69.875	0.0954	22.397267536232\\
69.875	0.096	14.8947957435303\\
69.875	0.0966	7.39232395082945\\
69.875	0.0972	-0.110147841871367\\
69.875	0.0978	-7.61261963457309\\
69.875	0.0984	-15.1150914272739\\
69.875	0.099	-22.6175632199756\\
69.875	0.0996	-30.1200350126765\\
69.875	0.1002	-37.6225068053773\\
69.875	0.1008	-45.124978598079\\
69.875	0.1014	-52.6274503907798\\
69.875	0.102	-60.1299221834815\\
69.875	0.1026	-67.6323939761824\\
69.875	0.1032	-75.1348657688832\\
69.875	0.1038	-82.6373375615849\\
69.875	0.1044	-90.1398093542857\\
69.875	0.105	-97.6422811469874\\
69.875	0.1056	-105.144752939688\\
69.875	0.1062	-112.647224732389\\
69.875	0.1068	-120.149696525092\\
69.875	0.1074	-127.652168317792\\
69.875	0.108	-135.154640110493\\
69.875	0.1086	-142.657111903195\\
69.875	0.1092	-150.159583695895\\
69.875	0.1098	-157.662055488597\\
69.875	0.1104	-165.164527281298\\
69.875	0.111	-172.666999073999\\
69.875	0.1116	-180.169470866701\\
69.875	0.1122	-187.671942659401\\
69.875	0.1128	-195.174414452103\\
69.875	0.1134	-202.676886244804\\
69.875	0.114	-210.179358037505\\
69.875	0.1146	-217.681829830207\\
69.875	0.1152	-225.184301622907\\
69.875	0.1158	-232.686773415609\\
69.875	0.1164	-240.18924520831\\
69.875	0.117	-247.69171700101\\
69.875	0.1176	-255.194188793712\\
69.875	0.1182	-262.696660586413\\
69.875	0.1188	-270.199132379114\\
69.875	0.1194	-277.701604171815\\
69.875	0.12	-285.204075964516\\
69.875	0.1206	-292.706547757218\\
69.875	0.1212	-300.209019549919\\
69.875	0.1218	-307.711491342619\\
69.875	0.1224	-315.213963135322\\
69.875	0.123	-322.716434928021\\
70.25	0.093	55.8151074319876\\
70.25	0.0936	48.5198794115822\\
70.25	0.0942	41.2246513911759\\
70.25	0.0948	33.9294233707715\\
70.25	0.0954	26.6341953503652\\
70.25	0.096	19.3389673299598\\
70.25	0.0966	12.0437393095535\\
70.25	0.0972	4.74851128914815\\
70.25	0.0978	-2.54671673125722\\
70.25	0.0984	-9.8419447516635\\
70.25	0.099	-17.1371727720689\\
70.25	0.0996	-24.4324007924752\\
70.25	0.1002	-31.7276288128796\\
70.25	0.1008	-39.0228568332859\\
70.25	0.1014	-46.3180848536913\\
70.25	0.102	-53.6133128740976\\
70.25	0.1026	-60.908540894502\\
70.25	0.1032	-68.2037689149083\\
70.25	0.1038	-75.4989969353137\\
70.25	0.1044	-82.79422495572\\
70.25	0.105	-90.0894529761254\\
70.25	0.1056	-97.3846809965307\\
70.25	0.1062	-104.679909016936\\
70.25	0.1068	-111.975137037343\\
70.25	0.1074	-119.270365057749\\
70.25	0.108	-126.565593078154\\
70.25	0.1086	-133.86082109856\\
70.25	0.1092	-141.156049118965\\
70.25	0.1098	-148.451277139371\\
70.25	0.1104	-155.746505159776\\
70.25	0.111	-163.041733180183\\
70.25	0.1116	-170.336961200588\\
70.25	0.1122	-177.632189220993\\
70.25	0.1128	-184.927417241399\\
70.25	0.1134	-192.222645261805\\
70.25	0.114	-199.517873282211\\
70.25	0.1146	-206.813101302617\\
70.25	0.1152	-214.108329323021\\
70.25	0.1158	-221.403557343428\\
70.25	0.1164	-228.698785363833\\
70.25	0.117	-235.994013384238\\
70.25	0.1176	-243.289241404644\\
70.25	0.1182	-250.58446942505\\
70.25	0.1188	-257.879697445456\\
70.25	0.1194	-265.174925465861\\
70.25	0.12	-272.470153486267\\
70.25	0.1206	-279.765381506672\\
70.25	0.1212	-287.060609527079\\
70.25	0.1218	-294.355837547483\\
70.25	0.1224	-301.651065567889\\
70.25	0.123	-308.946293588295\\
70.625	0.093	59.223060156939\\
70.625	0.0936	52.1350759088291\\
70.625	0.0942	45.0470916607183\\
70.625	0.0948	37.9591074126092\\
70.625	0.0954	30.8711231644984\\
70.625	0.096	23.7831389163885\\
70.625	0.0966	16.6951546682776\\
70.625	0.0972	9.60717042016859\\
70.625	0.0978	2.51918617205774\\
70.625	0.0984	-4.56879807605219\\
70.625	0.099	-11.6567823241621\\
70.625	0.0996	-18.744766572273\\
70.625	0.1002	-25.832750820382\\
70.625	0.1008	-32.9207350684928\\
70.625	0.1014	-40.0087193166028\\
70.625	0.102	-47.0967035647136\\
70.625	0.1026	-54.1846878128226\\
70.625	0.1032	-61.2726720609335\\
70.625	0.1038	-68.3606563090434\\
70.625	0.1044	-75.4486405571533\\
70.625	0.105	-82.5366248052642\\
70.625	0.1056	-89.6246090533732\\
70.625	0.1062	-96.712593301484\\
70.625	0.1068	-103.800577549595\\
70.625	0.1074	-110.888561797705\\
70.625	0.108	-117.976546045815\\
70.625	0.1086	-125.064530293926\\
70.625	0.1092	-132.152514542035\\
70.625	0.1098	-139.240498790145\\
70.625	0.1104	-146.328483038255\\
70.625	0.111	-153.416467286365\\
70.625	0.1116	-160.504451534476\\
70.625	0.1122	-167.592435782585\\
70.625	0.1128	-174.680420030696\\
70.625	0.1134	-181.768404278806\\
70.625	0.114	-188.856388526917\\
70.625	0.1146	-195.944372775027\\
70.625	0.1152	-203.032357023136\\
70.625	0.1158	-210.120341271247\\
70.625	0.1164	-217.208325519357\\
70.625	0.117	-224.296309767466\\
70.625	0.1176	-231.384294015576\\
70.625	0.1182	-238.472278263687\\
70.625	0.1188	-245.560262511797\\
70.625	0.1194	-252.648246759907\\
70.625	0.12	-259.736231008017\\
70.625	0.1206	-266.824215256127\\
70.625	0.1212	-273.912199504238\\
70.625	0.1218	-281.000183752348\\
70.625	0.1224	-288.088168000457\\
70.625	0.123	-295.176152248569\\
71	0.093	62.6310128818905\\
71	0.0936	55.750272406076\\
71	0.0942	48.8695319302606\\
71	0.0948	41.988791454447\\
71	0.0954	35.1080509786325\\
71	0.096	28.2273105028171\\
71	0.0966	21.3465700270026\\
71	0.0972	14.4658295511881\\
71	0.0978	7.58508907537362\\
71	0.0984	0.704348599559125\\
71	0.099	-6.17639187625628\\
71	0.0996	-13.0571323520708\\
71	0.1002	-19.9378728278843\\
71	0.1008	-26.8186133036997\\
71	0.1014	-33.6993537795142\\
71	0.102	-40.5800942553287\\
71	0.1026	-47.4608347311432\\
71	0.1032	-54.3415752069577\\
71	0.1038	-61.2223156827731\\
71	0.1044	-68.1030561585876\\
71	0.105	-74.9837966344021\\
71	0.1056	-81.8645371102166\\
71	0.1062	-88.7452775860311\\
71	0.1068	-95.6260180618465\\
71	0.1074	-102.506758537661\\
71	0.108	-109.387499013475\\
71	0.1086	-116.26823948929\\
71	0.1092	-123.148979965104\\
71	0.1098	-130.029720440919\\
71	0.1104	-136.910460916734\\
71	0.111	-143.791201392549\\
71	0.1116	-150.671941868363\\
71	0.1122	-157.552682344178\\
71	0.1128	-164.433422819992\\
71	0.1134	-171.314163295807\\
71	0.114	-178.194903771622\\
71	0.1146	-185.075644247437\\
71	0.1152	-191.95638472325\\
71	0.1158	-198.837125199066\\
71	0.1164	-205.71786567488\\
71	0.117	-212.598606150695\\
71	0.1176	-219.479346626509\\
71	0.1182	-226.360087102324\\
71	0.1188	-233.240827578139\\
71	0.1194	-240.121568053953\\
71	0.12	-247.002308529767\\
71	0.1206	-253.883049005583\\
71	0.1212	-260.763789481397\\
71	0.1218	-267.644529957211\\
71	0.1224	-274.525270433026\\
71	0.123	-281.40601090884\\
71.375	0.093	66.038965606841\\
71.375	0.0936	59.3654689033219\\
71.375	0.0942	52.6919721998029\\
71.375	0.0948	46.0184754962838\\
71.375	0.0954	39.3449787927648\\
71.375	0.096	32.6714820892457\\
71.375	0.0966	25.9979853857258\\
71.375	0.0972	19.3244886822076\\
71.375	0.0978	12.6509919786886\\
71.375	0.0984	5.97749527516953\\
71.375	0.099	-0.696001428350428\\
71.375	0.0996	-7.36949813186948\\
71.375	0.1002	-14.0429948353876\\
71.375	0.1008	-20.7164915389076\\
71.375	0.1014	-27.3899882424266\\
71.375	0.102	-34.0634849459457\\
71.375	0.1026	-40.7369816494638\\
71.375	0.1032	-47.4104783529838\\
71.375	0.1038	-54.0839750565028\\
71.375	0.1044	-60.7574717600219\\
71.375	0.105	-67.4309684635418\\
71.375	0.1056	-74.10446516706\\
71.375	0.1062	-80.777961870579\\
71.375	0.1068	-87.451458574099\\
71.375	0.1074	-94.124955277618\\
71.375	0.108	-100.798451981137\\
71.375	0.1086	-107.471948684656\\
71.375	0.1092	-114.145445388175\\
71.375	0.1098	-120.818942091694\\
71.375	0.1104	-127.492438795213\\
71.375	0.111	-134.165935498732\\
71.375	0.1116	-140.839432202252\\
71.375	0.1122	-147.51292890577\\
71.375	0.1128	-154.186425609289\\
71.375	0.1134	-160.859922312809\\
71.375	0.114	-167.533419016328\\
71.375	0.1146	-174.206915719848\\
71.375	0.1152	-180.880412423366\\
71.375	0.1158	-187.553909126886\\
71.375	0.1164	-194.227405830405\\
71.375	0.117	-200.900902533923\\
71.375	0.1176	-207.574399237443\\
71.375	0.1182	-214.247895940962\\
71.375	0.1188	-220.921392644481\\
71.375	0.1194	-227.594889347999\\
71.375	0.12	-234.268386051519\\
71.375	0.1206	-240.941882755038\\
71.375	0.1212	-247.615379458556\\
71.375	0.1218	-254.288876162074\\
71.375	0.1224	-260.962372865596\\
71.375	0.123	-267.635869569113\\
71.75	0.093	69.4469183317933\\
71.75	0.0936	62.9806654005697\\
71.75	0.0942	56.5144124693452\\
71.75	0.0948	50.0481595381225\\
71.75	0.0954	43.5819066068989\\
71.75	0.096	37.1156536756753\\
71.75	0.0966	30.6494007444517\\
71.75	0.0972	24.1831478132281\\
71.75	0.0978	17.7168948820045\\
71.75	0.0984	11.2506419507808\\
71.75	0.099	4.78438901955724\\
71.75	0.0996	-1.68186391166637\\
71.75	0.1002	-8.14811684288998\\
71.75	0.1008	-14.6143697741136\\
71.75	0.1014	-21.0806227053372\\
71.75	0.102	-27.5468756365608\\
71.75	0.1026	-34.0131285677835\\
71.75	0.1032	-40.479381499008\\
71.75	0.1038	-46.9456344302316\\
71.75	0.1044	-53.4118873614552\\
71.75	0.105	-59.8781402926788\\
71.75	0.1056	-66.3443932239015\\
71.75	0.1062	-72.8106461551261\\
71.75	0.1068	-79.2768990863506\\
71.75	0.1074	-85.7431520175733\\
71.75	0.108	-92.2094049487969\\
71.75	0.1086	-98.6756578800205\\
71.75	0.1092	-105.141910811244\\
71.75	0.1098	-111.608163742468\\
71.75	0.1104	-118.074416673691\\
71.75	0.111	-124.540669604915\\
71.75	0.1116	-131.006922536139\\
71.75	0.1122	-137.473175467362\\
71.75	0.1128	-143.939428398586\\
71.75	0.1134	-150.405681329809\\
71.75	0.114	-156.871934261033\\
71.75	0.1146	-163.338187192257\\
71.75	0.1152	-169.804440123479\\
71.75	0.1158	-176.270693054704\\
71.75	0.1164	-182.736945985927\\
71.75	0.117	-189.20319891715\\
71.75	0.1176	-195.669451848374\\
71.75	0.1182	-202.135704779597\\
71.75	0.1188	-208.601957710822\\
71.75	0.1194	-215.068210642045\\
71.75	0.12	-221.534463573268\\
71.75	0.1206	-228.000716504492\\
71.75	0.1212	-234.466969435715\\
71.75	0.1218	-240.933222366939\\
71.75	0.1224	-247.399475298163\\
71.75	0.123	-253.865728229386\\
72.125	0.093	72.8548710567438\\
72.125	0.0936	66.5958618978157\\
72.125	0.0942	60.3368527388875\\
72.125	0.0948	54.0778435799602\\
72.125	0.0954	47.8188344210312\\
72.125	0.096	41.559825262103\\
72.125	0.0966	35.3008161031748\\
72.125	0.0972	29.0418069442476\\
72.125	0.0978	22.7827977853194\\
72.125	0.0984	16.5237886263913\\
72.125	0.099	10.2647794674631\\
72.125	0.0996	4.00577030853492\\
72.125	0.1002	-2.25323885039325\\
72.125	0.1008	-8.51224800932141\\
72.125	0.1014	-14.7712571682496\\
72.125	0.102	-21.0302663271777\\
72.125	0.1026	-27.289275486105\\
72.125	0.1032	-33.5482846450332\\
72.125	0.1038	-39.8072938039613\\
72.125	0.1044	-46.0663029628895\\
72.125	0.105	-52.3253121218177\\
72.125	0.1056	-58.5843212807458\\
72.125	0.1062	-64.843330439674\\
72.125	0.1068	-71.1023395986031\\
72.125	0.1074	-77.3613487575303\\
72.125	0.108	-83.6203579164585\\
72.125	0.1086	-89.8793670753867\\
72.125	0.1092	-96.1383762343139\\
72.125	0.1098	-102.397385393242\\
72.125	0.1104	-108.65639455217\\
72.125	0.111	-114.915403711099\\
72.125	0.1116	-121.174412870027\\
72.125	0.1122	-127.433422028955\\
72.125	0.1128	-133.692431187883\\
72.125	0.1134	-139.951440346811\\
72.125	0.114	-146.210449505739\\
72.125	0.1146	-152.469458664667\\
72.125	0.1152	-158.728467823595\\
72.125	0.1158	-164.987476982524\\
72.125	0.1164	-171.246486141452\\
72.125	0.117	-177.505495300379\\
72.125	0.1176	-183.764504459307\\
72.125	0.1182	-190.023513618235\\
72.125	0.1188	-196.282522777164\\
72.125	0.1194	-202.541531936091\\
72.125	0.12	-208.800541095019\\
72.125	0.1206	-215.059550253946\\
72.125	0.1212	-221.318559412877\\
72.125	0.1218	-227.577568571804\\
72.125	0.1224	-233.836577730731\\
72.125	0.123	-240.095586889661\\
72.5	0.093	76.2628237816962\\
72.5	0.0936	70.2110583950634\\
72.5	0.0942	64.1592930084298\\
72.5	0.0948	58.107527621798\\
72.5	0.0954	52.0557622351653\\
72.5	0.096	46.0039968485326\\
72.5	0.0966	39.9522314618998\\
72.5	0.0972	33.900466075268\\
72.5	0.0978	27.8487006886353\\
72.5	0.0984	21.7969353020026\\
72.5	0.099	15.7451699153698\\
72.5	0.0996	9.69340452873712\\
72.5	0.1002	3.6416391421053\\
72.5	0.1008	-2.41012624452742\\
72.5	0.1014	-8.46189163116014\\
72.5	0.102	-14.5136570177929\\
72.5	0.1026	-20.5654224044247\\
72.5	0.1032	-26.6171877910574\\
72.5	0.1038	-32.6689531776901\\
72.5	0.1044	-38.7207185643229\\
72.5	0.105	-44.7724839509556\\
72.5	0.1056	-50.8242493375874\\
72.5	0.1062	-56.8760147242201\\
72.5	0.1068	-62.9277801108537\\
72.5	0.1074	-68.9795454974856\\
72.5	0.108	-75.0313108841183\\
72.5	0.1086	-81.083076270751\\
72.5	0.1092	-87.1348416573828\\
72.5	0.1098	-93.1866070440155\\
72.5	0.1104	-99.2383724306483\\
72.5	0.111	-105.290137817281\\
72.5	0.1116	-111.341903203914\\
72.5	0.1122	-117.393668590546\\
72.5	0.1128	-123.445433977178\\
72.5	0.1134	-129.497199363811\\
72.5	0.114	-135.548964750444\\
72.5	0.1146	-141.600730137076\\
72.5	0.1152	-147.652495523709\\
72.5	0.1158	-153.704260910342\\
72.5	0.1164	-159.756026296975\\
72.5	0.117	-165.807791683606\\
72.5	0.1176	-171.859557070239\\
72.5	0.1182	-177.911322456872\\
72.5	0.1188	-183.963087843505\\
72.5	0.1194	-190.014853230136\\
72.5	0.12	-196.066618616769\\
72.5	0.1206	-202.118384003402\\
72.5	0.1212	-208.170149390035\\
72.5	0.1218	-214.221914776666\\
72.5	0.1224	-220.273680163299\\
72.5	0.123	-226.325445549932\\
72.875	0.093	79.6707765066467\\
72.875	0.0936	73.8262548923094\\
72.875	0.0942	67.9817332779721\\
72.875	0.0948	62.1372116636358\\
72.875	0.0954	56.2926900492985\\
72.875	0.096	50.4481684349612\\
72.875	0.0966	44.6036468206239\\
72.875	0.0972	38.7591252062875\\
72.875	0.0978	32.9146035919503\\
72.875	0.0984	27.070081977613\\
72.875	0.099	21.2255603632757\\
72.875	0.0996	15.3810387489384\\
72.875	0.1002	9.53651713460204\\
72.875	0.1008	3.69199552026475\\
72.875	0.1014	-2.15252609407253\\
72.875	0.102	-7.99704770840981\\
72.875	0.1026	-13.8415693227462\\
72.875	0.1032	-19.6860909370826\\
72.875	0.1038	-25.5306125514198\\
72.875	0.1044	-31.3751341657571\\
72.875	0.105	-37.2196557800944\\
72.875	0.1056	-43.0641773944308\\
72.875	0.1062	-48.9086990087681\\
72.875	0.1068	-54.7532206231062\\
72.875	0.1074	-60.5977422374426\\
72.875	0.108	-66.4422638517799\\
72.875	0.1086	-72.2867854661172\\
72.875	0.1092	-78.1313070804536\\
72.875	0.1098	-83.9758286947908\\
72.875	0.1104	-89.8203503091281\\
72.875	0.111	-95.6648719234654\\
72.875	0.1116	-101.509393537803\\
72.875	0.1122	-107.353915152139\\
72.875	0.1128	-113.198436766476\\
72.875	0.1134	-119.042958380814\\
72.875	0.114	-124.887479995151\\
72.875	0.1146	-130.732001609488\\
72.875	0.1152	-136.576523223824\\
72.875	0.1158	-142.421044838161\\
72.875	0.1164	-148.265566452498\\
72.875	0.117	-154.110088066835\\
72.875	0.1176	-159.954609681172\\
72.875	0.1182	-165.799131295509\\
72.875	0.1188	-171.643652909846\\
72.875	0.1194	-177.488174524184\\
72.875	0.12	-183.332696138519\\
72.875	0.1206	-189.177217752858\\
72.875	0.1212	-195.021739367194\\
72.875	0.1218	-200.86626098153\\
72.875	0.1224	-206.710782595868\\
72.875	0.123	-212.555304210204\\
73.25	0.093	83.0787292315981\\
73.25	0.0936	77.4414513895563\\
73.25	0.0942	71.8041735475153\\
73.25	0.0948	66.1668957054744\\
73.25	0.0954	60.5296178634326\\
73.25	0.096	54.8923400213907\\
73.25	0.0966	49.2550621793489\\
73.25	0.0972	43.617784337308\\
73.25	0.0978	37.9805064952661\\
73.25	0.0984	32.3432286532252\\
73.25	0.099	26.7059508111834\\
73.25	0.0996	21.0686729691415\\
73.25	0.1002	15.4313951271006\\
73.25	0.1008	9.79411728505875\\
73.25	0.1014	4.15683944301691\\
73.25	0.102	-1.48043839902493\\
73.25	0.1026	-7.11771624106495\\
73.25	0.1032	-12.7549940831068\\
73.25	0.1038	-18.3922719251486\\
73.25	0.1044	-24.0295497671905\\
73.25	0.105	-29.6668276092323\\
73.25	0.1056	-35.3041054512732\\
73.25	0.1062	-40.9413832933151\\
73.25	0.1068	-46.5786611353569\\
73.25	0.1074	-52.2159389773979\\
73.25	0.108	-57.8532168194397\\
73.25	0.1086	-63.4904946614815\\
73.25	0.1092	-69.1277725035225\\
73.25	0.1098	-74.7650503455643\\
73.25	0.1104	-80.4023281876061\\
73.25	0.111	-86.039606029648\\
73.25	0.1116	-91.6768838716889\\
73.25	0.1122	-97.3141617137298\\
73.25	0.1128	-102.951439555772\\
73.25	0.1134	-108.588717397814\\
73.25	0.114	-114.225995239855\\
73.25	0.1146	-119.863273081897\\
73.25	0.1152	-125.500550923938\\
73.25	0.1158	-131.137828765979\\
73.25	0.1164	-136.775106608021\\
73.25	0.117	-142.412384450062\\
73.25	0.1176	-148.049662292104\\
73.25	0.1182	-153.686940134146\\
73.25	0.1188	-159.324217976187\\
73.25	0.1194	-164.961495818228\\
73.25	0.12	-170.598773660269\\
73.25	0.1206	-176.236051502311\\
73.25	0.1212	-181.873329344353\\
73.25	0.1218	-187.510607186394\\
73.25	0.1224	-193.147885028435\\
73.25	0.123	-198.785162870478\\
73.625	0.093	86.4866819565486\\
73.625	0.0936	81.0566478868031\\
73.625	0.0942	75.6266138170568\\
73.625	0.0948	70.1965797473113\\
73.625	0.0954	64.7665456775649\\
73.625	0.096	59.3365116078194\\
73.625	0.0966	53.906477538073\\
73.625	0.0972	48.4764434683275\\
73.625	0.0978	43.0464093985811\\
73.625	0.0984	37.6163753288347\\
73.625	0.099	32.1863412590892\\
73.625	0.0996	26.7563071893428\\
73.625	0.1002	21.3262731195973\\
73.625	0.1008	15.8962390498509\\
73.625	0.1014	10.4662049801054\\
73.625	0.102	5.03617091035903\\
73.625	0.1026	-0.393863159386456\\
73.625	0.1032	-5.82389722913285\\
73.625	0.1038	-11.2539312988793\\
73.625	0.1044	-16.6839653686247\\
73.625	0.105	-22.1139994383711\\
73.625	0.1056	-27.5440335081166\\
73.625	0.1062	-32.974067577863\\
73.625	0.1068	-38.4041016476103\\
73.625	0.1074	-43.8341357173549\\
73.625	0.108	-49.2641697871013\\
73.625	0.1086	-54.6942038568477\\
73.625	0.1092	-60.1242379265932\\
73.625	0.1098	-65.5542719963387\\
73.625	0.1104	-70.9843060660851\\
73.625	0.111	-76.4143401358315\\
73.625	0.1116	-81.8443742055779\\
73.625	0.1122	-87.2744082753234\\
73.625	0.1128	-92.7044423450689\\
73.625	0.1134	-98.1344764148153\\
73.625	0.114	-103.564510484562\\
73.625	0.1146	-108.994544554308\\
73.625	0.1152	-114.424578624054\\
73.625	0.1158	-119.854612693799\\
73.625	0.1164	-125.284646763545\\
73.625	0.117	-130.714680833291\\
73.625	0.1176	-136.144714903037\\
73.625	0.1182	-141.574748972783\\
73.625	0.1188	-147.004783042528\\
73.625	0.1194	-152.434817112274\\
73.625	0.12	-157.864851182022\\
73.625	0.1206	-163.294885251767\\
73.625	0.1212	-168.724919321514\\
73.625	0.1218	-174.15495339126\\
73.625	0.1224	-179.584987461003\\
73.625	0.123	-185.015021530751\\
74	0.093	89.894634681501\\
74	0.0936	84.67184438405\\
74	0.0942	79.4490540866\\
74	0.0948	74.2262637891499\\
74	0.0954	69.003473491699\\
74	0.096	63.7806831942489\\
74	0.0966	58.557892896798\\
74	0.0972	53.3351025993479\\
74	0.0978	48.1123123018979\\
74	0.0984	42.8895220044469\\
74	0.099	37.666731706996\\
74	0.0996	32.443941409545\\
74	0.1002	27.2211511120959\\
74	0.1008	21.9983608146449\\
74	0.1014	16.775570517194\\
74	0.102	11.5527802197439\\
74	0.1026	6.32998992229386\\
74	0.1032	1.10719962484291\\
74	0.1038	-4.11559067260714\\
74	0.1044	-9.3383809700581\\
74	0.105	-14.5611712675091\\
74	0.1056	-19.7839615649582\\
74	0.1062	-25.0067518624091\\
74	0.1068	-30.229542159861\\
74	0.1074	-35.4523324573111\\
74	0.108	-40.6751227547611\\
74	0.1086	-45.8979130522121\\
74	0.1092	-51.1207033496621\\
74	0.1098	-56.3434936471122\\
74	0.1104	-61.5662839445631\\
74	0.111	-66.7890742420141\\
74	0.1116	-72.0118645394641\\
74	0.1122	-77.2346548369142\\
74	0.1128	-82.4574451343651\\
74	0.1134	-87.6802354318161\\
74	0.114	-92.9030257292661\\
74	0.1146	-98.1258160267171\\
74	0.1152	-103.348606324167\\
74	0.1158	-108.571396621617\\
74	0.1164	-113.794186919068\\
74	0.117	-119.016977216518\\
74	0.1176	-124.239767513968\\
74	0.1182	-129.462557811419\\
74	0.1188	-134.68534810887\\
74	0.1194	-139.90813840632\\
74	0.12	-145.13092870377\\
74	0.1206	-150.353719001221\\
74	0.1212	-155.576509298671\\
74	0.1218	-160.799299596121\\
74	0.1224	-166.022089893571\\
74	0.123	-171.244880191023\\
};
\end{axis}

\begin{axis}[%
width=4.927496cm,
height=3.870968cm,
at={(0cm,5.376344cm)},
scale only axis,
xmin=56,
xmax=74,
tick align=outside,
xlabel={$L_{cut}$},
xmajorgrids,
ymin=0.093,
ymax=0.123,
ylabel={$D_{rlx}$},
ymajorgrids,
zmin=-16.1627688144179,
zmax=49.5376851700437,
zlabel={$x_2$},
zmajorgrids,
view={-140}{50},
legend style={at={(1.03,1)},anchor=north west,legend cell align=left,align=left,draw=white!15!black}
]
\addplot3[only marks,mark=*,mark options={},mark size=1.5000pt,color=mycolor1] plot table[row sep=crcr,]{%
74	0.123	0.706917161850943\\
72	0.113	-0.725188933625476\\
61	0.095	-2.23969084196957\\
56	0.093	-0.663995623424443\\
};
\addplot3[only marks,mark=*,mark options={},mark size=1.5000pt,color=black] plot table[row sep=crcr,]{%
69	0.104	-2.27707711636407\\
};

\addplot3[%
surf,
opacity=0.7,
shader=interp,
colormap={mymap}{[1pt] rgb(0pt)=(0.0901961,0.239216,0.0745098); rgb(1pt)=(0.0945149,0.242058,0.0739522); rgb(2pt)=(0.0988592,0.244894,0.0733566); rgb(3pt)=(0.103229,0.247724,0.0727241); rgb(4pt)=(0.107623,0.250549,0.0720557); rgb(5pt)=(0.112043,0.253367,0.0713525); rgb(6pt)=(0.116487,0.25618,0.0706154); rgb(7pt)=(0.120956,0.258986,0.0698456); rgb(8pt)=(0.125449,0.261787,0.0690441); rgb(9pt)=(0.129967,0.264581,0.0682118); rgb(10pt)=(0.134508,0.26737,0.06735); rgb(11pt)=(0.139074,0.270152,0.0664596); rgb(12pt)=(0.143663,0.272929,0.0655416); rgb(13pt)=(0.148275,0.275699,0.0645971); rgb(14pt)=(0.152911,0.278463,0.0636271); rgb(15pt)=(0.15757,0.281221,0.0626328); rgb(16pt)=(0.162252,0.283973,0.0616151); rgb(17pt)=(0.166957,0.286719,0.060575); rgb(18pt)=(0.171685,0.289458,0.0595136); rgb(19pt)=(0.176434,0.292191,0.0584321); rgb(20pt)=(0.181207,0.294918,0.0573313); rgb(21pt)=(0.186001,0.297639,0.0562123); rgb(22pt)=(0.190817,0.300353,0.0550763); rgb(23pt)=(0.195655,0.303061,0.0539242); rgb(24pt)=(0.200514,0.305763,0.052757); rgb(25pt)=(0.205395,0.308459,0.0515759); rgb(26pt)=(0.210296,0.311149,0.0503624); rgb(27pt)=(0.215212,0.313846,0.0490067); rgb(28pt)=(0.220142,0.316548,0.0475043); rgb(29pt)=(0.22509,0.319254,0.0458704); rgb(30pt)=(0.230056,0.321962,0.0441205); rgb(31pt)=(0.235042,0.324671,0.04227); rgb(32pt)=(0.240048,0.327379,0.0403343); rgb(33pt)=(0.245078,0.330085,0.0383287); rgb(34pt)=(0.250131,0.332786,0.0362688); rgb(35pt)=(0.25521,0.335482,0.0341698); rgb(36pt)=(0.260317,0.33817,0.0320472); rgb(37pt)=(0.265451,0.340849,0.0299163); rgb(38pt)=(0.270616,0.343517,0.0277927); rgb(39pt)=(0.275813,0.346172,0.0256916); rgb(40pt)=(0.281043,0.348814,0.0236284); rgb(41pt)=(0.286307,0.35144,0.0216186); rgb(42pt)=(0.291607,0.354048,0.0196776); rgb(43pt)=(0.296945,0.356637,0.0178207); rgb(44pt)=(0.302322,0.359206,0.0160634); rgb(45pt)=(0.307739,0.361753,0.0144211); rgb(46pt)=(0.313198,0.364275,0.0129091); rgb(47pt)=(0.318701,0.366772,0.0115428); rgb(48pt)=(0.324249,0.369242,0.0103377); rgb(49pt)=(0.329843,0.371682,0.00930909); rgb(50pt)=(0.335485,0.374093,0.00847245); rgb(51pt)=(0.341176,0.376471,0.00784314); rgb(52pt)=(0.346925,0.378826,0.00732741); rgb(53pt)=(0.352735,0.381168,0.00682184); rgb(54pt)=(0.358605,0.383497,0.00632729); rgb(55pt)=(0.364532,0.385812,0.00584464); rgb(56pt)=(0.370516,0.388113,0.00537476); rgb(57pt)=(0.376552,0.390399,0.00491852); rgb(58pt)=(0.38264,0.39267,0.00447681); rgb(59pt)=(0.388777,0.394925,0.00405048); rgb(60pt)=(0.394962,0.397164,0.00364042); rgb(61pt)=(0.401191,0.399386,0.00324749); rgb(62pt)=(0.407464,0.401592,0.00287258); rgb(63pt)=(0.413777,0.40378,0.00251655); rgb(64pt)=(0.420129,0.40595,0.00218028); rgb(65pt)=(0.426518,0.408102,0.00186463); rgb(66pt)=(0.432942,0.410234,0.00157049); rgb(67pt)=(0.439399,0.412348,0.00129873); rgb(68pt)=(0.445885,0.414441,0.00105022); rgb(69pt)=(0.452401,0.416515,0.000825833); rgb(70pt)=(0.458942,0.418567,0.000626441); rgb(71pt)=(0.465508,0.420599,0.00045292); rgb(72pt)=(0.472096,0.422609,0.000306141); rgb(73pt)=(0.478704,0.424596,0.000186979); rgb(74pt)=(0.485331,0.426562,9.63073e-05); rgb(75pt)=(0.491973,0.428504,3.49981e-05); rgb(76pt)=(0.498628,0.430422,3.92506e-06); rgb(77pt)=(0.505323,0.432315,0); rgb(78pt)=(0.512206,0.434168,0); rgb(79pt)=(0.519282,0.435983,0); rgb(80pt)=(0.526529,0.437764,0); rgb(81pt)=(0.533922,0.439512,0); rgb(82pt)=(0.54144,0.441232,0); rgb(83pt)=(0.549059,0.442927,0); rgb(84pt)=(0.556756,0.444599,0); rgb(85pt)=(0.564508,0.446252,0); rgb(86pt)=(0.572292,0.447889,0); rgb(87pt)=(0.580084,0.449514,0); rgb(88pt)=(0.587863,0.451129,0); rgb(89pt)=(0.595604,0.452737,0); rgb(90pt)=(0.603284,0.454343,0); rgb(91pt)=(0.610882,0.455948,0); rgb(92pt)=(0.618373,0.457556,0); rgb(93pt)=(0.625734,0.459171,0); rgb(94pt)=(0.632943,0.460795,0); rgb(95pt)=(0.639976,0.462432,0); rgb(96pt)=(0.64681,0.464084,0); rgb(97pt)=(0.653423,0.465756,0); rgb(98pt)=(0.659791,0.46745,0); rgb(99pt)=(0.665891,0.469169,0); rgb(100pt)=(0.6717,0.470916,0); rgb(101pt)=(0.677195,0.472696,0); rgb(102pt)=(0.682353,0.47451,0); rgb(103pt)=(0.687242,0.476355,0); rgb(104pt)=(0.691952,0.478225,0); rgb(105pt)=(0.696497,0.480118,0); rgb(106pt)=(0.700887,0.482033,0); rgb(107pt)=(0.705134,0.483968,0); rgb(108pt)=(0.709251,0.485921,0); rgb(109pt)=(0.713249,0.487891,0); rgb(110pt)=(0.71714,0.489876,0); rgb(111pt)=(0.720936,0.491875,0); rgb(112pt)=(0.724649,0.493887,0); rgb(113pt)=(0.72829,0.495909,0); rgb(114pt)=(0.731872,0.49794,0); rgb(115pt)=(0.735406,0.499979,0); rgb(116pt)=(0.738904,0.502025,0); rgb(117pt)=(0.742378,0.504075,0); rgb(118pt)=(0.74584,0.506128,0); rgb(119pt)=(0.749302,0.508182,0); rgb(120pt)=(0.752775,0.510237,0); rgb(121pt)=(0.756272,0.51229,0); rgb(122pt)=(0.759804,0.514339,0); rgb(123pt)=(0.763384,0.516385,0); rgb(124pt)=(0.767022,0.518424,0); rgb(125pt)=(0.770731,0.520455,0); rgb(126pt)=(0.774523,0.522478,0); rgb(127pt)=(0.77841,0.524489,0); rgb(128pt)=(0.782391,0.526491,0); rgb(129pt)=(0.786402,0.528496,0); rgb(130pt)=(0.790431,0.530506,0); rgb(131pt)=(0.794478,0.532521,0); rgb(132pt)=(0.798541,0.534539,0); rgb(133pt)=(0.802619,0.53656,0); rgb(134pt)=(0.806712,0.538584,0); rgb(135pt)=(0.81082,0.540609,0); rgb(136pt)=(0.81494,0.542635,0); rgb(137pt)=(0.819074,0.54466,0); rgb(138pt)=(0.823219,0.546686,0); rgb(139pt)=(0.827374,0.548709,0); rgb(140pt)=(0.831541,0.55073,0); rgb(141pt)=(0.835716,0.552749,0); rgb(142pt)=(0.8399,0.554763,0); rgb(143pt)=(0.844092,0.556774,0); rgb(144pt)=(0.848292,0.558779,0); rgb(145pt)=(0.852497,0.560778,0); rgb(146pt)=(0.856708,0.562771,0); rgb(147pt)=(0.860924,0.564756,0); rgb(148pt)=(0.865143,0.566733,0); rgb(149pt)=(0.869366,0.568701,0); rgb(150pt)=(0.873592,0.57066,0); rgb(151pt)=(0.877819,0.572608,0); rgb(152pt)=(0.882047,0.574545,0); rgb(153pt)=(0.886275,0.576471,0); rgb(154pt)=(0.890659,0.578362,0); rgb(155pt)=(0.895333,0.580203,0); rgb(156pt)=(0.900258,0.581999,0); rgb(157pt)=(0.905397,0.583755,0); rgb(158pt)=(0.910711,0.585479,0); rgb(159pt)=(0.916164,0.587176,0); rgb(160pt)=(0.921717,0.588852,0); rgb(161pt)=(0.927333,0.590513,0); rgb(162pt)=(0.932974,0.592166,0); rgb(163pt)=(0.938602,0.593815,0); rgb(164pt)=(0.94418,0.595468,0); rgb(165pt)=(0.949669,0.59713,0); rgb(166pt)=(0.955033,0.598808,0); rgb(167pt)=(0.960233,0.600507,0); rgb(168pt)=(0.965232,0.602233,0); rgb(169pt)=(0.969992,0.603992,0); rgb(170pt)=(0.974475,0.605791,0); rgb(171pt)=(0.978643,0.607636,0); rgb(172pt)=(0.98246,0.609532,0); rgb(173pt)=(0.985886,0.611486,0); rgb(174pt)=(0.988885,0.613503,0); rgb(175pt)=(0.991419,0.61559,0); rgb(176pt)=(0.99345,0.617753,0); rgb(177pt)=(0.99494,0.619997,0); rgb(178pt)=(0.995851,0.622329,0); rgb(179pt)=(0.996226,0.624763,0); rgb(180pt)=(0.996512,0.627352,0); rgb(181pt)=(0.996788,0.630095,0); rgb(182pt)=(0.997053,0.632982,0); rgb(183pt)=(0.997308,0.636004,0); rgb(184pt)=(0.997552,0.639152,0); rgb(185pt)=(0.997785,0.642416,0); rgb(186pt)=(0.998006,0.645786,0); rgb(187pt)=(0.998217,0.649253,0); rgb(188pt)=(0.998416,0.652807,0); rgb(189pt)=(0.998605,0.656439,0); rgb(190pt)=(0.998781,0.660138,0); rgb(191pt)=(0.998946,0.663897,0); rgb(192pt)=(0.9991,0.667704,0); rgb(193pt)=(0.999242,0.67155,0); rgb(194pt)=(0.999372,0.675427,0); rgb(195pt)=(0.99949,0.679323,0); rgb(196pt)=(0.999596,0.68323,0); rgb(197pt)=(0.99969,0.687139,0); rgb(198pt)=(0.999771,0.691039,0); rgb(199pt)=(0.999841,0.694921,0); rgb(200pt)=(0.999898,0.698775,0); rgb(201pt)=(0.999942,0.702592,0); rgb(202pt)=(0.999974,0.706363,0); rgb(203pt)=(0.999994,0.710077,0); rgb(204pt)=(1,0.713725,0); rgb(205pt)=(1,0.717341,0); rgb(206pt)=(1,0.720963,0); rgb(207pt)=(1,0.724591,0); rgb(208pt)=(1,0.728226,0); rgb(209pt)=(1,0.731867,0); rgb(210pt)=(1,0.735514,0); rgb(211pt)=(1,0.739167,0); rgb(212pt)=(1,0.742827,0); rgb(213pt)=(1,0.746493,0); rgb(214pt)=(1,0.750165,0); rgb(215pt)=(1,0.753843,0); rgb(216pt)=(1,0.757527,0); rgb(217pt)=(1,0.761217,0); rgb(218pt)=(1,0.764913,0); rgb(219pt)=(1,0.768615,0); rgb(220pt)=(1,0.772324,0); rgb(221pt)=(1,0.776038,0); rgb(222pt)=(1,0.779758,0); rgb(223pt)=(1,0.783484,0); rgb(224pt)=(1,0.787215,0); rgb(225pt)=(1,0.790953,0); rgb(226pt)=(1,0.794696,0); rgb(227pt)=(1,0.798445,0); rgb(228pt)=(1,0.8022,0); rgb(229pt)=(1,0.805961,0); rgb(230pt)=(1,0.809727,0); rgb(231pt)=(1,0.8135,0); rgb(232pt)=(1,0.817278,0); rgb(233pt)=(1,0.821063,0); rgb(234pt)=(1,0.824854,0); rgb(235pt)=(1,0.828652,0); rgb(236pt)=(1,0.832455,0); rgb(237pt)=(1,0.836265,0); rgb(238pt)=(1,0.840081,0); rgb(239pt)=(1,0.843903,0); rgb(240pt)=(1,0.847732,0); rgb(241pt)=(1,0.851566,0); rgb(242pt)=(1,0.855406,0); rgb(243pt)=(1,0.859253,0); rgb(244pt)=(1,0.863106,0); rgb(245pt)=(1,0.866964,0); rgb(246pt)=(1,0.870829,0); rgb(247pt)=(1,0.8747,0); rgb(248pt)=(1,0.878577,0); rgb(249pt)=(1,0.88246,0); rgb(250pt)=(1,0.886349,0); rgb(251pt)=(1,0.890243,0); rgb(252pt)=(1,0.894144,0); rgb(253pt)=(1,0.898051,0); rgb(254pt)=(1,0.901964,0); rgb(255pt)=(1,0.905882,0)},
mesh/rows=49]
table[row sep=crcr,header=false] {%
%
56	0.093	-0.663995623001483\\
56	0.0936	0.340037992859436\\
56	0.0942	1.34407160872036\\
56	0.0948	2.34810522458127\\
56	0.0954	3.35213884044219\\
56	0.096	4.35617245630306\\
56	0.0966	5.36020607216398\\
56	0.0972	6.3642396880249\\
56	0.0978	7.36827330388576\\
56	0.0984	8.37230691974668\\
56	0.099	9.3763405356076\\
56	0.0996	10.3803741514685\\
56	0.1002	11.3844077673293\\
56	0.1008	12.3884413831902\\
56	0.1014	13.3924749990512\\
56	0.102	14.3965086149121\\
56	0.1026	15.400542230773\\
56	0.1032	16.4045758466339\\
56	0.1038	17.4086094624948\\
56	0.1044	18.4126430783557\\
56	0.105	19.4166766942166\\
56	0.1056	20.4207103100775\\
56	0.1062	21.4247439259384\\
56	0.1068	22.4287775417993\\
56	0.1074	23.4328111576602\\
56	0.108	24.4368447735211\\
56	0.1086	25.440878389382\\
56	0.1092	26.4449120052429\\
56	0.1098	27.4489456211039\\
56	0.1104	28.4529792369648\\
56	0.111	29.4570128528257\\
56	0.1116	30.4610464686865\\
56	0.1122	31.4650800845475\\
56	0.1128	32.4691137004083\\
56	0.1134	33.4731473162693\\
56	0.114	34.4771809321302\\
56	0.1146	35.4812145479911\\
56	0.1152	36.485248163852\\
56	0.1158	37.4892817797129\\
56	0.1164	38.4933153955738\\
56	0.117	39.4973490114346\\
56	0.1176	40.5013826272955\\
56	0.1182	41.5054162431564\\
56	0.1188	42.5094498590174\\
56	0.1194	43.5134834748783\\
56	0.12	44.5175170907392\\
56	0.1206	45.5215507066001\\
56	0.1212	46.525584322461\\
56	0.1218	47.5296179383219\\
56	0.1224	48.5336515541828\\
56	0.123	49.5376851700437\\
56.375	0.093	-0.986886731156005\\
56.375	0.0936	0.0032585535312819\\
56.375	0.0942	0.993403838218569\\
56.375	0.0948	1.98354912290586\\
56.375	0.0954	2.97369440759314\\
56.375	0.096	3.96383969228043\\
56.375	0.0966	4.95398497696772\\
56.375	0.0972	5.944130261655\\
56.375	0.0978	6.93427554634223\\
56.375	0.0984	7.92442083102952\\
56.375	0.099	8.91456611571681\\
56.375	0.0996	9.9047114004041\\
56.375	0.1002	10.8948566850913\\
56.375	0.1008	11.8850019697787\\
56.375	0.1014	12.875147254466\\
56.375	0.102	13.8652925391533\\
56.375	0.1026	14.8554378238406\\
56.375	0.1032	15.8455831085279\\
56.375	0.1038	16.8357283932152\\
56.375	0.1044	17.8258736779024\\
56.375	0.105	18.8160189625897\\
56.375	0.1056	19.806164247277\\
56.375	0.1062	20.7963095319643\\
56.375	0.1068	21.7864548166515\\
56.375	0.1074	22.7766001013388\\
56.375	0.108	23.7667453860261\\
56.375	0.1086	24.7568906707133\\
56.375	0.1092	25.7470359554007\\
56.375	0.1098	26.737181240088\\
56.375	0.1104	27.7273265247753\\
56.375	0.111	28.7174718094626\\
56.375	0.1116	29.7076170941497\\
56.375	0.1122	30.6977623788371\\
56.375	0.1128	31.6879076635244\\
56.375	0.1134	32.6780529482116\\
56.375	0.114	33.668198232899\\
56.375	0.1146	34.6583435175863\\
56.375	0.1152	35.6484888022735\\
56.375	0.1158	36.6386340869608\\
56.375	0.1164	37.6287793716481\\
56.375	0.117	38.6189246563353\\
56.375	0.1176	39.6090699410226\\
56.375	0.1182	40.5992152257099\\
56.375	0.1188	41.5893605103972\\
56.375	0.1194	42.5795057950845\\
56.375	0.12	43.5696510797718\\
56.375	0.1206	44.5597963644591\\
56.375	0.1212	45.5499416491464\\
56.375	0.1218	46.5400869338336\\
56.375	0.1224	47.5302322185209\\
56.375	0.123	48.5203775032082\\
56.75	0.093	-1.30977783931047\\
56.75	0.0936	-0.333520885796815\\
56.75	0.0942	0.642736067716839\\
56.75	0.0948	1.61899302123049\\
56.75	0.0954	2.59524997474421\\
56.75	0.096	3.57150692825786\\
56.75	0.0966	4.54776388177152\\
56.75	0.0972	5.52402083528517\\
56.75	0.0978	6.50027778879883\\
56.75	0.0984	7.47653474231248\\
56.75	0.099	8.45279169582614\\
56.75	0.0996	9.42904864933979\\
56.75	0.1002	10.4053056028534\\
56.75	0.1008	11.3815625563671\\
56.75	0.1014	12.3578195098808\\
56.75	0.102	13.3340764633945\\
56.75	0.1026	14.3103334169082\\
56.75	0.1032	15.2865903704218\\
56.75	0.1038	16.2628473239355\\
56.75	0.1044	17.2391042774491\\
56.75	0.105	18.2153612309628\\
56.75	0.1056	19.1916181844765\\
56.75	0.1062	20.1678751379901\\
56.75	0.1068	21.1441320915038\\
56.75	0.1074	22.1203890450174\\
56.75	0.108	23.0966459985311\\
56.75	0.1086	24.0729029520448\\
56.75	0.1092	25.0491599055585\\
56.75	0.1098	26.0254168590722\\
56.75	0.1104	27.0016738125858\\
56.75	0.111	27.9779307660995\\
56.75	0.1116	28.9541877196131\\
56.75	0.1122	29.9304446731268\\
56.75	0.1128	30.9067016266404\\
56.75	0.1134	31.8829585801541\\
56.75	0.114	32.8592155336678\\
56.75	0.1146	33.8354724871814\\
56.75	0.1152	34.811729440695\\
56.75	0.1158	35.7879863942088\\
56.75	0.1164	36.7642433477224\\
56.75	0.117	37.740500301236\\
56.75	0.1176	38.7167572547497\\
56.75	0.1182	39.6930142082634\\
56.75	0.1188	40.669271161777\\
56.75	0.1194	41.6455281152907\\
56.75	0.12	42.6217850688044\\
56.75	0.1206	43.5980420223181\\
56.75	0.1212	44.5742989758318\\
56.75	0.1218	45.5505559293454\\
56.75	0.1224	46.526812882859\\
56.75	0.123	47.5030698363727\\
57.125	0.093	-1.63266894746505\\
57.125	0.0936	-0.670300325125027\\
57.125	0.0942	0.292068297215053\\
57.125	0.0948	1.25443691955508\\
57.125	0.0954	2.21680554189516\\
57.125	0.096	3.17917416423518\\
57.125	0.0966	4.14154278657526\\
57.125	0.0972	5.10391140891528\\
57.125	0.0978	6.06628003125525\\
57.125	0.0984	7.02864865359533\\
57.125	0.099	7.99101727593535\\
57.125	0.0996	8.95338589827543\\
57.125	0.1002	9.91575452061539\\
57.125	0.1008	10.8781231429555\\
57.125	0.1014	11.8404917652955\\
57.125	0.102	12.8028603876356\\
57.125	0.1026	13.7652290099757\\
57.125	0.1032	14.7275976323157\\
57.125	0.1038	15.6899662546558\\
57.125	0.1044	16.6523348769958\\
57.125	0.105	17.6147034993358\\
57.125	0.1056	18.5770721216759\\
57.125	0.1062	19.5394407440159\\
57.125	0.1068	20.501809366356\\
57.125	0.1074	21.464177988696\\
57.125	0.108	22.426546611036\\
57.125	0.1086	23.3889152333761\\
57.125	0.1092	24.3512838557161\\
57.125	0.1098	25.3136524780562\\
57.125	0.1104	26.2760211003962\\
57.125	0.111	27.2383897227363\\
57.125	0.1116	28.2007583450762\\
57.125	0.1122	29.1631269674164\\
57.125	0.1128	30.1254955897564\\
57.125	0.1134	31.0878642120965\\
57.125	0.114	32.0502328344365\\
57.125	0.1146	33.0126014567766\\
57.125	0.1152	33.9749700791165\\
57.125	0.1158	34.9373387014566\\
57.125	0.1164	35.8997073237966\\
57.125	0.117	36.8620759461367\\
57.125	0.1176	37.8244445684767\\
57.125	0.1182	38.7868131908168\\
57.125	0.1188	39.7491818131568\\
57.125	0.1194	40.7115504354969\\
57.125	0.12	41.6739190578369\\
57.125	0.1206	42.636287680177\\
57.125	0.1212	43.598656302517\\
57.125	0.1218	44.561024924857\\
57.125	0.1224	45.5233935471971\\
57.125	0.123	46.4857621695371\\
57.5	0.093	-1.95556005561957\\
57.5	0.0936	-1.00707976445312\\
57.5	0.0942	-0.0585994732866766\\
57.5	0.0948	0.889880817879714\\
57.5	0.0954	1.83836110904616\\
57.5	0.096	2.78684140021261\\
57.5	0.0966	3.73532169137906\\
57.5	0.0972	4.68380198254545\\
57.5	0.0978	5.63228227371184\\
57.5	0.0984	6.58076256487828\\
57.5	0.099	7.52924285604468\\
57.5	0.0996	8.47772314721112\\
57.5	0.1002	9.42620343837751\\
57.5	0.1008	10.3746837295439\\
57.5	0.1014	11.3231640207104\\
57.5	0.102	12.2716443118769\\
57.5	0.1026	13.2201246030432\\
57.5	0.1032	14.1686048942097\\
57.5	0.1038	15.1170851853761\\
57.5	0.1044	16.0655654765425\\
57.5	0.105	17.0140457677089\\
57.5	0.1056	17.9625260588754\\
57.5	0.1062	18.9110063500418\\
57.5	0.1068	19.8594866412082\\
57.5	0.1074	20.8079669323746\\
57.5	0.108	21.756447223541\\
57.5	0.1086	22.7049275147074\\
57.5	0.1092	23.6534078058739\\
57.5	0.1098	24.6018880970404\\
57.5	0.1104	25.5503683882068\\
57.5	0.111	26.4988486793732\\
57.5	0.1116	27.4473289705396\\
57.5	0.1122	28.395809261706\\
57.5	0.1128	29.3442895528725\\
57.5	0.1134	30.2927698440389\\
57.5	0.114	31.2412501352053\\
57.5	0.1146	32.1897304263717\\
57.5	0.1152	33.1382107175381\\
57.5	0.1158	34.0866910087046\\
57.5	0.1164	35.035171299871\\
57.5	0.117	35.9836515910374\\
57.5	0.1176	36.9321318822038\\
57.5	0.1182	37.8806121733702\\
57.5	0.1188	38.8290924645366\\
57.5	0.1194	39.7775727557031\\
57.5	0.12	40.7260530468696\\
57.5	0.1206	41.674533338036\\
57.5	0.1212	42.6230136292024\\
57.5	0.1218	43.5714939203688\\
57.5	0.1224	44.5199742115352\\
57.5	0.123	45.4684545027017\\
57.875	0.093	-2.27845116377409\\
57.875	0.0936	-1.34385920378128\\
57.875	0.0942	-0.409267243788463\\
57.875	0.0948	0.525324716204352\\
57.875	0.0954	1.45991667619717\\
57.875	0.096	2.39450863618998\\
57.875	0.0966	3.3291005961828\\
57.875	0.0972	4.26369255617556\\
57.875	0.0978	5.19828451616831\\
57.875	0.0984	6.13287647616113\\
57.875	0.099	7.06746843615394\\
57.875	0.0996	8.00206039614676\\
57.875	0.1002	8.93665235613952\\
57.875	0.1008	9.87124431613233\\
57.875	0.1014	10.8058362761251\\
57.875	0.102	11.740428236118\\
57.875	0.1026	12.6750201961108\\
57.875	0.1032	13.6096121561037\\
57.875	0.1038	14.5442041160965\\
57.875	0.1044	15.4787960760892\\
57.875	0.105	16.413388036082\\
57.875	0.1056	17.3479799960748\\
57.875	0.1062	18.2825719560676\\
57.875	0.1068	19.2171639160604\\
57.875	0.1074	20.1517558760532\\
57.875	0.108	21.086347836046\\
57.875	0.1086	22.0209397960388\\
57.875	0.1092	22.9555317560317\\
57.875	0.1098	23.8901237160245\\
57.875	0.1104	24.8247156760173\\
57.875	0.111	25.7593076360101\\
57.875	0.1116	26.6938995960028\\
57.875	0.1122	27.6284915559957\\
57.875	0.1128	28.5630835159885\\
57.875	0.1134	29.4976754759813\\
57.875	0.114	30.4322674359741\\
57.875	0.1146	31.3668593959669\\
57.875	0.1152	32.3014513559597\\
57.875	0.1158	33.2360433159525\\
57.875	0.1164	34.1706352759453\\
57.875	0.117	35.105227235938\\
57.875	0.1176	36.0398191959308\\
57.875	0.1182	36.9744111559236\\
57.875	0.1188	37.9090031159164\\
57.875	0.1194	38.8435950759093\\
57.875	0.12	39.7781870359021\\
57.875	0.1206	40.712778995895\\
57.875	0.1212	41.6473709558878\\
57.875	0.1218	42.5819629158805\\
57.875	0.1224	43.5165548758733\\
57.875	0.123	44.4511468358662\\
58.25	0.093	-2.60134227192856\\
58.25	0.0936	-1.68063864310938\\
58.25	0.0942	-0.759935014290193\\
58.25	0.0948	0.16076861452899\\
58.25	0.0954	1.08147224334817\\
58.25	0.096	2.00217587216736\\
58.25	0.0966	2.9228795009866\\
58.25	0.0972	3.84358312980578\\
58.25	0.0978	4.76428675862491\\
58.25	0.0984	5.68499038744409\\
58.25	0.099	6.60569401626327\\
58.25	0.0996	7.52639764508245\\
58.25	0.1002	8.44710127390158\\
58.25	0.1008	9.36780490272082\\
58.25	0.1014	10.28850853154\\
58.25	0.102	11.2092121603592\\
58.25	0.1026	12.1299157891784\\
58.25	0.1032	13.0506194179976\\
58.25	0.1038	13.9713230468168\\
58.25	0.1044	14.8920266756359\\
58.25	0.105	15.8127303044551\\
58.25	0.1056	16.7334339332743\\
58.25	0.1062	17.6541375620935\\
58.25	0.1068	18.5748411909127\\
58.25	0.1074	19.4955448197318\\
58.25	0.108	20.416248448551\\
58.25	0.1086	21.3369520773702\\
58.25	0.1092	22.2576557061894\\
58.25	0.1098	23.1783593350086\\
58.25	0.1104	24.0990629638279\\
58.25	0.111	25.019766592647\\
58.25	0.1116	25.9404702214661\\
58.25	0.1122	26.8611738502854\\
58.25	0.1128	27.7818774791045\\
58.25	0.1134	28.7025811079237\\
58.25	0.114	29.6232847367429\\
58.25	0.1146	30.5439883655621\\
58.25	0.1152	31.4646919943813\\
58.25	0.1158	32.3853956232005\\
58.25	0.1164	33.3060992520196\\
58.25	0.117	34.2268028808388\\
58.25	0.1176	35.1475065096579\\
58.25	0.1182	36.0682101384771\\
58.25	0.1188	36.9889137672963\\
58.25	0.1194	37.9096173961156\\
58.25	0.12	38.8303210249348\\
58.25	0.1206	39.751024653754\\
58.25	0.1212	40.6717282825732\\
58.25	0.1218	41.5924319113923\\
58.25	0.1224	42.5131355402115\\
58.25	0.123	43.4338391690306\\
58.625	0.093	-2.92423338008302\\
58.625	0.0936	-2.01741808243747\\
58.625	0.0942	-1.11060278479192\\
58.625	0.0948	-0.203787487146315\\
58.625	0.0954	0.703027810499236\\
58.625	0.096	1.60984310814479\\
58.625	0.0966	2.51665840579039\\
58.625	0.0972	3.42347370343595\\
58.625	0.0978	4.33028900108144\\
58.625	0.0984	5.23710429872705\\
58.625	0.099	6.1439195963726\\
58.625	0.0996	7.05073489401815\\
58.625	0.1002	7.9575501916637\\
58.625	0.1008	8.86436548930925\\
58.625	0.1014	9.7711807869548\\
58.625	0.102	10.6779960846005\\
58.625	0.1026	11.584811382246\\
58.625	0.1032	12.4916266798916\\
58.625	0.1038	13.3984419775372\\
58.625	0.1044	14.3052572751827\\
58.625	0.105	15.2120725728282\\
58.625	0.1056	16.1188878704738\\
58.625	0.1062	17.0257031681194\\
58.625	0.1068	17.932518465765\\
58.625	0.1074	18.8393337634105\\
58.625	0.108	19.746149061056\\
58.625	0.1086	20.6529643587016\\
58.625	0.1092	21.5597796563472\\
58.625	0.1098	22.4665949539928\\
58.625	0.1104	23.3734102516384\\
58.625	0.111	24.280225549284\\
58.625	0.1116	25.1870408469294\\
58.625	0.1122	26.0938561445751\\
58.625	0.1128	27.0006714422206\\
58.625	0.1134	27.9074867398662\\
58.625	0.114	28.8143020375118\\
58.625	0.1146	29.7211173351573\\
58.625	0.1152	30.6279326328028\\
58.625	0.1158	31.5347479304484\\
58.625	0.1164	32.441563228094\\
58.625	0.117	33.3483785257395\\
58.625	0.1176	34.2551938233851\\
58.625	0.1182	35.1620091210306\\
58.625	0.1188	36.0688244186762\\
58.625	0.1194	36.9756397163218\\
58.625	0.12	37.8824550139674\\
58.625	0.1206	38.7892703116129\\
58.625	0.1212	39.6960856092585\\
58.625	0.1218	40.602900906904\\
58.625	0.1224	41.5097162045496\\
58.625	0.123	42.4165315021952\\
59	0.093	-3.2471244882376\\
59	0.0936	-2.35419752176563\\
59	0.0942	-1.46127055529371\\
59	0.0948	-0.568343588821733\\
59	0.0954	0.324583377650242\\
59	0.096	1.21751034412216\\
59	0.0966	2.11043731059414\\
59	0.0972	3.00336427706605\\
59	0.0978	3.89629124353797\\
59	0.0984	4.78921821000989\\
59	0.099	5.68214517648187\\
59	0.0996	6.57507214295379\\
59	0.1002	7.4679991094257\\
59	0.1008	8.36092607589768\\
59	0.1014	9.2538530423696\\
59	0.102	10.1467800088416\\
59	0.1026	11.0397069753135\\
59	0.1032	11.9326339417855\\
59	0.1038	12.8255609082574\\
59	0.1044	13.7184878747294\\
59	0.105	14.6114148412013\\
59	0.1056	15.5043418076733\\
59	0.1062	16.3972687741452\\
59	0.1068	17.2901957406172\\
59	0.1074	18.1831227070891\\
59	0.108	19.076049673561\\
59	0.1086	19.968976640033\\
59	0.1092	20.8619036065049\\
59	0.1098	21.7548305729769\\
59	0.1104	22.6477575394489\\
59	0.111	23.5406845059208\\
59	0.1116	24.4336114723927\\
59	0.1122	25.3265384388646\\
59	0.1128	26.2194654053366\\
59	0.1134	27.1123923718085\\
59	0.114	28.0053193382805\\
59	0.1146	28.8982463047525\\
59	0.1152	29.7911732712244\\
59	0.1158	30.6841002376963\\
59	0.1164	31.5770272041682\\
59	0.117	32.4699541706402\\
59	0.1176	33.3628811371121\\
59	0.1182	34.2558081035841\\
59	0.1188	35.148735070056\\
59	0.1194	36.041662036528\\
59	0.12	36.934589003\\
59	0.1206	37.8275159694719\\
59	0.1212	38.7204429359439\\
59	0.1218	39.6133699024157\\
59	0.1224	40.5062968688877\\
59	0.123	41.3992238353596\\
59.375	0.093	-3.57001559639207\\
59.375	0.0936	-2.69097696109372\\
59.375	0.0942	-1.81193832579538\\
59.375	0.0948	-0.932899690497095\\
59.375	0.0954	-0.0538610551987517\\
59.375	0.096	0.825177580099592\\
59.375	0.0966	1.70421621539793\\
59.375	0.0972	2.58325485069622\\
59.375	0.0978	3.46229348599451\\
59.375	0.0984	4.34133212129285\\
59.375	0.099	5.22037075659119\\
59.375	0.0996	6.09940939188954\\
59.375	0.1002	6.97844802718777\\
59.375	0.1008	7.85748666248611\\
59.375	0.1014	8.73652529778445\\
59.375	0.102	9.61556393308285\\
59.375	0.1026	10.4946025683811\\
59.375	0.1032	11.3736412036795\\
59.375	0.1038	12.2526798389778\\
59.375	0.1044	13.1317184742761\\
59.375	0.105	14.0107571095745\\
59.375	0.1056	14.8897957448727\\
59.375	0.1062	15.7688343801711\\
59.375	0.1068	16.6478730154694\\
59.375	0.1074	17.5269116507677\\
59.375	0.108	18.4059502860661\\
59.375	0.1086	19.2849889213643\\
59.375	0.1092	20.1640275566627\\
59.375	0.1098	21.0430661919611\\
59.375	0.1104	21.9221048272594\\
59.375	0.111	22.8011434625577\\
59.375	0.1116	23.6801820978559\\
59.375	0.1122	24.5592207331543\\
59.375	0.1128	25.4382593684527\\
59.375	0.1134	26.317298003751\\
59.375	0.114	27.1963366390493\\
59.375	0.1146	28.0753752743477\\
59.375	0.1152	28.954413909646\\
59.375	0.1158	29.8334525449443\\
59.375	0.1164	30.7124911802426\\
59.375	0.117	31.5915298155409\\
59.375	0.1176	32.4705684508392\\
59.375	0.1182	33.3496070861376\\
59.375	0.1188	34.2286457214359\\
59.375	0.1194	35.1076843567342\\
59.375	0.12	35.9867229920326\\
59.375	0.1206	36.8657616273309\\
59.375	0.1212	37.7448002626293\\
59.375	0.1218	38.6238388979275\\
59.375	0.1224	39.5028775332258\\
59.375	0.123	40.3819161685242\\
59.75	0.093	-3.89290670454659\\
59.75	0.0936	-3.02775640042188\\
59.75	0.0942	-2.16260609629717\\
59.75	0.0948	-1.29745579217246\\
59.75	0.0954	-0.432305488047746\\
59.75	0.096	0.432844816076965\\
59.75	0.0966	1.29799512020168\\
59.75	0.0972	2.16314542432639\\
59.75	0.0978	3.02829572845104\\
59.75	0.0984	3.89344603257575\\
59.75	0.099	4.75859633670046\\
59.75	0.0996	5.62374664082517\\
59.75	0.1002	6.48889694494983\\
59.75	0.1008	7.35404724907454\\
59.75	0.1014	8.21919755319925\\
59.75	0.102	9.08434785732402\\
59.75	0.1026	9.94949816144873\\
59.75	0.1032	10.8146484655734\\
59.75	0.1038	11.6797987696982\\
59.75	0.1044	12.5449490738228\\
59.75	0.105	13.4100993779475\\
59.75	0.1056	14.2752496820722\\
59.75	0.1062	15.1403999861969\\
59.75	0.1068	16.0055502903217\\
59.75	0.1074	16.8707005944463\\
59.75	0.108	17.735850898571\\
59.75	0.1086	18.6010012026957\\
59.75	0.1092	19.4661515068205\\
59.75	0.1098	20.3313018109452\\
59.75	0.1104	21.1964521150699\\
59.75	0.111	22.0616024191946\\
59.75	0.1116	22.9267527233192\\
59.75	0.1122	23.791903027444\\
59.75	0.1128	24.6570533315687\\
59.75	0.1134	25.5222036356934\\
59.75	0.114	26.3873539398181\\
59.75	0.1146	27.2525042439428\\
59.75	0.1152	28.1176545480675\\
59.75	0.1158	28.9828048521922\\
59.75	0.1164	29.8479551563169\\
59.75	0.117	30.7131054604416\\
59.75	0.1176	31.5782557645663\\
59.75	0.1182	32.443406068691\\
59.75	0.1188	33.3085563728157\\
59.75	0.1194	34.1737066769405\\
59.75	0.12	35.0388569810652\\
59.75	0.1206	35.9040072851899\\
59.75	0.1212	36.7691575893146\\
59.75	0.1218	37.6343078934393\\
59.75	0.1224	38.499458197564\\
59.75	0.123	39.3646085016887\\
60.125	0.093	-4.21579781270111\\
60.125	0.0936	-3.36453583975003\\
60.125	0.0942	-2.51327386679895\\
60.125	0.0948	-1.66201189384788\\
60.125	0.0954	-0.81074992089674\\
60.125	0.096	0.0405120520543392\\
60.125	0.0966	0.891774025005418\\
60.125	0.0972	1.7430359979565\\
60.125	0.0978	2.59429797090752\\
60.125	0.0984	3.44555994385865\\
60.125	0.099	4.29682191680973\\
60.125	0.0996	5.14808388976081\\
60.125	0.1002	5.99934586271183\\
60.125	0.1008	6.85060783566291\\
60.125	0.1014	7.70186980861405\\
60.125	0.102	8.55313178156518\\
60.125	0.1026	9.40439375451626\\
60.125	0.1032	10.2556557274673\\
60.125	0.1038	11.1069177004185\\
60.125	0.1044	11.9581796733695\\
60.125	0.105	12.8094416463206\\
60.125	0.1056	13.6607036192717\\
60.125	0.1062	14.5119655922227\\
60.125	0.1068	15.3632275651739\\
60.125	0.1074	16.2144895381249\\
60.125	0.108	17.065751511076\\
60.125	0.1086	17.9170134840271\\
60.125	0.1092	18.7682754569782\\
60.125	0.1098	19.6195374299293\\
60.125	0.1104	20.4707994028804\\
60.125	0.111	21.3220613758315\\
60.125	0.1116	22.1733233487824\\
60.125	0.1122	23.0245853217336\\
60.125	0.1128	23.8758472946847\\
60.125	0.1134	24.7271092676358\\
60.125	0.114	25.5783712405869\\
60.125	0.1146	26.429633213538\\
60.125	0.1152	27.280895186489\\
60.125	0.1158	28.1321571594401\\
60.125	0.1164	28.9834191323912\\
60.125	0.117	29.8346811053422\\
60.125	0.1176	30.6859430782933\\
60.125	0.1182	31.5372050512444\\
60.125	0.1188	32.3884670241955\\
60.125	0.1194	33.2397289971466\\
60.125	0.12	34.0909909700977\\
60.125	0.1206	34.9422529430488\\
60.125	0.1212	35.7935149159999\\
60.125	0.1218	36.644776888951\\
60.125	0.1224	37.496038861902\\
60.125	0.123	38.3473008348531\\
60.5	0.093	-4.53868892085563\\
60.5	0.0936	-3.70131527907813\\
60.5	0.0942	-2.86394163730068\\
60.5	0.0948	-2.02656799552318\\
60.5	0.0954	-1.18919435374573\\
60.5	0.096	-0.35182071196823\\
60.5	0.0966	0.485552929809216\\
60.5	0.0972	1.32292657158666\\
60.5	0.0978	2.16030021336411\\
60.5	0.0984	2.99767385514156\\
60.5	0.099	3.83504749691906\\
60.5	0.0996	4.67242113869651\\
60.5	0.1002	5.50979478047395\\
60.5	0.1008	6.3471684222514\\
60.5	0.1014	7.18454206402885\\
60.5	0.102	8.02191570580641\\
60.5	0.1026	8.85928934758385\\
60.5	0.1032	9.69666298936136\\
60.5	0.1038	10.5340366311388\\
60.5	0.1044	11.3714102729163\\
60.5	0.105	12.2087839146937\\
60.5	0.1056	13.0461575564711\\
60.5	0.1062	13.8835311982486\\
60.5	0.1068	14.7209048400261\\
60.5	0.1074	15.5582784818035\\
60.5	0.108	16.395652123581\\
60.5	0.1086	17.2330257653585\\
60.5	0.1092	18.070399407136\\
60.5	0.1098	18.9077730489134\\
60.5	0.1104	19.7451466906909\\
60.5	0.111	20.5825203324684\\
60.5	0.1116	21.4198939742458\\
60.5	0.1122	22.2572676160233\\
60.5	0.1128	23.0946412578008\\
60.5	0.1134	23.9320148995782\\
60.5	0.114	24.7693885413557\\
60.5	0.1146	25.6067621831332\\
60.5	0.1152	26.4441358249106\\
60.5	0.1158	27.2815094666881\\
60.5	0.1164	28.1188831084655\\
60.5	0.117	28.956256750243\\
60.5	0.1176	29.7936303920204\\
60.5	0.1182	30.6310040337979\\
60.5	0.1188	31.4683776755754\\
60.5	0.1194	32.3057513173529\\
60.5	0.12	33.1431249591304\\
60.5	0.1206	33.9804986009078\\
60.5	0.1212	34.8178722426853\\
60.5	0.1218	35.6552458844627\\
60.5	0.1224	36.4926195262402\\
60.5	0.123	37.3299931680177\\
60.875	0.093	-4.86158002901016\\
60.875	0.0936	-4.03809471840628\\
60.875	0.0942	-3.21460940780241\\
60.875	0.0948	-2.3911240971986\\
60.875	0.0954	-1.56763878659473\\
60.875	0.096	-0.744153475990913\\
60.875	0.0966	0.0793318346129581\\
60.875	0.0972	0.902817145216829\\
60.875	0.0978	1.72630245582059\\
60.875	0.0984	2.54978776642446\\
60.875	0.099	3.37327307702833\\
60.875	0.0996	4.19675838763214\\
60.875	0.1002	5.02024369823596\\
60.875	0.1008	5.84372900883983\\
60.875	0.1014	6.66721431944364\\
60.875	0.102	7.49069963004757\\
60.875	0.1026	8.31418494065144\\
60.875	0.1032	9.13767025125526\\
60.875	0.1038	9.96115556185913\\
60.875	0.1044	10.7846408724629\\
60.875	0.105	11.6081261830668\\
60.875	0.1056	12.4316114936706\\
60.875	0.1062	13.2550968042744\\
60.875	0.1068	14.0785821148783\\
60.875	0.1074	14.9020674254821\\
60.875	0.108	15.7255527360859\\
60.875	0.1086	16.5490380466898\\
60.875	0.1092	17.3725233572937\\
60.875	0.1098	18.1960086678976\\
60.875	0.1104	19.0194939785014\\
60.875	0.111	19.8429792891053\\
60.875	0.1116	20.666464599709\\
60.875	0.1122	21.4899499103129\\
60.875	0.1128	22.3134352209168\\
60.875	0.1134	23.1369205315206\\
60.875	0.114	23.9604058421245\\
60.875	0.1146	24.7838911527284\\
60.875	0.1152	25.6073764633321\\
60.875	0.1158	26.430861773936\\
60.875	0.1164	27.2543470845398\\
60.875	0.117	28.0778323951436\\
60.875	0.1176	28.9013177057475\\
60.875	0.1182	29.7248030163513\\
60.875	0.1188	30.5482883269552\\
60.875	0.1194	31.3717736375591\\
60.875	0.12	32.1952589481629\\
60.875	0.1206	33.0187442587668\\
60.875	0.1212	33.8422295693707\\
60.875	0.1218	34.6657148799744\\
60.875	0.1224	35.4892001905783\\
60.875	0.123	36.3126855011822\\
61.25	0.093	-5.18447113716462\\
61.25	0.0936	-4.37487415773438\\
61.25	0.0942	-3.56527717830414\\
61.25	0.0948	-2.75568019887396\\
61.25	0.0954	-1.94608321944372\\
61.25	0.096	-1.13648624001348\\
61.25	0.0966	-0.326889260583243\\
61.25	0.0972	0.482707718846996\\
61.25	0.0978	1.29230469827718\\
61.25	0.0984	2.10190167770742\\
61.25	0.099	2.91149865713766\\
61.25	0.0996	3.72109563656784\\
61.25	0.1002	4.53069261599802\\
61.25	0.1008	5.34028959542826\\
61.25	0.1014	6.1498865748585\\
61.25	0.102	6.95948355428879\\
61.25	0.1026	7.76908053371903\\
61.25	0.1032	8.57867751314927\\
61.25	0.1038	9.38827449257946\\
61.25	0.1044	10.1978714720096\\
61.25	0.105	11.0074684514399\\
61.25	0.1056	11.8170654308701\\
61.25	0.1062	12.6266624103004\\
61.25	0.1068	13.4362593897306\\
61.25	0.1074	14.2458563691608\\
61.25	0.108	15.055453348591\\
61.25	0.1086	15.8650503280212\\
61.25	0.1092	16.6746473074515\\
61.25	0.1098	17.4842442868817\\
61.25	0.1104	18.293841266312\\
61.25	0.111	19.1034382457422\\
61.25	0.1116	19.9130352251723\\
61.25	0.1122	20.7226322046026\\
61.25	0.1128	21.5322291840328\\
61.25	0.1134	22.3418261634631\\
61.25	0.114	23.1514231428933\\
61.25	0.1146	23.9610201223235\\
61.25	0.1152	24.7706171017537\\
61.25	0.1158	25.580214081184\\
61.25	0.1164	26.3898110606142\\
61.25	0.117	27.1994080400443\\
61.25	0.1176	28.0090050194746\\
61.25	0.1182	28.8186019989048\\
61.25	0.1188	29.628198978335\\
61.25	0.1194	30.4377959577653\\
61.25	0.12	31.2473929371956\\
61.25	0.1206	32.0569899166258\\
61.25	0.1212	32.866586896056\\
61.25	0.1218	33.6761838754862\\
61.25	0.1224	34.4857808549164\\
61.25	0.123	35.2953778343467\\
61.625	0.093	-5.50736224531914\\
61.625	0.0936	-4.71165359706254\\
61.625	0.0942	-3.91594494880593\\
61.625	0.0948	-3.12023630054932\\
61.625	0.0954	-2.32452765229272\\
61.625	0.096	-1.52881900403611\\
61.625	0.0966	-0.733110355779502\\
61.625	0.0972	0.0625982924771051\\
61.625	0.0978	0.858306940733655\\
61.625	0.0984	1.65401558899026\\
61.625	0.099	2.44972423724687\\
61.625	0.0996	3.24543288550353\\
61.625	0.1002	4.04114153376008\\
61.625	0.1008	4.83685018201669\\
61.625	0.1014	5.6325588302733\\
61.625	0.102	6.42826747852996\\
61.625	0.1026	7.22397612678657\\
61.625	0.1032	8.01968477504317\\
61.625	0.1038	8.81539342329978\\
61.625	0.1044	9.61110207155633\\
61.625	0.105	10.4068107198129\\
61.625	0.1056	11.2025193680695\\
61.625	0.1062	11.9982280163262\\
61.625	0.1068	12.7939366645828\\
61.625	0.1074	13.5896453128394\\
61.625	0.108	14.385353961096\\
61.625	0.1086	15.1810626093526\\
61.625	0.1092	15.9767712576092\\
61.625	0.1098	16.7724799058659\\
61.625	0.1104	17.5681885541225\\
61.625	0.111	18.3638972023791\\
61.625	0.1116	19.1596058506356\\
61.625	0.1122	19.9553144988922\\
61.625	0.1128	20.7510231471488\\
61.625	0.1134	21.5467317954054\\
61.625	0.114	22.3424404436621\\
61.625	0.1146	23.1381490919187\\
61.625	0.1152	23.9338577401753\\
61.625	0.1158	24.7295663884319\\
61.625	0.1164	25.5252750366885\\
61.625	0.117	26.320983684945\\
61.625	0.1176	27.1166923332016\\
61.625	0.1182	27.9124009814582\\
61.625	0.1188	28.7081096297148\\
61.625	0.1194	29.5038182779715\\
61.625	0.12	30.2995269262281\\
61.625	0.1206	31.0952355744847\\
61.625	0.1212	31.8909442227413\\
61.625	0.1218	32.6866528709979\\
61.625	0.1224	33.4823615192545\\
61.625	0.123	34.2780701675111\\
62	0.093	-5.83025335347367\\
62	0.0936	-5.04843303639063\\
62	0.0942	-4.26661271930766\\
62	0.0948	-3.48479240222468\\
62	0.0954	-2.70297208514165\\
62	0.096	-1.92115176805868\\
62	0.0966	-1.1393314509757\\
62	0.0972	-0.357511133892729\\
62	0.0978	0.424309183190246\\
62	0.0984	1.20612950027322\\
62	0.099	1.9879498173562\\
62	0.0996	2.76977013443923\\
62	0.1002	3.55159045152215\\
62	0.1008	4.33341076860512\\
62	0.1014	5.11523108568815\\
62	0.102	5.89705140277118\\
62	0.1026	6.67887171985416\\
62	0.1032	7.46069203693713\\
62	0.1038	8.24251235402016\\
62	0.1044	9.02433267110308\\
62	0.105	9.80615298818606\\
62	0.1056	10.5879733052691\\
62	0.1062	11.3697936223521\\
62	0.1068	12.151613939435\\
62	0.1074	12.933434256518\\
62	0.108	13.715254573601\\
62	0.1086	14.497074890684\\
62	0.1092	15.278895207767\\
62	0.1098	16.06071552485\\
62	0.1104	16.842535841933\\
62	0.111	17.624356159016\\
62	0.1116	18.4061764760989\\
62	0.1122	19.1879967931819\\
62	0.1128	19.9698171102649\\
62	0.1134	20.7516374273479\\
62	0.114	21.5334577444309\\
62	0.1146	22.3152780615139\\
62	0.1152	23.0970983785968\\
62	0.1158	23.8789186956798\\
62	0.1164	24.6607390127628\\
62	0.117	25.4425593298457\\
62	0.1176	26.2243796469288\\
62	0.1182	27.0061999640117\\
62	0.1188	27.7880202810947\\
62	0.1194	28.5698405981778\\
62	0.12	29.3516609152608\\
62	0.1206	30.1334812323437\\
62	0.1212	30.9153015494267\\
62	0.1218	31.6971218665097\\
62	0.1224	32.4789421835927\\
62	0.123	33.2607625006756\\
62.375	0.093	-6.15314446162819\\
62.375	0.0936	-5.38521247571879\\
62.375	0.0942	-4.61728048980945\\
62.375	0.0948	-3.84934850390005\\
62.375	0.0954	-3.0814165179907\\
62.375	0.096	-2.3134845320813\\
62.375	0.0966	-1.54555254617196\\
62.375	0.0972	-0.777620560262562\\
62.375	0.0978	-0.00968857435327664\\
62.375	0.0984	0.758243411556123\\
62.375	0.099	1.52617539746547\\
62.375	0.0996	2.29410738337486\\
62.375	0.1002	3.06203936928415\\
62.375	0.1008	3.82997135519355\\
62.375	0.1014	4.59790334110289\\
62.375	0.102	5.36583532701235\\
62.375	0.1026	6.13376731292169\\
62.375	0.1032	6.90169929883109\\
62.375	0.1038	7.66963128474043\\
62.375	0.1044	8.43756327064978\\
62.375	0.105	9.20549525655912\\
62.375	0.1056	9.97342724246852\\
62.375	0.1062	10.7413592283779\\
62.375	0.1068	11.5092912142873\\
62.375	0.1074	12.2772232001965\\
62.375	0.108	13.0451551861059\\
62.375	0.1086	13.8130871720153\\
62.375	0.1092	14.5810191579247\\
62.375	0.1098	15.3489511438341\\
62.375	0.1104	16.1168831297435\\
62.375	0.111	16.8848151156529\\
62.375	0.1116	17.6527471015621\\
62.375	0.1122	18.4206790874716\\
62.375	0.1128	19.1886110733809\\
62.375	0.1134	19.9565430592903\\
62.375	0.114	20.7244750451997\\
62.375	0.1146	21.4924070311091\\
62.375	0.1152	22.2603390170183\\
62.375	0.1158	23.0282710029277\\
62.375	0.1164	23.7962029888371\\
62.375	0.117	24.5641349747464\\
62.375	0.1176	25.3320669606558\\
62.375	0.1182	26.0999989465652\\
62.375	0.1188	26.8679309324745\\
62.375	0.1194	27.635862918384\\
62.375	0.12	28.4037949042933\\
62.375	0.1206	29.1717268902027\\
62.375	0.1212	29.939658876112\\
62.375	0.1218	30.7075908620214\\
62.375	0.1224	31.4755228479307\\
62.375	0.123	32.2434548338401\\
62.75	0.093	-6.47603556978271\\
62.75	0.0936	-5.72199191504694\\
62.75	0.0942	-4.96794826031118\\
62.75	0.0948	-4.21390460557546\\
62.75	0.0954	-3.4598609508397\\
62.75	0.096	-2.70581729610393\\
62.75	0.0966	-1.95177364136822\\
62.75	0.0972	-1.19772998663245\\
62.75	0.0978	-0.443686331896743\\
62.75	0.0984	0.310357322839025\\
62.75	0.099	1.06440097757473\\
62.75	0.0996	1.8184446323105\\
62.75	0.1002	2.57248828704621\\
62.75	0.1008	3.32653194178198\\
62.75	0.1014	4.08057559651769\\
62.75	0.102	4.83461925125351\\
62.75	0.1026	5.58866290598928\\
62.75	0.1032	6.34270656072499\\
62.75	0.1038	7.09675021546076\\
62.75	0.1044	7.85079387019647\\
62.75	0.105	8.60483752493224\\
62.75	0.1056	9.35888117966795\\
62.75	0.1062	10.1129248344037\\
62.75	0.1068	10.8669684891395\\
62.75	0.1074	11.6210121438752\\
62.75	0.108	12.3750557986109\\
62.75	0.1086	13.1290994533467\\
62.75	0.1092	13.8831431080825\\
62.75	0.1098	14.6371867628182\\
62.75	0.1104	15.391230417554\\
62.75	0.111	16.1452740722897\\
62.75	0.1116	16.8993177270254\\
62.75	0.1122	17.6533613817612\\
62.75	0.1128	18.4074050364969\\
62.75	0.1134	19.1614486912327\\
62.75	0.114	19.9154923459685\\
62.75	0.1146	20.6695360007042\\
62.75	0.1152	21.4235796554399\\
62.75	0.1158	22.1776233101756\\
62.75	0.1164	22.9316669649114\\
62.75	0.117	23.6857106196471\\
62.75	0.1176	24.4397542743828\\
62.75	0.1182	25.1937979291186\\
62.75	0.1188	25.9478415838543\\
62.75	0.1194	26.7018852385901\\
62.75	0.12	27.4559288933259\\
62.75	0.1206	28.2099725480617\\
62.75	0.1212	28.9640162027974\\
62.75	0.1218	29.7180598575331\\
62.75	0.1224	30.4721035122689\\
62.75	0.123	31.2261471670046\\
63.125	0.093	-6.79892667793717\\
63.125	0.0936	-6.05877135437504\\
63.125	0.0942	-5.3186160308129\\
63.125	0.0948	-4.57846070725077\\
63.125	0.0954	-3.83830538368863\\
63.125	0.096	-3.0981500601265\\
63.125	0.0966	-2.35799473656436\\
63.125	0.0972	-1.61783941300229\\
63.125	0.0978	-0.877684089440208\\
63.125	0.0984	-0.137528765878074\\
63.125	0.099	0.602626557684061\\
63.125	0.0996	1.3427818812462\\
63.125	0.1002	2.08293720480827\\
63.125	0.1008	2.82309252837041\\
63.125	0.1014	3.56324785193254\\
63.125	0.102	4.30340317549474\\
63.125	0.1026	5.04355849905687\\
63.125	0.1032	5.78371382261901\\
63.125	0.1038	6.52386914618114\\
63.125	0.1044	7.26402446974322\\
63.125	0.105	8.00417979330535\\
63.125	0.1056	8.74433511686749\\
63.125	0.1062	9.48449044042962\\
63.125	0.1068	10.2246457639918\\
63.125	0.1074	10.9648010875538\\
63.125	0.108	11.7049564111159\\
63.125	0.1086	12.4451117346781\\
63.125	0.1092	13.1852670582402\\
63.125	0.1098	13.9254223818024\\
63.125	0.1104	14.6655777053645\\
63.125	0.111	15.4057330289266\\
63.125	0.1116	16.1458883524887\\
63.125	0.1122	16.8860436760509\\
63.125	0.1128	17.626198999613\\
63.125	0.1134	18.3663543231751\\
63.125	0.114	19.1065096467373\\
63.125	0.1146	19.8466649702994\\
63.125	0.1152	20.5868202938615\\
63.125	0.1158	21.3269756174236\\
63.125	0.1164	22.0671309409857\\
63.125	0.117	22.8072862645478\\
63.125	0.1176	23.54744158811\\
63.125	0.1182	24.2875969116721\\
63.125	0.1188	25.0277522352342\\
63.125	0.1194	25.7679075587964\\
63.125	0.12	26.5080628823585\\
63.125	0.1206	27.2482182059206\\
63.125	0.1212	27.9883735294828\\
63.125	0.1218	28.7285288530448\\
63.125	0.1224	29.468684176607\\
63.125	0.123	30.2088395001691\\
63.5	0.093	-7.1218177860917\\
63.5	0.0936	-6.39555079370319\\
63.5	0.0942	-5.66928380131469\\
63.5	0.0948	-4.94301680892619\\
63.5	0.0954	-4.21674981653769\\
63.5	0.096	-3.49048282414918\\
63.5	0.0966	-2.76421583176062\\
63.5	0.0972	-2.03794883937212\\
63.5	0.0978	-1.31168184698367\\
63.5	0.0984	-0.585414854595172\\
63.5	0.099	0.140852137793331\\
63.5	0.0996	0.867119130181834\\
63.5	0.1002	1.59338612257034\\
63.5	0.1008	2.31965311495884\\
63.5	0.1014	3.04592010734734\\
63.5	0.102	3.7721870997359\\
63.5	0.1026	4.4984540921244\\
63.5	0.1032	5.22472108451291\\
63.5	0.1038	5.95098807690147\\
63.5	0.1044	6.67725506928991\\
63.5	0.105	7.40352206167842\\
63.5	0.1056	8.12978905406692\\
63.5	0.1062	8.85605604645542\\
63.5	0.1068	9.58232303884392\\
63.5	0.1074	10.3085900312324\\
63.5	0.108	11.0348570236209\\
63.5	0.1086	11.7611240160094\\
63.5	0.1092	12.487391008398\\
63.5	0.1098	13.2136580007865\\
63.5	0.1104	13.939924993175\\
63.5	0.111	14.6661919855635\\
63.5	0.1116	15.3924589779519\\
63.5	0.1122	16.1187259703405\\
63.5	0.1128	16.844992962729\\
63.5	0.1134	17.5712599551175\\
63.5	0.114	18.297526947506\\
63.5	0.1146	19.0237939398945\\
63.5	0.1152	19.750060932283\\
63.5	0.1158	20.4763279246715\\
63.5	0.1164	21.20259491706\\
63.5	0.117	21.9288619094485\\
63.5	0.1176	22.655128901837\\
63.5	0.1182	23.3813958942255\\
63.5	0.1188	24.107662886614\\
63.5	0.1194	24.8339298790026\\
63.5	0.12	25.5601968713911\\
63.5	0.1206	26.2864638637796\\
63.5	0.1212	27.0127308561681\\
63.5	0.1218	27.7389978485566\\
63.5	0.1224	28.4652648409451\\
63.5	0.123	29.1915318333336\\
63.875	0.093	-7.44470889424622\\
63.875	0.0936	-6.73233023303129\\
63.875	0.0942	-6.01995157181642\\
63.875	0.0948	-5.30757291060155\\
63.875	0.0954	-4.59519424938662\\
63.875	0.096	-3.88281558817175\\
63.875	0.0966	-3.17043692695682\\
63.875	0.0972	-2.45805826574195\\
63.875	0.0978	-1.74567960452714\\
63.875	0.0984	-1.03330094331221\\
63.875	0.099	-0.320922282097342\\
63.875	0.0996	0.391456379117528\\
63.875	0.1002	1.1038350403324\\
63.875	0.1008	1.81621370154727\\
63.875	0.1014	2.5285923627622\\
63.875	0.102	3.24097102397712\\
63.875	0.1026	3.95334968519199\\
63.875	0.1032	4.66572834640692\\
63.875	0.1038	5.37810700762179\\
63.875	0.1044	6.09048566883666\\
63.875	0.105	6.80286433005153\\
63.875	0.1056	7.5152429912664\\
63.875	0.1062	8.22762165248133\\
63.875	0.1068	8.9400003136962\\
63.875	0.1074	9.65237897491102\\
63.875	0.108	10.3647576361259\\
63.875	0.1086	11.0771362973408\\
63.875	0.1092	11.7895149585558\\
63.875	0.1098	12.5018936197707\\
63.875	0.1104	13.2142722809855\\
63.875	0.111	13.9266509422005\\
63.875	0.1116	14.6390296034152\\
63.875	0.1122	15.3514082646302\\
63.875	0.1128	16.0637869258451\\
63.875	0.1134	16.7761655870599\\
63.875	0.114	17.4885442482749\\
63.875	0.1146	18.2009229094897\\
63.875	0.1152	18.9133015707046\\
63.875	0.1158	19.6256802319195\\
63.875	0.1164	20.3380588931344\\
63.875	0.117	21.0504375543492\\
63.875	0.1176	21.7628162155641\\
63.875	0.1182	22.475194876779\\
63.875	0.1188	23.1875735379939\\
63.875	0.1194	23.8999521992088\\
63.875	0.12	24.6123308604237\\
63.875	0.1206	25.3247095216386\\
63.875	0.1212	26.0370881828535\\
63.875	0.1218	26.7494668440684\\
63.875	0.1224	27.4618455052832\\
63.875	0.123	28.1742241664982\\
64.25	0.093	-7.76760000240074\\
64.25	0.0936	-7.06910967235945\\
64.25	0.0942	-6.37061934231821\\
64.25	0.0948	-5.67212901227691\\
64.25	0.0954	-4.97363868223567\\
64.25	0.096	-4.27514835219438\\
64.25	0.0966	-3.57665802215308\\
64.25	0.0972	-2.87816769211184\\
64.25	0.0978	-2.17967736207061\\
64.25	0.0984	-1.48118703202937\\
64.25	0.099	-0.782696701988073\\
64.25	0.0996	-0.0842063719467774\\
64.25	0.1002	0.614283958094404\\
64.25	0.1008	1.3127742881357\\
64.25	0.1014	2.01126461817694\\
64.25	0.102	2.70975494821829\\
64.25	0.1026	3.40824527825959\\
64.25	0.1032	4.10673560830082\\
64.25	0.1038	4.80522593834212\\
64.25	0.1044	5.5037162683833\\
64.25	0.105	6.2022065984246\\
64.25	0.1056	6.90069692846589\\
64.25	0.1062	7.59918725850713\\
64.25	0.1068	8.29767758854842\\
64.25	0.1074	8.99616791858961\\
64.25	0.108	9.6946582486309\\
64.25	0.1086	10.3931485786722\\
64.25	0.1092	11.0916389087135\\
64.25	0.1098	11.7901292387548\\
64.25	0.1104	12.488619568796\\
64.25	0.111	13.1871098988373\\
64.25	0.1116	13.8856002288785\\
64.25	0.1122	14.5840905589198\\
64.25	0.1128	15.2825808889611\\
64.25	0.1134	15.9810712190023\\
64.25	0.114	16.6795615490436\\
64.25	0.1146	17.3780518790849\\
64.25	0.1152	18.0765422091261\\
64.25	0.1158	18.7750325391674\\
64.25	0.1164	19.4735228692086\\
64.25	0.117	20.1720131992499\\
64.25	0.1176	20.8705035292912\\
64.25	0.1182	21.5689938593324\\
64.25	0.1188	22.2674841893737\\
64.25	0.1194	22.965974519415\\
64.25	0.12	23.6644648494563\\
64.25	0.1206	24.3629551794976\\
64.25	0.1212	25.0614455095388\\
64.25	0.1218	25.7599358395801\\
64.25	0.1224	26.4584261696213\\
64.25	0.123	27.1569164996626\\
64.625	0.093	-8.09049111055521\\
64.625	0.0936	-7.40588911168754\\
64.625	0.0942	-6.72128711281994\\
64.625	0.0948	-6.03668511395227\\
64.625	0.0954	-5.35208311508461\\
64.625	0.096	-4.66748111621695\\
64.625	0.0966	-3.98287911734928\\
64.625	0.0972	-3.29827711848168\\
64.625	0.0978	-2.61367511961407\\
64.625	0.0984	-1.92907312074641\\
64.625	0.099	-1.24447112187875\\
64.625	0.0996	-0.559869123011083\\
64.625	0.1002	0.124732875856523\\
64.625	0.1008	0.809334874724129\\
64.625	0.1014	1.49393687359179\\
64.625	0.102	2.17853887245951\\
64.625	0.1026	2.86314087132718\\
64.625	0.1032	3.54774287019484\\
64.625	0.1038	4.23234486906244\\
64.625	0.1044	4.91694686793005\\
64.625	0.105	5.60154886679771\\
64.625	0.1056	6.28615086566538\\
64.625	0.1062	6.97075286453304\\
64.625	0.1068	7.6553548634007\\
64.625	0.1074	8.33995686226825\\
64.625	0.108	9.02455886113592\\
64.625	0.1086	9.70916086000358\\
64.625	0.1092	10.3937628588713\\
64.625	0.1098	11.078364857739\\
64.625	0.1104	11.7629668566066\\
64.625	0.111	12.4475688554742\\
64.625	0.1116	13.1321708543418\\
64.625	0.1122	13.8167728532095\\
64.625	0.1128	14.5013748520772\\
64.625	0.1134	15.1859768509448\\
64.625	0.114	15.8705788498124\\
64.625	0.1146	16.5551808486801\\
64.625	0.1152	17.2397828475477\\
64.625	0.1158	17.9243848464154\\
64.625	0.1164	18.608986845283\\
64.625	0.117	19.2935888441506\\
64.625	0.1176	19.9781908430182\\
64.625	0.1182	20.6627928418859\\
64.625	0.1188	21.3473948407536\\
64.625	0.1194	22.0319968396213\\
64.625	0.12	22.7165988384889\\
64.625	0.1206	23.4012008373566\\
64.625	0.1212	24.0858028362242\\
64.625	0.1218	24.7704048350918\\
64.625	0.1224	25.4550068339595\\
64.625	0.123	26.1396088328272\\
65	0.093	-8.41338221870973\\
65	0.0936	-7.7426685510157\\
65	0.0942	-7.07195488332161\\
65	0.0948	-6.40124121562758\\
65	0.0954	-5.73052754793355\\
65	0.096	-5.05981388023952\\
65	0.0966	-4.38910021254549\\
65	0.0972	-3.71838654485146\\
65	0.0978	-3.04767287715748\\
65	0.0984	-2.37695920946345\\
65	0.099	-1.70624554176942\\
65	0.0996	-1.03553187407539\\
65	0.1002	-0.364818206381415\\
65	0.1008	0.305895461312616\\
65	0.1014	0.976609129006647\\
65	0.102	1.64732279670073\\
65	0.1026	2.31803646439477\\
65	0.1032	2.9887501320888\\
65	0.1038	3.65946379978283\\
65	0.1044	4.3301774674768\\
65	0.105	5.00089113517083\\
65	0.1056	5.67160480286486\\
65	0.1062	6.34231847055889\\
65	0.1068	7.01303213825292\\
65	0.1074	7.6837458059469\\
65	0.108	8.35445947364093\\
65	0.1086	9.02517314133496\\
65	0.1092	9.69588680902905\\
65	0.1098	10.3666004767231\\
65	0.1104	11.0373141444171\\
65	0.111	11.7080278121112\\
65	0.1116	12.3787414798051\\
65	0.1122	13.0494551474992\\
65	0.1128	13.7201688151932\\
65	0.1134	14.3908824828873\\
65	0.114	15.0615961505813\\
65	0.1146	15.7323098182753\\
65	0.1152	16.4030234859693\\
65	0.1158	17.0737371536633\\
65	0.1164	17.7444508213574\\
65	0.117	18.4151644890513\\
65	0.1176	19.0858781567454\\
65	0.1182	19.7565918244394\\
65	0.1188	20.4273054921334\\
65	0.1194	21.0980191598275\\
65	0.12	21.7687328275215\\
65	0.1206	22.4394464952156\\
65	0.1212	23.1101601629096\\
65	0.1218	23.7808738306036\\
65	0.1224	24.4515874982976\\
65	0.123	25.1223011659916\\
65.375	0.093	-8.73627332686425\\
65.375	0.0936	-8.07944799034385\\
65.375	0.0942	-7.4226226538234\\
65.375	0.0948	-6.765797317303\\
65.375	0.0954	-6.1089719807826\\
65.375	0.096	-5.4521466442622\\
65.375	0.0966	-4.79532130774174\\
65.375	0.0972	-4.13849597122135\\
65.375	0.0978	-3.481670634701\\
65.375	0.0984	-2.82484529818061\\
65.375	0.099	-2.16801996166015\\
65.375	0.0996	-1.51119462513975\\
65.375	0.1002	-0.854369288619409\\
65.375	0.1008	-0.197543952099011\\
65.375	0.1014	0.459281384421445\\
65.375	0.102	1.1161067209419\\
65.375	0.1026	1.7729320574623\\
65.375	0.1032	2.4297573939827\\
65.375	0.1038	3.08658273050315\\
65.375	0.1044	3.74340806702349\\
65.375	0.105	4.40023340354389\\
65.375	0.1056	5.05705874006429\\
65.375	0.1062	5.71388407658475\\
65.375	0.1068	6.37070941310515\\
65.375	0.1074	7.02753474962549\\
65.375	0.108	7.68436008614594\\
65.375	0.1086	8.34118542266634\\
65.375	0.1092	8.9980107591868\\
65.375	0.1098	9.6548360957072\\
65.375	0.1104	10.3116614322277\\
65.375	0.111	10.9684867687481\\
65.375	0.1116	11.6253121052683\\
65.375	0.1122	12.2821374417888\\
65.375	0.1128	12.9389627783092\\
65.375	0.1134	13.5957881148296\\
65.375	0.114	14.25261345135\\
65.375	0.1146	14.9094387878704\\
65.375	0.1152	15.5662641243908\\
65.375	0.1158	16.2230894609112\\
65.375	0.1164	16.8799147974316\\
65.375	0.117	17.536740133952\\
65.375	0.1176	18.1935654704724\\
65.375	0.1182	18.8503908069928\\
65.375	0.1188	19.5072161435132\\
65.375	0.1194	20.1640414800337\\
65.375	0.12	20.8208668165541\\
65.375	0.1206	21.4776921530745\\
65.375	0.1212	22.1345174895949\\
65.375	0.1218	22.7913428261152\\
65.375	0.1224	23.4481681626358\\
65.375	0.123	24.1049934991561\\
65.75	0.093	-9.05916443501877\\
65.75	0.0936	-8.41622742967195\\
65.75	0.0942	-7.77329042432518\\
65.75	0.0948	-7.13035341897836\\
65.75	0.0954	-6.48741641363159\\
65.75	0.096	-5.84447940828483\\
65.75	0.0966	-5.201542402938\\
65.75	0.0972	-4.55860539759124\\
65.75	0.0978	-3.91566839224447\\
65.75	0.0984	-3.2727313868977\\
65.75	0.099	-2.62979438155088\\
65.75	0.0996	-1.98685737620411\\
65.75	0.1002	-1.34392037085735\\
65.75	0.1008	-0.700983365510581\\
65.75	0.1014	-0.0580463601637575\\
65.75	0.102	0.584890645183066\\
65.75	0.1026	1.22782765052989\\
65.75	0.1032	1.87076465587666\\
65.75	0.1038	2.51370166122348\\
65.75	0.1044	3.15663866657019\\
65.75	0.105	3.79957567191695\\
65.75	0.1056	4.44251267726378\\
65.75	0.1062	5.08544968261054\\
65.75	0.1068	5.72838668795737\\
65.75	0.1074	6.37132369330408\\
65.75	0.108	7.0142606986509\\
65.75	0.1086	7.65719770399767\\
65.75	0.1092	8.30013470934455\\
65.75	0.1098	8.94307171469131\\
65.75	0.1104	9.58600872003814\\
65.75	0.111	10.2289457253849\\
65.75	0.1116	10.8718827307316\\
65.75	0.1122	11.5148197360784\\
65.75	0.1128	12.1577567414253\\
65.75	0.1134	12.800693746772\\
65.75	0.114	13.4436307521188\\
65.75	0.1146	14.0865677574656\\
65.75	0.1152	14.7295047628123\\
65.75	0.1158	15.3724417681591\\
65.75	0.1164	16.0153787735059\\
65.75	0.117	16.6583157788527\\
65.75	0.1176	17.3012527841994\\
65.75	0.1182	17.9441897895463\\
65.75	0.1188	18.587126794893\\
65.75	0.1194	19.2300638002399\\
65.75	0.12	19.8730008055867\\
65.75	0.1206	20.5159378109335\\
65.75	0.1212	21.1588748162802\\
65.75	0.1218	21.801811821627\\
65.75	0.1224	22.4447488269739\\
65.75	0.123	23.0876858323205\\
66.125	0.093	-9.38205554317329\\
66.125	0.0936	-8.7530068690001\\
66.125	0.0942	-8.12395819482697\\
66.125	0.0948	-7.49490952065378\\
66.125	0.0954	-6.86586084648059\\
66.125	0.096	-6.23681217230745\\
66.125	0.0966	-5.60776349813426\\
66.125	0.0972	-4.97871482396107\\
66.125	0.0978	-4.34966614978799\\
66.125	0.0984	-3.7206174756148\\
66.125	0.099	-3.09156880144161\\
66.125	0.0996	-2.46252012726848\\
66.125	0.1002	-1.83347145309534\\
66.125	0.1008	-1.20442277892215\\
66.125	0.1014	-0.575374104749017\\
66.125	0.102	0.0536745694242313\\
66.125	0.1026	0.682723243597422\\
66.125	0.1032	1.31177191777061\\
66.125	0.1038	1.94082059194375\\
66.125	0.1044	2.56986926611688\\
66.125	0.105	3.19891794029007\\
66.125	0.1056	3.82796661446321\\
66.125	0.1062	4.4570152886364\\
66.125	0.1068	5.08606396280959\\
66.125	0.1074	5.71511263698267\\
66.125	0.108	6.34416131115586\\
66.125	0.1086	6.97320998532905\\
66.125	0.1092	7.60225865950224\\
66.125	0.1098	8.23130733367543\\
66.125	0.1104	8.86035600784862\\
66.125	0.111	9.48940468202176\\
66.125	0.1116	10.1184533561948\\
66.125	0.1122	10.7475020303681\\
66.125	0.1128	11.3765507045412\\
66.125	0.1134	12.0055993787144\\
66.125	0.114	12.6346480528876\\
66.125	0.1146	13.2636967270607\\
66.125	0.1152	13.8927454012339\\
66.125	0.1158	14.5217940754071\\
66.125	0.1164	15.1508427495802\\
66.125	0.117	15.7798914237533\\
66.125	0.1176	16.4089400979265\\
66.125	0.1182	17.0379887720997\\
66.125	0.1188	17.6670374462728\\
66.125	0.1194	18.2960861204461\\
66.125	0.12	18.9251347946193\\
66.125	0.1206	19.5541834687924\\
66.125	0.1212	20.1832321429656\\
66.125	0.1218	20.8122808171387\\
66.125	0.1224	21.4413294913119\\
66.125	0.123	22.0703781654851\\
66.5	0.093	-9.70494665132776\\
66.5	0.0936	-9.0897863083282\\
66.5	0.0942	-8.4746259653287\\
66.5	0.0948	-7.85946562232914\\
66.5	0.0954	-7.24430527932958\\
66.5	0.096	-6.62914493633002\\
66.5	0.0966	-6.01398459333046\\
66.5	0.0972	-5.3988242503309\\
66.5	0.0978	-4.7836639073314\\
66.5	0.0984	-4.16850356433184\\
66.5	0.099	-3.55334322133228\\
66.5	0.0996	-2.93818287833273\\
66.5	0.1002	-2.32302253533328\\
66.5	0.1008	-1.70786219233372\\
66.5	0.1014	-1.09270184933416\\
66.5	0.102	-0.477541506334546\\
66.5	0.1026	0.137618836665013\\
66.5	0.1032	0.752779179664572\\
66.5	0.1038	1.36793952266413\\
66.5	0.1044	1.98309986566363\\
66.5	0.105	2.59826020866319\\
66.5	0.1056	3.21342055166275\\
66.5	0.1062	3.82858089466225\\
66.5	0.1068	4.44374123766181\\
66.5	0.1074	5.05890158066131\\
66.5	0.108	5.67406192366087\\
66.5	0.1086	6.28922226666043\\
66.5	0.1092	6.90438260966005\\
66.5	0.1098	7.51954295265961\\
66.5	0.1104	8.13470329565916\\
66.5	0.111	8.74986363865872\\
66.5	0.1116	9.36502398165817\\
66.5	0.1122	9.98018432465773\\
66.5	0.1128	10.5953446676573\\
66.5	0.1134	11.2105050106568\\
66.5	0.114	11.8256653536564\\
66.5	0.1146	12.440825696656\\
66.5	0.1152	13.0559860396555\\
66.5	0.1158	13.671146382655\\
66.5	0.1164	14.2863067256546\\
66.5	0.117	14.9014670686541\\
66.5	0.1176	15.5166274116536\\
66.5	0.1182	16.1317877546531\\
66.5	0.1188	16.7469480976527\\
66.5	0.1194	17.3621084406523\\
66.5	0.12	17.9772687836519\\
66.5	0.1206	18.5924291266514\\
66.5	0.1212	19.207589469651\\
66.5	0.1218	19.8227498126505\\
66.5	0.1224	20.4379101556501\\
66.5	0.123	21.0530704986496\\
66.875	0.093	-10.0278377594823\\
66.875	0.0936	-9.42656574765641\\
66.875	0.0942	-8.82529373583048\\
66.875	0.0948	-8.22402172400456\\
66.875	0.0954	-7.62274971217863\\
66.875	0.096	-7.0214777003527\\
66.875	0.0966	-6.42020568852678\\
66.875	0.0972	-5.81893367670085\\
66.875	0.0978	-5.21766166487492\\
66.875	0.0984	-4.616389653049\\
66.875	0.099	-4.01511764122307\\
66.875	0.0996	-3.41384562939714\\
66.875	0.1002	-2.81257361757127\\
66.875	0.1008	-2.21130160574535\\
66.875	0.1014	-1.61002959391942\\
66.875	0.102	-1.00875758209344\\
66.875	0.1026	-0.407485570267511\\
66.875	0.1032	0.193786441558473\\
66.875	0.1038	0.795058453384399\\
66.875	0.1044	1.39633046521027\\
66.875	0.105	1.9976024770362\\
66.875	0.1056	2.59887448886212\\
66.875	0.1062	3.20014650068805\\
66.875	0.1068	3.80141851251398\\
66.875	0.1074	4.40269052433985\\
66.875	0.108	5.00396253616577\\
66.875	0.1086	5.60523454799176\\
66.875	0.1092	6.20650655981774\\
66.875	0.1098	6.80777857164367\\
66.875	0.1104	7.40905058346959\\
66.875	0.111	8.01032259529552\\
66.875	0.1116	8.61159460712133\\
66.875	0.1122	9.21286661894732\\
66.875	0.1128	9.81413863077324\\
66.875	0.1134	10.4154106425992\\
66.875	0.114	11.0166826544252\\
66.875	0.1146	11.6179546662511\\
66.875	0.1152	12.219226678077\\
66.875	0.1158	12.8204986899029\\
66.875	0.1164	13.4217707017288\\
66.875	0.117	14.0230427135547\\
66.875	0.1176	14.6243147253806\\
66.875	0.1182	15.2255867372065\\
66.875	0.1188	15.8268587490325\\
66.875	0.1194	16.4281307608585\\
66.875	0.12	17.0294027726845\\
66.875	0.1206	17.6306747845103\\
66.875	0.1212	18.2319467963363\\
66.875	0.1218	18.8332188081622\\
66.875	0.1224	19.434490819988\\
66.875	0.123	20.0357628318141\\
67.25	0.093	-10.3507288676368\\
67.25	0.0936	-9.76334518698451\\
67.25	0.0942	-9.17596150633216\\
67.25	0.0948	-8.58857782567986\\
67.25	0.0954	-8.00119414502757\\
67.25	0.096	-7.41381046437522\\
67.25	0.0966	-6.82642678372292\\
67.25	0.0972	-6.23904310307063\\
67.25	0.0978	-5.65165942241833\\
67.25	0.0984	-5.06427574176604\\
67.25	0.099	-4.47689206111374\\
67.25	0.0996	-3.88950838046139\\
67.25	0.1002	-3.30212469980916\\
67.25	0.1008	-2.71474101915686\\
67.25	0.1014	-2.12735733850451\\
67.25	0.102	-1.53997365785216\\
67.25	0.1026	-0.952589977199864\\
67.25	0.1032	-0.365206296547512\\
67.25	0.1038	0.222177384104782\\
67.25	0.1044	0.80956106475702\\
67.25	0.105	1.39694474540937\\
67.25	0.1056	1.98432842606167\\
67.25	0.1062	2.57171210671396\\
67.25	0.1068	3.15909578736631\\
67.25	0.1074	3.74647946801855\\
67.25	0.108	4.33386314867084\\
67.25	0.1086	4.9212468293232\\
67.25	0.1092	5.50863050997555\\
67.25	0.1098	6.09601419062784\\
67.25	0.1104	6.68339787128019\\
67.25	0.111	7.27078155193249\\
67.25	0.1116	7.85816523258467\\
67.25	0.1122	8.44554891323708\\
67.25	0.1128	9.03293259388937\\
67.25	0.1134	9.62031627454172\\
67.25	0.114	10.207699955194\\
67.25	0.1146	10.7950836358463\\
67.25	0.1152	11.3824673164986\\
67.25	0.1158	11.9698509971509\\
67.25	0.1164	12.5572346778032\\
67.25	0.117	13.1446183584555\\
67.25	0.1176	13.7320020391078\\
67.25	0.1182	14.3193857197601\\
67.25	0.1188	14.9067694004124\\
67.25	0.1194	15.4941530810648\\
67.25	0.12	16.0815367617171\\
67.25	0.1206	16.6689204423694\\
67.25	0.1212	17.2563041230218\\
67.25	0.1218	17.843687803674\\
67.25	0.1224	18.4310714843263\\
67.25	0.123	19.0184551649787\\
67.625	0.093	-10.6736199757913\\
67.625	0.0936	-10.1001246263126\\
67.625	0.0942	-9.52662927683394\\
67.625	0.0948	-8.95313392735522\\
67.625	0.0954	-8.37963857787656\\
67.625	0.096	-7.80614322839784\\
67.625	0.0966	-7.23264787891918\\
67.625	0.0972	-6.65915252944046\\
67.625	0.0978	-6.08565717996186\\
67.625	0.0984	-5.51216183048314\\
67.625	0.099	-4.93866648100442\\
67.625	0.0996	-4.36517113152576\\
67.625	0.1002	-3.79167578204709\\
67.625	0.1008	-3.21818043256843\\
67.625	0.1014	-2.64468508308971\\
67.625	0.102	-2.07118973361099\\
67.625	0.1026	-1.49769438413227\\
67.625	0.1032	-0.924199034653611\\
67.625	0.1038	-0.350703685174892\\
67.625	0.1044	0.22279166430377\\
67.625	0.105	0.796287013782432\\
67.625	0.1056	1.36978236326115\\
67.625	0.1062	1.94327771273981\\
67.625	0.1068	2.51677306221853\\
67.625	0.1074	3.09026841169714\\
67.625	0.108	3.66376376117586\\
67.625	0.1086	4.23725911065452\\
67.625	0.1092	4.8107544601333\\
67.625	0.1098	5.38424980961196\\
67.625	0.1104	5.95774515909068\\
67.625	0.111	6.5312405085694\\
67.625	0.1116	7.10473585804795\\
67.625	0.1122	7.67823120752672\\
67.625	0.1128	8.25172655700538\\
67.625	0.1134	8.8252219064841\\
67.625	0.114	9.39871725596277\\
67.625	0.1146	9.97221260544148\\
67.625	0.1152	10.5457079549201\\
67.625	0.1158	11.1192033043988\\
67.625	0.1164	11.6926986538775\\
67.625	0.117	12.2661940033561\\
67.625	0.1176	12.8396893528349\\
67.625	0.1182	13.4131847023135\\
67.625	0.1188	13.9866800517922\\
67.625	0.1194	14.5601754012709\\
67.625	0.12	15.1336707507497\\
67.625	0.1206	15.7071661002283\\
67.625	0.1212	16.2806614497071\\
67.625	0.1218	16.8541567991857\\
67.625	0.1224	17.4276521486643\\
67.625	0.123	18.0011474981432\\
68	0.093	-10.9965110839458\\
68	0.0936	-10.4369040656408\\
68	0.0942	-9.87729704733567\\
68	0.0948	-9.31769002903059\\
68	0.0954	-8.7580830107255\\
68	0.096	-8.19847599242047\\
68	0.0966	-7.63886897411538\\
68	0.0972	-7.0792619558103\\
68	0.0978	-6.51965493750527\\
68	0.0984	-5.96004791920024\\
68	0.099	-5.40044090089515\\
68	0.0996	-4.84083388259006\\
68	0.1002	-4.28122686428503\\
68	0.1008	-3.72161984597994\\
68	0.1014	-3.16201282767491\\
68	0.102	-2.60240580936977\\
68	0.1026	-2.04279879106468\\
68	0.1032	-1.4831917727596\\
68	0.1038	-0.923584754454566\\
68	0.1044	-0.363977736149536\\
68	0.105	0.195629282155551\\
68	0.1056	0.755236300460638\\
68	0.1062	1.31484331876567\\
68	0.1068	1.87445033707075\\
68	0.1074	2.43405735537578\\
68	0.108	2.99366437368087\\
68	0.1086	3.55327139198596\\
68	0.1092	4.11287841029105\\
68	0.1098	4.67248542859613\\
68	0.1104	5.23209244690122\\
68	0.111	5.79169946520631\\
68	0.1116	6.35130648351122\\
68	0.1122	6.91091350181637\\
68	0.1128	7.47052052012145\\
68	0.1134	8.03012753842654\\
68	0.114	8.58973455673163\\
68	0.1146	9.14934157503666\\
68	0.1152	9.70894859334169\\
68	0.1158	10.2685556116468\\
68	0.1164	10.8281626299519\\
68	0.117	11.3877696482568\\
68	0.1176	11.9473766665619\\
68	0.1182	12.506983684867\\
68	0.1188	13.0665907031721\\
68	0.1194	13.6261977214772\\
68	0.12	14.1858047397823\\
68	0.1206	14.7454117580874\\
68	0.1212	15.3050187763924\\
68	0.1218	15.8646257946975\\
68	0.1224	16.4242328130025\\
68	0.123	16.9838398313076\\
68.375	0.093	-11.3194021921003\\
68.375	0.0936	-10.7736835049689\\
68.375	0.0942	-10.2279648178375\\
68.375	0.0948	-9.68224613070601\\
68.375	0.0954	-9.13652744357455\\
68.375	0.096	-8.5908087564431\\
68.375	0.0966	-8.04509006931164\\
68.375	0.0972	-7.49937138218019\\
68.375	0.0978	-6.95365269504879\\
68.375	0.0984	-6.40793400791733\\
68.375	0.099	-5.86221532078588\\
68.375	0.0996	-5.31649663365442\\
68.375	0.1002	-4.77077794652303\\
68.375	0.1008	-4.22505925939157\\
68.375	0.1014	-3.67934057226012\\
68.375	0.102	-3.1336218851286\\
68.375	0.1026	-2.58790319799715\\
68.375	0.1032	-2.0421845108657\\
68.375	0.1038	-1.49646582373424\\
68.375	0.1044	-0.950747136602843\\
68.375	0.105	-0.405028449471388\\
68.375	0.1056	0.140690237660067\\
68.375	0.1062	0.686408924791522\\
68.375	0.1068	1.23212761192298\\
68.375	0.1074	1.77784629905437\\
68.375	0.108	2.32356498618583\\
68.375	0.1086	2.86928367331728\\
68.375	0.1092	3.4150023604488\\
68.375	0.1098	3.96072104758025\\
68.375	0.1104	4.50643973471171\\
68.375	0.111	5.05215842184316\\
68.375	0.1116	5.5978771089745\\
68.375	0.1122	6.14359579610601\\
68.375	0.1128	6.68931448323747\\
68.375	0.1134	7.23503317036892\\
68.375	0.114	7.78075185750038\\
68.375	0.1146	8.32647054463183\\
68.375	0.1152	8.87218923176323\\
68.375	0.1158	9.41790791889468\\
68.375	0.1164	9.96362660602614\\
68.375	0.117	10.5093452931575\\
68.375	0.1176	11.055063980289\\
68.375	0.1182	11.6007826674204\\
68.375	0.1188	12.1465013545519\\
68.375	0.1194	12.6922200416834\\
68.375	0.12	13.2379387288149\\
68.375	0.1206	13.7836574159463\\
68.375	0.1212	14.3293761030778\\
68.375	0.1218	14.8750947902091\\
68.375	0.1224	15.4208134773406\\
68.375	0.123	15.966532164472\\
68.75	0.093	-11.6422933002548\\
68.75	0.0936	-11.110462944297\\
68.75	0.0942	-10.5786325883392\\
68.75	0.0948	-10.0468022323813\\
68.75	0.0954	-9.51497187642349\\
68.75	0.096	-8.98314152046567\\
68.75	0.0966	-8.45131116450784\\
68.75	0.0972	-7.91948080855002\\
68.75	0.0978	-7.3876504525922\\
68.75	0.0984	-6.85582009663437\\
68.75	0.099	-6.32398974067655\\
68.75	0.0996	-5.79215938471873\\
68.75	0.1002	-5.26032902876096\\
68.75	0.1008	-4.72849867280308\\
68.75	0.1014	-4.19666831684526\\
68.75	0.102	-3.66483796088738\\
68.75	0.1026	-3.13300760492956\\
68.75	0.1032	-2.60117724897168\\
68.75	0.1038	-2.06934689301386\\
68.75	0.1044	-1.53751653705609\\
68.75	0.105	-1.00568618109827\\
68.75	0.1056	-0.473855825140447\\
68.75	0.1062	0.0579745308174324\\
68.75	0.1068	0.589804886775255\\
68.75	0.1074	1.12163524273302\\
68.75	0.108	1.65346559869084\\
68.75	0.1086	2.18529595464867\\
68.75	0.1092	2.7171263106066\\
68.75	0.1098	3.24895666656442\\
68.75	0.1104	3.78078702252225\\
68.75	0.111	4.31261737848007\\
68.75	0.1116	4.84444773443778\\
68.75	0.1122	5.37627809039572\\
68.75	0.1128	5.90810844635354\\
68.75	0.1134	6.43993880231136\\
68.75	0.114	6.97176915826918\\
68.75	0.1146	7.50359951422701\\
68.75	0.1152	8.03542987018483\\
68.75	0.1158	8.56726022614265\\
68.75	0.1164	9.09909058210047\\
68.75	0.117	9.63092093805824\\
68.75	0.1176	10.1627512940161\\
68.75	0.1182	10.694581649974\\
68.75	0.1188	11.2264120059317\\
68.75	0.1194	11.7582423618897\\
68.75	0.12	12.2900727178474\\
68.75	0.1206	12.8219030738053\\
68.75	0.1212	13.3537334297631\\
68.75	0.1218	13.8855637857208\\
68.75	0.1224	14.4173941416789\\
68.75	0.123	14.9492244976366\\
69.125	0.093	-11.9651844084094\\
69.125	0.0936	-11.4472423836252\\
69.125	0.0942	-10.9293003588409\\
69.125	0.0948	-10.4113583340567\\
69.125	0.0954	-9.89341630927248\\
69.125	0.096	-9.37547428448829\\
69.125	0.0966	-8.8575322597041\\
69.125	0.0972	-8.33959023491985\\
69.125	0.0978	-7.82164821013572\\
69.125	0.0984	-7.30370618535147\\
69.125	0.099	-6.78576416056728\\
69.125	0.0996	-6.26782213578304\\
69.125	0.1002	-5.7498801109989\\
69.125	0.1008	-5.23193808621471\\
69.125	0.1014	-4.71399606143046\\
69.125	0.102	-4.19605403664622\\
69.125	0.1026	-3.67811201186197\\
69.125	0.1032	-3.16016998707778\\
69.125	0.1038	-2.64222796229353\\
69.125	0.1044	-2.1242859375094\\
69.125	0.105	-1.60634391272521\\
69.125	0.1056	-1.08840188794096\\
69.125	0.1062	-0.570459863156771\\
69.125	0.1068	-0.0525178383725233\\
69.125	0.1074	0.46542418641161\\
69.125	0.108	0.983366211195801\\
69.125	0.1086	1.50130823598005\\
69.125	0.1092	2.0192502607643\\
69.125	0.1098	2.53719228554854\\
69.125	0.1104	3.05513431033273\\
69.125	0.111	3.57307633511698\\
69.125	0.1116	4.09101835990106\\
69.125	0.1122	4.6089603846853\\
69.125	0.1128	5.12690240946955\\
69.125	0.1134	5.64484443425374\\
69.125	0.114	6.16278645903799\\
69.125	0.1146	6.68072848382218\\
69.125	0.1152	7.19867050860631\\
69.125	0.1158	7.71661253339056\\
69.125	0.1164	8.23455455817475\\
69.125	0.117	8.75249658295894\\
69.125	0.1176	9.27043860774313\\
69.125	0.1182	9.78838063252743\\
69.125	0.1188	10.3063226573115\\
69.125	0.1194	10.8242646820959\\
69.125	0.12	11.34220670688\\
69.125	0.1206	11.8601487316644\\
69.125	0.1212	12.3780907564484\\
69.125	0.1218	12.8960327812326\\
69.125	0.1224	13.4139748060169\\
69.125	0.123	13.931916830801\\
69.5	0.093	-12.2880755165638\\
69.5	0.0936	-11.7840218229532\\
69.5	0.0942	-11.2799681293426\\
69.5	0.0948	-10.775914435732\\
69.5	0.0954	-10.2718607421214\\
69.5	0.096	-9.7678070485108\\
69.5	0.0966	-9.26375335490025\\
69.5	0.0972	-8.75969966128963\\
69.5	0.0978	-8.25564596767907\\
69.5	0.0984	-7.75159227406851\\
69.5	0.099	-7.2475385804579\\
69.5	0.0996	-6.74348488684728\\
69.5	0.1002	-6.23943119323678\\
69.5	0.1008	-5.73537749962617\\
69.5	0.1014	-5.23132380601561\\
69.5	0.102	-4.72727011240494\\
69.5	0.1026	-4.22321641879432\\
69.5	0.1032	-3.71916272518376\\
69.5	0.1038	-3.21510903157315\\
69.5	0.1044	-2.71105533796259\\
69.5	0.105	-2.20700164435203\\
69.5	0.1056	-1.70294795074142\\
69.5	0.1062	-1.1988942571308\\
69.5	0.1068	-0.694840563520245\\
69.5	0.1074	-0.190786869909687\\
69.5	0.108	0.313266823700872\\
69.5	0.1086	0.817320517311487\\
69.5	0.1092	1.32137421092216\\
69.5	0.1098	1.82542790453272\\
69.5	0.1104	2.32948159814333\\
69.5	0.111	2.83353529175395\\
69.5	0.1116	3.33758898536439\\
69.5	0.1122	3.84164267897506\\
69.5	0.1128	4.34569637258568\\
69.5	0.1134	4.84975006619624\\
69.5	0.114	5.35380375980685\\
69.5	0.1146	5.85785745341747\\
69.5	0.1152	6.36191114702797\\
69.5	0.1158	6.86596484063858\\
69.5	0.1164	7.37001853424914\\
69.5	0.117	7.8740722278597\\
69.5	0.1176	8.37812592147026\\
69.5	0.1182	8.88217961508093\\
69.5	0.1188	9.38623330869143\\
69.5	0.1194	9.89028700230222\\
69.5	0.12	10.3943406959127\\
69.5	0.1206	10.8983943895233\\
69.5	0.1212	11.4024480831339\\
69.5	0.1218	11.9065017767444\\
69.5	0.1224	12.4105554703551\\
69.5	0.123	12.9146091639656\\
69.875	0.093	-12.6109666247184\\
69.875	0.0936	-12.1208012622814\\
69.875	0.0942	-11.6306358998444\\
69.875	0.0948	-11.1404705374075\\
69.875	0.0954	-10.6503051749705\\
69.875	0.096	-10.1601398125335\\
69.875	0.0966	-9.6699744500965\\
69.875	0.0972	-9.17980908765958\\
69.875	0.0978	-8.68964372522265\\
69.875	0.0984	-8.19947836278567\\
69.875	0.099	-7.70931300034869\\
69.875	0.0996	-7.2191476379117\\
69.875	0.1002	-6.72898227547478\\
69.875	0.1008	-6.23881691303779\\
69.875	0.1014	-5.74865155060087\\
69.875	0.102	-5.25848618816383\\
69.875	0.1026	-4.76832082572685\\
69.875	0.1032	-4.27815546328986\\
69.875	0.1038	-3.78799010085288\\
69.875	0.1044	-3.29782473841595\\
69.875	0.105	-2.80765937597897\\
69.875	0.1056	-2.31749401354205\\
69.875	0.1062	-1.82732865110506\\
69.875	0.1068	-1.33716328866808\\
69.875	0.1074	-0.846997926231154\\
69.875	0.108	-0.356832563794171\\
69.875	0.1086	0.133332798642812\\
69.875	0.1092	0.623498161079851\\
69.875	0.1098	1.11366352351678\\
69.875	0.1104	1.60382888595376\\
69.875	0.111	2.09399424839074\\
69.875	0.1116	2.58415961082761\\
69.875	0.1122	3.07432497326465\\
69.875	0.1128	3.56449033570163\\
69.875	0.1134	4.05465569813856\\
69.875	0.114	4.54482106057554\\
69.875	0.1146	5.03498642301253\\
69.875	0.1152	5.52515178544945\\
69.875	0.1158	6.01531714788644\\
69.875	0.1164	6.50548251032342\\
69.875	0.117	6.99564787276034\\
69.875	0.1176	7.48581323519727\\
69.875	0.1182	7.97597859763425\\
69.875	0.1188	8.46614396007124\\
69.875	0.1194	8.95630932250833\\
69.875	0.12	9.4464746849452\\
69.875	0.1206	9.93664004738218\\
69.875	0.1212	10.4268054098192\\
69.875	0.1218	10.9169707722561\\
69.875	0.1224	11.4071361346931\\
69.875	0.123	11.8973014971301\\
70.25	0.093	-12.9338577328729\\
70.25	0.0936	-12.4575807016096\\
70.25	0.0942	-11.9813036703462\\
70.25	0.0948	-11.5050266390828\\
70.25	0.0954	-11.0287496078195\\
70.25	0.096	-10.5524725765561\\
70.25	0.0966	-10.0761955452928\\
70.25	0.0972	-9.59991851402941\\
70.25	0.0978	-9.12364148276612\\
70.25	0.0984	-8.64736445150277\\
70.25	0.099	-8.17108742023942\\
70.25	0.0996	-7.69481038897607\\
70.25	0.1002	-7.21853335771277\\
70.25	0.1008	-6.74225632644942\\
70.25	0.1014	-6.26597929518607\\
70.25	0.102	-5.78970226392266\\
70.25	0.1026	-5.31342523265926\\
70.25	0.1032	-4.83714820139591\\
70.25	0.1038	-4.36087117013255\\
70.25	0.1044	-3.88459413886926\\
70.25	0.105	-3.40831710760591\\
70.25	0.1056	-2.93204007634256\\
70.25	0.1062	-2.45576304507921\\
70.25	0.1068	-1.97948601381586\\
70.25	0.1074	-1.50320898255256\\
70.25	0.108	-1.02693195128921\\
70.25	0.1086	-0.550654920025863\\
70.25	0.1092	-0.0743778887624558\\
70.25	0.1098	0.401899142500895\\
70.25	0.1104	0.878176173764246\\
70.25	0.111	1.3544532050276\\
70.25	0.1116	1.83073023629083\\
70.25	0.1122	2.3070072675543\\
70.25	0.1128	2.78328429881765\\
70.25	0.1134	3.259561330081\\
70.25	0.114	3.73583836134435\\
70.25	0.1146	4.2121153926077\\
70.25	0.1152	4.68839242387099\\
70.25	0.1158	5.16466945513434\\
70.25	0.1164	5.6409464863977\\
70.25	0.117	6.11722351766099\\
70.25	0.1176	6.59350054892434\\
70.25	0.1182	7.06977758018775\\
70.25	0.1188	7.54605461145104\\
70.25	0.1194	8.02233164271445\\
70.25	0.12	8.49860867397786\\
70.25	0.1206	8.97488570524115\\
70.25	0.1212	9.45116273650456\\
70.25	0.1218	9.92743976776785\\
70.25	0.1224	10.4037167990311\\
70.25	0.123	10.8799938302946\\
70.625	0.093	-13.2567488410274\\
70.625	0.0936	-12.7943601409377\\
70.625	0.0942	-12.331971440848\\
70.625	0.0948	-11.8695827407582\\
70.625	0.0954	-11.4071940406685\\
70.625	0.096	-10.9448053405787\\
70.625	0.0966	-10.482416640489\\
70.625	0.0972	-10.0200279403992\\
70.625	0.0978	-9.55763924030958\\
70.625	0.0984	-9.09525054021987\\
70.625	0.099	-8.63286184013009\\
70.625	0.0996	-8.17047314004037\\
70.625	0.1002	-7.70808443995071\\
70.625	0.1008	-7.24569573986093\\
70.625	0.1014	-6.78330703977122\\
70.625	0.102	-6.32091833968144\\
70.625	0.1026	-5.85852963959167\\
70.625	0.1032	-5.39614093950195\\
70.625	0.1038	-4.93375223941223\\
70.625	0.1044	-4.47136353932257\\
70.625	0.105	-4.00897483923279\\
70.625	0.1056	-3.54658613914307\\
70.625	0.1062	-3.08419743905336\\
70.625	0.1068	-2.62180873896358\\
70.625	0.1074	-2.15942003887392\\
70.625	0.108	-1.6970313387842\\
70.625	0.1086	-1.23464263869448\\
70.625	0.1092	-0.772253938604649\\
70.625	0.1098	-0.309865238514931\\
70.625	0.1104	0.152523461574788\\
70.625	0.111	0.614912161664563\\
70.625	0.1116	1.07730086175417\\
70.625	0.1122	1.53968956184394\\
70.625	0.1128	2.00207826193366\\
70.625	0.1134	2.46446696202344\\
70.625	0.114	2.92685566211316\\
70.625	0.1146	3.38924436220287\\
70.625	0.1152	3.85163306229259\\
70.625	0.1158	4.31402176238231\\
70.625	0.1164	4.77641046247197\\
70.625	0.117	5.23879916256169\\
70.625	0.1176	5.70118786265152\\
70.625	0.1182	6.16357656274113\\
70.625	0.1188	6.62596526283096\\
70.625	0.1194	7.08835396292068\\
70.625	0.12	7.55074266301051\\
70.625	0.1206	8.01313136310011\\
70.625	0.1212	8.47552006318995\\
70.625	0.1218	8.93790876327967\\
70.625	0.1224	9.40029746336927\\
70.625	0.123	9.8626861634591\\
71	0.093	-13.5796399491819\\
71	0.0936	-13.1311395802658\\
71	0.0942	-12.6826392113497\\
71	0.0948	-12.2341388424336\\
71	0.0954	-11.7856384735175\\
71	0.096	-11.3371381046014\\
71	0.0966	-10.8886377356852\\
71	0.0972	-10.4401373667691\\
71	0.0978	-9.99163699785305\\
71	0.0984	-9.54313662893696\\
71	0.099	-9.09463626002082\\
71	0.0996	-8.64613589110473\\
71	0.1002	-8.19763552218865\\
71	0.1008	-7.74913515327256\\
71	0.1014	-7.30063478435642\\
71	0.102	-6.85213441544028\\
71	0.1026	-6.40363404652413\\
71	0.1032	-5.95513367760805\\
71	0.1038	-5.5066333086919\\
71	0.1044	-5.05813293977587\\
71	0.105	-4.60963257085973\\
71	0.1056	-4.16113220194364\\
71	0.1062	-3.7126318330275\\
71	0.1068	-3.26413146411141\\
71	0.1074	-2.81563109519533\\
71	0.108	-2.36713072627924\\
71	0.1086	-1.9186303573631\\
71	0.1092	-1.47012998844696\\
71	0.1098	-1.02162961953081\\
71	0.1104	-0.573129250614727\\
71	0.111	-0.124628881698584\\
71	0.1116	0.323871487217389\\
71	0.1122	0.772371856133589\\
71	0.1128	1.22087222504967\\
71	0.1134	1.66937259396582\\
71	0.114	2.1178729628819\\
71	0.1146	2.56637333179805\\
71	0.1152	3.01487370071408\\
71	0.1158	3.46337406963022\\
71	0.1164	3.91187443854625\\
71	0.117	4.36037480746234\\
71	0.1176	4.80887517637859\\
71	0.1182	5.25737554529451\\
71	0.1188	5.70587591421076\\
71	0.1194	6.15437628312691\\
71	0.12	6.60287665204305\\
71	0.1206	7.05137702095908\\
71	0.1212	7.49987738987534\\
71	0.1218	7.94837775879137\\
71	0.1224	8.3968781277074\\
71	0.123	8.84537849662365\\
71.375	0.093	-13.9025310573364\\
71.375	0.0936	-13.4679190195939\\
71.375	0.0942	-13.0333069818514\\
71.375	0.0948	-12.5986949441088\\
71.375	0.0954	-12.1640829063664\\
71.375	0.096	-11.7294708686239\\
71.375	0.0966	-11.2948588308814\\
71.375	0.0972	-10.8602467931389\\
71.375	0.0978	-10.4256347553965\\
71.375	0.0984	-9.99102271765395\\
71.375	0.099	-9.55641067991144\\
71.375	0.0996	-9.12179864216898\\
71.375	0.1002	-8.68718660442653\\
71.375	0.1008	-8.25257456668402\\
71.375	0.1014	-7.81796252894156\\
71.375	0.102	-7.383350491199\\
71.375	0.1026	-6.94873845345649\\
71.375	0.1032	-6.51412641571397\\
71.375	0.1038	-6.07951437797152\\
71.375	0.1044	-5.64490234022907\\
71.375	0.105	-5.21029030248656\\
71.375	0.1056	-4.77567826474404\\
71.375	0.1062	-4.34106622700159\\
71.375	0.1068	-3.90645418925908\\
71.375	0.1074	-3.47184215151663\\
71.375	0.108	-3.03723011377417\\
71.375	0.1086	-2.60261807603166\\
71.375	0.1092	-2.16800603828909\\
71.375	0.1098	-1.73339400054658\\
71.375	0.1104	-1.29878196280413\\
71.375	0.111	-0.864169925061617\\
71.375	0.1116	-0.42955788731922\\
71.375	0.1122	0.00505415042334789\\
71.375	0.1128	0.439666188165802\\
71.375	0.1134	0.874278225908313\\
71.375	0.114	1.30889026365082\\
71.375	0.1146	1.74350230139328\\
71.375	0.1152	2.17811433913573\\
71.375	0.1158	2.6127263768783\\
71.375	0.1164	3.04733841462064\\
71.375	0.117	3.48195045236309\\
71.375	0.1176	3.91656249010578\\
71.375	0.1182	4.35117452784812\\
71.375	0.1188	4.78578656559068\\
71.375	0.1194	5.22039860333314\\
71.375	0.12	5.65501064107571\\
71.375	0.1206	6.08962267881816\\
71.375	0.1212	6.52423471656073\\
71.375	0.1218	6.95884675430318\\
71.375	0.1224	7.39345879204552\\
71.375	0.123	7.8280708297882\\
71.75	0.093	-14.225422165491\\
71.75	0.0936	-13.8046984589221\\
71.75	0.0942	-13.3839747523532\\
71.75	0.0948	-12.9632510457843\\
71.75	0.0954	-12.5425273392154\\
71.75	0.096	-12.1218036326466\\
71.75	0.0966	-11.7010799260777\\
71.75	0.0972	-11.2803562195088\\
71.75	0.0978	-10.85963251294\\
71.75	0.0984	-10.4389088063711\\
71.75	0.099	-10.0181850998022\\
71.75	0.0996	-9.59746139323335\\
71.75	0.1002	-9.17673768666452\\
71.75	0.1008	-8.75601398009564\\
71.75	0.1014	-8.33529027352682\\
71.75	0.102	-7.91456656695789\\
71.75	0.1026	-7.49384286038901\\
71.75	0.1032	-7.07311915382013\\
71.75	0.1038	-6.65239544725125\\
71.75	0.1044	-6.23167174068243\\
71.75	0.105	-5.81094803411355\\
71.75	0.1056	-5.39022432754467\\
71.75	0.1062	-4.96950062097579\\
71.75	0.1068	-4.54877691440691\\
71.75	0.1074	-4.12805320783809\\
71.75	0.108	-3.70732950126921\\
71.75	0.1086	-3.28660579470034\\
71.75	0.1092	-2.8658820881314\\
71.75	0.1098	-2.44515838156252\\
71.75	0.1104	-2.02443467499364\\
71.75	0.111	-1.60371096842482\\
71.75	0.1116	-1.18298726185606\\
71.75	0.1122	-0.76226355528712\\
71.75	0.1128	-0.341539848718242\\
71.75	0.1134	0.0791838578506372\\
71.75	0.114	0.499907564419516\\
71.75	0.1146	0.920631270988395\\
71.75	0.1152	1.34135497755722\\
71.75	0.1158	1.7620786841261\\
71.75	0.1164	2.18280239069497\\
71.75	0.117	2.60352609726374\\
71.75	0.1176	3.02424980383262\\
71.75	0.1182	3.4449735104015\\
71.75	0.1188	3.86569721697037\\
71.75	0.1194	4.28642092353937\\
71.75	0.12	4.70714463010825\\
71.75	0.1206	5.12786833667712\\
71.75	0.1212	5.548592043246\\
71.75	0.1218	5.96931574981477\\
71.75	0.1224	6.39003945638365\\
71.75	0.123	6.81076316295253\\
72.125	0.093	-14.5483132736454\\
72.125	0.0936	-14.1414778982501\\
72.125	0.0942	-13.7346425228549\\
72.125	0.0948	-13.3278071474596\\
72.125	0.0954	-12.9209717720644\\
72.125	0.096	-12.5141363966691\\
72.125	0.0966	-12.1073010212738\\
72.125	0.0972	-11.7004656458786\\
72.125	0.0978	-11.2936302704834\\
72.125	0.0984	-10.8867948950881\\
72.125	0.099	-10.4799595196929\\
72.125	0.0996	-10.0731241442976\\
72.125	0.1002	-9.6662887689024\\
72.125	0.1008	-9.25945339350716\\
72.125	0.1014	-8.85261801811191\\
72.125	0.102	-8.44578264271661\\
72.125	0.1026	-8.03894726732136\\
72.125	0.1032	-7.63211189192606\\
72.125	0.1038	-7.22527651653081\\
72.125	0.1044	-6.81844114113562\\
72.125	0.105	-6.41160576574038\\
72.125	0.1056	-6.00477039034513\\
72.125	0.1062	-5.59793501494988\\
72.125	0.1068	-5.19109963955464\\
72.125	0.1074	-4.78426426415939\\
72.125	0.108	-4.37742888876414\\
72.125	0.1086	-3.9705935133689\\
72.125	0.1092	-3.56375813797359\\
72.125	0.1098	-3.15692276257835\\
72.125	0.1104	-2.7500873871831\\
72.125	0.111	-2.3432520117878\\
72.125	0.1116	-1.93641663639266\\
72.125	0.1122	-1.52958126099736\\
72.125	0.1128	-1.12274588560211\\
72.125	0.1134	-0.715910510206868\\
72.125	0.114	-0.309075134811621\\
72.125	0.1146	0.0977602405836819\\
72.125	0.1152	0.504595615978815\\
72.125	0.1158	0.911430991374118\\
72.125	0.1164	1.31826636676942\\
72.125	0.117	1.72510174216461\\
72.125	0.1176	2.1319371175598\\
72.125	0.1182	2.5387724929551\\
72.125	0.1188	2.94560786835029\\
72.125	0.1194	3.3524432437456\\
72.125	0.12	3.7592786191409\\
72.125	0.1206	4.1661139945362\\
72.125	0.1212	4.57294936993139\\
72.125	0.1218	4.97978474532658\\
72.125	0.1224	5.38662012072189\\
72.125	0.123	5.79345549611708\\
72.5	0.093	-14.8712043817999\\
72.5	0.0936	-14.4782573375783\\
72.5	0.0942	-14.0853102933567\\
72.5	0.0948	-13.692363249135\\
72.5	0.0954	-13.2994162049134\\
72.5	0.096	-12.9064691606918\\
72.5	0.0966	-12.5135221164701\\
72.5	0.0972	-12.1205750722485\\
72.5	0.0978	-11.7276280280269\\
72.5	0.0984	-11.3346809838053\\
72.5	0.099	-10.9417339395837\\
72.5	0.0996	-10.548786895362\\
72.5	0.1002	-10.1558398511405\\
72.5	0.1008	-9.76289280691879\\
72.5	0.1014	-9.36994576269717\\
72.5	0.102	-8.9769987184755\\
72.5	0.1026	-8.58405167425383\\
72.5	0.1032	-8.19110463003221\\
72.5	0.1038	-7.7981575858106\\
72.5	0.1044	-7.40521054158899\\
72.5	0.105	-7.01226349736737\\
72.5	0.1056	-6.6193164531457\\
72.5	0.1062	-6.22636940892409\\
72.5	0.1068	-5.83342236470247\\
72.5	0.1074	-5.44047532048086\\
72.5	0.108	-5.04752827625924\\
72.5	0.1086	-4.65458123203763\\
72.5	0.1092	-4.2616341878159\\
72.5	0.1098	-3.86868714359429\\
72.5	0.1104	-3.47574009937262\\
72.5	0.111	-3.082793055151\\
72.5	0.1116	-2.6898460109295\\
72.5	0.1122	-2.29689896670777\\
72.5	0.1128	-1.90395192248616\\
72.5	0.1134	-1.51100487826454\\
72.5	0.114	-1.11805783404287\\
72.5	0.1146	-0.725110789821201\\
72.5	0.1152	-0.332163745599587\\
72.5	0.1158	0.0607832986219137\\
72.5	0.1164	0.453730342843642\\
72.5	0.117	0.846677387065256\\
72.5	0.1176	1.23962443128676\\
72.5	0.1182	1.63257147550848\\
72.5	0.1188	2.02551851972999\\
72.5	0.1194	2.41846556395183\\
72.5	0.12	2.81141260817333\\
72.5	0.1206	3.20435965239506\\
72.5	0.1212	3.59730669661667\\
72.5	0.1218	3.99025374083817\\
72.5	0.1224	4.38320078506001\\
72.5	0.123	4.77614782928151\\
72.875	0.093	-15.1940954899545\\
72.875	0.0936	-14.8150367769065\\
72.875	0.0942	-14.4359780638584\\
72.875	0.0948	-14.0569193508104\\
72.875	0.0954	-13.6778606377624\\
72.875	0.096	-13.2988019247144\\
72.875	0.0966	-12.9197432116664\\
72.875	0.0972	-12.5406844986184\\
72.875	0.0978	-12.1616257855704\\
72.875	0.0984	-11.7825670725224\\
72.875	0.099	-11.4035083594744\\
72.875	0.0996	-11.0244496464264\\
72.875	0.1002	-10.6453909333784\\
72.875	0.1008	-10.2663322203304\\
72.875	0.1014	-9.88727350728237\\
72.875	0.102	-9.50821479423428\\
72.875	0.1026	-9.1291560811863\\
72.875	0.1032	-8.75009736813826\\
72.875	0.1038	-8.37103865509027\\
72.875	0.1044	-7.99197994204229\\
72.875	0.105	-7.61292122899425\\
72.875	0.1056	-7.23386251594627\\
72.875	0.1062	-6.85480380289823\\
72.875	0.1068	-6.47574508985025\\
72.875	0.1074	-6.09668637680227\\
72.875	0.108	-5.71762766375429\\
72.875	0.1086	-5.33856895070625\\
72.875	0.1092	-4.95951023765815\\
72.875	0.1098	-4.58045152461017\\
72.875	0.1104	-4.20139281156213\\
72.875	0.111	-3.82233409851415\\
72.875	0.1116	-3.44327538546622\\
72.875	0.1122	-3.06421667241813\\
72.875	0.1128	-2.68515795937014\\
72.875	0.1134	-2.30609924632211\\
72.875	0.114	-1.92704053327412\\
72.875	0.1146	-1.54798182022603\\
72.875	0.1152	-1.1689231071781\\
72.875	0.1158	-0.789864394130177\\
72.875	0.1164	-0.410805681082024\\
72.875	0.117	-0.0317469680340992\\
72.875	0.1176	0.347311745013826\\
72.875	0.1182	0.726370458061979\\
72.875	0.1188	1.10542917110979\\
72.875	0.1194	1.48448788415806\\
72.875	0.12	1.86354659720598\\
72.875	0.1206	2.24260531025402\\
72.875	0.1212	2.62166402330195\\
72.875	0.1218	3.00072273634987\\
72.875	0.1224	3.37978144939802\\
72.875	0.123	3.75884016244595\\
73.25	0.093	-15.516986598109\\
73.25	0.0936	-15.1518162162346\\
73.25	0.0942	-14.7866458343602\\
73.25	0.0948	-14.4214754524858\\
73.25	0.0954	-14.0563050706114\\
73.25	0.096	-13.691134688737\\
73.25	0.0966	-13.3259643068626\\
73.25	0.0972	-12.9607939249882\\
73.25	0.0978	-12.5956235431138\\
73.25	0.0984	-12.2304531612394\\
73.25	0.099	-11.8652827793651\\
73.25	0.0996	-11.5001123974907\\
73.25	0.1002	-11.1349420156163\\
73.25	0.1008	-10.7697716337419\\
73.25	0.1014	-10.4046012518675\\
73.25	0.102	-10.0394308699931\\
73.25	0.1026	-9.6742604881187\\
73.25	0.1032	-9.3090901062443\\
73.25	0.1038	-8.94391972436989\\
73.25	0.1044	-8.57874934249554\\
73.25	0.105	-8.21357896062119\\
73.25	0.1056	-7.84840857874678\\
73.25	0.1062	-7.48323819687238\\
73.25	0.1068	-7.11806781499797\\
73.25	0.1074	-6.75289743312362\\
73.25	0.108	-6.38772705124927\\
73.25	0.1086	-6.02255666937486\\
73.25	0.1092	-5.6573862875004\\
73.25	0.1098	-5.29221590562599\\
73.25	0.1104	-4.92704552375159\\
73.25	0.111	-4.56187514187724\\
73.25	0.1116	-4.19670476000294\\
73.25	0.1122	-3.83153437812848\\
73.25	0.1128	-3.46636399625407\\
73.25	0.1134	-3.10119361437967\\
73.25	0.114	-2.73602323250532\\
73.25	0.1146	-2.37085285063085\\
73.25	0.1152	-2.0056824687565\\
73.25	0.1158	-1.64051208688215\\
73.25	0.1164	-1.2753417050078\\
73.25	0.117	-0.910171323133454\\
73.25	0.1176	-0.545000941258991\\
73.25	0.1182	-0.179830559384641\\
73.25	0.1188	0.185339822489709\\
73.25	0.1194	0.550510204364173\\
73.25	0.12	0.915680586238636\\
73.25	0.1206	1.28085096811299\\
73.25	0.1212	1.64602134998745\\
73.25	0.1218	2.0111917318618\\
73.25	0.1224	2.37636211373615\\
73.25	0.123	2.7415324956105\\
73.625	0.093	-15.8398777062635\\
73.625	0.0936	-15.4885956555627\\
73.625	0.0942	-15.137313604862\\
73.625	0.0948	-14.7860315541612\\
73.625	0.0954	-14.4347495034604\\
73.625	0.096	-14.0834674527596\\
73.625	0.0966	-13.7321854020589\\
73.625	0.0972	-13.3809033513581\\
73.625	0.0978	-13.0296213006574\\
73.625	0.0984	-12.6783392499566\\
73.625	0.099	-12.3270571992558\\
73.625	0.0996	-11.975775148555\\
73.625	0.1002	-11.6244930978543\\
73.625	0.1008	-11.2732110471536\\
73.625	0.1014	-10.9219289964528\\
73.625	0.102	-10.5706469457519\\
73.625	0.1026	-10.2193648950511\\
73.625	0.1032	-9.86808284435034\\
73.625	0.1038	-9.51680079364957\\
73.625	0.1044	-9.16551874294885\\
73.625	0.105	-8.81423669224807\\
73.625	0.1056	-8.4629546415473\\
73.625	0.1062	-8.11167259084652\\
73.625	0.1068	-7.76039054014575\\
73.625	0.1074	-7.40910848944503\\
73.625	0.108	-7.05782643874426\\
73.625	0.1086	-6.70654438804348\\
73.625	0.1092	-6.35526233734265\\
73.625	0.1098	-6.00398028664188\\
73.625	0.1104	-5.6526982359411\\
73.625	0.111	-5.30141618524033\\
73.625	0.1116	-4.95013413453967\\
73.625	0.1122	-4.59885208383884\\
73.625	0.1128	-4.24757003313812\\
73.625	0.1134	-3.89628798243734\\
73.625	0.114	-3.54500593173657\\
73.625	0.1146	-3.19372388103579\\
73.625	0.1152	-2.84244183033502\\
73.625	0.1158	-2.49115977963424\\
73.625	0.1164	-2.13987772893347\\
73.625	0.117	-1.7885956782327\\
73.625	0.1176	-1.43731362753192\\
73.625	0.1182	-1.08603157683115\\
73.625	0.1188	-0.734749526130372\\
73.625	0.1194	-0.383467475429597\\
73.625	0.12	-0.0321854247288229\\
73.625	0.1206	0.319096625971952\\
73.625	0.1212	0.670378676672726\\
73.625	0.1218	1.0216607273735\\
73.625	0.1224	1.37294277807428\\
73.625	0.123	1.72422482877505\\
74	0.093	-16.1627688144179\\
74	0.0936	-15.8253750948908\\
74	0.0942	-15.4879813753636\\
74	0.0948	-15.1505876558365\\
74	0.0954	-14.8131939363093\\
74	0.096	-14.4758002167821\\
74	0.0966	-14.138406497255\\
74	0.0972	-13.8010127777279\\
74	0.0978	-13.4636190582007\\
74	0.0984	-13.1262253386736\\
74	0.099	-12.7888316191464\\
74	0.0996	-12.4514378996193\\
74	0.1002	-12.1140441800922\\
74	0.1008	-11.776650460565\\
74	0.1014	-11.4392567410379\\
74	0.102	-11.1018630215107\\
74	0.1026	-10.7644693019835\\
74	0.1032	-10.4270755824563\\
74	0.1038	-10.0896818629292\\
74	0.1044	-9.7522881434021\\
74	0.105	-9.4148944238749\\
74	0.1056	-9.07750070434776\\
74	0.1062	-8.74010698482061\\
74	0.1068	-8.40271326529347\\
74	0.1074	-8.06531954576633\\
74	0.108	-7.72792582623919\\
74	0.1086	-7.39053210671204\\
74	0.1092	-7.05313838718484\\
74	0.1098	-6.71574466765765\\
74	0.1104	-6.3783509481305\\
74	0.111	-6.04095722860336\\
74	0.1116	-5.70356350907633\\
74	0.1122	-5.36616978954908\\
74	0.1128	-5.02877607002199\\
74	0.1134	-4.69138235049479\\
74	0.114	-4.35398863096759\\
74	0.1146	-4.01659491144051\\
74	0.1152	-3.67920119191342\\
74	0.1158	-3.34180747238622\\
74	0.1164	-3.00441375285902\\
74	0.117	-2.66702003333194\\
74	0.1176	-2.32962631380485\\
74	0.1182	-1.99223259427765\\
74	0.1188	-1.65483887475045\\
74	0.1194	-1.31744515522325\\
74	0.12	-0.980051435696168\\
74	0.1206	-0.642657716168969\\
74	0.1212	-0.30526399664177\\
74	0.1218	0.0321297228853155\\
74	0.1224	0.369523442412401\\
74	0.123	0.7069171619396\\
};
\end{axis}

\begin{axis}[%
width=4.927496cm,
height=3.870968cm,
at={(6.483547cm,16.129032cm)},
scale only axis,
xmin=56,
xmax=74,
tick align=outside,
xlabel={$L_{cut}$},
xmajorgrids,
ymin=0.093,
ymax=0.123,
ylabel={$D_{rlx}$},
ymajorgrids,
zmin=-111.704899191481,
zmax=75.4245651933675,
zlabel={$x_3$},
zmajorgrids,
view={-140}{50},
legend style={at={(1.03,1)},anchor=north west,legend cell align=left,align=left,draw=white!15!black}
]
\addplot3[only marks,mark=*,mark options={},mark size=1.5000pt,color=mycolor1] plot table[row sep=crcr,]{%
74	0.123	75.4245651900459\\
72	0.113	59.5782863482935\\
61	0.095	33.0003406266977\\
56	0.093	26.0569470415911\\
};
\addplot3[only marks,mark=*,mark options={},mark size=1.5000pt,color=black] plot table[row sep=crcr,]{%
69	0.104	47.4385858736507\\
};

\addplot3[%
surf,
opacity=0.7,
shader=interp,
colormap={mymap}{[1pt] rgb(0pt)=(0.0901961,0.239216,0.0745098); rgb(1pt)=(0.0945149,0.242058,0.0739522); rgb(2pt)=(0.0988592,0.244894,0.0733566); rgb(3pt)=(0.103229,0.247724,0.0727241); rgb(4pt)=(0.107623,0.250549,0.0720557); rgb(5pt)=(0.112043,0.253367,0.0713525); rgb(6pt)=(0.116487,0.25618,0.0706154); rgb(7pt)=(0.120956,0.258986,0.0698456); rgb(8pt)=(0.125449,0.261787,0.0690441); rgb(9pt)=(0.129967,0.264581,0.0682118); rgb(10pt)=(0.134508,0.26737,0.06735); rgb(11pt)=(0.139074,0.270152,0.0664596); rgb(12pt)=(0.143663,0.272929,0.0655416); rgb(13pt)=(0.148275,0.275699,0.0645971); rgb(14pt)=(0.152911,0.278463,0.0636271); rgb(15pt)=(0.15757,0.281221,0.0626328); rgb(16pt)=(0.162252,0.283973,0.0616151); rgb(17pt)=(0.166957,0.286719,0.060575); rgb(18pt)=(0.171685,0.289458,0.0595136); rgb(19pt)=(0.176434,0.292191,0.0584321); rgb(20pt)=(0.181207,0.294918,0.0573313); rgb(21pt)=(0.186001,0.297639,0.0562123); rgb(22pt)=(0.190817,0.300353,0.0550763); rgb(23pt)=(0.195655,0.303061,0.0539242); rgb(24pt)=(0.200514,0.305763,0.052757); rgb(25pt)=(0.205395,0.308459,0.0515759); rgb(26pt)=(0.210296,0.311149,0.0503624); rgb(27pt)=(0.215212,0.313846,0.0490067); rgb(28pt)=(0.220142,0.316548,0.0475043); rgb(29pt)=(0.22509,0.319254,0.0458704); rgb(30pt)=(0.230056,0.321962,0.0441205); rgb(31pt)=(0.235042,0.324671,0.04227); rgb(32pt)=(0.240048,0.327379,0.0403343); rgb(33pt)=(0.245078,0.330085,0.0383287); rgb(34pt)=(0.250131,0.332786,0.0362688); rgb(35pt)=(0.25521,0.335482,0.0341698); rgb(36pt)=(0.260317,0.33817,0.0320472); rgb(37pt)=(0.265451,0.340849,0.0299163); rgb(38pt)=(0.270616,0.343517,0.0277927); rgb(39pt)=(0.275813,0.346172,0.0256916); rgb(40pt)=(0.281043,0.348814,0.0236284); rgb(41pt)=(0.286307,0.35144,0.0216186); rgb(42pt)=(0.291607,0.354048,0.0196776); rgb(43pt)=(0.296945,0.356637,0.0178207); rgb(44pt)=(0.302322,0.359206,0.0160634); rgb(45pt)=(0.307739,0.361753,0.0144211); rgb(46pt)=(0.313198,0.364275,0.0129091); rgb(47pt)=(0.318701,0.366772,0.0115428); rgb(48pt)=(0.324249,0.369242,0.0103377); rgb(49pt)=(0.329843,0.371682,0.00930909); rgb(50pt)=(0.335485,0.374093,0.00847245); rgb(51pt)=(0.341176,0.376471,0.00784314); rgb(52pt)=(0.346925,0.378826,0.00732741); rgb(53pt)=(0.352735,0.381168,0.00682184); rgb(54pt)=(0.358605,0.383497,0.00632729); rgb(55pt)=(0.364532,0.385812,0.00584464); rgb(56pt)=(0.370516,0.388113,0.00537476); rgb(57pt)=(0.376552,0.390399,0.00491852); rgb(58pt)=(0.38264,0.39267,0.00447681); rgb(59pt)=(0.388777,0.394925,0.00405048); rgb(60pt)=(0.394962,0.397164,0.00364042); rgb(61pt)=(0.401191,0.399386,0.00324749); rgb(62pt)=(0.407464,0.401592,0.00287258); rgb(63pt)=(0.413777,0.40378,0.00251655); rgb(64pt)=(0.420129,0.40595,0.00218028); rgb(65pt)=(0.426518,0.408102,0.00186463); rgb(66pt)=(0.432942,0.410234,0.00157049); rgb(67pt)=(0.439399,0.412348,0.00129873); rgb(68pt)=(0.445885,0.414441,0.00105022); rgb(69pt)=(0.452401,0.416515,0.000825833); rgb(70pt)=(0.458942,0.418567,0.000626441); rgb(71pt)=(0.465508,0.420599,0.00045292); rgb(72pt)=(0.472096,0.422609,0.000306141); rgb(73pt)=(0.478704,0.424596,0.000186979); rgb(74pt)=(0.485331,0.426562,9.63073e-05); rgb(75pt)=(0.491973,0.428504,3.49981e-05); rgb(76pt)=(0.498628,0.430422,3.92506e-06); rgb(77pt)=(0.505323,0.432315,0); rgb(78pt)=(0.512206,0.434168,0); rgb(79pt)=(0.519282,0.435983,0); rgb(80pt)=(0.526529,0.437764,0); rgb(81pt)=(0.533922,0.439512,0); rgb(82pt)=(0.54144,0.441232,0); rgb(83pt)=(0.549059,0.442927,0); rgb(84pt)=(0.556756,0.444599,0); rgb(85pt)=(0.564508,0.446252,0); rgb(86pt)=(0.572292,0.447889,0); rgb(87pt)=(0.580084,0.449514,0); rgb(88pt)=(0.587863,0.451129,0); rgb(89pt)=(0.595604,0.452737,0); rgb(90pt)=(0.603284,0.454343,0); rgb(91pt)=(0.610882,0.455948,0); rgb(92pt)=(0.618373,0.457556,0); rgb(93pt)=(0.625734,0.459171,0); rgb(94pt)=(0.632943,0.460795,0); rgb(95pt)=(0.639976,0.462432,0); rgb(96pt)=(0.64681,0.464084,0); rgb(97pt)=(0.653423,0.465756,0); rgb(98pt)=(0.659791,0.46745,0); rgb(99pt)=(0.665891,0.469169,0); rgb(100pt)=(0.6717,0.470916,0); rgb(101pt)=(0.677195,0.472696,0); rgb(102pt)=(0.682353,0.47451,0); rgb(103pt)=(0.687242,0.476355,0); rgb(104pt)=(0.691952,0.478225,0); rgb(105pt)=(0.696497,0.480118,0); rgb(106pt)=(0.700887,0.482033,0); rgb(107pt)=(0.705134,0.483968,0); rgb(108pt)=(0.709251,0.485921,0); rgb(109pt)=(0.713249,0.487891,0); rgb(110pt)=(0.71714,0.489876,0); rgb(111pt)=(0.720936,0.491875,0); rgb(112pt)=(0.724649,0.493887,0); rgb(113pt)=(0.72829,0.495909,0); rgb(114pt)=(0.731872,0.49794,0); rgb(115pt)=(0.735406,0.499979,0); rgb(116pt)=(0.738904,0.502025,0); rgb(117pt)=(0.742378,0.504075,0); rgb(118pt)=(0.74584,0.506128,0); rgb(119pt)=(0.749302,0.508182,0); rgb(120pt)=(0.752775,0.510237,0); rgb(121pt)=(0.756272,0.51229,0); rgb(122pt)=(0.759804,0.514339,0); rgb(123pt)=(0.763384,0.516385,0); rgb(124pt)=(0.767022,0.518424,0); rgb(125pt)=(0.770731,0.520455,0); rgb(126pt)=(0.774523,0.522478,0); rgb(127pt)=(0.77841,0.524489,0); rgb(128pt)=(0.782391,0.526491,0); rgb(129pt)=(0.786402,0.528496,0); rgb(130pt)=(0.790431,0.530506,0); rgb(131pt)=(0.794478,0.532521,0); rgb(132pt)=(0.798541,0.534539,0); rgb(133pt)=(0.802619,0.53656,0); rgb(134pt)=(0.806712,0.538584,0); rgb(135pt)=(0.81082,0.540609,0); rgb(136pt)=(0.81494,0.542635,0); rgb(137pt)=(0.819074,0.54466,0); rgb(138pt)=(0.823219,0.546686,0); rgb(139pt)=(0.827374,0.548709,0); rgb(140pt)=(0.831541,0.55073,0); rgb(141pt)=(0.835716,0.552749,0); rgb(142pt)=(0.8399,0.554763,0); rgb(143pt)=(0.844092,0.556774,0); rgb(144pt)=(0.848292,0.558779,0); rgb(145pt)=(0.852497,0.560778,0); rgb(146pt)=(0.856708,0.562771,0); rgb(147pt)=(0.860924,0.564756,0); rgb(148pt)=(0.865143,0.566733,0); rgb(149pt)=(0.869366,0.568701,0); rgb(150pt)=(0.873592,0.57066,0); rgb(151pt)=(0.877819,0.572608,0); rgb(152pt)=(0.882047,0.574545,0); rgb(153pt)=(0.886275,0.576471,0); rgb(154pt)=(0.890659,0.578362,0); rgb(155pt)=(0.895333,0.580203,0); rgb(156pt)=(0.900258,0.581999,0); rgb(157pt)=(0.905397,0.583755,0); rgb(158pt)=(0.910711,0.585479,0); rgb(159pt)=(0.916164,0.587176,0); rgb(160pt)=(0.921717,0.588852,0); rgb(161pt)=(0.927333,0.590513,0); rgb(162pt)=(0.932974,0.592166,0); rgb(163pt)=(0.938602,0.593815,0); rgb(164pt)=(0.94418,0.595468,0); rgb(165pt)=(0.949669,0.59713,0); rgb(166pt)=(0.955033,0.598808,0); rgb(167pt)=(0.960233,0.600507,0); rgb(168pt)=(0.965232,0.602233,0); rgb(169pt)=(0.969992,0.603992,0); rgb(170pt)=(0.974475,0.605791,0); rgb(171pt)=(0.978643,0.607636,0); rgb(172pt)=(0.98246,0.609532,0); rgb(173pt)=(0.985886,0.611486,0); rgb(174pt)=(0.988885,0.613503,0); rgb(175pt)=(0.991419,0.61559,0); rgb(176pt)=(0.99345,0.617753,0); rgb(177pt)=(0.99494,0.619997,0); rgb(178pt)=(0.995851,0.622329,0); rgb(179pt)=(0.996226,0.624763,0); rgb(180pt)=(0.996512,0.627352,0); rgb(181pt)=(0.996788,0.630095,0); rgb(182pt)=(0.997053,0.632982,0); rgb(183pt)=(0.997308,0.636004,0); rgb(184pt)=(0.997552,0.639152,0); rgb(185pt)=(0.997785,0.642416,0); rgb(186pt)=(0.998006,0.645786,0); rgb(187pt)=(0.998217,0.649253,0); rgb(188pt)=(0.998416,0.652807,0); rgb(189pt)=(0.998605,0.656439,0); rgb(190pt)=(0.998781,0.660138,0); rgb(191pt)=(0.998946,0.663897,0); rgb(192pt)=(0.9991,0.667704,0); rgb(193pt)=(0.999242,0.67155,0); rgb(194pt)=(0.999372,0.675427,0); rgb(195pt)=(0.99949,0.679323,0); rgb(196pt)=(0.999596,0.68323,0); rgb(197pt)=(0.99969,0.687139,0); rgb(198pt)=(0.999771,0.691039,0); rgb(199pt)=(0.999841,0.694921,0); rgb(200pt)=(0.999898,0.698775,0); rgb(201pt)=(0.999942,0.702592,0); rgb(202pt)=(0.999974,0.706363,0); rgb(203pt)=(0.999994,0.710077,0); rgb(204pt)=(1,0.713725,0); rgb(205pt)=(1,0.717341,0); rgb(206pt)=(1,0.720963,0); rgb(207pt)=(1,0.724591,0); rgb(208pt)=(1,0.728226,0); rgb(209pt)=(1,0.731867,0); rgb(210pt)=(1,0.735514,0); rgb(211pt)=(1,0.739167,0); rgb(212pt)=(1,0.742827,0); rgb(213pt)=(1,0.746493,0); rgb(214pt)=(1,0.750165,0); rgb(215pt)=(1,0.753843,0); rgb(216pt)=(1,0.757527,0); rgb(217pt)=(1,0.761217,0); rgb(218pt)=(1,0.764913,0); rgb(219pt)=(1,0.768615,0); rgb(220pt)=(1,0.772324,0); rgb(221pt)=(1,0.776038,0); rgb(222pt)=(1,0.779758,0); rgb(223pt)=(1,0.783484,0); rgb(224pt)=(1,0.787215,0); rgb(225pt)=(1,0.790953,0); rgb(226pt)=(1,0.794696,0); rgb(227pt)=(1,0.798445,0); rgb(228pt)=(1,0.8022,0); rgb(229pt)=(1,0.805961,0); rgb(230pt)=(1,0.809727,0); rgb(231pt)=(1,0.8135,0); rgb(232pt)=(1,0.817278,0); rgb(233pt)=(1,0.821063,0); rgb(234pt)=(1,0.824854,0); rgb(235pt)=(1,0.828652,0); rgb(236pt)=(1,0.832455,0); rgb(237pt)=(1,0.836265,0); rgb(238pt)=(1,0.840081,0); rgb(239pt)=(1,0.843903,0); rgb(240pt)=(1,0.847732,0); rgb(241pt)=(1,0.851566,0); rgb(242pt)=(1,0.855406,0); rgb(243pt)=(1,0.859253,0); rgb(244pt)=(1,0.863106,0); rgb(245pt)=(1,0.866964,0); rgb(246pt)=(1,0.870829,0); rgb(247pt)=(1,0.8747,0); rgb(248pt)=(1,0.878577,0); rgb(249pt)=(1,0.88246,0); rgb(250pt)=(1,0.886349,0); rgb(251pt)=(1,0.890243,0); rgb(252pt)=(1,0.894144,0); rgb(253pt)=(1,0.898051,0); rgb(254pt)=(1,0.901964,0); rgb(255pt)=(1,0.905882,0)},
mesh/rows=49]
table[row sep=crcr,header=false] {%
%
56	0.093	26.0569470391285\\
56	0.0936	23.3017101145163\\
56	0.0942	20.5464731899042\\
56	0.0948	17.7912362652919\\
56	0.0954	15.0359993406796\\
56	0.096	12.2807624160675\\
56	0.0966	9.52552549145526\\
56	0.0972	6.77028856684319\\
56	0.0978	4.0150516422309\\
56	0.0984	1.25981471761861\\
56	0.099	-1.49542220699345\\
56	0.0996	-4.25065913160552\\
56	0.1002	-7.00589605621803\\
56	0.1008	-9.7611329808301\\
56	0.1014	-12.5163699054424\\
56	0.102	-15.2716068300545\\
56	0.1026	-18.0268437546665\\
56	0.1032	-20.782080679279\\
56	0.1038	-23.5373176038911\\
56	0.1044	-26.2925545285032\\
56	0.105	-29.0477914531155\\
56	0.1056	-31.8030283777277\\
56	0.1062	-34.55826530234\\
56	0.1068	-37.3135022269521\\
56	0.1074	-40.0687391515642\\
56	0.108	-42.8239760761765\\
56	0.1086	-45.5792130007885\\
56	0.1092	-48.334449925401\\
56	0.1098	-51.0896868500131\\
56	0.1104	-53.8449237746252\\
56	0.111	-56.6001606992374\\
56	0.1116	-59.3553976238495\\
56	0.1122	-62.1106345484618\\
56	0.1128	-64.8658714730741\\
56	0.1134	-67.6211083976862\\
56	0.114	-70.3763453222984\\
56	0.1146	-73.1315822469105\\
56	0.1152	-75.8868191715228\\
56	0.1158	-78.6420560961351\\
56	0.1164	-81.3972930207472\\
56	0.117	-84.1525299453597\\
56	0.1176	-86.9077668699717\\
56	0.1182	-89.6630037945838\\
56	0.1188	-92.4182407191961\\
56	0.1194	-95.1734776438084\\
56	0.12	-97.9287145684204\\
56	0.1206	-100.683951493033\\
56	0.1212	-103.439188417645\\
56	0.1218	-106.194425342257\\
56	0.1224	-108.949662266869\\
56	0.123	-111.704899191481\\
56.375	0.093	27.0744417352544\\
56.375	0.0936	24.3768255268801\\
56.375	0.0942	21.6792093185059\\
56.375	0.0948	18.9815931101311\\
56.375	0.0954	16.2839769017569\\
56.375	0.096	13.5863606933826\\
56.375	0.0966	10.8887444850081\\
56.375	0.0972	8.19112827663389\\
56.375	0.0978	5.49351206825941\\
56.375	0.0984	2.79589585988515\\
56.375	0.099	0.0982796515106656\\
56.375	0.0996	-2.59933655686359\\
56.375	0.1002	-5.29695276523807\\
56.375	0.1008	-7.99456897361233\\
56.375	0.1014	-10.6921851819868\\
56.375	0.102	-13.3898013903611\\
56.375	0.1026	-16.0874175987353\\
56.375	0.1032	-18.7850338071098\\
56.375	0.1038	-21.4826500154843\\
56.375	0.1044	-24.1802662238586\\
56.375	0.105	-26.8778824322328\\
56.375	0.1056	-29.5754986406073\\
56.375	0.1062	-32.2731148489818\\
56.375	0.1068	-34.970731057356\\
56.375	0.1074	-37.6683472657303\\
56.375	0.108	-40.3659634741045\\
56.375	0.1086	-43.063579682479\\
56.375	0.1092	-45.7611958908535\\
56.375	0.1098	-48.4588120992278\\
56.375	0.1104	-51.156428307602\\
56.375	0.111	-53.8540445159765\\
56.375	0.1116	-56.5516607243508\\
56.375	0.1122	-59.2492769327253\\
56.375	0.1128	-61.9468931410995\\
56.375	0.1134	-64.644509349474\\
56.375	0.114	-67.3421255578482\\
56.375	0.1146	-70.0397417662225\\
56.375	0.1152	-72.737357974597\\
56.375	0.1158	-75.4349741829715\\
56.375	0.1164	-78.1325903913457\\
56.375	0.117	-80.8302065997202\\
56.375	0.1176	-83.5278228080945\\
56.375	0.1182	-86.225439016469\\
56.375	0.1188	-88.9230552248432\\
56.375	0.1194	-91.6206714332177\\
56.375	0.12	-94.318287641592\\
56.375	0.1206	-97.0159038499664\\
56.375	0.1212	-99.7135200583407\\
56.375	0.1218	-102.411136266715\\
56.375	0.1224	-105.10875247509\\
56.375	0.123	-107.806368683464\\
56.75	0.093	28.0919364313802\\
56.75	0.0936	25.4519409392437\\
56.75	0.0942	22.8119454471071\\
56.75	0.0948	20.1719499549704\\
56.75	0.0954	17.5319544628339\\
56.75	0.096	14.8919589706975\\
56.75	0.0966	12.251963478561\\
56.75	0.0972	9.61196798642459\\
56.75	0.0978	6.97197249428791\\
56.75	0.0984	4.33197700215123\\
56.75	0.099	1.69198151001478\\
56.75	0.0996	-0.948013982121665\\
56.75	0.1002	-3.58800947425834\\
56.75	0.1008	-6.22800496639479\\
56.75	0.1014	-8.86800045853124\\
56.75	0.102	-11.5079959506677\\
56.75	0.1026	-14.1479914428044\\
56.75	0.1032	-16.787986934941\\
56.75	0.1038	-19.4279824270775\\
56.75	0.1044	-22.0679779192139\\
56.75	0.105	-24.7079734113504\\
56.75	0.1056	-27.3479689034871\\
56.75	0.1062	-29.9879643956235\\
56.75	0.1068	-32.6279598877602\\
56.75	0.1074	-35.2679553798966\\
56.75	0.108	-37.9079508720331\\
56.75	0.1086	-40.5479463641695\\
56.75	0.1092	-43.1879418563062\\
56.75	0.1098	-45.8279373484427\\
56.75	0.1104	-48.4679328405791\\
56.75	0.111	-51.1079283327156\\
56.75	0.1116	-53.7479238248523\\
56.75	0.1122	-56.3879193169889\\
56.75	0.1128	-59.0279148091254\\
56.75	0.1134	-61.6679103012618\\
56.75	0.114	-64.3079057933983\\
56.75	0.1146	-66.9479012855347\\
56.75	0.1152	-69.5878967776714\\
56.75	0.1158	-72.2278922698081\\
56.75	0.1164	-74.8678877619445\\
56.75	0.117	-77.5078832540812\\
56.75	0.1176	-80.1478787462177\\
56.75	0.1182	-82.7878742383541\\
56.75	0.1188	-85.4278697304906\\
56.75	0.1194	-88.0678652226272\\
56.75	0.12	-90.7078607147637\\
56.75	0.1206	-93.3478562069004\\
56.75	0.1212	-95.9878516990368\\
56.75	0.1218	-98.6278471911735\\
56.75	0.1224	-101.26784268331\\
56.75	0.123	-103.907838175446\\
57.125	0.093	29.1094311275062\\
57.125	0.0936	26.5270563516076\\
57.125	0.0942	23.944681575709\\
57.125	0.0948	21.3623067998101\\
57.125	0.0954	18.7799320239114\\
57.125	0.096	16.1975572480128\\
57.125	0.0966	13.6151824721142\\
57.125	0.0972	11.0328076962155\\
57.125	0.0978	8.45043292031664\\
57.125	0.0984	5.868058144418\\
57.125	0.099	3.28568336851936\\
57.125	0.0996	0.703308592620715\\
57.125	0.1002	-1.87906618327816\\
57.125	0.1008	-4.4614409591768\\
57.125	0.1014	-7.04381573507544\\
57.125	0.102	-9.62619051097408\\
57.125	0.1026	-12.2085652868727\\
57.125	0.1032	-14.7909400627716\\
57.125	0.1038	-17.3733148386702\\
57.125	0.1044	-19.9556896145689\\
57.125	0.105	-22.5380643904675\\
57.125	0.1056	-25.1204391663664\\
57.125	0.1062	-27.702813942265\\
57.125	0.1068	-30.2851887181637\\
57.125	0.1074	-32.8675634940623\\
57.125	0.108	-35.449938269961\\
57.125	0.1086	-38.0323130458596\\
57.125	0.1092	-40.6146878217585\\
57.125	0.1098	-43.1970625976571\\
57.125	0.1104	-45.7794373735558\\
57.125	0.111	-48.3618121494544\\
57.125	0.1116	-50.9441869253531\\
57.125	0.1122	-53.5265617012519\\
57.125	0.1128	-56.1089364771506\\
57.125	0.1134	-58.6913112530492\\
57.125	0.114	-61.2736860289478\\
57.125	0.1146	-63.8560608048467\\
57.125	0.1152	-66.4384355807454\\
57.125	0.1158	-69.020810356644\\
57.125	0.1164	-71.6031851325429\\
57.125	0.117	-74.1855599084415\\
57.125	0.1176	-76.7679346843402\\
57.125	0.1182	-79.350309460239\\
57.125	0.1188	-81.9326842361377\\
57.125	0.1194	-84.5150590120363\\
57.125	0.12	-87.0974337879352\\
57.125	0.1206	-89.6798085638338\\
57.125	0.1212	-92.2621833397325\\
57.125	0.1218	-94.8445581156313\\
57.125	0.1224	-97.42693289153\\
57.125	0.123	-100.009307667429\\
57.5	0.093	30.1269258236321\\
57.5	0.0936	27.6021717639712\\
57.5	0.0942	25.0774177043106\\
57.5	0.0948	22.5526636446496\\
57.5	0.0954	20.0279095849887\\
57.5	0.096	17.5031555253279\\
57.5	0.0966	14.978401465667\\
57.5	0.0972	12.4536474060062\\
57.5	0.0978	9.92889334634515\\
57.5	0.0984	7.40413928668431\\
57.5	0.099	4.87938522702348\\
57.5	0.0996	2.35463116736287\\
57.5	0.1002	-0.170122892298195\\
57.5	0.1008	-2.69487695195903\\
57.5	0.1014	-5.21963101161987\\
57.5	0.102	-7.7443850712807\\
57.5	0.1026	-10.2691391309415\\
57.5	0.1032	-12.7938931906026\\
57.5	0.1038	-15.3186472502634\\
57.5	0.1044	-17.843401309924\\
57.5	0.105	-20.3681553695849\\
57.5	0.1056	-22.8929094292459\\
57.5	0.1062	-25.4176634889068\\
57.5	0.1068	-27.9424175485676\\
57.5	0.1074	-30.4671716082285\\
57.5	0.108	-32.9919256678893\\
57.5	0.1086	-35.5166797275501\\
57.5	0.1092	-38.0414337872112\\
57.5	0.1098	-40.5661878468718\\
57.5	0.1104	-43.0909419065326\\
57.5	0.111	-45.6156959661935\\
57.5	0.1116	-48.1404500258543\\
57.5	0.1122	-50.6652040855154\\
57.5	0.1128	-53.1899581451762\\
57.5	0.1134	-55.714712204837\\
57.5	0.114	-58.2394662644979\\
57.5	0.1146	-60.7642203241587\\
57.5	0.1152	-63.2889743838195\\
57.5	0.1158	-65.8137284434804\\
57.5	0.1164	-68.3384825031412\\
57.5	0.117	-70.8632365628023\\
57.5	0.1176	-73.3879906224631\\
57.5	0.1182	-75.912744682124\\
57.5	0.1188	-78.4374987417848\\
57.5	0.1194	-80.9622528014459\\
57.5	0.12	-83.4870068611067\\
57.5	0.1206	-86.0117609207673\\
57.5	0.1212	-88.5365149804281\\
57.5	0.1218	-91.0612690400892\\
57.5	0.1224	-93.58602309975\\
57.5	0.123	-96.1107771594109\\
57.875	0.093	31.1444205197581\\
57.875	0.0936	28.6772871763351\\
57.875	0.0942	26.2101538329121\\
57.875	0.0948	23.743020489489\\
57.875	0.0954	21.275887146066\\
57.875	0.096	18.808753802643\\
57.875	0.0966	16.3416204592199\\
57.875	0.0972	13.8744871157971\\
57.875	0.0978	11.4073537723739\\
57.875	0.0984	8.94022042895085\\
57.875	0.099	6.47308708552782\\
57.875	0.0996	4.00595374210502\\
57.875	0.1002	1.53882039868176\\
57.875	0.1008	-0.928312944741265\\
57.875	0.1014	-3.39544628816429\\
57.875	0.102	-5.86257963158732\\
57.875	0.1026	-8.32971297501012\\
57.875	0.1032	-10.7968463184334\\
57.875	0.1038	-13.2639796618564\\
57.875	0.1044	-15.7311130052794\\
57.875	0.105	-18.1982463487022\\
57.875	0.1056	-20.6653796921255\\
57.875	0.1062	-23.1325130355485\\
57.875	0.1068	-25.5996463789716\\
57.875	0.1074	-28.0667797223944\\
57.875	0.108	-30.5339130658174\\
57.875	0.1086	-33.0010464092404\\
57.875	0.1092	-35.4681797526637\\
57.875	0.1098	-37.9353130960867\\
57.875	0.1104	-40.4024464395095\\
57.875	0.111	-42.8695797829325\\
57.875	0.1116	-45.3367131263556\\
57.875	0.1122	-47.8038464697788\\
57.875	0.1128	-50.2709798132016\\
57.875	0.1134	-52.7381131566246\\
57.875	0.114	-55.2052465000477\\
57.875	0.1146	-57.6723798434707\\
57.875	0.1152	-60.1395131868937\\
57.875	0.1158	-62.6066465303168\\
57.875	0.1164	-65.0737798737398\\
57.875	0.117	-67.5409132171631\\
57.875	0.1176	-70.0080465605861\\
57.875	0.1182	-72.4751799040089\\
57.875	0.1188	-74.9423132474319\\
57.875	0.1194	-77.4094465908552\\
57.875	0.12	-79.8765799342782\\
57.875	0.1206	-82.343713277701\\
57.875	0.1212	-84.810846621124\\
57.875	0.1218	-87.2779799645473\\
57.875	0.1224	-89.7451133079703\\
57.875	0.123	-92.2122466513931\\
58.25	0.093	32.1619152158839\\
58.25	0.0936	29.7524025886989\\
58.25	0.0942	27.3428899615137\\
58.25	0.0948	24.9333773343283\\
58.25	0.0954	22.5238647071433\\
58.25	0.096	20.1143520799581\\
58.25	0.0966	17.7048394527728\\
58.25	0.0972	15.2953268255878\\
58.25	0.0978	12.8858141984024\\
58.25	0.0984	10.4763015712174\\
58.25	0.099	8.06678894403217\\
58.25	0.0996	5.65727631684695\\
58.25	0.1002	3.24776368966172\\
58.25	0.1008	0.838251062476502\\
58.25	0.1014	-1.57126156470872\\
58.25	0.102	-3.98077419189372\\
58.25	0.1026	-6.39028681907894\\
58.25	0.1032	-8.79979944626439\\
58.25	0.1038	-11.2093120734494\\
58.25	0.1044	-13.6188247006346\\
58.25	0.105	-16.0283373278198\\
58.25	0.1056	-18.437849955005\\
58.25	0.1062	-20.8473625821903\\
58.25	0.1068	-23.2568752093755\\
58.25	0.1074	-25.6663878365605\\
58.25	0.108	-28.0759004637457\\
58.25	0.1086	-30.4854130909307\\
58.25	0.1092	-32.8949257181162\\
58.25	0.1098	-35.3044383453014\\
58.25	0.1104	-37.7139509724864\\
58.25	0.111	-40.1234635996716\\
58.25	0.1116	-42.5329762268568\\
58.25	0.1122	-44.942488854042\\
58.25	0.1128	-47.3520014812273\\
58.25	0.1134	-49.7615141084125\\
58.25	0.114	-52.1710267355975\\
58.25	0.1146	-54.5805393627827\\
58.25	0.1152	-56.9900519899679\\
58.25	0.1158	-59.3995646171531\\
58.25	0.1164	-61.8090772443384\\
58.25	0.117	-64.2185898715236\\
58.25	0.1176	-66.6281024987088\\
58.25	0.1182	-69.037615125894\\
58.25	0.1188	-71.447127753079\\
58.25	0.1194	-73.8566403802645\\
58.25	0.12	-76.2661530074497\\
58.25	0.1206	-78.6756656346347\\
58.25	0.1212	-81.0851782618199\\
58.25	0.1218	-83.4946908890054\\
58.25	0.1224	-85.9042035161904\\
58.25	0.123	-88.3137161433756\\
58.625	0.093	33.17940991201\\
58.625	0.0936	30.8275180010626\\
58.625	0.0942	28.4756260901154\\
58.625	0.0948	26.1237341791677\\
58.625	0.0954	23.7718422682206\\
58.625	0.096	21.4199503572731\\
58.625	0.0966	19.0680584463259\\
58.625	0.0972	16.7161665353785\\
58.625	0.0978	14.3642746244311\\
58.625	0.0984	12.0123827134837\\
58.625	0.099	9.66049080253651\\
58.625	0.0996	7.3085988915891\\
58.625	0.1002	4.95670698064168\\
58.625	0.1008	2.60481506969427\\
58.625	0.1014	0.25292315874708\\
58.625	0.102	-2.09896875220034\\
58.625	0.1026	-4.45086066314752\\
58.625	0.1032	-6.80275257409517\\
58.625	0.1038	-9.15464448504258\\
58.625	0.1044	-11.5065363959898\\
58.625	0.105	-13.8584283069372\\
58.625	0.1056	-16.2103202178846\\
58.625	0.1062	-18.562212128832\\
58.625	0.1068	-20.9141040397792\\
58.625	0.1074	-23.2659959507266\\
58.625	0.108	-25.6178878616738\\
58.625	0.1086	-27.9697797726212\\
58.625	0.1092	-30.3216716835686\\
58.625	0.1098	-32.6735635945161\\
58.625	0.1104	-35.0254555054632\\
58.625	0.111	-37.3773474164107\\
58.625	0.1116	-39.7292393273578\\
58.625	0.1122	-42.0811312383055\\
58.625	0.1128	-44.4330231492527\\
58.625	0.1134	-46.7849150602001\\
58.625	0.114	-49.1368069711473\\
58.625	0.1146	-51.4886988820947\\
58.625	0.1152	-53.8405907930421\\
58.625	0.1158	-56.1924827039895\\
58.625	0.1164	-58.5443746149369\\
58.625	0.117	-60.8962665258844\\
58.625	0.1176	-63.2481584368318\\
58.625	0.1182	-65.600050347779\\
58.625	0.1188	-67.9519422587264\\
58.625	0.1194	-70.3038341696738\\
58.625	0.12	-72.6557260806212\\
58.625	0.1206	-75.0076179915684\\
58.625	0.1212	-77.3595099025158\\
58.625	0.1218	-79.7114018134632\\
58.625	0.1224	-82.0632937244106\\
58.625	0.123	-84.4151856353578\\
59	0.093	34.1969046081358\\
59	0.0936	31.9026334134264\\
59	0.0942	29.6083622187168\\
59	0.0948	27.3140910240072\\
59	0.0954	25.0198198292978\\
59	0.096	22.7255486345882\\
59	0.0966	20.4312774398788\\
59	0.0972	18.1370062451692\\
59	0.0978	15.8427350504596\\
59	0.0984	13.5484638557502\\
59	0.099	11.2541926610406\\
59	0.0996	8.95992146633125\\
59	0.1002	6.66565027162142\\
59	0.1008	4.37137907691204\\
59	0.1014	2.07710788220265\\
59	0.102	-0.217163312506955\\
59	0.1026	-2.51143450721634\\
59	0.1032	-4.80570570192617\\
59	0.1038	-7.09997689663555\\
59	0.1044	-9.39424809134493\\
59	0.105	-11.6885192860545\\
59	0.1056	-13.9827904807642\\
59	0.1062	-16.2770616754738\\
59	0.1068	-18.5713328701831\\
59	0.1074	-20.8656040648925\\
59	0.108	-23.1598752596021\\
59	0.1086	-25.4541464543115\\
59	0.1092	-27.7484176490211\\
59	0.1098	-30.0426888437307\\
59	0.1104	-32.3369600384401\\
59	0.111	-34.6312312331497\\
59	0.1116	-36.9255024278591\\
59	0.1122	-39.2197736225687\\
59	0.1128	-41.5140448172783\\
59	0.1134	-43.8083160119877\\
59	0.114	-46.1025872066973\\
59	0.1146	-48.3968584014067\\
59	0.1152	-50.6911295961163\\
59	0.1158	-52.9854007908259\\
59	0.1164	-55.2796719855353\\
59	0.117	-57.5739431802451\\
59	0.1176	-59.8682143749545\\
59	0.1182	-62.1624855696639\\
59	0.1188	-64.4567567643735\\
59	0.1194	-66.7510279590831\\
59	0.12	-69.0452991537927\\
59	0.1206	-71.3395703485021\\
59	0.1212	-73.6338415432115\\
59	0.1218	-75.9281127379213\\
59	0.1224	-78.2223839326307\\
59	0.123	-80.5166551273403\\
59.375	0.093	35.2143993042621\\
59.375	0.0936	32.9777488257903\\
59.375	0.0942	30.7410983473187\\
59.375	0.0948	28.5044478688469\\
59.375	0.0954	26.2677973903753\\
59.375	0.096	24.0311469119035\\
59.375	0.0966	21.794496433432\\
59.375	0.0972	19.5578459549604\\
59.375	0.0978	17.3211954764884\\
59.375	0.0984	15.0845449980168\\
59.375	0.099	12.8478945195452\\
59.375	0.0996	10.6112440410736\\
59.375	0.1002	8.3745935626016\\
59.375	0.1008	6.13794308413003\\
59.375	0.1014	3.90129260565845\\
59.375	0.102	1.66464212718688\\
59.375	0.1026	-0.572008351284921\\
59.375	0.1032	-2.80865882975672\\
59.375	0.1038	-5.0453093082283\\
59.375	0.1044	-7.2819597867001\\
59.375	0.105	-9.51861026517167\\
59.375	0.1056	-11.7552607436435\\
59.375	0.1062	-13.991911222115\\
59.375	0.1068	-16.2285617005869\\
59.375	0.1074	-18.4652121790584\\
59.375	0.108	-20.70186265753\\
59.375	0.1086	-22.9385131360018\\
59.375	0.1092	-25.1751636144736\\
59.375	0.1098	-27.4118140929452\\
59.375	0.1104	-29.6484645714168\\
59.375	0.111	-31.8851150498886\\
59.375	0.1116	-34.1217655283601\\
59.375	0.1122	-36.3584160068319\\
59.375	0.1128	-38.5950664853035\\
59.375	0.1134	-40.8317169637753\\
59.375	0.114	-43.0683674422469\\
59.375	0.1146	-45.3050179207185\\
59.375	0.1152	-47.5416683991905\\
59.375	0.1158	-49.7783188776621\\
59.375	0.1164	-52.0149693561336\\
59.375	0.117	-54.2516198346054\\
59.375	0.1176	-56.4882703130772\\
59.375	0.1182	-58.7249207915488\\
59.375	0.1188	-60.9615712700204\\
59.375	0.1194	-63.1982217484922\\
59.375	0.12	-65.434872226964\\
59.375	0.1206	-67.6715227054356\\
59.375	0.1212	-69.9081731839071\\
59.375	0.1218	-72.1448236623789\\
59.375	0.1224	-74.3814741408507\\
59.375	0.123	-76.6181246193223\\
59.75	0.093	36.2318940003877\\
59.75	0.0936	34.0528642381539\\
59.75	0.0942	31.8738344759201\\
59.75	0.0948	29.6948047136862\\
59.75	0.0954	27.5157749514524\\
59.75	0.096	25.3367451892184\\
59.75	0.0966	23.1577154269846\\
59.75	0.0972	20.9786856647509\\
59.75	0.0978	18.7996559025169\\
59.75	0.0984	16.6206261402831\\
59.75	0.099	14.4415963780493\\
59.75	0.0996	12.2625666158156\\
59.75	0.1002	10.0835368535813\\
59.75	0.1008	7.90450709134757\\
59.75	0.1014	5.7254773291138\\
59.75	0.102	3.54644756688003\\
59.75	0.1026	1.36741780464627\\
59.75	0.1032	-0.811611957587729\\
59.75	0.1038	-2.99064171982172\\
59.75	0.1044	-5.16967148205549\\
59.75	0.105	-7.34870124428926\\
59.75	0.1056	-9.52773100652325\\
59.75	0.1062	-11.706760768757\\
59.75	0.1068	-13.8857905309908\\
59.75	0.1074	-16.0648202932246\\
59.75	0.108	-18.2438500554586\\
59.75	0.1086	-20.4228798176923\\
59.75	0.1092	-22.6019095799263\\
59.75	0.1098	-24.7809393421601\\
59.75	0.1104	-26.9599691043938\\
59.75	0.111	-29.1389988666276\\
59.75	0.1116	-31.3180286288616\\
59.75	0.1122	-33.4970583910956\\
59.75	0.1128	-35.6760881533294\\
59.75	0.1134	-37.8551179155631\\
59.75	0.114	-40.0341476777969\\
59.75	0.1146	-42.2131774400307\\
59.75	0.1152	-44.3922072022647\\
59.75	0.1158	-46.5712369644987\\
59.75	0.1164	-48.7502667267324\\
59.75	0.117	-50.9292964889664\\
59.75	0.1176	-53.1083262512002\\
59.75	0.1182	-55.287356013434\\
59.75	0.1188	-57.4663857756677\\
59.75	0.1194	-59.645415537902\\
59.75	0.12	-61.8244453001357\\
59.75	0.1206	-64.0034750623695\\
59.75	0.1212	-66.1825048246033\\
59.75	0.1218	-68.3615345868373\\
59.75	0.1224	-70.540564349071\\
59.75	0.123	-72.7195941113048\\
60.125	0.093	37.249388696514\\
60.125	0.0936	35.1279796505178\\
60.125	0.0942	33.0065706045218\\
60.125	0.0948	30.8851615585256\\
60.125	0.0954	28.7637525125297\\
60.125	0.096	26.6423434665337\\
60.125	0.0966	24.5209344205377\\
60.125	0.0972	22.3995253745418\\
60.125	0.0978	20.2781163285456\\
60.125	0.0984	18.1567072825496\\
60.125	0.099	16.0352982365537\\
60.125	0.0996	13.9138891905577\\
60.125	0.1002	11.7924801445615\\
60.125	0.1008	9.67107109856556\\
60.125	0.1014	7.5496620525696\\
60.125	0.102	5.42825300657364\\
60.125	0.1026	3.30684396057768\\
60.125	0.1032	1.18543491458149\\
60.125	0.1038	-0.935974131414469\\
60.125	0.1044	-3.05738317741043\\
60.125	0.105	-5.17879222340639\\
60.125	0.1056	-7.30020126940258\\
60.125	0.1062	-9.42161031539854\\
60.125	0.1068	-11.5430193613945\\
60.125	0.1074	-13.6644284073905\\
60.125	0.108	-15.7858374533864\\
60.125	0.1086	-17.9072464993824\\
60.125	0.1092	-20.0286555453786\\
60.125	0.1098	-22.1500645913745\\
60.125	0.1104	-24.2714736373705\\
60.125	0.111	-26.3928826833665\\
60.125	0.1116	-28.5142917293624\\
60.125	0.1122	-30.6357007753586\\
60.125	0.1128	-32.7571098213546\\
60.125	0.1134	-34.8785188673505\\
60.125	0.114	-36.9999279133465\\
60.125	0.1146	-39.1213369593427\\
60.125	0.1152	-41.2427460053389\\
60.125	0.1158	-43.3641550513348\\
60.125	0.1164	-45.4855640973308\\
60.125	0.117	-47.606973143327\\
60.125	0.1176	-49.7283821893229\\
60.125	0.1182	-51.8497912353189\\
60.125	0.1188	-53.9712002813149\\
60.125	0.1194	-56.092609327311\\
60.125	0.12	-58.214018373307\\
60.125	0.1206	-60.335427419303\\
60.125	0.1212	-62.4568364652989\\
60.125	0.1218	-64.5782455112951\\
60.125	0.1224	-66.6996545572911\\
60.125	0.123	-68.821063603287\\
60.5	0.093	38.2668833926398\\
60.5	0.0936	36.2030950628816\\
60.5	0.0942	34.1393067331235\\
60.5	0.0948	32.0755184033651\\
60.5	0.0954	30.0117300736069\\
60.5	0.096	27.9479417438488\\
60.5	0.0966	25.8841534140906\\
60.5	0.0972	23.8203650843327\\
60.5	0.0978	21.7565767545743\\
60.5	0.0984	19.6927884248162\\
60.5	0.099	17.629000095058\\
60.5	0.0996	15.5652117652999\\
60.5	0.1002	13.5014234355415\\
60.5	0.1008	11.4376351057833\\
60.5	0.1014	9.37384677602518\\
60.5	0.102	7.31005844626702\\
60.5	0.1026	5.24627011650887\\
60.5	0.1032	3.18248178675071\\
60.5	0.1038	1.11869345699256\\
60.5	0.1044	-0.945094872765594\\
60.5	0.105	-3.00888320252375\\
60.5	0.1056	-5.07267153228213\\
60.5	0.1062	-7.13645986204028\\
60.5	0.1068	-9.20024819179844\\
60.5	0.1074	-11.2640365215566\\
60.5	0.108	-13.3278248513147\\
60.5	0.1086	-15.3916131810729\\
60.5	0.1092	-17.4554015108311\\
60.5	0.1098	-19.5191898405892\\
60.5	0.1104	-21.5829781703474\\
60.5	0.111	-23.6467665001055\\
60.5	0.1116	-25.7105548298637\\
60.5	0.1122	-27.774343159622\\
60.5	0.1128	-29.8381314893802\\
60.5	0.1134	-31.9019198191384\\
60.5	0.114	-33.9657081488965\\
60.5	0.1146	-36.0294964786547\\
60.5	0.1152	-38.093284808413\\
60.5	0.1158	-40.157073138171\\
60.5	0.1164	-42.2208614679291\\
60.5	0.117	-44.2846497976875\\
60.5	0.1176	-46.3484381274457\\
60.5	0.1182	-48.4122264572038\\
60.5	0.1188	-50.476014786962\\
60.5	0.1194	-52.5398031167203\\
60.5	0.12	-54.6035914464785\\
60.5	0.1206	-56.6673797762367\\
60.5	0.1212	-58.7311681059948\\
60.5	0.1218	-60.794956435753\\
60.5	0.1224	-62.8587447655111\\
60.5	0.123	-64.9225330952693\\
60.875	0.093	39.2843780887656\\
60.875	0.0936	37.2782104752455\\
60.875	0.0942	35.2720428617251\\
60.875	0.0948	33.2658752482046\\
60.875	0.0954	31.2597076346842\\
60.875	0.096	29.2535400211641\\
60.875	0.0966	27.2473724076438\\
60.875	0.0972	25.2412047941234\\
60.875	0.0978	23.2350371806028\\
60.875	0.0984	21.2288695670825\\
60.875	0.099	19.2227019535624\\
60.875	0.0996	17.216534340042\\
60.875	0.1002	15.2103667265214\\
60.875	0.1008	13.2041991130011\\
60.875	0.1014	11.1980314994807\\
60.875	0.102	9.19186388596063\\
60.875	0.1026	7.18569627244028\\
60.875	0.1032	5.17952865891971\\
60.875	0.1038	3.17336104539936\\
60.875	0.1044	1.16719343187924\\
60.875	0.105	-0.838974181641106\\
60.875	0.1056	-2.84514179516168\\
60.875	0.1062	-4.85130940868203\\
60.875	0.1068	-6.85747702220237\\
60.875	0.1074	-8.86364463572249\\
60.875	0.108	-10.8698122492428\\
60.875	0.1086	-12.8759798627632\\
60.875	0.1092	-14.8821474762838\\
60.875	0.1098	-16.8883150898039\\
60.875	0.1104	-18.8944827033242\\
60.875	0.111	-20.9006503168446\\
60.875	0.1116	-22.9068179303649\\
60.875	0.1122	-24.9129855438855\\
60.875	0.1128	-26.9191531574056\\
60.875	0.1134	-28.925320770926\\
60.875	0.114	-30.9314883844463\\
60.875	0.1146	-32.9376559979667\\
60.875	0.1152	-34.9438236114872\\
60.875	0.1158	-36.9499912250074\\
60.875	0.1164	-38.9561588385277\\
60.875	0.117	-40.9623264520483\\
60.875	0.1176	-42.9684940655686\\
60.875	0.1182	-44.9746616790887\\
60.875	0.1188	-46.9808292926091\\
60.875	0.1194	-48.9869969061297\\
60.875	0.12	-50.99316451965\\
60.875	0.1206	-52.9993321331704\\
60.875	0.1212	-55.0054997466905\\
60.875	0.1218	-57.011667360211\\
60.875	0.1224	-59.0178349737314\\
60.875	0.123	-61.0240025872517\\
61.25	0.093	40.3018727848917\\
61.25	0.0936	38.3533258876091\\
61.25	0.0942	36.4047789903266\\
61.25	0.0948	34.456232093044\\
61.25	0.0954	32.5076851957615\\
61.25	0.096	30.5591382984792\\
61.25	0.0966	28.6105914011966\\
61.25	0.0972	26.6620445039141\\
61.25	0.0978	24.7134976066316\\
61.25	0.0984	22.764950709349\\
61.25	0.099	20.8164038120665\\
61.25	0.0996	18.8678569147842\\
61.25	0.1002	16.9193100175014\\
61.25	0.1008	14.9707631202189\\
61.25	0.1014	13.0222162229365\\
61.25	0.102	11.073669325654\\
61.25	0.1026	9.12512242837147\\
61.25	0.1032	7.17657553108893\\
61.25	0.1038	5.22802863380639\\
61.25	0.1044	3.27948173652385\\
61.25	0.105	1.33093483924154\\
61.25	0.1056	-0.617612058041232\\
61.25	0.1062	-2.56615895532377\\
61.25	0.1068	-4.51470585260608\\
61.25	0.1074	-6.46325274988862\\
61.25	0.108	-8.41179964717116\\
61.25	0.1086	-10.3603465444535\\
61.25	0.1092	-12.3088934417362\\
61.25	0.1098	-14.2574403390188\\
61.25	0.1104	-16.2059872363011\\
61.25	0.111	-18.1545341335836\\
61.25	0.1116	-20.1030810308662\\
61.25	0.1122	-22.0516279281487\\
61.25	0.1128	-24.0001748254313\\
61.25	0.1134	-25.9487217227138\\
61.25	0.114	-27.8972686199961\\
61.25	0.1146	-29.8458155172787\\
61.25	0.1152	-31.7943624145614\\
61.25	0.1158	-33.7429093118437\\
61.25	0.1164	-35.6914562091263\\
61.25	0.117	-37.640003106409\\
61.25	0.1176	-39.5885500036914\\
61.25	0.1182	-41.5370969009739\\
61.25	0.1188	-43.4856437982564\\
61.25	0.1194	-45.434190695539\\
61.25	0.12	-47.3827375928215\\
61.25	0.1206	-49.3312844901038\\
61.25	0.1212	-51.2798313873864\\
61.25	0.1218	-53.2283782846691\\
61.25	0.1224	-55.1769251819514\\
61.25	0.123	-57.125472079234\\
61.625	0.093	41.3193674810177\\
61.625	0.0936	39.4284412999732\\
61.625	0.0942	37.5375151189285\\
61.625	0.0948	35.6465889378837\\
61.625	0.0954	33.755662756839\\
61.625	0.096	31.8647365757943\\
61.625	0.0966	29.9738103947498\\
61.625	0.0972	28.082884213705\\
61.625	0.0978	26.1919580326603\\
61.625	0.0984	24.3010318516156\\
61.625	0.099	22.4101056705711\\
61.625	0.0996	20.5191794895263\\
61.625	0.1002	18.6282533084816\\
61.625	0.1008	16.7373271274369\\
61.625	0.1014	14.8464009463924\\
61.625	0.102	12.9554747653476\\
61.625	0.1026	11.0645485843031\\
61.625	0.1032	9.17362240325815\\
61.625	0.1038	7.28269622221364\\
61.625	0.1044	5.39177004116891\\
61.625	0.105	3.50084386012418\\
61.625	0.1056	1.60991767907944\\
61.625	0.1062	-0.281008501965289\\
61.625	0.1068	-2.17193468300979\\
61.625	0.1074	-4.06286086405453\\
61.625	0.108	-5.95378704509903\\
61.625	0.1086	-7.84471322614377\\
61.625	0.1092	-9.7356394071885\\
61.625	0.1098	-11.6265655882332\\
61.625	0.1104	-13.5174917692777\\
61.625	0.111	-15.4084179503225\\
61.625	0.1116	-17.299344131367\\
61.625	0.1122	-19.1902703124119\\
61.625	0.1128	-21.0811964934564\\
61.625	0.1134	-22.9721226745012\\
61.625	0.114	-24.8630488555459\\
61.625	0.1146	-26.7539750365904\\
61.625	0.1152	-28.6449012176354\\
61.625	0.1158	-30.5358273986799\\
61.625	0.1164	-32.4267535797246\\
61.625	0.117	-34.3176797607694\\
61.625	0.1176	-36.2086059418141\\
61.625	0.1182	-38.0995321228586\\
61.625	0.1188	-39.9904583039033\\
61.625	0.1194	-41.8813844849481\\
61.625	0.12	-43.7723106659928\\
61.625	0.1206	-45.6632368470373\\
61.625	0.1212	-47.554163028082\\
61.625	0.1218	-49.4450892091268\\
61.625	0.1224	-51.3360153901715\\
61.625	0.123	-53.2269415712162\\
62	0.093	42.3368621771435\\
62	0.0936	40.5035567123366\\
62	0.0942	38.6702512475299\\
62	0.0948	36.836945782723\\
62	0.0954	35.0036403179161\\
62	0.096	33.1703348531094\\
62	0.0966	31.3370293883024\\
62	0.0972	29.5037239234957\\
62	0.0978	27.6704184586886\\
62	0.0984	25.8371129938819\\
62	0.099	24.0038075290752\\
62	0.0996	22.1705020642682\\
62	0.1002	20.3371965994613\\
62	0.1008	18.5038911346544\\
62	0.1014	16.6705856698477\\
62	0.102	14.837280205041\\
62	0.1026	13.0039747402341\\
62	0.1032	11.1706692754271\\
62	0.1038	9.33736381062022\\
62	0.1044	7.50405834581352\\
62	0.105	5.67075288100659\\
62	0.1056	3.83744741619967\\
62	0.1062	2.00414195139297\\
62	0.1068	0.17083648658604\\
62	0.1074	-1.66246897822066\\
62	0.108	-3.49577444302759\\
62	0.1086	-5.32907990783428\\
62	0.1092	-7.16238537264121\\
62	0.1098	-8.99569083744814\\
62	0.1104	-10.8289963022548\\
62	0.111	-12.6623017670618\\
62	0.1116	-14.4956072318685\\
62	0.1122	-16.3289126966754\\
62	0.1128	-18.1622181614823\\
62	0.1134	-19.995523626289\\
62	0.114	-21.8288290910959\\
62	0.1146	-23.6621345559026\\
62	0.1152	-25.4954400207098\\
62	0.1158	-27.3287454855165\\
62	0.1164	-29.1620509503232\\
62	0.117	-30.9953564151303\\
62	0.1176	-32.828661879937\\
62	0.1182	-34.661967344744\\
62	0.1188	-36.4952728095507\\
62	0.1194	-38.3285782743576\\
62	0.12	-40.1618837391645\\
62	0.1206	-41.9951892039712\\
62	0.1212	-43.8284946687781\\
62	0.1218	-45.6618001335851\\
62	0.1224	-47.495105598392\\
62	0.123	-49.3284110631987\\
62.375	0.093	43.3543568732696\\
62.375	0.0936	41.5786721247007\\
62.375	0.0942	39.8029873761318\\
62.375	0.0948	38.0273026275624\\
62.375	0.0954	36.2516178789936\\
62.375	0.096	34.4759331304247\\
62.375	0.0966	32.7002483818555\\
62.375	0.0972	30.9245636332867\\
62.375	0.0978	29.1488788847175\\
62.375	0.0984	27.3731941361486\\
62.375	0.099	25.5975093875795\\
62.375	0.0996	23.8218246390106\\
62.375	0.1002	22.0461398904415\\
62.375	0.1008	20.2704551418724\\
62.375	0.1014	18.4947703933035\\
62.375	0.102	16.7190856447346\\
62.375	0.1026	14.9434008961655\\
62.375	0.1032	13.1677161475964\\
62.375	0.1038	11.3920313990275\\
62.375	0.1044	9.61634665045858\\
62.375	0.105	7.84066190188946\\
62.375	0.1056	6.06497715332034\\
62.375	0.1062	4.28929240475145\\
62.375	0.1068	2.51360765618233\\
62.375	0.1074	0.737922907613438\\
62.375	0.108	-1.03776184095545\\
62.375	0.1086	-2.81344658952435\\
62.375	0.1092	-4.58913133809369\\
62.375	0.1098	-6.36481608666259\\
62.375	0.1104	-8.14050083523148\\
62.375	0.111	-9.9161855838006\\
62.375	0.1116	-11.6918703323695\\
62.375	0.1122	-13.4675550809386\\
62.375	0.1128	-15.2432398295075\\
62.375	0.1134	-17.0189245780766\\
62.375	0.114	-18.7946093266455\\
62.375	0.1146	-20.5702940752144\\
62.375	0.1152	-22.3459788237838\\
62.375	0.1158	-24.1216635723526\\
62.375	0.1164	-25.8973483209215\\
62.375	0.117	-27.6730330694907\\
62.375	0.1176	-29.4487178180598\\
62.375	0.1182	-31.2244025666287\\
62.375	0.1188	-33.0000873151976\\
62.375	0.1194	-34.7757720637669\\
62.375	0.12	-36.5514568123358\\
62.375	0.1206	-38.3271415609047\\
62.375	0.1212	-40.1028263094738\\
62.375	0.1218	-41.8785110580429\\
62.375	0.1224	-43.6541958066118\\
62.375	0.123	-45.4298805551807\\
62.75	0.093	44.3718515693954\\
62.75	0.0936	42.6537875370643\\
62.75	0.0942	40.9357235047332\\
62.75	0.0948	39.2176594724019\\
62.75	0.0954	37.4995954400708\\
62.75	0.096	35.7815314077397\\
62.75	0.0966	34.0634673754084\\
62.75	0.0972	32.3454033430774\\
62.75	0.0978	30.627339310746\\
62.75	0.0984	28.909275278415\\
62.75	0.099	27.1912112460839\\
62.75	0.0996	25.4731472137528\\
62.75	0.1002	23.7550831814212\\
62.75	0.1008	22.0370191490902\\
62.75	0.1014	20.3189551167591\\
62.75	0.102	18.600891084428\\
62.75	0.1026	16.8828270520969\\
62.75	0.1032	15.1647630197656\\
62.75	0.1038	13.4466989874343\\
62.75	0.1044	11.7286349551032\\
62.75	0.105	10.0105709227721\\
62.75	0.1056	8.29250689044079\\
62.75	0.1062	6.57444285810971\\
62.75	0.1068	4.85637882577862\\
62.75	0.1074	3.13831479344753\\
62.75	0.108	1.42025076111622\\
62.75	0.1086	-0.297813271214864\\
62.75	0.1092	-2.01587730354618\\
62.75	0.1098	-3.73394133587726\\
62.75	0.1104	-5.45200536820835\\
62.75	0.111	-7.17006940053943\\
62.75	0.1116	-8.88813343287075\\
62.75	0.1122	-10.6061974652021\\
62.75	0.1128	-12.3242614975331\\
62.75	0.1134	-14.0423255298642\\
62.75	0.114	-15.7603895621953\\
62.75	0.1146	-17.4784535945264\\
62.75	0.1152	-19.1965176268579\\
62.75	0.1158	-20.914581659189\\
62.75	0.1164	-22.6326456915201\\
62.75	0.117	-24.3507097238514\\
62.75	0.1176	-26.0687737561825\\
62.75	0.1182	-27.7868377885136\\
62.75	0.1188	-29.5049018208449\\
62.75	0.1194	-31.2229658531762\\
62.75	0.12	-32.9410298855073\\
62.75	0.1206	-34.6590939178384\\
62.75	0.1212	-36.3771579501695\\
62.75	0.1218	-38.0952219825008\\
62.75	0.1224	-39.8132860148321\\
62.75	0.123	-41.5313500471632\\
63.125	0.093	45.3893462655215\\
63.125	0.0936	43.7289029494282\\
63.125	0.0942	42.0684596333349\\
63.125	0.0948	40.4080163172414\\
63.125	0.0954	38.7475730011481\\
63.125	0.096	37.0871296850548\\
63.125	0.0966	35.4266863689616\\
63.125	0.0972	33.7662430528683\\
63.125	0.0978	32.1057997367748\\
63.125	0.0984	30.4453564206815\\
63.125	0.099	28.7849131045882\\
63.125	0.0996	27.1244697884949\\
63.125	0.1002	25.4640264724014\\
63.125	0.1008	23.8035831563079\\
63.125	0.1014	22.1431398402146\\
63.125	0.102	20.4826965241214\\
63.125	0.1026	18.8222532080281\\
63.125	0.1032	17.1618098919346\\
63.125	0.1038	15.5013665758413\\
63.125	0.1044	13.840923259748\\
63.125	0.105	12.1804799436547\\
63.125	0.1056	10.5200366275612\\
63.125	0.1062	8.85959331146796\\
63.125	0.1068	7.19914999537468\\
63.125	0.1074	5.5387066792814\\
63.125	0.108	3.87826336318813\\
63.125	0.1086	2.21782004709485\\
63.125	0.1092	0.557376731001341\\
63.125	0.1098	-1.10306658509194\\
63.125	0.1104	-2.76350990118522\\
63.125	0.111	-4.42395321727849\\
63.125	0.1116	-6.08439653337177\\
63.125	0.1122	-7.74483984946528\\
63.125	0.1128	-9.40528316555856\\
63.125	0.1134	-11.0657264816518\\
63.125	0.114	-12.7261697977453\\
63.125	0.1146	-14.3866131138386\\
63.125	0.1152	-16.0470564299321\\
63.125	0.1158	-17.7074997460254\\
63.125	0.1164	-19.3679430621187\\
63.125	0.117	-21.0283863782122\\
63.125	0.1176	-22.6888296943055\\
63.125	0.1182	-24.3492730103987\\
63.125	0.1188	-26.009716326492\\
63.125	0.1194	-27.6701596425855\\
63.125	0.12	-29.3306029586788\\
63.125	0.1206	-30.9910462747721\\
63.125	0.1212	-32.6514895908654\\
63.125	0.1218	-34.3119329069589\\
63.125	0.1224	-35.9723762230522\\
63.125	0.123	-37.6328195391454\\
63.5	0.093	46.4068409616473\\
63.5	0.0936	44.8040183617918\\
63.5	0.0942	43.2011957619366\\
63.5	0.0948	41.5983731620809\\
63.5	0.0954	39.9955505622254\\
63.5	0.096	38.3927279623699\\
63.5	0.0966	36.7899053625144\\
63.5	0.0972	35.187082762659\\
63.5	0.0978	33.5842601628033\\
63.5	0.0984	31.9814375629478\\
63.5	0.099	30.3786149630923\\
63.5	0.0996	28.7757923632369\\
63.5	0.1002	27.1729697633812\\
63.5	0.1008	25.5701471635257\\
63.5	0.1014	23.9673245636704\\
63.5	0.102	22.364501963815\\
63.5	0.1026	20.7616793639595\\
63.5	0.1032	19.1588567641038\\
63.5	0.1038	17.5560341642483\\
63.5	0.1044	15.9532115643929\\
63.5	0.105	14.3503889645374\\
63.5	0.1056	12.7475663646817\\
63.5	0.1062	11.1447437648262\\
63.5	0.1068	9.54192116497074\\
63.5	0.1074	7.93909856511527\\
63.5	0.108	6.3362759652598\\
63.5	0.1086	4.73345336540433\\
63.5	0.1092	3.13063076554886\\
63.5	0.1098	1.52780816569339\\
63.5	0.1104	-0.0750144341620853\\
63.5	0.111	-1.67783703401756\\
63.5	0.1116	-3.28065963387303\\
63.5	0.1122	-4.88348223372873\\
63.5	0.1128	-6.4863048335842\\
63.5	0.1134	-8.08912743343967\\
63.5	0.114	-9.69195003329514\\
63.5	0.1146	-11.2947726331506\\
63.5	0.1152	-12.8975952330063\\
63.5	0.1158	-14.5004178328618\\
63.5	0.1164	-16.103240432717\\
63.5	0.117	-17.7060630325727\\
63.5	0.1176	-19.3088856324282\\
63.5	0.1182	-20.9117082322837\\
63.5	0.1188	-22.5145308321391\\
63.5	0.1194	-24.1173534319948\\
63.5	0.12	-25.7201760318503\\
63.5	0.1206	-27.3229986317058\\
63.5	0.1212	-28.9258212315613\\
63.5	0.1218	-30.528643831417\\
63.5	0.1224	-32.1314664312724\\
63.5	0.123	-33.7342890311279\\
63.875	0.093	47.4243356577733\\
63.875	0.0936	45.8791337741557\\
63.875	0.0942	44.333931890538\\
63.875	0.0948	42.7887300069203\\
63.875	0.0954	41.2435281233027\\
63.875	0.096	39.698326239685\\
63.875	0.0966	38.1531243560673\\
63.875	0.0972	36.6079224724497\\
63.875	0.0978	35.062720588832\\
63.875	0.0984	33.5175187052143\\
63.875	0.099	31.9723168215967\\
63.875	0.0996	30.427114937979\\
63.875	0.1002	28.8819130543611\\
63.875	0.1008	27.3367111707437\\
63.875	0.1014	25.791509287126\\
63.875	0.102	24.2463074035084\\
63.875	0.1026	22.7011055198907\\
63.875	0.1032	21.1559036362728\\
63.875	0.1038	19.6107017526554\\
63.875	0.1044	18.0654998690377\\
63.875	0.105	16.52029798542\\
63.875	0.1056	14.9750961018021\\
63.875	0.1062	13.4298942181845\\
63.875	0.1068	11.884692334567\\
63.875	0.1074	10.3394904509494\\
63.875	0.108	8.79428856733171\\
63.875	0.1086	7.24908668371404\\
63.875	0.1092	5.70388480009615\\
63.875	0.1098	4.15868291647871\\
63.875	0.1104	2.61348103286105\\
63.875	0.111	1.06827914924338\\
63.875	0.1116	-0.476922734374284\\
63.875	0.1122	-2.02212461799218\\
63.875	0.1128	-3.56732650160961\\
63.875	0.1134	-5.11252838522728\\
63.875	0.114	-6.65773026884494\\
63.875	0.1146	-8.20293215246261\\
63.875	0.1152	-9.7481340360805\\
63.875	0.1158	-11.2933359196979\\
63.875	0.1164	-12.8385378033156\\
63.875	0.117	-14.3837396869335\\
63.875	0.1176	-15.9289415705512\\
63.875	0.1182	-17.4741434541688\\
63.875	0.1188	-19.0193453377863\\
63.875	0.1194	-20.5645472214042\\
63.875	0.12	-22.1097491050218\\
63.875	0.1206	-23.6549509886395\\
63.875	0.1212	-25.2001528722571\\
63.875	0.1218	-26.7453547558748\\
63.875	0.1224	-28.2905566394925\\
63.875	0.123	-29.8357585231101\\
64.25	0.093	48.4418303538994\\
64.25	0.0936	46.9542491865195\\
64.25	0.0942	45.4666680191399\\
64.25	0.0948	43.9790868517598\\
64.25	0.0954	42.4915056843799\\
64.25	0.096	41.0039245170003\\
64.25	0.0966	39.5163433496205\\
64.25	0.0972	38.0287621822406\\
64.25	0.0978	36.5411810148607\\
64.25	0.0984	35.0535998474809\\
64.25	0.099	33.566018680101\\
64.25	0.0996	32.0784375127214\\
64.25	0.1002	30.5908563453413\\
64.25	0.1008	29.1032751779615\\
64.25	0.1014	27.6156940105818\\
64.25	0.102	26.128112843202\\
64.25	0.1026	24.6405316758221\\
64.25	0.1032	23.1529505084422\\
64.25	0.1038	21.6653693410624\\
64.25	0.1044	20.1777881736825\\
64.25	0.105	18.6902070063029\\
64.25	0.1056	17.2026258389228\\
64.25	0.1062	15.715044671543\\
64.25	0.1068	14.2274635041633\\
64.25	0.1074	12.7398823367835\\
64.25	0.108	11.2523011694036\\
64.25	0.1086	9.76472000202398\\
64.25	0.1092	8.27713883464389\\
64.25	0.1098	6.78955766726403\\
64.25	0.1104	5.3019764998844\\
64.25	0.111	3.81439533250455\\
64.25	0.1116	2.32681416512469\\
64.25	0.1122	0.83923299774483\\
64.25	0.1128	-0.648348169635028\\
64.25	0.1134	-2.13592933701489\\
64.25	0.114	-3.62351050439452\\
64.25	0.1146	-5.11109167177437\\
64.25	0.1152	-6.59867283915446\\
64.25	0.1158	-8.08625400653409\\
64.25	0.1164	-9.57383517391395\\
64.25	0.117	-11.061416341294\\
64.25	0.1176	-12.5489975086737\\
64.25	0.1182	-14.0365786760535\\
64.25	0.1188	-15.5241598434334\\
64.25	0.1194	-17.0117410108132\\
64.25	0.12	-18.4993221781931\\
64.25	0.1206	-19.986903345573\\
64.25	0.1212	-21.4744845129526\\
64.25	0.1218	-22.9620656803327\\
64.25	0.1224	-24.4496468477125\\
64.25	0.123	-25.9372280150924\\
64.625	0.093	49.4593250500254\\
64.625	0.0936	48.0293645988834\\
64.625	0.0942	46.5994041477416\\
64.625	0.0948	45.1694436965993\\
64.625	0.0954	43.7394832454572\\
64.625	0.096	42.3095227943154\\
64.625	0.0966	40.8795623431733\\
64.625	0.0972	39.4496018920315\\
64.625	0.0978	38.0196414408892\\
64.625	0.0984	36.5896809897474\\
64.625	0.099	35.1597205386054\\
64.625	0.0996	33.7297600874635\\
64.625	0.1002	32.2997996363213\\
64.625	0.1008	30.8698391851794\\
64.625	0.1014	29.4398787340374\\
64.625	0.102	28.0099182828953\\
64.625	0.1026	26.5799578317535\\
64.625	0.1032	25.1499973806112\\
64.625	0.1038	23.7200369294694\\
64.625	0.1044	22.2900764783274\\
64.625	0.105	20.8601160271855\\
64.625	0.1056	19.4301555760433\\
64.625	0.1062	18.0001951249014\\
64.625	0.1068	16.5702346737594\\
64.625	0.1074	15.1402742226173\\
64.625	0.108	13.7103137714755\\
64.625	0.1086	12.2803533203335\\
64.625	0.1092	10.8503928691914\\
64.625	0.1098	9.42043241804936\\
64.625	0.1104	7.99047196690753\\
64.625	0.111	6.56051151576548\\
64.625	0.1116	5.13055106462366\\
64.625	0.1122	3.70059061348138\\
64.625	0.1128	2.27063016233956\\
64.625	0.1134	0.840669711197506\\
64.625	0.114	-0.589290739944545\\
64.625	0.1146	-2.01925119108637\\
64.625	0.1152	-3.44921164222865\\
64.625	0.1158	-4.87917209337047\\
64.625	0.1164	-6.30913254451252\\
64.625	0.117	-7.73909299565457\\
64.625	0.1176	-9.16905344679662\\
64.625	0.1182	-10.5990138979384\\
64.625	0.1188	-12.0289743490805\\
64.625	0.1194	-13.4589348002228\\
64.625	0.12	-14.8888952513646\\
64.625	0.1206	-16.3188557025067\\
64.625	0.1212	-17.7488161536485\\
64.625	0.1218	-19.1787766047908\\
64.625	0.1224	-20.6087370559326\\
64.625	0.123	-22.0386975070744\\
65	0.093	50.476819746151\\
65	0.0936	49.104480011247\\
65	0.0942	47.7321402763428\\
65	0.0948	46.3598005414385\\
65	0.0954	44.9874608065345\\
65	0.096	43.6151210716303\\
65	0.0966	42.2427813367262\\
65	0.0972	40.870441601822\\
65	0.0978	39.4981018669178\\
65	0.0984	38.1257621320135\\
65	0.099	36.7534223971095\\
65	0.0996	35.3810826622055\\
65	0.1002	34.008742927301\\
65	0.1008	32.636403192397\\
65	0.1014	31.2640634574927\\
65	0.102	29.8917237225887\\
65	0.1026	28.5193839876845\\
65	0.1032	27.1470442527802\\
65	0.1038	25.7747045178762\\
65	0.1044	24.402364782972\\
65	0.105	23.030025048068\\
65	0.1056	21.6576853131635\\
65	0.1062	20.2853455782595\\
65	0.1068	18.9130058433552\\
65	0.1074	17.5406661084512\\
65	0.108	16.1683263735472\\
65	0.1086	14.7959866386429\\
65	0.1092	13.4236469037387\\
65	0.1098	12.0513071688345\\
65	0.1104	10.6789674339304\\
65	0.111	9.30662769902619\\
65	0.1116	7.93428796412218\\
65	0.1122	6.56194822921793\\
65	0.1128	5.18960849431369\\
65	0.1134	3.81726875940967\\
65	0.114	2.44492902450543\\
65	0.1146	1.07258928960141\\
65	0.1152	-0.299750445303061\\
65	0.1158	-1.67209018020708\\
65	0.1164	-3.0444299151111\\
65	0.117	-4.41676965001557\\
65	0.1176	-5.78910938491958\\
65	0.1182	-7.16144911982383\\
65	0.1188	-8.53378885472785\\
65	0.1194	-9.90612858963232\\
65	0.12	-11.2784683245363\\
65	0.1206	-12.6508080594406\\
65	0.1212	-14.0231477943446\\
65	0.1218	-15.3954875292488\\
65	0.1224	-16.7678272641531\\
65	0.123	-18.1401669990571\\
65.375	0.093	51.4943144422773\\
65.375	0.0936	50.1795954236111\\
65.375	0.0942	48.8648764049449\\
65.375	0.0948	47.5501573862782\\
65.375	0.0954	46.235438367612\\
65.375	0.096	44.9207193489458\\
65.375	0.0966	43.6060003302796\\
65.375	0.0972	42.2912813116131\\
65.375	0.0978	40.9765622929467\\
65.375	0.0984	39.6618432742805\\
65.375	0.099	38.3471242556141\\
65.375	0.0996	37.0324052369479\\
65.375	0.1002	35.7176862182814\\
65.375	0.1008	34.402967199615\\
65.375	0.1014	33.0882481809488\\
65.375	0.102	31.7735291622826\\
65.375	0.1026	30.4588101436163\\
65.375	0.1032	29.1440911249497\\
65.375	0.1038	27.8293721062835\\
65.375	0.1044	26.5146530876173\\
65.375	0.105	25.1999340689508\\
65.375	0.1056	23.8852150502844\\
65.375	0.1062	22.5704960316182\\
65.375	0.1068	21.2557770129517\\
65.375	0.1074	19.9410579942855\\
65.375	0.108	18.6263389756193\\
65.375	0.1086	17.3116199569531\\
65.375	0.1092	15.9969009382864\\
65.375	0.1098	14.6821819196202\\
65.375	0.1104	13.367462900954\\
65.375	0.111	12.0527438822876\\
65.375	0.1116	10.7380248636214\\
65.375	0.1122	9.42330584495494\\
65.375	0.1128	8.1085868262885\\
65.375	0.1134	6.79386780762229\\
65.375	0.114	5.47914878895608\\
65.375	0.1146	4.16442977028987\\
65.375	0.1152	2.84971075162321\\
65.375	0.1158	1.534991732957\\
65.375	0.1164	0.220272714290786\\
65.375	0.117	-1.09444630437588\\
65.375	0.1176	-2.40916532304209\\
65.375	0.1182	-3.7238843417083\\
65.375	0.1188	-5.03860336037474\\
65.375	0.1194	-6.35332237904117\\
65.375	0.12	-7.66804139770738\\
65.375	0.1206	-8.98276041637359\\
65.375	0.1212	-10.29747943504\\
65.375	0.1218	-11.6121984537065\\
65.375	0.1224	-12.9269174723727\\
65.375	0.123	-14.2416364910391\\
65.75	0.093	52.5118091384031\\
65.75	0.0936	51.2547108359747\\
65.75	0.0942	49.9976125335463\\
65.75	0.0948	48.7405142311175\\
65.75	0.0954	47.4834159286891\\
65.75	0.096	46.2263176262607\\
65.75	0.0966	44.9692193238322\\
65.75	0.0972	43.7121210214038\\
65.75	0.0978	42.4550227189752\\
65.75	0.0984	41.1979244165466\\
65.75	0.099	39.9408261141182\\
65.75	0.0996	38.6837278116898\\
65.75	0.1002	37.4266295092611\\
65.75	0.1008	36.1695312068327\\
65.75	0.1014	34.9124329044041\\
65.75	0.102	33.6553346019757\\
65.75	0.1026	32.3982362995473\\
65.75	0.1032	31.1411379971187\\
65.75	0.1038	29.8840396946903\\
65.75	0.1044	28.6269413922619\\
65.75	0.105	27.3698430898332\\
65.75	0.1056	26.1127447874046\\
65.75	0.1062	24.8556464849762\\
65.75	0.1068	23.5985481825478\\
65.75	0.1074	22.3414498801194\\
65.75	0.108	21.0843515776908\\
65.75	0.1086	19.8272532752624\\
65.75	0.1092	18.5701549728337\\
65.75	0.1098	17.3130566704053\\
65.75	0.1104	16.0559583679769\\
65.75	0.111	14.7988600655485\\
65.75	0.1116	13.5417617631199\\
65.75	0.1122	12.2846634606913\\
65.75	0.1128	11.0275651582629\\
65.75	0.1134	9.77046685583446\\
65.75	0.114	8.51336855340605\\
65.75	0.1146	7.25627025097742\\
65.75	0.1152	5.99917194854879\\
65.75	0.1158	4.74207364612039\\
65.75	0.1164	3.48497534369199\\
65.75	0.117	2.22787704126335\\
65.75	0.1176	0.970778738834952\\
65.75	0.1182	-0.286319563593679\\
65.75	0.1188	-1.54341786602208\\
65.75	0.1194	-2.80051616845071\\
65.75	0.12	-4.05761447087912\\
65.75	0.1206	-5.31471277330752\\
65.75	0.1212	-6.57181107573615\\
65.75	0.1218	-7.82890937816455\\
65.75	0.1224	-9.08600768059341\\
65.75	0.123	-10.3431059830214\\
66.125	0.093	53.5293038345292\\
66.125	0.0936	52.3298262483386\\
66.125	0.0942	51.1303486621478\\
66.125	0.0948	49.9308710759569\\
66.125	0.0954	48.7313934897663\\
66.125	0.096	47.5319159035757\\
66.125	0.0966	46.3324383173851\\
66.125	0.0972	45.1329607311945\\
66.125	0.0978	43.9334831450037\\
66.125	0.0984	42.7340055588131\\
66.125	0.099	41.5345279726225\\
66.125	0.0996	40.3350503864319\\
66.125	0.1002	39.1355728002411\\
66.125	0.1008	37.9360952140505\\
66.125	0.1014	36.7366176278599\\
66.125	0.102	35.5371400416693\\
66.125	0.1026	34.3376624554787\\
66.125	0.1032	33.1381848692879\\
66.125	0.1038	31.9387072830971\\
66.125	0.1044	30.7392296969065\\
66.125	0.105	29.5397521107159\\
66.125	0.1056	28.3402745245251\\
66.125	0.1062	27.1407969383345\\
66.125	0.1068	25.9413193521439\\
66.125	0.1074	24.7418417659533\\
66.125	0.108	23.5423641797627\\
66.125	0.1086	22.3428865935721\\
66.125	0.1092	21.1434090073813\\
66.125	0.1098	19.9439314211907\\
66.125	0.1104	18.7444538350001\\
66.125	0.111	17.5449762488095\\
66.125	0.1116	16.3454986626189\\
66.125	0.1122	15.146021076428\\
66.125	0.1128	13.9465434902374\\
66.125	0.1134	12.7470659040468\\
66.125	0.114	11.547588317856\\
66.125	0.1146	10.3481107316654\\
66.125	0.1152	9.14863314547461\\
66.125	0.1158	7.94915555928401\\
66.125	0.1164	6.74967797309341\\
66.125	0.117	5.55020038690259\\
66.125	0.1176	4.35072280071199\\
66.125	0.1182	3.1512452145214\\
66.125	0.1188	1.9517676283308\\
66.125	0.1194	0.752290042139975\\
66.125	0.12	-0.447187544050621\\
66.125	0.1206	-1.64666513024122\\
66.125	0.1212	-2.84614271643159\\
66.125	0.1218	-4.04562030262286\\
66.125	0.1224	-5.24509788881323\\
66.125	0.123	-6.44457547500406\\
66.5	0.093	54.5467985306548\\
66.5	0.0936	53.404941660702\\
66.5	0.0942	52.2630847907492\\
66.5	0.0948	51.1212279207962\\
66.5	0.0954	49.9793710508434\\
66.5	0.096	48.8375141808906\\
66.5	0.0966	47.695657310938\\
66.5	0.0972	46.5538004409852\\
66.5	0.0978	45.4119435710322\\
66.5	0.0984	44.2700867010794\\
66.5	0.099	43.1282298311266\\
66.5	0.0996	41.9863729611739\\
66.5	0.1002	40.8445160912208\\
66.5	0.1008	39.702659221268\\
66.5	0.1014	38.5608023513153\\
66.5	0.102	37.4189454813625\\
66.5	0.1026	36.2770886114097\\
66.5	0.1032	35.1352317414567\\
66.5	0.1038	33.9933748715039\\
66.5	0.1044	32.8515180015511\\
66.5	0.105	31.7096611315983\\
66.5	0.1056	30.5678042616453\\
66.5	0.1062	29.4259473916925\\
66.5	0.1068	28.2840905217399\\
66.5	0.1074	27.1422336517871\\
66.5	0.108	26.0003767818343\\
66.5	0.1086	24.8585199118816\\
66.5	0.1092	23.7166630419285\\
66.5	0.1098	22.5748061719758\\
66.5	0.1104	21.432949302023\\
66.5	0.111	20.2910924320702\\
66.5	0.1116	19.1492355621174\\
66.5	0.1122	18.0073786921644\\
66.5	0.1128	16.8655218222116\\
66.5	0.1134	15.7236649522588\\
66.5	0.114	14.581808082306\\
66.5	0.1146	13.4399512123532\\
66.5	0.1152	12.2980943424002\\
66.5	0.1158	11.1562374724474\\
66.5	0.1164	10.0143806024946\\
66.5	0.117	8.87252373254182\\
66.5	0.1176	7.73066686258903\\
66.5	0.1182	6.58880999263624\\
66.5	0.1188	5.44695312268345\\
66.5	0.1194	4.30509625273044\\
66.5	0.12	3.16323938277765\\
66.5	0.1206	2.02138251282486\\
66.5	0.1212	0.879525642872068\\
66.5	0.1218	-0.262331227081177\\
66.5	0.1224	-1.40418809703351\\
66.5	0.123	-2.5460449669863\\
66.875	0.093	55.564293226781\\
66.875	0.0936	54.4800570730661\\
66.875	0.0942	53.3958209193513\\
66.875	0.0948	52.3115847656361\\
66.875	0.0954	51.2273486119211\\
66.875	0.096	50.1431124582061\\
66.875	0.0966	49.0588763044912\\
66.875	0.0972	47.9746401507764\\
66.875	0.0978	46.8904039970612\\
66.875	0.0984	45.8061678433462\\
66.875	0.099	44.7219316896312\\
66.875	0.0996	43.6376955359162\\
66.875	0.1002	42.5534593822013\\
66.875	0.1008	41.4692232284863\\
66.875	0.1014	40.3849870747713\\
66.875	0.102	39.3007509210563\\
66.875	0.1026	38.2165147673413\\
66.875	0.1032	37.1322786136261\\
66.875	0.1038	36.0480424599114\\
66.875	0.1044	34.9638063061964\\
66.875	0.105	33.8795701524814\\
66.875	0.1056	32.7953339987662\\
66.875	0.1062	31.7110978450512\\
66.875	0.1068	30.6268616913364\\
66.875	0.1074	29.5426255376215\\
66.875	0.108	28.4583893839065\\
66.875	0.1086	27.3741532301915\\
66.875	0.1092	26.2899170764763\\
66.875	0.1098	25.2056809227615\\
66.875	0.1104	24.1214447690465\\
66.875	0.111	23.0372086153316\\
66.875	0.1116	21.9529724616166\\
66.875	0.1122	20.8687363079014\\
66.875	0.1128	19.7845001541864\\
66.875	0.1134	18.7002640004716\\
66.875	0.114	17.6160278467567\\
66.875	0.1146	16.5317916930417\\
66.875	0.1152	15.4475555393265\\
66.875	0.1158	14.3633193856115\\
66.875	0.1164	13.2790832318967\\
66.875	0.117	12.1948470781815\\
66.875	0.1176	11.1106109244665\\
66.875	0.1182	10.0263747707515\\
66.875	0.1188	8.94213861703656\\
66.875	0.1194	7.85790246332158\\
66.875	0.12	6.7736663096066\\
66.875	0.1206	5.68943015589139\\
66.875	0.1212	4.60519400217663\\
66.875	0.1218	3.52095784846142\\
66.875	0.1224	2.43672169474667\\
66.875	0.123	1.35248554103146\\
67.25	0.093	56.5817879229069\\
67.25	0.0936	55.5551724854297\\
67.25	0.0942	54.5285570479527\\
67.25	0.0948	53.5019416104753\\
67.25	0.0954	52.4753261729982\\
67.25	0.096	51.448710735521\\
67.25	0.0966	50.422095298044\\
67.25	0.0972	49.3954798605669\\
67.25	0.0978	48.3688644230895\\
67.25	0.0984	47.3422489856125\\
67.25	0.099	46.3156335481353\\
67.25	0.0996	45.2890181106582\\
67.25	0.1002	44.262402673181\\
67.25	0.1008	43.2357872357038\\
67.25	0.1014	42.2091717982266\\
67.25	0.102	41.1825563607495\\
67.25	0.1026	40.1559409232725\\
67.25	0.1032	39.1293254857951\\
67.25	0.1038	38.1027100483179\\
67.25	0.1044	37.076094610841\\
67.25	0.105	36.0494791733638\\
67.25	0.1056	35.0228637358864\\
67.25	0.1062	33.9962482984095\\
67.25	0.1068	32.9696328609323\\
67.25	0.1074	31.9430174234551\\
67.25	0.108	30.9164019859782\\
67.25	0.1086	29.889786548501\\
67.25	0.1092	28.8631711110236\\
67.25	0.1098	27.8365556735464\\
67.25	0.1104	26.8099402360695\\
67.25	0.111	25.7833247985923\\
67.25	0.1116	24.7567093611151\\
67.25	0.1122	23.7300939236379\\
67.25	0.1128	22.7034784861607\\
67.25	0.1134	21.6768630486836\\
67.25	0.114	20.6502476112066\\
67.25	0.1146	19.6236321737294\\
67.25	0.1152	18.597016736252\\
67.25	0.1158	17.5704012987751\\
67.25	0.1164	16.5437858612979\\
67.25	0.117	15.5171704238205\\
67.25	0.1176	14.4905549863436\\
67.25	0.1182	13.4639395488664\\
67.25	0.1188	12.4373241113892\\
67.25	0.1194	11.410708673912\\
67.25	0.12	10.3840932364351\\
67.25	0.1206	9.35747779895746\\
67.25	0.1212	8.33086236148074\\
67.25	0.1218	7.30424692400311\\
67.25	0.1224	6.27763148652639\\
67.25	0.123	5.25101604904876\\
67.625	0.093	57.5992826190327\\
67.625	0.0936	56.6302878977936\\
67.625	0.0942	55.6612931765542\\
67.625	0.0948	54.6922984553148\\
67.625	0.0954	53.7233037340754\\
67.625	0.096	52.7543090128363\\
67.625	0.0966	51.7853142915969\\
67.625	0.0972	50.8163195703576\\
67.625	0.0978	49.8473248491182\\
67.625	0.0984	48.8783301278788\\
67.625	0.099	47.9093354066397\\
67.625	0.0996	46.9403406854003\\
67.625	0.1002	45.9713459641609\\
67.625	0.1008	45.0023512429216\\
67.625	0.1014	44.0333565216822\\
67.625	0.102	43.0643618004431\\
67.625	0.1026	42.0953670792037\\
67.625	0.1032	41.1263723579643\\
67.625	0.1038	40.157377636725\\
67.625	0.1044	39.1883829154858\\
67.625	0.105	38.2193881942464\\
67.625	0.1056	37.2503934730069\\
67.625	0.1062	36.2813987517677\\
67.625	0.1068	35.3124040305283\\
67.625	0.1074	34.3434093092892\\
67.625	0.108	33.3744145880498\\
67.625	0.1086	32.4054198668107\\
67.625	0.1092	31.4364251455711\\
67.625	0.1098	30.4674304243317\\
67.625	0.1104	29.4984357030926\\
67.625	0.111	28.5294409818532\\
67.625	0.1116	27.5604462606141\\
67.625	0.1122	26.5914515393745\\
67.625	0.1128	25.6224568181353\\
67.625	0.1134	24.653462096896\\
67.625	0.114	23.6844673756566\\
67.625	0.1146	22.7154726544175\\
67.625	0.1152	21.7464779331779\\
67.625	0.1158	20.7774832119387\\
67.625	0.1164	19.8084884906993\\
67.625	0.117	18.83949376946\\
67.625	0.1176	17.8704990482206\\
67.625	0.1182	16.9015043269812\\
67.625	0.1188	15.9325096057421\\
67.625	0.1194	14.9635148845027\\
67.625	0.12	13.9945201632631\\
67.625	0.1206	13.0255254420244\\
67.625	0.1212	12.0565307207844\\
67.625	0.1218	11.0875359995457\\
67.625	0.1224	10.1185412783057\\
67.625	0.123	9.14954655706697\\
68	0.093	58.616777315159\\
68	0.0936	57.7054033101574\\
68	0.0942	56.7940293051561\\
68	0.0948	55.8826553001543\\
68	0.0954	54.9712812951529\\
68	0.096	54.0599072901514\\
68	0.0966	53.1485332851501\\
68	0.0972	52.2371592801485\\
68	0.0978	51.3257852751469\\
68	0.0984	50.4144112701456\\
68	0.099	49.503037265144\\
68	0.0996	48.5916632601427\\
68	0.1002	47.6802892551409\\
68	0.1008	46.7689152501396\\
68	0.1014	45.857541245138\\
68	0.102	44.9461672401367\\
68	0.1026	44.0347932351351\\
68	0.1032	43.1234192301335\\
68	0.1038	42.2120452251322\\
68	0.1044	41.3006712201307\\
68	0.105	40.3892972151293\\
68	0.1056	39.4779232101275\\
68	0.1062	38.5665492051262\\
68	0.1068	37.6551752001246\\
68	0.1074	36.7438011951233\\
68	0.108	35.832427190122\\
68	0.1086	34.9210531851204\\
68	0.1092	34.0096791801188\\
68	0.1098	33.0983051751173\\
68	0.1104	32.1869311701159\\
68	0.111	31.2755571651144\\
68	0.1116	30.364183160113\\
68	0.1122	29.4528091551113\\
68	0.1128	28.5414351501099\\
68	0.1134	27.6300611451086\\
68	0.114	26.718687140107\\
68	0.1146	25.8073131351057\\
68	0.1152	24.8959391301039\\
68	0.1158	23.9845651251026\\
68	0.1164	23.073191120101\\
68	0.117	22.1618171150994\\
68	0.1176	21.2504431100979\\
68	0.1182	20.3390691050968\\
68	0.1188	19.4276951000952\\
68	0.1194	18.5163210950936\\
68	0.12	17.6049470900921\\
68	0.1206	16.6935730850905\\
68	0.1212	15.782199080089\\
68	0.1218	14.8708250750874\\
68	0.1224	13.9594510700858\\
68	0.123	13.0480770650847\\
68.375	0.093	59.6342720112848\\
68.375	0.0936	58.7805187225213\\
68.375	0.0942	57.9267654337577\\
68.375	0.0948	57.0730121449938\\
68.375	0.0954	56.2192588562302\\
68.375	0.096	55.3655055674667\\
68.375	0.0966	54.5117522787029\\
68.375	0.0972	53.6579989899394\\
68.375	0.0978	52.8042457011757\\
68.375	0.0984	51.9504924124119\\
68.375	0.099	51.0967391236484\\
68.375	0.0996	50.2429858348848\\
68.375	0.1002	49.3892325461209\\
68.375	0.1008	48.5354792573573\\
68.375	0.1014	47.6817259685938\\
68.375	0.102	46.8279726798301\\
68.375	0.1026	45.9742193910665\\
68.375	0.1032	45.1204661023028\\
68.375	0.1038	44.266712813539\\
68.375	0.1044	43.4129595247755\\
68.375	0.105	42.559206236012\\
68.375	0.1056	41.705452947248\\
68.375	0.1062	40.8516996584844\\
68.375	0.1068	39.9979463697209\\
68.375	0.1074	39.1441930809572\\
68.375	0.108	38.2904397921936\\
68.375	0.1086	37.4366865034301\\
68.375	0.1092	36.5829332146661\\
68.375	0.1098	35.7291799259026\\
68.375	0.1104	34.8754266371391\\
68.375	0.111	34.0216733483755\\
68.375	0.1116	33.1679200596118\\
68.375	0.1122	32.314166770848\\
68.375	0.1128	31.4604134820845\\
68.375	0.1134	30.6066601933207\\
68.375	0.114	29.7529069045572\\
68.375	0.1146	28.8991536157937\\
68.375	0.1152	28.0454003270297\\
68.375	0.1158	27.1916470382662\\
68.375	0.1164	26.3378937495027\\
68.375	0.117	25.4841404607387\\
68.375	0.1176	24.6303871719751\\
68.375	0.1182	23.7766338832116\\
68.375	0.1188	22.9228805944481\\
68.375	0.1194	22.0691273056841\\
68.375	0.12	21.2153740169206\\
68.375	0.1206	20.361620728157\\
68.375	0.1212	19.5078674393935\\
68.375	0.1218	18.6541141506295\\
68.375	0.1224	17.800360861866\\
68.375	0.123	16.9466075731025\\
68.75	0.093	60.6517667074106\\
68.75	0.0936	59.8556341348849\\
68.75	0.0942	59.0595015623589\\
68.75	0.0948	58.263368989833\\
68.75	0.0954	57.4672364173073\\
68.75	0.096	56.6711038447816\\
68.75	0.0966	55.8749712722558\\
68.75	0.0972	55.0788386997299\\
68.75	0.0978	54.2827061272039\\
68.75	0.0984	53.4865735546782\\
68.75	0.099	52.6904409821525\\
68.75	0.0996	51.8943084096268\\
68.75	0.1002	51.0981758371006\\
68.75	0.1008	50.3020432645749\\
68.75	0.1014	49.5059106920492\\
68.75	0.102	48.7097781195234\\
68.75	0.1026	47.9136455469977\\
68.75	0.1032	47.1175129744715\\
68.75	0.1038	46.3213804019458\\
68.75	0.1044	45.5252478294201\\
68.75	0.105	44.7291152568944\\
68.75	0.1056	43.9329826843684\\
68.75	0.1062	43.1368501118425\\
68.75	0.1068	42.3407175393168\\
68.75	0.1074	41.544584966791\\
68.75	0.108	40.7484523942653\\
68.75	0.1086	39.9523198217396\\
68.75	0.1092	39.1561872492134\\
68.75	0.1098	38.3600546766877\\
68.75	0.1104	37.563922104162\\
68.75	0.111	36.7677895316363\\
68.75	0.1116	35.9716569591105\\
68.75	0.1122	35.1755243865844\\
68.75	0.1128	34.3793918140586\\
68.75	0.1134	33.5832592415329\\
68.75	0.114	32.7871266690072\\
68.75	0.1146	31.9909940964815\\
68.75	0.1152	31.1948615239553\\
68.75	0.1158	30.3987289514296\\
68.75	0.1164	29.6025963789036\\
68.75	0.117	28.8064638063777\\
68.75	0.1176	28.0103312338522\\
68.75	0.1182	27.2141986613265\\
68.75	0.1188	26.4180660888005\\
68.75	0.1194	25.6219335162746\\
68.75	0.12	24.8258009437486\\
68.75	0.1206	24.0296683712231\\
68.75	0.1212	23.2335357986972\\
68.75	0.1218	22.4374032261717\\
68.75	0.1224	21.6412706536453\\
68.75	0.123	20.8451380811198\\
69.125	0.093	61.6692614035369\\
69.125	0.0936	60.930749547249\\
69.125	0.0942	60.1922376909611\\
69.125	0.0948	59.4537258346729\\
69.125	0.0954	58.715213978385\\
69.125	0.096	57.9767021220969\\
69.125	0.0966	57.238190265809\\
69.125	0.0972	56.499678409521\\
69.125	0.0978	55.7611665532329\\
69.125	0.0984	55.022654696945\\
69.125	0.099	54.2841428406571\\
69.125	0.0996	53.5456309843692\\
69.125	0.1002	52.807119128081\\
69.125	0.1008	52.0686072717931\\
69.125	0.1014	51.3300954155052\\
69.125	0.102	50.5915835592173\\
69.125	0.1026	49.8530717029294\\
69.125	0.1032	49.1145598466412\\
69.125	0.1038	48.3760479903533\\
69.125	0.1044	47.6375361340652\\
69.125	0.105	46.8990242777772\\
69.125	0.1056	46.1605124214891\\
69.125	0.1062	45.4220005652012\\
69.125	0.1068	44.6834887089133\\
69.125	0.1074	43.9449768526254\\
69.125	0.108	43.2064649963374\\
69.125	0.1086	42.4679531400495\\
69.125	0.1092	41.7294412837614\\
69.125	0.1098	40.9909294274735\\
69.125	0.1104	40.2524175711856\\
69.125	0.111	39.5139057148976\\
69.125	0.1116	38.7753938586097\\
69.125	0.1122	38.0368820023216\\
69.125	0.1128	37.2983701460334\\
69.125	0.1134	36.5598582897455\\
69.125	0.114	35.8213464334576\\
69.125	0.1146	35.0828345771697\\
69.125	0.1152	34.3443227208816\\
69.125	0.1158	33.6058108645936\\
69.125	0.1164	32.8672990083057\\
69.125	0.117	32.1287871520174\\
69.125	0.1176	31.3902752957297\\
69.125	0.1182	30.6517634394413\\
69.125	0.1188	29.9132515831539\\
69.125	0.1194	29.1747397268655\\
69.125	0.12	28.436227870578\\
69.125	0.1206	27.6977160142897\\
69.125	0.1212	26.9592041580022\\
69.125	0.1218	26.2206923017134\\
69.125	0.1224	25.4821804454259\\
69.125	0.123	24.7436685891375\\
69.5	0.093	62.6867560996625\\
69.5	0.0936	62.0058649596126\\
69.5	0.0942	61.3249738195625\\
69.5	0.0948	60.6440826795122\\
69.5	0.0954	59.9631915394621\\
69.5	0.096	59.282300399412\\
69.5	0.0966	58.6014092593618\\
69.5	0.0972	57.9205181193117\\
69.5	0.0978	57.2396269792614\\
69.5	0.0984	56.5587358392113\\
69.5	0.099	55.8778446991612\\
69.5	0.0996	55.1969535591111\\
69.5	0.1002	54.5160624190607\\
69.5	0.1008	53.8351712790106\\
69.5	0.1014	53.1542801389605\\
69.5	0.102	52.4733889989104\\
69.5	0.1026	51.7924978588603\\
69.5	0.1032	51.11160671881\\
69.5	0.1038	50.4307155787599\\
69.5	0.1044	49.7498244387098\\
69.5	0.105	49.0689332986597\\
69.5	0.1056	48.3880421586093\\
69.5	0.1062	47.7071510185594\\
69.5	0.1068	47.0262598785093\\
69.5	0.1074	46.3453687384592\\
69.5	0.108	45.6644775984091\\
69.5	0.1086	44.983586458359\\
69.5	0.1092	44.3026953183087\\
69.5	0.1098	43.6218041782586\\
69.5	0.1104	42.9409130382085\\
69.5	0.111	42.2600218981584\\
69.5	0.1116	41.5791307581082\\
69.5	0.1122	40.8982396180579\\
69.5	0.1128	40.2173484780078\\
69.5	0.1134	39.5364573379577\\
69.5	0.114	38.8555661979076\\
69.5	0.1146	38.1746750578575\\
69.5	0.1152	37.4937839178074\\
69.5	0.1158	36.812892777757\\
69.5	0.1164	36.1320016377072\\
69.5	0.117	35.4511104976568\\
69.5	0.1176	34.7702193576065\\
69.5	0.1182	34.0893282175566\\
69.5	0.1188	33.4084370775063\\
69.5	0.1194	32.7275459374559\\
69.5	0.12	32.0466547974061\\
69.5	0.1206	31.3657636573557\\
69.5	0.1212	30.6848725173058\\
69.5	0.1218	30.0039813772555\\
69.5	0.1224	29.3230902372052\\
69.5	0.123	28.6421990971553\\
69.875	0.093	63.7042507957885\\
69.875	0.0936	63.0809803719762\\
69.875	0.0942	62.4577099481639\\
69.875	0.0948	61.8344395243516\\
69.875	0.0954	61.2111691005393\\
69.875	0.096	60.587898676727\\
69.875	0.0966	59.9646282529147\\
69.875	0.0972	59.3413578291024\\
69.875	0.0978	58.7180874052899\\
69.875	0.0984	58.0948169814778\\
69.875	0.099	57.4715465576655\\
69.875	0.0996	56.8482761338532\\
69.875	0.1002	56.2250057100407\\
69.875	0.1008	55.6017352862284\\
69.875	0.1014	54.9784648624161\\
69.875	0.102	54.355194438604\\
69.875	0.1026	53.7319240147917\\
69.875	0.1032	53.1086535909792\\
69.875	0.1038	52.4853831671669\\
69.875	0.1044	51.8621127433546\\
69.875	0.105	51.2388423195423\\
69.875	0.1056	50.61557189573\\
69.875	0.1062	49.9923014719177\\
69.875	0.1068	49.3690310481054\\
69.875	0.1074	48.7457606242931\\
69.875	0.108	48.1224902004808\\
69.875	0.1086	47.4992197766685\\
69.875	0.1092	46.8759493528562\\
69.875	0.1098	46.2526789290439\\
69.875	0.1104	45.6294085052316\\
69.875	0.111	45.0061380814193\\
69.875	0.1116	44.382867657607\\
69.875	0.1122	43.7595972337945\\
69.875	0.1128	43.1363268099824\\
69.875	0.1134	42.5130563861701\\
69.875	0.114	41.8897859623578\\
69.875	0.1146	41.2665155385457\\
69.875	0.1152	40.643245114733\\
69.875	0.1158	40.0199746909207\\
69.875	0.1164	39.3967042671084\\
69.875	0.117	38.7734338432961\\
69.875	0.1176	38.1501634194838\\
69.875	0.1182	37.5268929956715\\
69.875	0.1188	36.9036225718592\\
69.875	0.1194	36.2803521480469\\
69.875	0.12	35.6570817242346\\
69.875	0.1206	35.0338113004223\\
69.875	0.1212	34.41054087661\\
69.875	0.1218	33.7872704527977\\
69.875	0.1224	33.1640000289854\\
69.875	0.123	32.5407296051731\\
70.25	0.093	64.7217454919146\\
70.25	0.0936	64.1560957843401\\
70.25	0.0942	63.5904460767658\\
70.25	0.0948	63.0247963691911\\
70.25	0.0954	62.4591466616166\\
70.25	0.096	61.8934969540423\\
70.25	0.0966	61.3278472464679\\
70.25	0.0972	60.7621975388934\\
70.25	0.0978	60.1965478313186\\
70.25	0.0984	59.6308981237444\\
70.25	0.099	59.0652484161699\\
70.25	0.0996	58.4995987085954\\
70.25	0.1002	57.9339490010209\\
70.25	0.1008	57.3682992934464\\
70.25	0.1014	56.8026495858719\\
70.25	0.102	56.2369998782976\\
70.25	0.1026	55.6713501707231\\
70.25	0.1032	55.1057004631484\\
70.25	0.1038	54.5400507555742\\
70.25	0.1044	53.9744010479997\\
70.25	0.105	53.4087513404252\\
70.25	0.1056	52.8431016328504\\
70.25	0.1062	52.2774519252762\\
70.25	0.1068	51.7118022177017\\
70.25	0.1074	51.1461525101272\\
70.25	0.108	50.5805028025529\\
70.25	0.1086	50.0148530949784\\
70.25	0.1092	49.4492033874037\\
70.25	0.1098	48.8835536798292\\
70.25	0.1104	48.317903972255\\
70.25	0.111	47.7522542646805\\
70.25	0.1116	47.186604557106\\
70.25	0.1122	46.6209548495315\\
70.25	0.1128	46.055305141957\\
70.25	0.1134	45.4896554343825\\
70.25	0.114	44.9240057268082\\
70.25	0.1146	44.3583560192335\\
70.25	0.1152	43.7927063116592\\
70.25	0.1158	43.2270566040845\\
70.25	0.1164	42.6614068965105\\
70.25	0.117	42.0957571889353\\
70.25	0.1176	41.5301074813615\\
70.25	0.1182	40.9644577737863\\
70.25	0.1188	40.3988080662125\\
70.25	0.1194	39.8331583586373\\
70.25	0.12	39.2675086510635\\
70.25	0.1206	38.7018589434883\\
70.25	0.1212	38.1362092359145\\
70.25	0.1218	37.5705595283393\\
70.25	0.1224	37.0049098207655\\
70.25	0.123	36.4392601131904\\
70.625	0.093	65.7392401880406\\
70.625	0.0936	65.231211196704\\
70.625	0.0942	64.7231822053675\\
70.625	0.0948	64.2151532140306\\
70.625	0.0954	63.7071242226939\\
70.625	0.096	63.1990952313574\\
70.625	0.0966	62.6910662400207\\
70.625	0.0972	62.1830372486843\\
70.625	0.0978	61.6750082573474\\
70.625	0.0984	61.1669792660107\\
70.625	0.099	60.6589502746742\\
70.625	0.0996	60.1509212833375\\
70.625	0.1002	59.6428922920009\\
70.625	0.1008	59.1348633006642\\
70.625	0.1014	58.6268343093277\\
70.625	0.102	58.118805317991\\
70.625	0.1026	57.6107763266543\\
70.625	0.1032	57.1027473353176\\
70.625	0.1038	56.594718343981\\
70.625	0.1044	56.0866893526445\\
70.625	0.105	55.5786603613078\\
70.625	0.1056	55.0706313699709\\
70.625	0.1062	54.5626023786344\\
70.625	0.1068	54.0545733872978\\
70.625	0.1074	53.5465443959613\\
70.625	0.108	53.0385154046246\\
70.625	0.1086	52.5304864132881\\
70.625	0.1092	52.0224574219512\\
70.625	0.1098	51.5144284306145\\
70.625	0.1104	51.0063994392781\\
70.625	0.111	50.4983704479414\\
70.625	0.1116	49.9903414566049\\
70.625	0.1122	49.482312465268\\
70.625	0.1128	48.9742834739313\\
70.625	0.1134	48.4662544825949\\
70.625	0.114	47.9582254912582\\
70.625	0.1146	47.4501964999215\\
70.625	0.1152	46.9421675085848\\
70.625	0.1158	46.4341385172481\\
70.625	0.1164	45.9261095259117\\
70.625	0.117	45.418080534575\\
70.625	0.1176	44.9100515432378\\
70.625	0.1182	44.4020225519021\\
70.625	0.1188	43.8939935605649\\
70.625	0.1194	43.3859645692287\\
70.625	0.12	42.8779355778915\\
70.625	0.1206	42.3699065865553\\
70.625	0.1212	41.8618775952182\\
70.625	0.1218	41.3538486038819\\
70.625	0.1224	40.8458196125448\\
70.625	0.123	40.3377906212086\\
71	0.093	66.7567348841662\\
71	0.0936	66.3063266090676\\
71	0.0942	65.8559183339687\\
71	0.0948	65.4055100588698\\
71	0.0954	64.9551017837709\\
71	0.096	64.5046935086723\\
71	0.0966	64.0542852335736\\
71	0.0972	63.6038769584748\\
71	0.0978	63.1534686833759\\
71	0.0984	62.703060408277\\
71	0.099	62.2526521331783\\
71	0.0996	61.8022438580795\\
71	0.1002	61.3518355829806\\
71	0.1008	60.9014273078817\\
71	0.1014	60.4510190327831\\
71	0.102	60.0006107576842\\
71	0.1026	59.5502024825855\\
71	0.1032	59.0997942074864\\
71	0.1038	58.6493859323878\\
71	0.1044	58.1989776572889\\
71	0.105	57.7485693821902\\
71	0.1056	57.2981611070913\\
71	0.1062	56.8477528319925\\
71	0.1068	56.3973445568938\\
71	0.1074	55.9469362817949\\
71	0.108	55.4965280066963\\
71	0.1086	55.0461197315974\\
71	0.1092	54.5957114564985\\
71	0.1098	54.1453031813996\\
71	0.1104	53.694894906301\\
71	0.111	53.2444866312021\\
71	0.1116	52.7940783561035\\
71	0.1122	52.3436700810043\\
71	0.1128	51.8932618059057\\
71	0.1134	51.442853530807\\
71	0.114	50.9924452557084\\
71	0.1146	50.5420369806093\\
71	0.1152	50.0916287055106\\
71	0.1158	49.6412204304115\\
71	0.1164	49.1908121553129\\
71	0.117	48.7404038802138\\
71	0.1176	48.2899956051151\\
71	0.1182	47.8395873300165\\
71	0.1188	47.3891790549178\\
71	0.1194	46.9387707798187\\
71	0.12	46.48836250472\\
71	0.1206	46.0379542296214\\
71	0.1212	45.5875459545223\\
71	0.1218	45.1371376794236\\
71	0.1224	44.6867294043245\\
71	0.123	44.2363211292259\\
71.375	0.093	67.7742295802925\\
71.375	0.0936	67.3814420214317\\
71.375	0.0942	66.9886544625708\\
71.375	0.0948	66.5958669037095\\
71.375	0.0954	66.2030793448487\\
71.375	0.096	65.8102917859878\\
71.375	0.0966	65.4175042271268\\
71.375	0.0972	65.0247166682659\\
71.375	0.0978	64.6319291094048\\
71.375	0.0984	64.2391415505438\\
71.375	0.099	63.8463539916829\\
71.375	0.0996	63.4535664328221\\
71.375	0.1002	63.0607788739608\\
71.375	0.1008	62.6679913150999\\
71.375	0.1014	62.2752037562391\\
71.375	0.102	61.882416197378\\
71.375	0.1026	61.4896286385172\\
71.375	0.1032	61.0968410796561\\
71.375	0.1038	60.704053520795\\
71.375	0.1044	60.3112659619342\\
71.375	0.105	59.9184784030733\\
71.375	0.1056	59.525690844212\\
71.375	0.1062	59.1329032853512\\
71.375	0.1068	58.7401157264903\\
71.375	0.1074	58.3473281676293\\
71.375	0.108	57.9545406087684\\
71.375	0.1086	57.5617530499076\\
71.375	0.1092	57.1689654910463\\
71.375	0.1098	56.7761779321854\\
71.375	0.1104	56.3833903733246\\
71.375	0.111	55.9906028144635\\
71.375	0.1116	55.5978152556027\\
71.375	0.1122	55.2050276967416\\
71.375	0.1128	54.8122401378805\\
71.375	0.1134	54.4194525790199\\
71.375	0.114	54.0266650201588\\
71.375	0.1146	53.6338774612977\\
71.375	0.1152	53.2410899024367\\
71.375	0.1158	52.8483023435756\\
71.375	0.1164	52.455514784715\\
71.375	0.117	52.0627272258539\\
71.375	0.1176	51.6699396669928\\
71.375	0.1182	51.2771521081318\\
71.375	0.1188	50.8843645492711\\
71.375	0.1194	50.4915769904096\\
71.375	0.12	50.098789431549\\
71.375	0.1206	49.7060018726879\\
71.375	0.1212	49.3132143138268\\
71.375	0.1218	48.9204267549658\\
71.375	0.1224	48.5276391961052\\
71.375	0.123	48.1348516372441\\
71.75	0.093	68.7917242764183\\
71.75	0.0936	68.4565574337953\\
71.75	0.0942	68.1213905911723\\
71.75	0.0948	67.7862237485488\\
71.75	0.0954	67.4510569059257\\
71.75	0.096	67.1158900633027\\
71.75	0.0966	66.7807232206796\\
71.75	0.0972	66.4455563780564\\
71.75	0.0978	66.1103895354331\\
71.75	0.0984	65.7752226928101\\
71.75	0.099	65.440055850187\\
71.75	0.0996	65.104889007564\\
71.75	0.1002	64.7697221649405\\
71.75	0.1008	64.4345553223175\\
71.75	0.1014	64.0993884796944\\
71.75	0.102	63.7642216370714\\
71.75	0.1026	63.4290547944483\\
71.75	0.1032	63.0938879518249\\
71.75	0.1038	62.7587211092018\\
71.75	0.1044	62.4235542665788\\
71.75	0.105	62.0883874239557\\
71.75	0.1056	61.7532205813325\\
71.75	0.1062	61.4180537387092\\
71.75	0.1068	61.0828868960862\\
71.75	0.1074	60.7477200534631\\
71.75	0.108	60.4125532108401\\
71.75	0.1086	60.077386368217\\
71.75	0.1092	59.7422195255936\\
71.75	0.1098	59.4070526829705\\
71.75	0.1104	59.0718858403475\\
71.75	0.111	58.7367189977244\\
71.75	0.1116	58.4015521551012\\
71.75	0.1122	58.0663853124781\\
71.75	0.1128	57.7312184698546\\
71.75	0.1134	57.3960516272316\\
71.75	0.114	57.0608847846086\\
71.75	0.1146	56.7257179419857\\
71.75	0.1152	56.3905510993623\\
71.75	0.1158	56.0553842567392\\
71.75	0.1164	55.7202174141157\\
71.75	0.117	55.3850505714931\\
71.75	0.1176	55.0498837288696\\
71.75	0.1182	54.7147168862471\\
71.75	0.1188	54.3795500436231\\
71.75	0.1194	54.0443832010005\\
71.75	0.12	53.709216358377\\
71.75	0.1206	53.3740495157545\\
71.75	0.1212	53.038882673131\\
71.75	0.1218	52.7037158305079\\
71.75	0.1224	52.3685489878844\\
71.75	0.123	52.0333821452618\\
72.125	0.093	69.8092189725444\\
72.125	0.0936	69.5316728461592\\
72.125	0.0942	69.2541267197737\\
72.125	0.0948	68.9765805933882\\
72.125	0.0954	68.699034467003\\
72.125	0.096	68.4214883406178\\
72.125	0.0966	68.1439422142325\\
72.125	0.0972	67.8663960878473\\
72.125	0.0978	67.5888499614618\\
72.125	0.0984	67.3113038350766\\
72.125	0.099	67.0337577086914\\
72.125	0.0996	66.7562115823062\\
72.125	0.1002	66.4786654559205\\
72.125	0.1008	66.2011193295352\\
72.125	0.1014	65.92357320315\\
72.125	0.102	65.6460270767648\\
72.125	0.1026	65.3684809503795\\
72.125	0.1032	65.0909348239941\\
72.125	0.1038	64.8133886976088\\
72.125	0.1044	64.5358425712236\\
72.125	0.105	64.2582964448384\\
72.125	0.1056	63.9807503184529\\
72.125	0.1062	63.7032041920677\\
72.125	0.1068	63.4256580656825\\
72.125	0.1074	63.148111939297\\
72.125	0.108	62.8705658129118\\
72.125	0.1086	62.5930196865265\\
72.125	0.1092	62.3154735601411\\
72.125	0.1098	62.0379274337558\\
72.125	0.1104	61.7603813073706\\
72.125	0.111	61.4828351809854\\
72.125	0.1116	61.2052890545999\\
72.125	0.1122	60.9277429282145\\
72.125	0.1128	60.6501968018295\\
72.125	0.1134	60.372650675444\\
72.125	0.114	60.095104549059\\
72.125	0.1146	59.8175584226733\\
72.125	0.1152	59.5400122962883\\
72.125	0.1158	59.2624661699028\\
72.125	0.1164	58.9849200435178\\
72.125	0.117	58.7073739171319\\
72.125	0.1176	58.4298277907469\\
72.125	0.1182	58.1522816643615\\
72.125	0.1188	57.8747355379764\\
72.125	0.1194	57.5971894115905\\
72.125	0.12	57.319643285206\\
72.125	0.1206	57.0420971588201\\
72.125	0.1212	56.7645510324355\\
72.125	0.1218	56.4870049060496\\
72.125	0.1224	56.2094587796646\\
72.125	0.123	55.9319126532787\\
72.5	0.093	70.8267136686704\\
72.5	0.0936	70.606788258523\\
72.5	0.0942	70.3868628483756\\
72.5	0.0948	70.1669374382279\\
72.5	0.0954	69.9470120280805\\
72.5	0.096	69.7270866179331\\
72.5	0.0966	69.5071612077857\\
72.5	0.0972	69.2872357976382\\
72.5	0.0978	69.0673103874906\\
72.5	0.0984	68.8473849773432\\
72.5	0.099	68.6274595671957\\
72.5	0.0996	68.4075341570483\\
72.5	0.1002	68.1876087469007\\
72.5	0.1008	67.9676833367532\\
72.5	0.1014	67.7477579266058\\
72.5	0.102	67.5278325164584\\
72.5	0.1026	67.3079071063109\\
72.5	0.1032	67.0879816961633\\
72.5	0.1038	66.8680562860159\\
72.5	0.1044	66.6481308758684\\
72.5	0.105	66.428205465721\\
72.5	0.1056	66.2082800555734\\
72.5	0.1062	65.9883546454262\\
72.5	0.1068	65.7684292352787\\
72.5	0.1074	65.5485038251313\\
72.5	0.108	65.3285784149839\\
72.5	0.1086	65.1086530048365\\
72.5	0.1092	64.8887275946888\\
72.5	0.1098	64.6688021845414\\
72.5	0.1104	64.4488767743937\\
72.5	0.111	64.2289513642463\\
72.5	0.1116	64.0090259540989\\
72.5	0.1122	63.7891005439515\\
72.5	0.1128	63.569175133804\\
72.5	0.1134	63.3492497236566\\
72.5	0.114	63.1293243135092\\
72.5	0.1146	62.9093989033618\\
72.5	0.1152	62.6894734932143\\
72.5	0.1158	62.4695480830669\\
72.5	0.1164	62.2496226729195\\
72.5	0.117	62.0296972627716\\
72.5	0.1176	61.8097718526242\\
72.5	0.1182	61.5898464424768\\
72.5	0.1188	61.3699210323293\\
72.5	0.1194	61.1499956221819\\
72.5	0.12	60.9300702120345\\
72.5	0.1206	60.7101448018871\\
72.5	0.1212	60.4902193917396\\
72.5	0.1218	60.2702939815917\\
72.5	0.1224	60.0503685714443\\
72.5	0.123	59.8304431612969\\
72.875	0.093	71.8442083647963\\
72.875	0.0936	71.6819036708866\\
72.875	0.0942	71.5195989769772\\
72.875	0.0948	71.3572942830674\\
72.875	0.0954	71.1949895891578\\
72.875	0.096	71.0326848952482\\
72.875	0.0966	70.8703802013385\\
72.875	0.0972	70.7080755074289\\
72.875	0.0978	70.5457708135193\\
72.875	0.0984	70.3834661196097\\
72.875	0.099	70.2211614257001\\
72.875	0.0996	70.0588567317905\\
72.875	0.1002	69.8965520378806\\
72.875	0.1008	69.734247343971\\
72.875	0.1014	69.5719426500616\\
72.875	0.102	69.409637956152\\
72.875	0.1026	69.2473332622424\\
72.875	0.1032	69.0850285683325\\
72.875	0.1038	68.9227238744229\\
72.875	0.1044	68.7604191805133\\
72.875	0.105	68.5981144866037\\
72.875	0.1056	68.435809792694\\
72.875	0.1062	68.2735050987844\\
72.875	0.1068	68.1112004048748\\
72.875	0.1074	67.9488957109652\\
72.875	0.108	67.7865910170556\\
72.875	0.1086	67.624286323146\\
72.875	0.1092	67.4619816292361\\
72.875	0.1098	67.2996769353267\\
72.875	0.1104	67.1373722414169\\
72.875	0.111	66.9750675475075\\
72.875	0.1116	66.8127628535981\\
72.875	0.1122	66.6504581596882\\
72.875	0.1128	66.4881534657784\\
72.875	0.1134	66.325848771869\\
72.875	0.114	66.1635440779592\\
72.875	0.1146	66.0012393840498\\
72.875	0.1152	65.8389346901399\\
72.875	0.1158	65.6766299962305\\
72.875	0.1164	65.5143253023207\\
72.875	0.117	65.3520206084108\\
72.875	0.1176	65.1897159145014\\
72.875	0.1182	65.0274112205916\\
72.875	0.1188	64.8651065266822\\
72.875	0.1194	64.7028018327724\\
72.875	0.12	64.540497138863\\
72.875	0.1206	64.3781924449531\\
72.875	0.1212	64.2158877510437\\
72.875	0.1218	64.0535830571339\\
72.875	0.1224	63.891278363224\\
72.875	0.123	63.7289736693147\\
73.25	0.093	72.8617030609221\\
73.25	0.0936	72.7570190832503\\
73.25	0.0942	72.6523351055787\\
73.25	0.0948	72.5476511279066\\
73.25	0.0954	72.4429671502348\\
73.25	0.096	72.338283172563\\
73.25	0.0966	72.2335991948914\\
73.25	0.0972	72.1289152172196\\
73.25	0.0978	72.0242312395476\\
73.25	0.0984	71.9195472618758\\
73.25	0.099	71.8148632842042\\
73.25	0.0996	71.7101793065324\\
73.25	0.1002	71.6054953288603\\
73.25	0.1008	71.5008113511888\\
73.25	0.1014	71.3961273735169\\
73.25	0.102	71.2914433958451\\
73.25	0.1026	71.1867594181733\\
73.25	0.1032	71.0820754405015\\
73.25	0.1038	70.9773914628297\\
73.25	0.1044	70.8727074851579\\
73.25	0.105	70.7680235074861\\
73.25	0.1056	70.6633395298143\\
73.25	0.1062	70.5586555521425\\
73.25	0.1068	70.4539715744706\\
73.25	0.1074	70.3492875967988\\
73.25	0.108	70.2446036191272\\
73.25	0.1086	70.1399196414554\\
73.25	0.1092	70.0352356637834\\
73.25	0.1098	69.9305516861116\\
73.25	0.1104	69.8258677084398\\
73.25	0.111	69.7211837307684\\
73.25	0.1116	69.6164997530966\\
73.25	0.1122	69.5118157754243\\
73.25	0.1128	69.407131797753\\
73.25	0.1134	69.3024478200809\\
73.25	0.114	69.1977638424096\\
73.25	0.1146	69.0930798647373\\
73.25	0.1152	68.988395887066\\
73.25	0.1158	68.8837119093937\\
73.25	0.1164	68.7790279317223\\
73.25	0.117	68.6743439540496\\
73.25	0.1176	68.5696599763787\\
73.25	0.1182	68.4649759987064\\
73.25	0.1188	68.3602920210351\\
73.25	0.1194	68.2556080433628\\
73.25	0.12	68.1509240656915\\
73.25	0.1206	68.0462400880192\\
73.25	0.1212	67.9415561103478\\
73.25	0.1218	67.8368721326756\\
73.25	0.1224	67.7321881550042\\
73.25	0.123	67.627504177332\\
73.625	0.093	73.8791977570484\\
73.625	0.0936	73.8321344956144\\
73.625	0.0942	73.7850712341806\\
73.625	0.0948	73.7380079727463\\
73.625	0.0954	73.6909447113126\\
73.625	0.096	73.6438814498786\\
73.625	0.0966	73.5968181884446\\
73.625	0.0972	73.5497549270108\\
73.625	0.0978	73.5026916655765\\
73.625	0.0984	73.4556284041428\\
73.625	0.099	73.4085651427088\\
73.625	0.0996	73.3615018812748\\
73.625	0.1002	73.3144386198408\\
73.625	0.1008	73.2673753584068\\
73.625	0.1014	73.220312096973\\
73.625	0.102	73.173248835539\\
73.625	0.1026	73.126185574105\\
73.625	0.1032	73.079122312671\\
73.625	0.1038	73.032059051237\\
73.625	0.1044	72.9849957898032\\
73.625	0.105	72.9379325283692\\
73.625	0.1056	72.8908692669352\\
73.625	0.1062	72.8438060055012\\
73.625	0.1068	72.7967427440672\\
73.625	0.1074	72.7496794826334\\
73.625	0.108	72.7026162211994\\
73.625	0.1086	72.6555529597656\\
73.625	0.1092	72.6084896983314\\
73.625	0.1098	72.5614264368976\\
73.625	0.1104	72.5143631754634\\
73.625	0.111	72.4672999140296\\
73.625	0.1116	72.4202366525958\\
73.625	0.1122	72.3731733911616\\
73.625	0.1128	72.3261101297276\\
73.625	0.1134	72.2790468682938\\
73.625	0.114	72.2319836068596\\
73.625	0.1146	72.1849203454262\\
73.625	0.1152	72.1378570839915\\
73.625	0.1158	72.0907938225582\\
73.625	0.1164	72.0437305611235\\
73.625	0.117	71.9966672996902\\
73.625	0.1176	71.9496040382555\\
73.625	0.1182	71.9025407768222\\
73.625	0.1188	71.855477515388\\
73.625	0.1194	71.8084142539542\\
73.625	0.12	71.76135099252\\
73.625	0.1206	71.7142877310862\\
73.625	0.1212	71.667224469652\\
73.625	0.1218	71.6201612082182\\
73.625	0.1224	71.5730979467839\\
73.625	0.123	71.5260346853506\\
74	0.093	74.8966924531742\\
74	0.0936	74.907249907978\\
74	0.0942	74.917807362782\\
74	0.0948	74.9283648175856\\
74	0.0954	74.9389222723896\\
74	0.096	74.9494797271934\\
74	0.0966	74.9600371819974\\
74	0.0972	74.9705946368013\\
74	0.0978	74.9811520916051\\
74	0.0984	74.9917095464089\\
74	0.099	75.0022670012129\\
74	0.0996	75.0128244560167\\
74	0.1002	75.0233819108205\\
74	0.1008	75.0339393656243\\
74	0.1014	75.0444968204283\\
74	0.102	75.0550542752321\\
74	0.1026	75.0656117300362\\
74	0.1032	75.07616918484\\
74	0.1038	75.0867266396438\\
74	0.1044	75.0972840944478\\
74	0.105	75.1078415492516\\
74	0.1056	75.1183990040554\\
74	0.1062	75.1289564588592\\
74	0.1068	75.1395139136632\\
74	0.1074	75.150071368467\\
74	0.108	75.1606288232711\\
74	0.1086	75.1711862780749\\
74	0.1092	75.1817437328787\\
74	0.1098	75.1923011876825\\
74	0.1104	75.2028586424863\\
74	0.111	75.2134160972905\\
74	0.1116	75.2239735520943\\
74	0.1122	75.2345310068981\\
74	0.1128	75.2450884617019\\
74	0.1134	75.2556459165057\\
74	0.114	75.26620337131\\
74	0.1146	75.2767608261138\\
74	0.1152	75.2873182809176\\
74	0.1158	75.2978757357214\\
74	0.1164	75.3084331905252\\
74	0.117	75.318990645329\\
74	0.1176	75.3295481001328\\
74	0.1182	75.3401055549366\\
74	0.1188	75.3506630097409\\
74	0.1194	75.3612204645442\\
74	0.12	75.3717779193485\\
74	0.1206	75.3823353741523\\
74	0.1212	75.3928928289561\\
74	0.1218	75.4034502837599\\
74	0.1224	75.4140077385637\\
74	0.123	75.4245651933675\\
};
\end{axis}

\begin{axis}[%
width=4.927496cm,
height=3.870968cm,
at={(0cm,16.129032cm)},
scale only axis,
xmin=56,
xmax=74,
tick align=outside,
xlabel={$L_{cut}$},
xmajorgrids,
ymin=0.093,
ymax=0.123,
ylabel={$D_{rlx}$},
ymajorgrids,
zmin=-101.122897322466,
zmax=7.88686191364252,
zlabel={$x_4,x_4$},
zmajorgrids,
view={-140}{50},
legend style={at={(1.03,1)},anchor=north west,legend cell align=left,align=left,draw=white!15!black}
]
\addplot3[only marks,mark=*,mark options={},mark size=1.5000pt,color=mycolor1] plot table[row sep=crcr,]{%
74	0.123	-26.0353957891804\\
72	0.113	-20.9879322169279\\
61	0.095	-10.6920630070547\\
56	0.093	-10.9569379219045\\
};
\addplot3[only marks,mark=*,mark options={},mark size=1.5000pt,color=black] plot table[row sep=crcr,]{%
69	0.104	-15.5142132599949\\
};

\addplot3[%
surf,
opacity=0.7,
shader=interp,
colormap={mymap}{[1pt] rgb(0pt)=(0.0901961,0.239216,0.0745098); rgb(1pt)=(0.0945149,0.242058,0.0739522); rgb(2pt)=(0.0988592,0.244894,0.0733566); rgb(3pt)=(0.103229,0.247724,0.0727241); rgb(4pt)=(0.107623,0.250549,0.0720557); rgb(5pt)=(0.112043,0.253367,0.0713525); rgb(6pt)=(0.116487,0.25618,0.0706154); rgb(7pt)=(0.120956,0.258986,0.0698456); rgb(8pt)=(0.125449,0.261787,0.0690441); rgb(9pt)=(0.129967,0.264581,0.0682118); rgb(10pt)=(0.134508,0.26737,0.06735); rgb(11pt)=(0.139074,0.270152,0.0664596); rgb(12pt)=(0.143663,0.272929,0.0655416); rgb(13pt)=(0.148275,0.275699,0.0645971); rgb(14pt)=(0.152911,0.278463,0.0636271); rgb(15pt)=(0.15757,0.281221,0.0626328); rgb(16pt)=(0.162252,0.283973,0.0616151); rgb(17pt)=(0.166957,0.286719,0.060575); rgb(18pt)=(0.171685,0.289458,0.0595136); rgb(19pt)=(0.176434,0.292191,0.0584321); rgb(20pt)=(0.181207,0.294918,0.0573313); rgb(21pt)=(0.186001,0.297639,0.0562123); rgb(22pt)=(0.190817,0.300353,0.0550763); rgb(23pt)=(0.195655,0.303061,0.0539242); rgb(24pt)=(0.200514,0.305763,0.052757); rgb(25pt)=(0.205395,0.308459,0.0515759); rgb(26pt)=(0.210296,0.311149,0.0503624); rgb(27pt)=(0.215212,0.313846,0.0490067); rgb(28pt)=(0.220142,0.316548,0.0475043); rgb(29pt)=(0.22509,0.319254,0.0458704); rgb(30pt)=(0.230056,0.321962,0.0441205); rgb(31pt)=(0.235042,0.324671,0.04227); rgb(32pt)=(0.240048,0.327379,0.0403343); rgb(33pt)=(0.245078,0.330085,0.0383287); rgb(34pt)=(0.250131,0.332786,0.0362688); rgb(35pt)=(0.25521,0.335482,0.0341698); rgb(36pt)=(0.260317,0.33817,0.0320472); rgb(37pt)=(0.265451,0.340849,0.0299163); rgb(38pt)=(0.270616,0.343517,0.0277927); rgb(39pt)=(0.275813,0.346172,0.0256916); rgb(40pt)=(0.281043,0.348814,0.0236284); rgb(41pt)=(0.286307,0.35144,0.0216186); rgb(42pt)=(0.291607,0.354048,0.0196776); rgb(43pt)=(0.296945,0.356637,0.0178207); rgb(44pt)=(0.302322,0.359206,0.0160634); rgb(45pt)=(0.307739,0.361753,0.0144211); rgb(46pt)=(0.313198,0.364275,0.0129091); rgb(47pt)=(0.318701,0.366772,0.0115428); rgb(48pt)=(0.324249,0.369242,0.0103377); rgb(49pt)=(0.329843,0.371682,0.00930909); rgb(50pt)=(0.335485,0.374093,0.00847245); rgb(51pt)=(0.341176,0.376471,0.00784314); rgb(52pt)=(0.346925,0.378826,0.00732741); rgb(53pt)=(0.352735,0.381168,0.00682184); rgb(54pt)=(0.358605,0.383497,0.00632729); rgb(55pt)=(0.364532,0.385812,0.00584464); rgb(56pt)=(0.370516,0.388113,0.00537476); rgb(57pt)=(0.376552,0.390399,0.00491852); rgb(58pt)=(0.38264,0.39267,0.00447681); rgb(59pt)=(0.388777,0.394925,0.00405048); rgb(60pt)=(0.394962,0.397164,0.00364042); rgb(61pt)=(0.401191,0.399386,0.00324749); rgb(62pt)=(0.407464,0.401592,0.00287258); rgb(63pt)=(0.413777,0.40378,0.00251655); rgb(64pt)=(0.420129,0.40595,0.00218028); rgb(65pt)=(0.426518,0.408102,0.00186463); rgb(66pt)=(0.432942,0.410234,0.00157049); rgb(67pt)=(0.439399,0.412348,0.00129873); rgb(68pt)=(0.445885,0.414441,0.00105022); rgb(69pt)=(0.452401,0.416515,0.000825833); rgb(70pt)=(0.458942,0.418567,0.000626441); rgb(71pt)=(0.465508,0.420599,0.00045292); rgb(72pt)=(0.472096,0.422609,0.000306141); rgb(73pt)=(0.478704,0.424596,0.000186979); rgb(74pt)=(0.485331,0.426562,9.63073e-05); rgb(75pt)=(0.491973,0.428504,3.49981e-05); rgb(76pt)=(0.498628,0.430422,3.92506e-06); rgb(77pt)=(0.505323,0.432315,0); rgb(78pt)=(0.512206,0.434168,0); rgb(79pt)=(0.519282,0.435983,0); rgb(80pt)=(0.526529,0.437764,0); rgb(81pt)=(0.533922,0.439512,0); rgb(82pt)=(0.54144,0.441232,0); rgb(83pt)=(0.549059,0.442927,0); rgb(84pt)=(0.556756,0.444599,0); rgb(85pt)=(0.564508,0.446252,0); rgb(86pt)=(0.572292,0.447889,0); rgb(87pt)=(0.580084,0.449514,0); rgb(88pt)=(0.587863,0.451129,0); rgb(89pt)=(0.595604,0.452737,0); rgb(90pt)=(0.603284,0.454343,0); rgb(91pt)=(0.610882,0.455948,0); rgb(92pt)=(0.618373,0.457556,0); rgb(93pt)=(0.625734,0.459171,0); rgb(94pt)=(0.632943,0.460795,0); rgb(95pt)=(0.639976,0.462432,0); rgb(96pt)=(0.64681,0.464084,0); rgb(97pt)=(0.653423,0.465756,0); rgb(98pt)=(0.659791,0.46745,0); rgb(99pt)=(0.665891,0.469169,0); rgb(100pt)=(0.6717,0.470916,0); rgb(101pt)=(0.677195,0.472696,0); rgb(102pt)=(0.682353,0.47451,0); rgb(103pt)=(0.687242,0.476355,0); rgb(104pt)=(0.691952,0.478225,0); rgb(105pt)=(0.696497,0.480118,0); rgb(106pt)=(0.700887,0.482033,0); rgb(107pt)=(0.705134,0.483968,0); rgb(108pt)=(0.709251,0.485921,0); rgb(109pt)=(0.713249,0.487891,0); rgb(110pt)=(0.71714,0.489876,0); rgb(111pt)=(0.720936,0.491875,0); rgb(112pt)=(0.724649,0.493887,0); rgb(113pt)=(0.72829,0.495909,0); rgb(114pt)=(0.731872,0.49794,0); rgb(115pt)=(0.735406,0.499979,0); rgb(116pt)=(0.738904,0.502025,0); rgb(117pt)=(0.742378,0.504075,0); rgb(118pt)=(0.74584,0.506128,0); rgb(119pt)=(0.749302,0.508182,0); rgb(120pt)=(0.752775,0.510237,0); rgb(121pt)=(0.756272,0.51229,0); rgb(122pt)=(0.759804,0.514339,0); rgb(123pt)=(0.763384,0.516385,0); rgb(124pt)=(0.767022,0.518424,0); rgb(125pt)=(0.770731,0.520455,0); rgb(126pt)=(0.774523,0.522478,0); rgb(127pt)=(0.77841,0.524489,0); rgb(128pt)=(0.782391,0.526491,0); rgb(129pt)=(0.786402,0.528496,0); rgb(130pt)=(0.790431,0.530506,0); rgb(131pt)=(0.794478,0.532521,0); rgb(132pt)=(0.798541,0.534539,0); rgb(133pt)=(0.802619,0.53656,0); rgb(134pt)=(0.806712,0.538584,0); rgb(135pt)=(0.81082,0.540609,0); rgb(136pt)=(0.81494,0.542635,0); rgb(137pt)=(0.819074,0.54466,0); rgb(138pt)=(0.823219,0.546686,0); rgb(139pt)=(0.827374,0.548709,0); rgb(140pt)=(0.831541,0.55073,0); rgb(141pt)=(0.835716,0.552749,0); rgb(142pt)=(0.8399,0.554763,0); rgb(143pt)=(0.844092,0.556774,0); rgb(144pt)=(0.848292,0.558779,0); rgb(145pt)=(0.852497,0.560778,0); rgb(146pt)=(0.856708,0.562771,0); rgb(147pt)=(0.860924,0.564756,0); rgb(148pt)=(0.865143,0.566733,0); rgb(149pt)=(0.869366,0.568701,0); rgb(150pt)=(0.873592,0.57066,0); rgb(151pt)=(0.877819,0.572608,0); rgb(152pt)=(0.882047,0.574545,0); rgb(153pt)=(0.886275,0.576471,0); rgb(154pt)=(0.890659,0.578362,0); rgb(155pt)=(0.895333,0.580203,0); rgb(156pt)=(0.900258,0.581999,0); rgb(157pt)=(0.905397,0.583755,0); rgb(158pt)=(0.910711,0.585479,0); rgb(159pt)=(0.916164,0.587176,0); rgb(160pt)=(0.921717,0.588852,0); rgb(161pt)=(0.927333,0.590513,0); rgb(162pt)=(0.932974,0.592166,0); rgb(163pt)=(0.938602,0.593815,0); rgb(164pt)=(0.94418,0.595468,0); rgb(165pt)=(0.949669,0.59713,0); rgb(166pt)=(0.955033,0.598808,0); rgb(167pt)=(0.960233,0.600507,0); rgb(168pt)=(0.965232,0.602233,0); rgb(169pt)=(0.969992,0.603992,0); rgb(170pt)=(0.974475,0.605791,0); rgb(171pt)=(0.978643,0.607636,0); rgb(172pt)=(0.98246,0.609532,0); rgb(173pt)=(0.985886,0.611486,0); rgb(174pt)=(0.988885,0.613503,0); rgb(175pt)=(0.991419,0.61559,0); rgb(176pt)=(0.99345,0.617753,0); rgb(177pt)=(0.99494,0.619997,0); rgb(178pt)=(0.995851,0.622329,0); rgb(179pt)=(0.996226,0.624763,0); rgb(180pt)=(0.996512,0.627352,0); rgb(181pt)=(0.996788,0.630095,0); rgb(182pt)=(0.997053,0.632982,0); rgb(183pt)=(0.997308,0.636004,0); rgb(184pt)=(0.997552,0.639152,0); rgb(185pt)=(0.997785,0.642416,0); rgb(186pt)=(0.998006,0.645786,0); rgb(187pt)=(0.998217,0.649253,0); rgb(188pt)=(0.998416,0.652807,0); rgb(189pt)=(0.998605,0.656439,0); rgb(190pt)=(0.998781,0.660138,0); rgb(191pt)=(0.998946,0.663897,0); rgb(192pt)=(0.9991,0.667704,0); rgb(193pt)=(0.999242,0.67155,0); rgb(194pt)=(0.999372,0.675427,0); rgb(195pt)=(0.99949,0.679323,0); rgb(196pt)=(0.999596,0.68323,0); rgb(197pt)=(0.99969,0.687139,0); rgb(198pt)=(0.999771,0.691039,0); rgb(199pt)=(0.999841,0.694921,0); rgb(200pt)=(0.999898,0.698775,0); rgb(201pt)=(0.999942,0.702592,0); rgb(202pt)=(0.999974,0.706363,0); rgb(203pt)=(0.999994,0.710077,0); rgb(204pt)=(1,0.713725,0); rgb(205pt)=(1,0.717341,0); rgb(206pt)=(1,0.720963,0); rgb(207pt)=(1,0.724591,0); rgb(208pt)=(1,0.728226,0); rgb(209pt)=(1,0.731867,0); rgb(210pt)=(1,0.735514,0); rgb(211pt)=(1,0.739167,0); rgb(212pt)=(1,0.742827,0); rgb(213pt)=(1,0.746493,0); rgb(214pt)=(1,0.750165,0); rgb(215pt)=(1,0.753843,0); rgb(216pt)=(1,0.757527,0); rgb(217pt)=(1,0.761217,0); rgb(218pt)=(1,0.764913,0); rgb(219pt)=(1,0.768615,0); rgb(220pt)=(1,0.772324,0); rgb(221pt)=(1,0.776038,0); rgb(222pt)=(1,0.779758,0); rgb(223pt)=(1,0.783484,0); rgb(224pt)=(1,0.787215,0); rgb(225pt)=(1,0.790953,0); rgb(226pt)=(1,0.794696,0); rgb(227pt)=(1,0.798445,0); rgb(228pt)=(1,0.8022,0); rgb(229pt)=(1,0.805961,0); rgb(230pt)=(1,0.809727,0); rgb(231pt)=(1,0.8135,0); rgb(232pt)=(1,0.817278,0); rgb(233pt)=(1,0.821063,0); rgb(234pt)=(1,0.824854,0); rgb(235pt)=(1,0.828652,0); rgb(236pt)=(1,0.832455,0); rgb(237pt)=(1,0.836265,0); rgb(238pt)=(1,0.840081,0); rgb(239pt)=(1,0.843903,0); rgb(240pt)=(1,0.847732,0); rgb(241pt)=(1,0.851566,0); rgb(242pt)=(1,0.855406,0); rgb(243pt)=(1,0.859253,0); rgb(244pt)=(1,0.863106,0); rgb(245pt)=(1,0.866964,0); rgb(246pt)=(1,0.870829,0); rgb(247pt)=(1,0.8747,0); rgb(248pt)=(1,0.878577,0); rgb(249pt)=(1,0.88246,0); rgb(250pt)=(1,0.886349,0); rgb(251pt)=(1,0.890243,0); rgb(252pt)=(1,0.894144,0); rgb(253pt)=(1,0.898051,0); rgb(254pt)=(1,0.901964,0); rgb(255pt)=(1,0.905882,0)},
mesh/rows=49]
table[row sep=crcr,header=false] {%
%
56	0.093	-10.9569379221853\\
56	0.0936	-12.7602571101908\\
56	0.0942	-14.5635762981964\\
56	0.0948	-16.366895486202\\
56	0.0954	-18.1702146742076\\
56	0.096	-19.9735338622133\\
56	0.0966	-21.7768530502188\\
56	0.0972	-23.5801722382246\\
56	0.0978	-25.3834914262301\\
56	0.0984	-27.1868106142358\\
56	0.099	-28.9901298022414\\
56	0.0996	-30.793448990247\\
56	0.1002	-32.5967681782527\\
56	0.1008	-34.4000873662582\\
56	0.1014	-36.2034065542639\\
56	0.102	-38.0067257422694\\
56	0.1026	-39.8100449302751\\
56	0.1032	-41.6133641182807\\
56	0.1038	-43.4166833062862\\
56	0.1044	-45.220002494292\\
56	0.105	-47.0233216822975\\
56	0.1056	-48.8266408703032\\
56	0.1062	-50.6299600583089\\
56	0.1068	-52.4332792463146\\
56	0.1074	-54.2365984343202\\
56	0.108	-56.0399176223256\\
56	0.1086	-57.8432368103313\\
56	0.1092	-59.6465559983368\\
56	0.1098	-61.4498751863425\\
56	0.1104	-63.2531943743483\\
56	0.111	-65.0565135623536\\
56	0.1116	-66.8598327503594\\
56	0.1122	-68.6631519383649\\
56	0.1128	-70.4664711263706\\
56	0.1134	-72.2697903143763\\
56	0.114	-74.0731095023818\\
56	0.1146	-75.8764286903875\\
56	0.1152	-77.6797478783931\\
56	0.1158	-79.4830670663987\\
56	0.1164	-81.2863862544044\\
56	0.117	-83.0897054424099\\
56	0.1176	-84.8930246304157\\
56	0.1182	-86.6963438184214\\
56	0.1188	-88.4996630064268\\
56	0.1194	-90.3029821944326\\
56	0.12	-92.106301382438\\
56	0.1206	-93.9096205704437\\
56	0.1212	-95.7129397584495\\
56	0.1218	-97.516258946455\\
56	0.1224	-99.3195781344607\\
56	0.123	-101.122897322466\\
56.375	0.093	-10.5643587589389\\
56.375	0.0936	-12.3442430712366\\
56.375	0.0942	-14.1241273835348\\
56.375	0.0948	-15.9040116958327\\
56.375	0.0954	-17.6838960081307\\
56.375	0.096	-19.4637803204287\\
56.375	0.0966	-21.2436646327266\\
56.375	0.0972	-23.0235489450247\\
56.375	0.0978	-24.8034332573225\\
56.375	0.0984	-26.5833175696205\\
56.375	0.099	-28.3632018819186\\
56.375	0.0996	-30.1430861942165\\
56.375	0.1002	-31.9229705065145\\
56.375	0.1008	-33.7028548188123\\
56.375	0.1014	-35.4827391311104\\
56.375	0.102	-37.2626234434083\\
56.375	0.1026	-39.0425077557063\\
56.375	0.1032	-40.8223920680043\\
56.375	0.1038	-42.6022763803022\\
56.375	0.1044	-44.3821606926003\\
56.375	0.105	-46.1620450048981\\
56.375	0.1056	-47.9419293171961\\
56.375	0.1062	-49.7218136294942\\
56.375	0.1068	-51.5016979417923\\
56.375	0.1074	-53.2815822540902\\
56.375	0.108	-55.0614665663879\\
56.375	0.1086	-56.8413508786859\\
56.375	0.1092	-58.621235190984\\
56.375	0.1098	-60.401119503282\\
56.375	0.1104	-62.18100381558\\
56.375	0.111	-63.9608881278778\\
56.375	0.1116	-65.7407724401759\\
56.375	0.1122	-67.5206567524738\\
56.375	0.1128	-69.3005410647718\\
56.375	0.1134	-71.0804253770698\\
56.375	0.114	-72.8603096893677\\
56.375	0.1146	-74.6401940016657\\
56.375	0.1152	-76.4200783139636\\
56.375	0.1158	-78.1999626262616\\
56.375	0.1164	-79.9798469385597\\
56.375	0.117	-81.7597312508576\\
56.375	0.1176	-83.5396155631556\\
56.375	0.1182	-85.3194998754536\\
56.375	0.1188	-87.0993841877515\\
56.375	0.1194	-88.8792685000496\\
56.375	0.12	-90.6591528123474\\
56.375	0.1206	-92.4390371246454\\
56.375	0.1212	-94.2189214369436\\
56.375	0.1218	-95.9988057492415\\
56.375	0.1224	-97.7786900615395\\
56.375	0.123	-99.5585743738372\\
56.75	0.093	-10.1717795956924\\
56.75	0.0936	-11.9282290322826\\
56.75	0.0942	-13.684678468873\\
56.75	0.0948	-15.4411279054633\\
56.75	0.0954	-17.1975773420537\\
56.75	0.096	-18.9540267786441\\
56.75	0.0966	-20.7104762152343\\
56.75	0.0972	-22.4669256518247\\
56.75	0.0978	-24.2233750884149\\
56.75	0.0984	-25.9798245250053\\
56.75	0.099	-27.7362739615957\\
56.75	0.0996	-29.4927233981859\\
56.75	0.1002	-31.2491728347763\\
56.75	0.1008	-33.0056222713665\\
56.75	0.1014	-34.7620717079569\\
56.75	0.102	-36.5185211445471\\
56.75	0.1026	-38.2749705811375\\
56.75	0.1032	-40.0314200177279\\
56.75	0.1038	-41.7878694543181\\
56.75	0.1044	-43.5443188909086\\
56.75	0.105	-45.3007683274988\\
56.75	0.1056	-47.0572177640892\\
56.75	0.1062	-48.8136672006796\\
56.75	0.1068	-50.5701166372698\\
56.75	0.1074	-52.3265660738601\\
56.75	0.108	-54.0830155104503\\
56.75	0.1086	-55.8394649470407\\
56.75	0.1092	-57.595914383631\\
56.75	0.1098	-59.3523638202214\\
56.75	0.1104	-61.1088132568118\\
56.75	0.111	-62.8652626934019\\
56.75	0.1116	-64.6217121299923\\
56.75	0.1122	-66.3781615665825\\
56.75	0.1128	-68.1346110031729\\
56.75	0.1134	-69.8910604397634\\
56.75	0.114	-71.6475098763535\\
56.75	0.1146	-73.4039593129439\\
56.75	0.1152	-75.1604087495341\\
56.75	0.1158	-76.9168581861245\\
56.75	0.1164	-78.6733076227149\\
56.75	0.117	-80.4297570593052\\
56.75	0.1176	-82.1862064958956\\
56.75	0.1182	-83.942655932486\\
56.75	0.1188	-85.6991053690762\\
56.75	0.1194	-87.4555548056666\\
56.75	0.12	-89.2120042422567\\
56.75	0.1206	-90.9684536788471\\
56.75	0.1212	-92.7249031154375\\
56.75	0.1218	-94.4813525520278\\
56.75	0.1224	-96.2378019886182\\
56.75	0.123	-97.9942514252084\\
57.125	0.093	-9.77920043244603\\
57.125	0.0936	-11.5122149933286\\
57.125	0.0942	-13.2452295542114\\
57.125	0.0948	-14.978244115094\\
57.125	0.0954	-16.7112586759768\\
57.125	0.096	-18.4442732368593\\
57.125	0.0966	-20.177287797742\\
57.125	0.0972	-21.9103023586247\\
57.125	0.0978	-23.6433169195074\\
57.125	0.0984	-25.37633148039\\
57.125	0.099	-27.1093460412727\\
57.125	0.0996	-28.8423606021553\\
57.125	0.1002	-30.5753751630381\\
57.125	0.1008	-32.3083897239206\\
57.125	0.1014	-34.0414042848034\\
57.125	0.102	-35.774418845686\\
57.125	0.1026	-37.5074334065687\\
57.125	0.1032	-39.2404479674514\\
57.125	0.1038	-40.973462528334\\
57.125	0.1044	-42.7064770892167\\
57.125	0.105	-44.4394916500993\\
57.125	0.1056	-46.1725062109821\\
57.125	0.1062	-47.9055207718648\\
57.125	0.1068	-49.6385353327474\\
57.125	0.1074	-51.37154989363\\
57.125	0.108	-53.1045644545127\\
57.125	0.1086	-54.8375790153955\\
57.125	0.1092	-56.570593576278\\
57.125	0.1098	-58.3036081371607\\
57.125	0.1104	-60.0366226980434\\
57.125	0.111	-61.769637258926\\
57.125	0.1116	-63.5026518198088\\
57.125	0.1122	-65.2356663806913\\
57.125	0.1128	-66.9686809415741\\
57.125	0.1134	-68.7016955024567\\
57.125	0.114	-70.4347100633394\\
57.125	0.1146	-72.1677246242222\\
57.125	0.1152	-73.9007391851047\\
57.125	0.1158	-75.6337537459874\\
57.125	0.1164	-77.3667683068701\\
57.125	0.117	-79.0997828677527\\
57.125	0.1176	-80.8327974286354\\
57.125	0.1182	-82.5658119895181\\
57.125	0.1188	-84.2988265504007\\
57.125	0.1194	-86.0318411112835\\
57.125	0.12	-87.764855672166\\
57.125	0.1206	-89.4978702330487\\
57.125	0.1212	-91.2308847939315\\
57.125	0.1218	-92.9638993548141\\
57.125	0.1224	-94.6969139156969\\
57.125	0.123	-96.4299284765793\\
57.5	0.093	-9.38662126919951\\
57.5	0.0936	-11.0962009543745\\
57.5	0.0942	-12.8057806395495\\
57.5	0.0948	-14.5153603247245\\
57.5	0.0954	-16.2249400098997\\
57.5	0.096	-17.9345196950746\\
57.5	0.0966	-19.6440993802496\\
57.5	0.0972	-21.3536790654247\\
57.5	0.0978	-23.0632587505996\\
57.5	0.0984	-24.7728384357747\\
57.5	0.099	-26.4824181209497\\
57.5	0.0996	-28.1919978061246\\
57.5	0.1002	-29.9015774912997\\
57.5	0.1008	-31.6111571764748\\
57.5	0.1014	-33.3207368616498\\
57.5	0.102	-35.0303165468248\\
57.5	0.1026	-36.7398962319999\\
57.5	0.1032	-38.4494759171748\\
57.5	0.1038	-40.1590556023498\\
57.5	0.1044	-41.8686352875249\\
57.5	0.105	-43.5782149726998\\
57.5	0.1056	-45.287794657875\\
57.5	0.1062	-46.99737434305\\
57.5	0.1068	-48.706954028225\\
57.5	0.1074	-50.4165337134\\
57.5	0.108	-52.1261133985749\\
57.5	0.1086	-53.83569308375\\
57.5	0.1092	-55.545272768925\\
57.5	0.1098	-57.2548524541\\
57.5	0.1104	-58.964432139275\\
57.5	0.111	-60.6740118244501\\
57.5	0.1116	-62.3835915096251\\
57.5	0.1122	-64.0931711948001\\
57.5	0.1128	-65.8027508799752\\
57.5	0.1134	-67.5123305651501\\
57.5	0.114	-69.2219102503251\\
57.5	0.1146	-70.9314899355002\\
57.5	0.1152	-72.6410696206751\\
57.5	0.1158	-74.3506493058503\\
57.5	0.1164	-76.0602289910253\\
57.5	0.117	-77.7698086762002\\
57.5	0.1176	-79.4793883613753\\
57.5	0.1182	-81.1889680465504\\
57.5	0.1188	-82.8985477317252\\
57.5	0.1194	-84.6081274169004\\
57.5	0.12	-86.3177071020752\\
57.5	0.1206	-88.0272867872503\\
57.5	0.1212	-89.7368664724255\\
57.5	0.1218	-91.4464461576005\\
57.5	0.1224	-93.1560258427755\\
57.5	0.123	-94.8656055279504\\
57.875	0.093	-8.9940421059531\\
57.875	0.0936	-10.6801869154204\\
57.875	0.0942	-12.3663317248879\\
57.875	0.0948	-14.0524765343552\\
57.875	0.0954	-15.7386213438226\\
57.875	0.096	-17.42476615329\\
57.875	0.0966	-19.1109109627573\\
57.875	0.0972	-20.7970557722247\\
57.875	0.0978	-22.483200581692\\
57.875	0.0984	-24.1693453911595\\
57.875	0.099	-25.8554902006268\\
57.875	0.0996	-27.5416350100941\\
57.875	0.1002	-29.2277798195615\\
57.875	0.1008	-30.9139246290289\\
57.875	0.1014	-32.6000694384963\\
57.875	0.102	-34.2862142479636\\
57.875	0.1026	-35.972359057431\\
57.875	0.1032	-37.6585038668984\\
57.875	0.1038	-39.3446486763658\\
57.875	0.1044	-41.0307934858331\\
57.875	0.105	-42.7169382953005\\
57.875	0.1056	-44.4030831047679\\
57.875	0.1062	-46.0892279142353\\
57.875	0.1068	-47.7753727237027\\
57.875	0.1074	-49.46151753317\\
57.875	0.108	-51.1476623426373\\
57.875	0.1086	-52.8338071521048\\
57.875	0.1092	-54.519951961572\\
57.875	0.1098	-56.2060967710395\\
57.875	0.1104	-57.8922415805068\\
57.875	0.111	-59.5783863899742\\
57.875	0.1116	-61.2645311994415\\
57.875	0.1122	-62.9506760089089\\
57.875	0.1128	-64.6368208183763\\
57.875	0.1134	-66.3229656278437\\
57.875	0.114	-68.0091104373109\\
57.875	0.1146	-69.6952552467784\\
57.875	0.1152	-71.3814000562458\\
57.875	0.1158	-73.0675448657132\\
57.875	0.1164	-74.7536896751805\\
57.875	0.117	-76.4398344846478\\
57.875	0.1176	-78.1259792941153\\
57.875	0.1182	-79.8121241035827\\
57.875	0.1188	-81.4982689130499\\
57.875	0.1194	-83.1844137225174\\
57.875	0.12	-84.8705585319847\\
57.875	0.1206	-86.5567033414521\\
57.875	0.1212	-88.2428481509195\\
57.875	0.1218	-89.9289929603868\\
57.875	0.1224	-91.6151377698543\\
57.875	0.123	-93.3012825793214\\
58.25	0.093	-8.6014629427068\\
58.25	0.0936	-10.2641728764664\\
58.25	0.0942	-11.9268828102262\\
58.25	0.0948	-13.5895927439859\\
58.25	0.0954	-15.2523026777457\\
58.25	0.096	-16.9150126115053\\
58.25	0.0966	-18.577722545265\\
58.25	0.0972	-20.2404324790248\\
58.25	0.0978	-21.9031424127844\\
58.25	0.0984	-23.5658523465443\\
58.25	0.099	-25.2285622803039\\
58.25	0.0996	-26.8912722140636\\
58.25	0.1002	-28.5539821478234\\
58.25	0.1008	-30.216692081583\\
58.25	0.1014	-31.8794020153429\\
58.25	0.102	-33.5421119491025\\
58.25	0.1026	-35.2048218828622\\
58.25	0.1032	-36.867531816622\\
58.25	0.1038	-38.5302417503816\\
58.25	0.1044	-40.1929516841413\\
58.25	0.105	-41.8556616179011\\
58.25	0.1056	-43.5183715516608\\
58.25	0.1062	-45.1810814854206\\
58.25	0.1068	-46.8437914191803\\
58.25	0.1074	-48.5065013529399\\
58.25	0.108	-50.1692112866997\\
58.25	0.1086	-51.8319212204594\\
58.25	0.1092	-53.4946311542191\\
58.25	0.1098	-55.1573410879789\\
58.25	0.1104	-56.8200510217385\\
58.25	0.111	-58.4827609554983\\
58.25	0.1116	-60.145470889258\\
58.25	0.1122	-61.8081808230177\\
58.25	0.1128	-63.4708907567775\\
58.25	0.1134	-65.1336006905372\\
58.25	0.114	-66.7963106242969\\
58.25	0.1146	-68.4590205580566\\
58.25	0.1152	-70.1217304918163\\
58.25	0.1158	-71.7844404255761\\
58.25	0.1164	-73.4471503593358\\
58.25	0.117	-75.1098602930954\\
58.25	0.1176	-76.7725702268552\\
58.25	0.1182	-78.435280160615\\
58.25	0.1188	-80.0979900943746\\
58.25	0.1194	-81.7607000281345\\
58.25	0.12	-83.423409961894\\
58.25	0.1206	-85.0861198956538\\
58.25	0.1212	-86.7488298294136\\
58.25	0.1218	-88.4115397631732\\
58.25	0.1224	-90.0742496969331\\
58.25	0.123	-91.7369596306926\\
58.625	0.093	-8.20888377946039\\
58.625	0.0936	-9.84815883751241\\
58.625	0.0942	-11.4874338955645\\
58.625	0.0948	-13.1267089536166\\
58.625	0.0954	-14.7659840116687\\
58.625	0.096	-16.4052590697207\\
58.625	0.0966	-18.0445341277727\\
58.625	0.0972	-19.6838091858249\\
58.625	0.0978	-21.3230842438769\\
58.625	0.0984	-22.962359301929\\
58.625	0.099	-24.6016343599811\\
58.625	0.0996	-26.2409094180331\\
58.625	0.1002	-27.8801844760852\\
58.625	0.1008	-29.5194595341372\\
58.625	0.1014	-31.1587345921894\\
58.625	0.102	-32.7980096502414\\
58.625	0.1026	-34.4372847082935\\
58.625	0.1032	-36.0765597663456\\
58.625	0.1038	-37.7158348243976\\
58.625	0.1044	-39.3551098824497\\
58.625	0.105	-40.9943849405017\\
58.625	0.1056	-42.6336599985539\\
58.625	0.1062	-44.2729350566059\\
58.625	0.1068	-45.912210114658\\
58.625	0.1074	-47.55148517271\\
58.625	0.108	-49.1907602307621\\
58.625	0.1086	-50.8300352888142\\
58.625	0.1092	-52.4693103468662\\
58.625	0.1098	-54.1085854049184\\
58.625	0.1104	-55.7478604629704\\
58.625	0.111	-57.3871355210224\\
58.625	0.1116	-59.0264105790745\\
58.625	0.1122	-60.6656856371266\\
58.625	0.1128	-62.3049606951787\\
58.625	0.1134	-63.9442357532307\\
58.625	0.114	-65.5835108112827\\
58.625	0.1146	-67.2227858693349\\
58.625	0.1152	-68.8620609273869\\
58.625	0.1158	-70.501335985439\\
58.625	0.1164	-72.140611043491\\
58.625	0.117	-73.7798861015431\\
58.625	0.1176	-75.4191611595952\\
58.625	0.1182	-77.0584362176473\\
58.625	0.1188	-78.6977112756992\\
58.625	0.1194	-80.3369863337515\\
58.625	0.12	-81.9762613918034\\
58.625	0.1206	-83.6155364498555\\
58.625	0.1212	-85.2548115079077\\
58.625	0.1218	-86.8940865659597\\
58.625	0.1224	-88.5333616240118\\
58.625	0.123	-90.1726366820637\\
59	0.093	-7.81630461621398\\
59	0.0936	-9.4321447985584\\
59	0.0942	-11.0479849809028\\
59	0.0948	-12.6638251632472\\
59	0.0954	-14.2796653455916\\
59	0.096	-15.8955055279361\\
59	0.0966	-17.5113457102805\\
59	0.0972	-19.1271858926249\\
59	0.0978	-20.7430260749693\\
59	0.0984	-22.3588662573138\\
59	0.099	-23.9747064396581\\
59	0.0996	-25.5905466220025\\
59	0.1002	-27.2063868043469\\
59	0.1008	-28.8222269866914\\
59	0.1014	-30.4380671690359\\
59	0.102	-32.0539073513802\\
59	0.1026	-33.6697475337247\\
59	0.1032	-35.2855877160691\\
59	0.1038	-36.9014278984134\\
59	0.1044	-38.517268080758\\
59	0.105	-40.1331082631023\\
59	0.1056	-41.7489484454468\\
59	0.1062	-43.3647886277912\\
59	0.1068	-44.9806288101356\\
59	0.1074	-46.59646899248\\
59	0.108	-48.2123091748244\\
59	0.1086	-49.8281493571689\\
59	0.1092	-51.4439895395133\\
59	0.1098	-53.0598297218577\\
59	0.1104	-54.6756699042021\\
59	0.111	-56.2915100865465\\
59	0.1116	-57.9073502688909\\
59	0.1122	-59.5231904512353\\
59	0.1128	-61.1390306335799\\
59	0.1134	-62.7548708159242\\
59	0.114	-64.3707109982686\\
59	0.1146	-65.9865511806131\\
59	0.1152	-67.6023913629574\\
59	0.1158	-69.2182315453019\\
59	0.1164	-70.8340717276462\\
59	0.117	-72.4499119099906\\
59	0.1176	-74.0657520923352\\
59	0.1182	-75.6815922746796\\
59	0.1188	-77.2974324570239\\
59	0.1194	-78.9132726393684\\
59	0.12	-80.5291128217127\\
59	0.1206	-82.1449530040572\\
59	0.1212	-83.7607931864017\\
59	0.1218	-85.3766333687461\\
59	0.1224	-86.9924735510906\\
59	0.123	-88.6083137334348\\
59.375	0.093	-7.42372545296746\\
59.375	0.0936	-9.01613075960427\\
59.375	0.0942	-10.6085360662411\\
59.375	0.0948	-12.2009413728778\\
59.375	0.0954	-13.7933466795146\\
59.375	0.096	-15.3857519861514\\
59.375	0.0966	-16.9781572927881\\
59.375	0.0972	-18.5705625994249\\
59.375	0.0978	-20.1629679060616\\
59.375	0.0984	-21.7553732126985\\
59.375	0.099	-23.3477785193352\\
59.375	0.0996	-24.9401838259719\\
59.375	0.1002	-26.5325891326087\\
59.375	0.1008	-28.1249944392455\\
59.375	0.1014	-29.7173997458823\\
59.375	0.102	-31.309805052519\\
59.375	0.1026	-32.9022103591558\\
59.375	0.1032	-34.4946156657926\\
59.375	0.1038	-36.0870209724293\\
59.375	0.1044	-37.6794262790661\\
59.375	0.105	-39.2718315857028\\
59.375	0.1056	-40.8642368923397\\
59.375	0.1062	-42.4566421989764\\
59.375	0.1068	-44.0490475056132\\
59.375	0.1074	-45.6414528122499\\
59.375	0.108	-47.2338581188867\\
59.375	0.1086	-48.8262634255235\\
59.375	0.1092	-50.4186687321602\\
59.375	0.1098	-52.011074038797\\
59.375	0.1104	-53.6034793454338\\
59.375	0.111	-55.1958846520705\\
59.375	0.1116	-56.7882899587073\\
59.375	0.1122	-58.380695265344\\
59.375	0.1128	-59.9731005719809\\
59.375	0.1134	-61.5655058786176\\
59.375	0.114	-63.1579111852543\\
59.375	0.1146	-64.7503164918911\\
59.375	0.1152	-66.3427217985279\\
59.375	0.1158	-67.9351271051647\\
59.375	0.1164	-69.5275324118014\\
59.375	0.117	-71.1199377184381\\
59.375	0.1176	-72.712343025075\\
59.375	0.1182	-74.3047483317118\\
59.375	0.1188	-75.8971536383484\\
59.375	0.1194	-77.4895589449853\\
59.375	0.12	-79.081964251622\\
59.375	0.1206	-80.6743695582588\\
59.375	0.1212	-82.2667748648956\\
59.375	0.1218	-83.8591801715323\\
59.375	0.1224	-85.4515854781692\\
59.375	0.123	-87.0439907848058\\
59.75	0.093	-7.03114628972116\\
59.75	0.0936	-8.60011672065025\\
59.75	0.0942	-10.1690871515794\\
59.75	0.0948	-11.7380575825084\\
59.75	0.0954	-13.3070280134376\\
59.75	0.096	-14.8759984443667\\
59.75	0.0966	-16.4449688752958\\
59.75	0.0972	-18.013939306225\\
59.75	0.0978	-19.5829097371541\\
59.75	0.0984	-21.1518801680833\\
59.75	0.099	-22.7208505990122\\
59.75	0.0996	-24.2898210299413\\
59.75	0.1002	-25.8587914608705\\
59.75	0.1008	-27.4277618917996\\
59.75	0.1014	-28.9967323227288\\
59.75	0.102	-30.5657027536579\\
59.75	0.1026	-32.1346731845871\\
59.75	0.1032	-33.7036436155162\\
59.75	0.1038	-35.2726140464451\\
59.75	0.1044	-36.8415844773743\\
59.75	0.105	-38.4105549083034\\
59.75	0.1056	-39.9795253392326\\
59.75	0.1062	-41.5484957701617\\
59.75	0.1068	-43.1174662010909\\
59.75	0.1074	-44.68643663202\\
59.75	0.108	-46.2554070629491\\
59.75	0.1086	-47.8243774938782\\
59.75	0.1092	-49.3933479248072\\
59.75	0.1098	-50.9623183557364\\
59.75	0.1104	-52.5312887866655\\
59.75	0.111	-54.1002592175946\\
59.75	0.1116	-55.6692296485238\\
59.75	0.1122	-57.2382000794529\\
59.75	0.1128	-58.807170510382\\
59.75	0.1134	-60.3761409413111\\
59.75	0.114	-61.9451113722401\\
59.75	0.1146	-63.5140818031693\\
59.75	0.1152	-65.0830522340984\\
59.75	0.1158	-66.6520226650276\\
59.75	0.1164	-68.2209930959567\\
59.75	0.117	-69.7899635268858\\
59.75	0.1176	-71.3589339578149\\
59.75	0.1182	-72.9279043887441\\
59.75	0.1188	-74.4968748196731\\
59.75	0.1194	-76.0658452506024\\
59.75	0.12	-77.6348156815313\\
59.75	0.1206	-79.2037861124605\\
59.75	0.1212	-80.7727565433897\\
59.75	0.1218	-82.3417269743188\\
59.75	0.1224	-83.9106974052479\\
59.75	0.123	-85.4796678361769\\
60.125	0.093	-6.63856712647475\\
60.125	0.0936	-8.18410268169612\\
60.125	0.0942	-9.72963823691771\\
60.125	0.0948	-11.2751737921391\\
60.125	0.0954	-12.8207093473607\\
60.125	0.096	-14.366244902582\\
60.125	0.0966	-15.9117804578035\\
60.125	0.0972	-17.457316013025\\
60.125	0.0978	-19.0028515682465\\
60.125	0.0984	-20.5483871234679\\
60.125	0.099	-22.0939226786894\\
60.125	0.0996	-23.6394582339108\\
60.125	0.1002	-25.1849937891324\\
60.125	0.1008	-26.7305293443537\\
60.125	0.1014	-28.2760648995753\\
60.125	0.102	-29.8216004547967\\
60.125	0.1026	-31.3671360100183\\
60.125	0.1032	-32.9126715652396\\
60.125	0.1038	-34.4582071204611\\
60.125	0.1044	-36.0037426756826\\
60.125	0.105	-37.5492782309041\\
60.125	0.1056	-39.0948137861255\\
60.125	0.1062	-40.640349341347\\
60.125	0.1068	-42.1858848965686\\
60.125	0.1074	-43.73142045179\\
60.125	0.108	-45.2769560070113\\
60.125	0.1086	-46.8224915622329\\
60.125	0.1092	-48.3680271174543\\
60.125	0.1098	-49.9135626726759\\
60.125	0.1104	-51.4590982278972\\
60.125	0.111	-53.0046337831187\\
60.125	0.1116	-54.5501693383403\\
60.125	0.1122	-56.0957048935617\\
60.125	0.1128	-57.6412404487832\\
60.125	0.1134	-59.1867760040046\\
60.125	0.114	-60.7323115592261\\
60.125	0.1146	-62.2778471144476\\
60.125	0.1152	-63.8233826696689\\
60.125	0.1158	-65.3689182248905\\
60.125	0.1164	-66.914453780112\\
60.125	0.117	-68.4599893353334\\
60.125	0.1176	-70.0055248905549\\
60.125	0.1182	-71.5510604457764\\
60.125	0.1188	-73.0965960009978\\
60.125	0.1194	-74.6421315562194\\
60.125	0.12	-76.1876671114406\\
60.125	0.1206	-77.7332026666622\\
60.125	0.1212	-79.2787382218838\\
60.125	0.1218	-80.8242737771052\\
60.125	0.1224	-82.3698093323267\\
60.125	0.123	-83.915344887548\\
60.5	0.093	-6.24598796322823\\
60.5	0.0936	-7.76808864274199\\
60.5	0.0942	-9.29018932225586\\
60.5	0.0948	-10.8122900017696\\
60.5	0.0954	-12.3343906812836\\
60.5	0.096	-13.8564913607973\\
60.5	0.0966	-15.3785920403111\\
60.5	0.0972	-16.900692719825\\
60.5	0.0978	-18.4227933993387\\
60.5	0.0984	-19.9448940788526\\
60.5	0.099	-21.4669947583664\\
60.5	0.0996	-22.9890954378801\\
60.5	0.1002	-24.5111961173941\\
60.5	0.1008	-26.0332967969078\\
60.5	0.1014	-27.5553974764217\\
60.5	0.102	-29.0774981559355\\
60.5	0.1026	-30.5995988354493\\
60.5	0.1032	-32.1216995149631\\
60.5	0.1038	-33.6438001944769\\
60.5	0.1044	-35.1659008739907\\
60.5	0.105	-36.6880015535046\\
60.5	0.1056	-38.2101022330185\\
60.5	0.1062	-39.7322029125322\\
60.5	0.1068	-41.2543035920461\\
60.5	0.1074	-42.7764042715598\\
60.5	0.108	-44.2985049510736\\
60.5	0.1086	-45.8206056305875\\
60.5	0.1092	-47.3427063101012\\
60.5	0.1098	-48.8648069896152\\
60.5	0.1104	-50.386907669129\\
60.5	0.111	-51.9090083486427\\
60.5	0.1116	-53.4311090281566\\
60.5	0.1122	-54.9532097076703\\
60.5	0.1128	-56.4753103871842\\
60.5	0.1134	-57.9974110666981\\
60.5	0.114	-59.5195117462118\\
60.5	0.1146	-61.0416124257257\\
60.5	0.1152	-62.5637131052395\\
60.5	0.1158	-64.0858137847533\\
60.5	0.1164	-65.6079144642671\\
60.5	0.117	-67.1300151437808\\
60.5	0.1176	-68.6521158232947\\
60.5	0.1182	-70.1742165028087\\
60.5	0.1188	-71.6963171823223\\
60.5	0.1194	-73.2184178618363\\
60.5	0.12	-74.74051854135\\
60.5	0.1206	-76.2626192208638\\
60.5	0.1212	-77.7847199003777\\
60.5	0.1218	-79.3068205798915\\
60.5	0.1224	-80.8289212594053\\
60.5	0.123	-82.3510219389191\\
60.875	0.093	-5.85340879998182\\
60.875	0.0936	-7.35207460378797\\
60.875	0.0942	-8.85074040759423\\
60.875	0.0948	-10.3494062114003\\
60.875	0.0954	-11.8480720152065\\
60.875	0.096	-13.3467378190127\\
60.875	0.0966	-14.8454036228188\\
60.875	0.0972	-16.3440694266251\\
60.875	0.0978	-17.8427352304311\\
60.875	0.0984	-19.3414010342374\\
60.875	0.099	-20.8400668380435\\
60.875	0.0996	-22.3387326418497\\
60.875	0.1002	-23.8373984456558\\
60.875	0.1008	-25.336064249462\\
60.875	0.1014	-26.8347300532682\\
60.875	0.102	-28.3333958570744\\
60.875	0.1026	-29.8320616608805\\
60.875	0.1032	-31.3307274646867\\
60.875	0.1038	-32.8293932684928\\
60.875	0.1044	-34.3280590722991\\
60.875	0.105	-35.8267248761051\\
60.875	0.1056	-37.3253906799114\\
60.875	0.1062	-38.8240564837175\\
60.875	0.1068	-40.3227222875238\\
60.875	0.1074	-41.8213880913298\\
60.875	0.108	-43.320053895136\\
60.875	0.1086	-44.8187196989422\\
60.875	0.1092	-46.3173855027484\\
60.875	0.1098	-47.8160513065545\\
60.875	0.1104	-49.3147171103607\\
60.875	0.111	-50.8133829141668\\
60.875	0.1116	-52.3120487179731\\
60.875	0.1122	-53.8107145217791\\
60.875	0.1128	-55.3093803255854\\
60.875	0.1134	-56.8080461293915\\
60.875	0.114	-58.3067119331977\\
60.875	0.1146	-59.8053777370039\\
60.875	0.1152	-61.30404354081\\
60.875	0.1158	-62.8027093446162\\
60.875	0.1164	-64.3013751484224\\
60.875	0.117	-65.8000409522284\\
60.875	0.1176	-67.2987067560347\\
60.875	0.1182	-68.7973725598409\\
60.875	0.1188	-70.296038363647\\
60.875	0.1194	-71.7947041674532\\
60.875	0.12	-73.2933699712593\\
60.875	0.1206	-74.7920357750655\\
60.875	0.1212	-76.2907015788718\\
60.875	0.1218	-77.7893673826778\\
60.875	0.1224	-79.2880331864841\\
60.875	0.123	-80.7866989902901\\
61.25	0.093	-5.46082963673541\\
61.25	0.0936	-6.93606056483384\\
61.25	0.0942	-8.41129149293249\\
61.25	0.0948	-9.88652242103092\\
61.25	0.0954	-11.3617533491295\\
61.25	0.096	-12.836984277228\\
61.25	0.0966	-14.3122152053264\\
61.25	0.0972	-15.787446133425\\
61.25	0.0978	-17.2626770615235\\
61.25	0.0984	-18.737907989622\\
61.25	0.099	-20.2131389177206\\
61.25	0.0996	-21.688369845819\\
61.25	0.1002	-23.1636007739176\\
61.25	0.1008	-24.6388317020161\\
61.25	0.1014	-26.1140626301146\\
61.25	0.102	-27.5892935582131\\
61.25	0.1026	-29.0645244863117\\
61.25	0.1032	-30.5397554144101\\
61.25	0.1038	-32.0149863425087\\
61.25	0.1044	-33.4902172706072\\
61.25	0.105	-34.9654481987056\\
61.25	0.1056	-36.4406791268043\\
61.25	0.1062	-37.9159100549027\\
61.25	0.1068	-39.3911409830014\\
61.25	0.1074	-40.8663719110998\\
61.25	0.108	-42.3416028391982\\
61.25	0.1086	-43.8168337672969\\
61.25	0.1092	-45.2920646953953\\
61.25	0.1098	-46.7672956234939\\
61.25	0.1104	-48.2425265515924\\
61.25	0.111	-49.7177574796908\\
61.25	0.1116	-51.1929884077895\\
61.25	0.1122	-52.6682193358879\\
61.25	0.1128	-54.1434502639864\\
61.25	0.1134	-55.618681192085\\
61.25	0.114	-57.0939121201834\\
61.25	0.1146	-58.5691430482821\\
61.25	0.1152	-60.0443739763805\\
61.25	0.1158	-61.519604904479\\
61.25	0.1164	-62.9948358325776\\
61.25	0.117	-64.470066760676\\
61.25	0.1176	-65.9452976887745\\
61.25	0.1182	-67.4205286168732\\
61.25	0.1188	-68.8957595449715\\
61.25	0.1194	-70.3709904730702\\
61.25	0.12	-71.8462214011686\\
61.25	0.1206	-73.3214523292671\\
61.25	0.1212	-74.7966832573658\\
61.25	0.1218	-76.2719141854642\\
61.25	0.1224	-77.7471451135627\\
61.25	0.123	-79.2223760416612\\
61.625	0.093	-5.068250473489\\
61.625	0.0936	-6.52004652587982\\
61.625	0.0942	-7.97184257827075\\
61.625	0.0948	-9.42363863066157\\
61.625	0.0954	-10.8754346830525\\
61.625	0.096	-12.3272307354433\\
61.625	0.0966	-13.7790267878341\\
61.625	0.0972	-15.2308228402251\\
61.625	0.0978	-16.6826188926159\\
61.625	0.0984	-18.1344149450068\\
61.625	0.099	-19.5862109973976\\
61.625	0.0996	-21.0380070497885\\
61.625	0.1002	-22.4898031021794\\
61.625	0.1008	-23.9415991545702\\
61.625	0.1014	-25.3933952069611\\
61.625	0.102	-26.845191259352\\
61.625	0.1026	-28.2969873117429\\
61.625	0.1032	-29.7487833641337\\
61.625	0.1038	-31.2005794165245\\
61.625	0.1044	-32.6523754689155\\
61.625	0.105	-34.1041715213063\\
61.625	0.1056	-35.5559675736972\\
61.625	0.1062	-37.007763626088\\
61.625	0.1068	-38.459559678479\\
61.625	0.1074	-39.9113557308698\\
61.625	0.108	-41.3631517832606\\
61.625	0.1086	-42.8149478356515\\
61.625	0.1092	-44.2667438880424\\
61.625	0.1098	-45.7185399404333\\
61.625	0.1104	-47.1703359928241\\
61.625	0.111	-48.6221320452149\\
61.625	0.1116	-50.0739280976059\\
61.625	0.1122	-51.5257241499967\\
61.625	0.1128	-52.9775202023876\\
61.625	0.1134	-54.4293162547784\\
61.625	0.114	-55.8811123071692\\
61.625	0.1146	-57.3329083595602\\
61.625	0.1152	-58.784704411951\\
61.625	0.1158	-60.2365004643419\\
61.625	0.1164	-61.6882965167327\\
61.625	0.117	-63.1400925691236\\
61.625	0.1176	-64.5918886215145\\
61.625	0.1182	-66.0436846739054\\
61.625	0.1188	-67.4954807262961\\
61.625	0.1194	-68.9472767786872\\
61.625	0.12	-70.3990728310779\\
61.625	0.1206	-71.8508688834688\\
61.625	0.1212	-73.3026649358598\\
61.625	0.1218	-74.7544609882506\\
61.625	0.1224	-76.2062570406415\\
61.625	0.123	-77.6580530930322\\
62	0.093	-4.67567131024259\\
62	0.0936	-6.1040324869258\\
62	0.0942	-7.53239366360913\\
62	0.0948	-8.96075484029222\\
62	0.0954	-10.3891160169755\\
62	0.096	-11.8174771936586\\
62	0.0966	-13.2458383703419\\
62	0.0972	-14.6741995470252\\
62	0.0978	-16.1025607237083\\
62	0.0984	-17.5309219003916\\
62	0.099	-18.9592830770748\\
62	0.0996	-20.3876442537579\\
62	0.1002	-21.8160054304412\\
62	0.1008	-23.2443666071243\\
62	0.1014	-24.6727277838077\\
62	0.102	-26.1010889604909\\
62	0.1026	-27.5294501371741\\
62	0.1032	-28.9578113138573\\
62	0.1038	-30.3861724905405\\
62	0.1044	-31.8145336672237\\
62	0.105	-33.2428948439069\\
62	0.1056	-34.6712560205902\\
62	0.1062	-36.0996171972733\\
62	0.1068	-37.5279783739567\\
62	0.1074	-38.9563395506398\\
62	0.108	-40.384700727323\\
62	0.1086	-41.8130619040063\\
62	0.1092	-43.2414230806894\\
62	0.1098	-44.6697842573727\\
62	0.1104	-46.0981454340559\\
62	0.111	-47.526506610739\\
62	0.1116	-48.9548677874224\\
62	0.1122	-50.3832289641055\\
62	0.1128	-51.8115901407888\\
62	0.1134	-53.239951317472\\
62	0.114	-54.6683124941551\\
62	0.1146	-56.0966736708384\\
62	0.1152	-57.5250348475216\\
62	0.1158	-58.9533960242048\\
62	0.1164	-60.381757200888\\
62	0.117	-61.8101183775711\\
62	0.1176	-63.2384795542545\\
62	0.1182	-64.6668407309378\\
62	0.1188	-66.0952019076208\\
62	0.1194	-67.5235630843042\\
62	0.12	-68.9519242609873\\
62	0.1206	-70.3802854376705\\
62	0.1212	-71.8086466143538\\
62	0.1218	-73.2370077910369\\
62	0.1224	-74.6653689677203\\
62	0.123	-76.0937301444034\\
62.375	0.093	-4.28309214699618\\
62.375	0.0936	-5.68801844797179\\
62.375	0.0942	-7.09294474894739\\
62.375	0.0948	-8.49787104992288\\
62.375	0.0954	-9.90279735089848\\
62.375	0.096	-11.3077236518741\\
62.375	0.0966	-12.7126499528496\\
62.375	0.0972	-14.1175762538252\\
62.375	0.0978	-15.5225025548007\\
62.375	0.0984	-16.9274288557764\\
62.375	0.099	-18.3323551567519\\
62.375	0.0996	-19.7372814577274\\
62.375	0.1002	-21.1422077587031\\
62.375	0.1008	-22.5471340596786\\
62.375	0.1014	-23.9520603606542\\
62.375	0.102	-25.3569866616297\\
62.375	0.1026	-26.7619129626054\\
62.375	0.1032	-28.1668392635809\\
62.375	0.1038	-29.5717655645564\\
62.375	0.1044	-30.976691865532\\
62.375	0.105	-32.3816181665076\\
62.375	0.1056	-33.7865444674832\\
62.375	0.1062	-35.1914707684587\\
62.375	0.1068	-36.5963970694344\\
62.375	0.1074	-38.0013233704099\\
62.375	0.108	-39.4062496713854\\
62.375	0.1086	-40.811175972361\\
62.375	0.1092	-42.2161022733364\\
62.375	0.1098	-43.6210285743122\\
62.375	0.1104	-45.0259548752877\\
62.375	0.111	-46.4308811762631\\
62.375	0.1116	-47.8358074772389\\
62.375	0.1122	-49.2407337782143\\
62.375	0.1128	-50.64566007919\\
62.375	0.1134	-52.0505863801654\\
62.375	0.114	-53.455512681141\\
62.375	0.1146	-54.8604389821166\\
62.375	0.1152	-56.2653652830921\\
62.375	0.1158	-57.6702915840677\\
62.375	0.1164	-59.0752178850433\\
62.375	0.117	-60.4801441860188\\
62.375	0.1176	-61.8850704869944\\
62.375	0.1182	-63.28999678797\\
62.375	0.1188	-64.6949230889455\\
62.375	0.1194	-66.0998493899212\\
62.375	0.12	-67.5047756908966\\
62.375	0.1206	-68.9097019918722\\
62.375	0.1212	-70.3146282928479\\
62.375	0.1218	-71.7195545938234\\
62.375	0.1224	-73.124480894799\\
62.375	0.123	-74.5294071957744\\
62.75	0.093	-3.89051298374989\\
62.75	0.0936	-5.27200440901777\\
62.75	0.0942	-6.65349583428576\\
62.75	0.0948	-8.03498725955353\\
62.75	0.0954	-9.41647868482153\\
62.75	0.096	-10.7979701100894\\
62.75	0.0966	-12.1794615353573\\
62.75	0.0972	-13.5609529606253\\
62.75	0.0978	-14.9424443858932\\
62.75	0.0984	-16.3239358111612\\
62.75	0.099	-17.705427236429\\
62.75	0.0996	-19.0869186616968\\
62.75	0.1002	-20.4684100869648\\
62.75	0.1008	-21.8499015122327\\
62.75	0.1014	-23.2313929375007\\
62.75	0.102	-24.6128843627686\\
62.75	0.1026	-25.9943757880366\\
62.75	0.1032	-27.3758672133044\\
62.75	0.1038	-28.7573586385723\\
62.75	0.1044	-30.1388500638402\\
62.75	0.105	-31.5203414891081\\
62.75	0.1056	-32.9018329143761\\
62.75	0.1062	-34.283324339644\\
62.75	0.1068	-35.664815764912\\
62.75	0.1074	-37.0463071901798\\
62.75	0.108	-38.4277986154477\\
62.75	0.1086	-39.8092900407157\\
62.75	0.1092	-41.1907814659835\\
62.75	0.1098	-42.5722728912515\\
62.75	0.1104	-43.9537643165194\\
62.75	0.111	-45.3352557417873\\
62.75	0.1116	-46.7167471670552\\
62.75	0.1122	-48.0982385923231\\
62.75	0.1128	-49.4797300175911\\
62.75	0.1134	-50.861221442859\\
62.75	0.114	-52.2427128681269\\
62.75	0.1146	-53.6242042933949\\
62.75	0.1152	-55.0056957186627\\
62.75	0.1158	-56.3871871439306\\
62.75	0.1164	-57.7686785691985\\
62.75	0.117	-59.1501699944664\\
62.75	0.1176	-60.5316614197344\\
62.75	0.1182	-61.9131528450024\\
62.75	0.1188	-63.2946442702702\\
62.75	0.1194	-64.6761356955383\\
62.75	0.12	-66.057627120806\\
62.75	0.1206	-67.4391185460739\\
62.75	0.1212	-68.8206099713419\\
62.75	0.1218	-70.2021013966098\\
62.75	0.1224	-71.5835928218778\\
62.75	0.123	-72.9650842471456\\
63.125	0.093	-3.49793382050336\\
63.125	0.0936	-4.85599037006364\\
63.125	0.0942	-6.21404691962391\\
63.125	0.0948	-7.57210346918419\\
63.125	0.0954	-8.93016001874446\\
63.125	0.096	-10.2882165683047\\
63.125	0.0966	-11.6462731178649\\
63.125	0.0972	-13.0043296674253\\
63.125	0.0978	-14.3623862169854\\
63.125	0.0984	-15.7204427665458\\
63.125	0.099	-17.078499316106\\
63.125	0.0996	-18.4365558656663\\
63.125	0.1002	-19.7946124152265\\
63.125	0.1008	-21.1526689647868\\
63.125	0.1014	-22.5107255143471\\
63.125	0.102	-23.8687820639074\\
63.125	0.1026	-25.2268386134676\\
63.125	0.1032	-26.5848951630279\\
63.125	0.1038	-27.9429517125881\\
63.125	0.1044	-29.3010082621485\\
63.125	0.105	-30.6590648117086\\
63.125	0.1056	-32.017121361269\\
63.125	0.1062	-33.3751779108292\\
63.125	0.1068	-34.7332344603896\\
63.125	0.1074	-36.0912910099497\\
63.125	0.108	-37.44934755951\\
63.125	0.1086	-38.8074041090703\\
63.125	0.1092	-40.1654606586305\\
63.125	0.1098	-41.5235172081908\\
63.125	0.1104	-42.8815737577511\\
63.125	0.111	-44.2396303073112\\
63.125	0.1116	-45.5976868568716\\
63.125	0.1122	-46.9557434064318\\
63.125	0.1128	-48.3137999559922\\
63.125	0.1134	-49.6718565055523\\
63.125	0.114	-51.0299130551126\\
63.125	0.1146	-52.387969604673\\
63.125	0.1152	-53.7460261542332\\
63.125	0.1158	-55.1040827037936\\
63.125	0.1164	-56.4621392533537\\
63.125	0.117	-57.8201958029139\\
63.125	0.1176	-59.1782523524743\\
63.125	0.1182	-60.5363089020346\\
63.125	0.1188	-61.8943654515947\\
63.125	0.1194	-63.2524220011551\\
63.125	0.12	-64.6104785507152\\
63.125	0.1206	-65.9685351002756\\
63.125	0.1212	-67.3265916498359\\
63.125	0.1218	-68.6846481993961\\
63.125	0.1224	-70.0427047489565\\
63.125	0.123	-71.4007612985166\\
63.5	0.093	-3.10535465725695\\
63.5	0.0936	-4.43997633110962\\
63.5	0.0942	-5.77459800496229\\
63.5	0.0948	-7.10921967881484\\
63.5	0.0954	-8.4438413526675\\
63.5	0.096	-9.77846302652006\\
63.5	0.0966	-11.1130847003726\\
63.5	0.0972	-12.4477063742253\\
63.5	0.0978	-13.7823280480778\\
63.5	0.0984	-15.1169497219306\\
63.5	0.099	-16.4515713957832\\
63.5	0.0996	-17.7861930696357\\
63.5	0.1002	-19.1208147434884\\
63.5	0.1008	-20.4554364173409\\
63.5	0.1014	-21.7900580911936\\
63.5	0.102	-23.1246797650462\\
63.5	0.1026	-24.4593014388988\\
63.5	0.1032	-25.7939231127515\\
63.5	0.1038	-27.128544786604\\
63.5	0.1044	-28.4631664604567\\
63.5	0.105	-29.7977881343093\\
63.5	0.1056	-31.1324098081619\\
63.5	0.1062	-32.4670314820145\\
63.5	0.1068	-33.8016531558673\\
63.5	0.1074	-35.1362748297198\\
63.5	0.108	-36.4708965035724\\
63.5	0.1086	-37.805518177425\\
63.5	0.1092	-39.1401398512776\\
63.5	0.1098	-40.4747615251302\\
63.5	0.1104	-41.8093831989828\\
63.5	0.111	-43.1440048728354\\
63.5	0.1116	-44.4786265466881\\
63.5	0.1122	-45.8132482205407\\
63.5	0.1128	-47.1478698943934\\
63.5	0.1134	-48.4824915682459\\
63.5	0.114	-49.8171132420985\\
63.5	0.1146	-51.1517349159511\\
63.5	0.1152	-52.4863565898037\\
63.5	0.1158	-53.8209782636565\\
63.5	0.1164	-55.155599937509\\
63.5	0.117	-56.4902216113616\\
63.5	0.1176	-57.8248432852142\\
63.5	0.1182	-59.1594649590669\\
63.5	0.1188	-60.4940866329193\\
63.5	0.1194	-61.8287083067721\\
63.5	0.12	-63.1633299806246\\
63.5	0.1206	-64.4979516544773\\
63.5	0.1212	-65.83257332833\\
63.5	0.1218	-67.1671950021826\\
63.5	0.1224	-68.5018166760352\\
63.5	0.123	-69.8364383498877\\
63.875	0.093	-2.71277549401066\\
63.875	0.0936	-4.02396229215549\\
63.875	0.0942	-5.33514909030055\\
63.875	0.0948	-6.64633588844549\\
63.875	0.0954	-7.95752268659055\\
63.875	0.096	-9.26870948473538\\
63.875	0.0966	-10.5798962828803\\
63.875	0.0972	-11.8910830810254\\
63.875	0.0978	-13.2022698791703\\
63.875	0.0984	-14.5134566773153\\
63.875	0.099	-15.8246434754602\\
63.875	0.0996	-17.1358302736052\\
63.875	0.1002	-18.4470170717502\\
63.875	0.1008	-19.7582038698951\\
63.875	0.1014	-21.0693906680401\\
63.875	0.102	-22.3805774661851\\
63.875	0.1026	-23.6917642643301\\
63.875	0.1032	-25.0029510624751\\
63.875	0.1038	-26.3141378606199\\
63.875	0.1044	-27.625324658765\\
63.875	0.105	-28.9365114569099\\
63.875	0.1056	-30.247698255055\\
63.875	0.1062	-31.5588850531998\\
63.875	0.1068	-32.8700718513448\\
63.875	0.1074	-34.1812586494898\\
63.875	0.108	-35.4924454476347\\
63.875	0.1086	-36.8036322457797\\
63.875	0.1092	-38.1148190439246\\
63.875	0.1098	-39.4260058420697\\
63.875	0.1104	-40.7371926402146\\
63.875	0.111	-42.0483794383596\\
63.875	0.1116	-43.3595662365045\\
63.875	0.1122	-44.6707530346495\\
63.875	0.1128	-45.9819398327945\\
63.875	0.1134	-47.2931266309395\\
63.875	0.114	-48.6043134290843\\
63.875	0.1146	-49.9155002272294\\
63.875	0.1152	-51.2266870253743\\
63.875	0.1158	-52.5378738235194\\
63.875	0.1164	-53.8490606216642\\
63.875	0.117	-55.1602474198091\\
63.875	0.1176	-56.4714342179542\\
63.875	0.1182	-57.7826210160993\\
63.875	0.1188	-59.093807814244\\
63.875	0.1194	-60.4049946123891\\
63.875	0.12	-61.716181410534\\
63.875	0.1206	-63.027368208679\\
63.875	0.1212	-64.3385550068241\\
63.875	0.1218	-65.6497418049689\\
63.875	0.1224	-66.960928603114\\
63.875	0.123	-68.2721154012588\\
64.25	0.093	-2.32019633076425\\
64.25	0.0936	-3.60794825320147\\
64.25	0.0942	-4.89570017563892\\
64.25	0.0948	-6.18345209807615\\
64.25	0.0954	-7.47120402051348\\
64.25	0.096	-8.75895594295082\\
64.25	0.0966	-10.046707865388\\
64.25	0.0972	-11.3344597878254\\
64.25	0.0978	-12.6222117102627\\
64.25	0.0984	-13.9099636327001\\
64.25	0.099	-15.1977155551374\\
64.25	0.0996	-16.4854674775746\\
64.25	0.1002	-17.773219400012\\
64.25	0.1008	-19.0609713224493\\
64.25	0.1014	-20.3487232448866\\
64.25	0.102	-21.636475167324\\
64.25	0.1026	-22.9242270897613\\
64.25	0.1032	-24.2119790121985\\
64.25	0.1038	-25.4997309346359\\
64.25	0.1044	-26.7874828570732\\
64.25	0.105	-28.0752347795105\\
64.25	0.1056	-29.3629867019479\\
64.25	0.1062	-30.6507386243851\\
64.25	0.1068	-31.9384905468225\\
64.25	0.1074	-33.2262424692598\\
64.25	0.108	-34.5139943916971\\
64.25	0.1086	-35.8017463141344\\
64.25	0.1092	-37.0894982365717\\
64.25	0.1098	-38.3772501590091\\
64.25	0.1104	-39.6650020814463\\
64.25	0.111	-40.9527540038837\\
64.25	0.1116	-42.240505926321\\
64.25	0.1122	-43.5282578487582\\
64.25	0.1128	-44.8160097711957\\
64.25	0.1134	-46.1037616936329\\
64.25	0.114	-47.3915136160703\\
64.25	0.1146	-48.6792655385076\\
64.25	0.1152	-49.9670174609448\\
64.25	0.1158	-51.2547693833823\\
64.25	0.1164	-52.5425213058195\\
64.25	0.117	-53.8302732282567\\
64.25	0.1176	-55.1180251506942\\
64.25	0.1182	-56.4057770731315\\
64.25	0.1188	-57.6935289955687\\
64.25	0.1194	-58.9812809180062\\
64.25	0.12	-60.2690328404433\\
64.25	0.1206	-61.5567847628807\\
64.25	0.1212	-62.8445366853181\\
64.25	0.1218	-64.1322886077553\\
64.25	0.1224	-65.4200405301928\\
64.25	0.123	-66.7077924526299\\
64.625	0.093	-1.92761716751784\\
64.625	0.0936	-3.19193421424745\\
64.625	0.0942	-4.45625126097718\\
64.625	0.0948	-5.7205683077068\\
64.625	0.0954	-6.98488535443653\\
64.625	0.096	-8.24920240116614\\
64.625	0.0966	-9.51351944789576\\
64.625	0.0972	-10.7778364946255\\
64.625	0.0978	-12.0421535413551\\
64.625	0.0984	-13.3064705880848\\
64.625	0.099	-14.5707876348145\\
64.625	0.0996	-15.8351046815441\\
64.625	0.1002	-17.0994217282738\\
64.625	0.1008	-18.3637387750034\\
64.625	0.1014	-19.6280558217331\\
64.625	0.102	-20.8923728684628\\
64.625	0.1026	-22.1566899151925\\
64.625	0.1032	-23.4210069619221\\
64.625	0.1038	-24.6853240086517\\
64.625	0.1044	-25.9496410553814\\
64.625	0.105	-27.2139581021111\\
64.625	0.1056	-28.4782751488408\\
64.625	0.1062	-29.7425921955704\\
64.625	0.1068	-31.0069092423001\\
64.625	0.1074	-32.2712262890298\\
64.625	0.108	-33.5355433357594\\
64.625	0.1086	-34.7998603824892\\
64.625	0.1092	-36.0641774292187\\
64.625	0.1098	-37.3284944759484\\
64.625	0.1104	-38.5928115226782\\
64.625	0.111	-39.8571285694078\\
64.625	0.1116	-41.1214456161375\\
64.625	0.1122	-42.3857626628671\\
64.625	0.1128	-43.6500797095969\\
64.625	0.1134	-44.9143967563265\\
64.625	0.114	-46.1787138030561\\
64.625	0.1146	-47.4430308497858\\
64.625	0.1152	-48.7073478965154\\
64.625	0.1158	-49.9716649432452\\
64.625	0.1164	-51.2359819899748\\
64.625	0.117	-52.5002990367044\\
64.625	0.1176	-53.7646160834341\\
64.625	0.1182	-55.0289331301639\\
64.625	0.1188	-56.2932501768934\\
64.625	0.1194	-57.5575672236232\\
64.625	0.12	-58.8218842703527\\
64.625	0.1206	-60.0862013170824\\
64.625	0.1212	-61.3505183638122\\
64.625	0.1218	-62.6148354105418\\
64.625	0.1224	-63.8791524572715\\
64.625	0.123	-65.143469504001\\
65	0.093	-1.5350380042712\\
65	0.0936	-2.77592017529321\\
65	0.0942	-4.01680234631533\\
65	0.0948	-5.25768451733722\\
65	0.0954	-6.49856668835935\\
65	0.096	-7.73944885938135\\
65	0.0966	-8.98033103040325\\
65	0.0972	-10.2212132014254\\
65	0.0978	-11.4620953724474\\
65	0.0984	-12.7029775434694\\
65	0.099	-13.9438597144914\\
65	0.0996	-15.1847418855133\\
65	0.1002	-16.4256240565354\\
65	0.1008	-17.6665062275574\\
65	0.1014	-18.9073883985794\\
65	0.102	-20.1482705696014\\
65	0.1026	-21.3891527406236\\
65	0.1032	-22.6300349116455\\
65	0.1038	-23.8709170826675\\
65	0.1044	-25.1117992536896\\
65	0.105	-26.3526814247115\\
65	0.1056	-27.5935635957336\\
65	0.1062	-28.8344457667556\\
65	0.1068	-30.0753279377776\\
65	0.1074	-31.3162101087996\\
65	0.108	-32.5570922798215\\
65	0.1086	-33.7979744508436\\
65	0.1092	-35.0388566218656\\
65	0.1098	-36.2797387928877\\
65	0.1104	-37.5206209639097\\
65	0.111	-38.7615031349317\\
65	0.1116	-40.0023853059537\\
65	0.1122	-41.2432674769757\\
65	0.1128	-42.4841496479978\\
65	0.1134	-43.7250318190197\\
65	0.114	-44.9659139900417\\
65	0.1146	-46.2067961610638\\
65	0.1152	-47.4476783320857\\
65	0.1158	-48.6885605031079\\
65	0.1164	-49.9294426741299\\
65	0.117	-51.1703248451518\\
65	0.1176	-52.4112070161739\\
65	0.1182	-53.6520891871959\\
65	0.1188	-54.8929713582178\\
65	0.1194	-56.13385352924\\
65	0.12	-57.3747357002618\\
65	0.1206	-58.6156178712839\\
65	0.1212	-59.856500042306\\
65	0.1218	-61.0973822133279\\
65	0.1224	-62.3382643843501\\
65	0.123	-63.5791465553719\\
65.375	0.093	-1.1424588410249\\
65.375	0.0936	-2.35990613633919\\
65.375	0.0942	-3.57735343165359\\
65.375	0.0948	-4.79480072696788\\
65.375	0.0954	-6.01224802228239\\
65.375	0.096	-7.22969531759668\\
65.375	0.0966	-8.44714261291097\\
65.375	0.0972	-9.66458990822537\\
65.375	0.0978	-10.8820372035398\\
65.375	0.0984	-12.0994844988542\\
65.375	0.099	-13.3169317941685\\
65.375	0.0996	-14.5343790894829\\
65.375	0.1002	-15.7518263847973\\
65.375	0.1008	-16.9692736801115\\
65.375	0.1014	-18.1867209754259\\
65.375	0.102	-19.4041682707403\\
65.375	0.1026	-20.6216155660547\\
65.375	0.1032	-21.839062861369\\
65.375	0.1038	-23.0565101566834\\
65.375	0.1044	-24.2739574519978\\
65.375	0.105	-25.4914047473121\\
65.375	0.1056	-26.7088520426265\\
65.375	0.1062	-27.9262993379409\\
65.375	0.1068	-29.1437466332553\\
65.375	0.1074	-30.3611939285696\\
65.375	0.108	-31.5786412238839\\
65.375	0.1086	-32.7960885191984\\
65.375	0.1092	-34.0135358145127\\
65.375	0.1098	-35.2309831098271\\
65.375	0.1104	-36.4484304051414\\
65.375	0.111	-37.6658777004558\\
65.375	0.1116	-38.8833249957702\\
65.375	0.1122	-40.1007722910845\\
65.375	0.1128	-41.318219586399\\
65.375	0.1134	-42.5356668817133\\
65.375	0.114	-43.7531141770276\\
65.375	0.1146	-44.970561472342\\
65.375	0.1152	-46.1880087676564\\
65.375	0.1158	-47.4054560629708\\
65.375	0.1164	-48.622903358285\\
65.375	0.117	-49.8403506535993\\
65.375	0.1176	-51.0577979489138\\
65.375	0.1182	-52.2752452442282\\
65.375	0.1188	-53.4926925395424\\
65.375	0.1194	-54.7101398348569\\
65.375	0.12	-55.9275871301712\\
65.375	0.1206	-57.1450344254856\\
65.375	0.1212	-58.3624817208\\
65.375	0.1218	-59.5799290161143\\
65.375	0.1224	-60.7973763114288\\
65.375	0.123	-62.014823606743\\
65.75	0.093	-0.749879677778495\\
65.75	0.0936	-1.94389209738517\\
65.75	0.0942	-3.13790451699197\\
65.75	0.0948	-4.33191693659865\\
65.75	0.0954	-5.52592935620532\\
65.75	0.096	-6.719941775812\\
65.75	0.0966	-7.91395419541868\\
65.75	0.0972	-9.10796661502548\\
65.75	0.0978	-10.3019790346322\\
65.75	0.0984	-11.4959914542389\\
65.75	0.099	-12.6900038738456\\
65.75	0.0996	-13.8840162934523\\
65.75	0.1002	-15.0780287130591\\
65.75	0.1008	-16.2720411326658\\
65.75	0.1014	-17.4660535522725\\
65.75	0.102	-18.6600659718791\\
65.75	0.1026	-19.8540783914859\\
65.75	0.1032	-21.0480908110926\\
65.75	0.1038	-22.2421032306993\\
65.75	0.1044	-23.4361156503061\\
65.75	0.105	-24.6301280699128\\
65.75	0.1056	-25.8241404895196\\
65.75	0.1062	-27.0181529091262\\
65.75	0.1068	-28.212165328733\\
65.75	0.1074	-29.4061777483396\\
65.75	0.108	-30.6001901679463\\
65.75	0.1086	-31.7942025875531\\
65.75	0.1092	-32.9882150071597\\
65.75	0.1098	-34.1822274267665\\
65.75	0.1104	-35.3762398463732\\
65.75	0.111	-36.5702522659799\\
65.75	0.1116	-37.7642646855867\\
65.75	0.1122	-38.9582771051932\\
65.75	0.1128	-40.1522895248\\
65.75	0.1134	-41.3463019444067\\
65.75	0.114	-42.5403143640134\\
65.75	0.1146	-43.7343267836202\\
65.75	0.1152	-44.9283392032269\\
65.75	0.1158	-46.1223516228337\\
65.75	0.1164	-47.3163640424403\\
65.75	0.117	-48.510376462047\\
65.75	0.1176	-49.7043888816538\\
65.75	0.1182	-50.8984013012606\\
65.75	0.1188	-52.0924137208671\\
65.75	0.1194	-53.286426140474\\
65.75	0.12	-54.4804385600805\\
65.75	0.1206	-55.6744509796873\\
65.75	0.1212	-56.8684633992941\\
65.75	0.1218	-58.0624758189008\\
65.75	0.1224	-59.2564882385076\\
65.75	0.123	-60.4505006581142\\
66.125	0.093	-0.357300514532085\\
66.125	0.0936	-1.52787805843116\\
66.125	0.0942	-2.69845560233023\\
66.125	0.0948	-3.8690331462293\\
66.125	0.0954	-5.03961069012837\\
66.125	0.096	-6.21018823402744\\
66.125	0.0966	-7.3807657779264\\
66.125	0.0972	-8.55134332182558\\
66.125	0.0978	-9.72192086572454\\
66.125	0.0984	-10.8924984096237\\
66.125	0.099	-12.0630759535227\\
66.125	0.0996	-13.2336534974218\\
66.125	0.1002	-14.4042310413208\\
66.125	0.1008	-15.5748085852199\\
66.125	0.1014	-16.745386129119\\
66.125	0.102	-17.915963673018\\
66.125	0.1026	-19.0865412169171\\
66.125	0.1032	-20.2571187608162\\
66.125	0.1038	-21.4276963047153\\
66.125	0.1044	-22.5982738486143\\
66.125	0.105	-23.7688513925134\\
66.125	0.1056	-24.9394289364125\\
66.125	0.1062	-26.1100064803115\\
66.125	0.1068	-27.2805840242106\\
66.125	0.1074	-28.4511615681097\\
66.125	0.108	-29.6217391120086\\
66.125	0.1086	-30.7923166559078\\
66.125	0.1092	-31.9628941998068\\
66.125	0.1098	-33.133471743706\\
66.125	0.1104	-34.3040492876049\\
66.125	0.111	-35.474626831504\\
66.125	0.1116	-36.6452043754031\\
66.125	0.1122	-37.8157819193021\\
66.125	0.1128	-38.9863594632012\\
66.125	0.1134	-40.1569370071003\\
66.125	0.114	-41.3275145509992\\
66.125	0.1146	-42.4980920948984\\
66.125	0.1152	-43.6686696387974\\
66.125	0.1158	-44.8392471826966\\
66.125	0.1164	-46.0098247265956\\
66.125	0.117	-47.1804022704946\\
66.125	0.1176	-48.3509798143938\\
66.125	0.1182	-49.5215573582929\\
66.125	0.1188	-50.6921349021918\\
66.125	0.1194	-51.862712446091\\
66.125	0.12	-53.03328998999\\
66.125	0.1206	-54.203867533889\\
66.125	0.1212	-55.3744450777882\\
66.125	0.1218	-56.5450226216872\\
66.125	0.1224	-57.7156001655864\\
66.125	0.123	-58.8861777094852\\
66.5	0.093	0.0352786487143248\\
66.5	0.0936	-1.11186401947703\\
66.5	0.0942	-2.2590066876686\\
66.5	0.0948	-3.40614935585995\\
66.5	0.0954	-4.55329202405142\\
66.5	0.096	-5.70043469224277\\
66.5	0.0966	-6.84757736043412\\
66.5	0.0972	-7.99472002862558\\
66.5	0.0978	-9.14186269681693\\
66.5	0.0984	-10.2890053650085\\
66.5	0.099	-11.4361480331999\\
66.5	0.0996	-12.5832907013912\\
66.5	0.1002	-13.7304333695827\\
66.5	0.1008	-14.877576037774\\
66.5	0.1014	-16.0247187059655\\
66.5	0.102	-17.1718613741568\\
66.5	0.1026	-18.3190040423484\\
66.5	0.1032	-19.4661467105398\\
66.5	0.1038	-20.6132893787311\\
66.5	0.1044	-21.7604320469226\\
66.5	0.105	-22.9075747151139\\
66.5	0.1056	-24.0547173833054\\
66.5	0.1062	-25.2018600514969\\
66.5	0.1068	-26.3490027196883\\
66.5	0.1074	-27.4961453878797\\
66.5	0.108	-28.643288056071\\
66.5	0.1086	-29.7904307242625\\
66.5	0.1092	-30.9375733924538\\
66.5	0.1098	-32.0847160606453\\
66.5	0.1104	-33.2318587288368\\
66.5	0.111	-34.3790013970281\\
66.5	0.1116	-35.5261440652196\\
66.5	0.1122	-36.6732867334109\\
66.5	0.1128	-37.8204294016024\\
66.5	0.1134	-38.9675720697937\\
66.5	0.114	-40.1147147379852\\
66.5	0.1146	-41.2618574061767\\
66.5	0.1152	-42.409000074368\\
66.5	0.1158	-43.5561427425595\\
66.5	0.1164	-44.7032854107508\\
66.5	0.117	-45.8504280789422\\
66.5	0.1176	-46.9975707471336\\
66.5	0.1182	-48.1447134153252\\
66.5	0.1188	-49.2918560835164\\
66.5	0.1194	-50.438998751708\\
66.5	0.12	-51.5861414198993\\
66.5	0.1206	-52.7332840880907\\
66.5	0.1212	-53.8804267562822\\
66.5	0.1218	-55.0275694244735\\
66.5	0.1224	-56.1747120926651\\
66.5	0.123	-57.3218547608564\\
66.875	0.093	0.427857811960735\\
66.875	0.0936	-0.695849980523008\\
66.875	0.0942	-1.81955777300675\\
66.875	0.0948	-2.94326556549049\\
66.875	0.0954	-4.06697335797435\\
66.875	0.096	-5.19068115045809\\
66.875	0.0966	-6.31438894294172\\
66.875	0.0972	-7.43809673542557\\
66.875	0.0978	-8.56180452790932\\
66.875	0.0984	-9.68551232039317\\
66.875	0.099	-10.8092201128769\\
66.875	0.0996	-11.9329279053605\\
66.875	0.1002	-13.0566356978444\\
66.875	0.1008	-14.1803434903281\\
66.875	0.1014	-15.304051282812\\
66.875	0.102	-16.4277590752956\\
66.875	0.1026	-17.5514668677795\\
66.875	0.1032	-18.6751746602632\\
66.875	0.1038	-19.798882452747\\
66.875	0.1044	-20.9225902452307\\
66.875	0.105	-22.0462980377145\\
66.875	0.1056	-23.1700058301983\\
66.875	0.1062	-24.293713622682\\
66.875	0.1068	-25.4174214151659\\
66.875	0.1074	-26.5411292076495\\
66.875	0.108	-27.6648370001333\\
66.875	0.1086	-28.7885447926171\\
66.875	0.1092	-29.9122525851009\\
66.875	0.1098	-31.0359603775846\\
66.875	0.1104	-32.1596681700684\\
66.875	0.111	-33.2833759625521\\
66.875	0.1116	-34.407083755036\\
66.875	0.1122	-35.5307915475196\\
66.875	0.1128	-36.6544993400034\\
66.875	0.1134	-37.7782071324872\\
66.875	0.114	-38.9019149249709\\
66.875	0.1146	-40.0256227174548\\
66.875	0.1152	-41.1493305099384\\
66.875	0.1158	-42.2730383024223\\
66.875	0.1164	-43.396746094906\\
66.875	0.117	-44.5204538873897\\
66.875	0.1176	-45.6441616798735\\
66.875	0.1182	-46.7678694723573\\
66.875	0.1188	-47.891577264841\\
66.875	0.1194	-49.0152850573249\\
66.875	0.12	-50.1389928498086\\
66.875	0.1206	-51.2627006422923\\
66.875	0.1212	-52.3864084347762\\
66.875	0.1218	-53.5101162272599\\
66.875	0.1224	-54.6338240197438\\
66.875	0.123	-55.7575318122273\\
67.25	0.093	0.820436975207144\\
67.25	0.0936	-0.279835941568876\\
67.25	0.0942	-1.38010885834512\\
67.25	0.0948	-2.48038177512115\\
67.25	0.0954	-3.58065469189739\\
67.25	0.096	-4.68092760867341\\
67.25	0.0966	-5.78120052544944\\
67.25	0.0972	-6.88147344222568\\
67.25	0.0978	-7.9817463590017\\
67.25	0.0984	-9.08201927577795\\
67.25	0.099	-10.182292192554\\
67.25	0.0996	-11.28256510933\\
67.25	0.1002	-12.3828380261062\\
67.25	0.1008	-13.4831109428823\\
67.25	0.1014	-14.5833838596585\\
67.25	0.102	-15.6836567764345\\
67.25	0.1026	-16.7839296932107\\
67.25	0.1032	-17.8842026099868\\
67.25	0.1038	-18.9844755267628\\
67.25	0.1044	-20.0847484435391\\
67.25	0.105	-21.1850213603151\\
67.25	0.1056	-22.2852942770912\\
67.25	0.1062	-23.3855671938674\\
67.25	0.1068	-24.4858401106435\\
67.25	0.1074	-25.5861130274196\\
67.25	0.108	-26.6863859441956\\
67.25	0.1086	-27.7866588609718\\
67.25	0.1092	-28.8869317777479\\
67.25	0.1098	-29.9872046945241\\
67.25	0.1104	-31.0874776113002\\
67.25	0.111	-32.1877505280762\\
67.25	0.1116	-33.2880234448523\\
67.25	0.1122	-34.3882963616285\\
67.25	0.1128	-35.4885692784046\\
67.25	0.1134	-36.5888421951807\\
67.25	0.114	-37.6891151119568\\
67.25	0.1146	-38.7893880287329\\
67.25	0.1152	-39.889660945509\\
67.25	0.1158	-40.9899338622852\\
67.25	0.1164	-42.0902067790613\\
67.25	0.117	-43.1904796958373\\
67.25	0.1176	-44.2907526126135\\
67.25	0.1182	-45.3910255293897\\
67.25	0.1188	-46.4912984461656\\
67.25	0.1194	-47.591571362942\\
67.25	0.12	-48.6918442797179\\
67.25	0.1206	-49.792117196494\\
67.25	0.1212	-50.8923901132703\\
67.25	0.1218	-51.9926630300463\\
67.25	0.1224	-53.0929359468225\\
67.25	0.123	-54.1932088635984\\
67.625	0.093	1.21301613845355\\
67.625	0.0936	0.136178097385141\\
67.625	0.0942	-0.940659943683386\\
67.625	0.0948	-2.0174979847518\\
67.625	0.0954	-3.09433602582033\\
67.625	0.096	-4.17117406688874\\
67.625	0.0966	-5.24801210795715\\
67.625	0.0972	-6.32485014902568\\
67.625	0.0978	-7.40168819009409\\
67.625	0.0984	-8.47852623116262\\
67.625	0.099	-9.55536427223115\\
67.625	0.0996	-10.6322023132994\\
67.625	0.1002	-11.7090403543681\\
67.625	0.1008	-12.7858783954365\\
67.625	0.1014	-13.862716436505\\
67.625	0.102	-14.9395544775734\\
67.625	0.1026	-16.016392518642\\
67.625	0.1032	-17.0932305597104\\
67.625	0.1038	-18.1700686007788\\
67.625	0.1044	-19.2469066418473\\
67.625	0.105	-20.3237446829157\\
67.625	0.1056	-21.4005827239843\\
67.625	0.1062	-22.4774207650527\\
67.625	0.1068	-23.5542588061212\\
67.625	0.1074	-24.6310968471896\\
67.625	0.108	-25.707934888258\\
67.625	0.1086	-26.7847729293266\\
67.625	0.1092	-27.861610970395\\
67.625	0.1098	-28.9384490114635\\
67.625	0.1104	-30.0152870525319\\
67.625	0.111	-31.0921250936003\\
67.625	0.1116	-32.1689631346688\\
67.625	0.1122	-33.2458011757373\\
67.625	0.1128	-34.3226392168058\\
67.625	0.1134	-35.3994772578742\\
67.625	0.114	-36.4763152989426\\
67.625	0.1146	-37.5531533400111\\
67.625	0.1152	-38.6299913810795\\
67.625	0.1158	-39.7068294221481\\
67.625	0.1164	-40.7836674632165\\
67.625	0.117	-41.8605055042849\\
67.625	0.1176	-42.9373435453534\\
67.625	0.1182	-44.014181586422\\
67.625	0.1188	-45.0910196274903\\
67.625	0.1194	-46.1678576685589\\
67.625	0.12	-47.2446957096272\\
67.625	0.1206	-48.3215337506957\\
67.625	0.1212	-49.3983717917644\\
67.625	0.1218	-50.4752098328327\\
67.625	0.1224	-51.5520478739012\\
67.625	0.123	-52.6288859149696\\
68	0.093	1.60559530169996\\
68	0.0936	0.552192136339158\\
68	0.0942	-0.501211029021761\\
68	0.0948	-1.55461419438245\\
68	0.0954	-2.60801735974337\\
68	0.096	-3.66142052510418\\
68	0.0966	-4.71482369046487\\
68	0.0972	-5.76822685582579\\
68	0.0978	-6.82163002118659\\
68	0.0984	-7.8750331865474\\
68	0.099	-8.9284363519082\\
68	0.0996	-9.98183951726901\\
68	0.1002	-11.0352426826298\\
68	0.1008	-12.0886458479906\\
68	0.1014	-13.1420490133515\\
68	0.102	-14.1954521787122\\
68	0.1026	-15.2488553440731\\
68	0.1032	-16.302258509434\\
68	0.1038	-17.3556616747946\\
68	0.1044	-18.4090648401556\\
68	0.105	-19.4624680055163\\
68	0.1056	-20.5158711708772\\
68	0.1062	-21.569274336238\\
68	0.1068	-22.6226775015989\\
68	0.1074	-23.6760806669596\\
68	0.108	-24.7294838323204\\
68	0.1086	-25.7828869976812\\
68	0.1092	-26.836290163042\\
68	0.1098	-27.8896933284029\\
68	0.1104	-28.9430964937636\\
68	0.111	-29.9964996591244\\
68	0.1116	-31.0499028244853\\
68	0.1122	-32.103305989846\\
68	0.1128	-33.156709155207\\
68	0.1134	-34.2101123205678\\
68	0.114	-35.2635154859285\\
68	0.1146	-36.3169186512894\\
68	0.1152	-37.3703218166502\\
68	0.1158	-38.423724982011\\
68	0.1164	-39.4771281473718\\
68	0.117	-40.5305313127326\\
68	0.1176	-41.5839344780934\\
68	0.1182	-42.6373376434543\\
68	0.1188	-43.690740808815\\
68	0.1194	-44.7441439741759\\
68	0.12	-45.7975471395366\\
68	0.1206	-46.8509503048975\\
68	0.1212	-47.9043534702583\\
68	0.1218	-48.9577566356191\\
68	0.1224	-50.01115980098\\
68	0.123	-51.0645629663406\\
68.375	0.093	1.99817446494626\\
68.375	0.0936	0.968206175293176\\
68.375	0.0942	-0.0617621143600218\\
68.375	0.0948	-1.09173040401311\\
68.375	0.0954	-2.12169869366642\\
68.375	0.096	-3.1516669833195\\
68.375	0.0966	-4.18163527297258\\
68.375	0.0972	-5.2116035626259\\
68.375	0.0978	-6.24157185227898\\
68.375	0.0984	-7.27154014193218\\
68.375	0.099	-8.30150843158538\\
68.375	0.0996	-9.33147672123846\\
68.375	0.1002	-10.3614450108917\\
68.375	0.1008	-11.3914133005447\\
68.375	0.1014	-12.4213815901981\\
68.375	0.102	-13.4513498798511\\
68.375	0.1026	-14.4813181695043\\
68.375	0.1032	-15.5112864591574\\
68.375	0.1038	-16.5412547488106\\
68.375	0.1044	-17.5712230384638\\
68.375	0.105	-18.6011913281169\\
68.375	0.1056	-19.6311596177702\\
68.375	0.1062	-20.6611279074233\\
68.375	0.1068	-21.6910961970765\\
68.375	0.1074	-22.7210644867296\\
68.375	0.108	-23.7510327763828\\
68.375	0.1086	-24.781001066036\\
68.375	0.1092	-25.8109693556891\\
68.375	0.1098	-26.8409376453423\\
68.375	0.1104	-27.8709059349954\\
68.375	0.111	-28.9008742246485\\
68.375	0.1116	-29.9308425143017\\
68.375	0.1122	-30.9608108039549\\
68.375	0.1128	-31.9907790936081\\
68.375	0.1134	-33.0207473832612\\
68.375	0.114	-34.0507156729144\\
68.375	0.1146	-35.0806839625676\\
68.375	0.1152	-36.1106522522207\\
68.375	0.1158	-37.1406205418739\\
68.375	0.1164	-38.1705888315271\\
68.375	0.117	-39.2005571211802\\
68.375	0.1176	-40.2305254108334\\
68.375	0.1182	-41.2604937004866\\
68.375	0.1188	-42.2904619901396\\
68.375	0.1194	-43.320430279793\\
68.375	0.12	-44.3503985694459\\
68.375	0.1206	-45.3803668590992\\
68.375	0.1212	-46.4103351487524\\
68.375	0.1218	-47.4403034384055\\
68.375	0.1224	-48.4702717280587\\
68.375	0.123	-49.5002400177118\\
68.75	0.093	2.39075362819278\\
68.75	0.0936	1.38422021424731\\
68.75	0.0942	0.377686800301717\\
68.75	0.0948	-0.628846613643759\\
68.75	0.0954	-1.63538002758935\\
68.75	0.096	-2.64191344153483\\
68.75	0.0966	-3.6484468554803\\
68.75	0.0972	-4.65498026942578\\
68.75	0.0978	-5.66151368337125\\
68.75	0.0984	-6.66804709731684\\
68.75	0.099	-7.67458051126232\\
68.75	0.0996	-8.6811139252078\\
68.75	0.1002	-9.68764733915339\\
68.75	0.1008	-10.6941807530989\\
68.75	0.1014	-11.7007141670445\\
68.75	0.102	-12.7072475809899\\
68.75	0.1026	-13.7137809949355\\
68.75	0.1032	-14.7203144088809\\
68.75	0.1038	-15.7268478228264\\
68.75	0.1044	-16.7333812367719\\
68.75	0.105	-17.7399146507174\\
68.75	0.1056	-18.746448064663\\
68.75	0.1062	-19.7529814786085\\
68.75	0.1068	-20.7595148925541\\
68.75	0.1074	-21.7660483064996\\
68.75	0.108	-22.772581720445\\
68.75	0.1086	-23.7791151343906\\
68.75	0.1092	-24.785648548336\\
68.75	0.1098	-25.7921819622816\\
68.75	0.1104	-26.7987153762271\\
68.75	0.111	-27.8052487901725\\
68.75	0.1116	-28.8117822041181\\
68.75	0.1122	-29.8183156180636\\
68.75	0.1128	-30.8248490320092\\
68.75	0.1134	-31.8313824459547\\
68.75	0.114	-32.8379158599001\\
68.75	0.1146	-33.8444492738457\\
68.75	0.1152	-34.8509826877912\\
68.75	0.1158	-35.8575161017367\\
68.75	0.1164	-36.8640495156822\\
68.75	0.117	-37.8705829296276\\
68.75	0.1176	-38.8771163435732\\
68.75	0.1182	-39.8836497575188\\
68.75	0.1188	-40.8901831714642\\
68.75	0.1194	-41.8967165854099\\
68.75	0.12	-42.9032499993552\\
68.75	0.1206	-43.9097834133008\\
68.75	0.1212	-44.9163168272464\\
68.75	0.1218	-45.9228502411918\\
68.75	0.1224	-46.9293836551374\\
68.75	0.123	-47.9359170690827\\
69.125	0.093	2.78333279143931\\
69.125	0.0936	1.80023425320144\\
69.125	0.0942	0.81713571496357\\
69.125	0.0948	-0.165962823274299\\
69.125	0.0954	-1.14906136151217\\
69.125	0.096	-2.13215989975004\\
69.125	0.0966	-3.1152584379879\\
69.125	0.0972	-4.09835697622577\\
69.125	0.0978	-5.08145551446364\\
69.125	0.0984	-6.06455405270151\\
69.125	0.099	-7.04765259093938\\
69.125	0.0996	-8.03075112917713\\
69.125	0.1002	-9.01384966741512\\
69.125	0.1008	-9.99694820565287\\
69.125	0.1014	-10.9800467438909\\
69.125	0.102	-11.9631452821286\\
69.125	0.1026	-12.9462438203666\\
69.125	0.1032	-13.9293423586043\\
69.125	0.1038	-14.9124408968422\\
69.125	0.1044	-15.8955394350801\\
69.125	0.105	-16.878637973318\\
69.125	0.1056	-17.8617365115558\\
69.125	0.1062	-18.8448350497937\\
69.125	0.1068	-19.8279335880317\\
69.125	0.1074	-20.8110321262694\\
69.125	0.108	-21.7941306645073\\
69.125	0.1086	-22.7772292027452\\
69.125	0.1092	-23.760327740983\\
69.125	0.1098	-24.7434262792209\\
69.125	0.1104	-25.7265248174588\\
69.125	0.111	-26.7096233556965\\
69.125	0.1116	-27.6927218939345\\
69.125	0.1122	-28.6758204321723\\
69.125	0.1128	-29.6589189704102\\
69.125	0.1134	-30.642017508648\\
69.125	0.114	-31.6251160468859\\
69.125	0.1146	-32.6082145851238\\
69.125	0.1152	-33.5913131233616\\
69.125	0.1158	-34.5744116615996\\
69.125	0.1164	-35.5575101998373\\
69.125	0.117	-36.5406087380752\\
69.125	0.1176	-37.5237072763131\\
69.125	0.1182	-38.5068058145511\\
69.125	0.1188	-39.4899043527887\\
69.125	0.1194	-40.4730028910268\\
69.125	0.12	-41.4561014292644\\
69.125	0.1206	-42.4391999675024\\
69.125	0.1212	-43.4222985057403\\
69.125	0.1218	-44.4053970439782\\
69.125	0.1224	-45.388495582216\\
69.125	0.123	-46.3715941204538\\
69.5	0.093	3.1759119546856\\
69.5	0.0936	2.21624829215546\\
69.5	0.0942	1.25658462962519\\
69.5	0.0948	0.296920967095048\\
69.5	0.0954	-0.662742695435213\\
69.5	0.096	-1.62240635796536\\
69.5	0.0966	-2.58207002049562\\
69.5	0.0972	-3.54173368302588\\
69.5	0.0978	-4.50139734555603\\
69.5	0.0984	-5.46106100808629\\
69.5	0.099	-6.42072467061644\\
69.5	0.0996	-7.38038833314658\\
69.5	0.1002	-8.34005199567684\\
69.5	0.1008	-9.29971565820711\\
69.5	0.1014	-10.2593793207374\\
69.5	0.102	-11.2190429832675\\
69.5	0.1026	-12.1787066457978\\
69.5	0.1032	-13.1383703083279\\
69.5	0.1038	-14.0980339708581\\
69.5	0.1044	-15.0576976333884\\
69.5	0.105	-16.0173612959186\\
69.5	0.1056	-16.9770249584489\\
69.5	0.1062	-17.936688620979\\
69.5	0.1068	-18.8963522835093\\
69.5	0.1074	-19.8560159460394\\
69.5	0.108	-20.8156796085697\\
69.5	0.1086	-21.7753432710999\\
69.5	0.1092	-22.7350069336301\\
69.5	0.1098	-23.6946705961603\\
69.5	0.1104	-24.6543342586905\\
69.5	0.111	-25.6139979212206\\
69.5	0.1116	-26.573661583751\\
69.5	0.1122	-27.5333252462812\\
69.5	0.1128	-28.4929889088114\\
69.5	0.1134	-29.4526525713416\\
69.5	0.114	-30.4123162338717\\
69.5	0.1146	-31.371979896402\\
69.5	0.1152	-32.3316435589321\\
69.5	0.1158	-33.2913072214625\\
69.5	0.1164	-34.2509708839926\\
69.5	0.117	-35.2106345465228\\
69.5	0.1176	-36.170298209053\\
69.5	0.1182	-37.1299618715833\\
69.5	0.1188	-38.0896255341133\\
69.5	0.1194	-39.0492891966437\\
69.5	0.12	-40.0089528591739\\
69.5	0.1206	-40.9686165217041\\
69.5	0.1212	-41.9282801842344\\
69.5	0.1218	-42.8879438467645\\
69.5	0.1224	-43.8476075092948\\
69.5	0.123	-44.8072711718248\\
69.875	0.093	3.56849111793201\\
69.875	0.0936	2.63226233110947\\
69.875	0.0942	1.69603354428693\\
69.875	0.0948	0.759804757464394\\
69.875	0.0954	-0.176424029358259\\
69.875	0.096	-1.1126528161808\\
69.875	0.0966	-2.04888160300334\\
69.875	0.0972	-2.98511038982588\\
69.875	0.0978	-3.92133917664842\\
69.875	0.0984	-4.85756796347107\\
69.875	0.099	-5.79379675029361\\
69.875	0.0996	-6.73002553711603\\
69.875	0.1002	-7.66625432393869\\
69.875	0.1008	-8.60248311076123\\
69.875	0.1014	-9.53871189758388\\
69.875	0.102	-10.4749406844063\\
69.875	0.1026	-11.411169471229\\
69.875	0.1032	-12.3473982580515\\
69.875	0.1038	-13.283627044874\\
69.875	0.1044	-14.2198558316967\\
69.875	0.105	-15.1560846185191\\
69.875	0.1056	-16.0923134053418\\
69.875	0.1062	-17.0285421921643\\
69.875	0.1068	-17.964770978987\\
69.875	0.1074	-18.9009997658095\\
69.875	0.108	-19.8372285526319\\
69.875	0.1086	-20.7734573394546\\
69.875	0.1092	-21.7096861262771\\
69.875	0.1098	-22.6459149130998\\
69.875	0.1104	-23.5821436999223\\
69.875	0.111	-24.5183724867447\\
69.875	0.1116	-25.4546012735674\\
69.875	0.1122	-26.3908300603899\\
69.875	0.1128	-27.3270588472126\\
69.875	0.1134	-28.263287634035\\
69.875	0.114	-29.1995164208575\\
69.875	0.1146	-30.1357452076802\\
69.875	0.1152	-31.0719739945027\\
69.875	0.1158	-32.0082027813254\\
69.875	0.1164	-32.9444315681478\\
69.875	0.117	-33.8806603549704\\
69.875	0.1176	-34.816889141793\\
69.875	0.1182	-35.7531179286157\\
69.875	0.1188	-36.6893467154381\\
69.875	0.1194	-37.6255755022607\\
69.875	0.12	-38.5618042890832\\
69.875	0.1206	-39.4980330759058\\
69.875	0.1212	-40.4342618627285\\
69.875	0.1218	-41.3704906495509\\
69.875	0.1224	-42.3067194363736\\
69.875	0.123	-43.242948223196\\
70.25	0.093	3.96107028117842\\
70.25	0.0936	3.0482763700636\\
70.25	0.0942	2.13548245894856\\
70.25	0.0948	1.22268854783374\\
70.25	0.0954	0.309894636718695\\
70.25	0.096	-0.602899274396123\\
70.25	0.0966	-1.51569318551105\\
70.25	0.0972	-2.42848709662599\\
70.25	0.0978	-3.3412810077408\\
70.25	0.0984	-4.25407491885585\\
70.25	0.099	-5.16686882997067\\
70.25	0.0996	-6.07966274108549\\
70.25	0.1002	-6.99245665220053\\
70.25	0.1008	-7.90525056331535\\
70.25	0.1014	-8.81804447443039\\
70.25	0.102	-9.73083838554521\\
70.25	0.1026	-10.6436322966603\\
70.25	0.1032	-11.5564262077751\\
70.25	0.1038	-12.4692201188899\\
70.25	0.1044	-13.3820140300049\\
70.25	0.105	-14.2948079411198\\
70.25	0.1056	-15.2076018522348\\
70.25	0.1062	-16.1203957633496\\
70.25	0.1068	-17.0331896744647\\
70.25	0.1074	-17.9459835855795\\
70.25	0.108	-18.8587774966943\\
70.25	0.1086	-19.7715714078093\\
70.25	0.1092	-20.6843653189242\\
70.25	0.1098	-21.5971592300392\\
70.25	0.1104	-22.509953141154\\
70.25	0.111	-23.4227470522688\\
70.25	0.1116	-24.3355409633839\\
70.25	0.1122	-25.2483348744987\\
70.25	0.1128	-26.1611287856138\\
70.25	0.1134	-27.0739226967286\\
70.25	0.114	-27.9867166078434\\
70.25	0.1146	-28.8995105189584\\
70.25	0.1152	-29.8123044300733\\
70.25	0.1158	-30.7250983411883\\
70.25	0.1164	-31.6378922523031\\
70.25	0.117	-32.5506861634179\\
70.25	0.1176	-33.463480074533\\
70.25	0.1182	-34.3762739856479\\
70.25	0.1188	-35.2890678967627\\
70.25	0.1194	-36.2018618078778\\
70.25	0.12	-37.1146557189926\\
70.25	0.1206	-38.0274496301075\\
70.25	0.1212	-38.9402435412225\\
70.25	0.1218	-39.8530374523374\\
70.25	0.1224	-40.7658313634523\\
70.25	0.123	-41.6786252745671\\
70.625	0.093	4.35364944442495\\
70.625	0.0936	3.46429040901762\\
70.625	0.0942	2.5749313736103\\
70.625	0.0948	1.68557233820309\\
70.625	0.0954	0.796213302795763\\
70.625	0.096	-0.0931457326114469\\
70.625	0.0966	-0.982504768018657\\
70.625	0.0972	-1.87186380342598\\
70.625	0.0978	-2.76122283883319\\
70.625	0.0984	-3.65058187424052\\
70.625	0.099	-4.53994090964773\\
70.625	0.0996	-5.42929994505494\\
70.625	0.1002	-6.31865898046226\\
70.625	0.1008	-7.20801801586947\\
70.625	0.1014	-8.09737705127679\\
70.625	0.102	-8.986736086684\\
70.625	0.1026	-9.87609512209133\\
70.625	0.1032	-10.7654541574985\\
70.625	0.1038	-11.6548131929057\\
70.625	0.1044	-12.5441722283131\\
70.625	0.105	-13.4335312637203\\
70.625	0.1056	-14.3228902991276\\
70.625	0.1062	-15.2122493345348\\
70.625	0.1068	-16.1016083699421\\
70.625	0.1074	-16.9909674053494\\
70.625	0.108	-17.8803264407566\\
70.625	0.1086	-18.7696854761639\\
70.625	0.1092	-19.6590445115711\\
70.625	0.1098	-20.5484035469784\\
70.625	0.1104	-21.4377625823856\\
70.625	0.111	-22.327121617793\\
70.625	0.1116	-23.2164806532003\\
70.625	0.1122	-24.1058396886075\\
70.625	0.1128	-24.9951987240148\\
70.625	0.1134	-25.884557759422\\
70.625	0.114	-26.7739167948292\\
70.625	0.1146	-27.6632758302366\\
70.625	0.1152	-28.5526348656438\\
70.625	0.1158	-29.4419939010511\\
70.625	0.1164	-30.3313529364583\\
70.625	0.117	-31.2207119718655\\
70.625	0.1176	-32.1100710072728\\
70.625	0.1182	-32.9994300426802\\
70.625	0.1188	-33.8887890780873\\
70.625	0.1194	-34.7781481134947\\
70.625	0.12	-35.6675071489018\\
70.625	0.1206	-36.5568661843091\\
70.625	0.1212	-37.4462252197164\\
70.625	0.1218	-38.3355842551236\\
70.625	0.1224	-39.224943290531\\
70.625	0.123	-40.1143023259381\\
71	0.093	4.74622860767124\\
71	0.0936	3.88030444797175\\
71	0.0942	3.01438028827204\\
71	0.0948	2.14845612857243\\
71	0.0954	1.28253196887283\\
71	0.096	0.416607809173229\\
71	0.0966	-0.449316350526374\\
71	0.0972	-1.31524051022598\\
71	0.0978	-2.18116466992558\\
71	0.0984	-3.0470888296253\\
71	0.099	-3.91301298932478\\
71	0.0996	-4.77893714902439\\
71	0.1002	-5.6448613087241\\
71	0.1008	-6.51078546842359\\
71	0.1014	-7.37670962812331\\
71	0.102	-8.24263378782291\\
71	0.1026	-9.10855794752251\\
71	0.1032	-9.97448210722212\\
71	0.1038	-10.8404062669217\\
71	0.1044	-11.7063304266213\\
71	0.105	-12.5722545863209\\
71	0.1056	-13.4381787460206\\
71	0.1062	-14.3041029057201\\
71	0.1068	-15.1700270654198\\
71	0.1074	-16.0359512251194\\
71	0.108	-16.9018753848189\\
71	0.1086	-17.7677995445187\\
71	0.1092	-18.6337237042181\\
71	0.1098	-19.4996478639179\\
71	0.1104	-20.3655720236175\\
71	0.111	-21.2314961833171\\
71	0.1116	-22.0974203430167\\
71	0.1122	-22.9633445027163\\
71	0.1128	-23.829268662416\\
71	0.1134	-24.6951928221155\\
71	0.114	-25.5611169818151\\
71	0.1146	-26.4270411415148\\
71	0.1152	-27.2929653012143\\
71	0.1158	-28.158889460914\\
71	0.1164	-29.0248136206136\\
71	0.117	-29.8907377803131\\
71	0.1176	-30.7566619400128\\
71	0.1182	-31.6225860997124\\
71	0.1188	-32.4885102594119\\
71	0.1194	-33.3544344191117\\
71	0.12	-34.2203585788111\\
71	0.1206	-35.0862827385108\\
71	0.1212	-35.9522068982105\\
71	0.1218	-36.81813105791\\
71	0.1224	-37.6840552176097\\
71	0.123	-38.5499793773092\\
71.375	0.093	5.13880777091765\\
71.375	0.0936	4.29631848692577\\
71.375	0.0942	3.45382920293366\\
71.375	0.0948	2.61133991894178\\
71.375	0.0954	1.76885063494979\\
71.375	0.096	0.926361350957905\\
71.375	0.0966	0.0838720669659097\\
71.375	0.0972	-0.758617217026085\\
71.375	0.0978	-1.60110650101797\\
71.375	0.0984	-2.44359578501007\\
71.375	0.099	-3.28608506900196\\
71.375	0.0996	-4.12857435299384\\
71.375	0.1002	-4.97106363698583\\
71.375	0.1008	-5.81355292097783\\
71.375	0.1014	-6.65604220496982\\
71.375	0.102	-7.4985314889617\\
71.375	0.1026	-8.3410207729537\\
71.375	0.1032	-9.18351005694569\\
71.375	0.1038	-10.0259993409376\\
71.375	0.1044	-10.8684886249296\\
71.375	0.105	-11.7109779089216\\
71.375	0.1056	-12.5534671929136\\
71.375	0.1062	-13.3959564769054\\
71.375	0.1068	-14.2384457608975\\
71.375	0.1074	-15.0809350448894\\
71.375	0.108	-15.9234243288813\\
71.375	0.1086	-16.7659136128733\\
71.375	0.1092	-17.6084028968653\\
71.375	0.1098	-18.4508921808573\\
71.375	0.1104	-19.2933814648492\\
71.375	0.111	-20.1358707488412\\
71.375	0.1116	-20.9783600328332\\
71.375	0.1122	-21.820849316825\\
71.375	0.1128	-22.663338600817\\
71.375	0.1134	-23.505827884809\\
71.375	0.114	-24.3483171688009\\
71.375	0.1146	-25.1908064527929\\
71.375	0.1152	-26.0332957367849\\
71.375	0.1158	-26.8757850207769\\
71.375	0.1164	-27.7182743047688\\
71.375	0.117	-28.5607635887607\\
71.375	0.1176	-29.4032528727528\\
71.375	0.1182	-30.2457421567448\\
71.375	0.1188	-31.0882314407365\\
71.375	0.1194	-31.9307207247288\\
71.375	0.12	-32.7732100087205\\
71.375	0.1206	-33.6156992927125\\
71.375	0.1212	-34.4581885767046\\
71.375	0.1218	-35.3006778606965\\
71.375	0.1224	-36.1431671446885\\
71.375	0.123	-36.9856564286803\\
71.75	0.093	5.53138693416406\\
71.75	0.0936	4.71233252587979\\
71.75	0.0942	3.8932781175954\\
71.75	0.0948	3.07422370931113\\
71.75	0.0954	2.25516930102674\\
71.75	0.096	1.43611489274247\\
71.75	0.0966	0.617060484458193\\
71.75	0.0972	-0.20199392382608\\
71.75	0.0978	-1.02104833211035\\
71.75	0.0984	-1.84010274039474\\
71.75	0.099	-2.65915714867901\\
71.75	0.0996	-3.47821155696329\\
71.75	0.1002	-4.29726596524768\\
71.75	0.1008	-5.11632037353195\\
71.75	0.1014	-5.93537478181634\\
71.75	0.102	-6.75442919010061\\
71.75	0.1026	-7.573483598385\\
71.75	0.1032	-8.39253800666927\\
71.75	0.1038	-9.21159241495354\\
71.75	0.1044	-10.0306468232378\\
71.75	0.105	-10.8497012315221\\
71.75	0.1056	-11.6687556398065\\
71.75	0.1062	-12.4878100480908\\
71.75	0.1068	-13.3068644563751\\
71.75	0.1074	-14.1259188646594\\
71.75	0.108	-14.9449732729437\\
71.75	0.1086	-15.7640276812281\\
71.75	0.1092	-16.5830820895123\\
71.75	0.1098	-17.4021364977967\\
71.75	0.1104	-18.221190906081\\
71.75	0.111	-19.0402453143653\\
71.75	0.1116	-19.8592997226497\\
71.75	0.1122	-20.6783541309338\\
71.75	0.1128	-21.4974085392182\\
71.75	0.1134	-22.3164629475025\\
71.75	0.114	-23.1355173557868\\
71.75	0.1146	-23.9545717640711\\
71.75	0.1152	-24.7736261723554\\
71.75	0.1158	-25.5926805806398\\
71.75	0.1164	-26.4117349889241\\
71.75	0.117	-27.2307893972084\\
71.75	0.1176	-28.0498438054927\\
71.75	0.1182	-28.8688982137771\\
71.75	0.1188	-29.6879526220613\\
71.75	0.1194	-30.5070070303457\\
71.75	0.12	-31.3260614386298\\
71.75	0.1206	-32.1451158469142\\
71.75	0.1212	-32.9641702551986\\
71.75	0.1218	-33.7832246634829\\
71.75	0.1224	-34.6022790717673\\
71.75	0.123	-35.4213334800514\\
72.125	0.093	5.92396609741047\\
72.125	0.0936	5.12834656483381\\
72.125	0.0942	4.33272703225714\\
72.125	0.0948	3.53710749968047\\
72.125	0.0954	2.74148796710381\\
72.125	0.096	1.94586843452714\\
72.125	0.0966	1.15024890195048\\
72.125	0.0972	0.354629369373811\\
72.125	0.0978	-0.440990163202855\\
72.125	0.0984	-1.23660969577952\\
72.125	0.099	-2.03222922835619\\
72.125	0.0996	-2.82784876093274\\
72.125	0.1002	-3.62346829350952\\
72.125	0.1008	-4.41908782608607\\
72.125	0.1014	-5.21470735866285\\
72.125	0.102	-6.0103268912394\\
72.125	0.1026	-6.80594642381618\\
72.125	0.1032	-7.60156595639285\\
72.125	0.1038	-8.3971854889694\\
72.125	0.1044	-9.19280502154618\\
72.125	0.105	-9.98842455412273\\
72.125	0.1056	-10.7840440866995\\
72.125	0.1062	-11.5796636192761\\
72.125	0.1068	-12.3752831518528\\
72.125	0.1074	-13.1709026844294\\
72.125	0.108	-13.9665222170061\\
72.125	0.1086	-14.7621417495827\\
72.125	0.1092	-15.5577612821594\\
72.125	0.1098	-16.3533808147362\\
72.125	0.1104	-17.1490003473127\\
72.125	0.111	-17.9446198798894\\
72.125	0.1116	-18.7402394124661\\
72.125	0.1122	-19.5358589450427\\
72.125	0.1128	-20.3314784776194\\
72.125	0.1134	-21.1270980101961\\
72.125	0.114	-21.9227175427726\\
72.125	0.1146	-22.7183370753494\\
72.125	0.1152	-23.5139566079259\\
72.125	0.1158	-24.3095761405027\\
72.125	0.1164	-25.1051956730794\\
72.125	0.117	-25.9008152056559\\
72.125	0.1176	-26.6964347382327\\
72.125	0.1182	-27.4920542708094\\
72.125	0.1188	-28.2876738033859\\
72.125	0.1194	-29.0832933359627\\
72.125	0.12	-29.8789128685393\\
72.125	0.1206	-30.6745324011159\\
72.125	0.1212	-31.4701519336927\\
72.125	0.1218	-32.2657714662693\\
72.125	0.1224	-33.061390998846\\
72.125	0.123	-33.8570105314226\\
72.5	0.093	6.31654526065688\\
72.5	0.0936	5.54436060378794\\
72.5	0.0942	4.77217594691888\\
72.5	0.0948	3.99999129004993\\
72.5	0.0954	3.22780663318088\\
72.5	0.096	2.45562197631182\\
72.5	0.0966	1.68343731944287\\
72.5	0.0972	0.911252662573816\\
72.5	0.0978	0.139068005704871\\
72.5	0.0984	-0.633116651164187\\
72.5	0.099	-1.40530130803313\\
72.5	0.0996	-2.17748596490219\\
72.5	0.1002	-2.94967062177125\\
72.5	0.1008	-3.72185527864019\\
72.5	0.1014	-4.49403993550925\\
72.5	0.102	-5.26622459237819\\
72.5	0.1026	-6.03840924924725\\
72.5	0.1032	-6.81059390611631\\
72.5	0.1038	-7.58277856298525\\
72.5	0.1044	-8.35496321985431\\
72.5	0.105	-9.12714787672326\\
72.5	0.1056	-9.89933253359231\\
72.5	0.1062	-10.6715171904613\\
72.5	0.1068	-11.4437018473304\\
72.5	0.1074	-12.2158865041994\\
72.5	0.108	-12.9880711610683\\
72.5	0.1086	-13.7602558179374\\
72.5	0.1092	-14.5324404748063\\
72.5	0.1098	-15.3046251316754\\
72.5	0.1104	-16.0768097885444\\
72.5	0.111	-16.8489944454134\\
72.5	0.1116	-17.6211791022824\\
72.5	0.1122	-18.3933637591514\\
72.5	0.1128	-19.1655484160204\\
72.5	0.1134	-19.9377330728894\\
72.5	0.114	-20.7099177297584\\
72.5	0.1146	-21.4821023866275\\
72.5	0.1152	-22.2542870434964\\
72.5	0.1158	-23.0264717003655\\
72.5	0.1164	-23.7986563572344\\
72.5	0.117	-24.5708410141034\\
72.5	0.1176	-25.3430256709726\\
72.5	0.1182	-26.1152103278416\\
72.5	0.1188	-26.8873949847105\\
72.5	0.1194	-27.6595796415796\\
72.5	0.12	-28.4317642984485\\
72.5	0.1206	-29.2039489553175\\
72.5	0.1212	-29.9761336121867\\
72.5	0.1218	-30.7483182690556\\
72.5	0.1224	-31.5205029259247\\
72.5	0.123	-32.2926875827935\\
72.875	0.093	6.70912442390329\\
72.875	0.0936	5.96037464274195\\
72.875	0.0942	5.2116248615805\\
72.875	0.0948	4.46287508041928\\
72.875	0.0954	3.71412529925783\\
72.875	0.096	2.96537551809649\\
72.875	0.0966	2.21662573693516\\
72.875	0.0972	1.46787595577371\\
72.875	0.0978	0.719126174612484\\
72.875	0.0984	-0.0296236065489666\\
72.875	0.099	-0.778373387710303\\
72.875	0.0996	-1.52712316887164\\
72.875	0.1002	-2.27587295003309\\
72.875	0.1008	-3.02462273119431\\
72.875	0.1014	-3.77337251235576\\
72.875	0.102	-4.5221222935171\\
72.875	0.1026	-5.27087207467855\\
72.875	0.1032	-6.01962185583977\\
72.875	0.1038	-6.76837163700111\\
72.875	0.1044	-7.51712141816256\\
72.875	0.105	-8.2658711993239\\
72.875	0.1056	-9.01462098048535\\
72.875	0.1062	-9.76337076164657\\
72.875	0.1068	-10.512120542808\\
72.875	0.1074	-11.2608703239694\\
72.875	0.108	-12.0096201051307\\
72.875	0.1086	-12.7583698862921\\
72.875	0.1092	-13.5071196674534\\
72.875	0.1098	-14.2558694486148\\
72.875	0.1104	-15.0046192297762\\
72.875	0.111	-15.7533690109375\\
72.875	0.1116	-16.5021187920989\\
72.875	0.1122	-17.2508685732602\\
72.875	0.1128	-17.9996183544216\\
72.875	0.1134	-18.748368135583\\
72.875	0.114	-19.4971179167443\\
72.875	0.1146	-20.2458676979057\\
72.875	0.1152	-20.994617479067\\
72.875	0.1158	-21.7433672602284\\
72.875	0.1164	-22.4921170413897\\
72.875	0.117	-23.2408668225511\\
72.875	0.1176	-23.9896166037125\\
72.875	0.1182	-24.7383663848739\\
72.875	0.1188	-25.4871161660351\\
72.875	0.1194	-26.2358659471967\\
72.875	0.12	-26.9846157283579\\
72.875	0.1206	-27.7333655095193\\
72.875	0.1212	-28.4821152906807\\
72.875	0.1218	-29.230865071842\\
72.875	0.1224	-29.9796148530035\\
72.875	0.123	-30.7283646341647\\
73.25	0.093	7.1017035871497\\
73.25	0.0936	6.37638868169597\\
73.25	0.0942	5.65107377624224\\
73.25	0.0948	4.92575887078863\\
73.25	0.0954	4.20044396533478\\
73.25	0.096	3.47512905988117\\
73.25	0.0966	2.74981415442744\\
73.25	0.0972	2.02449924897371\\
73.25	0.0978	1.29918434351998\\
73.25	0.0984	0.573869438066254\\
73.25	0.099	-0.151445467387362\\
73.25	0.0996	-0.87676037284109\\
73.25	0.1002	-1.60207527829482\\
73.25	0.1008	-2.32739018374855\\
73.25	0.1014	-3.05270508920228\\
73.25	0.102	-3.77801999465601\\
73.25	0.1026	-4.50333490010973\\
73.25	0.1032	-5.22864980556335\\
73.25	0.1038	-5.95396471101708\\
73.25	0.1044	-6.67927961647081\\
73.25	0.105	-7.40459452192454\\
73.25	0.1056	-8.12990942737827\\
73.25	0.1062	-8.85522433283188\\
73.25	0.1068	-9.58053923828572\\
73.25	0.1074	-10.3058541437393\\
73.25	0.108	-11.0311690491931\\
73.25	0.1086	-11.7564839546468\\
73.25	0.1092	-12.4817988601005\\
73.25	0.1098	-13.2071137655543\\
73.25	0.1104	-13.9324286710079\\
73.25	0.111	-14.6577435764616\\
73.25	0.1116	-15.3830584819153\\
73.25	0.1122	-16.1083733873691\\
73.25	0.1128	-16.8336882928228\\
73.25	0.1134	-17.5590031982765\\
73.25	0.114	-18.2843181037301\\
73.25	0.1146	-19.0096330091839\\
73.25	0.1152	-19.7349479146376\\
73.25	0.1158	-20.4602628200913\\
73.25	0.1164	-21.185577725545\\
73.25	0.117	-21.9108926309987\\
73.25	0.1176	-22.6362075364524\\
73.25	0.1182	-23.3615224419062\\
73.25	0.1188	-24.0868373473597\\
73.25	0.1194	-24.8121522528137\\
73.25	0.12	-25.5374671582672\\
73.25	0.1206	-26.262782063721\\
73.25	0.1212	-26.9880969691748\\
73.25	0.1218	-27.7134118746284\\
73.25	0.1224	-28.4387267800822\\
73.25	0.123	-29.1640416855357\\
73.625	0.093	7.49428275039622\\
73.625	0.0936	6.7924027206501\\
73.625	0.0942	6.09052269090398\\
73.625	0.0948	5.38864266115797\\
73.625	0.0954	4.68676263141185\\
73.625	0.096	3.98488260166584\\
73.625	0.0966	3.28300257191984\\
73.625	0.0972	2.58112254217372\\
73.625	0.0978	1.87924251242771\\
73.625	0.0984	1.17736248268159\\
73.625	0.099	0.47548245293558\\
73.625	0.0996	-0.226397576810427\\
73.625	0.1002	-0.928277606556549\\
73.625	0.1008	-1.63015763630256\\
73.625	0.1014	-2.33203766604868\\
73.625	0.102	-3.03391769579468\\
73.625	0.1026	-3.73579772554081\\
73.625	0.1032	-4.43767775528681\\
73.625	0.1038	-5.13955778503282\\
73.625	0.1044	-5.84143781477894\\
73.625	0.105	-6.54331784452495\\
73.625	0.1056	-7.24519787427107\\
73.625	0.1062	-7.94707790401719\\
73.625	0.1068	-8.64895793376331\\
73.625	0.1074	-9.35083796350932\\
73.625	0.108	-10.0527179932553\\
73.625	0.1086	-10.7545980230014\\
73.625	0.1092	-11.4564780527475\\
73.625	0.1098	-12.1583580824936\\
73.625	0.1104	-12.8602381122396\\
73.625	0.111	-13.5621181419856\\
73.625	0.1116	-14.2639981717317\\
73.625	0.1122	-14.9658782014777\\
73.625	0.1128	-15.6677582312238\\
73.625	0.1134	-16.3696382609699\\
73.625	0.114	-17.0715182907159\\
73.625	0.1146	-17.773398320462\\
73.625	0.1152	-18.475278350208\\
73.625	0.1158	-19.1771583799541\\
73.625	0.1164	-19.8790384097001\\
73.625	0.117	-20.5809184394461\\
73.625	0.1176	-21.2827984691922\\
73.625	0.1182	-21.9846784989384\\
73.625	0.1188	-22.6865585286844\\
73.625	0.1194	-23.3884385584306\\
73.625	0.12	-24.0903185881765\\
73.625	0.1206	-24.7921986179226\\
73.625	0.1212	-25.4940786476687\\
73.625	0.1218	-26.1959586774148\\
73.625	0.1224	-26.8978387071609\\
73.625	0.123	-27.5997187369068\\
74	0.093	7.88686191364252\\
74	0.0936	7.20841675960423\\
74	0.0942	6.52997160556572\\
74	0.0948	5.85152645152732\\
74	0.0954	5.17308129748892\\
74	0.096	4.49463614345052\\
74	0.0966	3.81619098941212\\
74	0.0972	3.13774583537372\\
74	0.0978	2.45930068133532\\
74	0.0984	1.78085552729681\\
74	0.099	1.10241037325852\\
74	0.0996	0.423965219220122\\
74	0.1002	-0.254479934818391\\
74	0.1008	-0.932925088856678\\
74	0.1014	-1.61137024289519\\
74	0.102	-2.28981539693359\\
74	0.1026	-2.96826055097199\\
74	0.1032	-3.64670570501039\\
74	0.1038	-4.32515085904879\\
74	0.1044	-5.00359601308719\\
74	0.105	-5.68204116712559\\
74	0.1056	-6.3604863211641\\
74	0.1062	-7.0389314752025\\
74	0.1068	-7.7173766292409\\
74	0.1074	-8.3958217832793\\
74	0.108	-9.0742669373177\\
74	0.1086	-9.7527120913561\\
74	0.1092	-10.4311572453945\\
74	0.1098	-11.109602399433\\
74	0.1104	-11.7880475534713\\
74	0.111	-12.4664927075097\\
74	0.1116	-13.1449378615482\\
74	0.1122	-13.8233830155865\\
74	0.1128	-14.501828169625\\
74	0.1134	-15.1802733236634\\
74	0.114	-15.8587184777017\\
74	0.1146	-16.5371636317402\\
74	0.1152	-17.2156087857786\\
74	0.1158	-17.894053939817\\
74	0.1164	-18.5724990938554\\
74	0.117	-19.2509442478938\\
74	0.1176	-19.9293894019322\\
74	0.1182	-20.6078345559707\\
74	0.1188	-21.286279710009\\
74	0.1194	-21.9647248640475\\
74	0.12	-22.6431700180858\\
74	0.1206	-23.3216151721243\\
74	0.1212	-24.0000603261627\\
74	0.1218	-24.6785054802011\\
74	0.1224	-25.3569506342396\\
74	0.123	-26.0353957882778\\
};
\end{axis}
\end{tikzpicture}%
	\caption{Parameter maps reconstructed with LS surface fitting to the internal parameters estimated from the data sample of length 2000.}
\end{figure}
\begin{table}[!h]
	\centering
	\caption{Estimated polynomial coefficients for the sample length 4000.}
	\small
	\input{betas_C_ny_0_nu_4_size_4000.tex}
\end{table}
\begin{figure}[!h]
	\centering
	\input{Theta_surfaces_C_ny_0_nu_4_size_4000.tikz}
	\caption{Parameter maps reconstructed with LS surface fitting to the internal parameters estimated from the data sample of length 4000.}
\end{figure}
\end{document}